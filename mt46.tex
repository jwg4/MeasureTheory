\frfilename{mt46.tex}
\versiondate{26.8.13}
\copyrightdate{2000}

\def\chaptername{Pointwise compact sets of measurable functions}
\def\sectionname{Introduction}

\newchapter{46}

This chapter collects results inspired by problems in functional
analysis.   \S\S461 and 466 look directly at measures on
linear topological spaces.   The primary applications are of course to
Banach spaces, but as usual we quickly find ourselves considering weak
topologies.   In \S461 I look at `barycenters', or centres of mass, of
probability measures, with the basic theorems on existence and location
of barycenters of given measures and the construction of measures with
given barycenters.   In \S466 I examine topological measures on linear
spaces in terms of the classification developed in Chapter 41.   A
special class of normed spaces, those with `Kadec norms', is
particularly important, and in \S467 I sketch the theory of the most
interesting Kadec norms, the `locally uniformly rotund' norms.

In the middle sections of the chapter, I give an account of the theory
of pointwise compact sets of
measurable functions, as developed by A.Bellow, M.Talagrand and myself.
The first step is to examine pointwise compact sets of continuous
functions (\S462);  these have been extensively studied because they
represent an effective tool for investigating weakly compact sets in
Banach spaces, but here I give only results which are important in measure theory,
with a little background material.   In \S463 I present results on the
relationship between the two most important topologies on spaces of
measurable functions, {\it not} identifying functions which are equal
almost everywhere:  the pointwise topology and the topology of
convergence in measure.   These topologies have very different natures
but nevertheless interact in striking ways.   In particular, we have
important theorems giving conditions under which a pointwise compact set
of measurable functions will be compact for the topology of convergence
in measure (463G, 463L).

The remaining two sections are devoted to some remarkable ideas due to
Talagrand.   The first, `Talagrand's measure' (\S464), is a special
measure on $\Cal PI$ (or $\ell^{\infty}(I)$), extending the usual
measure of $\Cal PI$ in a canonical way.   In \S465 I turn to the theory
of `stable' sets of measurable functions, showing how a concept arising
naturally in the theory of pointwise compact sets led to a
characterization of Glivenko-Cantelli classes in the theory of empirical
measures.

\discrpage

