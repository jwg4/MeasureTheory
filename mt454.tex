\frfilename{mt454.tex}
\versiondate{6.12.05/6.2.07}
\copyrightdate{2001}

\def\chaptername{Perfect measures, disintegrations and processes}
\def\sectionname{Measures on product spaces}

\newsection{454}

A central concern of probability theory is the study of `processes',
that is, families $\family{t}{T}{X_t}$ of random variables thought of as
representing the evolution of a system in time.   The representation of
such processes as random variables in the modern sense, that is,
measurable functions on an abstract probability space, was one of the
first challenges faced by Kolmogorov.   In this
section I give a version of Kolmogorov's theorem on the extension of
consistent families of measures on subproducts to a measure on the whole
product (454D).   It turns out that some restriction on the marginal
measures is necessary, and `perfectness' seems to be an
appropriate hypothesis, necessarily satisfied if the factor spaces are
standard Borel spaces or the marginal measures are Radon measures.   If
we have marginal measures with stronger properties then we shall be
able to infer corresponding properties of the measure on the product
space (454A, generalizing 451J).

The apparatus here makes it easy to describe joint distributions of
arbitrary families of real-valued random variables (454J-454P),
extending the ideas of \S271.
For the sake of the theorem that almost all Brownian paths are
continuous (477B)
I briefly investigate measures on $C(T)$, where $T$ is a Polish space
(454Q-454S).  %454Q 454R 454S

\leader{454A}{Theorem} Let $\familyiI{(X_i,\Sigma_i,\mu_i)}$ be a
non-empty family of totally finite measure spaces.   Set
$X=\prod_{i\in I}X_i$ and let $\mu$ be a measure on $X$ which is inner
regular with
respect to the $\sigma$-algebra $\Tensorhat_{i\in I}\Sigma_i$ generated
by $\{\pi_i^{-1}[E]:i\in I,\,E\in\Sigma_i\}$, where $\pi_i:X\to X_i$ is
the coordinate map for each $i\in I$.   Suppose that every $\pi_i$ is
\imp.

(a) If $\Cal K\subseteq\Cal PX$ is a family of sets which is closed
under finite unions and countable intersections, and $\mu_i$ is inner
regular with respect to
$\Cal K_i=\{K:K\subseteq X_i,\,\pi_i^{-1}[K]\in\Cal K\}$ for every
$i\in I$, then $\mu$ is inner regular with respect to $\Cal K$.

(b)(i) If every $\mu_i$ is a compact measure, so is $\mu$;

\quad(ii) if every $\mu_i$ is a countably compact measure, so is $\mu$;

\quad(iii) if every $\mu_i$ is a perfect measure, so is $\mu$.

\proof{ If $X$ is empty this is all trivial, so we may suppose that
$X\ne\emptyset$.

\medskip

{\bf (a)} Set $\Cal A=\{\pi_i^{-1}[E]:i\in I,\,E\in\Sigma_i\}$.   If
$A\in\Cal A$, $V\in\Sigma$ and $\mu(A\cap V)>0$, there is a
$K\in\Cal K\cap\Cal A$ such that $K\subseteq A$ and $\mu(K\cap V)>0$.
\Prf\ Express $A$ as $\pi_i^{-1}[E]$, where $E\in\Sigma_i$;  take
$L\in\Cal K_i$ such that $L\subseteq E$ and
$\mu_iL>\mu_iE-\mu(A\cap V)$, and set $K=\pi_i^{-1}[L]$.\ \Qed

By 412C, $\mu\restr\Tensorhat_{i\in I}\Sigma_i$ is inner regular with
respect to $\Cal K$;  by 412Ab, so is $\mu$.

\medskip

{\bf (b)(i)-(ii)} Suppose that every $\mu_i$ is (countably) compact.
Then for each $i\in I$ we can find a (countably) compact class
$\Cal K_i\subseteq\Cal PX_i$ such that $\mu_i$ is inner regular with
respect
to $\Cal K_i$.   Set $\Cal L=\{\pi_i^{-1}[K]:i\in I,\,K\in\Cal K_i\}$.
Then $\Cal L$ is (countably) compact (451H).   So there is a
(countably) compact $\Cal K\supseteq\Cal L$ which is closed under finite
unions and countable intersections (342D, 413R).   Now $\mu$ is inner
regular with respect to $\Cal K$, by (a), and therefore (countably)
compact.

\medskip

\quad{\bf (iii)} Let $\Tau_0$ be a countably generated
$\sigma$-subalgebra of $\Tensorhat_{i\in I}\Sigma_i$.   Then there must
be some countable subfamily $\Cal E$ of
$\{\pi_i^{-1}[E]:i\in I,\,E\in\Sigma_i\}$ such that $\Tau_0$ is included
in the $\sigma$-algebra generated by $\Cal E$ (use 331Gd).   Set
$\Cal E_i=\{E:E\in\Sigma_i,\,\pi_i^{-1}[E]\in\Cal E\}$ for each $i$, so
that $\Cal E_i$ is countable, and let $\Sigma'_i$ be the
$\sigma$-algebra
generated by $\Cal E_i$.   Then $\mu_i\restr\Sigma'_i$ is compact
(451F).   Applying (i), we see that
$\mu\restr\Tensorhat_{i\in I}\Sigma'_i$ is compact, therefore perfect;
while $\Tau_0\subseteq\Tensorhat_{i\in I}\Sigma'_i$.   As $\Tau_0$ is
arbitrary, $\mu\restr\Tensorhat_{i\in I}\Sigma_i$ is perfect (451F).
But as the completion of $\mu$ is exactly the completion of
$\mu\restr\Tensorhat_{i\in I}\Sigma_i$, $\mu$ also is perfect, by 451Gc.
}%end of proof of 454A

\leader{454B}{Corollary} Let $\familyiI{X_i}$ be a family of Polish
spaces with product $X$.   Then any totally finite Baire measure on $X$
is a compact measure.

\proof{ If $\mu$ is a Baire measure on $X$, then its domain
$\CalBa(X)$ is $\Tensorhat_{i\in I}\Cal B(X_i)$, where $\Cal B(X_i)$
is the Borel $\sigma$-algebra of $X_i$ for each $i\in I$ (4A3Na).
So each
image measure $\mu_i$ on $X_i$ is a Borel measure, therefore tight (that
is, inner regular with respect to the closed compact sets, 433Ca),
and by 454A(b-i) $\mu$ is compact.
}%end of proof of 454B

\leader{454C}{Theorem}\cmmnt{ ({\smc Marczewski \& Ryll-Nardzewski 53})} 
Let $(X,\Sigma,\mu)$ be a perfect totally finite measure
space and $(Y,\Tau,\nu)$ any totally finite measure space.   Let
$\Sigma\otimes\Tau$ be the algebra of subsets of $X\times Y$ generated
by $\{E\times F:E\in\Sigma$, $F\in\Tau\}$.   If
$\lambda_0:\Sigma\otimes\Tau\to\coint{0,\infty}$ is a non-negative
finitely additive functional such that $\lambda_0(E\times Y)=\mu E$ and
$\lambda_0(X\times F)=\nu F$ whenever $E\in\Sigma$ and $F\in\Tau$, then
$\lambda_0$ has a unique extension to a measure defined on\cmmnt{ the
$\sigma$-algebra} $\Sigma\tensorhat\Tau$\cmmnt{ generated by
$\Sigma\otimes\Tau$}.

\proof{{\bf (a)} By 413Kb, it will be enough to show that
$\lim_{n\to\infty}\lambda_0W_n=0$ for every non-increasing sequence
$\sequencen{W_n}$ in $\Sigma\otimes\Tau$ with empty intersection.   Take
such a sequence.   Each $W_n$ must belong to the algebra generated by
some finite subset of $\{E\times F:E\in\Sigma,\,F\in\Tau\}$, so there
must be a countable set $\Cal E\subseteq\Sigma$ such that every $W_n$
belongs to the algebra generated by $\{E\times F:E\in\Cal
E,\,F\in\Tau\}$;  let $\Sigma_0$ be the $\sigma$-subalgebra of $\Sigma$
generated by $\Cal E$, so that every $W_n$ belongs to
$\Sigma_0\tensorhat\Tau$.

\medskip

{\bf (b)} By 451F, $\mu\restr\Sigma_0$ is a compact measure;  let
$\Cal K\subseteq\Cal PX$ be a compact class such that
$\mu\restr\Sigma_0$ is
inner regular with respect to $\Cal K$.   We may suppose that $\Cal K$
is the family of closed sets for a compact topology on $X$ (342Da).
Let $\Cal W$ be the family of those elements $W$ of
$\Sigma_0\otimes\Tau$
such that every horizontal section $W^{-1}[\{y\}]$ belongs to $\Cal K$.
Then $\Cal W$ is closed under finite unions and intersections.

\medskip

{\bf (c)} If $W\in\Sigma_0\otimes\Tau$ and $\epsilon>0$, then there is a
$W'\in\Cal W$ such that $W'\subseteq W$ and
$\lambda_0(W\setminus W')\le\epsilon$.   \Prf\ Express $W$ as
$\bigcup_{i\le n}E_i\times F_i$, where $E_i\in\Sigma_0$ and $F_i\in\Tau$
for each $i\le n$.   (Cf.\ 315Kb\formerly{3{}15J}.)   
For each $i\le n$, take
$K_i\in\Cal K\cap\Sigma_0$ such that
$\mu(E_i\setminus K_i)\le\bover1{n+1}\epsilon$, and set
$W'=\bigcup_{i\le n}K_i\times F_i$.   Then $W'\in\Cal W$,
$W'\subseteq W$ and

$$\eqalign{\lambda_0(W\setminus W')
&\le\sum_{i=0}^n\lambda_0((E_i\times F_i)\setminus(K_i\times F_i))
\le\sum_{i=0}^n\lambda_0((E_i\setminus K_i)\times Y)\cr
&=\sum_{i=0}^n\mu_0(E_i\setminus K_i)
\le\epsilon.\text{ \Qed}\cr}$$

\medskip

{\bf (d)} Take any $\epsilon>0$.   Then for each $n\in\Bbb N$ we can
find $W'_n\in\Cal W$ such that $W'_n\subseteq W_n$ and
$\lambda_0(W_n\setminus W'_n)\le 2^{-n}\epsilon$.   Set
$V_n=\bigcap_{i\le n}W'_i$, so that $V_n\in\Cal W$ and


\Centerline{$\lambda_0(W_n\setminus V_n)
\le\sum_{i=0}^n\lambda_0(W_i\setminus W'_i)
\le 2\epsilon$}

\noindent for each $n$, and $\sequencen{V_n}$ is non-increasing, with
empty intersection.

Because $V_n\in\Sigma_0\otimes\Tau$, its projection $H_n=V_n[X]$ belongs
to $\Tau$, for each $n$.   Of course $\sequencen{H_n}$ is
non-increasing;  also $\bigcap_{n\in\Bbb N}H_n=\emptyset$.   \Prf\ If
$y\in Y$, then $\sequencen{V_n^{-1}[\{y\}]}$ is a non-increasing
sequence in $\Cal K$ with empty intersection, because $\bigcap_{n\in\Bbb
N}V_n\subseteq\bigcap_{n\in\Bbb N}W_n$ is empty.  But $\Cal K$ is a
compact class, so there must be some $n$ such that $V_n^{-1}[\{y\}]$ is
empty, that is, $y\notin H_n$.\ \QeD\  Accordingly $\lim_{n\to\infty}\nu
H_n=0$.   But as $V_n\subseteq X\times H_n$,
$\lim_{n\to\infty}\lambda_0V_n=0$.

This means that $\lim_{n\to\infty}\lambda_0W_n\le 2\epsilon$.   But as
$\epsilon$ is arbitrary, $\lim_{n\to\infty}\lambda_0W_n=0$, as required.
}%end of proof of 454C

\leader{454D}{Theorem}\cmmnt{ ({\smc Kolmogorov 33}, \S III.4)} Let
$\familyiI{(X_i,\Sigma_i,\mu_i)}$ be a family
of totally finite perfect measure spaces.   Set $X=\prod_{i\in I}X_i$,
and write $\bigotimes_{i\in I}\Sigma_i$ for the algebra of subsets of
$X$ generated by $\{\pi_i^{-1}[E]:i\in I,\,E\in\Sigma_i\}$, where
$\pi_i:X\to X_i$ is the coordinate map for each $i\in I$.   Suppose that
$\lambda_0:\bigotimes_{i\in I}\Sigma_i\to\coint{0,\infty}$ is a
non-negative finitely additive functional such that
$\lambda_0\pi_i^{-1}[E]=\mu_iE$ whenever $i\in I$ and $E\in\Sigma_i$.
Then $\lambda_0$ has a unique extension to a measure $\lambda$ with
domain $\Tensorhat_{i\in I}\Sigma_i$, and $\lambda$ is perfect.

\proof{{\bf (a)} The argument follows the same pattern as that of 454C.
This time, take a non-increasing sequence $\sequencen{W_n}$ in
$\bigotimes_{i\in I}\Sigma_i$ with empty intersection.   Each $W_n$
belongs to the algebra generated by some finite subset of
$\{\pi_i^{-1}[E]:i\in I,\,E\in\Sigma_i\}$, so we can find countable sets
$\Cal E_i\subseteq\Sigma_i$ such that every $W_n$ belongs to the
subalgebra generated by $\{\pi_i^{-1}[E]:i\in I,\,E\in\Cal E_i\}$.   Let
$\Tau_i$ be the $\sigma$-subalgebra of $\Sigma_i$ generated by
$\Cal E_i$, so that every $W_n$ belongs to $\bigotimes_{i\in I}\Tau_i$.

\medskip

{\bf (b)} For each $i\in I$, $\mu_i\restrp\Tau_i$ is compact (451F);
let $\frak T_i$ be a compact topology on $X_i$ such that
$\mu_i\restrp\Tau_i$
is inner regular with respect to the closed sets (342F).   Let $\frak T$
be the product topology on $X$, so that $\frak T$ is compact (3A3J).
Let $\Cal W$ be the family of closed sets in $X$ belonging to
$\bigotimes_{i\in I}\Tau_i$.

\medskip

{\bf (c)} If $W\in\bigotimes_{i\in I}\Tau_i$ and $\epsilon>0$, there is
a $W'\in\Cal W$ such that $W'\subseteq W$ and
$\lambda_0(W\setminus W')\le\epsilon$.   \Prf\ We can express $W$ as
$\bigcup_{k\le n}\bigcap_{i\in J_k}\pi_i^{-1}[E_{ki}]$ where each $J_k$
is a finite
subset of $I$ and $E_{ki}\in\Sigma_i$ for $k\le n$, $i\in J_k$ (cf.\
315Kb).   Let $\langle\epsilon_{ki}\rangle_{k\le n,i\in J_k}$ be a
family of strictly positive numbers with sum at most $\epsilon$.   For
each $k\le n$, $i\in J_k$ take a closed set $K_{ki}\in\Tau_i$ such that
$K_{ki}\subseteq E_{ki}$ and
$\mu_i(E_{ki}\setminus K_{ki})\le\epsilon_{ki}$, and set
$W'=\bigcup_{k\le n}\bigcap_{i\in J_k}\pi_i^{-1}[K_{ki}]$.\ \Qed

\medskip

{\bf (d)} Take any $\epsilon>0$.   Then for each $n\in\Bbb N$ we can
find $W'_n\in\Cal W$ such that $W'_n\subseteq W_n$ and
$\lambda_0(W_n\setminus W'_n)\le 2^{-n}\epsilon$.   Set
$V_n=\bigcap_{i\le n}W'_i$.   Then $\sequencen{V_n}$ is a non-increasing
sequence of closed sets in the compact space $X$, and has empty
intersection, so there is some $n$ such that $V_n$ is empty, and

\Centerline{$\lambda_0W_n\le\sum_{i=0}^n\lambda_0(W_i\setminus W'_i)
\le 2\epsilon$.}

\noindent As $\epsilon$ is arbitrary, $\lim_{n\to\infty}\lambda_0W_n=0$.

\medskip

{\bf (e)} As $\sequencen{W_n}$ is arbitrary, $\lambda_0$ has a unique
countably additive extension to $\Tensorhat_{i\in I}\Sigma_i$, by 413Kb,
as before.   Of course the extension is perfect, by 454A(b-iii).
}%end of proof of 454D

\leader{454E}{Corollary} Let $\familyiI{(X_i,\Sigma_i,\mu_i)}$ be a
family of perfect measure spaces.
Let $\Cal C$ be the family of subsets of $X=\prod_{i\in I}X_i$
expressible in the form $X\cap\bigcap_{i\in J}\pi_i^{-1}[E_i]$ where
$J\subseteq I$ is finite and $E_i\in\Sigma_i$ for every $i\in I$,
writing $\pi_i(x)=x(i)$ for $x\in X$, $i\in I$.   Suppose that
$\lambda_0:\Cal C\to\Bbb R$ is a functional such that (i)
$\lambda_0\pi_i^{-1}[E]=\mu_iE$ whenever $i\in I$ and $E\in\Sigma_i$
(ii) $\lambda_0C
=\lambda_0(C\cap\pi_i^{-1}[E])+\lambda_0(C\setminus\pi_i^{-1}[E])$
whenever $C\in\Cal C$, $i\in I$ and $E\in\Sigma_i$.   Then $\lambda_0$
has a unique extension to a measure on $\Tensorhat_{i\in I}\Sigma_i$,
which is necessarily perfect.

\proof{ By 326E\formerly{3{}26Q},
$\lambda_0$ has an extension to an additive functional
on $\bigotimes_{i\in I}\Sigma_i$, so we can apply 454D.
}%end of proof of 454E

\leader{454F}{Corollary} Let $\familyiI{(X_i,\Sigma_i)}$ be a family of
standard Borel spaces.   Set $X=\prod_{i\in I}X_i$, and let
$\bigotimes_{i\in I}\Sigma_i$ be the algebra of subsets of $X$ generated
by $\{\pi_i^{-1}[E]:i\in I,\,E\in\Sigma_i\}$, where $\pi_i:X\to X_i$ is
the coordinate map for each $i$.   Let
$\lambda_0:\bigotimes_{i\in I}\Sigma_i\to\coint{0,\infty}$ be a
non-negative finitely
additive functional such that all the marginal functionals
$E\mapsto\lambda_0\pi_i^{-1}[E]:\Sigma_i\to\coint{0,\infty}$ are
countably additive.   Then $\lambda_0$ has a unique extension to a
measure defined on $\Tensorhat_{i\in I}\Sigma_i$, which is a compact
measure.

\proof{ This follows immediately from 454D and 454A if we note that all
the measures $\lambda_0\pi_i^{-1}$ are necessarily compact, therefore
perfect (451M).
}%end of proof of 454F

\vleader{48pt}{454G}{Corollary} Let $\familyiI{X_i}$ be a family of
sets, and
$\Sigma_i$ a $\sigma$-algebra of subsets of $X_i$ for each $i\in I$.
Suppose that for each finite set $J\subseteq I$ we are given a totally
finite measure $\mu_J$ on $Z_J=\prod_{i\in J}X_i$ with domain
$\Tensorhat_{i\in J}\Sigma_i$ such that (i) whenever $J$, $K$ are finite
subsets of $I$ and $J\subseteq K$, then the canonical projection from
$Z_K$ to $Z_J$ is \imp\ (ii) every marginal measure $\mu_{\{i\}}$ on
$Z_{\{i\}}\cong X_i$ is perfect.   Then there is a unique measure $\mu$
defined on $\Tensorhat_{i\in I}\Sigma_i$ such that the canonical
projection $\tilde\pi_J:\prod_{i\in I}X_i\to Z_J$ is \imp\ for every
finite $J\subseteq I$.

\proof{ All we need to observe is that

\Centerline{$\bigotimes_{i\in I}\Sigma_i
=\{\tilde\pi_J^{-1}[V]:J\in[I]^{<\omega},\,
  V\in\bigotimes_{i\in J}\Sigma_i\}$.}

\noindent Because all the canonical projections from $X_K$ onto $X_J$
are \imp, we have $\mu_JV=\mu_KV'$ whenever $J$, $K$ are finite subsets
of $I$, $V\in\bigotimes_{i\in J}\Sigma_i$,
$V'\in\bigotimes_{i\in K}\Sigma_i$ and
$\tilde\pi_J^{-1}[V]=\tilde\pi_K^{-1}[V']$.   So we have
a functional $\lambda_0:\bigotimes_{i\in I}\Sigma_i\to\coint{0,\infty}$
such that $\lambda_0\tilde\pi_J^{-1}[V]=\mu_JV$ whenever $J\subseteq I$
is finite and $V\in\bigotimes_{i\in J}\Sigma_i$.   It is easy to check
that $\lambda_0$ is finitely additive and satisfies the conditions of
454D.   So $\lambda_0$ can be extended to a measure $\mu$ defined on
$\Tensorhat_{i\in I}\Sigma_i$.

If $J\subseteq I$ is finite, then $\mu_J$ and $\mu\tilde\pi_J^{-1}$
agree on $\bigotimes_{i\in J}\Sigma_J$ and therefore (by the Monotone
Class Theorem, 136C) on $\Tensorhat_{i\in J}\Sigma_i$;  that is,
$\tilde\pi_J$
is \imp.   To see that $\mu$ itself is unique, observe that the
conditions define its values on $\bigotimes_{i\in I}\Sigma_i$ and
therefore on
$\Tensorhat_{i\in I}\Sigma_i$, by the Monotone Class Theorem once more.
}%end of proof of 454G

\leader{454H}{Corollary} Let $\sequencen{(X_n,\Sigma_n)}$ be a sequence
of standard Borel spaces.   For each $n\in\Bbb N$ set
$Z_n=\prod_{i<n}X_i$ and $\Tau_n=\Tensorhat_{i<n}\Sigma_i$.
\cmmnt{(For $n=0$, we have $Z_0=\{\emptyset\}$,
$\Tau_0=\{\emptyset,Z_0\}$.)}   For $n\in\Bbb N$, $W\in\Tau_{n+1}$,
$z\in Z_n$ write $W[\{z\}]=\{\xi:\xi\in X_n,\,(z,\xi)\in W\}$;  set
$X=\prod_{n\in\Bbb N}X_n$ and write $\tilde\pi_n$ for the canonical
projection of $X$ onto $Z_n$.   Suppose that for each $n\in\Bbb N$ and
$z\in Z_n$ we are given a probability measure $\nu_z$ on $X_n$ with
domain $\Sigma_n$ such that $z\mapsto\nu_z(E)$ is $\Tau_n$-measurable
for every $E\in\Sigma_n$.   Then there is a unique probability measure
$\mu$ on $X=\prod_{n\in\Bbb N}X_n$, with domain
$\Sigma=\Tensorhat_{n\in\Bbb N}\Sigma_n$, such that, writing
$\tilde\mu_n$ for
the image measure $\mu\tilde\pi_n^{-1}$ on $Z_n$,

\Centerline{$\tilde\mu_{n+1}(W)=\biggerint\nu_zW[\{z\}]\tilde\mu_n(dz)$}

\noindent for every $n\in\Bbb N$ and $W\in\Tau_{n+1}$, and

$$\eqalign{\int fd\tilde\mu_{n+1}
&=\iint\ldots\iint f(\xi_0,\ldots,\xi_n)
  \nu_{(\xi_0,\ldots,\xi_{n-1})}(d\xi_n)\cr
&\qquad\qquad\qquad\qquad\nu_{(\xi_0,\ldots,\xi_{n-2})}(d\xi_{n-1})
  \ldots\nu_{\xi_0}(d\xi_1)\nu_{\emptyset}(d\xi_0)\cr}$$

\noindent for every $n\in\Bbb N$ and $\tilde\mu_{n+1}$-integrable
real-valued function $f$.

\proof{{\bf (a)} We must check that the first formula given actually
defines $\tilde\mu_{n+1}(W)$ for every $W\in\Tau_{n+1}$.   Of course
this is
an induction on $n$.   $\tilde\mu_0$ will be the unique probability
measure on
the singleton set $Z_0$.   Given that $\dom\tilde\mu_n=\Tau_n$, the
class $\Cal W$ of sets $W\subseteq Z_{n+1}$ for which
$\int\nu_zW[\{z\}]\tilde\mu_n(dz)$ is defined
will contain all sets of the form $\prod_{i\le n}E_i$ where
$E_i\in\Sigma_i$ for every $i\le n$, just because the function
$z\mapsto\nu_zE_n$ is $\Tau_n$-measurable.   Since $\Cal W$ is
closed under increasing sequential unions and differences of comparable
sets, the Monotone Class Theorem (136B) tells us that it includes the
$\sigma$-algebra generated by the cylinder sets, which is $\Tau_{n+1}$.

\medskip

{\bf (b)} Accordingly $\family{z}{Z_n}{\nuprime_z}$ is a disintegration of
$\tilde\mu_{n+1}$ over $\tilde\mu_n$, where $\nuprime_z$ is the measure
with
domain $\Tau_{n+1}$ defined by writing $\nuprime_z(W)=\nu_zW[\{z\}]$ for
$W\in\Tau_{n+1}$, $z\in Z_n$.   By 452F,

$$\eqalign{\int_{Z_{n+1}}fd\tilde\mu_{n+1}
&=\int_{Z_n}\int_{Z_{n+1}}f(w)\nuprime_z(dw)\tilde\mu_n(dz)\cr
&=\int_{Z_n}\int_{X_n}f(z,\xi_{n+1})
   \nu_z(d\xi_{n+1})\tilde\mu_n(dz)\cr}$$

\noindent for every $\tilde\mu_{n+1}$-integrable function $f$.   (The
second
equality can be regarded as an application of the change-of-variable
formula 235Gb applied to the $(\nu_z,\nuprime_z)$-\imp\ function
$\xi\mapsto (z,\xi):X_n\to Z_{n+1}$.)   Now a direct induction
yields the
general formula for $\int fd\tilde\mu_{n+1}$ in the statement of this
corollary.

\medskip

{\bf (c)} The canonical maps from $Z_{n+1}$ to $Z_n$ are all \imp, just
because every $\nu_z$ is a probability measure.   We therefore have a
well-defined functional
$\lambda_0:\bigotimes_{n\in\Bbb N}\Sigma_n\to[0,1]$ defined by setting
$\lambda_0\tilde\pi_n^{-1}[W]=\tilde\mu_nW$ whenever $n\in\Bbb N$ and
$W\in\bigotimes_{i<n}\Sigma_i$, and this $\lambda_0$ is finitely
additive;  moreover, each marginal measure $\lambda_0\pi_n^{-1}$, where
$\pi_n:X\to X_n$ is the coordinate map, is countably additive, because
it is expressible as an image measure of $\tilde\mu_{n+1}$ on $X_n$.

\medskip

{\bf (d)} Everything so far has been valid for any sequence
$\sequencen{(X_n,\Sigma_n)}$ of sets with attached $\sigma$-algebras.
But at this point we note that the marginal measures
$\lambda_0\pi_n^{-1}$ must be perfect, because $(X_n,\Sigma_n)$ is a
standard Borel space.   So Theorem 454D gives the result.
}%end of proof of 454H

\cmmnt{
\leader{454I}{Remarks} In 454F and 454H the hypotheses call
for `standard Borel spaces' $(X_i,\Sigma_i)$.   As the proofs make
clear, what is needed in each case is that `every totally finite measure
with domain $\Sigma_i$ must be perfect'.   We have already seen other
ways in which this can be true:  for instance, if $X$ is any Radon
Hausdorff space (434C), and $\Sigma$ its Borel $\sigma$-algebra.
Further examples are in 454Xd, 454Xh-454Xi %454Xh 454Xe 454Xi
and 454Yb-454Yc.
Indeed, even weaker hypotheses can be fully adequate.   In 454H, for
instance, it will be quite enough if all the marginal measures on the
factors $X_n$ are perfect;  in view of 454A and 451E, this will be so
iff all the measures $\tilde\mu_n$ on the partial products $Z_n$ are
perfect.   It may be difficult to be sure of this unless either
we have some argument from the nature of the factor spaces
$(X_n,\Sigma_n)$, as
suggested above, or a clear understanding of the marginal measures.   In
applications such as 455A below, however, there may be other approaches
available.
}%end of comment

\vleader{72pt}{454J}{Distributions of random \dvrocolon{processes}}\cmmnt{
For the next few paragraphs I shift to probabilists' notation.

\medskip

\noindent}{\bf Proposition} Let $(\Omega,\Sigma,\mu)$ be a probability
space and $\familyiI{X_i}$ a family of real-valued random variables on
$\Omega$\cmmnt{ (see \S271)}.

(i) There is a unique complete probability measure $\nu$ on $\BbbR^I$,
measuring every Baire set and inner regular with respect to the zero
sets, such that

\Centerline{$\nu\{x:x\in\BbbR^I$,
  $x(i_r)\le\alpha_r$ for every $r\le n)
=\Pr(X_{i_r}\le\alpha_r$ for every $r\le n)$}

\noindent whenever $i_0,\ldots,i_n\in I$ and
$\alpha_0,\ldots,\alpha_n\in\Bbb R$.

(ii) If $i_0,\ldots,i_n\in I$ and $\tilde\pi(x)=(x(i_0),\ldots,x(i_n))$
for $x\in\BbbR^I$, then the image measure $\nu\tilde\pi^{-1}$ on
$\BbbR^{n+1}$ is the joint distribution of
$X_{i_0},\ldots,X_{i_n}$\cmmnt{ as defined in 271C}.

(iii) $\nu$ is a compact measure.   If $I$ is countable then $\nu$ is a
Radon measure.

(iv) If every $X_i$ is defined everywhere on $\Omega$, then the function
$\omega\mapsto\familyiI{X_i(\omega)}:\Omega\to\BbbR^I$ is \imp\ for
$\hat\mu$ and $\nu$, where $\hat\mu$ is the completion of $\mu$.

\proof{{\bf (a)} Completing $\mu$, and adjusting the $X_i$ on negligible
sets, does not change any of the joint distributions of families
$X_{i_0},\ldots,X_{i_n}$ (271Ad), so we may suppose henceforth that
$\mu$ is complete and that every $X_i$ is defined on the whole of
$\Omega$.   Set $\phi(\omega)=\familyiI{X_i(\omega)}$ for
$\omega\in\Omega$.   Then
$\{F:F\subseteq\BbbR^I$, $\phi^{-1}[F]\in\Sigma\}$ is a $\sigma$-algebra
of subsets of $\BbbR^I$ containing $\{x:x(i)\le\alpha\}$ whenever
$i\in I$ and $\alpha\in\Bbb R$, so includes the Baire $\sigma$-algebra
$\CalBa(\BbbR^I)$ of $\BbbR^I$ (4A3Na).
If we define $\nu_0F=\mu\phi^{-1}[F]$ for $F\in\CalBa(\BbbR^I)$, $\nu_0$
is a Baire measure on $\BbbR^I$ for which $\phi$ is \imp.   We are
supposing that $\mu$ is complete, so $\phi$ is still \imp\ for $\mu$ and
the completion $\nu$ of $\nu_0$ (234Ba\formerly{2{}35Hc}).
Since $\nu_0$ is inner
regular with respect to the zero sets (412D), so is $\nu$ (412Ha), and
of course $\nu$ measures every Baire set.   By 454B, $\nu_0$ is compact,
so $\nu$ also is (451Ga).

\medskip

{\bf (b)} If $i_0,\ldots,i_n\in I$ and
$\alpha_0,\ldots,\alpha_n\in\Bbb R$, then

$$\eqalign{\Pr(X_{i_r}\le\alpha_r\text{ for every }r\le n)
&=\mu\{\omega:X_{i_r}(\omega)\le\alpha_r\text{ for }r\le n\}\cr
&=\nu\{x:\phi(x)(i_r)\le\alpha_r\text{ for }r\le n\}\cr
&=\nu\{x:x(i_r)\le\alpha_r\text{ for }r\le n\}.\cr}$$

\medskip

{\bf (c)} If $i_0,\ldots,i_n\in I$ and we set
$\tilde\pi(x)=(x(i_0),\ldots,x(i_n))$ for $x\in\BbbR^I$, then
$\nu\tilde\pi^{-1}$ is a Radon measure (451O).   Since it agrees with
the distribution of $X_{i_0},\ldots,X_{i_n}$ on all sets of the form
$\{z:z(r)\le\alpha_r$ for $r\le n\}$, it must be exactly the
distribution of $X_{i_0},\ldots,X_{i_n}$ (271Ba).

\medskip

{\bf (d)} If $I$ is countable, then $\BbbR^I$ is Polish, so $\nu$ is a
Radon measure (433Cb).

\medskip

{\bf (e)} The only point I have not covered is the uniqueness of $\nu$.
But suppose that $\nuprime$ is another measure on $\BbbR^I$ with the
properties described in (i).   If $i_0,\ldots,i_n\in I$ and
$\tilde\pi(x)=(x(i_0),\ldots,x(i_n))$ for $x\in\BbbR^I$, then the image
measures $\nu\tilde\pi^{-1}$ and $\nuprime\tilde\pi^{-1}$ on
$\BbbR^{n+1}$ are both the distribution of $X_{i_0},\ldots,X_{i_n}$, by
the argument of (c) above.   This means that $\nu$ and $\nuprime$ agree
on the algebra of subsets of $\BbbR^I$ generated by sets of the form
$\{x:x(i)\in E\}$ where $i\in I$ and $E\subseteq\Bbb R$ is Borel.   By
454D, they agree on all zero sets, and must be equal (412L).
}%end of proof of 454J

\leader{454K}{Definition} In the context of 454J, I will call $\nu$
the ({\bf joint}) {\bf distribution} of the process $\familyiI{X_i}$.

\cmmnt{Note that if $I=n\in\Bbb N\setminus\{0\}$, then $\nu$ is a
Radon measure on $\BbbR^n$, so is the distribution of
$\ofamily{i}{n}{X_i}$ in the sense of 271C.}

\leader{454L}{\dvrocolon{Independence}}\dvAnew{2009}\cmmnt{ With this
extension of
the notion of `distribution' we have a straightforward reformulation of the
characterization of independence in 272G.

\medskip

\noindent}{\bf Theorem} Let $(\Omega,\Sigma,\mu)$ be a probability
space and $\familyiI{X_i}$ a family of real-valued random variables on
$\Omega$, with distribution $\nu$ on $\BbbR^I$.   Then $\familyiI{X_i}$ is
independent iff $\nu$ is the c.l.d.\
product of the marginal measures on $\Bbb R$.

\proof{{\bf (a)} For $i\in I$, write $\nu_i$ for the marginal measure
$\mu\pi_i^{-1}$ on $\Bbb R$, taking $\pi_i(x)=x(i)$ as usual.
If $J\subseteq I$ is finite, and $\tilde\pi_J(x)=x\restr J$, then
$\nu\tilde\pi_J^{-1}$ is the distribution (in the sense of Chapter 27) of
$\family{i}{J}{X_i}$, by 454J(iii).   In particular, $\nu_i$ is the
distribution of $X_i$ for each $i$.

\medskip

{\bf (b)} If $\nu$ is the product measure $\prod_{i\in I}\nu_i$, and
$J\subseteq I$ is finite, then
$\nu\tilde\pi_J^{-1}$ is the product measure $\prod_{i\in J}\nu_i$
(254Oa), so $\family{i}{J}{X_i}$ is independent (272G).   As $J$ is
arbitrary, $\familyiI{X_i}$ is independent (272Bb).

\medskip

{\bf (c)} Conversely, if $\familyiI{X_i}$ is independent, then
$\nu$ agrees with $\lambda=\prod_{i\in I}\nu_i$ on all sets of the form
$\{x:x(i)\le\alpha_i$ for $i\in J\}$ where $J\subseteq I$ is finite and
$\family{i}{J}{\alpha_i}\in\BbbR^J$.   By the uniqueness assertion in
454J(i), $\nu=\lambda$.
}%end of proof of 454L

\leader{454M}{}\dvAnew{2009}\cmmnt{ The fundamental existence theorem
454G takes a more direct form in this context.

\medskip

\noindent}{\bf Proposition} Let $I$ be a set, and suppose that for each
finite $J\subseteq I$ we are given a Radon probability measure $\nu_J$ on
$\BbbR^J$ such that whenever $K$ is a finite subset of $I$ and
$J\subseteq K$, then the canonical projection from $\BbbR^K$ to $\BbbR^J$
is \imp.   Then there is a unique complete probability measure $\nu$ on
$\BbbR^I$, measuring every Baire set and inner regular with respect to the
zero sets, such that the canonical projection from $\BbbR^I$ to $\BbbR^J$
is \imp\ for every finite $J\subseteq I$.

\proof{ For finite $J\subseteq I$, let $\mu_J$ be the restriction of
$\nu_J$ to the Borel $\sigma$-algebra $\Cal B(\BbbR^J)$.   Then the
canonical projection from $\BbbR^K$ to $\BbbR^J$
is \imp\ for $\mu_K$ and $\mu_J$ whenever $J\subseteq K$ are finite subsets
of $I$.   Moreover, $\mu_{\{i\}}$ is a Borel measure on $\Bbb R$, therefore
perfect, for every $i\in I$.   By 454G, we have a
unique Baire probability measure
$\mu$ on $\BbbR^I$ such that the projections $\BbbR^I\to\BbbR^J$ are
$(\mu,\mu_J)$-\imp\ for all finite $J\subseteq I$.   Let $\nu$ be the
completion of $\mu$;  then the projections are $(\nu,\nu_J)$-\imp\ because
$\nu_J$ is always the completion of $\mu_J$.   Finally, $\nu$ is unique
because $\nu\restr\CalBa(\BbbR^I)$ must have the defining property for
$\mu$.
}%end of proof of 454M

\leader{454N}{}\dvAnew{2009}\cmmnt{ We know that Radon measures are often
determined
by the integrals they give to continuous functions (415I).   If we look at
distributions we get a stronger result for probability measures.

\medskip

\noindent}{\bf Proposition} Let $\Omega$ be a Hausdorff space, $\mu$ and
$\nu$ two Radon probability measures on $\Omega$, and $\familyiI{X_i}$ a
family of continuous functions separating the points of $\Omega$.   If
$\mu$ and $\nu$ give $\familyiI{X_i}$ the same distribution, they are
equal.

\proof{{\bf (a)} If $K$ and $L$ are disjoint compact subsets of $\Omega$,
there is an open set $G$ such that $K\subseteq G\subseteq X\setminus L$
and $\mu G=\nu G$.   \Prf\
$W_i=\{(\omega,\omega'):X_i(\omega)\ne X_i(\omega')\}$ is an open subset of
$\Omega\times\Omega$, and $\bigcup_{i\in I}W_i$ includes the compact set
$K\times L$.   So there is a finite set $J\subseteq I$ such that
$K\times L\subseteq\bigcup_{i\in J}W_i$.   Define $f:\Omega\to\BbbR^J$ by
setting $f(\omega)(i)=X_i(\omega)$ for $\omega\in\Omega$ and $i\in J$;
then $f$ is continuous, and $f[K]\cap f[L]=\emptyset$.   Also the image
measures $\mu f^{-1}$ and $\nu f^{-1}$ must be the same, because they are
both the common distribution of $\family{i}{J}{X_i}$.   Set
$G=\Omega\setminus f^{-1}[L]$;  this works.\ \Qed

\medskip

{\bf (b)} Now if $E\subseteq\Omega$ is a Borel set, and $\epsilon>0$,
there are compact sets $K\subseteq E$, $L\subseteq\Omega\setminus E$ such
that $\mu K\ge\mu E-\epsilon$ and
$\nu L\ge\nu(\Omega\setminus E)-\epsilon$.   Let $G$ be an open set such
that $\mu G=\nu G$ and $K\subseteq G\subseteq\Omega\setminus L$.   Then

$$\eqalign{\mu E
&\le\epsilon+\mu K\le\epsilon+\mu G
=\epsilon+\nu G
\le\epsilon+\nu(\Omega\setminus L)\cr
&=\epsilon+1-\nu L
\le 2\epsilon+1-\nu(\Omega\setminus E)
=2\epsilon+\nu E.\cr}$$

\noindent As $\epsilon$ is arbitrary, $\mu E\le\nu E$;  similarly,
$\nu E\le\mu E$.   As $E$ is arbitrary, $\mu$ and $\nu$ agree on the Borel
sets and must coincide.
}%end of proof of 454N

\vleader{72pt}{454O}{Proposition}\dvAformerly{4{}54L} Let
$(\Omega,\Sigma,\mu)$ be a probability
space and $\familyiI{X_i}$ a family of random variables on $\Omega$
with distribution $\nu$.    If $\sequencen{i_n}$ is a sequence in $I$
and $f:\BbbR^{\Bbb N}\to\Bbb R$ is a Borel measurable function, then
the random variables

\Centerline{$\omega\mapsto f(\sequencen{X_{i_n}(\omega)}):
  \bigcap_{n\in\Bbb N}\dom X_{i_n}\to\Bbb R$,
\quad$x\mapsto f(\sequencen{x(i_n)}):\BbbR^I\to\Bbb R$}

\noindent have the same distribution.

\proof{ Completing $\mu$ and extending each $X_i$ to the whole of
$\Omega$, we may suppose that $\phi:\Omega\to\BbbR^I$ is \imp\ for $\mu$
and $\nu$, where
$\phi(\omega)=\familyiI{X_i(\omega)}$ for $\omega\in\Omega$.   Now, for
any $\alpha\in\Bbb R$,

$$\eqalign{\nu\{x:f(\sequencen{x(i_n)})\le\alpha\}
&=\mu\{\omega:f(\sequencen{\phi(\omega)(i_n)})\le\alpha\}\cr
&=\mu\{\omega:f(\sequencen{X_{i_n}(\omega)})\le\alpha\},\cr}$$

\noindent so the distributions are the same.
}%end of proof of 454O

\leader{454P}{Theorem}\dvAformerly{4{}54M} Let $I$ be a set.

(a) Let $\nu$ and $\nuprime$ be Baire probability measures on $\BbbR^I$
such that $\int e^{if(x)}\nu(dx)=\int e^{if(x)}\nuprime(dx)$ for every
continuous linear functional $f:\BbbR^I\to\Bbb R$.   Then
$\nu=\nuprime$.

(b) Let $\family{j}{I}{X_j}$ and $\family{j}{I}{Y_j}$ be two families of
random variables such that

\Centerline{$\Expn(\exp(i\sum_{r=0}^n\alpha_rX_{j_r}))
=\Expn(\exp(i\sum_{r=0}^n\alpha_rY_{j_r}))$}

\noindent whenever $j_0,\ldots,j_n\in I$ and
$\alpha_0,\ldots,\alpha_n\in\Bbb R$.   Then $\family{j}{I}{X_j}$ and
$\family{j}{I}{Y_j}$ have the same distribution.

\proof{{\bf (a)} For each finite set $J\subseteq I$, write
$\tilde\pi_J(x)=x\restr J$ for $x\in X$.   Then we have Radon measures
$\mu_J$ and $\mu'_J$ on $\BbbR^J$ defined by saying that
$\mu_JF=\nu\tilde\pi_J^{-1}[F]$,
$\mu'_JF=\nuprime\tilde\pi_J^{-1}[F]$ for Borel sets
$F\subseteq\BbbR^J$.
If $\family{j}{J}{\alpha_j}\in\BbbR^J$, then

$$\eqalign{\int\exp(i\sum_{j\in J}\alpha_jz(j))\mu_J(dz)
&=\int\exp(i\sum_{j\in J}\alpha_jx(j))\nu(dx)\cr
&=\int\exp(i\sum_{j\in J}\alpha_jx(j))\nuprime(dx)
=\int\exp(i\sum_{j\in J}\alpha_jz(j))\mu'_J(dz),\cr}$$

\noindent so $\mu_J$ and $\mu'_J$ have the same characteristic function,
therefore are equal (285M).   This is true for every $J$, so $\nu$ and
$\nuprime$ are equal, by 454D.

\medskip

{\bf (b)} Taking $\nu$ and $\nuprime$ to be the two distributions, (a)
(with
454O) tells us that their restrictions to the Baire $\sigma$-algebra of
$\BbbR^I$ are the same, so they must be identical.
}%end of proof of 454P

\leader{454Q}
{Continuous \dvrocolon{processes}}\dvAformerly{4{}54N}\cmmnt{ The original,
and still by far the most important, context for 454D is when every
$(X_i,\Sigma_i)$ is $\Bbb R$ with its Borel $\sigma$-algebra, so that
$X=\prod_{i\in I}X_i$ can be identified with $\BbbR^I$.   In the
discussion so far, the set $I$ has been an abstract set, except in the
very special case of 454H.   But some of the most important applications
(to which I shall come in \S455) involve index sets carrying
a topological structure;  for instance, $I$ could be the unit interval
$[0,1]$ or the half-line $\coint{0,\infty}$.   In such a case, we have a
wide variety of subspaces of $\BbbR^I$ (for instance, the space of
continuous functions) marked out as special, and it is important to know
when, and in what sense, our measures on the product space $\BbbR^I$
can be regarded as, or replaced by, measures on the subspace of
interest.   In the next few paragraphs I look briefly at spaces of
continuous functions on Polish spaces.

\medskip

\noindent}{\bf Lemma} Let $T$ be a separable metrizable space and
$(X,\Sigma,\mu)$ a semi-finite measure space.   Let $\frak T$ be a
topology on $X$ such that $\mu$ is inner regular with respect to the
closed sets.

(a) Let $\phi:X\times T\to\Bbb R$ be a function such that (i) for each
$x\in X$, $t\mapsto\phi(x,t)$ is continuous (ii) for each $t\in T$,
$x\mapsto\phi(x,t)$ is $\Sigma$-measurable.   Then $\mu$ is inner
regular with respect to $\Cal K=\{K:K\subseteq X,\,\phi\restr K\times T$
is continuous$\}$.

(b) Let $\theta:X\to C(T)$ be a function such that
$x\mapsto\theta(x)(t)$ is $\Sigma$-measurable for every $t\in T$.   Give
$C(T)$ the topology $\frak T_c$ of uniform convergence on compact
subsets of $T$.   Then $\theta$ is almost continuous.

\proof{ The result is trivial if $T$ is empty, so we may suppose that
$T\ne\emptyset$.

\medskip

{\bf (a)} Take $E\in\Sigma$ and $\gamma<\mu E$;  take
$F\in\Sigma$ such that $F\subseteq E$ and $\gamma<\mu F<\infty$.   Let
$\Cal U$ be a countable base for the topology of $T$ consisting of
non-empty sets, $D$ a countable dense subset of $T$ and $\Cal V$ a
countable base for the topology of $\Bbb R$.   For $U\in\Cal U$,
$V\in\Cal V$ set

\Centerline{$E_{UV}=\{x:\phi(x,t)\in V$ for every $t\in U\cap D\}$;}

\noindent then $E_{UV}\in\Sigma$.   Let
$\langle\epsilon_{UV}\rangle_{U\in\Cal U,V\in\Cal V}$ be a family of
strictly positive numbers with sum at most $\mu F-\gamma$.
For each $U\in\Cal U$, $V\in\Cal V$ take a closed set
$F_{UV}\subseteq F\setminus E_{UV}$ such that
$\mu F_{UV}\ge\mu(F\setminus E_{UV})-\epsilon_{UV}$.   Consider

\Centerline{$K
=\bigcap_{U\in\Cal U,V\in\Cal V}F_{UV}\cup(F\cap E_{UV})$.}

\noindent Then $K\subseteq E$ and $\mu K\ge\gamma$.

If $x\in K$, $t\in T$ and $\phi(x,t)\in V_0\in\Cal V$, let $V\in\Cal V$
be such that $\phi(x,t)\in V$ and $\overline{V}\subseteq V_0$.   Then
$\{t':\phi(x,t')\in V\}$ is an open set containing $t$, so there is some
$U\in\Cal U$ such that $t\in U$ and $\phi(x,t')\in V$ for every
$t'\in U$.   This
means that $x\in E_{UV}$, so that $(K\setminus F_{UV})\times U$ contains
$(x,t)$, and is a relatively open set in $K\times T$.   If
$(x',t')\in(K\setminus F_{UV})\times U$, then $x'\in E_{UV}$, so
$\phi(x',t'')\in V$ whenever $t''\in U\cap D$;  as $D$ is dense,
$\phi(x',t'')\in\overline{V}$ whenever $t''\in U$;  in particular,
$\phi(x',t')\in\overline{V}\subseteq V_0$.   This shows that
$(K\times T)\cap\phi^{-1}[V_0]$ is relatively open in $K\times T$;  as
$V_0$ is arbitrary, $\phi\restr K\times T$ is continuous.

So $K\in\Cal K$.   As $E$ and $\gamma$ are arbitrary, $\mu$ is inner
regular with respect to $\Cal K$.

\medskip

{\bf (b)} Set $\phi(x,t)=\theta(x)(t)$ for $x\in X$, $t\in T$.   Because
$\theta(x)\in C(T)$ for every $x$, $\phi$ is continuous in the second
variable;  and the hypothesis on $\theta$ is just that $\phi$ is
measurable in the first variable.   So $\mu$ is inner regular with
respect to $\Cal K$ as described in (a).   But $\theta\restr K$ is
continuous for every $K\in\Cal K$, by 4A2G(g-ii).   So $\theta$ is
almost continuous.
}%end of proof of 454Q

\leader{454R}{Proposition}\dvAformerly{4{}54O} Let $T$ be an analytic
metrizable
space\cmmnt{ (e.g., a Polish space, or any Souslin-F subset of a
Polish space)}, and $\mu$ a
probability measure on $C(T)$ with domain the $\sigma$-algebra $\Sigma$
generated by the evaluation functionals $f\mapsto f(t):C(T)\to\Bbb R$
for $t\in T$.   Give $C(T)$ the topology $\frak T_c$ of uniform
convergence on compact subsets of $T$.   Then the completion of $\mu$ is
a $\frak T_c$-Radon measure.

\proof{ If $T$ is empty this is trivial, so let us suppose henceforth
that $T\ne\emptyset$.

\medskip

{\bf (a)} Let $D$ be a countable dense subset of $T$.   Let
$\pi:C(T)\to\BbbR^D$ be the restriction map.
Set $X=\pi[C(T)]\subseteq\BbbR^D$;  then $X$, with the topology it
inherits from $\BbbR^D$, is a separable metrizable space.   Note that,
because $D$ is dense, $\pi$ is injective.

We need to know that $\pi$ is an isomorphism between $(C(T),\Sigma)$ and
$(X,\Cal B)$, where $\Cal B$ is the Borel $\sigma$-algebra of $X$.
\Prf\ Since
the Borel $\sigma$-algebra of $\BbbR^D$ is just the $\sigma$-algebra
generated
by the functionals $g\mapsto g(t):\BbbR^D\to\Bbb R$ as $t$ runs over
$D$ (4A3Dc/4A3E), $\Cal B$ is the $\sigma$-algebra of subsets of $X$
generated by the functionals $g\mapsto g(t):X\to\Bbb R$ for $t\in D$.
So $\pi$ is surely $(\Sigma,\Cal B)$-measurable.   On the other hand, if
$t\in X$, there is a sequence $\sequencen{t_n}$ in $D$ converging to
$t$, so that $\pi^{-1}(g)(t)=\lim_{n\to\infty}g(t_n)$ for every
$g\in X$, and $g\mapsto\pi^{-1}(g)(t):X\to\Bbb R$ is $\Cal B$-measurable.
Accordingly $\pi^{-1}$ is $(\Cal B,\Sigma)$-measurable.\ \Qed

\medskip

{\bf (b)} Let $\sequencen{U_n}$ be a sequence running over a base for
the topology of $T$, with no $U_n$ empty.   For each $n\in\Bbb N$,
$g\in\BbbR^D$ set

\Centerline{$\omega_n(g)=\sup_{t,u\in U_n\cap D}\min(1,g(t)-g(u))$,}

\noindent so that $\omega_n:\BbbR^D\to[0,1]$ is $\Tau$-measurable,
where $\Tau$ is the Borel (or Baire) algebra of $\BbbR^D$.   For
$g\in\BbbR^D$, $g\in X$ iff $g$ has an extension to a continuous
function on $T$, that is,

\Centerline{for every $t\in T$, $k\in\Bbb N$ there is an $n\in\Bbb N$
such that $t\in U_n$ and $\omega_n(g)\le 2^{-k}$.}

\noindent Turning this round, $\BbbR^D\setminus X$ is the projection
onto the first coordinate of the set

\Centerline{$Q=\bigcup_{k\in\Bbb N}\bigcap_{n\in\Bbb N}
  \{(g,t)$: either $t\notin
U_n$ or $\omega_n(g)>2^{-k}\}\subseteq\BbbR^D\times T$.}

\noindent But (because every $U_n$ is an open set and every $\omega_n$
is Borel measurable) $Q$ is a Borel set in the
analytic space $\BbbR^D\times T$.   So $Q$ is analytic (423Eb) and
$\BbbR^D\setminus X$ is analytic (423Bb).   Since $\BbbR^D$, being
Polish (4A2Qc), is a Radon space (434Kb), $X$ is a Radon space (434Fd).

\medskip

{\bf (c)} The image measure $\nu=\mu\pi^{-1}$ on $X$ is a Borel
probability measure.   Because $X$ is a Radon space, $\nu$ is tight, and
its completion $\hat\nu$
is a Radon measure.

By 454Qb, $\pi^{-1}:X\to C(T)$ is almost continuous if we give $C(T)$
the topology $\frak T_c$.   So the image measure
$\lambda=\hat\nu(\pi^{-1})^{-1}$ is a Radon measure for $\frak T_c$
(418I).   But of course $\lambda$ is the completion of $\mu$, just
because $\pi$ is a bijection and $\hat\nu$ is the completion of $\nu$.
}%end of proof of 454R

\leader{454S}{Corollary}\dvAformerly{4{}54P} Let
$T$ be an analytic metrizable space.

(a) $C(T)$, with either the topology $\frak T_p$ of uniform convergence
on finite subsets of $T$ or the topology $\frak T_c$ of uniform
convergence on compact subsets of $T$, is a measure-compact Radon space.

(b) Let $\mu$ be a Baire probability measure on $\BbbR^T$ such that
$\mu^*C(T)=1$.   Then the subspace measure $\hat\mu_C$ on $C(T)$ induced by
the completion of $\mu$ is a Radon measure on $C(T)$ if $C(T)$ is given
either $\frak T_p$ or $\frak T_c$.
$\mu$ itself is $\tau$-additive and has a unique
extension $\tilde\mu$ which is a Radon measure on $\BbbR^T$;
$\hat\mu_C$ is the subspace measure on $C(T)$ induced by
$\tilde\mu$.

\proof{{\bf (a)} Let $\mu$ be a probability measure on $C(T)$ which is
either a Baire measure or a Borel measure with respect to either
$\frak T_p$ or $\frak T_c$.    Let $\tilde\mu$ be the completion of
$\mu\restr\Sigma$, where $\Sigma$ is the $\sigma$-algebra generated by
the functionals $f\mapsto f(t)$;  because all these are
$\frak T_p$-continuous, $\Sigma$ is certainly included in the Baire
$\sigma$-algebra for $\frak T_p$, so that $\Sigma\subseteq\dom\mu$.
454R tells us that $\tilde\mu$ is
a Radon measure for $\frak T_c$.   Because $\frak T_p$ is a coarser
Hausdorff topology, $\tilde\mu$ is also a Radon measure for $\frak T_p$.
Also $\tilde\mu$ must extend $\mu$, because its domain includes that of
$\mu$ and the completion of $\mu$ must extend $\tilde\mu$ (in fact, of
course, this means that $\tilde\mu$ is actually the completion of
$\mu$).   Now $\tilde\mu$ is $\tau$-additive (for either topology), so
$\mu$ also is;  as $\mu$ is arbitrary, $C(T)$ is measure-compact (for
either topology).   On the other hand, if $\mu$ is a Borel measure (for
either topology), it must be tight for that topology;  so that $C(T)$ is
a Radon space.

\medskip

{\bf (b)} Write $\mu_C$ for the subspace measure on $C(T)$.   Recall
that the domain $\Sigma$ of $\mu$ is just the $\sigma$-algebra generated
by the functionals $f\mapsto f(t):\BbbR^T\to\Bbb R$, as $t$ runs over
$T$ (4A3Na), so that the domain $\Sigma_C$ of $\mu_C$ is the
$\sigma$-algebra of subsets of $C(T)$ generated by the functionals
$f\mapsto f(t):C(T)\to\Bbb R$.   By 454R, the completion of
$\mu_C$ is a Radon measure on $C(T)$ if we give $C(T)$ the topology
$\frak T_c$ of uniform convergence on compact subsets of $T$, and therefore
also for the weaker Hausdorff topology $\frak T_p$.   Because
the $\mu_C$-negligible sets for $\mu_C$ are just the intersections of
$C(T)$ with $\mu$-negligible sets (214Cb), the completion of $\mu_C$ is the
subspace measure $\hat\mu_C$ induced by the completion of $\mu$
(214Ib\formerly{2{}14Xb}).

The embedding $C(T)\embedsinto\BbbR^T$ is of course continuous for
$\frak T_c$ and the product topology on $\BbbR^T$, so we have a Radon
image measure $\tilde\mu$ on $\BbbR^T$ defined by saying that
$\tilde\mu E=\hat\mu_C(E\cap C(T))$ whenever $E\cap C(T)$ is measured by
$\hat\mu_C$.   If $E\in\Sigma$, then

\Centerline{$\tilde\mu E=\hat\mu_C(E\cap C(T))
=\mu_C(E\cap C(T))=\mu^*(E\cap C(T))=\mu E$}

\noindent because $\mu^*C(T)=1$, so $\tilde\mu$ extends $\mu$.   Of
course $\tilde\mu C(T)=1$ and the subspace measure on $C(T)$ induced by
$\tilde\mu$ is just $\hat\mu_C$.

Finally, because $\mu$ has an extension to a Radon measure, it must
itself be $\tau$-additive.   Because $\Sigma$ includes a base for
the topology of $\BbbR^T$, $\mu$ can have only one extension to a Radon
measure on $\BbbR^T$ (415H(iv)).
}%end of proof of 454S

\leader{454T}{}\dvAformerly{4{}54Q}\cmmnt{ A bit out of order,
I give an elementary remark
on completion regular measures on products of compact spaces.

\medskip

\noindent}{\bf Proposition} Let $\familyiI{X_i}$ be a family of compact
Hausdorff spaces, and $\mu$ a completion regular topological measure on
$X=\prod_{i\in I}X_i$.   Then the marginal measure $\mu_i$ of $\mu$ on
$X_i$ is completion regular for each $i\in I$.

\proof{ Let $\pi_i:X\to X_i$ be the coordinate map.   If
$E\subseteq X_i$ and $\gamma<\mu_iE$, there is a zero set
$Z\subseteq\pi_i^{-1}[E]$ such that $\mu Z\ge\gamma$.   Now
$\pi_i[Z]\subseteq E$ is a zero set in $X_i$ (4A2G(c-ii), using
4A2B(f-i)), and

\Centerline{$\mu_i\pi_i[Z]=\mu\pi_i^{-1}[\pi_i[Z]]\ge\mu Z\ge\gamma$.}

\noindent As $E$ and $\gamma$ are arbitrary, $\mu_i$ is inner regular with
respect to the zero sets, so is completion regular.
}%end of proof of 454T

\exercises{\leader{454X}{Basic exercises $\pmb{>}$(a)}
%\spheader 454Xa
Let $\mu$ be Lebesgue measure on $[0,1]$, and $\Sigma$ its domain.   Let
$X_0$, $X_1\subseteq[0,1]$ be disjoint sets of full outer measure.
For each $i$, let $\Sigma_i$ be the relative $\sigma$-algebra on
$X_i$.   Show that we have
a finitely additive functional $\lambda$ defined on
$\Sigma_0\otimes\Sigma_1$ by the formula

\Centerline{$\lambda((E\cap X_0)\times(F\cap X_1))=\mu(E\cap F)$ for all
$E$, $F\in\Sigma$,}

\noindent and that $\lambda$ has no extension to a measure on
$X_0\times X_1$.
%454F

\spheader 454Xb Adapt the example of 419K to provide a counter-example
for 454G if we omit the hypothesis that the marginal measures
$\mu_{\{i\}}$ must be perfect.
%454G

\spheader 454Xc Adapt the example of 419K/454Xb to provide a
counter-example for 454H if we omit the hypothesis that the
$(X_n,\Sigma_n)$ must be standard Borel spaces.   \Hint{if
$z\in\prod_{i\le n}X_i$, try $\nu_z(E)=1$ if $z(n)\in E$, $0$
otherwise.}
%454H

\sqheader 454Xd Let $X$ be a set and $\Sigma$ a $\sigma$-algebra of
subsets of $X$.   Let us say that $(X,\Sigma)$ has the {\bf perfect
measure property} if every totally finite measure with domain $\Sigma$
is perfect.   Show that (i) if $(X,\Sigma)$ has the perfect measure
property, so does $(E,\Sigma_E)$ for any $E\in\Sigma$, where $\Sigma_E$
is the subspace $\sigma$-algebra on $E$ (ii) if
$\familyiI{(X_i,\Sigma_i)}$ is a family of spaces with the perfect
measure property, then $(\prod_{i\in I}X_i,\Tensorhat_{i\in I}\Sigma_i)$
has the perfect measure property.
%454I

\spheader 454Xe Let $(X,\Sigma)$ be a space with the perfect measure
property, and $\Tau$ the smallest $\sigma$-algebra including $\Sigma$
and closed under Souslin's operation.   Show that $(X,\Tau)$ has the
perfect measure property.
%454I

\spheader 454Xf Let $X$ be a set and $\Sigma$ a $\sigma$-algebra of
subsets of $X$.   Let us say that $(X,\Sigma)$ has the {\bf countably
compact measure property} if every totally finite measure with domain
$\Sigma$ is countably compact.   Show that (i) if $(X,\Sigma)$ has the
countably compact measure property it has the perfect measure property
(ii) if $(X,\Sigma)$ has the countably compact measure property so does
$(E,\Sigma_E)$ for every $E\in\Sigma$, where $\Sigma_E$ is the subspace
$\sigma$-algebra on $E$
(iii) if $\familyiI{(X_i,\Sigma_i)}$ is a family of spaces with the
countably compact measure property, then
$(\prod_{i\in I}X_i,\Tensorhat_{i\in I}\Sigma_i)$
has the countably compact measure property.
%454I

\spheader 454Xg Suppose that $(X,\Sigma)$ has the countably compact
measure property.   (i) Let $\mu$ be a totally finite measure with
domain $\Sigma$, $(Y,\Tau,\nu)$ a measure space, and $f:X\to Y$ an \imp\
function.   Show that $\mu$ has a disintegration $\family{y}{Y}{\mu_y}$
over $\nu$ which is consistent with $f$.
(ii) Let $Y$ be any set, $\Tau$ a $\sigma$-algebra of subsets of $Y$,
and $\lambda$ a probability measure with domain $\Sigma\tensorhat\Tau$.
Show that there is a family $\family{y}{Y}{\mu_y}$ of probability
measures on $X$ such that
$\lambda W=\int\mu_yW^{-1}[\{y\}]\nu(dy)$ for every
$W\in\Sigma\tensorhat\Tau$, where $\nu$ is the marginal measure of
$\lambda$ on $Y$.   \Hint{452M.}
%454I

\spheader 454Xh(i) Let $X$ be any set, and $\Sigma$ the
countable-cocountable algebra on $X$.   Show that $(X,\Sigma)$ has the
countably compact measure property.  (ii) Show that any standard Borel
space has the countably compact measure property.
%454I

\spheader 454Xi Let $X$ be a Radon Hausdorff space, and
$\Sigma_{\text{um}}$ the
algebra of universally measurable sets in $X$ (434D).   Show that
$(X,\Sigma_{\text{um}})$ has the countably compact measure property.
%454I

\sqheader 454Xj Let $\familyiI{X_i}$ be an independent
family of real-valued random
variables.   Show that its distribution is
a quasi-Radon measure on $\BbbR^I$.   \Hint{415E.}
%454L

\spheader 454Xk Let $I$ be a set and $\nu$, $\nuprime$ two quasi-Radon
measures on $\BbbR^I$ such that
$\int e^{if(x)}\nu(dx)=\int e^{if(x)}\nuprime(dx)$ for every continuous
linear functional $f:\BbbR^I\to\Bbb R$.   Show that $\nu=\nuprime$.
%454P

\sqheader 454Xl Let $\Sigma$ be the $\sigma$-algebra of subsets of
$C(\coint{0,\infty})$ generated by the functionals $f\mapsto f(t)$ for
$t\ge 0$.   Give $C(\coint{0,\infty})$ the topology $\frak T_c$ of
uniform convergence on compact sets.   (i) Show that $\frak T_c$ is
Polish, and that $\Sigma\cap\frak T_c$ is a base for $\frak T_c$ which
generates $\Sigma$ as $\sigma$-algebra.   (ii) Use this to give a quick
proof of 454R in this case.
%454R

\spheader 454Xm Let $T$ be a Polish space, and $\frak T_c$ the topology
on $C(T)$ of uniform convergence on compact sets.   Show that if
$\frak T$ is any Hausdorff topology on $C(T)$, coarser than $\frak T_c$,
such that all the functionals $f\mapsto f(t)$, for $t\in T$, are Baire
measurable for $\frak T$, then $(C(T),\frak T)$ is a measure-compact
Radon space.
%454S

\spheader 454Xn Give an example of a metrizable space $\Omega$ with a
continuous injective function $X:\Omega\to[0,1]$ and two different
quasi-Radon probability measures $\mu$, $\nu$ on $\Omega$ giving the same
distribution to the random variable $X$.
%454N

\leader{454Y}{Further exercises (a)}
%\spheader 454Ya
In 454Ab, show that $\mu$ is weakly $\alpha$-favourable (definition:
451V) if every $\mu_i$ is.
%454A

\spheader 454Yb
Let $\Sigma$ be the algebra of Lebesgue measurable
subsets of $\Bbb R$.   Show that $(\BbbR,\Sigma)$ has the perfect
measure property (454Xd) iff $\frak c$ is measure-free.
%454I

\spheader 454Yc Let $\Cal B$ be the Borel $\sigma$-algebra of $\omega_1$
with its
order topology.   Show that $(\omega_1,\Cal B)$ has the perfect measure
property.   \Hint{439Yf.}
%454I

\spheader 454Yd Let $(X,\Sigma,\mu)$ be a semi-finite measure space with
a topology such that $\mu$ is inner regular with respect to the closed
sets, $T$ a second-countable space and $Y$ a separable metrizable space.
Suppose that $\phi:X\times T\to Y$ is continuous in the second variable
and measurable in the first, as in 454Q.   Show that $\mu$ is inner
regular with respect to $\Cal K=\{K:K\subseteq X,\,\phi\restr K\times T$
is continuous$\}$.
%454Q

}%end of exercises

\endnotes{
\Notesheader{454} 454A generalizes Theorem 451J, which gave the same
result (with essentially the same proof) for product measures.   One of
the themes of this section is the idea that we can deduce properties of
measures on product spaces from properties of their marginal measures,
that is, the image measures on the factors.   The essence of
`compactness', `countable compactness' and `perfectness' is that we can
find enough points in the measure space to do what we want.   (See, for
instance, the characterization of compactness in 343B, or Pachl's
characterization of countable compactness in 452Ye.)   Since the
canonical feature of a product space is that we put in every point the
Axiom of Choice provides us with, it's perhaps not surprising that such
properties can be inherited by measures on product spaces.

Theorems 454C and 454D can be regarded as further variations on the same
theme.   A finitely additive non-negative functional on an algebra of
sets will have an extension to a measure if, and only if, it is
sequentially smooth in the sense that the measures of a decreasing
sequence of sets with empty intersection converge to zero (413K).   If
we have a decreasing sequence of sets, with measures bounded away from
zero, but with empty intersection, one interpretation of the phenomenon
is that some points which ought to have been present got left out of the
sets.   What 454D tells us is that perfectness (and countable
additivity) of the marginal measures is enough to ensure that there are
enough points in the product to stop this happening.   In effect, 454C
tells us that it will be enough if every marginal but one is perfect.

These results are of course associated with the projective limit
constructions in 418M-418Q.  %418M 418O 418P 418Q
In the theorems there we had Radon
measures, so that they were actually compact rather than perfect;  in
return for the stronger hypothesis on the measures, we could handle
projective limits corresponding to rather small subsets of the product
spaces (see the formulae in 418O-418Q).  %418O 418P 418Q
Just as in \S418, the patterns
change when we have countable rather than uncountable families to deal
with (418P-418Q, 454H).

In 454J-454P, I insist rather arbitrarily that `the' joint
distribution of a family $\familyiI{X_i}$ of real-valued random
variables is the completion of a Baire measure on $\BbbR^I$.   Of course
all the ideas can also be expressed in terms of the Baire measure
itself, but I have sought a formulation which is consistent with the
rules set out in \S271.   When $I$ is countable, we get a Radon measure
(454J(iii)), as in the finite-dimensional case.   There are other cases
in which the distribution is a quasi-Radon measure (454Xj).   As
always, we can ask whether the distribution is $\tau$-additive;  in this
case it will have a canonical extension to a quasi-Radon measure (415N).
Important examples of this phenomenon are described in 455H and 456O.
Because $\BbbR^I$ has a linear topological space structure, we have a
notion of `characteristic function' for any probability measure on
$\BbbR^I$ measuring the zero sets, and the characteristic function of a
Baire measure determines that measure (454P, 454Xk).

In 454R, $C(T)$, with $\frak T_c$, has a
countable network (4A2Oe), so the subspace measure $\mu_C$
induced by $\mu$ on $C(T)$ must be a $\tau$-additive topological measure
with respect to $\frak T_c$ (414O) and has a unique extension to a
quasi-Radon
measure on $C(T)$ (415M).   The hard bit is the next step, showing that
$C(T)$, under $\frak T_c$, is a Radon space;  this is the real point of
454Q-454R.   For the most important case, in which $T=\coint{0,\infty}$,
we have a useful simplification, because $\frak T_c$ is actually Polish
(454Xl).   Even in this case, however, we need to observe that the
measure we are seeking is a little more complicated than a simple
completion of a measure on $\BbbR^T$.
We must complete the {\it subspace} measure on $C(T)$, and $C(T)$ is far
from being a measurable set.   The measure $\tilde\mu$ of 454S will not
as a rule be completion regular, for instance.   Spaces of continuous
functions are so important that it is worth noticing that the results
here will be valid for various topologies on $C(T)$ (454Xm).

}%end of notes

\discrpage

