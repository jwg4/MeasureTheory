\frfilename{mt5a4.tex}
\versiondate{15.8.13}
\copyrightdate{2007}

\def\chaptername{Appendix}
\def\sectionname{General topology}

\def\Engelking{{\smc Engelking 89}}

\newsection{5A4}

The principal new topological concepts required in this volume are
some of the standard cardinal functions of topology (5A4A-5A4B).
As usual, there are some particularly interesting phenomena involving
compact spaces (5A4C).   For special purposes in \S513, we need to
know some non-trivial facts about metrizable spaces (5A4D).
The rest of the section is made up of scraps which are either elementary or
standard.

\leader{5A4A}{Definitions} Let $(X,\frak T)$ be a topological space.

\spheader 5A4Aa The {\bf weight} of $X$, $w(X)$, is the least cardinal
of any base for $\frak T$.

\spheader 5A4Ab The {\bf $\pi$-weight} of $X$ is
$\pi(X)=\ci(\frak T\setminus\{\emptyset\})$, the smallest cardinal of any
$\pi$-base for $\frak T$.

\spheader 5A4Ac The {\bf density} $d(X)$ of
$X$ is the smallest cardinal of any dense subset of $X$.

\spheader 5A4Ad The {\bf cellularity} of $X$ is

\Centerline{$c(X)
\cmmnt{\mskip5mu=\cdownarrow(\frak T\setminus\{\emptyset\})}
=\sup\{\#(\Cal G):\Cal G\subseteq\frak T\setminus\{\emptyset\}$ is
disjoint$\}$.}

\noindent The {\bf saturation} of $X$ is

\Centerline{$\sat(X)
\cmmnt{\mskip5mu=\sat^{\downarrow}(\frak T\setminus\{\emptyset\})}
=\sup\{\#(\Cal G)^+:\Cal G\subseteq\frak T\setminus\{\emptyset\}$ is
disjoint$\}$\dvro{.}{,}}

\cmmnt{\noindent that is, the smallest cardinal $\kappa$ such that
there is no disjoint family of $\kappa$ non-empty open sets.}

\spheader 5A4Ae The {\bf tightness} of $X$, $t(X)$, is the smallest
cardinal $\kappa$ such that whenever $A\subseteq X$ and
$x\in\overline{A}$ there is a $B\in[A]^{\le\kappa}$ such that
$x\in\overline{B}$.   \cmmnt{(Recall that
$[A]^{\le\kappa}=\{B:B\subseteq A$, $\#(B)\le\kappa\}$.)}
%can drop I think

\spheader 5A4Af The {\bf Nov\'ak number}
$n(X)$ is the smallest cardinal of any family of nowhere dense subsets
of $X$ covering $X$;  or $\infty$ if there is no such family.

\spheader 5A4Ag(i) The {\bf Lindel\"of number} $L(X)$ is the least cardinal $\kappa$
such that every open cover of $X$ has a subcover with cardinal at most
$\kappa$.

\quad(ii) The {\bf hereditary Lindel\"of number} $\hL(X)$ is
$\sup_{Y\subseteq X}L(Y)$.
%5{}31A

\spheader 5A4Ah(i) If $x\in X$, the {\bf character of $x$ in
$X$}, $\chi(x,X)$, is the smallest cardinal of any
base of neighbourhoods of $x$.

\quad(ii) The {\bf character} of $X$ is $\chi(X)=\sup_{x\in X}\chi(x,X)$.

\spheader 5A4Ai The {\bf network weight} of $X$, $\nw(X)$,
is the smallest cardinal of any network for $\frak T$.

%weight:  w(X) Engelking p 12
%character \chi(x,X) \chi(X)  Engelking p 12
%density  d(X)  Engelking p 25
%tightness  t(X)  Engelking p 60
%cellularity=Souslin number  c(X)  Engelking p 59
%network weight  nw(X)  Engelking p 127
%L(X)  Engelking p 127  hL(X) Engelking p 222

\cmmnt{\medskip

\noindent{\bf Remark} Recall that $X$ is called `second-countable'
iff $w(X)\le\omega$, `separable' iff $d(X)\le\omega$, `ccc'
iff $c(X)\le\omega$ (that is, $\sat(X)\le\omega_1$), `Lindel\"of' if
$L(X)\le\omega$, `hereditarily Lindel\"of' if $\hL(X)\le\omega$,
`first-countable' if $\chi(X)\le\omega$
and `countably tight' iff $t(X)\le\omega$.
}%end of comment

\leader{5A4B}{Proposition} Let $(X,\frak T)$ be a topological space.

(a)

\Centerline{$c(X)\le d(X)
%5{}21Oa
%\le\min(\#(X),\pi(X))
\le\pi(X)\le w(X)$,}
%5{}27E
%\le 2^{\#(X)}$,}

%\Centerline{$t(X)\le\min(w(X),\#(X))$,}

%\Centerline{$\max(c(X),L(X))\le\hL(X)\le\min(w(X),\#(X))$,}

\Centerline{$\#(\frak T)\le 2^{\nw(X)}$,}
%5{}31A

\Centerline{$\chi(X)\le w(X)\le\max(\#(X),\chi(X))$.}
%5{}31Xk

\noindent $\sat(X)=c(X)^+$ unless $\sat(X)$ is weakly inaccessible,
in which case $\sat(X)=c(X)$.
%5{}14Da
%If $X$ is regular then $w(X)\le 2^{d(X)}$.

(b) If $Y$ is a subspace of $X$,
then $w(Y)\le w(X)$,
%5{}A4Bh
$\nw(Y)\le\nw(X)$ and
%5{}31A
$\chi(y,Y)\le\chi(y,X)$ for every $y\in Y$.
%5{}31P
%$\chi(Y)\le\chi(X)$ and $t(Y)\le t(X)$.
%If $Y$ is open, then $\sat(Y)\le\sat(X)$,
%$c(Y)\le c(X)$ and $\pi(Y)\le\pi(X)$.

(c) If a topological space $Y$ is a continuous
image of $X$, then $d(Y)\le d(X)$,
%5{}36Cf
$c(Y)\le c(X),$
%5{}A4Be
$L(Y)\le L(X)$
%5{}34K
and $\nw(Y)\le\nw(X)$.
%5{}21 notes
%$\sat(Y)\le\sat(X)$

(d) If $\Cal G$ is a family of open subsets
of $X$, then there is a subfamily $\Cal H\subseteq\Cal G$ such that
$\#(\Cal H)<\sat(X)$ and
$\overline{\bigcup\Cal H}=\overline{\bigcup\Cal G}$.
%5{}14Db 5{}16Qb

(e) Let $\familyiI{X_i}$ be a family of non-empty topological spaces
with product $X$, and $\lambda$ a cardinal such that
$\#(I)\le 2^{\lambda}$.   Then

%\Centerline{$w(X)\le\max(\omega,\#(I),\sup_{i\in I}w(X_i))$,}

\Centerline{$d(X)\le\max(\omega,\lambda,\sup_{i\in I}d(X_i))$}
%5{}15H %5{}21N %5{}31Xd  cf. 4A2B(e-ii)

%\Centerline{$\pi(X)\le\max(\omega,\#(I),\sup_{i\in I}\pi(X_i))$}

\noindent and

\Centerline{$c(X)
=\sup_{J\subseteq I\text{ is finite}}c(\prod_{i\in J}X_i)$.}
%5{}14Ud
%\le\max(\omega,\sup_{i\in I}\min(d(X_i),2^{c(X_i)}))$.}

(f) If $\Cal G$ is any family of open subsets of $X$, there is an
$\Cal H\subseteq\Cal G$ such that $\#(\Cal H)\le\hL(X)$ and
$\bigcup\Cal H=\bigcup\Cal G$.
%5{}31A

(g) If $X$ is Hausdorff then $\#(X)\le 2^{\max(c(X),\chi(X))}$.
%5{}31Xk

(h) Suppose that $X$ is metrizable.

\quad(i) $d(X)=w(X)$.

\quad(ii) $d(Y)\le d(X)$ for every $Y\subseteq X$.
%5{}29B    %5{}31 5{}28   %5{}34 %5{}28
So any discrete subset of $X$ has cardinal at most $d(X)$.
%5{}31M

%\quad(ii) $c(X)=d(X)$.

\quad(iii) Let $\rho$ be a metric on $X$ defining its topology.   Then $X$
is separable iff there is no uncountable $A\subseteq X$ such that
$\inf_{x,y\in A,x\ne y}\rho(x,y)>0$.
%5{}34D

\proof{{\bf (a)} Let $D\subseteq X$
be a dense set with cardinal $d(X)$.   If
$\Cal G\subseteq\frak T\setminus\{\emptyset\}$ is disjoint, we have a
surjection from $D\cap\bigcup\Cal G$ to $\Cal G$, so
$\#(\Cal G)\le d(X)$;  as $\Cal G$ is arbitrary, $c(X)\le d(X)$.

Let $\Cal U$ be a $\pi$-base for $\frak T$ with cardinal $\pi(X)$.
Then there is a set $D\subseteq X$,
with cardinal at most $\#(\Cal U)$, meeting every non-empty
member of $\Cal U$;  now
$D$ is dense, so $d(X)\le\#(D)\le\pi(X)$.

Any base for $\frak T$ is a $\pi$-base for $\frak T$, so
$\pi(X)\le w(X)$.

Let $\Cal E$ be a network for $\frak T$ with cardinal $\nw(X)$;  then
$\frak T\subseteq\{\bigcup\Cal E':\Cal E'\subseteq\Cal E\}$, so
$\#(\frak T)\le 2^{\#(\Cal E)}=2^{\nw(X)}$.

If $\Cal U$ is a base for $\frak T$ with cardinal $w(X)$, and $x\in X$, then
$\Cal U_x=\{U:x\in U\in\Cal U\}$ is a base of neighbourhoods of $x$, so
$\chi(x,X)\le\#(\Cal U_x)\le w(X)$;  as $x$ is arbitrary,
$\chi(X)\le w(X)$.

If $X$ is finite, every point $x$ of $X$ has a smallest neighbourhood
$V_x$, and $\{V_x:x\in X\}$ is a base for $\frak T$, so $w(X)\le\#(X)$.
If $X$ is infinite, then for each $x\in X$ choose a base $\Cal U_x$ of
neighbourhoods of $x$ with $\#(\Cal U_x)=\chi(x,X)\le\chi(X)$.   Set
$\Cal U=\{\interior U:U\in\bigcup_{x\in X}\Cal U_x\}$;
then $\Cal U$ is a base for $\frak T$ so

\Centerline{$w(X)\le\#(\Cal U)\le\max(\omega,\#(X),\chi(X))
=\max(\#(X),\chi(X))$.}

Taking $P$ to be the partially ordered set
$(\frak T\setminus\{\emptyset\},\subseteq)$, $c(X)=c^{\downarrow}(P)$ and
$\sat(X)=\sat^{\downarrow}(P)$, so 513B, inverted, tells us that
$\sat(X)=c(X)^+$ unless $\sat(X)$ is weakly inaccessible,
in which case $\sat(X)=c(X)$.

\medskip

{\bf (b)} If $\Cal U$ is a base for $\frak T$, then
$\{U\cap Y:U\in\Cal U\}$ is a base for the topology of $Y$,
so $w(Y)\le w(X)$.
If $\Cal E$ is a network for $\frak T$, then
$\{E\cap Y:E\in\Cal E\}$ is a network for the topology of $Y$,
so $\nw(Y)\le\nw(X)$.   If $y\in Y$ and $\Cal V$ is a base of
neighbourhoods of $y$ in $X$, then $\{V\cap Y:V\in\Cal V\}$ is a base of
neighbourhoods of $y$ in $Y$, so $\chi(y,Y)\le\chi(y,X)$.

\medskip

{\bf (c)} Let $f:X\to Y$ be a continuous surjection.   If $D\subseteq X$ is
dense, then $f[D]$ is
dense in $Y$ (3A3Eb), and $d(Y)\le\#(f[D])\le\#(D)$;  as $D$ is arbitrary,
$d(Y)\le d(X)$.

If $\Cal H$ is a disjoint family of non-empty open sets in $Y$, then
$\Cal G=\{f^{-1}[H]:H\in\Cal H\}$ is a disjoint family of non-empty open
sets in $X$, so $\#(\Cal H)=\#(\Cal G)\le c(X)$;  as $\Cal H$ is arbitrary,
$c(Y)\le c(X)$.

If $\Cal H$ is an open cover of $Y$, then
$\Cal G=\{f^{-1}[H]:H\in\Cal H\}$ is an open cover of $X$;  let
$\Cal G_0\in[\Cal G]^{\le L(X)}$ be a subcover;  choose
$\Cal H_0\subseteq\Cal H$ such that $\#(\Cal H_0)=\Cal G_0$ and
$\Cal G_0=\{f^{-1}[H]:H\in\Cal H_0\}$;  then $\Cal H_0$ covers $Y$.
As $\Cal H$ is arbitrary, $L(Y)\le L(X)$.

If $\Cal A$ is a network for $\frak T$, then $\{f[A]:A\in\Cal A\}$ is a
network for the topology of $Y$, so $\nw(Y)\le\nw(X)$.

\medskip

{\bf (d)} Let $\Cal V$ be a maximal disjoint family of non-empty open sets
included in members of $\Cal G$.   Then $\#(\Cal V)<\sat(X)$.   Let
$\Cal H\subseteq\Cal G$ be such that $\#(\Cal H)\le\#(\Cal V)$ and every
member of $\Cal V$ is included in a member of $\Cal H$.   If $G\in\Cal G$
then $G\setminus\overline{\bigcup\Cal H}$ meets no member of $\Cal V$, so
must be empty;  so this $\Cal H$ serves.

\medskip

{\bf (e)(i)} By \Engelking, 2.3.15,
$d(X)\le\max(\omega,\lambda,\sup_{i\in I}d(X_i))$.

\medskip

\quad{\bf (ii)} Set
$\kappa=\sup_{J\subseteq I\text{ is finite}}c(\prod_{i\in J}X_i)$.
All the
finite products $\prod_{i\in J}X_i$ are continuous images of $X$, so
$c(X)\ge\kappa$, by (c).   \Quer\ Suppose, if possible, that
$c(X)>\kappa$.   Let $\Cal V$ be the usual base for the topology of $X$,
consisting of sets of the form $\prod_{i\in I}G_i$ where $G_i\subseteq X_i$
is open for every $i$ and $\{i:G_i\ne X_i\}$ is finite.
Let $\ofamily{\xi}{\kappa^+}{W_{\xi}}$ be a disjoint family of non-empty
open sets in $X$.   For each $\xi<\kappa$ let
$W'_{\xi}\subseteq W_{\xi}$ be a non-empty member of $\Cal V$, so that
$W'_{\xi}$ is determined by a coordinates in a finite subset $I_{\xi}$
of $I$.   By the $\Delta$-system Lemma (4A1Db) there is a set
$A\subseteq\kappa^+$, with cardinal $\kappa^+$, such that
$\family{\xi}{A}{I_{\xi}}$ is a $\Delta$-system with root $J$ say.
For $\xi\in A$ express $W'_{\xi}$ as $U_{\xi}\cap V_{\xi}$ where
$U_{\xi}$ is determined by coordinates in $J$ and $V_{\xi}$ is
determined by coordinates in $I_{\xi}\setminus J$.   Now for distinct
$\xi$, $\eta\in A$,

\Centerline{$\emptyset=W'_{\xi}\cap W'_{\eta}
=U_{\xi}\cap U_{\eta}\cap V_{\xi}\cap V_{\eta}$.}

\noindent Since $V_{\xi}$ and $V_{\eta}$ and $U_{\xi}\cap U_{\eta}$ are
determined by coordinates in the disjoint sets $I_{\xi}\setminus J$,
$I_{\eta}\setminus J$ and $J$ respectively, one of them must be empty,
and this can only be $U_{\xi}\cap U_{\eta}$.   Thus
$\family{\xi}{A}{U_{\xi}}$ is disjoint.   But now observe that each
$U_{\xi}$ is of the form $\pi_J^{-1}[H_{\xi}]$ where
$H_{\xi}\subseteq\prod_{i\in J}X_i$ is a non-empty open set and
$\pi_J(x)=x\restr J$ for every $x\in X$.   So $\family{\xi}{A}{H_{\xi}}$
witnesses that $c(\prod_{i\in J}X_i)\ge\kappa^+$, which contradicts the
definition of $\kappa$.\ \Bang

Thus $c(X)=\sup_{J\subseteq I\text{ is finite}}c(\prod_{i\in J}X_i)$.

\medskip

{\bf (f)} This is just because $L(\bigcup\Cal G)\le\hL(X)$.

\medskip

{\bf (g)} If $X$ is finite, $c(X)=\#(X)$ and the result is trivial.
Otherwise, set $\kappa=\max(c(X),\chi(X))$ and for each $x\in X$ let
$\ofamily{\xi}{\kappa}{U_{\xi}(x)}$ run over a base of neighbourhoods of
$x$ consisting of open sets.   Let $f:[X]^2\to[\kappa]^2$ be such that
whenever $x$, $y\in X$ are distinct then there are $\xi$,
$\eta\in f(\{x,y\})$ such that $U_{\xi}(x)$ and $U_{\eta}(y)$ are disjoint.
\Quer\ If $\#(X)>2^{\kappa}$ then by the Erd\H{o}s-Rado theorem
(5A1Ga) there is a $C\subseteq X$ such that $\#(C)>\kappa$ and $f$ is
constant on $[C]^2$;  let $\{\xi,\eta\}$ be the constant value.
For $x\in C$ set $G_x=U_{\xi}(x)\cap U_{\eta}(x)$;  then
$\family{x}{C}{G_x}$ is a disjoint family of non-empty open sets, so
$c(X)\ge\#(C)>\kappa$.\ \Bang

\medskip

{\bf (h)} Fix a metric $\rho$ on $X$ defining its topology.

\medskip

\quad{\bf (i)} If $d(X)<\omega$ then $X$ is finite and the result is
trivial.
Otherwise, let $D$ be a dense subset of $X$ with cardinal $d(X)$;
setting $U(x,\epsilon)=\{y:\rho(y,x)<\epsilon\}$,
$\{U(x,2^{-n}):x\in D$, $n\in\Bbb N\}$ is a base for $\frak T$, so
$w(X)\le\max(\#(D),\omega)=d(X)$.   Since we know from (a) that
$d(X)\le w(X)$, we have equality.

\medskip

\quad{\bf (ii)} Put (i) together with (b) above to see that $d(Y)\le d(X)$.
If $Y$ is discrete, then $\#(Y)=d(Y)\le d(X)$.

\medskip

\quad{\bf (iii)}($\alpha$) If there is an uncountable $A\subseteq X$ such
that $\inf_{x,y\in A,x\ne y}\rho(x,y)>0$, then $A$ is not separable in its
subspace topology, so $X$ is not separable, by (ii).
($\beta$) If there is no such $A$, then
for each $n\in\Bbb N$ let $A_n$ be a maximal subset of
$X$ such that $\rho(x,y)\ge 2^{-n}$ for all distinct $x$, $y\in A_n$.
In this case $\bigcup_{n\in\Bbb N}A_n$ is dense, so
$d(X)\le\max(\omega,\sup_{n\in\Bbb N}\#(A_n))=\omega$ and $X$ is separable.
}%end of proof of 5A4B

\leader{5A4C}{Compactness} Let $X$ be a compact Hausdorff space.

\spheader 5A4Ca{\bf (i)} $\nw(X)=w(X)$.
%5{}31A
\prooflet{(\Engelking, 3.1.19.)}

\medskip

\quad{\bf (ii)} There is a set
$Y\subseteq X$, with cardinal at most the cardinal power
$d(X)^{\omega}$, which meets every
non-empty G$_{\delta}$ subset of $X$.
%5{}21N
\prooflet{\Prf\ Let $D\subseteq X$ be a dense set with cardinal $d(X)$.
For each sequence $\omega\in D^{\Bbb N}$ choose a cluster point
$x_{\omega}$ of $\sequencen{\omega(n)}$;  set
$Y=\{x_{\omega}:\omega\in D^{\Bbb N}\}$.   Then
$\#(Y)\le\#(D^{\Bbb N})=d(X)^{\omega}$.   If $\sequencen{G_n}$ is a
sequence of open sets in $X$ with non-empty intersection, take
$x\in\bigcap_{n\in\Bbb N}G_n$ and choose inductively a sequence
$\sequencen{H_n}$ of open sets such that $x\in H_n$ and
$\overline{H}_{n+1}\subseteq H_n\cap G_n$ for every $n$.   Let
$\omega\in D^{\Bbb N}$ be such that $\omega(n)\in H_n$ for every $n$;
then

\Centerline{$x_{\omega}\in Y\cap\bigcap_{n\in\Bbb N}\overline{H}_n
\subseteq\bigcap_{n\in\Bbb N}G_n$.}

\noindent As $\sequencen{G_n}$ is arbitrary, $Y$ is a suitable set.\ \Qed}
%works for countably compact X, and we could have Y countably compact too

\spheader 5A4Cb
If $X$ is perfectly normal it is first-countable.
%5{}31Q
\prooflet{(Every singleton set in $X$ is a zero set (4A2Fi), so is
a G$_{\delta}$ set;  by 4A2Kf, $X$ is first-countable.)}

\spheader 5A4Cc
If $w(X)\le\kappa$, $X$ is
homeomorphic to a closed subspace of $[0,1]^{\kappa}$.
%5{}33D %5{}36C
\prooflet{(\Engelking, 3.2.5.)}

\spheader 5A4Cd{\bf (i)} If $Y$ is a Hausdorff space and $f:X\to Y$ is a
continuous irreducible surjection, then $d(X)=d(Y)$.
%5{}31Q
\prooflet{\Prf\ We know that $d(Y)\le d(X)$ (5A4Bc).   In the other
direction, let $D\subseteq Y$ be a dense set with cardinal $d(Y)$, and
$C\subseteq X$ a set with cardinal $\#(D)$ such that $f[C]=D$.   If
$G\subseteq X$ is open and not empty, $f[X\setminus G]$ is a closed proper
subset of $Y$ (because $f$ is irreducible),
so $D\not\subseteq f[X\setminus G]$ and
$C\not\subseteq X\setminus G$.   As $G$ is arbitrary, $C$ is dense, and
witnesses that $d(X)\le d(Y)$.\ \Qed}
%$\pi(X)=\pi(Y)$

\medskip

\quad{\bf (ii)} If $f:X\to\{0,1\}^{\kappa}$ is
a continuous irreducible surjection, where $\kappa\ge\omega$, then
$\chi(x,X)\ge\kappa$ for every $x\in X$.
\prooflet{\Prf\ Let $\Cal V$ be a base of neighbourhoods of $x$ with cardinal
$\chi(x,X)$.   For each $\xi<\kappa$, set
$G_{\xi}=\{y:y\in X$, $f(y)(\xi)=f(x)(\xi)\}$.   For $V\in\Cal V$, set
$I_V=\{\xi:\xi<\kappa$, $V\subseteq G_{\xi}\}$;  then
$f[V]\subseteq\{z:z\in\{0,1\}^{\kappa}$, $z\restr I_V=f(x)\restr I_V\}$;
but $f[X\setminus V]$ is a closed proper subset of $\{0,1\}^{\kappa}$, so
$\interior f[V]$ is non-empty and $I_V$ is finite.   As $\Cal V$ is a base
of neighbourhoods of $x$, $\kappa=\bigcup_{V\in\Cal V}I_V$.
As $\kappa$ is infinite, $\Cal V$ is infinite, and
$\kappa\le\#(\Cal V)=\chi(x,X))$.\ \Qed}

\medskip

\quad{\bf (iii)}
So if there is a continuous
surjection from $X$ onto $[0,1]^{\kappa}$, there is a non-empty closed
$K\subseteq X$ such that $\chi(x,K)\ge\kappa$ for every $x\in K$.
%5{}31M %5{}31N %5{}31P
\prooflet{\Prf\ Let $f:X\to[0,1]^{\kappa}$ be a continuous surjection.
Set $Z=f^{-1}[\{0,1\}]^{\kappa}$.
By 4A2G(i-i), there is a closed $K\subseteq Z$ such that
$f\restr K$ is an irreducible surjection onto $\{0,1\}^{\kappa}$, and we
can use (ii).\ \Qed}

\medskip

\quad{\bf (iv)} If $Y$ and $Z$ are Hausdorff spaces and $f:X\to Y$,
$g:Y\to Z$ are continuous irreducible surjections then 
$gf:X\to Z$ is irreducible.
%5{}31Q
\prooflet{(If $F\subseteq X$ is a closed proper subset, then $f[F]$ is
a closed proper subset of $Y$ and $g[f[F]]\ne Z$.)}

\spheader 5A4Ce If $\sequencen{x_n}$
is a sequence in $X$ with at most one cluster point in
$X$, then $\sequencen{x_n}$ is convergent.
%5{}36C
\prooflet{\Prf\ Because $X$ is compact, $\sequencen{x_n}$ has at least one
cluster point;  let $x$ be such a point.   \Quer\ If $\sequencen{x_n}$ does
not converge to $x$, let $G$ be an open set containing $x$ such that
$J=\{n:n\in\Bbb N$, $x_n\notin G\}$ is infinite.   Then there must be a
point $y$ in $\bigcap_{n\in\Bbb N}\overline{\{x_i:i\in J\setminus n\}}$;
and now $y$ is a cluster point of $\sequencen{x_n}$ in $X\setminus G$, so
cannot be equal to $x$.\ \Bang\Qed}

\spheader 5A4Cf Let $Y$ be a Hausdorff space and
$f:X\to Y$ a continuous function.   If $\Cal E$ is a non-empty
downwards-directed family of closed subsets of $X$, then
$f[\bigcap\Cal E]=\bigcap_{F\in\Cal E}f[F]$.
%5{}36C
\prooflet{\Prf\ Of course
$f[\bigcap\Cal E]\subseteq\bigcap_{F\in\Cal E}f[F]$.
If $y\in\bigcap_{F\in\Cal E}f[F]$, then
$\{F\cap f^{-1}[\{y\}]:F\in\Cal E\}$ is a downwards-directed family of
closed subsets of $X$, so has non-empty intersection;  and any point of the
intersection witnesses that $y\in f[\bigcap\Cal E]$.\ \Qed}

\leader{5A4D}{Vietoris topologies:  Proposition}
Let $X$ be a separable metrizable space and $\Cal K$
the set of compact subsets of $X$ with the topology induced by the Vietoris
topology on the set of closed subsets of $X$\cmmnt{ (4A2T)}.
%Note that this is the topology defined by a Hausdorff metric.

(a) $\Cal K$ is second-countable.
%actually it is separable and metrizable.

(b) If $Y$ is a topological space and $R\subseteq Y\times X$ is
usco-compact, then $y\mapsto R[\{y\}]:Y\to\Cal K$ is Borel measurable.
%5{}13N

(c) There is a sequence $\sequencen{f_n}$ of Borel measurable
functions from $\Cal K\setminus\{\emptyset\}$ to $X$ such that
$\{f_n(K):n\in\Bbb N\}$ is a dense subset of $K$ for every
$K\in\Cal K\setminus\{\emptyset\}$.
%5{}13L
%maybe bits of (b) and (c) work without "separable"

\proof{{\bf (a)} Let $\Cal U$ be a
countable base for the topology of $X$.
Let $\Cal V$ be the family of sets of the form

\Centerline{$\{K:K\in\Cal K$, $K\cap U_i\ne\emptyset$ for $i<n$,
$K\subseteq\bigcup_{i<n}U_i\}$}

\noindent where $n\in\Bbb N$ and $U_i\in\Cal U$ for $i<n$;  then
$\Cal V$ is a countable family of
open sets in $\Cal K$ and is a base for the topology of $\Cal K$.

\medskip

{\bf (b)} If $G\subseteq X$ is open, then

\Centerline{$\{y:y\in Y$, $R[\{y\}]\subseteq G\}
=Y\setminus R^{-1}[X\setminus G]$}

\noindent is open.   Also $G$ can be expressed as
$\bigcup_{n\in\Bbb N}F_n$ where every $F_n$ is closed, so

\Centerline{$\{y:R[\{y\}]\cap G\ne\emptyset\}
=\bigcup_{n\in\Bbb N}R^{-1}[F_n]$}

\noindent is F$_{\sigma}$, therefore Borel.   Thus

\Centerline{$\Cal W=\{W:W\subseteq\Cal K$, $\{y:R[\{y\}]\in W\}$ is
Borel$\}$}

\noindent includes a
subbase for the topology of $\Cal K$.   It therefore includes a base;
because $\Cal K$ is second-countable, therefore hereditarily Lindel\"of,
every open set is a countable union of members of
$\Cal W$ and belongs to $\Cal W$, that is,
$y\mapsto R[\{y\}]$ is Borel measurable.

\medskip

{\bf (c)(i)} Note first that if $G\subseteq X$ is open, then
$K\mapsto\overline{K\cap G}:\Cal K\to\Cal K$ is Borel measurable.   \Prf\
If $H\subseteq X$ is open, then

\Centerline{$\{K:\overline{K\cap G}\cap H\ne\emptyset\}
=\{K:K\cap(G\cap H)\ne\emptyset\}$}

\noindent is open.   Next, we can express $H$ as the union
$\bigcup_{n\in\Bbb N}H_n$ of a non-decreasing
sequence of open sets such that
$\overline{H}_n\subseteq H$ for every $n$, so

\Centerline{$\{K:\overline{K\cap G}\subseteq H\}
=\bigcup_{n\in\Bbb N}\{K:K\cap G\subseteq\overline{H}_n\}
=\bigcup_{n\in\Bbb N}\{K:K\cap(G\setminus\overline{H}_n)=\emptyset\}$}

\noindent is F$_{\sigma}$, therefore Borel.   As in (b), this is enough.\
\Qed

\medskip

\quad{\bf (ii)} Let $\sequencen{U_n}$ run over a base for the topology of
$X$.   For each $n\in\Bbb N$ define $g_n:\Cal K\to\Cal K$ by setting

$$\eqalign{g_n(K)
&=\overline{K\cap U_n}\text{ if }K\cap U_n\ne\emptyset,\cr
&=K\text{ otherwise}.\cr}$$

\noindent Since $\{K:K\cap U_n\ne\emptyset\}$ is open,
(i) tells us that $g_n$ is Borel measurable.   Set $h_n=g_n\ldots g_1g_0$;
then $h_n$ also is Borel measurable, for each $n$.   Now, for each
$K\in\Cal K\setminus\{\emptyset\}$, $\sequencen{h_n(K)}$ is a
non-increasing sequence of non-empty compact sets, so has non-empty
intersection.   Morover, for each $n$, $h_n(K)$ is either disjoint from
$U_n$ or included in $\overline{U}_n$;  so
$\bigcap_{n\in\Bbb N}h_n(K)$ has exactly one point;  call this point
$f(K)$.   Of course $f(K)\in h_0(K)\subseteq K$.

Now $f:\Cal K\setminus\{\emptyset\}\to X$ is Borel measurable.   \Prf\
If $F\subseteq X$ is closed, then

\Centerline{$f^{-1}[F]
=\bigcap_{n\in\Bbb N}\{K:F\cap h_n(K)\ne\emptyset\}$}

\noindent is a Borel set because every $h_n$ is Borel measurable and
$\{K:F\cap K=\emptyset\}$ is open.\ \Qed

\medskip

\quad{\bf (iii)} Set $f_n=fg_n$ for each $n$.   Then $f_n(K)\in K$ for
every $n\in\Bbb N$ and $K\in\Cal K\setminus\{\emptyset\}$,
$f_n:\Cal K\setminus\{\emptyset\}\to X$ is Borel measurable for each $n$,
and $f_n(K)\in\overline{K\cap U_n}$ whenever $K\cap U_n\ne\emptyset$;
so $\{f_n(K):n\in\Bbb N\}$ is dense in $K$ for every
$K\in\Cal K\setminus\{\emptyset\}$.
}%end of proof of 5A4D

\leader{5A4E}{Category and the Baire property}
Let $X$ be a topological space;  write $\widehat{\Cal B}(X)$ for its
Baire-property algebra\cmmnt{ (4A3Q)}.

\spheader 5A4Ea Suppose that
$\familyiI{G_i}$ is a disjoint family of open sets and
$\familyiI{E_i}$ is a family of nowhere dense sets.   Then
$\bigcup_{i\in I}G_i\cap E_i$ is nowhere dense.
%5{}17M
\prooflet{(Elementary;  see (a-i) of the proof of 4A3R.)}

\spheader 5A4Eb If $A\subseteq X$ and
$H=\bigcup\{G:G\subseteq X$ is open, $G\cap A$ is meager$\}$, then
$H\cap A$ is meager.
\prooflet{(Again see the proof of 4A3Ra.)}

\spheader 5A4Ec Let $Y$ be another topological space.

\medskip

\quad{\bf (i)}
If $A\subseteq X$ is nowhere dense in $X$, then $A\times Y$ is
nowhere dense in $X\times Y$.
\prooflet{($\overline{A\times Y}=\overline{A}\times Y$.)}
So if $A\subseteq X$ is meager in $X$, then $A\times Y$ is meager in
$X\times Y$.

\medskip

\quad{\bf (ii)} $\widehat{\Cal B}(X)\tensorhat\widehat{\Cal B}(Y)
\subseteq\widehat{\Cal B}(X\times Y)$.
%5{}54E
\prooflet{\Prf\ If $E\in\widehat{\Cal B}(X)$, let $G\subseteq X$ be such
that $E\symmdiff G$ is meager;  then

\Centerline{$E\times Y=(G\times Y)\symmdiff((E\symmdiff G)\times Y)
\in\widehat{\Cal B}(X\times Y)$.}

\noindent
Similarly, $X\times F\in\widehat{\Cal B}(X\times Y)$ for every
$F\in\widehat{\Cal B}(Y)$.   Because $\widehat{\Cal B}(X\times Y)$ is a
$\sigma$-algebra of sets, it includes
$\widehat{\Cal B}(X)\tensorhat\widehat{\Cal B}(Y)$.\ \Qed}

\medskip

\quad{\bf (iii)} If $Y$ is compact, Hausdorff
and not empty, then a set $A\subseteq X$ is
meager in $X$ iff $A\times Y$ is meager in $X\times Y$.   \prooflet{\Prf\
We saw in (i) that if $A$ is meager then $A\times Y$ is meager.   In the
other direction, if $A\times Y$ is meager in $X\times Y$,
let $\sequencen{W_n}$ be a sequence of dense open subsets of $X\times Y$
such that $\bigcap_{n\in\Bbb N}W_n$ is disjoint from $A\times Y$.   Choose
$\sequencen{V_n}$ inductively, as follows.   $V_0=X\times Y$.   Given that
$V_n$ is an open subset of $X\times Y$ such that $\pi_1[V_n]$ is dense in
$X$, where $\pi_1$ is the projection from $X\times Y$ onto $X$, then
$\pi_1[V_n\cap W_n]$ is dense in $X$.   Set

\Centerline{$\Cal V_n=\{G\times H:G\subseteq X$ is open, $H\subseteq Y$ is
open, $G\times\overline{H}\subseteq V_n\cap W_n\}$.}

\noindent Because $Y$ is regular, $\bigcup\Cal V_n$ is dense in
$V_n\cap W_n$ and $\pi_1[\bigcup\Cal V_n]$ is dense in $X$.
Let $\Cal V'_n\subseteq\Cal V_n$ be a maximal family such that
$\pi_1[V]\cap\pi_1[V']$ is empty whenever $V$, $V'\in\Cal V'_n$ are
disjoint;  because $G'\times H\in\Cal V_n$ whenever $G\times H\in\Cal V_n$
and $G'$ is an open subset of $G$, $\pi_1[\bigcup\Cal V'_n]$ is dense in
$X$.   Set $V_{n+1}=\bigcup\Cal V'_n$, and continue.

If
$x\in\bigcap_{n\in\Bbb N}\pi_1[V_n]$, $\sequencen{\overline{V_n[\{x\}]}}$
is a non-increasing sequence of non-empty closed subsets of $Y$, so there
is a
$y\in\bigcap_{n\in\Bbb N}\overline{V_n[\{x\}]}$, because $Y$ is compact.
For each $n$, there is a $V\in\Cal V_n$ such that $x\in\pi_1[V]$,
so $V_{n+1}[\{x\}]=V[\{x\}]$ and $(x,y)\in W_n$.   Thus
$x\in\pi_1[\bigcap_{n\in\Bbb N}W_n]$ and $x\notin A$.   As $x$ is
arbitrary, $A$ is disjoint from $\bigcap_{n\in\Bbb N}\pi_1[V_n]$ and is
meager.\ \Qed}

%works for weakly \alpha-favourable Y

\leaveitout{If $\add\Cal M(X)<\sat(Y)$, we have nwd $A_{\xi}\subseteq X$,
disjoint open $H_{\xi}\subseteq Y$, $\bigcup_{\xi}A_{\xi}\times H_{\xi}$
nwd.}

\spheader 5A4Ed Suppose that $X$ is completely regular and ccc.

\medskip

\quad{\bf (i)} 
Every nowhere dense subset of $X$ is included in a nowhere dense
zero set.   \prooflet{\Prf\ If $E\subseteq X$ is nowhere dense, let
$\Cal G$ be a maximal disjoint family of cozero sets included in
$X\setminus E$.   Because $X$ is ccc, $\Cal G$ is countable, and
$G=\bigcup\Cal G$ is a cozero set.   Because $X$ is completely regular,
$X\setminus(G\cup E)$ is nowhere dense and $X\setminus G$ is a nowhere
dense zero set including $E$.\ \Qed}

\medskip

\quad{\bf (ii)} 
Every meager subset of $X$ is included in a meager Baire set.
%5{}54E
\prooflet{(By (i), it is included in the union of a sequence of nowhere
dense zero sets.)}

\leader{5A4F}{Normal and paracompact spaces (a)}
For a normal space $X$ and an infinite
set $I$, the following are equiveridical: (i) there is a
continuous surjection from $X$ onto $[0,1]^I$;  (ii) there is a
continuous surjection from a closed subset of $X$ onto
$\{0,1\}^I$.
%5{}31L
\prooflet{\Prf\ (i)$\Rightarrow$(ii) is elementary, as $\{0,1\}^I$ is a
closed subset of $[0,1]^I$.   So suppose that (ii) is true.
The map $x\mapsto\sum_{n=0}^{\infty}2^{-n-1}$ is a
continuous surjection from $\{0,1\}^{\Bbb N}$ onto $[0,1]$;  there is
therefore a continuous surjection from $\{0,1\}^{I\times\Bbb N}$ onto
$[0,1]^I$;  but $I$ is infinite, so $\{0,1\}^I$ is homeomorphic to
$\{0,1\}^{I\times\Bbb N}$.   We therefore have a continuous surjection from
$\{0,1\}^I$ onto $[0,1]^I$.   Accordingly there is a continuous surjection
$f$ from a closed subset $F$ of $X$ onto $[0,1]^I$.
Set $f_i(x)=f(x)(i)$ for $x\in F$ and
$i\in I$;  by Tietze's theorem (4A2F(d-ix)),
there is a continuous $g_i:X\to[0,1]$ extending $f_i$;  now
$x\mapsto\familyiI{g_i(x)}:X\to[0,1]^I$ is a continuous
surjection, and (i) is true.\ \Qed}

\spheader 5A4Fb Suppose that $X$ is a paracompact Hausdorff space and
$\Cal G$ is an open cover of $X$.   Then there is a continuous
pseudometric
$\rho:X\times X\to\coint{0,\infty}$ such that whenever
$\emptyset\ne A\subseteq X$
and $\sup_{x,y\in A}\rho(x,y)\le 1$ there is a $G\in\Cal G$ such
that $A\subseteq G$.
\prooflet{\Prf\ There is an open cover $\Cal H$ of $X$ which is a
`star-refinement' of $\Cal G$, that is, for every $x\in X$ there is a
$G\in\Cal G$ including $\bigcup\{H:x\in H\in\Cal H\}$ (\Engelking, 5.1.12).
Next, there is a `locally finite resolution of the identity subordinate to
$\Cal H$', that is, a family $\familyiI{f_i}$ of continuous functions from
$X$ to $[0,1]$ such that $\familyiI{f_i^{-1}[\,\ocint{0,1}\,]}$ is a
locally finite refinement of $\Cal H$ and
$\sum_{i\in I}f_i(x)=1$ for every $x\in X$ (\Engelking, 5.1.9).
Set $\rho(x,y)=2\sum_{i\in I}|f_i(x)-f_i(y)|$.   Then $\rho$ is a
pseudometric on $X$, and is continuous
because $\familyiI{f_i^{-1}[\,\ocint{0,1}\,]}$ is locally finite.
If $A\subseteq X$ is a non-empty set such that $\rho(x,y)\le 1$ for all
$x$, $y\in A$, take any $x\in A$.  There is a $G\in\Cal G$ such that
$H\subseteq G$ whenever $H\in\Cal H$ and $x\in H$.   Set
$J=\{i:i\in I$, $f_i(x)>0\}$;  then $\sum_{i\in J}f_i(x)=1$.
If $y\in A$, then $\sum_{i\in J}|f_i(x)-f_i(y)|\le\bover12$, so
there is an $i\in J$ such that $f_i(y)>0$.   Set
$U=f_i^{-1}[\,\ocint{0,1}\}\,]$;  then $x$ and $y$ both belong to $U$.
Let $H\in\Cal H$ be such that $U\subseteq H$;  then $x\in H$ so
$H\subseteq G$, and $y\in H$ so $y\in G$.   Thus $A\subseteq G$, as
required.\ \Qed}
%5{}34Xj

\leader{5A4G}{Baire $\sigma$-algebras (a)} Let $X$ be a topological space.
Write $\CalBa_0(X)$ for the set of
cozero sets in $X$ and for ordinals $\alpha>0$ set

\Centerline{$\CalBa_{\alpha}(X)=\{\bigcup_{n\in\Bbb N}(X\setminus E_n):
\sequencen{E_n}$ is a sequence in
$\bigcup_{\beta<\alpha}\CalBa_{\beta}(X)\}$.}

\noindent Then the Baire $\sigma$-algebra $\CalBa(X)$ of $X$ is
$\bigcup_{\alpha<\omega_1}\CalBa_{\alpha}(X)$.
%5{}51F
\prooflet{\Prf\ Inducing on $\alpha$, we see that $\CalBa_{\alpha}(X)$
is included in the Baire $\sigma$-algebra of $X$ for every $\alpha$;
and $\bigcup_{\alpha<\omega_1}\CalBa_{\alpha}(X)$ is a $\sigma$-algebra of
sets containing every cozero set, so includes the Baire $\sigma$-algebra.\
\Qed}

\spheader 5A4Gb\dvAnew{2014}{\bf (i)} 
If $\familyiI{X_i}$ is a family of separable
metrizable spaces with product $X$, then
$\#(\CalBa(X))\le\max(\frak c,\#(I)^{\omega})$.   \prooflet{\Prf\
By 4A3Na, $\CalBa(X)=\Tensorhat_{i\in I}\Cal B(X_i)$, where $\Cal B(X_i)$
is the Borel $\sigma$-algebra of $X_i$ for each $i$.   By 4A3Fa,
$\#(\Cal B(X_i))\le\frak c$ for each $i$, so

\Centerline{$\Cal E=\{\{x:x\in X$, $x(i)\in E\}:
i\in E$, $E\in\Cal B(X_i)\}$}

\noindent has cardinal at most $\max(\frak c,\#(I))$ and the
$\sigma$-algebra $\CalBa(X)$ it generates has cardinal at most
$\max(\frak c,\#(I))^{\omega}=\max(\frak c,\#(I)^{\omega})$ (4A1O).\ \Qed}

\medskip

\quad{\bf (ii)} If $\kappa\ge 2$ is a cardinal, 
then the set\cmmnt{ $F$} of 
Baire measurable functions from $\{0,1\}^{\kappa}$ to $\{0,1\}^{\omega}$
has cardinal $\kappa^{\omega}$.   \prooflet{\Prf\ The map

\Centerline{$f\mapsto\sequence{i}{\{x:f(x)(i)=1\}}:
F\to\CalBa(\{0,1\}^{\kappa})^{\Bbb N}$}

\noindent is bijective, so 

\Centerline{$\#(F)=\#(\CalBa(\{0,1\}^{\kappa}))^{\omega}
\le(\max(\frak c,\kappa)^{\omega})^{\omega}=\kappa^{\omega}$.}

\noindent Of course $\#(\CalBa(\{0,1\})^{\kappa})\ge\kappa$ so 
$\#(F)=\kappa^{\omega}$.\ \Qed}


\leader{5A4H}{Proposition} If $X$ is a compact metrizable space and
$(Y,\rho)$ a complete separable metric space, then
$C(X;Y)$, with the topology of uniform convergence, is Polish.
\prooflet{(\Engelking, 4.3.13 and 4.2.18.)}

\leader{5A4I}{Old friends (a)} The weight of the Stone-\v{C}ech
compactification $\beta\Bbb N$ is $\frak c$.
%5{}36C 5{}31Xd
\prooflet{(\Engelking, 3.6.11.)}

\spheader 5A4Ib(i) For any infinite $I$, there is a continuous surjection
from $\{0,1\}^I$ onto $[0,1]^I$.
%5{}31L
\prooflet{(Immediate from 5A4Fa, or otherwise.)}

\quad(ii) There is a continuous surjection from $[0,1]$ onto
$[0,1]^{\Bbb N}$.
%5{}31L
\prooflet{(The Cantor set $C\subseteq[0,1]$ is homeomorphic to
$\{0,1\}^{\Bbb N}$ (4A2Uc), so again 5A4Fa gives the result.   Or see
416Yi.)}

\spheader 5A4Ic If $X$ is a non-empty zero-dimensional
compact metrizable space without isolated points, it is homeomorphic to
$\{0,1\}^{\Bbb N}$.
%5{}{}38H
\prooflet{\Prf\ Let $\frak B$ be the algebra of open-and-closed subsets
of $X$.   Because $X$ has no isolated points, $\frak B$ is atomless
(316Lb).   We know that $X$ is second-countable (4A2P(a-ii));  let $\Cal U$
be a countable base for its topology;  then every member of $\frak B$ is
open, so expressible as a union of members of $\Cal U$, and compact, so
expressible as the union of a finite subset of $\Cal U$.   Accordingly
$\frak B$ is countable;  and as $X\ne\emptyset$, $\frak B\ne\{0\}$.
By 316M, $\frak B$ is isomorphic to the algebra
of open-and-closed subsets of $\{0,1\}^{\Bbb N}$;  by 311J, $X$ and
$\{0,1\}^{\Bbb N}$ are homeomorphic.\ \Qed}

\spheader 5A4Id Let $X$ be a non-empty zero-dimensional Polish space in
which no non-empty open set is compact.   Then $X$ is homeomorphic to $\NN$
with its usual topology.  \prooflet{\Prf\ Let $\rho$ be a complete metric
on $X$ defining its topology.   (i) If $U\subseteq X$ is a non-empty open
set and $\epsilon>0$, there is a partition $\sequencen{U_n}$ of $U$ into
non-empty open-and-closed sets of diameter at most $\epsilon$.
To see this, note that as $U$ is not compact,
there is a sequence $\sequencen{x_n}$ in $U$ with no cluster point in $U$
(4A2Le).   Let $\Cal V$ be the family of subsets of $U$, of diameter at
most $\epsilon$, which are open-and-closed in $X$ and contain $x_i$ for at
most finitely many $i$.   Because $X$ is zero-dimensional, $\Cal V$ is a
base for the subspace topology of $U$.
Because $U$ is Lindel\"of (4A2P(a-iii)), there is
a sequence $\sequencen{V_n}$ in $\Cal V$ covering $U$;  set
$V'_n=V_n\setminus\bigcup_{i<n}V_i$ for each $n$, so that
$\sequencen{V'_n}$ is a disjoint sequence in $\Cal V$ covering $U$.
Because no $V'_n$ can contain infinitely many of the $x_i$,
$I=\{n:V'_n\ne\emptyset\}$ is infinite, and we can re-enumerate
$\family{n}{I}{V'_n}$ as $\sequencen{U_n}$ to get an appropriate sequence.
(ii) Now set $S=\bigcup_{n\in\Bbb N}\BbbN^n$ and define
$\family{\sigma}{S}{U_{\sigma}}$ inductively in such a way that
$U_{\emptyset}=X$ and

\inset{$\sequencen{U_{\sigma^{\smallfrown}\fraction{n}}}$
is a partition of $U_{\sigma}$
into non-empty open-and-closed sets of diameter at most $2^{-k}$ whenever
$k\in\Bbb N$ and $\sigma\in\BbbN^k$.}

\noindent For $\alpha\in\NN$, $\sequence{k}{U_{\alpha\restr k}}$ is a
non-increasing sequence of non-empty closed sets and
$\diam U_{\alpha\restr k}\le 2^{-k+1}$ for every $k\ge 1$, so there is
exactly one point in $\bigcap_{k\in\Bbb N}U_{\alpha\restr k}$;  let
$f(\alpha)$ be this point.   This defines a function $f:\NN\to X$.
(iii) Because $\sequencen{U_{\sigma^{\smallfrown}\fraction{n}}}$
is a partition of $U_{\sigma}$ for every $\sigma$, $f$ is a bijection.
(iv) If $G\subseteq X$ is open, then

\Centerline{$f^{-1}[G]=\{\alpha:$ there is some $k\in\Bbb N$ such that
$U_{\alpha\restr k}\subseteq G\}$}

\noindent is open, so $f$ is continuous.   (v) If $H\subseteq\NN$ is open,
then

\Centerline{$f[H]=\bigcup\{U_{\sigma}:\sigma\in S$,
$\{\alpha:\sigma\subseteq\alpha\in\NN\}\subseteq H\}$}

\noindent is open, so $f$ is a homeomorphism.\ \Qed}
%5{}A4Je

\spheader 5A4Ie If $X$ is a non-empty Polish space without
isolated points, then it has a dense G$_{\delta}$ set which is
homeomorphic to $\NN$ with its usual topology.
%5{}22Vb
\prooflet{\Prf\ Let $\Cal U$ be a countable base for the topology of $X$,
and $D$ a countable dense subset of $X$;   set

\Centerline{$Y=X\setminus(D\cup\bigcup_{U\in\Cal U}\partial U)$.}

\noindent
Then $Y$ is a G$_{\delta}$ set in $X$, so is Polish (4A2Qd).
Because $X$ has no isolated points, $Y$ is comeager in $X$ and is dense and
not empty.
Because $\{U\cap Y:U\in\Cal U\}$ is a base for the topology of $Y$
consisting of relatively open-and-closed sets, $Y$ is zero-dimensional.
If $V\subseteq Y$ is a non-empty relatively open set, let $G\subseteq X$ be
an open set such that $G\cap Y=V$;  then $D\cap G$ is non-empty, so there
is a sequence in $V$ converging (in $X$) to a point in
$D\cap G\subseteq X\setminus V$, and $V$
cannot be compact.   By (d), $Y$ is homeomorphic to $\NN$.\ \Qed}

%\endnotes{
%\Notesheader{5A4}

%}%end of notes

\discrpage
