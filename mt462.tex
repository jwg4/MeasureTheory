\frfilename{mt462.tex}
\versiondate{30.6.07}
\copyrightdate{2007}

\def\chaptername{Pointwise compact sets of measurable functions}
\def\sectionname{Pointwise compact sets of continuous functions}

\newsection{462}

In preparation for the main work of this chapter, beginning in the next
section, I offer a few pages on spaces of continuous functions under
their `pointwise' topologies (462Ab).   There is an extensive general
theory of such spaces, described in {\smc Arkhangel'skii 92};  here I
present only those fragments which seem directly relevant to the theory
of measures on normed spaces and spaces of functions.   In particular,
I star the paragraphs 462C-462D, which are
topology and functional analysis
rather than measure theory.   They are here because although this
material is well known, and may be found in many places, I think that
the ideas, as well as the results, are essential for any understanding
of measures on linear topological spaces.

Measure theory enters the section in the proof of 462E, in the form of
an application of the Riesz representation theorem, though 462E itself
remains visibly part of functional analysis.   In the rest of the
section, however, we come to results which are pure measure theory.
For (countably) compact spaces $X$, the Radon measures on $C(X)$ are the
same for the pointwise and norm topologies (462I).
This fact has extensive implications for the theory of separately
continuous functions (462K) and for the theory of convex hulls in linear
topological spaces (462L).

\leader{462A}{Definitions (a)} A regular Hausdorff space $X$ is {\bf
angelic} if whenever $A$ is a subset of $X$ which is relatively
countably compact in $X$, then (i) its closure $\overline{A}$ is compact
(ii) every point of $\overline{A}$ is the limit of a sequence in $A$.

\cmmnt{(A {\bf Fr\'echet-Urysohn} space is a topological space in
which, for any set $A$, every point of the closure of $A$ is a limit of
a sequence in $A$.   So (ii) here may be written as `every compact
subspace of $X$ is a Fr\'echet-Urysohn space'.)}

\spheader 462Ab If $X$ is any set and $A$ a subset of $\Bbb R^X$, then
the topology of {\bf pointwise convergence} on $A$ is that inherited
from the usual product topology of $\Bbb R^X$\cmmnt{;  that is, the
coarsest topology on $A$ for which the map $f\mapsto f(x):A\to\Bbb R$ is
continuous for every $x\in X$}.   I shall commonly use the symbol
$\frak T_p$ for such a
topology.   In this context, I will say that a sequence or filter is
{\bf pointwise convergent} if it is convergent for the topology of
pointwise convergence.   Note that if $A$ is a linear subspace of
$\Bbb R^X$ then
$\frak T_p$ is a linear space topology on $A$\cmmnt{ (4A4Ba)}.

\vleader{48pt}{*462B}{Proposition}\cmmnt{ ({\smc Pryce 71})} Let
$(X,\frak T)$ be an angelic regular Hausdorff space.

(a) Any subspace of $X$ is angelic.

(b) If $\frak S$ is a regular topology on $X$ finer than $\frak T$, then
$\frak S$ is angelic.

(c) Any countably compact subset of $X$ is compact and sequentially
compact.

\proof{{\bf (a)} Let $Y$ be any subset of $X$.   Then of course the
subspace topology on $Y$ is regular and Hausdorff.   If $A\subseteq Y$
is relatively countably compact in $Y$, then $A$ is
relatively countably compact in $X$, so $\overline{A}$, the closure of
$A$ in $X$, is compact.   Now if $x\in\overline{A}$, there is a sequence
$\sequencen{x_n}$ in $A$ converging to $x$;   but $\sequencen{x_n}$ must
have a cluster point in $Y$, and (because $\frak T$ is Hausdorff) this
cluster point can only be $x$.   Accordingly $\overline{A}\subseteq Y$
and is the closure of $A$ in $Y$.   Thus $A$ is relatively compact in
$Y$.   Moreover, any point of $\overline{A}$ is the limit of a sequence
in $A$.   As $A$ is arbitrary, $Y$ is angelic.

\medskip

{\bf (b)} By hypothesis, $\frak S$ is regular, and it is Hausdorff
because it is finer than $\frak T$.   Now suppose that $A\subseteq X$ is
$\frak S$-relatively countably compact.   Because the identity map from
$(X,\frak S)$ to $(X,\frak T)$ is continuous, $A$ is
$\frak T$-relatively countably compact (4A2G(f-iv)), and the
$\frak T$-closure $\overline{A}$ of $A$ is $\frak T$-compact.

Let $\Cal F$ be any ultrafilter on $X$ containing $A$.
Then $\Cal F$ has a $\frak T$-limit $x\in X$.   \Quer\ If $\Cal F$ is
not $\frak S$-convergent to $x$,
there is an $H\in\frak S$ such that $x\in H$ and
$X\setminus H\in\Cal F$, so that $A\setminus H\in\Cal F$.
Now $x$ belongs to the
$\frak T$-closure of $A\setminus H$, because $A\setminus H\in\Cal F$ and
$\Cal F$ is $\frak T$-convergent to $x$;  because $\frak T$ is angelic,
there is a sequence $\sequencen{x_n}$ in $A\setminus H$ which
$\frak T$-converges to $x$.   But now $\sequencen{x_n}$ has a
$\frak S$-cluster point $x'$.   $x'$ must also be a $\frak T$-cluster
point of $\sequencen{x_n}$, so $x'=x$;  but every $x_n$ belongs to the
$\frak S$-closed set $X\setminus H$, so $x'\notin H$, which is
impossible.\ \Bang

Thus every ultrafilter on $X$ containing $A$ is $\frak S$-convergent.
Because $\frak S$ is regular, the $\frak S$-closure $\tilde A$ of $A$ is
$\frak S$-compact (3A3De).

Again because $\frak S$ is finer than $\frak T$, and $\frak T$ is
Hausdorff, the two topologies must agree on $\tilde A=\overline{A}$.
But now every point of $\overline{A}$ is the $\frak T$-limit of a
sequence in $A$, so every point of $\tilde A$ is the $\frak S$-limit of
a sequence in $A$.   As $A$ is arbitrary, $\frak S$ is angelic.

\medskip

{\bf (c)} If $K\subseteq X$ is countably compact, then of course it is
relatively countably compact in its subspace topology, so (being
angelic) must be compact in its subspace topology.   If
$\sequencen{x_n}$ is a sequence in $K$, let $x$ be any cluster point of
$\sequencen{x_n}$.   If $\{n:x_n=x\}$ is infinite, then this immediately
provides us with a subsequence converging to $x$.   Otherwise, take $n$
such that $x\ne x_i$ for $i\ge n$.   Since $x\in\overline{\{x_i:i\ge
n\}}$, and $\{x_i:i\ge n\}$ is relatively countably compact, there is a
sequence $\sequence{i}{y_i}$ in $\{x_i:i\ge n\}$ converging to $x$.
The topology of $X$ being Hausdorff, $\{y_i:i\in\Bbb N\}$ must be
infinite, and $\sequence{i}{y_i}$, $\sequence{i}{x_i}$ have a common
subsequence which converges to $x$.   As $\sequence{i}{x_i}$ is
arbitrary, $K$ is sequentially compact.
}%end of proof of 462B

\vleader{60pt}{*462C}{Theorem}\cmmnt{ ({\smc Pryce 71})} Let $X$ be a
topological space such that there is a sequence $\sequencen{X_n}$ of
relatively
countably compact subsets of $X$, covering $X$, with the property that a
function $f:X\to\Bbb R$ is continuous whenever $f\restr X_n$ is
continuous for every $n\in\Bbb N$.   Then the space $C(X)$ of continuous
real-valued functions on $X$ is angelic in its topology of pointwise
convergence.

\proof{ Of course $C(X)$ is regular and Hausdorff under $\frak T_p$,
because $\Bbb R^X$ is, so we need attend only to the rest of the
definition in 462Aa.   Let $A\subseteq C(X)$ be relatively countably
compact for $\frak T_p$.

\medskip

{\bf (a)} Since $\{f(x):f\in A\}$, being a continuous image of $A$, must
be relatively countably compact
in $\Bbb R$ (4A2G(f-iv)), therefore relatively compact (4A2Le), for
every $x\in X$, the
closure $\overline{A}$ of $A$ in $\Bbb R^X$ is compact, by Tychonoff's
theorem.

\Quer\ Suppose, if possible, that $\overline{A}\not\subseteq C(X)$;  let
$g\in\overline{A}$ be a discontinuous function.   By the hypothesis of
the theorem, there is an $n\in\Bbb N$ such that $g\restr X_n$ is not
continuous;  take $x^*\in X_n$ such that $g\restr X_n$ is discontinuous
at $x^*$.   Let $\epsilon>0$ be such that for every neighbourhood $U$ of
$x^*$ in $X_n$ there is a point $x\in U$ such that
$|g(x)-g(x^*)|\ge\epsilon$.

Choose sequences $\sequence{i}{f_i}$ in $A$ and $\sequence{i}{x_i}$ in
$X_n$ as follows.   Given $\langle f_i\rangle_{i<m}$ and
$\langle x_i\rangle_{i<m}$, choose $x_m\in X_n$ such that
$|f_i(x_m)-f_i(x^*)|\le 2^{-m}$ for every $i<m$ and
$|g(x^*)-g(x_m)|\ge\epsilon$.   Now choose
$f_m\in A$ such that $|f_m(x^*)-g(x^*)|\le 2^{-m}$ and
$|f_m(x_i)-g(x_i)|\le 2^{-m}$ for every $i\le m$.   Continue.

At the end of the induction, take a cluster point $x$ of
$\sequence{i}{x_i}$ in $X$ and a cluster point $f$ of
$\sequence{i}{f_i}$ in $C(X)$.   Because $|f_i(x_m)-f_i(x^*)|\le 2^{-m}$
whenever $i<m$, $f_i(x)=f_i(x^*)$ for every $i$, and $f(x)=f(x^*)$.
Because $|f_m(x^*)-g(x^*)|\le 2^{-m}$ for every $m$, $f(x^*)=g(x^*)$.
Because $|f_m(x_i)-g(x_i)|\le 2^{-m}$ whenever $i\le m$, $f(x_i)=g(x_i)$
for every $i$, $|g(x^*)-f(x_i)|\ge\epsilon$ for every $i$, and
$|g(x^*)-f(x)|\ge\epsilon$;  but this is impossible, because
$f(x)=f(x^*)=g(x^*)$.\ \Bang

Thus the compact set $\overline{A}\subseteq C(X)$ is the closure of $A$
in $C(X)$, and $A$ is relatively compact in $C(X)$.

\medskip

{\bf (b)} Now take any $g\in\overline{A}$.   There are countable sets
$D\subseteq X$, $B\subseteq A$ such that

\inset{whenever $I\subseteq B\cup\{g\}$ is finite, $n\in\Bbb N$,
$\epsilon>0$ and $x\in X_n$, there is a $y\in D\cap X_n$ such that
$|f(y)-f(x)|\le\epsilon$ for every $f\in I$;}

\inset{whenever $J\subseteq D$ is finite and $\epsilon>0$ there is an
$f\in B$ such that $|f(x)-g(x)|\le\epsilon$ for every $x\in J$.}

\noindent\Prf\ For any finite set $I\subseteq\Bbb R^X$ and $n\in\Bbb N$,
the set $Q_{In}=\{\langle f(x)\rangle_{f\in I}:x\in X_n\}$ is a subset
of the separable metrizable space $\BbbR^I$, so is itself separable, and
there is a countable dense set $D_{In}\subseteq X_n$ such that
$Q'_{In}=\{\langle f(x)\rangle_{f\in I}:x\in D_{In}\}$ is dense in
$Q_{In}$.   Similarly, because $g\in\overline{A}$, we can choose for any
finite set $J\subseteq X$ a sequence $\sequence{i}{f_{Ji}}$ in $A$ such
that $\lim_{i\to\infty}f_{Ji}(x)=g(x)$ for every $x\in J$.

Now construct $\sequence{m}{D_m}$, $\sequence{m}{B_m}$ inductively by
setting

\Centerline{$D_m=\bigcup\{D_{In}:n\in\Bbb N,\,
  I\subseteq\{g\}\cup\bigcup_{i<m}B_i$ is finite$\}$,}

\Centerline{$B_m=\{f_{Jk}:k\in\Bbb N,\,J\subseteq\bigcup_{i<m}D_i$ is
finite$\}$.}

\noindent At the end of the induction, set $D=\bigcup_{m\in\Bbb N}D_m$
and $B=\bigcup_{m\in\Bbb N}B_m$.   Since the construction clearly
ensures that $\sequence{m}{D_m}$ and $\sequence{m}{B_m}$ are
non-decreasing sequences of countable sets, $D$ and $B$ are countable,
and we shall have $D_{In}\subseteq D$ whenever $n\in\Bbb N$ and
$I\subseteq B\cup\{g\}$ is finite, while $f_{Ji}\in B$ whenever
$i\in\Bbb N$ and $J\subseteq D$ is finite.   Thus we have suitable sets
$D$ and $B$.\ \Qed

By the second condition on $D$ and $B$, there must be a sequence
$\sequence{i}{f_i}$ in $B$ such that $g(x)=\lim_{i\to\infty}f_i(x)$ for
every $x\in D$.   In fact $g(y)=\lim_{i\to\infty}f_i(y)$ for every $y\in
X$.   \Prf\Quer\ Otherwise, there is an $\epsilon>0$ such that
$J=\{i:|g(y)-f_i(y)|\ge\epsilon\}$ is infinite.   Let $n$ be such that
$y\in X_n$.   For each $m\in\Bbb N$, $I_m=\{f_i:i\le m\}$ is a finite
subset of $B$, so there is an $x_m\in D_{I_mn}$ such that
$|f(x_m)-f(y)|\le 2^{-m}$ for every $f\in I_m\cup\{g\}$.   Let
$x^*\in X$ be a cluster point of $\sequence{m}{x_m}$, and
$h\in C(X)$ a cluster point of $\langle f_i\rangle_{i\in J}$.   Then

\inset{because $g(x)=\lim_{i\to\infty}f_i(x)$ for every $x\in D$,
$g(x_m)=h(x_m)$ for every $m\in\Bbb N$, and $g(x^*)=h(x^*)$;}

\inset{because $|g(y)-f_i(y)|\ge\epsilon$ for every $i\in J$,
$|g(y)-h(y)|\ge\epsilon$;}

\inset{because $|f_i(x_m)-f_i(y)|\le 2^{-m}$ whenever $i\le m$,
$f_i(x^*)=f_i(y)$
for every $i\in\Bbb N$, and $h(x^*)=h(y)$;}

\inset{because $|g(x_m)-g(y)|\le 2^{-m}$ for every $m$, $g(x^*)=g(y)$;}

\noindent but this means that $g(y)=g(x^*)=h(x^*)=h(y)\ne g(y)$, which is
absurd.\ \Bang\Qed

So $g=\lim_{i\to\infty}f_i$ for $\frak T_p$.   As $g$ is arbitrary, $A$
has both properties required in 462Aa;  as $A$ is arbitrary, $C(X)$ is
angelic.
}%end of proof of 462C

\cmmnt{\medskip

\noindent{\bf Remark} For a slight strengthening of this result, see
462Ya.}

\leader{*462D}{Theorem} Let $U$ be any normed space.   Then it is
angelic in its weak topology.

\proof{ Write $X$ for the unit ball of the dual space $U^*$, with its
weak* topology.   Then $X$ is compact (3A5F).   We have
a natural map $u\mapsto\hat u:U\to\Bbb R^X$ defined by setting
$\hat u(x)=x(u)$ for $x\in X$ and $u\in U$.
By the definition of the weak*
topology, $\hat u\in C(X)$ for every $u\in U$.   The weak topology of
$U$ is normally defined in terms of all functionals $u\mapsto f(u)$, for
$f\in U^*$;  but as every member of $U^*$ is a scalar multiple of some
$x\in X$, we can equally regard the weak topology of $U$ as defined just
by the functionals $u\mapsto x(u)=\hat u(x)$, for $x\in X$.   But this
means that the map $u\mapsto\hat u$ is a homeomorphism between $U$, with
its weak topology, and its image $\widehat U$ in $C(X)$, with the
topology of pointwise convergence.

Since $C(X)$ is $\frak T_p$-angelic (462C), so is $\widehat U$ (462Ba),
and $U$ is angelic in its weak topology.
}%end of proof of 462D

\leader{462E}{Theorem} Let $X$ be a locally compact Hausdorff space, and
$C_0(X)$ the Banach lattice of continuous real-valued functions on $X$
which vanish at infinity\cmmnt{ (436I)}.
Write $\frak T_p$ for the topology of pointwise convergence on $C_0(X)$.

(i) $C_0(X)$ is $\frak T_p$-angelic.

(ii) A sequence $\sequencen{u_n}$ in $C_0(X)$ is weakly convergent to
$u\in C_0(X)$ iff it is $\frak T_p$-convergent to $u$ and norm-bounded.

(iii) A subset $K$ of $C_0(X)$ is weakly compact iff it is norm-bounded
and $\frak T_p$-countably compact.

\proof{{\bf (a)} Let $X^*=X\cup\{x_{\infty}\}$ be the one-point
compactification of $X$ (3A3O).   Then $C(X^*)$ is angelic in its topology
$\frak T^*_p$ of pointwise convergence, by 462C.   Set
$V=\{g:g\in C(X^*)$, $g(x_{\infty})=0\}$.   By 462Ba, $V$ is angelic in the
subspace topology induced by $\frak T_p^*$.   Now observe that we have a
natural bijection $g\mapsto g\restr X:V\to C_0(X)$, and that this is a
homeomorphism for the topologies of pointwise convergence on $V$ and
$C_0(X)$.   So $C_0(X)$ is angelic under $\frak T_p$.

\medskip

{\bf (b)} Since all the maps $u\mapsto u(x)$, where $x\in X$, are
bounded linear functionals on $C_0(X)$, $\frak T_p$ is coarser than
the weak topology $\frak T_s$;  so
a $\frak T_s$-convergent sequence is $\frak T_p$-convergent to the same
limit, and a $\frak T_s$-compact set is $\frak T_p$-compact, therefore
$\frak T_p$-countably compact.

\medskip

{\bf (c)} If $K\subseteq C_0(X)$ is $\frak T_s$-compact, then
$f[K]\subseteq\Bbb R$ must be compact, therefore bounded, for every
$f\in C_0(X)^*$;  by the Uniform Boundedness Theorem (3A5Hb), $K$ is
norm-bounded.   Applying this to
$\{u\}\cup\{u_n:n\in\Bbb N\}$, we see that any $\frak T_s$-convergent
sequence $\sequencen{u_n}$ with limit $u$ is norm-bounded.

\medskip

{\bf (d)} Suppose that $\sequencen{u_n}$ is norm-bounded and
$\frak T_p$-convergent to $u$.   By Lebesgue's Dominated Convergence
Theorem,
$\lim_{n\to\infty}\int u_nd\mu=\int u\,d\mu$ for every totally finite
Radon measure $\mu$ on $X$.   But by the Riesz Representation Theorem
(in the form 436K), this says just that $\lim_{n\to\infty}f(u_n)=f(u)$
for every positive linear functional $f$ on $C_0(X)$.   Since every
member of $C_0(X)^*$ is expressible as the difference of two positive
linear functionals (356Dc), $\lim_{n\to\infty}f(u_n)=f(u)$ for every
$f\in U^*$, that is, $\sequencen{u_n}$ is $\frak T_s$-convergent to $u$.

Putting this together with (b) and (c), we see that (ii) is true.

\medskip

{\bf (e)} Now suppose that $K\subseteq C_0(X)$ is norm-bounded and
$\frak T_p$-countably compact.   Any
sequence $\sequencen{u_n}$ in $K$ has a subsequence which is
$\frak T_p$-convergent to a point of $K$ (462Bc), and this subsequence is
also $\frak T_s$-convergent, by (c).   This means that $K$ is
sequentially compact, therefore
countably compact, in $C_0(X)$ for the topology $\frak T_s$.   Since
$\frak T_s$ is angelic (462D), $K$ is $\frak T_s$-compact, by 462Bc again.
}%end of proof of 462E

\leader{462F}{Lemma} Let $X$ be a topological space, and $Q$ a relatively
countably compact subset of $X$.   Suppose that $K\subseteq C_b(X)$ is
$\|\,\|_{\infty}$-bounded and $\frak T_p$-countably
compact, where $\frak T_p$ is the
topology of pointwise convergence on $C_b(X)$.   Then the map
$u\mapsto u\restr Q:K\to C_b(Q)$ is continuous for $\frak T_p$ on $K$ and
the weak topology of the Banach space $C_b(Q)$.

\proof{ We have a natural map $x\mapsto\hat x:X\to\BbbR^K$ defined by
writing $\hat x(u)=u(x)$ for every $u\in K$ and
$x\in X$.   By the definition of $\frak T_p$, $\hat x\in C(K)$ for every
$x\in X$, if we take $C(K)$ to be the space of $\frak T_p$-continuous
real-valued functions on $K$;  and $x\mapsto\hat x:X\to C(K)$ is
continuous for the given topology on $X$ and the topology of pointwise
convergence on $C(K)$ because $K\subseteq C(X)$.   It follows that
$\{\hat x:x\in Q\}$ is relatively countably compact for the topology of
pointwise convergence on $C(K)$ (4A2G(f-iv)).   But now
$Z=\overline{\{\hat x:x\in Q\}}$ must be actually compact for
the topology of pointwise convergence on $C(K)$, by 462C.

Next, consider the natural map $u\mapsto\hat u:K\to\BbbR^Z$ defined by
setting $\hat u(f)=f(u)$ for $f\in Z$ and $u\in K$.   Just as in the last
paragraph, this is a continuous function from $K$ to $C(Z)$, if we give
$K$, $Z$ and $C(Z)$ their topologies of pointwise convergence.   So
$L=\{\hat u:u\in K\}$ is countably compact for the topology of pointwise
convergence on $C(Z)$ (4A2G(f-vi)).
Moreover, it is norm-bounded, because

$$\eqalign{\sup_{\phi\in L}\|\phi\|_{\infty}
&=\sup_{u\in K,f\in Z}|\hat u(f)|
=\sup_{u\in K,f\in Z}|f(u)|
=\sup_{u\in K,x\in Q}|\hat x(u)|\cr
&=\sup_{u\in K,x\in Q}|u(x)|
\le\sup_{u\in K,x\in X}|u(x)|
=\sup_{u\in K}\|u\|_{\infty}\cr}$$

\noindent is finite.   So 462E(iii) tells us that $L$ is weakly compact in
$C(Z)$.   (Note that $C(Z)=C_0(Z)$ because $Z$ is compact.)
Since the weak topology on $C(Z)$ is finer than the pointwise topology,
while the pointwise topology is Hausdorff,
the two topologies on $L$ coincide;  it follows that
$u\mapsto\hat u:K\to C(Z)$ is continuous for $\frak T_p$ and the weak
topology on $C(Z)$.

Now we have an operator $T:C(Z)\to\Bbb R^Q$ defined by setting

\Centerline{$(T\phi)(x)=\phi(\hat x)$}

\noindent for $\phi\in C(Z)$ and $x\in Q$.   Because
$x\mapsto\hat x:Q\to Z$ is continuous, $T\phi\in C(Q)$ for every
$\phi\in C(Z)$, and of
course $T$, regarded as a linear operator from $C(Z)$ to $C_b(Q)$,
has norm at most $1$.   So $T$ is continuous for the weak topologies of
$C(Z)$ and $C_b(Q)$ (2A5If), and
$u\mapsto T\hat u:K\to C_b(Q)$ is continuous for $\frak T_p$ and the weak
topology of $C_b(Q)$.

But if $u\in K$ and $x\in Q$,

\Centerline{$(T\hat u)(x)=\hat u(\hat x)=\hat x(u)=u(x)$,}

\noindent so $T\hat u=u\restr Q$.   Accordingly
$u\mapsto u\restr Q:K\to C_b(Q)$ is continuous for $\frak T_p$ and the weak
topology on $C_b(Q)$.
}%end of proof of 462F

\leader{462G}{Proposition} Let $X$ be a countably compact topological
space.   Then a subset of $C_b(X)$ is weakly compact iff it is
norm-bounded and compact for the topology\cmmnt{ $\frak T_p$} of
pointwise convergence.

\proof{ A weakly compact subset of $C_b(X)$ is
norm-bounded and $\frak T_p$-compact by the same arguments as in (b)-(c)
of the proof of 462E.   In the other direction, taking $Q=X$ in 462F, we
see that a norm-bounded $\frak T_p$-compact set is weakly compact.
}%end of proof of 462G

\leader{462H}{Lemma} Let $X$ be a topological space, $Q$ a
relatively countably
compact subset of $X$, and $\mu$ a totally finite measure on $C_b(X)$ which
is Radon for the topology $\frak T_p$ of pointwise convergence
on $C_b(X)$.   Let
$T:C_b(X)\to C_b(Q)$ be the restriction map.   Then the image measure
$\nu=\mu T^{-1}$ on $C_b(Q)$ is Radon for the norm topology of $C_b(Q)$.

\proof{{\bf (a)} $T$ is almost continuous for $\frak T_p$ and the weak
topology of $C_b(Q)$.   \Prf\ If $E\in\dom\mu$ and $\mu E>\gamma\ge 0$,
then there is a $\frak T_p$-compact set $K\subseteq E$ such that
$\mu K>\gamma$.   Since all the balls
$\{f:f\in C_b(X)$, $\|f\|_{\infty}\le k\}$ are $\frak T_p$-closed, we may
suppose that $K$ is norm-bounded.   Now $T\restr K$ is continuous for
$\frak T_p$ and the weak topology, by 462F.\ \QeD\   By 418I, $\nu$ is a
Radon measure for the weak topology of $C_b(Q)$.

\medskip

{\bf (b)} I show next that if $F\in\dom\nu$, $\nu F>0$ and
$\epsilon>0$, there is some $g\in C_b(Q)$ such that
$\nu(F\cap B(g,\epsilon))>0$, where
$B(g,\epsilon)=\{h:\|h-g\|_{\infty}\le\epsilon\}$.   \Prf\
Since all the balls $B(g,\epsilon)$ are convex
and norm-closed, they are weakly closed (3A5Ee) and measured by $\nu$.
\Quer\ Suppose, if possible, that $F\cap B(g,\epsilon)$ is
$\nu$-negligible for every $g\in C_b(Q)$.   Set $E=T^{-1}[F]$.
As in (a), there is a
$\|\,\|_{\infty}$-bounded $\frak T_p$-compact set $K\subseteq E$
such that $\mu K>0$.
Choose $\sequencen{K_n}$ and $\sequencen{f_n}$ as follows.   $K_0=K$.
Given that $K_n\subseteq E$ is non-negligible and $\frak T_p$-compact,
and that $\langle f_i\rangle_{i<n}$ is a finite sequence
in $C_b(X)$, then the convex hull

\Centerline{$\Gamma_n=\{\sum_{i=0}^{n-1}\alpha_iTf_i:\alpha_i\ge 0$
for every $i<n$, $\sum_{i=0}^{n-1}\alpha_i=1\}$}

\noindent of the finite set $\{Tf_i:i<n\}$ is
norm-compact in $C_b(Q)$, so there is a finite set $D_n\subseteq\Gamma_n$
such that for every $g\in\Gamma_n$ there is a $g'\in D_n$ such that
$\|g-g'\|_{\infty}\le\bover12\epsilon$.   Now

\Centerline{$H_n=\{f:\|Tf-g\|_{\infty}>\epsilon$ for every $g\in D_n\}$}

\noindent is a $\frak T_p$-open set and

\Centerline{$E\setminus H_n
=\bigcup_{g\in D_n}T^{-1}[F\cap B(g,\epsilon)]$}

\noindent is $\mu$-negligible,
so $\mu(K_n\cap H_n)>0$ and we can find a non-negligible
$\frak T_p$-closed set $K_{n+1}\subseteq K_n\cap H_n$;  choose
$f_n\in K_{n+1}$.   Continue.

At the end of the induction, let $f^*\in K$ be a cluster point of
$\sequencen{f_n}$ for $\frak T_p$.
Since $T\restr K:K\to C_b(Q)$ is continuous for
$\frak T_p$ and the weak topology of $C_b(Q)$,
$Tf^*$ is a cluster point of
$\sequencen{Tf_n}$ for the weak topology on $C_b(Q)$.   The set
$\Gamma=\bigcup_{n\in\Bbb N}\Gamma_n$ is convex, so its
norm-closure $\overline{\Gamma}$ is also convex (2A5Eb), therefore
closed for the weak topology (3A5Ee), and contains $Tf^*$.   So
there is a $g\in\Gamma$ such that
$\|Tf^*-g\|_{\infty}\le\bover12\epsilon$.
Now there is some $n$ such that
$g\in\Gamma_n$.   Let $g'\in D_n$ be such that
$\|g-g'\|_{\infty}\le\bover12\epsilon$, so that
$\|Tf^*-g'\|_{\infty}\le\epsilon$.
But $f_i\in K_{n+1}$ for every $i\ge n$,
so $f^*\in K_{n+1}\subseteq H_n$ and $\|Tf^*-g'\|_{\infty}>\epsilon$,
which is impossible.\ \Bang\Qed

\medskip

{\bf (c)} What this means is that if we take $\Cal K_n$ to be the family
of subsets of $C_b(Q)$ which can be covered by finitely many balls of
radius at most $2^{-n}$, then $\nu$ is inner regular with respect
to $\Cal K_n$ (see 412Aa), and therefore with respect to
$\Cal K=\bigcap_{n\in\Bbb N}\Cal K_n$ (412Ac).   But $\Cal K$ is just
the set of subsets of $C_b(Q)$ which are totally bounded for the
norm-metric $\rho$ on $C_b(Q)$.

At the same time, $\nu$ is inner regular with respect to the $\rho$-closed
sets in $C_b(Q)$.   \Prf\ If $\nu F>\gamma$,
there is a $\|\,\|_{\infty}$-bounded $\frak T_p$-compact set
$K\subseteq T^{-1}[F]$ such that $\mu K\ge\gamma$;
now $T[K]$ is weakly compact,
therefore weakly closed and $\rho$-closed in $C_b(Q)$, while
$T[K]\subseteq F$ is measured by $\nu$ and

\Centerline{$\nu T[K]=\mu T^{-1}[T[K]]\ge\mu K\ge\gamma$.  \Qed}

\medskip

{\bf (d)} By 412Ac again, $\nu$ must be inner regular with respect to the
family of $\rho$-closed $\rho$-totally bounded sets;  because $C_b(Q)$ is
$\rho$-complete, these are the $\rho$-compact sets.   Next, every
$\rho$-compact set is weakly compact, therefore weakly closed, and is
measured by $\nu$, by (a);  and $\nu$, being the image of a complete
totally finite measure, is complete and totally finite.   Consequently
every $\rho$-closed set is measured by $\nu$ (use 412Ja) and $\nu$ is a
$\rho$-Radon measure, as claimed.
}%end of proof of 462H

\leader{462I}{Theorem} Let $X$ be a countably compact topological space.
Then the totally finite Radon measures on $C(X)$ are the same for the
topology of pointwise convergence and the norm topology.

\proof{ Write $\frak T_p$ for the topology of
pointwise convergence on $C(X)$ and $\frak T_{\infty}$ for the norm
topology.   Because $\frak T_p\subseteq\frak T_{\infty}$ and $\frak T_p$ is
Hausdorff, every totally
finite $\frak T_{\infty}$-Radon measure is $\frak T_p$-Radon (418I).
On the other hand, 462H, with $Q=X$, tells us that every $\frak T_p$-Radon
measure is $\frak T_{\infty}$-Radon.
}%end of proof of 462I

\leader{462J}{Corollary} Let $X$ be a countably compact Hausdorff space,
and give $C(X)$ its topology of pointwise convergence.   If $\mu$ is any
Radon measure on $C(X)$, it is inner regular with respect to the family
of compact metrizable subsets of $C(X)$.

\proof{ In the language of 462I, $\mu$ is inner regular with respect to
the family of $\frak T_{\infty}$-compact sets;  but as $\frak
T_p\subseteq\frak T_{\infty}$, the two topologies agree on all such
sets, and they are compact and metrizable for $\frak T_p$.
}%end of proof of 462J

\leader{462K}{Proposition} Let $X$ be a topological space, $Y$ a
Hausdorff space, $f:X\times Y\to\Bbb R$ a bounded separately continuous
function, and $\nu$ a totally finite Radon measure on $Y$.   Set
$\phi(x)=\int f(x,y)\nu(dy)$ for every $x\in X$.   Then $\phi\restr Q$
is continuous for every relatively countably compact set $Q\subseteq X$.

\proof{ For $y\in Y$, set $u_y(x)=f(x,y)$ for every $x\in X$.   Then
every $u_y$ is continuous and bounded, because $f$ is
bounded and continuous in the first
variable, and $y\mapsto u_y:Y\to C_b(X)$ is continuous, if we give $C_b(X)$
the topology $\frak T_p$ of pointwise convergence, because $f$ is
continuous in the second variable.   We therefore have a
$\frak T_p$-Radon image measure $\mu$ on $C_b(X)$, by 418I.

Let $T:C_b(X)\to C_b(Q)$ be the restriction map.
By 462H, the image measure $\lambda=\mu T^{-1}$
is a Radon measure for the norm
topology of $C_b(Q)$.   Now recall that $f$ is bounded.   If
$|f(x,y)|\le M$ for all $x\in X$, $y\in Y$, then $\|u_y\|_{\infty}\le M$
for every
$y\in Y$, and the ball $B(0,M)$ in $C_b(Q)$ is $\lambda$-conegligible.
By 461F, applied to the subspace measure on $B(0,M)$,
$\nu$ has a barycenter $h$ in $C_b(Q)$.   Now we can compute $h$ by
the formulae

$$\eqalignno{h(x)
&=\int_{C_b(Q)}g(x)\lambda(dg)\cr
\noalign{\noindent (because $g\mapsto g(x)$ belongs to $C_b(X)^*$)}
&=\int_{C_b(X)}u(x)\mu(du)
=\int_Yu_y(x)\nu(dy)\cr
\noalign{\noindent (by 235G)}
&=\int_Yf(x,y)\nu(dy)
=\phi(x),\cr}$$

\noindent for every $x\in Q$.   So $\phi=h$ is continuous.
}%end of proof of 462K

\leader{462L}{Corollary} Let $X$ be a topological space such that

\inset{\noindent
whenever $h\in\BbbR^X$ is such that $h\restr Q$ is continuous
for every relatively countably compact $Q\subseteq X$, then $h$ is
continuous.}

\noindent Write $\frak T_p$ for the topology of pointwise convergence on
$C(X)$.   Let $K\subseteq C(X)$ be a $\frak T_p$-compact
set such that $\{h(x):h\in K$, $x\in Q\}$ is bounded for any
relatively countably compact set $Q\subseteq X$.   Then the
$\frak T_p$-closed convex hull of $K$, taken in $C(X)$, is
$\frak T_p$-compact.

\proof{ If $K$ is empty, this is trivial;  suppose that $K\ne\emptyset$.
Since $\sup_{h\in K}|h(x)|$ is finite for every $x\in X$,
the closed convex hull $\overline{\Gamma(K)}$ of $K$, taken in
$\Bbb R^X$, is closed and included in a product of closed bounded
intervals, therefore compact.    If
$h\in\overline{\Gamma(K)}$, then there is a Radon probability measure
$\mu$ on $K$ such that $h$ is the barycenter of $\mu$ (461I), so that
$h(x)=\int f(x)\mu(df)$ for every $x\in X$.

If $Q\subseteq X$ is relatively
countably compact, then $h\restr Q$ is continuous.   \Prf\ Of course we may
suppose that $Q$ is non-empty.   Consider its closure $Z=\overline{Q}$.
We have a continuous linear operator $T:\BbbR^X\to\BbbR^Z$ defined by
setting $Tf=f\restr Z$ for every $f\in\BbbR^X$.   $L=T[K]$ is compact in
$\BbbR^Z$, and $L\subseteq C(Z)$;  moreover,

\Centerline{$\sup_{g\in L}\|g\|_{\infty}
=\sup_{f\in K,x\in Z}|f(x)|
=\sup_{f\in K,x\in Q}|f(x)|$}

\noindent is finite.   Since $T\restr K:K\to L$ is continuous, the image
measure $\nu=\mu(T\restr K)^{-1}$ on $L$ is a Radon measure.
If $x\in Z$, then

\Centerline{$h(x)=\int_Kf(x)\mu(df)=\int_K(Tf)(x)\mu(df)
=\int_Lg(x)\nu(dg)$.}

The map $(x,g)\mapsto g(x):Z\times L\to\Bbb R$ is separately continuous,
because $L\subseteq C(Z)$ is being given its topology of pointwise
convergence, and bounded.    Also every sequence in $Q$ has a cluster point
in $X$ which must also belong to $Z$, and $Q$ is relatively countably
compact in $Z$.   By 462K, $h\restr Q$ is continuous, as required.\ \Qed

Thus the $\frak T_p$-compact set $\overline{\Gamma(K)}$ is included in
$C(X)$, and must be the closed convex hull of $K$ in $C(X)$.
}%end of proof of 462L

\cmmnt{\medskip

\noindent{\bf Remark} The hypothesis

\inset{\noindent
whenever $h\in\Bbb R^X$ is such that $h\restr Q$ is continuous
for every relatively countably compact $Q\subseteq X$, then $h$ is
continuous}

\noindent is clumsy, but seems the best way to cover the large number of
potential applications of the ideas here.   Besides the obvious case of
countably compact spaces $X$, we have all first-countable spaces (for
which, of course, the other hypotheses can be relaxed, as in 462Xc), and
all $k$-spaces.   (A {\bf $k$-space} is a topological space $X$ such
that a set $G\subseteq X$ is open iff $G\cap K$ is relatively open in
$K$ for every compact set $K\subseteq X$;  see {\smc Engelking 89},
3.3.18 {\it et seq.}   In particular, all locally compact spaces are
$k$-spaces.)
}%end of comment

\exercises{\leader{462X}{Basic exercises (a)}
%\spheader 462Xa
(i) Show that $\Bbb R$, with the right-facing Sorgenfrey
topology, is angelic.   (ii) Show that any metrizable space is angelic.
(iii) Show that the one-point compactification of an angelic locally
compact Hausdorff space is angelic.   (iv) Find a first-countable regular Hausdorff space which is not angelic.
%462B

\sqheader 462Xb Let $X$ be any countably compact topological space.
Show that a norm-bounded sequence in $C_b(X)$ which is pointwise
convergent is weakly convergent.
%462G

\spheader 462Xc Let $X$ be a first-countable topological space,
$(Y,\Tau,\nu)$ a totally finite measure space, and $f:X\times Y\to\Bbb
R$ a bounded function such that $y\mapsto f(x,y)$ is measurable for
every $x\in X$, and $x\mapsto f(x,y)$ is continuous for almost every
$y\in Y$.   Show that $x\mapsto\int f(x,y)\nu(dy)$ is continuous.
%462K

\spheader 462Xd Give an example of a $\frak T_p$-compact subset $K$ of
$C([0,1])$ such that the convex hull of $K$ is not relatively
compact in $C([0,1])$.
%462L

\leader{462Y}{Further exercises (a)}
%\spheader 462Ya
Let $X$ be a topological space such that there is a
sequence $\sequencen{X_n}$ of
relatively countably compact subsets of $X$, covering
$X$, with the property that a function $f:X\to\Bbb R$ is continuous
whenever $f\restr X_n$ is continuous for every $n\in\Bbb N$.   Let
$\frak T_p$ be the topology of pointwise convergence on $C(X)$.   Show
that, for a set $K\subseteq C(X)$, the following are equiveridical:  (i)
$\phi[K]$ is bounded for every $\frak T_p$-continuous function
$\phi:C(X)\to\Bbb R$;  (ii) whenever $\sequencen{f_n}$ is a sequence in
$K$ and $A\subseteq X$ is countable, there is a cluster point of
$\sequencen{f_n\restr A}$ in $C(A)$ for the topology of pointwise
convergence on $C(A)$;  (iii) $K$ is relatively compact in $C(X)$ for
$\frak T_p$.  (See {\smc Asanov \& Velichko 81}.)
%462C

\spheader 462Yb Let $U$ be a metrizable locally convex linear
topological space.   Show that it is angelic in its weak topology.
\Hint{start with the case in which $U$ is complete, using Grothendieck's
theorem and the full strength of 462C, with $X=U^*$.}
%462D

\spheader 462Yc In 462K, show that the conclusion remains valid for any
totally finite $\tau$-additive topological measure $\nu$ on $Y$ which is
inner regular with respect to the relatively countably compact subsets
of $Y$.
%462K

\spheader 462Yd Show that if $X$ is any compact topological space (more
generally, any topological space such that $X^n$ is Lindel\"of for every
$n\in\Bbb N$), then $C(X)$, with its topology of pointwise convergence,
is countably tight.
%462+

\spheader 462Ye(i)
Let $X$ and $Y$ be Polish spaces, and write $B_1(X;Y)$ for
the set of functions $f:X\to Y$ such that $f^{-1}[H]$ is G$_{\delta}$ in
$X$ for every closed set $H\subseteq Y$ ({\smc Kuratowski 66}, \S31).
Show that $B_1(X;Y)$, with the topology of pointwise convergence inherited
from $Y^X$, is angelic.   \Hint{{\smc Bourgain Fremlin \& Talagrand 78}.}
(ii) Let $X$ be a Polish space.   Show that the space $\tildeClll(X)$ of
438P-438Q is angelic.
}%end of exercises

\leader{462Z}{Problem} Let $K$ be a compact Hausdorff space.   Is
$C(K)$, with the topology of pointwise convergence, necessarily a
pre-Radon space?   \cmmnt{(Compare 454S.)}

%Is C(K) Borel-measure-compact?
%Is C(K) Borel-measure-complete?  or even measure-compact?
%Not if K is one-point compactification
%of a discrete space with cardinal not of measure zero, as then C(K)
%has a subspace homeomorphic to K.   But what if every cardinal has
%measure zero?

\endnotes{
\Notesheader{462} The theory of pointwise convergence in spaces of
continuous functions is intimately connected with the theory of
separately continuous functions of two variables.   For if $X$ and $Y$
are topological spaces, and $f:X\times Y\to\Bbb R$ is any separately
continuous function, then we have natural maps $x\mapsto f_x:X\to C(Y)$
and $y\mapsto f^y:Y\to C(X)$, writing $f_x(y)=f^y(x)=f(x,y)$, which are
continuous if $C(X)$ and $C(Y)$ are given their topologies of pointwise
convergence;  and if $X$ is a topological space and $Y$ is any subset of
$C(X)$ with its topology of pointwise convergence, the map
$(x,y)\mapsto y(x):X\times Y\to\Bbb R$ is separately continuous.   I
include a back-and-forth shuffle between $C(X)$ and separately
continuous functions in 462H-462K-462L as a demonstration of the
principle that all the theorems here can be expressed in both languages.

462Yb is a compendium of \v Smulian's theorem with part of Eberlein's
theorem;
462E and 462L can be thought of as the centre of Krein's theorem.
There are many alternative routes to these results, which may be found
in {\smc K\"othe 69} or {\smc Grothendieck 92}.   In particular, 462E
can be proved without using measure theory;  see, for instance, 
{\smc Fremlin 74}, A2F.

Topological spaces homeomorphic to compact 
uniformly bounded subsets of $C(X)$, where $X$
is some compact space and $C(X)$ is given its topology of pointwise
convergence, are called {\bf Eberlein compacta};  see 467O-467P.

A positive answer to A.Bellow's problem (463Za below) would imply a 
positive answer to 462Z;  so if the continuum hypothesis, for instance, 
is true,
then $C(K)$ is pre-Radon in its topology of pointwise convergence for
any compact space $K$.

}%end of notes

\discrpage

\leaveitout{If $X$ and $Y$ are compact Hausdorff spaces, of which $Y$ is
ccc, and $f:X\times Y\to\Bbb R$ a separately continuous function, then
$f$ is Baire measurable.

\proof{{\bf (a)} To begin with, suppose that $X$ itself is countably
compact.   For $y\in Y$, set $u_y(x)=f(x,y)$ for every $x\in X$.   Then
every $u_y$ is continuous, because $f$ is continuous in the first
variable, and $y\mapsto u_y:Y\to C(X)$ is continuous, if we give $C(X)$
the topology $\frak T_p$ of pointwise convergence, because $f$ is
continuous in the second variable.   We therefore have a
$\frak T_p$-Radon image measure $\mu$ on $C(X)$, by 418I.

By 462I, $\mu$ is a Radon measure for the norm topology of $C(X)$.   Now
recall that $f$ is bounded.   If
$|f(x,y)|\le M$ for all $x\in X$, $y\in Y$, then $\|u_y\|_{\infty}\le M$
for every
$y\in Y$, and the ball $B(0,M)$ in $C(X)$ is $\mu$-conegligible.   By
461F, applied to the subspace measure on $B(0,M)$,
$\mu$ has a barycenter $h$ in $C(X)$.   Now we can compute $h$ by
the formulae

$$\eqalignno{h(x)
&=\int g(x)\mu(dg)\cr
\noalign{\noindent (because $g\mapsto g(x)$ belongs to $C(X)^*$)}
&=\int u_y(x)\nu(dy)\cr
\noalign{\noindent (by 235G)}
&=\int f(x,y)\nu(dy)
=\phi(x),\cr}$$

\noindent for every $x\in X$.   So $\phi=h$ is continuous.

\medskip

{\bf (b)} For the general result, we have only to apply (a) to $f\restr
C$ for each countably compact $C\subseteq X$.
}%end of proof of 462K

}%end of leaveitout

