\frfilename{mt435.tex}
\versiondate{16.8.08}
\copyrightdate{1999}

\def\chaptername{Topologies and measures II}
\def\sectionname{Baire measures}

\newsection{435}

Imitating the programme of \S434, I apply a similar analysis to Baire
measures, starting with a simple-minded classification\cmmnt{ (435A)}.   This time the central section (435D-435H) is devoted to
`measure-compact'
spaces, those on which all (totally finite) Baire measures are
$\tau$-additive.

\cmmnt{
\leader{435A}{Types of Baire measures} In 434A I looked at a
list of four
properties which a Borel measure may or may not possess:  inner
regularity with respect to closed sets, inner regularity with respect to
zero sets, tightness\cmmnt{ (that is, inner regularity with respect to closed compact sets)}, and
$\tau$-additivity.   Since every (semi-finite) Baire measure is inner
regular with respect to the zero sets (412D), only two of the four are
important considerations for Baire measures:  tightness and
$\tau$-additivity.   On the other hand,
there is a new question we can ask.   Given a Baire measure on a
topological space, when can it be extended to a Borel measure?   And in
the case of a positive answer, we can ask whether the extension is
unique, and whether we can find extensions to Borel measures satisfying
the properties considered in 434A.

We already have some information on this.   If $X$ is a completely
regular space, and $\mu$ is a $\tau$-additive effectively locally finite
Baire measure on $X$, then $\mu$ has a (unique) extension to a
$\tau$-additive Borel measure (415N).
While if $\mu$ is tight, the extension will also be tight (cf.\ 416C).   Perhaps I should
remark immediately that while there can be only one $\tau$-additive
Borel measure extending $\mu$, there might be another Borel measure, not
$\tau$-additive, also extending $\mu$;  see 435Xa.   Of course if there
is any completion regular Borel measure extending $\mu$, there is only
one;  moreover, if $\mu$ is $\sigma$-finite, and there is a completion
regular Borel measure extending $\mu$, this is the only Borel measure
extending $\mu$.   (For every Borel set will be measured by the
completion of $\mu$.)

A possible division of Baire measures is therefore into classes

\inset{(E) measures which are not $\tau$-additive,}

\inset{(F) measures which are $\tau$-additive, but not tight,}

\inset{(G) tight measures,}

\noindent and within these classes we can distinguish measures with no
extension to a Borel measure (type E$_0$), measures with more than one
extension to a Borel measure (types E$_1$, F$_1$ and G$_1$), measures
with exactly one extension to a Borel measure which is not completion
regular (types E$_2$, F$_2$ and G$_2$) and measures with an extension to
a completion regular Borel measure (types E$_3$, F$_3$ and G$_3$).   For
examples, see 439M and 439O (E$_0$), 439N (E$_2$), 439J
(E$_3$), 435Xc (F$_1$), 435Xd (F$_2$), 415Xc and 434Xa (F$_3$), 435Xa
(G$_1$), 435Xb (G$_2$) and the
restriction of Lebesgue measure to the Baire subsets of $\Bbb R$
(G$_3$);   other
examples may be constructed as direct sums of these.

A separate question we can ask of a Baire measure is whether it can be
extended to a Radon measure.   For this there is a straightforward
criterion (435B), which shows that (at least for totally finite measures
on completely regular spaces) only the types F$_1$ and F$_2$ are divided
by this question.   (If a Baire measure $\mu$ can be extended to a Radon
measure, it is surely
$\tau$-additive.   If $\mu$ is tight, it satisfies the criteria of 435B,
so has an extension to a Radon
measure.   If $\mu$ has an extension to a completion regular Borel
measure $\mu_1$ {\it and} has an extension to a Radon measure $\mu_2$,
then the completion $\hat\mu$ of $\mu$ extends $\mu_1$, while $\mu_2$
extends $\hat\mu$;   so $\mu_1$ is the restriction of $\mu_2$ to the
Borel sets and $\mu_2=\hat\mu_1=\hat\mu$ and $\mu$, like $\mu_2$, is
tight, by 412Hb or otherwise.
Thus no measure of type F$_3$ can be extended to a Radon measure.)

As with the classification of Borel measures that I offered in \S434,
any restriction on the topology of the underlying space may eliminate
some of these possibilities.   For instance, because a semi-finite Baire
measure is inner regular with respect to the closed sets, we can have no
(semi-finite) measure of classes E or F on a compact Hausdorff space.
On a locally compact Hausdorff space we can have no effectively locally
finite Baire measure of class F (435Xe), while on a
K-analytic Hausdorff space we can have no locally finite Baire
measure of class E (432F).   In a metrizable space, or a regular
space with a countable network (e.g., a regular analytic Hausdorff space),
the Baire and Borel $\sigma$-algebras coincide
(4A3Kb), so we can have no measures of type E$_0$, E$_1$,
F$_1$ or G$_1$.
}%end of comment

\leader{435B}{Theorem} Let $X$ be a Hausdorff space and $\mu$ a
locally finite Baire measure on $X$.   Then the following are
equiveridical:

\inset{(i) $\mu$ has an extension to a Radon measure on $X$;

(ii) for every non-negligible Baire set $E\subseteq X$ there is a
compact set $K\subseteq E$ such that $\mu^*K>0$.}

\noindent If $\mu$ is totally finite, we can add

\inset{(iii) $\sup\{\mu^*K:K\subseteq X$ is compact$\}=\mu X$.}

\proof{ Because $\mu$ is inner regular with respect to the closed sets
(412D), this is just a special case of 416P.
}%end of proof of 435B

\leader{435C}{Theorem}\cmmnt{ ({\smc \Marik\ 57})} Let $X$ be a
normal countably paracompact space.   Then any semi-finite Baire measure
on $X$ has an extension to a semi-finite Borel measure which is inner
regular with respect to the closed sets.

\proof{{\bf (a)} Let $\nu$ be a semi-finite Baire measure on $X$.   Let
$\Cal K$ be the family of those closed subsets of
$X$ which are included in zero sets of finite measure, and set
$\phi_0K=\nu^*K$ for $K\in\Cal K$.   Then $\Cal K$ and $\phi_0$ satisfy
the conditions of 413I, that is,

\inset{$\emptyset\in\Cal K$,}

\inset{($\dagger$) $K\cup K'\in\Cal K$ whenever $K$, $K'\in\Cal K$ are
disjoint,}

\inset{($\ddagger$) $\bigcap_{n\in\Bbb N}K_n\in\Cal K$ whenever
$\sequencen{K_n}$ is a sequence in $\Cal K$,}

\inset{($\alpha$)
$\phi_0K=\phi_0L+\sup\{\phi_0K':K'\in\Cal K,\,K'\subseteq K\setminus L\}$
whenever $K$, $L\in\Cal K$ and $L\subseteq K$,}

\inset{($\beta$) $\inf_{n\in\Bbb N}\phi_0K_n=0$ whenever
$\sequencen{K_n}$ is
a non-increasing sequence in $\Cal K$ with empty intersection.}

\Prf\ The first three are trivial.

\medskip

\quad\grheada\ Take $K$, $L\in\Cal K$ with $L\subseteq K$, and set
$\gamma=\sup\{\phi_0 K':K'\in\Cal K,\,K'\subseteq K\setminus L\}$.   (i)
If $K'\subseteq K\setminus L$ is closed, then (because $X$ is normal)
there is a zero set $F$ including $K'$ and disjoint from $L$
(4A2F(d-iv)), so

\Centerline{$\phi_0K'+\phi_0L
=\nu^*((K'\cup L)\cap F)+\nu^*((K'\cup L)\setminus F)
=\nu^*(K'\cup L)
\le\nu^*K$.}

\noindent As $K'$ is arbitrary, $\gamma+\phi_0L\le\phi_0K$.
(ii) Let $\epsilon>0$.   Let $F_0$ be a zero set of finite measure
including $K$.   Because $\nu$ is inner regular with respect to the zero
sets (412D), there is a zero set $F\subseteq F_0\setminus L$ such that
$\nu F\ge\nu_*(F_0\setminus L)-\epsilon$ (413Ee), so that
$\nu(F_0\setminus F)\le\nu^*L+\epsilon$ (413Ec).   Set $K'=K\cap F$.
Then

\Centerline{$\nu^*K
=\nu^*(K\setminus F)+\nu^*(K\cap F)
\le\nu(F_0\setminus F)+\nu^*K'
\le \nu^*L+\epsilon+\gamma$.}

\noindent As $\epsilon$ is arbitrary, $\nu^*K\le\nu^*L+\gamma$.

\medskip

\quad\grheadb\ If $\sequencen{K_n}$ is a non-increasing sequence in
$\Cal K$ with empty intersection, then (because $X$ is countably
paracompact)
there is a sequence $\sequencen{G_n}$ of open sets such that
$K_n\subseteq G_n$ for every $n$ and $\bigcap_{n\in\Bbb N}G_n=\emptyset$
(4A2Ff).   Because $X$ is normal, there are zero sets $F_n$ such
that $K_n\subseteq F_n\subseteq G_n$ for each $n$ (4A2F(d-iv) again), 
so that
$\bigcap_{n\in\Bbb N}F_n=\emptyset$.   We may suppose that $F_0$ has
finite measure.   In this case,

\Centerline{$\lim_{n\to\infty}\nu^*K_n
\le\lim_{n\to\infty}\nu(\bigcap_{i\le n}F_i)
=0$.}

\noindent Thus $\Cal K$ and $\phi_0$ satisfy the conditions ($\alpha$)
and ($\beta$) as well.\ \Qed

\medskip

{\bf (b)} By 413I, there is a complete locally determined measure $\mu$
on $X$, extending $\phi_0$ and inner regular with respect to $\Cal K$.
If $F\subseteq X$ is closed, then $F\cap K\in\Cal K$ for every
$K\in\Cal K$, so $F\in\dom\mu$ (413F(ii));  accordingly $\mu$ is a
topological measure, and because $\nu$ also is inner regular with
respect to $\Cal K$, $\mu$ must extend $\nu$.   So the restriction of
$\mu$ to the Borel
sets is a Borel extension of $\nu$ which is inner regular with respect
to the closed sets.
}%end of proof of 435C

\cmmnt{\medskip

\noindent{\bf Remark} If $X$ is normal, but not countably paracompact,
the result may fail;  see 439O.   I have stated the result in terms of
`countable paracompactness', but the formally distinct `countable
metacompactness' is also sufficient (435Ya).   If we are told that the
Baire measure is $\tau$-additive and effectively locally finite, we have
a much stronger result (415M).
}%end of comment

\leader{435D}{}\cmmnt{ Just as with the `Radon' spaces of \S434, we
can look at classes of topological spaces defined by the behaviour of
the Baire measures they carry.   The class which has aroused most
interest is the following.

\medskip

\noindent}{\bf Definition} A completely regular topological space $X$ is
{\bf measure-compact}\cmmnt{ (sometimes called {\bf almost Lindel\"of})} if
every totally finite Baire measure on $X$ is
$\tau$-additive\cmmnt{, that is, has an extension to a quasi-Radon
measure on $X$ (415N)}.

\leader{435E}{}\cmmnt{ The following lemma will make our path easier.

\medskip

\noindent}{\bf Lemma} Let $X$ be a completely regular topological space
and $\nu$ a totally finite Baire measure on $X$.   Suppose that
$\sup_{G\in\Cal G}\nu G=\nu X$ whenever $\Cal G$ is an upwards-directed
family of cozero sets with union $X$.   Then $\nu$ is $\tau$-additive.

\proof{ Let $\Cal G$ be an upwards-directed family of open Baire sets
such that $G^*=\bigcup\Cal G$ also is a Baire set, and $\epsilon>0$.
Because $\nu$ is inner regular with respect to the zero sets, there is a
zero set $F\subseteq G^*$ such that $\nu F\ge\nu G^*-\epsilon$.   Let
$\Cal G'$ be the family of cozero sets included in members of $\Cal G$;
because $X$ is completely regular, so that the cozero sets are a base
for its topology, $\bigcup\Cal G'=G^*$, and of course $\Cal G'$ is
upwards-directed.   Now

\Centerline{$\Cal H=\{G\cup(X\setminus F):G\in\Cal G'\}$}

\noindent is an upwards-directed family of cozero sets with union $X$,
so there is a $G_0\in\Cal G'$ such that
$\nu(G_0\cup(X\setminus F))\ge\nu X-\epsilon$.   In this case

\Centerline{$\sup_{G\in\Cal G}\nu G
\ge\nu G_0\ge\nu X-\epsilon-\nu(X\setminus F)=\nu F-\epsilon
\ge\nu G^*-2\epsilon$.}

\noindent As $\Cal G$ and $\epsilon$ are arbitrary, $\nu$ is
$\tau$-additive.
}%end of proof of 435E

\leader{435F}{Elementary facts (a)} If $X$ is a completely regular space
which is not measure-compact, there are a Baire probability measure
$\mu$ on $X$ and a
cover of $X$ by $\mu$-negligible cozero sets.   \prooflet{\Prf\ There is
a totally finite Baire measure $\nu$ on $X$ which is not
$\tau$-additive. By 435E, there is an upwards-directed family $\Cal G$
of cozero sets, covering $X$, such that $\sup_{G\in\Cal G}\nu G<\nu X$.
Let $\sequencen{G_n}$ be a sequence in $\Cal G$ such that
$\sup_{n\in\Bbb N}\nu G_n=\sup_{G\in\Cal G}\nu G$.   Then
$\gamma=\nu(X\setminus\bigcup_{n\in\Bbb N}G_n)>0$.   Set

\Centerline{$\mu H=\Bover1{\gamma}\nu(H\setminus\bigcup_{n\in\Bbb
N}G_n)$}

\noindent for Baire sets $H\subseteq X$;  then $\mu$ is a Baire
probability measure and $\Cal G$ is a cover of $X$ by $\mu$-negligible
cozero sets.\ \Qed
}%end of prooflet

\spheader 435Fb Regular Lindel\"of spaces are measure-compact.
\prooflet{(For if a Lindel\"of space can be covered by negligible open
sets, it can be covered by countably many negligible open sets, so is
itself negligible.)}   \cmmnt{In particular, compact Hausdorff spaces,
indeed all regular K-analytic
Hausdorff spaces (422Gg), are measure-compact.}

\cmmnt{Note that regular Lindel\"of spaces are normal and paracompact
(4A2H(b-i)), so their measure-{\vthsp}compactness is also a consequence
of 435C and 434Hb.}

\spheader 435Fc An open subset of a measure-compact space need not be
measure-compact\cmmnt{ (435Xi(i))}.   A continuous image of a
measure-compact space need not be measure-compact\cmmnt{ (435Xi(ii))}.
$\BbbN^{\frakc}$ is not measure-compact\cmmnt{ (439P)}.   The
product of two measure-compact spaces need not
be measure-compact\cmmnt{ (439Q)}.

\spheader 435Fd If $X$ is a measure-compact completely regular space it
is Borel-measure-compact.   \prooflet{\Prf\ Let $\mu$ be a non-zero
totally finite Borel measure on $X$ and $\Cal G$ an open cover of $X$.
Let $\nu$ be the restriction of $\mu$ to the Baire $\sigma$-algebra of
$X$, so that $\nu$ is $\tau$-additive.   Let $\Cal U$ be the set of
cozero sets $U\subseteq X$ included in members of $\Cal G$;  because the
family of cozero sets is a base for the topology of $X$,
$\bigcup\Cal U=X$, and there is some $U\in\Cal U$ such that $\nu U>0$.
This means that there is some $G\in\Cal G$ such that $\mu G>0$.   By
434H(a-v), $X$ is Borel-measure-compact.\ \Qed}

\leader{435G}{Proposition} A Souslin-F subset of a measure-compact
completely regular space is measure-compact.

\proof{{\bf (a)} Let $X$ be a measure-compact completely regular space,
$\family{\sigma}{S}{F_{\sigma}}$ a Souslin scheme consisting of closed
subsets of $X$ with kernel $A$, $\nu$ a totally finite Baire measure on
$A$, and $\Cal G$ an upwards-directed family of (relatively) cozero
subsets of $A$ covering $A$.   Let $\nu_1$ be the Baire measure on $X$
defined by setting $\nu_1H=\nu(A\cap H)$ for every Baire subset $H$ of
$X$.   Because $X$ is measure-compact, $\nu_1$ has an extension to a
quasi-Radon measure $\mu$ on $X$.   Let $\mu_A$ be the subspace measure
on $A$.

\medskip

{\bf (b)} By 431B, $A$ is measured by $\mu$.   In fact
$\mu A=\nu A$.  \Prf\ The construction of $\mu$ given in
415K-415N %415K 415L 415M 415N
ensures that
$\mu F=\nu_1^*F$ for every closed set $F$, and this is in any case a
consequence of the facts that $\mu$ is $\tau$-additive and $\dom\nu_1$
includes a base for the topology.   For each $\sigma\in S$, in
particular, $\mu F_{\sigma}=\nu_1^*F_{\sigma}$;  let
$F'_{\sigma}\supseteq F_{\sigma}$ be a Baire set such that
$\nu_1F'_{\sigma}=\nu_1^*F_{\sigma}$.   Then

\Centerline{$\mu F'_{\sigma}=\nu_1F'_{\sigma}=\nu_1^*F_{\sigma}
=\mu F_{\sigma}$}

\noindent and $\mu(F'_{\sigma}\setminus F_{\sigma})=0$ for every
$\sigma\in S$.   Let $A'$ be the kernel of the Souslin scheme
$\family{\sigma}{S}{F'_{\sigma}}$.   Then $A\subseteq A'$ and

\Centerline{$\mu(A'\setminus A)
\le\sum_{\sigma\in S}\mu(F'_{\sigma}\setminus F_{\sigma})=0$,}

\noindent so $\mu A=\mu A'$.   On the other hand, writing $\hat\nu_1$
for the completion of $\nu_1$, $A'$ is measured by $\hat\nu_1$, by
431A, so that (because $\mu$ extends $\nu_1$)

\Centerline{$\mu A=\mu A'=\mu^*A'\le\nu_1^*A'
=(\nu_1)_*A'\le\mu_*A'=\mu A'$.}

\noindent Thus $\mu A=\nu_1^*A'$.   But of course

\Centerline{$\nu A=\nu_1X=\nu_1^*A=\nu_1^*A'$,}

\noindent so that $\mu A=\nu A$.\ \Qed

Since we surely have

\Centerline{$\mu X=\nu_1X=\nu A$,}

\noindent we see that $\mu(X\setminus A)=0$.

\medskip

{\bf (c)} It follows that $\mu F=\nu F$ for every (relatively) zero set
$F\subseteq A$.   \Prf\ There is a closed set $F'\subseteq X$ such that
$F=A\cap F'$.   Now if $H\subseteq X$ is a Baire set including $F'$,
$H\cap A$ is a (relatively) Baire set including $F$, so $\nu
F\le\nu(H\cap A)=\nu_1H$;  as $H$ is arbitrary, $\nu F\le\nu_1^*F'$.
But $\nu_1^*F'=\mu F'$, as remarked in (b) above,
and $\mu(X\setminus A)=0$, so

\Centerline{$\mu F=\mu F'=\nu_1^*F'\ge\nu F$.}

\noindent On the other hand, $A\setminus F$ is (relatively) cozero, so
there is a non-decreasing sequence $\sequencen{F_n}$ of (relatively)
zero
subsets of $A$ with union $A\setminus F$, and

\Centerline{$\mu(A\setminus F)=\lim_{n\to\infty}\mu
F_n\ge\lim_{n\to\infty}\nu F_n=\nu(A\setminus F)$.}

\noindent Since we already know that $\mu A=\nu A$, it follows that

\Centerline{$\mu F=\mu A-\mu(A\setminus F)\le\nu A-\nu(A\setminus F)
=\nu F$,}

\noindent and $\mu F=\nu F$.\ \Qed

\medskip

{\bf (d)} The set

\Centerline{$\{E:E\subseteq A$ is a (relative) Baire set,
$\mu E=\nu E\}$}

\noindent therefore contains every (relatively) zero set, and by the
Monotone Class Theorem (136C) contains every (relatively) Baire set.
What this means is that $\mu$ actually extends $\nu$;  so the subspace
measure $\mu_A=\mu\restrp\Cal PA$ also extends $\nu$.   But $\mu_A$ is a
quasi-Radon measure (415B), therefore $\tau$-additive, and $\nu$ must
also be $\tau$-additive.
}%end of proof of 435G

\leader{435H}{Corollary} A Baire subset of a measure-compact completely
regular space is measure-compact.

\proof{ Put 435G and 421L together.
}%end of proof of 435H

\exercises{
\leader{435X}{Basic exercises $\pmb{>}$(a)}
%\spheader 435Xa
Give $\omega_1+1$ its order topology.   (i) Show that its  Baire
$\sigma$-algebra $\Sigma$ is just the family of sets
$E\subseteq\omega_1+1$ such that either $E$ or its complement is a
countable subset of $\omega_1$.   (ii) Show that there is a unique Baire
probability measure $\nu$ on $\omega_1+1$ such that $\nu\{\xi\}=0$
for every $\xi<\omega_1$.   (iii) Show that $\nu$ is $\tau$-additive.
(iv) Show that there is exactly one Radon measure on $\omega_1+1$
extending $\nu$, but that the measure $\mu$ of 434Xf is another Borel
measure also extending $\nu$.
%435A

\sqheader 435Xb Let $I$ be a set of cardinal $\omega_1$, endowed with
its discrete topology, and $X=I\cup\{\infty\}$ its one-point
compactification (3A3O).   Let $\mu$ be the Dirac measure on $X$
concentrated at $\infty$.   (i) Show that every subset of $X$ is
a Borel set.   (ii) Show that $\{\infty\}$ is not a zero set.   (iii)
Let $\nu$ be the restriction of $\mu$ to the Baire $\sigma$-algebra of
$X$.   Show that $\nu$ is tight.
Show that $\mu$ is the unique Borel measure extending $\nu$ ({\it
hint\/}:  you will need 419G), but is not completion regular.   (iv)
Show that the subspace measure $\nu_I$ on $I$ is the
countable-cocountable measure on $I$, and is not a Baire measure, nor
has any extension to a Baire measure on $I$.   (v) Show that $X$ is
measure-compact.
%435A

\spheader 435Xc On $\BbbR^{\omega_1}$ let $\mu$ be the Baire measure
defined by saying that $\mu E=1$ if $\chi\omega_1\in E$, $0$ otherwise.
(i) Show that $\mu$ is $\tau$-additive, but not tight.   \Hint{4A3P.}
(ii) Show that the
map $\xi\mapsto\chi\xi:\omega_1+1\to\BbbR^{\omega_1}$ is continuous, so
that $\mu$ has more than one extension to a Borel measure.   (iii) Show
that $\mu$ has an extension to a Radon measure.
%435A

\spheader 435Xd Set $X=\omega_1+1$ with the topology $\Cal
P\omega_1\cup\{X\setminus A:A\subseteq\omega_1$ is countable$\}$.   Let
$\mu$ be the Baire measure on $X$ defined by saying that, for Baire sets
$E\subseteq X$, $\mu E=1$ if $\omega_1\in E$, $0$ otherwise.   (i) Show
that a function $f:X\to\Bbb R$ is continuous iff $\{\xi:\xi\in
X,\,f(\xi)\ne f(\omega_1)\}$ is countable;  show that $X$ is completely
regular and Hausdorff.   (ii) Show that $\mu$ is
$\tau$-additive.   (iii) Show that every subset of $X$ is Borel.   (iv)
Show that the only Borel measure extending $\mu$ is the Dirac measure
concentrated at $\omega_1$, and that this is a Radon measure.   (v)
Show that all compact subsets of $X$ are finite, so that $\mu$ is not
tight.   *(vi) Show that $X$ is Lindel\"of.
%435A

\spheader 435Xe Let $X$ be a locally compact Hausdorff space and $\mu$
an effectively locally finite $\tau$-additive Baire measure on $X$.
Show that $\mu$ is tight.
\Hint{the relatively compact cozero sets cover $X$;  use 414Ea and
412D.}
%435A

\sqheader 435Xf Let $X$ be a completely regular space and $\mu$ a
totally finite $\tau$-additive Borel measure on $X$.   Let $\mu_0$ be
the restriction of $\mu$ to the Baire $\sigma$-algebra of $X$.   Show
that $\mu F=\mu_0^*F$ for every closed set $F\subseteq X$.
%435C

\spheader 435Xg Show that if a semi-finite Baire measure $\nu$ on a
normal countably paracompact space is extended to a Borel measure $\mu$
by the construction in 435C, then the measure algebra of $\nu$ becomes
embedded as an order-dense subalgebra of the measure algebra of $\mu$,
so that $L^1(\mu)$ can be identified with $L^1(\nu)$.
%435C

\spheader 435Xh Show
that a Borel-measure-compact normal countably paracompact
space is measure-compact.
%435F, 435C

\spheader 435Xi(i) Show that $\omega_1+1$ is measure-compact, in its
order topology, but that its open subset $\omega_1$ is not (cf.\ 434Xl).
(ii) Show
that a discrete space of cardinal $\omega_1$ is measure-compact, but
that it has a continuous image which is not measure-compact.
%435G

\spheader 435Xj Let $X$ be a metacompact completely regular space and
$\nu$ a totally finite strictly positive Baire measure on $X$.   Show
that $X$ is Lindel\"of, so that $\nu$ has an extension to a quasi-Radon
measure on $X$.   \Hint{if $\Cal H$ is a point-finite open cover of $X$,
not containing $\emptyset$, then for each $H\in\Cal H$ choose a
non-empty cozero set $G_H\subseteq H$;  show that
$\{H:\nu G_H\ge\delta\}$ is finite for every $\delta>0$.}
%435H

\spheader 435Xk A completely regular space $X$ is {\bf strongly
measure-compact} ({\smc Moran 69}) if
$\mu X=\sup\{\mu^*K:K\subseteq X$ is compact$\}$ for every totally
finite Baire measure $\mu$ on $X$.   (i) Show that a
completely regular Hausdorff space $X$ is strongly measure-compact iff
every totally finite Baire measure on $X$ has an extension to a Radon
measure iff $X$ is measure-compact and pre-Radon.   (ii) Show that a Souslin-F subset of a
strongly measure-compact completely regular space is strongly
measure-compact.   (iii) Show that a discrete space of cardinal
$\omega_1$ is strongly measure-compact.   (iv) Show that a countable
product of strongly measure-compact completely regular spaces is
strongly measure-compact.   (v) Show that $\BbbN^{\omega_1}$ is not
strongly measure-compact.   \Hint{take a non-trivial probability measure
on $\Bbb N$ and consider its power on $\BbbN^{\omega_1}$.}  (vi) Show
that if $X$ and $Y$ are completely regular
spaces, $X$ is measure-compact and $Y$ is strongly measure-compact then
$X\times Y$ is measure-compact.
%435+

\spheader 435Xl (T.D.Austin) Let $X$ be a topological space, $\mu$ an
atomless Baire probability measure on $X$ and $\hat\mu$ its completion.
Show that there is a continuous function $f:X\to[0,1]$ which is \imp\ for
$\hat\mu$ and Lebesgue measure on $[0,1]$.
\Hint{Check the case $X=[0,1]$ first.   For the general case,
let $Z$ be the set of continuous functions from $X$ to
$[0,1]$ with the complete metric induced by $\|\,\|_{\infty}$, and set
$\alpha(f)=\max\{\mu f^{-1}[\{t\}]:t\in[0,1]\}$ for $f\in Z$.
Show that $\interior\{f:\alpha(f)\le\epsilon\}$ is dense in
$Z$ for every $\epsilon>0$, so that there is an $f\in Z$ such that
$\mu f^{-1}$ is atomless.}
%435+

\spheader 435Xm Let $X$ be a normal space and $\mu$ a
complete $\sigma$-finite
topological probability measure on $X$ which is inner regular
with respect to the closed sets.
(i) Let $\nu$ be the restriction of $\mu$ to the Baire $\sigma$-algebra
of $X$.    Show that $\mu$ and $\nu$ have isomorphic measure algebras.
(ii) Show that if $\mu$ is an atomless probability measure there is a
continuous $f:X\to[0,1]$ which is \imp\ for $\mu$ and Lebesgue measure.
%435Xl

\spheader 435Xn Let $X$ be a topological space and $\Cal G$ the family of
cozero sets in $X$.
Show that a functional $\psi:\Cal G\to\coint{0,\infty}$ can
be extended to a Baire measure on $X$ iff $\psi$ is modular (definition:
413Xq) and
$\lim_{n\to\infty}\psi G_n=0$ whenever $\sequencen{G_n}$ is a
non-increasing sequence in $\Cal G$ with empty intersection.   \Hint{if
$\psi$ satisfies the conditions, first check that $\psi\emptyset=0$ and
that $\psi G\le\psi H$ whenever $G\subseteq H$;  now apply 413I with
$\phi K=\inf\{\psi G:K\subseteq G\in\Cal G\}$ for zero sets $K$.}
%435+

\spheader 435Xo Let $X$ be a countably compact topological space and $\mu$
a totally finite Baire measure on $X$.   Show that $\mu$ has an extension
to a Borel measure which is inner regular with respect to the closed sets.
\Hint{413O.}
%435F

\leader{435Y}{Further exercises (a)}
%\spheader 435Ya
Show that a normal countably metacompact space (434Yn)
is countably paracompact.
%435C

\spheader 435Yb
Let $X$ be a completely regular Hausdorff space and $\beta X$ its
Stone-\v Cech
compactification.   Show that $X$ is measure-compact iff whenever $\nu$
is a Radon measure on $\beta X$ such that $\nu X=0$, there is a
$\nu$-negligible Baire subset of $\beta X$ including $X$.
%435D

\leaveitout{Let $X$ be a completely regular Hausdorff space and
$\beta X$ its Stone-\v Cech
compactification.   Show that $X$ is measure-compact iff whenever
$K\subseteq\beta X\setminus X$ is a compact set which is the support of
a Radon measure, then there is a zero set $H$ of $\beta X$ such
that $K\subseteq H\subseteq\beta X\setminus X$.
}%is this true?

}%end of exercises

\endnotes{
\Notesheader{435} The principal reason for studying Baire measures is
actually outside the main line of this chapter.   For a completely
regular Hausdorff space $X$, write $C_b(X)$ for the $M$-space of bounded
continuous real-valued functions on $X$.   Then
$C_b(X)^*=C_b(X)^{\sim}$ is an $L$-space\cmmnt{ (356N)}, and inside
$C_b(X)^*$ we have the bands generated by the tight, smooth and
sequentially smooth
functionals (see 437A and 437F below), all identifiable, if we choose, 
with spaces
of `signed Baire measures'.   {\smc Wheeler 83} argues convincingly that
for the questions a functional analyst naturally asks, these Baire
measures are often an effective aid.

From the point of view of the arguments in this section, the most
fundamental difference between `Baire' and `Borel'
measures lies in their action on subspaces.   If $X$ is a topological
space and $A$ is a subset of $X$, then any Borel or Baire measure $\mu$
on $A$ provides us with a measure $\mu_1$ of the same type on $X$,
setting $\mu_1E=\mu(A\cap E)$ for the appropriate sets $E$.   In the
other direction, if $\mu$ is a Borel measure on $X$, then the subspace
measure $\mu_A$ is a Borel measure on $A$, because the Borel
$\sigma$-algebra of
$A$ is just the subspace $\sigma$-algebra derived from the Borel algebra
of $X$ (4A3Ca).   But if $\mu$ is a Baire measure on $X$, it does
not follow that $\mu_A$ is a Baire measure on $A$;  this is because (in
general) not every continuous function $f:A\to[0,1]$ has a continuous
extension to $X$, so that not every zero set in $A$ is the intersection
of $A$ with a zero set in $X$ (see 435Xb).   The analysis of those pairs
$(X,A)$ for which the Baire $\sigma$-algebra of $A$ is just the subspace
algebra derived from the Baire sets in $X$ is a challenging problem in
general topology which I pass by here.   For the moment I note only that
avoiding it is the principal technical problem in the proof of 435G.

I do not know if I ought to apologise for
`countably tight' spaces (434N),
`first-countable' spaces (434R),
`metacompact' spaces (438J),
`normal countably paracompact' spaces (435C),
`quasi-dyadic' spaces (434O) and
`sequential' spaces (436F).
General topology is notorious for invoking
arcane terminology to stretch arguments to their utmost limit of
generality, and even specialists may find their patience tried by
definitions which seem to have only one theorem each.   In 438J, for
instance, it is obvious that the original result concerned metrizable
spaces (438H), and you may well feel at first that the extension is a
baroque over-elaboration.   On the other hand, there
are (if you look for them) some very interesting metacompact spaces
({\smc Engelking 89}, \S5.3), and metacompactness has taken its
place in the standard lists.   In this book I try to follow a rule of
introducing a class of topological spaces only when it is both genuinely
interesting, from the point of view of general topology, and
also a support for an idea which
is interesting from the point of view of measure theory.
}%end of notes

\discrpage

