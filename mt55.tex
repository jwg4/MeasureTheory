\frfilename{mt55.tex}
\versiondate{7.1.15}
\copyrightdate{2007}

\def\chaptername{Possible worlds}

\newchapter{55}

In my original plans for this volume, I hoped to cover the most important
consistency proofs relating to undecidable questions in measure theory.
Unhappily my ignorance of forcing means that for the majority of results I
have nothing useful to offer.   I have therefore restricted my account to
the very narrow range of ideas in which I feel I have achieved
some understanding beyond what I have read in the standard texts.

For a measure theorist, by far the most important forcings are those
of `adding random reals'.   I give three sections (\S\S552-553 and 555)
to these.
Without great difficulty, we can determine the behaviour of the cardinals
in Cicho\'n's diagram (552B, 552C, 552F-552I), %552F 552G 552H 552I
at least if many random reals are
added.   Going deeper, there are things to be said about outer
measure and Sierpi\'nski sets (552D, 552E),
and extensions of Radon measures
(552N).   In the same section I give a version of the fundamental
result that simple iteration of random real forcings gives random
real forcings (552P).   In \S553 I collect results which are connected with
other topics dealt with above (Rothberger's property, precalibers,
ultrafilters, cellularity,
trees, medial limits, universally measurable sets) and in
which the arguments seem to me to develop properties of measure algebras
which may be of independent interest.
In preparation for this work, and also for \S554, I start with a section
(\S551) devoted to a rather technical
general account of forcings with quotients of
$\sigma$-algebras of sets, aiming to find effective representations of
names for points, sets, functions, measure algebras and filters.

Very similar ideas can also take us a long way with Cohen real forcing
(\S554).   Here I give little more than obvious parallels to the first part
of \S552, with an account of Freese-Nation numbers sufficient to support
Carlson's theorem that a Borel lifting for Lebesgue measure can exist when
the continuum hypothesis is false (554I).

One of the most remarkable applications of random reals is in Solovay's
proof that if it is consistent to suppose that there is a
two-valued-measurable cardinal, then it is consistent to suppose that there
is an atomlessly-measurable cardinal (555D).   By taking a bit of trouble
over the lemmas, we can get a good deal more, including the corresponding
theorem relating supercompact cardinals to the normal measure axiom (555N);
and similar techniques show the possibility of interesting \pssqa s
(555G, 555K).

I end the chapter with something quite different (\S556).
A familiar phenomenon
in ergodic theory is that once one has proved a theorem for ergodic
transformations one can expect, possibly at the cost of substantial effort,
but without having to find any really new idea, a corresponding result
for general measure-preserving transformations.
There is more than one way to look at this, but here I present a method in
which the key step, in each application,
is an appeal to the main theorem of forcing.   A similar approach gives a
description of the completion of the asymptotic density algebra.   The
technical details take up a good deal of space, but are based on the same
principles as those in \S551, and are essentially straightforward.

\discrpage

