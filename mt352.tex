\frfilename{mt352.tex}
\versiondate{22.9.03}
\copyrightdate{1998}
     
\def\chaptername{Riesz spaces}
\def\sectionname{Riesz spaces}
\def\pols{partially ordered linear space}
\def\med{\mathop{\text{med}}}
     
\newsection{352}
     
In this section I sketch those fragments of the theory we need which can
be expressed as theorems about general Riesz spaces or vector lattices.
I begin with the definition (352A) and most elementary properties
(352C-352F).
In 352G-352J I discuss Riesz homomorphisms and the associated subspaces
(Riesz subspaces, solid linear subspaces);  I mention product spaces
(352K, 352T) and quotient spaces (352Jb, 352U) and the form the
representation theorem 351Q takes in the present context
(352L-352M).   Most of the second half of the section
concerns the theory of `bands' in Riesz spaces, with the algebras of
complemented bands (352Q) and projection bands (352S) and a description
of bands generated by upwards-directed sets (352V).   I conclude with a
description of `$f$-algebras' (352W).
     
\leader{352A}{}\cmmnt{ I repeat a definition from 241E.
     
\medskip
     
\noindent}{\bf Definition} A {\bf Riesz space} or {\bf vector lattice}
is a partially ordered linear space which is a lattice.
     
\leader{352B}{Lemma} If $U$ is a \pols, then it is a Riesz space iff
$\sup\{0,u\}$ is defined for every $u\in U$.
     
\proof{ If $U$ is a lattice, then of course $\sup\{u,0\}$ is defined for
every $u$.   If $\sup\{u,0\}$ is defined for every $u$, and $v_1$, $v_2$
are any two members of $U$, consider $w=v_1+\sup\{0,v_2-v_1\}$;  by
351Db, $w=\sup\{v_1,v_2\}$.   Next,
     
\Centerline{$\inf\{v_1,v_2\}=-\sup\{-v_1,-v_2\}$}
     
\noindent must also be defined in $U$;  as $v_1$ and $v_2$ are
arbitrary, $U$ is a lattice.
}%end of proof of 352B
     
\leader{352C}{Notation} In any Riesz space $U$ I will write
     
\Centerline{$u^+=u\vee 0$,\quad $u^-=(-u)\vee 0=(-u)^+$,\quad
$|u|=u\vee(-u)$}
     
\noindent where (as in any lattice) $u\vee v=\sup\{u,v\}$ (and $u\wedge
v=\inf\{u,v\}$).
     
\dvro{A}{I mention immediately a term which will be useful:  a} family
$\familyiI{u_i}$ in $U$ is {\bf disjoint} if
$|u_i|\wedge|u_j|=0$ for all distinct $i$, $j\in I$.   Similarly, a set
$C\subseteq U$ is {\bf disjoint} if $|u|\wedge|v|=0$ for all distinct
$u$, $v\in C$.
     
\leader{352D}{Elementary identities} Let $U$ be a Riesz space.
\cmmnt{ The translation-invariance of the order, and its invariance
under positive scalar multiplication, reversal under negative
multiplication, lead directly to the following, which are in effect
special cases of 351D:}
     
\Centerline{$u+(v\vee w)=(u+v)\vee(u+w)$,
\quad$u+(v\wedge w)=(u+v)\wedge(u+w)$,}
     
\Centerline{$\alpha(u\vee v)=\alpha u\vee\alpha v$,\quad
$\alpha(u\wedge v)=\alpha u\wedge\alpha v$ if $\alpha\ge 0$,}
     
\Centerline{$-(u\vee v)=(-u)\wedge(-v)$.}
     
\noindent Combining and elaborating on these facts, we get
     
\Centerline{$u^+-u^-
\prooflet{\mskip5mu =(u\vee 0)-((-u)\vee 0)
=u+(0\vee(-u))-((-u)\vee 0)}
=u$,}
     
\Centerline{$u^++u^-
\prooflet{\mskip5mu =2u^+-u
=(2u\vee 0)-u
=u\vee(-u)}
=|u|$,}
     
\Centerline{$u\ge 0
\iff -u\le 0
\iff u^-=0
\iff u=u^+
\iff u=|u|$,}
     
\Centerline{$|-u|=|u|$,\quad$|\,|u|\,|=|u|$,\quad$|\alpha u|
=|\alpha||u|$\dvro{,}{}}
     
\prooflet{\hfill (looking at the cases $\alpha\ge 0$, $\alpha\le 0$
separately),}
     
\ifwithproofs{
$$\eqalign{u\vee v+u\wedge v
&=u+(0\vee(v-u))+v+((u-v)\wedge 0)\cr
&=u+(0\vee(v-u))+v-((v-u)\vee 0)
=u+v,\cr}$$
}\else{
$$\eqalign{u\vee v+u\wedge v
=u+v,\cr}$$
}\fi
     
\Centerline{$u\vee v
\prooflet{\mskip5mu =u+(0\vee(v-u))}
=u+(v-u)^+$,}
     
\Centerline{$u\wedge v
\prooflet{\mskip5mu =u+(0\wedge(v-u))
=u-(-0\vee(u-v))}
=u-(u-v)^+$,}
     
\Centerline{$u\vee v
\prooflet{\mskip5mu =\Bover12(2u\vee 2v)
=\Bover12(u+v+(u-v)\vee(v-u))}
=\Bover12(u+v+|u-v|)$,}
     
\Centerline{$u\wedge v
\prooflet{\mskip5mu =u+v-u\vee v}
=\Bover12(u+v-|u-v|)$,}
     
\Centerline{$u^+\vee u^-
\prooflet{\mskip5mu =u\vee(-u)\vee 0}
=|u|$,
\quad$u^+\wedge u^-
\prooflet{\mskip5mu =u^++u^--(u^+\vee u^-)}
=0$,}
     
\ifwithproofs
\Centerline{$|u+v|=(u+v)\wedge((-u)+(-v))\le(|u|+|v|)\wedge(|u|+|v|)
=|u|+|v|$,}
\else
\Centerline{$|u+v|\le|u|+|v|$,}
\fi
     
\ifwithproofs
\Centerline{$||u|-|v||=(|u|-|v|)\vee(|v|-|u|)
\le|u-v|+|v-u|=|u-v|$,}
\else
\Centerline{$||u|-|v||\le|u-v|$,}
\fi
     
\Centerline{$|u\vee v|\le|u|+|v|$}
     
\prooflet{\hfill (because
$-|u|\le u\vee v\le|u|\vee|v|\le|u|+|v|$)}
     
\noindent for $u$, $v$, $w\in U$ and $\alpha\in\Bbb R$.
     
\leader{352E}{Distributive laws} Let $U$ be a Riesz space.
     
\spheader 352Ea If $A$, $B\subseteq U$ have suprema $a$, $b$ in $U$,
then $C=\{u\wedge v:u\in A,\,v\in B\}$ has supremum $a\wedge b$.
\prooflet{\Prf\ Of course $u\wedge v\le a\wedge b$ for all $u\in A$,
$v\in B$, so $a\wedge b$ is an upper bound for $C$.   Now suppose that
$c$ is any upper bound for $C$.   If $u\in A$ and $v\in B$ then
     
\Centerline{$u-(u-v)^+=u\wedge v\le c$,
\quad$u\le c+(u-v)^+\le c+(a-v)^+$}
     
\noindent (because $(a-v)^+=\sup\{a-v,0\}\ge\sup\{u-v,0\}=(u-v)^+$).
As $u$ is arbitrary, $a\le c+(a-v)^+$ and $a\wedge v\le c$.   Now turn
the argument round:
     
\Centerline{$v=(a\wedge v)+(v-a)^+\le c+(v-a)^+\le c+(b-a)^+$,}
     
\noindent and this is true for every $v\in B$, so $b\le c+(b-a)^+$ and
$a\wedge b\le c$.   As $c$ is arbitrary, $a\wedge b=\sup C$, as
claimed.\ \Qed\ }
     
\spheader 352Eb\cmmnt{ Similarly, or applying (a) to $-A$ and $-B$,}
$\inf\{u\vee v:u\in A,\,v\in B\}=\inf A\vee\inf B$ whenever $A$,
$B\subseteq U$ and the right-hand-side is defined.
     
\spheader 352Ec\cmmnt{ In particular,} $U$ is a distributive
lattice\cmmnt{ (definition:  3A1Ic)}.
     
\leader{352F}{Further identities
and \dvrocolon{inequalities}}\cmmnt{ At a
slightly deeper level we have the following facts.
     
\medskip
     
\noindent}{\bf Proposition} Let $U$ be a Riesz space.
     
(a) If $u$, $v$, $w\ge 0$ in $U$ then 
$u\wedge(v+w)\le (u\wedge v)+(u\wedge w)$.
     
(b) If $u_0,\ldots,u_n\in U$ are disjoint, then
$|\sum_{i=0}^n\alpha_i u_i|=\sum_{i=0}^n|\alpha_i||u_i|$ for any
$\alpha_0,\ldots,\alpha_n\in\Bbb R$.
     
(c) If $u$, $v\in U$ then
     
\Centerline{$u^+\wedge v^+\le(u+v)^+\le u^++v^+$.}
     
(d) If $u_0,\ldots,u_m,v_0,\ldots,v_n\in U^+$ and
$\sum_{i=0}^mu_i=\sum_{j=0}^nv_j$, then there is a family $\langle
w_{ij}\rangle_{i\le m,j\le n}$ in $U^+$ such that
$\sum_{i=0}^mw_{ij}=v_j$ for every $j\le n$ and $\sum_{j=0}^nw_{ij}=u_i$
for every $i\le m$.
     
\proof{{\bf (a)}
     
$$\eqalign{u\wedge(v+w)
&\le[(u+w)\wedge(v+w)]\wedge u\cr
&\le[(u\wedge v)+w]\wedge[(u\wedge v)+u]
=(u\wedge v)+(u\wedge w).\cr}$$
     
\medskip
     
{\bf (b)(i)}\grheada\ A simple induction, using (a) for the inductive
step,  shows that if $v_0,\ldots,v_m$, $w_0,\ldots,w_n$ are non-negative
then
$\sum_{i=0}^mv_i\wedge\sum_{j=0}^nw_j
\le\sum_{i=0}^m\sum_{j=0}^nv_i\wedge w_j$.
\grheadb\ Next, if $u\wedge v=0$ then
     
\Centerline{$(u-v)^+=u-(u\wedge v)=u$,\quad$(u-v)^-=(v-u)^+=v-(v\wedge
u)=v$,}
     
\Centerline{$|u-v|=(u-v)^++(u-v)^-=u+v=|u+v|$,}
     
\noindent so if $|u|\wedge|v|=0$ then
     
$$\eqalign{(u^++v^+)\wedge(u^-+v^-)
&\le (u^+\wedge u^-)+(u^+\wedge v^-)+(v^+\wedge u^-)+(v^+\wedge v^-)\cr
&\le 0+(|u|\wedge|v|)+(|v|\wedge|u|)+0=0\cr}$$
     
\noindent and
     
\Centerline{$|u+v|=|(u^++v^+)-(u^-+v^-)|=u^++v^++u^-+v^-=|u|+|v|$.}
     
\noindent \grheadc\ Finally, if $|u|\wedge|v|=0$ and $\alpha$,
$\beta\in\Bbb R$,
     
\Centerline{$|\alpha u|\wedge|\beta v|
=|\alpha||u|\wedge|\beta||v|
\le(|\alpha|+|\beta|)|u|\wedge(|\alpha|+|\beta|)|v|
=(|\alpha|+|\beta|)(|u|\wedge|v|)
=0$.}
     
\medskip
     
\quad{\bf (ii)} We may therefore proceed by induction.   The case $n=0$
is trivial.   For the inductive step to $n+1$, setting
$u'_i=\alpha_iu_i$ we have $|u'_i|\wedge|u'_j|=0$ for all $i\ne j$, by
(i-$\gamma$).   By (i-$\alpha$),
     
\Centerline{$|u'_{n+1}|\wedge|\sum_{i=0}^nu'_i|
\le|u'_{n+1}|\wedge\sum_{i=0}^n|u'_i|
\le\sum_{i=0}^n|u'_{n+1}|\wedge|u'_i|=0$,}
     
\noindent so by (i-$\beta$) and the inductive hypothesis
     
\Centerline{$|\sum_{i=0}^{n+1}u'_i|
=|u'_{n+1}|+|\sum_{i=0}^nu'_i|
=\sum_{i=0}^{n+1}|u'_i|$}
     
\noindent as required.
     
\medskip
     
{\bf (c)} By 352E,
     
\Centerline{$u^+\wedge v^+=(u\vee 0)\wedge(v\vee 0)=(u\wedge v)\vee 0$.}
     
\noindent Now
     
\Centerline{$u\wedge v=\Bover12(u+v-|u-v|)\le\Bover12(u+v+|u+v|)
=(u+v)^+$,}
     
\noindent and of course $0\le(u+v)^+$, so $u^+\wedge v^+\le(u+v)^+$.
     
For the other inequality we need only note that $u+v\le u^++v^+$
(because $u\le u^+$, $v\le v^+$) and $0\le u^++v^+$.
     
\medskip
     
{\bf (d)} Write $w$ for the common value of $\sum_{i=0}^mu_i$ and
$\sum_{j=0}^nv_j$.
     
Induce on
$k=\#(\{(i,j):i\le m,\,j\le n,\,u_i\wedge v_j>0\})$.   If $k=0$, that
is, $u_i\wedge v_j=0$ for all $i$, $j$, then (by (a), used repeatedly)
we must have $w\wedge
w=0$, that is, $w=0$, and we can take $w_{ij}=0$ for all $i$, $j$.   For
the inductive step to $k\ge 1$, take $i^*$, $j^*$ such that
$w^*=u_{i^*}\wedge v_{j^*}>0$.   Set
     
\Centerline{$\tilde u_{i^*}=u_{i^*}-w^*$,\quad $\tilde u_i=u_i$ for
$i\ne i^*$,}
     
\Centerline{$\tilde v_{j^*}=v_{j^*}-w^*$,\quad $\tilde v_j=v_j$ for
$j\ne j^*$.}
     
\noindent Then
$\sum_{i=0}^m\tilde u_i=\sum_{j=0}^n\tilde v_j=w-w^*$ and
$\tilde u_i\wedge\tilde v_j\le u_i\wedge v_j$ for all $i$, $j$,
while $\tilde u_{i^*}\wedge\tilde v_{j^*}=0$;  so that
     
\Centerline{$\#(\{(i,j):\tilde u_i\wedge\tilde v_j>0\})<k$.}
     
\noindent By the inductive hypothesis, there are $\tilde w_{ij}\ge 0$,
for $i\le m$ and $j\le n$, such that
$\tilde u_i=\sum_{j=0}^n\tilde w_{ij}$ for each $i$ and
$\tilde v_j=\sum_{i=0}^m\tilde w_{ij}$ for each
$j$.   Set $w_{i^*j^*}=\tilde w_{i^*j^*}+w^*$,
$w_{ij}=\tilde w_{ij}$ for
$(i,j)\ne(i^*,j^*)$;  then $u_i=\sum_{j=0}^nw_{ij}$ and
$v_j=\sum_{i=0}^mw_{ij}$, so the induction proceeds.
}%end of proof of 352F
     
\leader{352G}{Riesz homomorphisms:  Proposition} Let $U$ be a Riesz
space, $V$ a \pols\ and $T:U\to V$ a linear operator.   Then the
following
are equiveridical:
     
(i) $T$ is a Riesz homomorphism in the sense of 351H;
     
(ii) $(Tu)^+=\sup\{Tu,0\}$ is defined and equal to $T(u^+)$ for every
$u\in U$;
     
(iii) $\sup\{Tu,-Tu\}$ is defined and equal to $T|u|$ for every 
$u\in U$;
     
(iv) $\inf\{Tu,Tv\}=0$ in $V$ whenever $u\wedge v=0$ in $U$.
     
\proof{ (i)$\Rightarrow$(iii) and (i)$\Rightarrow$(iv) are special cases
of 351Hc.   For
(iii)$\Rightarrow$(ii) we have
     
\Centerline{$\sup\{Tu,0\}=\Bover12Tu+\sup\{\Bover12Tu,-\Bover12Tu\}
=\Bover12Tu+\Bover12T|u|=T(u^+)$.}
     
     
\noindent For (ii)$\Rightarrow$(i), argue as follows.   If (ii) is true
and $u$, $v\in U$, then
     
\Centerline{$Tu\wedge Tv
=\inf\{Tu,Tv\}
=Tu+\inf\{0,Tv-Tu\}
=Tu-\sup\{0,T(u-v)\}$}
     
\noindent is defined and equal to
     
     
\Centerline{$Tu-T((u-v)^+)
=T(u-(u-v)^+)
=T(u\wedge v)$.}
     
\noindent Inducing on $n$,
     
\Centerline{$\inf_{i\le n}Tu_i=T(\inf_{i\le n}u_i)$}
     
\noindent for all $u_0,\ldots,u_n\in U$;  in particular, if $\inf_{i\le
n}u_i=0$ then $\inf_{i\le n}Tu_i=0$;  which is the definition I gave of
Riesz homomorphism.
     
Finally, for (iv)$\Rightarrow$(ii), we know from (iv) that
$0=\inf\{T(u^+),T(u^-)\}$, so $-T(u^+)=\inf\{0,-Tu\}$ and
$T(u^+)=\sup\{0,Tu\}$.
}%end of proof of 352G
     
\leader{352H}{Proposition} If $U$ and $V$ are Riesz spaces and 
$T:U\to V$ is a bijective Riesz homomorphism, then $T$ is a
partially-ordered-linear-space isomorphism, and $T^{-1}:V\to U$ is a
Riesz homomorphism.
     
\proof{ Use 352G(ii).  If $v\in V$, set $u=T^{-1}v$;  then $T(u^+)=v^+$
so $T^{-1}(v^+)=u^+=(T^{-1}v)^+$.   Thus $T^{-1}$ is a Riesz
homomorphism;  in particular, it is order-preserving, so $T$ is an
isomorphism for the order structures as well as for the linear
structures.
}%end of proof of 352H
     
\leader{352I}{Riesz subspaces (a)} If $U$ is a \pols, a {\bf Riesz
subspace} of $U$ is a linear subspace $V$ such that $\sup\{u,v\}$ and
$\inf\{u,v\}$ are defined in $U$ and belong to $V$ for every $u$,
$v\in V$.   In this case\cmmnt{ they are the supremum and infimum of
$\{u,v\}$
in $V$, so} $V$, with the induced order and linear structure, is a Riesz
space in its own right, and the embedding map $u\mapsto u:V\to U$ is a
Riesz homomorphism.
     
\spheader 352Ib Generally, if $U$ is a Riesz space, $V$ is a \pols\ and
$T:U\to V$ is a Riesz homomorphism, then $T[U]$ is a Riesz subspace of
$V$\prooflet{ (because, by 351Hc, $Tu\vee Tu'=T(u\vee u')$,
$T(u\wedge u')=Tu\wedge Tu'$ belong to $T[U]$ for all $u$, $u'\in U$)}.
     
\spheader 352Ic If $U$ is a Riesz space and $V$ is a linear subspace of
$U$, then $V$ is a Riesz subspace of $U$ iff $|u|\in V$ for every
$u\in V$.   \prooflet{\Prf\ In this case,
     
\Centerline{$u\vee v=\Bover12(u+v+|u-v|)$,
\quad$u\wedge v=\Bover12(u+v-|u-v|)$}
     
\noindent belong to $V$ for all $u$, $v\in V$.\ \Qed}
     
\leader{352J}{Solid subsets (a)} If $U$ is a Riesz space, a
subset $A$ of $U$ is solid\cmmnt{ (in the sense of 351I)} iff
$v\in A$ whenever $u\in A$ and $|v|\le|u|$.   \prooflet{\Prf\ ($\alpha$)
If $A$
is solid, $u\in V$ and $|v|\le|u|$, then there is some $w\in A$ such
that $-w\le u\le w$;  in this case $|v|\le|u|\le w$ and $-w\le v\le w$
and $v\in A$.   ($\beta$) Suppose that $A$ satisfies the condition.   If
$u\in A$, then $|u|\in A$ and $-|u|\le u\le|u|$.   If $w\in A$ and
$-w\le u\le w$ then $-u\le w$, $|u|\le w=|w|$ and $u\in A$.   Thus $A$
is solid.\ \Qed\ }   In particular, if $A$ is
solid, then $v\in A$ iff $|v|\in A$.
     
For any set $A\subseteq U$, the set
     
\Centerline{$\{u:$ there is some $v\in A$ such that $|u|\le|v|\}$}
     
\noindent is a solid subset of $U$, the smallest solid set including
$A$;  we call it the {\bf solid hull} of $A$ in $U$.
     
Any solid linear subspace of $U$ is a Riesz subspace\prooflet{ (use
352Fc)}.   If $V\subseteq U$ is a Riesz subspace, 
then the solid hull of $V$ in $U$ is
     
\Centerline{$\{u:$ there is some $v\in V$ such that $|u|\le v\}$}
     
\noindent and is a solid linear subspace of $U$.
     
\spheader 352Jb If $T$ is a Riesz homomorphism from a Riesz space $U$ to
a \pols\ $V$, then its kernel $W$ is a solid linear subspace of $U$.
\prooflet{\Prf\ If $u\in W$ and $|v|\le|u|$, then
$T|u|=\sup\{Tu,T(-u)\}=0$, while $-|u|\le v\le |u|$, so that $-0\le
Tv\le 0$ and $v\in W$.\ \Qed\ }   Now the quotient space $U/W$\cmmnt{,
as defined in 351J,} is a Riesz space, and is isomorphic, as \pols, to
the Riesz space $T[U]$.   \prooflet{\Prf\ Because $U/W$ is the linear
space quotient of $V$ by the kernel of
the linear operator $T$, we have an induced linear space isomorphism
$S:U/W\to T[U]$ given by setting $Su^{\ssbullet}=Tu$ for every $u\in U$.
If $p\ge 0$ in $U/W$ there is a $u\in U^+$ such that $u^{\ssbullet}=p$
(351J), so that $Sp=Tu\ge 0$.   On the other hand, if $p\in U/W$ and
$Sp\ge 0$, take $u\in U$ such that $u^{\ssbullet}=p$.   We have
     
\Centerline{$T(u^+)=(Tu)^+=(Sp)^+=Sp=Tu$,}
     
\noindent so that $T(u^-)=Tu^+-Tu=0$ and $u^-\in W$,
$p=(u^+)^{\ssbullet}\ge 0$.   Thus $Sp\ge 0$ iff $p\ge 0$, and $S$ is a
partially-ordered-linear-space isomorphism.\ \Qed\ }
     
\cmmnt{
\spheader 352Jc Because a subset of a Riesz space is a solid linear
subspace iff it is the kernel of a Riesz homomorphism (see 352U below), such subspaces are sometimes called {\bf ideals}.
}%end of comment
     
\leader{352K}{Products} If $\langle U_i\rangle_{i\in I}$ is any family
of Riesz spaces, then the product \pols\
$U=\prod_{i\in I}U_i$\cmmnt{ (351L)} is a Riesz space, with
     
\Centerline{$u\vee v=\langle u(i)\vee v(i)\rangle_{i\in I}$,
\quad$u\wedge v=\langle u(i)\wedge v(i)\rangle_{i\in I}$,
\quad$|u|=\langle|u(i)|\rangle_{i\in I}$}
     
\noindent for all $u$, $v\in U$.
     
\leader{352L}{Theorem} Let $U$ be any Riesz space.   Then there are a
set $X$, a filter $\Cal F$ on $X$ and a Riesz subspace of the Riesz
space $\Bbb R^X|\Cal F$\cmmnt{ (definition:  351M)} which is
isomorphic, as Riesz space, to $U$.
     
\proof{ By 351Q, we can find such $X$ and $\Cal F$ and an injective
Riesz homomorphism $T:U\to\Bbb R^X|\Cal F$.   By 352K, or otherwise,
$\Bbb R^X$ is a Riesz space;  by 352Jb, $\Bbb R^X|\Cal F$ is a Riesz
space (recall that it is a quotient of $\Bbb R^X$ by a solid linear
subspace, as explained in 351M);  by 352Ib, $T[U]$ is a Riesz subspace
of $\Bbb R^X|\Cal F$;  and by 352H it is isomorphic to $U$.
}%end of proof of 352L
     
\leader{352M}{Corollary} Any identity involving the operations $+$, $-$,
$\vee$, $\wedge$, $^+$, $^-$, $|\,\,|$ and scalar multiplication, and
the relation $\le$, which is valid in $\Bbb R$, is valid in all Riesz
spaces.
     
\cmmnt{\medskip
     
\noindent{\bf Remark} I suppose some would say that a strict proof of
this must begin with a formal description of what the phrase \lq any
identity involving the operations$\ldots$' means.   However I think it
is clear in practice what is involved.   Given a proposed identity like
     
\Centerline{$0\le\sum_{i=0}^n|\alpha_i||u_i|-|\sum_{i=0}^n\alpha_iu_i|
\le\sum_{i\ne j}(|\alpha_i|+|\alpha_j|)(|u_i|\wedge|u_j|)$,}
     
\noindent (compare 352Fb), then to check that it is valid in all Riesz
spaces you need only check (i) that it is true in $\Bbb R$ (ii) that it
is true in $\Bbb R^X$ (iii) that it is true in any $\Bbb R^X|\Cal F$
(iv) that it is true in any Riesz subspace of $\Bbb R^X|\Cal F$;  and
you can hope that
the arguments for (ii)-(iv) will be nearly trivial, since
(ii) is generally nothing but a coordinate-by-coordinate repetition of
(i), and (iii) and (iv) involve only transformations of the formula by
Riesz homomorphisms which preserve its structure.
}%end of comment
     
\leader{352N}{Order-density and order-continuity} Let $U$
be a Riesz space.
     
\spheader 352Na {\bf Definition}
A Riesz subspace $V$ of $U$ is
{\bf quasi-order-dense} if for every $u>0$ in $U$ there is a $v\in V$
such that $0<v\le u$;  it is {\bf order-dense} if $u=\sup\{v:v\in
V,\,0\le v\le u\}$ for every $u\in U^+$.
     
\spheader 352Nb If $U$ is a Riesz space and $V$ is a quasi-order-dense
Riesz subspace of $U$, then the embedding $V\embedsinto U$ is
order-continuous.   \prooflet{\Prf\ Let $A\subseteq V$ be a non-empty
set such that $\inf A=0$ in $V$.   \Quer\ If $0$ is not the infimum of
$A$ in $U$, then there is a $u>0$ such that $u$ is a lower bound for $A$
in $U$;  now there is a
$v\in V$ such that $0<v\le u$, and $v$ is a lower bound for $A$ in
$V$ which is strictly greater than $0$.\ \Bang\   Thus $0=\inf A$ in
$U$.   As $A$ is arbitrary, the embedding is order-continuous.\ \Qed\
}
     
\spheader 352Nc(i) If $V\subseteq U$ is an order-dense Riesz subspace,
it is quasi-order-dense.   (ii) If $V$ is a quasi-order-dense Riesz
subspace of $U$ and $W$ is a quasi-order-dense Riesz subspace of $V$,
then $W$ is a
quasi-order-dense Riesz subspace of $U$.   (iii) If $V$ is an
order-dense Riesz subspace of $U$ and $W$ is an order-dense Riesz
subspace of $V$, then $W$ is an order-dense Riesz subspace of $U$.
\cmmnt{(Use (b).)}
(iv) If $V$ is a quasi-order-dense solid linear subspace of $U$ and $W$
is a quasi-order-dense Riesz subspace of $U$ then $V\cap W$ is
quasi-order-dense\cmmnt{ in $V$, therefore} in $U$.
     
\spheader 352Nd \dvro{A}{I ought somewhere to remark that a} Riesz
homomorphism\cmmnt{, being a lattice homomorphism,} is
order-continuous iff it preserves arbitrary suprema and infima\cmmnt{;
compare 313L(b-iv) and (b-v)}.
     
\spheader 352Ne If $V$ is a Riesz subspace of $U$, we say that it is
{\bf regularly embedded} in $U$ if the identity map from $V$ to $U$ is
order-continuous\cmmnt{, that is, whenever $A\subseteq V$ is non-empty
and has infimum $0$ in $V$, then $0$ is still its
greatest lower bound in $U$.   Thus quasi-order-dense Riesz subspaces
and solid linear subspaces are regularly embedded}.
     
\leader{352O}{Bands} Let $U$ be a Riesz space.
     
\spheader 352Oa {\bf Definition}    A {\bf
band} or {\bf normal subspace} of $U$ is an order-closed solid linear
subspace.
     
\spheader 352Ob If $V\subseteq U$ is a solid
linear subspace then it is a band iff $\sup A\in V$ whenever 
$A\subseteq V^+$ is a non-empty, upwards-directed subset of $V$ with a 
supremum in
$U$.   \prooflet{\Prf\ Of course the condition is necessary;  I have to
show that it is sufficient.   (i) Let $A\subseteq V$ be any non-empty
upwards-directed set with a supremum in $V$.   Take any $u_0\in A$ and
set
$A_1=\{u-u_0:u\in A,\,u\ge u_0\}$.   Then $A_1$ is a non-empty
upwards-directed subset of $V^+$, and $u_0+A_1=\{u:u\in A,\,u\ge u_0\}$
has the same upper bounds as $A$, so $\sup A_1=\sup A-u_0$ is defined in
$U$ and belongs to $V$.   Now $\sup A=u_0+\sup A_1$ also belongs to $V$.
(ii) If $A\subseteq V$ is non-empty, downwards-directed and has an
infimum in $U$, then $-A\subseteq V$ is upwards-directed, so $\inf
A=\sup(-A)$ belongs to $V$.   Thus $V$ is order-closed.\ \Qed\ }
     
\spheader 352Oc For any set $A\subseteq U$
set $A^{\perp}=\{v:v\in U,\,|u|\wedge|v|=0$ for every $u\in A\}$.   Then
$A^{\perp}$ is a band.  \prooflet{\Prf\ (i) Of course $0\in A^{\perp}$.
(ii) If $v$, $w\in A^{\perp}$ and $u\in A$, then
     
\Centerline{$|u|\wedge|v+w|\le(|u|\wedge|v|)+(|u|\wedge|w|)=0$,}
     
\noindent so $v+w\in A^{\perp}$.   (iii) If $v\in A^{\perp}$ and
$|w|\le|v|$ then
     
\Centerline{$0\le|u|\wedge|w|\le|u|\wedge|v|=0$}
     
\noindent for every $u\in A$, so $w\in A^{\perp}$.   (iv) If $v\in
A^{\perp}$ then $nv\in A^{\perp}$ for every $n$, by (ii).   So if
$\alpha\in\Bbb R$, take $n\in\Bbb N$ such that $|\alpha|\le n$;  then
     
\Centerline{$|\alpha v|=|\alpha||v|\le n|v|\in A^{\perp}$}
     
\noindent and $\alpha v\in A^{\perp}$.   Thus $A^{\perp}$ is a solid
linear subspace of $U$.   (v) If $B\subseteq (A^{\perp})^+$ is non-empty
and upwards-directed and has a supremum $w$ in $U$, then
     
\Centerline{$|u|\wedge|w|=|u|\wedge w=\sup_{v\in B}|u|\wedge v=0$}
     
\noindent by 352Ea, so $w\in A^{\perp}$.   Thus $A^{\perp}$ is a
band.\ \Qed\ }
     
\spheader 352Od For any $A\subseteq U$, $A\subseteq
(A^{\perp})^{\perp}$.   Also $B^{\perp}\subseteq A^{\perp}$ whenever
$A\subseteq B$.   So\prooflet{
     
\Centerline{$A^{\perp\perp\perp}\subseteq A^{\perp}\subseteq
A^{\perp\perp\perp}$}
     
\noindent and} $A^{\perp}=A^{\perp\perp\perp}$.
     
\spheader 352Oe If $W$ is another Riesz space and $T:U\to W$ is an
order-continuous Riesz homomorphism then its kernel is a band.
\prooflet{(For $\{0\}$ is order-closed in $W$ and the inverse image of
an order-closed set by an order-continuous order-preserving function is
order-closed.)}
     
\leader{352P}{Complemented bands} Let $U$ be a Riesz space.
A band $V\subseteq U$ is {\bf complemented} if $V^{\perp\perp}=V$, that
is, if $V$ is of the form $A^{\perp}$ for some
$A\subseteq U$\cmmnt{ (352Od)}.
In this case its {\bf complement} is the complemented band $V^{\perp}$.
     
\vleader{36pt}{352Q}{Theorem} In any Riesz space $U$, the set $\frak C$ 
of complemented bands forms a Dedekind complete Boolean algebra, with
     
\Centerline{$V\hskip.2em\Bcapshort_{\frak C}\hskip.2em W=V\cap W$,
\quad$V\hskip.2em\Bcupshort_{\frak C}\hskip.2em W=(V+W)^{\perp\perp}$,}
     
\Centerline{$1_{\frak C}=U$,
\quad $0_{\frak C}=\{0\}$,
\quad$1_{\frak C}\hskip.2em\Bsetminusshort_{\frak C}\hskip.2em
V=V^{\perp}$,}
     
\Centerline{$V\hskip.2em\Bsubseteqshort_{\frak C}\hskip.2em W\iff
V\subseteq W$}
     
\noindent for $V$, $W\in\frak C$.
     
\proof{ To show that $\frak C$ is a Boolean algebra, I use the
identification of Boolean algebras with complemented distributive
lattices (311L).
     
\medskip
     
{\bf (a)} Of course $\frak C$ is partially ordered by $\subseteq$.   If
$V$, $W\in\frak C$ then
     
\Centerline{$V\cap W=V^{\perp\perp}\cap W^{\perp\perp}=(V^{\perp}\cup
W^{\perp})^{\perp}\in\frak C$,}
     
\noindent and $V\cap W$ must be $\inf\{V,W\}$ in $C$.   The map
$V\mapsto V^{\perp}:\frak C\to\frak C$ is an order-reversing permutation,
so that $V\subseteq W$ iff $W^{\perp}\subseteq V^{\perp}$ and
$V\vee W=\sup\{V,W\}$ will be $(V^{\perp}\cap W^{\perp})^{\perp}$;  thus
$\frak C$ is a lattice.   Note also that $V\vee W$ must be the smallest
complemented band including $V+W$, that is, it is $(V+W)^{\perp\perp}$.
     
\medskip
     
{\bf (b)} If $V_1$, $V_2$, $W\in\frak C$ then $(V_1\vee V_2)\wedge
W=(V_1\wedge W)\vee(V_2\wedge W)$.   \Prf\ Of course
$(V_1\vee V_2)\wedge W\supseteq(V_1\wedge W)\vee(V_2\wedge W)$.   \Quer\
Suppose, if possible, that there is a $u\in(V_1\vee V_2)\cap
W\setminus((V_1\cap W)\vee(V_2\cap W))$.   Then $u\notin((V_1\cap
W)^{\perp}\cap(V_2\cap W)^{\perp})^{\perp}$, so there is a $v\in(V_1\cap
W)^{\perp}\cap(V_2\cap W)^{\perp}$ such that $u_1=|u|\wedge|v|>0$.   Now
$u_1\in V_1\vee V_2=(V_1^{\perp}\cap V_2^{\perp})^{\perp}$ so $u_1\notin
V_1^{\perp}\cap V_2^{\perp}$;  say $u_1\notin V_j^{\perp}$, and there is
a $v_j\in V_j$ such that $u_2=u_1\wedge|v_j|>0$.   In this case we still
have $u_2\in (V_j\cap W)^{\perp}$, because $u_2\le|v|$, but also $u_2\in
V_j$ and $u_2\in W$ because $u_2\le|u|$;  but this means that
$u_2=u_2\wedge u_2=0$, which is absurd.\ \Bang\  Thus
$(V_1\vee V_2)\wedge W\subseteq(V_1\wedge W)\vee(V_2\wedge W)$ and the
two are equal.\ \Qed
     
\medskip
     
{\bf (c)} Now if $V\in\frak C$,
     
\Centerline{$V\wedge V^{\perp}=\{0\}$}
     
\noindent is the least member of $\frak C$, because if
$v\in V\cap V^{\perp}$ then $|v|=|v|\wedge|v|=0$.   By 311L, $\frak C$
has a Boolean algebra structure, with the Boolean relations described;
by 312M, this structure is uniquely defined.
     
\medskip
     
{\bf (d)} Finally, if $\Cal V\subseteq\frak C$ is non-empty, then
     
\Centerline{$\bigcap\Cal V=(\bigcup_{V\in\Cal
V}V^{\perp})^{\perp}\in\frak C$}
     
\noindent and is $\inf\Cal V$ in $\frak C$.   So $\frak C$ is Dedekind
complete.
}%end of proof of 352Q
     
\leader{352R}{Projection bands} Let $U$ be a Riesz space.
     
\spheader 352Ra A {\bf projection band} in $U$ is a set
$V\subseteq U$ such that $V+V^{\perp}=U$.   In this case $V$ is a
complemented band.   \prooflet{\Prf\ If $v\in V^{\perp\perp}$ then $v$
is expressible as $v_1+v_2$ where $v_1\in V$ and $v_2\in V^{\perp}$.
Now $|v|=|v_1|+|v_2|\ge|v_2|$ (352Fb), so
     
\Centerline{$|v_2|=|v_2|\wedge|v_2|\le|v_2|\wedge|v|=0$}
     
\noindent and $v=v_1\in V$.  Thus $V=V^{\perp\perp}$ is a complemented
band.\ \QeD}   Observe that $U=V^{\perp}+V^{\perp\perp}$ so $V^{\perp}$
also is a projection band.
     
\spheader 352Rb\cmmnt{ Because $V\cap V^{\perp}$ is always $\{0\}$, we
must have} $U=V\oplus V^{\perp}$ for any projection band $V\subseteq U$;
\cmmnt{accordingly} there is a corresponding {\bf band projection}
$P_V:U\to U$ defined by setting $P(v+w)=v$ whenever $v\in V$, $w\in
V^{\perp}$.   In this context I will say that $v$ is the {\bf component}
of $v+w$ in $V$.   The kernel of $P$ is $V^{\perp}$, the set of values
is $V$, and $P^2=P$.   \cmmnt{Moreover,}
$P$ is an order-continuous Riesz homomorphism.   \prooflet{\Prf\ (i) $P$
is a linear operator because $V$ and $V^{\perp}$ are linear subspaces.
(ii) If $v\in V$ and $w\in V^{\perp}$ then $|v+w|=|v|+|w|$, by 352Fb, so
$P|v+w|=|v|=|P(v+w)|$;  consequently $P$ is a Riesz homomorphism (352G).
(iii) If $A\subseteq U$ is downwards-directed and has infimum $0$, then
$Pu\le u$ for every $u\in A$, so $\inf P[A]=0$;  thus $P$ is
order-continuous.\ \Qed\ }
     
\spheader 352Rc Note that for any band projection $P$, and any $u\in U$,
we have\cmmnt{ $|Pu|\wedge|u-Pu|=0$, so that $|u|=|Pu|+|u-Pu|$ and (in
particular)} $|Pu|\le|u|$;  \cmmnt{consequently} $P[W]\subseteq W$ for
any solid linear subspace $W$ of $U$.
     
\spheader 352Rd A linear operator $P:U\to U$ is a band projection iff
$Pu\wedge(u-Pu)=0$ for every $u\in U^+$.   \prooflet{\Prf\ I remarked in
(c) that the condition is satisfied for any band projection.
Now suppose that $P$ has the property.   (i) For any $u\in U^+$,
$Pu\ge 0$ and $u-Pu\ge 0$;  in particular, $P$ is a positive linear
operator.   (ii) If $u$, $v\in U^+$ then
$u-Pu\le(u+v)-P(u+v)$, so
     
\Centerline{$Pv\wedge(u-Pu)\le P(u+v)\wedge((u+v)-P(u+v))=0$}
     
\noindent and $Pv\wedge(u-Pu)=0$.   (iii) If $u$, $v\in U$ then $|Pv|\le
P|v|$, $|u-Pu|\le|u|-P|u|$ (because $w\mapsto w-Pw$ is a positive linear
operator), so
     
\Centerline{$|Pv|\wedge|u-Pu|\le P|v|\wedge(|u|-P|u|)=0$.}
     
\noindent (iv) Setting $V=P[U]$, we see that $u-Pu\in V^{\perp}$ for
every $u\in U$, so that
     
\Centerline{$u=u+(u-Pu)\in V+V^{\perp}$}
     
\noindent for every $u$, and $U=V+V^{\perp}$;  thus $V$ is a projection
band.   (v) Since $Pu\in V$ and $u-Pu\in V^{\perp}$ for every $u\in U$,
$P$ is the band projection onto $V$.\ \Qed}
     
\leader{352S}{Proposition} Let $U$ be any Riesz space.
     
(a) The family $\frak B$ of projection bands in $U$ is a subalgebra of
the Boolean algebra $\frak C$ of complemented bands in $U$.
     
(b) For $V\in\frak B$ let $P_V:U\to V$ be the corresponding projection.
Then for any $e\in U^+$,
     
\Centerline{$P_{V\cap W}e=P_Ve\wedge P_We=P_VP_We$,
\quad$P_{V\vee W}e=P_Ve\vee P_We$}
     
\noindent for all $V$, $W\in\frak B$.   In particular, band projections
commute.
     
(c) If $V\in\frak B$ then the algebra of projection bands in $V$ is just
the principal ideal of $\frak B$ generated by $V$.
     
\proof{{\bf (a)} Of course $0_{\frak C}=\{0\}\in\frak B$.   If
$V\in \frak B$ then $V^{\perp}=1_{\frak C}\Bsetminus V$ belongs to
$\frak B$.   If now $W$ is another member of $\frak B$, then
     
\Centerline{$(V\cap W)+(V\cap W)^{\perp}
\supseteq (V\cap W)+V^{\perp}+W^{\perp}$.}
     
\noindent But if $u\in U$ then we can express $u$ as $v+v'$, where $v\in
V$ and $v'\in V^{\perp}$, and $v$ as $w+w'$, where $w\in W$ and $w'\in
W^{\perp}$;  and as $|w|\le|v|$, we also have $w\in V$, so that
     
\Centerline{$u=w+v'+w'\in (V\cap W)+V^{\perp}+W^{\perp}$.}
     
\noindent This shows that $V\cap W\in\frak B$.   Thus $\frak B$ is
closed under intersection and complements
and is a subalgebra of $\frak C$.
     
\medskip
     
(b) If $V$, $W\in\frak B$ and $e\in U^+$, we have $e=e_1+e_2+e_3+e_4$
where
     
\Centerline{$e_1=P_WP_Ve\in V\cap W$,
\quad$e_2=P_{W^{\perp}}P_Ve\in V\cap W^{\perp}$,}
     
\Centerline{$e_3=P_WP_{V^{\perp}}e\in V^{\perp}\cap W$,
\quad$e_4=P_{W^{\perp}}P_{V^{\perp}}e\in V^{\perp}\cap W^{\perp}$,}
     
\Centerline{$e_1+e_2=P_Ve$,
\quad$e_1+e_3=P_We$.}
     
\noindent Now $e_2+e_3+e_4$ belongs to $(V\cap W)^{\perp}$, so $e_1$
must be the component of $e$ in $V\cap W$;  similarly $e_4$ is the
component of $e$ in $V^{\perp}\cap W^{\perp}$, and $e_1+e_2+e_3$ is the
component of $e$ in $V\vee W$.    But as $e_2\wedge e_3=0$, we have
     
\Centerline{$P_{V\cap W}e=e_1=(e_1+e_2)\wedge(e_1+e_3)
=P_Ve\wedge P_We$,}
     
\Centerline{$P_{V\vee W}e=e_1+e_2+e_3=(e_1+e_2)\vee(e_1+e_3)=P_Ve\vee
P_We$,}
     
\noindent as required.
     
It follows that
     
\Centerline{$P_VP_W=P_{V\cap W}=P_{W\cap V}=P_WP_V$.}
     
     
\medskip
     
{\bf (c)} If $V$, $W\in\frak B$ and $W\subseteq V$, then of course $W$
is a band in the Riesz space $V$ (because $V$ is order-closed in $U$, so
that for any set $A\subseteq W$ its supremum in $U$ will be its supremum
in $V$).   For any $v\in V$, we have an expression of it as $w+w'$,
where $w\in W$ and $w'\in W^{\perp}$, taken in $U$;  but as
$|w|+|w'|=|w+w'|=|v|\in V$, $w'$ belongs to $V$, and is in
$W^{\perp}_V$, the band in $V$ orthogonal to $W$.   Thus
$W+W^{\perp}_V=V$ and $W$ is a projection band in $V$.   Conversely, if
$W$ is a projection band in $V$, then $W^{\perp}$ (taken in $U$)
includes $W^{\perp}_V+V^{\perp}$, so that
     
\Centerline{$W+W^{\perp}\supseteq
W+W^{\perp}_V+V^{\perp}=V+V^{\perp}=U$}
     
\noindent and $W\in\frak
B$.
     
Thus the algebra of projection bands in $V$ is, as a set, equal to the
principal ideal $\frak B_V$;  because their orderings agree, or
otherwise, their Boolean algebra structures coincide.
}%end of proof of 352S
     
\vleader{60pt}{352T}{Products again (a)} If $U=\prod_{i\in I}U_i$ is a product
of Riesz spaces, then for any $J\subseteq I$ we have a subspace
     
\Centerline{$V_J=\{u:u\in U,\,u(i)=0$ for all $i\in I\setminus J\}$}
     
\noindent of $U$, canonically isomorphic to $\prod_{i\in J}U_i$.   Each
$V_J$ is a projection band, its complement being $V_{I\setminus J}$;
the map $J\mapsto V_J$ is a Boolean homomorphism from $\Cal PI$ to the
algebra $\frak B$ of projection bands in $U$, and 
$\langle V_{\{i\}}\rangle_{i\in I}$ is a partition of unity in $\frak B$.
     
\spheader 352Tb Conversely, if $U$ is a Riesz space and
$(V_0,\ldots,V_n)$ is a {\it finite} partition of unity in the algebra
$\frak B$ of projection bands in $U$, then every element of $U$ is
uniquely expressible as $\sum_{i=0}^nu_i$ where $u_i\in V_i$ for each
$i$.   \prooflet{(Induce on $n$.)}   This decomposition corresponds to a
Riesz space isomorphism between $U$ and $\prod_{i\le n}V_i$.
     
\leader{352U}{Quotient spaces (a)} If $U$ is a Riesz space and $V$ is a
solid linear subspace, then the quotient \pols\ $U/V$\cmmnt{ (351J)}
is a Riesz space;  if $U$ and $W$ are Riesz spaces and $T:U\to W$ a
Riesz homomorphism, then the kernel $V$ of $T$ is a solid linear
subspace of
$U$ and the Riesz subspace $T[U]$ of $W$ is isomorphic to $U/V$\cmmnt{
(352Jb)}.
     
\spheader 352Ub Suppose that $U$ is a Riesz space and $V$ a solid linear
subspace.   Then the canonical map from $U$ to $U/V$ is
order-continuous iff $V$ is a band.   \prooflet{\Prf\ (i) If $u\mapsto
u^{\ssbullet}$ is order-continuous, its kernel $V$ is a band, by 352Oe.
(ii) If $V$ is a band, and $A\subseteq U$ is non-empty and
downwards-directed and has infimum $0$, let $p\in U/V$ be any lower
bound for $\{u^{\ssbullet}:u\in A\}$.   Express $p$ as $w^{\ssbullet}$.
Then $((w-u)^+)^{\ssbullet}=(w^{\ssbullet}-u^{\ssbullet})^+=0$, that is,
$(w-u)^+\in V$ for every $u\in A$.   But this means that
     
\Centerline{$w^+=\sup_{u\in A}(w-u)^+\in V$,
\quad$p^+=(w^+)^{\ssbullet}=0$,}
     
\noindent that is, $p\le 0$.   As $p$ is arbitrary, $\inf_{u\in
A}u^{\ssbullet}=0$;  as $A$ is arbitrary, $u\mapsto u^{\ssbullet}$ is
order-continuous.\ \Qed
}%end of prooflet
     
\leader{352V}{Principal bands} Let $U$ be a Riesz space.   Evidently the
intersection of any family of Riesz subspaces of $U$ is a Riesz
subspace, the intersection of any family of solid linear subspaces is a
solid linear subspace, the intersection of any family of bands is a
band;  we may\cmmnt{ therefore} speak of the band generated by a subset $A$ of
$U$, the intersection of all the bands including $A$.  \cmmnt{Now we
have the following description of the band generated by a single
element.}
     
\medskip
     
\noindent{\bf Lemma} Let $U$ be a Riesz space.
     
(a) If $A\subseteq U^+$ is upwards-directed and $2w\in A$ for every
$w\in A$, then an element $u$ of $U$ belongs to the band generated by
$A$ iff $|u|=\sup_{w\in A}|u|\wedge w$.
     
(b) If $u\in U$ and $w\in U^+$, then $u$ belongs to the band in $U$
generated by $w$ iff $|u|=\sup_{n\in\Bbb N}|u|\wedge nw$.
     
\proof{{\bf (a)} Let $W$ be the band generated by $A$ and $W'$ the set
of elements of $U$ satisfying the condition.
     
\medskip
     
\quad{\bf (i)} If $u\in W'$ then $|u|\wedge w\in W$ for every $w\in A$,
because $W$ is a solid linear subspace;  because $W$ is also
order-closed, $|u|$ and $u$ belong to $W$.   Thus $W'\subseteq W$.
     
\medskip
     
\quad{\bf (ii)} Now $W'$ is a band.
     
\medskip
     
\Prf\grheada\ If $u\in W'$ and $|v|\le|u|$ then
     
\Centerline{$\sup_{w\in A}|v|\wedge w
=\sup_{w\in A}|v|\wedge|u|\wedge w
=|v|\wedge\sup_{w\in A}|u|\wedge w
=|v|\wedge|u|=|v|$}
     
\noindent by 352Ea, so $v\in W'$.
     
\medskip
     
\qquad\grheadb\ If $u$, $v\in W'$ then, for any $w_1$, $w_2\in A$ there
is a $w\in A$ such that $w\ge w_1\vee w_2$.   Now $w_1+w_2\le 2w\in
A$, and
     
\Centerline{$(|u|+|v|)\wedge 2w
\ge(|u|\wedge w_1)+(|v|\wedge w_2)$.}
     
\noindent So any upper bound for $\{(|u|+|v|)\wedge w:w\in A\}$ must
also be an upper bound for
$\{|u|\wedge w:w\in A\}+\{|v|\wedge w:w\in A\}$ and therefore greater
than or equal to
     
$$\eqalign{\sup(\{|u|\wedge w:w\in A\}+\{|v|\wedge w:w\in A\})
&=\sup_{w\in A}|u|\wedge w+\sup_{w\in A}|v|\wedge w\cr
&=|u|+|v|\cr}$$
     
\noindent (351Dc).   But this means that
$\sup_{w\in A}(|u|+|v|)\wedge w$ must be $|u|+|v|$, and $|u|+|v|$
belongs to $W'$;  it follows from ($\alpha$) that $u+v$ belongs to $W'$.
     
\medskip
     
\qquad\grheadc\ Just as in 352Oc, we now have
     
\Centerline{$nu\in W'$ for every $n\in\Bbb N$, $u\in W'$,}
     
\noindent and therefore $\alpha u\in W'$ for every $\alpha\in\Bbb R$,
$u\in W'$, since $|\alpha u|\le|nu|$ if $|\alpha|\le n$.   Thus $W'$ is
a solid linear subspace of $U$.
     
\medskip
     
\qquad\grheadd\  Now suppose that $C\subseteq (W')^+$ has a supremum $v$
in $U$.   Then any upper bound of $\{v\wedge w:w\in A\}$ must also be an
upper bound of $\{u\wedge w:u\in C,\,w\in A\}$ and greater than or equal
to $u=\sup_{w\in A}u\wedge w$ for every $u\in C$, therefore greater than
or equal to $v=\sup C$.   Thus $v=\sup_{w\in A}v\wedge w$ and $v\in W'$.
As $C$ is arbitrary, $W'$ is a band (352Ob).\ \Qed
     
\medskip
     
\quad{\bf (iii)} Since $A$ is obviously included in $W'$, $W'$ must
include $W$;  putting this together with (i), $W=W'$, as claimed.
     
\medskip
     
{\bf (b)} Apply (a) with $A=\{nw:n\in\Bbb N\}$.
}%end of proof of 352V
     
\leader{352W}{$f$-algebras}\cmmnt{ Some of the most important Riesz
spaces have multiplicative structures as well as their order and linear
structures.   A particular class of these structures appears
sufficiently often for it to be useful to develop a little of its
theory.   The following definition is a common approach.
     
\medskip
     
} {\bf (a) Definition} An {\bf $f$-algebra} is a Riesz space
$U$ with a multiplication $\times:U\times U\to U$ such that
     
\Centerline{$u\times(v\times w)=(u\times v)\times w$,}
     
\Centerline{$(u+v)\times w=(u\times w)+(v\times w)$,
\quad$u\times(v+w)=(u\times v)+(u\times w)$,}
     
\Centerline{$\alpha(u\times v)=(\alpha u)\times v=u\times(\alpha v)$}
     
\noindent for all $u$, $v$, $w\in U$ and $\alpha\in\Bbb R$, and
     
\Centerline{$u\times v\ge 0$ whenever $u$, $v\ge 0$,}
     
\Centerline{if $u\wedge v=0$ then 
$(u\times w)\wedge v=(w\times u)\wedge v=0$ for every $w\ge 0$.}
     
\noindent An $f$-algebra is {\bf commutative} if $u\times v=v\times u$
for all $u$, $v$.
     
\spheader 352Wb Let $U$ be an $f$-algebra.
     
\medskip
     
\quad{\bf (i)} If $u\wedge v=0$ in $U$, then
$u\times v=0$.   \prooflet{\Prf\
$u\wedge(u\times v)=0$ so $(u\times v)\wedge(u\times v)=0$.\ \Qed}
     
\medskip
     
\quad{\bf (ii)} $u\times u\ge 0$ for every $u\in U$.   \prooflet{\Prf\
     
$$\eqalign{(u^+-u^-)\times(u^+-u^-)
&=u^+\times u^+-u^+\times u^--u^-\times u^++u^-\times u^-\cr
&=u^+\times u^++u^-\times u^-
\ge 0.\text{ \Qed}\cr}$$
}%end of prooflet
     
\medskip
     
\quad{\bf (iii)} If $u$, $v\in U$ then $|u\times v|=|u|\times|v|$.
\prooflet{\Prf\ $u^+\times v^+$, $u^+\times v^-$, $u^-\times v^+$ and
$u^+\times u^-$ are disjoint, so
     
$$\eqalign{|u\times v|
&=|u^+\times v^+-u^+\times v^--u^-\times v^++u^-\times v^-|\cr
&=u^+\times v^++u^+\times v^-+u^-\times v^++u^-\times v^-\cr
&=|u|\times|v|\cr}$$
     
\noindent by 352Fb.\ \Qed}
     
\medskip
     
\quad{\bf (iv)} If $v\in U^+$ the
maps $u\mapsto u\times v$, $u\mapsto v\times u:U\to U$ are Riesz
homomorphisms.
\prooflet{\Prf\ The first four clauses of the definition in (a) ensure
that they are linear operators.   If $u\in U$, then
     
\Centerline{$|u|\times v=|u\times v|$,
\quad$v\times|u|=|v\times u|$}
     
\noindent by (iii), so we have Riesz homomorphisms, by 352G(iii).\ \Qed}
     
\medskip
     
{\bf (c)} Let $\familyiI{U_i}$ be a family of $f$-algebras, with Riesz
space product $U$ (352K).   If we set
$u\times v=\familyiI{u(i)\times v(i)}$ for all $u$, $v\in U$, then $U$
becomes an $f$-algebra.
     
\exercises{\leader{352X}{Basic exercises $\pmb{>}$(a)}
%\spheader 352Xa
Let $U$ be any Riesz space.   Show that
$|u^+-v^+|\le|u-v|$ for all $u$, $v\in U$.
%352F
     
\sqheader 352Xb
Let $U$, $V$ be a Riesz spaces and $T:U\to V$ a linear operator.   Show
that
the following are equiveridical:  (i) $T$ is a Riesz homomorphism;
(ii) $T(u\vee v)=Tu\vee Tv$ for all $u$, $v\in U$;
(iii) $T(u\wedge v)=Tu\wedge Tv$ for all $u$, $v\in U$;
(iv) $|Tu|=T|u|$ for every $u\in U$.
%352G
     
\spheader 352Xc Let $U$ be a Riesz space and $V$ a solid linear
subspace;  for $u\in U$ write $u^{\ssbullet}$ for the corresponding
element of $U/V$.   Show that if $A\subseteq U$ is solid then
$\{u^{\ssbullet}:u\in A\}$ is solid in $U/W$.
%352J
     
\spheader 352Xd Let $U$ be a Riesz space.   Show that
$\med(\alpha u,\alpha v,\alpha w)=\alpha\med(u,v,w)$ for all $u$, $v$,
$w\in U$ and all $\alpha\in\Bbb R$.   \Hint{3A1Ic, 352M.}
%352L
     
\spheader 352Xe Let $U$ and $V$ be Riesz spaces and $T:U\to V$ a Riesz
homomorphism with kernel $W$.   Show that if $W$ is a band in $U$ and
$T[U]$ is regularly embedded in $V$ then $T$ is order-continuous.
%352O
     
\spheader 352Xf Give $U=\BbbR^2$ its lexicographic ordering (351Xa).
Show that it has a band $V$ which is not complemented.
%352P  [Riesz space because totally ordered]
     
\spheader 352Xg Let $U$ be a Riesz space and $\frak C$ the algebra of
complemented bands in $U$.   Show that for any $V\in\frak C$ the algebra
of complemented bands in $V$ is just the principal ideal of $\frak C$
generated by $V$.
%352Q
     
\sqheader 352Xh Let $U=C([0,1])$ be the space of continuous functions
from $[0,1]$ to $\Bbb R$, with its usual linear and order structures, so
that it is a Riesz subspace of $\BbbR^{[0,1]}$.   Set $V=\{u:u\in
U,\,u(t)=0$ if $t\le\bover12\}$.   Show that $V$ is a band in $U$ and
that $V^{\perp}=\{u:u(t)=0$ if $t\ge\bover12\}$, so that $V$ is
complemented but is not a projection band.
%352R
     
\spheader 352Xi Show that the Boolean homomorphism $J\mapsto V_J:\Cal
PI\to\frak B$ of 352Ta is order-continuous.
%352T
     
\spheader 352Xj Let $U$ be a Riesz space and $A\subseteq U^+$ an
upwards-directed set.   Show that the band generated by $A$ is
$\{u:|u|=\sup_{n\in\Bbb N,w\in A}|u|\wedge nw\}$.
%352V
     
\sqheader 352Xk(i) Let $X$ be any set.   Setting
$(u\times v)(x)=u(x)v(x)$ for $u$, $v\in\Bbb R^X$, $x\in X$, show that
$\Bbb R^X$
is a commutative $f$-algebra.   (ii) With the same definition of
$\times$, show
that $\ell^{\infty}(X)$ is an $f$-algebra.   (iii) If $X$ is a
topological space, show that $C(X)$, $C_b(X)$ are $f$-algebras.   (iv)
If $(X,\Sigma,\mu)$ is a measure space, show that $L^0(\mu)$ and
$L^{\infty}(\mu)$ (\S241, \S243) are $f$-algebras.
%352W
     
\spheader 352Xl  Let $U\subseteq\BbbR^{\Bbb Z}$ be the set of sequences
$u$ such that $\{n:u(n)\ne 0\}$ is bounded above in $\Bbb Z$.   For $u$,
$v\in U$ (i) say that $u\le v$ if either $u=v$ or there is an $n\in\Bbb
Z$ such that $u(n)<v(n)$, $u(i)=v(i)$ for every $i>n$ (ii) say that
$(u*v)(n)=\sum_{i=-\infty}^{\infty}u(i)v(n-i)$ for every $n\in\Bbb Z$.
Show that $U$ is an $f$-algebra under this ordering and multiplication.
%352W
     
\spheader 352Xm Let $U$ be an $f$-algebra.  (i) Show that any
complemented band in $U$
is an ideal in the ring $(U,+,\times)$.   (ii) Show that if $P:U\to U$
is a band projection, then $P(u\times v)=Pu\times Pv$ for every $u$,
$v\in U$.
%352W
     
\spheader 352Xn Let $U$ be an $f$-algebra with multiplicative identity
$e$.   Show that $u-\gamma e\le\Bover1{\gamma}u^2$ for every $u\in U$,
$\gamma>0$.   \Hint{$(u^+-\gamma e)^2\ge 0$.}
%352W
     
%\leader{352Y}{Further exercises (a)}
}%end of exercises
     
\cmmnt{
\Notesheader{352} In this section we begin to see a striking
characteristic of the theory of Riesz spaces:  repeated reflections of
results in Boolean algebra.   Without spelling out a complete list, I
mention the distributive laws (313Bc, 352Ea) and the behaviour of
order-continuous homomorphisms (313Pa, 313Qa, 352N, 352Oe, 352Ub,
352Xe).
Riesz subspaces correspond to subalgebras, solid linear subspaces to
ideals and Riesz homomorphisms to Boolean homomorphisms.
We even have a correspondence, though a weaker one, between the
representation theorems available;  every Boolean algebra is isomorphic
to a subalgebra of a power of $\Bbb Z_2$ (311D-311E), while every
Riesz space is isomorphic to a Riesz subspace of a quotient of a power
of $\Bbb R$ (352L).   It would be a closer parallel if every Riesz space
were embeddable in some $\Bbb R^X$;  I must emphasize that the
differences are as important as the agreements.   Subspaces of
$\BbbR^X$ are of great importance, but are by no means adequate for our
needs.   And of course the details -- for instance, the identities in
352D-352F, or 352V -- frequently involve new techniques in the case of
Riesz spaces.   Elsewhere, as in 352G, I find myself arguing rather from
the opposite side, when applying results from the theory of general
partially ordered linear spaces, which has little to do with Boolean
algebra.
     
In the theory of bands in Riesz spaces -- corresponding to order-closed
ideals in Boolean algebras -- we have a new complication in the form of
bands which are not complemented, which does not arise in the Boolean
algebra context;  but it disappears again when we come to specialize to
Archimedean Riesz spaces (353B).   (Similarly, order-density and
quasi-order-density coincide in both Boolean algebras (313K) and
Archimedean Riesz spaces (353A).)   Otherwise the algebra of
complemented bands in a Riesz space looks very like the algebra of
order-closed ideals in a Boolean algebra (314Yh, 352Q).   The algebra
of projection bands in a Riesz space (352S) would correspond, in a
Boolean algebra, to the algebra itself.
     
I draw your attention to 352H.   The result is nearly trivial, but it
amounts to saying that the theory of Riesz spaces will be `algebraic',
like the theories of groups or linear spaces, rather than `analytic',
like the theories of \pols s or topological spaces, in which we can have
bijective morphisms which are not isomorphisms.
     
}%end of comment
     
\discrpage
     
