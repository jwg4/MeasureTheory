\frfilename{mt477.tex}
\versiondate{4.1.08/2.1.10}
\copyrightdate{2008}

\def\hp{\mathop{\text{hp}}\nolimits}

\def\chaptername{Geometric measure theory}
\def\sectionname{Brownian motion}

\newsection{477}

I presented \S455 with an extraordinary omission:  the leading example of a
L\'evy process, and the inspiration for the whole project, was
relegated to an anonymous example (455Xg).   In this section I will take
the subject up again.   The theorem that the sum of independent normally
distributed random variables is again normally distributed (274B), when
translated into the language of this volume, tells us that we have a family
$\langle\lambda_t\rangle_{t>0}$ of centered normal distributions such that
$\lambda_{s+t}=\lambda_s*\lambda_t$ for all $s$, $t>0$.   Consequently we
have a corresponding example of a L\'evy process on $\Bbb R$, and this is
the process which we call `Brownian motion' (477A).   This is special in
innumerable ways, but one of them is central:  we can represent it in such
a way that sample paths are continuous (477B), that is, as a Radon measure
on the space of continuous paths starting at $0$.   In this form, it also
appears as a limit, for the narrow topology, of interpolations of
random walks (477C).

For the geometric ideas of \S479, we need Brownian motion in three
dimensions;  the $r$-dimensional theory of 477D-477G gives no new
difficulties.   The simplest expression of Brownian motion in $\BbbR^r$ is
just to take a product measure (477Da), but in order to apply the results
of \S455, and match the construction with the ideas of \S456, a fair bit of
explanation is necessary.   The geometric properties of Brownian motion
begin with the invariant transformations of 477E.   As for all L\'evy
processes, we have a strong Markov property, and Theorem 455U translates
easily into the new formulation (477G), as does the theory of hitting
times (477I).   I conclude with a classic result on maximal values
which will be useful later (477J), and with proofs that
almost all Brownian paths are nowhere differentiable (477K) and have
zero two-dimensional Hausdorff measure (477L).

\leader{477A}{Brownian motion:  Theorem} There are a probability space
$(\Omega,\Sigma,\mu)$ and a family $\langle X_t\rangle_{t\ge 0}$ of
real-valued random variables on $\Omega$ such that

(i) $X_0=0$ almost everywhere;

(ii) whenever $0\le s<t$ then $X_t-X_s$ is normally
distributed with expectation $0$ and variance $t-s$;

(iii) $\langle X_t\rangle_{t\ge 0}$ has independent increments.

\ifwithproofs\medskip

\noindent{\bf First proof} In 455P, take $U=\Bbb R$ and
$\lambda_t$, for $t>0$, to be the distribution of a normal random
variable with expectation $0$ and variance $t$;  that is, the distribution
with probability density function
$x\mapsto\Bover1{\sqrt{2\pi t}}e^{-x^2/2t}$.   By 272T\formerly{2{}72S},
$\lambda_{s+t}=\lambda_s*\lambda_t$ for all $s$, $t>0$.
If $\epsilon>0$, then

\def\shortstrut{\hbox{\vrule height7.5pt depth2.5pt width0pt}}

\Centerline{$\lim_{t\downarrow 0}\lambda_t\ooint{-\epsilon,\epsilon}
=\lim_{t\downarrow 0}
\lambda_1\bigl]-\Bover{\epsilon}{\sqrt t},\Bover{\epsilon}{\sqrt t}\bigr[
=1$,}

\noindent so $\langle\lambda_t\rangle_{t>0}$ satisfies the conditions of
455P.   Accordingly we have a probability measure $\hat\mu$ on
$\Omega=\BbbR^{\coint{0,\infty}}$ for which, setting
$X_t(\omega)=\omega(t)$, $\langle X_t\rangle_{t\ge 0}$ has the required
properties, as noted in 455Q-455R.

\medskip

\noindent{\bf Second proof} Let $\mu_L$ be Lebesgue measure on $\Bbb R$,
and for $t\ge 0$ set $u_t=\chi[0,t]^{\ssbullet}$ in $L^2(\mu_L)$, so that
$\innerprod{u_s}{u_t}=\min(s,t)$ for $s$, $t\ge 0$.   By 456C, there is a
centered Gaussian distribution $\mu$ on $\BbbR^{\coint{0,\infty}}$ with
covariance matrix $\langle\min(s,t)\rangle_{s,t\ge 0}$.   Set
$X_t(x)=x(t)$ for $x\in\BbbR^{\coint{0,\infty}}$.   Then $X_0$ has
expectation and variance both $0$, that is, $X_0=0$ a.e.   If
$0\le s<t$, then $X_t-X_s$ is a linear combination of $X_s$ and $X_t$, so
is normally distributed with expectation $0$, and its variance is

\Centerline{$\Expn(X_t-X_s)^2
=\Expn(X_t)^2-2\Expn(X_t\times X_s)+\Expn(X_s)^2
=t-2s+s=t-s$.}

\noindent If $0\le t_0<\ldots<t_n$ and $Y_i=X_{t_{i+1}}-X_{t_i}$ for
$i<n$, then $\langle Y_i\rangle_{i<n}$ has a centered Gaussian
distribution, by 456Ba.   Also, if $i<j<n$, then

$$\eqalign{\Expn(Y_i\times Y_j)
&=\Expn(X_{t_{i+1}}\times X_{t_{j+1}})
  -\Expn(X_{t_{i+1}}\times X_{t_j})
  -\Expn(X_{t_i}\times X_{t_{j+1}})
  +\Expn(X_{t_i}\times X_{t_j})\cr
&=t_{i+1}-t_{i+1}-t_i+t_i
=0.\cr}$$

\noindent So 456E assures us that $\langle Y_i\rangle_{i<n}$ is
independent.

Thus $\langle X_t\rangle_{t\ge 0}$ satisfies the conditions required.
\else\fi %end of proof of 477A

\leader{477B}{}\cmmnt{ These constructions of Brownian motion are
sufficient to show that there is a process, satisfying the defining
conditions (i)-(iii), which can be studied with the tools of measure
theory.   From 455H we see that we have a Radon measure on the space
of \callal\ functions representing the process, and from 455P that we
have the option of moving to the \cadlag\ functions, with a corresponding
description of the strong Markov property in terms of \imp\ functions, as
in 455U.   But there is no hint yet of the most important property of
Brownian motion, that `sample paths are continuous'.   With some simple
inequalities from Chapter 27 and the ideas of 454Q-454S, %454Q 454R 454S
we can find a proof of this, as follows.

\medskip

\noindent}{\bf Theorem} Let
$\langle X_t\rangle_{t\ge 0}$ be as in 477A, and $\hat\mu$ the
distribution of the process $\langle X_t\rangle_{t\ge 0}$.
Let $C(\coint{0,\infty})_0$ be
the set of continuous functions
$\omega:\coint{0,\infty}\to\Bbb R$ such that $\omega(0)=0$.
Then $C(\coint{0,\infty})_0$ has full outer measure for
$\hat\mu$, and the subspace measure $\mu_W$ on $C(\coint{0,\infty})_0$
induced by $\hat\mu$
is a Radon measure when $C(\coint{0,\infty})_0$ is given the topology
$\frak T_c$ of uniform convergence on compact sets.

\proof{{\bf (a)} The main part of the argument here (down to
the end of (e)) is devoted to showing that
$\hat\mu^*C(\coint{0,\infty})=1$;
the result will then follow easily from 454Sb.

\medskip

{\bf (b)} \Quer\ Suppose, if possible, that
$\hat\mu^*C(\coint{0,\infty})<1$.
Then there is a non-negligible Baire set
$H\subseteq\BbbR^{\coint{0,\infty}}\setminus C(\coint{0,\infty})$.
There is a countable set $D\subseteq\coint{0,\infty}$ such that $H$ is
determined by coordinates in $D$ (4A3Nb);  we may suppose that $D$
includes $\Bbb Q\cap\coint{0,\infty}$.

\medskip

{\bf (c)} (The key.)  Let $q$, $q'$ be rational numbers such that
$0\le q<q'$, and $\epsilon>0$.   Then

\Centerline{$\Pr(\sup_{t\in D\cap[q,q']}|X_t-X_q|>\epsilon)
\le\Bover{18\sqrt{q'-q}}{\epsilon\sqrt{2\pi}}
   e^{-\epsilon^2/18(q'-q)}$.}

\noindent\Prf\ If $q=t_0<t_1<\ldots<t_n=q'$, set
$Y_i=X_{t_i}-X_{t_{i-1}}$ for $1\le i\le n$, so that
$X_{t_m}-X_q=\sum_{i=1}^mY_i$ for $1\le m\le n$, and $Y_1,\ldots,Y_n$
are independent.   By Etemadi's lemma (272V\formerly{2{}72U}),

$$\eqalignno{\Pr(\sup_{i\le n}|X_{t_i}-X_q|>\epsilon)
&\le 3\max_{i\le n}\Pr(|X_{t_i}-X_q|>\Bover13\epsilon)\cr
&=3\max_{1\le i\le n}\Pr(\Bover1{\sqrt{t_i-q}}|X_{t_i}-X_q|
                         >\Bover{\epsilon}{3\sqrt{t_i-q}})\cr
&=6\max_{1\le i\le n}\Bover1{\sqrt{2\pi}}
  \int_{\epsilon/3\sqrt{t_i-q}}^{\infty}e^{-x^2/2}dx\cr
\displaycause{because $\Bover1{\sqrt{t_i-q}}(X_{t_i}-X_q)$ is
standard normal}
&=\Bover6{\sqrt{2\pi}}
  \int_{\epsilon/3\sqrt{q'-q}}^{\infty}e^{-x^2/2}dx\cr
&\le\Bover{18\sqrt{q'-q}}{\epsilon\sqrt{2\pi}}
   e^{-\epsilon^2/18(q'-q)}\cr}$$

\noindent by 274Ma.   Thus if $I\subseteq[q,q']$ is any finite set
containing $q$ and $q'$,

\Centerline{$\Pr(\sup_{t\in I}|X_t-X_q|>\epsilon)
\le\Bover{18\sqrt{q'-q}}{\epsilon\sqrt{2\pi}}
   e^{-\epsilon^2/18(q'-q)}$.}

\noindent Taking $\sequencen{I_n}$ to be a non-decreasing sequence of
finite sets with union $D\cap[q,q']$, starting from $I_0=\{q,q'\}$, we
get

$$\eqalign{\Pr(\sup_{t\in D\cap[q,q']}|X_t-X_q|>\epsilon)
&=\lim_{n\to\infty}\Pr(\sup_{t\in I_n}|X_t-X_q|>\epsilon)\cr
&\le\Bover{18\sqrt{q'-q}}{\epsilon\sqrt{2\pi}}
   e^{-\epsilon^2/18(q'-q)},\cr}$$

\noindent as required.\ \Qed

\medskip

{\bf (d)} If $\epsilon>0$ and $n\ge 1$, then

$$\eqalign{\Pr(\text{there are }t,\,u\in D\cap[0,n]&\text{ such that }
|t-u|\le\Bover1{n^2}\text{ and }|X_t-X_u|>3\epsilon)\cr
&\le\Bover{18n^2}{\epsilon\sqrt{2\pi}}e^{-n^2\epsilon^2/18}.\cr}$$

\noindent\Prf\ Divide $[0,n]$ into $n^3$ intervals $[q_i,q_{i+1}]$ of
length $1/n^2$.   For each of these,

\Centerline{$\Pr(\sup_{t\in D\cap[q_i,q_{i+1}]}|X_t-X_{q_i}|>\epsilon)
\le\Bover{18}{n\epsilon\sqrt{2\pi}}e^{-n^2\epsilon^2/18}$.}

\noindent So

\Centerline{$\Pr(\text{there are }i<n^3,\,t\in D\cap[q_i,q_{i+1}]
\text{ such that }|X_t-X_{q_i}|>\epsilon)$}

\noindent is at most
$\Bover{18n^2}{\epsilon\sqrt{2\pi}}e^{-n^2\epsilon^2/18}$.

But if
$t$, $u\in[0,n]$ and $|t-u|\le 1/n^2$ and $|X_t-X_u|>3\epsilon$,
there must be an $i<n^3$ such that both $t$ and $u$ belong to
$[q_i,q_{i+2}]$, so that either there is a $t'\in D\cap[q_i,q_{i+1}]$
such that $|X_{t'}-X_{q_i}|>\epsilon$ or there is a
$t'\in D\cap[q_{i+1},q_{i+2}]$ such that
$|X_{t'}-X_{q_{i+1}}|>\epsilon$.   So

$$\eqalign{\Pr(&\text{there are }t,\,u\in D\cap[0,n]\text{ such that }
|t-u|\le\Bover1{n^2}\text{ and }|X_t-X_u|> 3\epsilon)\cr
&\quad\le\Pr(\text{there are }i<n^3,\,t\in D\cap[q_i,q_{i+1}]
\text{ such that }\text{ and }|X_t-X_{q_i}|>\epsilon)\cr
&\quad\le\Bover{18n^2}{\epsilon\sqrt{2\pi}}e^{-n^2\epsilon^2/18},\cr}$$

\noindent as required.\ \Qed

\medskip

{\bf (e)} So if we take $G_{\epsilon n}$ to be the Baire set

$$\eqalign{\{\omega:\omega\in\BbbR^{\coint{0,\infty}},
  \text{ there are }t,\,
  u\in D\cap[0,n]&\text{ such that }|t-u|\le\Bover1{n^2}\cr
&\text{ and }|\omega(t)-\omega(u)|> 3\epsilon\},\cr}$$

\noindent we have

\Centerline{$\hat\mu G_{\epsilon n}
\le\Bover{18n^2}{\epsilon\sqrt{2\pi}}e^{-n^2\epsilon^2/18}$,}

\ifdim\pagewidth>467pt\fontdimen3\tenrm=2pt\fi
\ifdim\pagewidth>467pt\fontdimen4\tenrm=1.67pt\fi
\noindent and $\lim_{n\to\infty}\hat\mu G_{\epsilon n}=0$.   We can
therefore find a strictly increasing sequence $\sequence{k}{n_k}$
in $\Bbb N$ such that $\sum_{k=1}^{\infty}\hat\mu(G_{1/k,n_k})<\hat\mu H$,
so that there is
an $\omega\in H\setminus\bigcup_{k\ge 1}G_{1/k,n_k}$.
\fontdimen3\tenrm=1.67pt
\fontdimen4\tenrm=1.11pt

What this means is that if $k\ge 1$ and $t$, $u\in D\cap[0,n_k]$ are
such that $|t-u|\le\Bover1{n_k^2}$, then
$|\omega(t)-\omega(u)|\le\Bover3k$.   Since
$n_k\to\infty$ as $k\to\infty$, there is a continuous function
$\omega':\coint{0,\infty}\to\Bbb R$ such
that $\omega'\restr D=\omega\restr D$.
But $H$ is determined by coordinates in
$D$, so $\omega'$ belongs to $H\cap C(\coint{0,\infty})$,
which is supposed to be empty.\ \Bang

\medskip

{\bf (f)} Thus $\hat\mu^*C(\coint{0,\infty})=1$.
Since $\hat\mu\{\omega:\omega(0)=0\}=1$,
$C(\coint{0,\infty})\setminus C(\coint{0,\infty})_0$
is $\hat\mu$-negligible and
$C(\coint{0,\infty})_0$ is of full outer measure for $\hat\mu$.
By 454Sb, the subspace
measure $\hat\mu_C$ on $C(\coint{0,\infty})$ induced by $\hat\mu$ is a
Radon measure for $\frak T_c$;  now $C(\coint{0,\infty})_0$ is
$\hat\mu_C$-conegligible.   The subspace measure $\mu_W$ on
$C(\coint{0,\infty})_0$ induced by $\hat\mu$ is also the subspace measure
induced by $\hat\mu_C$, so is a Radon measure for the topology on
$C(\coint{0,\infty})_0$ induced by $\frak T_c$.
}%end of proof of 477B

\cmmnt{\medskip

\noindent{\bf Remark} We can put this together with the ideas of 455H.
Following the First Proof of 477A, and using 455Pc, we see that
there is a unique Radon
measure $\tilde\mu$ on $\BbbR^{\coint{0,\infty}}$ (for the topology
$\frak T_p$ of
pointwise convergence) extending $\hat\mu$.   The identity map
$\iota:C(\coint{0,\infty})_0\to\BbbR^{\coint{0,\infty}}$ is continuous for
$\frak T_c$ and $\frak T_p$, so the image measure $\mu_W\iota^{-1}$ is a
Radon measure on $\BbbR^{\coint{0,\infty}}$ (418I).   If
$E\subseteq\BbbR^{\coint{0,\infty}}$ is a Baire set, then

\Centerline{$\mu_W\iota^{-1}[E]=\mu_W(E\cap C(\coint{0,\infty})_0)
=\hat\mu E$,}

\noindent so $\mu_W\iota^{-1}$ agrees with $\tilde\mu$ on Baire sets,
and the two must be equal.
Now $C(\coint{0,\infty})_0$ is $\tilde\mu$-conegligible,
just because its complement has empty inverse image under $\iota$.
So $\mu_W$ is also the subspace measure on $C(\coint{0,\infty})_0$
induced by $\tilde\mu$.

Equally, since of course $C(\coint{0,\infty})_0$ is a subspace of the set
$\Cdlg$ of \cadlag\ functions from $\coint{0,\infty}$ to $\Bbb R$,
$\mu_W$ is the subspace measure induced by the measure $\ddot\mu$ of
Theorem 455O.
}%end of comment

\leader{*477C}{}\cmmnt{ I star the next theorem because it is very hard
work and will not be relied on later.
Nevertheless I think the statement, at
least, should be part of your general picture of Brownian motion.

\medskip

\noindent}{\bf Theorem} For $\alpha>0$, define
$f_{\alpha}:\BbbR^{\Bbb N}\to\Omega=C(\coint{0,\infty})_0$ by setting
$f_{\alpha}(z)(t)
=\sqrt\alpha(\sum_{i<n}z(i)+\Bover1{\alpha}(t-n\alpha)z(n))$
when $z\in\BbbR^{\Bbb N}$, $n\in\Bbb N$ and
$n\alpha\le t\le(n+1)\alpha$.   Give $\Omega$ its topology $\frak T_c$ of
uniform convergence on compact sets, and $\BbbR^{\Bbb N}$ its product
topology;  then $f_{\alpha}$ is continuous.
For a Radon probability measure $\nu$ on $\Bbb R$,
let $\mu_{\nu\alpha}$ be the image Radon measure
$\nu^{\Bbb N}f_{\alpha}^{-1}$ on $\Omega$, where $\nu^{\Bbb N}$ is the
product measure on $\BbbR^{\Bbb N}$.   Let $\mu_W$ be the Radon measure of
477B, and $U$ a neighbourhood of $\mu_W$ in the space
$P_{\text{R}}(\Omega)$ of Radon probability measures on $\Omega$
for the narrow topology\cmmnt{ (437Jd)}.   Then there is a
$\delta>0$ such
that $\mu_{\nu\alpha}\in U$ whenever $\alpha\in\ocint{0,\delta}$ and $\nu$
is a Radon probability measure on $\Bbb R$ with mean $0=\int x\,\nu(dx)$
and variance $1=\int x^2\nu(dx)$ and

%\vskip2pt

\hfill$\biggerint_{\{x:|x|\ge\delta/\sqrt{\alpha}\}}x^2\nu(dx)\le\delta$.
\hfill($\dagger$)

\cmmnt{\medskip

\noindent{\bf Remark} The idea is that, for a given $\alpha$ and $\nu$, we
consider a random walk with independent identically distributed steps,
with expectation 0 and variance $\alpha$, at time intervals of $\alpha$,
and then interpolate to get a continuous function on $\coint{0,\infty}$;
and that if the step-lengths are small the result should look like Brownian
motion.   Moreover, this ought not to depend on the distribution $\nu$;
but in order to apply the Central Limit Theorem in a sufficiently uniform
way, we need the extra regularity condition ($\dagger$).
On first reading you may well prefer to fix on a
particular distribution $\nu$ with mean $0$ and expectation $1$ (e.g., the
distribution which gives measure $\bover12$ to each of $\{1\}$ and
$\{-1\}$), so that ($\dagger$) is satisfied whenever $\alpha$ is small
enough compared with $\delta$.}

\proof{ For $\delta>0$ I will write $Q(\delta)$ for the set
of pairs $(\nu,\alpha)$ such that $\nu$ is a
Radon probability measure on $\Bbb R$ with mean $0$ and variance $1$,
$0<\alpha\le\delta$ and
$\int_{\{x:|x|\ge\delta/\sqrt{\alpha}\}}x^2\nu(dx)\le\delta$.
Note that $Q(\delta')\subseteq Q(\delta)$ when $\delta'\le\delta$.

\medskip

{\bf (a)(i)} If $\gamma$, $\epsilon>0$ there is a $\delta>0$ such
that whenever $(\nu,\alpha)\in Q(\delta)$, $s$, $t\ge 0$
are multiples of $\alpha$
such that $t-s\ge\gamma$, and $I\subseteq\Bbb R$ is an interval (open,
closed or half-open), then

\Centerline{$
|\mu_{\nu\alpha}\{\omega:\omega\in\Omega,\,\omega(t)-\omega(s)\in I\}
-\Bover1{\sqrt{2\pi(t-s)}}\int_Ie^{-x^2/2(t-s)}dx|\le\epsilon$.}

\noindent\Prf\ For $\delta>0$, $x\in\Bbb R$ set
$\psi_{\delta}(x)=x^2$ if $|x|>\delta$, $0$ if $|x|\le\delta$.
Let $\eta>0$ be such that whenever $Y_1,\ldots,Y_k$ are
independent random variables with finite variance and zero expectation,
$\sum_{i=1}^k\Var(Y_i)=1$ and
$\sum_{i=1}^k\Expn(\psi_{\eta}(Y_i))\le\eta$, then

\Centerline{$|\Pr(\sum_{i=1}^kY_i\le\beta)
  -\Bover1{\sqrt{2\pi}}\int_{-\infty}^{\beta}e^{-x^2/2}dx|
\le\Bover{\epsilon}2$}

\noindent for every $\beta\in\Bbb R$ (274F);  observe that in this case

$$\eqalignno{|\Pr(\sum_{i=1}^k&Y_i<\beta)
  -\Bover1{\sqrt{2\pi}}\int_{-\infty}^{\beta}e^{-x^2/2}dx|\cr
&=\lim_{\beta'\uparrow\beta}|\Pr(\sum_{i=1}^kY_i\le\beta')
  -\Bover1{\sqrt{2\pi}}\int_{-\infty}^{\beta'}e^{-x^2/2}dx|
\le\Bover{\epsilon}2\cr}$$

\noindent for every $\beta\in\Bbb R$, so that

\Centerline{$|\Pr(\sum_{i=1}^kY_i\in J)
  -\Bover1{\sqrt{2\pi}}\int_Je^{-x^2/2}dx|
\le\epsilon$}

\noindent for every interval $J\subseteq\Bbb R$.

Set $\delta=\min(\eta,\eta\sqrt{\gamma})$.
If $I\subseteq\Bbb R$ is an interval,
$(\nu,\alpha)\in Q(\delta)$ and $s$, $t$ are multiples of $\alpha$ such
that $t-s\ge\gamma$, set $j=\Bover{s}{\alpha}$,
$k=\Bover{t-s}{\alpha}$ and $J=\sqrt{t-s}I=\sqrt{\bover{k}{\alpha}}I$.
Then

$$\eqalign{\mu_{\nu\alpha}\{\omega:\omega(t)-\omega(s)\in I\}
&=\nu^{\Bbb N}\{z:f_{\alpha}(z)(t)-f_{\alpha}(z)(s)\in I\}\cr
&=\nu^{\Bbb N}\{z:\sqrt{\alpha}\sum_{i=j}^{j+k-1}z(i)\in I\}
=\Pr(\sum_{i=0}^{k-1}Y_i\in J)\cr}$$

\noindent where $Y_i(z)=\Bover1{\sqrt{k}}z(j+i)$.   For each $i$,
the mean and variance of $Y_i$ are $0$ and $\bover1k$, because the mean
and expectation of $\nu$ are $0$ and $1$.   Next,

$$\eqalign{\sum_{i=0}^{k-1}\Expn(\psi_{\eta}(Y_i))
&=k\int_{\{x:|x|>\eta\sqrt{k}\}}\Bover1kx^2\nu(dx)
\le\int_{\{x:|x|>\eta\sqrt{\gamma/\alpha}\}}x^2\nu(dx)\cr
&\le\int_{\{x:|x|>\delta/\sqrt{\alpha}\}}x^2\nu(dx)
\le\delta
\le\eta,\cr}$$

\noindent so by the choice of $\eta$,

$$\eqalign{
&|\mu_{\nu\alpha}\{\omega:\omega\in\Omega,\,\omega(t)-\omega(s)\in I\}
  -\Bover1{\sqrt{2\pi(t-s)}}\int_Ie^{-x^2/2(t-s)}dx|\cr
&\mskip230mu
=|\Pr(\sum_{i=0}^{k-1}Y_i\in J)
  -\Bover1{\sqrt{2\pi}}\int_Je^{-x^2/2}dx|
\le\epsilon.  \text{ \Qed}\cr}$$

\medskip

\quad{\bf (ii)} If $\gamma$, $\epsilon>0$, there is a $\delta>0$
such that

\Centerline{$\mu_{\nu\alpha}
  \{\omega:\diam(\omega[\,[\beta,\beta+\gamma]\,])
>12\epsilon\}
\le\Bover{3\sqrt{\gamma}}{\epsilon}e^{-\epsilon^2/2\gamma}$}

\noindent whenever $(\nu,\alpha)\in Q(\delta)$ and
$\beta\ge 0$.   \Prf\ Let $\eta>0$ be such that

\Centerline{$6(\eta+\Bover{\sqrt{\gamma+\eta}}{\epsilon\sqrt{2\pi}}
   e^{-\epsilon^2/2(\gamma+\eta)})
\le\Bover{3\sqrt{\gamma}}{\epsilon}e^{-\epsilon^2/2\gamma}$,}

\noindent and let $\delta_0>0$ be such that

\Centerline{$
|\mu_{\nu\alpha}\{\omega:\omega\in\Omega,\,\omega(t)-\omega(s)\in I\}
-\Bover1{\sqrt{2\pi(t-s)}}\int_Ie^{-x^2/2(t-s)}dx|\le\eta$}

\noindent whenever $I\subseteq\Bbb R$ is an interval,
$(\nu,\alpha)\in Q(\delta_0)$
and $s$ and $t$ are
multiples of $\alpha$ such that $t-s\ge\bover14\gamma$.
Set $\delta=\min(\bover14\gamma,\bover12\eta,\delta_0)$.

Fix $(\nu,\alpha)\in Q(\delta)$.   Applying the last formula with
$I=[-\epsilon,\epsilon]$ and then taking complements,

$$\eqalign{\mu_{\nu\alpha}\{\omega:|\omega(t)-\omega(s)|>\epsilon\}
&\le\eta+\Bover{2}{\sqrt{2\pi(t-s)}}
  \int_{\epsilon}^{\infty}e^{-x^2/2(t-s)}dx\cr
&=\eta+\Bover{2}{\sqrt{2\pi}}
  \int_{\epsilon/\sqrt{t-s}}^{\infty}e^{-x^2/2}dx\cr
&\le\eta+\Bover{2}{\sqrt{2\pi}}
  \int_{\epsilon/\sqrt{\gamma+\eta}}^{\infty}e^{-x^2/2}dx
\le\eta
  +\Bover{\sqrt{\gamma+\eta}}
  {\epsilon}e^{-\epsilon^2/2(\gamma+\eta)}\cr}$$

\noindent whenever $s$, $t$ are multiples of $\alpha$ such that
$\bover14\gamma\le t-s\le\gamma+\eta$, using
274Ma for the last step, as in part (c) of the proof of 477B.
Now if $s$, $t$ are multiples of $\alpha$ such
that $s\le t\le\gamma+\eta$, either $t-s\ge\bover14\gamma$ and

\Centerline{$\mu_{\nu\alpha}\{\omega:|\omega(t)-\omega(s)|>2\epsilon\}
\le\eta+\Bover{\sqrt{\gamma+\eta}}
  {\epsilon}e^{-\epsilon^2/2(\gamma+\eta)}$,}

\noindent or $t\le s+\bover14\gamma$ and
there is a multiple $u$ of $\alpha$ such that
$t+\bover14\gamma\le u\le t+\bover12\gamma$, in which case

$$\eqalign{\mu_{\nu\alpha}\{\omega:|\omega(t)-\omega(s)|>2\epsilon\}
&\le\mu_{\nu\alpha}\{\omega:|\omega(u)-\omega(s)|>\epsilon\}
  +\mu_{\nu\alpha}\{\omega:|\omega(u)-\omega(t)|>\epsilon\}\cr
&\le 2(\eta+\Bover{\sqrt{\gamma+\eta}}
  {\epsilon}e^{-\epsilon^2/2(\gamma+\eta)}).\cr}$$

Let $j$, $k$ be such that $\beta-\bover12\eta<j\alpha\le\beta$ and
$\beta+\gamma\le k\alpha<\beta+\gamma+\bover12\eta$.
We have

$$\eqalignno{
\mu_{\nu\alpha}\{\omega&:\diam(\omega[\,[\beta,\beta+\gamma]\,])
  >12\epsilon\}\cr
&=\nu^{\Bbb N}\{z:\diam(f_{\alpha}(z)[\,[\beta,\beta+\gamma]\,])
   >12\epsilon\}\cr
&\le\nu^{\Bbb N}\{z:\sup_{t\in[\beta,\beta+\gamma]}
   |f_{\alpha}(z)(t)-f_{\alpha}(z)(j\alpha)|
   >6\epsilon\}\cr
&\le\nu^{\Bbb N}\{z:\text{there is an }l\text{ such that }j<l\le k
  \text{ and }
  |f_{\alpha}(z)(l\alpha)-f_{\alpha}(z)(j\alpha)|>6\epsilon\}\cr
\displaycause{because $f_{\alpha}(z)$ is linear between its
determining values at multiples of $\alpha$}
&=\nu^{\Bbb N}\{z:\text{there is an }l\text{ such that }j<l\le k
  \text{ and }
  |\sum_{i=j}^{l-1}z(i)|>\Bover{6\epsilon}{\sqrt{\alpha}}\}\cr
&\le 3\sup_{j<l\le k}
  \nu^{\Bbb N}\{z:|\sum_{i=j}^{l-1}z(i)|
  >\Bover{2\epsilon}{\sqrt{\alpha}}\}\cr
\displaycause{Etemadi's lemma, 272V}
&=3\sup_{j<l\le k}
  \mu_{\nu\alpha}\{\omega:|\omega(l\alpha)-\omega(j\alpha)|
  >2\epsilon\}\cr
&\le 6(\eta+\Bover{\sqrt{\gamma+\eta}}{\epsilon}
  e^{-\epsilon^2/2(\gamma+\eta)})
\le\Bover{3\sqrt{\gamma}}{\epsilon}e^{-\epsilon^2/2\gamma},\cr}$$

\noindent as required.\ \Qed

\medskip

\quad{\bf (iii)}
If $\gamma$, $\epsilon>0$ there is a $\delta>0$ such that

\Centerline{$\mu_{\nu\alpha}\{\omega:\text{there are }s,\,t\in[0,\gamma]
   \text{ such that }|t-s|\le\delta
   \text{ and }|\omega(t)-\omega(s)|>\epsilon\}\le\epsilon$}

\noindent whenever $(\nu,\alpha)\in Q(\delta)$.
\Prf\ Set $\eta=\epsilon/12$, and let $k\ge 1$ be such that
$\Bover{6\gamma k}{\eta}e^{-k^2\eta^2/2}\le\epsilon$;  set
$m=\lfloor 2k^2\gamma\rfloor$.
By (ii), there is a $\delta\in\ocint{0,\bover1{2k^2}}$
such that

\Centerline{$\mu_{\nu\alpha}\{\omega:
  \diam(\omega[\,[\beta,\beta+\Bover1{k^2}]\,])>12\eta\}
\le\Bover{3}{k\eta}e^{-k^2\eta^2/2}$}

\noindent whenever $(\nu,\alpha)\in Q(\delta)$ and
$\beta\ge 0$.   Now, for such $\nu$ and $\alpha$,

$$\eqalign{&\mu_{\nu\alpha}\{\omega:\text{there are }s,\,t\in[0,\gamma]
   \text{ such that }|t-s|\le\delta
   \text{ and }|\omega(t)-\omega(s)|>\epsilon\}\cr
&\mskip150mu
\le\mu_{\nu\alpha}(\bigcup_{i<m}
  \{\omega:\diam(\omega[\,[\Bover{i}{2k^2},\Bover{i+2}{2k^2}]\,]>12\eta\})
  \cr
&\mskip150mu
\le\Bover{3m}{k\eta}e^{-k^2\eta^2/2}
\le\Bover{6\gamma k}{\eta}e^{-k^2\eta^2/2}
\le\epsilon,\cr}$$

\noindent as required.\ \Qed

\medskip

{\bf (b)} Suppose that $0=t_0<t_1<\ldots<t_n$ and that $E_0,\ldots,E_{n-1}$
are intervals in $\Bbb R$;  set
$E=\{\omega:\omega\in\Omega$, $\omega(t_{i+1})-\omega(t_i)\in E_i$ for
$i<n\}$.   Then for every $\epsilon>0$ there is a $\delta>0$ such
that $\mu_WE\le 3\epsilon+\mu_{\nu\alpha}E$ whenever
$(\nu,\alpha)\in Q(\delta)$.   \Prf\ Of course

\Centerline{$\mu_WE
=\prod_{i<n}\Bover1{\sqrt{2\pi(t_{i+1}-t_i)}}
  \int_{E_i}e^{-x^2/2(t_{i+1}-t_i)}dx$.}

\noindent For $\eta>0$ and $i<n$, let $F_{i\eta}$ be the interval
$\{x:[x-2\eta,x+2\eta]\subseteq E_i\}$.
Set $\gamma=\bover12\min_{i<n}(t_{i+1}-t_i)$;  let
$\eta\in\ocint{0,\gamma}$ be such that

\Centerline{$\prod_{i<n}\Bover1{\sqrt{2\pi\gamma_i}}
  \int_{F_{i\eta}}e^{-x^2/2\gamma_i}dx
\ge\mu_WE-\epsilon$}

\noindent whenever $|\gamma_i-(t_{i+1}-t_i)|\le\eta$ for every $i<n$.
Next, by (a-i) and (a-iii), there is a $\delta\in\ocint{0,\bover12\eta}$
such that

$$\eqalign{\prod_{i<n}\Bover1{\sqrt{2\pi(s_{i+1}-s_i)}}
   &\int_{F_{i\eta}}e^{-x^2/2(s_{i+1}-s_i)}dx\cr
&\le\epsilon+\prod_{i<n}\mu_{\nu\alpha}\{\omega:
    \omega(s_{i+1})-\omega(s_i)\in F_{i\eta}\},\cr}$$

$$\eqalign{\mu_{\nu\alpha}\{\omega:
  \text{there are }&s,\,t\in[0,t_n+\eta]\cr
&\text{ such that }|s-t|\le\delta
  \text{ and }|\omega(s)-\omega(t)|>\eta\}
\le\epsilon\cr}$$

\noindent whenever $(\nu,\alpha)\in Q(\delta)$ and
$s_0,\ldots,s_n$ are multiples of $\alpha$ such that
$s_{i+1}-s_i\ge\gamma$ for every $i\le n$.   Take any
$(\nu,\alpha)\in Q(\delta)$,
and for each $i\le n$ let $s_i$ be a multiple of $\alpha$ such
that $t_i\le s_i\le t_i+\alpha$.   Then

$$\eqalign{\{\omega:
   \omega&(s_{i+1})-\omega(s_i)\in F_{i\eta}\text{ for every }i<n\}
\setminus E\cr
&=\bigcup_{i<n}\{\omega:\omega(s_{i+1})-\omega(s_i)\in F_{i\eta},\,
\omega(t_{i+1})-\omega(t_i)\notin E_i\}\cr
&\subseteq\bigcup_{i<n}\{\omega:
|(\omega(s_{i+1})-\omega(s_i))-(\omega(t_{i+1}-\omega(t_i))|>2\eta\}\cr
&\subseteq\bigcup_{i\le n}\{\omega:|\omega(s_i)-\omega(t_i)|>\eta\}\cr
&\subseteq\{\omega:\text{there are }s,\,t\in[0,t_n+\eta]\cr
&\mskip100mu\text{ such that }
|s-t|\le\delta\text{ and }|\omega(s_i)-\omega(t_i)|>\eta\},\cr}$$

\noindent so

\Centerline{$\mu_{\nu\alpha}\{\omega:
  \omega(s_{i+1})-\omega(s_i)\in F_{i\eta}$ for every $i<n\}
\le\epsilon+\mu_{\nu\alpha}E$.}

\noindent Next, if $s_i=k_i\alpha$ for each $i$,

$$\eqalignno{\mu_{\nu\alpha}\{\omega:\omega&(s_{i+1})-\omega(s_i)
  \in F_{i\eta}\text{ for every }i<n\}\cr
&=\nu^{\Bbb N}\{z:f_{\alpha}(z)(s_{i+1})-f_{\alpha}(z)(s_i)
  \in F_{i\eta}
  \text{ for every }i<n\}\cr
&=\nu^{\Bbb N}\{z:\sqrt{\alpha}\sum_{j=k_i}^{k_{i+1}-1}z(j)\in F_{i\eta}
  \text{ for every }i<n\}\cr
&=\prod_{i<n}\nu^{\Bbb N}\{z:\sqrt{\alpha}\sum_{j=k_i}^{k_{i-1}}z(j)
  \in F_{i\eta}\}\cr
&=\prod_{i<n}\mu_{\nu\alpha}\{\omega:\omega(s_{i+1})-\omega(s_i)
  \in F_{i\eta}\}.\cr}$$

\noindent So

$$\eqalignno{\mu_WE
&\le\epsilon+\prod_{i<n}\Bover1{\sqrt{2\pi(s_{i+1}-s_i)}}
   \int_{F_{i\eta}}e^{-x^2/2(s_{i+1}-s_i)}dx\cr
\displaycause{because $|(s_{i+1}-s_i)-(t_{i+1}-t_i)|\le\alpha\le\eta$
for $i<n$}
&\le 2\epsilon
  +\prod_{i<n}\mu_{\nu\alpha}\{\omega:\omega(s_{i+1})-\omega(s_i)
      \in F_{i\eta}\}\cr
\displaycause{because $s_{i+1}-s_i\ge\gamma$ for every $i<n$}
&=2\epsilon+\mu_{\nu\alpha}\{\omega:\omega(s_{i+1})-\omega(s_i)
  \in F_{i\eta}\text{ for every }i<n\}\cr
&\le 3\epsilon+\mu_{\nu\alpha}E,\cr}$$

\noindent as required.\ \Qed

\medskip

{\bf (c)(i)} For $k\in\Bbb N$ let $\delta_k>0$ be such that
$\mu_{\nu\alpha}G_k\le 2^{-k}$ whenever $(\nu,\alpha)\in Q(\delta_k)$,
where

\Centerline{$G_k=\{\omega:$ there are $s$, $t\in[0,k]$ such that
   $|t-s|\le\delta_k$ and $|\omega(t)-\omega(s)|>2^{-k}\}$;}

\noindent such exists by (a-iii) above.   For $k$, $n\in\Bbb N$ set
$H_{kn}=\bigcup_{i\le n}G_{k+i}$.   If $k\in\Bbb N$ and
$\sequencen{\omega'_n}$ is a sequence such that
$\omega'_n\in\Omega\setminus H_{kn}$ for every $n\in\Bbb N$,
$\{\omega'_n:n\in\Bbb N\}$ is relatively compact in $\Omega$.   \Prf\
If $\gamma\ge 0$ and $\epsilon>0$, there is an $n\in\Bbb N$
such that $2^{-k-n}\le\epsilon$ and $k+n\ge\gamma$;  now for $m\ge n$,
$\omega'_m\notin G_{k+n}$ so
$|\omega'_m(t)-\omega'_m(s)|\le\epsilon$ whenever
$s$, $t\in[0,\gamma]$ and $|s-t|\le\delta_{k+n}$.   Of course there is a
$\delta\in\ocint{0,\delta_{k+n}}$ such that
$|\omega'_m(s)-\omega'_m(t)|\le\epsilon$ whenever $m<k+n$ and $s$,
$t\in[0,\gamma]$ are such that $|s-t|\le\delta$.   Since $\omega'_n(0)=0$
for every $n$, the conditions of
4A2U(e-ii) are satisfied, and $\{\omega'_n:n\in\Bbb N\}$ is relatively
compact in $C(\coint{0,\infty})$, therefore in its closed subset $\Omega$.\
\Qed

Now if we have a compact set
$K\subseteq\Omega$, an open set $G\subseteq\Omega$ including $K$, and
$k\in\Bbb N$, there are an $n\in\Bbb N$ and a finite set
$I\subseteq\coint{0,\infty}$ such that $\omega'\in G\cup H_{kn}$ whenever
$\omega\in K$, $\omega'\in\Omega$ and $|\omega'(s)-\omega(s)|\le 2^{-n}$
for every $s\in I$.   \Prf\Quer\ Otherwise, let $\sequence{i}{q_i}$
enumerate $\Bbb Q\cap\coint{0,\infty}$.   For each $n\in\Bbb N$ we have
$\omega_n\in K$ and $\omega'_n\in\Omega\setminus(G\cup H_{kn})$ such that
$|\omega'_n(q_i)-\omega_n(q_i)|\le 2^{-n}$ for every $i\le n$.
Since the topology $\frak T_c$ on $\Omega$ is metrizable (4A2U(e-i)),
and both
$\{\omega_n:n\in\Bbb N\}$ and $\{\omega'_n:n\in\Bbb N\}$ are relatively
compact, there is a strictly increasing sequence $\sequence{i}{n_i}$ such
that $\omega=\lim_{i\to\infty}\omega_{n_i}$ and
$\omega'=\lim_{i\to\infty}\omega'_{n_i}$ are both defined (use 4A2Lf
twice).   Since $|\omega'(q)-\omega(q)|
=\lim_{i\to\infty}|\omega'_{n_i}(q)-\omega_{n_i}(q)|$ is zero for
every $q\in\Bbb Q\cap\coint{0,\infty}$, $\omega=\omega'$;  but
$\omega\in K$ and $\omega'\notin G$, so this is impossible.\ \Bang\Qed

\medskip

\quad{\bf (ii)} Suppose that $G\subseteq\Omega$ is open and
$\gamma<\mu_WE$.   Then there is a $\delta>0$ such that
$\mu_{\nu\alpha}G>\gamma$ whenever $(\nu,\alpha)\in Q_{\delta}$.
\Prf\ Let $K\subseteq G$ be a compact set such that
$\mu_WK>\gamma$.   Let $k\in\Bbb N$, $\epsilon>0$ be such that
$\mu_WK\ge\gamma+\epsilon+2^{-k+1}$.
By (i), there are an $n\in\Bbb N$ and
a finite set $I\subseteq\coint{0,\infty}$ such that
$\omega'\in G\cup H_{kn}$ whenever $\omega'\in\Omega$, $\omega\in K$ and
$|\omega'(t)-\omega(t)|\le 2^{-n}$ for every $t\in I$;  of course we can
suppose that $0\in I$ and that $\#(I)\ge 2$.   Enumerate $I$ in increasing
order as $\langle t_i\rangle_{i\le m}$.   For $z\in\BbbZ^m$, set

\Centerline{$E_z
=\{\omega:\omega\in\Omega$,
 $\lfloor 2^nm(\omega(t_{i+1}-\omega(t_i))\rfloor=z(i)$ for every $i<m\}$;}

\noindent set $D=\{z:z\in\BbbZ^m$, $E_z\cap K\ne\emptyset\}$ and
$F=\bigcup_{z\in D}E_z$.   If $z\in D$ and $\omega'\in E_z$, there is
an $\omega\in K\cap E_z$, in which case

\Centerline{$|(\omega'(t_{i+1})-\omega'(t_i))-(\omega(t_{i+1}-\omega(t_i))|
\le\Bover{2^{-n}}m$ for every $i<m$,}

\Centerline{$|\omega'(t_i)-\omega(t_i)|\le 2^{-n}$ for every $i\le m$}

\noindent and $\omega'\in G\cup H_{kn}$.   Thus $F\subseteq G\cup H_{kn}$.
As $K$ is compact, $\{\omega(t_i):\omega\in K\}$ is bounded for every $i$
and $D$ is finite.
By (b) there is a $\delta>0$ such that $\delta\le\delta_{k+i}$
for every $i\le n$ and

\Centerline{$\mu_WE_z\le\Bover{\epsilon}{1+\#(D)}+\mu_{\nu\alpha}E_z$}

\noindent whenever $z\in D$ and $(\nu,\alpha)\in Q(\delta)$.
Now, for such $\nu$ and $\alpha$,

$$\eqalignno{\epsilon+2^{-k+1}+\gamma
&\le\mu_WK
\le\mu_WF
=\sum_{z\in D}\mu_WE_z
\le\epsilon+\sum_{z\in D}\mu_{\nu\alpha}E_z\cr
&=\epsilon+\mu_{\nu\alpha}F
\le\epsilon+\mu_{\nu\alpha}G+\sum_{i=0}^n\mu_{\nu\alpha}G_{k+i}\cr
&\le\epsilon+\mu_{\nu\alpha}G+\sum_{i=0}^n2^{-k-i}
<\epsilon+2^{-k+1}+\mu_{\nu\alpha}G\cr}$$

\noindent and $\mu_{\nu\alpha}G>\gamma$, as required.\ \Qed

\medskip

\quad{\bf (iii)} So if $U$ is a neighbourhood of $\mu_W$ for the narrow
topology on $P_{\text{R}}(\Omega)$,
there is a $\delta>0$ such that $\mu_{\nu\alpha}\in U$
whenever $(\nu,\alpha)\in Q(\delta)$.   \Prf\ There are
open sets $G_0,\ldots,G_n$ and $\gamma_0,\ldots,\gamma_n$ such that
$\gamma_i<\mu_WG_i$ for each $i<n$ and $U$ includes
$\{\mu:\mu\in P_{\text{R}}(\Omega)$, $\mu G_i>\gamma_i$ for every $i<n\}$.
But from (ii) we see that for each $i\le n$ there will be a
$\delta'_i>0$ such that $\mu_{\nu\alpha}G_i>\gamma_i$ for every
$i$ whenever $(\nu,\alpha)\in Q(\delta'_i)$;  so setting
$\delta=\min_{i\le n}\delta'_i$ we get the result.\ \Qed

And this is just the conclusion declared in the
statement of the theorem, rephrased in the
language developed in the course of the proof.
}%end of proof of 477C

\leader{477D}{Multidimensional Brownian motion}\cmmnt{ In \S\S478-479
we shall need the theory of Brownian motion in
$r$-dimensional space.   I sketch the relevant details.}
Fix an integer $r\ge 1$.

\spheader 477Da Let $\mu_{W1}$ be the Radon probability
measure on $\Omega_1=C(\coint{0,\infty})_0$ described
in 477B;  I will call it {\bf one-dimensional Wiener measure}.
We can identify the power $\Omega_1^r$ with
$\Omega=C(\coint{0,\infty};\BbbR^r)_0$, the space of continuous functions
$\omega:\coint{0,\infty}\to\BbbR^r$ such that $\omega(0)=0$,
with the topology of uniform convergence on compact sets;  note that
$\Omega_1$ is Polish\cmmnt{ (4A2U(e-i))}, so $\Omega_1^r$ also is.
Because $\Omega_1$ is separable and metrizable, the c.l.d.\
product measure $\mu_{W1}^r$ measures every Borel
set\cmmnt{ (4A3Dc, 4A3E)}, while it is inner regular with respect to
the compact sets\cmmnt{ (412Sb)}, so
it is a Radon measure.   I will say that
$\mu_W=\mu_{W1}^r$, interpreted as a measure on
$C(\coint{0,\infty};\BbbR^r)_0$, is {\bf $r$-dimensional Wiener measure}.

\cmmnt{As observed in 477B,}
$\mu_{W1}$ is the subspace measure on $\Omega_1$ induced by the
distribution $\hat\mu$ of the process
$\langle X_t\rangle_{t\ge 0}$ in 477A.   \cmmnt{So }$\mu_W$ here,
regarded as a
measure on $C(\coint{0,\infty})_0^r$, is the subspace measure induced by
$\hat\mu^r$ on
$(\BbbR^{\coint{0,\infty}})^r
  \cong\BbbR^{\coint{0,\infty}\times r}$\cmmnt{ (254La)}.

\spheader 477Db For $\omega\in\Omega$, $t\ge 0$ and $i<r$,
set $X_t^{(i)}(\omega)=\omega(t)(i)$.   Then
$\langle X_t^{(i)}\rangle_{t\ge 0,i<r}$ is a centered
Gaussian process, with covariance matrix

$$\eqalign{\Expn(X_s^{(i)}\times X_t^{(j)})
&=0\text{ if }i\ne j,\cr
&=\min(s,t)\text{ if }i=j.\cr}$$

\prooflet{\noindent\Prf\ Taking $\mu$, $\hat\mu$ and $\hat\mu^r$ as in (a),
$\hat\mu^r$, like $\hat\mu$, is a centered Gaussian
distribution (456Be);  but it is easy to check from the formula in 454J(i)
that $\hat\mu^r$ can be identified with the distribution of the family
$\langle X_t^{(i)}\rangle_{t\ge 0,i<r}$.   So
$\langle X_t^{(i)}\rangle_{t\ge 0,i<r}$ is a centered Gaussian process.
As for the covariance matrix, if $i\ne j$ then $X_s^{(i)}$ and $X_t^{(j)}$
are determined by different factors in the product $\Omega=\Omega_1^r$,
so must be independent;  while if $i=j$ then $(X^{(i)}_s,X^{(i)}_t)$ have
the same joint distribution as $(X_s,X_t)$ in 477A.\ \Qed}

\spheader 477Dc\cmmnt{ We shall need a variety of characterizations of
the Radon measure $\mu_W$.

\medskip

\quad}{\bf (i)} $\mu_W$ is the only Radon probability measure on $\Omega$
such that the process $\langle X_t^{(i)}\rangle_{t\ge 0,i<r}$ described in
(b) is a Gaussian process with the covariance matrix there.
\prooflet{\Prf\ Suppose $\nu$ is another measure with these properties.
The distribution of $\langle X_t^{(i)}\rangle_{t\ge 0,i<r}$ (with respect
to $\nu$) must be a centered Gaussian process on
$\BbbR^{r\times\coint{0,\infty}}\cong(\BbbR^{\coint{0,\infty}})^r$, and
because it has the same covariance matrix it must be equal to $\hat\mu^r$,
by 456Bb.   But this says just that $\langle X_t^{(i)}\rangle_{t\ge 0,i<r}$
has the same joint distribution with respect to $\mu_W$ and $\nu$.   By
454N, $\nu=\mu_W$.\ \Qed}

\medskip

\quad{\bf (ii)}
Another way of looking at the family
$\langle X^{(i)}_t\rangle_{i<r,t\ge 0}$ is to write $X_t(\omega)=\omega(t)$
for $t\ge 0$, so that $\langle X_t\rangle_{t\ge 0}$ is now a family of
$\BbbR^r$-valued random variables defined on $\Omega$.
We can describe its distribution in terms matching those of 455Q and 477A,
which become

\inset{(i) $X_0=0$ everywhere\cmmnt{ (on $\Omega$, that is)};

(ii) whenever $0\le s<t$ then $\Bover1{\sqrt{t-s}}(X_t-X_s)$
has the standard Gaussian distribution $\mu_G^r$\cmmnt{ (that is,
$\omega\mapsto\Bover1{\sqrt{t-s}}(\omega(t)-\omega(s))$ is \imp\ for
$\mu_W$ and $\mu_G^r$)};

(iii) whenever $0\le t_1<\ldots<t_n$, then $X_{t_2}-X_{t_1},\ldots,
X_{t_n}-X_{t_{n-1}}$ are independent\cmmnt{ (that is, taking $\Tau_i$ to
be the $\sigma$-algebra
$\{\{\omega:\omega(t_{i+1})-\omega(t_i)\in E\}:E\subseteq\BbbR^r$
is a Borel set$\}$, $\Tau_1,\ldots,\Tau_{n-1}$ are independent)}.
}

\noindent Note that
these properties also determine the Radon measure $\mu_W$.
\prooflet{\Prf\ Once again, suppose $\nu$ is a Radon probability measure on
$\Omega$ for which (ii) and (iii) are true.   We wish to show that $\mu_W$
and $\nu$ give the same distribution to
$\langle X^{(i)}_t\rangle_{i<r,t\ge 0}$.   If $0=t_0<t_1<\ldots<t_n$, we
know that $\mu_W$ and $\nu$ give the same distribution to each of the
differences $Y_j=X_{t_{j+1}}-X_{t_j}$ (or, if you prefer, to each of the
families $\ofamily{i}{r}{Y^{(i)}_j}$, where
$Y^{(i)}_j=X^{(i)}_{t_{j+1}}-X^{(i)}_{t_j}$);  moreover,
if $\Sigma_j$ is the $\sigma$-algebra generated by $\{Y^{(i)}_j:i<r\}$ for
each $j$, then $\mu_W$ and $\nu$ agree that $\ofamily{j}{n}{\Sigma_j}$ is
independent.   So $\mu_WE=\nu E$ whenever $E$ is of the form
$\bigcap_{j<n}E_j$ where $E_j\in\Sigma_j$ for each $j<n$.   By the Monotone
Class Theorem, $\mu_W$ and $\nu$ agree on the $\sigma$-algebra
$\Sigma$ generated by
sets of this type, which is the $\sigma$-algebra generated by
$\{Y^{(i)}_j:i<r$, $j<n\}$.   But as $X_{t_j}=\sum_{i<j}Y_i$ for every
$j\le n$, every $X^{(i)}_{t_j}$ is $\Sigma$-measurable, and $\mu_W$ and
$\nu$ give the same distribution to
$\langle X^{(i)}_{t_j}\rangle_{i<r,j\le n}$.  As this is true whenever
$0=t_0<\ldots<t_n$, $\mu_W$ and $\nu$ give the same distribution to the
whole family $\langle X^{(i)}_t\rangle_{i<r,t\ge 0}$, and must be equal.\
\Qed}

\cmmnt{
\spheader 477Dd In order to apply Theorem 455U, we need to go a little
deeper, in order to relate the product-measure definition of $\mu_W$ to the
construction in 455P.   I will use the ideas of part (b) of the proof of
455R.   Consider the process $\langle X^{(i)}_t\rangle_{t\ge 0,i<r}$
and the associated distribution $\hat\mu^r$ on
$(\BbbR^{\coint{0,\infty}})^r\cong(\BbbR^r)^{\coint{0,\infty}}$.
Setting $X_t=\ofamily{i}{r}{X^{(i)}_t}$, $\langle X_t\rangle_{t\ge 0}$ is
an $\BbbR^r$-valued process satisfying the conditions of 455Q with
$U=\BbbR^r$.   \Prf\ $X_0=0$ a.e.\  because every $X^{(i)}_0$ is zero a.e.
If $0\le s<t$ then $X_t-X_s=\ofamily{i}{r}{X^{(i)}_t-X^{(i)}_s}$ has the
same distribution as $X_{t-s}$ because $X^{(i)}_t-X^{(i)}_s$ has the same
distribution as $X^{(i)}_{t-s}$ for each $i$ and
$\ofamily{i}{r}{X^{(i)}_t-X^{(i)}_s}$,
$\ofamily{i}{r}{X^{(i)}_{t-s}}$ are both independent.   If
$0\le t_0<t_1<\ldots<t_n$ then
$\langle X^{(i)}_{t_{j+1}}-X^{(i)}_{t_j}\rangle_{i<r,j<n}$ is independent
so $\ofamily{j}{n}{X_{t_{j+1}}-X_{t_j}}$ is independent (using 272K, or
otherwise).   Finally, when $t\downarrow 0$,
$X_t\to 0$ in measure because $X^{(i)}_t\to 0$ in
measure for each $i$.\ \Qed

For $t>0$, let $\lambda_t$ be the distribution of
$X_t$.   Then $\lambda_t$ is the centered Gaussian distribution
on $\BbbR^r$ with
covariance matrix $\langle\sigma_{ij}\rangle_{i,j<r}$ where
$\sigma_{ij}=t$
if $i=j$ and $0$ if $i\ne j$ (456Ba, with $T(\omega)=\omega(t)$ for
$\omega\in\BbbR^{\coint{0,\infty}\times r}
\cong(\BbbR^r)^{\coint{0,\infty}}$).
By 455R, the process of 455P can be applied to
$\langle\lambda_t\rangle_{t>0}$ to give us a measure
$\hat\nu$ on $(\BbbR^r)^{\coint{0,\infty}}$,
the completion of a Baire measure, such that

$$\eqalign{\hat\nu\{\omega:\omega(t_i)\in F_i\text{ for every }i\le n\}
&=\Pr(X_{t_i}\in F_i\text{ for every }i\le n)\cr
&=\hat\mu^r\{\omega:\omega(t_i)\in F_i\text{ for every }i\le n\}\cr}$$

\noindent whenever $F_0,\ldots,F_n\subseteq\BbbR^r$ are Borel sets and
$t_0,\ldots,t_n\in\coint{0,\infty}$.
Since sets of this kind generate the Baire $\sigma$-algebra of
$(\BbbR^r)^{\coint{0,\infty}}$,
$\hat\nu$ must be equal to $\hat\mu^r$, that is, $\hat\mu^r$
is the result of applying 455P to $\langle\lambda_t\rangle_{t>0}$.

By 455H, $\hat\nu$
has a unique extension to a measure $\tilde\nu$ on
$(\BbbR^r)^{\coint{0,\infty}}$ which is a Radon measure for the product
topology.   But if we write
$\iota:C(\coint{0,\infty};\BbbR^r)_0\to(\BbbR^r)^{\coint{0,\infty}}$ for
the identity map, the image measure $\mu_W\iota^{-1}$ is a Radon measure on
$(\BbbR^r)^{\coint{0,\infty}}$ for the
product topology and extends $\hat\mu^r$, so must be equal to $\tilde\nu$.
Thus $\tilde\nu C(\coint{0,\infty};\BbbR^r)_0=1$.
Of course $C(\coint{0,\infty};\BbbR^r)_0$ is included in the space of
\callal\ functions from $\coint{0,\infty}$ to $\BbbR^r$, so that we have a
strengthening of the results in \S455.   Similarly, writing $\ddot\nu$ for
the subspace measure induced by $\hat\nu$ or $\hat\mu^r$
on the space $\Cdlg$ of \cadlag\ functions from
$\coint{0,\infty}$ to $\BbbR^r$, $\mu_W$ is the subspace measure
on $C(\coint{0,\infty};\BbbR^r)_0$ induced by $\ddot\nu$.

By 4A3Wa, every Baire subset of $\Cdlg$ is the intersection of $\Cdlg$ with
a Baire subset of $(\BbbR^r)^{\coint{0,\infty}}$, and is therefore measured
by $\ddot\nu$.   In particular,
$C(\coint{0,\infty};\BbbR^r)$ and $C(\coint{0,\infty};\BbbR^r)_0$ are
measured by $\ddot\nu$ (4A3Wd).
}%end of comment

\leader{477E}{Invariant transformations of Wiener measure:  Proposition}
Let $r\ge 1$ be an integer, and $\mu_W$ Wiener measure on
$\Omega=C(\coint{0,\infty};\BbbR^r)_0$.
Let $\hat\mu^r$ be the product measure on
$(\BbbR^{\coint{0,\infty}})^r$ as described in 477D.

(a) Suppose that
$f:(\BbbR^{\coint{0,\infty}})^r\to(\BbbR^{\coint{0,\infty}})^r$
is \imp\ for $\hat\mu^r$, and that $\Omega_0\subseteq\Omega$ is a
$\mu_W$-conegligible set such that $f[\Omega_0]\subseteq\Omega_0$.
Then $f\restr\Omega_0$ is \imp\ for the subspace measure induced by
$\mu_W$ on $\Omega_0$.

(b) Suppose that
$\hat T:\BbbR^{r\times\coint{0,\infty}}\to\BbbR^{r\times\coint{0,\infty}}$
is a linear operator such that, for $i$, $j<r$ and $s$, $t\ge 0$,

$$\eqalign{\int(\hat T\omega)(i,s)(\hat T\omega)(j,t)\hat\mu^r(d\omega)
&=\min(s,t)\text{ if }i=j,\cr
&=0\text{ if }i\ne j.\cr}$$

\noindent Then, identifying $\BbbR^{r\times\coint{0,\infty}}$
with $(\BbbR^{\coint{0,\infty}})^r$,
$\hat T$ is \imp\ for $\hat\mu^r$.

(c) Suppose that $t\ge 0$.   Define $S_t:\Omega\to\Omega$ by setting
$(S_t\omega)(s)=\omega(s+t)-\omega(s)$ for $s\ge 0$ and $\omega\in\Omega$.
Then $S_t$ is \imp\ for $\mu_W$.

(d) Let $T:\BbbR^r\to\BbbR^r$ be an orthogonal
transformation.   Define $\tilde T:\Omega\to\Omega$ by setting
$(\tilde T\omega)(t)=T(\omega(t))$ for $t\ge 0$ and $\omega\in\Omega$.
Then $\tilde T$ is an automorphism of $(\Omega,\mu_W)$.

(e) Suppose that $\alpha>0$.   Define $U_{\alpha}:\Omega\to\Omega$ by
setting $U_{\alpha}(\omega)(t)=\Bover1{\sqrt{\alpha}}\omega(\alpha t)$ for
$t\ge 0$ and $\omega\in\Omega$.   Then $U_{\alpha}$ is an automorphism of
$(\Omega,\mu_W)$.

(f) Set

\Centerline{$\Omega_0=\{\omega:\omega\in\Omega$,
$\lim_{t\to\infty}\Bover1t\omega(t)=0\}$,}

\noindent and define $R:\Omega_0\to\Omega_0$ by setting

$$\eqalign{(R\omega)(t)
&=t\omega(\Bover1t)\text{ if }t>0,\cr
&=0\text{ if }t=0.\cr}$$

\noindent Then $\Omega_0$ is $\mu_W$-conegligible and $R$ is an
automorphism of $\Omega_0$ with its subspace measure.

(g) Suppose that $1\le r'\le r$, and that $\mu_W^{(r')}$ is Wiener measure
on $C(\coint{0,\infty};\BbbR^{r'})_0$.   Define
$P:\Omega\to C(\coint{0,\infty};\BbbR^{r'})_0$ by setting
$(P\omega)(t)(i)=\omega(t)(i)$ for $t\ge 0$, $i<r'$ and $\omega\in\Omega$.
Then $\mu_W^{(r')}$ is the image measure $\mu_WP^{-1}$.

\proof{ The following arguments will unscrupulously identify
$C(\coint{0,\infty};\BbbR^r)_0$ with $C(\coint{0,\infty})_0^r$, and
$\BbbR^{r\times\coint{0,\infty}}$ with $(\BbbR^r)^{\coint{0,\infty}}$ and
$(\BbbR^{\coint{0,\infty}})^r$.

\medskip

{\bf (a)} Because $\mu_W$ is the subspace measure on $\Omega$ induced by
$\hat\mu^r$ (477Da), the subspace measure $\nu$
on $\Omega_0$ induced by $\mu_W$
is also the subspace measure on $\Omega_0$ induced by $\hat\mu^r$ (214Ce).
If $E\subseteq\Omega_0$ is measured by $\nu$, there is an
$F\in\dom\hat\mu^r$ such that $E=F\cap\Omega_0$, and now

\Centerline{$\nu E=\hat\mu^rF=\hat\mu^rf^{-1}[F]=\nu(\Omega_0\cap f^{-1}[F])
=\nu(f\restr\Omega_0)^{-1}[E]$.}

\noindent As $E$ is arbitrary, $f\restr\Omega_0$ is \imp\ for $\nu$.

\medskip

{\bf (b)} By 456Ba, $\hat\mu^r\hat T^{-1}$ is a centered Gaussian
distribution on $\BbbR^{r\times\coint{0,\infty}}$.   The hypothesis
asserts that its covariance matrix is the same as that of $\hat\mu^r$
(477Db), so that $\hat\mu^r=\hat\mu^r\hat T^{-1}$ (456Bb), that is,
$\hat T$ is \imp\ for $\hat\mu^r$.

\medskip

{\bf (c)} Define $\hat S_t:\BbbR^{r\times\coint{0,\infty}}
  \to\BbbR^{r\times\coint{0,\infty}}$ by setting
$(\hat S_t\omega)(i,s)=\omega(i,s+t)-\omega(i,s)$, this time for
$\omega\in\BbbR^{r\times\coint{0,\infty}}$, $i<r$ and $s\ge 0$.
Then $\hat S_t$ is linear, and for $s$, $u\in\coint{0,\infty}$, $i$, $j<r$

$$\eqalign{\int(\hat S_t\omega)&(i,s)(\hat S_t\omega)(j,u)
   \mu^r(d\omega)\cr
&=\int(\omega(i,s+t)-\omega(t))(\omega(j,u+t)-\omega(j,t))\hat\mu^r(d\omega)
   \cr
&=0\text{ if }i\ne j,\cr
&=\min(s+t,u+t)-\min(s+t,t)\cr
&\mskip100mu-\min(t,u+t)+\min(t,t)
=\min(s,u)\text{ if }i=j.\cr}$$

\noindent By (b), $\hat S_t$ is \imp\ for $\hat\mu^r$.   Now
$\hat S_t[\Omega]\subseteq\Omega$, so $S_t=\hat S_t\restr\Omega$ is \imp\
for $\mu W$, by (a).

\medskip

{\bf (d)} If we define
$\hat T:(\BbbR^r)^{\coint{0,\infty}}\to(\BbbR^r)^{\coint{0,\infty}}$
by setting
$(\hat T)(\omega)(t)=T(\omega(t))$ for $x\in(\BbbR^r)^{\coint{0,\infty}}$ and
$t\ge 0$, then $\hat T$ is linear.   Suppose that $T$ is defined by the
matrix $\langle\alpha_{ij}\rangle_{i,j<r}$.
For $\omega\in\BbbR^{r\times\coint{0,\infty}}$, $t\in\coint{0,\infty}$
and $i<r$,

\Centerline{$(\hat T\omega)(i,t)
=\sum_{k=0}^{r-1}\alpha_{ik}\omega(k,t)$.}

\noindent So, for $i$, $j<r$ and $s$, $t\ge 0$,

$$\eqalign{\int(\hat T\omega)(i,s)(\hat T\omega)(j,t)\hat\mu^r(d\omega)
&=\sum_{k=0}^{r-1}\sum_{l=0}^{r-1}\alpha_{ik}\alpha_{jl}
  \int\omega(k,s)\omega(l,t)\hat\mu^r(d\omega)\cr
&=\sum_{k=0}^{r-1}\alpha_{ik}\alpha_{jk}\min(s,t)\cr
&=\min(s,t)\text{ if }i=j,\cr
&=0\text{ if }i\ne j.\cr}$$

\noindent So $\hat T$ is $\hat\mu^r$-\imp.   If we think of $\hat T$ as
operating on $(\BbbR^r)^{\coint{0,\infty}}$, then $\hat T(\omega)=T\omega$
is continuous for every $\omega\in C(\coint{0,\infty};\BbbR^r)$, so
$\hat T[\Omega]\subseteq\Omega$ and $\tilde T=\hat T\restr\Omega$ is
$\mu_W$-\imp.

Now the same argument applies to $T^{-1}$, so
$\tilde T^{-1}=(T^{-1})\sptilde$ also is \imp, and $\tilde T$ is an
automorphism of $(\Omega,\mu_W)$.

\medskip

{\bf (e)} This time, we have
$\hat U_{\alpha}:\BbbR^{r\times\coint{0,\infty}}
\to\BbbR^{r\times\coint{0,\infty}}$ defined by the formula
$(\hat U_{\alpha}\omega)(i,t)=\Bover1{\sqrt{\alpha}}\omega(i,\alpha t)$
for $i<r$, $t\ge 0$ and $\omega\in\BbbR^{r\times\coint{0,\infty}}$.
Once again, $\hat U_{\alpha}$ is linear.   This time,


$$\eqalign{\int(\hat U_{\alpha}\omega)(i,s)(\hat U_{\alpha}\omega)(j,t)
  \hat\mu^r(d\omega)
&=\Bover1{\alpha}\int\omega(i,\alpha s)\omega(j,\alpha t)\hat\mu^r(d\omega)
   \cr
&=0\text{ if }i\ne j,\cr
&=\Bover1{\alpha}\min(\alpha s,\alpha t)=\min(s,t)\text{ if }i=j.\cr}$$

\noindent As before, it follows that $\hat U_{\alpha}$ is $\hat\mu^r$-\imp,
so that $U_{\alpha}=\hat U_{\alpha}\restr\Omega$ is $\mu_W$-\imp.
In the same way as in
(d),  $U_{\alpha}^{-1}=U_{1/\alpha}$ is $\mu_W$-\imp, so
$U_{\alpha}$ is an automorphism of $(\Omega,\mu_W)$.

\medskip

{\bf (f)} Define
$\hat R:\BbbR^{r\times\coint{0,\infty}}\to\BbbR^{r\times\coint{0,\infty}}$
by setting

$$\eqalign{\hat R(\omega)(i,t)
&=t\omega(i,\Bover1t)\text{ if }i<r\text{ and }t>0,\cr
&=\omega(i,0)\text{ if }i<r\text{ and }t=0.\cr}$$

\noindent Then, if $i$, $j<r$ and $s$, $t>0$,

$$\eqalign{\int(\hat R\omega)(i,s)(\hat R\omega)(j,t)\hat\mu^r(d\omega)
&=st\int\omega(i,\Bover1s)\omega(j,\Bover1t)\hat\mu^r(d\omega)\cr
&=0\text{ if }i\ne j,\cr
&=st\min(\Bover1s,\Bover1t)=\min(s,t)\text{ if }i=j.\cr}$$

\noindent If $s=0$, then
$(\hat R\omega)(i,s)=\omega(i,s)=0$ for almost every $\omega$, so
that $\int(\hat R\omega)(i,s)(\hat R\omega)(j,t)\hat\mu^r(d\omega)=0$;
and similarly if $t=0$.
So $\hat R$ is $\hat\mu^r$-\imp.

At this point I think we need a new argument.   Consider the set

\Centerline{$E
=\{\omega:\omega\in(\BbbR^r)^{\coint{0,\infty}}$,
$\lim_{q\in\Bbb Q,q\downarrow 0}\omega(q)=0\}$.}

\noindent Then $E$ is a Baire set in
$(\BbbR^r)^{\coint{0,\infty}}\cong\BbbR^{r\times\coint{0,\infty}}$.
Since $E\supseteq\Omega$, $\hat\mu^rE=1$.   Consequently
$\hat\mu^r\hat R^{-1}[E]=1$.   But, for
$\omega\in(\BbbR^r)^{\coint{0,\infty}}$,

\Centerline{$\omega\in\hat R^{-1}[E]
\iff 0
=\lim_{q\in\Bbb Q,q\downarrow 0}(\hat R\omega)(q)
=\lim_{q\in\Bbb Q,q\downarrow 0}q\omega(\Bover1q)
=\lim_{q\in\Bbb Q,q\to\infty}\Bover1q\omega(q)$.}

\noindent So

$$\eqalignno{\Omega_0
&=\{\omega:\omega\in\Omega,\,\lim_{t\to\infty}\Bover1t\omega(t)=0\}
=\{\omega:\omega\in\Omega,\,
  \lim_{q\in\Bbb Q,q\to\infty}\Bover1q\omega(q)=0\}\cr
\displaycause{because every member of $\Omega$ is continuous}
&=\Omega\cap\hat R^{-1}[E]\cr}$$

\noindent is $\mu_W$-conegligible.   Next, for $\omega\in\Omega_0$,
$\hat R\omega$ is continuous on $\ooint{0,\infty}$ and

$$\eqalign{0
&=\omega(0)
=\lim_{t\to\infty}\Bover1t\omega(t)=\lim_{t\downarrow 0}\omega(t)\cr
&=(\hat R\omega)(0)
=\lim_{t\to\infty}(\hat R\omega)(\Bover1t)
=\lim_{t\downarrow 0}t(\hat R\omega)(\Bover1t)\cr
&=\lim_{t\downarrow 0}(\hat R\omega)(t)
=\lim_{t\to\infty}\Bover1t(\hat R\omega)(t),\cr}$$

\noindent so $\hat R\omega\in\Omega_0$.   By (a), $R=\hat R\restr\Omega_0$
is \imp\ for the subspace measure $\nu=(\mu_W)_{\Omega_0}$;
being an involution, it is an automorphism of $(\Omega_0,\nu)$.

\medskip

{\bf (g)} If we identify $\mu_W$ and $\mu_W^{(r')}$ with
$\mu_{W1}^r$ and $\mu_{W1}^{r'}$, as in 477Da, this is elementary.
}%end of proof of 477E

\leader{477F}{Proposition} Let $r\ge 1$ be an integer.   Then
Wiener measure on $\Omega=C(\coint{0,\infty};\BbbR^r)_0$ is strictly
positive for the topology $\frak T_c$ of uniform convergence on compact
sets.

\proof{{\bf (a)} Let $\Cal I$ be a partition of $\coint{0,\infty}$ into
bounded intervals (open, closed or half-open).   As usual, set
$X_t(\omega)=\omega(t)$ for $t\in\coint{0,\infty}$ and
$\omega\in(\BbbR^r)^{\coint{0,\infty}}$.   Define
$\langle Y_t\rangle_{t\ge 0}$, $\langle Z_t\rangle_{t\ge 0}$ as follows.
If $t\in I\in\Cal I$, $a=\inf I$ and $b=\sup I$, then

$$\eqalign{Y_t
&=X_a+\Bover{t-a}{b-a}(X_b-X_a)\text{ if }a<b,\cr
&=X_a=X_t=X_b\text{ if }a=b,\cr
Z_t&=X_t-Y_t.\cr}$$

\medskip

\quad{\bf (i)} With respect to the centered Gaussian distribution
$\hat\mu^r$ on
$(\BbbR^{\coint{0,\infty}})^r\cong(\BbbR^r)^{\coint{0,\infty}}$,
$(\langle Y_t\rangle_{t\ge 0},\langle Z_t\rangle_{t\ge 0})$
is a centered Gaussian process.   \Prf\ The map
$\omega\mapsto(\langle Y_t(\omega)\rangle_{t\ge 0},
\langle Z_t(\omega)\rangle_{t\ge 0})$ is linear and continuous, so we can
apply 456Ba (strictly speaking, we apply this to the family
$(\langle Y^{(i)}_t(\omega)\rangle_{i<r,t\ge 0},
\langle Z^{(i)}_t(\omega)\rangle_{i<r,t\ge 0})$ regarded as linear operators
from $\BbbR^{r\times\coint{0,\infty}}$ to its square).\ \Qed

\medskip

\quad{\bf (ii)} If $s$, $t\in\coint{0,\infty}$ then
$\Expn(Y_s\times Y_t)=\Expn(X_s\times Y_t)$.   \Prf\ Let $I$, $J$ be the
members of $\Cal I$ containing $s$, $t$ respectively;  set $a=\min I$,
$b=\max I$, $c=\min J$ and $d=\max J$.

\qquad{\bf case 1} If $a=b$ then $Y_s=X_s$ and we can stop.

\qquad{\bf case 2} If $a<b\le c$, then

\Centerline{$\Expn(X_a\times X_c)=\Expn(X_a\times X_d)=a$,
\quad$\Expn(X_a\times Y_t)=a$,}

\Centerline{$\Expn(X_b\times X_c)=\Expn(X_b\times X_d)=b$,
\quad$\Expn(X_b\times Y_t)=b$,}

\Centerline{$\Expn(X_s\times X_c)=\Expn(X_s\times X_d)=s$,
\quad$\Expn(X_s\times Y_t)=s$,}

\Centerline{$\Expn(Y_s\times Y_t)
=a+\Bover{s-a}{b-a}(b-a)=s=\Expn(X_s\times Y_t)$.}

\qquad{\bf case 3} If $d\le a<b$, then

\Centerline{$c=\Expn(X_s\times X_c)
=\Expn(X_a\times X_c)=\Expn(X_b\times X_c)=\Expn(Y_s\times X_c)$,}

\Centerline{$d=\Expn(X_s\times X_d)
=\Expn(X_a\times X_d)=\Expn(X_b\times X_d)=\Expn(Y_s\times X_d)$;}

\noindent since $Y_t$ is a convex combination of $X_c$ and $X_d$,
$\Expn(Y_s\times Y_t)=\Expn(X_s\times Y_t)$.

\qquad{\bf case 4} If $a=c<b=d$, then

\Centerline{$\Expn(X_a\times Y_t)
=\Expn(X_a\times X_a)+\Bover{t-a}{b-a}\,\Expn(X_a\times(X_b-X_a))=a$,}

\Centerline{$\Expn(X_b\times Y_t)
=\Expn(X_b\times X_a)+\Bover{t-a}{b-a}\,\Expn(X_b\times(X_b-X_a))
=a+\Bover{t-a}{b-a}(b-a)=t$,}

\Centerline{$\Expn(X_s\times Y_t)
=\Expn(X_s\times X_a)+\Bover{t-a}{b-a}\,\Expn(X_s\times(X_b-X_a))
=a+\Bover{t-a}{b-a}(s-a)$,}

$$\eqalign{\Expn(Y_s\times Y_t)
&=\Expn(X_a\times Y_t)+\Bover{s-a}{b-a}\,\Expn((X_b-X_a)\times Y_t)\cr
&=a+\Bover{s-a}{b-a}(t-a)=\Expn(X_s\times Y_t).  \text{ \Qed}\cr}$$

\medskip

\quad{\bf (iii)} Accordingly $\Expn(Z_s\times Y_t)=0$ for all $s$,
$t\ge 0$.   It follows that if $\Sigma_1$, $\Sigma_2$ are the
$\sigma$-algebras of subsets of $(\BbbR^r)^{\coint{0,\infty}}$ defined by
$\{Y_t:t\ge 0\}$ and $\{Z_t:t\ge 0\}$ respectively, $\Sigma_1$ and
$\Sigma_2$ are $\hat\mu^r$-independent (456Eb).

\medskip

{\bf (b)} Observe next that if
$x_1,\ldots,x_n\in\BbbR^r$, $0<t_1\le\ldots\le t_n$ and $\delta>0$, then
$\hat\mu^r\{\omega:\|\omega(t_i)-x_i\|\le\delta$
for $1\le i\le n\}$ is greater than $0$.   \Prf\ Set $x_0=0$ in
$\BbbR^r$ and $t_0=0$.   For each $i<n$, the distribution of
$\Bover1{\sqrt{t_{i+1}-t_i}}(X_{t_{i+1}}-X_{t_i})$ is the standard Gaussian
distribution $\mu_G^r$
on $\BbbR^r$, which has strictly positive probability density
function with respect to Lebesgue measure.   So

$$\eqalign{\Pr(\|(X_{t_{i+1}}-X_{t_i})-(x_{i+1}-x_i)\|\le\Bover1n\delta)
&=\mu_G^r\{x:\|\sqrt{t_{i+1}-t_i}x-x_{i+1}+x_i\|\le\Bover1n\delta\}\cr
&>0.\cr}$$

\noindent Next, $X_{t_1}-X_{t_0},\ldots,X_{t_n}-X_{t_{n-1}}$ are
independent, so

$$\eqalign{0
&<\prod_{i<n}\Pr(\|(X_{t_{i+1}}-X_{t_i})-(x_{i+1}-x_i)\|\le\Bover1n\delta)
  \cr
&=\Pr(\|(X_{t_{i+1}}-X_{t_i})-(x_{i+1}-x_i)\|\le\Bover1n\delta
   \text{ for every }i<n)\cr
&\le\Pr(\|X_{t_i}-x_i\|\le\delta\text{ for every }i\le n).
   \text{ \Qed}\cr}$$

\medskip

{\bf (c)} Let $G\subseteq\Omega$ be a non-empty $\frak T_c$-open set.
Then there are
$\omega_0\in\Omega$, $m\in\Bbb N$ and $\delta>0$ such that $G$ includes

\Centerline{$V=\{\omega:\omega\in\Omega$,
$\|\omega(t)-\omega_0(t)\|\le 6\delta$ for every $t\in[0,m]\}$.}

\noindent For $n\in\Bbb N$, let $F_n$ be

$$\eqalign{\{\omega:\omega&\in(\BbbR^r)^{\coint{0,\infty}},
\|\omega(q)-\omega(q')\|\le\delta\cr
&\mskip50mu
\text{ whenever }q,\,q'\in\Bbb Q\cap[0,m]\text{ and }
  |q-q'|\le 2^{-n}\}.\cr}$$

\noindent Then $\Omega\subseteq\bigcup_{n\in\Bbb N}F_n$ so there is an
$n\ge 0$ such that $\omega_0\in F_n$ and
$\hat\mu^rF_n>0$.   Let $\Cal I$ be
$\{\coint{2^{-n}k,2^{-n}(k+1)}:k\in\Bbb N\}$, and let
$\langle Y_t\rangle_{t\ge 0}$, $\langle Z_t\rangle_{t\ge 0}$ be the
families of random variables defined from $\Cal I$ by the method of (a)
above, with corresponding independent $\sigma$-algebras $\Sigma_1$,
$\Sigma_2$.   If $\omega\in F_n$, then $\|Z_t(\omega)\|\le\delta$
for every $t\in\Bbb Q\cap[0,m]$.
\Prf\ If $t\in\coint{a,b}\in\Cal I$, then
$b-a=2^{-n}$ so $\|X_t(\omega)-X_a(\omega)\|$,
$\|X_t(\omega)-X_b(\omega)\|$ and therefore
$\|Z_t(\omega)\|=\|X_t(\omega)-Y_t(\omega)\|$ are all at most $\delta$.\
\QeD\  Set

\Centerline{$F=\{\omega:\omega\in(\BbbR^r)^{\coint{0,\infty}}$,
$\|Z_t(\omega)\|\le\delta$ for every
$t\in\Bbb Q\cap[0,m]\}$;}

\noindent then $F\in\Sigma_2$ and $\hat\mu^rF>0$.

Next, set

\Centerline{$E=\{\omega:\omega\in(\BbbR^r)^{\coint{0,\infty}}$,
$\|\omega(2^{-n}k)-\omega_0(2^{-n}k)\|\le\delta$ for $1\le k\le 2^nm\}$.}

\noindent By (b), $\hat\mu^rE>0$.   But since
$Y_{2^{-n}k}(\omega)=X_{2^{-n}k}(\omega)=\omega(2^{-n}k)$ whenever
$k\le 2^nm$, $E\in\Sigma_1$.    Accordingly
$\hat\mu^r(E\cap F)=\hat\mu^rE\cdot\hat\mu^rF>0$.   But
$E\cap F\cap\Omega\subseteq V$.   \Prf\ If $\omega\in E\cap F\cap\Omega$,
then $t\mapsto X_t(\omega)$, $t\mapsto Y_t(\omega)$ and
$t\mapsto Z_t(\omega)$ are all continuous, so $\|Z_t(\omega)\|\le\delta$
for every $t\in[0,m]$.
If $t\in[0,m]$, let $k<2^nm$ be such that
$2^{-n}k\le t\le 2^{-n}(k+1)$.   Then

$$\eqalign{\|\omega(t)-\omega_0(t)\|
&\le\|\omega(t)-\omega(2^{-n}k)\|
   +\|\omega(2^{-n}k)-\omega_0(2^{-n}k)\|
   +\|\omega_0(2^{-n}k)-\omega_0(t)\|\cr
&\le\|Z_t(\omega)\|+\|Y_t(\omega)-Y_{2^{-n}k}(\omega)\|
   +2\delta\cr
&\le\delta+\|\omega(2^{-n}(k+1))-\omega(2^{-n}k)\|
   +2\delta\cr
&\le\|\omega(2^{-n}(k+1))-\omega_0(2^{-n}(k+1))\|
   +\|\omega_0(2^{-n}(k+1))-\omega_0(2^{-n}k)\|\cr
&\mskip200mu
   +\|\omega_0(2^{-n}k)-\omega(2^{-n}k)\|
   +3\delta\cr
&\le 6\delta.\cr}$$

\noindent As $t$ is arbitrary, $\omega\in V$.\ \Qed

Accordingly

\Centerline{$\mu_WG
\ge\mu_W(E\cap F\cap\Omega)=\hat\mu^r(E\cap F)>0$.}

\noindent As $G$ is arbitrary, $\mu_W$ is strictly positive.
}%end of proof of 477F


\leader{477G}{The strong Markov \dvrocolon{property}}\cmmnt{ With the
identification in 477Dd, we are ready for
one of the most important properties of Brownian motion.

\medskip

\noindent}{\bf Theorem} Suppose that $r\ge 1$,
$\mu_W$ is Wiener measure on $\Omega=C(\coint{0,\infty};\BbbR^r)_0$ and
$\Sigma$ is its domain.   For $t\ge 0$ let $\Sigma_t$ be

\Centerline{$\{F:F\in\Sigma$, $\omega'\in F$ whenever $\omega\in F$,
$\omega'\in\Omega$ and $\omega'\restr[0,t]=\omega\restr[0,t]\}$,}

\Centerline{$\Sigma^+_t=\bigcap_{s>t}\Sigma_s$,}

\noindent and let
$\tau:\Omega\to[0,\infty]$ be a stopping time adapted to the family
$\langle\Sigma^+_t\rangle_{t\ge 0}$.
Define $\phi_{\tau}:\Omega\times\Omega\to\Omega$ by saying that

$$\eqalign{\phi_{\tau}(\omega,\omega')(t)
&=\omega(t)\text{ if }t\le\tau(\omega),\cr
&=\omega(\tau(\omega))+\omega'(t-\tau(\omega))
   \text{ if }t\ge\tau(\omega).\cr}$$

\noindent Then $\phi_{\tau}$ is \imp\ for $\mu_W\times\mu_W$ and $\mu_W$.

\proof{{\bf (a)}
At this point I apply the general theory of \S455 in something like
its full strength.   As in 477Dd, let
$\langle\lambda_t\rangle_{t>0}$ be the
standard family of Gaussian distributions on $\BbbR^r$, $\ddot\nu$
the corresponding measure on the space $\Cdlg$ of \cadlag\ functions from
$\coint{0,\infty}$ to $\BbbR^r$, and $\ddot\Sigma$ its domain;  then
$\ddot\nu\Omega=1$ and $\mu_W$ is the subspace measure on
$\Omega$ induced by $\ddot\nu$.   As in 455U, let $\ddot\Sigma_t$ be

\Centerline{$\{F:F\in\ddot\Sigma$, $\omega'\in F$ whenever $\omega\in F$,
   $\omega'\in\Cdlg$ and $\omega'\restr[0,t]=\omega\restr[0,t]\}$,}

\noindent and $\ddot\Sigma^+_t=\bigcap_{s>t}\ddot\Sigma_s$, for $t>0$.

\medskip

{\bf (b)} For $\omega\in\Cdlg$ set

\Centerline{$\ddot\tau(\omega)
=\inf\{t:$ there is an $\omega'\in\Omega$ such that
$\omega'\restr[0,t]=\omega\restr[0,t]$ and $\tau(\omega')\le t\}$,}

\noindent counting $\inf\emptyset$ as $\infty$.   Then $\ddot\tau$ is a
stopping time adapted to $\langle\ddot\Sigma^+_t\rangle_{t\ge 0}$.
\Prf\ For $t\ge 0$, set

\Centerline{$F_t=\{\omega:\omega\in\Omega$, $\tau(\omega)<t\}\in\Sigma_t$}

\noindent (455Lb), and

\Centerline{$F'_t=\{\omega:\omega\in\Cdlg$, there is an $\omega'\in F_t$
such that $\omega\restr[0,t]=\omega'\restr[0,t]\}$.}

\noindent Since $F_t\in\Sigma_t$, $F'_t\cap\Omega=F_t$;  as $\Omega$ is
$\ddot\nu$-conegligible and $\ddot\nu$ is complete, $F'_t\in\ddot\Sigma$;
now of course $F'_t\in\ddot\Sigma_t$.   If $\omega\in F'_t$, let
$\omega'\in F_t$ be such that $\omega\restr[0,t]=\omega'\restr[0,t]$;  then
$\omega'$ witnesses that $\ddot\tau(\omega)\le\tau(\omega')<t$.   If
$\omega\in\Cdlg$ and $\ddot\tau(\omega)<t$, let $q<t$ and
$\omega'\in\Omega$ be such that $q$ is rational,
$\omega\restr[0,q]=\omega'\restr[0,q]$ and
$\tau(\omega')<q$;  then

\Centerline{$\omega\in F'_q\in\ddot\Sigma_q\subseteq\ddot\Sigma_t$.}

\noindent This shows that

\Centerline{$\{\omega:\omega\in\Cdlg$, $\ddot\tau(\omega)<t\}
=\bigcup_{q\in[0,t]\cap\Bbb Q}F'_q
\in\ddot\Sigma_t$.}

\noindent By 455Lb in the other direction, $\ddot\tau$ is a stopping time
adapted to $\langle\ddot\Sigma^+_t\rangle_{t\ge 0}$.\ \Qed

\medskip

{\bf (c)} By 455U, $\ddot\phi:\Cdlg\times\Cdlg\to\Cdlg$ is \imp\ for
$\ddot\nu\times\ddot\nu$ and $\ddot\nu$, where

$$\eqalign{\ddot\phi(\omega,\omega')(t)
&=\omega(t)\text{ if }t<\ddot\tau(\omega),\cr
&=\omega(\ddot\tau(\omega))+\omega'(t-\ddot\tau(\omega))
   \text{ if }t\ge\ddot\tau(\omega).\cr}$$

\noindent Now $\phi_{\tau}=\ddot\phi\restr\Omega\times\Omega$
and $\mu_W\times\mu_W$ is the subspace measure on
$\Omega\times\Omega$ induced by $\ddot\nu\times\ddot\nu$ (251Q),
so $\phi_{\tau}$ also is \imp.
}%end of proof of 477G

%see mt47bits, 477M \query

\leader{477H}{Some families of \dvrocolon{$\sigma$-algebras}}\cmmnt{ The
$\sigma$-algebras considered in Theorem 477G can be looked at in other ways
which are sometimes useful.

\medskip

\noindent}{\bf Proposition} Let $r\ge 1$
be an integer, $\mu_W\,\,r$-dimensional Wiener measure on
$\Omega=C(\coint{0,\infty};\BbbR^r)_0$ and $\Sigma$ its domain.   Set
$X^{(i)}_t(\omega)=\omega(t)(i)$ for $t\ge 0$ and $i<r$.   For
$I\subseteq\coint{0,\infty}$, let $\Tau_I$ be the $\sigma$-algebra
of subsets of $\Omega$
generated by $\{X^{(i)}_s-X^{(i)}_t:s$, $t\in I$, $i<r\}$, and
$\hat\Tau_I$ the $\sigma$-algebra
$\{E\symmdiff F:E\in\Tau_I$, $\mu_WF=0\}$.

(a) $\Tau_{\coint{0,\infty}}$ is the Borel $\sigma$-algebra of $\Omega$
either for the topology of pointwise convergence inherited from
$(\BbbR^r)^{\coint{0,\infty}}$ or $\BbbR^{r\times\coint{0,\infty}}$, or for
the topology of uniform convergence on compact sets.

(b) If $\Cal I$ is a family of subsets of $\coint{0,\infty}$ such that
for all distinct $I$, $J\in\Cal I$ either $\sup I\le\inf J$ or
$\sup J\le\inf I$ (counting $\inf\emptyset$ as $\infty$ and $\sup\emptyset$
as $0$), then $\family{I}{\Cal I}{\hat\Tau_I}$ is an
independent family of $\sigma$-algebras.

(c) For $t\ge 0$, let $\Sigma_t$ be
the $\sigma$-algebra of sets $F\in\Sigma$
such that $\omega'\in F$ whenever $\omega\in F$, $\omega'\in\Omega$ and
$\omega'\restr[0,t]=\omega\restr[0,t]$, and
$\Sigma^+_t=\bigcap_{s>t}\Sigma_s$.   Write $\hat\Tau_{[0,t]}^+$ for
$\bigcap_{s>t}\hat\Tau_{[0,s]}$.   Then, for any $t\ge 0$,

%\Centerline{$\Tau_{[0,t]}\subseteq\Sigma_t\subseteq\Sigma^+_t
%\subseteq\hat\Tau_{[0,t]}^+=\hat\Tau_{[0,t]}$.}

\Centerline{$\Tau_{[0,t]}\subseteq\Sigma_t\subseteq\Sigma^+_t
\subseteq\hat\Tau_{[0,t]}^+=\hat\Tau_{[0,t]}=\hat\Tau_{\coint{0,t}}$.}

(d) On the tail $\sigma$-algebra
$\bigcap_{t\ge 0}\hat\Tau_{\coint{t,\infty}}$, $\mu_W$
takes only the values $0$ and $1$.

\proof{{\bf (a)} Write $\Cal B(\Omega,\frak T_p)$,
$\Cal B(\Omega,\frak T_c)$ for the Borel algebras under the topologies
$\frak T_p$, $\frak T_c$ of pointwise convergence and uniform convergence
on compact sets.   Then
$\Tau_{\coint{0,\infty}}\subseteq\Cal B(\Omega,\frak T_p)$ because the
functionals $X^{(i)}_t$ are all $\frak T_p$-continuous, and
$\Cal B(\Omega,\frak T_p)\subseteq\Cal B(\Omega,\frak T_c)$ because
$\frak T_p\subseteq\frak T_c$.

Now $\Tau_{\coint{0,\infty}}$ includes a base for $\frak T_c$.
\Prf\ Suppose that $\omega\in\Omega$,
$n\in\Bbb N$ and
$V=\{\omega':\omega'\in\Omega,\,
  \sup_{t\in[0,n]}\|\omega'(t)-\omega(t)\|<2^{-n}\}$.   Then
(because every member of $\Omega$ is continuous)

\Centerline{$V=\bigcup_{m\in\Bbb N}\bigcap_{q\in\Bbb Q\cap[0,n]}
  \{\omega':\sum_{i=0}^{r-1}|X^{(i)}_q(\omega')-X^{(i)}_q(\omega)|^2
  \le 2^{-2n}-2^{-m}\}$}

\noindent belongs to $\Tau_{\coint{0,\infty}}$;  but such sets form a base
for $\frak T_c$.\ \QeD\

Since $\frak T_c$ is separable and metrizable,
$\Cal B(\Omega,\frak T_c)\subseteq\Tau_{\coint{0,\infty}}$ (4A3Da).

\medskip

{\bf (b)(i)} Suppose to begin with that $\Cal I$ is finite and
every member of $\Cal I$ is finite.
If we enumerate $\{0\}\cup\bigcup\Cal I$ in ascending order as
$\langle t_j\rangle_{j\le n}$,
$\langle X^{(i)}_{t_{j+1}}-X^{(i)}_{t_j}\rangle_{j<n,i<r}$ is an
independent family of real-valued random variables.   Taking
$J_I=\{t_j:j<n$, $t_j\in I$, $t_{j+1}\in I\}$ for $I\in\Cal I$,
$\{X^{(i)}_{t_{j+1}}-X^{(i)}_{t_j}:j\in J_I\}$ generates $\Tau_I$ for each
$I\in\Cal I$, because of the separation property of the members of
$\Cal I$, and $\family{I}{\Cal I}{J_I}$ is disjoint.   By 272K,
$\family{I}{\Cal I}{\Tau_I}$ is independent.

\medskip

\quad{\bf (ii)} Now suppose only that $\Cal I$ is finite and not empty.
For $I\in\Cal I$,
set $\Tau'_I=\bigcup_{J\subseteq I\text{ is finite}}\Tau_J$ for
$I\in\Cal I$;  then $\Tau'_I$ is an algebra of sets, and $\Tau_I$ is the
$\sigma$-algebra generated by $\Tau'_I$.   If $E_I\in\Tau'_I$ for
$I\in\Cal I$, there are $J_I\in[I]^{<\omega}$ such that $E_I\in\Tau_{J_I}$
for $I\in\Cal I$, so
$\mu_W(\bigcap_{I\in\Cal I}E_I)=\prod_{I\in\Cal I}\mu_W(E_I)$, by
($\alpha$).   Inducing on $n$, and using the Monotone Class Theorem
for the inductive step, we see that
$\mu_W(\bigcap_{I\in\Cal I}E_I)=\prod_{I\in\Cal I}\mu_W(E_I)$
whenever $E_I\in\Tau_I$ for every $I\in\Cal I$ and
$\#(\{I:E_I\notin\Tau'_I\})\le n$.   At the end of the induction, with
$n=\#(\Cal I)$, we have
$\mu_W(\bigcap_{I\in\Cal I}E_I)=\prod_{I\in\Cal I}\mu_W(E_I)$
whenever $E_I\in\Tau_I$ for every $I\in\Cal I$;  that is,
$\family{I}{\Cal I}{\Tau_I}$ is independent.

\medskip

\quad{\bf (iii)} Thus $\family{I}{\Cal J}{\Tau_I}$ is independent for every
non-empty finite $\Cal J\subseteq\Cal I$, and $\family{I}{\Cal I}{\Tau_I}$
is independent (272Bb).

\medskip

{\bf (c)(i)} If $s$, $s'\le t$ and $i<r$
then $X^{(i)}_s$, $X^{(i)}_{s'}$ and $X^{(i)}_s-X^{(i)}_{s'}$ are
$\Sigma_t$-measurable, so $\Tau_{[0,t]}\subseteq\Sigma_t$.
Of course $\Sigma_t\subseteq\Sigma^+_t$.

\medskip

\quad{\bf (ii)} $\Sigma_t\subseteq\hat\Tau_{[0,t]}$.   \Prf\ Suppose that
$F\in\Sigma_t$.   Set $D=[0,t]\cap(\Bbb Q\cup\{t\})$, and set
$g(\omega)=\omega\restr D$ for $\omega\in\Omega$;  then
$g:\Omega\to(\BbbR^r)^D$ is continuous (when $\Omega$ is given the topology
of pointwise convergence inherited from $(\BbbR^r)^{\coint{0,\infty}}$,
for definiteness), and $F=g^{-1}[g[F]]$.   Now the Borel $\sigma$-algebra
of $(\BbbR^r)^D\cong\BbbR^{r\times D}$ is the $\sigma$-algebra generated
by the functionals $\omega\mapsto\omega(t)(i):(\BbbR^r)^D\to\Bbb R$ for
$t\in D$ and $i<r$ (4A3D(c-i)), and for such $t$ and $i$,
$\omega\mapsto g(\omega)(t)(i)=X_t^{(i)}(\omega)$ is
$\Tau_{[0,t]}$-measurable;  so $g$ is $\Tau_{[0,t]}$-measurable.   Now
there is a sequence $\sequencen{K_n}$ of compact subsets of $F$ such that
$\sup_{n\in\Bbb N}\mu_WK_n=\mu_WF$.    In this case,
$g[K_n]\subseteq(\BbbR^r)^D$ is compact and $K'_n=g^{-1}[g[K_n]]$ belongs
to $\Tau_{[0,t]}$, for each $n$.   So $F'=\bigcup_{n\in\Bbb N}K'_n$ belongs
to $\Tau_{[0,t]}$, and $\mu_W(F\setminus F')=0$.

Similarly, applying the same argument to $\Omega\setminus F$, we have an
$F''\in\Tau_{[0,t]}$ such that $F''\supseteq F$ and
$\mu_W(F''\setminus F)=0$.   So $F\in\hat\Tau_{[0,t]}$.\ \Qed

Consequently $\Sigma_t^+\subseteq\hat\Tau_{[0,t]}^+$.

\medskip

\quad{\bf (iii)}\grheada\
Let $\Cal A$ be the family of those sets $G\in\Sigma$
such that $\chi G$ has a conditional expectation on
$\hat\Tau_{[0,t]}^+$ which is $\Tau_{[0,t]}$-measurable.
Then $\Cal A$ is a Dynkin class (definition:  136A).
If $E\in\Tau_{[0,t]}$ and $F\in\Tau_{\coint{s,\infty}}$ where $s>t$,
then $(\mu_WF)\chi\Omega$ is a conditional
expectation of $\chi F$ on $\hat\Tau_{[0,t]}^+$, because
$\hat\Tau_{[0,t]}^+\subseteq\hat\Tau_{[0,s]}$ and
$\Tau_{\coint{s,\infty}}$ are independent.   As
$E\in\hat\Tau_{[0,t]}^+$,
$(\mu_WF)\chi E$ is a conditional expectation of $\chi(E\cap F)$ on
$\hat\Tau_{[0,t]}^+$ (233Eg), and $E\cap F\in\Cal A$.   Since
$\Cal E=\{E\cap F:E\in\Sigma_t$,
$F\in\bigcup_{s>t}\Tau_{\coint{s,\infty}}\}$ is closed
under finite intersections, the Monotone Class Theorem, in the form 136B,
shows that $\Cal A$ includes the $\sigma$-algebra $\Tau$ generated by
$\Cal E$;  note that $\Tau$ includes
$\Tau_{[0,t]}\cup\Tau_{\coint{s,\infty}}$ whenever
$s>t$.   Now $X^{(i)}_u-X^{(i)}_s$ is $\Tau$-measurable whenever
$0\le s\le u$.   \Prf\ If $u\le t$, $X^{(i)}_u-X^{(i)}_s$ is
$\Tau_{[0,t]}$-measurable, therefore $\Tau$-measurable;  if $t<s$,
$X^{(i)}_u-X^{(i)}_s$ is
$\Tau_{\coint{s,\infty}}$-measurable, therefore $\Tau$-measurable.
If $s\le t<u$,
let $\sequencen{t_n}$ be a sequence in $\ocint{t,u}$ with
limit $t$.   Then

\Centerline{$X^{(i)}_u-X^{(i)}_s
=\lim_{n\to\infty}(X^{(i)}_u-X^{(i)}_{t_n})+(X^{(i)}_t-X^{(i)}_s)$}

\noindent is $\Tau$-measurable.\ \Qed

\medskip

\qquad\grheadb\ This means that $\Tau$ includes $\Tau_{\coint{0,\infty}}$.
It follows that $\Cal A=\Sigma$, because for any $G\in\Sigma$ there is a
$G'\in\Tau_{\coint{0,\infty}}$ such that $G\symmdiff G'$ is negligible,
and now $\chi G$ and $\chi G'$ have the same conditional expectations.
So $\hat\Tau_{[0,t]}^+=\hat\Tau_{[0,t]}$.   \Prf\ Of course
$\hat\Tau_{[0,t]}^+\supseteq\hat\Tau_{[0,t]}$.
If $H\in\hat\Tau_{[0,t]}^+$ there is a $\Tau_{[0,t]}$-measurable function
$g$ which is a conditional expectation of
$\chi H$ on $\hat\Tau_{[0,t]}^+$.   But in this case $g\eae\chi H$,
so, setting $E=\{\omega:g(\omega)=1\}\in\Tau_{[0,t]}$, $E\symmdiff H$ is
negligible and $H\in\hat\Tau_{[0,t]}$.   Thus
$\hat\Tau_{[0,t]}^+\subseteq\hat\Tau_{[0,t]}$.\ \Qed

\medskip

\qquad\grheadc\ Observe next that $X^{(i)}_t$ is
$\Tau_{\coint{0,t}}$-measurable for $i<r$.    \Prf\ If $t=0$ then
$X^{(i)}_t$ is the constant function with value $0$.
Otherwise, there is a strictly
increasing sequence $\sequencen{s_n}$ in $\coint{0,t}$ with limit $t$, so
that $X^{(i)}_t=\lim_{n\to\infty}X^{(i)}_{s_n}$ is the limit of a sequence
of $\Tau_{\coint{0,t}}$-measurable functions and is itself
$\Tau_{\coint{0,t}}$-measurable.\ \QeD\   But this means that
$\Tau_{[0,t]}=\Tau_{\coint{0,t}}$, so
$\hat\Tau_{[0,t]}^+\subseteq\hat\Tau_{[0,t]}=\hat\Tau_{\coint{0,t}}$.
In the other direction, of course
$\Tau_{\coint{0,t}}\subseteq\Tau_{[0,t]}^+$ and
$\hat\Tau_{\coint{0,t}}\subseteq\hat\Tau_{[0,t]}^+$, so we have equality.

\medskip

{\bf (d)} Set $\Tau'=\bigcup_{t\ge 0}\Tau_{[0,t]}$.
If $E\in\bigcap_{t\ge 0}\hat\Tau_{\coint{t,\infty}}$, then
$\mu_W(E\cap F)=\mu_WE\cdot\mu_WF$ for every $F\in\Tau'$.   By the
Monotone Class Theorem again, $\mu_W(E\cap F)=\mu_WE\cdot\mu_WF$ for every
$F$ in the $\sigma$-algebra
generated by $\Tau'$, which is $\Cal B(\Omega)$, by (a).
Now $\mu_W\LLcorner E$ (definition:  234M\formerly{2{}34E}) and
$(\mu_WE)\mu_W$
are Radon measures on $\Omega$ (416S) which agree on $\Cal B(\Omega)$, so
must be identical.   In particular,

\Centerline{$\mu_WE=(\mu_W\LLcorner E)(E)=(\mu_WE)^2$}

\noindent and $\mu_WE$ must be either $0$ or $1$.
}%end of proof of 477H

\leader{477I}{Hitting times}\cmmnt{ In 455M I introduced `hitting times'.
I give a paragraph now to these in the special case of Brownian motion;
such stopping times will dominate
the applications of the theory in \S\S478-479.} Take $r\ge 1$, and let
$\mu_W$ be Wiener measure on $\Omega=C(\coint{0,\infty};\BbbR^r)_0$ and
$\Sigma$ its domain;
for $t\ge 0$ define $\Sigma^+_t$ and $\Tau_{[0,t]}$ as in 477G and 477H.
Give $\Omega$ its topology of uniform convergence on compact sets.

\spheader 477Ia Suppose that $A\subseteq\BbbR^r$.
For $\omega\in\Omega$ set
$\tau(\omega)=\inf\{t:t\in\coint{0,\infty}$, $\omega(t)\in A\}$, counting
$\inf\emptyset$ as $\infty$.
I will call $\tau$ the {\bf Brownian hitting time}
to $A$, or the {\bf Brownian exit time} from $\BbbR^r\setminus A$.
I will say that the {\bf Brownian hitting probability} of $A$, or
the {\bf Brownian exit probability} of $\BbbR^r\setminus A$, is
$\hp(A)=\mu_W\{\omega:\tau(\omega)<\infty\}$ if this is defined.
More generally, I will write

\Centerline{$\hp^*(A)=\mu_W^*\{\omega:\tau(\omega)<\infty\}
=\mu_W^*\{\omega:\omega^{-1}[A]\ne\emptyset\}$,}

\noindent the {\bf outer Brownian hitting probability},
for any $A\subseteq\BbbR^r$.

\spheader 477Ib If $A\subseteq\BbbR^r$ is analytic,
the Brownian hitting time to $A$
is a stopping time adapted to the family
$\langle\Sigma^+_t\rangle_{t\ge 0}$.   \prooflet{\Prf\ Let $\Cdlg$ be the
space of \cadlag\ functions from $\coint{0,\infty}$ to $\BbbR^r$, and
define $\ddot\Sigma$ as in the proof
of 477G;  let $\ddot\tau$ be the hitting time on $\Cdlg$ defined by $A$.
By 455Ma, $\ddot\tau$ is $\ddot\Sigma$-measurable, so
$\tau=\ddot\tau\restr\Omega$ is $\Sigma$-measurable.   Now (as in 455Mb)
$\{\omega:\tau(\omega)<t\}\in\Sigma_t$ for every $t$, so
$\tau$ is adapted to $\langle\Sigma^+_t\rangle_{t\ge 0}$.\ \Qed}

In particular, there is a well-defined Brownian hitting probability of $A$.

\spheader 477Ic Let $F\subseteq\BbbR^r$ be a closed set, and $\tau$ the
Brownian hitting time to $F$.

\medskip

\quad{\bf (i)} If $\tau(\omega)<\infty$, then

\Centerline{$\tau(\omega)
=\inf\omega^{-1}[F]=\min\omega^{-1}[F]$}

\noindent because $\omega$ is continuous.   If $0\notin F$ and
$\tau(\omega)<\infty$, then
$\omega(\tau(\omega))\in\partial F$.

\medskip

\quad{\bf (ii)} $\tau$ is lower semi-continuous.
\prooflet{\Prf\ For any $t\in\coint{0,\infty}$,

\Centerline{$\{\omega:\tau(\omega)>t\}
=\{\omega:\omega(s)\notin F$ for every $s\le t\}$}

\noindent is open in $\Omega$.\ \QeD}

\medskip

\quad{\bf (iii)} $\tau$ is adapted to $\langle\Tau_{[0,t]}\rangle_{t\ge 0}$.
\prooflet{\Prf\
Let $\sequencen{G_n}$ be a non-increasing sequence of open sets including
$F$ such that $F=\bigcap_{n\in\Bbb N}\overline{G}_n$.
Then, for $\omega\in\Omega$ and $t>0$,

$$\eqalignno{\tau(\omega)\le t
&\iff\omega[\,[0,t]\,]\cap F\ne\emptyset\cr
\displaycause{because $\omega$ is continuous}
&\iff\omega[\,[0,t]\,]\cap G_n\ne\emptyset\text{ for every }n\in\Bbb N\cr
\displaycause{because $\omega[\,[0,t]\,]$ is compact}
&\iff\text{ for every }n\in\Bbb N\text{ there is a rational }q\le t
  \text{ such that }\omega(q)\in G_n.\cr}$$

\noindent So

\Centerline{$\{\omega:\tau(\omega)\le t\}
=\bigcap_{n\in\Bbb N}\bigcup_{q\in\Bbb Q\cap[0,t]}
   \{\omega:\omega(q)\in G_n\}\in\Tau_{[0,t]}$.}

\noindent Of course $\{\omega:\tau(\omega)=0\}$ is either $\Omega$ (if
$0\in F$) or $\emptyset$ (if $0\notin F$), so belongs to
$\Tau_{[0,0]}$.\ \Qed}

In the language of 477G, we have $\Tau_{[0,t]}\subseteq\Sigma_t$ for every
$t\ge 0$\cmmnt{ (477Hc)}, so $\tau$ must also be adapted to
$\langle\Sigma_t\rangle_{t\ge 0}$.

\spheader 477Id If $A\subseteq\BbbR^r$ is any set, then

\Centerline{$\hp^*(A)
=\min\{\hp(B):B\supseteq A$ is an analytic set$\}
=\min\{\hp(E):E\supseteq A$ is a G$_{\delta}$ set$\}$.}

\prooflet{\noindent\Prf\ Of course

$$\eqalign{\hp^*(A)
&\le\inf\{\hp(B):B\supseteq A\text{ is an analytic set}\}\cr
&=\min\{\hp(B):B\supseteq A\text{ is an analytic set}\}\cr
&\le\inf\{\hp(E):E\supseteq A\text{ is a G}_{\delta}\text{ set}\}
=\min\{\hp(E):E\supseteq A\text{ is a G}_{\delta}\text{ set}\}\cr}$$

\noindent just because $\hp^*$ is an order-preserving function.   If
$\gamma>\hp^*(A)$, there is a compact $K\subseteq\Omega$ such that
$\omega^{-1}[A]=\emptyset$ for every $\omega\in K$ and
$\mu_WK\ge 1-\gamma$.   Now
$F=\{\omega(t):\omega\in K$, $t\in\coint{0,\infty}\}$ is a K$_{\sigma}$ set
not meeting $A$, so $E=\BbbR^r\setminus F$ is a G$_{\delta}$ set including
$A$.   Since $\omega^{-1}[E]$ is empty for every $\omega\in K$,
$\hp(E)\le\mu_W(\Omega\setminus K)\le\gamma$.   As $\gamma$ is arbitrary,

\Centerline{$\inf\{\hp(E):E\supseteq A$ is a G$_{\delta}$ set$\}
\le\hp^*(A)$}

\noindent and we have equality throughout.\ \Qed}

\spheader 477Ie If $A\subseteq\BbbR^r$ is analytic, then
$\hp(A)=\sup\{\hp(K):K\subseteq A$ is compact$\}$.   \prooflet{\Prf\
Suppose that $\gamma<\hp(A)$.
Set $E=\{(\omega,t):\omega\in\Omega$, $t\ge 0$, $\omega(t)\in A\}$.
Then $E$ is analytic and $\hp(A)=\mu_W\pi_1[E]$, where
$\pi_1(\omega,t)=\omega$ for $(\omega,t)\in E$.   Let $\lambda$ be the
subspace measure $(\mu_W)_{\pi_1[E]}$.   By 433D, there is a Radon measure
$\lambda'$ on $E$ such that $\lambda=\lambda'\pi_1^{-1}$.
Then $\lambda'E=\hp(A)>\gamma$, so there is a compact set
$L\subseteq E$ such that $\lambda'L\ge\gamma$.   Set
$K=\{\omega(t):(\omega,t)\in L\}$;  then $K\subseteq\BbbR^r$ is compact,
and

\Centerline{$\hp(K)=\mu_W\{\omega:\omega(t)\in K$ for some $t\ge 0\}
\ge\mu_W\pi_1[L]\ge\lambda'L\ge\gamma$.}

\noindent As $\gamma$ is arbitrary,
$\hp(A)\le\sup\{\hp(K):K\subseteq A$ is compact$\}$;  the reverse
inequality is trivial.\ \Qed}

\cmmnt{\medskip

\noindent{\bf Remark} 477Id-477Ie are characteristic of Choquet capacities
(432J-432L);  see 478Xe below.}

\leader{477J}{}\cmmnt{ As an example of the use of 477G, I give
a classical result on one-dimensional Brownian motion.

\medskip

\noindent}{\bf Proposition} Let $\mu_W$ be Wiener measure
on $\Omega=C(\coint{0,\infty})_0$.   Set
$X_t(\omega)=\omega(t)$ for $\omega\in\Omega$.
Then

\Centerline{$\Pr(\max_{s\le t}X_s\ge\alpha)
=2\Pr(X_t\ge\alpha)
=\Bover2{\sqrt{2\pi}}\int_{\alpha/\sqrt t}^{\infty}e^{-u^2/2}du$}

\noindent whenever $t>0$ and $\alpha\ge 0$.

\proof{ Let $\tau$ be the Brownian hitting time to
$F=\{x:x\in\Bbb R$, $x\ge\alpha\}$;  because $F$ is closed,
$\tau$ is a stopping time adapted to
$\langle\Sigma_t\rangle_{t\ge 0}$, as in 477Ic.   Let
$\phi_{\tau}:\Omega\times\Omega\to\Omega$ be the corresponding \imp\
function as in 477G, and set $E=\{\omega:\tau(\omega)<t\}$.
Note that as $\omega(\tau(\omega))=\alpha$ whenever $\tau(\omega)$ is
finite, $\Pr(\tau=t)\le\Pr(X_t=\alpha)=0$, and

\Centerline{$\mu_WE=\Pr(\tau\le t)=\Pr(\max_{s\le t}X_s\ge\alpha)$.}

\noindent Now

$$\eqalignno{\Pr(X_t\ge\alpha)
&=\mu_W\{\omega:\omega(t)\ge\alpha\}
=\mu_W^2\{(\omega,\omega'):\phi_{\tau}(\omega,\omega')(t)\ge\alpha\}\cr
&=\mu_W^2\{(\omega,\omega'):\tau(\omega)\le t,\,
   \phi_{\tau}(\omega,\omega')(t)\ge\alpha\}\cr
\displaycause{because if $\tau(\omega)>t$ then
$\phi_{\tau}(\omega,\omega')(t)=\omega(t)<\alpha$}
&=\mu_W^2\{(\omega,\omega'):\tau(\omega)<t,\,
   \phi_{\tau}(\omega,\omega')(t)\ge\alpha\}\cr
\displaycause{because $\{\omega:\tau(\omega)=t\}$ is negligible}
&=\mu_W^2\{(\omega,\omega'):\tau(\omega)<t,\,
   \omega(\tau(\omega))+\omega'(t-\tau(\omega))\ge\alpha\}\cr
&=\mu_W^2\{(\omega,\omega'):\tau(\omega)<t,\,
   \omega'(t-\tau(\omega))\ge 0\}\cr
&=\int_E\mu_W\{\omega':\omega'(t-\tau(\omega))\ge 0\}\mu_W(d\omega)\cr
&=\Bover12\mu_WE
=\Bover12\Pr(\max_{s\le t}X_s\ge\alpha).\cr}$$

To compute the value, observe that $X_t$ has the same distribution as
$\sqrt tZ$ where $Z$ is a standard normal random variable, so that

\Centerline{$\Pr(X_t\ge\alpha)
=\Pr(Z\ge\bover{\alpha}{\sqrt t})
=\Bover1{\sqrt{2\pi}}\int_{\alpha/\sqrt{t}}^{\infty}e^{-u^2/2}du$.}
}%end of proof of 477J

\leader{477K}{Typical Brownian \dvrocolon{paths}}\cmmnt{ A vast amount is
known concerning the nature of `typical' members of
$\Omega$;   that is to say, a great many
interesting $\mu_W$-conegligible sets have been found.   Here I will
give only a couple of basic results;  the first because it is essential to any
picture of Brownian motion, and the second because it is relevant to a
question in \S479.   Others are in 478M, 478Yi and 479R.

\medskip

\noindent}{\bf Proposition} Let $\mu_W$
be\cmmnt{ one-dimensional} Wiener
measure on $\Omega=C(\coint{0,\infty})_0$.
Then $\mu_W$-almost every element of $\Omega$ is nowhere differentiable.

\proof{ Note first that if $\eta>0$ and $Z$ is a standard normal random
variable, then $\Pr(|Z|\le\eta)\le\eta$, because
the maximum value of the probability density function of $Z$
is $\Bover1{\sqrt{2\pi}}\le\Bover12$.   For $m$, $n$, $k\in\Bbb N$, set

\Centerline{$F_m=\{\omega:\omega\in\Omega$ and
there is a $t\in\coint{0,m}$ such that
$\limsup_{s\downarrow 0}\Bover{|\omega(s)-\omega(t)|}{s-t}<m\}$,}

$$\eqalign{E_{mnk}
&=\{\omega:\omega\in\Omega,\,
        |\omega(2^{-n}(k+2))-\omega(2^{-n}(k+1))|\le 3\cdot 2^{-n}m,\cr
&\mskip200mu
        |\omega(2^{-n}(k+3))-\omega(2^{-n}(k+2))|\le 5\cdot 2^{-n}m,\cr
&\mskip200mu
        |\omega(2^{-n}(k+4))-\omega(2^{-n}(k+3))|\le 7\cdot 2^{-n}m\},\cr}$$
	
\Centerline{$E_{mn}=\bigcup_{k<2^nm}E_{mnk}$.}

\noindent Now we can estimate the measure of $E_{mnk}$, because for any
$\alpha$, $t\ge 0$, $2^{n/2}(X_{t+2^{-n}}-X_t)$ has a standard normal
distribution (taking $X_t(\omega)=\omega(t)$, as usual), so

\Centerline{$\Pr(|X_{t+2^{-n}}-X_t|\le\alpha)\le 2^{n/2}\alpha$;}

\noindent since $E_{mnk}$ is the intersection of three independent sets of
this type,

\Centerline{$\mu_WE_{mnk}
\le 2^{n/2}\cdot 3\cdot 2^{-n}m\cdot2^{n/2}\cdot 5
   \cdot 2^{-n}m\cdot2^{n/2}\cdot 7\cdot 2^{-n}m
=105\,m^32^{-3n/2}$.}

\noindent Accordingly

\Centerline{$\mu_WE_{mn}\le\sum_{k<2^nm}\mu_WE_{mnk}\le 105\,m^42^{-n/2}$.}

Next, observe that
$F_m\subseteq\bigcup_{l\in\Bbb N}\bigcap_{n\ge l}E_{mn}$.   \Prf\
If $\omega\in F_m$, let $t\in\coint{0,m}$ be such that
$\limsup_{s\downarrow 0}\Bover{|\omega(s)-\omega(t)|}{s-t}<m$,
and $l\in\Bbb N$ such that
$|\omega(s)-\omega(t)|\le m(s-t)$ whenever $t<s\le t+4\cdot 2^{-l}$.
Take any $n\ge l$.   Then there is a $k<2^nm$ such that
$2^{-n}k\le t<2^{-n}(k+1)$.   In this case,

\Centerline{$|\omega(2^{-n}(k+j))-\omega(t)|\le 2^{-n}jm$}

\noindent for $1\le j\le 4$,

\Centerline{$|\omega(2^{-n}(k+j+1))-\omega(2^{-n}(k+j))|\le (2j+1)2^{-n}m$}

\noindent for $1\le j\le 3$, and $\omega\in E_{mnk}\subseteq E_{mn}$.\ \Qed

Since

\Centerline{$\mu_W(\bigcup_{l\in\Bbb N}\bigcap_{n\ge l}E_{mn})
\le\liminf_{n\to\infty}\mu_WE_{mn}=0$,}

\noindent $F_m$ is negligible.   So $F=\bigcup_{m\in\Bbb N}F_m$ is
negligible.   But $F$ includes any member of $\omega$ which is
differentiable at any point of $\ooint{0,\infty}$, and more.   So almost
every path is nowhere differentiable.
}%end of proof of 477K

\leader{477L}{Theorem} Let $r\ge 1$ be an integer,
and $\mu_W$ Wiener measure on $\Omega=C(\coint{0,\infty};\BbbR^r)_0$;
for $s>0$ let $\mu_{Hs}$ be $s$-dimensional Hausdorff measure on
$\BbbR^r$.

(a)\cmmnt{ ({\smc Taylor 53})}
$\{\omega(t):t\in\coint{0,\infty}\}$ is $\mu_{H2}$-negligible
for $\mu_W$-almost every $\omega$.

(b) Now suppose that $r\ge 2$.   For
$\omega\in\Omega$, let $F_{\omega}$ be the compact set
$\{\omega(t):t\in[0,1]\}$.
Then for $\mu_W$-almost every $\omega\in\Omega$,
$\mu_{Hs}F_{\omega}=\infty$ for every $s\in\ooint{0,2}$.

\proof{{\bf (a)(i)} For $0\le s\le t$ and $\omega\in\Omega$ set
$K_{st}(\omega)=\{\omega(u):s\le u\le t\}$ and
$d_{st}(\omega)=\diam K_{st}(\omega)$.   Note that
$d_{st}:\Omega\to\coint{0,\infty}$ is continuous
(for the topology of uniform convergence on compact sets, of course).

\medskip

\qquad\grheada\ If $0\le s\le t$ then $\Expn(d_{st}^2)\le 8r(t-s)$.
\Prf\ As $\langle X_{u+s}-X_s\rangle_{u\ge 0}$ and
$\langle X_u\rangle_{u\ge 0}$ have the same distribution, $d_{st}$ has
the same distribution as $d_{0,t-s}$, and we may suppose that $s=0$.
[If you prefer:  if $S_s:\Omega\to\Omega$ is the shift operator of 477Ec,
$K_{st}(\omega)=\omega(s)+K_{0,t-s}(S_s\omega)$, so
$d_{st}(\omega)=d_{0,t-s}(S_s\omega)$, while $S_s$ is \imp.]
In this case,

\Centerline{$d_{0t}(\omega)^2
\le 4\max_{s\in[0,t]}\|\omega(s)\|^2
\le 4\sum_{j=0}^{r-1}\max_{s\in[0,t]}\omega(s)(j)^2$.}

\noindent For each $j<r$,

$$\eqalignno{\int\max_{s\in[0,t]}\omega(s)(j)^2\mu_W(d\omega)
&=\int_0^{\infty}\mu_W\{\omega:
   \max_{s\in[0,t]}\omega(s)(j)^2\ge\beta\}d\beta\cr
&\le\int_0^{\infty}\mu_W\{\omega:\max_{s\in[0,t]}\omega(s)(j)
   \ge\sqrt\beta\}\cr
&\mskip100mu
   +\mu_W\{\omega:\min_{s\in[0,t]}\omega(s)(j)
      \le-\sqrt\beta\}
   d\beta\cr
&=2\int_0^{\infty}\mu_W\{\omega:\max_{s\in[0,t]}\omega(s)(j)
   \ge\sqrt\beta\}d\beta\cr
\displaycause{because $\mu_W$ is invariant under reflections
in $\BbbR^r$, see 477Ed}
&=4\int_0^{\infty}\mu_W\{\omega:\omega(t)(j)\ge\sqrt\beta\}d\beta\cr
\displaycause{by 477J, applied to the $j$th coordinate projection of
$\Omega$ onto $C(\coint{0,\infty})_0$, which is \imp, by 477Da or
477Ed and 477Eg}
&=2\int_0^{\infty}\mu_W\{\omega:\omega(t)(j)^2\ge\beta\}d\beta\cr
\displaycause{again because $\mu_W$ is symmetric}
&=2\int_{\Omega}\omega(t)(j)^2\mu_W(d\omega)
=2\Expn(tZ^2)\cr
\displaycause{where $Z$ is a standard normal random variable}
&=2t.\cr}$$

\noindent Summing,

\Centerline{$\Expn(d_{0t}^2)
\le 4\sum_{j=0}^{r-1}\int_{\Omega}\max_{s\in[0,t]}\omega(s)(j)^2
    \mu_W(d\omega)
\le 8rt$.  \Qed}

\medskip

\qquad\grheadb\ For any $\epsilon>0$, $\Pr(d_{01}\le\epsilon)>0$.   \Prf\
$\{\omega:d_{01}(\omega)\le\epsilon\}$ is a neighbourhood of $0$ for the
topology of uniform convergence on compact sets, so has non-zero measure,
by 477F.\ \Qed

\medskip

\quad{\bf (ii)}
For a non-empty finite set $I\subseteq\coint{0,\infty}$ and
$\omega\in\Omega$ set

\Centerline{$g_I(\omega)
=\sum_{j=0}^{n-1}d_{t_{j-1},t_j}(\omega)^2$}

\noindent where $\langle t_j\rangle_{j\le n}$ enumerates $I$ in increasing
order.   For $0\le s\le t$ and $\omega\in\Omega$ set

\Centerline{$h_{st}(\omega)
=\inf_{\{s,t\}\subseteq I\subseteq[s,t]\text{ is finite}}g_I(\omega)$;}

\noindent then $h_{st}$ is $\Tau_{[s,t]}$-measurable, in the language of
477H.   \Prf\ The point is that
$I\mapsto g_I(\omega)$ is a continuous function of the members of $I$,
at least if we restrict attention to sets $I$ of a fixed size.
So if $D$ is any countable dense subset of $[s,t]$ containing $s$ and $t$,

\Centerline{$h_{st}
=\inf_{\{s,t\}\subseteq I\subseteq D\text{ is finite}}g_I$.}

\noindent On the other hand, if $I\subseteq D$ is enumerated as
$\langle t_i\rangle_{i\le n}$,

\Centerline{$g_I(\omega)
=\sum_{i=0}^{n-1}\max_{u,u'\in D\cap[t_i,t_{i+1}]}
  \|\omega(u)-\omega(u')\|^2$,}

\noindent so $g_I$ is $\Tau_{[s,t]}$-measurable.\ \Qed

\medskip

\quad{\bf (iii)} We need the following facts about the $h_{st}$.

\medskip

\qquad\grheada\ If $0\le s\le t$, then the distribution of $h_{st}$ is the
same as the distribution of $h_{0,t-s}$, again because
$\langle X_{s+u}-X_s\rangle_{u\ge 0}$ has the same distribution as
$\langle X_u\rangle_{u\ge 0}$.   [In the language suggested in the proof of
(i-$\alpha$),
we have $g_{s+I}(\omega)=g_I(S_s(\omega))$ for any $\omega\in\Omega$
and non-empty finite $I\subseteq\coint{0,\infty}$, so
$h_{st}(\omega)=h_{0,t-s}(S_s\omega)$.]

\medskip

\qquad\grheadb\ $h_{0t}$ has finite expectation.   \Prf\
$h_{0t}\le g_{\{0,t\}}=d_{0t}^2$, so we can use (i).\ \Qed

\medskip

\qquad\grheadc\ $h_{su}\le h_{st}+h_{tu}$ if $s\le t\le u$.   \Prf\ If
$\{s,t\}\subseteq I\subseteq[s,t]$ and $\{t,u\}\subseteq J\subseteq[t,u]$
then $\{s,u\}\subseteq I\cup J\subseteq[s,u]$ and
$g_{I\cup J}=g_I+g_J$.\ \QeD\

\medskip

\qquad\grheadd\ If $s\le t\le u$ then $h_{st}$ and $h_{tu}$ are
independent, because
$\Tau_{[s,t]}$ and $\Tau_{[t,u]}$ are independent (477H(b-i)).

\medskip

\qquad\grheade\ The
distribution of $h_{0t}$ is the same as the distribution of $th_{01}$
whenever $t\ge 0$.
\Prf\ The case $t=0$ is trivial.   For $t>0$, define
$U_t:\Omega\to\Omega$ by saying that
$U_t(\omega)(s)=\Bover1{\sqrt{t}}\omega(ts)$, as in 477Ee.   Then

\Centerline{$K_{su}(U_t(\omega))=\Bover1{\sqrt{t}}K_{ts,tu}(\omega)$,
\quad$d_{su}(U_t(\omega))=\Bover1{\sqrt{t}}d_{ts,tu}(\omega)$,}

\Centerline{$g_I(U_t(\omega))=\Bover1tg_{tI}(\omega)$,
\quad$h_{su}(U_t(\omega))=\Bover1th_{ts,tu}(\omega)$,}

\noindent whenever $s\le u$, $\{s,u\}\subseteq I\subseteq[s,u]$
and $\omega\in\Omega$, and

$$\eqalignno{\mu_W\{\omega:th_{01}(\omega)\ge\alpha\}
&=\mu_W\{\omega:th_{01}(U_t(\omega))\ge\alpha\}\cr
\displaycause{because $U_t$ is an automorphism of $(\Omega,\mu_W)$}
&=\mu_W\{\omega:h_{0t}(\omega)\ge\alpha\}\cr}$$

\noindent for every $\alpha\in\Bbb R$.\ \Qed

\medskip

\qquad{\bf($\pmb{\zeta}$)} Consequently

\Centerline{$\Expn(h_{st})=\Expn(h_{0,t-s})=(t-s)\Expn(h_{01})$}

\noindent whenever $s\le t$, and

\Centerline{$\Expn(h_{st})+\Expn(h_{tu})=\Expn(h_{su})$}

\noindent whenever $s\le t\le u$.   Since $h_{st}+h_{tu}\ge h_{su}$, by
(iii), we must have $h_{st}+h_{tu}\eae h_{su}$.

\medskip

\qquad{\bf($\pmb{\eta}$)} For any $\eta>0$, $\Pr(h_{01}\le 4\eta^2)>0$.
\Prf\ By
(i-$\beta$),
 $\Pr(d_{01}\le 2\eta)>0$, and $h_{01}\le g_{\{0,1\}}=d_{01}^2$.\ \Qed

\medskip

\quad{\bf (iv)} For $t\ge 0$ let $\phi_t$ be the characteristic function of
$h_{0t}$, that is, $\phi_t(\alpha)=\Expn(\exp(i\alpha h_{0t}))$ for
$\alpha\in\Bbb R$ (285Ab).   Working through the facts listed above, we see
that

$$\eqalignno{\phi_t(1)
&=\Expn(\exp(ih_{0t}))
=\Expn(\exp(ith_{01}))\cr
\displaycause{by (iii-$\epsilon$)}
&=\phi_1(t),\cr
\phi_1(s)\phi_1(t)
&=\phi_s(1)\phi_t(1)
=\Expn(\exp(ih_{0s}))\Expn(\exp(ih_{0t}))\cr
&=\Expn(\exp(ih_{0s}))\Expn(\exp(ih_{s,s+t}))\cr
\displaycause{by (iii-$\alpha$)}
&=\Expn(\exp(ih_{0s})\exp(ih_{s,s+t}))\cr
\displaycause{because $h_{0s}$ and $h_{s,s+t}$ are independent, by (c-iv)}
&=\Expn(\exp(i(h_{0s}+h_{s,s+t})))
=\Expn(ih_{0,s+t})\cr
\displaycause{by (iii-$\zeta$)}
&=\phi_1(s+t),\cr}$$

\noindent for all $s$, $t\ge 0$;  while $\phi_1$ is differentiable, because
$h_{01}$ has finite expectation ((iii-$\alpha$) above and 285Fd).
It follows that there is a
$\gamma\in\Bbb R$ such that $\phi_1(t)=e^{i\gamma t}$ for every
$t\in\Bbb R$.   \Prf\ Set $\gamma=\Bover1i\phi_1'(0)=\Expn(h_{01})$
(285Fd) and $\psi(t)=e^{-i\gamma t}\phi_1(t)$ for $t\in\Bbb R$.
If $t>0$, then

\Centerline{$\phi'_1(t)
=\lim_{s\downarrow 0}\Bover1s(\phi_1(t+s)-\phi_1(t))
=\phi_1(t)\lim_{s\downarrow 0}\Bover1s(\phi_1(s)-1)
=i\phi_1(t)\gamma$}

\noindent and $\psi'(t)=0$.
Since $\psi$ is continuous on $\coint{0,\infty}$ (285Fb), it must be
constant, and $\phi_1(t)=e^{i\gamma t}\psi(0)=e^{i\gamma t}$ for every
$t\ge 0$.   As for negative $t$, we have

\Centerline{$\phi_1(t)=\overline{\phi_1(-t)}=\overline{e^{-i\gamma t}}
=e^{i\gamma t}$}

\noindent for $t\le 0$, by 285Fc.\ \Qed

\medskip

\quad{\bf (v)}
Thus we see that $h_{01}$ has the same characteristic function
as the distribution concentrated at $\gamma$, and this must therefore be
the distribution of $h_{01}$ (285M);
that is, $h_{01}\eae\gamma$.   Now (iii-$\eta$) tells us that $\gamma=0$.

Since $h_{0t}$ has the same distribution as $th_{01}$, $h_{0t}\eae 0$ for
every $t\ge 0$.   But now observe that if $t\ge 0$, $\omega\in\Omega$ and
$h_{0t}(\omega)=0$, then for any $\eta>0$
there is a finite $I\subseteq[0,t]$, containing
$0$ and $t$, such that $g_I(\omega)\le\eta^2$.
This means that $K_{0t}(\omega)$ can be covered by finitely many
sets $K_{t_j,t_{j+1}}(\omega)$ with
$\sum_{j=0}^{n-1}\diam K_{t_j,t_{j+1}}(\omega)^2\le\eta^2$.   All the
diameters here must of course be less than or equal to $\eta$.   As
$\eta$ is arbitrary, $\mu_{H2}K_{0t}(\omega)=0$.

For each $t\ge 0$, this is true for almost every $\omega$.   But this means
that, for almost every $\omega$, $\mu_{H2}K_{0n}(\omega)=0$ for every $n$,
and $\mu_{H2}\{\omega(t):t\ge 0\}=0$, as claimed.

\medskip

{\bf (b)(i)} To begin with, take a fixed $s\in\ooint{0,2}$.
Let $\mu_{L1}$ be Lebesgue measure on $[0,1]$.
For each $\omega\in\Omega$, let $\zeta_{\omega}$ be the image
measure $\mu_{L1}(\omega\restr[0,1])^{-1}$ on $F_{\omega}$.   Then

$$\eqalignno{\int_{\Omega}\int_{F_{\omega}}\int_{F_{\omega}}
   &\Bover1{\|x-y\|^s}\zeta_{\omega}(dx)\zeta_{\omega}(dy)
    \mu_W(d\omega)\cr
&=\int_{\Omega}\int_0^1\int_0^1
   \Bover1{\|\omega(t)-\omega(u)\|^s}dt\,du\,\mu_W(d\omega)\cr
&=\int_0^1\int_0^1\int_{\Omega}
   \Bover1{\|\omega(t)-\omega(u)\|^s}\mu_W(d\omega)\,dt\,du\cr
\displaycause{of course
$(\omega,t,u)\mapsto\Bover1{\|\omega(t)-\omega(u)\|^s}$ is continuous and
non-negative, so there is no difficulty with the change in order of
integration}
&=2\int_0^1\int_u^1\int_{\Omega}
   \Bover1{\|\omega(t)-\omega(u)\|^s}\mu_W(d\omega)\,dt\,du\cr
&=2\int_0^1\int_u^1\int_{\Omega}
   \Bover1{\|\omega(t-u)\|^s}\mu_W(d\omega)\,dt\,du\cr
\displaycause{because $X_t-X_u$ has the same distribution as $X_{t-u}$, as
in (a-i-$\alpha$)}
&=2\int_0^1\int_0^{1-u}\int_{\Omega}
   \Bover1{\|\omega(t)\|^s}\mu_W(d\omega)\,dt\,du
\le 2\int_0^1\int_{\Omega}
   \Bover1{\|\omega(t)\|^s}\mu_W(d\omega)\,dt\cr
&=2\int_0^1\int_{\BbbR^r}\Bover1{(\sqrt{2\pi t})^r}
   \Bover1{\|x\|^s}e^{-\|x\|^2/2t}\mu(dx)\,dt\cr
\displaycause{here $\mu$ is Lebesgue measure on $\BbbR^r$}
&=\Bover2{(\sqrt{2\pi})^r}\int_0^1\Bover1{t^{r/2}}\int_0^{\infty}
   \Bover1{\alpha^s}\cdot r\beta_r\alpha^{r-1}e^{-\alpha^2/2t}
   d\alpha\,dt\cr
&=\Bover{2r\beta_r}{(\sqrt{2\pi})^r}\int_0^1\Bover1{t^{r/2}}
   \int_0^{\infty}
   \Bover{(\beta\sqrt t)^{r-1}}{(\beta\sqrt t)^s}e^{-\beta^2/2}
   \sqrt{t}d\beta\,dt\cr
&=\Bover{2r\beta_r}{(\sqrt{2\pi})^r}\int_0^1
   \Bover1{t^{s/2}}dt
   \int_0^{\infty}\beta^{r-s-1}e^{-\beta^2/2}d\beta
<\infty\cr}$$

\noindent because $\Bover{s}2<1$ and $r-s>0$.
So $\biggerint_{F_{\omega}}\biggerint_{F_{\omega}}
   \Bover1{\|x-y\|^s}\zeta_{\omega}(dx)\zeta_{\omega}(dy)$
is finite for almost every
$\omega$.   Since $\zeta_{\omega}$ is always a probability measure
with support included in $F_{\omega}$,
$\mu_{Hs}F_{\omega}=\infty$ for all such $\omega$ (471Tb).

\medskip

\quad{\bf (ii)} Setting $s_n=2-2^{-n}$ for each $n$, we see that, for almost
every $\omega\in\Omega$, $\mu_{Hs_n}F_{\omega}=\infty$ for every $n$.
But for
any such $\omega$, $\mu_{Hs}F_{\omega}=\infty$ for every $s\in\ooint{0,2}$,
by 471L.
}%end of proof of 477L

\exercises{\leader{477X}{Basic exercises (a)}
%\spheader 477Xa
Use 272Yc\formerly{2{}72Ye} to simplify the formulae in the proof of 477B.
%477B

\spheader 477Xb Let $\mu_W$ be Wiener measure on
$\Omega=C(\coint{0,\infty})_0$,
and set $X_t(\omega)=\omega(t)$ for $t\ge 0$ and $\omega\in\Omega$.
Let $f$ be a real-valued tempered function on $\Bbb R$ (definition:
284D).   For $x\in\Bbb R$ and $0<t<b$, let
$\nu_x^{(t,b)}$ be the distribution of a normally distributed random
variable with mean $x$ and variance $b-t$, so that
$g(x,t)=\int f(y)\nu_x^{(t,b)}(dy)$ can be regarded as
the expectation of $f(X_b)$ given that $X_t=x$.   (i) Show that
$g$ satisfies the {\bf backwards heat equation}
$2\Bover{\partial g}{\partial t}+\Bover{\partial^2g}{\partial x^2}=0$.
(ii) Interpret this in terms of the disintegration
$\nu_x^{(t,b)}=\int\nu_z^{(u,b)}\nu_x^{(t,u)}(dz)$ as $u\downarrow t$.
%477B

\spheader 477Xc(i) Show that the measure $\hat\mu^r$ of 477Da can be
constructed directly by applying 455A with
$(X_t,\Cal B_t)=(\BbbR^r,\Cal B(\BbbR^r))$ for every $t\ge 0$ and suitable
Gaussian distributions $\nu_x^{(s,t)}$ on $\BbbR^r$.   (ii) Show that the
measure $\hat\mu^r$ can be constructed by applying 455A to
$T=r\times\coint{0,\infty}$ with its lexicographic ordering and suitable
Gaussian distributions $\nu_x^{(s,t)}$ on $\Bbb R$.
%477D

\spheader 477Xd Let $r\ge 1$ be an integer.
(i) Show that there is a centered Gaussian process
$\langle Y_t\rangle_{t\in[0,1]}=\langle Y^{(i)}_t\rangle_{t\in[0,1],i<r}$
such that
$\Expn(Y^{(i)}_s\times Y^{(j)}_t)=0$ if $i\ne j$, $\min(s,t)-st$ otherwise.
(ii) Show that if $\langle X_t\rangle_{t\ge 0}$ is ordinary $r$-dimensional
Brownian motion, then $\family{t}{[0,1]}{Y_t}$ has the same distribution as
$\family{t}{[0,1]}{X_t-tX_1}$.
(iii) Show that the process $\family{t}{[0,1]}{Y_t}$
(the {\bf Brownian bridge})
can be represented by a Radon probability
measure $\mu_{\text{bridge}}$ on the space $C([0,1];\BbbR^r)_{00}$ of
continuous functions from $[0,1]$ to $\BbbR^r$ taking the value $0$ at both
ends of the interval.   (iv) For
$\omega\in C([0,1];\BbbR^r)_{00}$ define
$\Reverse{\omega}\in C([0,1];\BbbR^r)_{00}$ by setting
$\Reverse{\omega}(t)=\omega(1-t)$ for $t\in[0,1]$.   Show that
$\omega\mapsto\Reverse{\omega}$ is an automorphism of
$(C([0,1];\BbbR^r)_{00},\mu_{\text{bridge}})$.
%477D

\spheader 477Xe Let $(\Omega,\Sigma,\mu)$ be a complete
probability space and
$\langle X_t\rangle_{t\ge 0}$ a family of real-valued random variables on
$\Omega$ with independent increments.   For $I\subseteq\coint{0,\infty}$
let $\Tau_I$ be the $\sigma$-algebra generated by
$\{X_s-X_t:s$, $t\in I\}$.
Let $\Cal I$ be a family of subsets of $\coint{0,\infty}$ such that
for all distinct $I$, $J\in\Cal I$ either $\sup I\le\inf J$ or
$\sup J\le\inf I$.   Show that $\family{I}{\Cal I}{\Tau_I}$ is an
independent family of $\sigma$-algebras.
%477H

\spheader 477Xf Suppose that $H\subseteq\BbbR^r$ is an F$_{\sigma}$ set
and that $\tau:\Omega\to[0,\infty]$ is the Brownian hitting time to $H$,
as defined in 477I.   Show that $\tau$ is Borel measurable.
%477I

\spheader 477Xg Let $\mu_W$ be one-dimensional Wiener measure, and $\tau$
the hitting time to $\{1\}$.   Show that the distribution of
$\tau$ has probability density function
$x\mapsto\Bover1{x\sqrt{2\pi x}}e^{-1/2x}$ for $x>0$.
%477J

\spheader 477Xh Let $\mu_W$ be one-dimensional Wiener measure on
$\Omega=C(\coint{0,\infty})_0$.   Show that, for $\mu_W$-almost every
$\omega\in\Omega$, the total variation $\Var_{[s,t]}(\omega)$ is infinite
whenever $0\le s<t$.
%477K

\leader{477Y}{Further exercises (a)}
%\spheader 477Ya
Write $D_n$ for
$\{2^{-n}i:i\in\Bbb N\}$ and $D=\bigcup_{n\in\Bbb N}D_n$,
the set of dyadic rationals in
$\coint{0,\infty}$.   For $d\in D$ define $f_d\in C(\coint{0,\infty})$
as follows.   If $n\in\Bbb N$, $f_n(t)=0$ if $t\le n$, $t-n$ if
$n\le t\le n+1$, $1$ if $t\ge n+1$.   If $d=2^{-n}k$ where $n\ge 1$ and
$k\in\Bbb N$ is
odd, $f_d(t)=\max(0,\Bover1{\sqrt{2^{n+1}}}(1-2^n|t-d|))$ for $t\ge 0$.
Now let $\family{d}D{Z_d}$ be an independent family of standard normal
distributions, and set $\omega_n(t)=\sum_{d\in D_n}f_d(t)Z_d$ for $t\ge 0$,
so that each $\omega_n$ is a random continuous function on
$\coint{0,\infty}$.   Show that for any $n\in\Bbb N$ and $\epsilon>0$,

\Centerline{$\Pr(\sup_{t\in[0,n]}|\omega_{n+1}(t)-\omega_n(t)|\ge\epsilon)
\le 2^nn\cdot\Bover{2}{\sqrt{2\pi}}
  \biggerint_{2\epsilon\sqrt{2^n}}^{\infty}e^{-x^2/2}dx$,}

\noindent and hence that $\sequencen{\omega_n}$ converges almost surely to a
continuous function.   Explain how to interpret this as a
construction of Wiener measure on $\Omega=C(\coint{0,\infty})_0$,
as the image measure $\mu_G^Dg^{-1}$ where $g:\BbbR^D\to\Omega$ is almost
continuous (for the topology $\frak T_c$ on $\Omega$) and $\mu_G^D$ is the
product of copies of the standard normal distribution $\mu_G$.
%477B

\spheader 477Yb\dvAformerly{4{}56Ye}
Fix $p\in\ooint{0,1}\setminus\{\bover12\}$.   (i) For $\alpha\in\Bbb R$ set
$h_{\alpha}(t)=|t-\alpha|^{p-\bover12}-|t|^{p-\bover12}$ when
this is defined.   Show that $h_{\alpha}\in\eusm L^2(\mu_L)$,
where $\mu_L$ is Lebesgue measure,
and that $\|h_{\alpha}\|^2_2=|\alpha|^{2p}\|h_1\|^2_2$ and
$\|h_{\alpha}-h_{\beta}\|_2=\|h_{\alpha-\beta}\|_2$ for all $\alpha$,
$\beta\in\Bbb R$.
(ii) Show that there is a centered Gaussian process
$\family{\alpha}{\Bbb R}{X_{\alpha}}$ such that
$\Expn(X_{\alpha}\times X_{\beta})
=|\alpha|^{2p}+|\beta|^{2p}-|\alpha-\beta|^{2p}$ for all $\alpha$,
$\beta\in\Bbb R$.
(iii) Show that such a process can be represented by a Radon measure on
$C(\Bbb R)$.   \Hint{477Ya.}   (This is {\bf fractional Brownian motion}.)
%477B

\spheader 477Yc Let $\mu_W$ be Wiener measure on
$\Omega=C(\coint{0,\infty};\BbbR^r)_0$, where $r\ge 1$,
and set $X_t(\omega)=\omega(t)$ for $t\ge 0$ and $\omega\in\Omega$.
Let $f$ be a real-valued tempered function on $\BbbR^r$ (definition:
284Wa).   For $x\in\BbbR^r$ and $0<t<b$, let
$\nu_x^{(t,b)}$ be the distribution of $x+X_{b-t}$, so that
$g(x,t)=\int f(y)\nu_x^{(t,b)}(dy)$ can be regarded as
the expectation of $f(X_b)$ given that $X_t=x$.   Show that
$g$ satisfies the backwards heat equation
$2\Bover{\partial g}{\partial t}+\sum_{i=0}^{r-1}\Pdd{g}{\xi_i}=0$.
%477Xb 477B

\spheader 477Yd Let $r\ge 1$ be an integer,
and $\nu$ a Radon probability measure on $\BbbR^r$ such that
$x\mapsto\varinnerprod{a}{x}$ has expectation $0$ and variance
$\|a\|^2$ for every $a\in\BbbR^r$.
Let $\Omega$ be $C(\coint{0,\infty};\BbbR^r)_0$, and
for $\alpha>0$ define $f_{\alpha}:(\BbbR^r)^{\Bbb N}\to\Omega$ by setting
$f_{\alpha}(x)(t)
=\sqrt{\alpha}(\sum_{i<n}z(i)+\Bover1{\alpha}(t-n\alpha)z(n))$ when
$z\in(\BbbR^r)^{\Bbb N}$, $n\in\Bbb N$ and $n\alpha\le t\le(n+1)\alpha$;
let $\mu_{\alpha}$ be the image Radon measure $\nu^{\Bbb N}f_{\alpha}^{-1}$
on $\Omega$.   Show that Wiener measure $\mu_W$ is the limit
$\lim_{\alpha\downarrow 0}\mu_{\alpha}$ for the narrow topology.
%477C calculate characteristic function of X_t

\spheader 477Ye Let $\mu_W$ be Wiener measure on $C(\coint{0,\infty})_0$,
and $\gamma>\bover12$.   Show that
$\lim_{t\to\infty}\Bover1{t^{\gamma}}\omega(t)
=\lim_{t\downarrow 0}\Bover1{t^{1-\gamma}}\omega(t)=0$
for $\mu_W$-almost every $\omega$.
%477E


\spheader 477Yf Write out a proof of 477G which works directly from the
Gaussian-distribution
characterization of Wiener measure, without appealing to results from
\S455 other than 455L.   (I think you will need to start with stopping
times taking finitely and countably many values, as in 455C;  but
you will find great simplifications.)
%477G

\spheader 477Yg Let $\hat\mu$ be the Gaussian distribution on
$\BbbR^{\coint{0,\infty}}$ corresponding to Brownian motion, as in
477A.   For $t\ge 0$ let
$\Sigma_t$ be the family of Baire subsets of $\BbbR^{\coint{0,\infty}}$
determined by coordinates in $[0,t]$, and
$\Sigma^+_t=\bigcap_{s>t}\Sigma_s$.
For $\omega\in\BbbR^{\coint{0,\infty}}$ set
$\tau(\omega)=\inf\{q:q\in\Bbb Q$, $\omega(q)\ge 1\}$, counting
$\inf\emptyset$ as $\infty$.   Show that $\tau$ is a stopping time adapted
to $\langle\Sigma^+_t\rangle_{t\ge 0}$.
For $\omega$, $\omega'\in\BbbR^{\coint{0,\infty}}$
define $\phi_{\tau}(\omega,\omega')\in\BbbR^{\coint{0,\infty}}$ by setting
$\phi_{\tau}(\omega,\omega')(t)=\omega(t)$ if $t\le\tau(\omega)$,
$\omega(\tau(\omega))+\omega'(t-\tau(\omega))$ if $t>\tau(\omega)$.
Show that $\phi_{\tau}$ is not \imp\ for the product measure
$\hat\mu^2$ and $\hat\mu$.    \Hint{show that
$\{\omega:\tau(\omega)\in D\}$ is negligible for every countable set $D$.}
%477G

\spheader 477Yh Let $\mu_W$ be one-dimensional Wiener measure
on $\Omega=C(\coint{0,\infty})_0$.   (i) Show
by induction on $k$ that $\Pr($there are $t_0<t_1<\ldots<t_k\le t$ such
that $X_{t_j}=(-1)^j$ for every $j\le k)
=\Pr($there is an $s\le t$ such that
$X_s=2k+1)$ for any $t\ge 0$.   \Hint{477J.}  (ii) For $\omega\in\Omega$,
$k\in\Bbb N$ define $\tau_k(\omega)$ by saying that
$\tau_0(\omega)=\inf\{t:|\omega(t)|=1\}$,
$\tau_{k+1}(\omega)=\inf\{t:t>\tau_k(\omega)$,
$\omega(t)=-\omega(\tau_k(\omega))\}$.   Show that $\tau_k$ is a stopping
time adapted to the family $\langle\Sigma_t\rangle_{t\ge 0}$ of 477H,
and is finite $\mu_W$-a.e.   (iii) Set
$E_k=\{\omega:\tau_k(\omega)\le 1<\tau_{k+1}(\omega)\}$,
$p_k=\mu_WE_k$, $F_k=\{\omega:$ there is an $s\le 1$ such that
$\omega(s)=2k+1\}$, $q_k=\mu_WF_k$.   Show that

\Centerline{$q_k
=\Bover12p_k+\sum_{j=k+1}^{\infty}p_j
=\Bover2{\sqrt{2\pi}}\int_{2k+1}^{\infty}e^{-x^2/2}dx$.}

\noindent (iv) Show that
$\Pr(\tau_0\le 1)=\sum_{k=0}^{\infty}p_k=2\sum_{k=0}^{\infty}(-1)^kq_k$.
%477J

\spheader 477Yi Let $\mu_W$ be Wiener measure on
$\Omega=C(\coint{0,\infty};\BbbR^r)_0$, where $r\ge 1$.
Show that, for $\mu_W$-almost every $\omega\in\Omega$,
$\{\Bover{\omega(t)}{\|\omega(t)\|}:t\ge t_0$, $\omega(t)\ne 0\}$
is dense in $\partial B(\tbf{0},1)$ for every $t_0\ge 0$.
%477I mt47bits out of order query

\spheader 477Yj(i) Show that, for any $r\ge 1$,
the topology $\frak T_c$ of uniform
convergence on compact sets is a complete linear space topology on
$\Omega=C(\coint{0,\infty};\BbbR^r)_0$.   (ii) Show that Wiener measure on
$\Omega$ is a centered Gaussian measure in the sense of 466N.
\Hint{466Ye.}
%477D out of order query
}%end of exercises

\endnotes{
\Notesheader{477} The `first proof' of 477A calls on a result which is
twenty-five %checked 2010
pages into \S455, and you will probably be glad to be assured
that all it really needs is 455A, fragments from the proof
of 455P, and part (a) of the proof of 455R.
So it is not enormously harder than the `second proof', based on the
elementary theory of Gaussian processes.

With this theorem, we have two routes to the first target:  setting up a
measure space with a family of random variables representing Brownian
motion.   I repeat that this is a secondary issue.   Brownian
motion begins with
the family of joint distributions of finite indexed sets
$(X_{t_0},\ldots,X_{t_n})$ satisfying the properties listed in 477A.
It is one of the triumphs of Kolmogorov's theory of probability that these
distributions can be represented by a family of
real-valued functions on a set with a countably additive measure;  but they
would still be of the highest importance if they could not.   In order to
show that it can be done, we
can use either the time-dependent approach based on conditional
expectations, as in 455A, 455P and the `first proof' of 477A, or the
timeless Gaussian-distribution approach through 456C, as in the `second
proof' of 477A.   Both, of course, depend on Kolmogorov's theorem 454D.
They have different advantages, and it will be very useful to be able
to call on the intuitions of both.   The `first proof' leads us naturally
into the theory of L\'evy processes, in which other families of
distributions replace normal distributions.

To get to the continuity of
sample paths, we need to do quite a bit more, and the proof of 477B is one
way of filling the gap.   At this point it becomes tempting to abandon both
proofs of 477A and start again with the method of 477Ya, the
`L\'evy-Ciesielski' construction, {\it not} using
Kolmogorov's theorem.   But if we do this, we shall have to devise a new
argument to prove the strong Markov property 477G, rather than quoting
455U.   Of course the special properties of Gaussian processes mean that a
direct proof of 477G is still quite a bit easier than the general results
of \S455 (477Yf).   I make no claim that one
approach is `better' than another;  they all throw light on the result.

What I here call `Wiener measure' (477B) is a particular realization of
Brownian motion.
It is so convincing that it is tempting to regard it as `the' real basis
of Brownian motion.   I do not mean to assert this in any way which
will bind me in future.   But (as a measure theorist, rather than a
probabilist) I think that the specific measures of 477B and 477D
are worth as much attention as any.
One reason for not insisting that the space
$C(\coint{0,\infty})_0$ is the only right place to start is that we may at
any moment wish to move to something smaller, as in 477Ef.   The approach
here gives a very direct language in which to express
theorems of the form `almost
every Brownian path is $\ldots$' (477K, 477L), and every such theorem
carries an implicit suggestion that we could move to a conegligible set
and a subspace measure.

In 477C I sketch an alternative characterization of one-dimensional Wiener
measure.   Five pages seem to be rather a lot for a proof of
something which surely has to be true, if we can get the hypotheses right;
but I do not see a genuinely
shorter route, and I think in fact that the indigestibility of the argument
as presented is due to compression more than to pedantry.
At least I have tried to put the key step into part (a-i) of the proof.
We have to use the Central Limit Theorem;  we have to use a
finite-approximation version of it, rather than a limit version;  the ideas
of this proof do not demand Lindeberg's formulation, but this is what we
have to hand in Volume 2;  and if we are going to consider interpolations
for general random walks, we need something to force a sufficient degree
of equicontinuity, and ($\dagger$) is what comes naturally from the result
in 274F.   It is surely obvious that I have been half-hearted in the
generality of the theorem as given.   There can be no reason for insisting
on steps being at uniform time intervals, or on stationary processes, or
even on variances being exactly correct, provided that everything averages
out nicely in the limit.   The idea does require that steps be
independent, but after that we just need hypotheses adequate for the
application of the Central Limit Theorem.

Clearly we can also look for $r$-dimensional versions of the theorem.   I
have not done so because they would inevitably demand vector-valued
versions of the Central Limit Theorem, and while a combination of
the ideas of
\S\S274 and 456 would take us a long way, it would not belong to this
section.   However I give 477Yd as an example which can be dealt with
without much general theory.

Already in the elementary results
477Eb-477Eg % 477Eb 477Ec 477Ed 477Ee 477Ef 477Eg
we see that Wiener measure is a remarkable construction.
It is a general principle that the more symmetries an object has, the more
important it is;  this one has a surprising symmetry (477Ef), which is even
better.   I take it as confirmation that we have a good representation,
that all these symmetries can be represented by actual \imp\ functions,
rather than leaving them as manipulations of distributions.

477F is a natural result, and a further confirmation that in
$C(\coint{0,\infty};\BbbR^r)_0$ we have got hold
of an appropriate space of functions.   The proof I give depends on an
aspect of the structure developed in 477Ya.

The next really important result is the `strong Markov property' (477G).
This is clearly a central property of Brownian motion.   It may not
be quite so clear what the formulation here is trying to say.   As in 477E,
I am expressing the result in terms of an \imp\ function.   This makes no
sense unless we have a probability space $\Omega$ in which we can put two
elements $\omega$, $\omega'$ together to form a third;  so we are more or
less forced to look at a space of paths.   But not all spaces of paths will
do.   For an indication of what can happen if we work with
the wrong realization, see 477Yg.

In 477H we have two kinds of zero-one law.   One, 477Hd, is explicit; the
tail $\sigma$-algebra $\bigcap_{t\ge 0}\hat\Tau_{\coint{t,\infty}}$ behaves
like the tail of an independent sequence of  $\sigma$-algebras (272O).
But the formula `$\bigcap_{s>t}\Sigma_s\subseteq\hat\Tau_{[0,t]}$' in
477Hc can
be thought of as a relative zero-one law.   There are many events (e.g.,
$\{\omega:\liminf_{s\downarrow t}\bover1{s-t}\omega(s)\ge 0\}$)
which belong to $\bigcap_{s>t}\Sigma_s$ and not to $\Tau_{[0,t]}$ or
$\Sigma_t$, but all collapse to events in $\Tau_{[0,t]}$
if we rearrange them on appropriate negligible sets.   This is really a
special case of 455T.

The formulae in the first application of the strong Markov property (477J)
demand a bit of concentrated attention, but
I think that the key step at the end of the proof (moving from $\mu_W^2$ to
$\int\ldots d\mu_W$) faithfully represents the intuition:  once we have
reached the level $\alpha$, we have an even chance of rising farther.
For the discrete case, 272Yc is a version of the same idea.
From the distribution of the
hitting time to $\{\alpha\}$ we can deduce the distribution of the
hitting time to $\{-1,1\}$ (477Yh);   but I do not know of a corresponding
exact result for the hitting time to the unit circle for two-dimensional
Brownian motion.

I expect you have been shown a continuous
function which is nowhere differentiable.   In 477K we see that
`almost every'
function is of this type.   What a hundred and fifty years ago seemed to be
an exotic counter-example now presents itself as a representative of the
typical case.
The very crude estimates in the proof of 477K are supposed to furnish a
straightforward proof of the result, without asking for anything which
might lead to refinements.   Of course there is much more to be said,
starting with 477Xh.   In 477L we have an interesting result which will
be useful in \S479, when
I return to geometric properties of Brownian paths.


}%end of notes

\discrpage

