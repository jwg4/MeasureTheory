\frfilename{mt45.tex}
\versiondate{5.6.09}
\copyrightdate{1998}

\def\chaptername{Perfect measures and disintegrations}
\def\sectionname{Introduction}

\newchapter{45}

One of the most remarkable features of countably additive
measures is that they provide us with a framework for probability
theory, as described in Chapter 27.   The extraordinary achievements of
probability theory since Kolmogorov are to a large extent possible
because of the rich variety of probability measures which can be
constructed.   We have already seen image measures (234C\formerly{1{}12E})
and product measures (\S254).   The former are elementary,
but a glance at the index will confirm that they have many
surprises to offer;  the latter are obviously fundamental to any idea of
what probability theory means.   In this chapter I will look at some
further constructions.   The most important are those associated with
`disintegrations' or `regular conditional probabilities' (\S\S452-453)
and methods for confirming the existence of measures on product spaces
with given images on subproducts (\S454, 455A).   We find that these
constructions have to be based on measure spaces of special types;  the
measures involved in the principal results are the Radon measures of
Chapter 41 (of course), the compact and perfect measures of Chapter 34,
and an intermediate class, the `countably compact' measures of
{\smc Marczewski 53} (451B).   So the first section of this chapter is a
systematic discussion of compact, countably compact and perfect
measures.

A `disintegration', when present, is likely to provide us with a
particularly effective instrument for studying a measure, analogous to
Fubini's theorem for product measures (see 452F).   \S\S452-453
therefore concentrate on theorems guaranteeing the existence of
disintegrations compatible with some pre-existing structure, typically
an \imp\ function (452I, 452O, 453K) or a product structure
(452M).   Both depend on the existence of
suitable liftings, and for the topological version in \S453 we need a
`strong' lifting, so much of that section is devoted to the study of
such liftings.

One of the central concerns of probability theory is to understand
`stochastic processes', that is, models of systems evolving
randomly over time.   If we think of our state space as consisting
of functions, so that a whole possible history is described by a
random function of time, it is natural to think of our functions as
members of some set $\prod_{n\in\Bbb N}Z_n$ (if we think of
observations as being taken at discrete time intervals) or
$\prod_{t\in\coint{0,\infty}}Z_t$ (if we regard our system as
evolving continuously), where $Z_t$ represents the set of possible
states
of the system at time $t$.   We are therefore led to consider
measures on such product spaces, and the new idea is that we may
have some definite intuition concerning the joint distribution of
{\it finite} strings $(f(t_0),\ldots,f(t_n))$ of values of our
random function, that is to say, we may think we know something
about the image measures on finite products
$\prod_{i\le n}Z_{t_i}$.   So we come immediately to a fundamental
question:
given a (probability) measure $\mu_J$ on $\prod_{i\in J}Z_i$ for
each finite $J\subseteq T$, when will there be a measure on
$\prod_{i\in T}Z_i$ compatible with every $\mu_J$?   In \S454 I give the
most important generally applicable existence theorems for such
measures, and in 455A-455E %455A 455C 455E
I show how they can be applied to a general
construction for models of Markov processes.   These models enable us to
discuss the Markov property either in terms of disintegrations or in terms
of conditional expectations (455C, 455O), and for L\'evy processes, in
terms of \imp\ functions (455U).

The abstract theory of \S454 yields measures on product spaces which,
from the point of view of a probabilist, are unnaturally large, often
much larger than intuition suggests.   Some of the most powerful results in
the theory of Markov processes, such as the strong Markov property (455O),
depend on moving to much smaller spaces;  most notably the space of
\cadlag\ functions (455G), but the larger space of \callal\ functions is
also of interest.   The most important example, Brownian motion,
will have to wait for Chapter
47, but I give the basic general theory of L\'evy processes in complete
metric groups.

One of the defining characteristics of Brownian motion is the fact that
all its finite-dimensional marginals are Gaussian distributions.
Stochastic processes with this
property form a particularly interesting class, which I examine in
\S456.   From the point of view of this volume, one of their most
striking properties is Talagrand's theorem
that, regarded as measures on powers $\BbbR^I$, they are $\tau$-additive
(456O).

The next two sections look again at some of the ideas of the
previous sections when interpreted as answers to questions of the form
`can all the measures in such-and-such a family be simultaneously
extended to a single measure?'   If we seek only a {\it finitely}
additive common extension, there is a reasonably convincing general
result (457A);  but countably additive measures remain puzzling even in
apparently simple circumstances (457Z).
In \S458 I introduce `relatively independent' families of
$\sigma$-algebras, with the associated concept of `relative product' of
measures.   Finally, in \S459, I give some basic results on
symmetric measures and exchangeable random variables, with De Finetti's
theorem (459C) and corresponding theorems on representing
permutation-invariant measures on products as mixtures of product
measures (459E, 459H).

\discrpage

