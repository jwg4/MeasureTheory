\frfilename{mt555.tex}
\versiondate{23.10.14}
\copyrightdate{2005}

\def\chaptername{Possible worlds}
\def\sectionname{Solovay's construction of \rvm\ cardinals}

\Loadfourteens

\def\BbbPk{\Bbb P_{\kappa}}
\def\BbbQk{\Bbb Q_{\kappa}}
\long\def\doubleinset#1{\inset{\inset{\parindent=-20pt #1}}}
\def\Etwovmc{$\exists$2vmc}
\def\Eqmc{$\exists$qmc}
\def\Eamc{$\exists$amc}
\def\Fn{\mathop{\text{Fn}}\nolimits}
\def\pssqa{power set $\sigma$-quotient algebra}
\def\VVdash{\mskip5mu\vrule height 7.5pt depth 2.5pt width 0.5pt
  \mskip2.5mu\vrule height 7.5pt depth 2.5pt width 0.5pt
  \vrule height 2.75pt depth -2.25pt width 4pt\mskip2mu}
\def\VVdPk{\VVdash_{\Bbb P_{\kappa}}}
\def\VVdQk{\VVdash_{\Bbb Q_{\kappa}}}
\def\VVdP{\VVdash_{\Bbb P}}

\newsection{555}

While all the mathematical ideas of Chapter 54 were
expressed as arguments in ZFC, many would be of little interest
if it appeared that there could be no \am\ cardinals.   In this section I
present R.M.Solovay's theorem that if there is a \2vm\ cardinal in the
original universe, then there is a forcing notion $\Bbb P$ such that

\Centerline{$\VVdP$ there is an \am\ cardinal}

\noindent(555D).   Varying $\Bbb P$ we find that we can force models with
other kinds of \qm\ cardinal (555G, 555K);  starting from a stronger
hypothesis we can reach the normal measure axiom (555N).

\leader{555A}{Notation}\cmmnt{ As in \S\S552-553,} I will write
$(\frak B_{\kappa},\bar\nu_{\kappa})$ for the measure algebra of the usual
measure on $\{0,1\}^{\kappa}$, and $\BbbPk$ for the forcing notion
$\frak B_{\kappa}^+=\frak B_{\kappa}\setminus\{0\}$, active downwards.
\cmmnt{In this context, as in 525A,}
$\ofamily{\eta}{\kappa}{e_{\eta}}$ will be the standard
generating family in $\frak B_{\kappa}$.

\cmmnt{As in \S554, }I will write $\frak G_{\kappa}$ for the category
algebra of $\{0,1\}^{\kappa}$, and $\BbbQk$ for the forcing notion
$\frak G_{\kappa}^+$, active downwards.
\cmmnt{Recall that }$\frak G_{\kappa}$ is isomorphic to the regular
open algebra $\RO(\{0,1\}^{\kappa})$\cmmnt{ (514If)}.

\leader{555B}{Theorem} Suppose that $X$ is a set,
and $\Cal I$ a proper $\sigma$-ideal of subsets of
$X$ containing singletons.
Let $\Bbb P=(P,\le,\Bbbone,\updownarrows)$ be a ccc forcing notion, and
$\dot{\Cal I}$ a $\Bbb P$-name such that

\Centerline{$\VVdP\,\dot{\Cal I}
=\{J:$ there is an $I\in\check{\Cal I}$ such that $J\subseteq I\}$.}

\noindent Then

(a)(i) If $\dot J$ is a $\Bbb P$-name and $p\in P$ is such that
$p\VVdP\,\dot J\in\dot{\Cal I}$, there is an $I\in\Cal I$ such that
$p\VVdP\,\dot J\subseteq\check I$.

\quad(ii)

\doubleinset{$\VVdP\,\dot{\Cal I}$ is the ideal of subsets of $\check X$
generated by $\check{\Cal I}$;  it is a proper
$\sigma$-ideal containing singletons.}

(b) $\VVdP\,\add\dot{\Cal I}=(\add\Cal I)\var2spcheck$.

(c) If $\Cal I$ is $\omega_1$-saturated in $\Cal PX$, then

\Centerline{$\VVdP\,\dot{\Cal I}$ is $\omega_1$-saturated in
$\Cal P\check X$, so $\Cal P\check X/\dot{\Cal I}$ is ccc and
Dedekind complete.}

(d) If $X=\lambda$ is a regular uncountable
cardinal and $\Cal I$ is a normal ideal on $\lambda$, then

\Centerline{$\VVdP\,\dot{\Cal I}$ is a normal ideal on $\check\lambda$.}

\proof{{\bf (a)(i)} We have

\Centerline{$p\VVdP$ there is an $I\in\check{\Cal I}$ such that
$\dot J\subseteq I$.}

\noindent Set

\Centerline{$A=\{q:$ there is an $I\in\Cal I$ such that
$q\VVdP\,\dot J\subseteq\check I\}$.}

\noindent If $p'$ is stronger than $p$, there is a $q\in A$ stronger than
$p'$.   Let $A'\subseteq A$ be a maximal
antichain.   Then $A'$ is countable and for each $q\in A'$ there is an
$I_q\in\Cal I$ such that $q\VVdP\,\dot J\subseteq\check I_q$.   Set
$I=\bigcup_{q\in A}I_q$;  because $\Cal I$ is a $\sigma$-ideal,
$I\in\Cal I$.   Now $q\VVdP\,\dot J\subseteq\check I$ for every
$q\in A'$.   If $p'$ is stronger than $p$ there is a $q\in A'$ which is
compatible with $p'$,
so $p\VVdP\,\dot J\subseteq\check I$, as required.

\medskip

\quad{\bf (ii)} Because

\Centerline{$\VVdP\,\check{\Cal I}$ is a family of subsets of $\check X$
closed under finite unions,}

\noindent we have

\Centerline{$\VVdP\,\dot{\Cal I}$ is an ideal of subsets of $\check X$.}

\noindent Because

\Centerline{$\VVdP\,\check I\in\dot{\Cal I}$}

\noindent whenever $I\in\Cal I$,

\Centerline{$\VVdP\,\check{\Cal I}\subseteq\dot{\Cal I}$ and
$\dot{\Cal I}$ is the ideal generated by $\check{\Cal I}$.}

Since $X\notin\Cal I$, (i) tells us that

\Centerline{$\VVdP\,\check X\notin\dot{\Cal I}$.}

\noindent Since $\{x\}\in\Cal I$ for every $x\in X$,

\Centerline{$\VVdP\,\{x\}\in\dot{\Cal I}$ for every $x\in\check X$.}

Thus

\Centerline{$\VVdP\,\dot{\Cal I}$ is a proper ideal of $\Cal P\check X$
containing singletons.}

\noindent I defer the final step to (b-i) below.

\medskip

{\bf (b)} Set $\theta=\add\Cal I$.

\medskip

\quad{\bf (i)} Suppose that $p\in P$ and that $\dot{\Cal A}$ is a
$\Bbb P$-name such that

\Centerline{$p\VVdP\,\dot{\Cal A}\subseteq\dot{\Cal I}$ and
$\#(\dot{\Cal A})<\check\theta$.}

\noindent Then there are a $q$ stronger than $p$, a $\delta<\theta$ and a
family $\ofamily{\xi}{\delta}{\dot A_{\xi}}$ of $\Bbb P$-names such that

\Centerline{$q\VVdP\,\dot{\Cal A}=\{\dot A_{\xi}:\xi<\check\delta\}$.}

\noindent For each $\xi<\delta$, $q\VVdP\,\dot A_{\xi}\in\dot{\Cal I}$,
so we have an $I_{\xi}\in\Cal I$ such that
$q\VVdP\,\dot A_{\xi}\subseteq\check I_{\xi}$.   Set
$I=\bigcup_{\xi<\delta}I_{\xi}\in\Cal I$.   Then

\Centerline{$q\VVdP\,\dot A_{\xi}\subseteq\check I$ for every
$\xi<\check\delta$, so $\bigcup\dot{\Cal A}\subseteq\check I$ and
$\bigcup\dot{\Cal A}\in\dot{\Cal I}$.}

\noindent As $p$ and $\dot{\Cal A}$ are arbitrary,

\Centerline{$\VVdP\,\add\dot{\Cal I}\ge\check\theta$.}

In particular, since we certainly have
$\theta\ge\omega_1$,

\Centerline{$\VVdP\,\dot{\Cal I}$ is a $\sigma$-ideal.}

\medskip

\quad{\bf (ii)} In the other direction, there is a family
$\ofamily{\xi}{\theta}{I_{\xi}}$ in $\Cal I$ with no upper bound in
$\Cal I$.    Now $\VVdP\,\check I_{\xi}\in\dot\Cal I$ for every
$\xi<\theta$.   \Quer\ If $p\in P$ is such that

\Centerline{$p\VVdP\,\bigcup_{\xi<\theta}\check I_{\xi}\in\dot\Cal I$,}

\noindent then there is an $I\in\Cal I$ such that

\Centerline{$p\VVdP\,\bigcup_{\xi<\check\theta}\check I_{\xi}
  \subseteq\check I$}

\noindent and $\bigcup_{\xi<\theta}I_{\xi}\subseteq I\in\Cal I$.\
\BanG\  So

\Centerline{$\VVdP\,
\bigcup_{\xi<\check\theta}\check I_{\xi}\notin\dot\Cal I$
and $\add\dot\Cal I\le\check\theta$.}

\medskip

{\bf (c)} Let $p\in P$ and a family
$\ofamily{\eta}{\omega_1}{\dot A_{\eta}}$ of $\Bbb P$-names be such
that

\Centerline{$p\VVdP\,\ofamily{\eta}{\omega_1}{\dot A_{\eta}}$ is a
disjoint family of subsets of $\check X$.}

\noindent For each $x\in X$,
$\ofamily{\eta}{\omega_1}{\widehat{p}
\Bcap\Bvalue{\check x\in\dot A_{\eta}}}$
is a disjoint family in $\RO(\Bbb P)$, where

\Centerline{$\widehat{p}
=\interior\overline{\{q:q\text{ is stronger than }p\}}$}

\noindent is the regular open set corresponding to $p$.   So there is an
$\alpha_x<\omega_1$ such that
$\widehat{p}\Bcap\Bvalue{\check x\in\dot A_{\eta}}=0$ for every
$\eta\ge\alpha_x$, that is, $p\VVdP\,\check x\notin\dot A_{\eta}$ for every
$\eta\ge\alpha_x$.    Because $\Cal I$ is $\omega_1$-saturated, therefore
$\omega_2$-additive (542B-542C), there is an $\alpha<\omega_1$ such that
$I=\{x:x\in X$, $\alpha_x\ge\alpha\}$ belongs to $\Cal I$.
Now $p\VVdP\,\check x\notin\dot A_{\alpha}$
for every $x\in X\setminus I$, that is,

\Centerline{$p\VVdP\,\dot A_{\alpha}\subseteq\check I$ and
$\dot A_{\alpha}\in\dot{\Cal I}$.}

\noindent As $p$ and $\ofamily{\eta}{\omega_1}{\dot A_{\eta}}$ are
arbitrary, $\VVdP\,\dot{\Cal I}$ is $\omega_1$-saturated.

Thus $\VVdP\,\Cal P\check X/\dot{\Cal I}$ is ccc.   But since we know from
(b) that $\VVdP\,\dot{\Cal I}$ is a $\sigma$-ideal, and of course
$\VVdP\,\Cal P\check X$ is Dedekind complete, we have

\Centerline{$\VVdP\,\Cal P\check X/\dot{\Cal I}$ is Dedekind
$\sigma$-complete, therefore Dedekind complete.}

\medskip

{\bf (d)} Suppose that $p\in P$ and that
$\ofamily{\xi}{\lambda}{\dot A_{\xi}}$ is a $\Bbb P$-name such that

\Centerline{$p\VVdP\,\dot A_{\xi}\in\dot{\Cal I}$ for every
$\xi<\check\lambda$.}

\noindent For each $\xi<\lambda$ we have an $I_{\xi}\in\Cal I$ such that
$p\VVdP\,\dot A_{\xi}\subseteq\check I_{\xi}$;  let $I$ be the diagonal
union

\Centerline{$\{\xi:\xi<\lambda$, $\xi\in\bigcup_{\eta<\xi}I_{\eta}\}$.}

\noindent Because $\Cal I$ is a normal ideal on $\lambda$, $I\in\Cal I$.
Now suppose that $q$ is stronger than $p$ and that $\xi<\lambda$ is such
that

\Centerline{$q\VVdP\,\check\xi\in\bigcup_{\eta<\check\xi}\dot A_{\eta}$.}

\noindent Then

\Centerline{$q\VVdP\,\check\xi\in\bigcup_{\eta<\check\xi}\check I_{\eta}$,}

\noindent so $\xi\in\bigcup_{\eta<\xi}I_{\eta}$ and $\xi\in I$ and
$q\VVdP\,\check\xi\in\check I$.   As $q$ and $\xi$ are arbitrary,

\Centerline{$p\VVdP$ the diagonal union of
$\ofamily{\xi}{\check\lambda}{\dot A_{\xi}}$ is included in $\check I$ and
belongs to $\dot{\Cal I}$.}

\noindent As $p$ and $\ofamily{\xi}{\lambda }{\dot A_{\xi}}$ are arbitrary,

\Centerline{$\VVdP\,\dot{\Cal I}$ is normal.}
}%end of proof of 555B

\leader{555C}{Theorem} Let $(X,\Cal PX,\mu)$ be a probability space such
that $\mu\{x\}=0$ for every $x\in X$, and $\Cal N$ the null ideal of
$\mu$.   Let $\kappa>0$ be a cardinal.   Then we can find a $\BbbPk$-name
$\dot\mu$ such that

(i) $\VVdPk\,\dot\mu$ is a probability
measure with domain $\Cal P\check X$, zero on singletons;

(ii) if $\dot{\Cal N}$ is a $\BbbPk$-name for the ideal of subsets
of $\check X$ generated by $\check{\Cal N}$\cmmnt{, as in 555B}, then

\Centerline{$\VVdPk\,\dot{\Cal N}$ is the null ideal of $\dot\mu$.}

\proof{{\bf (a)} For each function $\sigma:X\to\frak B_{\kappa}$,
write $\vec\sigma$ for the $\BbbPk$-name

\Centerline{$\{(\check x,\sigma(x)):x\in X$, $\sigma(x)\ne 0\}$.}

\noindent Then

\Centerline{$\VVdPk\,\vec\sigma\subseteq\check X$}

\noindent and $\Bvalue{\check x\in\vec\sigma}=\sigma(x)$ for every
$x\in X$.   Moreover, if $\dot A$ is any $\BbbPk$-name such that
$\VVdPk\,\dot A\subseteq\check X$, then $\VVdPk\,\dot A=\vec\sigma$, where
$\sigma(x)=\Bvalue{\check x\in\dot A}$ for $x\in X$.

\medskip

{\bf (b)} For $\sigma\in\frak B_{\kappa}^X$, the functional

\Centerline{$a\mapsto
\int\bar\nu_{\kappa}(\sigma(x)\Bcap a)\,\mu(dx)$}

\noindent is additive and dominated by $\bar\nu_{\kappa}$, so
there is a unique
$u_{\sigma}\in L^{\infty}(\frak B_{\kappa})$ such that

\Centerline{$\int_au_{\sigma}\,d\bar\nu_{\kappa}
=\int\bar\nu_{\kappa}(\sigma(x)\Bcap a)\mu(dx)$}

\noindent for every $a\in\frak B_{\kappa}$
(365E, 365D(d-ii)), and $0\le u_{\sigma}\le\chi 1$.
Observe that if $\sigma$, $\tau\in\frak B_{\kappa}^X$,
$a\in\frak B_{\kappa}^+$ and $a\VVdPk\vec\sigma=\vec\tau$, then
$a\Bcap\sigma(x)=a\Bcap\tau(x)$ for every $x\in X$, so that
$u_{\sigma}\times\chi a=u_{\tau}\times\chi a$.

\medskip

{\bf (c)} For $u\in L^{\infty}(\frak B_{\kappa})$
let $\vec u$ be the corresponding
$\BbbPk$-name for a real number (5A3L, identifying $\frak B_{\kappa}$ with
$\RO(\BbbPk)$ as usual).   Consider the $\BbbPk$-name

\Centerline{$\dot\mu=\{((\vec\sigma,\vec u_{\sigma}),\Bbbone):
\sigma\in\frak B_{\kappa}^X\}$.}

\noindent Then

\Centerline{$\VVdPk\,\dot\mu$ is a function from
$\Cal P\check X$ to $[0,1]$.}

\noindent\Prf\ Because $0\le u_{\sigma}\le\chi 1$,
$\VVdPk\,\vec u_{\sigma}\in[0,1]$ for each $\sigma$, and
$\VVdPk\,\dot\mu\subseteq\Cal P\check X\times[0,1]$.
If $\sigma$, $\tau\in\frak B_{\kappa}^X$ and $a\in\frak B_{\kappa}^+$
are such that $a\VVdPk\,\vec\sigma=\vec\tau$, then
$u_{\sigma}\times\chi a=u_{\tau}\times\chi a$, by (b);
but this means that
$a\VVdPk\,\vec u_{\sigma}=\vec u_{\tau}$ (5A3M).   So
$\VVdPk\,\dot\mu$ is a function (5A3Ha).
If $a\in P$ and
$\dot A$ are such that $a\VVdPk\,\dot A\subseteq\check X$, then there is a
$\sigma\in\frak B_{\kappa}^X$ such that
$a\VVdPk\,\dot A=\vec\sigma$, so that
$\VVdPk\,\dom\dot\mu=\Cal P\check X$ (5A3Hb).\ \Qed

\medskip

{\bf (d)} Now we have to check the properties of $\dot\mu$.

\medskip

\quad{\bf (i)} $\VVdPk\,\dot\mu$ is countably additive.
\Prf\ Suppose that $\sequencen{\dot A_n}$ is a sequence of
$\BbbPk$-names and $a\in\frak B_{\kappa}^+$ is such that

\Centerline{$a\VVdPk\,\sequencen{\dot A_n}$ is a disjoint sequence of
subsets of $\check X$.}

\noindent For each $n\in\Bbb N$, let
$\sigma_n\in\frak B_{\kappa}^X$ be such that
$a\VVdPk\,\dot A_n=\vec\sigma_n$;  then $\sequencen{a\Bcap\sigma_n(x)}$
is a disjoint sequence in $\frak B_{\kappa}$ for each $x\in X$.   Set
$\sigma(x)=\sup_{n\in\Bbb N}\sigma_n(x)$ for each $x$.   Then
$\Bvalue{\check x\in\vec\sigma}
=\sup_{n\in\Bbb N}\Bvalue{\check x\in\vec\sigma_n}$
for each $x$, so $\VVdPk\,\vec\sigma=\bigcup_{n\in\Bbb N}\vec\sigma_n$.
Now for any $b\Bsubseteq a$,

$$\eqalign{\int_bu_{\sigma}d\bar\nu_{\kappa}
&=\int\bar\nu_{\kappa}(b\Bcap\sigma(x))\mu(dx)
=\int\sum_{n=0}^{\infty}\bar\nu_{\kappa}(b\Bcap\sigma_n(x))\mu(dx)\cr
&=\sum_{n=0}^{\infty}\int\bar\nu_{\kappa}(b\Bcap\sigma_n(x))\mu(dx)
=\sum_{n=0}^{\infty}\int_bu_{\sigma_n}d\bar\nu_{\kappa}.\cr}$$

\noindent So

\Centerline{$\chi a\times u_{\sigma}
=\sup_{n\in\Bbb N}\chi a\times\sum_{i=0}^nu_{\sigma_i}$}

\noindent in $L^0(\frak B_{\kappa})$, and

\Centerline{$a\VVdPk\,\dot\mu(\bigcup_{n\in\Bbb N}\dot A_n)
=\dot\mu(\bigcup_{n\in\Bbb N}\vec\sigma_n)
=\dot\mu(\vec\sigma)
=\vec u_{\sigma}
=\sum_{n=0}^{\infty}\vec u_{\sigma_n}
=\sum_{n=0}^{\infty}\dot\mu(\dot A_n)$.}

\noindent As $a$ and $\sequencen{\dot A_n}$ are arbitrary,
$\VVdPk\,\dot\mu$ is countably additive.\ \Qed

\medskip

\quad{\bf (ii)} Suppose that
$\dot y$ is a $\BbbPk$-name and $a\in\frak B_{\kappa}^+$ is such
that $a\VVdPk\,\dot y\in\check X$.   Take any $b$ stronger than
$a$ and $y\in X$ such that $b\VVdPk\,\dot y=\check y$.
Set $\sigma(y)=1$ and
$\sigma(x)=0$ for $x\in X\setminus\{y\}$.   Then

\Centerline{$\int u_{\sigma}d\bar\nu_{\kappa}=\mu\{y\}=0$,}

\noindent so

\Centerline{$b\VVdPk\,\dot\mu(\{\dot y\})
=\dot\mu(\{\check y\})=\dot\mu(\vec\sigma)
=\vec u_{\sigma}=0$.}

\noindent Thus

\Centerline{$\VVdPk\,\dot\mu$ is zero on singletons.}

\medskip

\quad{\bf (iii)} If $\sigma(x)=1$ for every $x\in X$, then
$u_{\sigma}=\chi 1$ so

\Centerline{$\VVdPk\dot\mu(\check X)=\vec{\chi 1}=1$.}

\medskip

{\bf (e)} $\VVdPk\,\dot{\Cal N}=\{A:\dot\mu A=0\}$.   \Prf\ Let
$a\in\frak B_{\kappa}^+$ and a $\BbbPk$-name $\dot A$ be such that
$a\VVdPk\,\dot A\subseteq\check X$.   Let
$\sigma\in\frak B_{\kappa}^X$ be such that
$a\VVdPk\,\dot A=\vec\sigma$;  then
$a\VVdPk\,\dot\mu(\dot A)=\vec u_{\sigma}$.

\medskip

\quad{\bf (i)} If $a\VVdPk\,\dot A\in\dot\Cal N$, then there is an
$I\in\Cal N$ such that
$a\VVdPk\,\vec\sigma\subseteq\check I$ (555B(a-i)),
that is, $a\Bcap\sigma(x)=0$ for
every $x\in X\setminus I$.   But this means that

\Centerline{$\int_au_{\sigma}d\bar\nu_{\kappa}
=\int\bar\nu_{\kappa}(\sigma(x)\Bcap a)\mu(dx)
\le\mu I=0$,}

\noindent and $\chi a\times u_{\sigma}=0$.   So

\Centerline{$a\VVdPk\,\dot\mu(\dot A)=\vec u_{\sigma}=0$.}

\medskip

\quad{\bf (ii)} If $a\VVdPk\,\dot\mu(\dot A)=0$, then
$\chi a\times u_{\sigma}=0$, so

\Centerline{$\int\bar\nu_{\kappa}(\sigma(x)\Bcap a)\mu(dx)
=\int_au_{\sigma}d\bar\nu_{\kappa}=0$.}

\noindent Set $I=\{x:a\Bcap\sigma(x)\ne 0\}$;  then $\mu I=0$ and

\Centerline{$a\VVdPk\,\dot A\subseteq\check I$ so $\dot A\in\dot{\Cal N}$.}

\noindent Putting these together we have what we need.\ \Qed
}%end of proof of 555C

\leader{555D}{Corollary}\cmmnt{ ({\smc Solovay 71})}
Suppose that $\lambda$ is a \2vm\ cardinal and
that $\kappa\ge\lambda$ is a cardinal.   Then

\Centerline{$\VVdPk\,\check\lambda$ is \am.}

\proof{ Putting 555C and 555B together,

\doubleinset{$\VVdPk$ there is a probability
measure $\mu$ with domain $\Cal P\check\lambda$, zero on singletons,
such that the null ideal of $\mu$ is $\check\lambda$-additive.}

\noindent By 552B and 543Bc,

\Centerline{$\VVdPk\,\check\lambda\le\frak c$ is a \rvm\ cardinal, so is
\am.}
}%end of proof of 555D

\leader{555E}{Theorem} Let $\lambda$ be a \2vm\ cardinal, and $\Cal I$
a $\lambda$-additive maximal proper ideal of $\Cal P\lambda$ containing
singletons;  let $\mu$ be the $\{0,1\}$-valued probability measure on
$\lambda$ with null ideal $\Cal I$.
Let $\kappa\ge\lambda$ be a cardinal, and define $\dot\mu$ from $\mu$ as
in\cmmnt{ Theorem} 555C.   Set
$\theta=\Tr_{\Cal I}(\lambda;\kappa)$\cmmnt{ (definition:  5A1La)}.
Then

\Centerline{$\VVdPk\,\dot\mu$ is \Mth\ with Maharam type
$\check\theta$.}

\proof{{\bf (a)}
Let $\ofamily{\alpha}{\theta}{g_{\alpha}}$ be a family in
$\kappa^{\lambda}$ such that
$\{\xi:g_{\alpha}(\xi)=g_{\beta}(\xi)\}\in\Cal I$ whenever
$\alpha<\beta<\theta$ (541F).   Because $\lambda\le\kappa$, we can suppose
that all the $g_{\alpha}$ are injective.   (Just arrange that
$g_{\alpha}(\xi)$ always belongs to some $J_{\xi}\in[\kappa]^{\kappa}$
where $\ofamily{\xi}{\lambda}{J_{\xi}}$ is disjoint.)
For $\alpha<\theta$ and $\xi<\lambda$ set
$\sigma_{\alpha}(\xi)=e_{g_{\alpha}(\xi)}$.   For
$\sigma\in\frak B_{\kappa}^{\lambda}$ let $\vec\sigma$ be the corresponding
$\BbbPk$-name for a subset of $\lambda$ as in the prood of 555C.
Then for any non-empty finite $K\subseteq\theta$ we have

\Centerline{$\VVdPk\,
\dot\mu(\bigcap_{\alpha\in\check K}\vec\sigma_{\alpha})
=(2^{-\#(K)})\var2spcheck$.}

\noindent\Prf\ Set $\sigma(\xi)=\inf_{\alpha\in K}\sigma_{\alpha}(\xi)$ for
each $\xi$, so that

\Centerline{$\VVdPk\,
\vec\sigma=\bigcap_{\alpha\in\check K}\vec\sigma_{\alpha}$.}

\noindent Set

\Centerline{$I
=\bigcup_{\alpha,\beta\in K\text{ are different}}
  \{\xi:\xi<\lambda,\,g_{\alpha}(\xi)=g_{\beta}(\xi)\}$;}

\noindent then $I\in\Cal I$.
If $a\in\frak B_{\kappa}$, let $J\in[\kappa]^{\le\omega}$
be such that $a$ belongs to the closed subalgebra of $\frak B_{\kappa}$
generated by $\{e_{\eta}:\eta\in J\}$.   Then

\Centerline{$\{\xi:\xi<\lambda,\,\bar\nu_{\kappa}(\sigma(\xi)\Bcap a)
   \ne 2^{-\#(K)}\bar\nu_{\kappa}(a)\}
\subseteq I\cup\bigcup_{\alpha\in K}g_{\alpha}^{-1}[J]
\in\Cal I$,}

\noindent so

\Centerline{$\int\bar\nu_{\kappa}(\sigma(\xi)\Bcap a)\mu(d\xi)
=2^{-\#(K)}\bar\nu_{\kappa}(a)$.}

\noindent This means that $u_{\sigma}$, as defined in the proof of 555C, is
just $2^{-\#(K)}\chi 1$ and
$\VVdPk\,\vec u_{\sigma}=(2^{-\#(K)})\var2spcheck$, that is,

\Centerline{$\VVdPk\,
\dot\mu(\bigcap_{\alpha\in\check K}\vec\sigma_{\alpha})
=(2^{-\#(K)})\var2spcheck$.  \Qed}

Thus

\doubleinset{$\VVdPk\,\ofamily{\alpha}{\check\theta}{\vec\sigma_{\alpha}}$
is a
stochastically independent family in $\Cal P\lambda$ of elements of measure
$\bover12$, and every principal ideal of the measure algebra of $\dot\mu$
has Maharam type at least $\check\theta$}

\noindent (331Ja).

\medskip

{\bf (b)} In the other direction, suppose that $a\in\frak B_{\kappa}^+$,
$\delta$ is a cardinal, $t>0$ is a rational number
and $\ofamily{\alpha}{\delta}{\dot A_{\alpha}}$ is a family of
$\BbbPk$-names such that

\Centerline{$a\VVdPk\,\dot A_{\alpha}\subseteq\check\lambda$,
$\dot\mu(\dot A_{\alpha}\symmdiff\dot A_{\beta})\ge 3\check t$
whenever $\alpha<\beta<\check\delta$.}

\noindent For each $\alpha<\delta$ let
$\sigma_{\alpha}\in\frak B_{\kappa}^{\lambda}$ be such that
$a\VVdPk\,\vec\sigma_{\alpha}=\dot A_{\alpha}$;  then

\Centerline{$\int\bar\nu_{\kappa}
(a\Bcap(\sigma_{\alpha}(\xi)\Bsymmdiff\sigma_{\beta}(\xi)))\mu(d\xi)
\ge 3t\bar\nu_{\kappa}a$}

\noindent whenever $\alpha<\beta<\delta$.   Let
$D\subseteq\frak B_{\kappa}$ be a set of size $\kappa$ which is dense for
the measure-algebra topology (521E(a-ii)), and for $\alpha<\delta$,
$\xi<\kappa$ take $d_{\alpha}(\xi)\in D$ such that
$\bar\nu_{\kappa}(d_{\alpha}(\xi)\Bsymmdiff\sigma_{\alpha}(\xi))
\le t\bar\nu_{\kappa}(a)$.   Then

\Centerline{$\int\bar\nu_{\kappa}
  (d_{\alpha}(\xi)\Bsymmdiff d_{\beta}(\xi))\mu(d\xi)
>0$}

\noindent and $\{\xi:d_{\alpha}(\xi)\ne d_{\beta}(\xi)\}\notin\Cal I$
whenever $\alpha<\beta<\delta$;  as $\Cal I$ is a maximal ideal,
$\{\xi:d_{\alpha}(\xi)=d_{\beta}(\xi)\}\in\Cal I$
whenever $\alpha<\beta<\delta$, and $\ofamily{\alpha}{\delta}{d_{\alpha}}$
witnesses that

\Centerline{$\delta\le\Tr_{\Cal I}(\lambda;D)=\Tr_{\Cal I}(\lambda;\kappa)
=\theta$.}

\noindent As
$a$, $t$ and $\ofamily{\alpha}{\delta}{\dot A_{\alpha}}$ are arbitrary,

\Centerline{$\VVdPk$ the Maharam type of $\dot\mu$ is at most
$\check\theta$;}

\noindent with (a), this means that

\Centerline{$\VVdPk\,\dot\mu$ is \Mth\ with Maharam type
$\check\theta$.}
}%end of proof of 555E

\leader{555F}{Proposition} Let $\lambda$ be a \2vm\ cardinal
and $\kappa>0$.   Let $\mu$ be a normal
witnessing probability on $\lambda$ and
$\dot\mu$ the corresponding $\BbbPk$-name for a measure on
$\check\lambda$, as in 555C.   Then

\doubleinset{
$\VVdPk$ the covering number of the null ideal of the product measure
$\dot\mu^{\Bbb N}$ on $\check\lambda^{\Bbb N}$ is $\check\lambda$.}

\proof{{\bf (a)} It may save a moment's thought later on if I remark now
that if $(Y,\Tau,\nu)$ is any measure space and $W\subseteq Y^{\Bbb N}$ is
negligible for the product measure $\nu^{\Bbb N}$, then there is a family
$\langle F_{ij}\rangle_{j\le i\in\Bbb N}$ in $\Tau$ such that

\Centerline{$W
\subseteq\bigcap_{n\in\Bbb N}\bigcup_{i\ge n}
  \{y:y\in Y^{\Bbb N}$, $y(j)\in F_{ij}$ for every $j\le i\}$,
\quad$\sum_{i=0}^{\infty}\prod_{j=0}^i\nu F_{ij}\le 1$.}

\noindent\Prf\ For each $k\in\Bbb N$, let $\sequence{i}{C_{ki}}$ be a
sequence of measurable cylinders such that
$W\subseteq\bigcup_{i\in\Bbb N}C_{ki}$ and
$\sum_{i=0}^{\infty}\lambda^{\Bbb N}C_{ki}\le 2^{-k-1}$;  let
$\sequence{i}{C_i}$ be a re-listing of the double family
$\langle C_{ki}\rangle_{k,i\in\Bbb N}$ with enough copies of the empty set
interleaved to ensure that $C_i$ is determined by coordinates less than or
equal to $i$ for each $i$;  express each $C_i$ as
$\{y:y(j)\in F_{ij}$ for $j\le i\}$.\ \Qed

\medskip

{\bf (b)} It will also help to be able to do some calculations with
sequences of $\BbbPk$-names for subsets of $\lambda$.
For $J\subseteq\kappa$ let $\frak C_J$ be the closed subalgebra
of $\frak B_{\kappa}$ generated by $\{e_{\eta}:\eta\in J\}$, and
$P_J:L^{\infty}(\frak B_{\kappa})\to L^{\infty}(\frak C_J)$ the
corresponding conditional
expectation operator;  see 242J, 254R and 365Q\formerly{3{}65R}
for the basic manipulations of these
operators.   Let $\Cal F$ be the normal ultrafilter $\{F:\mu F=1\}$.

\medskip

\quad{\bf (i)}
Let $\langle\dot A_{ij}\rangle_{j\le i\in\Bbb N}$ be a family of
$\BbbPk$-names such that

\Centerline{$\VVdPk\,\dot A_{ij}\subseteq\check\lambda$}

\noindent whenever $j\le i\in\Bbb N$.
For $j\le i\in\Bbb N$ and $\xi<\lambda$ set
$\sigma_{ij}(\xi)=\Bvalue{\check\xi\in\dot A_{ij}}$.
Then there is a countable
set $I_{\xi}\subseteq\kappa$ such that
$\sigma_{ij}(\xi)\in\frak C_{I_{\xi}}$ whenever
$j\le i\in\Bbb N$.   By 541Rb, %this is where we need "normal ideal"
there are an $F_0\in\Cal F$ and a countable
set $I\subseteq\kappa$ such that $I_{\xi}\cap I_{\xi'}\subseteq I$ for all
distinct $\xi$, $\xi'\in F_0$.

\medskip

\quad{\bf (ii)} Set $u_{ij\xi}=P_I(\chi\sigma_{ij}(\xi))$ for
$\xi<\lambda$.   Because $\#(L^{\infty}(\frak C_I))\le\frak c<\lambda$ and
$\Cal F$ is a $\lambda$-complete ultrafilter, there are
$u_{ij}\in L^{\infty}(\frak C_I)$ such that
$\{\xi:u_{ij\xi}=u_{ij}\}$ belongs to
$\Cal F$ whenever $j\le i\in\Bbb N$.   Set

\Centerline{$F
=F_0\cap\{\xi:u_{ij\xi}=u_{ij}$ whenever $j\le i\in\Bbb N\}$,}

\noindent so that $F\in\Cal F$.

\medskip

\quad{\bf (iii)} We have

\Centerline{$\VVdPk\,\dot\mu\dot A_{ij}=\vec u_{ij}$}

\noindent whenever $j\le i\in\Bbb N$.   \Prf\ If $a\in\frak B_{\kappa}$,
there is a countable $J\subseteq\kappa$ such that $a$ belongs to the
closed subalgebra $\frak C_J$ generated by $\{e_{\eta}:\eta\in J\}$;  we
may suppose that $I\subseteq J$.   Now
$\family{\xi}{F}{I_{\xi}\setminus I}$ is disjoint, so
$\{\xi:\xi\in F$, $I_{\xi}\cap J\not\subseteq I\}$ is countable, and
$F'=\{\xi:\xi\in F$, $I_{\xi}\cap J\subseteq I\}$ belongs to $\Cal F$.
For $\xi\in F'$ we have

\Centerline{$P_J(\chi\sigma_{ij}(\xi))=P_JP_{I_{\xi}}(\chi\sigma_{ij}(\xi))
=P_{J\cap I_{\xi}}(\chi\sigma_{ij}(\xi))\in\frak C_I$}

\noindent so that

\Centerline{$P_J(\chi\sigma_{ij}(\xi))=P_IP_J(\chi\sigma_{ij}(\xi))
=P_I(\chi\sigma_{ij}(\xi))=u_{ij}$;}

\noindent consequently

\Centerline{$\int_au_{ij}d\bar\nu_{\kappa}
=\int_a\chi\sigma_{ij}(\xi)d\bar\nu_{\kappa}
=\bar\nu_{\kappa}(a\Bcap\sigma_{ij}(\xi))$.}

\noindent Because $F'$ is $\mu$-conegligible,

\Centerline{$\int_au_{ij}d\bar\nu_{\kappa}
=\int\bar\nu_{\kappa}(a\Bcap\sigma_{ij}(\xi))\mu(d\xi)$.}

\noindent As this is true for every $a\in\frak B_{\kappa}$, the
construction in 555C gives $\VVdPk\,\dot\mu\dot A_{ij}=\vec u_{ij}$.\ \Qed

\medskip

\quad{\bf (iv)} Note also that if $i\in\Bbb N$ and
$\xi_0,\ldots,\xi_j\in F$ are distinct, then

\Centerline{$\int\prod_{j=0}^iu_{ij}d\bar\nu_{\kappa}
=\bar\nu_{\kappa}(\inf_{j\le i}\sigma_{ij}(\xi_j))$.}

\noindent\Prf\ The algebras $\frak C_{I\cup\{\eta\}}$, for
$\eta\in\kappa\setminus I$, are relatively stochastically independent over
$\frak C_I$ in the sense of 458L;  by 458H/458Le, the algebras
$\frak C_{I\cup I_{\xi}}$, for $\xi\in F$, are relatively stochastically
independent over $\frak C_I$;  but, disentangling the definitions, this is
exactly what we need to know.\ \Qed

\medskip

{\bf (c)} We are now ready for the central idea of the proof.
Suppose that $\dot W$ is a $\BbbPk$-name such that

\Centerline{$\VVdPk\,\dot W\subseteq\check\lambda^{\Bbb N}$ and
$\dot\mu^{\Bbb N}\dot W=0$.}

\noindent Then there is an $F\in\Cal F$ such that

\Centerline{$\VVdPk\,\dot W$ is disjoint from
$(F^{\Bbb N}\setminus\Delta)\var2spcheck$}

\noindent where
$\Delta=\bigcup_{j<k\in\Bbb N}\{x:x\in\lambda^{\Bbb N}$, $x(j)=x(k)\}$.
\Prf\ By (a), we have a family
$\langle\dot A_{ij}\rangle_{j\le i\in\Bbb N}$ of $\Bbb P_{\kappa}$-names
such that

\Centerline{$\VVdPk\,\dot A_{ij}\subseteq\check\lambda$ whenever
$j\le i\in\Bbb N$,}

\Centerline{$\VVdPk\,\dot W\subseteq\bigcap_{n\in\Bbb N}\bigcup_{i\ge n}
  \{x:x\in\check\lambda^{\Bbb N}$,
  $x(j)\in\dot A_{ij}$ for every $j\le i\}$,}

\Centerline{$\VVdPk\,\sum_{i=0}^{\infty}\prod_{j=0}^i\dot\mu\dot A_{ij}
\le 1$.}

\noindent Take $\sigma_{ij}(\xi)$, $I\in[\kappa]^{\le\omega}$,
$u_{ij}\in\frak C_I$ and $F\in\Cal F$ as in (b).   Suppose that
$x\in F^{\Bbb N}$ and $x(j)\ne x(k)$ for distinct $j$, $k\in\Bbb N$.   Then

\Centerline{$\Bvalue{\check x\in\dot W}
\Bsubseteq\inf_{n\in\Bbb N}\sup_{i\ge n}\inf_{j\le i}
  \Bvalue{x(j)\var2spcheck\in\dot A_{ij}}
=\inf_{n\in\Bbb N}\sup_{i\ge n}\inf_{j\le i}\sigma_{ij}(x(j))$.}

\noindent So

\Centerline{$\bar\nu_{\kappa}\Bvalue{\check x\in\dot W}
\le\inf_{n\in\Bbb N}\sum_{i=n}^{\infty}
  \bar\nu_{\kappa}(\inf_{j\le i}\sigma_{ij}(x(j)))
=\inf_{n\in\Bbb N}\sum_{i=n}^{\infty}
  \int\prod_{j\le i}u_{ij}d\bar\nu_{\kappa}$}

\noindent by (b-iv).   On the other hand, setting
$v_i=\prod_{j\le i}u_{ij}$ for $i\in\Bbb N$
and $w=\sup_{m\in\Bbb N}\sum_{i=0}^mv_i$, we have

\Centerline{$\VVdPk\,\vec w=\sum_{i=0}^{\infty}\vec v_i
=\sum_{i=0}^{\infty}\prod_{j=0}^i\dot\mu\dot A_{ij}
\le 1$.}

\noindent So $w\le\chi 1$ and

\Centerline{$\sum_{i=0}^{\infty}\int\prod_{j\le i}u_{ij}d\bar\nu_{\kappa}
=\int w\,d\bar\nu_{\kappa}\le 1$.}

\noindent Putting these together, we see that
$\bar\nu_{\kappa}\Bvalue{\check x\in\dot W}=0$ and
$\VVdPk\,\check x\notin\dot W$.   As $x$ is arbitrary,

\Centerline{$\VVdPk\,\dot W$ is disjoint from
$(F^{\Bbb N}\setminus\Delta)\var2spcheck$. \Qed}

\medskip

{\bf (d)} We are nearly home.   Suppose that $a\in\frak B_{\kappa}^+$
and that $\dot{\Cal W}$ is a $\BbbPk$-name such that

\Centerline{$a\VVdPk\,\dot\Cal W$ is a family of negligible sets in
$\check\lambda^{\Bbb N}$ and $\#(\dot\Cal W)<\check\lambda$.}

\noindent Take any $b$ stronger than $a$, $\theta<\lambda$ and family
$\ofamily{\zeta}{\theta}{\dot W_{\zeta}}$ of $\BbbPk$-names such that

\Centerline{$b\VVdPk\,\dot\Cal W=\{\dot W_{\zeta}:\zeta<\check\theta\}$.}

\noindent For each $\zeta<\theta$ let $\dot W'_{\zeta}$ be a $\BbbPk$-name
such that

\Centerline{$\VVdPk\,\dot W'_{\zeta}\subseteq\check\lambda^{\Bbb N}$ is
negligible,
\quad$b\VVdPk\,\dot W'_{\zeta}=\dot W_{\zeta}$.}

\noindent By (c), we have an $F_{\zeta}\in\Cal F$ such that

\Centerline{$\VVdPk\,\dot W'_{\zeta}
\cap(F_{\zeta}^{\Bbb N}\setminus\Delta)\var2spcheck=\emptyset$.}

\noindent Because $\theta<\lambda$, $\bigcap_{\zeta<\theta}F_{\zeta}$
belongs to $\Cal F$ and is infinite, and there is an
$x\in\bigcap_{\zeta<\theta}F_{\zeta}^{\Bbb N}\setminus\Delta$.   But now

\Centerline{$\VVdPk\,
\check x\notin\bigcup_{\zeta<\check\theta}\dot W'_{\zeta}$}

\noindent and

\Centerline{$b\VVdPk\,\check x\notin\bigcup\dot\Cal W$, so $\dot\Cal W$
does not cover $\check\lambda^{\Bbb N}$.}

\noindent As $b$ is arbitrary,

\Centerline{$a\VVdPk\,\dot\Cal W$ does not cover $\check\lambda^{\Bbb N}$;}

\noindent as $a$ and $\dot\Cal W$ are arbitrary,

\Centerline{
$\VVdPk$ the covering number of the null ideal of the product measure
on $\check\lambda^{\Bbb N}$ is at least $\check\lambda$.}

\noindent The reverse inequality is trivial, since

\Centerline{$\VVdPk\,\dot\mu\{\xi\}=0$}

\noindent for every $\xi<\check\lambda$;  so the proposition is proved.
}%end of proof of 555F

\leader{555G}{Cohen \dvrocolon{forcing}}\cmmnt{ If
we allow ourselves
to start from a measurable cardinal, we can find forcing constructions for
a variety of \pssqa{}s besides the probability algebras provided by Theorem
555C.   In view of \S554, an obvious construction is the following.

\medskip

\noindent}{\bf Theorem} Let $\lambda$ be a \2vm\ cardinal and
$\kappa\ge\lambda$ a cardinal.   Let $\Cal I$ be a
$\lambda$-additive maximal proper ideal of subsets of $\lambda$, and
$\dot{\Cal I}$ a $\Bbb Q_{\kappa}$-name for the ideal of subsets of
$\check\lambda$ generated by $\check{\Cal I}$\cmmnt{, as in 555B}.   Set
$\theta=\Tr_{\Cal I}(\lambda;\kappa)$.   Then

\Centerline{$\VVdQk\,
\Cal P\lambda/\dot{\Cal I}\cong\frak G_{\check\theta}$.}

\proof{{\bf (a)} For $\eta<\kappa$ let $e_{\eta}\in\frak G_{\kappa}$
be the equivalence class of $\{x:x\in\{0,1\}^{\kappa}$, $x(\eta)=1\}$;
for $L\subseteq\kappa$ let $\frak C_L$ be the closed subalgebra of
$\frak G_{\kappa}$ generated by $\{e_{\eta}:\eta\in L\}$.
For $\sigma\in\frak G_{\kappa}^{\lambda}$ let
$\vec\sigma$ be the $\BbbPk$-name
$\{(\check\xi,\sigma(\xi)):\xi<\lambda$, $\sigma(\xi)\ne 0\}$, so that
$\VVdQk\,\vec\sigma\subseteq\check\lambda$, and
$\Bvalue{\check\xi\in\vec\sigma}=\sigma(\xi)$ for any $\xi<\lambda$.

Write
$\Cal F=\{\lambda\setminus I:I\in\Cal I\}=\Cal P\lambda\setminus\Cal I$,
so that $\Cal F$ is a $\lambda$-complete ultrafilter on $\lambda$.

\medskip

{\bf (b)} For $z\in\Fn_{<\omega}(\kappa;\{0,1\})$ set

\Centerline{$v_z=\{x:z\subseteq x\in\{0,1\}^{\kappa}\}^{\ssbullet}
\in\frak G_{\kappa}$.}

\noindent Then $\{v_z:z\in\Fn_{<\omega}(\kappa;\{0,1\})\}$ is order-dense
in $\frak G_{\kappa}$.   For $A\subseteq\lambda$ and
$\tau\in\Fn_{<\omega}(\kappa;\{0,1\})^A$, set
$\sigma_{\tau}(\xi)=v_{\tau(\xi)}$ if $\xi\in A$, $0$ if
$\xi\in\lambda\setminus A$.   Note that

\Centerline{$\VVdQk\,\vec{\sigma}_{\tau}\subseteq(\dom\tau)\var2spcheck$.}

\noindent Now if $a\in\frak G_{\kappa}^+$
and $\dot C$ is a $\BbbQk$-name such that
$a\VVdQk\,\dot C\subseteq\check\lambda$, there is a countable set
$T\subseteq\bigcup_{A\subseteq\lambda}\Fn_{<\omega}(\kappa;\{0,1\})^A$
such that $a\VVdQk\,\dot C=\bigcup_{\tau\in\check T}\vec\sigma_{\tau}$.
\Prf\ Set $D=\{\xi:\xi<\lambda$, $\VVdQk\,\check\xi\notin\dot C\}$.
For each $\xi\in\lambda\setminus D$,
choose a sequence $\sequence{n}{\tau_n(\xi)}$ in
$\Fn_{<\omega}(\kappa;\{0,1\})$ such that
$\Bvalue{\check\xi\in\dot C}=\sup_{n\in\Bbb N}v_{\tau_n(\xi)}$.
If $\xi\in D$, then $\xi\notin\dom\tau_n$ and
$\sigma_{\tau_n}(\xi)=0$ for every $n$;  accordingly

\Centerline{$\Bvalue{\check\xi\in\dot C}
=0=\sup_{n\in\Bbb N}\Bvalue{\check\xi\in\vec\sigma_{\tau_n}}$.}

\noindent While if $\xi\in\lambda\setminus D$,

\Centerline{$\Bvalue{\check\xi\in\dot C}
=\sup_{n\in\Bbb N}v_{\tau_n(\xi)}
=\sup_{n\in\Bbb N}\sigma_{\tau_n}(\xi)
=\sup_{n\in\Bbb N}\Bvalue{\check\xi\in\vec\sigma_{\tau_n}}$.}

\noindent So

\Centerline{$a\VVdQk\,\dot C
=\bigcup_{n\in\Bbb N}\vec\sigma_{\tau_n}$}

\noindent and we can set $T=\{\tau_n:n\in\Bbb N\}$.\ \Qed

\medskip

{\bf (c)} There is a family $G_0\subseteq\kappa^{\lambda}$, with cardinal
$\theta$, such that
$\{\xi:g(\xi)=g'(\xi)\}\in\Cal I$ whenever $g$, $g'\in G$ are distinct
(541F again),
and we can suppose that every member of $G_0$ is injective (see
part (a) of the proof of 555E).   Let $G\supseteq G_0$
be a maximal family such that
$\{\xi:g(\xi)=g'(\xi)\}$ belongs to $\Cal I$ whenever $g$, $g'\in G$ are
distinct.   Then $\#(G)\le\theta$, by the definition of
$\Tr_{\Cal I}(\lambda;\kappa)$, so in fact $\#(G)=\theta$.
Enumerate $G$ as $\ofamily{\alpha}{\theta}{g_{\alpha}}$, and for
$\alpha<\theta$, $\xi<\lambda$ set
$\rho_{\alpha}(\xi)=e_{g_{\alpha}(\xi)}\in\frak G_{\kappa}$.

\medskip

{\bf (d)} Suppose that $\alpha<\theta$ and
$a\in\frak G_{\kappa}^+$ are such that

\Centerline{$a\VVdQk\,\vec\rho_{\alpha}^{\mskip4mu\ssbullet}$
is neither $0$ nor $1$ in $\Cal P\check\lambda/\dot{\Cal I}$.}

\noindent Then $g_{\alpha}^{-1}[\{\eta\}]\notin\Cal F$ for
every $\eta<\kappa$.   \Prf\Quer\ Otherwise, take $b$ to be one of
$a\Bcap e_{\eta}$, $a\Bsetminus e_{\eta}$ and non-zero, and any
$\xi\in g_{\alpha}^{-1}[\{\eta\}]$.
If $b\Bsubseteq e_{\eta}$ then $b\Bsubseteq\rho_{\alpha}(\xi)$ and
$b\VVdQk\,\check\xi\in\vec\rho_{\alpha}$.   If
$b\Bcap e_{\eta}=0$ then $\rho_{\alpha}(\xi)\Bcap b=0$ and
$b\VVdQk\,\check\xi\notin\vec\rho_{\alpha}$.   So

\doubleinset{$b\VVdQk\,g_{\alpha}^{-1}[\{\eta\}]\var2spcheck$ is either
included in or disjoint
from $\vec\rho_{\alpha}$, and $\vec\rho_{\alpha}^{\mskip4mu\ssbullet}$
is either $1$ or $0$ in $\Cal P\check\lambda/\dot{\Cal I}$.}

\noindent But this contradicts the assumption on $a$.\ \Bang\Qed

\medskip

{\bf (e)} $\VVdQk\,
\{\vec\rho_{\alpha}^{\mskip4mu\ssbullet}:
\alpha<\check\theta\}\setminus\{0,1\}$
is a Boolean-independent family in $\Cal P\check\lambda/\dot{\Cal I}$.

\Prf\Quer\ Otherwise, there must be disjoint finite sets $J$,
$K\subseteq\theta$ and $a\in\frak G_{\kappa}^+$ such that

\Centerline{$a\VVdQk\,\vec\rho_{\alpha}^{\mskip4mu\ssbullet}$
is neither $0$ nor $1$ in $\Cal P\check\lambda/\dot{\Cal I}$}

\noindent for every $\alpha\in J\cup K$, but

\Centerline{$a\VVdQk\,
\inf_{\alpha\in\check J}\vec\rho_{\alpha}^{\mskip4mu\ssbullet}
\Bsetminus\sup_{\alpha\in\check K}\vec\rho_{\alpha}^{\mskip4mu\ssbullet}
=0$.}

\noindent In this case,

\Centerline{$I=\{\xi:\xi<\lambda$,
there are distinct $\alpha$, $\beta\in J\cup K$ such that
$g_{\alpha}(\xi)=g_{\beta}(\xi)\}$}

\noindent belongs to $\Cal I$.   For $\xi<\lambda$ define
$\sigma(\xi)\in\frak G_{\kappa}$ by saying that

$$\eqalign{\sigma(\xi)
&=1\text{ if }\xi\in I,\cr
&=\inf_{\alpha\in J}e_{g_{\alpha}(\xi)}
  \Bsetminus\sup_{\alpha\in K}e_{g_{\alpha}(\xi)}
\text{ otherwise}.\cr}$$

\noindent For $\alpha\in J$ and $\xi\in\lambda\setminus I$,
$\sigma(\xi)\Bsubseteq\rho_{\alpha}(\xi)$;  so

\Centerline{$\VVdQk\,\vec\sigma\setminus\vec\rho_{\alpha}$ is included in
$\check I$ and belongs to $\dot{\Cal I}$, that is,
$\vec\sigma^{\ssbullet}\Bsubseteq\vec\rho_{\alpha}^{\mskip4mu\ssbullet}$.}

\noindent On the other hand, if $\alpha\in K$ and
$\xi\in\lambda\setminus I$,
$\sigma(\xi)\Bcap\rho_{\alpha}(\xi)=0$;  so

\Centerline{$\VVdQk\,\vec\sigma\cap\vec\rho_{\alpha}$ is included in
$\check I$ and belongs to $\dot{\Cal I}$, that is,
$\vec\sigma^{\ssbullet}\Bcap\vec\rho_{\alpha}^{\mskip4mu\ssbullet}=0$.}

\noindent So we have

\Centerline{$\VVdQk\,\vec\sigma^{\ssbullet}
\Bsubseteq\inf_{\alpha\in\check J}\vec\rho_{\alpha}^{\mskip4mu\ssbullet}
\Bsetminus\sup_{\alpha\in\check K}\vec\rho_{\alpha}^{\mskip4mu\ssbullet}$}

\noindent and

\Centerline{$a\VVdQk\,\vec\sigma\in\dot{\Cal I}$.}

Let $I'\in\Cal I$ be such that $a\VVdQk\,\vec\sigma\subseteq\check I'$
(see part (a) of the proof of 555B), and $L\in[\kappa]^{\le\omega}$ such
that $a\in\frak C_L$.    By (d), $g_{\alpha}^{-1}[\{\eta\}]\in\Cal I$ for
every $\alpha\in J\cup K$ and
$\eta<\kappa$, so there must be a $\xi\in\lambda\setminus(I'\cup I)$
such that $g_{\alpha}(\xi)\notin L$ for every $\alpha\in J\cup K$.
In this case, $\sigma(\xi)\in\frak C_{\kappa\setminus L}$ so
$a\Bcap\sigma(\xi)$ is non-zero.   But
$\sigma(\xi)=\Bvalue{\check\xi\in\vec\sigma}$ and
$a\VVdQk\,\check\xi\notin\vec\sigma$, so this is impossible.\ \Bang\Qed

\medskip

{\bf (f)} $\VVdQk$ the subalgebra of $\Cal P\check\lambda/\dot{\Cal I}$
generated by
$\{\vec\rho_{\alpha}^{\mskip4mu\ssbullet}:\alpha<\check\theta\}$
is order-dense in $\Cal P\check\lambda/\dot{\Cal I}$.

\Prf\ If $a\in\frak G_{\kappa}^+$ and $\dot c$ is a $\Bbb Q_{\kappa}$-name
such that

\Centerline{$a\VVdQk\,
\dot c\in(\Cal P\check\lambda/\dot{\Cal I})\setminus\{0\}$,}

\noindent there is a $\BbbQk$-name $\dot C$ such that

\Centerline{$a\VVdQk\,\dot C\subseteq\check\lambda$,
$\dot C\notin\dot{\Cal I}$ and $\dot c=\dot C^{\ssbullet}$.}

\noindent Take a countable set
$T\subseteq\bigcup_{A\subseteq\lambda}\Fn_{<\omega}(\kappa;\{0,1\})^A$
such that $a\VVdQk\,\dot C=\bigcup_{\tau\in\check T}\vec\sigma_{\tau}$,
as described in (b).
Since $\VVdQk\,\dot{\Cal I}$ is a $\sigma$-ideal, there are a $b$ stronger
than $a$ and a $\tau\in T$ such that
$b\VVdQk\,\vec\sigma_{\tau}\notin\dot{\Cal I}$.
Set $F_0=\dom\tau$;  since $\VVdQk\,\vec\sigma_{\tau}\subseteq\check F_0$,
$F_0\notin\Cal I$..   Since $\Cal I$ is a
$\sigma$-ideal, there is an $n\in\Bbb N$ such that
$F_1=\{\xi:\xi\in F_0$, $\#(\tau(\xi))=n\}\notin\Cal I$, and
$F_1\in\Cal F$.

Let $\langle h_i\rangle_{i<n}$ be a finite
sequence of functions from $\lambda$ to
$\kappa$ such that $\dom\tau(\xi)=\{h_i(\xi):i<n\}$ for every
$\xi\in F_1$.   As $G$ was maximal, there is for
each $i<n$ an $\alpha_i<\theta$ such that
$\{\xi:g_{\alpha_i}(\xi)=h_i(\xi)\}$ belongs to $\Cal F$;  set
$F_2=\{\xi:\xi\in F_1$, $g_{\alpha_i}(\xi)=h_i(\xi)$ for every $i<n\}$.
Note that if $i<j<n$ then $g_{\alpha_i}(\xi)\ne g_{\alpha_j}(\xi)$
for any $\xi\in F_1$, so
$\alpha_i\ne\alpha_j$.   Next, there must be an $L\subseteq n$ such that

$$\eqalign{F_3
=\{\xi:\xi\in F_2,\,\,&\tau(\xi)(g_{\alpha_i}(\xi))=1
  \text{ for every }i\in L,\cr
&\tau(\xi)(g_{\alpha_i}(\xi))=0
  \text{ for every }i\in n\setminus L\}\cr}$$

\noindent belongs to $\Cal F$.   Set $J=\{\alpha_i:i\in L\}$ and
$K=\{\alpha_i:i\in n\setminus L\}$;  of course $J\cap K=\emptyset$, because
all the $\alpha_i$ are different.
Then, for $\xi\in F_3$, $\dom\tau_{\xi}=\{g_{\alpha_i}(\xi):i<n\}$ and

$$\eqalign{\Bvalue{\check\xi\in\vec\sigma_{\tau}}
&=\sigma_{\tau}(\xi)
=v_{\tau(\xi)}
=\inf_{i\in L}e_{g_{\alpha_i}(\xi)}
   \Bsetminus\sup_{i\in n\setminus L}e_{g_{\alpha_i}(\xi)}\cr
&=\inf_{\alpha\in J}e_{g_{\alpha}(\xi)}
   \Bsetminus\sup_{\alpha\in K}e_{g_{\alpha}(\xi)}
=\inf_{\alpha\in J}\Bvalue{\check\xi\in\vec\rho_{\alpha}}
  \Bsetminus\sup_{\alpha\in K}\Bvalue{\check\xi\in\vec\rho_{\alpha}}.\cr}$$

\noindent Accordingly

\Centerline{$\VVdQk\,\vec\sigma_{\tau}
\symmdiff(\bigcap_{\alpha\in\check J}\vec\rho_{\alpha}
  \setminus\bigcup_{\alpha\in\check K}\vec\rho_{\alpha})$
is disjoint from $\check F_3$ and belongs to $\dot{\Cal I}$,}

\noindent so

\Centerline{$b\VVdQk\,\inf_{\alpha\in\check J}
     \vec\rho_{\alpha}^{\mskip4mu\ssbullet}
  \Bsetminus\sup_{\alpha\in\check K}\vec\rho_{\alpha}^{\mskip4mu\ssbullet}
  =\vec\sigma_{\tau}^{\ssbullet}\Bsubseteq\dot c$.}

\noindent As $a$ and $\dot c$ are arbitrary, this proves the result.\ \Qed

\medskip

{\bf (g)} If $\alpha<\theta$ is such that $g_{\alpha}\in G_0$, then

\Centerline{$\VVdQk\,\vec\rho_{\alpha}^{\mskip4mu\ssbullet}\ne 0$.}

\noindent\Prf\Quer\ Otherwise, there is a non-zero
$a\in\frak G_{\kappa}$ such that

\Centerline{$a\VVdQk\,\vec\rho_{\alpha}^{\mskip4mu\ssbullet}=0$,}

\noindent that is,

\Centerline{$a\VVdQk\,\{\xi:\vec\rho_{\alpha}(\xi)\ne 0\}\in\dot{\Cal I}$,}

\noindent and there is an $I\in\Cal I$ such that

\Centerline{$a\VVdQk\,
\{\xi:\vec\rho_{\alpha}(\xi)\ne 0\}\subseteq\check I$.}

\noindent In this case, for $\xi\in\lambda\setminus I$,

\Centerline{$a\VVdPk\,\vec\rho_{\alpha}(\xi)=0$,}

\noindent that is,

\Centerline{$0=a\Bcap\rho_{\alpha}(\xi)=a\Bcap e_{g_{\alpha}(\xi)}$.}

\noindent But $\lambda\setminus I$ is infinite and $g_{\alpha}$ is
injective, so $\{g_{\alpha}(\xi):\xi\in\lambda\setminus I\}$ is infinite
and $a=0$.\ \Bang\Qed

Similarly,

\Centerline{$\VVdQk\,\vec\rho_{\alpha}^{\mskip4mu\ssbullet}\ne 1$.}

\noindent As this is true whenever $g_{\alpha}\in G_0$, and
$\#(G_0)=\theta$, we see that

\Centerline{$\VVdQk\,\#(\{\alpha:\vec\rho_{\alpha}^{\mskip4mu\ssbullet}
  \notin\{0,1\}\})=\check\theta$.}

\medskip

{\bf (h)} Putting (e)-(g) together, and using 555Bc, 515N and 514Ih,

\doubleinset{$\VVdQk\,\Cal P\check\lambda/\dot{\Cal I}$ has a
Boolean-independent family of size $\check\theta$
generating an order-dense subalgebra;  being Dedekind complete, it is
isomorphic to $\RO(\{0,1\}^{\check\theta})\cong\frak G_{\check\theta}$.}

}%end of proof of 555G

\leader{555H}{Corollary} Suppose that $\lambda$ is a \2vm\ cardinal and
$\kappa=2^{\lambda}$.   Then

\Centerline{$\VVdQk$ there is a
non-trivial atomless $\sigma$-centered \pssqa.}

\proof{{\bf (a)} Note first that $\{0,1\}^{\frak c}$ is separable
(4A2B(e-ii)), so $\frak G_{\frak c}\cong\RO(\{0,1\}^{\frak c})$ is
$\sigma$-centered (514H(b-iii));  also, of course, it is atomless.

\medskip

{\bf (b)} Taking $\Cal I$ and $\dot{\Cal I}$ as in 555G, we have

\Centerline{$\VVdQk\,\Cal P\check\lambda/\dot{\Cal I}
\cong\frak G_{\check\theta}$}

\noindent where $\theta=\Tr_{\Cal I}(\lambda;\kappa)$.
But since $\theta$ lies between $\kappa$ and the cardinal power
$\kappa^{\lambda}=\kappa^{\omega}=\kappa$, we have

\Centerline{$\VVdQk\,\check\theta=\check\kappa=\frak c$}

\noindent (554B), and

\Centerline{$\VVdQk\,\Cal P\check\lambda/\dot{\Cal I}
\cong\frak G_{\frak c}$ is $\sigma$-centered and atomless}

\noindent by (a).
}%end of proof of 555H

\leader{555I}{}\cmmnt{ The next example relies on some interesting
facts which I have not yet had any
compelling reason to spell out.   I must begin with
a definition which has so far been confined to the exercises.

\medskip

\noindent}{\bf Definition} A Boolean algebra $\frak A$ has the
{\bf Egorov property} if whenever
$\sequencen{A_n}$ is a sequence of countable partitions of unity in
$\frak A$ then there is a countable partition $B$ of unity such that
$\{a:a\in A_n$, $a\Bcap b\ne 0\}$ is finite for every $b\in B$ and
$n\in\Bbb N$.

\leader{555J}{Lemma} (a) Suppose that $X$ is a set and $\#(X)<\frak b$.
Then $\Cal PX$ has the Egorov property.

(b) Let $\frak A$ be a Dedekind $\sigma$-complete
Boolean algebra with the Egorov property and $I$
a $\sigma$-ideal of $\frak A$.   Then $\frak A/I$ has the Egorov property.

(c) A ccc Boolean algebra has the Egorov property iff it is \wsid.

\proof{{\bf (a)} Let $\sequencen{A_n}$ be a sequence of
countable partitions of $X$;  enumerate each $A_n$ as
$\ofamily{i}{N_n}{a_{ni}}$ where $N_n\in\Bbb N\cup\{\omega\}$ for each
$n$.   For each $x\in X$ and $n\in\Bbb N$, let $f_x(n)<N_n$ be such that
$x\in a_{n,f_x(n)}$.   Because $\#(X)<\frak b$, there is an
$f:\Bbb N\to\Bbb N$ such that $\{n:f(n)<f_x(n)\}$ is finite for every
$x\in X$ (522C);  set $g(x)=\sup\{n:f(n)<f_x(n)\}$.   Now set
$b_n=\{x:x\in X$, $n=\max_{k\le g(x)}f_x(k)\}$
for each $n$, and
$B=\{b_n:n\in\Bbb N\}$;  then $B$ is a partition of $X$, and for any
$m$, $n\in\Bbb N$ we have $b_n\cap a_{mi}=\emptyset$ whenever
$\max(n,f(m))<i<N_m$.
So $\{a:a\in A_m$, $b_n\cap a\ne\emptyset\}$ is finite for
all $m$, $n\in\Bbb N$.   As $\sequencen{A_n}$ is arbitrary,
$\Cal PX$ has the Egorov property.

\medskip

{\bf (b)} Let $\sequencen{C_n}$ be a sequence of countable partitions of
unity in $\frak A/I$.   For each $n\in\Bbb N$, we can choose a countable
disjoint family $A_n\subseteq\frak A$ such that
$C_n=\{a^{\ssbullet}:a\in A_n\}$;  set
$A'_n=A_n\cup\{1\Bsetminus\sup A_n\}$, so that $A'_n$ is a countable
partition of unity in $\frak A$.   Let $B$ be a countable
partition of unity in
$\frak A$ such that $\{a:a\in A'_n,\,a\Bcap b\ne 0\}$ is finite for every
$n\in\Bbb N$.   Then $D=\{b^{\ssbullet}:b\in B\}$ is a countable partition
of unity in $\frak A/I$ and $\{c:c\in C_n$, $c\Bcap d\ne 0\}$ is finite for
every $d\in D$ and $n\in\Bbb N$.

\medskip

{\bf (c)} This is elementary, because every partition of unity in $\frak A$
is countable, so the Egorov property exactly matches (ii) of 316H.
}%end of proof of 555J

\leader{555K}{G{\l}\'owczy\'nski's
example}\cmmnt{ ({\smc G{\l}\'owczy\'nski 91}, {\smc Balcar Jech \&
Paz\'ak 05}, {\smc G{\l}\'owczy\'nski 08})} Let $\lambda$ be a
\2vm\ cardinal, and $\Bbb P$ a ccc forcing notion such that

\Centerline{$\VVdP\,\check\lambda<\frak m$\dvro{.}{}}

\noindent\cmmnt{(5A3P). }Then, taking $\Cal I$ to be the null ideal of a
witnessing measure on $\lambda$, and $\dot\Cal I$ to be a $\Bbb P$-name for
the ideal of subsets of $\check\lambda$ generated by
$\check{\Cal I}$\cmmnt{, as in 555B},

\doubleinset{$\VVdP\,\Cal P\check\lambda/\dot{\Cal I}$ is ccc, atomless,
Dedekind complete, \wsid, has Maharam type $\omega$ and is not a Maharam
algebra.}

\proof{ We know from 555B that

\Centerline{$\VVdP\,\dot{\Cal I}$ is a $\sigma$-ideal, and the
quotient $\Cal P\check\lambda/\dot{\Cal I}$ is ccc and
Dedekind complete.}

\noindent By 541P, because $\frak m\le\frak c$,

\Centerline{$\VVdP\,\Cal P\check\lambda/\dot{\Cal I}$ is atomless.}

\noindent By 555J, because $\frak m\le\frak b$,

\doubleinset{$\VVdP\,\Cal P\check\lambda$ has the Egorov property, so
$\Cal P\check\lambda/\dot{\Cal I}$ has the Egorov property and is \wsid.}

\noindent Moreover, because $\frak m\le\frak p$, 517Rc tells us that

\doubleinset{$\VVdP\,\Cal P\check\lambda$ is $\sigma$-generated by a
countable
set, so $\Cal P\check\lambda/\dot{\Cal I}$ is $\sigma$-generated by a
countable set and has countable Maharam type.}

\noindent Finally, since we certainly have

\doubleinset{$\VVdP\,\check\lambda<\frak m\le\frakmctbl\le\frak c$, so
there is a separable metrizable topology on $\check\lambda$,}

\noindent 539I shows that

\Centerline{$\VVdP$ there is no non-zero Maharam submeasure on
$\Cal P\check\lambda/\dot{\Cal I}$.}
}%end of proof of 555K

\leader{555L}{Supercompact cardinals and the normal measure
\dvrocolon{axiom}}\cmmnt{ If we allow ourselves to
go a good deal farther than `measurable
cardinal' we can use similar techniques to find a forcing language in which
NMA is true.

\medskip

\noindent}{\bf Definition} An uncountable cardinal $\kappa$ is
{\bf supercompact} if for every set $X$
there is a $\kappa$-additive maximal proper ideal $\Cal I$ of subsets of
$S=[X]^{<\kappa}$ such that

\inset{($\alpha$) $\{s:s\in S$, $x\notin s\}\in\Cal I$ for every $x\in X$,

($\beta$) if $A\subseteq S$, $A\notin\Cal I$ and $f:A\to X$ is such that
$f(s)\in s$ for every $s\in A$, then there is an $x\in X$ such that
$\{s:s\in A$, $f(s)=x\}\notin\Cal I$.}

\noindent\cmmnt{(Compare 545D.)}

\leader{555M}{Proposition} A supercompact cardinal is \2vm.

\proof{ If $\kappa$ is supercompact, let $\Cal I$ be a
$\kappa$-additive maximal ideal of subsets of
$S=[\kappa]^{<\kappa}$ satisfying ($\alpha$) of 555L.
Define $f:S\to\kappa$ by setting $f(s)=\min(\kappa\setminus s)$ for
$s\in S$.   Then $\Cal J=\{J:J\subseteq\kappa$, $f^{-1}[J]\in\Cal I\}$ is a
$\kappa$-additive maximal ideal of subsets of $\kappa$.   If $\xi<\kappa$,
then

\Centerline{$f^{-1}[\{\xi\}]=\{s:f(s)=\xi\}\subseteq\{s:\xi\notin s\}
\in\Cal I$,}

\noindent so $\{\xi\}\in\Cal J$.  Thus
$\Cal J$ contains all singletons, and witnesses that $\kappa$ is \2vm.
}%end of proof of 555M

\leader{555N}{Theorem}\cmmnt{ ({\smc Prikry 75},
{\smc Fleissner 91})} Suppose that
$\kappa$ is a supercompact cardinal.   Then

\Centerline{$\VVdPk$ the normal measure axiom and the product measure
extension axiom are true.}

\proof{{\bf (a)} Life will be a little easier if I start by pointing out
that we can work with a variation of NMA as stated in 545D.  First, for a
set $X$ and an uncountable cardinal $\lambda$
let $\ddagger(X,\lambda)$ be the statement

\inset{\noindent there is a $\lambda$-additive
probability measure $\nu$ on $S=[X]^{<\lambda}$, with domain $\Cal PS$,
such that

($\alpha$) $\nu\{s:x\in s\in S\}=1$ for every $x\in X$,

($\beta$) if $g:S\setminus\{\emptyset\}\to X$ is such that
$g(s)\in s$ for every $s\in S\setminus\{\emptyset\}$, then there is a
countable set $K\subseteq X$ such that
$\nu g^{-1}[K]=1$.}

\noindent Now the point is that if $\ddagger(\alpha,\frak c)$
is true for every
ordinal $\alpha$, then the normal measure axiom is true.   \Prf\ Let $X$ be
any set.   Since, as always, we are working with the
axiom of choice, $X$ is equipollent with some ordinal and
$\ddagger(X,\frak c)$ is true;  let $\nu$ be a measure on
$S=[X]^{<\frak c}$ as above.   Given $A\subseteq S$ and a function
$f:A\to X$ which is regressive in the sense of ($\beta$) in 545D, then we
can extend $f$ to a function $g:S\setminus\{\emptyset\}\to X$ which is
regressive in the sense of ($\beta$) here.   If $K$ is a countable set such
that $g^{-1}[K]$ is conegligible, and $A$ is not negligible, then there
must be a $\xi\in K$ such that $A\cap g^{-1}[\{\xi\}]=f^{-1}[\{\xi\}]$ is
not negligible, as required in 545D.\ \Qed

\medskip

{\bf (b)} For the time being (down to the end of (d) below),
fix an ordinal $\alpha$.   Let $\Cal I$ be a
$\kappa$-additive maximal ideal of subsets of $S=[\alpha]^{<\kappa}$ as in
555L, and $\nu$ the corresponding measure on $S$, setting $\nu A=0$ and
$\nu(S\setminus A)=1$ if $A\in\Cal I$.   By 555C, we
have a corresponding $\BbbPk$-name $\dot\nu$ for a measure on $\check S$.
Now

\Centerline{$\VVdPk\,\check S\subseteq[\check{\alpha}]^{<\check{\kappa}}$,}

\noindent so we have a $\BbbPk$-name $\dot\mu$ such that

\doubleinset{$\VVdPk\,\dot\mu$ is a measure with domain
$\Cal P([\check{\alpha}]^{<\check{\kappa}})$ and
$\dot\mu(D)=\dot\nu(D\cap\check S)$ for
every $D\subseteq[\check{\alpha}]^{<\check{\kappa}}$.}

\noindent By 555C,

\doubleinset{$\VVdPk\,\dot\nu$ is a $\check{\kappa}$-additive
probability measure,
so $\dot\mu$ is a $\check{\kappa}$-additive probability measure.}

\medskip

{\bf (c)} $\VVdPk\,
\dot\mu\{s:\xi\in s\in[\check{\alpha}]^{<\check{\kappa}}\}=1$ for every
$\xi<\check{\alpha}$.

\Prf\ If $a\in\frak B_{\kappa}^+$ and $\dot\xi$ are such that
$a\VVdP\,\dot\xi<\check{\alpha}$, take any $b$ stronger than $a$ and
$\xi<\alpha$ such that $b\VVdPk\,\dot\xi=\check\xi$.   Now
$I=\{s:s\in S$, $\xi\notin s\}\in\Cal I$ so

$$\eqalign{b\VVdPk\,&0=\dot\nu\check I
=\dot\nu\{s:s\in\check S,\,\check\xi\notin s\}
=\dot\mu\{s:s\in[\check\alpha]^{<\check\kappa},\,\check\xi\notin s\}\cr
&\text{ and }
1=\dot\mu\{s:s\in[\check\alpha]^{<\check\kappa},\,\check\xi\in s\}.\cr}$$

\noindent As $b$ and $\xi$ are arbitrary,

\Centerline{$a\VVdPk\,
\dot\mu\{s:\dot\xi\in s\in[\check\alpha]^{<\check\kappa}\}=1$;}

\noindent as $a$ and $\dot\xi$ are arbitrary,

\Centerline{$\VVdPk\,
\dot\mu\{s:\xi\in s\in[\check{\alpha}]^{<\check{\kappa}}\}=1$ for every
$\xi<\check{\alpha}$.  \Qed}

\medskip

{\bf (d)} Suppose that $a\in\frak B_{\kappa}^+$ and that $\dot f$ is a
$\BbbPk$-name such that

\Centerline{$a\VVdPk\,
\dot f:[\check\alpha]^{<\check\kappa}\setminus\{\emptyset\}
\to\check\alpha$ is a function
and $\dot f(s)\in s$ whenever
$\emptyset\ne s\in[\check\alpha]^{<\check\kappa}$.}

\noindent For each $s\in S\setminus\{\emptyset\}$, we have

\Centerline{$a\VVdPk\,\dot f(\check s)\in\check s$;}

\noindent because $\BbbPk$ is ccc, there is a non-empty countable set
$J_s\subseteq s$ such that $a\VVdPk\,\dot f(\check s)\in\check J_s$
(5A3Nc).   Let $\sequencen{h_n(s)}$ be a sequence running over $J_s$.
For each $n\in\Bbb N$, we have a $\beta_n<\alpha$ such that
$\{s:s\in S\setminus\{\emptyset\}$, $h_n(s)\ne\beta_n\}\in\Cal I$.
Set $K=\{\beta_n:n\in\Bbb N\}$;
since $\Cal I$ is a $\sigma$-ideal containing $\{\emptyset\}$,
$I=\{s:s\in S\setminus\{\emptyset\}$, $J_s\not\subseteq K\}
\cup\{\emptyset\}$ belongs to $\Cal I$.
But in this case, $\VVdPk\,\dot\mu\check I=\dot\nu\check I=0$ and

\Centerline{$a\VVdPk\,\dot f(\check s)\in\check J_s\subseteq\check K$}

\noindent whenever $s\in S\setminus I$, so

\Centerline{$a\VVdPk\,\dot\mu(\dot f^{-1}[\check K])
=\dot\nu(\check S\cap\dot f^{-1}[\check K])
\ge\dot\nu(\check S\setminus\check I)=1$,}

\noindent while of course $\VVdPk\,\check K$ is countable.

\medskip

{\bf (e)} What this means is that

\Centerline{$\VVdPk\,\ddagger(\check\alpha,\check\kappa)$}

\noindent for every ordinal $\alpha$;  since forcing adds no new ordinals
(5A3Na),

\Centerline{$\VVdPk\,\ddagger(\alpha,\check\kappa)$ for every ordinal
$\alpha$.}

\noindent But 552B, with 555M and 5A1Ec, tells us that

\Centerline{$\VVdPk\,\frak c=\check\kappa$, so $\ddagger(\alpha,\frak c)$
for every ordinal $\alpha$;}

\noindent with (a) above and 545E, we get

\Centerline{$\VVdPk$ NMA and PMEA.}
}%end of proof of 555N

\leader{555O}{}\cmmnt{ All forcing constructions of \qm\ cardinals
start from \2vm\ cardinals, and there is a reason for this.

\medskip

\noindent}{\bf Theorem}\cmmnt{ ({\smc Solovay 71})}
If $\kappa$ is an uncountable cardinal and $\Cal I$
is a proper $\kappa$-saturated $\kappa$-additive ideal of $\Cal P\kappa$
containing singletons, then

\doubleinset{$L(\Cal I)\,\vDash\,\kappa$ is \2vm\ and the generalized
continuum hypothesis is true.}

\cmmnt{\medskip

\noindent{\bf Remarks}
The proof employs techniques not used elsewhere in this treatise,
so I omit it entirely, to the point of not explaining what
$L(\Cal I)$ is or what the symbol $\vDash$ means;  I remark only
that $L(\Cal I)$ is a proper class containing every ordinal and the set
$\Cal I$, and that the
theorem says that the axioms of ZFC, together with `$\kappa$ is \2vm' and
the generalized continuum hypothesis, are true when relativized
appropriately to the class $L(\Cal I)$.   For more,
see {\smc Jech 78}, p.\ 416, Theorem 82a.

}%end of comment

\exercises{\leader{555X}{Basic exercises (a)}
%\spheader 555Xa
Suppose that $\lambda$ is a \rvm\ cardinal with witnessing
probability $\nu$, and $\kappa$ a cardinal.   Let $\dot\mu$ be the
$\BbbPk$-name for a measure on $\check\lambda$ as defined in 555C.
Show that

\Centerline{$\VVdPk\,\dot\mu\check A=(\nu A)\var2spcheck$}

\noindent for any $A\subseteq\lambda$.
%555C

\spheader 555Xb Suppose that $\lambda$ is a \2vm\ cardinal.   Set
$\kappa=2^{\lambda}$.   Show that

\doubleinset{$\VVdPk\,\check\lambda$ is an \am\ cardinal with a witnessing
probability with Maharam type $\frak c=2^{\check\lambda}$.}
%555E

\spheader 555Xc Suppose that $\lambda$ is a \2vm\ cardinal and that
$\kappa=(2^{\lambda})^{(+\omega)}$.   Show that

\doubleinset{$\VVdPk\,\check\lambda$ is an \am\ cardinal with a witnessing
probability with Maharam type less than $\frak c$.}
%555E

\spheader 555Xd Suppose that $\lambda$ is a \2vm\ cardinal and
$\kappa=2^{\lambda}$.   Show that

\Centerline{$\VVdPk$ there is a non-trivial atomless $\sigma$-linked
\pssqa.}
%555E

\spheader 555Xe Let $\frak A$ be a Dedekind $\sigma$-complete Boolean
algebra.   Show that
$\frak A$ has the Egorov property iff for every sequence $\sequencen{u_n}$
in $L^0=L^0(\frak A)$ there is a sequence $\sequencen{\alpha_n}$ in
$\ooint{0,\infty}$ such that $\{\alpha_nu_n:n\in\Bbb N\}$ is order-bounded
in $L^0$.
%555J

\leader{555Y}{Further exercises (a)}
%\spheader 555Ya
Suppose that $X$ is a set,
and $\Cal I$ a proper ideal of subsets of $X$ containing singletons.
Let $\Bbb P$ be a forcing notion such that $\sat\Bbb P\le\add\Cal I$,
and $\dot{\Cal I}$ a $\Bbb P$-name for the ideal of subsets of $\check X$
generated by $\check{\Cal I}$, as in 555B.
(i) Show that

\Centerline{$\VVdP\,\add\dot{\Cal I}=(\add\Cal I)\var2spcheck$.}

\noindent
(ii) Suppose that $\sat(\Cal PX/\Cal I)<\add\Cal I$.   Set
$\theta=\max(\sat\Bbb P,\sat(\Cal PX/\Cal I))$.   Show that

\Centerline{$\VVdP\,\check\theta$ is a cardinal,
$\dot{\Cal I}$ is $\check\theta$-saturated in
$\Cal P\check X$ and $\Cal P\check X/\dot{\Cal I}$ is Dedekind complete.}

\noindent(iii) Show that if $X=\lambda$ is a regular uncountable cardinal
and $\Cal I$ is a normal ideal on $\lambda$, then

\Centerline{$\VVdP\,\dot{\Cal I}$ is a normal ideal on $\check\lambda$.}
%555B

\spheader 555Yb
In 555B, show that if $\Cal I$ is $\theta$-saturated in $\Cal PX$,
where $\theta$ is
an uncountable cardinal such that $\cff[\theta]^{\le\omega}<\add\Cal I$,
then

\Centerline{$\VVdP\,\dot{\Cal I}$ is $\check\theta$-saturated in
$\Cal P\check X$.}
%555B

\spheader 555Yc Suppose that $\lambda$ is a \2vm\ cardinal, and that
$\Bbb P$ is a forcing notion with $\#(\Bbb P)<\lambda$.   Show that
$\VVdP\,\check\lambda$ is a \2vm\ cardinal.

\spheader 555Yd Suppose that $\kappa$ is a \2vm\ cardinal, and that
$\frak m=\frak c$.   Show that

\doubleinset{$\VVdPk\,\frak c$ is \rvm,
$\frak b=\frak d=\check\frak c$
and the shrinking number of the Lebesgue null ideal is at least
$\check\frak c$.}

\spheader 555Ye\dvAnew{2014} Let $\kappa$ be a cardinal.   Suppose that
for every set $X$
there is a $\kappa$-additive maximal proper ideal $\Cal I$ of subsets of
$S=[X]^{<\kappa}$ such that

\inset{($\alpha$) $\{s:s\in S$, $x\notin s\}\in\Cal I$ for every $x\in X$,

($\beta$) if $A\subseteq S$, $A\notin\Cal I$ and $f:A\to X$ is such that
$f(s)\in s$ for every $s\in A$, then there is an $x\in X$ such that
$\{s:s\in A$, $f(s)=x\}\notin\Cal I$.}

\noindent Show that $\kappa$ is supercompact.
%555L

\spheader 555Yf Let $\kappa$ be a supercompact cardinal.   Show that
$\square_{\lambda}$ is false for every $\lambda\ge\kappa$.
%555L  mt55bits Jech 03, 27.3
}%end of exercises

\leader{555Z}{Problems (a)}
%\spheader 555Za
In 555B, what can we say about the $\pi$-weight of
$\Cal P\check X/\dot{\Cal I}$?
%555K

\spheader 555Zb
Suppose that $\lambda$ is an \am\ cardinal with a normal witnessing
probability.   Let $\ofamily{\eta}{\omega_1}{A_{\eta}}$ be a
family of non-negligible subsets of $\lambda$.   Must there be a countable
set meeting every $A_{\eta}$?\cmmnt{   (See 555F and 521Xi.)}
%555F

\endnotes{
\Notesheader{555}
The point of Solovay's theorems 555D and 555O is that they are relative
consistency results.   Continuing the discussion in the notes to \S541,
write `\Etwovmc', `\Eqmc', `\Eamc' for the sentences `there is a
\2vm\ cardinal', `there is a \qm\
cardinal' and `there is an \am\ cardinal'.   I have already noted that
there are fundamental metamathematical reasons why
we cannot have a proof, in ZFC, that

\Centerline{if ZFC is consistent then ZFC + \Eqmc\ is
consistent}

\noindent unless ZFC is actually {\it inconsistent}.
But 555D tells us that

\Centerline{if ZFC + \Etwovmc\ is consistent, then ZFC + \Eamc\ is
consistent}

\noindent and 555O that

\Centerline{if ZFC + \Eqmc\ is consistent, then ZFC + \Etwovmc\ is
consistent.}

\noindent Since \Eqmc\ is actually a consequence of both \Etwovmc\ and \Eamc,
we see that

\Centerline{if one of ZFC + \Etwovmc, ZFC + \Eamc, ZFC + \Eqmc\ is
consistent, so are the others;}

\noindent that is, \Etwovmc, \Eamc\ and \Eqmc\ are equiconsistent in ZFC.
Of course they are not in general {\it equiveridical}
(unless all are disprovable);
as noted in 555O, if we start from a universe in which \Eqmc\ is true, we
can move to one in which \Etwovmc\ and CH are both true,
so that \Eamc\ is false, and
there are easier arguments to show that if we start with \Eamc, we can
move to

\Centerline{\Eamc\ + `there are no strongly inaccessible cardinals',}

\noindent so that we have \Eamc\ true but \Etwovmc\ false.

The reason for working through these
equiconsistency results is to show that assertions like NMA, PMEA and
\Eamc, which are of interest in measure theory and general topology, are no
more dangerous than appropriate assertions about large cardinals which have
been explored in depth ({\smc Jech 78}, chap.\ 6;
{\smc Kanamori 03}, \S22; {\smc Jech 03}, \S20),
and for which we can have corresponding confidence
that either they are consistent with ZFC, or that an eventual contradiction
would lead to an earthquake, and rescue (if it came) would be from outside
measure theory.

In \S544 I examined some of the consequences of supposing that there is an
\am\ cardinal;  for instance, that there are many Sierpi\'nski sets (544G).
It is not an accident that we get similar properties of random real models
(552E).   If we want to know if something might be implied by the
existence of an \am\ cardinal, the first step is to look at what can be
determined in the forcing models of 555C.   This is often straightforward;
for
instance, since $\frak d$ is not changed by random real forcing, and since
$\frak d$ must be much lower than any \2vm\ cardinal, it must be much lower
than any \am\ cardinal created by random real forcing.   But it is quite
unclear that the same can be said about \am\ cardinals in general (544Zd).
I offer 555F as another example of a phenomenon which appears in Solovay's
models but which is not known to be true for all \am\ cardinals (555Zb).

In \S546 I gave some of the Gitik-Shelah results showing that non-trivial
`simple' algebras cannot be \pssqa s.   Of course
this depends a good deal on what
we mean by `simple'.   Looking at the basic cardinal functions, we see that
(at least if there can be measurable cardinals) then there can be
non-trivial ccc \pssqa s which are $\sigma$-centered
or have countable Maharam
type (555H, 555K).   But they are still very far away from any algebra
which can be specified without (perhaps implicitly)
using a \2vm\ cardinal at some stage.
We cannot have a non-trivial \pssqa\ with countable $\pi$-weight (546Xb),
but I do not see how to rule out `small' $\pi$-weight in general (546Zd,
555Za).


}%end of notes

\discrpage

