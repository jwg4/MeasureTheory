\frfilename{mt344.tex}
\versiondate{22.3.06}

\def\chaptername{The lifting theorem}
\def\sectionname{Realization of automorphisms}

\newsection{344}

In 343Jb, I gave an example of a `good' (compact, complete) probability
space $X$ with an automorphism $\phi$ of its measure algebra such that
both $\phi$ and $\phi^{-1}$ are representable by functions from $X$ to
itself, but there is no such representation in which the two
functions are
inverses of each other.   The present section is an attempt to describe
the further refinements necessary to ensure that automorphisms of
measure algebras can be represented by automorphisms of the measure
spaces.   It turns out that in the most important contexts in which this
can be done, a little extra work yields a significant generalization:
the simultaneous realization of countably many homomorphisms by a
consistent family of functions.

I will describe three cases in which such simultaneous realizations can
be achieved:  Stone spaces (344A), perfect complete countably separated
spaces (344C) and suitable measures on $\{0,1\}^I$ (344E-344G).   The
arguments for 344C, suitably refined, give a complete description of
perfect complete countably separated strictly localizable spaces which
are not purely atomic (344I\cmmnt{, 344Xc}).   At the same time we
find that Lebesgue measure, and the usual measure on $\{0,1\}^I$, are
`homogeneous' in the strong sense that two measurable subspaces (of
non-zero measure) are isomorphic iff they have the same measure (344J,
344L).

\leader{344A}{Stone spaces}\cmmnt{ The first case is
immediate from the work of \S\S312, 313 and 321, as collected in
324E.} If $(Z,\Sigma,\mu)$ is\cmmnt{ actually} the Stone space of a
measure algebra $(\frak A,\bar\mu)$, then every order-continuous Boolean
homomorphism
$\phi:\frak A\to\frak A$ corresponds to a unique continuous function
$f_{\phi}:Z\to Z$\cmmnt{ (312Q)} which represents
$\phi$\cmmnt{ (324E)}.
\cmmnt{The uniqueness of $f_{\phi}$
means that we can be sure that} $f_{\phi\psi}=f_{\psi}f_{\phi}$ for all
order-continuous homomorphisms $\phi$ and $\psi$;
and\cmmnt{ of course} $f_{\iota}$ is the identity map on $Z$, so that
$f_{\phi^{-1}}$ will
have to be $f_{\phi}^{-1}$ whenever $\phi$ is invertible.
Thus in this\cmmnt{ special} case we can consistently, and
canonically,
represent all
order-continuous Boolean homomorphisms from $\frak A$ to itself.

\cmmnt{Now for two cases where we have to work for the results.}

\leader{344B}{Theorem} Let $(X,\Sigma,\mu)$ be a countably separated
measure space with measure algebra $\frak A$, and $G$ a countable
semigroup of Boolean homomorphisms from $\frak A$ to itself such that
every member of $G$ can be represented by some function from $X$ to
itself.   Then a family $\langle f_{\phi}\rangle_{\phi\in G}$ of such
representatives can be chosen in such a way that
$f_{\phi\psi}=f_{\psi}f_{\phi}$ for all $\phi$, $\psi\in G$;  and if the
identity automorphism $\iota$ belongs to $G$, then we may arrange that
$f_{\iota}$ is the identity function on $X$.

\proof{{\bf (a)} Because $G\cup\{\iota\}$ satisfies the same conditions
as $G$, we may suppose from the beginning that $\iota$ belongs to $G$
itself.   Let $\Cal A\subseteq\Sigma$ be a countable set separating the
points of $X$.   For each $\phi\in G$ take some representing
function $g_{\phi}:X\to X$;  take $g_{\iota}$ to be the identity
function.   If $\phi$, $\psi\in G$, then of course

\Centerline{$((g_{\phi}g_{\psi})^{-1}[E])^{\ssbullet}
=(g_{\psi}^{-1}[g_{\phi}^{-1}[E]])^{\ssbullet}
=\psi(g_{\phi}^{-1}[E])^{\ssbullet}
=\psi\phi E^{\ssbullet}
=(g_{\psi\phi}^{-1}[E])^{\ssbullet}$}

\noindent for every $E\in\Sigma$.   By 343F, the set

\Centerline{$H_{\phi\psi}=\{x:g_{\psi\phi}(x)\ne g_{\phi}g_{\psi}(x)\}$}

\noindent is negligible and belongs to $\Sigma$.

\medskip

{\bf (b)} Set

\Centerline{$H=\bigcup_{\phi,\psi\in G}
H_{\phi\psi}$;}

\noindent because $G$ is countable, $H$ also is measurable and
negligible.   Try defining $f_{\phi}:X\to X$ by setting
$f_{\phi}(x)=g_{\phi}(x)$ if $x\in X\setminus H$, $f_{\phi}(x)=x$ if
$x\in H$.   Because $H$ is measurable, $f_{\phi}^{-1}[E]\in\Sigma$ for
every $E\in\Sigma$;  because $H$ is negligible,

\Centerline{$(f_{\phi}^{-1}[E])^{\ssbullet}
=(g_{\phi}^{-1}[E])^{\ssbullet}=\phi E^{\ssbullet}$}

\noindent for every $E\in\Sigma$, and $f_{\phi}$ represents $\phi$, for
every $\phi\in G$.   Of course $f_{\iota}=g_{\iota}$ is the identity
function on $X$.

\medskip

{\bf (c)} If $\theta\in G$ then $f_{\theta}^{-1}[H]=H$.   \Prf\ (i) If
$x\in
H$ then $f_{\theta}(x)=x\in H$.   (ii) If $f_{\theta}(x)\in H$ and
$f_{\theta}(x)=x$ then of course $x\in H$.   (iii) If
$f_{\theta}(x)=g_{\theta}(x)\in H$ then there are $\phi$, $\psi\in
G$ such that $g_{\phi}g_{\psi}g_{\theta}(x)\ne
g_{\psi\phi}g_{\theta}(x)$.   So either

\Centerline{$g_{\psi}g_{\theta}(x)\ne g_{\theta\psi}(x)$,}

\noindent or

\Centerline{$g_{\phi}g_{\theta\psi}(x)\ne g_{\theta\psi\phi}(x)$}

\noindent or

\Centerline{$g_{\theta\psi\phi}(x)\ne g_{\psi\phi}g_{\theta}(x)$,}

\noindent and in any case $x\in H$.\ \Qed

\medskip

{\bf (d)} It follows that $f_{\phi}f_{\psi}=f_{\psi\phi}$ for every
$\phi$, $\psi\in G$.   \Prf\ (i) If $x\in H$ then

\Centerline{$f_{\phi}f_{\psi}(x)=x=f_{\psi\phi}(x)$.}

\noindent (ii) If $x\in X\setminus H$ then $f_{\psi}(x)\notin H$, by
(c), so

\Centerline{$f_{\phi}f_{\psi}(x)=g_{\phi}g_{\psi}(x)=g_{\psi\phi}(x)
=f_{\psi\phi}(x)$.   \Qed}
}%end of proof of 344B

\leader{344C}{Corollary} Let $(X,\Sigma,\mu)$ be a countably
separated perfect complete strictly localizable measure space with
measure algebra $\frak A$, and $G$ a countable semigroup of
order-continuous Boolean homomorphisms from $\frak A$ to itself.  Then
we can choose simultaneously, for each $\phi\in G$, a function
$f_{\phi}:X\to X$ representing
$\phi$, in such a way that $f_{\phi\psi}=f_{\psi}f_{\phi}$ for all
$\phi$, $\psi\in G$;  and if the identity automorphism $\iota$ belongs
to $G$, then we may arrange that $f_{\iota}$ is the identity function on
$X$.   In particular, if $\phi\in G$ is invertible, and
$\phi^{-1}\in G$, we shall have $f_{\phi^{-1}}=f_{\phi}^{-1}$;  so that
if moreover
$\phi$ and $\phi^{-1}$ are measure-preserving, $f_{\phi}$ will be an
automorphism of the measure space $(X,\Sigma,\mu)$.

\proof{ By 343K, $(X,\Sigma,\mu)$ is compact.   So 343B(v) tells us that
every member of $G$ is representable, and we can apply 344B.
}%end of proof of 344C

\cmmnt{\medskip

\noindent{\bf Reminder}  Spaces satisfying the conditions of this
corollary include Lebesgue measure on $\BbbR^r$, the usual measure on
$\{0,1\}^{\Bbb N}$, and their measurable subspaces;  see also 342J,
342Xe, 343H and 343Ye.}

\leader{344D}{}\cmmnt{ The third case I wish to present requires a
more elaborate argument.   I start with a kind of Schr\"oder-Bernstein
theorem for measurable spaces.

\medskip

\noindent}{\bf Lemma} Let $X$ and $Y$ be sets, and 
$\Sigma\subseteq\Cal PX$, $\Tau\subseteq\Cal PY\,\,\sigma$-algebras.   
Suppose that there are
$f:X\to Y$, $g:Y\to X$ such that $F=f[X]\in\Tau$, $E=g[Y]\in\Sigma$, $f$
is an isomorphism between $(X,\Sigma)$ and $(F,\Tau_F)$ and $g$ is an
isomorphism between $(Y,\Tau)$ and $(E,\Sigma_E)$, writing $\Sigma_E$,
$\Tau_F$ for the subspace $\sigma$-algebras\cmmnt{ (see 121A)}.   Then
$(X,\Sigma)$ and $(Y,\Tau)$ are isomorphic, and there is an isomorphism
$h:X\to Y$ which is covered by $f$ and $g$ in the sense that

\Centerline{$\{(x,h(x)):x\in X\}\subseteq\{(x,f(x)):x\in X\}
\cup\{(g(y),y):y\in Y\}$.}

\proof{ Set $X_0=X$, $Y_0=Y$, $X_{n+1}=g[Y_n]$ and $Y_{n+1}=f[X_n]$ for
each $n\in\Bbb N$;  then $\sequencen{X_n}$ is a non-increasing sequence
in $\Sigma$ and $\sequencen{Y_n}$ is a non-increasing sequence in
$\Tau$.   Set $X_{\infty}=\bigcap_{n\in\Bbb N}X_n$,
$Y_{\infty}=\bigcap_{n\in\Bbb N}Y_n$.   Then $f\restr X_{2k}\setminus
X_{2k+1}$ is an isomorphism between $X_{2k}\setminus X_{2k+1}$ and
$Y_{2k+1}\setminus Y_{2k+2}$, while
$g\restrp Y_{2k}\setminus Y_{2k+1}$ is an isomorphism between
$Y_{2k}\setminus Y_{2k+1}$ and $X_{2k+1}\setminus X_{2k+2}$;  and
$g\restrp Y_{\infty}$ is an isomorphism between $Y_{\infty}$ and
$X_{\infty}$.    So the formula

$$\eqalign{h(x)
&=f(x)\text{ if }x\in\bigcup_{k\in\Bbb N}X_{2k}\setminus X_{2k+1},\cr
&=g^{-1}(x)\text{ for other }x\in X\cr}$$

\noindent gives the required isomorphism between $X$ and $Y$.
}%end of proof of 344D

\cmmnt{\medskip

\noindent{\bf Remark} You will recognise the ordinary
Schr\"oder-Bernstein theorem (2A1G) as the case $\Sigma=\Cal PX$,
$\Tau=\Cal PY$.}

\leader{344E}{Theorem} Let $I$ be any set, and let $\mu$ be a
$\sigma$-finite measure on $X=\{0,1\}^I$ with domain the
$\sigma$-algebra $\CalBa_I$ generated by the sets $\{x:x(i)=1\}$ as $i$
runs over $I$;  write $\frak A$ for the measure algebra of $\mu$.
Let $G$ be a countable semigroup of order-continuous Boolean
homomorphisms from $\frak A$ to itself.   Then we can choose
simultaneously, for each $\phi\in G$, a function $f_{\phi}:X\to X$
representing $\phi$, in such a way that $f_{\phi\psi}=f_{\psi}f_{\phi}$
for all $\phi$, $\psi\in G$;  and if the identity automorphism $\iota$
belongs to $G$, then we may arrange that $f_{\iota}$ is the identity
function on $X$.   In particular, if $\phi\in G$ is invertible and
$\phi^{-1}\in G$, we shall have $f_{\phi^{-1}}=f_{\phi}^{-1}$;  so that
if moreover $\phi$ is measure-preserving, $f_{\phi}$ will be an
automorphism of the measure space $(X,\CalBa_I,\mu)$.

\proof{{\bf (a)} As in 344C, we may as well suppose from the beginning
that $\iota\in G$.   The case of finite $I$ is trivial, so I will
suppose that $I$ is infinite.      For $i\in I$, set
$E_i=\{x:x(i)=1\}$;  for $J\subseteq I$, let $\Cal B_J$ be the
$\sigma$-subalgebra of $\CalBa_I$ generated by $\{E_i:I\in J\}$.    For
$i\in I$, $\phi\in G$ choose
$F_{\phi i}\in\CalBa_I$ such that
$F_{\phi i}^{\ssbullet}=\phi E_i^{\ssbullet}$.    (Of course we take
$F_{\iota i}=E_i$ for every $i$.)
Let $\Cal J$ be the family of those subsets $J$ of $I$ such that
$F_{\phi i}\in\Cal B_J$ for every $i\in J$ and $\phi\in G$.

\medskip

{\bf (b)} For the purposes of this proof, I will say that a pair
$(J,\langle g_{\phi}\rangle_{\phi\in
G})$ is {\bf consistent} if $J\in\Cal J$ and, for each $\phi\in G$,
$g_{\phi}$ is a function from $X$ to itself such that

\inset{$g_{\phi}^{-1}[E_i]\in\Cal B_J$ and
$(g_{\phi}^{-1}[E_i])^{\ssbullet}=\phi E_i^{\ssbullet}$ whenever $i\in
J$, $\phi\in G$,}


\inset{$g_{\phi}^{-1}[E_i]=E_i$ whenever $i\in I\setminus J$, $\phi\in
G$,}

\inset{$g_{\phi}g_{\psi}=g_{\psi\phi}$ whenever $\phi$, $\psi\in G$,}

\inset{$g_{\iota}(x)=x$ for every $x\in X$.}

\noindent Now the key to the proof is the following fact:  if
$(J,\langle g_{\phi}\rangle_{\phi\in G})$ is consistent, and $\tilde J$
is a member of $\Cal J$ such that $\tilde J\setminus J$ is countably
infinite, then there is a family
$\family{\phi}{G}{tilde g_{\phi}}$ such that
$(\tilde J,\langle\tilde g_{\phi}\rangle_{\phi\in G})$ is
consistent and $\tilde g_{\phi}^{-1}[E_i]=g_{\phi}^{-1}[E_i]$ whenever
$i\in J$ and $\phi\in G$, that is,
$\tilde g_{\phi}(x)\restr J=g_{\phi}(x)\restr J$ whenever $\phi\in G$ and
$x\in X$.   The construction is as follows.

\medskip

\quad{\bf (i)} Start by fixing on any infinite set
$K\subseteq\tilde J\setminus J$ such that
$(\tilde J\setminus J)\setminus K$ also is infinite.   For
$z\in\{0,1\}^K$, set $V_z=\{x:x\in X,\,x\restr K=z\}$;  then
$V_z\in\Cal B_{\tilde J}$.   All the sets $V_z$, as $z$ runs over the
uncountable
set $\{0,1\}^K$, are disjoint, so they cannot all have non-zero measure
(because $\mu$ is $\sigma$-finite), and we can choose $z$
such that $V_z$ is $\mu$-negligible.

\medskip

\quad{\bf (ii)} Define $h_{\phi}:X\to X$, for $\phi\in G$, by setting

$$\eqalign{h_{\phi}(x)(i)&=g_{\phi}(x)(i)\text{ if }i\in J,\cr
&=x(i)\text{ if }i\in I\setminus\tilde J,\cr
&=x(i)\text{ if }i\in \tilde J\setminus J\text{ and }x\in V_z,\cr
&=1\text{ if }i\in\tilde J\setminus J\text{ and }
  x\in F_{\phi i}\setminus V_z,\cr
&=0\text{ if }i\in\tilde J\setminus J\text{ and }
  x\notin F_{\phi i}\cup V_z.\cr}$$

\noindent Because $V_z\in\Cal B_{\tilde J}$ and $\mu V_z=0$, we see
that

\inset{($\alpha$) $h_{\phi}^{-1}[E_i]=g_{\phi}^{-1}[E_i]\in\Cal B_J$ if
$i\in J$,}

\inset{($\beta$) $h_{\phi}^{-1}[E_i]\in\Cal B_{\tilde J}$ and
$h_{\phi}^{-1}[E_i]\symmdiff F_{\phi i}$ is negligible
if $i\in\tilde J\setminus J$,}

\noindent and consequently

\inset{($\gamma$) $(h_{\phi}^{-1}[E_i])^{\ssbullet}=\phi E_i^{\ssbullet}$
for every $i\in\tilde J$,}

\inset{($\delta$) $(h_{\phi}^{-1}[E])^{\ssbullet}=\phi E^{\ssbullet}$
for every $E\in\Cal B_{\tilde J}$}

\noindent (by 343Ab);  moreover,

\inset{($\epsilon$) $h_{\phi}^{-1}[E]=g_{\phi}^{-1}[E]$ for every
$E\in\Cal B_J$,}

\inset{($\zeta$) $h_{\phi}^{-1}[E]\in\Cal B_{\tilde J}$ for every
$E\in\Cal B_{\tilde J}$,}

\inset{($\eta$) $h_{\phi}^{-1}[E_i]=E_i$ if $i\in I\setminus\tilde J$,}

\noindent so that

\inset{($\theta$) $h_{\phi}^{-1}[E]\in\CalBa_I$ for every $E\in\CalBa_I$;}

\noindent finally, because $F_{\iota i}=E_i$,

\inset{($\iota$) $h_{\iota}(x)=x$ for every $x\in X$.}

\medskip

\quad{\bf (iii)} The next step is to note that if $\phi$, $\psi\in G$
then

\Centerline{$H_{\phi,\psi}=\{x:x\in X,\,h_{\phi}h_{\psi}(x)\ne
h_{\psi\phi}(x)\}$}

\noindent belongs to $\Cal B_{\tilde J}$ and is negligible.
\Prf

\Centerline{$H_{\phi,\psi}
=\bigcup_{i\in I}h_{\psi}^{-1}[h_{\phi}^{-1}[E_i]]\symmdiff
h_{\psi\phi}^{-1}[E_i]$.}

\noindent Now if $i\in J$, then $h_{\phi}^{-1}[E_i]
=g_{\phi}^{-1}[E_i]\in\Cal B_J$, so

\Centerline{$h_{\psi}^{-1}[h_{\phi}^{-1}[E_i]]
=h_{\psi}^{-1}[g_{\phi}^{-1}[E_i]]
=g_{\psi}^{-1}[g_{\phi}^{-1}[E_i]]
=g_{\psi\phi}^{-1}[E_i]
=h_{\psi\phi}^{-1}[E_i]$.}

\noindent Next, for $i\in I\setminus\tilde J$,

\Centerline{$h_{\psi}^{-1}[h_{\phi}^{-1}[E_i]]
=h_{\psi}^{-1}[E_i]
=E_i
=h_{\psi\phi}^{-1}[E_i]$.}

\noindent So

\Centerline{$H_{\phi,\psi}
=\bigcup_{i\in\tilde J\setminus J}
h_{\psi}^{-1}[h_{\phi}^{-1}[E_i]]\symmdiff h_{\psi\phi}^{-1}[E_i]$.}

\noindent But for any particular $i\in\tilde J\setminus J$,
$E_i$ and $h_{\phi}^{-1}[E_i]$ belong to $\Cal B_{\tilde J}$, so

\Centerline{$(h_{\psi}^{-1}[h_{\phi}^{-1}[E_i]])^{\ssbullet}
=\psi(h_{\phi}^{-1}[E_i])^{\ssbullet}
=\psi\phi E_i^{\ssbullet}
=(h_{\psi\phi}^{-1}[E_i])^{\ssbullet}$,}

\noindent and $h_{\psi}^{-1}[h_{\phi}^{-1}[E_i]]\symmdiff
h_{\psi\phi}^{-1}[E_i]$ is a negligible set, which by (ii-$\zeta$)
belongs to $\Cal B_{\tilde J}$.   So
$H_{\phi,\psi}$ is a countable union of sets of measure $0$ in
$\Cal B_{\tilde J}$ and is itself a negligible member of $\Cal B_{\tilde
J}$, as claimed.\ \Qed

\medskip

\quad{\bf (iv)} Set

\Centerline{$H=\bigcup_{\phi,\psi\in G}H_{\phi,\psi}\cup\bigcup_{\phi\in
G}h_{\phi}^{-1}[V_z]$.}

\noindent Then $H\in\Cal B_{\tilde J}$ and $\mu H=0$.   \Prf\  We
know that every $H_{\phi,\psi}$ is negligible and belongs to
$\Cal B_{\tilde J}$ ((iii) above), that every $h_{\phi}^{-1}[V_z]$
belongs to $\Cal B_{\tilde J}$ (by (ii-$\zeta$), and that
$(h_{\phi}^{-1}[V_z])^{\ssbullet}=\phi V_z^{\ssbullet}=0$, so that
$h_{\phi}^{-1}[V_z]$ is negligible, for every $\phi\in G$
(by (ii-$\delta$)).   Consequently $H$ is negligible and belongs to
$\Cal B_{\tilde J}$.\ \Qed\
Also, of course, $V_z=h_{\iota}^{-1}[V_z]\subseteq H$.

Next, $h_{\phi}(x)\notin H$ whenever $x\in X\setminus H$ and $\phi\in
G$.   \Prf\ If $\psi$, $\theta\in G$ then

\Centerline{$h_{\theta\psi}h_{\phi}(x)
=h_{\phi\theta\psi}(x)
=h_{\psi}h_{\phi\theta}(x)=h_{\psi}h_{\theta}h_{\phi}(x)$,}

\Centerline{$h_{\psi}h_{\phi}(x)=h_{\phi\psi}(x)\notin V_z$}

\noindent because

\Centerline{$x\notin H_{\theta\psi,\phi}\cup H_{\psi,\phi\theta}\cup
H_{\theta,\phi}\cup H_{\psi,\phi}\cup h_{\phi\psi}^{-1}[V_z]$;}

\noindent thus
$h_{\phi}(x)\notin H_{\psi,\theta}\cup h_{\psi}^{-1}[V_z]$;  as $\psi$
and $\theta$ are arbitrary, $h_{\phi}(x)\notin H$.\ \Qed

\medskip

\quad{\bf (v)} The next fact we need is that there is a bijection
$q:X\to H$ such that ($\alpha$) for $E\subseteq H$,
$E\in\Cal B_{\tilde J}$ iff
$q^{-1}[E]\in\Cal B_{\tilde J}$ ($\beta$) $q(x)(i)=x(i)$ for every
$i\in I\setminus(\tilde J\setminus J)$ and $x\in X$.
\Prf\ Fix any bijection
$r:\tilde J\setminus J\to\tilde J\setminus(J\cup K)$.   Consider the
maps $p_1:X\to H$, $p_2:H\to X$ given by

$$\eqalign{p_1(x)(i)&=x(r^{-1}(i))\text{ if }
  i\in\tilde J\setminus (J\cup K),\cr
&=z(i)\text{ if }i\in K,\cr
&=x(i)\text{ if }i\in X\setminus(\tilde J\setminus J),\cr
p_2(y)&=y\cr}$$

\noindent for $x\in X$, $y\in H$.   Then $p_1$ is actually an
isomorphism between $(X,\Cal B_{\tilde J})$ and 
$(V_z,\Cal B_{\tilde J}\cap\Cal PV_z)$.   So $p_1$, $p_2$ are 
isomorphisms between
$(X,\Cal B_{\tilde J})$, $(H,\Cal B_{\tilde J}\cap\Cal PH)$ and
measurable subspaces of $H$, $X$ respectively.   By 344D, there is an
isomorphism $q$ between $X$ and $H$ such that, for every $x\in X$,
either $q(x)=p_1(x)$ or $p_2(q(x))=x$.   Since $p_1(x)\restr
I\setminus(\tilde J\setminus J)=x\restr I\setminus(\tilde J\setminus J)$
for every $x\in X$, and $p_2(y)\restr I\setminus(\tilde J\setminus
J)=y\restr I\setminus(\tilde J\setminus J)$ for every $y\in H$,
$q(x)\restr I\setminus(\tilde J\setminus J)=x\restr I\setminus(\tilde
J\setminus J)$ for every $x\in X$.\ \Qed

\medskip

\quad{\bf (vi)} An incidental fact which will be used below is the
following:  if $i\in \tilde J$ and $\phi\in G$ then $g_{\phi}^{-1}[E_i]$
belongs to $\Cal B_{\tilde J}$, because it belongs to $\Cal B_J$ if
$i\in J$, and otherwise is equal to $E_i$.   Consequently
$g_{\phi}^{-1}[E]\in\Cal B_{\tilde J}$ for every $E\in\Cal B_{\tilde
J}$.

\medskip

\quad{\bf (vii)} I am at last ready to give a formula for $\tilde
g_{\phi}$.   For $\phi\in G$ set

$$\eqalign{\tilde g_{\phi}(x)&=h_{\phi}(x)
  \text{ if }x\in X\setminus H,\cr
&=qg_{\phi}q^{-1}(x)\text{ if }x\in H.\cr}$$

\noindent Now ($\tilde J,\langle \tilde g_{\phi}\rangle_{\phi\in G})$ is
consistent.   \Prf\

\qquad($\alpha$) If $i\in\tilde J$ and $\phi\in G$,

\Centerline{$\tilde g_{\phi}^{-1}[E_i]=(h_{\phi}^{-1}[E_i]\setminus
H)\cup q[g_{\phi}^{-1}[q^{-1}[E_i\cap H]]]\in \tilde\Cal B_J$}

\noindent because $H\in\Cal B_{\tilde J}$ and
$h_{\phi}^{-1}[E]$, $q^{-1}[H\cap E]$, $g_{\phi}^{-1}[E]$ and $q[E]$ all
belong to $\Cal B_{\tilde J}$ for every $E\in\Cal B_{\tilde J}$.   At
the same time, because $\tilde g_{\phi}$ agrees with $h_{\phi}$ on the
conegligible set $X\setminus H$,

\Centerline{$(\tilde g_{\phi}^{-1}[E_i])^{\ssbullet}
=(h_{\phi}^{-1}[E_i])^{\ssbullet}
=\phi E_i^{\ssbullet}$.}

\qquad($\beta$) If $i\in I\setminus\tilde J$, $\phi\in G$ and $x\in X$ then

\Centerline{$g_{\phi}(x)(i)=h_{\phi}(x)(i)=q(x)(i)=x(i)$,}

\noindent and if $x\in H$ then $q^{-1}(x)(i)$ also is equal to $x(i)$;
so $\tilde g_{\phi}(x)(i)=x(i)$.   But this means that 
$\tilde g_{\phi}^{-1}[E_i]=E_i$.

\qquad($\gamma$) If $\phi$, $\psi\in G$ and $x\in X\setminus H$, then

\Centerline{$\tilde g_{\psi}(x)=h_{\psi}(x)\in X\setminus H$}

\noindent by (iv) above.   So

\Centerline{$\tilde g_{\phi}\tilde g_{\psi}(x)
=h_{\phi}h_{\psi}(x)
=h_{\psi\phi}(x)
=\tilde g_{\psi\phi}(x)$}

\noindent because $x\notin H_{\phi,\psi}$.   While if $x\in H$, then

\Centerline{$\tilde g_{\psi}(x)=qg_{\psi}q^{-1}(x)\in H$,}

\noindent so

\Centerline{$\tilde g_{\phi}\tilde g_{\psi}(x)
=qg_{\phi}q^{-1}qg_{\psi}q^{-1}(x)
=qg_{\phi}g_{\psi}q^{-1}(x)
=qg_{\psi\phi}q^{-1}(x)
=\tilde g_{\psi\phi}(x)$.}

\noindent Thus $\tilde g_{\phi}\tilde g_{\psi}=\tilde g_{\psi\phi}$.

\qquad($\delta$) Because $g_{\iota}(x)=h_{\iota}(x)=x$ for every $x$,
$\tilde g_{\iota}(x)=x$ for every $x$.\ \Qed

\medskip

\quad{\bf (viii)} Finally, if $i\in J$ and $\phi\in G$,
$q^{-1}[E_i]=E_i$, so that $q[E_i]=E_i\cap H$.
Accordingly $q(x)\restr J=x\restr J$ for
every $x\in X$, while $q^{-1}(x)\restr J=x\restr J$ for $x\in H$.   So
$g_{\phi}q^{-1}(x)\restr J=g_{\phi}(x)\restr J$ for $x\in H$, and

$$\eqalign{\tilde g_{\phi}(x)\restr J
&=h_{\phi}(x)\restr J=g_{\phi}(x)\restr J\text{ if }x\in X\setminus H,\cr
&=qg_{\phi}q^{-1}(x)\restr J=g_{\phi}q^{-1}(x)\restr J=g_{\phi}(x)\restr J
  \text{ if }x\in H.\cr}$$

\noindent Thus $(\tilde J,\langle \tilde g_{\phi}\rangle_{\phi\in G})$
satisfies all the required conditions.

\medskip

{\bf (c)} The remaining idea we need is the following:  there is a
non-decreasing family $\langle J_{\xi}\rangle_{\xi\le\kappa}$ in $\Cal
J$, for some cardinal $\kappa$, such that $J_{\xi+1}\setminus J_{\xi}$
is countably infinite for every $\xi<\kappa$,
$J_{\xi}=\bigcup_{\eta<\xi}J_{\eta}$ for every limit ordinal
$\eta<\kappa$, and $J_{\kappa}=I$.   \Prf\ Recall that I am already
supposing that $I$ is infinite.   If $I$ is countable, set $\kappa=1$,
$J_0=\emptyset$, $J_1=I$.   Otherwise, set $\kappa=\#(I)$ and let
$\langle i_{\xi}\rangle_{\xi<\kappa}$ be an enumeration of $I$.   For
$i\in I$, $\phi\in G$ let $K_{\phi i}\subseteq I$ be a countable set
such that $F_{\phi i}\in\Cal B_{K_{\phi i}}$.   Choose the $J_{\xi}$
inductively, as follows.   The inductive hypothesis must include the
requirement that $\#(J_{\xi})\le\max(\omega,\#(\xi))$ for every $\xi$.
Start by setting $J_0=\emptyset$.   Given $\xi<\kappa$ and
$J_{\xi}\in\Cal J$ with $\#(J_{\xi})\le\max(\omega,\#(\xi))<\kappa$,
take an infinite set $L\subseteq\kappa\setminus J_{\xi}$ and set
$J_{\xi+1}=J_{\xi}\cup\bigcup_{n\in\Bbb N}L_n$, where

\Centerline{$L_0=L\cup\{i_{\xi}\}$,}

\Centerline{$L_{n+1}=\bigcup_{i\in L_n,\phi\in G}K_{\phi i}$}

\noindent for $n\in\Bbb N$, so that every $L_n$ is countable,

\Centerline{$F_{\phi i}\in\Cal B_{L_{n+1}}$ whenever $i\in L_n$,
$\phi\in G$}

\noindent  and $J_{\xi+1}\in\Cal J$;  since $L\subseteq
J_{\xi+1}\setminus J_{\xi}\subseteq\bigcup_{n\in\Bbb N}L_n$,
$J_{\xi+1}\setminus J_{\xi}$ is countably infinite, and

\Centerline{$\#(J_{\xi+1})=\max(\omega,\#(J_{\xi}))
\le\max(\omega,\#(\xi))=\max(\omega,\#(\xi+1))$.}

\noindent For non-zero limit ordinals $\xi<\kappa$, set
$J_{\xi}=\bigcup_{\eta<\xi}J_{\eta}$;  then

\Centerline{$\#(J_{\xi})
\le\max(\omega,\#(\xi),\sup_{\eta<\xi}\#(J_{\eta}))
\le\max(\omega,\#(\xi))$.}

\noindent Thus the induction proceeds.   Observing that the construction
puts $i_{\xi}$ into $J_{\xi+1}$ for every $\xi$, we see that
$J_{\kappa}$ will be the whole of $I$, as required.\ \Qed

\medskip

{\bf (d)} Now put (b) and (c) together, as follows.   Take
$\langle J_{\xi}\rangle_{\xi\le\kappa}$ from (c).   Set
$f_{\phi 0}(x)=x$ for every
$\phi\in G$, $x\in X$;  then, because $J_0=\emptyset$,
$(J_0,\langle f_{\phi 0}\rangle_{\phi\in G})$ is consistent in the sense
of (b).   Given that $(J_{\xi},\langle f_{\phi\xi}\rangle_{\phi\in G})$
is consistent, where $\xi<\kappa$, use the construction of (b) to find a
family $\langle f_{\phi,\xi+1}\rangle_{\phi\in G}$ such that
$(J_{\xi+1},\langle f_{\phi,\xi+1}\rangle_{\phi\in G})$ is consistent
and $f_{\phi,\xi+1}(x)(i)=f_{\phi\xi}(x)(i)$ for every $i\in J_{\xi}$ and
$x\in X$.   At a non-zero limit ordinal $\xi\le\kappa$, set

$$\eqalign{f_{\phi\xi}(x)(i)&=f_{\phi\eta}(x)(i)
  \text{ if }x\in X,\,\eta<\xi,\,i\in J_{\eta},\cr
&=x(i)\text{ if }i\in I\setminus J_{\xi}.\cr}$$

\noindent (The inductive hypothesis includes the requirement that
$f_{\phi\eta}(x)\restr J_{\zeta}=f_{\phi\zeta}(x)\restr J_{\zeta}$
whenever $\phi\in G$, $x\in X$ and $\zeta\le\eta<\xi$.)   To see that
$(J_{\xi},\langle f_{\phi\xi}\rangle_{\phi\in G})$
is consistent, the only non-trivial point to check is that

\Centerline{$f_{\phi,\xi}f_{\psi,\xi}=f_{\psi\phi,\xi}$}

\noindent for all $\phi$, $\psi\in G$.   But if $i\in J_{\xi}$ there is
some $\eta<\xi$ such that $i\in J_{\eta}$, and in this case

\Centerline{$f_{\psi,\xi}^{-1}[E_i]
=f_{\psi,\eta}^{-1}[E_i]\in\Cal B_{J_{\eta}}$}

\noindent is determined by coordinates in $J_{\eta}$, so that (because
$f_{\phi,\xi}(x)\restr J_{\eta}=f_{\phi,\eta}(x)\restr J_{\eta}$ for
every $x$)

\Centerline{$f_{\phi,\xi}^{-1}[f_{\psi,\xi}^{-1}[E_i]]
=f_{\phi,\eta}^{-1}[f_{\psi,\eta}^{-1}[E_i]]
=f_{\psi\phi,\eta}^{-1}[E_i]
=f_{\psi\phi,\xi}^{-1}[E_i]$;}

\noindent while if $i\in I\setminus J_{\xi}$ then

\Centerline{$f_{\psi\phi,\xi}^{-1}[E_i]=E_i
=f_{\phi,\xi}^{-1}[E_i]=f_{\psi,\xi}^{-1}[E_i]
=f_{\psi,\xi}^{-1}[f_{\phi,\xi}^{-1}[E_i]]$.}

\noindent Thus

\Centerline{$f_{\psi,\xi}^{-1}[f_{\phi,\xi}^{-1}[E_i]]
=f_{\psi\phi,\xi}^{-1}[E_i]$}

\noindent for every $i$, and
$f_{\phi,\xi}f_{\psi,\xi}=f_{\psi\phi,\xi}$.

On completing the induction, set $f_{\phi}=f_{\phi\kappa}$ for every
$\phi\in G$;  it is easy to see that
$\langle f_{\phi}\rangle_{\phi\in G}$ satisfies the conditions of the
theorem.
}%end of proof of 344E

\leader{344F}{Corollary} Let $I$ be any set, and let $\mu$ be a
$\sigma$-finite measure on $X=\{0,1\}^I$.   Suppose that $\mu$ is
the completion of its restriction to the
$\sigma$-algebra $\CalBa_I$ generated by the sets $\{x:x(i)=1\}$ as $i$
runs over $I$.  Write $\frak A$ for the measure algebra of $\mu$.
Let $G$ be a countable semigroup of order-continuous Boolean
homomorphisms from $\frak A$ to itself.   Then we can choose
simultaneously, for each $\phi\in G$, a function $f_{\phi}:X\to X$
representing $\phi$, in such a way that $f_{\phi\psi}=f_{\psi}f_{\phi}$
for all $\phi$, $\psi\in G$;  and if the identity automorphism $\iota$
belongs to $G$, then we may arrange that $f_{\iota}$ is the identity
function on $X$.   In particular, if $\phi\in G$ is invertible and
$\phi^{-1}\in G$, we shall have $f_{\phi^{-1}}=f_{\phi}^{-1}$;  so that
if moreover $\phi$ is measure-preserving, $f_{\phi}$ will be an
automorphism of the measure space $(X,\Sigma,\mu)$.

\proof{ Apply 344E to $\mu\restr\CalBa_I$;  of course $\frak A$ is
canonically isomorphic to the measure algebra of $\mu\restr\CalBa_I$
(322Da).   The functions $f_{\phi}$ provided by 344E still represent the
homomorphisms $\phi$ when re-interpreted as functions on the completed
measure space $(\{0,1\}^I,\mu)$, by 343Ac.
}%end of proof of 344F

\leader{344G}{Corollary} Let $I$ be any set, $\nu_I$ the usual measure
on $\{0,1\}^{I}$, and $\frak B_I$ its measure algebra.   Then any
measure-preserving automorphism of $\frak B_I$ is representable by a
measure space automorphism of $(\{0,1\}^I,\nu_I)$.

\leaveitout{344F}

\leader{344H}{Lemma}\dvArevised{2010} Let $(X,\Sigma,\mu)$ be a perfect
semi-finite measure space.   If $H\in\Sigma$ is a non-negligible set which
includes no atom, there is a
negligible subset of $H$ of cardinal $\frak c$.

\proof{{\bf (a)} Consider first the case in which $\mu$ is atomless,
compact and totally finite, and $H=X$.
Let $\Cal K\subseteq\Cal PX$ be a compact class such that 
$\mu$ is inner regular with respect to $\Cal K$.
Set $S^*=\bigcup_{n\in\Bbb N}\{0,1\}^n$, and
choose $\langle K_{\sigma}\rangle_{\sigma\in S^*}$ inductively, as follows.
$K_{\emptyset}$ is to be any non-negligible member of $\Cal K\cap\Sigma$.
Given that $\mu K_{\sigma}>0$, where
$\sigma\in\{0,1\}^n$, take $F_{\sigma}$, $F'_{\sigma}\subseteq K_{\sigma}$ 
to be disjoint non-negligible
measurable sets both of measure at most $3^{-n}$;  such exist because
$\mu$ is atomless (215C).   Choose
$K_{\sigma^{\smallfrown}\fraction{0}}\subseteq F_{\sigma}$,
$K_{\sigma^{\smallfrown}\fraction{1}}\subseteq F'_{\sigma}$ to be 
non-negligible members of $\Cal K\cap\Sigma$.

For each $w\in\{0,1\}^{\Bbb N}$, $\sequencen{K_{w\restr n}}$ is a
decreasing sequence of members of $\Cal K$ all of non-zero measure, so
has non-empty intersection;  choose a point
$x_w\in\bigcap_{n\in\Bbb N}K_{w\restr n}$.   Since
$K_{\sigma^{\smallfrown}\fraction{0}}
\cap K_{\sigma^{\smallfrown}\fraction{1}}=\emptyset$ for every 
$\sigma\in S^*$, all the $x_w$ are
distinct, and $A=\{x_w:w\in\{0,1\}^{\Bbb N}\}$ has cardinal $\frak c$.
Also

\Centerline{$A\subseteq\bigcup_{\sigma\in\{0,1\}^{n}}K_{\sigma}$}

\noindent which has measure at most $2^n3^{-(n-1)}$ for every $n\ge 1$,
so $\mu^*A=0$ and $A$ is negligible.

\medskip

{\bf (b)} Now consider the case in which $\mu$ is atomless and totally
finite and perfect, but not necessarily compact, while again $H=X$.   
In this case, by 215C, 
we can choose $\sequencen{E_n}$ inductively so that 
$\mu(E_n\cap E)=\bover12\mu E$ whenever $n\in\Bbb N$ and $E$ is an atom of
the subalgebra of $\Cal PX$ generated by $\{E_i:i<n\}$.
Now define $f:X\to\{0,1\}^{\Bbb N}$ by setting
$f(x)=\sequencen{\chi E_n(x)}$ for $x\in X$.   Consider the image measure
$\nu=\mu f^{-1}$ on $Y=f[X]\subseteq\{0,1\}^{\Bbb N}$.   
This is perfect.   \Prf\ If
$g:Y\to\Bbb R$ is $\Tau$-measurable, where $\Tau=\dom\nu$,
and $\nu F>0$, then
$gf:X\to\Bbb R$ is $\Sigma$-measurable and $\mu f^{-1}[F]>0$.   There is
therefore a compact set $K\subseteq gf[f^{-1}[F]]$ such that
$\mu(gf)^{-1}[K]>0$.   In this case, $K\subseteq g[F]$ and 
$\nu g^{-1}[K]>0$.\ \Qed

Next, for every $n\in\Bbb N$ and $\sigma\in\{0,1\}^n$,

\Centerline{$\nu\{y:y\in Y,\,y\restr n=\sigma\}
=\mu\{x:\,\Forall i<n,\,x\in E_i\iff \sigma(i)=1\}=2^{-n}\mu X$.}

\noindent So $\nu$ can have no atom of measure greater than $2^{-n}\mu X$; 
as $n$ is arbitrary, $\nu$ is atomless.   Thirdly, $(Y,\Tau,\nu)$ is
countably separated, because $\sequencen{\{y:y\in Y$, $y(n)=1\}}$ is a
sequence of measurable sets separating the points of $Y$.   By 343K, $\nu$
is compact;  by (a) here, there is a $\nu$-negligible set $B\subseteq Y$ of
cardinal $\frak c$.   Now $f^{-1}[B]$ is $\mu$-negligible, and because
$B\subseteq f[X]$, $\#(f^{-1}[B])\ge\#(B)=\frak c$.   We therefore have a
set $A\subseteq f^{-1}[B]$ of cardinal $\frak c$, and $A$ is
$\mu$-negligible.

\medskip

{\bf (c)} Finally, for the general case in which $\mu$ is just semi-finite
and perfect, and $H$ is a non-negligible subset of $X$ not including an
atom, let $E\subseteq H$ be a set of non-zero finite
measure.   Then the subspace measure $\mu_E$ is
atomless.   Also $\mu_E$ is perfect.   \Prf\ Let $f:E\to\Bbb R$ be a
measurable function.   Define $g:X\to\Bbb R$ by setting 

$$\eqalign{g(x)
&=e^{f(x)}\text{ if }x\in E,\cr
&=0\text{ if }x\in X\setminus E.\cr}$$

\noindent Then $g$ is measurable.   There is therefore a compact set
$K\subseteq g[E]$ such that $\mu g^{-1}[K]>0$.   Now 
$\ln[K]\subseteq f[E]$ is compact and 
$\mu_Ef^{-1}[\ln[K]]=\mu g^{-1}[K]>0$.\ \Qed

By (b), there is a $\mu_E$-negligible
set $A\subseteq E$ of cardinal $\frak c$, and of course $A$ is also a
$\mu$-negligible subset of $H$.
}%end of proof of 344H

\cmmnt{\medskip

\noindent{\bf Remark} I see that in this proof I have slipped into a
notation which is a touch more sophisticated than what I have used so
far.   See 3A1H for a note on the interpretations of `$\{0,1\}^n$',
`$\{0,1\}^{\Bbb N}$' which make sense of the formulae here.
}

\leader{344I}{Theorem} Let $(X,\Sigma,\mu)$ and $(Y,\Tau,\nu)$
be atomless, perfect, complete, strictly localizable, countably
separated measure spaces of the same non-zero magnitude.   Then they are
isomorphic.

\proof{{\bf (a)} The point is that the measure
algebra $(\frak A,\bar\mu)$ of $\mu$ has Maharam type $\omega$.  \Prf\
Let $\sequencen{E_n}$ be a sequence in $\Sigma$ separating the points of
$X$.   Let $\Sigma_0$ be
the $\sigma$-subalgebra of $\Sigma$ generated by $\{E_n:n\in\Bbb N\}$,
and $\frak A_0$ the order-closed subalgebra of $\frak A$ generated by
$\{E_n^{\ssbullet}:n\in\Bbb N\}$;  then $E^{\ssbullet}\in\frak A_0$ for
every $E\in\Sigma_0$, and $(X,\Sigma_0,\mu\restr\Sigma_0)$ is countably
separated.   Let $f:X\to\Bbb R$ be $\Sigma_0$-measurable and injective
(343E).   Of course $f$ is also $\Sigma$-measurable.   If
$a\in\frak A\setminus\{0\}$, express $a$ as $E^{\ssbullet}$ where
$E\in\Sigma$.   Because $(X,\Sigma,\mu)$ is perfect, there is a compact
$K\subseteq\Bbb R$ such that $K\subseteq f[E]$ and
$\mu f^{-1}[K]>0$.  $K$ is surely a Borel set, so
$f^{-1}[K]\in\Sigma_0$ and

\Centerline{$b=f^{-1}[K]^{\ssbullet}\in\frak A_0\setminus\{0\}$.}

\noindent   But because $f$ is injective, we also have
$f^{-1}[K]\subseteq E$ and $b\Bsubseteq a$.   As $a$ is arbitrary,
$\frak A_0$ is order-dense in $\frak A$;   but $\frak A_0$ is
order-closed, so must be the whole of $\frak A$.   Thus $\frak A$ is
$\tau$-generated by the countable set $\{E_n^{\ssbullet}:n\in\Bbb N\}$,
and $\tau(\frak A)\le\omega$.\ \Qed

On the other hand, because $\frak A$ is atomless, and not $\{0\}$, none
of its principal ideals can have finite Maharam type, and it is \Mth,
with type $\omega$.

\medskip

{\bf (b)} Writing $(\frak B,\bar\nu)$ for the measure algebra of $\nu$,
we see that the argument of (a) applies equally to $(\frak B,\bar\nu)$,
so that $(\frak A,\bar\mu)$ and $(\frak B,\bar\nu)$ are atomless
localizable
measure algebras, with Maharam type $\omega$ and the same magnitude.
Consequently they are isomorphic as measure algebras, by 332J.   Let
$\phi:\frak A\to\frak B$ be a measure-preserving isomorphism.

By 343K, both $\mu$ and $\nu$ are (locally) compact.   As they are also
complete and strictly localizable, 343B tells us that
there are functions $g:Y\to X$ and $f:X\to Y$ representing
$\phi$ and $\phi^{-1}$.   Now $fg:Y\to Y$ and $gf:X\to X$
represent the identity automorphisms on $\frak B$, $\frak A$, so by 343F
are equal almost everywhere to the identity functions on $Y$, $X$
respectively.   Set

\Centerline{$E=\{x:x\in X,\,gf(x)=x\}$,
\quad$F=\{y:y\in Y,\,fg(y)=y\}$;}

\noindent then both $E$ and $F$ are conegligible.   Of course
$f[E]\subseteq F$ (since $fgf(x)=f(x)$ for every $x\in E$), and
similarly $g[F]\subseteq E$;  consequently $f\restr E$, $g\restr F$ are
the two halves of a one-to-one correspondence between $E$ and $F$.
Because $\phi$ is measure-preserving, $\mu f^{-1}[H]=\nu H$ and
$\nu g^{-1}[G]=\mu G$ for every $G\in\Sigma$, $H\in\Tau$;
accordingly $f\restr E$ is an isomorphism between the subspace measures
on $E$ and $F$.

\medskip

{\bf (c)} By 344H, there is a
negligible set $A\subseteq E$ of cardinal $\frak c$.   Now $X$ and $Y$,
being countably separated, both have cardinal at most $\frak c$.
(There are injective functions from $X$ and $Y$ to $\Bbb R$.)   Set

\Centerline{$B=A\cup(X\setminus E)$,\quad $C=f[A]\cup(Y\setminus F)$.}

\noindent Then $B$ and $C$ are negligible subsets of $X$, $Y$
respectively, and both have cardinal $\frak c$ precisely, so there is a
bijection $h:B\to C$.   Set

$$\eqalign{f_1(x)&=f(x)\text{ if }x\in X\setminus B=E\setminus A,\cr
&=h(x)\text{ if }x\in B.\cr}$$

\noindent Then, because $\mu$ and $\nu$ are complete, $f_1$ is an
isomorphism between the measure spaces
$(X,\Sigma,\mu)$ and $(Y,\Tau,\nu)$, as required.
}%end of proof of 344I

\leader{344J}{Corollary} Suppose that $E$, $F$ are two Lebesgue
measurable subsets of $\BbbR^r$ of the same non-zero measure.   Then
the subspace measures on $E$ and $F$ are isomorphic.

\leader{344K}{Corollary} (a) A measure space is isomorphic to Lebesgue
measure on $[0,1]$ iff it is an atomless countably separated compact (or
perfect) complete probability space;  in this case it is also isomorphic
to the usual measure on $\{0,1\}^{\Bbb N}$.

(b) A measure space is isomorphic to Lebesgue measure on $\Bbb R$ iff it
is an atomless countably separated compact (or perfect) $\sigma$-finite
measure space which is not totally finite;  in this case it is also
isomorphic to Lebesgue measure on any Euclidean space $\BbbR^r$.

(c) Let $\mu$ be Lebesgue measure on $\Bbb R$.   If $0<\mu E<\infty$ and
we set $\nu F=\Bover1{\mu E}\mu F$ for every measurable $F\subseteq E$,
then $(E,\nu)$ is isomorphic to Lebesgue measure on $[0,1]$.

\leader{344L}{}\cmmnt{ The homogeneity property of Lebesgue
measure described in 344J is repeated in $\{0,1\}^I$ for any infinite $I$.

\medskip

\noindent}{\bf Theorem}\discrversionA{\footnote{Revised 2006.}}{} 
Let $I$ be an infinite set, and $\nu_I$ the usual
measure on $\{0,1\}^I$.   If $E\subseteq\{0,1\}^I$ is a
measurable set of non-zero measure, the subspace
measure on $E$ is isomorphic to $(\nu_IE)\nu_I$.

\proof{ For $J\subseteq I$ let $\nu_J$ be the usual measure on
$X_J=\{0,1\}^J$.

\medskip

{\bf (a)} If $I$ is countably infinite,
then the subspace measure on $E$ is perfect and complete and countably
separated, so is isomorphic to Lebesgue measure on the interval
$[0,\nu_IE]$, by 344I.
But by 344Kc, or otherwise, this is isomorphic,
up to a scalar multiple of the measure, to Lebesgue measure on $[0,1]$,
which is in turn isomorphic to $\nu_I$.

So henceforth we can suppose that $I$ is uncountable.

\medskip

{\bf (b)} By 254Oc there are a countable set $J\subseteq I$ and a set
$E'\subseteq E$, determined by coordinates in $J$, such that
$E\setminus E'$ is
negligible.   Identifying $X_I$ with $X_J\times X_{I\setminus J}$ (254N),
we can think of $E'$ as $V\times X_{I\setminus J}$ where $V$ is
measured by $\nu_J$ (see 254O).
Take $v_0\in V$ and set

\Centerline{$V'=V\setminus\{v_0\}$,
\quad$W'=X_J\setminus\{v_0\}$,
\quad$E''=V'\times X_{I\setminus J}$,
\quad$F''=W'\times X_{I\setminus J}$.}

\noindent Then by (a), applied to $V'$ and $W'$ in turn,
we have a bijection
$g:V'\to W'$ which, up to a scalar multiple of the measure, is an
isomorphism between the subspace measures.   Now the subspace measure on
$V'\times X_{I\setminus J}$ is just the product of the subspace
measure on $V'$ with $\nu_{I\setminus J}$
(251Q(ii)), so if we set $f_0(x,z)=(g(x),z)$ for $x\in V'$ and
$z\in X_{I\setminus J}$, then $f_0:E''\to F''$ is an isomorphism of
the subspace measures on $E''$ and
$F''$, up to a scalar multiple of the
measures as always.   On the other hand, $E\setminus E''$ and
$X_I\setminus F''$ are negligible and both have cardinal
$\#(X_{I\setminus J})=\#(X_I)$, so we have a bijection
$f_1:E\setminus E''\to X_I\setminus F''$.
Putting $f_0$ and $f_1$ together,
we have a bijection $f:E\to X_I$ which, up to a scalar multiple of the
measure, is an isomorphism of the subspace measure on $E$ with $\nu_I$.
}%end of proof of 344L

\exercises{
\leader{344X}{Basic exercises (a)}
%\spheader 344Xa
Let $(X,\Sigma,\mu)$ and $(Y,\Tau,\nu)$ be measure spaces, and suppose
that there are $E\in\Sigma$, $F\in\Tau$ such that $(X,\Sigma,\mu)$ is
isomorphic to the subspace $(F,\Tau_F,\nu_F)$, while $(Y,\Tau,\nu)$ is
isomorphic to $(E,\Sigma_E,\mu_E)$.   Show that $(X,\Sigma,\mu)$ and
$(Y,\Tau,\nu)$ are isomorphic.
%344D

\spheader 344Xb Let $(X,\Sigma,\mu)$ and $(Y,\Tau,\nu)$ be perfect
countably separated complete strictly localizable measure spaces with
isomorphic measure algebras.   Show that there are conegligible subsets
$X'\subseteq X$, $Y'\subseteq Y$ such that $X'$ and $Y'$, with the
subspace measures, are isomorphic.
%344I

\spheader 344Xc Let $(X,\Sigma,\mu)$ and $(Y,\Tau,\nu)$ be perfect
countably separated complete strictly localizable measure spaces with
isomorphic measure algebras.   Suppose that they are not purely atomic.
Show that they are isomorphic.
%344I

\spheader 344Xd Give an example of two perfect countably separated
complete probability spaces, with isomorphic measure algebras, which are
not isomorphic.
%344I

\spheader 344Xe Let $(Z,\Sigma,\mu)$ be the Stone space of a homogeneous
measure algebra.   Show that if $E$, $F\in\Sigma$ have the same non-zero
finite measure, then the subspace measures on $E$ and $F$ are
isomorphic.
%344J maybe should be in 344Y

\spheader 344Xf Let $(I^{\|},\Sigma,\mu)$ be the split interval with its
usual measure (343J), and $\frak A$ its measure algebra.  (i) Show that
every measure-preserving
automorphism of $\frak A$ is represented by a measure space
automorphism of $I^{\|}$.   (ii) Show that if $E$, $F\in\Sigma$ and
$\mu E=\mu F>0$ then the subspace measures on $E$ and $F$ are
isomorphic.
%344L

\vleader{72pt}{344Y}{Further exercises (a)}
%\spheader 344Ya
Let $X$ be a set, $\Sigma$ a $\sigma$-algebra of
subsets of $X$, $\Cal I$ a $\sigma$-ideal of $\Sigma$, and $\frak A$ the
quotient $\Sigma/\Cal I$.   Suppose that there is a countable set
$\Cal A\subseteq\Sigma$ separating the points of $X$.   Let $G$ be a
countable semigroup of Boolean homomorphisms from $\frak A$ to itself
such that every member of $G$ can be represented by some function from
$X$ to itself.   Show that a family
$\langle f_{\phi}\rangle_{\phi\in G}$ of such representatives can be
chosen in such a way that $f_{\phi\psi}=f_{\psi}f_{\phi}$ for all
$\phi$, $\psi\in G$;  and if the identity automorphism $\iota$ belongs
to $G$, then we may arrange that $f_{\iota}$ is the identity function on
$X$.
%344B

\spheader 344Yb Let $\frak A$, $\frak B$ be Dedekind $\sigma$-complete
Boolean algebras.   Suppose that each is isomorphic to a principal ideal
of the other.   Show that they are isomorphic.
%344D

\spheader 344Yc Let $I$ be an infinite set, and
write $\CalBa_I$ for the $\sigma$-algebra of subsets of $X=\{0,1\}^I$
generated by the sets $\{x:x(i)=1\}$ as $i$ runs over $I$.   Let
$\mu$ and $\nu$ be $\sigma$-finite measures on $X$, both with domain
$\CalBa_I$, and with measure algebras $(\frak A,\bar\mu)$,
$(\frak B,\bar\nu)$.   Show that any Boolean isomorphism
$\phi:\frak A\to\frak B$ is represented by a permutation $f:X\to X$ such
that $f^{-1}$ represents $\phi^{-1}:\frak B\to\frak A$, and hence that
$(\frak A,\bar\mu)$ is isomorphic to $(\frak B,\bar\nu)$ iff
$(X,\CalBa_I,\mu)$ is isomorphic to $(X,\CalBa_I,\nu)$.
%344E

\spheader 344Yd Let $I$ be any set, and write $\CalBa_I$ for the
$\sigma$-algebra of subsets of $X=\{0,1\}^I$ generated by the sets
$\{x:x(i)=1\}$ as $i$
runs over $I$.   Let $\Cal I$ be an $\omega_1$-saturated $\sigma$-ideal of
$\CalBa_I$,
and write $\frak A$ for the quotient Boolean algebra
$\Cal B/\Cal I$.   Let $G$ be a countable semigroup of order-continuous
Boolean homomorphisms from $\frak A$ to itself.   Show that we can
choose simultaneously, for each $\phi\in G$, a function $f_{\phi}:X\to
X$ representing $\phi$, in such a way that
$f_{\phi\psi}=f_{\psi}f_{\phi}$
for all $\phi$, $\psi\in G$;  and if the identity automorphism $\iota$
belongs to $G$, then we may arrange that $f_{\iota}$ is the identity
function on $X$.   In particular, if $\phi\in G$ is invertible and
$\phi^{-1}\in G$, $f_{\phi}$ will be an automorphism of the structure
$(X,\CalBa_I,\Cal I)$.
%344E

\spheader 344Ye Let $I$ be any set, and write $\CalBa_I$ for the
$\sigma$-algebra of subsets of $X=\{0,1\}^I$ generated by the sets
$\{x:x(i)=1\}$ as $i$
runs over $I$.   Let $\Cal I$, $\Cal J$ be $\omega_1$-saturated
$\sigma$-ideals of $\CalBa_I$.
Show that if the Boolean algebras $\CalBa_I/\Cal I$
and $\CalBa_I/\Cal J$ are isomorphic, so are the structures
$(X,\CalBa_I,\Cal I)$ and $(X,\CalBa_I,\Cal J)$.
%344E
}%end of exercises

\cmmnt{\Notesheader{344} In this section and the last, I have allowed
myself to drift some distance from the avowed subject of this chapter;
but it seemed a suitable place for this material, which is fundamental
to abstract measure theory.   We find that the concepts of \S\S342-343
are just what is needed to characterise Lebesgue measure (344K), and the
characterization shows that among non-negligible measurable subspaces of
$\BbbR^r$ the isomorphism classes are determined by a single parameter,
the measure of the subspace.   Of course a very large number of other
spaces -- indeed, most of those appearing in ordinary applications of
measure theory to other topics -- are perfect and countably separated
(for example, those of 342Xe and 343Ye), and therefore covered by this
classification.   I note that it includes, as a special case, the
isomorphism between Lebesgue measure on $[0,1]$ and the usual measure on
$\{0,1\}^{\Bbb N}$ already described in 254K.

In 344I, the first part of the proof is devoted to showing that a
perfect countably separated measure space has countable Maharam type;  I
ought perhaps to note here that we must resist the temptation to suppose
that
all countably separated measure spaces have countable Maharam type.   In
fact there are countably separated probability spaces with Maharam type
as high as $2^{\frak c}$.   The arguments are elementary but seem to fit
better into \S521 of Volume 5 than here.

I have offered three contexts in which automorphisms of measure algebras
are represented by automorphisms of measure spaces (344A, 344C, 344E).
In the first case, every automorphism can be represented simultaneously
in a consistent way.   In the other two cases, there is, I am sure, no
such consistent family of representations which can be constructed
within ZFC;  but the theorems I give offer consistent simultaneous
representations of countably many homomorphisms.   The question
arises, whether `countably many' is the true natural limit of
the arguments.   In fact it is possible to extend both results to
families of at most $\omega_1$ automorphisms.
%I hope to return to this in Volume 5.

Having successfully characterized Lebesgue measure -- or, what is very
nearly the same thing, the usual measure on $\{0,1\}^{\Bbb N}$ -- it is
natural to seek similar characterizations of the usual measures on
$\{0,1\}^{\kappa}$ for uncountable cardinals $\kappa$.   This seems to
be hard.   A variety of examples (some
touched on in the exercises to \S521)
show that none of the most natural conjectures can be provable in ZFC.

In fact the principal new ideas of this section do not belong
specifically to measure theory;  rather, they belong to the general
theory of $\sigma$-algebras and $\sigma$-ideals of sets.   In the case
of the Schr\"oder-Bernstein-type theorem 344D, this is obvious from the
formulation I give.   (See also 344Yb.)   In the case of 344B and 344E,
I offer generalizations in 344Ya-344Ye.   Of course the applications of
344B here, in 344C and its corollaries, depend on Maharam's theorem and
the concept of `compact' measure space.   The former has no
generalization to the wider context, and the value of the latter is
based on the
equivalences in Theorem 343B, which also do not have simple
generalizations.

The property described in 344J -- a measure space
$(X,\Sigma,\mu)$ in which any two measurable subsets of the same
non-zero measure are isomorphic -- seems to be a natural concept of
`homogeneity' for measure spaces;  it seems unreasonable to ask for all
sets of zero measure to be isomorphic, since finite sets of different
cardinalities can be expected to be of zero measure.   An extra
property, shared by Lebesgue measure and the usual measure on
$\{0,1\}^I$ and by the measure on the split interval (344Kc, 344L, 344Xf)
but not by
counting measure, would be the requirement that
measurable sets of different non-zero finite measures should be
isomorphic up
to a scalar multiple of the measure.   All these examples have the
further property, that all automorphisms of their measure algebras
correspond to automorphisms of the measure spaces.
}%end of notes


\discrpage

