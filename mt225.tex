\frfilename{mt225.tex}
\versiondate{16.8.15}
\copyrightdate{1996}

\def\chaptername{The Fundamental Theorem of Calculus}
\def\sectionname{Absolutely continuous functions}

\newsection{225}

We are now ready for a full characterization of the functions that can
appear as indefinite integrals (225E, 225Xf).   The essential idea
is that of `absolute continuity' (225B).   In the second half of the
section (225G-225N) I describe some of the relationships between
this concept and those we have already seen.

\leader{225A}{Absolute continuity of the
indefinite \dvrocolon{integral}}\cmmnt{ I
begin with an easy fundamental result from general measure theory.

\medskip

\noindent}{\bf Theorem} Let
$(X,\Sigma,\mu)$ be any measure space and $f$ an integrable real-valued
function defined on a conegligible subset of $X$.
Then for any $\epsilon>0$ there are a measurable set
$E$ of finite measure and a real number
$\delta>0$ such that $\int_F|f|\le\epsilon$ whenever $F\in\Sigma$ and
$\mu(F\cap E)\le\delta$.

\proof{ There is a non-decreasing sequence $\sequencen{g_n}$ of
non-negative simple functions such that $|f|\eae\lim_{n\to\infty}g_n$
and $\int|f|=\lim_{n\to\infty}\int g_n$.   Take $n\in\Bbb N$ such that
$\int g_n\ge\int|f|-\bover12\epsilon$.
Let $M>0$, $E\in\Sigma$ be such that $\mu E<\infty$ and
$g_n\le M\chi E$;  set $\delta=\epsilon/2M$.
If $F\in\Sigma$ and $\mu(F\cap E)\le\delta$, then

\Centerline{$\int_Fg_n=\int g_n\times\chi F
\le M\mu(F\cap E)\le\Bover12\epsilon$;}

\noindent  consequently

\Centerline{$\int_F|f|
=\int_Fg_n+\int_F|f|-g_n\le\Bover12\epsilon+\int|f|-g_n\le\epsilon$.}
}%end of proof of 225A

\leader{225B}{Absolutely continuous functions on $\Bbb R$:  Definition}
If $[a,b]$ is a non-empty
closed interval in $\Bbb R$ and $f:[a,b]\to\Bbb R$ is a function, we say
that $f$ is {\bf absolutely continuous} if for every $\epsilon>0$ there
is a $\delta>0$ such that $\sum_{i=1}^n|f(b_i)-f(a_i)|\le\epsilon$
whenever $a\le a_1\le b_1\le a_2\le b_2\le\ldots\le a_n\le b_n\le b$ and
$\sum_{i=1}^nb_i-a_i\le\delta$.

\cmmnt{\medskip

\noindent{\bf Remark} The phrase `absolutely continuous' is used in
various senses in measure theory, closely related (if you look at them
in the right way) but not identical;  you will need to keep the context
of each definition in clear focus.
}%end of comment

\leader{225C}{Proposition} Let $[a,b]$ be a non-empty closed interval in
$\Bbb R$.

(a) If $f:[a,b]\to\Bbb R$ is absolutely continuous, it is uniformly
continuous.

(b) If $f:[a,b]\to\Bbb R$ is absolutely continuous it is of bounded
variation on $[a,b]$, so is differentiable almost everywhere in $[a,b]$,
and its derivative is integrable over $[a,b]$.

(c) If $f$, $g:[a,b]\to\Bbb R$ are absolutely continuous, so are
$f+g$ and $cf$, for every $c\in\Bbb R$.

(d) If $f$, $g:[a,b]\to\Bbb R$ are absolutely continuous so is $f\times
g$.

(e) If $g:[a,b]\to [c,d]$ and $f:[c,d]\to\Bbb R$ are absolutely
continuous, and $g$ is non-decreasing, then the composition
$fg:[a,b]\to\Bbb R$ is absolutely continuous.

\proof{{\bf (a)} Let $\epsilon>0$.   Then there is a $\delta>0$ such
that $\sum_{i=1}^n|f(b_i)-f(a_i)|\le\epsilon$
whenever $a\le a_1\le b_1\le a_2\le b_2\le\ldots\le a_n\le b_n\le b$ and
$\sum_{i=1}^nb_i-a_i\le\delta$;  but of course now
$|f(y)-f(x)|\le\epsilon$ whenever $x$, $y\in[a,b]$ and
$|x-y|\le\delta$.   As $\epsilon$ is arbitrary, $f$ is uniformly
continuous.



\medskip

{\bf (b)} Let $\delta>0$ be such that
$\sum_{i=1}^n|f(b_i)-f(a_i)|\le 1$
whenever $a\le a_1\le b_1\le a_2\le b_2\le\ldots\le a_n\le b_n\le b$ and
$\sum_{i=1}^nb_i-a_i\le\delta$.   If
$a\le c=c_0\le c_1\le\ldots\le c_n\le d\le\min(b,c+\delta)$, then
$\sum_{i=1}^n|f(c_i)-f(c_{i-1})|\le 1$, so $\Var_{[c,d]}(f)\le 1$;
accordingly (inducing on $k$, using
224Cc for the inductive step) $\Var_{[a,\min(a+k\delta,b)]}(f)\le k$
for every $k$, and

\Centerline{$\Var_{[a,b]}(f)\le\lceil(b-a)/\delta\rceil<\infty$.}

It follows that $f'$ is integrable, by 224I.

\medskip

{\bf (c)(i)} Let $\epsilon>0$.   Then there are $\delta_1$,
$\delta_2>0$ such that

\Centerline{$\sum_{i=1}^n|f(b_i)-f(a_i)|\le\Bover12\epsilon$}

\noindent whenever
$a\le a_1\le b_1\le a_2\le b_2\le\ldots\le a_n\le b_n\le b$ and
$\sum_{i=1}^nb_i-a_i\le\delta_1$,

\Centerline{$\sum_{i=1}^n|g(b_i)-g(a_i)|\le\Bover12\epsilon$}

\noindent whenever $a\le a_1\le b_1\le a_2\le b_2\le\ldots\le a_n\le
b_n\le b$ and
$\sum_{i=1}^nb_i-a_i\le\delta_2$.   Set
$\delta=\min(\delta_1,\delta_2)>0$,
and suppose that $a\le a_1\le b_1\le a_2\le b_2\le\ldots\le a_n\le
b_n\le b$ and
$\sum_{i=1}^nb_i-a_i\le\delta$.   Then


\Centerline{$\sum_{i=1}^n|(f+g)(b_i)-(f+g)(a_i)|\le
\sum_{i=1}^n|f(b_i)-f(a_i)|+\sum_{i=1}^n|g(b_i)-g(a_i)|\le\epsilon$.}

\noindent As $\epsilon$ is arbitrary, $f+g$ is absolutely
continuous.

\medskip

\quad{\bf (ii)} Let $\epsilon>0$.   Then there is a
$\delta>0$ such that

\Centerline{$\sum_{i=1}^n|f(b_i)-f(a_i)|\le\Bover{\epsilon}{1+|c|}$}

\noindent whenever
$a\le a_1\le b_1\le a_2\le b_2\le\ldots\le a_n\le b_n\le b$ and
$\sum_{i=1}^nb_i-a_i\le\delta$.   Now

\Centerline{$\sum_{i=1}^n|(cf)(b_i)-(cf)(a_i)|\le\epsilon$}

\noindent whenever
$a\le a_1\le b_1\le a_2\le b_2\le\ldots\le a_n\le b_n\le b$ and
$\sum_{i=1}^nb_i-a_i\le\delta$.   As $\epsilon$ is arbitrary, $cf$ is
absolutely continuous.

\medskip

{\bf (d)} By either (a) or (b), $f$ and $g$ are bounded;  set
$M=\sup_{x\in[a,b]}|f(x)|$, $M'=\sup_{x\in[a,b]}|g(x)|$.   Let
$\epsilon>0$.   Then there are $\delta_1$, $\delta_2>0$ such that

\inset{$\sum_{i=1}^n|f(b_i)-f(a_i)|\le\epsilon$
whenever $a\le a_1\le b_1\le a_2\le b_2\le\ldots\le a_n\le b_n\le b$ and
$\sum_{i=1}^nb_i-a_i\le\delta_1$,}

\inset{$\sum_{i=1}^n|g(b_i)-g(a_i)|\le\epsilon$
whenever $a\le a_1\le b_1\le a_2\le b_2\le\ldots\le a_n\le b_n\le b$ and
$\sum_{i=1}^nb_i-a_i\le\delta_2$.}

\noindent Set $\delta=\min(\delta_1,\delta_2)>0$ and suppose that $a\le
a_1\le b_1\le\ldots\le b_n\le b$ and $\sum_{i=1}^nb_i-a_i\le\delta$.
Then

$$\eqalign{\sum_{i=1}^n|f(b_i)g(b_i)-f(a_i)g(a_i)|
&=\sum_{i=1}^n|(f(b_i)-f(a_i))g(b_i)+f(a_i)(g(b_i)-g(a_i))|\cr
&\le\sum_{i=1}^n|f(b_i)-f(a_i)||g(b_i)|+|f(a_i)||g(b_i)-g(a_i)|\cr
&\le\sum_{i=1}^n|f(b_i)-f(a_i)|M'+M|g(b_i)-g(a_i)|\cr
&\le\epsilon M'+M\epsilon
=\epsilon(M+M').\cr}$$

\noindent As $\epsilon$ is arbitrary, $f\times g$ is absolutely
continuous.

\medskip

{\bf (e)} Let $\epsilon>0$.   Then there is a $\delta>0$ such that
$\sum_{i=1}^n|f(d_i)-f(c_i)|\le\epsilon$ whenever $c\le c_1\le
d_1\le\ldots\le c_n\le d_n\le d$ and $\sum_{i=1}^nd_i-c_i\le\delta$;
and there is an $\eta>0$ such that
$\sum_{i=1}^n|g(b_i)-g(a_i)|\le\delta$ whenever $a\le a_1\le
b_1\le\ldots\le a_n\le b_n\le b$ and $\sum_{i=1}^nb_i-a_i\le\eta$.   Now
suppose that $a\le a_1\le b_1\le\ldots\le a_n\le b_n\le b$ and
$\sum_{i=1}^nb_i-a_i\le\eta$.   Because $g$ is non-decreasing, we have
$c\le g(a_1)\le\ldots\le g(b_n)\le d$ and
$\sum_{i=1}^ng(b_i)-g(a_i)\le\delta$, so $\sum_{i=1}^n|f(g(b_i))
-f(g(a_i))|\le\epsilon$;  as $\epsilon$ is arbitrary, $fg$ is absolutely
continuous.
}%end of proof of 225C

\leader{225D}{Lemma} Let $[a,b]$ be a non-empty closed interval in
$\Bbb R$ and $f:[a,b]\to\Bbb R$ an absolutely continuous function which
has zero derivative almost everywhere in $[a,b]$.
Then $f$ is constant on $[a,b]$.

\proof{ Let $x\in[a,b]$, $\epsilon>0$.   Let $\delta>0$ be
such that $\sum_{i=1}^n|f(b_i)-f(a_i)|\le\epsilon$
whenever $a\le a_1\le b_1\le a_2\le b_2\le\ldots\le a_n\le b_n\le b$ and
$\sum_{i=1}^nb_i-a_i\le\delta$.   Set $A=\{t:a<t<x,\,f'(t)$ exists
$=0\}$;  then $\mu A=x-a$, writing $\mu$ for Lebesgue measure.
Let $\Cal I$ be the set of non-empty non-singleton closed intervals
$[c,d]\subseteq[a,x]$ such that $|f(d)-f(c)|\le\epsilon (d-c)$;  then
every member of $A$ belongs to arbitrarily short members of $\Cal I$.
By Vitali's theorem (221A), there is a countable disjoint family
$\Cal I_0\subseteq\Cal I$ such that $\mu(A\setminus\bigcup\Cal I_0)=0$,
that is,

\Centerline{$x-a=\mu(\bigcup\Cal I_0)=\sum_{I\in\Cal I_0}\mu I$.}

\noindent Now there is a finite $\Cal I_1\subseteq\Cal I_0$ such that

\Centerline{$\mu(\bigcup\Cal I_1)
=\sum_{I\in\Cal I_1}\mu I\ge x-a-\delta$.}

\noindent If $\Cal I_1=\emptyset$, then $x\le a+\delta$ and
$|f(x)-f(a)|\le\epsilon$.   Otherwise, express $\Cal I_1$ as
$\{[c_0,d_0],\ldots,[c_n,d_n]\}$, where
$a\le c_0<d_0<c_1<d_1<\ldots<c_n<d_n\le x$.   Then

\Centerline{$(c_0-a)+\sum_{i=1}^n(c_i-d_{i-1})
+(x-d_n)=\mu([a,x]\setminus\bigcup\Cal I_1)\le\delta$,}

\noindent so

\Centerline{$|f(c_0)-f(a)|+\sum_{i=1}^n|f(c_i)-f(d_{i-1})|+|f(x)-f(d_n)|
\le\epsilon$.}

\noindent On the other hand, $|f(d_i)-f(c_i)|\le\epsilon(d_i-c_i)$ for
each $i$, so

\Centerline{$\sum_{i=0}^n|f(d_i)-f(c_i)|\le\epsilon\sum_{i=0}^nd_i-c_i
\le\epsilon(x-a)$.}

\noindent Putting these together,

$$\eqalign{|f(x)-f(a)|
&\le|f(c_0)-f(a)|+|f(d_0)-f(c_0)|+|f(c_1)-f(d_0)|+\ldots\cr
&\hskip10em+|f(d_n)-f(c_n)|+|f(x)-f(d_n)|\cr
&=|f(c_0)-f(a)|+\sum_{i=1}^n|f(c_i)-f(d_{i-1})|\cr
&\hskip10em+|f(x)-f(d_n)|+\sum_{i=0}^n|f(d_i)-f(c_i)|\cr
&\le\epsilon+\epsilon(x-a)
=\epsilon(1+x-a).\cr}$$

\noindent As $\epsilon$ is arbitrary, $f(x)=f(a)$.   As $x$ is
arbitrary, $f$ is constant.
}%end of proof of 225D

\leader{225E}{Theorem}   Let $[a,b]$ be a non-empty closed interval in
$\Bbb R$ and $F:[a,b]\to\Bbb R$ a function.   Then the following are
equiveridical:

(i) there is an integrable real-valued function $f$ such that
$F(x)=F(a)+\int_a^xf$ for every $x\in[a,b]$;

(ii) $\int_a^xF'$ exists and is
equal to $F(x)-F(a)$ for every $x\in[a,b]$;

(iii) $F$ is absolutely continuous.

\cmmnt{\medskip

\noindent{\bf Remark} Here, and for the rest of the section (except in
225Oa), integrals will
be taken with respect to Lebesgue measure on $\Bbb R$.
}

\proof{{\bf (i)$\Rightarrow$(iii)}  Assume (i).
Let $\epsilon>0$.   By 225A, there is a
$\delta>0$ such that $\int_H|f|\le\epsilon$ whenever $H\subseteq[a,b]$
and $\mu H\le\delta$, writing $\mu$ for Lebesgue measure as usual.   Now
suppose that $a\le a_1\le b_1\le a_2\le b_2\le\ldots\le a_n\le b_n\le b$
and $\sum_{i=1}^nb_i-a_i\le\delta$.   Consider $H=\bigcup_{1\le i\le
n}\coint{a_i,b_i}$.   Then $\mu H\le\delta$ and

\Centerline{$\sum_{i=1}^n|F(b_i)-F(a_i)|
=\sum_{i=1}^n|\int_{\coint{a_i,b_i}}f|
\le\sum_{i=1}^n\int_{\coint{a_i,b_i}}|f|
=\int_F|f|\le\epsilon$.}

\noindent   As
$\epsilon$ is arbitrary, $F$ is absolutely continuous.

\medskip

{\bf (iii)$\Rightarrow$(ii)} If $F$ is absolutely continuous, then it is
of bounded variation (225Cb),  so $\int_a^bF'$ exists (224I).   Set
$G(x)=\int_a^xF'$ for $x\in [a,b]$;
then $G'\eae F'$ (222E) and $G$ is absolutely continuous (by
(i)$\Rightarrow$(iii) just proved).   Accordingly $G-F$ is absolutely
continuous (225Cc) and is differentiable, with zero derivative, almost
everywhere.   It follows that $G-F$ must be constant (225D).   But as
$G(a)=0$, $G=F-F(a)$;  just as required by (ii).

\medskip

{\bf (ii)$\Rightarrow$(i)} is trivial.
}%end of proof of 225E

\leader{225F}{Integration by \dvrocolon{parts}}\cmmnt{ As an
application of this
result, I give a justification of a familiar formula.

\medskip

\noindent}{\bf Theorem} Let $f$ be a real-valued function which is
integrable over
an interval $[a,b]\subseteq\Bbb R$, and $g:[a,b]\to\Bbb R$ an absolutely
continuous function.   Suppose that $F$ is an indefinite integral of
$f$, so that $F(x)-F(a)=\int_a^xf$ for $x\in[a,b]$.   Then

\Centerline{$\int_a^bf\times g
=F(b)g(b) - F(a)g(a) -\int_a^bF\times g'$.}

\proof{ Set $h=F\times g$.   Because $F$ is absolutely continuous
(225E), so is $h$ (225Cd).   Consequently
$h(b)-h(a)=\int_a^bh'$, by (iii)$\Rightarrow$(ii) of 225E.   But
$h'=F'\times g+F\times g'$ wherever $F'$ and $g'$ are defined, which is
almost everywhere, and $F'\eae f$, by 222E again;  so
$h'\eae f\times g+F\times g'$.   Finally, $g$ and $F$ are continuous, therefore measurable,
and bounded, while $f$ and $g'$ are integrable (using 225E yet again),
so $f\times g$ and $F\times g'$ are integrable, and

\Centerline{$F(b)g(b)-F(a)g(a)
=h(b)-h(a)
=\int_a^bh'
=\int_a^bf\times g+\int_a^bF\times g'$,}

\noindent as required.
}%end of proof of 225F

\leader{225G}{}\cmmnt{ I come now to a group of results at a rather
deeper level than most of the work of this chapter, being closer to the
ideas of Chapter 26.

\medskip

\noindent}{\bf Proposition} Let $[a,b]$ be a non-empty closed interval
in $\Bbb R$ and $f:[a,b]\to\Bbb R$ an absolutely continuous function.

(a) $f[A]$ is negligible for every negligible set $A\subseteq\Bbb R$.

(b)\dvAnew{2013}
$f[E]$ is measurable for every measurable set $E\subseteq\Bbb R$.

\proof{{\bf (a)} Let $\epsilon>0$.   Then there is a $\delta>0$ such that
$\sum_{i=1}^n|f(b_i)-f(a_i)|\le\epsilon$ whenever
$a\le a_1\le b_1\ldots\le a_n\le b_n\le b$ and
$\sum_{i=1}^nb_i-a_i\le\delta$.   Now
there is a sequence $\sequence{k}{I_k}$ of closed intervals, covering
$A$, with $\sum_{k=0}^{\infty}\mu I_k\le\delta$.   For each
$m\in\Bbb N$, let $F_m$ be $[a,b]\cap\bigcup_{k\le m}I_k$.   Then
$\mu f[F_m]\le\epsilon$.   \Prf\  $F_m$ must be expressible as
$\bigcup_{i\le n}[c_i,d_i]$ where $n\le m$ and
$a\le c_0\le d_0\le\ldots\le c_n\le d_n\le b$.
For each $i\le n$ choose $x_i$, $y_i$ such that
$c_i\le x_i$, $y_i\le d_i$ and

\Centerline{$f(x_i)=\min_{x\in[c_i,d_i]}f(x)$,\quad
$f(y_i)=\max_{x\in[c_i,d_i]}f(x)$;}

\noindent such exist because $f$ is continuous (225Ca), so is bounded and
attains its bounds on $[c_i,d_i]$.   Set $a_i=\min(x_i,y_i)$,
$b_i=\max(x_i,y_i)$, so that $c_i\le a_i\le b_i\le d_i$.   Then

\Centerline{$\sum_{i=0}^nb_i-a_i\le\sum_{i=0}^nd_i-c_i=\mu F_m
\le\mu(\bigcup_{k\in\Bbb N}I_k)\le\delta$,}

\noindent so

$$\eqalign{\mu f[F_m]
&=\mu(\bigcup_{i\le m}f[\,[c_i,d_i]\,])
\le\sum_{i=0}^n\mu(f[\,[c_i,d_i]\,])\cr
&=\sum_{i=0}^n\mu[f(x_i),f(y_i)]
=\sum_{i=0}^n|f(b_i)-f(a_i)|
\le\epsilon.  \text{ \Qed}\cr}$$

\noindent But $\sequence{m}{f[F_m]}$ is a non-decreasing sequence
covering $f[A]$, so

\Centerline{$\mu^*f[A]\le\mu(\bigcup_{m\in\Bbb N}f[F_m])
=\sup_{m\in\Bbb N}\mu f[F_m]\le\epsilon$.}

\noindent As $\epsilon$ is arbitrary, $f[A]$ is negligible, as claimed.

\medskip

{\bf (b)} By 134Fb, there is a sequence $\sequencen{F_n}$ of closed subsets
of $E\cap[a,b]$ such that $\lim_{n\to\infty}\mu F_n=\mu(E\cap[a,b])$.
For each $n$, $F_n$ is closed and bounded, therefore compact (2A2F);  as
$f$ is continuous, $f[F_n]$ is compact (2A2Eb), therefore closed
(2A2F, in the other direction) and measurable (114G).   Next,
setting $A=E\cap[a,b]\setminus\bigcup_{n\in\Bbb N}F_n$,
$A$ is negligible, so $f[A]$ is negligible, by (a) here,
therefore measurable.   Consequently

\Centerline{$f[E]=f[E\cap[a,b]]=f[\bigcup_{n\in\Bbb N}F_n\cup A]
=\bigcup_{n\in\Bbb N}f[F_n]\cup f[A]$}

\noindent is measurable, as claimed.
}%end of proof of 225G

\leader{225H}{Semi-continuous functions}\cmmnt{ In preparation for the
last main result of this section, I give a general result concerning
measurable real-valued functions on subsets of $\Bbb R$.   It will be
convenient here, for once, to consider functions taking values in
$[-\infty,\infty]$.}   If $D\subseteq\BbbR^r$, a function
$g:D\to[-\infty,\infty]$ is {\bf lower
semi-continuous} if $\{x:g(x)>u\}$ is an open subset of
$D$\cmmnt{ (for the subspace topology, see 2A3C)} for
every $u\in[-\infty,\infty]$.   Any lower semi-continuous function is
Borel measurable, therefore Lebesgue
measurable\cmmnt{ (121B-121D)}. %121B 121C 121D
\cmmnt{Now we have the following result.}

\leader{225I}{Proposition}  Suppose that $r\ge 1$ and that $f$ is a
real-valued function, defined on
a subset $D$ of $\BbbR^r$, which is integrable over $D$.      Then for
any $\epsilon>0$ there is a lower semi-continuous function
$g:\BbbR^r\to[-\infty,\infty]$ such that $g(x)\ge f(x)$ for every
$x\in D$ and $\int_Dg$ is defined and not greater than
$\epsilon+\int_Df$.

\cmmnt{\medskip

\noindent{\bf Remarks} This is a result of great general importance, so
I give it in a fairly general form;  but for the present chapter all we
need is the case $r=1$, $D=[a,b]$ where $a\le b$.
}%end of comment

\proof{{\bf (a)} We can enumerate $\Bbb Q$ as $\sequencen{q_n}$.    By
225A, there is a $\delta>0$ such that $\int_F|f|\le\bover12{\epsilon}$
whenever $\mu_DF\le\delta$, where $\mu_D$ is the subspace measure on
$D$, so that $\mu_DF=\mu^*F$, the outer Lebesgue measure of $F$, for
every $F\in\Sigma_D$, the domain of $\mu_D$ (214A-214B).     For each
$n\in\Bbb N$, set

\Centerline{$\delta_n=2^{-n-1}\min(\Bover{\epsilon}{1+2|q_n|},\delta)$,}

\noindent so that
$\sum_{n=0}^{\infty}\delta_n|q_n|\le\bover12{\epsilon}$ and
$\sum_{n=0}^{\infty}\delta_n\le\delta$.   For each $n\in\Bbb N$, let
$E_n\subseteq\BbbR^r$ be a Lebesgue measurable set such that
$\{x:f(x)\ge q_n\}=D\cap E_n$, and choose an open set
$G_n\supseteq E_n\cap B(\tbf{0},n)$ such that
$\mu G_n\le\mu(E_n\cap B(\tbf{0},n))+\delta_n$ (134Fa), writing
$B(\tbf{0},n)$ for the ball
$\{x:\|x\|\le n\}$.   For $x\in\BbbR^r$, set

\Centerline{$g(x)=\sup\{q_n:x\in G_n\}$,}

\noindent allowing $-\infty$ as $\sup\emptyset$ and $\infty$ as the
supremum of a set with no upper bound in $\Bbb R$.

\medskip

{\bf (b)} Now check the properties of $g$.

\medskip

\quad{\bf (i)} $g$ is lower semi-continuous.   \Prf\ If
$u\in[-\infty,\infty]$, then

\Centerline{$\{x:g(x)>u\}=\bigcup\{G_n:q_n>u\}$}

\noindent is a union of open sets, therefore open.\ \Qed

\medskip

\quad{\bf (ii)} $g(x)\ge f(x)$ for every $x\in D$.   \Prf\ If
$x\in D$ and $\eta>0$, there is an $n\in\Bbb N$ such that
$\|x\|\le n$ and $f(x)-\eta\le q_n\le f(x)$;   now
$x\in E_n\subseteq G_n$
so $g(x)\ge q_n\ge f(x)-\eta$.   As $\eta$ is arbitrary,
$g(x)\ge f(x)$.\ \Qed

\medskip

\quad{\bf (iii)}  Consider the functions
$h_1$, $h_2:D\to\ocint{-\infty,\infty}$ defined by setting

$$\eqalign{h_1(x)
&=|f(x)|\text{ if }x\in D\cap\bigcup_{n\in\Bbb N}(G_n\setminus E_n),\cr
&=0\text{ for other }x\in D,\cr
h_2(x)&=\sum_{n=0}^{\infty}|q_n|\chi(G_n\setminus E_n)(x)
\text{ for every }x\in D.\cr}$$

\noindent Setting $F=\bigcup_{n\in\Bbb N}G_n\setminus E_n$,

\Centerline{$\mu F\le\sum_{n=0}^{\infty}\mu(G_n\setminus E_n)
\le\delta$,}

\noindent so

\Centerline{$\int_Dh_1
=\int_{D\cap F}|f|\le\Bover12\epsilon$}

\noindent by the choice of $\delta$.   As for $h_2$, we have
(by B.Levi's theorem)

\Centerline{$\int_Dh_2
=\sum_{n=0}^{\infty}|q_n|\mu_D(D\cap G_n\setminus F_n)
\le\sum_{n=0}^{\infty}|q_n|\mu(G_n\setminus F_n)
\le\Bover12\epsilon$}

\noindent -- because this is finite, $h_2(x)<\infty$ for almost every
$x\in D$.   Thus $\int_Dh_1+h_2\le\epsilon$.

\medskip

\quad{\bf (iv)} The point is that $g\le f+h_1+h_2$ everywhere in
$D$.   \Prf\ Take any $x\in D$.   If $n\in\Bbb N$
and $x\in G_n$, then either $x\in E_n$, in which case

\Centerline{$f(x)+h_1(x)+h_2(x)\ge f(x)\ge q_n$,}

\noindent or $x\in G_n\setminus E_n$, in which case

\Centerline{$f(x)+h_1(x)+h_2(x)\ge f(x)+|f(x)|+|q_n|\ge q_n$.}

\noindent Thus

\Centerline{$f(x)+h_1(x)+h_2(x)\ge\sup\{q_n:x\in G_n\}\ge g(x)$.   \Qed}

\noindent So $g\le f+h_1+h_2$ everywhere in $D$.

\medskip

\quad{\bf (v)} Putting (iii) and (iv) together,

\Centerline{$\int_Dg\le\int_Df+h_1+h_2\le\epsilon+\int_Df$,}

\noindent as required.
}%end of proof of 225I


\leader{225J}{}\cmmnt{ We need some results on Borel measurable sets
and functions which are of independent interest.

\medskip

\noindent}{\bf Theorem} Let $D$ be a subset of $\Bbb R$ and
$f:D\to\Bbb R$ any function.   Then

\Centerline{$E=\{x:x\in D,\,f$ is continuous at $x\}$}

\noindent is relatively Borel measurable in $D$, and

\Centerline{$F=\{x:x\in D,\,f$ is differentiable at $x\}$}

\noindent is Borel measurable in $\Bbb R$;  moreover, $f':F\to\Bbb R$ is
Borel measurable.

\wheader{225J}{0}{0}{0}{36pt}

\proof{{\bf (a)} For $k\in\Bbb N$ set

\Centerline{$\Cal G_k=\{\ooint{a,b}:a,\,b\in\Bbb R,\,
|f(x)-f(y)|\le 2^{-k}$ for all $x$, $y\in D\cap\ooint{a,b}\}$.}

\noindent Then $G_k=\bigcup\Cal G_k$ is an open set, so
$E_0=\bigcap_{k\in\Bbb N}G_k$ is a Borel set.   But $E=D\cap E_0$, so
$E$ is a relatively Borel subset of $D$.

\medskip

{\bf (b)(i)} I should perhaps say at once that when interpreting the
formula $f'(x)=\lim_{h\to 0}(f(x+h)-f(x))/h$, I insist on the
restrictive definition

\Centerline{$a=\lim_{h\to 0}\bover{f(x+h)-f(x)}{h}$}

\noindent if

\qquad\quad for every $\epsilon>0$ there is a $\delta>0$ such that
$\Bover{f(x+h)-f(x)}{h}$ is defined and

\hfill$|\Bover{f(x+h)-f(x)}{h}-a|\le\epsilon$ whenever
$0<|h|\le\delta$.\qquad\qquad\quad

\noindent So $f'(x)$ can be defined only if there is some $\delta>0$
such that the whole interval $[x-\delta,x+\delta]$ lies within the
domain $D$ of $f$.

\medskip

\quad{\bf (ii)} For $p$, $q$, $q'\in\Bbb Q$ and $k\in\Bbb N$ set

$$\eqalign{H(k,p,q,q')
&=\{x:x\in E\cap\ooint{q,q'},\,|f(y)-f(x)-p(y-x)|\le 2^{-k}|y-x|
\text{ for every }y\in\ooint{q,q'}\}\cr&
\mskip300mu\text{ if }\ooint{q,q'}\subseteq D\cr
&=\emptyset\text{ otherwise}.\cr}$$
%display works in smallprint version

\noindent Then
$H(k,p,q,q')=E\cap\ooint{q,q'}\cap\overline{H(k,p,q,q')}$.   \Prf\ If
$x\in E\cap\ooint{q,q'}\cap\overline{H(k,p,q,q')}$ there is a sequence
$\sequencen{x_n}$ in $H(k,p,q,q')$ converging to $x$.
Because $f$ is continuous at $x$,

$$\eqalign{|f(y)-f(x)-p(y-x)|
&=\lim_{n\to\infty}|f(y)-f(x_n)-p(y-x_n)|\cr&
\le 2^{-k}\lim_{n\to\infty}2^{-k}|y-x_n|
=2^{-k}|y-x|\cr}$$

\noindent for every $y\in\ooint{q,q'}$, so that
$x\in H(k,p,q,q')$.\ \QeD\   Consequently $H(k,p,q,q')$ is a Borel set.
\Prf\ There is a Borel set $E_0$ such that $E=E_0\cap D$, by (a), so that
if $\ooint{q,q'}\subseteq D$ then

\Centerline{$H(k,p,q,q')=E\cap\ooint{q,q'}\cap\overline{H(k,p,q,q')}
=E_0\cap\ooint{q,q'}\cap\overline{H(k,p,q,q')}$}

\noindent is Borel.   Otherwise, of course, $H(k,p,q,q')$ is Borel because
it is empty.\ \Qed

\medskip

\quad{\bf (iii)} Now

\Centerline{$F
=\bigcap_{k\in\Bbb N}\bigcup_{p,q,q'\in\Bbb Q}H(k,p,q,q')$.}

\noindent \Prf\ ($\alpha$) Suppose $x\in F$, that is, $f'(x)$ is
defined;  say $f'(x)=a$.   Take any $k\in\Bbb N$.   Then there are
$p\in\Bbb Q$, $\delta>0$ such that
$|p-a|\le 2^{-k-1}$ and $[x-\delta,x+\delta]\subseteq D$ and
$|\bover{f(x+h)-f(x)}{h}-a|\le 2^{-k-1}$ whenever $0<|h|\le\delta$;
now take
$q\in\Bbb Q\cap\coint{x-\delta,x}$, $q'\in\Bbb Q\cap\ocint{x,x+\delta}$
and see that $x\in H(k,p,q,q')$.   As $x$ is arbitrary,
$F\subseteq\bigcap_{k\in\Bbb N}\bigcup_{p,q,q'\in\Bbb Q}H(k,p,q,q')$.
($\beta$) If
$x\in\bigcap_{k\in\Bbb N}\bigcup_{p,q,q'\in\Bbb Q}H(k,p,q,q')$, then for
each $k\in\Bbb N$ choose $p_k$, $q_k$,
$q'_k\in\Bbb Q$ such that $x\in H(k,p_k,q_k,q'_k)$.   If $h\ne 0$,
$x+h\in\ooint{q_k,q'_k}$ then $|\bover{f(x+h)-f(x)}{h}-p_k|\le 2^{-k}$.
But this means, first, that $|p_k-p_l|\le 2^{-k}+2^{-l}$ for every $k$,
$l$ (since surely there is some $h\ne 0$ such that
$x+h\in\ooint{q_k,q'_k}\cap\ooint{q_l,q'_l}$), so that
$\sequence{k}{p_k}$ is a Cauchy sequence, with limit $a$ say;  and,
second, that $|\bover{f(x+h)-f(x)}{h}-a|\le 2^{-k}+|a-p_k|$ whenever
$h\ne 0$ and $x+h\in\ooint{q_k,q'_k}$, so that $f'(x)$ is defined and
equal to $a$.\ \Qed

\medskip

\quad{\bf (iv)} Because $\Bbb Q$ is countable, all the unions
$\bigcup_{p,q,q'\in\Bbb Q}H(k,p,q,q')$ are Borel sets, so $F$ also is.

\medskip

\quad{\bf (v)} Now enumerate $\BbbQ^3$ as
$\sequence{i}{(p_i,q_i,q'_i)}$, and set
$H'_{ki}=H(k,p_i,q_i,q'_i)\setminus\bigcup_{j<i}H(k,p_j,q_j,q'_j)$ for
each $k$, $i\in\Bbb N$.   Every $H'_{ki}$ is Borel measurable,
$\sequence{i}{H'_{ki}}$ is disjoint, and

\Centerline{$\bigcup_{i\in\Bbb N}H'_{ki}
=\bigcup_{i\in\Bbb N}H(k,p_i,q_i,q'_i)\supseteq F$}

\noindent for each $k$.   Note that
$|f'(x)-p|\le 2^{-k}$ whenever $x\in F\cap H(k,p,q,q')$, so if we set
$f_k(x)=p_i$ for every $x\in H'_{ki}$ we shall have a Borel
measurable function
$f_k$ such that $|f(x)-f_k(x)|\le 2^{-k}$ for every $x\in F$.
Accordingly $f'=\lim_{k\to\infty}f_k\restr F$ is Borel measurable.
}%end of proof of 225J

\leader{225K}{Proposition} Let $[a,b]$ be a non-empty closed interval in
$\Bbb R$, and $f:[a,b]\to\Bbb R$ a function.   Set
$F=\{x:x\in\ooint{a,b},\,f'(x)$ is defined$\}$.   Then $f$ is absolutely
continuous iff (i) $f$ is continuous (ii) $f'$ is integrable over $F$
(iii) $f[\,[a,b]\setminus F]$ is negligible.

\proof{{\bf (a)} Suppose first that $f$ is absolutely continuous.   Then
$f$ is surely continuous (225Ca) and $f'$ is integrable over $[a,b]$,
therefore over $F$ (225E);  also $[a,b]\setminus F$ is negligible, so
$f[\,[a,b]\setminus F]$ is negligible, by 225G.

\medskip

{\bf (b)} So now suppose that $f$ satisfies the conditions.   Set
$f^*(x)=|f'(x)|$ for $x\in F$, $0$ for
$x\in[a,b]\setminus F$.   Then $f(b)\le f(a)+\int_a^bf^*$.

\medskip

\Prf\ {\bf (i)} Because $F$ is a Borel set and $f'$ is a Borel
measurable function (225J), $f^*$ is measurable.   Let $\epsilon>0$.
Let $G$ be an open subset of
$\Bbb R$ such that $f[\,[a,b]\setminus F]\subseteq G$ and
$\mu G\le\epsilon$ (134Fa again).   Let $g:\Bbb R\to[0,\infty]$ be a lower
semi-continuous function such that $f^*(x)\le g(x)$ for every
$x\in[a,b]$ and $\int_a^bg\le\int_a^bf^*+\epsilon$ (225I).   Consider

\Centerline{$A=\{x:a\le x\le b,\,\mu([f(a),f(x)]\setminus G)
  \le 2\epsilon(x-a)+\int_a^xg\}$,}

\noindent interpreting $[f(a),f(x)]$ as $\emptyset$ if $f(x)<f(a)$.
Then $a\in A\subseteq[a,b]$, so $c=\sup A$ is defined and belongs to
$[a,b]$.

Because $f$ is continuous, the function
$x\mapsto\mu([f(a),f(x)]\setminus G)$ is continuous;  also
$x\mapsto 2\epsilon(x-a)+\int_a^xg$ is certainly continuous, so
$c\in A$.

\medskip

\quad{\bf (ii)} \Quer\ If $c\in F$, so that $f^*(c)=|f'(c)|$, then there
is a $\delta>0$ such that

\Centerline{$a\le c-\delta\le c+\delta\le b$,}

\Centerline{$g(x)\ge g(c)-\epsilon\ge|f'(c)|-\epsilon$ whenever
$|x-c|\le\delta$,}

\Centerline{$|\Bover{f(x)-f(c)}{x-c}-f'(c)|\le\epsilon$ whenever
$|x-c|\le\delta$.}

\noindent Consider $x=c+\delta$.   Then $c<x\le b$ and

$$\eqalignno{\mu([f(a),f(x)]\setminus G)
&\le\mu([f(a),f(c)]\setminus G)+|f(x)-f(c)|\cr
&\le2\epsilon(c-a)+\int_a^cg+\epsilon(x-c)+|f'(c)|(x-c)\cr
&\le2\epsilon(c-a)+\int_a^cg
         +\epsilon(x-c)+\int_c^x(g+\epsilon)\cr
\noalign{\noindent (because $g(t)\ge |f'(c)|-\epsilon$ whenever
$c\le t\le x$)}
&=2\epsilon(x-a)+\int_a^xg,\cr}$$

\noindent so $x\in A$;  but $c$ is supposed to be an upper bound of
$A$.\ \Bang

Thus $c\in[a,b]\setminus F$.

\medskip

\quad{\bf (iii)} \Quer\ Now suppose, if possible, that
$c<b$.   We know that $f(c)\in G$, so there is an $\eta>0$ such that
$[f(c)-\eta,f(c)+\eta]\subseteq G$;  now there is a $\delta>0$
such that $|f(x)-f(c)|\le\eta$ whenever $x\in[a,b]$ and
$|x-c|\le\delta$.   Set $x=\min(c+\delta,b)$;  then $c<x\le b$ and
$[f(c),f(x)]\subseteq G$, so

\Centerline{$\mu([f(a),f(x)]\setminus G)
=\mu([f(a),f(c)]\setminus G)
\le 2\epsilon(c-a)+\int_a^cg
\le 2\epsilon(x-a)+\int_a^xg$}

\noindent and once again $x\in A$, even though $x>\sup A$.\ \Bang

\medskip

\quad{\bf (iv)} We conclude that $c=b$, so that $b\in A$.   But this
means that

$$\eqalign{f(b)-f(a)
&\le\mu([f(a),f(b)])
\le\mu([f(a),f(b)]\setminus G)+\mu G\cr
&\le 2\epsilon(b-a)+\int_a^bg+\epsilon
\le 2\epsilon(b-a)+\int_a^bf^*+\epsilon+\epsilon\cr
&=2\epsilon(1+b-a)+\int_a^bf^*.\cr}$$

\noindent As $\epsilon$ is arbitrary, $f(b)-f(a)\le\int_a^bf^*$, as
claimed.\ \Qed

\medskip

{\bf (c)} Similarly, or applying (b) to $-f$,
$f(a)-f(b)\le\int_a^bf^*$, so that $|f(b)-f(a)|\le\int_a^bf^*$.

Of course the argument applies equally to any subinterval of $[a,b]$, so
$|f(d)-f(c)|\le\int_c^df^*$ whenever
$a\le c\le d\le b$.   Now let $\epsilon>0$.   By 225A once more, there
is a $\delta>0$ such that $\int_Ef^*\le\epsilon$ whenever
$E\subseteq[a,b]$ and $\mu E\le\delta$.   Suppose that
$a\le a_1\le b_1\le\ldots\le a_n\le b_n\le b$ and
$\sum_{i=1}^nb_i-a_i\le\delta$.   Then

\Centerline{$\sum_{i=1}^n|f(b_i)-f(a_i)|
\le\sum_{i=1}^n\int_{a_i}^{b_i}f^*
=\int_{\bigcupop_{i\le n}[a_i,b_i]}f^*
\le\epsilon$.}

\noindent So $f$ is absolutely continuous, as claimed.
}%end of proof of 225K

\leader{225L}{Corollary} Let $[a,b]$ be a non-empty closed interval in
$\Bbb R$.   Let $f:[a,b]\to\Bbb R$ be a continuous function which is
differentiable on the open interval $\ooint{a,b}$.   If its derivative
$f'$ is integrable over $[a,b]$, then $f$ is absolutely continuous, and
$f(b)-f(a)=\int_a^bf'$.

\proof{ $f[\,[a,b]\setminus F]=\{f(a),f(b)\}$ is surely negligible, so
$f$ is absolutely continuous, by 225K;  consequently
$f(b)-f(a)=\int_a^bf'$, by 225E.
}%end of proof of 225L

\leader{225M}{Corollary} Let $[a,b]$ be a non-empty closed interval in
$\Bbb R$, and $f:[a,b]\to\Bbb R$ a continuous function.   Then $f$ is
absolutely continuous iff it is continuous and of bounded variation and
$f[A]$ is negligible for every negligible $A\subseteq[a,b]$.

\proof{{\bf (a)} Suppose that $f$ is absolutely continuous.   By
225C(a-b) it is continuous and of bounded variation, and by 225G we have
$f[A]$ negligible for every negligible $A\subseteq [a,b]$.

\medskip

{\bf (b)} So now suppose that $f$ satisfies the conditions.   Set
$F=\{x:x\in\ooint{a,b},\,f'(x)$ is defined$\}$.   By 224I once more,
$[a,b]\setminus F$ is negligible, so $f[\,[a,b]\setminus F]$ is
negligible.   Moreover, also by 224I, $f'$ is integrable over $[a,b]$
or $F$.   So the conditions of 225K are satisfied and $f$ is absolutely
continuous.
}%end of proof of 225M

\leader{225N}{The Cantor function}\cmmnt{ I should mention the
standard example of a continuous function of bounded variation which is
not absolutely continuous.}   Let $C\subseteq[0,1]$ be the Cantor
set\cmmnt{ (134G)}.
Recall that the `Cantor function' is a non-decreasing continuous
function $f:[0,1]\to[0,1]$ such that $f'(x)$ is defined and equal to
zero for every $x\in[0,1]\setminus C$,
but $f(0)=0<1=f(1)$\cmmnt{ (134H)}.   \cmmnt{Of
course} $f$ is of bounded variation and not absolutely continuous.   $C$
is negligible and $f[C]=[0,1]$ is not.   If $x\in C$, then for every
$n\in\Bbb N$ there is an interval of length $3^{-n}$, containing $x$, on
which $f$ increases by $2^{-n}$;  so $f$ cannot be differentiable at
$x$, and the set $F=\dom f'$ of 225K is precisely $[0,1]\setminus C$, so
that $f[\,[0,1]\setminus F]=[0,1]$.

\vleader{48pt}{225O}{Complex-valued functions}\cmmnt{ As usual, I
spell out
the results above in the forms applicable to complex-valued functions.

\medskip

} {\bf (a)} Let
$(X,\Sigma,\mu)$ be any measure space and $f$ an integrable
complex-valued
function defined on a conegligible subset of $X$.
Then for any $\epsilon>0$ there are a measurable set
$E$ of finite measure and a real number
$\delta>0$ such that $\int_F|f|\le\epsilon$ whenever $F\in\Sigma$ and
$\mu(F\cap E)\le\delta$.  \prooflet{(Apply 225A to $|f|$.)}

\spheader 225Ob If $[a,b]$ is a non-empty
closed interval in $\Bbb R$ and $f:[a,b]\to\Bbb C$ is a function, we say
that $f$ is {\bf absolutely continuous} if for every $\epsilon>0$ there
is a $\delta>0$ such that $\sum_{i=1}^n|f(b_i)-f(a_i)|\le\epsilon$
whenever $a\le a_1\le b_1\le a_2\le b_2\le\ldots\le a_n\le b_n\le b$ and
$\sum_{i=1}^nb_i-a_i\le\delta$.   Observe that $f$ is absolutely
continuous iff its real and imaginary parts are both absolutely
continuous.

\spheader 225Oc Let $[a,b]$ be a non-empty closed interval in
$\Bbb R$.

\quad(i) If $f:[a,b]\to\Bbb C$ is absolutely continuous it is of bounded
variation on $[a,b]$, so is differentiable almost everywhere in $[a,b]$,
and its derivative is integrable over $[a,b]$.

\quad(ii) If $f$, $g:[a,b]\to\Bbb C$ are absolutely continuous, so are
$f+g$ and $\zeta f$, for any $\zeta\in\Bbb C$, and $f\times g$.

\quad(iii) If $g:[a,b]\to[c,d]$ is monotonic and absolutely continuous,
and
$f:[c,d]\to\Bbb C$ is absolutely continuous, then $fg:[a,b]\to\Bbb C$ is
absolutely continuous.

\spheader 225Od Let $[a,b]$ be a non-empty closed interval in
$\Bbb R$ and $F:[a,b]\to\Bbb C$ a function.   Then the following are
equiveridical:

\quad(i) there is an integrable complex-valued function $f$ such that
$F(x)=F(a)+\int_a^xf$ for every $x\in[a,b]$;

\quad(ii) $\int_a^xF'$ exists and is
equal to $F(x)-F(a)$ for every $x\in[a,b]$;

\quad(iii) $F$ is absolutely continuous.

\cmmnt{\noindent (Apply 225E to the real and imaginary parts of $F$.)}

\spheader 225Oe Let $f$ be an integrable complex-valued function
on an interval $[a,b]\subseteq\Bbb R$, and $g:[a,b]\to\Bbb C$ an
absolutely continuous function.   Set $F(x)=\int_a^xf$ for $x\in[a,b]$.
Then

\Centerline{$\int_a^bf\times g
=F(b)g(b) - F(a)g(a) -\int_a^bF\times g'$.}

\cmmnt{\noindent (Apply 225F to the real and imaginary parts of $f$
and $g$.)}

\spheader 225Of Let $f$ be a continuous complex-valued function
on a closed interval $[a,b]\subseteq\Bbb R$, and suppose that $f$ is
differentiable at every point of the open interval $\ooint{a,b}$, with
$f'$ integrable over $[a,b]$.   Then $f$ is absolutely continuous.
\cmmnt{(Apply 225L to the real and imaginary parts of $f$.)}

\cmmnt{\spheader 225Og For a result corresponding to 225M, see 264Yp.}

\exercises{
\leader{225X}{Basic exercises (a)} 
%\spheader 225Xa
Show directly from the definition in
225B (without appealing to
225E) that any absolutely continuous real-valued function on a closed
interval $[a,b]$ is expressible as the difference of non-decreasing
absolutely continuous functions.
%225B

\spheader 225Xb Show directly from the definition in 225B and
the Mean Value Theorem (without appealing to 225K) that if a function
$f$ is continuous on a closed interval $[a,b]$, differentiable on the
open interval $\ooint{a,b}$, and has bounded derivative in
$\ooint{a,b}$, then $f$ is absolutely continuous, so that
$f(x)=f(a)+\int_a^xf'$ for every $x\in[a,b]$.
%225E

\spheader 225Xc Show that if $f:[a,b]\to\Bbb R$ is absolutely
continuous, then $\Var f=\int_a^b|f'|$.   \Hint{put 224I and
225E together.}
%225E

\spheader 225Xd Let $g:\Bbb R\to\Bbb R$ be a
non-decreasing function which is absolutely continuous on every bounded
interval; let $\mu_g$ be the associated Lebesgue-Stieltjes measure
(114Xa), and $\Sigma_g$ its domain.   Show that $\int_Eg'=\mu_gE$  for
any $E\in\Sigma_g$, if we allow $\infty$ as a value of the
integral.   \Hint{start with intervals $E$.}
%225E

\spheader 225Xe Let $g:[a,b]\to\Bbb R$ be a non-decreasing
absolutely continuous function, and $f:[g(a),g(b)]\to\Bbb R$ a
continuous function.   Show that $\int_{g(a)}^{g(b)}f(t)dt=
\int_a^bf(g(t))g'(t)dt$.   \Hint{set $F(x)=\int_{g(a)}^xf$,
$G=Fg$ and consider $\int_a^bG'(t)dt$.   See also 263J.}
%225E

\spheader 225Xf Suppose that $I\subseteq\Bbb R$ is any
non-trivial interval (bounded or unbounded, open, closed or half-open,
but not empty or a singleton), and $f:I\to\Bbb R$ a function.   Show
that
$f$ is absolutely continuous on every closed bounded subinterval of $I$
iff there is a function $g$ such that $\int_a^bg=f(b)-f(a)$ whenever
$a\le b$ in $I$, and in this case $g$ is integrable iff $f$ is of
bounded variation on $I$.
%225E

\spheader 225Xg Show that
$\biggerint_0^1\Bover{\ln x}{x-1}dx=\sum_{n=1}^{\infty}\Bover1{n^2}$.
\Hint{use 225F to find $\int_0^1x^n\ln x\,dx$, and recall that
$\bover1{1-x}=\sum_{n=0}^{\infty}x^n$ for $0\le x<1$.}
%225F

\spheader 225Xh(i) Show that $\int_0^1t^adt$ is finite for
every $a>-1$.   (ii) Show that $\int_1^{\infty}t^ae^{-t}dt$ is finite
for every $a\in\Bbb R$.   \Hint{show that there is an $M$ such
that $t^a\le Me^{t/2}$ for $t\ge 1$.}   (iii) Show that
$\Gamma(a)=\int_0^{\infty}t^{a-1}e^{-t}dt$ is defined for every $a>0$.
(iv) Show that $\Gamma(a+1)=a\Gamma(a)$ for every $a>0$.   (v) Show that
$\Gamma(n+1)=n!$ for every $n\in\Bbb N$.

($\Gamma$ is of course the {\bf gamma function}.)
%225F

\spheader 225Xi Show that if $b>0$ then
$\int_0^{\infty}u^{b-1}e^{-u^2/2}du=2^{(b-2)/2}\Gamma(\bover{b}2)$.
\Hint{consider $f(t)=t^{(b-2)/2}e^{-t}$, $g(u)=u^2/2$ in 225Xe.}
%225Xe, 225Xh, 225F

\spheader 225Xj Suppose that $f$, $g$ are
lower semi-continuous functions, defined on subsets of $\BbbR^r$,
and taking values in $\ocint{-\infty,\infty}$.   (i) Show that $f+g$,
$f\wedge g$ and $f\vee g$ are lower semi-continuous, and that
$\alpha f$ is lower
semi-continuous for every $\alpha\ge 0$.   (ii) Show that if $f$ and $g$
are non-negative, then $f\times g$ is lower semi-continuous.
(iii) Show that if $f$ is non-negative and $g$ is continuous, then
$f\times g$ is lower semi-continuous.   (iv) Show that if $f$ is
non-decreasing then the composition $fg$ is lower semi-continuous.
%225H

\spheader 225Xk Let $A$ be a non-empty family of lower
semi-continuous functions defined on subsets of $\BbbR^r$ and taking
values in $[-\infty,\infty]$.   Set $g(x)=\sup\{f(x):f\in A,\,x\in\dom
f\}$ for $x\in
D=\bigcup_{f\in A}\dom f$.   Show that $g$ is lower
semi-continuous.
%225H

\spheader 225Xl Let $f:[a,b]\to\Bbb R$ be an absolutely continuous
function, where $a\le b$.   (i) Show that $|f|:[a,b]\to\Bbb R$ is
absolutely continuous.   (ii)  Show that $gf$ is absolutely continuous
whenever $g:f[\,[a,b]\,]\to\Bbb R$ is absolutely continuous and
$g'$ is bounded.   (iii) Show that if $g:[a,b]\to\Bbb R$ is absolutely
continuous and $\inf_{x\in[a,b]}|g(x)|>0$ then
$f/g$ is absolutely continuous.
%225K

\spheader 225Xm Suppose that $f:[a,b]\to\Bbb R$ is continuous, and
differentiable at all but countably many
points of $[a,b]$.   Show that $f$ is absolutely continuous iff it is of
bounded variation.
%225K

\spheader 225Xn Let $f:\coint{0,\infty}\to\Bbb C$ be a function
which is absolutely continuous on $[0,a]$ for every
$a\in\coint{0,\infty}$ and has Laplace transform
$F(s)=\int_0^{\infty}e^{-sx}f(x)dx$ defined on $\{s:\Real s>S\}$.
Suppose also that $\lim_{x\to\infty}e^{-Sx}f(x)=0$.   Show
that $f'$ has Laplace transform $sF(s)-f(0)$ defined whenever $\Real
s>S$.   ({\it Hint\/}:  show that

\Centerline{$f(x)e^{-sx}-f(0)
=\int_0^x\Bover{d}{dt}(f(t)e^{-st})dt$}

\noindent for every $x\ge 0$.)
%225O 225F

\leader{225Y}{Further exercises (a)}
%\spheader 225Ya
Show that the composition of two absolutely
continuous functions need not be absolutely continuous.   \Hint{224Xb.}
%225B

\spheader 225Yb Let $f:[a,b]\to\Bbb R$ be a continuous function, where
$a<b$.   Set $G=\{x:x\in\ooint{a,b},\,\exists\,y\in\ocint{x,b}$ such
that $f(x)<f(y)\}$.   Show that $G$ is open and is expressible as a
disjoint union of intervals $\ooint{c,d}$ where $f(c)\le f(d)$.   Use
this to prove 225D without calling on Vitali's theorem.
%225D 2A2I

\spheader 225Yc Let $f:[a,b]\to\Bbb R$ be a function of bounded
variation and $\gamma>0$.   Show that there is an absolutely continuous
function $g:[a,b]\to\Bbb R$ such that $|g'(x)|\le\gamma$ wherever the
derivative is defined and $\{x:x\in[a,b],\,f(x)\ne g(x)\}$ has measure
at most $\bover1{\gamma}\Var f$.   \Hint{reduce to the case of
non-decreasing $f$.   Apply 225Yb to the function
$x\mapsto f(x)-\gamma x$ and show that $\gamma\mu G\le\Var f$.
Set $g(x)=f(x)$ for $x\in\ooint{a,b}\setminus G$.}
%225Yb 225D

\spheader 225Yd\dvAnew{2013}
Let $f:\Bbb R\to\Bbb R$ be a function which is absolutely
continuous on every bounded interval.   Show that
$\Var f\le\bover12\Var f'+\int|f|$.
%225E mt22bits

\spheader 225Ye Let $f$ be a non-negative measurable real-valued
function defined on a subset $D$ of $\BbbR^r$, where $r\ge 1$.   Show
that for any $\epsilon>0$ there is a lower semi-continuous
function $g:\BbbR^r\to[-\infty,\infty]$ such that $g(x)\ge f(x)$ for
every $x\in D$ and $\int_Dg-f\le\epsilon$.
%225I

\spheader 225Yf Let $f$ be a measurable real-valued function
defined on a subset $D$ of $\BbbR^r$, where $r\ge 1$.   Show that for
any $\epsilon>0$ there is a lower semi-continuous function
$g:\BbbR^r\to[-\infty,\infty]$ such that $g(x)\ge f(x)$ for every
$x\in D$ and
$\mu^*\{x:x\in D,\,g(x)>f(x)\}\le\epsilon$.   \Hint{134Yd, 134Fb.}
%225I

\spheader 225Yg(i) Show that if $f$ is a Lebesgue measurable real function
then all its Dini derivates are Lebesgue measurable.
(ii) Show that if $f$ is a Borel measurable real function
then all its Dini derivates are Borel measurable.
%225J mt22bits
}%end of exercises

\endnotes{
\Notesheader{225} There is a good deal more to say about absolutely
continuous functions;  I will return to the topic in the next section
and in Chapter 26.   I shall rarely make direct use of the results
from 225H on in their full strengths,
but it seems to me that this kind of investigation is
necessary for any clear picture of the relationships between such
concepts as absolute continuity and bounded variation.    Of course, in
order to apply these results, we do need a store of simple kinds
of absolutely continuous function, differentiable functions with
bounded derivative forming the most important class (225Xb).   A larger
family of the same kind is the class of `Lipschitz' functions
(262Bc).

The definition of `absolutely continuous function' is ordinarily set
out for closed bounded intervals, as in 225B.   The point is that for
other intervals the simplest generalizations of this formulation do not
seem quite appropriate.   In 225Xf I try to suggest the kind of demands
one might make on functions defined on other types of interval.

I should remark that the real prize is still not quite within our grasp.
I have been able to give a reasonably satisfactory formulation of simple
integration by parts (225F), at least for bounded intervals -- a further
limiting process is necessary to deal with unbounded intervals.   But a
companion method from advanced calculus, integration by substitution,
remains elusive.   The best I think we can do at this point is 225Xe,
which insists on a continuous integrand $f$.   It is the case that the
result is valid for general integrable $f$, but there are some further
subtleties to be mastered on the way;  the necessary ideas are given in
the much more general results 235A and 263D below, and applied to the
one-dimensional case in 263J.

On the way to the characterization of absolutely continuous functions in
225K, I find myself calling on one of the fundamental relationships
between Lebesgue measure and the topology of $\BbbR^r$ (225I).   The
technique here can be adapted to give many variations of the result;
see 225Ye-225Yf.   If you have not seen semi-continuous functions
before, 225Xj-225Xk give a partial idea of their properties.    In 225J
I give a fragment of `descriptive set theory', the study of the kinds of
set which can arise from the formulae of analysis.   These ideas too
will re-surface elsewhere (compare 225Yg and also the proof of 262M
below) and will be of great importance in Volumes 4 and 5.
}%end of notes

\discrpage

