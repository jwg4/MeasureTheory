\frfilename{mt315.tex}
\versiondate{13.11.12}
\copyrightdate{1994}

\def\chaptername{Boolean algebras}
\def\sectionname{Products and free products}

\newsection{315}

I describe here two algebraic constructions of fundamental importance.
They are very different in character, indeed may be regarded as
opposites, despite the common use of the word
`product'.   The first part of the section
(315A-315H) % 315A 315B 315C 315D 315E 315F 315G 315H
deals with the easier construction, the `simple product';
the second part (315I-315Q) % 315I 315J 315K 315L 315M 315N 315O 315P 315Q
with the `free product'.
These constructions lead to descriptions of projective and inductive
limits (315R-315S).

\leader{315A}{Products of Boolean algebras (a)} Let
$\langle\frak A_i\rangle_{i\in I}$ be any family of Boolean algebras.
Set $\frak A=\prod_{i\in I}\frak A_i$, with the natural ring structure

\Centerline{$a\Bsymmdiff b
=\langle a(i)\Bsymmdiff b(i)\rangle_{i\in I}$,}

\Centerline{$a\Bcap b=\langle a(i)\Bcap b(i)\rangle_{i\in I}$}

\noindent for $a$, $b\in\frak A$.   Then $\frak A$ is a
\dvro{Boolean algebra.}{ring (3A2H);  it
is a Boolean ring because

\Centerline{$a\Bcap a=\langle a(i)\Bcap a(i)\rangle_{i\in I}=a$}

\noindent for every $a\in\frak A$;  and it is a Boolean algebra because
if we set $1_{\frak A}=\langle 1_{\frak A_i}\rangle_{i\in I}$, then
$1_{\frak A}\Bcap a=a$ for every $a\in\frak A$.}
I will call $\frak A$ the {\bf simple product} of the family
$\langle\frak A_i\rangle_{i\in I}$.

\cmmnt{I should perhaps remark that when $I=\emptyset$ then $\frak A$
becomes $\{\emptyset\}$, to be interpreted as the singleton Boolean
algebra.}


\spheader 315Ab The Boolean operations on $\frak A$ are now defined by
the formulae

\Centerline{$a\Bcup b=\langle a(i)\Bcup b(i)\rangle_{i\in I}$,
\quad$a\Bsetminus b=\langle a(i)\Bsetminus b(i)\rangle_{i\in I}$
}

\noindent for all $a$, $b\in\frak A$.


\vleader{48pt}{315B}{Theorem} Let $\langle\frak A_i\rangle_{i\in I}$ be a
family of Boolean algebras, and $\frak A$ their simple product.

(a) The maps $a\mapsto \pi_i(a)=a(i):\frak A\to\frak A_i$ are all
Boolean homomorphisms.

(b) If $\frak B$ is any other Boolean algebra, then a map
$\phi:\frak B\to\frak A$ is a Boolean homomorphism iff
$\pi_i\phi:\frak B\to\frak A_i$ is a Boolean homomorphism for every $i\in I$.

\proof{ Verification of
these facts amounts just to applying the definitions with attention.
}%end of proof of 315B

\leader{315C}{Products of partially ordered sets (a)}\cmmnt{ It is perhaps
worth spelling out the following elementary definition.} 
If $\langle P_i\rangle_{i\in I}$ is any family of partially ordered sets, 
its {\bf product} is the set $P=\prod_{i\in I}P_i$ ordered by saying that 
$p\le q$ iff $p(i)\le q(i)$ for every $i\in I$\cmmnt{;  it is easy to check
that $P$ is now a partially ordered set}.

\spheader 315Cb The point is that if $\frak A$ is the simple
product of a family $\langle\frak A_i\rangle_{i\in I}$ of Boolean
algebras, then the ordering of $\frak A$ is just the product partial
order:

\Centerline{$a\Bsubseteq b \cmmnt{ \iff a\Bcap b=a\iff a(i)\Bcap
b(i)=a(i)\Forall  i\in I}\iff a(i)\Bsubseteq b(i)\Forall i\in I$.
}

\cmmnt{Now we have the following elementary, but
extremely useful, general facts about products of partially ordered
sets.
}

\leader{315D}{Proposition} Let $\langle P_i\rangle_{i\in I}$ be a family
of non-empty partially ordered sets with product $P$.

(a) For any non-empty set $A\subseteq P$ and $q\in P$,

\quad (i) $\sup A=q$ in $P$ iff $\sup_{p\in A}p(i)=q(i)$ in $P_i$ for
every $i\in I$,

\quad (ii) $\inf A=q$ in $P$ iff $\inf_{p\in A}p(i)=q(i)$ in $P_i$ for
every $i\in I$.

(b) The coordinate maps $p\mapsto\pi_i(p)=p(i):P\to P_i$ are all
order-preserving and order-continuous.

(c) For any partially ordered set $Q$ and function $\phi:Q\to P$,
$\phi$ is order-preserving iff $\pi_i\phi$ is order-preserving for every
$i\in I$.

(d) For any partially ordered set $Q$ and order-preserving function
$\phi:Q\to P$,

\quad (i) $\phi$ is order-continuous iff $\pi_i\phi$ is order-continuous
for every $i$,

\quad (ii) $\phi$ is sequentially order-continuous iff $\pi_i\phi$ is
sequentially order-continuous for every $i$.

(e)(i) $P$ is Dedekind complete iff every $P_i$ is Dedekind complete.

\quad (ii) $P$ is Dedekind $\sigma$-complete iff every $P_i$ is Dedekind
$\sigma$-complete.

\proof{ All these are elementary verifications.   Of course parts (b),
(d) and (e) rely on (a).
}%end of proof of 315D

\leader{315E}{Factor algebras as principal ideals}\cmmnt{ Because
Boolean
algebras have least elements, we have a second type of canonical homomorphism
associated with their products.}   If $\langle\frak A_i\rangle_{i\in I}$
is a family of Boolean algebras with simple product $\frak A$, define
$\theta_i:\frak A_i\to\frak A$ by setting $(\theta_i a)(i)=a$,
$(\theta_i a)(j)=0_{\frak A_j}$ if $i\in I$, $a\in \frak A_i$ and $j\in
I\setminus\{i\}$.   Each $\theta_i$ is a ring homomorphism, and is a
Boolean isomorphism between $\frak A_i$ and the principal ideal of
$\frak A$ generated by $\theta_i(1_{\frak A_i})$.   The family
$\langle\theta_i(1_{\frak A_i})\rangle_{i\in I}$ is a partition of unity
in $\frak A$.

\cmmnt{Associated with these embeddings is the following important
result.}

\vleader{72pt}{315F}{Proposition} Let $\frak A$ be a Boolean algebra and
$\langle e_i\rangle_{i\in I}$ a partition of unity in $\frak A$.
Suppose

\quad{\it either} (i) that $I$ is finite

\quad{\it or} (ii) that $I$ is countable and $\frak A$ is Dedekind
$\sigma$-complete

\quad{\it or} (iii) that $\frak A$ is Dedekind complete.

\noindent Then the map $a\mapsto\langle a\Bcap e_i\rangle_{i\in I}$ is a
Boolean isomorphism between $\frak A$ and $\prod_{i\in I}\frak A_{e_i}$,
writing $\frak A_{e_i}$ for the principal ideal of $\frak A$ generated
by $e_i$ for each $i$.

\proof{ The given map is a Boolean homomorphism because each of the maps
$a\mapsto a\Bcap e_i:\frak A\to\frak A_{e_i}$ is (312J).   It is
injective because $\sup_{i\in I}e_i=1$, so if
$a\in\frak A\setminus\{0\}$ there is an $i$ such that
$a\Bcap e_i\ne 0$.   It is
surjective because $\langle e_i\rangle_{i\in I}$ is disjoint and if
$c\in\prod_{i\in I}\frak A_{e_i}$ then $a=\sup_{i\in I}c(i)$ is defined
in $\frak A$ and

\Centerline{$a\Bcap e_j=\sup_{i\in I}c(i)\Bcap e_j=c(j)$}

\noindent for every $j\in I$ (using 313Ba).   The three alternative
versions of the
hypotheses of this proposition are designed to ensure that the
supremum is always well-defined in $\frak A$.
}%end of proof of 315F

\leader{315G}{Algebras of sets and their \dvrocolon{quotients}}\cmmnt{ The Boolean
algebras of measure theory are mostly presented as algebras of sets or
quotients of algebras of sets, so it is perhaps worth spelling out the
ways in which the product construction applies to such algebras.
\medskip

\noindent}{\bf Proposition} Let $\langle X_i\rangle_{i\in I}$ be a
family
of sets, and $\Sigma_i$ an algebra of subsets of $X_i$ for each $i$.

(a) The simple product $\prod_{i\in I}\Sigma_i$ may be identified with
the algebra

\Centerline{$\Sigma=\{E:E\subseteq X,\,\{x:(x,i)\in E\}\in\Sigma_i$ for
every $i\in I\}$}

\noindent of subsets of $X=\{(x,i):i\in I,\,x\in X_i\}$, with the
canonical homomorphisms $\pi_i:\Sigma\to\Sigma_i$ being given by

\Centerline{$\pi_iE=\{x:(x,i)\in E\}$}

\noindent for each $E\in\Sigma$.

(b) Now suppose that $\Cal J_i$ is an ideal of $\Sigma_i$ for each $i$.
Then $\prod_{i\in I}\Sigma_i/\Cal J_i$ may be identified with
$\Sigma/\Cal J$, where

\Centerline{$\Cal J=\{E:E\in\Sigma,\,\{x:(x,i)\in E\}\in\Cal J_i$ for
every $i\in I\}$,}

\noindent and the canonical homomorphisms
$\tilde\pi_i:\Sigma/\Cal J\to\Sigma_i/\Cal J_i$ are given by the formula
$\tilde\pi_i(E^{\ssbullet})=(\pi_iE)^{\ssbullet}$ for every
$E\in\Sigma$.

\proof{{\bf (a)} It is easy to check that $\Sigma$ is a subalgebra of
$\Cal PX$, and that the map
$E\mapsto\langle\pi_iE\rangle_{i\in I}:\Sigma\to\prod_{i\in I}\Sigma_i$ is a Boolean isomorphism.

\medskip

{\bf (b)} Again, it is easy to check that $\Cal J$ is an ideal of
$\Sigma$, that the proposed formula for $\tilde\pi_i$ does indeed define
a map from $\Sigma/\Cal J$ to $\Sigma_i/\Cal J_i$, and that
$E^{\ssbullet}\mapsto\langle\tilde\pi_iE^{\ssbullet}\rangle_{i\in I}$ is
an isomorphism between $\Sigma/\Cal J$ and
$\prod_{i\in I}\Sigma_i/\Cal J_i$.
}%end of proof of 315G

\leader{*315H}{}\cmmnt{ There is a particular kind of simple
product which arises naturally when we look at regular open algebras.

\medskip

\noindent}{\bf Proposition}\dvAnew{2011}
Let $X$ be a topological space, and
$\Cal U$ a disjoint family of open subsets of $X$ with union
dense in $X$.   Then the regular open algebra $\RO(X)$ is isomorphic to
the simple product $\prod_{U\in\Cal U}\RO(U)$.

\proof{ By 314R(b-i), $G\mapsto G\cap U$ is a Boolean homomorphism from
$\RO(X)$ onto $\RO(U)$, for any $U\in\Cal U$.
By 315B, we have a Boolean homomorphism
$G\mapsto\pi G
=\family{U}{\Cal U}{G\cap U}:\RO(X)\to\prod_{U\in\Cal U}\RO(U)$.
If $G\in\RO(X)\setminus\{\emptyset\}$, then
$G\cap\bigcup\Cal U\ne\emptyset$, because $\bigcup\Cal U$ is dense;
now there is a $U\in\Cal U$ such that $G\cap U\ne\emptyset$, so
$\pi G$ is non-zero in the Boolean algebra $\prod_{U\in\Cal U}\RO(U)$.
As $G$ is arbitrary, $\pi$ is injective (3A2Db).

To see that $\pi$ is surjective, suppose that we are given a family
$\family{U}{\Cal U}{V_U}$ with $V_U\in\RO(U)$ for every $U\in\Cal U$.
Set $H=\bigcup_{U\in\Cal U}V_U$,
$G=\interior\overline{H}\in\RO(X)$.   Then, for any $U\in\Cal U$,
(writing $\interior_U$ and $\overline{\phantom{H}}^{(U)}$ for interior and
closure in the subspace
topology on $U$, as in part (b) of the proof of 314R)

\Centerline{$G\cap U
=U\cap\interior\overline{H}
=\interior_U\overline{H\cap U}^{(U)}
=\interior_U\overline{V}_U^{(U)}
=V_U$,}

\noindent so $\pi G=\family{U}{\Cal U}{V_U}$.   Thus $\pi$ is bijective and
is a Boolean isomorphism.
}%end of proof of 315H

\leader{315I}{Free products} \cmmnt{I come now to the second
construction of this section.

\medskip

}{\bf (a) Definition}\dvAformerly{3{}15J} 
Let $\langle\frak A_i\rangle_{i\in I}$ be a family
of Boolean algebras.   For each $i\in I$, let $Z_i$ be
the Stone space of $\frak A_i$.   Set $Z=\prod_{i\in I}Z_i$, with the
product topology.   Then the {\bf free product} of
$\langle\frak A_i\rangle_{i\in I}$ is the algebra $\frak A$ of
open-and-closed sets in
$Z$;  I will denote it by $\bigotimes_{i\in I}\frak A_i$.

\spheader 315Ib For $i\in I$ and $a\in \frak A_i$, the
set\cmmnt{ $\widehat a\subseteq Z_i$ representing $a$
is an open-and-closed subset of $Z_i$;  because $z\mapsto z(i):Z\to Z_i$
is continuous,} $\varepsilon_i(a)=\{z:z(i)\in\widehat a\}$
\cmmnt{ is open-and-closed, so} belongs to $\frak A$.   In this
context I will call
$\varepsilon_i:\frak A_i\to\frak A$ the {\bf canonical map}.

\spheader 315Ic The topological space $Z$ may be identified with the
Stone space of the Boolean algebra $\frak A$.   \prooflet{\Prf\ By
Tychonoff's
theorem (3A3J), $Z$ is compact.   If $z\in Z$ and $G$ is an open
subset of $Z$ containing $z$, then there are $J$,
$\langle G_j\rangle_{i\in J}$
such that $J$ is a finite subset of $I$, $G_j$ is an open subset of
$Z_j$ for each $j\in J$, and

\Centerline{$z\in\{w:w\in Z,\,w(j)\in G_j$ for every
$j\in J\}\subseteq G$.}

\noindent Because each $Z_j$ is zero-dimensional, we can find an
open-and-closed set $E_j\subseteq Z_j$ such that
$z(j)\in E_j\subseteq G_j$.   Now

\Centerline{$H=Z\cap\bigcap_{j\in J}\{w:w(j)\in E_j\}$}

\noindent is a finite intersection of open-and-closed subsets of $Z$, so
is open-and-closed;  and $z\in H\subseteq G$.   As $z$ and $G$ are
arbitrary, $Z$ is zero-dimensional.   Finally, $Z$, being the product of
Hausdorff spaces, is Hausdorff.   So the result follows from 311J.\ \Qed}

\leader{315J}{Theorem}\dvAformerly{3{}15I}
Let $\langle\frak A_i\rangle_{i\in I}$ be a
family of Boolean algebras, with free product $\frak A$.

(a) The canonical map $\varepsilon_i:\frak A_i\to\frak A$ is a Boolean
homomorphism for every $i\in I$.

(b) For any Boolean algebra $\frak B$ and any family
$\langle\phi_i\rangle_{i\in I}$ such that $\phi_i$ is a Boolean
homomorphism from $\frak A_i$ to $\frak B$ for every $i$, there is a
unique Boolean homomorphism $\phi:\frak A\to\frak B$ such that
$\phi_i=\phi\varepsilon_i$ for each $i$.

\proof{ These are both consequences of 312Q-312R.   As in 315I, write
$Z_i$ for the Stone space of $\frak A$, and $Z$ for $\prod_{i\in I}Z_i$,
identified with the Stone space of $\frak A$, as observed in 315Ic.
The maps $\varepsilon_i:\frak A_i\to\frak A$ are defined as the
homomorphisms corresponding to the continuous maps
$z\mapsto\tilde\varepsilon_i(z)=z(i):Z\to Z_i$, so (a) is surely true.

Now suppose that we are given a Boolean homomorphism
$\phi_i:\frak A_i\to\frak B$ for each $i\in I$.   Let $W$ be the Stone
space of $\frak B$, and let $\tilde\phi_i:W\to Z_i$ be the continuous
function corresponding to $\phi_i$.   By 3A3Ib, the map
$w\mapsto\tilde\phi(w)=\langle\tilde\phi_i(w)\rangle_{i\in I}:W\to Z$ is
continuous, and corresponds to a Boolean homomorphism
$\phi:\frak A\to\frak B$;  because
$\tilde\phi_i=\tilde\varepsilon_i\tilde\phi$,
$\phi\varepsilon_i=\phi_i$ for each $i$.   Moreover, $\phi$ is the only
Boolean homomorphism with this property, because if
$\psi:\frak A\to\frak B$ is a Boolean homomorphism such that
$\psi\varepsilon_i=\phi_i$ for every $i$, then $\psi$ corresponds to a
continuous function $\tilde\psi:W\to Z$, and we must have
$\tilde\varepsilon_i\tilde\psi=\tilde\phi_i$ for each $i$, so that
$\tilde\psi=\tilde\phi$ and $\psi=\phi$.   This proves (b).
}%end of proof of 315J

\leader{315K}{}\cmmnt{ Of course 315J is the defining property of the
free product (see 315Xi below).   I list a few further basic facts.

\medskip

\noindent}{\bf Proposition}\dvAformerly{3{}15J} 
Let $\langle\frak A_i\rangle_{i\in I}$ be a
family of Boolean algebras, and $\frak A$ their free product;  write
$\varepsilon_i:\frak A_i\to\frak A$ for the canonical homomorphisms.

(a) $\frak A$ is the subalgebra of itself generated by
$\bigcup_{i\in I}\varepsilon_i[\frak A_i]$.

(b) Write $C$ for the set of those members of $\frak A$ expressible in
the form $\inf_{j\in J}\varepsilon_j(a_j)$, where $J\subseteq I$ is
finite and $a_j\in\frak A_j$ for every $j$.   Then every member of
$\frak A$ is expressible as the supremum of a disjoint finite subset of
$C$.   In particular, $C$ is order-dense in $\frak A$.

(c) Every $\varepsilon_i$ is order-continuous.

(d) $\frak A=\{0_{\frak A}\}$ iff there is some $i\in I$ such that
$\frak A_i=\{0_{\frak A_i}\}$.

(e) Now suppose that $\frak A_i\ne\{0_{\frak A_i}\}$ for every $i\in I$.

\quad (i) $\varepsilon_i$ is injective for every $i\in I$.

\quad (ii) If $J\subseteq I$ is finite and $a_j$ is a non-zero member of
$\frak A_j$ for each $j\in J$, then
$\inf_{j\in J}\varepsilon_j(a_j)\ne 0$.

\quad (iii) If $i$, $j$ are distinct members of $I$, $a\in\frak A_i$ and
$b\in\frak A_j$, then $\varepsilon_i(a)=\varepsilon_j(b)$ iff either
$a=0_{\frak A_i}$ and $b=0_{\frak A_j}$ or $a=1_{\frak A_i}$ and
$b=1_{\frak
A_j}$.

\proof{ As usual, write $Z_i$ for the Stone space of $\frak A_i$, and
$Z=\prod_{i\in I}Z_i$, identified with the Stone space of $\frak A$
(315Ic).

\medskip

{\bf (a)} Write $\frak A'$ for the subalgebra of $\frak A$ generated by
$\bigcup_{i\in I}\varepsilon_i[\frak A_i]$.   Then
$\varepsilon_i:\frak A_i\to\frak A'$ is a Boolean homomorphism for each
$i$, so by 315Jb
there is a Boolean homomorphism $\phi:\frak A\to\frak A'$ such that
$\phi\varepsilon_i=\varepsilon_i$ for each $i$.   Now, regarding $\phi$
as a Boolean homomorphism from $\frak A$ to itself, the uniqueness
assertion of 315Jb (with $\frak B=\frak A$) shows that $\phi$ must be
the identity, so that $\frak A'=\frak A$.

\medskip

{\bf (b)} Write $\Cal D$ for the set of finite partitions of unity in
$\frak A$ consisting of members of $C$, and $A$ for the set of members
of $\frak A$ expressible in the form $\sup D'$ where $D'$ is a subset of
a member of $\Cal D$.   Then $A$ is a subalgebra of $\frak A$.   \Prf\
(i) $1_{\frak A}\in C$ (set $J=\emptyset$ in the definition of members
of $C$) so $\{1_{\frak A}\}\in\Cal D$ and $0_{\frak A}$, $1_{\frak A}\in
A$.   (ii) Note that if $c$, $d\in C$ then $c\Bcap d\in C$.   (iii) If
$a$, $b\in A$, express them as $\sup D'$, $\sup E'$ where
$D'\subseteq D\in\Cal D$, $E'\subseteq E\in\Cal D$.   Then

\Centerline{$F=\{d\Bcap e:d\in D,\,e\in E\}\in\Cal D$,}

\noindent so

\Centerline{$1_{\frak A}\Bsetminus a=\sup(D\setminus D')\in A$,}

\Centerline{$a\Bcup b=\sup\{f:f\in F,\,f\Bsubseteq a\Bcup b\}\in A$.
\Qed}

\noindent Also, $\varepsilon_i[\frak A_i]\subseteq A$ for each $i\in I$.
\Prf\ If $a\in\frak A_i$, then
$\{\varepsilon_i(a),\varepsilon_i(1_{\frak A_i}
\Bsetminus a)\}\in\Cal D$, so $\varepsilon_i(a)\in A$.  \Qed

So (a) tells us that $A=\frak A$, and every member of $\frak A$ is a
finite disjoint union of members of $C$.

\medskip

{\bf (c)} If $i\in I$ and $A\subseteq\frak A_i$ and $\inf A=0$ in
$\frak A_i$, take any non-zero $c\in\frak A$.   By (b), we can find a
finite $J\subseteq I$ and a family $\langle a_j\rangle_{j\in J}$ such
that $c'=\inf_{j\in J}\varepsilon_j(a_j)\Bsubseteq c$ and $c'\ne 0$.
Regarding $c'$ as a subset of $Z$, we have a point $z\in c'$.   Adding
$i$ to $J$ and setting $a_i=1_{\frak A_i}$ if necessary, we may suppose
that $i\in J$.   Now $c'\ne 0_{\frak A}$ so $a_i\ne 0_{\frak A_i}$ and
there is an $a\in A$ such that $a_i\notBsubseteq a$, so there is a
$t\in\widehat a_i\setminus\widehat a$.   In this case, setting $w(i)=t$,
$w(j)=z(j)$ for $j\ne i$, we have
$w\in c'\setminus\varepsilon_i(a)$, and $c'$, $c$ are not included in
$\varepsilon_i(a)$.   As $c$ is arbitrary, this shows that
$\inf\varepsilon_i[A]=0$.   As $A$ is arbitrary, $\varepsilon_i$ is
order-continuous.

\medskip

{\bf (d)} The point is that $\frak A=\{0_{\frak A}\}$ iff $Z=\emptyset$,
which is so iff some $Z_i$ is empty.

\medskip

{\bf (e)(i)} Because no $Z_i$ is empty, all the coordinate maps from $Z$
to $Z_i$ are surjective, so the corresponding homomorphisms
$\varepsilon_i$ are injective (312Sa).

\medskip

\quad{\bf (ii)} Because $J$ is finite,

\Centerline{$\inf_{j\in J}\varepsilon_j(a_j)
=\{z:z\in Z,\,z(j)\in\widehat a_j$ for every $j\in J\}$}

\noindent is not empty.

\medskip

\quad{\bf (iii)} If $\varepsilon_i(a)=\varepsilon_j(b)=0_{\frak A}$ then
(using (i)) $a=0_{\frak A_i}$ and $b=0_{\frak A_j}$;
if $\varepsilon_i(a)=\varepsilon_j(b)=1_{\frak A}$ then
$a=1_{\frak A_i}$ and $b=1_{\frak A_j}$.   \Quer\ If $\varepsilon_i(a)
=\varepsilon_j(b)\in\frak A\setminus\{0_{\frak A},1_{\frak A}\}$, then
there are $t\in\widehat a$ and $u\in Z_j\setminus\widehat b$.   Now
there is a $z\in Z$ such that $z(i)=t$ and $z(j)=u$, so that
$z\in\varepsilon_i(a)\setminus\varepsilon_j(b)$.   \Bang
}%end of proof of 315K

\leader{315L}{Proposition}\dvAformerly{3{}15K}
Let $\langle \frak A_i\rangle_{i\in I}$ be
any family of Boolean algebras, and $\langle J_k\rangle_{k\in K}$ any
partition\cmmnt{ (that is, disjoint cover)} of $I$.   Then the
free product $\frak A$ of $\familyiI{\frak A_i}$ is isomorphic to the free product $\frak B$ of
$\langle\frak B_k\rangle_{k\in K}$, where each $\frak B_k$ is the free
product of $\langle \frak A_i\rangle_{i\in J_k}$.

\proof{ Write $\varepsilon_i:\frak A_i\to\frak A$,
$\varepsilon_i':\frak A_i\to\frak B_k$ and $\delta_k:\frak B_k\to\frak B$
for the canonical homomorphisms
when $k\in K$ and $i\in J_k$.   Then the homomorphisms
$\delta_k\varepsilon'_i:\frak A_i\to\frak B$ correspond to a
homomorphism $\phi:\frak A\to\frak B$ such that
$\phi\varepsilon_i=\delta_k\varepsilon'_i$ whenever $i\in J_k$.   Next,
for each $k$, the homomorphisms $\varepsilon_i:\frak A_i\to\frak A$, for
$i\in J_k$, correspond to a homomorphism $\psi_k:\frak B_k\to\frak A$
such that $\psi_k\varepsilon_i'=\varepsilon_i$ for $i\in J_k$;
and the family $\langle\psi_k\rangle_{k\in K}$ corresponds to a
homomorphism $\psi:\frak B\to\frak A$ such that $\psi\delta_k=\psi_k$
for
$k\in K$.   Consequently

\Centerline{$\psi\phi\varepsilon_i
=\psi\delta_k\varepsilon_i'
=\psi_k\varepsilon_i'
=\varepsilon_i$}

\noindent whenever $k\in K$, $i\in J_k$.   Once again using the
uniqueness assertion
in 315Jb, $\psi\phi$ is the identity homomorphism on $\frak A$.   On the
other hand, if we look at $\phi\psi:\frak B\to\frak B$, then we see that

\Centerline{$\phi\psi\delta_k\varepsilon'_i
=\phi\psi_k\varepsilon_i'
=\phi\varepsilon_i
=\delta_k\varepsilon'_i$}

\noindent whenever $k\in K$, $i\in J_k$.   Now, for given $k$,
$\{b:b\in\frak B_k,\,\phi\psi\delta_kb=\delta_kb\}$ is a subalgebra of
$\frak B_k$ including $\bigcup_{i\in J_k}\varepsilon'_i[\frak A_i]$, and
must be the whole of $\frak B_k$, by 315Ka.   So $\{b:b\in\frak
B,\,\phi\psi b=b\}$ is a subalgebra of $\frak B$ including
$\bigcup_{k\in
K}\delta_k[\frak B_k]$, and is the whole of $\frak B$.   Thus $\phi\psi$
is the identity on $\frak B$ and $\phi$, $\psi$ are the two halves of an
isomorphism between $\frak A$ and $\frak B$.
}%end of proof of 315L

\leader{315M}
{Algebras of sets and their \dvrocolon{quotients}}\cmmnt{ Once again I
devote a paragraph to spelling out the application of the construction
to the algebras most important to us.

\medskip

\noindent}{\bf Proposition}\dvAformerly{3{}15L} 
Let $\langle X_i\rangle_{i\in I}$ be a
family of sets, and $\Sigma_i$ an algebra of subsets of $X_i$ for each
$i$.

(a) The free product $\bigotimes_{i\in I}\Sigma_i$ may be identified
with the algebra $\Sigma$ of subsets of $X=\prod_{i\in I}X_i$ generated
by the set $\{\varepsilon_i(E):i\in I,\,E\in\Sigma_i\}$, where
$\varepsilon_i(E)=\{x:x\in X,\,x(i)\in E\}$.

(b) Now suppose that $\Cal J_i$ is an ideal of $\Sigma_i$ for each $i$.
Then $\bigotimes_{i\in I}\Sigma_i/\Cal J_i$ may be identified with
$\Sigma/\Cal J$, where $\Cal J$ is the ideal of $\Sigma$ generated by
$\{\varepsilon_i(E):i\in I,\,E\in\Cal J_i\}$;  the corresponding
canonical maps $\tilde\varepsilon_i:\Sigma_i/\Cal J_i\to\Sigma/\Cal J$
being defined by the formula
$\tilde\varepsilon_i(E^{\ssbullet})=(\varepsilon_i(E))^{\ssbullet}$ for
$i\in I$, $E\in\Sigma_i$.

\proof{ I start by proving (b) in detail;  the argument for (a) is then
easy to extract.   Write $\frak A_i=\Sigma_i/\Cal J_i$, $\frak
A=\Sigma/\Cal J$.

\medskip

{\bf (i)}
Fix $i\in I$ for the moment.   By the definition of $\Sigma$,
$\varepsilon_i(E)\in\Sigma$ for $E\in\Sigma_i$, and it is easy to check
that $\varepsilon_i:\Sigma_i\to\Sigma$ is a Boolean homomorphism.
Again, because $\varepsilon_i(E)\in\Cal J$ whenever $E\in\Cal J_i$, the
kernel of the homomorphism
$E\mapsto(\varepsilon_i(E))^{\ssbullet}:\Sigma_i\to\frak A$ includes
$\Cal J_i$, so the formula for $\tilde\varepsilon_i$ defines a
homomorphism from $\frak A_i$ to $\frak A$.

Now let $\frak C=\bigotimes_{i\in I}\frak A_i$ be the free product, and
write $\varepsilon'_i:\frak A_i\to\frak C$ for the canonical
homomorphisms.   By 315J, there is a Boolean homomorphism $\phi:\frak
C\to\frak A$ such that $\phi\varepsilon'_i=\tilde\varepsilon_i$ for each
$i$.   The set

\Centerline{$\{H:H\in\Sigma,\,H^{\ssbullet}\in\phi[\frak C]\}$}

\noindent is a subalgebra of $\Sigma$ including
$\varepsilon_i[\Sigma_i]$ for every $i$, so is $\Sigma$ itself, and
$\phi$ is surjective.

\medskip

{\bf (ii)} We need a simple description of the ideal $\Cal J$, as
follows:  a set $H\in\Sigma$ belongs to $\Cal J$ iff there are a finite
$K\subseteq I$ and a family $\langle F_k\rangle_{k\in K}$ such that
$F_k\in\Cal J_k$ for each $k$ and
$H\subseteq\bigcup_{k\in K}\varepsilon_k(F_k)$.
For evidently such sets have to belong to $\Cal
J$, since the $\varepsilon_k(F_k)$ will be in $\Cal J$, while the family
of all these sets is an ideal containing $\varepsilon_i(F)$ whenever
$i\in I$ and $F\in\Cal J_i$.

\medskip

{\bf (iii)} Now we can see that $\phi:\frak C\to\frak A$ is injective.
\Prf\ Take any
non-zero $c\in \frak C$.   By 315Kb, we can find a finite $J\subseteq I$
and a family $\langle a_j\rangle_{j\in J}$ in $\prod_{j\in J}\frak A_j$
such that $0\ne\inf_{j\in J}\varepsilon'_ja_j\Bsubseteq c$.   Express
each $a_j$ as $E_j^{\ssbullet}$, where $E_j\in\Sigma_j$, and consider
$H=X\cap\bigcap_{j\in J}\varepsilon_j(E_j)\in\Sigma$.   Then

\Centerline{$H^{\ssbullet}=\inf_{j\in J}\tilde\varepsilon_ja_j
=\phi(\inf_{j\in J}\varepsilon'_ja_j)\Bsubseteq\phi(c)$.}

\noindent Also, because $\varepsilon'_ja_j\ne 0$, $E_j\notin\Cal J_j$
for each $j$.   But it follows that $H\notin\Cal J$, because if
$K\subseteq I$ is finite and $F_k\in\Cal J_k$ for each $k\in K$, set
$E_i=X_i$ for $i\in I\setminus J$, $F_i=\emptyset$ for
$i\in I\setminus K$;  then there is an $x\in X$ such that
$x(i)\in E_i\setminus F_i$ for
each $i\in I$, so that
$x\in H\setminus\bigcup_{k\in K}\varepsilon_k(F_k)$.   By the
criterion of (ii), $H\notin\Cal J$.   So

\Centerline{$0\ne E^{\ssbullet}\Bsubseteq\phi(c)$.}

\noindent As $c$ is arbitrary, the kernel of $\phi$ is $\{0\}$, and
$\phi$ is injective.   \Qed

So $\phi:\frak C\to\frak A$ is the required isomorphism.

\medskip

{\bf (iv)} This proves (b).   Reading through the arguments above, it is
easy to see the simplifications which compose a proof of (a), reading
$\Sigma_i$ for $\frak A_i$ and $\{\emptyset\}$ for $\Cal J_i$.
}%end of proof of 315M

\leader{315N}{Notation}\dvAformerly{3{}15M}\cmmnt{ Free products 
are sufficiently surprising that I
think it worth taking a moment to look at a pair of examples relevant to
the kinds of application I wish to make of the concept in the next
chapter.   First let me introduce a somewhat more direct notation which
seems appropriate for the free product of finitely many factors.}   If
$\frak A$ and $\frak B$ are two Boolean algebras, I write $\frak
A\otimes\frak B$ for their free product, and for $a\in\frak A$,
$b\in\frak B$ I write $a\otimes b$ for
$\varepsilon_1(a)\Bcap\varepsilon_2(b)$, where $\varepsilon_1:\frak
A\to\frak A\otimes\frak B$,
$\varepsilon_2:\frak B\to\frak A\otimes\frak B$ are the canonical maps.
Observe that $(a_1\otimes
b_1)\Bcap(a_2\otimes b_2)=(a_1\Bcap a_2)\otimes(b_1\Bcap b_2)$, and that
the maps $a\mapsto a\otimes b_0$, $b\mapsto a_0\otimes b$ are always
ring homomorphisms.   Now\cmmnt{ 315K(e-ii) tells us that} $a\otimes
b=0$ only
when one of $a$, $b$ is $0$.   In the context of 315M, we can identify
$E\otimes F$ with $E\times F$ for $E\in\Sigma_1$ and $F\in\Sigma_2$, and
$E^{\ssbullet}\otimes F^{\ssbullet}$ with $(E\times F)^{\ssbullet}$.

\leader{315O}{Lemma}\dvAformerly{3{}15N} 
Let $\frak A$, $\frak B$ be Boolean algebras.

(a) Any element of $\frak A\otimes\frak B$ is expressible as
$\sup_{i\in I}a_i\otimes b_i$ where $\langle a_i\rangle_{i\in I}$ is a
finite partition of unity in $\frak A$.

(b) If $c\in\frak A\otimes\frak B$ is non-zero there are non-zero
$a\in\frak A$, $b\in\frak B$ such that $a\otimes b\Bsubseteq c$.

\proof{{\bf (a)} Let $C$ be the set of elements of $\frak A\otimes\frak
B$ representable in this form.   Then $C$ is a subalgebra of $\frak
A\otimes\frak B$.   \Prf\ (i) If $\langle a_i\rangle_{i\in I}$, $\langle
a'_j\rangle_{j\in J}$ are finite partitions of unity in $\frak A$, and
$b_i$, $b'_j$ members of $\frak B$ for $i\in I$ and $j\in J$, then
$\langle a_i\Bcap a'_j\rangle_{i\in I,j\in J}$ is a partition of unity
in $\frak A$, and

$$\eqalign{(\sup_{i\in I}a_i\otimes b_i)
  \Bcap(\sup_{j\in J}a'_j\otimes b'_j)
&=\sup_{i\in I,j\in J}(a_i\otimes b_i)\Bcap(a'_j\otimes b'_j)\cr
&=\sup_{i\in I,j\in J}(a_i\Bcap a'_j)\otimes(b_i\Bcap b'_j)
\in C.}$$

\noindent So $c\Bcap c'\in C$ for all $c$, $c'\in C$.   (ii) If $\langle
a_i\rangle_{i\in I}$ is a finite partition of unity in $\frak A$ and
$b_i\in\frak B$ for each $i$, then

\Centerline{$1\Bsetminus\sup_{i\in I}a_i\otimes b_i
=(\sup_{i\in I}a_i\otimes 1)\Bsetminus(\sup_{i\in I}a_i\otimes b_i)
=\sup_{i\in I}a_i\otimes(1\Bsetminus b_i)
\in C$.}

\noindent Thus $1\Bsetminus c\in C$ for every $c\in C$.   \Qed

Since $a\otimes 1=(a\otimes 1)\Bcup((1\Bsetminus a)\otimes 0)$ and
$1\otimes b$ belong to $C$ for every $a\in\frak A$ and $b\in\frak B$, $C$
must be the whole of $\frak A\otimes\frak B$, by 315Ka.

\medskip

{\bf (b)} Now this follows at once, as well as being a special case of
315Kb.
}%end of proof of 315O

\leader{315P}{Example} $\frak A=\Cal P\Bbb N\otimes\Cal P\Bbb N$ is not
Dedekind $\sigma$-complete.   \prooflet{\Prf\ Consider
$A=\{\{n\}\otimes\{n\}:n\in\Bbb N\}\subseteq\frak A$.   \Quer\ If $A$
has a least upper bound $c$ in $\frak A$, then $c$ is expressible as a
supremum $\sup_{j\le k}a_j\otimes b_j$, by 315Kb.   Because $k$ is
finite, there must be distinct $m$, $n$ such that $\{j:m\in
a_j\}=\{j:n\in a_j\}$.   Now
$\{n\}\times\{n\}\Bsubseteq c$, so there is a $j\le k$ such that

\Centerline{$(a_j\Bcap\{n\})\otimes(b_j\Bcap\{n\})
=(\{n\}\otimes\{n\})\Bcap(a_j\otimes b_j)\ne 0$,}

\noindent so that neither $a_j\Bcap\{n\}$ nor $b_j\Bcap\{n\}$ is empty,
that is, $n\in a_j\cap b_j$.   But this means that $m\in a_j$, so that

\Centerline{$(a_j\otimes b_j)\Bcap(\{m\}\otimes\{n\})
=(a_j\Bcap\{m\})\otimes(b_j\Bcap\{n\})
\ne 0$,}

\noindent and $c\Bcap(\{m\}\otimes\{n\})\ne 0$, even though
$a\Bcap(\{m\}\otimes\{n\})=0$ for every $a\in A$.   \Bang\   Thus we
have found a countable subset of $\frak A$ with no supremum in $\frak
A$, and $\frak A$ is not Dedekind $\sigma$-complete.   \Qed}

\leader{315Q}{Example} Now let $\frak A$ be any non-trivial atomless
Boolean algebra, and $\frak B$ the free product $\frak A\otimes\frak A$.
Then the identity homomorphism from $\frak A$ to itself induces a
homomorphism $\phi:\frak B\to\frak A$ given by setting
$\phi(a\otimes b)=a\Bcap b$ for every $a$, $b\in\frak A$.   \cmmnt{The
point I wish to make is
that} $\phi$ is not order-continuous.   \prooflet{\Prf\ Let $C$ be the
set $\{a\otimes b:a,\,b\in\frak A,\,a\Bcap b=0\}$.   Then
$\phi(c)=0_{\frak A}$ for every $c\in C$.   If $d\in \frak B$ is
non-zero, then by 315Ob there are non-zero $a$, $b\in\frak A$ such that
$a\otimes b\Bsubseteq d$;  now, because $\frak A$ is atomless, there is
a non-zero $a'\Bsubseteq a$ such that $a\Bsetminus a'\ne 0$.   At least
one of $b\Bsetminus a'$, $b\Bsetminus(a\Bsetminus a')$ is non-zero;
suppose the former.   Then $a'\otimes(b\Bsetminus a')$ is a non-zero
member of $C$ included in $d$.   As $d$ is arbitrary, this shows that
$\sup C=1_{\frak B}$.   So

\Centerline{$\sup_{c\in C}\phi(c)=0_{\frak A}\ne1_{\frak A}
=\phi(\sup C)$,}

\noindent and $\phi$ is not order-continuous.   \Qed}

Thus the free product\cmmnt{ (unlike the product, see 315Dd)} does not
respect order-continuity.

\leader{*315R}{Projective and inductive limits:  Proposition}
Let $\familyiI{\frak A_i}$
be a family of Boolean algebras, and $R$ a subset of $I\times I$;  suppose
that $\pi_{ji}:\frak A_i\to\frak A_j$ is a Boolean homomorphism for each
$(i,j)\in R$.

(a) There are a Boolean algebra $\frak C$ and a family $\familyiI{\pi_i}$
such that

\inset{$\pi_i:\frak C\to\frak A_i$ is a Boolean homomorphism for each
$i\in I$,

$\pi_j=\pi_{ji}\pi_i$ whenever $(i,j)\in R$,}

\noindent and whenever $\frak B$, $\familyiI{\phi_i}$ are such that

\inset{$\frak B$ is a Boolean algebra,

$\phi_i:\frak B\to\frak A_i$ is a Boolean homomorphism for each
$i\in I$,

$\phi_j=\pi_{ji}\phi_i$ whenever $(i,j)\in R$,}

\noindent then there is a unique Boolean homomorphism
$\phi:\frak B\to\frak C$ such that $\pi_i\phi=\phi_i$ for every $i\in I$.

(b) There are a Boolean algebra $\frak C$ and a family $\familyiI{\pi_i}$
such that

\inset{$\pi_i:\frak A_i\to\frak C$ is a Boolean homomorphism for each
$i\in I$,

$\pi_i=\pi_j\pi_{ji}$ whenever $(i,j)\in R$,}

\noindent and whenever $\frak B$, $\familyiI{\phi_i}$ are such that

\inset{$\frak B$ is a Boolean algebra,

$\phi_i:\frak A_i\to\frak B$ is a Boolean homomorphism for each
$i\in I$,

$\phi_i=\phi_j\pi_{ji}$ whenever $(i,j)\in R$,}

\noindent then there is a unique Boolean homomorphism
$\phi:\frak C\to\frak B$ such that $\phi\pi_i=\phi_i$ for every $i\in I$.

\proof{{\bf (a)} Let $\frak A$ be the simple product
$\prod_{i\in I}\frak A_i$, and set

\Centerline{$\frak C
=\{a:a\in\frak A$, $a(j)=\pi_{ji}a(i)$ whenever $(i,j)\in R\}$.}

\noindent Because every $\pi_{ji}$ is a Boolean homomorphism, $\frak C$ is
a subalgebra of $\frak A$.   Set $\pi_i(a)=a(i)$ for $i\in I$ and
$a\in\frak C$;  then $\pi_i:\frak C\to\frak A_i$ is a Boolean homomorphism
for every $i$, and $\pi_j=\pi_{ji}\pi_i$ whenever $(i,j)\in R$.

Now suppose that $\frak B$ and $\familyiI{\phi_i}$ have the declared
properties.   For $b\in\frak B$, set $\phi b=\familyiI{\phi_ib}\in\frak A$;
because $\phi_j=\pi_{ji}\phi_i$ whenever $(i,j)\in R$, $\phi b\in\frak C$.
Of course $\phi b$ is the unique member of $\frak C$ such that
$\pi_i\phi b=\phi_ib$ for every $i\in I$.   And $\phi:\frak B\to\frak A$ is
a Boolean homomorphism by 315Bb, so $\phi:\frak B\to\frak C$ is a Boolean
homomorphism.

\medskip

{\bf (b)}
This time, let $\frak A=\bigotimes_{i\in I}\frak A_i$ be the free
product of $\familyiI{\frak A_i}$;  for each $i\in I$, let
$\varepsilon_i:\frak A_i\to\frak A$ be the canonical map.   Let $J$ be the
ideal of $\frak A$ generated by elements of the form
$\varepsilon_i a\Bsymmdiff\varepsilon_j\pi_{ji}a$ where $(i,j)\in R$ and
$a\in\frak A_i$;  let $\frak C$ be the quotient algebra $\frak A/J$, and
set $\pi_ia=(\varepsilon_ia)^{\ssbullet}\in\frak C$ for $i\in I$ and
$a\in\frak A_i$.   Then every $\pi_i$ is a Boolean homomorphism, and if
$(i,j)\in R$ and $a\in\frak A_i$, then

\Centerline{$\pi_ia=(\varepsilon_ia)^{\ssbullet}
=(\varepsilon_j\pi_{ji}a)^{\ssbullet}
=\pi_j\pi_{ji}a$}

\noindent because $\varepsilon_ia\Bsymmdiff\varepsilon_j\pi_{ji}a$ belongs to
$J$.

Once again, suppose that $\frak B$ and $\familyiI{\phi_i}$ have the
properties declared in this part of the proposition.   By 315Jb, there is a
Boolean homomorphism $\tilde\phi:\frak A\to\frak B$ such that
$\tilde\phi\varepsilon_i=\phi_i$ for every $i\in I$.   Now the kernel of
$\tilde\phi$ includes $J$.   \Prf\ The kernel of $\tilde\phi$ is an ideal
of $\frak A$, so all we have to check is that it contains
$\varepsilon_ia\Bsymmdiff\varepsilon_j\pi_{ji}a$ whenever $(i,j)\in R$ and
$a\in\frak A_i$;  but in this case

\Centerline{$\tilde\phi(\varepsilon_ia\Bsymmdiff\varepsilon_j\pi_{ji}a)
=\tilde\phi\varepsilon_ia\Bsymmdiff\tilde\phi\varepsilon_j\pi_{ji}a
=\phi_ia\Bsymmdiff\phi_j\pi_{ji}a
=\phi_ia\Bsymmdiff\phi_ia
=0$.  \Qed}

Accordingly there is a unique ring homomorphism
$\phi:\frak C\to\frak B$ defined by saying that
$\phi c^{\ssbullet}=\tilde\phi c$ for every $c\in\frak A$ (3A2G).   As

\Centerline{$\phi 1_{\frak C}=\phi(1_{\frak A}^{\ssbullet})
=\tilde\phi 1_{\frak A}=1_{\frak B}$,}

\noindent $\phi$ is a Boolean homomorphism.   Now, of course,

\Centerline{$\phi\pi_ia=\phi(\varepsilon_ia)^{\ssbullet}
=\tilde\phi\varepsilon_ia=\phi_ia$}

\noindent whenever $i\in I$ and $a\in\frak A_i$.

To see that $\phi$ is unique, observe that if $\phi':\frak C\to\frak B$ has
the same property, then we have a Boolean homomorphism
$\tilde\phi':\frak A\to\frak B$ defined by setting
$\tilde\phi'c=\phi'c^{\ssbullet}$ for every $c\in\frak A$;  in which case

\Centerline{$\tilde\phi'\varepsilon_ia
=\phi'(\varepsilon_ia)^{\ssbullet}=\phi'\pi_ia=\phi_ia$}

\noindent whenever $i\in I$ and $a\in\frak A_i$, so that
$\tilde\phi'=\tilde\phi$ and $\phi'=\phi$.
}%end of proof of 315R

\leader{*315S}{Definitions} In 315Ra, we call $\frak A$, together with
$\familyiI{\pi_i}$, `the' {\bf projective limit} of
$(\familyiI{\frak A_i},\langle\pi_{ji}\rangle_{(i,j)\in R})$;  in
315Rb, we call $\frak A$, together with
$\familyiI{\pi_i}$, `the' {\bf inductive limit} of
$(\familyiI{\frak A_i},\langle\pi_{ji}\rangle_{(i,j)\in R})$.


\exercises{
\leader{315X}{Basic exercises (a)}
%\spheader 315Xa
Let $\langle\frak A_i\rangle_{i\in I}$ be any family of Boolean algebras,
with simple product $\frak A$,
and $\pi_i:\frak A\to\frak A_i$ the coordinate homomorphisms.   Suppose
we have another Boolean algebra $\frak A'$, with homomorphisms
$\pi'_i:\frak A'\to\frak A_i$, such that for every Boolean algebra
$\frak B$ and every family $\langle\phi_i\rangle_{i\in I}$ of
homomorphisms from $\frak B$ to the $\frak A_i$ there is a unique
homomorphism $\phi:\frak B\to\frak A'$ such that $\phi_i=\pi'_i\phi$ for
every $i$.   Show that there is a unique isomorphism $\psi:\frak
A\to\frak A'$ such that $\pi_i'\psi=\pi_i$ for every $i\in I$.
%315B

\spheader 315Xb\dvAnew{2012}
Let $\familyiI{\frak A_i}$ be a family of Boolean algebras
with simple product $\frak A=\prod_{i\in I}\frak A_i$.   (i) Show that
$\frak A$ is Dedekind complete iff every $\frak A_i$ is Dedekind complete.
(ii) Show that
$\frak A$ is Dedekind $\sigma$-complete iff every $\frak A_i$ is Dedekind
$\sigma$-complete.
%315C 

\spheader 315Xc\dvAnew{2012}
Let $\familyiI{\frak A_i}$ be a family of Boolean algebras
with simple product $\frak A=\prod_{i\in I}\frak A_i$.   Suppose that for
every $i\in I$ we are given a subalgebra $\frak B_i$ of $\frak A_i$.
(i) Show that the simple product $\frak B=\prod_{i\in I}\frak B_i$ is a
subalgebra of $\frak A$.   (ii) Show that $\frak B$ is order-closed in
$\frak A$ iff $\frak B_i$ is order-closed in $\frak A_i$ for every
$i\in I$.
%315C 

\spheader 315Xd Let $\langle P_i\rangle_{i\in I}$ be
a family of non-empty partially ordered sets, with product partially
ordered set $P$.   Show that $P$ is a lattice iff every $P_i$ is a
lattice, and that in this case it is the product lattice in the sense
that $p\vee q=\langle p(i)\vee q(i)\rangle_{i\in I}$, $p\wedge q=\langle
p(i)\wedge q(i)\rangle_{i\in I}$ for all $p$, $q\in P$.
%315D

\spheader 315Xe Let $\langle\frak A_i\rangle_{i\in I}$ be a
family of Boolean algebras with simple product $\frak A$.   For each
$i\in I$ let $Z_i$ be the Stone space of $\frak A_i$, and let $Z$ be the
Stone space of $\frak A$.   (i)
Show that the coordinate maps from $\frak A$
onto $\frak A_i$ induce homeomorphisms between the $Z_i$ and
open-and-closed subsets $Z_i^*$ of $Z$.   (ii) Show that
$\langle Z_i^*\rangle_{i\in I}$ is disjoint.   (iii) Show that
$\bigcup_{i\in I}Z_i^*$ is dense in $Z$,
and is equal to $Z$ iff $\{i:\frak A_i\ne\{0\}\}$ is finite.
%315G

\spheader 315Xf Let $\langle\frak A_i\rangle_{i\in I}$ be a
family of Boolean algebras, with simple product $\frak A$.   Suppose
that for each $i\in I$ we are given an ideal $I_i$ of $\frak A_i$.
Show that $I=\prod_{i\in I}I_i$ is an ideal of $\frak A$, and that
$\frak A/I$ may be identified, as Boolean algebra, with
$\prod_{i\in I}\frak A_i/I_i$.
%315G

\spheader 315Xg Let $\langle X_i\rangle_{i\in I}$ be any family
of topological spaces.   Let $X$ be their disjoint union
$\{(x,i):i\in I,\,x\in X_i\}$, with the disjoint union topology;  that
is, a set
$G\subseteq X$ is open in $X$ iff $\{x:(x,i)\in G\}$ is open in $X_i$
for every $i\in I$.   Show that the algebra of open-and-closed
subsets of $X$ can be identified, as Boolean algebra, with the simple
product of the algebras of open-and-closed sets of the $X_i$.
%315G

\spheader 315Xh Show that the topological product of any family of
zero-dimensional spaces is zero-dimensional.
%315I

\spheader 315Xi Let $\langle\frak A_i\rangle_{i\in I}$ be any
family of Boolean algebras, with free product $\frak A$, and
$\varepsilon_i:\frak A_i\to\frak A$ the canonical homomorphisms.
Suppose we have another Boolean algebra $\frak A'$, with homomorphisms
$\varepsilon'_i:\frak A_i\to\frak A'$, such that for every Boolean
algebra $\frak B$ and every family $\langle\phi_i\rangle_{i\in I}$ of
homomorphisms from the $\frak A_i$ to $\frak B$ there is a unique
homomorphism $\phi:\frak A'\to\frak B$ such that
$\phi_i=\phi\varepsilon'_i$ for every $i$.   Show that there is a unique
isomorphism $\psi:\frak A\to\frak A'$ such that
$\varepsilon_i'=\psi\varepsilon_i$ for every $i\in I$.
%315J

\spheader 315Xj Let $I$ be any set, and let $\frak A$ be the
algebra of open-and-closed sets of $\{0,1\}^I$;  for each $i\in I$ set
$a_i=\{x:x\in\{0,1\}^I,\,x(i)=1\}\in\frak A$.   Show that for any
Boolean algebra $\frak B$ and any family $\langle b_i\rangle_{i\in I}$ in
$\frak B$ there is a unique Boolean homomorphism
$\phi:\frak A\to\frak B$ such that $\phi(a_i)=b_i$ for every $i\in I$.
%315J

\spheader 315Xk Let $\langle\frak A_i\rangle_{i\in I}$,
$\langle\frak B_j\rangle_{j\in J}$ be two families of Boolean algebras.
Show that there is a natural injective homomorphism
$\phi:\prod_{i\in I}\frak A_i\otimes\prod_{j\in J}\frak B_j
\to\prod_{i\in I,j\in J}\frak A_i\otimes\frak B_j$ defined by saying
that

\Centerline{$\phi(a\otimes b)
=\langle a(i)\otimes b(j)\rangle_{i\in I,j\in J}$}

\noindent for $a\in\prod_{i\in I}\frak A_i$,
$b\in\prod_{j\in J}\frak B_j$.   Show that $\phi$ is surjective if $I$
and $J$ are finite.
%315K

\spheader 315Xl Let $\langle J(i)\rangle_{i\in I}$ be a family
of sets, with product $Q=\prod_{i\in I}J(i)$.   Let
$\langle\frak A_{ij}\rangle_{i\in I,j\in J(i)}$ be a family of Boolean
algebras.   Describe a natural injective homomorphism
$\phi:\bigotimes_{i\in I}\prod_{j\in J(i)}\frak A_{ij}
\to\prod_{q\in Q}\bigotimes_{i\in I}\frak A_{i,q(i)}$.
%315K, 315Xk

\spheader 315Xm Let $\frak A$ and $\frak B$ be Boolean algebras with
partitions of unity $\langle a_i\rangle_{i\in I}$,
$\langle b_j\rangle_{j\in J}$.   Show that
$\langle a_i\otimes b_j\rangle_{i\in I,j\in J}$ is a partition of unity
in $\frak A\otimes\frak B$.
%315K, 315Xk

\spheader 315Xn Let $\frak A$ and $\frak B$ be Boolean algebras and
$a\in\frak A$, $b\in\frak B$.   Write $\frak A_a$, $\frak B_b$ for the
corresponding principal ideals.   Show that there is a canonical
isomorphism between $\frak A_a\otimes\frak B_b$ and the principal ideal
of $\frak A\otimes\frak B$ generated by $a\otimes b$.
%315K, 315Xk

\spheader 315Xo Let $\langle\frak A_i\rangle_{i\in I}$ be any
family of Boolean algebras, with free product
$\bigotimes_{i\in I}\frak A_i$, and $\varepsilon_i:\frak A_i\to\frak A$
the canonical maps.   Show
that $\varepsilon_i[\frak A_i]$ is an order-closed subalgebra of
$\frak A$ for every $i$.
%315K

\spheader 315Xp Let $\frak A$ be a Boolean algebra.   Let us say
that a family $\langle\frak A_i\rangle_{i\in I}$ of subalgebras of
$\frak A$ is {\bf Boolean-independent} if $\inf_{j\in J}a_j\ne 0$
whenever $J\subseteq I$ is finite and $a_j\in\frak A_j\setminus\{0\}$
for every $j\in J$.   Show that in this case the subalgebra of $\frak A$
generated by $\bigcup_{i\in I}\frak A_i$ is isomorphic to the free
product $\bigotimes_{i\in I}\frak A_i$.
%315L

\spheader 315Xq Let $\langle\frak A_i\rangle_{i\in I}$ and
$\langle\frak B_i\rangle_{i\in I}$ be two families of Boolean algebras,
and suppose that for each $i\in I$ we are given a Boolean homomorphism
$\phi_i:\frak A_i\to\frak B_i$ with kernel
$K_i\normalsubgroup\frak A_i$.   Show that the $\phi_i$ induce a Boolean
homomorphism
$\phi:\bigotimes_{i\in I}\frak A_i\to\bigotimes_{i\in I}\frak B_i$ with
kernel generated by $\bigcup_{i\in I}\varepsilon[K_i]$, where
$\varepsilon_i:\frak A_i\to\frak A$ is the canonical homomorphism.
Show that if every $\phi_i$ is surjective, so is $\phi$.
%315M

\spheader 315Xr Let $\langle\frak A_i\rangle_{i\in I}$ be any
family of non-trivial Boolean algebras.   Show that if $J\subseteq I$
and $\frak B_j$ is a subalgebra of $\frak A_j$ for each $j\in J$,
then $\bigotimes_{j\in J}\frak B_j$ is canonically embedded as a
subalgebra of $\bigotimes_{i\in I}\frak A_i$.
%315M, 315Xq

\spheader 315Xs Let $\frak A$ and $\frak B$ be Boolean algebras, neither
$\{0\}$.   Show that any element of $\frak A\otimes\frak B$ is uniquely
expressible as $\sup_{i\in I}a_i\otimes b_i$ where
$\langle a_i\rangle_{i\in I}$ is a partition of unity in $\frak A$, with
no $a_i$
equal to $0$, and $b_i\ne b_j$ in $\frak B$ for $i\ne j$.
%315O


\leader{315Y}{Further exercises (a)}
%\spheader 315Ya
Let $\langle\frak A_i\rangle_{i\in I}$ and
$\langle\frak B_i\rangle_{i\in I}$ be two families of Boolean algebras,
and suppose that we are given Boolean homomorphisms
$\phi_i:\frak A_i\to\frak B_i$ for each $i$;  let
$\phi:\bigotimes_{i\in I}\frak A_i\to\bigotimes_{i\in I}\frak B_i$ be
the induced homomorphism.   (i)
Show that if every $\phi_i$ is order-continuous, so is $\phi$.   (ii)
Show that if every $\phi_i$ is sequentially order-continuous, so is
$\phi$.
%315J %see mt31bits

\spheader 315Yb Let $\familyiI{Z_i}$ be any family of topological spaces
with product $Z$.   For $i\in I$, $z\in Z$ set
$\tilde\varepsilon_i(z)=z(i)$.   Show that if $M\subseteq Z_i$ is
nowhere dense in $Z_i$ then $\tilde\varepsilon_i^{-1}[M]$ is nowhere
dense in $Z$.   Use this to prove 315Kc.
%315K

\spheader 315Yc Let $\langle\frak A_i\rangle_{i\in I}$ be a
family of Boolean algebras, and suppose that we are given subalgebras
$\frak B_i$ of $\frak A_i$ for each $i$;  set
$\frak A=\bigotimes_{i\in I}\frak A_i$ and
$\frak B=\bigotimes_{i\in I}\frak B_i$, and  let
$\phi:\frak B\to\frak A$ be the homomorphism induced by the embeddings
$\frak B_i\embedsinto\frak A_i$.   (i) Show that if every $\frak B_i$ is
order-closed in $\frak A_i$, then $\phi[\frak B]$ is order-closed in
$\frak A$.
(ii) Show that if every $\frak B_i$ is a $\sigma$-subalgebra of
$\frak A_i$, then $\phi[\frak B]$ is a $\sigma$-subalgebra in $\frak A$.
%315K %see mt31bits

\spheader 315Yd Let $\langle X_i\rangle_{i\in I}$ be a family of
topological spaces, with product $X$.   Let $\RO(X_i)$, $\RO(X)$ be
the corresponding regular open algebras.   Show that $\RO(X)$ can be
identified with the Dedekind completion of
$\bigotimes_{i\in I}\RO(X_i)$.
%315K

\spheader 315Ye Use the ideas of 315Xj and 315M to give an
alternative construction of `free product', for which 315J and
315K(e-ii) are true, and
which does not depend on the concept of Stone space nor on any other use
of the axiom of choice.   ({\it Hint\/}:  show that for any Boolean
algebra $\frak A$ there is a canonical surjection from the algebra
$\Cal E_{\frak A}$ onto $\frak A$, where $\Cal E_J$ is the algebra of
subsets of $\{0,1\}^J$ generated by sets of the form $\{x:x(j)=1\}$;
show that for such algebras $\Cal E_J$, at least, the method of
315I-315J can be used;  now apply the method of 315M to describe
$\bigotimes_{i\in I}\frak A_i$ as a quotient of $\Cal E_J$ where
$J=\{(a,i):i\in I,\,a\in\frak A_i\}$.   Finally check 315K(e-ii).)
%315M

\spheader 315Yf Let $\frak A$ and $\frak B$ be
Boolean algebras.   Show that $\frak A\otimes\frak B$ is Dedekind
complete iff {\it either} $\frak A=\{0\}$ {\it or} $\frak B=\{0\}$ {\it
or} $\frak A$ is finite and $\frak B$ is Dedekind complete {\it or}
$\frak B$ is finite and $\frak A$ is Dedekind complete.
%315P

\spheader 315Yg Let $\langle P_i\rangle_{i\in I}$ be any family
of partially ordered spaces.   (i) Give a construction of a partially
ordered space $P$, together with a family of order-preserving maps
$\varepsilon_i:P_i\to P$, such that whenever $Q$ is a partially ordered
set and $\phi_i:P_i\to Q$ is order-preserving for every $i\in I$, there
is a unique order-preserving map $\phi:P\to Q$ such that
$\phi_i=\phi\varepsilon_i$ for every $i$.  (ii) Show that $\phi$ will be
order-continuous iff every $\phi_i$ is.   (iii) Show that $P$ will be
Dedekind complete iff every $P_i$ is, but (except in trivial cases) is
not a lattice.
%315Q

\spheader 315Yh Let $\familyiI{\frak A_i}$
be a family of Boolean algebras, and $R$ a subset of $I\times I$;  suppose
that $\pi_{ji}:\frak A_i\to\frak A_j$ is a Boolean homomorphism for each
$(i,j)\in R$.   For each $i\in I$, let
$Z_i$ be the Stone space of $\frak A_i$;  for $(i,j)\in R$, let
$f_{ji}:Z_j\to Z_i$ be the continuous function corresponding to $\pi_{ji}$.
Show that the Stone space of the inductive limit of the system
$(\familyiI{\frak A_i},\langle\pi_{ji}\rangle_{(i,j)\in R})$ can be
identified with
$\{z:z\in\prod_{i\in I}Z_i$, $f_{ji}(z(j))=z(i)$ whenever $(i,j)\in R\}$.
%315S
}%end of exercises

\cmmnt{
\Notesheader{315} In this section I find myself asking for slightly more
sophisticated algebra than seems necessary elsewhere.   The point is
that simple products and free products are best regarded as defined by
the properties described in 315B and 315J.   That is, it is sometimes
right to think of {\it a} simple product of a family
$\langle\frak A_i\rangle_{i\in I}$ of Boolean algebras as being a structure $(\frak
A,\langle\pi_i\rangle_{i\in I})$ where $\frak A$ is a Boolean algebra,
$\pi_i:\frak A\to\frak A_i$ is a homomorphism for every $i\in I$, and
every family of homomorphisms from a Boolean algebra $\frak B$ to the
$\frak A_i$ can be uniquely represented by a single homomorphism from
$\frak B$ to $\frak A$.   Similarly, reversing the direction of the
homomorphisms, we can speak of {\it a} free product (it would be natural
to say `coproduct') $(\frak A,\langle\varepsilon_i\rangle_{i\in I})$ of
$\langle\frak A_i\rangle_{i\in I}$.   On such definitions, it is
elementary that any two simple products, or free products, are
isomorphic in the obvious sense (315Xa, 315Xi), and very general
arguments from abstract algebra, not
restricted to Boolean algebras (see {\smc Bourbaki 68}, IV.3.2), show
that they exist.   (But in order to prove such basic facts as that the
$\pi_i$ are surjective, or that the $\varepsilon_i$ are, except when the
construction collapses altogether, injective, we do of course have to
look at the special properties of Boolean algebras.)
Now in the case of simple products, the Cartesian product construction
is so direct and so familiar that there seems no need to trouble our
imaginations with any other.   But in the case of free products, things
are more complicated.   I have given primacy to the construction in
terms of Stone
spaces because I believe that this is the fastest route to
effective mental pictures.   But in some ways this approach seems to be
inappropriate.   If you take what in my view is a tenable position, and
say that a Boolean algebra is best regarded as the limit of its finite
subalgebras, then you might prefer a construction of a free product as
a limit of free products of finitely many finite subalgebras.   Or
you might feel that it is wrong to rely on the axiom of choice to prove
a result which certainly does not need it (see 315Ye).

Because I believe that the universal mapping theorem 315J is the right
basis for the study of free products, I am naturally led to use it as
the starting point for proofs of theorems about free products, as in
315L.   But 315K(e-ii) seems to lie deeper.   (Note, for instance, that
in 315M we {\it do} need the axiom of choice, in part (a-iii) of the proof,
since without it the product $\prod_{i\in I}X_i$ could be empty.)

Both `simple product' and `free product' are essentially algebraic
constructions involving the category of Boolean algebras and Boolean
homomorphisms, and any relationships with such concepts as
order-continuity can be regarded as accidental, in so far as there are
accidents in mathematics.   315Cb and 315D show
that simple products behave very straightforwardly when the
homomorphisms involved are order-continuous.   315Q, 315Xo and
315Ya-315Yc show that free products are much more complex and subtle.

For finite products, we have a kind of distributivity;
$(\frak A\times\frak B)\otimes\frak C$ can be identified with
$(\frak A\otimes\frak C)\times(\frak B\otimes\frak C)$ (315Xk, 315Xl).
There are contexts in which this makes it seem more natural to write
$\frak A\oplus\frak B$ in place of $\frak A\times\frak B$, and indeed I
have already spoken of a `direct sum' of measure spaces (214L) in terms
which correspond closely to the simple product of algebras of sets
described
in 315Ga.   Generally, the simple product corresponds to disjoint unions
of Stone spaces (315Xe) and the free product to products of Stone
spaces.   But the simple product is indeed the product Boolean algebra,
in the ordinary category sense;  the universal mapping theorem 315B is
exactly of the type we expect from products of topological spaces
(3A3Ib) or partially ordered sets (315Dc), etc.   It is the `free
product' which is special to Boolean algebras.   The nearest analogy
that I know
of elsewhere is with the concept of `tensor product' of linear spaces
(cf. \S253).

It is perhaps worth noting that projective limits of systems of Boolean
algebras have a straightforward description in terms of the algebras
themselves (315Ra), while
inductive limits have a similarly direct description in terms of Stone
spaces (315Yh).
}%end of notes

\discrpage

