\frfilename{mt353.tex}
\versiondate{28.2.04}
\copyrightdate{1998}
     
\def\chaptername{Riesz spaces}
\def\sectionname{Archimedean and Dedekind complete Riesz spaces}
     
\newsection{353}
     
I take a few pages over elementary properties of Archimedean
and Dedekind ($\sigma$)-complete Riesz spaces.
     
\leader{353A}{Proposition} Let $U$ be an Archimedean Riesz space.   Then
every quasi-order-dense Riesz subspace of $U$ is order-dense.
     
\proof{ Let $V\subseteq U$ be a quasi-order-dense Riesz subspace, and
$u\ge 0$ in $U$.   Set $A=\{v:v\in V,\,v\le u\}$.   \Quer\ Suppose, if
possible, that $u$ is not the least upper bound of $A$.   Then there is
a $u_1<u$ such that $v\le u_1$ for every $v\in A$.   Because $0\in A$,
$u_1\ge 0$.   Because $V$ is quasi-order-dense, there is a $v>0$ in $V$
such that $v\le u-u_1$.   Now $nv\le u_1$ for every $n\in\Bbb N$.
\Prf\ Induce on $n$.   For $n=0$ this is trivial.   For the inductive
step, given $nv\le u_1$, then $(n+1)v\le u_1+v\le u$, so $(n+1)v\in A$
and $(n+1)v\le u_1$.   Thus the induction proceeds.\ \QeD\ But this is
impossible, because $v>0$ and $U$ is supposed to be Archimedean.\ \Bang
     
So $u=\sup A$.   As $u$ is arbitrary, $V$ is order-dense.
}%end of proof of 353A
     
\vleader{108pt}{353B}{Proposition} Let $U$ be an Archimedean Riesz space.   Then
     
(a) for every $A\subseteq U$, the band generated by $A$ is
$A^{\perp\perp}$,
     
(b) every band in $U$ is complemented.
     
\proof{{\bf (a)} Let $V$ be the band generated by $A$.   Then $V$ is
surely included in $A^{\perp\perp}$, because this is a band including
$A$ (352O).   \Quer\ Suppose, if possible,
that $V\ne A^{\perp\perp}$.   Then there is a 
$w\in A^{\perp\perp}\setminus V$, so that $|w|\notin V$.   Set
$B=\{v:v\in V,\,v\le |w|\}$;  then $B$ is upwards-directed and
non-empty.  Because $V$ is order-closed, $|w|$ cannot be the supremum of
$A$, and there is a $u_0>0$ such that $|w|-u_0\ge v$ for every $v\in B$.
Now $u_0\wedge|w|\ne 0$, so $u_0\notin A^{\perp}$, and there is a
$u_1\in A$
such that $v=u_0\wedge|u_1|>0$.   In this case $nv\in B$ for every
$n\in\Bbb N$.   \Prf\ Induce on $n$.   For $n=0$ this is trivial.   For
the inductive step, given that $nv\in B$, then $nv\le |w|-u_0$ so $(n+1)v\le nv+u_0\le|w|$;  but also $(n+1)v\le nv+|u_1|\in V$, so $(n+1)v\in B$.\
\QeD\  But this means that $nv\le|w|$ for every $n$, which is
impossible, because $U$ is Archimedean.\ \BanG
     
\medskip
     
{\bf (b)} Now if $V\subseteq U$ is any band, it is surely the band
generated by itself, so is equal to $V^{\perp\perp}$, and is
complemented (352P).
}%end of proof of 353B
     
\cmmnt{\medskip
     
\noindent{\bf Remark} We may therefore speak of the {\bf band algebra}
of an Archimedean Riesz space, rather than the `complemented band
algebra' (352Q).}
     
\leader{353C}{Corollary} Let $U$ be an Archimedean Riesz space and $v\in
U$.   Let $V$ be the band in $U$ generated by $v$.   If $u\in U$, then
$u\in V$ iff there is no $w$ such that $0<w\le|u|$ and $w\wedge|v|=0$.
     
\proof{ By 353B, $V=\{v\}^{\perp\perp}$.
Now, for $u\in U$,
     
\Centerline{$u\notin V
\iff \,\exists\,w\in\{v\}^{\perp},\,|u|\wedge|w|>0
\iff \,\exists\,w\in\{v\}^{\perp},\,0<w\le|u|$.}
     
\noindent Turning this round, we have the condition announced.
}%end of proof of 353C
     
\leader{353D}{Proposition} Let $U$ be an Archimedean Riesz space and
$V$ an order-dense Riesz subspace of $U$.   Then the map $W\mapsto
W\cap V$ is an isomorphism between the band algebras of $U$ and $V$.
     
\proof{ If $W\subseteq U$ is a band, then $W\cap V$ is surely a band in
$V$ (it is order-closed in $V$ because it is the inverse image of the
order-closed set $W$ under the embedding $V\embedsinto U$, which is
order-continuous by 352Nc and 352Nb).   If $W$, $W'$ are distinct bands
in $U$, say
$W'\not\subseteq W$, then $W'\not\subseteq W^{\perp\perp}$, by 353B, so
$W'\cap W^{\perp}\ne\{0\}$;  because $V$ is order-dense, $V\cap W'\cap
W^{\perp}\ne\{0\}$, and $V\cap W'\ne V\cap W$.   Thus $W\mapsto W\cap V$
is injective.
     
If $Q\subseteq V$ is a band in $V$, then its complementary band in $V$
is just $Q^{\perp}\cap V$, where $Q^{\perp}$ is taken in $U$.   So
(because $V$, like $U$, is Archimedean, by 351Rc) 
$Q=(Q^{\perp}\cap V)^{\perp}\cap V=W\cap V$, where 
$W=(Q^{\perp}\cap V)^{\perp}$ is a band in $U$.   Thus
the map $W\mapsto W\cap V$ is an order-preserving bijection between the
two band algebras.   By 312M, it is a Boolean isomorphism, as claimed.
}%end of proof of 353D
     
\leader{353E}{Lemma} Let $U$ be an Archimedean Riesz space and
$V\subseteq U$ a band such that $\sup\{v:v\in V,\,0\le v\le u\}$ is
defined for every $u\in U^+$.   Then $V$ is a projection band.
     
\proof{ Take any $u\in U^+$ and set $v=\sup\{v':v'\in V^+,\,v'\le u\}$,
$w=u-v$.   $v\in V$ because $V$ is a band.   Also $w\in V^{\perp}$.
\Prf\Quer\ If not, there is some $v_0\in V^+$ such that $w\wedge v_0>0$.
Now for any $n\in \Bbb N$ we see that
     
\Centerline{$nv_0\le u\Longrightarrow nv_0\le v
\Longrightarrow (n+1)v_0\le v+w=u$,}
     
\noindent so an induction on $n$ shows that $nv_0\le u$ for every $n$;
which is impossible, because $U$ is supposed to be
Archimedean.\ \Bang\QeD\  Accordingly $u=v+w\in V+V^{\perp}$.   As $u$
is arbitrary,
$U^+\subseteq V+V^{\perp}$, and $V$ is a projection band (352R).
}%end of proof of 353E
     
\leader{353F}{Lemma} Let $U$ be an Archimedean Riesz space.   If
$A\subseteq U$ is non-empty and bounded above and $B$ is the set of its upper bounds, then $\inf(B-A)=0$.
\proof{ \Quer\ If not, let $w>0$ be a lower bound for $B-A$.   If 
$u\in A$ and $v\in B$, then $v-u\ge w$, that is, $u\le v-w$;  as $u$ is
arbitrary,
$v-w\in B$.   Take any $u_0\in A$, $v_0\in B$.   Inducing on $n$, we see
that $v_0-nw\in B$ for every $n\in\Bbb N$, so that $v_0-nw\ge u_0$,
$nw\le v_0-u_0$ for every $n$;  but this is impossible, because $U$ is supposed to be Archimedean.\ \Bang
}%end of proof of 353F
     
\leader{353G}{Dedekind completeness}\cmmnt{ Recall that a
partially ordered set $P$ is Dedekind ($\sigma$)-complete if (countable)
non-empty sets with upper and lower bounds have suprema and infima in
$P$ (314A).}  For a Riesz space $U$, $U$ is Dedekind complete iff every
non-empty upwards-directed subset of $U^+$ with an upper bound has a
least upper bound, and is Dedekind $\sigma$-complete iff every
non-decreasing sequence in $U^+$ with an upper bound has a least upper
bound.   \prooflet{\Prf\ (Compare 314Bc.)  (i) Suppose that any
non-empty
upwards-directed order-bounded subset of $U^+$ has an upper bound, and
that $A\subseteq U$ is any
non-empty set with an upper bound.   Take $u_0\in A$ and set
     
\Centerline{$B=\{u_0\vee u_1\vee\ldots\vee u_n-u_0:u_1,\ldots,u_n\in
A\}$.}
     
\noindent Then $B$ is an upwards-directed subset of $U^+$, and if $w$ is
an upper bound of $A$ then $w-u_0$ is an upper bound of $B$.   So $\sup
B$ is defined in $U$, and in this case $u_0+\sup B=\sup A$.   As $A$ is
arbitrary, $U$ is Dedekind complete.   (ii) Suppose that
order-bounded non-decreasing sequences in $U^+$ have suprema, and that
$A\subseteq U$ is any countable non-empty set with an upper bound.   Let
$\sequencen{u_n}$ be a sequence running over $A$, and set
$v_n=\sup_{i\le n}u_i-u_0$ for each $n$.   Then $\sequencen{v_n}$ is a
non-decreasing order-bounded sequence in $U^+$, and $u_0+\sup_{n\in\Bbb
N}v_n=\sup A$.   (iii) Finally, still supposing that order-bounded
non-decreasing sequences in $U^+$ have suprema, if $A\subseteq U$ is
non-empty, countable and bounded below, $\inf A$ will be defined and
equal to $-\sup(-A)$.\ \Qed}%end of prooflet
     
\leader{353H}{Proposition} Let $U$ be a Dedekind $\sigma$-complete Riesz
space.
     
(a) $U$ is Archimedean.
     
(b) For any $v\in U$ the band generated by $v$ is a projection band.
     
(c) If $u$, $v\in U$, then $u$ is uniquely expressible as $u_1+u_2$,
where $u_1$ belongs to the band generated by $v$ and $|u_2|\wedge|v|=0$.
     
\proof{{\bf (a)} Suppose that $u$, $v\in U$ are such that $nu\le v$ for
every $n\in\Bbb
N$.   Then $nu^+\le v^+$ for every $n$, and $A=\{nu^+:n\in\Bbb N\}$ is a
countable non-empty upwards-directed set with an upper bound;  say
$w=\sup A$.   Since $A+u^+\subseteq A$, $w+u^+=\sup(A+u^+)\le w$, and
$u\le u^+\le 0$.   As $u$, $v$ are arbitrary, $U$ is Archimedean.
     
\medskip
     
{\bf (b)} Let $V$ be the band generated by $v$.   Take any $u\in U^+$
and set $A=\{v':v'\in V,\,0\le v'\le u\}$.   Then
$\{u\wedge n|v|:n\in\Bbb N\}$ is a countable set with an upper bound, so has a supremum $u_1$ say
in $U$.   Now $u_1$ is an upper bound for $A$.   \Prf\ If $v'\in A$,
then
     
\Centerline{$v'=\sup_{n\in\Bbb N}v'\wedge n|v|\le u_1$}
     
\noindent by 352Vb.\ \QeD\   Since $u\wedge n|v|\in A\subseteq V$ for
every $n$, $u_1\in V$ and $u_1=\sup A$.
     
As $u$ is arbitrary, 353E tells us that $V$ is a projection band.
     
\medskip
     
{\bf (c)} Again let $V$ be the band generated by $v$.   Then
$\{v\}^{\perp\perp}$ is a band containing $v$, so
     
\Centerline{$\{v\}\subseteq V\subseteq\{v\}^{\perp\perp}$,
\quad$\{v\}^{\perp}\supseteq V^{\perp}
  \supseteq\{v\}^{\perp\perp\perp}=\{v\}^{\perp}$}
     
\noindent (352Od), and $V^{\perp}=\{v\}^{\perp}$.
     
Now, if $u\in U$, $u$ is uniquely expressible in the form $u_1+u_2$
where $u_1\in V$ and $u_2\in V^{\perp}$, by (b).   But
     
\Centerline{$u_2\in V^{\perp}\iff u_2\in\{v\}^{\perp}\iff
|u_2|\wedge|v|=0$.}
     
\noindent So we have the result.
}%end of proof of 353H
     
\leader{353I}{Proposition} In a Dedekind complete Riesz space, all bands
are projection bands.
     
\proof{ Use 353E, noting that the sets $\{v:v\in V,\,0\le v\le u\}$
there are always non-empty, upwards-directed and bounded above, so
always have suprema.
}%end of proof of 353I
     
\vleader{72pt}{353J}{Proposition} (a) Let $U$ be a Dedekind 
$\sigma$-complete Riesz space.
     
\quad(i) If $V$ is a solid linear subspace of $U$,   then $V$
is (in itself) Dedekind $\sigma$-complete.
     
\quad(ii) If $V$ is a
sequentially order-closed Riesz subspace of $U$ then $V$ is Dedekind
$\sigma$-complete.
     
\quad(iii) If $V$ is a sequentially
order-closed solid linear subspace of $U$, the canonical map from $U$ to
$V$ is sequentially order-continuous, and the quotient Riesz space $U/V$
also is Dedekind $\sigma$-complete.
     
(b) Let $U$ be a Dedekind complete Riesz space.
     
\quad(i) If $V$ is a solid
linear subspace of $U$, then $V$ is (in itself) Dedekind complete.
     
\quad(ii) If $V\subseteq U$ is an order-closed Riesz subspace then $V$
is Dedekind complete.
     
\proof{{\bf (a)(i)} If $\sequencen{u_n}$ is a non-decreasing sequence in
$V^+$ with an upper bound $v\in V$, then $w=\sup_{n\in\Bbb N}u_n$ is
defined in $U$;  but as $0\le w\le v$, $w\in V$ and 
$w=\sup_{n\in\Bbb N}u_n$ in $V$.   Thus $V$ is Dedekind $\sigma$-complete.
     
\medskip
     
{\bf (ii)} If $\sequencen{u_n}$ is a non-decreasing order-bounded
sequence in $W$, then $u=\sup_{n\in\Bbb N}u_n$ is defined in $U$;  but
because $V$ is sequentially order-closed, $u\in V$ and 
$u=\sup_{n\in\Bbb N}u_n$ in $V$.
     
\medskip
     
{\bf (iii)} Let $\sequencen{u_n}$ be a non-decreasing sequence in $U$
with supremum $u$.   Then of course $u^{\ssbullet}$ is an upper bound
for $A=\{u_n^{\ssbullet}:n\in\Bbb N\}$ in $U/V$.   Now let $p$ be any
other upper bound for $A$.   Express $p$ as $v^{\ssbullet}$.   Then for
each $n\in\Bbb N$ we have $u_n^{\ssbullet}\le p$, so that 
$(u_n-v)^+\in V$.   Because $V$ is sequentially order-closed, 
$(u-v)^+=\sup_{n\in\Bbb N}(u_n-v)^+\in V$ and $u^{\ssbullet}\le p$.   
Thus $u^{\ssbullet}$ is the least upper bound of $A$.   By 351Gb,
$u\mapsto u^{\ssbullet}$ is sequentially order-continuous.
     
Now suppose that $\sequencen{p_n}$ is a non-decreasing sequence in
$(U/V)^+$ with an upper bound $p\in (U/V)^+$.   Let $u\in U^+$ be such
that $u^{\ssbullet}=p$, and for each $n\in\Bbb N$ let $u_n\in U^+$ be
such that $u_n^{\ssbullet}=p_n$.   Set $v_n=u\wedge\sup_{i\le n}u_i$ for
each $n$;  then $v_n^{\ssbullet}=p_n$ for each $n$, and
$\sequencen{v_n}$ is a non-decreasing order-bounded sequence in $U$.
Set $v=\sup_{n\in\Bbb N}v_n$;  by the last paragraph,
$v^{\ssbullet}=\sup_{n\in\Bbb N}p_n$ in $U/V$.   As $\sequencen{p_n}$ is
arbitrary, $U/V$ is Dedekind $\sigma$-complete, as claimed.
     
\medskip
     
{\bf (b)} The argument is the same as parts (i) and (ii) of the proof of
(a).
}%end of proof of 353J
     
\leader{353K}{Proposition} Let $U$ be a Riesz space and $V$ a
quasi-order-dense Riesz subspace of $U$ which is (in itself) Dedekind
complete.   Then $V$ is a solid linear subspace of $U$.
     
\proof{ Suppose that $v\in V$, $u\in U$ and that $|u|\le|v|$.   Consider
$A=\{w:w\in V,\,0\le w\le u^+\}$.   Then $A$ is a non-empty subset of
$V$ with an upper bound in $V$ (viz., $|v|$).   So $A$ has a supremum
$v_0$ in $V$.   Because the embedding $V\embedsinto U$ is order-continuous
(352Nb), $v_0$ is the supremum of $A$ in $U$.   But as $V$ is
order-dense (353A),
$v_0=u^+$ and $u^+\in V$.   Similarly, $u^-\in V$ and $u\in V$.   As
$u$, $v$ are arbitrary, $V$ is solid.
}%end of proof of 353K
     
\leader{353L}{Order units} Let $U$ be a Riesz space.
     
\spheader 353La  An element $e$ of $U^+$ is an {\bf order unit} in $U$
if $U$ is the solid linear subspace of itself generated by $e$\cmmnt{;
that is, if for every $u\in U$ there is an $n\in\Bbb N$ such that
$|u|\le ne$}.
\prooflet{(For  the solid linear subspace generated by $v\in U^+$ is
$\bigcup_{n\in\Bbb N}[-nv,nv]$.)}
     
\spheader 353Lb An element $e$ of $U^+$ is a {\bf weak order unit} in
$U$ if $U$ is the principal band generated by $e$\cmmnt{;  that is, if
$u=\sup_{n\in\Bbb N}u\wedge ne$ for every $u\in U^+$ (352Vb)}.
     
Of course an order unit is a weak order unit.
     
\spheader 353Lc If $U$ is Archimedean, then an element $e$ of $U^+$ is a
weak order unit iff\prooflet{ $\{e\}^{\perp\perp}=U$ (353B), that is,
iff
$\{e\}^{\perp}=\{0\}$ (because
     
\Centerline{$\{e\}^{\perp}=\{0\}
\Longrightarrow\{e\}^{\perp\perp}=\{0\}^{\perp}=U
\Longrightarrow\{e\}^{\perp}=\{e\}^{\perp\perp\perp}=U^{\perp}=\{0\}$,)}
     
\noindent that is, iff} $u\wedge e>0$ whenever $u>0$.
     
\leader{353M}{Theorem} Let $U$ be an Archimedean Riesz space with order
unit $e$.   Then it can be embedded as an order-dense and norm-dense
Riesz subspace of $C(X)$, where $X$ is a
compact Hausdorff space, in such a way that $e$ corresponds to $\chi X$;
moreover, this embedding is essentially unique.
     
\cmmnt{\medskip
     
\noindent{\bf Remark} Here $C(X)$ is the space of all continuous
functions from $X$ to $\Bbb R$;  because $X$ is compact, they are all
bounded, so that $\chi X$ is an order unit in $C(X)=C_b(X)$.
}%end of comment
     
\proof{{\bf (a)} Let $X$ be the set of Riesz homomorphisms $x$ from $U$
to $\Bbb R$ such that $x(e)=1$.   Define $T:U\to\Bbb R^X$ by setting
$(Tu)(x)=x(u)$ for $x\in X$, $u\in U$;  then it is easy to check that
$T$ is a Riesz homomorphism, just because every member of $X$ is a Riesz
homomorphism, and of course $Te=\chi X$.
     
\medskip
     
{\bf (b)} The key to the proof is the fact that $X$ separates the points
of $U$, that is, that $T$ is injective.   I choose the following method
to show this.
Suppose that $w\in U$ and $w>0$.   Because $U$ is Archimedean, there is
a $\delta>0$ such that $(w-\delta e)^+\ne 0$.   Now there is an $x\in X$ such that $x(w)\ge\delta$.
\Prf\ (i) By 351O, there is a solid linear subspace $V$ of $U$ such that
$(w-\delta e)^+\notin V$ and whenever $u\wedge v=0$ in $U$ then one of
$u$, $v$ belongs to $V$.   (ii) Because $V\ne U$, $e\notin V$, so no
non-zero multiple of $e$ can belong to $V$.   Also observe that if $u$,
$v\in U\setminus V$, then one of $(u-v)^+$, $(v-u)^+$ must belong to
$V$, while neither $u=u\wedge v+(u-v)^+$ nor $v=u\wedge v+(v-u)^+$ does;
so $u\wedge v\notin V$.   (iii) For each $u\in U$ set
$A_u=\{\alpha:\alpha\in\Bbb R,\,(u-\alpha e)^+\in V\}$.   Then
     
\Centerline{$\alpha\ge\beta\in A_u
\Longrightarrow 0\le(u-\alpha e)^+\le(u-\beta e)^+\in V
\Longrightarrow \alpha\in A_u$.}
     
\noindent Also $A_u$ is
non-empty and bounded below, because if $\alpha\ge 0$ is such that
$-\alpha e\le u\le\alpha e$ then $\alpha\in A_u$ and $-\alpha-1\notin
A_u$ (since $(u-(-\alpha-1)e)^+\ge e\notin V$).
(iv) Set $x(u)=\inf A_u$ for every $u\in U$;  then $\alpha\in
A_u$ for every $\alpha>x(u)$, $\alpha\notin A_u$ for every
$\alpha<x(u)$.   (v) If $u$, $v\in U$, $\alpha>x(u)$ and $\beta>x(v)$
then
     
\Centerline{$((u+v)-(\alpha+\beta)e)^+\le(u-\alpha e)^++(v-\beta e)^+
\in V$}
     
\noindent (352Fc), so $\alpha+\beta\in A_{u+v}$;  as $\alpha$ and
$\beta$ are arbitrary, $x(u+v)\le x(u)+x(v)$.   (vi) If $u$, $v\in U$
and $\alpha<x(u)$, $\beta<x(v)$ then
     
\Centerline{$((u+v)-(\alpha+\beta)e)^+
\ge(u-\alpha e)^+\wedge(v-\beta e)^+\notin V$,}
     
\noindent using (ii) of this argument and 352Fc, so
$\alpha+\beta\notin A_{u+v}$.   As $\alpha$ and $\beta$ are arbitrary,
$x(u+v)\ge x(u)+x(v)$.   (vii) Thus $x:U\to\Bbb R$ is additive.   (viii)
If $u\in U$, $\gamma>0$ then
     
\Centerline{$\alpha\in A_u
\Longrightarrow (\gamma u-\alpha\gamma e)^+=\gamma(u-\alpha e)^+\in V
\Longrightarrow \gamma\alpha\in A_{\gamma u}$;}
     
\noindent thus $A_{\gamma u}\supseteq\gamma A_u$;  similarly,
$A_u\supseteq\gamma^{-1}A_{\gamma u}$ so $A_{\gamma u}=\gamma A_u$ and
$x(\gamma u)=\gamma x(u)$.   (ix) Consequently $x$ is linear, since we
know already from (vii) that $x(0u)=0.x(u)$, $x(-u)=-x(u)$.   (x) If
$u\ge 0$ then $u+\alpha e\ge\alpha e\notin V$ for every $\alpha>0$, that
is, $-\alpha\notin A_u$ for every $\alpha>0$, and $x(u)\ge 0$;  thus $x$
is a positive linear functional.   (xi) If $u\wedge v=0$, then one of
$u$, $v$ belongs to $V$, so $\min(x(u),x(v))\le 0$ and (using (x))
$\min(x(u),x(v))=0$;  thus $x$ is a Riesz homomorphism (352G(iv)).
(xii) $A_e=\coint{1,\infty}$ so $x(e)=1$.   Thus $x\in X$.   (xiii)
$\delta\notin A_w$ so $x(w)\ge\delta$.\ \Qed
     
\medskip
     
{\bf (c)} Thus $Tw\ne 0$ whenever $w>0$;  consequently $|Tw|=T|w|\ne 0$
whenever $w\ne 0$, and $T$ is injective.   I now
have to define the topology of $X$.   This is just the subspace topology
on $X$ if we regard $X$ as a subset of $\BbbR^U$ with its product
topology.
To see that $X$ is compact, observe that if for each $u\in U$ we choose
an $\alpha_u$ such that $|u|\le\alpha_ue$, then $X$ is a subspace of
$Q=\prod_{u\in U}[-\alpha_u,\alpha_u]$.
Because $Q$ is a product of compact spaces, it is compact, by
Tychonoff's theorem (3A3J).   Now $X$ is a closed subset of $Q$.
\Prf\ $X$ is just the intersection of the sets
     
\Centerline{$\{x:x(u+v)=x(u)+x(v)\}$,
\quad$\{x:x(\alpha u)=\alpha x(u)\}$,}
     
\Centerline{$\{x:x(u^+)=\max(x(u),0)\}$,
\quad$\{x:x(e)=1\}$}
     
\noindent as $u$, $v$ run over $U$ and $\alpha$ over $\Bbb R$;  and each
of these is closed, so $X$ is an intersection of closed sets and
therefore itself closed.\ \QeD\  Consequently $X$ also is compact.
Moreover, the coordinate functionals $x\mapsto x(u)$ are continuous on
$Q$, therefore on $X$ also, that is, $Tu:X\to\Bbb R$ is a continuous
function for every $u\in U$.
     
Note also that because $Q$ is a product of Hausdorff spaces, $Q$ and $X$
are Hausdorff (3A3Id).
     
\medskip
     
{\bf (d)} So $T$ is a Riesz homomorphism from $U$ to $C(X)$.   Now
$T[U]$ is a Riesz subspace of $C(X)$, containing $\chi X$, and such that
if $x$, $y\in X$ are distinct there is an $f\in T[U]$ such that
$f(x)\ne f(y)$ (because there is surely a $u\in U$ such that 
$x(u)\ne y(u)$).   By the Stone-Weierstrass theorem (281A), $T[U]$ is
$\|\,\|_{\infty}$-dense in $C(X)$.
     
Consequently it is also order-dense.   \Prf\ If $f>0$ in $C(X)$, set
$\epsilon=\bover13\|f\|_{\infty}$, and let $u\in U$ be such that
$\|f-Tu\|_{\infty}\le\epsilon$;  set $v=(u-\epsilon e)^+$.   Since
     
\Centerline{$0<(f-2\epsilon\chi X)^+\le(Tu-\epsilon\chi X)^+\le f^+=f$,}
     
\noindent $0<Tv\le f$.   As $f$ is arbitrary, $T[U]$ is
quasi-order-dense, therefore order-dense (353A).\ \Qed
     
\medskip
     
{\bf (e)} I have still to show that the representation is (essentially)
unique.   Suppose, then, that we have another representation of $U$ as a
norm-dense Riesz subspace of $C(Z)$, with $e$ this time corresponding to
$\chi Z$;  to simplify the notation, let us suppose that $U$ is actually
a subspace of $C(Z)$.   Then for each $z\in Z$, we have a functional
$\hat z:U\to\Bbb R$ defined by setting $\hat z(u)=u(z)$ for every $u\in
U$;  of course $\hat z$ is a Riesz homomorphism such that $\hat z(e)=1$,
that is, $\hat z\in X$.   Thus we have a function $z\mapsto\hat z:Z\to
X$.   For any $u\in U$, the function
$z\mapsto\hat z(u)=u(z)$ is continuous, so the function $z\mapsto\hat z$
is continuous (3A3Ib).   If $z_1$, $z_2$ are distinct members of $Z$,
there is an $f\in C(Z)$ such that $f(z_1)\ne f(z_2)$ (3A3Bf);  now
there is a $u\in U$ such that
$\|f-u\|_{\infty}\le\bover13|f(z_1)-f(z_2)|$, so that $u(z_1)\ne u(z_2)$
and $\hat z_1\ne\hat z_2$.   Thus $z\mapsto\hat z$ is injective.
Finally, it is also surjective.   \Prf\ Suppose that $x\in X$.   Set
$V=\{u:u\in U,\,x(u)=0\}$;  then $V$ is a solid linear subspace of $U$
(352Jb), not containing $e$.   For $z\in V^+$ set $G_v=\{z:v(z)>1\}$.
Because $e\notin V$, $G_v\ne Z$.    $\Cal G=\{G_v:v\in V^+\}$ is an
upwards-directed family of open sets in $Z$, not containing $Z$;
consequently, because $Z$ is compact, $\Cal G$ cannot be an open cover
of $Z$.   Take $z\in Z\setminus\bigcup\Cal G$.   Then $v(z)\le 1$ for
every $v\in V^+$;  because $\alpha|v|\in V^+$ whenever $v\in V$,
$\alpha\ge 0$, we must have $v(z)=0$ for every $v\in V$.   Now, given
any $u\in U$, consider $v=u-x(u)e$.   Then $x(v)=0$ so $v\in V$ and
$v(z)=0$, that is,
     
\Centerline{$u(z)=(v+x(u)e)(z)=v(z)+x(u)e(z)=x(u)$.}
     
\noindent  As $u$ is arbitrary, $\hat z=x$;  as $x$ is arbitrary, we
have the result.\ \Qed
     
Thus $z\mapsto\hat z$ is a continuous bijection from the compact
Hausdorff space $Z$ to the compact Hausdorff space $X$;  it must
therefore be a homeomorphism (3A3Dd).
     
This argument shows that if $U$ is embedded as a norm-dense Riesz
subspace of $C(Z)$, where $Z$ is compact and Hausdorff, then $Z$ must be
homeomorphic to $X$.   But it shows also that a homeomorphism is
canonically defined by the embedding;  $z\in Z$ corresponds to the Riesz
homomorphism $u\mapsto u(z)$ in $X$.
}%end of proof of 353M
     
\leader{353N}{Lemma} Let $U$ be a Riesz space, $V$ an Archimedean Riesz
space and $S$, $T:U\to V$ Riesz homomorphisms such that $Su\wedge Tu'=0$
in $V$ whenever $u\wedge u'=0$ in $U$.   Set $W=\{u:Su=Tu\}$.   Then $W$
is a solid linear subspace of $U$;  if $S$ and $T$ are order-continuous,
$W$ is a band.
     
\proof{{\bf (a)} It is easy to check that, because $S$ and $T$ are Riesz
homomorphisms, $W$ is a Riesz subspace of $U$.
     
\medskip
     
{\bf (b)} If $w\in W$ and $0\le u\le w$ in $U$, then $Su\le Tu$.
\Prf\Quer\ Otherwise, set $e=Sw=Tw$, and let $V_e$ be the solid linear
subspace of $V$ generated by $e$, so that $V_e$ is an Archimedean Riesz
space with order unit, containing both $Su$ and $Tu$.   By 353M (or its
proof), there is a Riesz homomorphism $x:V_e\to\Bbb R$ such that
$x(e)=1$ and $x(Su)>x(Tu)$.   Take $\alpha$ such that
$x(Su)>\alpha>x(Tu)$, and consider $u'=(u-\alpha w)^+$,
$u''=(\alpha w-u)^+$.    Then
     
\Centerline{$x(Su')=\max(0,x(Su)-\alpha x(Sw))=\max(0,x(Su)-\alpha)>0$,}
     
\Centerline{$x(Tu'')=\max(0,\alpha x(Tw)-x(Tu))
=\max(0,\alpha-x(Tu))>0$,}
     
\noindent so
     
\Centerline{$x(Su'\wedge Tu'')=\min(x(Su'),x(Tu''))>0$}
     
\noindent and $Su'\wedge Tu''>0$, while $u'\wedge u''=0$.\ \Bang\Qed
     
Similarly, $Tu\le Su$ and $u\in W$.   As $u$ and $w$ are arbitrary, $W$
is a solid linear subspace.
     
\medskip
     
{\bf (c)} Finally, suppose that $S$ and $T$ are order-continuous, and
that $A\subseteq W$ is a non-empty upwards-directed set with supremum
$u$ in $U$.   Then
     
\Centerline{$Su=\sup S[A]=\sup T[A]=Tu$}
     
\noindent and $u\in W$.   As $u$ and $A$ are arbitrary, $W$ is a band
(352Ob).
}%end of proof of 353N
     
\leader{353O}{\dvrocolon{$f$-algebras}}\cmmnt{ I give two results on
$f$-algebras, intended to clarify the connexions between the
multiplicative and lattice structures of the Riesz spaces in Chapter 36.
     
\wheader{353O}{4}{2}{2}{72pt}

\noindent}{\bf Proposition} Let $U$ be an Archimedean
$f$-algebra\cmmnt{ (352W)}.   Then
     
(a) the multiplication is separately
order-continuous in the sense that the maps $u\mapsto u\times w$,
$u\mapsto w\times u$ are
order-continuous for every $w\in U^+$;
     
(b) the multiplication is commutative.
     
\proof{{\bf (a)} Let $A\subseteq U$ be a non-empty set with infimum $0$,
and $v_0\in U^+$ a lower bound for $\{u\times w:u\in A\}$.   Fix $u_0\in
A$.   If $u\in A$ and $\delta>0$, then
$v_0\wedge(u_0-\bover1{\delta}u)^+\le\delta
u_0\times w$.   \Prf\ Set $v=v_0\wedge(u_0-\bover1{\delta}u)^+$.   Then
     
\Centerline{$\delta v\wedge(u-\delta u_0)^+
\le(\delta u_0-u)^+\wedge(u-\delta u_0)^+=0$,}
     
\noindent so $v\wedge(u-\delta u_0)^+=0$ and $v\wedge((u-\delta
u_0)^+\times w)=0$.   But
     
\Centerline{$v\le v_0\le u\times w
\le(u-\delta u_0)^+\times w+\delta u_0\times w$,}
     
\noindent so
     
\Centerline{$v
\le((u-\delta u_0)^+\times w)\wedge v+(\delta u_0\times w)\wedge v
\le\delta u_0\times w$,}
     
\noindent by 352Fa.\ \Qed

\woddheader{353O}{0}{0}{0}{24pt}
     
Taking the infimum over $u$, and using the distributive laws (352E), we
get
     
\Centerline{$v_0\wedge u_0\le\delta u_0\times w$.}
     
\noindent Taking the infimum over $\delta$, and using the hypothesis
that $U$ is Archimedean,
     
\Centerline{$v_0\wedge u_0=0$.}
     
\noindent But this means that $v_0\wedge(u_0\times w)=0$, while
$v_0\le u_0\times w$, so $v_0=0$.   As $v_0$ is arbitrary, $\inf_{u\in
A}u\times w=0$;  as $A$ is arbitrary, $u\mapsto u\times w$ is
order-continuous.   Similarly, $u\mapsto w\times u$ is order-continuous.
     
\medskip
     
{\bf (b)(i)} Fix $v\in U^+$, and for $u\in U$ set
     
\Centerline{$Su=u\times v$,\quad $Tu=v\times u$.}
     
\noindent Then $S$ and $T$ are both order-continuous Riesz homomorphisms
from $U$ to itself (352W(b-iv) and (a) above).   Also, $Su\wedge Tu'=0$
whenever $u\wedge u'=0$.   \Prf\
     
\Centerline{$0=(u\times v)\wedge u'=(u\times v)\wedge(v\times u')$.
\Qed}
     
\noindent So $W=\{u:u\times v=v\times u\}$ is a band in $U$ (353N).
Of course $v\in W$ (because $Sv=Tv=v^2$).   If $u\in W^{\perp}$, then
$v\wedge|u|=0$ so $Su=Tu=0$ (352W(b-i)), and $u\in W$;  but this means
that $W^{\perp}=\{0\}$ and $W=W^{\perp\perp}=U$ (353Bb).   Thus $v\times
u=u\times v$ for every $u\in U$.
     
\medskip
     
\quad{\bf (ii)} This is true for every $v\in U^+$.   Of course it
follows that $v\times u=u\times v$ for every $u$, $v\in U$, so that
multiplication is commutative.
}%end of proof of 353O
     
\leader{353P}{Proposition} Let $U$ be an Archimedean $f$-algebra with
multiplicative identity $e$.
     
(a) $e$ is a weak order unit in $U$.
     
(b) If $u$, $v\in U$ then $u\times v=0$ iff $|u|\wedge|v|=0$.
     
(c) If $u\in U$ has a multiplicative inverse $u^{-1}$ then $|u|$ also
has a multiplicative inverse;  if $u\ge 0$ then $u^{-1}\ge 0$ and $u$ is
a weak order unit.
     
(d) If $V$ is another Archimedean $f$-algebra with multiplicative
identity $e'$, and $T:U\to V$ is a positive linear operator such that
$Te=e'$, then $T$ is a Riesz homomorphism iff $T(u\times v)=Tu\times Tv$
for all $u$, $v\in U$.
     
\proof{{\bf (a)} $e=e^2\ge 0$ by 352W(b-ii).   If $u\in U$ and
$e\wedge|u|=0$ then $|u|=(e\times|u|)\wedge|u|=0$;  by 353Lc, $e$ is a
weak order unit.
     
\medskip
     
{\bf (b)} If $|u|\wedge|v|=0$ then $u\times v=0$, by 352W(b-i).   If
$w=|u|\wedge|v|>0$, then $w^2\le|u|\times|v|$.   Let $n\in\Bbb N$ be
such that $nw\not\le e$, and set $w_1=(nw-e)^+$, $w_2=(e-nw)^+$.   Then
     
$$\eqalign{0
&\ne w_1
=w_1\times e
=w_1\times w_2+w_1\times(e\wedge nw)\cr
&=w_1\times(e\wedge nw)
\le (nw)^2
\le n^2|u|\times|v|
=n^2|u\times v|,\cr}$$
     
\noindent so $u\times v\ne 0$.
     
\medskip
     
{\bf (c)} $u\times u^{-1}=e$ so $|u|\times|u^{-1}|=|e|=e$ (352W(b-iii)),
and $|u^{-1}|=|u|^{-1}$.   (Recall that inverses in any semigroup with
identity are unique, so that we need have no inhibitions in using the
formulae $u^{-1}$, $|u|^{-1}$.)
     
Now suppose that $u\ge 0$.
Then $u^{-1}=|u^{-1}|\ge 0$.   If $u\wedge|v|=0$ then
     
\Centerline{$e\wedge|v|=(u\times u^{-1})\wedge|v|=0$,}
     
\noindent so $v=0$;  accordingly $u$ is a weak order unit.
     
\medskip
     
{\bf (d)(i)} If $T$ is multiplicative, and $u\wedge v=0$ in $U$, then
$Tu\times Tv=T(u\times v)=0$ and $Tu\wedge Tv=0$, by (b).  So $T$ is a
Riesz homomorphism, by 352G.
     
\medskip
     
\quad{\bf (ii)} Accordingly I shall henceforth assume that $T$ is a
Riesz homomorphism and seek to show that it is multiplicative.
     
If $u$, $v\in U^+$, then $T(u\times v)$ and $Tu\times Tv$ both belong to
the band generated by $Tu$.
\Prf\ Write $W$ for this band.   ($\alpha$) For any $n\ge 1$ we have
$(v-ne)^2\ge 0$, that is, $2nv\le v^2+n^2e$, so
     
\Centerline{$n(v-ne)\le 2nv-n^2e\le v^2$.}
     
\noindent Consequently
     
\Centerline{$T(u\times v)-nTu
=T(u\times v)-nT(u\times e)
=T(u\times(v-ne))
\le\Bover1nT(u\times v^2)$}
     
\noindent because $v'\mapsto T(u\times v')$ is a positive linear
operator;  as $V$ is Archimedean,
$\inf_{n\in\Bbb N}(T(u\times v)-nTu)^+=0$ and $T(u\times
v)=\sup_{n\in\Bbb N}T(u\times v)\wedge nTu$ belongs to $W$.
($\beta$) If $w\wedge|Tu|=0$ then
     
\Centerline{$w\wedge|Tu\times Tv|=w\wedge(|Tu|\times|Tv|)=0$;}
     
\noindent so $Tu\times Tv\in W^{\perp\perp}=W$.\ \Qed
     
\medskip
     
\quad{\bf (iii)} Fix
$v\in U^+$.   For $u\in U$, set $S_1u=Tu\times Tv$ and $S_2u=T(u\times
v)$.   Then $S_1$ and $S_2$ are both Riesz homomorphisms from $U$ to
$V$.   If $u\wedge u'=0$ in $U$, then $S_1u\wedge S_2u'=0$ in $V$,
because (by (ii) just above) $S_1u$ belongs to the band generated by
$Tu$, while $S_2u'$ belongs to the band generated by $Tu'$, and
$Tu\wedge Tu'=T(u\wedge u')=0$.   By 353N, $W=\{u:S_1u=S_2u\}$ is a
solid linear subspace of $U$.   Of course it contains $e$, since
     
\Centerline{$S_1e=Te\times Tv=e'\times Tv=Tv=T(e\times v)=S_2e$.}
     
\noindent In fact $u\in W$ for every $u\in U^+$.   \Prf\ As noted in
(ii) just above,
$u-ne\le\bover1nu^2$ for every $n\ge 1$.   So
     
$$\eqalign{|S_1u-S_2u|
&=|S_1(u-ne)^++S_1(u\wedge ne)-S_2(u-ne)^+-S_2(u\wedge ne)|\cr
&\le S_1(u-ne)^+ + S_2(u-ne)^+
\le\Bover1n(S_1u^2+S_2u^2)\cr}$$
     
\noindent for every $n\ge 1$, and $|S_1u-S_2u|=0$, that is,
$S_1u=S_2u$.\ \Qed
     
So $W=U$, that is, $Tu\times Tv=T(u\times v)$ for every $u\in U$.  And
this is true for every $v\in U^+$.   It follows at once that it is true
for every $v\in U$, so that $T$ is multiplicative, as claimed.
}%end of proof of 353P

\leader{353Q}{Proposition}\dvAnew{2010} %out of order query
Let $U$ be a Riesz space and $V$ an
order-dense Riesz subspace of $U$.   If $V$ is Archimedean, so is $U$.

\proof{ \Quer\ Otherwise, let $u'$, $u\in U$ be such that $u'>0$ and
$nu'\le u$ for every $n\in\Bbb N$.   Let $v'\in V$ be such that 
$0<v'\le u'$;  set $\tilde u=u-v'$;  let $v\in V$ be such that $v\le u$ 
but $v\not\le\tilde u$.   (This
is where we need $V$ to be order-dense rather than just quasi-order-dense.)
Let $w\in V$ be such that $w>0$ and $w\le(v-\tilde u)^+$;  note that
$w\le u-\tilde u=v'$.

Because $V$ is Archimedean, there is an $n\ge 1$ such that $nw\not\le v$.
In this case, 

\Centerline{$0<(nw-v)^+\le((n+1)v'-(v+v'))^+\le (u-(v+v'))^+
=(\tilde u-v)^+$}

\noindent but

\Centerline{$(nw-v)^+\wedge(\tilde u-v)^+\le nw\wedge n(\tilde u-v)^+
=n((v-\tilde u)^+\wedge(\tilde u-v)^+)=0$,}

\noindent which is impossible.\ \Bang
}%end of proof of 353Q
     
\exercises{
\leader{353X}{Basic exercises $\pmb{>}$(a)}
%\spheader 353Xa
Let $U$ be a Riesz space in which every band is complemented.   Show
that $U$ is Archimedean.
%353B
     
\spheader 353Xb A Riesz space $U$ has the {\bf principal projection
property} iff the
band generated by any single member of $U$ is a projection band.   Show
that any Dedekind $\sigma$-complete Riesz space has the principal
projection property, and that any Riesz space with the principal
projection property is Archimedean.
%353H
     
\sqheader 353Xc Fill in the missing part (b-iii) of 353J.
%353J
     
\spheader 353Xd Let $U$ be an Archimedean $f$-algebra with an
order-unit which is a multiplicative identity.   Show that $U$ can be
identified, as $f$-algebra, with a subspace of $C(X)$ for some compact
Hausdorff space
$X$.
%353P
     
\leader{353Y}{Further exercises (a)}
%\spheader 353Ya
Let $U$ be a Riesz space in which every quasi-order-dense solid linear
subspace is order-dense.   Show that $U$ is Archimedean.
%353A
     
\spheader 353Yb Let $X$ be a completely regular Hausdorff space.   Show
that $C(X)$ is Dedekind complete iff $C_b(X)$ is Dedekind complete iff
$X$ is extremally disconnected.
%353G
     
\spheader 353Yc Let $X$ be a compact Hausdorff space.   Show that $C(X)$
is Dedekind $\sigma$-complete iff $\overline{G}$ is open for every
cozero set $G\subseteq X$.   (Cf.\ 314Yf.)   Show that in this case $X$
is zero-dimensional.
%353Yb
%works for normal spaces I think, but these not yet defined
     
\spheader 353Yd Let $U$ be an Archimedean Riesz space such that
$\{u_n:n\in\Bbb N\}$ has
a supremum in $U$ whenever $\sequencen{u_n}$ is a sequence in $U$ such
that $u_m\wedge u_n=0$ whenever $m\ne n$.   Show that $U$ has the
principal projection property, but need not be Dedekind
$\sigma$-complete.
%353H, 353Xa
     
\spheader 353Ye Let $U$ be an Archimedean Riesz space.   Show that the
following are equiveridical:  (i) $U$ has the countable sup property
(241Ye) (ii) for every
$A\subseteq U$ there is a countable $B\subseteq A$ such that $A$ and $B$
have the same upper bounds;  
(iii) every order-bounded disjoint subset of $U^+$ is countable.
%353K+
     
\spheader 353Yf Let $U$ be an Archimedean $f$-algebra.   Show that an
element $e$ of $U$ is a multiplicative identity iff $e^2=e$ and $e$ is a
weak order unit.   \Hint{start by showing that under these conditions,
$e\times u=0\Rightarrow u=0$.}
%353O
     
\spheader 353Yg Let $U$ be an Archimedean $f$-algebra with a
multiplicative identity.   Show that if $u\in U$ then $u$ is invertible
iff $|u|$ is invertible.
%353P
% u^{-1} = u\times|u|^{-2}
}%end of exercises
     
\cmmnt{
\Notesheader{353} As in the last section, many of the results above have
parallels in the theory of Boolean algebras;  thus 353A corresponds to
313K, 353G corresponds in part to remarks in 314Bc and 314Xa, and
353J corresponds to 314C-314E.   Riesz spaces are more complicated;
for instance, principal ideals in Boolean algebras are straightforward,
while in Riesz spaces we have to distinguish between the solid linear
subspace generated by an element and the band generated by the same
element.   Thus an `order unit' in a Boolean ring would just be an
identity, while in a Riesz space we must distinguish between `order
unit' and `weak
order unit'.   As this remark may suggest to you, (Archimedean) Riesz
spaces are actually closer in spirit to arbitrary Boolean rings than to
the Boolean
algebras we have been concentrating on so far;  to the point that in
\S361 below I will return briefly to general Boolean rings.
     
Note that the standard definition of `order-dense' in Boolean
algebras, as given in 313J, corresponds to the definition of
`quasi-order-dense' in Riesz spaces (352Na);  the point here being that
Boolean algebras behave like Archimedean Riesz spaces, in which there is
no need to make a distinction.
     
I give the representation theorem 353M more for completeness than
because we need it in any formal sense.   In 351Q and 352L I have given
representation theorems for general partially ordered linear spaces, and
general Riesz spaces, as quotients of spaces of functions;  in 368F
below I give a theorem for Archimedean Riesz spaces corresponding rather
more closely to the expressions of the $L^p$ spaces as quotients of
spaces of measurable functions.   In 353M, by contrast, we have a
theorem expressing Archimedean Riesz spaces with order units as true
spaces of functions, rather than as spaces of equivalence classes of
functions.   All these theorems are important in forming an appropriate
mental picture of ordered linear spaces, as in 352M.
     
I give a bare-handed proof of 353M, using only the Riesz space structure
of $C(X)$;  if you know a little about extreme points of dual unit balls
you can approach from that direction instead, using 354Yj.   The point
is that (as part (d) of the proof of 353M 
makes clear) the space $X$ can be
regarded as a subset of the normed space dual $U^*$ of $U$ with its
weak* topology.   In this treatise generally, and in the present
chapter in particular, I allow myself to be slightly prejudiced against
normed-space methods;  you can find them in any book on functional
analysis, and I prefer here to develop techniques like those in part (b)
of the proof of 353M, which will be a useful preparation for such
theorems as 368E.
     
There is a very close analogy between 353M and the Stone representation
of Boolean algebras (311E, 311I-311K). %311I 311J 311K
Just as the proof of 311E
looked at the set of ring homomorphisms from $\frak A$ to the elementary
Boolean algebra $\Bbb Z_2$, so the proof of 353M looks at Riesz
homomorphisms from $U$ to the elementary $M$-space $\Bbb R$.   Later on,
the most important $M$-spaces, from the point of view of this treatise,
will be the $L^{\infty}$ spaces of \S363, explicitly defined in terms of
Stone representations (363A).
     
Of the two parts of 353O, it is (a) which is most important for the
purposes of this book.   The $f$-algebras we shall encounter in Chapter
36 can be seen to be commutative for different, and more elementary,
reasons.  The (separate) order-continuity of multiplication, however, is
not always immediately obvious.   Similarly, the uniferent Riesz
homomorphisms we shall encounter can generally be seen to be
multiplicative without relying on the arguments of 353Pd.
}%end of notes
     
     
\discrpage
     
