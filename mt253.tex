\frfilename{mt253.tex}
\versiondate{18.4.08}

\def\chaptername{Product measures}
\def\sectionname{Tensor products}

\newsection{253}

The theorems of the last section show that the integrable functions on a
product of two measure spaces can be effectively studied in terms of
integration on each factor space separately.   In this section I present
a very striking relationship between the $L^1$ space of a product
measure and the $L^1$ spaces of its factors, which actually determines
the product $L^1$ up to isomorphism as Banach lattice.   I start with a
brief note on bilinear operators (253A) and a description of the canonical
bilinear operator from $L^1(\mu)\times L^1(\nu)$ to $L^1(\mu\times\nu)$
(253B-253E).   The main theorem of the section is 253F, showing that
this canonical map is universal for continuous bilinear operators from
$L^1(\mu)\times L^1(\nu)$ to Banach spaces;  it also determines the
ordering of $L^1(\mu\times\nu)$ (253G).   I end with a description of a
fundamental type of conditional expectation operator (253H) and notes
on products of indefinite-integral measures (253I) and upper integrals
of special kinds of function (253J, 253K).

\leader{253A}{Bilinear operators}\cmmnt{ Before looking at any of the
measure theory in this section, I introduce a concept from the theory of
linear spaces.

\medskip

} {\bf (a)} Let $U$, $V$ and $W$ be linear spaces over
$\Bbb  R$\cmmnt{ (or, indeed, any field)}.   A map
$\phi:U\times V\to W$ is {\bf bilinear} if it is linear in each variable
separately, that is,

\Centerline{$\phi(u_1+u_2,v)=\phi(u_1,v)+\phi(u_2,v)$,}

\Centerline{$\phi(u,v_1+v_2)=\phi(u,v_1)+\phi(u,v_2)$,}

\Centerline{$\phi(\alpha u,v)=\alpha\phi(u,v)=\phi(u,\alpha v)$}

\noindent for all $u$, $u_1$, $u_2\in U$, $v$, $v_1$, $v_2\in V$ and
scalars $\alpha$.   Observe that such a $\phi$ gives rise to, and in
turn can be defined by, a linear operator
$T:U\to\eurm L(V;W)$\cmmnt{, writing
$\eurm L(V;W)$ for the space of linear operators from $V$ to $W$}, where

\Centerline{$(Tu)(v)=\phi(u,v)$}

\noindent for all $u\in U$, $v\in V$.  \cmmnt{Hence, or otherwise, we
can see, for instance, that} $\phi(0,v)=\phi(u,0)=0$ whenever $u\in U$ and
$v\in V$.

If $W'$ is another linear space\cmmnt{ over the same field,} and
$S:W\to W'$ is a linear operator, then $S\phi:U\times V\to W'$ is
bilinear.

\spheader 253Ab Now suppose that $U$, $V$ and $W$ are
normed spaces, and $\phi:U\times V\to W$ a bilinear operator.
\cmmnt{Then we
say that} $\phi$ is {\bf bounded} if
$\sup\{\|\phi(u,v)\|:\|u\|\le 1,\,\|v\|\le 1\}$ is finite, and in this
case we call this supremum the
norm $\|\phi\|$ of $\phi$.    \cmmnt{Note that}
$\|\phi(u,v)\|\le\|\phi\|\|u\|\|v\|$ for all $u\in U$, $v\in
V$\prooflet{ (because

\Centerline{$\|\phi(u,v)\|=\alpha\beta\|\phi(\alpha^{-1}u,\beta^{-1}v)\|
\le\alpha\beta\|\phi\|$}

\noindent whenever $\alpha>\|u\|$, $\beta>\|v\|$)}.

If $W'$ is another normed space and $S:W\to W'$ is a bounded linear
operator, then $S\phi:U\times V\to W'$ is a bounded bilinear operator,
and $\|S\phi\|\le\|S\|\|\phi\|$.

\leader{253B}{Definition}\cmmnt{ The most important bilinear operators of
this section are based on the following idea.}   Let $f$ and $g$ be
real-valued functions.   I will write $f\otimes g$ for the function
$(x,y)\mapsto f(x)g(y):\dom f\times\dom g\to\Bbb R$.

\leader{253C}{Proposition}
(a) Let $X$ and $Y$ be sets, and $\Sigma$, $\Tau\,\,\sigma$-algebras of
subsets of $X$, $Y$ respectively.   If $f$ is a $\Sigma$-measurable
real-valued function defined on a subset of $X$, and $g$ is a
$\Tau$-measurable real-valued function defined on a subset of $Y$, then
$f\otimes g$\cmmnt{, as defined in 253B,} is
$\Sigma\tensorhat\Tau$-measurable.

(b) Let $(X,\Sigma,\mu)$ and $(Y,\Tau,\nu)$ be measure spaces, and
$\lambda$ the c.l.d.\ product measure on $X\times Y$.  If
$f\in\eusm L^0(\mu)$ and $g\in\eusm L^0(\nu)$, then
$f\otimes g\in\eusm L^0(\lambda)$.

\cmmnt{\medskip

\noindent{\bf Remark} Recall from 241A that $\eusm L^0(\mu)$ is the
space of $\mu$-virtually measurable real-valued functions defined
on $\mu$-conegligible subsets of $X$.}

\proof{{\bf (a)} The point is that $f\otimes\chi Y$ is
$\Sigma\tensorhat\Tau$-measurable, because for any $\alpha\in\Bbb R$
there is an $E\in\Sigma$ such that

\Centerline{$\{x:f(x)\ge\alpha\}=E\cap\dom f$,}

\noindent so that

\Centerline{$\{(x,y):(f\otimes\chi Y)(x,y)\ge\alpha\}
=(E\cap\dom f)\times Y=(E\times Y)\cap\dom(f\otimes\chi Y)$,}

\noindent and of course $E\times Y\in\Sigma\tensorhat\Tau$.   Similarly,
$\chi X\otimes g$ is $\Sigma\tensorhat\Tau$-measurable and
$f\otimes g=(f\otimes\chi Y)\times(\chi X\otimes g)$ is
$\Sigma\tensorhat\Tau$-measurable.

\medskip

{\bf (b)} Let $E\in\Sigma$, $F\in\Tau$ be conegligible subsets of $X$,
$Y$ respectively such that $E\subseteq\dom f$, $F\subseteq\dom g$,
$f\restr E$ is $\Sigma$-measurable and $g\restr F$ is $\Tau$-measurable.
Write $\Lambda$ for the domain of $\lambda$.   Then
$\Sigma\tensorhat\Tau\subseteq\Lambda$ (251Ia).   Also $E\times F$ is
$\lambda$-conegligible, because

$$\eqalign{\lambda((X\times Y)\setminus(E\times F))
&\le\lambda((X\setminus E)\times Y)+\lambda(X\times(Y\setminus F))\cr
&=\mu(X\setminus E)\cdot\nu Y+\mu X\cdot\nu(Y\setminus F)
=0\cr}$$

\noindent (also from 251Ia).   So $\dom(f\otimes g)\supseteq E\times F$
is conegligible.   Also, by (a),
$(f\otimes g)\restr(E\times F)
=(f\restr E)\otimes(g\restr F)$ is $\Sigma\tensorhat\Tau$-measurable,
therefore $\Lambda$-measurable, and $f\otimes g$ is virtually
measurable.   Thus $f\otimes g\in\eusm L^0(\lambda)$, as claimed.
}%end of proof of 253C

\leader{253D}{}\cmmnt{ Now we can apply the ideas of 253B-253C to
integrable functions.

\medskip

\noindent}{\bf Proposition} Let $(X,\Sigma,\mu)$ and
$(Y,\Tau,\nu)$ be measure spaces, and write $\lambda$ for the c.l.d.\
product measure on $X\times Y$.   If $f\in\eusm L^1(\mu)$ and
$g\in\eusm L^1(\nu)$, then $f\otimes g\in\eusm L^1(\lambda)$ and
$\int f\otimes g\,d\lambda=\int f\,d\mu\int g\,d\nu$.

\cmmnt{\medskip

\noindent{\bf Remark} I follow \S242 in writing $\eusm L^1(\mu)$ for
the space of $\mu$-integrable real-valued functions.
}

\proof{{\bf (a)} Consider first the
case $f=\chi E$, $g=\chi F$ where $E\in\Sigma$, $F\in\Tau$ have finite
measure;  then $f\otimes g=\chi(E\times F)$ is $\lambda$-integrable with
integral

\Centerline{$\lambda(E\times F)=\mu E\cdot\nu F
=\int f\,d\mu\cdot\int g\,d\nu$,}

\noindent by 251Ia.

\medskip

{\bf (b)}  It follows at once that $f\otimes g$ is $\lambda$-simple,
with $\int f\otimes g\,d\lambda=\int f\,d\mu\int g\,d\nu$, whenever $f$
is a $\mu$-simple function and $g$ is a $\nu$-simple function.

\medskip

{\bf (c)} If $f$ and $g$ are non-negative integrable functions, there
are non-decreasing sequences $\sequencen{f_n}$, $\sequencen{g_n}$ of
non-negative simple
functions converging almost everywhere to $f$, $g$ respectively;   now
note that if $E\subseteq X$, $F\subseteq Y$ are conegligible, $E\times
F$ is conegligible in $X\times Y$, as remarked in the proof of 253C, so
the non-decreasing sequence $\sequencen{f_n\times g_n}$ of
$\lambda$-simple functions converges almost everywhere to $f\otimes g$,
and

\Centerline{$\int f\otimes g\,d\lambda
=\lim_{n\to\infty}\int f_n\otimes g_nd\lambda
=\lim_{n\to\infty}\int f_nd\mu\int g_nd\nu
=\int f\,d\mu\int g\,d\nu$}

\noindent by B.Levi's theorem.

\medskip

{\bf (d)} Finally, for general $f$ and $g$, we can express them as
the differences $f^+-f^-$, $g^+-g^-$ of non-negative integrable
functions, and see that

\Centerline{$\int f\otimes g\,d\lambda
=\int f^+\otimes g^+-f^+\otimes g^--f^-\otimes g^+
+f^-\otimes g^-d\lambda
=\int f\,d\mu\int g\,d\nu.$}
}%end of proof of 253D

\leader{253E}{The canonical map $L^1\times L^1\to L^1$}\cmmnt{ I
continue the argument from 253D.   Because $E\times F$ is conegligible
in $X\times Y$ whenever $E$ and $F$ are
conegligible subsets of $X$ and $Y$,
$f_1\otimes g_1=f\otimes g\,\,\lambda$-a.e. whenever $f=f_1\,\,\mu$-a.e.
and $g=g_1\,\,\nu$-a.e.}   We may\cmmnt{ therefore} define
$u\otimes v\in L^1(\lambda)$, for $u\in L^1(\mu)$ and $v\in L^1(\nu)$,
by saying that
$u\otimes v=(f\otimes g)^{\ssbullet}$ whenever $u=f^{\ssbullet}$ and
$v=g^{\ssbullet}$.   \dvro{The}{

Now if $f$, $f_1$, $f_2\in\eusm L^1(\mu)$, $g$, $g_1$,
$g_2\in\eusm L^1(\nu)$ and $a\in\Bbb R$,

\Centerline{$(f_1+f_2)\otimes g=(f_1\otimes g)+(f_2\otimes g)$,}

\Centerline{$f\otimes(g_1+g_2)=(f\otimes g_1)+(f\otimes g_2)$,}

\Centerline{$(af)\otimes g=a(f\otimes g)=f\otimes(ag)$.}

\noindent It follows at once that the} map $(u,v)\mapsto u\otimes v$ is
bilinear.

\cmmnt{Moreover, if $f\in\eusm L^1(\mu)$ and $g\in\eusm L^1(\nu)$,
$|f|\otimes|g|=|f\otimes g|$, so
$\int|f\otimes g|d\lambda=\int|f|d\mu\int|g|d\nu$.   Accordingly}

\Centerline{$\|u\otimes v\|_1=\|u\|_1\|v\|_1$}

\noindent for all $u\in L^1(\mu)$, $v\in L^1(\nu)$.   In particular, the
bilinear operator $\otimes$ is bounded, with norm $1$ (except in the trivial
case in which one of $L^1(\mu)$, $L^1(\nu)$ is $0$-dimensional).

\leader{253F}{}\cmmnt{ We are now ready for the main theorem of this
section.

\medskip

\noindent}{\bf Theorem}
Let $(X,\Sigma,\mu)$ and $(Y,\Tau,\nu)$ be measure spaces, and let
$\lambda$ be the c.l.d.\ product measure on $X\times Y$.   Let $W$ be
any Banach space and $\phi:L^1(\mu)\times L^1(\nu)\to W$ a bounded bilinear
operator.   Then there is a unique bounded linear operator
$T:L^1(\lambda)\to W$ such that $T(u\otimes v)=\phi(u,v)$ for all
$u\in L^1(\mu)$ and $v\in L^1(\nu)$, and $\|T\|=\|\phi\|$.

\proof{{\bf (a)} The centre of the argument is the following
fact:   if $E_0,\ldots,E_n$ are measurable sets of finite measure in
$X$, $F_0,\ldots,F_n$ are measurable sets of finite measure in $Y$,
$a_0,\ldots,a_n\in\Bbb R$ and
$\sum_{i=0}^na_i\chi(E_i\times F_i)=0\,\,\lambda$-a.e., then
$\sum_{i=0}^na_i\phi(\chi E_i^{\ssbullet},\chi F_i^{\ssbullet})=0$ in $W$.
\Prf\ We
can find a disjoint family $\langle G_j\rangle_{j\le m}$ of measurable
sets of finite measure in $X$ such that each $E_i$ is expressible as a
union of some subfamily of the $G_j$;  so that $\chi E_i$ is
expressible in the form $\sum_{j=0}^mb_{ij}\chi G_j$ (see 122Ca).
Similarly, we
can find a disjoint family $\langle H_k\rangle_{k\le l}$ of measurable
sets of finite measure in $Y$ such that each $\chi F_i$ is expressible
as $\sum_{k=0}^lc_{ik}\chi H_k$.   Now

\Centerline{$\sum_{j=0}^m\sum_{k=0}^l
\bigl(\sum_{i=0}^na_ib_{ij}c_{ik}\bigr)
\chi(G_j\times H_k)
=\sum_{i=0}^na_i\chi(E_i\times F_i)=0\,\,\lambda\text{-a.e.}$}

\noindent Because the $G_j\times H_k$ are disjoint, and
$\lambda(G_j\times H_k)=\mu G_j\cdot\nu H_k$ for all $j$, $k$, it
follows that for
every $j\le m$, $k\le l$ we have either $\mu G_j=0$ or $\nu H_k=0$ or
$\sum_{i=0}^na_ib_{ij}c_{ik}=0$.   In any of these three cases,
$\sum_{i=0}^na_ib_{ij}c_{ik}\phi(\chi G_j^{\ssbullet},\chi
H_k^{\ssbullet})=0$ in $W$.   But this means that

\Centerline{$0=\sum_{j=0}^m\sum_{k=0}^l
\bigl(\sum_{i=0}^na_ib_{ij}c_{ik}\bigr)
\phi(\chi G_j^{\ssbullet},\chi H_k^{\ssbullet})
=\sum_{i=0}^na_i\phi(\chi E_i^{\ssbullet},\chi F_i^{\ssbullet}),$}

\noindent as claimed.\ \Qed

\medskip

{\bf (b)} It follows that if  $E_0,\ldots,E_n$, $E'_0,\ldots,E'_m$  are
measurable sets of finite measure in $X$, $F_0,\ldots,F_n$,
$F'_0,\ldots,F'_m$ are measurable sets of finite measure in $Y$,
$a_0,\ldots,a_n,a'_0,\ldots,a'_m\in\Bbb R$ and
$\sum_{i=0}^na_i\chi(E_i\times F_i)=\sum_{i=0}^ma'_i\chi(E'_i\times
F'_i)\,\,\lambda$-a.e., then

\Centerline{$\sum_{i=0}^na_i\phi(\chi
E_i^{\ssbullet},\chi F_i^{\ssbullet})=\sum_{i=0}^ma'_i\phi(\chi
{E'_i}^{\ssbullet},\chi {F'_i}^{\ssbullet})$}

\noindent in $W$.   Let $M$ be the linear
subspace of $L^1(\lambda)$ generated by

\Centerline{$\{\chi(E\times F)^{\ssbullet}:E\in\Sigma,\,\mu E<\infty,\,
  F\in\Tau,\,\nu F<\infty\};$}

\noindent then we have a unique map $T_0:M\to W$ such that

\Centerline{$T_0(\sum_{i=0}^na_i\chi(E_i\times F_i)^{\ssbullet})
=\sum_{i=0}^na_i\phi(\chi E_i^{\ssbullet},\chi F_i^{\ssbullet})$}

\noindent whenever $E_0,\ldots,E_n$  are measurable sets of finite
measure in $X$, $F_0,\ldots,F_n$ are measurable sets of finite measure
in $Y$ and $a_0,\ldots,a_n\in\Bbb R$.   Of course $T_0$ is linear.

\medskip

{\bf (c)} Some of the same calculations show that
$\|T_0u\|\le\|\phi\|\|u\|_1$ for every $u\in M$.   \Prf\ If $u\in M$,
then, by the arguments of (a), we can express $u$ as
$\sum_{j=0}^m\sum_{k=0}^la_{jk}\chi(G_j\times
H_k)^{\ssbullet}$, where $\langle G_j\rangle_{j\le m}$ and $\langle
H_k\rangle_{k\le l}$ are disjoint families of sets of finite measure.
Now

$$\eqalign{\|T_0u\|
&=\|\sum_{j=0}^m\sum_{k=0}^la_{jk}\phi(\chi G_j^{\ssbullet},\chi
  H_k^{\ssbullet})\|
\le\sum_{j=0}^m\sum_{k=0}^l|a_{jk}|\|\phi(\chi G_j^{\ssbullet},\chi
  H_k^{\ssbullet})\|\cr
&\le\sum_{j=0}^m\sum_{k=0}^l|a_{jk}|\|\phi\|\|\chi
  G_j^{\ssbullet}\|_1\|\chi H_k^{\ssbullet}\|_1
=\|\phi\|\sum_{j=0}^m\sum_{k=0}^l|a_{jk}|\mu G_j\cdot\nu H_k\cr
&=\|\phi\|\sum_{j=0}^m\sum_{k=0}^l|a_{jk}|\lambda(G_j\times H_k)
=\|\phi\|\|u\|_1,\cr}$$

\noindent as claimed.\ \Qed

\medskip

{\bf (d)} The next point is to observe that $M$ is dense in
$L^1(\lambda)$ for $\|\,\|_1$.   \Prf\ Repeating the ideas above once
again, we observe that if $E_0,\ldots,E_n$ are sets of finite measure in
$X$ and $F_0,\ldots,F_n$ are sets of finite measure in $Y$, then
$\chi(\bigcup_{i\le n}E_i\times F_i)^{\ssbullet}\in M$;  this is
because,
expressing each $E_i$ as a union of $G_j$, where the $G_j$ are disjoint,
we have

\Centerline{$\bigcup_{i\le n}E_i\times F_i=\bigcup_{j\le m}G_j\times
F'_j,$}

\noindent where $F'_j=\bigcup\{F_i:G_j\subseteq E_i\}$ for each $j$;
now $\langle G_j\times F'_j\rangle_{j\le m}$ is disjoint, so

\Centerline{$\chi(\bigcup_{j\le m}G_j\times F_j)^{\ssbullet}
=\sum_{j=0}^m\chi(G_j\times F'_j)^{\ssbullet}\in M.$}

\noindent So 251Ie tells us that whenever $\lambda H<\infty$ and
$\epsilon>0$ there is a $G$ such that $\lambda(H\symmdiff G)\le\epsilon$
and $\chi G^{\ssbullet}\in M$;  now

\Centerline{$\|\chi H^{\ssbullet}-\chi
G^{\ssbullet}\|_1=\lambda(G\symmdiff H)\le\epsilon$,}

\noindent so $\chi H^{\ssbullet}$
is approximated arbitrarily closely by members of $M$, and belongs to
the closure $\overline{M}$ of $M$ in $L^1(\lambda)$.   Because $M$ is a
linear subspace of $L^1(\lambda)$, so is $\overline{M}$ (2A4Cb);
accordingly $\overline{M}$ contains the equivalence classes of all
$\lambda$-simple functions;  but these are dense in $L^1(\lambda)$
(242Mb), so $\overline{M}=L^1(\lambda)$, as claimed.\ \Qed

\medskip

{\bf (e)} Because $W$ is a Banach space, it follows that there is a
bounded linear operator $T:L^1(\lambda)\to W$ extending $T_0$, with
$\|T\|=\|T_0\|\le\|\phi\|$ (2A4I).    Now $T(u\otimes v)=\phi(u,v)$ for
all $u\in L^1(\mu)$, $v\in L^1(\nu)$.   \Prf\ If $u=\chi E^{\ssbullet}$ and
$v=\chi F^{\ssbullet}$, where $E$, $F$ are measurable sets of finite
measure, then

\Centerline{$T(u\otimes v)=T(\chi(E\times F)^{\ssbullet})
=T_0(\chi(E\times F)^{\ssbullet})
=\phi(\chi E^{\ssbullet},\chi F^{\ssbullet})=\phi(u,v).$}

\noindent Because $\phi$ and $\otimes$ are bilinear and $T$ is linear,

\Centerline{$T(f^{\ssbullet}\otimes g^{\ssbullet})
=\phi(f^{\ssbullet},g^{\ssbullet})$}

\noindent whenever $f$ and $g$ are simple functions.   Now whenever
$u\in L^1(\mu)$, $v\in L^1(\nu)$ and $\epsilon>0$, there are simple
functions $f$, $g$ such that $\|u-f^{\ssbullet}\|_1\le\epsilon$,
$\|v-g^{\ssbullet}\|_1\le\epsilon$ (242Mb again);  so that

$$\eqalign{\|\phi(u,v)-\phi(f^{\ssbullet},g^{\ssbullet})\|
&\le\|\phi(u-f^{\ssbullet},v-g^{\ssbullet})\|
+\|\phi(u,g^{\ssbullet}-v)\|
+\|\phi(f^{\ssbullet}-u,v)\|\cr
&\le\|\phi\|(\epsilon^2+\epsilon\|u\|_1+\epsilon\|v\|_1).}$$

\noindent Similarly

\Centerline{$\|u\otimes v-f^{\ssbullet}\otimes g^{\ssbullet}\|_1
\le\epsilon(\epsilon+\|u\|_1+\|v\|_1),$}

\noindent so

\Centerline{$\|T(u\otimes v)-T(f^{\ssbullet}\otimes g^{\ssbullet})\|
\le\epsilon\|T\|(\epsilon+\|u\|_1+\|v\|_1);$}

\noindent because $T(f^{\ssbullet}\otimes g^{\ssbullet})
=\phi(f^{\ssbullet},g^{\ssbullet})$,

\Centerline{$\|T(u\otimes v)-\phi(u,v)\|
\le\epsilon(\|T\|+\|\phi\|)(\epsilon+\|u\|_1
+\|v\|_1).$}

\noindent As $\epsilon$ is arbitrary, $T(u\otimes v)=\phi(u,v)$, as
required.\ \Qed

\medskip

{\bf (f)} The argument of (e) ensured that $\|T\|\le\|\phi\|$.
Because $\|u\otimes v\|_1\le\|u\|_1\|v\|_1$ for all $u\in L^1(\mu)$ and
$v\in L^1(\nu)$, $\|\phi(u,v)\|\le\|T\|\|u\|_1\|v\|_1$ for all $u$, $v$,
and $\|\phi\|\le\|T\|$;  so $\|T\|=\|\phi\|$.

\medskip

{\bf (g)} Thus $T$ has the required properties.   To see that it is
unique, we have only to observe that any bounded linear operator
$S:L^1(\lambda)\to W$ such that $S(u\otimes v)=\phi(u,v)$ for all $u\in
L^1(\mu)$, $v\in L^1(\nu)$ must agree with $T$ on objects of the form
$\chi(E\times F)^{\ssbullet}$ where $E$ and $F$ are of finite measure,
and
therefore on every member of $M$;  because $M$ is dense and both $S$ and
$T$ are continuous, they agree everywhere in $L^1(\lambda)$.
}

\leader{253G}{The order structure of \dvrocolon{$L^1$}}\cmmnt{ In
253F I have treated the $L^1$ spaces exclusively as
normed linear spaces.   In general, however, the order structure of an
$L^1$ space  (see 242C) is as important as its norm.   The map
$\otimes:L^1(\mu)\times L^1(\nu)\to L^1(\lambda)$ respects the order
structures of the three spaces in the following strong sense.

\medskip

\noindent}{\bf Proposition} Let $(X,\Sigma,\mu)$ and $(Y,\Tau,\nu)$ be
measure spaces, and $\lambda$ the c.l.d.\ product measure on $X\times
Y$.   Then

(a)  $u\otimes v\ge 0$ in
$L^1(\lambda)$ whenever $u\ge 0$ in $L^1(\mu)$ and $v\ge 0$ in
$L^1(\nu)$.

(b) The positive cone $\{w:w\ge 0\}$ of
$L^1(\lambda)$ is precisely the closed
convex hull $C$ of $\{u\otimes v:u\ge 0,\,v\ge 0\}$ in $L^1(\lambda)$.

*(c) Let $W$ be any Banach lattice, and $T:L^1(\lambda)\to W$ a
bounded linear operator.   Then
the following are equiveridical:

\quad(i) $Tw\ge 0$ in $W$ whenever $w\ge 0$ in
$L^1(\lambda)$;

\quad(ii) $T(u\otimes v)\ge 0$ in $W$ whenever $u\ge 0$ in
$L^1(\mu)$ and $v\ge 0$ in $L^1(\nu)$.

\proof{{\bf (a)} If $u$, $v\ge 0$ then they are expressible as
$f^{\ssbullet}$, $g^{\ssbullet}$ where $f\in\eusm L^1(\mu)$,
$g\in\eusm L^1(\nu)$, $f\ge 0$ and $g\ge 0$.
Now $f\otimes g\ge 0$ so $u\otimes v=(f\otimes g)^{\ssbullet}\ge 0$.

\medskip

{\bf (b)(i)} Write $L^1(\lambda)^+$ for $\{w:w\in L^1(\lambda),\,w\ge
0\}$.   Then $L^1(\lambda)^+$ is a closed convex set in $L^1(\lambda)$
(242De);  by (a), it contains $u\otimes v$ whenever $u\in L^1(\mu)^+$ and
$v\in L^1(\nu)^+$, so it must include $C$.

\medskip

\quad{\bf (ii)}($\alpha$)
Of course $0=0\otimes 0\in C$.  ($\beta$) If $u\in M$, as defined in
the proof of 253F, and $u>0$, then $u$ is expressible as $\sum_{j\le
m,k\le l}a_{jk}\chi(G_j\times H_k)^{\ssbullet}$, where $G_0,\ldots,G_m$
and $H_0,\ldots,H_l$ are disjoint sequences of sets of finite measure,
as in (a) of the proof of 253F.   Now $a_{jk}$ can be negative only if
$\chi(G_j\times H_k)^{\ssbullet}=0$, so replacing every $a_{jk}$ by
$\max(0,a_{jk})$  if necessary, we can suppose that $a_{jk}\ge 0$ for
all $j$, $k$.     Not all the $a_{jk}$ can be zero, so $a=\sum_{j\le
m,k\le l}a_{jk}>0$, and

\Centerline{$u
=\sum_{j\le m,k\le l}
  \Bover{a_{jk}}a\cdot a\chi(G_j\times H_k)^{\ssbullet}
=\sum_{j\le m,k\le l}\Bover{a_{jk}}a
  \cdot(a\chi G_j^{\ssbullet})\otimes\chi H_k^{\ssbullet}
\in C$.}

\noindent ($\gamma$) If $w\in L^1(\lambda)^+$ and $\epsilon>0$, express
$w$ as $h^{\ssbullet}$ where $h\ge 0$ in $\eusm L^1(\lambda)$.   There
is a simple function $h_1\ge 0$ such that $h_1\leae h$ and
$\int h\le\int h_1+\epsilon$.   Express $h_1$ as
$\sum_{i=0}^na_i\chi H_i$
where $\lambda H_i<\infty$ and $a_i\ge 0$ for each $i$, and for each
$i\le n$ choose sets $G_{i0},\ldots,G_{im_i}\in\Sigma$,
$F_{i0},\ldots,F_{im_i}\in\Tau$, all of finite measure, such that
$G_{i0},\ldots,G_{im_i}$ are disjoint and
$\lambda(H_i\symmdiff\bigcup_{j\le m_i}G_{ij}\times
F_{ij})\le\epsilon/(n+1)(a_i+1)$, as in (d) of the proof of 253F.   Set

\Centerline{$w_0
=\sum_{i=0}^na_i\sum_{j=0}^{m_i}\chi(G_{ij}\times F_{ij})^{\ssbullet}$.}

\noindent Then $w_0\in C$ because $w_0\in M$ and $w_0\ge 0$.   Also

$$\eqalign{\|w-w_0\|_1
&\le\|w-h_1^{\ssbullet}\|_1+\|h_1^{\ssbullet}-w_0\|_1\cr
&\le\int(h-h_1)d\lambda
  +\sum_{i=0}^na_i\int|\chi H_i-\sum_{j=0}^{m_i}\chi(G_{ij}\times
  F_{ij})|d\lambda\cr
&\le\epsilon+\sum_{i=0}^na_i\lambda(H\symmdiff\bigcup_{j\le
m_i}G_{ij}\times F_{ij})
\le 2\epsilon.\cr}$$

\noindent As $\epsilon$ is arbitrary and $C$ is closed, $w\in C$.   As
$w$ is arbitrary, $L^1(\lambda)^+\subseteq C$ and $C=L^1(\lambda)^+$.

\medskip

{\bf (c)} Part (a) tells us that (i)$\Rightarrow$(ii).   For the reverse
implication, we need a fragment from the theory of Banach lattices:
$W^+=\{w:w\in W,\,w\ge 0\}$ is a closed set in $W$.
\Prf\ If $w$, $w'\in W$, then

\Centerline{$w=(w-w')+w'\le|w-w'|+w'\le|w-w'|+|w'|$,}

\Centerline{$-w=(w'-w)-w'\le|w-w'|-w'\le|w-w'|+|w'|$,}

\Centerline{$|w|\le|w-w'|+|w'|$,
\quad$|w|-|w'|\le|w-w'|$,}

\noindent because $|w|=w\vee(-w)$ and the order of $W$ is
translation-invariant (241Ec).   Similarly, $|w'|-|w|\le|w-w'|$ and
$||w|-|w'||\le|w-w'|$, so $\||w|-|w'|\|\le\|w-w'\|$, by the definition
of Banach lattice (242G).   Setting $\phi(w)=|w|-w$, we see that
$\|\phi(w)-\phi(w')\|\le 2\|w-w'\|$ for all $w$, $w'\in W$, so that
$\phi$ is continuous.

Now, because the order is invariant under multiplication by positive
scalars,

\Centerline{$w\ge 0\iff 2w\ge 0\iff w\ge-w\iff w=|w|\iff\phi(w)=0$,}

\noindent so $W^+=\{w:\phi(w)=0\}$ is closed.\ \Qed

Now suppose that (ii) is true, and set
$C_1=\{w:w\in L^1(\lambda),\,Tw\ge 0\}$.   Then $C_1$ contains
$u\otimes v$ whenever $u$, $v\ge 0$;  but also
it is convex, because $T$ is linear, and closed, because $T$ is
continuous and $C_1=T^{-1}[W^+]$.   By (b), $C_1$ includes
$\{w:w\in L^1(\lambda),\,w\ge 0\}$, as required by (i).
}%end of proof of 253G

\leader{253H}{Conditional \dvrocolon{expectations}}\cmmnt{ The ideas
of this section and the preceding one provide us with some of the most
important examples of conditional expectations.

\medskip

\noindent}{\bf Theorem} Let $(X,\Sigma,\mu)$ and $(Y,\Tau,\nu)$ be
complete probability spaces, with c.l.d.\ product
$(X\times Y,\Lambda,\lambda)$.   Set $\Lambda_1=\{E\times Y:E\in\Sigma\}$.
Then $\Lambda_1$ is a $\sigma$-subalgebra of $\Lambda$.   Given a
$\lambda$-integrable real-valued function $f$, set

\Centerline{$g(x,y)=\int f(x,z)\nu(dz)$}

\noindent whenever $x\in X$, $y\in Y$ and the integral is defined in
$\Bbb R$.   Then $g$ is a conditional expectation of $f$ on $\Lambda_1$.

\proof{ We know that $\Lambda_1\subseteq\Lambda$, by 251Ia, and
$\Lambda_1$ is a $\sigma$-algebra of sets because $\Sigma$ is.
Fubini's theorem (252B, 252C)
tells us that $f_1(x)=\int f(x,z)\nu(dz)$ is defined for almost every
$x$, and therefore that $g=f_1\otimes\chi Y$ is defined almost everywhere
in $X\times Y$.   $f_1$ is $\mu$-virtually measurable;  because $\mu$ is
complete, $f_1$ is $\Sigma$-measurable, so $g$ is $\Lambda_1$-measurable
(since $\{(x,y):g(x,y)\le\alpha\}=\{x:f_1(x)\le\alpha\}\times Y$ for
every $\alpha\in\Bbb R$).   Finally, if $W\in\Lambda_1$, then
$W=E\times Y$ for some $E\in\Sigma$, so

$$\eqalignno{\int_Wg\,d\lambda
&=\int (f_1\otimes\chi Y)\times(\chi E\otimes\chi Y)d\lambda
=\int f_1\times\chi E\,d\mu\int\chi Y\,d\nu\cr
\noalign{\noindent (by 253D)}
&=\iint\chi E(x)f(x,y)\nu(dy)\mu(dx)
=\int f\times\chi(E\times Y)d\lambda\cr
\noalign{\noindent (by Fubini's theorem)}
&=\int_Wf\,d\lambda.\cr}$$

\noindent So $g$ is a conditional expectation of $f$.
}%end of proof of 253H

\leader{253I}{}\cmmnt{ This is a convenient moment to set out a
useful result on products of indefinite-integral measures.

\medskip

\noindent}{\bf Proposition} Let $(X,\Sigma,\mu)$ and $(Y,\Tau,\nu)$ be
measure spaces, and $f\in\eusm L^0(\mu)$, $g\in\eusm L^0(\nu)$
non-negative functions.   Let $\mu'$, $\nuprime$ be the corresponding
indefinite-integral measures\cmmnt{ (see \S234)}.   Let $\lambda$ be
the c.l.d.\ product of $\mu$ and $\nu$, and $\lambda'$ the
indefinite-integral measure defined from $\lambda$ and
$f\otimes g\in\eusm L^0(\lambda)$\cmmnt{ (253Cb)}.   Then $\lambda'$ is
the c.l.d.\ product of $\mu'$ and $\nuprime$.

\proof{  Write $\theta$ for the c.l.d.\ product of $\mu'$ and $\nuprime$.

\medskip

{\bf (a)} If we replace $\mu$ by its completion, we do not change $\mu'$
(234Ke);  at the same time, we do not change $\lambda$, by 251T.
The same applies to $\nu$.   So it will be enough to prove the result on
the assumption that $\mu$ and $\nu$ are complete;  in which case $f$ and
$g$ are measurable and have measurable domains.

Set $F=\{x:x\in\dom f,\,f(x)>0\}$ and $G=\{y:y\in\dom g,\,g(y)>0\}$, so
that $F\times G=\{w:w\in\dom(f\otimes g),\,(f\otimes g)(w)>0\}$.   Then
$F$ is $\mu'$-conegligible and $G$ is $\nuprime$-conegligible, so
$F\times G$ is $\theta$-conegligible as well as
$\lambda'$-conegligible.   Because both $\theta$ and $\lambda'$ are
complete (251Ic, 234I), it will be enough to show that the subspace
measures $\theta_{F\times G}$, $\lambda'_{F\times G}$ on $F\times G$ are
equal.   But note that $\theta_{F\times G}$ can be identified with the
product of $\mu_F'$ and $\nu_G'$, where $\mu_F'$ and $\nu_G'$ are the
subspace measures on $F$, $G$ respectively (251Q(ii-$\alpha$)).   At the
same time, $\mu_F'$ is the indefinite-integral measure defined from the
subspace measure $\mu_F$ on $F$ and the function $f\restr F$,
$\nu_G'$ is the indefinite-integral measure defined from the subspace
measure $\nu_G$ on $G$ and $g\restr G$, and $\lambda'_{F\times G}$ is
defined from the subspace measure $\lambda_{F\times G}$ and
$(f\restr F)\otimes(g\restr G)$.   Finally, by 251Q again,
$\lambda_{F\times G}$ is the product of $\mu_F$ and $\nu_G$.

What all this means is that it will be enough to deal with the case in
which $F=X$ and $G=Y$, that is, $f$ and $g$ are everywhere defined and
strictly positive;  which is what I will suppose from now on.

\medskip

{\bf (b)} In this case $\dom\mu'=\Sigma$ and $\dom\nuprime=\Tau$ (234La).
Similarly, $\dom\lambda'=\Lambda$ is just the domain of $\lambda$.   Set

\Centerline{$F_n=\{x:x\in X,\,2^{-n}\le f(x)\le 2^n\}$,
\quad$G_n=\{y:y\in Y,\,2^{-n}\le g(y)\le 2^n\}$}

\noindent for $n\in\Bbb N$.

\medskip

{\bf (c)} Set

\Centerline{$\Cal A=\{W:W\in\dom\theta\cap\dom\lambda'$,
$\theta(W)=\lambda'(W)\}$.}

\noindent If $\mu'E$ and $\nuprime H$ are defined and finite, then
$f\times\chi E$ and $g\times\chi H$ are integrable, so

$$\eqalign{\lambda'(E\times H)
&=\int(f\otimes g)\times\chi(E\times H)d\lambda
=\int(f\times\chi E)\otimes(g\times\chi H)d\lambda\cr
&=\int f\times\chi E\,d\mu\cdot\int g\times\chi H\,d\nu
=\theta(E\times H)\cr}$$

\noindent by 253D and 251Ia, that is, $E\times H\in\Cal A$.   If we now
look at
$\Cal A_{EH}=\{W:W\subseteq X\times Y$, $W\cap(E\times H)\in\Cal A\}$,
then we see that

\inset{$\Cal A_{EH}$ contains $E'\times H'$ for every $E'\in\Sigma$,
$H'\in\Tau$,}

\inset{if $\sequencen{W_n}$ is a non-decreasing sequence in
$\Cal A_{EH}$ then $\bigcup_{n\in\Bbb N}W_n\in\Cal A_{EH}$,}

\inset{if $W$, $W'\in\Cal A_{EH}$ and $W\subseteq W'$ then
$W'\setminus W\in\Cal A_{EH}$.}

\noindent Thus $\Cal A_{EH}$ is a Dynkin class of subsets of
$X\times Y$, and by the Monotone Class Theorem (136B) includes
the $\sigma$-algebra generated by
$\{E'\times H':E'\in\Sigma,\,H'\in\Tau\}$, which is
$\Sigma\tensorhat\Tau$.

\medskip

{\bf (d)} Now suppose that $W\in\Lambda$.   In this case
$W\in\dom\theta$ and $\theta W\le\lambda'W$.   \Prf\ Take $n\in\Bbb N$,
and $E\in\Sigma$, $H\in\Tau$ such that $\mu'E$ and $\nuprime H$ are
both finite.   Set $E'=E\cap F_n$, $H'=H\cap G_n$ and $W'=W\cap(E'\times
H')$.   Then $W'\in\Lambda$, while $\mu E'\le 2^n\mu'E$ and $\nu H'\le
2^n\nuprime H$ are finite.   By 251Ib there is a $V\in\Sigma\tensorhat\Tau$
such that $V\subseteq W'$ and $\lambda V=\lambda W'$.  Similarly, there
is a $V'\in\Sigma\tensorhat\Tau$ such that $V'\subseteq(E'\times
H')\setminus W'$ and $\lambda V'=\lambda((E'\times H')\setminus W')$.
This means that $\lambda((E'\times H')\setminus(V\cup V'))=0$, so
$\lambda'((E'\times H')\setminus(V\cup V'))=0$.   But
$(E'\times H')\setminus(V\cup V')\in\Cal A$, by (c), so $\theta((E'\times H')\setminus(V\cup V'))=0$ and $W'\in\dom\theta$, while

\Centerline{$\theta W'=\theta V=\lambda'V
\le\lambda'W$.}

Since $E$ and $H$ are arbitrary, $W\cap(F_n\times G_n)\in\dom\theta$
(251H) and $\theta(W\cap(F_n\times G_n))\le\lambda'W$.   Since
$\sequencen{F_n}$, $\sequencen{G_n}$ are non-decreasing sequences with
unions $X$, $Y$ respectively,

\Centerline{$\theta W=\sup_{n\in\Bbb N}\theta(W\cap(F_n\times G_n))
\le\lambda'W$.  \Qed}

\medskip

{\bf (e)} In the same way, $\lambda'W$ is defined and less than or equal
to $\theta W$ for every $W\in\dom\theta$.   \Prf\ The arguments are very
similar, but a refinement seems to be necessary at the last stage.
Take $n\in\Bbb N$, and $E\in\Sigma$, $H\in\Tau$ such that $\mu E$
and $\nu H$ are both finite.   Set $E'=E\cap F_n$, $H'=H\cap G_n$ and
$W'=W\cap(E'\times H')$.   Then $W'\in\dom\theta$, while $\mu'E'\le
2^n\mu E$ and $\nuprime H'\le 2^n\nu H$ are finite.   This time, there are
$V$, $V'\in\Sigma\tensorhat\Tau$ such that $V\subseteq W'$,
$V'\subseteq(E'\times H')\setminus W'$, $\theta V=\theta W'$ and
$\theta V'=\theta((E'\times H')\setminus W')$.   Accordingly

\Centerline{$\lambda'V+\lambda'V'=\theta V+\theta V'
=\theta(E'\times H')=\lambda'(E'\times H')$,}

\noindent so that $\lambda'W'$ is defined and equal to $\theta W'$.

What this means is that $W\cap(F_n\times G_n)\cap(E\times H)\in\Cal A$
whenever $\mu E$ and $\nu H$ are finite.   So
$W\cap(F_n\times G_n)\in\Lambda$, by 251H;
as $n$ is arbitrary, $W\in\Lambda$ and $\lambda'W$ is defined.

\Quer\ Suppose, if possible, that $\lambda'W>\theta W$.   Then there is
some $n\in\Bbb N$ such that $\lambda'(W\cap(F_n\times G_n))>\theta W$.
Because $\lambda$ is semi-finite, 213B tells us that there is some
$\lambda$-simple function $h$ such that
$h\le(f\otimes g)\times\chi(W\cap(F_n\times G_n))$ and
$\int h\,d\lambda>\theta W$;  setting $V=\{(x,y):h(x,y)>0\}$, we see that $V\subseteq W\cap(F_n\times G_n)$,
$\lambda V$ is defined and finite and $\lambda'V>\theta W$.   Now there
must be sets $E\in\Sigma$, $H\in\Tau$ such that $\mu E$ and $\nu F$ are
both finite and
$\lambda(V\setminus(E\times H))<4^{-n}(\lambda'V-\theta W)$.   But in this case $V\in\Lambda\subseteq\dom\theta$ (by (d)), so we can apply the argument just above to $V$ and conclude that
$V\cap(E\times H)=V\cap(F_n\times G_n)\cap(E\times H)$ belongs to
$\Cal A$.   And now

$$\eqalign{\lambda'V
&=\lambda'(V\cap(E\times H))+\lambda'(V\setminus(E\times H))\cr
&\le\theta(V\cap(E\times H))+4^n\lambda(V\setminus(E\times H))
<\theta V+\lambda'V-\theta W
\le\lambda'V,\cr}$$

\noindent which is absurd.\ \Bang

So $\lambda'W$ is defined and not greater than $\theta W$.\
\Qed

\medskip

{\bf (f)} Putting this together with (d), we see that $\lambda'=\theta$,
as claimed.
}%end of proof of 253I

\cmmnt{\medskip

\noindent{\bf Remark} If $\mu'$ and $\nuprime$ are totally finite, so that
they are `truly continuous' with respect to $\mu$ and $\nu$ in the sense
of 232Ab, then $f$ and $g$ are integrable, so $f\otimes g$ is
$\lambda$-integrable, and $\theta=\lambda'$ is truly continuous with
respect to $\lambda$.

The proof above can be simplified using fragments of the general theory
of complete locally determined spaces, which will be given in \S412 %412J
in Volume 4.}%end of comment

\leader{*253J}{Upper \dvrocolon{integrals}}\cmmnt{ The idea of 253D
can be repeated in terms of upper integrals, as follows.

\medskip

\noindent}{\bf Proposition} Let $(X,\Sigma,\mu)$ and $(Y,\Tau,\nu)$ be
$\sigma$-finite measure spaces, with c.l.d.\ product measure $\lambda$.
Then for any functions $f$ and $g$, defined on conegligible subsets of
$X$ and $Y$ respectively, and taking values in $[0,\infty]$,

\Centerline{$\overlineint f\otimes g\,d\lambda
=\overlineint fd\mu\cdot\overlineint g\,d\nu$.}

\cmmnt{\medskip

\noindent{\bf Remark} Here $(f\otimes g)(x,y)=f(x)g(y)$ for all
$x\in\dom f$ and $y\in\dom g$, taking $0\cdot\infty=0$, as in \S135.
}%end of comment

\proof{{\bf (a)} I show first that
$\overline{\int}f\otimes g\le\overline{\int}f\overline{\int}g$.   \Prf\
If $\overline{\int}f=0$, then $f=0$ a.e., so $f\otimes g=0$ a.e.\ and
the result is immediate.   The same argument applies if
$\overline{\int}g=0$.   If both $\overline{\int}f$ and
$\overline{\int}g$ are non-zero, and either is infinite, the result is
trivial.   So let us suppose that both are finite.   In this case there
are integrable $f_0$, $g_0$ such that $f\leae f_0$, $g\leae g_0$,
$\overline{\int}f=\int f_0$ and $\overline{\int}g=\int g_0$ (133Ja/135Ha).
So $f\otimes g\leae f_0\otimes g_0$, and

\Centerline{$
\overlineint f\otimes g\le\int f_0\otimes g_0
=\int f_0\int g_0=\overlineint f\overlineint g$,}

\noindent by 253D.\ \Qed

\medskip

{\bf (b)} For the reverse inequality, we need consider only the case in
which $\overline{\int}f\otimes g$ is finite, so that there is a
$\lambda$-integrable function $h$ such that $f\otimes g\leae h$ and
$\overline{\int}f\otimes g=\int h$.   Set

\Centerline{$f_0(x)=\int h(x,y)\nu(dy)$}

\noindent whenever this is defined in $\Bbb R$, which is almost
everywhere, by Fubini's theorem (252B-252C).   Then
$f_0(x)\ge f(x)\overline{\int}g\,d\nu$ for every
$x\in\dom f_0\cap\dom f$, which is a conegligible set in $X$;  so

\Centerline{$
\overlineint f\otimes g=\int h\,d\lambda
=\int f_0d\mu\ge\overlineint f\overlineint g$,}

\noindent as required.
}%end of proof of 253J

\leader{*253K}{}\cmmnt{ A similar argument applies to upper
integrals of sums, as follows.

\medskip

\noindent}{\bf Proposition} Let $(X,\Sigma,\mu)$ and $(Y,\Tau,\nu)$ be
probability spaces, with c.l.d.\ product measure $\lambda$.   Then for
any real-valued functions $f$, $g$ defined on conegligible subsets of
$X$, $Y$ respectively,

\Centerline{$
\overlineint f(x)+g(y)\,\lambda(d(x,y))
=\overlineint f(x)\mu(dx)+\overlineint g(y)\nu(dy)$,}

\noindent at least when the right-hand side is defined in
$[-\infty,\infty]$.

\proof{Set $h(x,y)=f(x)+g(y)$ for $x\in\dom f$ and $y\in\dom g$, so that
$\dom h$ is $\lambda$-conegligible.

\medskip

{\bf (a)} As in 253J, I start by showing that
$\overline{\int}h\le\overline{\int}f+\overline{\int}g$.   \Prf\ If
either $\overline{\int}f$ or $\overline{\int}g$ is $\infty$, this is
trivial.   Otherwise, take integrable functions $f_0$, $g_0$ such that
$f\leae f_0$ and $g\leae g_0$.   Set
$h_0=(f_0\otimes\chi Y)+(\chi X\otimes g_0)$;  then
$h\le h_0\,\,\lambda$-a.e., so

\Centerline{$
\overlineint h\,d\lambda\le\int h_0d\lambda
=\int f_0d\mu+\int g_0d\nu$.}

\noindent As $f_0$, $g_0$ are arbitrary,
$\overline{\int}h\le\overline{\int}f+\overline{\int}g$.\ \Qed

\medskip

{\bf (b)} For the reverse inequality, suppose that $h\le h_0$ for
$\lambda$-almost every $(x,y)$, where $h_0$ is $\lambda$-integrable.
Set $f_0(x)=\int h_0(x,y)\nu(dy)$ whenever this is defined in $\Bbb R$.
Then $f_0(x)\ge f(x)+\overline{\int}g\,d\nu$ whenever $x\in\dom
f\cap\dom f_0$, so

\Centerline{$\int h_0\,d\lambda
=\int f_0d\mu\ge\overlineint fd\mu+\overlineint g\,d\nu$.}

\noindent As $h_0$ is arbitrary,
$\overline{\int}h\ge\overline{\int}f+\overline{\int}g$, as required.
}%end of proof of 253K

\leader{253L}{Complex spaces} As usual, the ideas\cmmnt{ of 253F and
253H} apply essentially unchanged to complex $L^1$ spaces.   Writing
$L^1_{\Bbb C}(\mu)$, etc., for the complex $L^1$ spaces involved, we
have the following\cmmnt{ results}.
Throughout, let $(X,\Sigma,\mu)$ and $(Y,\Tau,\nu)$
be measure spaces, and $\lambda$ the c.l.d.\ product measure on
$X\times Y$.

\spheader 253La If $f\in\eusm L^0_{\Bbb C}(\mu)$ and
$g\in\eusm L^0_{\Bbb C}(\nu)$ then $f\otimes g$, defined by the formula
$(f\otimes g)(x,y)=f(x)g(y)$ for $x\in\dom f$ and $y\in\dom g$, belongs
to $\eusm L^0_{\Bbb C}(\lambda)$.

\spheader 253Lb If $f\in\eusm L^1_{\Bbb C}(\mu)$ and
$g\in\eusm L^1_{\Bbb C}(\nu)$ then
$f\otimes g\in\eusm L^1_{\Bbb C}(\lambda)$ and
$\int f\otimes g\,d\lambda=\int fd\mu\int g\,d\nu$.

\spheader 253Lc We have a bilinear operator $(u,v)\mapsto u\otimes v:
L^1_{\Bbb C}(\mu)\times L^1_{\Bbb C}(\nu)\to L^1_{\Bbb C}(\lambda)$
defined by writing $f^{\ssbullet}\otimes g^{\ssbullet}=(f\otimes
g)^{\ssbullet}$ for all $f\in\eusm L^1_{\Bbb C}(\mu)$, $g\in \eusm
L^1_{\Bbb C}(\nu)$.

\spheader 253Ld If $W$ is any complex Banach space and
$\phi:L^1_{\Bbb C}(\mu)\times L^1_{\Bbb C}(\nu)\to W$ is any bounded
bilinear operator, then there is a unique bounded linear operator
$T:L^1_{\Bbb C}(\lambda)\to W$ such that $T(u\otimes v)=\phi(u,v)$ for
every $u\in L^1_{\Bbb C}(\mu)$ and $v\in L^1_{\Bbb C}(\nu)$, and
$\|T\|=\|\phi\|$.

\spheader 253Le If $\mu$ and $\nu$ are complete probability measures,
and $\Lambda_1=\{E\times Y:E\in\Sigma\}$, then for any $f\in\eusm
L^1_{\Bbb C}(\lambda)$ we have a conditional expectation $g$ of $f$ on
$\Lambda_1$ given by setting $g(x,y)=\int f(x,z)\nu(dz)$ whenever this
is defined.

\exercises{
\leader{253X}{Basic exercises $\pmb{>}$(a)}
%\spheader 253Xa
Let $U$, $V$ and $W$ be linear
spaces.   Show that the set of
bilinear operators from $U\times V$ to $W$ has a natural linear structure
agreeing with those of $\eurm L(U;\eurm L(V;W))$ and $\eurm L(V;\eurm
L(U;W))$, writing $\eurm L(U;W)$
for the linear space of linear operators from $U$ to $W$.

\sqheader 253Xb Let $U$, $V$ and $W$ be normed spaces.   (i)
Show that for a
bilinear operator $\phi:U\times V\to W$ the following are equiveridical:
($\alpha$) $\phi$ is bounded in the sense of 253Ab;  ($\beta$) $\phi$ is
continuous;  ($\gamma$) $\phi$ is continuous at some point of $U\times
V$.   (ii) Show that the space of bounded bilinear operators from $U\times V$
to $W$ is a linear subspace of the space of all bilinear operators from
$U\times V$ to $W$, and that the functional $\|\,\|$ defined in 253Ab is
a norm, agreeing with the norms of $\eurm B(U;\eurm B(V;W))$ and $\eurm
B(V;\eurm B(U;W))$, writing $\eurm B(U;W)$ for the normed space of
bounded linear operators from $U$ to $W$.

\spheader 253Xc Let
$(X_1,\Sigma_1,\mu_1),\ldots,(X_n,\Sigma_n,\mu_n)$ be
measure spaces, and $\lambda$ the c.l.d.\ product measure on
$X_1\times\ldots\times X_n$, as described in 251W.   Let $W$ be a
Banach space, and suppose that
$\phi:L^1(\mu_1)\times\ldots\times L^1(\mu_n)\to W$ is {\bf multilinear}
(that is, linear in each variable
separately) and {\bf bounded} (that is,
$\|\phi\|=\sup\{\phi(u_1,\ldots,u_n):
\|u_i\|_1\le 1\Forall i\le n\}<\infty$).   Show that there is a unique
bounded linear operator
$T:L^1(\lambda)\to W$ such that $T\otimes=\phi$, where
$\otimes:L^1(\mu_1)\times\ldots\times L^1(\mu_n)\to L^1(\lambda)$ is a
canonical multilinear operator (to be defined).

\spheader 253Xd Let $(X,\Sigma,\mu)$ and $(Y,\Tau,\nu)$ be measure
spaces, and $\lambda$ the c.l.d.\ product measure on $X\times Y$.   Show
that if $A\subseteq L^1(\mu)$ and $B\subseteq L^1(\nu)$ are both
uniformly integrable, then $\{u\otimes v:u\in A,\,v\in B\}$ is uniformly
integrable in $L^1(\lambda)$.

\sqheader 253Xe Let $(X,\Sigma,\mu)$ and $(Y,\Tau,\nu)$ be measure
spaces and $\lambda$ the c.l.d.\ product measure on $X\times Y$.
Show that

\quad (i) we have a bilinear operator
$(u,v)\mapsto u\otimes v:L^0(\mu)\times L^0(\nu)\to L^0(\lambda)$
given by setting
$f^{\ssbullet}\otimes g^{\ssbullet}=(f\otimes g)^{\ssbullet}$ for all
$f\in\eusm L^0(\mu)$ and $g\in \eusm L^0(\nu)$;

\quad (ii) if $1\le p\le\infty$ then $u\otimes v\in L^p(\lambda)$ and
$\|u\otimes v\|_p=\|u\|_p\|v\|_p$ for all $u\in L^p(\mu)$ and
$v\in L^p(\nu)$;

\quad (iii) if $u$, $u'\in L^2(\mu)$ and $v$, $v'\in L^2(\nu)$ then the
inner product $\innerprod{u\otimes v}{u'\otimes v'}$, taken in
$L^2(\lambda)$, is just $\innerprod{u}{u'}\innerprod{v}{v'}$;

\quad (iv) the map
$(u,v)\mapsto u\otimes v:L^0(\mu)\times L^0(\nu)\to L^0(\lambda)$
is continuous if $L^0(\mu)$, $L^0(\nu)$ and $L^0(\lambda)$
are all given their topologies of convergence in measure.

\spheader 253Xf In 253Xe, assume that $\mu$ and $\nu$ are semi-finite.
Show that if $u_0,\ldots,u_n$ are linearly independent members of
$L^0(\mu)$ and $v_0,\ldots,v_n\in L^0(\nu)$ are not all $0$, then
$\sum_{i=0}^nu_i\otimes v_i\ne 0$ in $L^0(\lambda)$.   ({\it Hint\/}:
start by finding sets $E\in\Sigma$, $F\in\Tau$ of finite measure such
that $u_0\times\chi E^{\ssbullet},\ldots,u_n\times\chi E^{\ssbullet}$
are linearly independent and
$v_0\times\chi F^{\ssbullet},\ldots,v_n\times\chi F^{\ssbullet}$
are not all $0$.)

\spheader 253Xg In 253Xe, assume that $\mu$ and $\nu$ are semi-finite.
If $U$, $V$ are linear subspaces of $L^0(\mu)$ and $L^0(\nu)$
respectively, write $U\otimes V$ for the linear subspace of
$L^0(\lambda)$ generated by $\{u\otimes v:u\in U,\,v\in V\}$.
Show that if $W$ is any linear space and $\phi:U\times V\to W$ is a
bilinear operator, there is a unique linear operator $T:U\otimes V\to W$
such that $T(u\otimes v)=\phi(u,v)$ for all $u\in U$, $v\in V$.
({\it Hint\/}: start by showing that if $u_0,\ldots,u_n\in U$ and
$v_0,\ldots,v_n\in V$ are such that $\sum_{i=0}^nu_i\otimes v_i=0$, then
$\sum_{i=0}^n\phi(u_i,v_i)=0$ -- do this by expressing the $u_i$ as
linear combinations of some linearly independent family and applying
253Xf.)

\sqheader 253Xh Let $(X,\Sigma,\mu)$ and $(Y,\Tau,\nu)$ be complete
probability spaces, with c.l.d.\ product measure $\lambda$.   Suppose
that $p\in[1,\infty]$ and that $f\in\eusm L^p(\lambda)$.   Set
$g(x)=\int f(x,y)\nu(dy)$ whenever this is defined.   Show that
$g\in\eusm L^p(\mu)$ and that $\|g\|_p\le\|f\|_p$.   ({\it Hint\/}:  253H,
244M.)

\spheader 253Xi Let $(X,\Sigma,\mu)$ and $(Y,\Tau,\nu)$ be measure
spaces, with c.l.d.\ product measure $\lambda$, and
$p\in\coint{1,\infty}$.   Show that $\{w:w\in L^p(\lambda),\,w\ge 0\}$
is the closed convex hull in $L^p(\lambda)$ of
$\{u\otimes v:u\in L^p(\mu),\,v\in L^p(\nu),\,u\ge 0,\,v\ge 0\}$
(see 253Xe(ii) above).

\leader{253Y}{Further exercises (a)}
%\spheader 253Ya
Let $(X,\Sigma,\mu)$ and $(Y,\Tau,\nu)$ be measure spaces, and
$\lambda_0$ the primitive product measure on $X\times Y$.  Show that if
$f\in\eusm L^0(\mu)$ and $g\in\eusm L^0(\nu)$, then
$f\otimes g\in\eusm L^0(\lambda_0)$.
%253C

\spheader 253Yb Let $(X,\Sigma,\mu)$ and
$(Y,\Tau,\nu)$ be measure spaces, and $\lambda_0$ the primitive
product measure on $X\times Y$.   Show that if $f\in\eusm L^1(\mu)$ and
$g\in\eusm L^1(\nu)$, then $f\otimes g\in\eusm L^1(\lambda_0)$ and
$\int f\otimes g\,d\lambda_0=\int f\,d\mu\int g\,d\nu$.
%253Ya, 253C

\spheader 253Yc Let $(X,\Sigma,\mu)$ and $(Y,\Tau,\nu)$ be
measure spaces, and $\lambda_0$, $\lambda$ the primitive and c.l.d.\
product measures on $X\times Y$.   Show that the embedding 
$\eusm L^1(\lambda_0)\embedsinto\eusm L^1(\lambda)$
induces a Banach lattice isomorphism between $L^1(\lambda_0)$ and
$L^1(\lambda)$.
%253D

\spheader 253Yd Let $(X,\Sigma,\mu)$, $(Y,\Tau,\nu)$ be strictly
localizable measure spaces, with c.l.d.\ product measure $\lambda$.
Show that $L^{\infty}(\lambda)$ can be identified with $L^1(\lambda)^*$.
Show that under this identification
$\{w:w\in L^{\infty}(\lambda),\,w\ge 0\}$ is the weak*-closed convex hull of
$\{u\otimes v:u\in L^{\infty}(\mu),\,v\in L^{\infty}(\nu),
\,u\ge 0,\,v\ge 0\}$.
%253G, 253Xe

\spheader 253Ye Find a version of 253J valid when one of $\mu$, $\nu$
is not $\sigma$-finite.
%253J

\spheader 253Yf Let $(X,\Sigma,\mu)$ be any measure space and $V$ any
Banach space.   Write $\eusm L^1_V=\eusm L^1_V(\mu)$ for the set of
functions $f$
such that ($\alpha$) $\dom f$ is a conegligible subset of $X$ ($\beta$)
$f$ takes values in $V$ ($\gamma$) there is a conegligible set
$D\subseteq\dom
f$ such that $f[D]$ is separable and $D\cap f^{-1}[G]\in\Sigma$ for
every open set $G\subseteq V$ ($\delta$) the integral
$\int\|f(x)\|\mu(dx)$ is finite.   (These are the {\bf Bochner
integrable} functions from $X$ to
$V$.)   For $f$, $g\in\eusm L^1_V$ write $f\sim g$ if
$f=g\,\,\mu$-a.e.;  let $L^1_V$ be the set of equivalence classes
in $\eusm L^1_V$ under $\sim$.   Show that

\quad (i) $f+g$, $cf\in\eusm L^1_V$ for all $f$, 
$g\in\eusm L^1_V$, $c\in\Bbb R$;

\quad (ii) $L^1_V$ has a natural linear space structure, defined by
writing $f^{\ssbullet}+g^{\ssbullet}=(f+g)^{\ssbullet}$,
$cf^{\ssbullet}=(cf)^{\ssbullet}$ for $f$, $g\in\eusm L^1_V$ and
$c\in\Bbb R$;

\quad (iii) $L^1_V$ has a norm $\|\,\|$, defined by writing
$\|f^{\ssbullet}\|=\int\|f(x)\|\mu(dx)$ for $f\in\eusm L^1_V$;

\quad (iv) $L^1_V$ is a Banach space under this norm;

\quad (v) there is a natural map
$\otimes:\eusm L^1\times V\to\eusm L^1_V$ defined by writing
$(f\otimes v)(x)=f(x)v$ when $f\in\eusm L^1=\eusm L^1_{\Bbb R}(\mu)$,
$v\in V$ and $x\in\dom f$;

\quad (vi) there is a canonical bilinear operator
$\otimes:L^1\times V\to L^1_V$ defined by writing
$f^{\ssbullet}\otimes v=(f\otimes v)^{\ssbullet}$ for $f\in\eusm L^1$ and
$v\in V$;

\quad (vii) whenever $W$ is a Banach space and 
$\phi:L^1\times V\to W$ is a bounded bilinear operator, 
there is a unique bounded
linear operator $T:L^1_V\to W$ such that $T(u\otimes v)=\phi(u,v)$
for all $u\in L^1$ and $v\in V$, and $\|T\|=\|\phi\|$.   
%new 2008
(When $W=V$ and
$\phi(u,v)=(\int u)v$ for $u\in L^1$ and $v\in V$, $Tf^{\ssbullet}$ 
is called the {\bf Bochner integral} of $f$.)

%253+

\spheader 253Yg Let $(X,\Sigma,\mu)$ and $(Y,\Tau,\nu)$ be
measure spaces, and
$\lambda_0$ the primitive product measure on $X\times Y$.   If $f$ is a
$\lambda_0$-integrable function, write $f_x(y)=f(x,y)$ whenever this is
defined.   Show that we have a map $x\mapsto f_x^{\ssbullet}$ from a
conegligible subset
$D_0$ of $X$ to $L^1(\nu)$.   Show that this map is a Bochner integrable
function, as defined in 253Yf, 
%new 2008
and that its Bochner integral is $\int fd\lambda_0$.
%253Yf, 253+

\spheader 253Yh Let $(X,\Sigma,\mu)$ and $(Y,\Tau,\nu)$ be
measure spaces, and
suppose that $\phi$ is a function from $X$ to a separable subset of
$L^1(\nu)$ which is measurable in the sense that $\phi^{-1}[G]\in\Sigma$
for every open $G\subseteq L^1(\nu)$.   Show that
there is a $\Lambda$-measurable function $f$ from $X\times Y$ to $\Bbb
R$, where $\Lambda$ is the domain of the c.l.d.\ product measure on
$X\times Y$, such that $\phi(x)=f_x^{\ssbullet}$ for every $x\in X$,
writing $f_x(y)=f(x,y)$ for $x\in X$, $y\in Y$.
%253+

\spheader 253Yi Let $(X,\Sigma,\mu)$ and $(Y,\Tau,\nu)$ be measure
spaces, and $\lambda$ the c.l.d.\ product measure on $X\times Y$.   Show
that 253Yg provides a canonical identification between $L^1(\lambda)$
and $L^1_{L^1(\nu)}(\mu)$.
%253Yg, 253Yi, 253+

\spheader 253Yj Let $(X,\Sigma,\mu)$ and $(Y,\Tau,\nu)$ be complete
locally determined measure spaces, with c.l.d.\ product measure
$\lambda$.   (i) Suppose that $K\in\eusm L^2(\lambda)$,
$f\in\eusm L^2(\mu)$.   Show that $h(y)=\int K(x,y)f(x)dx$ is defined for almost
all $y\in Y$ and that $h\in\eusm L^2(\nu)$.   ({\it Hint\/}:  to see
that $h$ is defined a.e., consider $\int_{E\times F}K(x,y)f(x)d(x,y)$
for $\mu E$, $\nu F<\infty$;  to see that $h\in\eusm L^2$ consider
$\int h\times g$ where $g\in\eusm L^2(\nu)$.)   (ii) Show that the map
$f\mapsto h$ corresponds to a bounded linear operator
$T_K:L^2(\mu)\to L^2(\nu)$.   (iii) Show that the map $K\mapsto T_K$
corresponds to a bounded linear operator, of norm at most $1$, from
$L^2(\lambda)$ to $\eurm B(L^2(\mu);L^2(\nu))$.
%253+

\spheader 253Yk Suppose that $p$, $q\in[1,\infty]$ and that
$\bover1p+\bover1q=1$, interpreting $\bover1{\infty}$ as $0$ as usual.
Let $(X,\Sigma,\mu)$, $(Y,\Tau,\nu)$ be complete locally determined
measure spaces with c.l.d.\ product measure $\lambda$.    Show that the
ideas of 253Yj can be used to define a bounded linear operator, of norm
at most $1$, from $L^p(\lambda)$ to $\eurm B(L^q(\mu);L^p(\nu))$.
%253Yj, 253+

\spheader 253Yl In 253Xc, suppose that $W$ is a Banach lattice.   Show
that the following are equiveridical:  (i) $Tu\ge 0$ whenever
$u\in L^1(\lambda)$;  (ii) $\phi(u_1,\ldots,u_n)\ge 0$ whenever
$u_i\ge 0$ in $L^1(\mu_i)$ for each $i\le n$.
%253G, 253Xc
}%end of exercises

\endnotes{
\Notesheader{253} Throughout the main arguments of this section, I have
written the results in terms of the c.l.d.\ product measure;  of course
the isomorphism noted in 253Yc means that they could just as well have
been expressed in terms of the primitive product measure.   The
more restricted notion of integrability with respect to the primitive
product measure is indeed the one appropriate for the ideas of 253Yg.

Theorem 253F is a `universal mapping theorem';
it asserts that every bounded bilinear operator on $L^1(\mu)\times
L^1(\nu)$ factors through $\otimes:L^1(\mu)\times L^1(\nu)\to
L^1(\lambda)$, at least if the range space is a Banach space.   It is
easy to see that this property defines the pair $(L^1(\lambda),\otimes)$
up to Banach space isomorphism, in the following sense:  if $V$ is a
Banach space, and $\psi:L^1(\mu)\times L^1(\nu)\to V$ is a bounded
bilinear operator such that for every bounded bilinear operator $\phi$ from
$L^1(\mu)\times L^1(\nu)$ to any Banach space $W$ there is a unique
bounded linear operator $T:V\to W$ such that $T\psi = \phi$ and
$\|T\|=\|\phi\|$, then there is an isometric Banach space isomorphism
$S:L^1(\lambda)\to V$ such that $S\otimes=\psi$.    There is of course a
general theory of bilinear operators between Banach spaces;  in the language
of this theory, $L^1(\lambda)$ is, or is isomorphic to, the
`projective tensor product' of $L^1(\mu)$ and $L^1(\nu)$.   For an
introduction to
this subject, see {\smc Defant \& Floret 93}, \S I.3, or 
{\smc Semadeni 71}, \S20.   
I should perhaps emphasise, for the sake of those who
have not encountered tensor products before, that this theorem is
special to $L^1$ spaces.   While some of the same ideas can be applied
to other function spaces (see 253Xe-253Xg), there is no other class to
which 253F applies.

There is also a theory of tensor products of Banach lattices, for which
I do not think we are quite ready (it needs general ideas about ordered
linear spaces for which I mean to wait until Chapter 35 in the next volume).   However
253G shows that the ordering, and therefore the Banach lattice
structure, of
$L^1(\lambda)$ is determined by the ordering of $L^1(\mu)$ and
$L^1(\nu)$ and the map
$\otimes:L^1(\mu)\times L^1(\nu)\to L^1(\lambda)$.

The conditional expectation operators described in 253H are of very
great importance, largely because in this special context we have a
realization of the conditional expectation operator as a function $P_0$
from $\eusm L^1(\lambda)$ to $\eusm L^1(\lambda\restr\Lambda_1)$, not
just as a function from
$L^1(\lambda)$ to $L^1(\lambda\restr\Lambda_1)$, as in 242J.
As described here, $P_0(f+f')$ need not be equal, in the strict sense,
to $P_0f+P_0f'$;  it can have a larger domain.   In applications,
however, one might be willing to restrict attention to the linear space
$\eusm U$  of bounded $\Sigma\tensorhat\Tau$-measurable functions
defined everywhere on $X\times Y$, so that $P_0$ becomes an operator
from $\eusm U$ to itself (see 252P).
}%end of notes

\discrpage

