\frfilename{mt283.tex}
\versiondate{31.3.13}

\def\chaptername{Fourier analysis}
\def\sectionname{Fourier transforms I}
\copyrightdate{1994}

\newsection{283}

I turn now to the theory of Fourier transforms on $\Bbb R$.   In the
first of two sections on the subject, I present those parts of the
elementary theory which can be dealt with using the methods of the
previous section on Fourier series.   I find no way of making sense of
the theory, however, without introducing a fragment of L.Schwartz'
theory of distributions, which I present in \S284.
As in \S282, of course, this treatment also is nothing but a start in
the topic.

The whole theory can also be done in $\BbbR^r$.   I leave this
extension to the exercises, however, since there are few new ideas, the
formulae are significantly more complicated, and I shall not, in this
volume at least, have any use for the multidimensional versions of these
particular theorems, though some of the same ideas will appear, in
multidimensional form, in \S285.

\leader{283A}{Definitions} Let $f$ be a complex-valued function which is
integrable over $\Bbb R$.


\spheader 283Aa The {\bf Fourier transform} of $f$ is the
function $\varhatf:\Bbb R\to\Bbb C$ defined by setting

\Centerline{$\varhatf(y)=\Bover1{\sqrt{2\pi}}\int_{-\infty}^{\infty}
e^{-iyx}f(x)dx$}

\noindent for every $y\in\Bbb R$.   \cmmnt{(Of course the integral is
always defined because $x\mapsto e^{-iyx}$ is bounded and continuous,
therefore measurable.)}

\spheader 283Ab The {\bf inverse Fourier transform} of $f$ is
the function $\varcheckf:\Bbb R\to\Bbb C$ defined by setting

\Centerline{$\varcheckf(y)=\Bover1{\sqrt{2\pi}}\int_{-\infty}^{\infty}
e^{iyx}f(x)dx$}

\noindent for every $y\in\Bbb R$.

\cmmnt{
\leader{283B}{Remarks (a)} It is a mildly vexing feature of the theory
of Fourier transforms -- vexing, that is,
for outsiders like myself -- that there is in fact
no standard definition of `Fourier transform'.
The commonest definitions are, I think,

\Centerline{$\varhatf(y)=\Bover1{\sqrt{2\pi}}\int_{-\infty}^{\infty}
e^{\mp iyx}f(x)dx$,}

\Centerline{$\varhatf(y)=\int_{-\infty}^{\infty}
e^{\mp iyx}f(x)dx$,}

\Centerline{$\varhatf(y)=\int_{-\infty}^{\infty}
e^{\mp 2\pi iyx}f(x)dx$,}

\noindent corresponding to inverse transforms

\Centerline{$\varcheckf(y)=\Bover1{\sqrt{2\pi}}\int_{-\infty}^{\infty}
e^{\pm iyx}f(x)dx$,}

\Centerline{$\varcheckf(y)=\Bover1{{2\pi}}\int_{-\infty}^{\infty}
e^{\pm iyx}f(x)dx$,}

\Centerline{$\varcheckf(y)=\int_{-\infty}^{\infty}
e^{\pm 2\pi iyx}f(x)dx$.}

\noindent I leave it to you to check that the whole theory can be
carried through with any of these six pairs, and to investigate other
possibilities (see 283Xa-283Xb below).

\spheader 283Bb The phrases `Fourier transform', `inverse
Fourier transform' make it plain that $(\varhatf)\varspcheck$ is
supposed to be $f$, at least some of the time.   This is indeed the
case, but the class of $f$ for which this is true in the literal sense
is somewhat constrained, and we shall have to wait a little while before
investigating it.

\spheader 283Bc No amount of juggling with constants, in the
manner of (a) above, can make $\varhatf$ and $\varcheckf$ quite the
same.   However, on the definitions I have chosen, we do have
$\varcheckf(y)=\varhatf(-y)$ for every $y$, so that $\varcheckf$ and
$\varhatf$ will share essentially all the properties of interest to us
here;  in particular, everything in the next proposition will be valid
with $\varspcheck$ in place of $\varsphat$, if you change signs at the
right points in parts (c), (h) and (i).
}%end of comment

\leader{283C}{Proposition} Let $f$ and $g$ be complex-valued functions
which are integrable over $\Bbb R$.

(a) $(f+g)\varsphat=\varhatf+\varhat{g}$.

(b) $(cf)\varsphat=c\varhatf$ for every $c\in\Bbb C$.

(c) If $c\in\Bbb R$ and $h(x)=f(x+c)$ whenever this is
defined, then $\varhat{h}(y)=e^{icy}\varhatf(y)$ for every $y\in\Bbb R$.

(d) If $c\in\Bbb R$ and $h(x)=e^{icx}f(x)$ for every
$x\in\dom f$, then $\varhat{h}(y)=\varhatf(y-c)$ for every
$y\in\Bbb R$.

(e) If $c>0$ and $h(x)=f(cx)$ whenever this is defined,
then $\varhat{h}(y)=\Bover1c\varhatf(\Bover{y}{c})$
for every $y\in\Bbb R$.

(f) $\varhatf:\Bbb R\to\Bbb C$ is continuous.

(g) $\lim_{y\to\infty}\varhatf(y)=\lim_{y\to-\infty}\varhatf(y)=0$.

(h) If $\int_{-\infty}^{\infty}|xf(x)|dx<\infty$, then $\varhatf$ is
differentiable, and its derivative is

\Centerline{$\varhatf\vthsp'(y)=-\Bover{i}{\sqrt{2\pi}}
\int_{-\infty}^{\infty}e^{-iyx}xf(x)dx$}

\noindent for every $y\in\Bbb R$.

(i) If $f$ is absolutely continuous on every bounded interval and $f'$
is integrable, then $(f')\varsphat(y)=iy\varhatf(y)$ for every $y\in\Bbb
R$.

\proof{{\bf (a)} and {\bf (b)} are trivial, and {\bf (c)}, {\bf (d)} and
{\bf (e)} are elementary substitutions.

\medskip

{\bf (f)} If $\sequencen{y_n}$ is any convergent sequence in $\Bbb R$
with limit $y$, then

$$\eqalign{\varhatf(y)
&=\Bover1{\sqrt{2\pi}}
\int_{-\infty}^{\infty}\lim_{n\to\infty}e^{-iy_nx}f(x)dx\cr
&=\lim_{n\to\infty}\Bover1{\sqrt{2\pi}}
\int_{-\infty}^{\infty}e^{-iy_nx}f(x)dx
=\lim_{n\to\infty}\varhatf(y_n)\cr}$$

\noindent by Lebesgue's Dominated Convergence Theorem, because
$|e^{-iy_nx}f(x)|\le|f(x)|$ for every $n\in\Bbb N$ and $x\in\dom f$.   As
$\sequencen{y_n}$ is arbitrary, $\varhatf$ is continuous.

\medskip

{\bf (g)} This is just the Riemann-Lebesgue lemma (282E).

\medskip

{\bf (h)} The point is that $|\pd{}{y}e^{-iyx}f(x)|=|xf(x)|$ whenever
$x\in\dom f$ and $y\in\Bbb R$.   So by 123D

$$\eqalign{\varhatf\vthsp'(y)
&=\Bover1{\sqrt{2\pi}}\Bover{d}{dy}
     \int_{-\infty}^{\infty}e^{-iyx}f(x)dx
=\Bover1{\sqrt{2\pi}}\Bover{d}{dy}\int_{\dom f}
     e^{-iyx}f(x)dx\cr
&=\Bover1{\sqrt{2\pi}}\int_{\dom f}
     \Bover{\partial}{\partial y}e^{-iyx}f(x)dx
=\Bover1{\sqrt{2\pi}}\int_{-\infty}^{\infty}
     -ixe^{-iyx}f(x)dx\cr
&=-\Bover{i}{\sqrt{2\pi}}\int_{-\infty}^{\infty}
     xe^{-iyx}f(x)dx.\cr}$$

\medskip

{\bf (i)} Because $f$ is absolutely continuous on every bounded
interval,

\Centerline{$f(x)=f(0)+\int_0^xf'$ for $x\ge 0$,\quad
$f(x)=f(0)-\int_x^0f'$ for $x\le 0$.}

\noindent Because $f'$ is integrable,

\Centerline{$\lim_{x\to\infty}f(x)=f(0)+\int_0^{\infty}f'$,\quad
$\lim_{x\to-\infty}f(x)=f(0)-\int_{-\infty}^0f'$}

\noindent both exist.   Because $f$ also is integrable, both limits must
be zero.   Now we can integrate by parts (225F) to see that

$$\eqalign{(f')\varsphat(y)
&=\Bover1{\sqrt{2\pi}}\int_{-\infty}^{\infty}e^{-iyx}f'(x)dx
=\Bover1{\sqrt{2\pi}}\lim_{a\to\infty}\int_{-a}^{a}e^{-iyx}f'(x)dx\cr
&=\Bover1{\sqrt{2\pi}}\bigl(\lim_{a\to\infty}e^{-iya}f(a)
-\lim_{a\to-\infty}e^{-iya}f(a)\bigr)
+\Bover{iy}{\sqrt{2\pi}}\int_{-\infty}^{\infty}e^{-iyx}f(x)dx\cr
&=iy\varhatf(y).\cr}$$
}%end of proof of 283C

\leader{283D}{Lemma} (a) $\lim_{a\to\infty}\int_{0}^a\bover{\sin
x}{x}dx=\bover{\pi}2$, $\lim_{a\to\infty}\int_{-a}^a\bover{\sin
x}{x}dx=\pi$.

(b) There is a $K<\infty$ such that $|\int_a^b\bover{\sin cx}{x}dx|\le
K$ whenever $a\le b$ and $c\in\Bbb R$.

\proof{{\bf (a)(i)} Set

\Centerline{$F(a)=\int_0^a\Bover{\sin x}{x}dx$ if $a\ge 0$,\quad
$F(a)=-\int_{-a}^0\Bover{\sin x}{x}dx$ if $a\le 0$,}

\noindent so that $F(a)=-F(-a)$ and
$\int_a^b\bover{\sin x}{x}dx=F(b)-F(a)$ for all $a\le b$.

If $0<a\le b$, then by 224J

\Centerline{$|\int_a^b\Bover{\sin x}{x}dx|
\le(\Bover1b+\Bover1a-\Bover1b)\sup_{c\in[a,b]}
  |\int_a^c\sin x\,dx|
\le\Bover1a\sup_{c\in[a,b]}|\cos a-\cos c|
\le\Bover2a$.}

\noindent In particular, $|F(n)-F(m)|\le\bover2{m}$ if $0<m\le n$ in
$\Bbb N$, and $\sequencen{F(n)}$ is a Cauchy sequence with limit
$\gamma$ say;  now

\Centerline{$|\gamma - F(a)|
=\lim_{n\to\infty}|F(n)-F(a)|\le\Bover2a$}

\noindent for every $a>0$, so $\lim_{a\to\infty}F(a)=\gamma$.
Of course we also have

\Centerline{$\lim_{a\to\infty}\int_{-a}^a\Bover{\sin x}{x}dx
=\lim_{a\to\infty}(F(a)-F(-a))
=\lim_{a\to\infty}2F(a)=2\gamma$.}

\medskip

\quad{\bf (ii)} So now I have to calculate $\gamma$.    For this,
observe first that

\Centerline{$2\gamma
=\lim_{a\to\infty}\int_{-\pi a}^{\pi a}\Bover{\sin x}{x}dx
=\lim_{a\to\infty}\int_{-\pi}^{\pi}\Bover{\sin at}{t}dt$}

\noindent (substituting $x=t/a$).   Next,

\Centerline{$\lim_{t\to 0}\Bover1t-\Bover1{2\sin{1\over 2}t}=\lim_{u\to
0}\Bover{\sin u-u}{2u\sin u}=0$,}

\noindent so

\Centerline{$\int_{-\pi}^{\pi}
  \bigl|\Bover1t-\Bover1{2\sin{1\over 2}t}\bigr|dt<\infty$,}

\noindent and by the Riemann-Lebesgue lemma (282Fb)

\Centerline{$\lim_{a\to\infty}\int_{-\pi}^{\pi}
   \bigl(\Bover1t-\Bover1{2\sin{1\over 2}t}\bigr)\sin at\,dt=0$.}

\noindent But we know that

\Centerline{$\int_{-\pi}^{\pi}
\Bover{\sin(n+{1\over 2})t\pushbottom{3.5pt}}{2\sin{1\over 2}t}dt
=\pi$}

\noindent for every $n$ (using 282Dc), so we must have

$$\eqalign{\lim_{a\to\infty}\int_{-a}^a\Bover{\sin t}{t}dt
&=\lim_{a\to\infty}\int_{-\pi}^{\pi}\Bover{\sin at}{t}dt
=\lim_{a\to\infty}\int_{-\pi}^{\pi}
      \Bover{\sin at}{2\sin{1\over 2}t}dt\cr
&=\lim_{n\to\infty}\int_{-\pi}^{\pi}
    \Bover{\sin(n+{1\over 2})t\pushbottom{3.5pt}}{2\sin{1\over 2}t}dt
=\pi,\cr}$$

\noindent and $\gamma=\pi/2$, as claimed.

\medskip

{\bf (b)} Because $F$ is continuous and

\Centerline{$\lim_{a\to\infty}F(a)=\gamma=\Bover{\pi}2$, \quad
$\lim_{a\to-\infty}F(a)=-\gamma=-\Bover{\pi}2$,}

\noindent $F$ is bounded;  say $|F(a)|\le K_1$ for all $a\in\Bbb R$.
Try $K=2K_1$.   Now suppose that $a<b$ and $c\in\Bbb R$.   If $c>0$,
then

\Centerline{$|\int_a^b\Bover{\sin cx}{x}dx|
=|\int_{ac}^{bc}\Bover{\sin t}{t}dt|
=|F(bc)-F(ac)|\le 2K_1=K$,}

\noindent substituting $x=t/c$.   If $c<0$, then

\Centerline{$|\int_a^b\Bover{\sin cx}{x}dx|
=|-\int_a^b\Bover{\sin(-c)x}{x}dx|
\le K$;}

\noindent while if $c=0$ then

\Centerline{$|\int_a^b\Bover{\sin cx}{x}dx|
=0\le K$.}
}%end of proof of 283D

\leader{283E}{}\cmmnt{ The hardest work of this section will lie in
the `pointwise inversion theorems' 283I and 283K below.   I begin
however with a relatively easy, and at least equally important, result,
showing (among other things) that an integrable function $f$ can
(essentially) be recovered from its Fourier transform.

\medskip

\noindent}{\bf Lemma} Whenever $c<d$ in $\Bbb R$,

$$\eqalign{\lim_{a\to\infty}\int_{-a}^{a}
     e^{-iyx}\,\Bover{e^{idy}-e^{icy}}{y}dy
&=2\pi i\text{ if }c<x<d,\cr
&=\pi i\text{ if }x=c\text{ or }x=d,\cr
&=0\text{ if }x<c\text{ or } x>d.\cr}$$

\proof{ We know that for any $b>0$

\Centerline{$\lim_{a\to\infty}\int_{-a}^{a}\Bover{\sin bx}{x}dx
=\lim_{a\to\infty}\int_{-ab}^{ab}\Bover{\sin t}{t}dt
=\pi$}

\noindent (subsituting $x=t/b$), and therefore that for any $b<0$

\Centerline{$\lim_{a\to\infty}\int_{-a}^{a}\Bover{\sin bx}{x}dx
=-\lim_{a\to\infty}\int_{-a}^{a}\Bover{\sin(-b)x}{x}dx
=-\pi$.}

\noindent Now consider, for $x\in\Bbb R$,

\Centerline{$\lim_{a\to\infty}\int_{-a}^{a}e^{-iyx}
\,\Bover{e^{idy}-e^{icy}}{y}dy$.}

\noindent First note that all the integrals $\int_{-a}^a$ exist, because

\Centerline{$\lim_{y\to 0}\Bover{e^{idy}-e^{icy}}{y}=i(d-c)$}

\noindent is finite, and the integrand is certainly continuous except at
$0$.   Now we have

$$\eqalignno{\int_{-a}^ae^{-iyx}&\Bover{e^{idy}-e^{icy}}{y}dy\cr
&=\int_{-a}^a\Bover{e^{i(d-x)y}-e^{i(c-x)y}}{y}dy\cr
&=\int_{-a}^a\Bover{\cos(d-x)y-\cos(c-x)y}{y}dy
 +i\int_{-a}^a\Bover{\sin(d-x)y-\sin(c-x)y}{y}dy\cr
&=i\int_{-a}^a\Bover{\sin(d-x)y-\sin(c-x)y}{y}dy\cr}$$

\noindent because $\cos$ is an even function, so

\Centerline{$\int_{-a}^{a}\Bover{\cos(d-x)y-\cos(c-x)y}{y}dy=0$}

\noindent for every $a\ge 0$.   (Once again, this integral exists
because

\Centerline{$\lim_{y\to 0}\Bover{\cos(d-x)y-\cos(c-x)y}{y}=0$.)}

\noindent   Accordingly

$$\eqalign{\lim_{a\to\infty}\int_{-a}^a
e^{-iyx}\Bover{e^{idy}-e^{icy}}{y}dy
&=i\lim_{a\to\infty}\int_{-a}^a\Bover{\sin(d-x)y}{y}dy
   -i\lim_{a\to\infty}\int_{-a}^a\Bover{\sin(c-x)y}{y}dy\cr
&=i\pi-i\pi=0\text{ if }x<c,\cr
&=i\pi-0=\pi i\text{ if }x=c,\cr
&=i\pi+i\pi=2\pi i\text{ if }c<x<d,\cr
&=0+i\pi=\pi i\text{ if }x=d,\cr
&=-i\pi+i\pi=0\text{ if }x>d.\cr}$$
}%end of proof of 283E

\leader{283F}{Theorem} Let $f$ be a complex-valued function which is
integrable over $\Bbb R$, and $\varhatf$ its Fourier transform.   Then
whenever $c\le d$ in $\Bbb R$,

$$\int_c^df=\Bover{i}{\sqrt{2\pi}}\lim_{a\to\infty}
\int_{-a}^{a}\Bover{e^{icy}-e^{idy}}{y}\varhatf(y)dy.$$

\proof{ If $c=d$ this is trivial;  let us suppose that $c<d$.

\medskip

{\bf (a)} Writing

$$\theta_a(x)=\biggerint_{-a}^{a}e^{-iyx}
\,\Bover{e^{idy}-e^{icy}}{y}dy$$

\noindent for $x\in\Bbb R$ and $a\ge 0$, 283E tells us that

\Centerline{$\lim_{a\to\infty}\theta_a(x)=2\pi i\theta(x)$}

\noindent where $\theta=\bover12(\chi[c,d]+\chi\ooint{c,d})$ takes the
value $1$ inside the interval $[c,d]$, $0$ outside and the value
$\bover12$ at the endpoints.   At the same time,

$$\eqalignno{|\theta_a(x)|
&=|\int_{-a}^a\Bover{\sin(d-x)y-\sin(c-x)y}{y}dy|\cr
\displaycause{see the proof of 283E}
&\le|\int_{-a}^a\Bover{\sin(d-x)y}{y}dy|
  +|\int_{-a}^a\Bover{\sin(c-x)y}{y}dy|
\le 2K\cr}$$

\noindent for all $a\ge 0$ and $x\in\Bbb R$, where $K$ is the constant of
283Db.   Consequently $|f\times\theta_a|\le 2K|f|$ everywhere on
$\dom f$, for every $a\ge 0$, and
(applying Lebesgue's Dominated Convergence
Theorem to sequences $\sequencen{f\times\theta_{a_n}}$, where
$a_n\to\infty$)

\Centerline{$\lim_{a\to\infty}
\int f\times\theta_a=2\pi i\int f\times\theta=2\pi i\int_c^df$.}

\medskip

{\bf (b)} Now consider the limit in the statement of the theorem.   We
have

$$\eqalignno{\int_{-a}^a
\bover{e^{icy}-e^{idy}}{y}\varhatf(y)dy
&=\bover1{\sqrt{2\pi}}\int_{-a}^a
\int_{-\infty}^{\infty}
\bover{e^{icy}-e^{idy}}{y}e^{-iyx}f(x)dxdy\cr
&=\bover1{\sqrt{2\pi}}\int_{-\infty}^{\infty}\int_{-a}^a
\bover{e^{icy}-e^{idy}}{y}e^{-iyx}f(x)dydx\cr
&=-\bover1{\sqrt{2\pi}}\int_{-\infty}^{\infty}f\times\theta_a\cr}$$

\noindent by Fubini's and Tonelli's theorems (252H), using the fact that
$(e^{icy}-e^{idy})/y$ is bounded to see that

\Centerline{$\int_{-\infty}^{\infty}\int_{-a}^{a}
  \bigl|\bover{e^{icy}-e^{idy}}{y}e^{-iyx}f(x)\bigr|dydx$}

\noindent is finite.   Accordingly

$$\eqalign{\bover{i}{\sqrt{2\pi}}\lim_{a\to\infty}\int_{-a}^a
\bover{e^{icy}-e^{idy}}{y}\varhatf(y)dy
&=-\bover{i}{2\pi}\lim_{a\to\infty}
\int_{-\infty}^{\infty}f\times\theta_a\cr
&=-\bover{i}{2\pi}2\pi i\int_c^df
=\int_c^df,\cr}$$

\noindent as required.
}%end of proof of 283F

\leader{283G}{Corollary} If $f$ and $g$ are
complex-valued functions which are integrable over $\Bbb R$, then
$\varhatf=\varhat{g}$ iff $f\eae g$.

\proof{ If $f\eae g$ then of course

\Centerline{$\varhatf(y)
=\Bover1{\sqrt{2\pi}}\int_{-\infty}^{\infty}e^{-iyx}f(x)dx
=\Bover1{\sqrt{2\pi}}\int_{-\infty}^{\infty}e^{-iyx}g(x)dx
=\varhat{g}(y)$}

\noindent for every $y\in\Bbb R$.   Conversely, if
$\varhatf=\varhat{g}$, then by the last theorem

\Centerline{$\int_c^df=\int_c^dg$}

\noindent for all $c\le d$, so $f=g$ almost everywhere, by 222D.
}%end of proof of 283G

\leader{283H}{Lemma} Let $f$ be a complex-valued function which is
integrable over $\Bbb R$, and $\varhatf$ its Fourier transform.   Then

\Centerline{$\Bover1{\sqrt{2\pi}}\int_{-a}^{a}e^{ixy}\varhatf(y)dy
=\Bover1{\pi}\int_{-\infty}^{\infty}\Bover{\sin a(x-t)}{x-t}f(t)dt
=\Bover1{\pi}\int_{-\infty}^{\infty}\Bover{\sin at}{t}f(x-t)dt$}

\noindent whenever $a>0$ and $x\in\Bbb R$.

\proof{ We have

\Centerline{$\int_{-a}^a\int_{-\infty}^{\infty}|e^{ixy}e^{-iyt}f(t)|dtdy
\le 2a\int_{-\infty}^{\infty}|f(t)|dt<\infty$,}

\noindent so (because the function $(t,y)\mapsto e^{ixy}e^{-iyt}f(t)$ is
surely jointly measurable) we may reverse the order of integration, and
get

$$\eqalign{\Bover1{\sqrt{2\pi}}\int_{-a}^ae^{ixy}\varhatf(y)dy
&=\Bover1{2\pi}\int_{-a}^a\int_{-\infty}^{\infty}
   e^{ixy}e^{-iyt}f(t)dt\,dy\cr
&=\Bover1{2\pi}\int_{-\infty}^{\infty}f(t)\int_{-a}^a
   e^{i(x-t)y}dy\,dt\cr
&=\Bover1{2\pi}\int_{-\infty}^{\infty}
   \Bover{2\sin(x-t)a}{x-t}f(t)dt
=\Bover1{\pi}\int_{-\infty}^{\infty}
   \Bover{\sin au}{u}f(x-u)du,\cr}$$

\noindent substituting $t=x-u$.
}%end of proof of 283H


\leader{283I}{Theorem} Let $f$ be a complex-valued function which is
integrable over $\Bbb R$, and suppose that $f$ is differentiable at
$x\in\Bbb R$.   Then

\Centerline{$f(x)
=\Bover{1}{\sqrt{2\pi}}
\lim_{a\to\infty}\int_{-a}^ae^{ixy}\varhatf(y)dy
=\Bover{1}{\sqrt{2\pi}}
\lim_{a\to\infty}\int_{-a}^ae^{-ixy}\varcheckf(y)dy$.}

\proof{ Set $g(u)=f(x)$ if $|u|\le 1$, $0$ otherwise, and observe that
$\lim_{u\to 0}\bover1u(f(x-u)-g(u))=-f'(x)$ is finite, so that there is
a $\delta\in\ocint{0,1}$ such that

\Centerline{$K
=\sup_{0<|u|\le\delta}\bigl|\Bover{f(x-u)-g(u)}{u}\bigr|<\infty$.}

\noindent Consequently

$$\eqalign{\int_{-\infty}^{\infty}\bigl|\Bover{f(x-u)-g(u)}{u}\bigr|du
&\le\Bover1{\delta}\int_{-\infty}^{-\delta}|f(x-u)|du
   +\Bover1{\delta}\int_{-1}^1|g|\cr
&{\hskip 8em plus 0pt minus 0pt}
   +\int_{-\delta}^{\delta}K
   +\Bover1{\delta}\int_{\delta}^{\infty}|f(x-u)|du\cr
&\le\Bover{1}{\delta}\int_{-\infty}^{\infty}|f|
   + \Bover2{\delta}|f(x)|+ 2\delta K
<\infty.\cr}$$

\noindent By the Riemann-Lebesgue lemma (282Fb again),

\Centerline{$\lim_{a\to\infty}\int_{-\infty}^{\infty}
\Bover{\sin au}{u}(f(x-u)-g(u))du=0$.}

\noindent If we now examine $\int\bover{\sin au}{u}g(u)du$, we get

$$\eqalign{\int_{-\infty}^{\infty}\Bover{\sin au}{u}g(u)du
&=\int_{-1}^1\Bover{\sin au}{u}f(x)du
=f(x)\int_{-a}^a\Bover{\sin v}{v}dv,\cr}$$

\noindent substituting $u=v/a$.   So we get

$$\eqalign{\lim_{a\to\infty}\int_{-\infty}^{\infty}\Bover{\sin
au}{u}f(x-u)du
&=\lim_{a\to\infty}\int_{-\infty}^{\infty}\Bover{\sin au}{u}g(u)du\cr
&=\lim_{a\to\infty}f(x)\int_{-a}^a\Bover{\sin v}{v}dv
=\pi f(x),\cr}$$

\noindent by 283Da.   Accordingly

$$\Bover{1}{\sqrt{2\pi}}
\lim_{a\to\infty}\int_{-a}^ae^{ixy}\varhatf(y)dy
=\Bover{1}{\pi}\lim_{a\to\infty}\int_{-\infty}^{\infty}\Bover{\sin
au}{u}f(x-u)du
= f(x),$$

\noindent using 283H.    As for the second equality,

$$\eqalignno{\Bover1{\sqrt{2\pi}}\lim_{a\to\infty}\int_{-a}^a
e^{-ixy}\varcheckf(y)dy
&=\Bover1{\sqrt{2\pi}}\lim_{a\to\infty}\int_{-a}^a
e^{-ixy}\varhatf(-y)dy\cr
&=\Bover1{\sqrt{2\pi}}\lim_{a\to\infty}\int_{-a}^a
e^{ixu}\varhatf(u)du
=f(x)\cr}$$

\noindent (substituting $y=-u$).
}%end of proof of 283I

\cmmnt{\medskip

\noindent{\bf Remark} Compare 282L.}%end of comment

\leader{283J}{Corollary} Let $f:\Bbb R\to\Bbb C$ be an integrable
function such that $f$ is differentiable and $\varhatf$ is integrable.
Then $f=(\varhatf)\varspcheck=(\varcheckf)\varsphat$.

\proof{ Because $\varhatf$ is integrable,

\Centerline{$\varhatf\varcheck{\phantom{f}}(x)=\lim_{a\to\infty}
\Bover1{\sqrt{2\pi}}\int_{-a}^ae^{ixy}\varhatf(y)dy = f(x)$}

\noindent for every $x\in\Bbb R$.   Similarly,

\Centerline{$\varcheckf\varhat{\phantom{f}}(x)=\lim_{a\to\infty}
\Bover1{\sqrt{2\pi}}\int_{-a}^ae^{-ixy}\varcheckf(y)dy = f(x).$}
}%end of proof of 283J

\cmmnt{\medskip

\noindent{\bf Remark} See also 283Wk below.
}

\leader{283K}{}\cmmnt{ The next proposition gives a class of functions
to which the last corollary can be applied.

\medskip

\noindent}{\bf Proposition} Suppose that $f$ is a twice-differentiable
function from $\Bbb R$ to $\Bbb C$ such that $f$, $f'$ and $f''$ are all
integrable.   Then $\varhatf$ is integrable.

\proof{ Because $f'$ and $f''$ are integrable, $f$ and $f'$
are absolutely continuous on any bounded interval (225L).   So by 283Ci
we have

\Centerline{$(f'')\varsphat(y)=iy(f')\varsphat(y)=-y^2\varhatf(y)$}

\noindent for every $y\in\Bbb R$.   At the same time, by 283Cf-283Cg,
$(f'')\varsphat$ and $\varhatf$ must be bounded;  say
$|\varhatf(y)|+|(f'')\varsphat(y)|\le K$ for every $y\in\Bbb R$.   Now

\Centerline{$|\varhatf(y)|\le\Bover{K}{1+y^2}$}

\noindent for every $y$, so that

\Centerline{$\int_{-\infty}^{\infty}|\varhatf|
\le K\int_{-\infty}^{-1}\Bover{1}{y^2}dy
  +2K+K\int_{1}^{\infty}\Bover{1}{y^2}dy
=4K<\infty$.}
}%end of proof of 283K

\cmmnt{\medskip

\noindent{\bf Remark} Compare 282Rb.}

\leader{283L}{}\cmmnt{ I turn now to the result corresponding to 282O,
using a slightly different approach.

\medskip

\noindent}{\bf Theorem} Let $f$ be a complex-valued function which is
integrable over $\Bbb R$, with Fourier transform $\varhatf$ and inverse
Fourier transform $\varcheckf$, and suppose that $f$ is of bounded
variation on some neighbourhood of $x\in\Bbb R$.   Set
$a=\lim_{t\in\dom f,t\uparrow x}f(t)$,
$b=\lim_{t\in\dom f,t\downarrow x}f(t)$.   Then

\Centerline{$\Bover1{\sqrt{2\pi}}\lim_{\gamma\to\infty}
     \int_{-\gamma}^{\gamma}e^{ixy}\varhatf(y)dy
=\Bover1{\sqrt{2\pi}}\lim_{\gamma\to\infty}
     \int_{-\gamma}^{\gamma}e^{-ixy}\varcheckf(y)dy
=\Bover12(a+b)$.}

\proof{{\bf (a)} The limits $\lim_{t\in\dom f,t\uparrow x}f(t)$ and
$\lim_{t\in\dom f,t\downarrow x}f(t)$ exist because $f$ is of bounded
variation near $x$ (224F).   Recall from 283Db that there is a constant
$K<\infty$ such that

\Centerline{$|\int_{\gamma}^{\delta}\Bover{\sin cx}{x}dx|\le K$}

\noindent whenever $\gamma\le \delta$ and $c\in\Bbb R$.

\medskip

{\bf (b)} Let $\epsilon>0$.   The hypothesis is that there is some
$\delta>0$ such that $\Var_{[x-\delta,x+\delta]}(f)<\infty$.
Consequently

\Centerline{$\lim_{\eta\downarrow 0}\Var_{\ocint{x,x+\eta}}(f)
=\lim_{\eta\downarrow 0}\Var_{\coint{x-\eta,x}}(f)=0$}

\noindent (224E).   There is therefore an $\eta>0$ such that

\Centerline{$\max(\Var_{\coint{x-\eta,x}}(f),
               \Var_{\ocint{x,x+\eta}}(f))\le\epsilon$.}

\noindent Of course

\Centerline{$|f(t)-f(u)|\le\Var_{\coint{x-\eta,x}}(f)\le\epsilon$}

\noindent whenever $t$, $u\in\dom f$ and $x-\eta\le t\le u<x$, so we
shall have

\Centerline{$|f(t)-a|\le\epsilon$ for every
$t\in\dom f\cap\coint{x-\eta,x}$,}

\noindent and similarly

\Centerline{$|f(t)-b|\le\epsilon$ whenever $t\in\dom f
\cap\ocint{x,x+\eta}$.}

\medskip

{\bf (c)} Now set

\Centerline{$g_1(t)=f(t)$ when $t\in\dom f$ and $|x-t|>\eta$, $0$
otherwise,}

\Centerline{$g_2(t)=a$ when $x-\eta\le t<x$, $b$ when $x<t\le x+\eta$,
$0$ otherwise,}

\Centerline{$g_3=f-g_1-g_2$.}

\noindent Then $f=g_1+g_2+g_3$;  each $g_j$ is integrable;  $g_1$ is
zero on a neighbourhood of $x$;

\Centerline{$\sup_{t\in\dom g_3,t\ne x}|g_3(t)|\le\epsilon$,}

\Centerline{$\Var_{\coint{x-\eta,x}}(g_3)\le\epsilon$,
\quad $\Var_{\ocint{x,x+\eta}}(g_3)\le\epsilon$.}

\medskip

{\bf (d)} Consider the three parts $g_1$, $g_2$, $g_3$ separately.

\medskip

\quad{\bf (i)} For the first, we have

\Centerline{$\lim_{\gamma\to\infty}\Bover1{\sqrt{2\pi}}
   \int_{-\gamma}^{\gamma}e^{ixy}\varhat{g}_1(y)dy=0$}

\noindent by 283I.

\medskip

\quad{\bf (ii)} Next,

$$\eqalignno{\Bover{1}{\sqrt{2\pi}}
        \int_{-\gamma}^{\gamma}e^{ixy}\varhat{g}_2(y)dy
&=\Bover1{\pi}\int_{-\infty}^{\infty}
        \Bover{\sin(x-t)\gamma}{x-t}g_2(t)dt\cr
\noalign{\noindent (by 283H)}
&=\Bover{a}{\pi}\int_{x-\eta}^{x}\Bover{\sin(x-t)\gamma}{x-t}dt
   +\Bover{b}{\pi}\int_{x}^{x+\eta}\Bover{\sin(x-t)\gamma}{x-t}dt\cr
&=\Bover{a}{\pi}\int_0^{\gamma\eta}\Bover{\sin u}{u}du
   +\Bover{b}{\pi}\int_{0}^{\gamma\eta}\Bover{\sin u}{u}du\cr
\noalign{\noindent (substituting $t=x-\bover{1}{\gamma}u$ in the first
integral, $t=-x+\bover1{\gamma}u$ in the second)}
&\to \Bover{a+b}{2}\text{ as }\gamma\to\infty\cr}$$

\noindent by 283Da.

\medskip

\quad{\bf (iii)} As for the third, we have, for any $\gamma>0$,

$$\eqalignno{\bigl|\Bover1{\sqrt{2\pi}}\int_{-\gamma}^{\gamma}e^{ixy}
     \varhat{g}_3(y)dy\bigr|
&=\Bover1{\pi}\bigl|\int_{-\infty}^{\infty}
     \Bover{\sin(x-t)\gamma}{x-t}g_3(t)dt\bigr|
=\Bover1{\pi}\bigl|\int_{-\infty}^{\infty}
     \Bover{\sin t\gamma}{t}g_3(x-t)dt\bigr|\cr
&\le\Bover1{\pi}\bigl|\int_{-\eta}^0
     \Bover{\sin t\gamma}{t}g_3(x-t)dt\bigr|
   +\Bover1{\pi}\bigl|\int_0^{\eta}
     \Bover{\sin t\gamma}{t}g_3(x-t)dt\bigr|\cr
&\le \Bover{K}{\pi}\Bigl(\sup_{t\in\dom g_3\cap\ooint{x-\eta,x}}|g_3(t)|
   +\Var_{\ooint{x-\eta,x}}(g_3)\cr
  &{\hskip 8em plus 0pt minus 0pt}
   +\sup_{t\in\dom g_3\cap\ooint{x,x+\eta}}|g_3(t)|
   +\Var_{\ooint{x,x+\eta}}(g_3)\Bigr)\cr
&\le 4\epsilon\Bover{K}{\pi},\cr}$$

\noindent using 224J again
to bound the integrals in terms of the variation
and supremum of $g_3$ and integrals of $\bover{\sin\gamma t}{t}$ over
subintervals.

\medskip

{\bf (e)} We therefore have

$$\eqalign{\limsup_{\gamma\to\infty}\bigl|\Bover1{\sqrt{2\pi}}
        \int_{-\gamma}^{\gamma}
   &e^{ixy}\varhatf(y)dy-\Bover{a+b}{2}\bigr|\cr
&\le\limsup_{\gamma\to\infty}\Bover1{\sqrt{2\pi}}
     \bigl|\int_{-\gamma}^{\gamma}e^{ixy}\varhat{g}_1(y)dy\bigr|\cr
&{\hskip 6em plus 0pt minus 0pt}
     +\limsup_{\gamma\to\infty}\bigl|\Bover1{\sqrt{2\pi}}
     \int_{-\gamma}^{\gamma}e^{ixy}\varhat{g}_2(y)dy
     -\Bover{a+b}{2}\bigr|\cr
&{\hskip 6em plus 0pt minus 0pt}
     +\limsup_{\gamma\to\infty}\Bover1{\sqrt{2\pi}}
     \bigl|\int_{-\gamma}^{\gamma}e^{ixy}\varhat{g}_3ydy\bigr|\cr
&\le 0+0+\Bover{4K}{\pi}\epsilon\cr}$$

\noindent by the calculations in (d).   As $\epsilon$ is arbitrary,

\Centerline{$\lim_{\gamma\to\infty}\Bover1{\sqrt{2\pi}}
     \int_{-\gamma}^{\gamma}e^{ixy}\varhatf(y)dy-\Bover{a+b}{2}=0$.}

\medskip

{\bf (f)} This is the first half of the theorem.   But of course the
second half follows at once, because

$$\eqalign{\Bover1{\sqrt{2\pi}}\lim_{\gamma\to\infty}
    \int_{-\gamma}^{\gamma}e^{-ixy}\varcheckf(y)dy
&=\Bover1{\sqrt{2\pi}}\lim_{\gamma\to\infty}
    \int_{-\gamma}^{\gamma}e^{-ixy}\varhatf(-y)dy\cr
&=\Bover1{\sqrt{2\pi}}\lim_{\gamma\to\infty}
    \int_{-\gamma}^{\gamma}e^{ixy}\varhatf(y)dy
=\Bover{a+b}2.\cr}$$
}%end of proof of 283L

\cmmnt{\medskip

\noindent{\bf Remark} You will see that this argument uses some of the
same ideas as those in 282O-282P.   It is more direct because (i) I am
not using any concept corresponding to Fej\'er sums (though a very
suitable one is available;  see 283Xf)
(ii) I do not trouble to give the result concerning uniform convergence
of the Fej\'er integrals when $f$ is continuous and of bounded variation
(283Xj)
(iii) I do not give any pointer to the significance of the fact that if
$f$ is of bounded variation then $\sup_{y\in\Bbb
R}|y\varhatf(y)|<\infty$ (283Xk).
}%end of comment

\leader{283M}{}\cmmnt{ Corresponding to 282Q, we have the following.

\medskip

\noindent}{\bf Theorem} Let $f$ and $g$ be complex-valued functions
which are integrable over $\Bbb R$, and $f*g$ their convolution product,
defined by setting

\Centerline{$(f*g)(x)=\int_{-\infty}^{\infty}f(t)g(x-t)dt$}

\noindent whenever this is defined\cmmnt{ (255E)}.   Then

\Centerline{$(f*g)\varsphat(y)
=\sqrt{2\pi}\varhatf(y)\varhat{g}(y)$,\quad
$(f*g)\varspcheck(y)
=\sqrt{2\pi}\varcheckf(y)\varcheck{g}(y)$}

\noindent for every $y\in\Bbb R$.

\proof{ For any $y$,

$$\eqalignno{(f*g)\varsphat(y)
&=\Bover1{\sqrt{2\pi}}\int_{-\infty}^{\infty}e^{-iyx}(f*g)(x)dx\cr
&=\Bover1{\sqrt{2\pi}}\int_{-\infty}^{\infty}\int_{-\infty}^{\infty}
e^{-iy(t+u)}f(t)g(u)dtdu\cr
\noalign{\noindent (using 255G)}
&=\Bover1{\sqrt{2\pi}}\int_{-\infty}^{\infty}e^{-iyt}f(t)dt
\int_{-\infty}^{\infty}e^{-iyu}g(u)du
=\sqrt{2\pi}\varhatf(y)\varhat{g}(y).\cr}$$

\noindent Now, of course,

\Centerline{$(f*g)\varspcheck(y)
=(f*g)\varsphat(-y)
=\sqrt{2\pi}\varhatf(-y)\varhat{g}(-y)
=\sqrt{2\pi}\varcheckf(y)\varcheck{g}(y)$.}
}%end of proof of 283M

\leader{283N}{}\cmmnt{ I show how to compute a special Fourier
transform, which
will be used repeatedly in the next section.

\medskip

\noindent}{\bf Lemma} For $\sigma>0$, set
$\psi_{\sigma}(x)=\bover1{\sigma\sqrt{2\pi}}e^{-x^2/2\sigma^2}$ for
$x\in\Bbb R$.   Then its Fourier transform and inverse Fourier transform
are

\Centerline{$\varhat{\psi}_{\sigma}=\varcheck{\psi}_{\sigma}
=\Bover1{\sigma}\psi_{1/\sigma}$.}

\noindent In particular, $\varhat{\psi}_1=\psi_1$.

\proof{{\bf (a)} I begin with the special case $\sigma=1$, using the
Maclaurin series

\Centerline{$e^{-iyx}=\sum_{k=0}^{\infty}\Bover{(-iyx)^k}{k!}$}

\noindent and the expressions for
$\int_{-\infty}^{\infty}x^ke^{-x^2/2}dx$ from \S263.

Fix $y\in\Bbb R$.    Writing

\Centerline{$g_k(x)=\Bover{(-iyx)^k}{k!}e^{-x^2/2}$,\quad
$h_n(x)=\sum_{k=0}^ng_k(x)$, \quad$h(x)=e^{|yx|-x^2/2}$,}

\noindent we see that

\Centerline{$|g_k(x)|\le \Bover{|yx|^k}{k!}e^{-x^2/2}$,}

\noindent so that

\Centerline{$|h_n(x)|\le\sum_{k=0}^{\infty}|g_k(x)|
\le e^{|yx|}e^{-x^2/2}=h(x)$}

\noindent for every $n$;  moreover, $h$ is integrable, because
$|h(x)|\le e^{-|x|}$ whenever $|x|\ge 2(1+|y|)$.   Consequently, using
Lebesgue's Dominated Convergence Theorem,

$$\eqalignno{\varhat{\psi}_1(y)
&=\Bover1{{2\pi}}\int_{-\infty}^{\infty}
  \lim_{n\to\infty}h_n\
=\Bover1{{2\pi}}\lim_{n\to\infty}\int_{-\infty}^{\infty}
  h_n\cr
&=\Bover1{{2\pi}}\sum_{k=0}^{\infty}
  \int_{-\infty}^{\infty}g_k
=\Bover1{{2\pi}}\sum_{k=0}^{\infty}\Bover{(-iy)^k}{k!}
  \int_{-\infty}^{\infty}x^ke^{-x^2/2}dx\cr
&=\Bover1{{2\pi}}\sum_{j=0}^{\infty}\Bover{(-iy)^{2j}}{(2j)!}
  \Bover{(2j)!}{2^jj!}\sqrt{2\pi}\cr
\noalign{\noindent (by 263H)}
&=\Bover1{\sqrt{2\pi}}\sum_{j=0}^{\infty}\Bover{(-y^2)^j}{2^jj!}
=\Bover1{\sqrt{2\pi}}e^{-y^2/2}
=\psi_1(y),\cr}$$

\noindent as claimed.

\medskip

{\bf (b)} For the general case,
$\psi_{\sigma}(x)=\Bover1{\sigma}\psi_1(\Bover{x}{\sigma})$, so that

\Centerline{$\varhat{\psi}_{\sigma}(y)
=\Bover1{\sigma}\cdot\sigma\varhat{\psi}_1(\sigma y)
=\Bover1\sigma\psi_{1/\sigma}(y)$}

\noindent by 283Ce.   Of course we now have

\Centerline{$\varcheck{\psi}_{\sigma}(y)=\varhat{\psi}_{\sigma}(-y)
=\Bover1{\sigma}\psi_{1/\sigma}(y)$}

\noindent because $\psi_{1/\sigma}$ is an even function.
}%end of proof of 283N


\leader{283O}{}\cmmnt{ To lead into the ideas of the next section, I
give the following very simple fact.

\medskip

\noindent}{\bf Proposition} Let $f$ and $g$ be two
complex-valued functions which are integrable over $\Bbb R$.   Then
$\int_{-\infty}^{\infty}f\times\varhat{g}
=\int_{-\infty}^{\infty}\varhatf\times g$ and
$\int_{-\infty}^{\infty}f\times\varcheck{g}
  =\int_{-\infty}^{\infty}\varcheckf\times g$.

\proof{  Of course

\Centerline{$\int_{-\infty}^{\infty}\int_{-\infty}^{\infty}
|e^{-ixy}f(x)g(y)|dxdy
=\int_{-\infty}^{\infty}|f|\int_{-\infty}^{\infty}|g|<\infty$,}

\noindent so

$$\eqalign{\int_{\infty}^{\infty}f\times\varhat{g}
&=\Bover1{\sqrt{2\pi}}\int_{-\infty}^{\infty}\int_{-\infty}^{\infty}
f(y)e^{-iyx}g(x)dxdy\cr
&=\Bover1{\sqrt{2\pi}}\int_{-\infty}^{\infty}\int_{-\infty}^{\infty}
f(y)e^{-ixy}g(x)dydx
=\int_{-\infty}^{\infty}\varhatf\times g.\cr}$$

\noindent For the other half of the proposition, replace every
$e^{-ixy}$ in the argument by $e^{ixy}$.
}%end of proof of 283O

\exercises{\leader{283W}{Higher dimensions} I offer a series of
exercises designed to provide hints on how the work of this section may
be done in the
$r$-dimensional case, where $r\ge 1$.

\spheader 283Wa Let $f$ be an integrable complex-valued function
defined almost everywhere in $\BbbR^r$.   Its {\bf Fourier transform}
is the function $\varhatf:\BbbR^r\to\Bbb C$ defined by the formula

\Centerline{$\varhatf(y)
=\Bover1{(\sqrt{2\pi})^r}\int e^{-iy\dotproduct x}f(x)dx$,}

\noindent writing $y\dotproduct x=\eta_1\xi_1+\ldots+\eta_r\xi_r$ for
$x=(\xi_1,\ldots,\xi_r)$ and $y=(\eta_1,\ldots,\eta_r)\in\BbbR^r$, and
$\int\ldots dx$ for integration with respect to Lebesgue measure on
$\BbbR^r$.   Similarly, the {\bf inverse Fourier transform} of $f$ is
the function $\varcheckf$ given by

\Centerline{$\varcheckf(y)
=\Bover1{(\sqrt{2\pi})^r}\int e^{iy\dotproduct x}f(x)dx=\varhatf(-y)$.}

\noindent Show that, for any integrable complex-valued function $f$ on
$\BbbR^r$,

\quad (i) $\varhatf:\BbbR^r\to\Bbb C$ is continuous;

\quad (ii) $\lim_{\|y\|\to\infty}\varhatf(y)=0$, writing
$\|y\|=\sqrt{y\dotproduct y}$ as usual;

\quad (iii) if $\int\|x\||f(x)|dx<\infty$, then $\varhatf$ is
differentiable, and

\Centerline{$\Pd{}{\eta_j}\varhatf(y)
=-\Bover{i}{(\sqrt{2\pi})^r}\int e^{-iy\dotproduct x}\xi_jf(x)dx$}

\noindent for $j\le r$, $y\in\BbbR^r$, always taking $\xi_j$ to be the
$j$th coordinate of $x\in\BbbR^r$;

\quad (iv) if $j\le r$ and $\pd{f}{\xi_j}$ is defined everywhere and is
integrable, then
$(\pd{f}{\xi_j})\varsphat(y)=i\eta_j\varhatf(y)$
for every $y\in\Bbb R^r$.   \prooflet{(Use 225L to show that if
$e\in\BbbR^r$ is a unit vector, then
$\gamma\mapsto f(x+\gamma e)$ is absolutely continuous on every bounded
interval for almost every $x$.)}
%283C

\spheader 283Wb Show that if $f_1,\ldots,f_r$ are integrable
complex-valued functions on $\Bbb R$ with Fourier transforms
$g_1,\ldots,g_r$, and we write $f(x)=f_1(\xi_1)\ldots f_r(\xi_r)$ for
$x=(\xi_1,\ldots,\xi_r)\in\BbbR^r$, then the Fourier transform of $f$
is $y\mapsto g_1(\eta_1)\ldots g_r(\eta_r)$.

\spheader 283Wc Let $f$ be an integrable complex-valued function
on $\BbbR^r$, and $\varhatf$ its Fourier transform.   If $c\le d$ in
$\BbbR^r$, show that

$$\int_{[c,d]}f
=(\bover{i}{\sqrt{2\pi}})^r
\lim_{\alpha_1,\ldots,\alpha_r\to\infty}
\int_{[-a,a]}\prod_{j=1}^r
\bover{e^{i\gamma_j\eta_j}-e^{i\delta_j\eta_j}}{\eta_j}\varhatf(y)dy,$$

\noindent setting $a=(\alpha_1,\ldots)$, $c=(\gamma_1,\ldots)$,
$d=(\delta_1,\ldots)$.
%283F

\spheader 283Wd
Let $f$ be an integrable complex-valued function on $\BbbR^r$, and
$\varhatf$ its Fourier transform.   Show that if we write

\Centerline{$B_{\infty}(\tbf{0},a)=\{y:|\eta_j|\le a
  \text{ for every }j\le r\}$,}

\noindent then

\Centerline{$\Bover1{(\sqrt{2\pi})^r}\int_{B_{\infty}(\tbf{0},a)}
e^{ix\dotproduct y}\varhatf(y)dy
=\int \phi_a(t)f(x-t)dt$}

\noindent for every $a\ge 0$, where

\Centerline{$\phi_a(t)
=\Bover1{\pi^r}\prod_{j=1}^r\Bover{\sin a\tau_j}{\tau_j}$}

\noindent for $t=(\tau_1,\ldots,\tau_r)\in\BbbR^r$.
%283H

\spheader 283We Show that
$\biggerint_{\BbbR^r}\Bover{1}{1+\|x\|^{r+1}}dx<\infty$.

\spheader 283Wf Let $f:\BbbR^r\to\Bbb C$ be an integrable function.
Show that if all the partial derivatives
$\Bover{\partial^kf}{\partial\xi_j^k}$, for $k\le r+1$ and
$j\le r$, are defined almost everywhere and integrable,
then $\varhat{f}$ is integrable.
%283Wa(iii) 283We 283K

\spheader 283Wg\dvAformerly{2{}83Wj}
Show that if $f$ and $g$ are integrable
complex-valued functions on $\BbbR^r$, then (defining convolution as in
255L) $(f*g)\varsphat=(\sqrt{2\pi})^r\varhatf\times\varhat{g}$.
%283M

\spheader 283Wh Let $f$ and $g$ be integrable complex-valued
functions on $\BbbR^r$.   Show that
$f*\varcheck{g}=(\sqrt{2\pi})^r(\varhatf\times g)\varspcheck$.
%283M 283Wg

\spheader 283Wi\dvAformerly{2{}83We} For $\sigma>0$, define
$\psi_{\sigma}:\Bbb R^r\to\Bbb C$ by setting

\Centerline{$\psi_{\sigma}(x)=\Bover1{(\sigma\sqrt{2\pi})^r}
e^{-x\dotproduct x/2\sigma^2}$,
\quad$(\varhat{\psi}_{\sigma})\varspcheck=\psi_{\sigma}$.}

\noindent for every $x\in\BbbR^r$.   Show that

\Centerline{$\varhat{\psi}_{\sigma}
=\varcheck{\psi}_{\sigma}=\Bover1{\sigma^r}\psi_{1/\sigma}$.}
%283N

\spheader 283Wj Defining $\psi_{\sigma}$ as in (e), show that
$\lim_{\sigma\to 0}(f*\psi_{\sigma})(x)=f(x)$ whenever $x\in\BbbR^r$ and
$f:\BbbR^r\to\Bbb C$ is continuous and either integrable or bounded.
(Cf.\ 261Ye, 262Yi.)
%283Wi

\spheader 283Wk Show that if $f:\BbbR^r\to\Bbb C$ is continuous
and integrable, and $\varhatf$ also is integrable, then
$f=\varhatf\varcheck{\phantom{f}}$.
({\it Hint\/}:  Show that both are equal at every point to
$\lim_{\sigma\to 0}(\sqrt{2\pi})^r
  (\varhatf\times\varhat{\psi}_{\sigma})\varspcheck$.)
%283Wj 283Wh 283J

\spheader 283Wl Show that if $f$ and $g$ are integrable
complex-valued functions on $\BbbR^r$, then $\int
f\times\varhat{g}=\int\varhatf\times g$.

\spheader 283Wm(i) Show that
$\int_{2k\pi}^{2(k+1)\pi}\bover{\sin t}{t\sqrt{t}}dt>0$ for every
$k\in\Bbb N$, and hence that
$\int_0^{\infty}\bover{\sin t}{t\sqrt{t}}dt>0$.

\quad{(ii)} Set $f_1(\xi)=1/\sqrt{|\xi|}$ for $0<|\xi|\le 1$, $0$ for
other $\xi$.   Show that
$\lim_{a\to\infty}\bover1{\sqrt{a}}\int_{-a}^a\varhatf_1(\eta)d\eta$
exists in $\Bbb R$ and is greater than $0$.

\quad{(iii)} Construct an integrable function $f_2$, zero on some
neighbourhood of $0$, such that there are infinitely many $m\in\Bbb N$
for which $|\int_{-m}^m\varhatf_2(\eta)d\eta|\ge\bover1{\sqrt{m}}$.
\Hint{take $f_2(\xi)=2^{-k}\sin m_k\xi$ for $k+1\le\xi<k+2$, for a
sufficiently rapidly increasing sequence $\sequence{k}{m_k}$.}

\quad{(iv)} Set $f(x)=f_1(\xi_1)f_2(\xi_2)$ for $x\in\BbbR^2$.   Show
that $f$ is integrable, that $f$ is zero in a neighbourhood of
$\tbf{0}$, but that

\Centerline{$\limsup_{a\to\infty}\Bover1{2\pi}|\int_{B_{\infty}(\tbf{0},
a)}\varhatf(y)dy|>0$,}

\noindent defining $B_{\infty}$ as in 283Wd.
}%end of 283W

\exercises{
\leader{283X}{Basic exercises (a)}
%\spheader 283Xa
Confirm that the six alternative
definitions of the transforms $\varhatf$, $\varcheckf$ offered in 283B
all lead to the same theory;  find the constants involved in the new
versions of 283Ch, 283Ci, 283L, 283M and 283N.
%283B

\spheader 283Xb If we redefined $\varhatf(y)$ to be
$\alpha\int_{-\infty}^{\infty}e^{i\beta xy}f(x)dx$, what would
$\varcheckf(y)$ be?
%283B

\spheader 283Xc Show that nearly every $2\pi$ would disappear
from the theorems of this section if we defined a measure $\nu$ on
$\Bbb R$ by saying that $\nu E=\bover1{\sqrt{2\pi}}\mu E$ for every Lebesgue measurable set $E$, where $\mu$ is Lebesgue measure, and wrote

\Centerline{$\varhatf(y)=\int_{-\infty}^{\infty}e^{-iyx}f(x)\nu(dx)$,
\quad$\varcheckf(y)=\int_{-\infty}^{\infty}e^{iyx}f(x)\nu(dx)$,}

\Centerline{$(f*g)(x)=\int_{-\infty}^{\infty}f(t)g(x-t)\nu(dt)$.}

\noindent What is
$\lim_{a\to\infty}\int_{-a}^a\bover{\sin t}{t}\nu(dt)$?
%283B+

\sqheader 283Xd Let $f$ be an integrable complex-valued function
on $\Bbb R$, with Fourier transform $\varhatf$.   Show that
(i) if $g(x)=f(-x)$ whenever this
is defined, then $\varhat{g}(y)=\varhatf(-y)$ for
every $y\in\Bbb R$;
(ii) if $g(x)=\overline{f(x)}$ whenever this is defined, then
$\varhat{g}(y)=\overline{\varhatf(-y)}$ for every $y$.
%283C

\spheader 283Xe Let $f$ be an integrable complex-valued function
on $\Bbb R$, with Fourier transform $\varhatf$.   Show that

\Centerline{$\int_c^d\varhatf(y)dy=\Bover{i}{\sqrt{2\pi}}
\int_{-\infty}^{\infty}\Bover{e^{-idx}-e^{-icx}}{x}f(x)dx$}

\noindent whenever $c\le d$ in $\Bbb R$.
%283F

\sqheader 283Xf For an integrable complex-valued function $f$ on
$\Bbb R$, let its {\bf Fej\'er integrals} be

\Centerline{$\sigma_c(x)
=\Bover1{c\sqrt{2\pi}}\int_0^c
  \bigl(\int_{-a}^{a}e^{ixy}\varhatf(y)dy\bigr)da$}

\noindent for $c>0$.   Show that

\Centerline{$\sigma_c(x)=
\Bover1{\pi}\int_{-\infty}^{\infty}
  \Bover{1-\cos ct}{ct^2}f(x-t)dt$.}
%283F

\spheader 283Xg Show that
$\biggerint_{-\infty}^{\infty}\Bover{1-\cos at}{at^2}dt=\pi$ for every $a>0$.   \Hint{integrate by parts and use 283Da.}   Show that

\Centerline{$\lim_{a\to\infty}\int_{\delta}^{\infty}
         \Bover{1-\cos at}{at^2}dt
=\lim_{a\to\infty}\sup_{t\ge\delta}\Bover{1-\cos at}{at^2}
=0$}

\noindent for every $\delta>0$.
%283Xf, 283D, 283F

\wheader{283Xh}{0}{0}{0}{72pt}

\spheader 283Xh Let $f$ be an integrable complex-valued function
on $\Bbb R$, and define its Fej\'er integrals $\sigma_a$ as in 283Xf
above.   Show that if $x\in\Bbb R$, $c\in\Bbb C$ are such that

\Centerline{$\lim_{\delta\downarrow 0}\Bover1{\delta}\int_0^{\delta}
    |f(x+t)+f(x-t)-2c|dt=0$,}

\noindent then $\lim_{a\to\infty}\sigma_a(x)=c$.   \Hint{adapt
the argument of 282H.}
%283Xf, 283Xg

\sqheader 283Xi Let $f$ be an integrable complex-valued function
on $\Bbb R$, and define its Fej\'er integrals $\sigma_a$ as in 283Xf
above.   Show that $f(x)=\lim_{a\to\infty}\sigma_a(x)$ for almost every
$x\in\Bbb R$.
%283Xf, 283Xh

\spheader 283Xj Let $f:\Bbb R\to\Bbb C$ be a continuous integrable
complex-valued function of bounded variation, and define its Fej\'er
integrals $\sigma_a$ as in 283Xf above.   Show that
$f(x)=\lim_{a\to\infty}\sigma_a(x)$ uniformly for $x\in\Bbb R$.
%283Xf, 283Xi

\sqheader 283Xk Let $f$ be an integrable complex-valued function
of bounded variation on $\Bbb R$, and $\varhatf$ its Fourier transform.
Show that $\sup_{y\in\Bbb R}|y\varhatf(y)|<\infty$.
%283F

\spheader 283Xl Let $f$ and $g$ be integrable complex-valued
functions on $\Bbb R$.   Show that
$f*\varcheck{g}=\sqrt{2\pi}(\varhatf\times g)\varspcheck$.
%283M

\spheader 283Xm Let $f$ be an integrable complex-valued function
on $\Bbb R$, and fix $x\in\Bbb R$.    Set

\Centerline{$\hat f_x(y)=\int_{-\infty}^{\infty}f(t)\cos y(x-t)dt$}

\noindent for $y\in\Bbb R$.   Show that

\quad (i) if $f$ is differentiable at $x$,

\Centerline{$f(x)
=\Bover1{\pi}\lim_{a\to\infty}\int_{0}^a\hat f_x(y)dy$;}

\quad (ii) if there is a neighbourhood of $x$ in which $f$ has bounded
variation, then

\Centerline{$\Bover1{\pi}\lim_{a\to\infty}\int_{0}^a\hat f_x(y)dy
=\Bover12(\lim_{t\in\dom f,t\uparrow 0}f(t)
 +\lim_{t\in\dom f,t\downarrow 0}f(t))$;}

\quad (iii) if $f$ is twice differentiable and $f'$, $f''$ are
integrable then $\hat f_x$ is integrable and
$f(x)=\bover1{\pi}\int_0^{\infty}\hat f_x$.
(The formula

\Centerline{$f(x)=\Bover1{\pi}\int_{0}^{\infty}\bigl(
\int_{-\infty}^{\infty}f(t)\cos y(x-t)dt\bigr)dy$,}

\noindent valid for such functions $f$, is called {\bf Fourier's
integral formula}.)

\spheader 283Xn Show that if $f$ is a complex-valued function of
bounded variation, defined almost everywhere in $\Bbb R$, and converging
to $0$ (along its domain) at $\pm\infty$, then

\Centerline{$g(y)
=\Bover1{\sqrt{2\pi}}\lim_{a\to\infty}\int_{-a}^ae^{-iyx}f(x)dx$}

\noindent is defined in $\Bbb C$ for every $y\ne 0$, and that the limit
is uniform in any region bounded away from $0$.

\spheader 283Xo Let $f$ be an integrable complex-valued function
on $\Bbb R$.   Set

\Centerline{$\varhatf_c(y)=\Bover1{\sqrt{2\pi}}\int_{-\infty}^{\infty}
\cos yx\,f(x)dx$,\quad
$\varhatf_s(y)=\Bover1{\sqrt{2\pi}}\int_{-\infty}^{\infty}
\sin yx\,f(x)dx$}

\noindent for $y\in\Bbb R$.   Show that

\Centerline{$\Bover1{\sqrt{2\pi}}\int_{-a}^ae^{ixy}\varhatf(y)dy
=\sqrt{\Bover2{\pi}}\int_0^a\cos xy\,\varhatf_c(y)dy
+\sqrt{\Bover2{\pi}}\int_0^a\sin xy\,\varhatf_s(y)dy$}

\noindent for every $x\in\Bbb R$ and $a\ge 0$.

\spheader 283Xp Use the fact that
$\int_{0}^a\int_{0}^{\infty}e^{-xy}\sin
y\,dxdy=\int_{0}^{\infty}\int_{0}^ae^{-xy}\sin y\,dydx$ whenever
$a\ge 0$ to show that 
$\int_0^{\infty}\bover1{1+x^2}dx=\lim_{a\to\infty}\int_0^a\bover{\sin y}{y}dy$.

\sqheader 283Xq Show that if $f(x)=e^{-\sigma|x|}$, where
$\sigma>0$, then
$\varhatf(y)=\bover{2\sigma}{\sqrt{2\pi}(\sigma^2+y^2)}$.
Hence, or otherwise, find the Fourier transform of
$y\mapsto\Bover{1}{1+y^2}$.

\spheader 283Xr Find the inverse Fourier transform of the
characteristic function of a bounded interval in $\Bbb R$.   Show that
in a formal sense 283F can be regarded as a special case of 283O.

\spheader 283Xs Let $f$ be a non-negative integrable function on
$\Bbb R$, with Fourier transform $\varhatf$.   Show that
\discrcenter{468pt}{$\sum_{j=0}^n\sum_{k=0}^n
  a_j\bar a_k\varhatf(y_j-y_k)\ge 0$ }whenever $y_0,\ldots,y_n$ in
$\Bbb R$ and $a_0,\ldots,a_n\in\Bbb C$.

\spheader 283Xt Let $f$ be an integrable complex-valued function on
$\Bbb R$.   Show that $\tilde f(x)=\sum_{n=-\infty}^{\infty}f(x+2\pi n)$
is defined in $\Bbb C$ for almost every $x$.
\Hint{$\sum_{n=-\infty}^{\infty}\int_{-\pi}^{\pi}|f(x+2\pi n)|dx
<\infty$.}   Show that $\tilde f$ is periodic.   Show that the
Fourier coefficients of $\tilde f\restr\ocint{-\pi,\pi}$ are
$\langle\bover{1}{\sqrt{2\pi}}\varhatf(k)\rangle_{k\in\Bbb Z}$.

\leader{283Y}{Further exercises (a)}
%\spheader 283Ya
Show that if $f:\Bbb R\to\Bbb C$ is
absolutely continuous in every bounded interval, $f'$ is of bounded
variation on $\Bbb R$, and
$\lim_{x\to\infty}f(x)=\lim_{x\to-\infty}f(x)=0$, then

\Centerline{$g(y)=\Bover1{\sqrt{2\pi}}\lim_{a\to\infty}
\int_{-a}^ae^{-iyx}f(x)dx
=-\Bover{i}{y\sqrt{2\pi}}\lim_{a\to\infty}
\int_{-a}^ae^{-iyx}f'(x)dx$}

\noindent is defined, with

\Centerline{$y^2|g(y)|\le\Bover4{\sqrt{2\pi}}\Var_{\Bbb R}(f')$,}

\noindent for every $y\ne 0$.
%283C

\spheader 283Yb Let $f:\Bbb R\to\Bbb C$ be an integrable
function which is absolutely continuous on every bounded interval, and
suppose that its derivative $f'$ is of bounded variation on $\Bbb R$.
Show that $\varhatf$ is integrable and that
$f=\varhatf\varcheck{\phantom{f}}$.   \Hint{225Yd, 283Ci, 283Xk.}
%283Xk 283K

\spheader 283Yc Let $f:\Bbb R\to\coint{0,\infty}$ be an even
function such that $f$ is convex on $\coint{0,\infty}$
and $\lim_{x\to\infty}f(x)=0$.

\quad{(i)} Show that, for any $y>0$ and $k\in\Bbb N$,
$\int_{-2k\pi/y}^{2k\pi/y}e^{-iyx}f(x)dx\ge 0$.

\quad{(ii)} Show that
$g(y)=\bover1{\sqrt{2\pi}}\lim_{a\to\infty}\int_{-a}^ae^{-iyx}f(x)dx$
exists in $\coint{0,\infty}$ for every $y\ne 0$.

\quad{(iii)} For $n\in\Bbb N$, set $f_n(x)
=e^{-|x|/(n+1)}f(x)$ for every $x$.   Show that $f_n$ is integrable and
convex on $\coint{0,\infty}$.

\quad{(iv)} Show that $g(y)=\lim_{n\to\infty}\varhatf_{n}(y)$ for
every $y\ne 0$.

\quad{(vi)} Show that if $f$ is integrable then

\Centerline{$\int_{-a}^a\varhatf
=\Bover4{\sqrt{2\pi}}\int_{0}^{\infty}\Bover{\sin at}{t}f(t)dt
\le\Bover{4a}{\sqrt{2\pi}}\int_0^{\pi/a}f
\le 2\sqrt{2\pi}f(0)
$}

\noindent for every $a\ge 0$.   Hence show that whether $f$ is
integrable or not, $g$ is integrable and $f_n=(\varhatf_n)\varspcheck$
for every $n$.

\quad{(vii)} Show that
$\lim_{a\downarrow 0}\sup_{n\in\Bbb N}\int_{-a}^a\varhatf_n=0$.

\quad(viii) Show that if $f'$ is bounded (on its domain) then
$\{\varhatf_n:n\in\Bbb N\}$ is uniformly integrable ({\it hint\/}: use
(vii) and 283Ya), so that $\lim_{n\to\infty}\|\varhat f_n-g\|_1=0$ and
$f=\varcheck{g}$.

\quad(ix) Show that if $f'$ is unbounded then for every $\epsilon>0$ we
can find $h_1$, $h_2:\Bbb R\to\coint{0,\infty}$, both even, convex and
converging to $0$ at $\infty$, such that $f=h_1+h_2$, $h'_1$ is bounded,
$\int h_2\le\epsilon$ and $h_2(0)\le\epsilon$.   Hence show that in this
case also $f=\varcheck{g}$.

\spheader 283Yd Suppose that $f:\Bbb R\to\Bbb R$ is even,
twice differentiable and convergent to $0$ at $\infty$, that $f''$ is
continuous and that $\{x:f''(x)=0\}$
is bounded in $\Bbb R$.   Show that $f$ is the Fourier transform of an
integrable function.   \Hint{use 283Yc and 283Yb.}
%283 notes

\spheader 283Ye Let $g:\Bbb R\to\Bbb R$ be an odd function of
bounded variation such that $\int_1^{\infty}\bover1xg(x)dx=\infty$.
Show that $g$ cannot be the Fourier transform of any integrable function
$f$.   ({\it Hint\/}:  show that if $g=\varhatf$ then
%283Yd

\Centerline{$-i\int_0^1f
=\Bover{2}{\sqrt{2\pi}}\lim_{a\to\infty}
  \int_0^a\Bover{1-\cos x}{x}g(x)dx=\infty$.)}
}%end of exercises

\endnotes{
\Notesheader{283}
I have tried in this section to give the elementary theory of Fourier
transforms of integrable functions on $\Bbb R$, with an eye to the
extension of the concept which will be attempted in the next section.
Following \S282, I have given prominence to two theorems (283I and
283L) describing conditions for the inversion of the Fourier transform
to return to the original function;   we find ourselves looking at
improper integrals
$\lim_{a\to\infty}\int_{-a}^a$, just as earlier we needed to look at
symmetric sums $\lim_{n\to\infty}\sum_{k=-n}^n$.   I do not go quite so
far as in \S282, and in particular I leave the study of
square-integrable functions for the moment, since their Fourier
transforms may not be describable by the simple formulae used here.

One of the most fundamental obstacles in the subject is the lack of any
effective criteria for determining which functions are the Fourier
transforms of integrable functions.   (Happily, things are better for
square-integrable functions;  see 284O-284P.)   In 283Yc-283Yd I sketch
an argument showing that `ordinary' non-oscillating {\it even} functions
which converge to $0$ at $\pm\infty$ are Fourier transforms of
integrable functions.   Strikingly, this is not true of {\it odd}
functions;  thus $y\mapsto\Bover{1}{\ln(e+y^2)}$ is the Fourier
transform of an integrable function, but
$y\mapsto\Bover{\arctan y}{\ln(e+y^2)}$ is not (283Ye).

In 283W I sketch the corresponding theory of Fourier transforms in
$\Bbb R^r$.   There are few surprises.   One point to note is that where
in the one-dimensional case we ask for a well-behaved second derivative,
in the $r$-dimensional case we may need to differentiate $r+1$ times
(283Wf).   Another is that we lose the `localization principle'.   In
the one-dimensional case, if $f$ is integrable and zero on an interval
$\ooint{c,d}$, then
$\lim_{a\to\infty}\int_{-a}^ae^{ixy}\varhatf(y)dy=0$ for every
$x\in\ooint{c,d}$;  this is immediate from either 283I or 283L.   But in
higher dimensions the most natural formulation of a corresponding result
is false (283Wm).




}%end of notes

\discrpage

