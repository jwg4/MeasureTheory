\frfilename{mt341.tex}
\versiondate{9.4.10}
\copyrightdate{2000}

\def\chaptername{Liftings}
\def\sectionname{The lifting theorem}

\newsection{341}

I embark directly on the principal theorem of this chapter (341K, `every
non-trivial complete strictly localizable measure space has a lifting'),
using the minimum of advance preparation.   341A-341B give the
definition
of `lifting';  the main argument is in 341F-341K, using the concept of
`lower density' (341C-341E) and a theorem on martingales from \S275.
In 341P I describe an alternative way of thinking about liftings in
terms of the Stone space of the measure algebra.

\leader{341A}{Definition} Let $(X,\Sigma,\mu)$ be a measure space,
and $\frak A$ its measure algebra.   By a {\bf lifting} for $\frak A$ (or
for $(X,\Sigma,\mu)$, or for $\mu$) I shall mean

{\it either} a Boolean homomorphism $\theta:\frak A\to\Sigma$ such that
$(\theta a)^{\ssbullet}=a$ for every $a\in\frak A$

{\it or} a Boolean homomorphism $\phi:\Sigma\to\Sigma$ such that (i)
$\phi E=\emptyset$ whenever $\mu E=0$ (ii) $\mu(E\symmdiff\phi E)=0$ for
every $E\in\Sigma$.

\cmmnt{
\leader{341B}{Remarks (a)} I trust that the ambiguities permitted by
this terminology will not cause any confusion.   The point is that there
is a natural one-to-one correspondence between liftings
$\theta:\frak A\to\Sigma$ and liftings $\phi:\Sigma\to\Sigma$ given by the formula

\Centerline{$\theta E^{\ssbullet}=\phi E$ for every $E\in\Sigma$.}

\noindent\prooflet{\Prf\ (i) Given a lifting $\theta:\frak A\to\Sigma$,
the formula defines a Boolean homomorphism $\phi:\Sigma\to\Sigma$ such
that

\Centerline{$\phi\emptyset=\theta 0=\emptyset$,
\quad$(E\Bsymmdiff\phi E)^{\ssbullet}
=E^{\ssbullet}\Bsymmdiff(\theta E^{\ssbullet})^{\ssbullet}
=0\,\Forall\,E\in\Sigma$,}

\noindent so that $\phi$ is a lifting.   (ii) Given a lifting
$\phi:\Sigma\to\Sigma$, the kernel of $\phi$ includes
$\{E:\mu E=0\}$, so there is a Boolean homomorphism
$\theta:\frak A\to\Sigma$
such that $\theta E^{\ssbullet}=\phi E$ for every $E$ (3A2G), and now

\Centerline{$(\theta E^{\ssbullet})^{\ssbullet}
=(\phi E)^{\ssbullet}=E^{\ssbullet}$}

\noindent for every $E\in\Sigma$, so $\theta$ is a lifting.\ \Qed}

I suppose that the word `lifting' applies most naturally to functions
from $\frak A$ to $\Sigma$;  but for applications in measure theory the
other type of lifting is used at least equally often.


\header{341Bb}{\bf (b)} Note that if $\phi:\Sigma\to\Sigma$ is a lifting
then $\phi^2=\phi$.   \prooflet{\Prf\ For any $E\in\Sigma$,

\Centerline{$\phi^2E\Bsymmdiff\phi E=\phi(E\Bsymmdiff\phi E)=\emptyset$.
\Qed}
}

\noindent If $\phi$ is associated with $\theta:\frak A\to\Sigma$, then
$\phi\theta a=\theta a$ for every $a\in\frak A$.   \prooflet{\Prf\
$\phi\theta a=\theta((\theta a)^{\ssbullet})=\theta a$.\ \Qed}

\spheader 341Bc In the theorems to follow, there will occasionally intrude
a hypothesis `$\mu X>0$'.   The point is that if we have a measure space
$(X,\Sigma,\mu)$ which is trivial in the sense that $\mu X=0$, then the
only candidate for a `lifting' $\phi:\Sigma\to\Sigma$ is the constant
function with value $\emptyset$;  and if $X\ne\emptyset$ this is not a
Boolean homomorphism in the sense of this book.   The simplest way of
dealing with these cases is to rule them out of the discussion.
}%end of comment

\leader{341C}{Definition} Let $(X,\Sigma,\mu)$ be a measure space,
and $\frak A$ its measure algebra.   By a {\bf lower density} for
$\frak A$ (or for $(X,\Sigma,\mu)$, or for $\mu$) I shall mean

{\it either} a function $\undtheta:\frak A\to\Sigma$ such that
(i) $(\undtheta a)^{\ssbullet}=a$ for every $a\in\frak A$
(ii) $\undtheta0=\emptyset$
(iii) $\undtheta(a\Bcap b)=\undtheta a\cap\undtheta b$ for all $a$,
$b\in\frak A$

{\it or} a function $\undphi:\Sigma\to\Sigma$ such that (i)
$\undphi E=\undphi F$ whenever $E$, $F\in\Sigma$ and
$\mu(E\symmdiff F)=0$ (ii) $\mu(E\symmdiff\undphi E)=0$
for every $E\in\Sigma$ (iii) $\undphi\emptyset=\emptyset$ (iv)
$\undphi(E\cap F)=\undphi E\Bcap\undphi F$ for all $E$, $F\in\Sigma$.

\leader{341D}{Remarks (a)} As in 341B, there is a natural one-to-one
correspondence between lower densities
$\undtheta:\frak A\to\Sigma$ and lower densities
$\undphi:\Sigma\to\Sigma$ given by the formula

\Centerline{$\undtheta E^{\ssbullet}=\undphi E$ for
every $E\in\Sigma$.}

\cmmnt{\noindent (For the requirement $\undphi E=\undphi F$
whenever $E^{\ssbullet}=F^{\ssbullet}$ in $\frak A$ means that every
$\undphi$ corresponds to a function $\undtheta$, and
the other clauses match each other directly.)}

\cmmnt{\header{341Db}{\bf (b)} As before, if $\undphi:\Sigma\to\Sigma$
is a lower density then $\undphi^2=\undphi$.   If
$\undphi$ is associated with $\undtheta:\frak
A\to\Sigma$, then $\undphi\undtheta
=\undtheta$.
}

\spheader 341Dc\cmmnt{ It will be convenient, in the course of the
proofs of 341F-341H below, to have the following concept available.}
If $(X,\Sigma,\mu)$ is a measure space with measure algebra $\frak A$,
a {\bf partial lower density} of $\frak A$ is a function
$\undtheta:\frak B\to\Sigma$ such that (i) the domain $\frak B$
of $\undtheta$ is a subalgebra of $\frak A$ (ii)
$(\undtheta b)^{\ssbullet}=b$ for every $b\in\frak B$ (iii)
$\undtheta0=\emptyset$ (iv)
$\undtheta(a\Bcap b)=\undtheta a\cap\undtheta b$ for all $a$,
$b\in\frak B$.

Similarly, if $\Tau$ is a subalgebra of $\Sigma$, a function
$\undphi:\Tau\to\Sigma$ is a {\bf partial lower density} if (i) $\undphi
E=\undphi F$ whenever $E$, $F\in\Tau$ and $\mu(E\symmdiff F)=0$ (ii)
$\mu(E\symmdiff\undphi E)=0$ for every $E\in\Tau$ (iii)
$\undphi\emptyset=\emptyset$
(iv) $\undphi(E\cap F)=\undphi E\cap\undphi F$ for all $E$, $F\in\Tau$.

\cmmnt{\spheader 341Dd Note that lower densities and partial lower
densities
are order-preserving;  if $a\Bsubseteq b$ in $\frak A$, and $\undtheta$
is a lower density for $\frak A$, then

\Centerline{$\undtheta a=\undtheta(a\Bcap b)=\undtheta a\Bcap\undtheta
b\Bsubseteq\undtheta b$.}
}

\cmmnt{\spheader 341De Of course a Boolean homomorphism from $\frak A$ to
$\Sigma$, or from $\Sigma$ to itself, is a lifting iff it is a lower
density.
}%end of comment

\leader{341E}{Example} Let $\mu$ be Lebesgue measure on $\BbbR^r$,
where $r\ge 1$, and $\Sigma$ its domain.   For $E\in\Sigma$ set

\Centerline{$\intstar E
=\{x:x\in\BbbR^r,\,\lim_{\delta\downarrow 0}
\Bover{\mu(E\cap B(x,\delta))}{\mu B(x,\delta)}=1\}$.}

\noindent (Here $B(x,\delta)$ is the closed ball with centre $x$ and
radius $\delta$.)   Then $\intstar$ is a lower density for $\mu$;  we may
call it {\bf lower Lebesgue density}.   \prooflet{\Prf\ (You may prefer
at first to suppose that $r=1$, so that
$B(x,\delta)=[x-\delta,x+\delta]$
and $\mu B(x,\delta)=2\delta$.)   By 261Db (or 223B, for the
one-dimensional case) $E\symmdiff\intstar E$ is negligible for
every $E$;  in particular, $\intstar E\in\Sigma$ for every
$E\in\Sigma$.   If $E\symmdiff F$ is negligible, then
$\mu(E\cap B(x,\delta))=\mu(F\cap B(x,\delta))$ for every $x$ and
$\delta$, so $\intstar E=\intstar F$.   If $E\subseteq F$, then
$\mu(E\cap B(x,\delta))\le \mu(F\cap B(x,\delta))$ for every
$x$, $\delta$, so $\intstar E\subseteq\intstar F$;
consequently $\intstar(E\cap F)\subseteq\intstar E\cap\intstar F$ for all
$E$, $F\in\Sigma$. If $E$, $F\in\Sigma$ and
$x\in\intstar E\cap\intstar F$, then

$$\eqalign{\mu(E\cap F\cap B(x,\delta))
&=\mu(E\cap B(x,\delta))+\mu(F\cap B(x,\delta))
-\mu((E\cup F)\cap B(x,\delta))\cr
&\ge\mu(E\cap B(x,\delta))+\mu(F\cap B(x,\delta))
-\mu(B(x,\delta))\cr}$$

\noindent for every $\delta$, so

\Centerline{$\Bover{\mu(E\cap F\cap B(x,\delta))}{\mu B(x,\delta)}
\ge\Bover{\mu(E\cap B(x,\delta))}{\mu B(x,\delta)}
  +\Bover{\mu(F\cap B(x,\delta))}{\mu B(x,\delta)}-1
\to 1$}

\noindent as $\delta\downarrow 0$, and $x\in\intstar(E\cap F)$.
Thus $\intstar(E\cap F)=\intstar E\cap\intstar F$
for all $E$, $F\in\Sigma$, and $\intstar$ is a lower density.\ \Qed}

\cmmnt{\medskip

\noindent{\bf Remark} In Chapter 47 of Volume 4 I will return to the
operator $\intstar$ in a context in which an alternative name, `essential
interior', is more natural.
}

\leader{341F}{}\cmmnt{ The hard work of this section is in the proof
of 341H below.   To make it a little more digestible, I extract two
parts of the proof as separate lemmas.

\medskip

\noindent}{\bf Lemma} Let $(X,\Sigma,\mu)$ be a probability space and
$\frak A$ its measure algebra.   Let $\frak B$ be a closed subalgebra of
$\frak A$ and $\undtheta:\frak B\to\Sigma$ a partial lower
density.   Then for any $e\in\frak A$ there is a partial lower density
$\undtheta_1$, extending $\undtheta$, defined on the
subalgebra $\frak B_1$ of $\frak A$ generated by $\frak B\cup\{e\}$.

\proof{{\bf (a)} Because $\frak B$ is order-closed, therefore Dedekind
complete in itself (314Ea),

\Centerline{$v=\upr(e,\frak B)=\inf\{a:a\in\frak B,\,a\Bsupseteq e\}$,
\quad$w=\upr(1\Bsetminus e,\frak B)$}

\noindent are defined in $\frak B$.   Let $E\in\Sigma$ be such
that $E^{\ssbullet}=e$.

\medskip

{\bf (b)} We have a function $\undtheta_1:\frak B_1\to\Sigma$ defined by
writing

\Centerline{$\undtheta_1((a\Bcap e)\Bcup(b\Bsetminus e))
=\bigl(\undtheta((a\Bcap v)\Bcup(b\Bsetminus v))\cap E\bigr)
   \cup\bigl(\undtheta((a\Bsetminus w)\Bcup(b\Bcap w))
     \setminus E\bigr)$}

\noindent for $a$, $b\in\frak B$.   \Prf\ By 312N, every element of
$\frak B_1$ is expressible as $(a\Bcap e)\Bcup(b\Bsetminus e)$ for some
$a$, $b\in\frak B$.   If $a$, $a'$, $b$, $b'\in\frak B$ are such that
$(a\Bcap e)\Bcup(b\Bsetminus e)=(a'\Bcap e)\Bcup(b'\Bsetminus e)$, then
$a\Bcap e=a'\Bcap e$ and $b\Bsetminus e=b'\Bsetminus e$, that is,

\Centerline{$a\Bsymmdiff a'\Bsubseteq 1\Bsetminus e\Bsubseteq w$,
\quad$b\Bsymmdiff b'\Bsubseteq e\Bsubseteq v$.}

\noindent This means that
$e\Bsubseteq 1\Bsetminus(a\Bsymmdiff a')\in\frak B$ and
$1\Bsetminus e\Bsubseteq 1\Bsetminus(b\Bsymmdiff b')\in\frak B$.   So we
also have $v\Bsubseteq 1\Bsetminus(a\Bsymmdiff a')$ and
$w\Bsubseteq 1\Bsetminus(b\Bsymmdiff b')$.   Accordingly

\Centerline{$a\Bcap v=a'\Bcap v$,
\quad$b\Bcap w=b'\Bcap w$.
\quad$a\Bsetminus w=a'\Bsetminus w$,
\quad$b\Bsetminus v=b'\Bsetminus v$.}

\noindent But this means that

$$\eqalign{\bigl(\undtheta((a\Bcap v)&\Bcup(b\Bsetminus v))\cap E\bigr)
   \cup\bigl(\undtheta((a\Bsetminus w)\Bcup(b\Bcap w))
      \setminus E\bigr)\cr
&=\bigl(\undtheta((a'\Bcap v)\Bcup(b'\Bsetminus v))\cap E\bigr)
   \cup\bigl(\undtheta((a'\Bsetminus w)\Bcup(b'\Bcap w))
      \setminus E\bigr).\cr}$$

\noindent Thus the formula given defines $\undtheta_1$ uniquely.\ \Qed

\medskip

{\bf (c)} Now $\undtheta_1$ is a partial lower density.

\medskip

\Prf{\bf (i)} If $a$, $b\in\frak B$,

$$\eqalign{(\undtheta_1((a\Bcap e)\Bcup(b\Bsetminus e)))^{\ssbullet}
&=\bigl(\bigl(\undtheta((a\Bcap v)\Bcup(b\Bsetminus v))\cap E\bigr)
   \cup\bigl(\undtheta((a\Bsetminus w)\Bcup(b\Bcap w))\setminus E\bigr)
   \bigr)^{\ssbullet}\cr
&=\bigl(((a\Bcap v)\Bcup(b\Bsetminus v))\Bcap e\bigr)
   \cup\bigl(((a\Bsetminus w)\Bcup(b\Bcap w))\Bsetminus e\bigr)\cr
&=(a\Bcap e)
   \cup(b\Bsetminus e).\cr}$$

\noindent So $(\undtheta_1c)^{\ssbullet}=c$ for every $c\in\frak B_1$.

\medskip

\quad{\bf (ii)}

\Centerline{$\undtheta_1(0)
=\bigl(\undtheta((0\Bcap v)\Bcup(0\Bsetminus v))\cap E\bigr)
   \cup\bigl(\undtheta((0\Bsetminus w)\Bcup(0\Bcap w))\setminus E\bigr)
=\emptyset$.}

\medskip

\quad{\bf (iii)} If $a$, $a'$, $b$, $b'\in\frak B$, then

$$\eqalign{\undtheta_1(((a\Bcap e)&\Bcup(b\Bsetminus e))
  \Bcap((a'\Bcap e)\Bcup(b'\Bsetminus e)))\cr
&=\undtheta_1((a\Bcap a'\Bcap e)\Bcup(b\cap b'\Bsetminus e))\cr
&=\bigl(\undtheta((a\Bcap a'\Bcap v)\Bcup(b\Bcap b'\Bsetminus v))
   \cap E\bigr)
   \cup\bigl(\undtheta((a\Bcap a'\Bsetminus w)\Bcup(b\Bcap b'\Bcap w))
   \setminus E\bigr)\cr
&=\bigl(\undtheta(((a\Bcap v)\Bcup(b\Bsetminus v))
   \Bcap((a'\Bcap v)\Bcup(b'\Bsetminus v)))
   \cap E\bigr)\cr
&\qquad\qquad \cup\bigl(\undtheta(((a\Bsetminus w)\Bcup(b\Bcap w))
   \Bcap((a'\Bsetminus w)\Bcup(b'\Bcap w)))
   \setminus E\bigr)\cr
&=\bigl(\undtheta((a\Bcap v)\Bcup(b\Bsetminus v))
   \cap\undtheta((a'\Bcap v)\Bcup(b'\Bsetminus v))
   \cap E\bigr)\cr
&\qquad\qquad \cup\bigl(\undtheta((a\Bsetminus w)\Bcup(b\Bcap w))
   \cap\undtheta((a'\Bsetminus w)\Bcup(b'\Bcap w))
   \setminus E\bigr)\cr
&=\bigl(\bigl(\undtheta((a\Bcap v)\Bcup(b\Bsetminus v))
   \cap E\bigr)
  \cup\bigl(\undtheta((a\Bsetminus w)\Bcup(b\Bcap w))
   \setminus E\bigr)\bigr)\cr
&\qquad\qquad
  \cap\bigl(\bigl(\undtheta((a'\Bcap v)\Bcup(b'\Bsetminus v))
   \cap E\bigr)
  \cup\bigl(\undtheta((a'\Bsetminus w)\Bcup(b'\Bcap w))
   \setminus E\bigr)\bigr)\cr
&=\undtheta_1((a\Bcap e)\Bcup(b\Bsetminus e))
  \cap\undtheta_1((a'\Bcap e)\Bcup(b'\Bsetminus e)).\cr}$$

\noindent So $\undtheta_1(c\Bcap c')=\undtheta_1(c)\Bcap\undtheta_1(c')$
for all $c$, $c'\in\frak B_1$.\ \Qed

\wheader{341F}{6}{2}{2}{72pt}

{\bf (d)} If $a\in\frak B$, then

$$\eqalign{\undtheta_1(a)
&=\undtheta_1((a\Bcap e)\Bcup(a\Bsetminus e))\cr
&=\bigl(\undtheta((a\Bcap v)\Bcup(a\Bsetminus v))\cap E\bigr)
   \cup\bigl(\undtheta((a\Bsetminus w)\Bcup(a\Bcap w))
     \setminus E\bigr)\cr
&=\bigl(\undtheta(a)\cap E\bigr)
   \cup\bigl(\undtheta(a)\setminus E\bigr)
=\undtheta a.\cr}$$

\noindent Thus $\undtheta_1$ extends $\undtheta$, as required.
}%end of proof of 341F

\leader{341G}{Lemma} Let $(X,\Sigma,\mu)$ be a probability space and
$(\frak A,\bar\mu)$ its measure algebra.   Suppose we have a sequence
$\sequencen{\undtheta_n}$ of partial lower densities such that, for each
$n$, (i) the domain $\frak B_n$ of $\undtheta_n$ is a closed subalgebra
of $\frak A$ (ii) $\frak B_n\subseteq\frak B_{n+1}$ and
$\undtheta_{n+1}$ extends $\undtheta_n$.   Let $\frak B$ be the closed
subalgebra of $\frak A$ generated by $\bigcup_{n\in\Bbb N}\frak B_n$.
Then there is a partial lower density $\undtheta$, with domain $\frak
B$, extending every $\undtheta_n$.

\proof{{\bf (a)} For each $n$, set

\Centerline{$\Sigma_n=\{E:E\in\Sigma,\,E^{\ssbullet}\in\frak B_n\}$,}

\noindent and set

\Centerline{$\Sigma_{\infty}=\{E:E\in\Sigma,\,E^{\ssbullet}\in\frak
B\}$.}

\noindent Then (because all the $\frak B_n$, $\frak B$ are
$\sigma$-subalgebras of $\frak A$, and $E\mapsto E^{\ssbullet}$ is
sequentially order-continuous) all the $\Sigma_n$, $\Sigma_{\infty}$ are
$\sigma$-subalgebras of $\Sigma$.
We need to know that $\Sigma_{\infty}$ is just the
$\sigma$-algebra $\Sigma_{\infty}^*$ of subsets of $X$ generated by
$\bigcup_{n\in\Bbb N}\Sigma_n$.   \Prf\ Because $\Sigma_{\infty}$ is a
$\sigma$-algebra including $\bigcup_{n\in\Bbb N}\Sigma_n$,
$\Sigma_{\infty}^*\subseteq\Sigma_{\infty}$.   On the other hand,
$\frak B^*=\{E^{\ssbullet}:E\in\Sigma^*_{\infty}\}$ is a
$\sigma$-subalgebra of $\frak A$ including $\frak B_n$ for every
$n\in\Bbb N$.   Because $\frak A$ is ccc, $\frak B^*$ is
(order-\nobreak)closed (316Fb), so includes $\frak B$.
This means that if $E\in\Sigma_{\infty}$ there must be an
$F\in\Sigma_{\infty}^*$ such that $E^{\ssbullet}=F^{\ssbullet}$.   But
now $(E\symmdiff F)^{\ssbullet}=0\in\frak B_0$, so
$E\symmdiff F\in\Sigma_0\subseteq \Sigma_{\infty}^*$, and $E$ also
belongs to $\Sigma^*_{\infty}$.   This shows that
$\Sigma_{\infty}\subseteq\Sigma_{\infty}^*$ and the two algebras are
equal.\ \Qed

\medskip

{\bf (b)} For each $n\in\Bbb N$, we have the partial lower density
$\undtheta_n:\frak B_n\to\Sigma$.   Since
$(\undtheta_na)^{\ssbullet}=a\in\frak B_n$ for every
$a\in\frak B_n$, $\undtheta_n$ takes all its values in $\Sigma_n$.
For $n\in\Bbb N$, let $\undphi_n:\Sigma_n\to\Sigma_n$ be the
lower density corresponding to $\undtheta_n$ (341Ba), that is,
$\undphi_nE=\undtheta_nE^{\ssbullet}$ for every
$E\in\Sigma_n$.

\medskip

{\bf (c)} For $a\in\frak A$, $n\in\Bbb N$ choose $G_a\in\Sigma$,
$g_{an}$ such that $G_a^{\ssbullet}=a$ and $g_{an}$ is a conditional
expectation of $\chi G_a$ on $\Sigma_n$;  that is,

\Centerline{$\int_Eg_{an}=\int_E\chi G_a=\mu(E\cap
G_a)=\bar\mu(E^{\ssbullet}\Bcap a)$}

\noindent for every $E\in\Sigma_n$.   As remarked in 233Db, such a
function $g_{an}$ can always be found, and moreover we may take it to be
$\Sigma_n$-measurable and defined everywhere on $X$.
Now if $a\in\frak B$, $\lim_{n\to\infty}g_{an}(x)$ exists and is equal
to $\chi G_a(x)$ for almost every $x$.   \Prf\ By L\'evy's martingale
theorem (275I), $\lim_{n\to\infty}g_{an}$ is defined almost everywhere
and is a conditional expectation of $\chi G_a$ on the $\sigma$-algebra
generated by $\bigcup_{n\in\Bbb N}\Sigma_n$.   As observed in (a), this
is just $\Sigma_{\infty}$;  and as $\chi G_a$ is itself
$\Sigma_{\infty}$-measurable, it is also a conditional expectation of
itself on $\Sigma_{\infty}$, and must be equal almost everywhere to
$\lim_{n\to\infty}g_{an}$.\ \Qed

\medskip

{\bf (d)} For $a\in\frak B$, $k\ge 1$, $n\in\Bbb N$ set

\Centerline{$H_{kn}(a)
=\{x:x\in X,\,g_{an}(x)\ge 1-2^{-k}\}\in\Sigma_n$,
\quad $\tilde H_{kn}(a)=\undphi_n(H_{kn}(a))$,}

\Centerline{$\undtheta a
=\bigcap_{k\ge 1}\bigcup_{n\in\Bbb N}\bigcap_{m\ge n}\tilde H_{km}(a)$.}

\noindent The rest of the proof is devoted to showing that
$\undtheta:\frak B\to\Sigma$ has the required properties.

\medskip

{\bf (e)} $G_0$ is negligible, so every $g_{0n}$ is zero almost
everywhere, every $H_{kn}(0)$ is negligible and every $\tilde H_{kn}(0)$
is empty;  so $\undtheta 0=\emptyset$.

\medskip

{\bf (f)} If $a\Bsubseteq b$ in $\frak B$, then $\undtheta a\subseteq
\undtheta b$.  \Prf\  $G_a\setminus G_b$ is negligible, $g_{an}\le
g_{bn}$ almost everywhere for every $n$, every $H_{kn}(a)\setminus
H_{kn}(b)$ is negligible, $\tilde H_{kn}(a)\subseteq\tilde H_{kn}(b)$
for every $n$ and $k$, and $\undtheta a\subseteq \undtheta b$.\ \Qed

\medskip

{\bf (g)}  If $a$, $b\in\frak B$ then $\undtheta(a\Bcap b)=\undtheta
a\cap\undtheta b$.
\Prf\ $\chi G_{a\Bcap b}\geae\chi G_a+\chi G_b-1$ so
$g_{a\Bcap b,n}\geae g_{an}+g_{bn}-1$ for every $n$.   Accordingly

\Centerline{$H_{k+1,n}(a)\cap H_{k+1,n}(b)\setminus H_{kn}(a\Bcap b)$}

\noindent is negligible, and (because $\undphi_n$ is a lower density)

\Centerline{$\tilde H_{kn}(a\Bcap b)\supseteq
\undphi_n(H_{k+1,n}(a)\cap H_{k+1,n}(b))
=\tilde H_{k+1,n}(a)\cap\tilde H_{k+1,n}(b)$}

\noindent for all $k\ge 1$, $n\in\Bbb N$.   Now, if $x\in\undtheta
a\cap\undtheta b$, then, for any $k\ge 1$, there are $n_1$, $n_2\in\Bbb
N$ such that

\Centerline{$x\in\bigcap_{m\ge n_1}\tilde H_{k+1,m}(a)$,
\quad$x\in\bigcap_{m\ge n_2}\tilde H_{k+1,m}(b)$.}

\noindent But this means that

\Centerline{$x\in\bigcap_{m\ge\max(n_1,n_2)}\tilde H_{km}(a\Bcap b)$.}

\noindent As $k$ is arbitrary, $x\in\undtheta (a\Bcap b)$;  as $x$ is
arbitrary, $\undtheta a\cap\undtheta b\subseteq\undtheta(a\Bcap b)$.
We know already from (f) that $\undtheta(a\Bcap b)\subseteq\undtheta
a\cap\undtheta b$, so
$\undtheta(a\Bcap b)=\undtheta a\cap\undtheta b$.\ \Qed

\medskip

{\bf (h)}
If $a\in\frak B$, then $\undtheta a^{\ssbullet}=a$.   \Prf\
$\sequencen{g_{an}}\to\chi G_a$ a.e., so setting

\Centerline{$V_a
=\bigcap_{k\ge 1}\bigcup_{n\in\Bbb N}\bigcap_{m\ge n}H_{km}(a)
=\{x:\liminf_{n\to\infty}g_{an}(x)\ge 1\}$,}

\noindent $V_a\symmdiff G_a$ is negligible, and
$V_a^{\ssbullet}=a$;  but

\Centerline{$\undtheta a\symmdiff V_a
\subseteq\bigcup_{k\ge 1,n\in\Bbb N}H_{kn}(a)\symmdiff\tilde H_{kn}(a)$}

\noindent is negligible, so $\undtheta a^{\ssbullet}$ is also equal
to $a$.\ \QeD\   Thus $\undtheta$ is a partial lower density with domain
$\frak B$.

\medskip

{\bf (i)} Finally, $\undtheta$ extends $\undtheta_n$ for every
$n\in\Bbb N$.   \Prf\ If $a\in\frak B_n$, then $G_a\in\Sigma_m$ for every $m\ge n$, so $g_{am}\eae\chi G_a$ for every $m\ge n$;  $H_{km}(a)\symmdiff G_a$ is negligible for $k\ge 1$, $m\ge n$;

\Centerline{$\tilde H_{km}=\undphi_mG_a=\undtheta_m a=\undtheta_na$}

\noindent for $k\ge 1$, $m\ge n$ (this is where I use the hypothesis
that $\undtheta_{m+1}$ extends $\undtheta_m$ for every $m$); and

$$\eqalign{\undtheta a
&=\bigcap_{k\ge 1}\bigcup_{r\in\Bbb N}
  \bigcap_{m\ge r}\tilde H_{km}(a)\cr
&=\bigcap_{k\ge 1}\bigcup_{r\ge n}\bigcap_{m\ge r}\tilde H_{km}(a)
=\bigcap_{k\ge 1}\bigcup_{r\ge n}\undtheta_na
=\undtheta_n a.\text{ \Qed}\cr}$$

The proof is complete.
}%end of proof of 341G


\leader{341H}{}\cmmnt{ Now for the first main theorem.

\medskip

\noindent}{\bf Theorem} Let $(X,\Sigma,\mu)$ be any strictly
localizable measure space.   Then it has a lower density
$\undphi:\Sigma\to\Sigma$.   If $\mu X>0$ we can take $\undphi X=X$.

\proof{{\bf: Part A} I deal first with the case of probability spaces.
Let $(X,\Sigma,\mu)$ be a probability space, and $(\frak A,\bar\mu)$ its
measure algebra.

\medskip

{\bf (a)} Set $\kappa=\#(\frak A)$ and enumerate $\frak A$ as $\langle
a_{\xi}\rangle_{\xi<\kappa}$.   For $\xi\le\kappa$ let $\frak A_{\xi}$
be the closed subalgebra of $\frak A$ generated by
$\{a_{\eta}:\eta<\xi\}$.
I seek to define a lower density $\undtheta:\frak A\to\Sigma$ as the
last of a family $\langle\undtheta_{\xi}\rangle_{\xi\le\kappa}$, where
$\undtheta_{\xi}:\frak A_{\xi}\to\Sigma$ is a partial lower
density for each $\xi$.
The inductive hypothesis will be that $\undtheta_{\xi}$ extends
$\undtheta_{\eta}$ whenever $\eta\le\xi\le\kappa$.

To start the induction, we have $\frak A_0=\{0,1\}$,
$\undtheta_00=\emptyset$, $\undtheta_01=X$.
\medskip

{\bf (b)} {\it Inductive step to a successor ordinal $\xi$}   Given a
successor ordinal $\xi\le\kappa$, express it as $\zeta+1$;  we are
supposing that $\undtheta_{\zeta}:\frak A_{\zeta}\to\Sigma$ has
been defined.   Now $\frak A_{\xi}$ is the subalgebra of $\frak A$
generated by $\frak A_{\zeta}\cup\{a_{\zeta}\}$ (because this is a
closed subalgebra, by 323K).   So 341F tells us that $\undtheta_{\zeta}$
can be extended to a partial lower density $\undtheta_{\xi}$ with domain
$\frak A_{\xi}$.

\medskip

{\bf (c)} {\it Inductive step to a non-zero limit ordinal $\xi$ of
countable cofinality} In this case, there is a strictly increasing
sequence $\sequencen{\zeta(n)}$ with supremum $\xi$.   Applying 341G
with $\frak B_n=\frak A_{\zeta(n)}$, we see that there is a partial
lower density $\undtheta_{\xi}$, with domain the closed subalgebra
$\frak B$ generated by $\bigcup_{n\in\Bbb N}\frak A_{\zeta(n)}$,
extending every $\undtheta_{\zeta(n)}$.   Now
$\frak A_{\zeta(n)}\subseteq\frak A_{\xi}$ for every $n$, so
$\frak B\subseteq\frak A_{\xi}$;  but also, if $\eta<\xi$, there is an
$n\in\Bbb N$ such that $\eta<\zeta(n)$, so that
$a_{\eta}\in\frak A_{\zeta(n)}\subseteq\frak B$;  as $\eta$ is arbitrary,
$\frak A_{\xi}\subseteq\frak B$ and $\frak A_{\xi}=\frak B$.   Again, if
$\eta<\xi$, there is an $n$ such that $\eta\le\zeta(n)$, so that
$\undtheta_{\zeta(n)}$ extends $\undtheta_{\eta}$ and $\undtheta_{\xi}$
extends $\undtheta_{\eta}$.   Thus the induction continues.

\medskip


{\bf (d)} {\it Inductive step to a limit ordinal $\xi$ of uncountable
cofinality} In this case,
$\frak A_{\xi}=\bigcup_{\eta<\xi}\frak A_{\eta}$.
\Prf\ Because $\frak A$ is ccc, every member $a$ of
$\frak A_{\xi}$ must be in the closed subalgebra of $\frak A$ generated
by some countable subset $A$ of $\{a_{\eta}:\eta<\xi\}$ (331Gd-331Ge).
Now $A$ can be expressed as $\{a_{\eta}:\eta\in I\}$ for some countable
$I\subseteq\xi$.   As $I$ cannot be cofinal with $\xi$, there is a
$\zeta<\xi$ such that $\eta<\zeta$ for every $\eta\in I$, so that
$A\subseteq\frak A_{\zeta}$ and $a\in\frak A_{\zeta}$.\ \Qed

But now, because $\undtheta_{\zeta}$ extends
$\undtheta_{\eta}$ whenever $\eta\le\zeta<\xi$, we have a
function $\undtheta_{\xi}:\frak A_{\xi}\to\Sigma$ defined by
writing $\undtheta_{\xi}a=\undtheta_{\eta}a$ whenever
$\eta<\xi$ and $a\in\frak A_{\eta}$.   Because the family $\{\frak
A_{\eta}:\eta<\xi\}$ is totally ordered and every
$\undtheta_{\eta}$ is a partial lower density,
$\undtheta_{\xi}$ is a partial lower density.

Thus the induction proceeds when $\xi$ is a limit ordinal of uncountable
cofinality.

\medskip

{\bf (e)} The induction stops when we reach
$\undtheta_{\kappa}:\frak A\to\Sigma$, which is a lower
density such that $\undtheta_{\kappa}1=X$.   Setting
$\undphi E=\undtheta_{\kappa}E^{\ssbullet}$,
$\undphi$ is a lower density such that $\undphi X=X$.

\medskip

{\bf Part B} The general case of a strictly localizable
measure space
follows easily.   First, if $\mu X=0$, then $\frak A=\{0\}$ and we
can set $\undphi 0=\emptyset$.   Second, if $\mu$ is totally finite
but not zero, we can replace it by $\nu$, where $\nu E=\mu
E/\mu X$ for every $E\in\Sigma$;  a lower density for $\nu$ is
also a lower density for $\mu$.   Third, if $\mu$ is not totally finite,
let $\langle X_i\rangle_{i\in I}$ be a decomposition of $X$ (211E).
There is surely some $j$ such that $\mu X_j>0$;  replacing $X_j$ by
$X_j\cup\bigcup\{X_i:i\in I,\,\mu X_i=0\}$, we may assume that $\mu
X_i>0$ for every $i\in I$.   For each
$i\in I$, let $\undphi_i:\Sigma_i\to\Sigma_i$ be a
lower density for $\mu_i$, where $\Sigma_i=\Sigma\cap\Cal PX_i$ and
$\mu_i=\mu\restr\Sigma_i$, such that $\undphi_iX_i=X_i$.   Then it is
easy to check that we
have a lower density $\undphi:\Sigma\to\Sigma$ given by setting

\Centerline{$\undphi E=\bigcup_{i\in I}\undphi_i(E\cap X_i)$}

\noindent for every $E\in\Sigma$, and that $\undphi X=X$.
}%end of proof of 341H

\leader{341I}{}\cmmnt{ The next step is to give a method of moving
from lower densities to liftings.   I start with an elementary remark on
lower densities on complete measure spaces.

\medskip

\noindent}{\bf Lemma} Let $(X,\Sigma,\mu)$ be a complete measure space
with
measure algebra $\frak A$.

(a) Suppose that $\undtheta:\frak A\to\Sigma$ is a lower density and
$\undtheta_1:\frak A\to\Cal PX$ is a function such that
$\undtheta_10=\emptyset$,
$\undtheta_1(a\Bcap b)=\undtheta_1a\cap\undtheta_1b$ for all
$a$, $b\in\frak A$ and $\undtheta_1a\supseteq\undtheta a$ for all
$a\in\frak A$.   Then $\undtheta_1$ is a lower density.   If
$\undtheta_1$
is a Boolean homomorphism, it is a lifting.

(b) Suppose that $\undphi:\Sigma\to\Sigma$ is a lower density and
$\undphi_1:\Sigma\to\Cal PX$ is a function such that
$\undphi_1E=\undphi_1F$ whenever $E\symmdiff F$ is negligible,
$\undphi_1\emptyset=\emptyset$,
$\undphi_1(E\Bcap F)=\undphi_1E\cap\undphi_1F$ for all
$E$, $F\in\Sigma$ and $\undphi_1E\supseteq\undphi E$ for all
$E\in\Sigma$.
Then $\undphi_1$ is a lower density.   If $\undphi_1$ is a Boolean
homomorphism, it is a lifting.

\proof{{\bf (a)} All I have to check is that $\undtheta_1a\in\Sigma$ and
$(\undtheta_1a)^{\ssbullet}=a$ for every $a\in\frak A$.   But

\Centerline{$\undtheta a\subseteq\undtheta_1a$,
\quad$\undtheta(1\Bsetminus a)\subseteq\undtheta_1(1\Bsetminus a)$,
\quad$\undtheta_1a\cap\undtheta_1(1\Bsetminus a)
=\undtheta_10=\emptyset$.}

\noindent So

\Centerline{$\undtheta a\subseteq\undtheta_1a\subseteq
X\setminus\undtheta(1\Bsetminus a)$.}

\noindent Since

\Centerline{$(\undtheta a)^{\ssbullet}=a
=(X\setminus\undtheta(1\Bsetminus a))^{\ssbullet}$,}

\noindent and $\mu$ is complete, $\undtheta_1$ is a lower density.   If
it is a Boolean homomorphism, then it is also a lifting (341De).

\medskip

{\bf (b)} This follows by the same argument, or by looking at the
functions from $\frak A$ to $\Sigma$ defined by $\undphi$ and
$\undphi_1$ and using (a).
}%end of proof of 341I

\leader{341J}{Proposition} Let $(X,\Sigma,\mu)$ be a complete
measure space such that $\mu X>0$, and $\frak A$ its measure algebra.

(a) If $\undtheta:\frak A\to\Sigma$ is any lower density, there is a
lifting $\theta:\frak A\to\Sigma$ such that $\theta a\supseteq\undtheta
a$
for every $a\in\frak A$.

(b) If $\undphi:\Sigma\to\Sigma$ is any lower density, there is a
lifting $\phi:\Sigma\to\Sigma$ such that $\phi E\supseteq\undphi E$ for
every $E\in\Sigma$.

\proof{{\bf (a)}  For each $x\in\undtheta 1$, set

\Centerline{$I_x=\{a:a\in\frak A,\,x\in\undtheta(1\Bsetminus a)\}$.}

\noindent Then $I_x$ is a proper ideal of $\frak A$.   \Prf\ We have

\inset{$0\in I_x$, because $x\in\undtheta 1$,}

\inset{if $b\Bsubseteq a\in I_x$ then $b\in I_x$, because
$x\in\undtheta(1\Bsetminus a)\subseteq\undtheta(1\Bsetminus b)$,}

\inset{if $a$, $b\in I_x$ then $a\Bcup b\in I_x$, because
$x\in\undtheta(1\Bsetminus a)\cap\undtheta(1\Bsetminus
b)=\undtheta(1\Bsetminus(a\Bcup b))$,}

\inset{$1\notin I_x$ because $x\notin\emptyset=\undtheta 0$.\ \Qed}

\noindent For $x\in X\setminus\undtheta 1$, set $I_x=\{0\}$;  this is
also a proper ideal of $\frak A$, because $\frak A\ne\{0\}$.   By 311D,
there is a surjective Boolean homomorphism $\pi_x:\frak A\to\{0,1\}$
such that $\pi_xd=0$ for every $d\in I_x$.

Define $\theta:\frak A\to\Cal PX$ by setting

\Centerline{$\theta a=\{x:x\in X,\,\pi_x(a)=1\}$}

\noindent for every $a\in\frak A$.   It is easy to check that, because
every $\pi_x$ is a surjective Boolean homomorphism, $\theta$ is a
Boolean homomorphism.   Now for any $a\in\frak A$, $x\in X$,

\Centerline{$x\in\undtheta a\Longrightarrow 1\Bsetminus a\in I_x
\Longrightarrow \pi_x(1\Bsetminus a)=0\Longrightarrow \pi_xa=1
\Longrightarrow x\in\theta a$.}

\noindent Thus $\theta a\supseteq\undtheta a$ for every $a\in\frak A$.
By 341I, $\theta$ is a lifting, as required.

\medskip

{\bf (b)} Repeat the argument above, or apply it, defining $\undtheta$
by setting $\undtheta(E^{\ssbullet})=\undphi E$ for every $E\in\Sigma$,
and $\phi$ by setting $\phi E=\theta(E^{\ssbullet})$ for every $E$.
}%end of proof of 341J

\leader{341K}{The Lifting Theorem} Every complete strictly localizable
measure space of non-zero measure has a lifting.

\proof{ By 341H, it has a lower density, so by 341J it has a lifting.}

\cmmnt{
\leader{341L}{Remarks} If we count 341F-341K as a single argument, it
may be the longest proof, after Carleson's theorem (\S286), which I have
yet presented in this treatise, and
perhaps it will be helpful if I suggest ways of looking at its
components.

\header{341La}{\bf (a)} The first point is that the theorem should be
thought of as one about probability spaces.   The shift to general
strictly localizable spaces (Part B of the proof of 341H) is purely a
matter of technique.   I would not have presented it if I did not think
that it's worth doing, for a variety of reasons, but there is no
significant idea needed, and if -- for instance -- the result were valid
only for $\sigma$-finite spaces, it would still be one of the great
theorems of mathematics.   So the rest of these remarks will be directed
to the ideas needed in probability spaces.

\header{341Lb}{\bf (b)} All the proofs I know of the theorem depend in
one way or another on an inductive construction.   We do not, of course,
need a transfinite induction written out in the way I have presented it
in 341H above.   Essentially the same proof can be presented as an
application of Zorn's Lemma;  if we take $P$ to be the set of partial
lower densities, then the arguments of 341G and part (A-d) of the proof
of 341H can be adapted to prove that any totally ordered subset of $P$
has an upper bound in $P$, while the argument of 341F shows that any
maximal element of $P$ must have domain $\frak A$.   I think it is
purely a matter of taste which form one prefers.   I suppose I have used
the ordinal-indexed form largely because that seemed appropriate for
Maharam's theorem in the last chapter.

\header{341Lc}{\bf (c)} There are then three types of inductive step to
examine, corresponding to 341F, 341G and (A-d) in 341H.   The first and
last are easier than the second.   Seeking the one-step extension of
$\undtheta:\frak B\to\Sigma$ to $\undtheta_1:\frak B_{1}\to\Sigma$,
the natural model to use is the one-step extension of a
Boolean homomorphism presented in 312O.    The situation here is
rather more complicated, as $\undtheta_1$ is not fully specified by the
value of $\undtheta_1e$, and we do in fact have more freedom at this
point than is entirely welcome.   The formula used in the proof of 341F
is derived from {\smc Graf \& Weizs\"acker 76}.

\header{341Ld}{\bf (d)} At this point I must call attention to the way
in which the whole proof is dominated by the choice of {\it closed}
subalgebras as the domains of our partial liftings.   This is what makes
the inductive step to a limit ordinal $\xi$ of countable cofinality
difficult, because $\frak A_{\xi}$ will ordinarily be larger than
$\bigcup_{\eta<\xi}\frak A_{\eta}$.   But it is absolutely essential in
the one-step extensions as treated here.   (I will return to this point in
\S535 of Volume 5.   See also 341Ye.)

Because we are dealing with a ccc algebra $\frak A$, the requirement
that the $\frak A_{\xi}$ should be closed is not a problem when
$\cf\xi$ is uncountable, since in this case
$\bigcup_{\eta<\xi}\frak A_{\eta}$ is already a closed subalgebra;
this is the only idea needed in (A-d) of 341H.

\header{341Le}{\bf (e)} So we are left with the inductive step to $\xi$
when $\cf\xi=\omega$, which is 341G.   Here we actually need some
measure theory, and a particularly striking bit.   (You will see that
the {\it measure} $\mu$, as opposed to the algebras $\Sigma$ and
$\frak A$ and the homomorphism $E\mapsto E^{\ssbullet}$ and the ideal of
negligible sets, is simply not mentioned anywhere else in the whole
argument.)

\medskip

\quad{\bf (i)}
The central idea is to use the fact that bounded martingales converge to
define $\undtheta a$ in terms of a sequence of conditional
expectations.   Because I have chosen a fairly direct assault on the
problem, some of the surrounding facts are not perhaps so clearly
visible as they might have been if I had used a more leisurely route.
For each $a\in\frak A$, I start by choosing a representative
$G_a\in\Sigma$;  let me emphasize that this is a crude application of
the axiom of choice, and that the different sets $G_a$ are in no way
coordinated.   (The theorem we are proving is that
they {\it can} be coordinated, but we have not reached that point yet.)
Next, I choose, arbitrarily, a conditional expectation $g_{an}$ of
$\chi G_a$ on each
$\Sigma_n$.   Once again, the choices are not coordinated;
but the martingale theorem assures us that $g_a=\lim_{n\to\infty}g_{an}$
is defined almost everywhere, and is equal almost everywhere to
$\chi G_a$ if $a\in\frak B$.   Of course I could have gone to the $g_{an}$
directly, without mentioning the $G_a$;  $g_{an}$ is a Radon-Nikod\'ym
derivative of the countably additive functional
$E\mapsto\bar\mu(E^{\ssbullet}\Bcap a):\Sigma_n\to\Bbb R$.   Now the
$g_{an}$, like the $G_a$, are not uniquely defined.   But they are
defined `up to a negligible set', so that any alternative functions
$g'_{an}$ would have $g'_{an}\eae g_{an}$.   This means that the sets
$H_{kn}(a)=\{x:g_{an}(x)\ge 1-2^{-k}\}$ are also defined `up to a
negligible set', and consequently the sets
$\tilde H_{kn}(a)=\undphi_{n}(H_{kn}(a))$ are uniquely defined.   I point
this out to show that it is not a complete miracle that we have formulae

\Centerline{$\tilde H_{kn}(a)\subseteq\tilde H_{kn}(b)$ if
$a\Bsubseteq b$,}

\Centerline{$\tilde H_{kn}(a\Bcap b)
\supseteq\tilde H_{k+1,n}(a)\cap\tilde H_{k+1,n}(b)$
for all $a$, $b\in\frak A$}

\noindent which do not ask us to turn a blind eye to any negligible
sets.   I note in passing that I could have defined the
$\tilde H_{kn}(a)$ without mentioning the $g_{an}$;  in fact

\Centerline{$\tilde H_{kn}(a)
=\undtheta_{n}(\sup\{c:c\in\frak B_{n},\,
\bar\mu(a\Bcap d)\ge 1-2^{-k}\bar\mu d$ whenever $d\in\frak B_{n}$
and $d\Bsubseteq c\})$.}

\medskip

\quad{\bf (ii)} Now, with the sets $\tilde H_{kn}(a)$ in hand, we can
look at

\Centerline{$\tilde V_a=\bigcap_{k\ge 1}\bigcup_{n\in\Bbb
N}\bigcap_{m\ge n}\tilde H_{kn}(a)$;}

\noindent because $g_{an}\to \chi G_a$ a.e., $\tilde V_a\symmdiff G_a$
is negligible and $\tilde V_a^{\ssbullet}=a$ for every
$a\in\frak A_{\xi}$.   The rest of the argument amounts to checking that
$a\mapsto\tilde V_a$ will serve for $\undtheta$.

\spheader 341Lf The arguments above apply to all probability spaces, and
show that every probability space has a lower density.   The next step
is to convert a lower density into a lifting.   It is here that we need
to assume completeness.   The point is that we can find a Boolean
homomorphism $\theta:\frak A\to\Cal PX$ such that $\undtheta
a\subseteq\theta a$ for every $a$;   this corresponds just to extending
the ideals $I_x=\{a:x\in\undtheta(1\Bsetminus a)\}$ to maximal ideals
(and giving a moment's thought to $x\in X\setminus\undtheta 1$).   In
order to ensure that $\theta a\in\Sigma$ and $(\theta a)^{\ssbullet}=a$,
we have to observe that $\theta a$ is sandwiched between $\undtheta a$
and $X\setminus\undtheta(1\Bsetminus a)$, which differ by a negligible
set;  so that if $\mu$ is complete all will be well.

\spheader 341Lg The fact that completeness is needed at only one point
in the argument makes it natural to wonder whether the theorem might be
true for probability spaces in general.   (I will come later, in
341M, to non-strictly-localizable spaces.)   There is as yet no
satisfactory answer to this.   For Borel measure on $\Bbb R$, the
question is known to be undecidable from the ordinary axioms of set
theory (including the axiom of choice, but not the continuum hypothesis,
as usual);  I will give the easy part of the argument in \S535;
see {\smc Burke 93} for the rest.
But I conjecture that there is a counter-example under the ordinary
axioms (see 341Z below).

\spheader 341Lh Quite apart from whether completeness is needed in the
argument, it is not absolutely clear why measure theory is required.
The general question of whether a lifting exists can be formulated for
any triple $(X,\Sigma,\Cal I)$ where $X$ is a set, $\Sigma$ is a
$\sigma$-algebra of subsets of $X$, and $\Cal I$ is a $\sigma$-ideal of
$\Sigma$.   (See 341Ya below.)   S.Shelah has given an example of such a
triple without a lifting in which two of the basic
properties of the
measure-theoretic case are satisfied:  $(X,\Sigma,\Cal I)$ is
`complete' in the sense that every subset of any member of $\Cal I$
belongs to $\Sigma$ (and therefore to $\Cal I$), and $\Cal I$ is
$\omega_1$-saturated in $\Sigma$ in the sense of 316C (see {\smc Shelah
Sh636}).   But many other cases are known (e.g., 341Yb) in which liftings
do exist.

\spheader 341Li It is of course possible to prove 341K without
mentioning `lower densities', and there are even some advantages in
doing so.   The idea is to follow the lines of 341H, but with `liftings'
instead of `lower densities' throughout.   The inductive step to a
successor ordinal is actually easier, because we have a Boolean
homomorphism $\theta$ in 341F to extend, and we can use 312O as it
stands if we can choose the pair $E$, $F=X\setminus E$ correctly.   The
inductive step to an ordinal of uncountable cofinality remains
straightforward.   But in the inductive step to an ordinal of countable
cofinality, we find that in 341G we get no help from assuming that the
$\undtheta_n$ are actually liftings;  we are still led to to a lower
density $\undtheta$.   So at this point we have to interpolate the
argument of 341J to convert this lower density into a lifting.

I have chosen the more leisurely exposition, with the extra concept,
partly in order to get as far as possible without assuming completeness
of the measure and partly because lower densities are an important tool
for further work (see \S\S345-346).

\spheader 341Lj For more light on the argument of 341G see also 363Xe
and 363Yf below.
}%end of comment 341L

\leader{341M}{}\cmmnt{ I remarked above that the shift from
probability spaces to general strictly localizable spaces was simply a
matter of technique.   The question of which spaces have liftings is
also primarily a matter concerning probability spaces, as the next
result shows.

\medskip

\noindent}{\bf Proposition} Let $(X,\Sigma,\mu)$ be a complete locally
determined space with $\mu X>0$.   Then it has a lifting iff it has a
lower density iff it is strictly localizable.

\proof{ If $(X,\Sigma,\mu)$ is strictly localizable then it has a
lifting, by 341K.   A lifting is already a lower density, and if
$(X,\Sigma,\mu)$ has a lower density it has a lifting, by 341J.   So we
have only to prove that if it has a lifting then it is strictly
localizable.

Let $\theta:\frak A\to\Sigma$ be a lifting, where $\frak A$ is the
measure algebra of $(X,\Sigma,\mu)$.   Let $C$ be a partition of unity
in $\frak A$ consisting of elements of finite measure (322Ea).   Set
$\Cal A=\{\theta c:c\in C\}$.   Because $C$ is disjoint, so is
$\Cal A$.   Because $\sup C=1$ in $\frak A$, every set of positive
measure meets some member of $\Cal A$ in a set of positive measure.   So
the conditions of 213O are satisfied, and $(X,\Sigma,\mu)$ is strictly
localizable.
}%end of proof of 341M

\leader{341N}{Extension of partial \dvrocolon{liftings}}\cmmnt{ The
following facts are obvious from the proof of 341H, but it will be
useful to have them out in the open.

\wheader{341N}{6}{2}{2}{48pt}

\noindent}{\bf Proposition} Let $(X,\Sigma,\mu)$ be a probability space
and $\Tau$ a $\sigma$-subalgebra of $\Sigma$.

(a) Any partial lower density $\undphi_0:\Tau\to\Sigma$ has an extension
to a lower density $\undphi:\Sigma\to\Sigma$.

(b) Suppose now that $\mu$ is complete.   If $\undphi_0$ is a Boolean
homomorphism, it has an extension to a lifting $\phi$ for $\mu$.

%query:  example of partial lower density on a subalgebra which has
%no extension?

\proof{{\bf (a)} In Part A of the proof of 341H, let $\frak A_{\xi}$ be
the closed subalgebra of $\frak A$ generated by
$\{E^{\ssbullet}:E\in\Tau\}\cup\{a_{\eta}:\eta<\xi\}$, and set
$\undtheta_0E^{\ssbullet}=\undphi_0E$ for every $E\in\Tau$.   Proceed
with the induction as before.   The only difference is that we no longer
have a guarantee that $\undphi X=X$.

\medskip

{\bf (b)} Suppose now that $\undphi_0$ is a Boolean homomorphism and $\mu$
is complete.   341J
tells us that there is a lifting $\phi:\Sigma\to\Sigma$ such that $\phi
E\supseteq\undphi E$ for every $E\in\Sigma$.   But if $E\in\Tau$ we must
have $\phi E\supseteq\undphi_0E$,

\Centerline{$\phi E\setminus\undphi_0E
=\phi E\cap\undphi_0(X\setminus E)
\subseteq\phi E\cap\phi(X\setminus E)=\emptyset$,}

\noindent so that $\phi E=\undphi_0E$, and $\phi$ extends $\undphi_0$.
}%end of proof of 341N


\leader{341O}{Liftings and Stone spaces}\cmmnt{ The arguments of this
section so far involve repeated use of the axiom of choice, and offer no
suggestion that any liftings (or lower densities) are in any sense
`canonical'.   There is however one context in which we have a
distinguished
lifting.}   Suppose that we have the Stone space $(Z,\Tau,\nu)$ of a
measure algebra $(\frak A,\bar\mu)$;  \cmmnt{as in 311E,} I think of
$Z$ as being the set of surjective Boolean homomorphisms from $\frak A$
to $\Bbb Z_2$, so that each $a\in\frak A$ corresponds to the
open-and-closed set $\widehat{a}=\{z:z(a)=1\}$.   Then we have a lifting
$\theta:\frak A\to\Tau$ defined by setting $\theta a=\widehat{a}$ for
each $a\in\frak A$.   \cmmnt{(I am identifying
$\frak A$ with the measure algebra of $\nu$, as in 321J.)}   The
corresponding
lifting $\phi:\Tau\to\Tau$ is defined by taking $\phi E$ to be that
unique open-and-closed set such that $E\symmdiff\phi E$ is
negligible\cmmnt{ (or, if you prefer, meager)}.

\cmmnt{Generally, liftings can be described in terms of Stone spaces,
as follows.}

\leader{341P}{Proposition} Let
$(X,\Sigma,\mu)$ be a measure space, $(\frak A,\bar\mu)$ its measure
algebra, and $(Z,\Tau,\nu)$ the Stone space of $(\frak A,\bar\mu)$ with
its canonical measure.

(a) There is a one-to-one correspondence between liftings
$\theta:\frak A\to\Sigma$ and functions $f:X\to Z$ such that
$f^{-1}[\widehat{a}]\in\Sigma$
and $(f^{-1}[\widehat{a}])^{\ssbullet}=a$ for every $a\in\frak A$,
defined by the formula

\Centerline{$\theta a=f^{-1}[\widehat{a}]$ for every $a\in\frak A$.}

(b) If $(X,\Sigma,\mu)$ is complete and locally determined, then a
function $f:X\to Z$ satisfies the conditions of (a) iff ($\alpha$) it is
inverse-measure-preserving ($\beta$) the homomorphism it induces between
the measure algebras of $\mu$ and $\nu$ is the canonical
isomorphism defined by the construction of $Z$.

\proof{ Recall that $\Tau$ is just the set
$\{\widehat{a}\symmdiff M:a\in\frak A$, $M\subseteq Z$ is meager$\}$,
and that $\nu(\widehat{a}\symmdiff M)=\bar\mu a$ for all such $a$, $M$;
while the canonical isomorphism $\pi$ between $\frak A$ and the measure
algebra of $\nu$ is defined by the formula

\Centerline{$\pi F^{\ssbullet}=a$ whenever $F\in\Tau$, $a\in\frak A$ and
$F\symmdiff\widehat{a}$ is meager}

\noindent (341K).

\medskip

{\bf (a)} If $\theta:\frak A\to\Sigma$ is any Boolean homomorphism, then
for every $x\in X$ we have a surjective Boolean homomorphism
$f_{\theta}(x):\frak A\to\Bbb Z_2$ defined by saying that
$f_{\theta}(x)(a)=1$ if $x\in\theta a$, $0$ otherwise.   $f_{\theta}$ is
a function from $X$ to $Z$.   We can recover
$\theta$ from $f_{\theta}$ by the formula

\Centerline{$\theta
a=\{x:f_{\theta}(x)(a)=1\}=\{x:f_{\theta}(x)\in\widehat{a}\}
=f_{\theta}^{-1}[\widehat{a}]$.}

\noindent So $f_{\theta}^{-1}[\widehat{a}]\in\Sigma$ and, if $\theta$ is
a lifting,


\Centerline{$(f_{\theta}^{-1}[\widehat{a}])^{\ssbullet}=(\theta
a)^{\ssbullet}=a$.}

\noindent for every $a\in\frak A$.

Similarly, given a function $f:X\to Z$ with this property, then we can
set $\theta a=f^{-1}[\widehat{a}]$ for every $a\in\frak A$ to obtain a
lifting $\theta:\frak A\to\Sigma$;  and of course we now have

\Centerline{$f(x)(a) = 1\iff f(x)\in\widehat{a}\iff x\in\theta a$,}

\noindent so $f_{\theta}=f$.

\medskip

{\bf (b)} Assume now that $(X,\Sigma,\mu)$ is complete and locally
determined.

\medskip

\quad{\bf (i)} Let $f:X\to Z$ be the function associated with a lifting
$\theta$, as in (a).   I show first that $f$ is \imp.
\Prf\ If $F\in\Tau$, express it as $\widehat{a}\symmdiff M$, where
$a\in\frak A$ and $M\subseteq Z$ is meager.   By 322F, $\frak A$ is
\wsid, so $M$ is nowhere dense (316I).
Consider $f^{-1}[M]$.  If $E\subseteq X$ is measurable and of finite
measure, then $E\cap f^{-1}[M]$ has a measurable envelope $H$ (132Ee).
\Quer\ If $\mu H>0$, then $b=H^{\ssbullet}\ne 0$ and $\widehat{b}$ is a
non-empty open set in $Z$.   Because $M$ is nowhere dense, there is a
non-zero $a\in\frak A$ such that
$\widehat{a}\subseteq\widehat{b}\setminus
M$.   Now $\mu(f^{-1}[\widehat{b}]\symmdiff H)=0$, so
$f^{-1}[\widehat{a}]\setminus
H$ is negligible, and $f^{-1}[\widehat{a}]\cap H$ is a non-negligible
measurable set disjoint from $E\cap f^{-1}[M]$ and included in $H$;
which is
impossible.\ \BanG\   Thus $H$ and $E\cap f^{-1}[M]$ are negligible.
This is true for every measurable set $E$ of finite measure.  Because
$\mu$ is complete and locally determined, $f^{-1}[M]\in\Sigma$ and
$\mu f^{-1}[M]=0$.
So $f^{-1}[F]=f^{-1}[\widehat{a}]\symmdiff f^{-1}[M]$ is measurable, and

\Centerline{$\mu f^{-1}[F]=\mu f^{-1}[\widehat{a}]=\mu\theta a=\bar\mu
a=\nu\widehat{a}=\nu F$.}

\noindent As $F$ is arbitrary, $f$ is \imp.\ \Qed

It follows at once that for any $F\in\Tau$,

\Centerline{$f^{-1}[F]^{\ssbullet}=a=\pi F^{\ssbullet}$}
\noindent where $a$ is that element of $\frak A$ such that $M=F\symmdiff
a$ is meager, because in this case $f^{-1}[\widehat{a}]^{\ssbullet}=a$,
by (a), while $f^{-1}[M]$ is negligible.   So $\pi$ is the
homomorphism induced by $f$.

\medskip

\quad{\bf (ii)} Now suppose that $f:X\to Z$ is an
inverse-measure-preserving function such that $f^{-1}[F]^{\ssbullet}=\pi
F^{\ssbullet}$ for every $F\in\Tau$.   Then, in particular,

\Centerline{$f^{-1}[\widehat{a}]^{\ssbullet}=\pi
\widehat{a}^{\ssbullet}=a$}

\noindent for every $a\in\frak A$, so that $f$ satisfies the conditions
of (a).
}%end of proof of 341P

\leader{341Q}{Corollary} Let $(X,\Sigma,\mu)$ be a strictly
localizable measure space, $(\frak A,\bar\mu)$ its measure algebra, and
$Z$ the Stone space of $\frak A$;  suppose that $\mu X>0$.   For
$E\in\Sigma$ write $E^*$ for the
open-and-closed subset of $Z$ corresponding to
$E^{\ssbullet}\in\frak A$.   Then there is a function $f:X\to Z$ such
that $E\symmdiff f^{-1}[E^*]$ is negligible for every $E\in\Sigma$.
If $\mu$ is complete, then $f$ is \imp.

\proof{ Let $\hat\mu$ be the completion of $\mu$, and $\hat\Sigma$ its
domain.   Then we can identify $(\frak A,\bar\mu)$ with the measure
algebra of $\hat\mu$ (322Da).   Let $\theta:\frak A\to\hat\Sigma$ be a
lifting, and $f:X\to Z$ the
corresponding function.   If $E\in\Sigma$ then $E^*=\widehat{a}$ where
$a=E^{\ssbullet}$, so
$E\symmdiff f^{-1}[E^*]=E\symmdiff\theta E^{\ssbullet}$ is negligible.
If $\mu$ is itself complete, so that $\hat\Sigma=\Sigma$, then $f$ is
\imp, by 341Pb.
}%end of proof of 341Q

\exercises{
\leader{341X}{Basic exercises (a)} Let $(X,\Sigma,\mu)$ be a measure
space and $\phi:\Sigma\to\Sigma$ a function.   Show that $\phi$ is a
lifting iff it is a lower density and $\phi E\cup\phi(X\setminus E)=X$
for every $E\in\Sigma$.
%341A

\sqheader 341Xb Let $\nu_{\Bbb N}$ be the usual measure on
$X=\{0,1\}^{\Bbb N}$, and $\Tau_{\Bbb N}$ its domain.
For $x\in X$ and $n\in\Bbb N$ set
$U_n(x)=\{y:y\in X,\,y\restr n=x\restr n\}$.   For $E\in\Tau_{\Bbb N}$ set
$\undphi E=\{x:\lim_{n\to\infty}2^n\mu(E\cap U_n(x))=1\}$.   Show
that $\undphi$ is a lower density for $\nu_{\Bbb N}$.
%341E

\sqheader 341Xc Let $\frak A$ be a Boolean algebra, $I$ an ideal of
$\frak A$, and $\frak B$ a countable subalgebra of the quotient algebra
$\frak A/I$.
Show that there is a Boolean homomorphism $\theta:\frak B\to\frak A$
such that
$(\theta b)^{\ssbullet}=b$ for every $b\in\frak B$.
\Hint{let $\sequencen{b_n}$ run over $\frak B$;  let $\frak B_n$ be the
subalgebra of $\frak B$ generated
by $\{b_i:i<n\}$;  given $\theta\restr\frak B_n$, show that there is an
$a_n\in\frak A$ such that $a_n^{\ssbullet}=b_n$ and
$\theta b'\Bsubseteq a_n\Bsubseteq \theta b''$
whenever $b'$, $b''\in\frak B_n$ and $b'\Bsubseteq b_n\Bsubseteq b''$.}
%new 2003;  341F

\sqheader 341Xd Let $P$ be the set of all lower densities of a complete
measure space $(X,\Sigma,\mu)$, with measure algebra $\frak A$, ordered
by saying that $\undtheta\le\undtheta'$ if
$\undtheta a\subseteq\undtheta'a$
for every $a\in\frak A$.   Show that any non-empty totally ordered
subset of $P$ has an upper bound
in $P$.   Show that if $\undtheta\in P$, $a\in\frak A\setminus\{0\}$ and
$x\in X\setminus(\undtheta a\cup\undtheta(1\Bsetminus a))$, then
$\undtheta':\frak A\to\Sigma$ is a lower density, where
$\undtheta'b=\undtheta b\cup\{x\}$ if either $a\Bsubseteq b$ or there is
a $c\in\frak A$ such that
$x\in\undtheta c$ and $a\Bcap c\Bsubseteq b$, and
$\undtheta'b=\undtheta b$ otherwise.   Hence prove 341J.
%341J

\spheader 341Xe Let $(X,\Sigma,\mu)$ and $(Y,\Tau,\nu)$ be
measure spaces and suppose that there is an inverse-measure-preserving
function $f:X\to Y$ such that the associated homomorphism from the
measure algebra of $\nu$ to that of $\mu$ (324M) is an isomorphism.
Show that for every lifting $\phi$ for $(Y,\Tau,\nu)$ we have a
corresponding lifting $\psi$ of $(X,\Sigma,\mu)$ defined uniquely by the
formula

\Centerline{$\psi(f^{-1}[F])=f^{-1}[\phi F]$ for every $F\in\Tau$.}
%341P

\spheader 341Xf Let $(X,\Sigma,\mu)$ be a measure space, and write
$\eusm L^{\infty}(\Sigma)$ for the linear space of all bounded
$\Sigma$-measurable functions from $X$ to $\Bbb R$.   Show that for any
lifting $\phi:\Sigma\to\Sigma$ of $\mu$ there is a unique linear
operator $T:L^{\infty}(\mu)\to\eusm L^{\infty}(\Sigma)$ such that
$T(\chi E)^{\ssbullet}=\chi(\phi E)$ for every $E\in\Sigma$ and
$Tu\ge 0$ in $\eusm L^{\infty}(\Sigma)$ whenever $u\ge 0$ in
$L^{\infty}(\mu)$.   Show that (i) $(Tu)^{\ssbullet}=u$ and
$\sup_{x\in X}|(Tu)(x)|=\|u\|_{\infty}$ for every $u\in L^{\infty}(\mu)$
(ii) $T(u\times v)=Tu\times Tv$ for all $u$, $v\in L^{\infty}(\mu)$.

\spheader 341Xg Let $\mu$ be Lebesgue measure on $[0,1]$.
Write $\eusm L^1_{\Sigma_L}$ for the
linear space of integrable functions $f:[0,1]\to\Bbb R$.   Show that there
is no operator $T:L^1(\mu)\to\eusm L^1_{\Sigma_L}$ such that (i)
$(Tu)^{\ssbullet}=u$ for every $u\in L^1(\mu)$ (ii) $Tu\ge Tv$ whenever
$u\ge v$ in $L^1(\mu)$.
\Hint{Let $F\subseteq\eusm L^1_{\Sigma_L}$ be the countable set
$\{n\chi[2^{-n}k,2^{-n}(k+1)]:n\in\Bbb N$, $k<2^n\}$.
Show that if $T$
satisfies (i) then there is an $x\in\{0,1\}^{\Bbb N}$ such that
$T(f^{\ssbullet})(x)=f(x)$ for every $f\in F$;  find a sequence
$\sequencen{f_n}$ in $F$ such that $\{f_n^{\ssbullet}:n\in\Bbb N\}$ is
bounded above in $L^1(\mu)$ but $\sup_{n\in\Bbb N}f_n(x)=\infty$.}
%341Xf

\leader{341Y}{Further exercises (a)} Let $X$ be a set, $\Sigma$ an
algebra of subsets of $X$ and $\Cal I$ an ideal of $\Sigma$;  let
$\frak A$ be the quotient Boolean algebra $\Sigma/\Cal I$.   We say that
a function $\theta:\frak A\to\Sigma$ is a {\bf lifting} if it is a
Boolean homomorphism and $(\theta a)^{\ssbullet}=a$ for every
$a\in\frak A$, and that $\undtheta:\frak A\to\Sigma$ is a {\bf lower density} if $\undtheta 0=\emptyset$,
$\undtheta(a\Bcap b)=\undtheta a\cap\undtheta b$ for all $a$,
$b\in\frak A$, and $(\undtheta a)^{\ssbullet}=a$ for
every $a\in\frak A$.

Show that if $(X,\Sigma,\Cal I)$ is `complete' in the sense that
$F\in\Sigma$ whenever $F\subseteq E\in\Cal I$, and if $X\notin\Cal I$,
and $\undtheta:\frak A\to\Sigma$ is a lower density, then there is a
lifting $\theta:\frak A\to\Sigma$ such that
$\undtheta a\subseteq\theta a$ for every $a\in\frak A$.
%341J

\spheader 341Yb Let $X$ be a Baire space, $\widehat{\Cal B}$
Baire-property algebra of $X$ (314Yd)
and $\Cal M$ the ideal of meager subsets of $X$.   Show that there is a
lifting $\theta$ from $\widehat{\Cal B}/\Cal M$ to $\widehat{\Cal B}$
such that $\theta G^{\ssbullet}\supseteq G$ for every open
$G\subseteq X$.   \Hint{in
341Ya, set $\undtheta(G^{\ssbullet})=G$ for every regular open set
$G$.}
%341Ya 341J

\spheader 341Yc Let $(X,\Sigma,\mu)$ be a \Mth\
probability space with Maharam type $\kappa\ge\omega$.    Let
$\CalBa_{\kappa}$
be the Baire $\sigma$-algebra of $Y=\{0,1\}^{\kappa}$, that is, the
$\sigma$-algebra of subsets of $Y$ generated by the family
$\{\{x:x(\xi)=1\}:\xi<\kappa\}$, and let $\nu$ be the restriction to
$\CalBa_{\kappa}$ of the usual
measure on $\{0,1\}^{\kappa}$.  Show that there is an
inverse-measure-preserving function $f:X\to Y$ which induces an
isomorphism between the measure algebras of $\mu$ and $\nu$.

\spheader 341Yd Let $(X,\Sigma,\mu)$ be a complete \Mth\ probability
space with Maharam type $\kappa\ge\omega$, and give
$Y=\{0,1\}^{\kappa}$ its usual measure $\nu_{\kappa}$.  Show that
there is an
inverse-measure-preserving function $f:X\to Y$ which induces an
isomorphism between the measure algebras of $\mu$ and $\nu_{\kappa}$.
%341Yc

\header{341Ye}{\bf *(e)} Give an example of a
complete probability space $(X,\Sigma,\mu)$,
a subalgebra $\Tau$ of $\Sigma$, and a partial lower density
$\undphi:\Tau\to\Sigma$ which has no extension to a lower density for
$\mu$.   \Hint{There is a subset of $\{0,1\}^{\frakc}$, of cardinal
$\frak c$, which is non-negligible for the usual measure on
$\{0,1\}^{\frakc}$.}
%341N mt34bits
}%end of exercises


\leader{341Z}{Problems (a)} Can we construct, using the ordinary axioms
of mathematics\cmmnt{ (including the axiom of choice, but not the
continuum
hypothesis)}, a probability space $(X,\Sigma,\mu)$ with no lifting?

\header{341Zb}{\bf (b)} Set $\kappa=\omega_3$.  \cmmnt{(There is a
reason for taking $\omega_3$ here;  see 535E in Volume 5.)}
Let $\CalBa_{\kappa}$ be the Baire $\sigma$-algebra of
$\{0,1\}^{\kappa}$\cmmnt{ (as in 341Yc)}, and $\mu$ the
restriction to $\CalBa_{\kappa}$ of the usual measure on
$\{0,1\}^{\kappa}$.   Can we show that $\mu$ has no lifting?

\cmmnt{
\Notesheader{341} Innumerable variations of the proof of 341K have been
devised, as each author has struggled with the technical complications.
I have discussed the reasons for my own choices in 341L.

The theorem has a curious history.   It was originally announced by von
Neumann, but he seems never to have written his proof down, and the
first published proof is that of {\smc Maharam 58}.   That argument is
based on Maharam's theorem, 341Xe and 341Yd, which show that it is
enough to find liftings for every $\{0,1\}^{\kappa}$;  this requires
most of the ideas presented above, but feels more concrete, and some of
the details are slightly simpler.   The argument as I have written it
owes a great deal to {\smc Ionescu Tulcea \& Ionescu Tulcea 69}.

The lifting theorem and Maharam's theorem are the twin pillars of modern
abstract measure theory.   But there remains a degree of mystery about
the lifting theorem which is absent from the other.   The first point is
that there is nothing canonical about the liftings we can construct,
except in the quite exceptional case of Stone spaces (341O).   Even
when there is a more or less canonical lower density present (341E,
341Xb), the conversion of this into a lifting requires arbitrary
choices, as in 341J.   While we can distinguish some liftings as being
somewhat more regular than others, I know of no criterion which marks
out any particular lifting for Lebesgue measure, for instance, among the
rest.   Perhaps associated with this arbitrariness is the extreme
difficulty of deciding whether liftings of any given type exist.
Neither positive nor negative results are easily come by (I will present
a few in the later sections of this chapter), and the nature of the
obstacles remains quite unclear.
}

\discrpage
