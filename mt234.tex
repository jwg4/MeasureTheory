\frfilename{mt234.tex}
\versiondate{11.4.09}
\copyrightdate{2008}

\def\chaptername{The Radon-Nikod\'ym theorem}
\def\sectionname{Operations on measures}

\newsection{234}

I take a few pages to describe some standard constructions.   The ideas are
straightforward, but a number of details need to be worked
out if they are to be securely integrated into the general framework I
employ.   The first step is to formally introduce \imp\ functions
(234A-234B), the most important class of transformations between measure
spaces.   For construction of new measures, we have the notions of
image measure (234C-234E), %234C 234D 234E
sum of measures (234G-234H) and indefinite-integral measure
(234I-234O).  %234I 234J 234K 234L 234M 234N 234O
Finally I mention a way of ordering the measures on a given set
(234P-234Q).


\leader{234A}{Inverse-measure-preserving
functions}\dvAformerly{2{}35G}\cmmnt{ It is high
time that I introduced the nearest thing in measure theory to a
`morphism'.}   If $(X,\Sigma,\mu)$ and $(Y,\Tau,\nu)$ are measure
spaces, a function $\phi:X\to Y$ is {\bf \imp} if
$\phi^{-1}[F]\in\Sigma$ and $\mu(\phi^{-1}[F])=\nu F$ for every
$F\in\Tau$.

\leader{234B}{Proposition} Let $(X,\Sigma,\mu)$ and $(Y,\Tau,\nu)$ be
measure spaces, and $\phi:X\to Y$ an \imp\ function.


(a)\dvAformerly{2{}35Hc} If $\hat\mu$, $\hat\nu$ are the completions of 
$\mu$, $\nu$
respectively, $\phi$ is also \imp\ for $\hat\mu$ and $\hat\nu$.

(b)\dvAformerly{2{}35Xe} $\mu$ is a probability measure iff $\nu$ is a probability measure.

(c)\dvAformerly{2{}35Xe}
 $\mu$ is totally finite iff $\nu$ is totally finite.

(d)\dvAformerly{2{}35Xe}(i)
If $\nu$ is $\sigma$-finite, then $\mu$ is $\sigma$-finite.

\quad(ii) If $\nu$ is semi-finite and $\mu$ is $\sigma$-finite, then $\nu$
is $\sigma$-finite.

(e)\dvAformerly{2{}35Xe}(i)
If $\nu$ is $\sigma$-finite and atomless, then $\mu$ is atomless.

\quad(ii) If $\nu$ is semi-finite and $\mu$ is purely atomic, then
$\nu$ is purely atomic.

(f)\dvAnew{2008}(i) $\mu^*\phi^{-1}[B]\le\nu^*B$ for
every $B\subseteq Y$.

\quad(ii) $\mu^*A\le\nu^*\phi[A]$ for every $A\subseteq X$.

(g)\dvAformerly{2{}35Hc}
 If $(Z,\Lambda,\lambda)$ is another measure space, and
$\psi:Y\to Z$ is \imp,  then $\psi\phi:X\to Z$ is \imp.

\proof{{\bf (a)} If $\hat\nu$ measures $F$, there are $F'$, $F''\in\Tau$
such that $F'\subseteq F\subseteq F''$ and $\nu(F''\setminus F')=0$.
Now

\Centerline{$\phi^{-1}[F']\subseteq\phi^{-1}[F]\subseteq\phi^{-1}[F'']$,
\quad$\mu(\phi^{-1}[F'']\setminus\phi^{-1}[F'])=\nu(F''\setminus F')=0$,}

\noindent so $\hat\mu$ measures $\phi^{-1}[F]$ and

\Centerline{$\hat\mu(\phi^{-1}[F])=\mu\phi^{-1}[F']=\nu F'=\hat\nu F$.}

\noindent As $F$ is arbitrary, $\phi$ is \imp\ for $\hat\mu$ and $\hat\nu$.

\medskip

{\bf (b)-(c)} are surely obvious.

\medskip

{\bf (d)(i)} If $\sequencen{F_n}$ is a cover of $Y$ by sets of finite
measure for $\nu$, then $\sequencen{\phi^{-1}[F_n]}$ is a cover of $X$ by
sets of finite measure for $\mu$.

\medskip

\quad{\bf (ii)} Let $\Cal F\subseteq\Tau$ be a disjoint family of
non-$\nu$-negligible sets.   Then $\family{F}{\Cal F}{\phi^{-1}[F]}$ is a
disjoint family of non-$\mu$-negligible sets.   By 215B(iii), $\Cal F$ is
countable.   By 215B(iii) in the opposite direction, $\nu$ is
$\sigma$-finite.

\medskip

{\bf (e)(i)} Suppose that $E\in\Sigma$ and $\mu E>0$.   Let
$\sequencen{F_n}$ be a cover of $Y$ by sets of finite measure for $\nu$.
Because $\nu$ is atomless, we can find, for each $n$, a
finite partition $\family{i}{I_n}{F_{ni}}$
of $F_n$ such that $\nu F_{ni}<\mu E$ for every $i\in I_n$ (use 215D
repeatedly).   Now
$X=\bigcup_{n\in\Bbb N,i\in I_n}\phi^{-1}[F_{ni}]$, so there are
$n\in\Bbb N$ and $i\in I_n$ with

\Centerline{$0<\mu(E\cap\phi^{-1}[F_{ni})
\le\mu\phi^{-1}[F_{ni}]=\nu F_{ni}<\mu E$,}

\noindent and $E$ is not a $\mu$-atom.   As $E$ is arbitrary, $\mu$ is
atomless.

\medskip

\quad{\bf (ii)} Suppose that $F\in\Tau$ and $\nu F>0$.   Because $\nu$ is
semi-finite, there is an $F_1\subseteq F$ such that $0<\nu F_1<\infty$.
Now $\mu\phi^{-1}[F_1]>0$;  because $\mu$ is purely atomic, there is a
$\mu$-atom $E\subseteq\phi^{-1}[F_1]$.

Let $\Cal G$ be the set of those $G\in\Tau$ such that $G\subseteq F_1$ and
$\mu(E\cap\phi^{-1}[G])=0$.   Then the union of any sequence in $\Cal G$
belongs to $\Cal G$, so by 215Ac there is an $H\in\Cal G$ such that
$\nu(G\setminus H)=0$ whenever $G\in\Cal G$.   Consider $F_1\setminus H$.
We have

\Centerline{$\nu(F_1\setminus H)
=\mu(\phi^{-1}[F_1]\setminus\phi^{-1}[H])
\ge\mu(E\setminus\phi^{-1}[H])
=\mu E>0$.}

\noindent If $G\in\Tau$ and $G\subseteq F_1\setminus H$,
then one of $E\cap\phi^{-1}[G]$, $E\setminus\phi^{-1}[G]$ is
$\mu$-negligible.
In the former case, $G\in\Cal G$ and $G=G\setminus H$ is $\nu$-negligible.
In the
latter case, $F_1\setminus G\in\Cal G$ and $(F_1\setminus H)\setminus G$ is
$\nu$-negligible.   As $G$ is arbitrary, $F_1\setminus H$ is a $\nu$-atom
included in $F$;  as $F$ is arbitrary, $\nu$ is purely atomic.

\medskip

{\bf (f)(i)} Let $F\in\Tau$ be such that $B\subseteq F$ and $\nu^*B=\nu F$
(132Aa);  then $\phi^{-1}[B]\subseteq\phi^{-1}[F]$ so

\Centerline{$\mu^*\phi^{-1}[B]\le\mu\phi^{-1}[F]=\nu F=\nu^*B$.}

\medskip

\quad{\bf (ii)} $\mu^*A\le\mu^*(\phi^{-1}[\phi[A]])\le\nu^*\phi[A]$ by (i).

\medskip

{\bf (g)} For any $W\in\Lambda$,

\Centerline{$\mu(\psi\phi)^{-1}[W]=\mu\phi^{-1}[\psi^{-1}[W]]
=\nu\psi^{-1}[W]=\lambda W$.}
}%end of proof of 234B


\leader{234C}{Image \dvrocolon{measures}}\dvAformerly{1{}12E}\cmmnt{ The
following
construction is one of the commonest ways in which new measure spaces
appear.

\medskip

\noindent}{\bf Proposition} Let $(X,\Sigma,\mu)$ be a measure space, $Y$
any set, and $\phi:X\to Y$ a function.   Set

\Centerline{$\Tau=\{F:F\subseteq Y,\,\phi^{-1}[F]\in\Sigma\}$,
\quad$\nu F=\mu(\phi^{-1}[F])$ for every $F\in\Tau$.}

\noindent Then $(Y,\Tau,\nu)$ is a measure space.

\proof{{\bf (a)} $\emptyset=\phi^{-1}[\emptyset]\in\Sigma$ so
$\emptyset\in\Tau$.

\medskip

{\bf (b)} If $F\in\Tau$, then $\phi^{-1}[F]\in\Sigma$, so
$X\setminus\phi^{-1}[F]\in\Sigma$;  but
$X\setminus\phi^{-1}[F]=\phi^{-1}[Y\setminus F]$, so $Y\setminus
F\in\Tau$.

\medskip

{\bf (c)} If $\sequencen{F_n}$ is a sequence in $\Tau$,
then $\phi^{-1}[F_n]\in\Sigma$ for every $n$, so
$\bigcup_{n\in\Bbb N}\phi^{-1}[F_n]\in\Sigma$;  but
$\phi^{-1}[\bigcup_{n\in\Bbb N}F_n]=\bigcup_{n\in\Bbb N}\phi^{-1}[F_n]$,
so $\bigcup_{n\in\Bbb N}F_n\in\Tau$.

Thus $\Tau$ is a $\sigma$-algebra.

\medskip

{\bf (d)} $\nu\emptyset=\mu\phi^{-1}[\emptyset]=\mu\emptyset=0$.

\medskip

{\bf (e)} If $\sequencen{F_n}$ is a disjoint sequence in $\Tau$, then
$\sequencen{\phi^{-1}[F_n]}$ is a disjoint sequence in $\Sigma$, so

\Centerline{$\nu(\bigcup_{n\in\Bbb N}F_n)
=\mu\phi^{-1}[\bigcup_{n\in\Bbb N}F_n]
=\mu(\bigcup_{n\in\Bbb N}\phi^{-1}[F_n])
=\sum_{n=0}^{\infty}\mu\phi^{-1}[F_n]
=\sum_{n=0}^{\infty}\nu F_n$.}

\noindent So $\nu$ is a measure.

}%end of proof of 234C

\leader{234D}{Definition}\dvAformerly{1{}12F}
 In the context of 234C, $\nu$ is called
the {\bf image measure}\cmmnt{ or {\bf push-forward measure}};  
I will denote it $\mu\phi^{-1}$.

\cmmnt{\medskip

\noindent{\bf Remark} I ought perhaps to say that this construction does
not always produce exactly the `right' measure on $Y$;  there are
circumstances in which some modification of the measure $\mu\phi^{-1}$
described here is more useful.   But I will note these explicitly when
they occur;  when I use the unadorned phrase `image measure' I shall
mean the measure constructed above.
}%end of comment

\leader{234E}{Proposition} Let $(X,\Sigma,\mu)$ be a measure space, $Y$ a
set and $\phi:X\to Y$ a function;  let $\mu\phi^{-1}$ be the image measure
on $Y$.

(a) $\phi$ is \imp\ for $\mu$ and $\mu\phi^{-1}$.

(b)\dvAformerly{2{}12Bd} If $\mu$ is complete, so is $\mu\phi^{-1}$.

(c)\dvAformerly{1{}12Xd} If $Z$ is another set, and $\psi:Y\to Z$ a function,
then the image measures $\mu(\psi\phi)^{-1}$ and $(\mu\phi^{-1})\psi^{-1}$
on $Z$ are the same.

\proof{{\bf (a)} Immediate from the definitions.

\medskip

{\bf (b)} Write $\nu$ for $\mu\phi^{-1}$ and $\Tau$ for its domain.   If
$\nu^*B=0$, then $\mu^*\phi^{-1}[B]=0$, by 234B(f-i);  as $\mu$ is
complete, $\phi^{-1}[B]\in\Sigma$, so $B\in\Tau$.   As $B$ is arbitrary,
$\nu$ is complete.

\medskip

{\bf (c)} For $G\subseteq Z$ and $u\in[0,\infty]$,

$$\eqalign{(\mu(\psi\phi)^{-1})(G)&\text{ is defined and equal to }u\cr
&\iff\mu((\psi\phi)^{-1}[G])\text{ is defined and equal to }u\cr
&\iff\mu(\phi^{-1}[\psi^{-1}[G]])\text{ is defined and equal to }u\cr
&\iff(\mu\phi^{-1})(\psi^{-1}[G]))\text{ is defined and equal to }u\cr
&\iff((\mu\phi^{-1})\psi^{-1})(G)\text{ is defined and equal to }u.\cr}$$

}%end of proof of 234E


\leader{*234F}{}\dvAformerly{1{}32G}\cmmnt{ In the opposite direction, the
following construction of a pull-back measure is sometimes useful.

\medskip

\noindent}{\bf Proposition} Let $X$ be a set, $(Y,\Tau,\nu)$ a measure
space, and
$\phi:X\to Y$ a function such that $\phi [X]$ has full outer measure in $Y$.
Then there is a measure $\mu$ on $X$, with domain
$\Sigma=\{\phi^{-1}[F]:F\in\Tau\}$, such that $\phi$ is \imp\ for $\mu$ and
$\nu$.

\proof{ The check that $\Sigma$ is a $\sigma$-algebra of subsets of
$X$ is straightforward;  all we need to know is that
$\phi^{-1}[\emptyset]=\emptyset$,
$X\setminus\phi^{-1}[F]=\phi^{-1}[Y\setminus F]$ for every $F\subseteq Y$,
and that
$\phi^{-1}[\bigcup_{n\in\Bbb N}F_n]=\bigcup_{n\in\Bbb N}\phi^{-1}[F_n]$ for
every sequence $\sequencen{F_n}$ of subsets of $Y$.   The key fact is
that if $F_1$, $F_2\in\Tau$ and $\phi^{-1}[F_1]=\phi^{-1}[F_2]$,
then $\phi[X]$ does not meet $F_1\symmdiff F_2$;  because $\phi[X]$ has full
outer measure, $F_1\symmdiff F_2$ is $\nu$-negligible and
$\nu F_1=\nu F_2$.   Accordingly the formula $\mu\phi^{-1}[F]=\nu F$
does define a function $\mu:\Sigma\to[0,\infty]$.   Now

\Centerline{$\mu\emptyset=\mu\phi^{-1}[\emptyset]=\nu\emptyset=0$.}

\noindent Next, if $\sequencen{E_n}$ is a disjoint sequence in $\Sigma$,
choose $F_n\in\Tau$ such that $E_n=\phi^{-1}[F_n]$ for each $n\in\Bbb N$.
The sequence $\sequencen{F_n}$ need not be disjoint, but if we set
$F'_n=F_n\setminus\bigcup_{i<n}F_i$ for each $n\in\Bbb N$, then
$\sequencen{F'_n}$ is disjoint and

\Centerline{$E_n=E_n\setminus\bigcup_{i<n}E_i=\phi^{-1}[F'_n]$}

\noindent for each $n$;  so

\Centerline{$\mu(\bigcup_{n\in\Bbb N}E_n)
=\nu(\bigcup_{n\in\Bbb N}F'_n)
=\sum_{n=0}^{\infty}\nu F'_n
=\sum_{n=0}^{\infty}\mu E_n$.}

\noindent As $\sequencen{E_n}$ is arbitrary, $\mu$ is a measure on
$X$, as required.
}%end of proof of 234F


\leader{234G}{Sums of \dvrocolon{measures}}\dvAformerly{1{}12Ya;  
1{}12Xe for sum of two measures}\cmmnt{ I come now to a quite
different way of building measures.   The idea is an obvious one, but the
technical details, in the general case I wish to examine, need watching.

\medskip

\noindent}{\bf Proposition}
Let $X$ be a set, and $\familyiI{\mu_i}$ a family of measures on $X$.
For each $i\in I$, let $\Sigma_i$ be the
domain of $\mu_i$.   Set $\Sigma=\Cal PX\cap\bigcap_{i\in I}\Sigma_i$ and
define $\mu:\Sigma\to[0,\infty]$ by setting $\mu E=\sum_{i\in I}\mu_iE$ for
every $E\in\Sigma$.   Then $\mu$ is a measure on $X$.

\proof{ $\Sigma$ is a $\sigma$-algebra of subsets of $X$ because every
$\Sigma_i$ is.   (Apply 111Ga with
$\frak S=\{\Sigma_i:i\in I\}\cup\{\Cal PX\}$.)
Of course $\mu$ takes values in $[0,\infty]$ (226A).
$\mu\emptyset=0$ because $\mu_i\emptyset=0$ for every $i$.   If
$\sequencen{E_n}$ is a disjoint sequence in $\Sigma$ with union $E$, then

$$\eqalignno{\mu E
&=\sum_{i\in I}\mu_iE
=\sum_{i\in I}\sum_{n=0}^{\infty}\mu_iE_n
=\sum_{n=0}^{\infty}\sum_{i\in I}\mu_iE_n\cr
\displaycause{226Af}
&=\sum_{n=0}^{\infty}\mu E_n.\cr}$$

\noindent So $\mu$ is a measure.
}%end of proof of 234G

\medskip

\noindent{\bf Remark} In this context, I will call $\mu$ the {\bf sum} of
the family $\familyiI{\mu_i}$.

\leader{234H}{Proposition}\dvAnew{2008}
 Let $X$ be a set and $\familyiI{\mu_i}$ a family
of complete measures on $X$ with sum $\mu$.

(a) $\mu$ is complete.

(b)(i) A subset of $X$ is $\mu$-negligible iff it is $\mu_i$-negligible for
every $i\in I$.

\quad(ii) A subset of $X$ is $\mu$-conegligible iff it is
$\mu_i$-conegligible for every $i\in I$.

(c)\dvAformerly{2{}12Xh, for sum of two measures}
 Let $f$ be a function from a subset of $X$ to $[-\infty,\infty]$.
Then $\int fd\mu$ is defined in $[-\infty,\infty]$ iff $\int fd\mu_i$ is
defined in $[-\infty,\infty]$ for every $i$ and one of
$\sum_{i\in I}f^+d\mu_i$, $\sum_{i\in I}f^-d\mu_i$ is finite, and in this
case $\int fd\mu=\sum_{i\in I}\int fd\mu_i$.

\proof{ Write $\Sigma_i=\dom\mu_i$ for $i\in I$,
$\Sigma=\Cal PX\cap\bigcap_{i\in I}\Sigma_i=\dom\mu$.

\medskip

{\bf (a)} If $E\subseteq F\in\Sigma$ and $\mu F=0$, then $\mu_iF=0$ for
every $i\in I$;  because $\mu_i$ is complete, $E_i\in\Sigma_i$ for
every $i\in I$, and $E\in\Sigma$.

\medskip

{\bf (b)} This now follows at once, since a set $A\subseteq X$ is
$\mu$-negligible iff $\mu A=0$.

\medskip

{\bf (c)(i)} Note first that (b-ii) tells us that, under either hypothesis,
$\dom f$ is conegligible both for $\mu$ and for every $\mu_i$, so that
if we extend $f$ to $X$ by giving it the value $0$ on $X\setminus\dom f$
then neither $\int fd\mu$ nor $\sum_{i\in I}\int fd\mu_i$ is affected.   So
let us assume from now on that $f$ is defined everywhere on $X$.   Now it
is plain that
either hypothesis ensures that $f$ is $\Sigma$-measurable, that is, is
$\Sigma_i$-measurable for every $i\in I$.

\medskip

\quad{\bf (ii)} Suppose that $f$ is non-negative.
For $n\in\Bbb N$ set
$f_n(x)=\sum_{k=1}^{4^n}2^{-n}\chi\{x:f(x)\ge 2^{-n}k\}$, so that
$\sequencen{f_n}$ is a non-decreasing sequence with supremum $f$.  We have

$$\eqalign{\int f_nd\mu
&=\sum_{k=1}^{4^n}2^{-n}\mu\{x:f(x)\ge 2^{-n}k\}
=\sum_{k=1}^{4^n}\sum_{i\in I}2^{-n}\mu_i\{x:f(x)\ge 2^{-n}k\}\cr
&=\sum_{i\in I}\sum_{k=1}^{4^n}2^{-n}\mu_i\{x:f(x)\ge 2^{-n}k\}
=\sum_{i\in I}\int f_nd\mu_i\cr}$$

\noindent for every $n$, so

$$\eqalign{\int fd\mu
&=\sup_{n\in\Bbb N}\int f_nd\mu
=\sup_{n\in\Bbb N}\sup_{J\subseteq I\text{ is finite}}
   \sum_{i\in J}\int f_nd\mu_i
=\sup_{J\subseteq I\text{ is finite}}\sup_{n\in\Bbb N}
   \sum_{i\in J}\int f_nd\mu_i\cr
&=\sup_{J\subseteq I\text{ is finite}}
   \lim_{n\to\infty}\sum_{i\in J}\int f_nd\mu_i
=\sup_{J\subseteq I\text{ is finite}}
   \sum_{i\in J}\lim_{n\to\infty}\int f_nd\mu_i
=\sum_{i\in I}\int fd\mu_i.\cr}$$

\medskip

\quad{\bf (iii)} Generally,

$$\eqalign{\int fd\mu&\text{ is defined in }[\infty,\infty]\cr
&\iff\int f^+d\mu\text{ and }\int f^-d\mu
  \text{ are defined and at most one is infinite}\cr
&\iff\sum_{i\in I}\int f^+d\mu_i\text{ and }\sum_{i\in I}\int f^-d\mu_i
  \text{ are defined and at most one is infinite}\cr
&\iff\int fd\mu_i\text{ is defined for every }i
  \text{ and at most one of }\sum_{i\in I}\int f^+d\mu_i,\cr
&\mskip400mu
  \sum_{i\in I}\int f^-d\mu_i\text{ is infinite},\cr}$$

\noindent and in this case

\Centerline{$\int fd\mu
=\int f^+d\mu-\int f^-d\mu
=\sum_{i\in I}\int f^+d\mu_i-\sum_{i\in I}\int f^-d\mu_i
=\sum_{i\in I}\int fd\mu_i$.}
}%end of proof of 234H

\leader{234I}{Indefinite-integral
\dvrocolon{measures}}\dvAformerly{2{}34A}\cmmnt{ Extending an
idea already used in 232D, we are led to the following construction;  once
again, we need to take care over the formal details if we want
to get full value from it.

\medskip

\noindent}{\bf Theorem} Let $(X,\Sigma,\mu)$ be a measure
space, and $f$ a non-negative $\mu$-virtually measurable real-valued
function defined on a conegligible subset of $X$.   Write
$\nu F=\int f\times\chi F\,d\mu$ whenever $F\subseteq X$ is such that
the integral is
defined in $[0,\infty]$\cmmnt{ according to the conventions of 133A}.
Then $\nu$ is a complete measure on $X$, and its domain includes
$\Sigma$.
%the trouble with taking [0,\infty]-valued functions is that if
%I allow  f=\infty  on a non-negligible set then  \nu  becomes
%non-semi-finite for a silly reason

\proof{{\bf (a)} Write $\Tau$ for the domain of $\nu$, that is, the
family of sets $F\subseteq X$ such that $\int f\times\chi F\,d\mu$ is
defined in
$[0,\infty]$, that is, $f\times\chi F$ is $\mu$-virtually measurable
(133A).   Then $\Tau$ is a $\sigma$-algebra of subsets of $X$.   \Prf\
For each $F\in\Tau$ let $H_F\subseteq X$ be a $\mu$-conegligible set
such that $f\times\chi F\restr H_F$ is $\Sigma$-measurable.   Because
$f$ itself is $\mu$-virtually measurable, $X\in\Tau$.   If $F\in\Tau$,
then

\Centerline{$f\times\chi(X\setminus F)\restr(H_X\cap H_F)
=f\restr(H_X\cap H_F)-(f\times\chi F)\restr(H_X\cap H_F)$}

\noindent is $\Sigma$-measurable, while $H_X\cap H_F$ is
$\mu$-conegligible, so $X\setminus F\in\Tau$.   If $\sequencen{F_n}$ is
a sequence in $\Tau$ with union $F$, set
$H=\bigcap_{n\in\Bbb N}H_{F_n}$;
then $H$ is conegligible, $f\times\chi F_n\restr H$ is $\Sigma$-measurable
for every $n\in\Bbb N$, and
$f\times\chi F=\sup_{n\in\Bbb N}f\times\chi F_n$, so
$f\times\chi F\restr H$ is $\Sigma$-measurable, and $F\in\Tau$.   Thus
$\Tau$ is a $\sigma$-algebra.   If $F\in\Sigma$, then
$f\times\chi F\restr H_X$ is $\Sigma$-measurable, so $F\in\Tau$.\ \Qed

\medskip

{\bf (b)} Next, $\nu$ is a measure.   \Prf\ Of course
$\nu F\in[0,\infty]$
for every $F\in\Tau$.   $f\times\chi\emptyset=0$ wherever it is defined,
so $\nu\emptyset=0$.   If $\sequencen{F_n}$ is a disjoint sequence in
$\Tau$ with union $F$, then
$f\times\chi F=\sum_{n=0}^{\infty}f\times\chi F_n$.
If $\nu F_m=\infty$ for some $m$, then we surely have
$\nu F=\infty=\sum_{n=0}^{\infty}\nu F_n$.   If $\nu F_m<\infty$ for
each $m$ but $\sum_{n=0}^{\infty}\nu F_n=\infty$, then

\Centerline{$\int f\times\chi(\bigcup_{n\le m}F_n)
=\sum_{n=0}^m\int f\times\chi F_n\to\infty$}

\noindent as $m\to\infty$, so again
$\nu F=\infty=\sum_{n=0}^{\infty}\nu F_n$.   If
$\sum_{n=0}^{\infty}\nu F_n<\infty$ then by B.Levi's theorem

\Centerline{$\nu F=\int\sum_{n=0}^{\infty}f\times\chi F_n
=\sum_{n=0}^{\infty}\int f\times\chi F_n
=\sum_{n=0}^{\infty}\nu F_n$.
\Qed}

\medskip

{\bf (c)} Finally, $\nu$ is complete.   \Prf\ If $A\subseteq F\in\Tau$
and $\nu F=0$, then $f\times\chi F=0$ a.e., so $f\times\chi A=0$ a.e.\
and $\nu A$ is defined and equal to zero.\ \Qed
}%end of proof of 234I

\leader{234J}{Definition}\dvAformerly{2{}34B}
Let $(X,\Sigma,\mu)$ be a measure space, and
$\nu$ another measure on $X$ with domain $\Tau$.   I will call $\nu$ an
{\bf indefinite-integral measure} over $\mu$, or sometimes a {\bf
completed indefinite-integral measure}, if it can be obtained by the
method of 234I
from some non-negative virtually measurable function $f$ defined almost
everywhere on $X$.   In this case, $f$ is a Radon-Nikod\'ym derivative 
of $\nu$ with respect to $\mu$ in the sense of 232Hf.
\cmmnt{As in 232Hf, the phrase {\bf density function} is also used in
this context.}

\leader{234K}{Remarks}\dvAformerly{2{}34C}
Let $(X,\Sigma,\mu)$ be a measure space, and $f$
a $\mu$-virtually measurable non-negative real-valued function defined
almost everywhere
on $X$;  let $\nu$ be the associated indefinite-integral measure.

\spheader 234Ka\cmmnt{ There is a $\Sigma$-measurable function
$g:X\to\coint{0,\infty}$ such that $f=g\,\,\mu$-a.e.   \Prf\ Let
$H\subseteq\dom f$ be a measurable conegligible set such that
$f\restr H$ is measurable, and set $g(x)=f(x)$ for $x\in H$, $g(x)=0$
for $x\in X\setminus H$.\ \QeD\    In
this case, $\int f\times\chi E\,d\mu=\int g\times\chi E\,d\mu$ if either
is defined.   So $g$ is a Radon-Nikod\'ym derivative of $\nu$, and}
$\nu$ has a
Radon-Nikod\'ym derivative which is $\Sigma$-measurable and defined
everywhere.

%includes old 2{}34Ce
\spheader 234Kb If $E$ is $\mu$-negligible, 
then\cmmnt{ $f\times\chi E=0\,\,\mu$-a.e., so} $\nu E=0$.   
\cmmnt{Many authors are prepared to say `$\nu$ is
absolutely continuous with respect to $\mu$' in this context.   But if
$\nu$ is not totally finite, it need not be
absolutely continuous in the
$\epsilon$-$\delta$ sense of 232Aa (234Xh), and
further difficulties can arise if $\mu$ or $\nu$ is not $\sigma$-finite
(see 234Yk, 234Ym).
}%end of comment

\cmmnt{\spheader 234Kc I have defined `indefinite-integral measure' in
such a way as to produce a complete measure.   In my view this is what
makes best sense in most applications.   There are occasions on which it
seems more appropriate to use the measure $\nu_0:\Sigma\to[0,\infty]$
defined by
setting $\nu_0 E=\int_Efd\mu=\int f\times\chi E\,d\mu$ for $E\in\Sigma$.
I suppose I would call this the {\bf uncompleted indefinite-integral
measure} over $\mu$ defined by $f$.   ($\nu$ is always the
completion of $\nu_0$;  see 234Lb.)
}%end of comment

\spheader 234Kd\cmmnt{ Note the way in which I formulated the
definition of $\nu$:  `$\nu E=\int f\times\chi E\,d\mu$ if the integral
is defined',
rather than `$\nu E=\int_Efd\mu$'.   The point is that the longer
formula gives a rule for deciding what the domain of $\nu$ must be.   Of
course it is the case that} $\nu E=\int_Efd\mu$ for every
$E\in\dom\nu$\cmmnt{ (apply 214F to $f\times\chi E$)}.

\spheader 234Ke\dvAnew{2008}
\cmmnt{Because }$\mu$ and its completion define the 
same\cmmnt{ virtually measurable functions, the same null ideals and the same
integrals\cmmnt{ (212Eb, 212F)},
they define the same} indefinite-integral measures.

\leader{234L}{The domain of an indefinite-integral
\dvrocolon{measure}}\dvAformerly{2{}34D}\cmmnt{ It is sometimes useful
to have an explicit description of the domain of a measure constructed
in this way.

\medskip

\noindent}{\bf Proposition} Let $(X,\Sigma,\mu)$ be a measure space, $f$
a non-negative $\mu$-virtually measurable function defined almost
everywhere
in $X$, and $\nu$ the associated indefinite-integral measure.   Set
$G=\{x:x\in\dom f$, $f(x)>0\}$, and let $\hat\Sigma$ be the domain of the
completion $\hat\mu$ of $\mu$.

(a) The domain $\Tau$ of $\nu$ is
$\{E:E\subseteq X,\,E\cap G\in\hat\Sigma\}$;  \cmmnt{in particular,}
$\Tau\supseteq\hat\Sigma\supseteq\Sigma$.

(b) $\nu$ is the completion of its restriction to $\Sigma$.

(c) A set $A\subseteq X$ is $\nu$-negligible iff $A\cap G$ is
$\mu$-negligible.

(d) In particular, if $\mu$\cmmnt{ itself} is complete, then
$\Tau=\{E:E\subseteq X,\,E\cap G\in\Sigma\}$ and $\nu A=0$ iff
$\mu(A\cap G)=0$.

\proof{{\bf (a)(i)} If $E\in\Tau$, then $f\times\chi E$ is virtually
measurable, so there is a conegligible measurable set $H\subseteq\dom f$
such that $f\times\chi E\restr H$ is measurable.   Now $E\cap G\cap
H=\{x:x\in H,\,(f\times\chi E)(x)>0\}$ must belong to $\Sigma$, while
$E\cap G\setminus H$ is negligible, so belongs to $\hat\Sigma$, and
$E\cap G\in\hat\Sigma$.

\medskip

\quad{\bf (ii)} If $E\cap G\in\hat\Sigma$, let $F_1$, $F_2\in\Sigma$ be
such that $F_1\subseteq E\cap G\subseteq F_2$ and $F_2\setminus F_1$ is
negligible.   Let $H\subseteq\dom f$ be a conegligible set such that
$f\restr H$ is measurable.   Then $H'=H\setminus(F_2\setminus F_1)$ is
conegligible and $f\times\chi E\restr H'=f\times\chi F_1\restr H'$ is
measurable, so $f\times\chi E$ is virtually measurable and $E\in\Tau$.

\medskip

{\bf (b)} Thus the given formula does indeed describe $\Tau$.   If
$E\in\Tau$, let $F_1$, $F_2\in\Sigma$ be such that
$F_1\subseteq E\cap G\subseteq F_2$ and $\mu(F_2\setminus F_1)=0$.
Because $G$ itself also
belongs to $\hat\Sigma$, there are $G_1$, $G_2\in\Sigma$ such that
$G_1\subseteq G\subseteq G_2$ and $\mu(G_2\setminus G_1)=0$.   Set
$F_2'=F_2\cup(X\setminus G_1)$.   Then $F_2'\in\Sigma$ and
$F_1\subseteq E\subseteq F'_2$.   But
$(F'_2\setminus F_1)\cap G
\subseteq(G_2\setminus G_1)\cup(F_2\setminus F_1)$ is $\mu$-negligible, so
$\nu(F'_2\setminus F_1)=0$.

This shows that if $\nuprime$ is the completion of $\nu\restr\Sigma$
and $\Tau'$ is its domain, then $\Tau\subseteq\Tau'$.   But as
$\nu$ is complete, it surely extends $\nuprime$, so $\nu=\nuprime$, as
claimed.

\medskip

{\bf (c)} Now take any $A\subseteq X$.   Because $\nu$ is complete,

$$\eqalign{A\text{ is }\nu\text{-negligible}
&\iff\nu A=0\cr
&\iff\int f\times\chi A\,d\mu=0\cr
&\iff f\times\chi A=0\,\,\mu\text{-a.e.}\cr
&\iff A\cap G\text{ is }\mu\text{-negligible.}\cr}$$

\medskip

{\bf (d)} This is just a restatement of (a) and (c) when $\mu=\hat\mu$.
}%end of proof of 234L

\leader{234M}{Corollary}\dvAformerly{2{}34E}
 If $(X,\Sigma,\mu)$ is a complete measure
space and $G\in\Sigma$, then the indefinite-integral measure over $\mu$
defined by $\chi G$ is just the measure $\mu\LLcorner G$ defined by
setting

\Centerline{$(\mu\LLcorner G)(F)=\mu(F\cap G)$ whenever $F\subseteq X$
and $F\cap G\in\Sigma$.}

\proof{ 234Ld.
}%end of proof of 234M

\leader{*234N}{}\cmmnt{ The next two results will not be relied on in
this
volume, but I include them for future reference, and to give an idea of
the scope of these ideas.

\medskip

\noindent}{\bf Proposition}\dvAformerly{2{}34F}
Let $(X,\Sigma,\mu)$ be a measure space, and
$\nu$ an indefinite-integral measure over $\mu$.

(a) If $\mu$ is semi-finite, so is $\nu$.

(b) If $\mu$ is complete and locally determined, so is $\nu$.

(c) If $\mu$ is localizable, so is $\nu$.

(d) If $\mu$ is strictly localizable, so is $\nu$.

(e) If $\mu$ is $\sigma$-finite, so is $\nu$.

(f) If $\mu$ is atomless, so is $\nu$.

\proof{ By 234Ka, we may express $\nu$ as the indefinite integral of a
$\Sigma$-measurable function $f:X\to\coint{0,\infty}$.   Let $\Tau$ be
the domain of $\nu$, and $\hat\Sigma$ the domain of the completion
$\hat\mu$ of
$\mu$;  set $G=\{x:x\in X,\,f(x)>0\}\in\Sigma$.

\medskip

{\bf (a)} Suppose that $E\in\Tau$ and that $\nu E=\infty$.   Then
$E\cap G$ cannot be $\mu$-negligible.
Because $\mu$ is semi-finite, there is a non-negligible $F\in\Sigma$
such that $F\subseteq E\cap G$ and $\mu F<\infty$.   Now
$F=\bigcup_{n\in\Bbb N}\{x:x\in F$, $2^{-n}\le f(x)\le n\}$, so there
is an $n\in\Bbb N$ such
that $F'=\{x:x\in F$, $2^{-n}\le f(x)\le n\}$ is non-negligible.
Because $f$ is measurable, $F'\in\Sigma\subseteq\Tau$ and
$2^{-n}\mu F'\le\nu F'\le n\mu F'$.   Thus we have found an
$F'\subseteq E$ such
that $0<\nu F'<\infty$.   As $E$ is arbitrary, $\nu$ is semi-finite.

\medskip

{\bf (b)} We already know that $\nu$ is complete (234Lb) and semi-finite.   Now
suppose that $E\subseteq X$ is such that $E\cap F\in\Tau$, that is,
$E\cap F\cap G\in\Sigma$ (234Ld), whenever $F\in\Tau$ and $\nu F<\infty$.   Then
$E\cap G\cap F\in\Sigma$ whenever
$F\in\Sigma$ and $\mu F<\infty$.   \Prf\ Set
$F_n=\{x:x\in F\cap G,\,f(x)\le n\}$.   Then
$\nu F_n\le n\mu F<\infty$, so $E\cap G\cap F_n\in\Sigma$ for every $n$.
But this means that
$E\cap G\cap F=\bigcup_{n\in\Bbb N}E\cap G\cap F_n\in\Sigma$.\ \QeD\   
Because $\mu$ is locally determined, $E\cap G\in\Sigma$ and $E\in\Tau$.   
As $E$ is arbitrary, $\nu$ is locally determined.

\medskip

{\bf (c)} Let $\Cal F\subseteq\Tau$ be any set.   Set
$\Cal E=\{F\cap G:F\in\Cal F\}$, so that $\Cal E\subseteq\hat\Sigma$.   By
212Ga, $\hat\mu$
is localizable, so $\Cal E$ has an essential supremum $H\in\hat\Sigma$.
But now, for any $H'\in\Tau$,
$H'\cup(X\setminus G)=(H'\cap G)\cup(X\setminus G)$ belongs to
$\hat\Sigma$, so

$$\eqalign{\nu(F\setminus H')&=0\text{ for every }F\in\Cal F\cr
&\iff\hat\mu(F\cap G\setminus H')=0\text{ for every }F\in\Cal F\cr
&\iff\hat\mu(E\setminus H')=0\text{ for every }E\in\Cal E\cr
&\iff\hat\mu(E\setminus(H'\cup(X\setminus G)))=0
  \text{ for every }E\in\Cal E\cr
&\iff\hat\mu(H\setminus((H'\cup(X\setminus G)))=0\cr
&\iff\hat\mu(H\cap G\setminus H')=0\cr
&\iff\nu(H\setminus H')=0.\cr}$$

\noindent Thus $H$ is also an essential supremum of $\Cal F$ in $\Tau$.
As $\Cal F$ is arbitrary, $\nu$ is localizable.

\medskip

{\bf (d)} Let
$\langle X_i\rangle_{i\in I}$ be a decomposition of $X$ for $\mu$;
then it is also a decomposition for $\hat\mu$ (212Gb).
Set $F_0=X\setminus G$, $F_n=\{x:x\in G,\,n-1<f(x)\le n\}$ for $n\ge 1$.
Then $\langle X_i\cap F_n\rangle_{i\in I,n\in\Bbb N}$ is a decomposition
for $\nu$.   \Prf\ (i) $\langle X_i\rangle_{i\in I}$ and
$\langle F_n\rangle_{n\in\Bbb N}$ are partitions of $X$ into members of $\Sigma\subseteq\Tau$, so
$\langle X_i\cap F_n\rangle_{i\in I,n\in\Bbb N}$ also is.
(ii)  $\nu(X_i\cap F_0)=0$,
$\nu(X_i\cap F_n)\le n\mu X_i<\infty$ for $i\in I$, $n\ge 1$.
(iii) If $E\subseteq X$ and $E\cap X_i\cap F_n\in\Tau$ for every
$i\in I$ and $n\in\Bbb N$ then
$E\cap X_i\cap G=\bigcup_{n\in\Bbb N}E\cap X_i\cap F_n\cap G$ belongs to
$\hat\Sigma$ for every $i$, so $E\cap G\in\hat\Sigma$ and
$E\in\Tau$.
(iv) If $E\in\Tau$, then of course

\Centerline{$\sum_{i\in I,n\in\Bbb N}\nu(E\cap X_i\cap F_n)
=\sup_{J\subseteq I\times\Bbb N\text{ is finite}}
  \sum_{(i,n)\in J}\nu(E\cap X_i\cap F_n)\le\nu E$.}

\noindent So if $\sum_{i\in I,n\in\Bbb N}\nu(E\cap X_i\cap F_n)=\infty$
it is surely equal to $\nu E$.   If the sum is finite, then
$K=\{i:i\in I,\,\nu(E\cap X_i)>0\}$ must be countable.   But for
$i\in I\setminus K$,
$\int_{E\cap X_i}fd\mu=0$, so $f=0\,\,\mu$-a.e.\ on $E\cap X_i$, that
is, $\hat\mu(E\cap G\cap X_i)=0$.   Because
$\langle X_i\rangle_{i\in I}$ is a decomposition for $\hat\mu$,
$\hat\mu(E\cap G\cap\bigcup_{i\in I\setminus K}X_i)=0$ and
$\nu(E\cap\bigcup_{i\in I\setminus K}X_i)=0$.   But this
means that

\Centerline{$\nu E=\sum_{i\in K}\nu(E\cap X_i)
=\sum_{i\in K,n\in\Bbb N}\nu(E\cap X_i\cap F_n)
=\sum_{i\in I,n\in\Bbb N}\nu(E\cap X_i\cap F_n)$.}

\noindent As $E$ is arbitrary,
$\langle X_i\cap F_n\rangle_{i\in I,n\in\Bbb N}$ is a decomposition for
$\nu$.\ \QeD\   So $\nu$ is strictly localizable.

\medskip

{\bf (e)} If $\mu$ is $\sigma$-finite, then in (d) we can take $I$ to be
countable, so that $I\times\Bbb N$ also is countable, and $\nu$ will be
$\sigma$-finite.

\medskip

{\bf (f)} If $\mu$ is atomless, so is $\hat\mu$ (212Gd).   If $E\in\Tau$
and $\nu E>0$, then $\hat\mu(E\cap G)>0$, so there is an $F\in\hat\Sigma$
such that $F\subseteq E\cap G$ and
neither $F$ nor $E\cap G\setminus F$ is $\hat\mu$-negligible.   But in this case both $\nu F=\int_Ffd\mu$ and
$\nu(E\setminus F)=\int_{E\setminus F}fd\mu$ must be greater than $0$
(122Rc).   As $E$ is arbitrary, $\nu$ is atomless.
}%end of proof of 234N

\leader{*234O}{}\cmmnt{ For localizable measures, there is a
straightforward description of the associated indefinite-integral
measures.

\medskip

\noindent}{\bf Theorem}\dvAformerly{2{}34G}
 Let $(X,\Sigma,\mu)$ be a localizable measure
space.   Then a measure $\nu$, with domain $\Tau\supseteq\Sigma$, is an
indefinite-integral measure over $\mu$ iff ($\alpha$) $\nu$ is
semi-finite
and zero on $\mu$-negligible sets ($\beta$) $\nu$ is the completion of
its restriction to $\Sigma$ ($\gamma$) whenever $\nu E>0$ there is an
$F\subseteq E$ such that $F\in\Sigma$, $\mu F<\infty$ and $\nu F>0$.

\proof{{\bf (a)} If $\nu$ is an indefinite-integral measure over $\nu$,
then by 234Na, 234Kb and 234Lb it is semi-finite, zero on
$\mu$-negligible sets and the completion of its restriction to $\Sigma$.
Now suppose that $E\in\Tau$ and $\nu E>0$.   Then there is an
$E_0\in\Sigma$
such that $E_0\subseteq E$ and $\nu E_0=\nu E$, by 234Lb.   If
$f:X\to\Bbb R$ is a $\Sigma$-measurable
Radon-Nikod\'ym derivative of $\nu$ (234Ka), and $G=\{x:f(x)>0\}$, then
$\mu(G\cap E_0)>0$;  because $\mu$ is semi-finite, there is an
$F\in\Sigma$ such that
$F\subseteq G\cap E_0$ and $0<\mu F<\infty$, in which case $\nu F>0$.

\medskip

{\bf (b)} So now suppose that $\nu$ satisfies the conditions.

\medskip

\quad{\bf (i)} Set $\Cal E=\{E:E\in\Sigma,\,\nu E<\infty\}$.   For
each $E\in\Cal E$, consider $\nu_E:\Sigma\to\Bbb R$, setting
$\nu_EG=\nu(G\cap E)$ for every $G\in\Sigma$.   Then $\nu_E$ is
countably additive and truly continuous with respect to $\mu$.   \Prf\
$\nu_E$ is countably additive, just as in 231De.   Because $\nu$ is zero
on $\mu$-negligible sets, $\nu_E$ must
be absolutely continuous with respect to $\mu$, by 232Ba.   Since
$\nu_E$ clearly satisfies condition ($\gamma$) of 232Bb,
it must be truly continuous.\ \Qed

By 232E, there is a $\mu$-integrable function $f_E$ such that
$\nu_EG=\int_Gf_Ed\mu$ for every $G\in\Sigma$, and we may suppose that
$f_E$ is $\Sigma$-measurable (232He).   Because $\nu_E$ is non-negative, $f_E\ge
0\,\,\mu$-almost everywhere.

\medskip

\quad{\bf (ii)} If $E$, $F\in\Cal E$ then $f_E=f_F\,\,\mu$-a.e. on
$E\cap F$, because

\Centerline{$\int_Gf_Ed\mu=\nu G=\int_Gf_Fd\mu$}

\noindent whenever $G\in\Sigma$ and $G\subseteq E\cap F$.   Because
$(X,\Sigma,\mu)$ is localizable, there is a measurable $f:X\to\Bbb R$
such that $f_E=f\,\,\mu$-a.e.\ on $E$ for every $E\in\Cal E$ (213N).
Because every $f_E$ is non-negative almost everywhere, we may suppose
that
$f$ is non-negative, since surely $f_E=f\vee\tbf{0}\,\,\mu$-a.e. on $E$
for
every $E\in\Cal E$.

\medskip

\quad{\bf (iii)} Let $\nuprime$ be the indefinite-integral measure
defined by $f$.   If $E\in\Cal E$ then

\Centerline{$\nu E=\int_Ef_Ed\mu=\int_Efd\mu=\nuprime E$.}
\noindent For $E\in\Sigma\setminus\Cal E$, we have

\Centerline{$\nuprime E
\ge\sup\{\nuprime F:F\in\Cal E,\,F\subseteq E\}
=\sup\{\nu F:F\in\Cal E,\,F\subseteq E\}=\nu E=\infty$}

\noindent because $\nu$ is semi-finite.   Thus $\nuprime$ and $\nu$
agree on $\Sigma$.  But since each is the completion of its restriction
to $\Sigma$, they must be equal.
}%end of proof of 234O

\leader{234P}{Ordering \dvrocolon{measures}}\cmmnt{ There are many ways in
which one measure can dominate another.   Here I will describe one of the
simplest.

\medskip

\noindent}{\bf Definition}\dvAnew{2008}
Let $\mu$, $\nu$ be two measures on a set $X$.
I will say that $\mu\le\nu$ if $\mu E$ is defined, and $\mu E\le\nu E$,
whenever $\nu$ measures $E$.

\leader{234Q}{Proposition}\dvAnew{2008}
Let $X$ be a set, and write $\Mu$ for the set of
all measures on $X$.

(a) Defining $\le$ as in 234P, $(\Mu,\le)$ is a partially ordered set.

(b) If $\mu$, $\nu\in\Mu$, then $\mu\le\nu$ iff there is a $\lambda\in\Mu$
such that $\mu+\lambda=\nu$.

(c) If $\mu\le\nu$ in $\Mu$ and $f$ is a $[-\infty,\infty]$-valued
function,
defined on a subset of $X$, such that $\int fd\nu$ is defined in
$[-\infty,\infty]$, then $\int fd\mu$ is defined;  if $f$ is non-negative,
$\int fd\mu\le\int fd\nu$.

\proof{{\bf (a)} Of course $\mu\le\mu$ for every $\mu\in\Mu$.
If $\mu\le\nu$ and $\nu\le\lambda$ in $\Mu$, then
$\dom\lambda\subseteq\dom\nu\subseteq\dom\mu$, and
$\mu E\le\nu E\le\lambda E$ whenever $\lambda$ measures $E$.
If $\mu\le\nu$ and $\nu\le\mu$ then
$\dom\mu\subseteq\dom\nu\subseteq\dom\mu$ and $\mu E\le\nu E\le\mu E$ for
every $E$ in their common domain, so $\mu=\nu$.

\medskip

{\bf (b)(i)}
If $\mu+\lambda=\nu$, then the definitions in 234G and 234P make
it plain that $\mu\le\nu$.

\medskip

\quad{\bf (ii)}\grheada\ In the reverse direction, if $\mu\le\nu$,
write $\Tau$ for the domain of $\nu$.
Define $\lambda:\Tau\to[0,\infty]$ by setting

\Centerline{$\lambda G
=\sup\{\nu F-\mu F:F\in\Tau,\,F\subseteq G,\,\mu F<\infty\}$}

\noindent for $G\in\Tau$.
Then $\lambda\in\Mu$.   \Prf\ Of course $\dom\lambda=\Tau$ is a
$\sigma$-algebra, and $\lambda\emptyset=0$.   Suppose that
$\sequencen{G_n}$ is a
disjoint sequence in $\Tau$ with union $G$.
If $F\in\Tau$, $F\subseteq G$ and $\mu F<\infty$, then

$$\eqalign{\nu F-\mu F
&=\sum_{n=0}^{\infty}\nu(F\cap G_n)-\sum_{n=0}^{\infty}\mu(F\cap G_n)\cr
&=\sum_{n=0}^{\infty}\nu(F\cap G_n)-\mu(F\cap G_n)
\le\sum_{n=0}^{\infty}\lambda G_n;\cr}$$

\noindent as $F$ is arbitrary,
$\lambda G\le\sum_{n=0}^{\infty}\lambda G_n$.   If
$\gamma<\sum_{n=0}^{\infty}\lambda G_n$, there are an $m\in\Bbb N$ such
that $\gamma<\sum_{n=0}^m\lambda G_n$, and $F_0,\ldots,F_m$ such that
$F_n\in\Tau$, $F_n\subseteq G_n$ and $\mu F_n<\infty$ for every $n\le m$,
while $\sum_{n=0}^m\nu F_n-\mu F_n\ge\gamma$.   Set
$F=\bigcup_{n\le m}F_n$;  then $F\in\Tau$, $F\subseteq G$ and
$\mu F<\infty$, so

\Centerline{$\lambda G\ge\nu F-\mu F
=\sum_{n=0}^m\nu F_n-\mu F_n\ge\gamma$.}

\noindent As $\gamma$ is arbitrary,
$\lambda G\ge\sum_{n=0}^{\infty}\lambda G_n$;  as $\sequencen{G_n}$ is
arbitrary, $\lambda$ is countably additive.\ \Qed

\medskip

\qquad\grheadb\ Now $\mu+\lambda=\nu$.   \Prf\ The domain of $\mu+\lambda$
is $\dom\mu\cap\dom\lambda=\Tau=\dom\nu$.    Take $G\in\Tau$.   If
$\mu G=\infty$, then $\nu G=\infty=(\mu+\lambda)G$.   Otherwise,

\Centerline{$(\mu+\lambda)G\ge\mu G+\nu G-\mu G=\nu G$.}

\noindent So if $\nu G=\infty$ we shall certainly have
$\nu G=(\mu+\lambda)G$.   Finally, if $\nu G<\infty$ then

$$\eqalign{(\mu+\lambda)G
&=\mu G+\sup\{\nu F-\mu F:F\in\Tau,\,F\subseteq G\}\cr
&=\sup\{\nu F+\mu(G\setminus F):F\in\Tau,\,F\subseteq G\}\cr
&\le\sup\{\nu F+\nu(G\setminus F):F\in\Tau,\,F\subseteq G\}
=\nu G,\cr}$$

\noindent so again we have equality.\ \Qed

Thus we have an appropriate expression of $\nu$ as a sum of measures.

\medskip

{\bf (c)(i)} If $f$ is non-negative, put (b) and 234Hc together.

\medskip

\quad{\bf (ii)} In general, if $\int fd\nu$ is defined, so are both
$\int f^+d\nu$ and $\int f^-d\nu$, and at most one is infinite;
so $\int f^+d\mu$ and $\int f^-d\mu$ are defined
and at most one is infinite.
}%end of proof of 234Q

\exercises{
\leader{234X}{Basic exercises (a)}
%\spheader 234Xa
\dvAformerly{1{}32Xk, 2{}14Xp}
Let $(X,\Sigma,\mu)$ and $(Y,\Tau,\nu)$ be measure spaces, and
$\phi:X\to Y$ an \imp\ function.   Let $A\subseteq X$ be a set of full
outer measure in $X$.   Show that
$\phi[A]$ has full outer measure in $Y$, and
that $\phi\restr A$ is \imp\ for the subspace measures on $A$ and
$\phi[A]$.
%234B

\spheader 234Xb\dvAformerly{1{}12Xg}
Let $(X,\Sigma,\mu)$ be a measure space, $Y$ a set and
$\phi:X\to Y$ a function.   Show that if $\mu$ is point-supported, so is
the image measure $\mu\phi^{-1}$.
%234D

\spheader 234Xc\dvAnew{2008}
Give an example of a probability space $(X,\Sigma,\mu)$, a
set $Y$, and a function $\phi:X\to Y$ such that the completion of the image
measure $\mu\phi^{-1}$ is not the image of the completion of $\mu$.
\Hint{$\#(X)=3$.}
%234E

\spheader 234Xd\dvAnew{2008}
 Let $X$, $Y$ be sets, $\phi:X\to Y$ a function
and $\familyiI{\mu_i}$ a family of
measures on $X$ with sum $\mu$.   Writing $\mu_i\phi^{-1}$,
$\mu\phi^{-1}$ for the image measures on $Y$, show that
$\mu\phi^{-1}=\sum_{i\in I}\mu_i\phi^{-1}$.
%234G

\spheader 234Xe\dvAnew{2008}
Let $X$ be a set.
(i) Show that if $\familyiI{\mu_i}$ is a countable family of
$\sigma$-finite measures on $X$, and $\mu=\sum_{i\in I}\mu_i$ is
semi-finite, then $\mu$ is $\sigma$-finite.
(ii) Show that if $\familyiI{\mu_i}$ is a family of purely atomic measures
on $X$, and $\mu=\sum_{i\in I}\mu_i$ is semi-finite, then $\mu$ is purely
atomic.
(iii) Show that if $\familyiI{\mu_i}$ is any family of point-supported
measures on $X$, then $\sum_{i\in I}\mu_i$ is point-supported.
%234G

\sqheader 234Xf\dvAnew{2008;  see 1{}12Xe}
Let $X$ be a set, and write $\Mu$ for the set of all measures on $X$.   For
$\mu\in\Mu$ and $\alpha\in\coint{0,\infty}$, define $\alpha\mu$ by saying
that if $\alpha>0$ then
$(\alpha\mu)(E)=\alpha\mu E$ for $E\in\dom\mu$, while if $\alpha=0$ then
$(\alpha\mu)(E)=0$ for every $E\subseteq X$.   (i) Show that
$\alpha\mu\in\Mu$ for all $\alpha\in\coint{0,\infty}$ and $\mu\in\Mu$.
(ii) Show that $(\alpha+\beta)\mu=\alpha\mu+\beta\mu$,
$\alpha(\beta\mu)=(\alpha\beta)\mu$, $\alpha(\mu+\nu)=\alpha\mu+\alpha\nu$
for all $\alpha$, $\beta\in\coint{0,\infty}$ and $\mu$, $\nu\in\Mu$.
%234G

\spheader 234Xg\dvAformerly{2{}12Xj}
 Let $X$ be a set, and $\familyiI{\mu_i}$ a family of
complete measures on $X$ with sum $\mu$.
Show that a $[-\infty,\infty]$-valued
function $f$ defined on a subset of $X$ is $\mu$-integrable iff it is
$\mu_i$-integrable for every $i\in I$ and $\sum_{i\in I}\int|f|d\mu_i$ is
finite.
%234H

\spheader 234Xh\dvAformerly{2{}34Xa}
Let $\mu$ be Lebesgue measure on $[0,1]$, and set $f(x)=\bover1x$ for
$x>0$. Let $\nu$ be the associated indefinite-integral measure.   Show
that the
domain of $\nu$ is equal to the domain of $\mu$.   Show that for every
$\delta\in\ocint{0,\bover12}$ there is a measurable set $E$ such that
$\mu E=\delta$ but $\nu E=\bover1{\delta}$.
%234K

\spheader 234Xi\dvAnew{2008}
 Let $(X,\Sigma,\mu)$ be a measure space.   (i) Show that if
$\nu_1$ and $\nu_2$ are indefinite-integral measures over $\mu$, so is
$\nu_1+\nu_2$.   (ii) Show that if $\familyiI{\nu_i}$ is a countable family
of indefinite-integral measures over $\mu$, and $\nu=\sum_{i\in I}\nu_i$ is
semi-finite, then $\nu$ is an indefinite-integral measure over $\mu$.
%234L

\spheader 234Xj\dvAformerly{2{}34Xb}
 Let $(X,\Sigma,\mu)$ be a measure space, and $\nu$ an
indefinite-integral measure over $\mu$.   Show that if $\mu$ is purely
atomic, so is $\nu$.
%234N

\spheader 234Xk\dvAformerly{2{}34Xc}
 Let $\mu$ be a point-supported measure.   Show that any
indefinite-integral measure over $\mu$ is point-supported.
%234N

\spheader 234Xl\dvAnew{2008}
 Let $X$ be a set, and $\Mu$ the set of measures on $X$,
with the partial ordering defined in 234P.   Show that (i) $\Mu$ has
greatest and least members (to be described);
(ii) if $\familyiI{\mu_i}$
and $\familyiI{\nu_i}$ are families in $\Mu$ such that $\mu_i\le\nu_i$ for
every $i$, then $\sum_{i\in I}\mu_i\le\sum_{i\in I}\nu_i$;   (iii) if we
define scalar multiplication as in 234Xf, then
$\alpha\mu\le\mu$ whenever $\mu\in\Mu$ and $\alpha\in[0,1]$;   (iv) writing
$\hat\mu$ for the completion of $\mu$, $\hat\mu\le\mu$ and
$\hat\mu\le\hat\nu$ whenever $\mu$, $\nu\in\Mu$ and $\mu\le\nu$;  (v)
writing $\tilde\mu$ for the c.l.d.\ version of $\mu$, $\tilde\mu\le\mu$ for
every $\mu\in\Mu$;  (vi) whenever $A\subseteq\Mu$ is upwards-directed, it
has a least upper bound in $\Mu$.
%234Q

\spheader 234Xm\dvAnew{2008}
 Write out an elementary direct proof of 234Qc not depending
on 234Qb.
%234Q

\spheader 234Xn\dvAformerly{211Xd(v)} Let $(X,\Sigma,\mu)$ and
$(Y,\Tau,\nu)$ be measure spaces and $\phi:X\to Y$ an \imp\ function.
Show that if $\mu$ is $\sigma$-finite and purely atomic then
$\nu$ is purely atomic.

\leader{234Y}{Further exercises (a)}
%\spheader 234Ya
\dvAformerly{2{}35Ya}
Write $\nu$ for Lebesgue measure on
$Y=[0,1]$, and $\Tau$ for its domain.    Let $A\subseteq[0,1]$ be a set
such that $\nu^*A=\nu^*([0,1]\setminus A)=1$, and set
$X=[0,1]\cup\{x+1:x\in A\}\cup\{x+2:x\in[0,1]\setminus A\}$.   Let
$\mu_{LX}$ be the subspace measure induced on $X$ by Lebesgue measure,
and set $\mu E=\bover13\mu_{LX}E$ for $E\in\Sigma=\dom\mu_{LX}$.
Define $\phi:X\to Y$ by writing $\phi(x)=x$ if $x\in[0,1]$,
$\phi(x)=x-1$ if $x\in X\cap\ocint{1,2}$ and $\phi(x)=x-2$ if
$x\in X\cap\ocint{2,3}$.   Show that $\nu$ is the image measure
$\mu\phi^{-1}$, but that $\nu^*A>\mu^*\phi^{-1}[A]$.
%234B

\spheader 234Yb\dvAnew{2008}
Look for interesting examples of probability spaces
$(X,\Sigma,\mu)$ and $(Y,\Tau,\nu)$ for which there are functions
$\phi:X\to Y$ such that $\phi[E]\in\Tau$ and $\nu\phi[E]=\mu E$ for every
$E\in\Sigma$.   \Hint{254K, 343J.}
%234B

\spheader 234Yc\dvAnew{2008}
Let $\mu$ be
two-dimensional Lebesgue measure on the unit square $[0,1]^2$, and let
$\phi:[0,1]^2\to[0,1]$ be the projection onto the first coordinate, so that
$\phi(\xi_1,\xi_2)=\xi_1$ for $\xi_1$, $\xi_2\in[0,1]$.   Show that the
image measure $\mu\phi^{-1}$ is Lebesgue
one-dimensional measure on $[0,1]$.
%234E

\spheader 234Yd\dvAformerly{1{}32Yf}
In 234F, show that the image measure $\mu\phi^{-1}$
extends $\nu$, and is equal to $\nu$ if and only if $F\in\Tau$ for
every $F\subseteq Y\setminus\phi[X]$.
%234F

\spheader 234Ye\dvAnew{2008}
 Let $(Y,\Tau,\nu)$ be a complete measure space, $X$ a set
and $\phi:X\to Y$ a surjection.   Set

\Centerline{$\Sigma=\{E:E\subseteq X$, $\phi[E]\in\Tau$,
$\nu(\phi[E]\cap\phi[X\setminus E])=0\}$,
\quad$\mu E=\nu\phi[E]$ for $E\in\Tau$.}

\noindent Show that $\mu$ is the
completion of the measure constructed by the process of 234F.
%234F

\spheader 234Yf\dvAnew{2008}
Let $X$ be a set, and $\Mu$ the set of measures on $X$.
Show that $\Mu$, with addition as defined for two measures by the formulae
of 234G, is a commutative semigroup with identity;  describe the identity.
%234G

\spheader 234Yg\dvAnew{2008}
Give an example of a set $X$, probability measures
$\mu_1$, $\mu_2$ on $X$ and a set $A\subseteq X$ such that $A$ is
both $\mu_1$-negligible and $\mu_2$-negligible, but is not
$\mu$-negligible, where $\mu=\mu_1+\mu_2$.
%234G 

\spheader 234Yh\dvAformerly{2{}14Ya}
In 214O, show that if we set
$\nu E=\sup_{I\in\Cal I}\mu^*(E\cap I)$ for every $E\in\Sigma$, then
$\nu$ is a measure, while $\mu=\nu+\lambda$.
%234H

\spheader 234Yi\dvAformerly{2{}34Yd}
 Let $(X,\Sigma,\mu)$ be an atomless semi-finite measure
space and
$\nu$ an indefinite-integral measure over $\mu$.   Show that the
following are equiveridical:  (i) for every $\epsilon>0$ there is a
$\delta>0$ such that $\nu E\le\epsilon$ whenever $\mu E\le\delta$ (ii)
$\nu$ has a Radon-Nikod\'ym derivative expressible as the sum of a
bounded function and an integrable function.
%234K

\spheader 234Yj\dvAformerly{2{}34Yf}
 Let $(X,\Sigma,\mu)$ be a measure space and $\nu$ an
indefinite-integral measure over $\mu$, with Radon-Nikod\'ym derivative
$f$.   Show that the c.l.d.\ version of $\nu$ is the indefinite-integral
measure defined by $f$ over the c.l.d.\ version of $\mu$.
%234N

\spheader 234Yk\dvAformerly{2{}34Ya}
Let $(X,\Sigma,\mu)$ be a semi-finite measure space which is not
localizable.   Show that there is a measure $\nu:\Sigma\to[0,\infty]$
such that $\nu E\le\mu E$ for every $E\in\Sigma$ but there is no
measurable
function $f$ such that $\nu E=\int_Efd\mu$ for every $E\in\Sigma$.
%234O

\spheader 234Yl\dvAformerly{2{}34Yb}
Let $(X,\Sigma,\mu)$ be a localizable measure space with
locally determined negligible sets.   Show that a measure $\nu$,
with domain $\Tau\supseteq\Sigma$, is an indefinite-integral measure
over $\mu$ iff ($\alpha$) $\nu$ is complete and semi-finite and zero on
$\mu$-negligible sets
($\beta$) whenever $\nu E>0$ there is an $F\subseteq E$ such that
$F\in\Sigma$ and $\mu F<\infty$ and $\nu F>0$.
%234O

\spheader 234Ym\dvAformerly{2{}34Yc}
 Give an example of a localizable measure space
$(X,\Sigma,\mu)$ and a complete semi-finite measure $\nu$ on $X$,
defined on
a $\sigma$-algebra $\Tau\supseteq\Sigma$, zero on $\mu$-negligible sets,
and such that whenever $\nu E>0$ there
is an $F\subseteq E$ such that $F\in\Sigma$ and $\mu F<\infty$ and
$\nu F>0$, but $\nu$ is not an indefinite-integral measure over $\mu$.
\Hint{216Yb.}
%234O 23bits

\spheader 234Yn\dvAformerly{2{}34Ye}
 Let $(X,\Sigma,\mu)$ be a localizable measure space, and
$\nu$ a complete localizable measure on $X$, with domain
$\Tau\supseteq\Sigma$, which is the completion of its restriction to
$\Sigma$.   Show that if we set
$\nu_1F=\sup\{\nu(F\cap E):E\in\Sigma$, $\mu E<\infty\}$ for every
$F\in\Tau$, then $\nu_1$ is an indefinite-integral measure over $\mu$,
and there is an $H\in\Sigma$ such that $\nu_1F=\nu(F\cap H)$ for every
$F\in\Tau$.
%234O

\spheader 234Yo\dvAnew{2008}
Let $X$ be a set, and $\Mu_{\text{sf}}$ the set of
semi-finite measures
on $X$.   For $\mu$, $\nu\in\Mu_{\text{sf}}$ say that $\mu\preccurlyeq\nu$
if $\dom\nu\subseteq\dom\mu$, $\mu F\le\nu F$ for every $F\in\dom\nu$, and
whenever $E\in\dom\mu$ and $\mu E>0$ there is an $F\in\dom\nu$ such that
$F\subseteq E$ and $0<\mu F<\infty$.   (i) Show that
$(\Mu_{\text{sf}},\preccurlyeq)$ is a
partially ordered set.   (ii) Show that if $A\subseteq\Mu_{\text{sf}}$
is a non-empty
set with an upper bound in $\Mu_{\text{sf}}$, then it has a least upper
bound $\lambda$
defined by saying that $\dom\lambda=\bigcap_{\mu\in A}\dom\mu$ and, for
$E\in\dom\lambda$,

$$\eqalign{\lambda E
&=\sup\{\sum_{i=0}^n\mu_iF_i:\mu_0,\ldots,\mu_n\in A,\,
   \langle F_i\rangle_{i\le n}\text{ is a partition of }E,\cr
&\mskip200mu
   F_i\in\dom\lambda\text{ for every }i\le n\}\cr
&=\sup\{\sum_{i=0}^n\mu_iF_i:\mu_0,\ldots,\mu_n\in A,\,
   F_0,\ldots,F_n\text{ are disjoint},\cr
&\mskip200mu
   F_i\in\dom\mu_i\text{ and }F_i\subseteq E\text{ for every }i\le n\}.
   \cr}$$

\noindent(iii) Suppose that $\mu$, $\nu\in\Mu_{\text{sf}}$ have completions $\hat\mu$,
$\hat\nu$ and c.l.d.\ versions $\tilde\mu$, $\tilde\nu$.
Show that $\tilde\mu\preccurlyeq\hat\mu\preccurlyeq\mu$.   Show that if
$\mu\preccurlyeq\nu$ then
$\hat\mu\preccurlyeq\hat\nu$ and $\tilde\mu\preccurlyeq\tilde\nu$.
%234P
}%end of exercises

\endnotes{\Notesheader{234}
One of the striking features of measure theory, compared with other
comparably abstract branches of pure mathematics, is the relative
unimportance of any notion of `morphism'.   The theory of groups, for
instance, is dominated by the concept of `homomorphism', and general
topology gives a similar place to `continuous function'.   In my view,
the nearest equivalent in measure theory is the idea of
`\imp\ function' (234A).   I mean in Volumes 3 and
4 to explore this concept more thoroughly.   In this volume I will
content myself with signalling such functions when they arise, and with
the basic facts listed in 234B.

Naturally linked with the idea of \imp\ function is the construction of
`image measures' (234C).   These appear everywhere in the subject, starting
with the not-quite-elementary 234Yc.   They are of such importance that it
is natural to explore variations, as in 234F and 234Yb, but in my view none
are of comparable significance.

Nearly half the section is taken up with `indefinite-integral measures'.
I have taken this part very carefully
because the ideas I wish to express here, in so far as they extend the
work of \S232, rely critically on the details of the formulation in 234I,
and it is easy to make a false step once we have left the relatively
sheltered context of complete $\sigma$-finite measures.   I believe that
if we take a little trouble at this point we can develop a theory
(234K-234N) %234K 234L 234M 234N
which will
offer a smooth path to later applications;  to see what I have in mind,
you can refer to the entries under
`indefinite-integral measure' in the index.   For the moment I mention
only a kind of Radon-Nikod\'ym theorem for localizable measures (234O).

The partial ordering described in 234P-234Q is only one of many which can
be considered, and for some purposes it seems unsatisfactory.   The most
important examples will appear in Chapter 41 of Volume 4, and have a
variety of special features for which it might be worth setting out further
abstractions.   However the version here has the merit of simplicity and
supports at least some of the relevant ideas (234Xl).   For an alternative
notion, see 234Yo.

}%end of notes

%look for ideas to add to \S412

\discrpage

