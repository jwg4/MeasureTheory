\frfilename{mt52.tex} 
\versiondate{7.10.13} 
\copyrightdate{2005} 
 
\def\chaptername{Cardinal functions of measure theory} 
 
\newchapter{52} 
 
From the point of view of this book, the most important cardinals are 
those associated with measures and measure algebras, especially, of 
course, Lebesgue measure and the usual measure $\nu_I$ of $\{0,1\}^I$. 
In this 
chapter I try to cover the principal known facts about these which are 
theorems of ZFC.   I start with a review of the theory for general measure 
spaces in \S521,  
including some material which returns to the classification 
scheme of Chapter 21, exploring relationships between (strict) 
localizability, magnitude and Maharam type.    
\S522 examines Lebesgue measure and the surprising 
connexions found by {\smc Bartoszy\'nski 84} and 
{\smc Raisonnier \& Stern 85} between the cardinals associated with the 
Lebesgue null 
ideal and the corresponding ones based on the ideal of meager subsets of 
$\Bbb R$.   \S523 looks at the 
measures $\nu_I$ for uncountable sets $I$, giving formulae for the 
additivities and cofinalities of their null ideals, and bounds for 
their covering numbers, uniformities and shrinking numbers.    
Remarkably, these cardinals 
are enough to tell us most of what we want to know concerning the 
cardinal functions of general Radon measures and semi-finite measure 
algebras (\S524).   These three sections are heavily dependent on the 
Galois-Tukey connections and Tukey functions of \S\S512-513.    
Precalibers do 
not seem to fit into this scheme, and the relatively partial information 
I have is in \S525.   The second half of the chapter deals with special 
topics which can be approached with the methods so far developed.    
In \S526 I return to the ideal of subsets of $\Bbb N$ 
with asymptotic density zero, seeking to locate it in the Tukey 
classification.   Further $\sigma$-ideals which are of interest in 
measure theory are the `skew products' of \S527.    
In \S528 I examine some interesting 
Boolean algebras, the `amoeba algebras' first introduced by 
{\smc Martin \& Solovay 70}, giving the results of {\smc Truss 88} 
on the connexions 
between different amoeba algebras and localization posets.   Finally, 
in \S529, I look at a handful of other structures, concentrating on 
results involving cardinals already described. 
 
\discrpage 
 
 
