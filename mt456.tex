\frfilename{mt456.tex}
\versiondate{19.5.10}
\copyrightdate{2010}

\def\chaptername{Perfect measures, disintegrations and processes}
\def\sectionname{Gaussian distributions}

\newsection{456}

Uncountable powers of $\Bbb R$ are not as a rule
measure-compact\cmmnt{ (439P, 455Xc;  see also 533J in Volume 5)}.
Accordingly distributions, in
the sense of 454K, need not be $\tau$-additive.   But some, at least,
of the distributions most important to us
are indeed $\tau$-additive, and therefore have interesting canonical
extensions.
This section is devoted to a remarkable result, taken from
{\smc Talagrand 81}, concerning a class of distributions which are of
great importance in probability theory.   It demands a combination of
techniques from classical probability theory and from the topological
measure theory of this volume.   I begin with the definition and
fundamental properties of what I call `centered Gaussian distributions'
(456A-456I).  %456A 456B 456C 456D 456E 456F 456G 456H 456I
These are fairly straightforward
adaptations of the classical finite-dimensional theory, and
will be useful in \S477 when we come to study Brownian motion.
Another relatively easy idea is that of `universal' Gaussian
distribution (456J-456L).  %456J 456K 456L
In 456M we come to a much deeper result, a step towards classifying the
ways in which a Gaussian family of $n$-dimensional random variables can
accumulate at $0$.   The ideas are combined in
456N-456O to complete the proof of Talagrand's theorem that Gaussian
distributions on powers of $\Bbb R$ are $\tau$-additive.

\leader{456A}{Definitions (a)} Write $\mu_G$ for the Radon probability
measure on $\Bbb R$ which is the distribution of a standard normal
random variable\cmmnt{, that is, the probability distribution with
density function $x\mapsto\Bover1{\sqrt{2\pi}}e^{-x^2/2}$ (274A)}.   For any set
$I$, write $\mu_G^{(I)}$ for the measure on $\BbbR^I$ which is the
product of copies of $\mu_G$;  this is
always quasi-Radon\cmmnt{ (415E/453J)};  if $I$ is countable, it is
Radon\cmmnt{ (417Q)};  if
$I=n\in\Bbb N\setminus\{0\}$, it is the probability distribution with
density function
$x\mapsto(2\pi)^{-n/2}e^{-x\dotproduct x/2}$\cmmnt{ (272I)};
if $I=\emptyset$, it is the unique probability measure on the singleton
set $\BbbR^{\emptyset}$.

\spheader 456Ab I will use the phrase {\bf centered Gaussian
distribution} to mean a measure $\mu$ on a power $\BbbR^I$ of $\Bbb R$
such that $\mu$ is the completion of a Baire measure\cmmnt{ (that is,
is a distribution in the sense of 454K)} and every
continuous linear functional $f:\BbbR^I\to\Bbb R$ is either zero almost
everywhere or is a normal random variable with zero
expectation.   \cmmnt{(Note that I call the
distribution concentrated at the point $0$ in $\BbbR^I$ a
`Gaussian distribution'.)}

\spheader 456Ac If $I$ is a set and $\mu$ is a centered Gaussian
distribution on $\BbbR^I$, its {\bf covariance matrix} is the
family $\langle\sigma_{ij}\rangle_{i,j\in I}$ where
$\sigma_{ij}=\int x(i)x(j)\mu(dx)$ for $i$, $j\in I$.   \cmmnt{(The
integral is always defined and finite because each function
$x\mapsto x(i)$ is either essentially constant or normally distributed,
and in either case is square-integrable.)}

\leader{456B}{}\cmmnt{ I start with some fundamental facts about
Gaussian distributions.

\medskip

\noindent}{\bf Proposition} (a) Suppose that $I$ and $J$ are sets, $\mu$
is a centered Gaussian distribution on $\BbbR^I$, and
$T:\BbbR^I\to\BbbR^J$ is a continuous linear operator.   Then there is a
unique centered Gaussian distribution on $\BbbR^J$ for which $T$ is
\imp;  if $J$ is countable, this is the image measure $\mu T^{-1}$.

(b) Let $I$ be a set, and $\mu$, $\nu$ two centered Gaussian
distributions on $\BbbR^I$.   If they have the same covariance matrices
they are equal.

(c) For any set $I$, $\mu_G^{(I)}$ is the centered Gaussian distribution
on $\BbbR^I$ with the identity matrix for its covariance matrix.

(d) Suppose that $I$ is a countable set.   Then a measure $\mu$ on
$\BbbR^I$ is a centered Gaussian distribution iff it is of the form
$\mu_G^{(\Bbb N)}T^{-1}$ where $T:\BbbR^{\Bbb N}\to\Bbb R^I$ is a
continuous linear operator.

(e) Suppose $\family{j}{J}{I_j}$ is a disjoint family of sets with union
$I$, and that for each $j\in J$ we have a centered Gaussian distribution
$\nu_j$ on $\BbbR^{I_j}$.   Then the product $\nu$ of the measures
$\nu_j$, regarded as a measure on $\BbbR^I$, is a centered Gaussian
distribution.

(f) Let $I$ be any set, $\mu$ a centered Gaussian distribution on
$\BbbR^I$ and $E\subseteq\BbbR^I$ a set such that $\mu$ measures $E$.
Writing $-E=\{-x:x\in E\}$, $\mu(-E)=\mu E$.

\proof{{\bf (a)(i)} For Baire sets $F\subseteq\BbbR^J$, set
$\nu F=\mu T^{-1}[F]$;  this is always defined because $T$ is continuous
(4A3Kc).   This makes $\nu$ a Baire probability measure on $\BbbR^J$ for
which $T$ is \imp.   Because $\mu$ is complete, $T$ is still \imp\ for
the completion $\hat\nu$ of $\nu$ (234Ba\formerly{2{}35Hc}).
If $g:\BbbR^J\to\Bbb R$ is
a continuous linear functional, so is $gT:\BbbR^I\to\Bbb R$;  now
$\nu\{y:g(y)\le\alpha\}=\mu\{x:gT(x)\le\alpha\}$ for every $\alpha$, so
$g$ and $gT$ have the same distribution, and are both either zero a.e.\
or normal random variables.   As $g$ is arbitrary, $\hat\nu$ is a
centered Gaussian distribution as defined in 456Ab.   Of course it is
the only such distribution on $\BbbR^J$ for which $T$ is \imp.

\medskip

\quad{\bf (ii)} Now suppose that $J$ is countable.   Then $\BbbR^J$
is Polish (4A2Qc), so $\nu$ is a Borel measure and $\hat\nu$ is a Radon
measure (433Cb).
$\BbbR^J$ has a countable network consisting of Borel sets, $\mu$ is
perfect (454A(b-iii)) and totally finite, and $T$ is measurable (418Bd),
so $\mu T^{-1}$ is a Radon measure (451O).   Thus $\hat\nu$ and
$\mu T^{-1}$ are Radon measures agreeing on the Borel sets and must be
equal.

\medskip

{\bf (b)} The point is that $\mu f^{-1}=\nu f^{-1}$ for every continuous
linear functional $f:\BbbR^I\to\Bbb R$.   \Prf\ By (a),
$\mu f^{-1}$ and $\nu f^{-1}$ are Radon measures on $\Bbb R$, and by the
definition of `Gaussian distribution' each is either a normal
distribution with expectation zero, or is concentrated at $0$.   By
4A4Be, we
can express $f$ in the form $f(x)=\sum_{i\in I}\beta_ix(i)$ for every
$x\in\BbbR^I$, where $\{i:\beta_i\ne 0\}$ is finite.   In this case

$$\eqalignno{\int t^2(\mu f^{-1})(dt)
&=\int f(x)^2\mu(dx)\cr
\displaycause{235G\formerly{2{}35I}}
&=\sum_{i,j\in I}\beta_i\beta_j\int x(i)x(j)\mu(dx)\cr
&=\sum_{i,j\in I}\beta_i\beta_j\int x(i)x(j)\nu(dx)
=\int t^2(\nu f^{-1})(dt)\cr}$$

\noindent because $\mu$ and $\nu$ have the same covariance matrices.
But this means that $\mu f^{-1}$ and $\nu f^{-1}$ have the same
variance;  if this is zero, they both give measure $1$ to $\{0\}$;
otherwise, they are normal distributions with the same expectation and
the same variance, so again are equal.\ \Qed

By 454P, $\mu=\nu$.

\medskip

{\bf (c)} Being a completion regular quasi-Radon probability measure
(415E), $\mu^{(I)}_G$ is the completion of a Baire probability measure
on $\BbbR^I$.   If $f:\BbbR^I\to\Bbb R$ is a continuous linear
functional, then it is expressible in the form
$f(z)=\sum_{i\in I}\beta_iz(i)$, where $J=\{i:\beta_i\ne 0\}$ is finite.
I need to show that $f$ is either zero a.e.\ or a normal random variable
with expectation $0$.   If $J=\emptyset$ then $f=0$ everywhere and we
can stop.
Otherwise, $f=\sum_{i\in J}\beta_i\pi_i$, where $\pi_i(x)=x(i)$ for
$i\in I$ and $x\in\BbbR^I$.   Now, with respect to the measure
$\mu_G^{(I)}$, $\family{i}{J}{\pi_i}$ is an independent family of normal
random variables with zero expectation (272G).   So
$\family{i}{J}{\beta_i\pi_i}$ is independent (272E), and $\beta_i\pi_i$
is normal for $i\in J$ (274Ae).   By 274B,
$f=\sum_{i\in J}\beta_i\pi_i$ is normal, and of course it has zero
expectation.
As $f$ is arbitrary, $\mu_G^{(I)}$ is a centered Gaussian distribution.

We have

\Centerline{$\int x(i)x(i)\mu_G^{(I)}(dx)=\int t^2\mu_G(dt)=1$}

\noindent for $i\in I$, and

\Centerline{$\int x(i)x(j)\mu_G^{(I)}(dx)
=\int t\mu_G(dt)\cdot\int t\mu_G(dt)=0$}

\noindent if $i$, $j\in I$ are distinct.   So the covariance matrix of
$\mu_G^{(J)}$ is the identity matrix.   By (b), it is the only centered
Gaussian distribution with this covariance matrix.

\medskip

{\bf (d)(i)} It follows from (c) and (a) that if
$\mu=\mu_G^{(\Bbb N)}T^{-1}$, where $T:\BbbR^{\Bbb N}\to\BbbR^I$ is a
continuous linear operator, then $\mu$ is a centered Gaussian
distribution.

\medskip

\quad{\bf (ii)} Now suppose that $\mu$ is a centered Gaussian
distribution on $\BbbR^I$.   Set $\pi_i(x)=x(i)$ for $i\in I$ and
$x\in\BbbR^I$;  for
$i\in I$, set $u_i=\pi_i^{\ssbullet}$ in $L^2=L^2(\mu)$.   By 4A4Jh,
there is a countable orthonormal family $\family{j}{J}{v_j}$ in $L^2$
such that every $v_j$ is a linear combination of the $u_i$, and every
$u_i$ is a linear combination of the $v_j$.   We may suppose that
$J\subseteq\Bbb N$.   For $i\in I$, express $u_i$ as
$\sum_{j\in J}\alpha_{ij}v_j$, where $\{j:\alpha_{ij}\ne 0\}$ is finite.
Define $T:\BbbR^{\Bbb N}\to\BbbR^I$ by setting
$(Tz)(i)=\sum_{j\in J}\alpha_{ij}z(j)$ for every $z\in\BbbR^{\Bbb N}$ and
$i\in I$.   Then $T$ is a continuous linear functional.   Set
$\nu=\mu_G^{(\Bbb N)}T^{-1}$, so that $\nu$ is a centered Gaussian
distribution on $\BbbR^I$, by (a).   Because $\BbbR^I$ is Polish, both
$\mu$ and $\nu$ must be Radon measures.

\medskip

\quad{\bf (iii)} Now $\mu$ and $\nu$ have the same covariance matrices.
\Prf\ If $i$, $i'\in I$ then

$$\eqalign{\int x(i)x(i')\mu(dx)
&=\innerprod{u_i}{u_{i'}}
=\sum_{j,j'\in J}\alpha_{ij}\alpha_{i'j'}\innerprod{v_j}{v_{j'}}\cr
&=\sum_{j\in J}\alpha_{ij}\alpha_{i'j}
=\sum_{j,j'\in J}\alpha_{ij}\alpha_{i'j'}
  \int z(j)z(j')\mu_G^{(\Bbb N)}(dz)\cr
&=\int (Tz)(i)(Tz)(i')\mu_G^{(\Bbb N)}(dz)
=\int x(i)x(i')\nu(dx). \text{ \Qed}\cr}$$

\noindent By (b), $\mu=\nu$ is of the required form.

\medskip

{\bf (e)} We must first confirm that $\nu$ is the completion of a Baire
measure.   \Prf\ If we write $\CalBa(\BbbR^{I_j})$ for the Baire
$\sigma$-algebra of $\BbbR^{I_j}$, then each $\nu_j$ is the completion
of its restriction $\nu_j\restr\CalBa(\BbbR^{I_j})$, so $\nu$ is also
the product of the measures $\nu_j\restr\CalBa(\BbbR^{I_j})$ (254I), and
is therefore the completion of its restriction to
$\Tensorhat_{j\in J}\CalBa(\BbbR^{I_j})$ (254Ff).   But as
$\CalBa(\BbbR^{I_j})=\Tensorhat_{i\in I_j}\CalBa(\Bbb R)$ for every $j$
(4A3Na), $\Tensorhat_{j\in J}\CalBa(\BbbR^{I_j})$ can be identified with
$\Tensorhat_{i\in I}\CalBa(\Bbb R)=\CalBa(\BbbR^I)$, so that $\nu$ is
indeed the completion of $\nu\restr\CalBa(\BbbR^I)$.\ \Qed

Now suppose that $f:\BbbR^I\to\Bbb R$ is a continuous linear functional.
Then we can express $f$ in the form $f(x)=\sum_{i\in K}\alpha_ix(i)$ for
every $x\in\BbbR^I$, where $K\subseteq I$ is finite.   Set
$L=\{j:K\cap I_j\ne\emptyset\}$ and $K_j=K\cap I_j$ for $j\in L$, so
that $L$ and every $K_j$ are finite;  for $j\in L$ and $x\in\BbbR^I$ set
$f_j(x)=\sum_{i\in K_j}\alpha_ix(i)$.   Now $f=\sum_{j\in L}f_j$.

If we set $g_j(y)=\sum_{i\in K_j}\alpha_iy(i)$ for $y\in\BbbR^{I_j}$,
then $g_j$ is either zero a.e.\ or a normal random variable with respect
to the probability measure $\nu_j$.   Since

\Centerline{$\nu\{x:f_j(x)\le\alpha\}=\nu\{x:g_j(x\restr I_j)\le\alpha\}
=\nu_j\{y:g_j(y)\le\alpha\}$}

\noindent for every $\alpha\in\Bbb R$, $f_j$ (regarded as a random
variable on $(\BbbR^I,\nu)$) has the same distribution as $g_j$
(regarded as a random variable on $(\BbbR^{I_j},\nu_j)$).   This is true
for every $j\in L$.   Moreover, the different $f_j$, as $j$ runs over
$L$, are independent.   So $f=\sum_{j\in L}f_j$ is the sum of
independent random variables which are all either normal or essentially
constant.   By 274B again, $f$ also is either normal or essentially
constant.   And of course its expectation is zero.   As $f$ is
arbitrary, this shows that $\nu$ is a centered Gaussian distribution.

\medskip

{\bf (f)} Set $Tx=-x$ for $x\in\BbbR^I$, so that $T$ is a continuous
linear operator and we have a unique centered Gaussian distribution
$\nu$ on $\BbbR^I$ such that $T$ is \imp\ for $\mu$ and $\nu$, by (a).
For any $i$, $j\in I$,

\Centerline{$\int x(i)x(j)\nu(dx)
=\int(Tx)(i)(Tx)(j)\mu(dx)
=\int x(i)x(j)\mu(dx)$,}

\noindent so $\mu$ and $\nu$ have the same covariance matrices and are
equal, by (b).   Accordingly

\Centerline{$\mu(-E)=\mu T^{-1}[E]=\nu E=\mu E$}

\noindent whenever $\mu$ measures $E$.
}%end of proof of 456B

\leader{456C}{}\cmmnt{ Since a Gaussian distribution is determined by
its covariance matrix (456Bb), we naturally seek descriptions of which
matrices can arise.

\medskip

\noindent}{\bf Theorem}\dvAformerly{4{}56Xa.}
Let $I$ be a set and $\langle\sigma_{ij}\rangle_{i,j\in I}$ a family of
real numbers.   Then the following are equiveridical:

(i) $\langle\sigma_{ij}\rangle_{i,j\in I}$ is the covariance matrix of a
centered Gaussian distribution on $\BbbR^I$;

(ii) there are a (real) Hilbert
space $U$ and a family $\familyiI{u_i}$ in $U$ such that
$\innerprod{u_i}{u_j}=\sigma_{ij}$ for all $i$, $j\in I$;

(iii) for every finite $J\subseteq I$,
$\langle\sigma_{ij}\rangle_{i,j\in J}$ is the covariance matrix of a
centered Gaussian distribution on $\BbbR^J$;

(iv) $\langle\sigma_{ij}\rangle_{i,j\in I}$ is
symmetric and positive semi-definite
in the sense that $\sigma_{ij}=\sigma_{ji}$ for all $i$, $j\in I$ and
$\sum_{i,j\in J}\alpha_i\alpha_j\sigma_{ij}\ge 0$ whenever
$J\subseteq I$ is finite and $\family{i}{J}{\alpha_i}\in\Bbb R^J$.

\proof{{\bf (i)$\Rightarrow$(ii)} If $\mu$ is a centered Gaussian
distribution on $\BbbR^I$ with covariance matrix
$\langle\sigma_{ij}\rangle_{i,j\in I}$, then $L^2(\mu)$ is a Hilbert space.
Setting $X_i(x)=x(i)$ for $x\in\BbbR^I$,
$u_i=X_i^{\ssbullet}$ belongs to the Hilbert space $L^2(\mu)$ for
every $i\in I$, and

\Centerline{$\innerprod{u_i}{u_j}=\int X_i\times X_jd\mu
=\int x(i)x(j)\mu(dx)=\sigma_{ij}$}

\noindent for all $i$, $j\in I$.

\medskip

{\bf (ii)$\Rightarrow$(iv)} In this context,

\Centerline{$\sigma_{ij}=\innerprod{u_i}{u_j}=\innerprod{u_j}{u_i}
=\sigma_{ji}$,}

\Centerline{$\sum_{i,j\in J}\alpha_i\alpha_j\sigma_{ij}
=\sum_{i,j\in J}\alpha_i\alpha_j\innerprod{u_i}{u_j}
=\|\sum_{i\in J}\alpha_iu_i\|^2
\ge 0$.}

\medskip

{\bf (iv)$\Rightarrow$(iii)} Here we have to know something about symmetric
matrices.   Given a family $\langle\sigma_{ij}\rangle_{i,j\in I}$
satisfying the conditions of (iv), and a
finite set $J\subseteq I$, we have a linear operator $T:\BbbR^J\to\BbbR^J$
defined by saying that $(Tz)(i)=\sum_{j\in J}\sigma_{ij}z(j)$ for
$z\in\BbbR^J$ and $i\in J$.   Give $\BbbR^J=\ell^2(J)$ its usual inner
product, so that $w\dotproduct z=\sum_{j\in J}w(j)z(j)$ for $w$,
$z\in\BbbR^J$;  then $\BbbR^J$ is a Hilbert space and

\Centerline{$Tw\dotproduct z
=\sum_{i\in J}\sum_{j\in J}\sigma_{ij}w(j)z(i)
=\sum_{j\in J}\sum_{i\in J}\sigma_{ji}z(i)w(j)
=w\dotproduct Tz$}

\noindent for all $w$, $z\in\BbbR^J$, so that $T$ is self-adjoint.
Moreover, if $z\in\BbbR^J$,

\Centerline{$Tz\dotproduct z
=\sum_{i,j\in J}\sigma_{ij}z(i)z(j)\ge 0$}

\noindent by the other condition on
$\langle\sigma_{ij}\rangle_{i,j\in I}$.

By 4A4M\footnote{Or, rather, its finite-dimensional special case, which
is easier;  you may know it under the slogan `symmetric matrices are
diagonalisable'.},
$\BbbR^J$ has an orthonormal basis consisting of eigenvectors for $T$;
if $\#(J)=n$, we have a basis $\ofamily{k}{n}{u_k}$ and a family
$\ofamily{k}{n}{\gamma_k}$ of real numbers such that $Tu_k=\gamma_ku_k$ for
each $k<n$.   We need to know that $\sum_{k<n}u_k(i)u_k(j)=1$ if $i=j$, $0$
otherwise.   \Prf\ Let $\family{i}{J}{v_i}$
be the standard basis of $\BbbR^J$,
so that $v_i(j)=1$ if $i=j$, $0$ if $i\ne j$.   Then

\Centerline{$v_i=\sum_{k<n}(v_i\dotproduct u_k)u_k
=\sum_{k<n}u_k(i)u_k$}

\noindent for $i\in J$, so

$$\eqalign{\sum_{k<n}u_k(i)u_k(j)
=\sum_{k,l<n}u_k(i)u_l(j)u_k\dotproduct u_l
=v_i\dotproduct v_j
&=1\text{ if }i=j,\cr
&=0\text{ otherwise.  \Qed}\cr}$$

Now $\gamma_k=Tu_k\dotproduct u_k\ge 0$, so $\sqrt{\gamma_k}$
is defined for each $k$, and we have a linear operator
$S:\BbbR^n\to\BbbR^J$ defined by setting $Se_k=\sqrt{\gamma_k}u_k$ for each
$k$, where $\ofamily{k}{n}{e_k}$ is the standard basis of $\BbbR^n$,
defined by saying that $e_k(l)=1$ if $k=l$, $0$ otherwise.

Taking $\mu_G^{(n)}$ to be the standard Gaussian distribution on $\BbbR^n$,
$\mu=\mu_G^{(n)}S^{-1}$ is a centered Gaussian distribution on $\BbbR^J$,
by 456Ba.   For $i$, $j\in J$,

$$\eqalign{\int w(i)w(j)\mu(dw)
&=\int (Sz)(i)(Sz)(j)\mu_G^{(n)}(dz)\cr
&=\int\sum_{k<n}\sqrt{\gamma_k}z(k)u_k(i)
  \cdot\sum_{l<n}\sqrt{\gamma_l}z(l)u_l(j)
   \mu_G^{(n)}(dz)\cr
&=\sum_{k<n}\sum_{l<n}\sqrt{\gamma_k\gamma_l}u_k(i)u_l(j)
  \int z(k)z(l)\mu_G^{(n)}(dz)\cr
&=\sum_{k<n}\gamma_k u_k(i)u_k(j)
=\sum_{k<n}Tu_k(i)u_k(j)\cr
&=\sum_{k<n,l\in J}\sigma_{il}u_k(l)u_k(j)
=\sum_{l\in J}\sigma_{il}\sum_{k<n}u_k(l)u_k(j)
=\sigma_{ij}.\cr}$$

\noindent So $\mu$ is the distribution we are looking for.

\medskip

{\bf (iii)$\Rightarrow$(i)} I seek to apply 454M.
For each finite $J\subseteq I$, let $\mu_J$ be
a centered Gaussian distribution on $\BbbR^J$ with covariance matrix
$\langle\sigma_{ij}\rangle_{i,j\in J}$;  by 456Bb, it is unique.   If
$K\subseteq I$ is finite and $J\subseteq K$, set $T_{KJ}z=z\restr J$ for
$z\in\BbbR^K$;  then $\mu_KT_{KJ}^{-1}$ is a centered Gaussian distribution on
$\BbbR^J$, by 456Ba, and its covariance matrix is that of $\mu_J$, so
$\mu_J=\mu_KT_{KJ}^{-1}$.
By 454M, we have a distribution $\mu$ on $\BbbR^I$, the completion of a
Baire probability measure, such that $\mu_J=\mu T_J^{-1}$ for every finite
$J\subseteq I$, setting $T_Jx=x\restr J$ for $x\in\BbbR^I$.

Applying this with $J=\{i,j\}$, we see that
$\int x(i)x(j)\mu(dx)=\sigma_{ij}$ for all $i$, $j\in I$.   To see that
$\mu$ is a centered Gaussian distribution in the sense of 456Ab, take
a continuous linear functional $f:\BbbR^I\to\Bbb R$.   Then there is a
finite family $\family{i}{J}{\beta_i}$ in $\Bbb R$
such that $f(x)=\sum_{i\in J}\beta_ix(i)$ for each $x\in\Bbb R^I$.
Setting $g(z)=\sum_{i\in J}\beta_iz(i)$ for $z\in\Bbb R^J$, we have
$f=gT_J$, so that the image distribution $\mu f^{-1}$ on $\Bbb R$ is
just $\mu_Jg^{-1}$, and (because $\mu_J$ is a centered Gaussian
distribution) is either normal or concentrated at $0$.   As $f$ is
arbitrary, $\mu$ itself is a centered Gaussian distribution.
}%end of proof of 456C

\leader{456D}{Gaussian \dvrocolon{processes}}\cmmnt{ I take a page
to spell out the connexion between centered Gaussian distributions,
and the processes considered in 454J-454K.

\medskip

\noindent}{\bf Definition} A family $\familyiI{X_i}$ of real-valued
random variables on a probability space is a
{\bf centered Gaussian process} if its distribution\cmmnt{ (454J)}
is a centered Gaussian distribution.

\leader{456E}{Independence and \dvrocolon{correlation}}\cmmnt{ We have an
important characterization of independence of families forming a Gaussian
process.   The essential idea is in (a) below.   I give the more elaborate
version (b) for the sake of an application in \S477.

\medskip

\noindent}{\bf Proposition}
(a)\dvAformerly{4{}56Xh} Let $\familyiI{X_i}$ be a centered
Gaussian process.   Then $\familyiI{X_i}$ is independent iff
$\Expn(X_i\times X_j)=0$ for all distinct $i$, $j\in I$.

(b)\dvAnew{2007}
Let $\familyiI{X_i}$ be a centered Gaussian process on a complete
probability space $(\Omega,\Sigma,\mu)$, and
$\Cal J$ a disjoint family of subsets of $I$;  for $J\in\Cal J$ let
$\Sigma_J$ be the $\sigma$-algebra of subsets of $\Omega$ generated by
$\{X_i^{-1}[F]:i\in J$, $F\subseteq\Bbb R$ is Borel$\}$.
Suppose that $\Expn(X_i\times X_j)=0$
whenever $J$, $J'$ are distinct members of $\Cal J$, $i\in J$ and
$j\in J'$.   Then $\family{J}{\Cal J}{\Sigma_J}$ is independent.

\proof{{\bf (a)(i)} If $\familyiI{X_i}$ is independent, and
$i$, $j\in I$ are
distinct, then $\Expn(X_i\times X_j)=\Expn(X_i)\Expn(X_j)=0$, by
272R\formerly{2{}72Q}.

\medskip

\quad{\bf (ii)}
If $\Expn(X_i\times X_j)=0$ for all distinct $i$, $j\in I$, let
$\mu$ be the distribution of $\familyiI{X_i}$ and
$\langle\sigma_{ij}\rangle_{i,j\in I}$ its covariance matrix.   Then
$\sigma_{ij}=0$ whenever $i\ne j$.   So if we take $\nu_i$ to be the
normal distribution on $\Bbb R$ with expectation $0$ and variance
$\sigma_{ii}$ (or the distribution concentrated at $0$ if $\sigma_{ii}=0$),
the product $\nu=\prod_{i\in I}\nu_i$ will be a centered
Gaussian distribution on $\BbbR^I$ (456Be) also with covariance matrix
$\langle\sigma_{ij}\rangle_{i,j\in I}$, and is equal to $\mu$, by 456Bb.
Thus $\mu$ is a product measure and $\familyiI{X_i}$ is independent
(454L).

\medskip

{\bf (b)} Set $K=\bigcup\Cal J$.
For each $J\in\Cal J$, let $\nu_J$ be the distribution of
$\family{i}{J}{X_i}$, and let $\nu=\prod_{J\in\Cal J}\nu_J$ be the product
measure on $\prod_{J\in\Cal J}\BbbR^J$, which we can identify with
$\BbbR^K$.   Then $\nu$ is a centered Gaussian
distribution, and its covariance matrix
$\langle\sigma_{ij}\rangle_{i,j\in K}$ is such that

$$\eqalign{\sigma_{ij}
&=\Expn(X_i\times X_j)\text{ if }i,\,j\text{ belong to the same member of }
   \Cal J,\cr
&=\Expn(X_i)\Expn(X_j)=0\text{ otherwise};\cr}$$

\noindent that is, it is the covariance matrix of the process
$\family{i}{K}{X_i}$.   Let $f:\Omega\to\BbbR^K$ be a function such that
$f(\omega)(i)=X_i(\omega)$ whenever $i\in K$ and $\omega\in\dom X_i$;
because $\mu$ is complete, $f$ is $(\Sigma,\CalBa(\BbbR^K))$-measurable.
For $J\in\Cal J$ and $\omega\in\Omega$
set $f_J(\omega)=f(\omega)\restr J$.

Now suppose that $\Cal J_0\subseteq\Cal J$ is non-empty
and finite and that $E_J\in\Sigma_J$ for each $J\in\Cal J_0$.   Then for
each $J\in\Cal J_0$ there is a Baire set $F_J\subseteq\BbbR^J$ such that
$E_J\symmdiff f_J^{-1}[F_J]$ is $\mu$-negligible, and
$\mu E_J=\nu_JF_J$.   Next, the distribution of $\family{i}{K}{X_i}$ is
a centered Gaussian distribution on $\BbbR^K$, and has covariance matrix
$\langle\Expn(X_i\times X_j)\rangle_{i,j\in K}
=\langle\sigma_{ij}\rangle_{i,j\in K}$, so it must be $\nu$.   But this
means that, setting
$F=\{x:x\in\BbbR^K$, $x\restr J\in F_J$ for every $J\in\Cal J_0\}
\in\CalBa(\BbbR^K)$,

$$\eqalign{\mu(\bigcap_{J\in\Cal J}E_J)
&=\mu(\bigcap_{J\in\Cal J}f_J^{-1}[F_J])
=\mu f^{-1}[F]\cr
&=\nu F
=\prod_{J\in\Cal J}\nu_JF_J
=\prod_{J\in\Cal J}\mu E_J.\cr}$$

\noindent As $\family{J}{\Cal J_0}{E_J}$ is arbitrary,
$\family{J}{\Cal J}{\Sigma_J}$ is independent.
}%end of proof of 456E

\leader{456F}{Proposition} Let $\familyiI{X_i}$ be
a family of random variables on a probability space
$(\Omega,\Sigma,\mu)$.   Then the following are equiveridical:

(i) the distribution of $\familyiI{X_i}$\cmmnt{, in the sense of
454K,} is a centered Gaussian distribution;

(ii) whenever $i_0,\ldots,i_n\in I$ and
$\alpha_0,\ldots,\alpha_n\in\Bbb R$ then $\sum_{r=0}^n\alpha_rX_{i_r}$
is either zero a.e.\ or a normal random variable with zero expectation;

(iii) whenever $i_0,\ldots,i_n\in I$ then the joint distribution of
$X_{i_0},\ldots,X_{i_n}$\cmmnt{, in the sense of 271C,} is a centered
Gaussian distribution;

(iv) whenever $J\subseteq I$ is finite then there is an independent
family $\family{k}{K}{Y_k}$ of standard normal random variables on
$\Omega$ such that each $X_i$, for $i\in J$, is almost everywhere equal
to a linear combination of the $Y_k$.

\proof{{\bf (ii)$\Leftrightarrow$(i)} is immediate from the definition in
456Ab and 454O.

\medskip

{\bf (i)$\Leftrightarrow$(iii)} is also direct from 456Ab and the
identification of the two concepts of `distribution' (454K).

\medskip

{\bf (iv)$\Rightarrow$(ii)} is direct from 274A-274B.

\medskip

{\bf (i)$\Rightarrow$(iv)} For $i\in J$ set $u_i=X_i^{\ssbullet}$ in
$L^2(\mu)$.   By 4A4Jh again there is an orthonormal family
$\family{k}{K}{v_k}$ in $L^2(\mu)$ such that each $v_k$ is a linear
combination of the $u_i$ and each $u_i$ is a linear combination of the
$v_k$.   Take $Y_k$ such that $Y_k^{\ssbullet}=v_k$ for each $k$;  then
each $X_i$ is equal almost everywhere to a linear combination of the
$Y_k$, while each $Y_k$ is equal almost everywhere to a linear
combination of the $X_i$.   As $\#(K)$ must be the dimension of the
linear span of $\{u_i:i\in J\}$, $K$ is finite.   Any linear combination
of the $Y_k$ is equal almost everywhere to a linear combination of the
$X_k$, so is either zero a.e.\ or a normal random variable with zero
expectation.   Because (ii)$\Rightarrow$(iii), $\family{k}{K}{Y_k}$ has
a centered Gaussian distribution $\nu$ say.   Each $Y_k$ has variance
$\innerprod{v_k}{v_k}=1$, so is a standard normal random variable.

The covariance matrix of $\nu$ is given by

$$\eqalignno{\int y(j)y(k)\nu(dy)
=\Expn(Y_j\times Y_k)
=\innerprod{v_j}{v_k}
&=1\text{ if }j=k,\cr
&=0\text{ otherwise}.\cr}$$

\noindent By 456E, $\family{k}{K}{Y_k}$ is independent, so we have found a
suitable family.
}%end of proof of 456F

\leader{456G}{}\cmmnt{ Now I start work on material for the main theorem
of this section.

\medskip

\noindent}{Lemma} Let $I$ be a finite set and $\mu$ a centered
Gaussian distribution on $\BbbR^I$.   Suppose that $\gamma\ge 0$ and
$\alpha=\mu\{x:\sup_{i\in I}|x(i)|\ge\gamma\}$.   Then
$\mu\{x:\sup_{i\in I}|x(i)|\ge\bover12\gamma\}\ge 2\alpha(1-\alpha)^3$.

\proof{{\bf (a)}
The case $\gamma=0$, $\alpha=1$ is trivial;  suppose that
$\gamma>0$.   We may suppose that $I$ has a total order $\le$.   Give
$\BbbR^{I\times 4}\cong(\BbbR^I)^4$ the product $\lambda$ of four
copies of $\mu$;  then $\lambda$ is a centered Gaussian distribution
(456Be).   Define $T:\BbbR^{I\times 4}\to\BbbR^I$ by setting
$(Ty)(i)=\bover12\sum_{r=0}^3y(i,r)$ for $i\in I$,
$y\in\BbbR^{I\times 4}$;  then $\lambda T^{-1}$ is a centered Gaussian
distribution on $\BbbR^I$ (456Ba).   Now $\lambda T^{-1}$ has the same
covariance matrix as $\mu$.   \Prf\ If $i$, $j\in I$ then

$$\eqalignno{\int x(i)x(j)(\lambda T^{-1})(dx)
&=\int (Ty)(i)(Ty)(j)\lambda(dy)
=\Bover14\sum_{r=0}^3\sum_{s=0}^3\int y(i,r)y(j,s)\lambda(dy)\cr
&=\Bover14\sum_{r=0}^3\int y(i,r)y(j,r)\lambda(dy)
=\Bover14\sum_{r=0}^3\int x(i)x(j)\mu(dx)\cr
\displaycause{because the map
$y\mapsto\familyiI{y(i,r)}:\BbbR^{I\times 4}\to\BbbR^I$ is \imp\ for
each $r$}
&=\int x(i)x(j)\mu(dx).  \text{ \Qed}\cr}$$

\noindent So $\lambda T^{-1}=\mu$, by 456Bb.

\medskip

{\bf (b)} Define

\Centerline{$E_{ir}=\{y:y\in\BbbR^{I\times 4}$,
  $|y(i,r)|\ge\gamma\}$}

\noindent for $r<4$ and $i\in I$, and

\Centerline{$E_r=\bigcup_{i\in I}E_{ir}$}

\noindent for  $r<4$, so that

\Centerline{$\lambda E_r=\lambda\{y:\sup_{i\in I}|y(i,r)|\ge\gamma\}
=\mu\{x:\sup_{i\in I}|x(i)|\ge\gamma\}=\alpha$.}

\noindent Set $E'_r=E_r\setminus\bigcup_{s\ne r}E_s$, so that

$$\eqalignno{\lambda E'_r
&=\lambda\{y:\sup_{i\in I}|y(i,r)|\ge\gamma,\,
  \sup_{i\in I}|y(i,s)|<\gamma\text{ for }s\ne r)\cr
&=\lambda\{y:\sup_{i\in I}|y(i,r)|\ge\gamma\}
  \cdot\prod_{s\ne r}\lambda\{y:\sup_{i\in I}|y(i,s)|<\gamma\}\cr
\displaycause{because these are independent events}
&=\alpha(1-\alpha)^3.\cr}$$

\medskip

{\bf (c)} Next, for $i\in I$ and $r<4$, set
$E'_{ir}=E_{ir}\setminus(\bigcup_{j<i}E_{jr}\cup\bigcup_{s\ne r}E_s)$,
so that $E'_r=\bigcup_{i\in I}E'_{ir}$.   Observe that
$\langle E'_{ir}\rangle_{i\in I,r<4}$ is disjoint.   Set

\Centerline{$F_{ir}=\{y:y\in E'_{ir}$,
$y(i,r)\sum_{s\ne r}y(i,s)\ge 0\}$.}

\noindent Then $\nu F_{ir}\ge\bover12\nu E'_{ir}$.   \Prf\ We can think
of $\BbbR^{I\times 4}$ as a product $\BbbR^J\times\BbbR^K$, where
$J=I\times\{r\}$ and $K=I\times(4\setminus\{r\})$.   In this case,
$\lambda$ becomes identified with a product $\lambda_J\times\lambda_K$,
where $\lambda_J$ and $\lambda_K$ are centered Gaussian distributions on
$\BbbR^J$ and $\BbbR^K$ respectively, and $E'_{ir}$ is of the form
$V\times W$, where

\Centerline{$V=\{v:v\in\BbbR^J$, $|v(i,r)|\ge\gamma$,
$|v(j,r)|<\gamma$ for $j<i\}$,}

\Centerline{$W=\{w:w\in\BbbR^K$, $|w(j,s)|<\gamma$ for every $j\in I$,
$s\ne r\}$.}

\noindent In the same representation, $F_{ir}$ becomes
$(V^+\times W^+)\cup(V^-\times W^-)$, where

\Centerline{$V^+=\{v:v\in V$, $v(i,r)\ge\gamma\}$,
\quad$V^-=\{v:v\in V$, $v(i,r)\le-\gamma\}$,}

\Centerline{$W^+=\{w:w\in W$, $\sum_{r\ne s}w(i,s)\ge 0\}$,
\quad$W^-=\{w:w\in W$, $\sum_{r\ne s}w(i,s)\le 0\}$.}

\noindent By 456Bf, $\lambda_JV^+=\lambda_JV^-$ and
$\lambda_KW^+=\lambda_KW^-$;  since $V^-=V\setminus V^+$, while
$W^+\cup W^-=W$, we have

\Centerline{$\lambda_JV^+=\lambda_JV^-=\Bover12\lambda_JV$,
\quad$\lambda_KW^+=\lambda_KW^-\ge\Bover12\lambda_KW$.}

\noindent But this means that

$$\eqalignno{\lambda F_{ir}
&=\lambda(V^+\times W^+)+\lambda(V^-\times W^-)\cr
&=\lambda_JV^+\cdot\lambda_KW^++\lambda_JV^-\cdot\lambda_KW^-
\ge\Bover12\lambda_JV\cdot\lambda_KW
=\Bover12\lambda E'_{ir},\cr}$$

\noindent as claimed.\ \Qed

\medskip

{\bf (d)} At this point, observe that if $y\in F_{ir}$ then
$|\sum_{s=0}^3y(i,s)|\ge|y(i,r)|\ge\gamma$.   So

$$\eqalign{\mu\{x:\sup_{i\in I}|x(i)|\ge\Bover12\gamma\})
&=\lambda T^{-1}[\{x:\sup_{i\in I}|x(i)|\ge\Bover12\gamma\})]\cr
&=\lambda\{y:\sup_{i\in I}
   |\Bover12\sum_{r=0}^3y(i,r)|\ge\Bover12\gamma\})\cr
&=\lambda\{y:\sup_{i\in I}|\sum_{r=0}^3y(i,r)|\ge\gamma)\cr
&\ge\lambda(\bigcup_{i\in I,r<4}F_{ir})
=\sum_{i\in I,r<4}\lambda F_{ir}\cr
&\ge\Bover12\sum_{r<4}\sum_{i\in I}\lambda E'_{ir}
\ge\Bover12\sum_{r<4}\lambda E'_r
=2\alpha(1-\alpha)^3,\cr}$$

\noindent which is what we set out to prove.
}%end of proof of 456G

\leader{456H}{The support of a Gaussian distribution:  Proposition} Let
$I$ be a set and $\mu$ a centered Gaussian distribution on $\BbbR^I$.
Write $Z$ for the set of those $x\in\BbbR^I$ such that $f(x)=0$ whenever
$f:\BbbR^I\to\Bbb R$ is a continuous linear functional and $f=0$ a.e.
Then $Z$ is a self-supporting closed linear subspace of $\BbbR^I$ with
full outer measure.   If $I$ is countable $Z$ is the support of $\mu$.

\proof{{\bf (a)} Being the intersection of a family of closed linear
subspaces, of course $Z$ is a closed linear subspace.

\medskip

{\bf (b)} $Z$ has full outer measure.   \Prf\ Let $F\subseteq\BbbR^I$
be a non-negligible zero set.   Let $J\subseteq I$ be a countable set
such that $F$ is determined by coordinates in $J$.   For $i\in I$ and
$x\in\BbbR^I$ set $\pi_i(x)=x(i)$;  then each $\pi_i$ is either normally
distributed or zero almost everywhere, so is square-integrable;  set
$u_i=\pi_i^{\ssbullet}$ in $L^2=L^2(\mu)$.
Let $\family{k}{K}{v_k}$ be a countable orthonormal family in $L^2$ such
that every $v_k$ is a linear combination of the $u_i$, for $i\in J$, and
every $u_i$, for $i\in J$, is a linear combination of the $v_k$ (4A4Jh
once more).
Extend $\family{k}{K}{v_k}$ to a Hamel basis $\family{l}{L}{v_l}$ of
$L^2$.   For every $i\in I$, we can express $u_i$ as
$\sum_{l\in L}\alpha_{il}v_l$, where $\{l:\alpha_{il}\ne 0\}$ is finite;
and the construction ensures that $\alpha_{il}=0$ if $i\in J$ and
$l\in L\setminus K$.

Consider the linear operator $T_0:\BbbR^K\to\BbbR^J$ defined by setting
$(T_0z)(i)=\sum_{k\in K}\alpha_{ik}z(k)$ for $z\in\BbbR^K$ and $i\in J$.
If we give $\BbbR^K$ the product measure $\mu_G^{(K)}$, then the image
measure $\mu_G^{(K)}T_0^{-1}$ is a Gaussian distribution (456Ba), with
covariance matrix

$$\eqalign{\sigma_{ii'}
&=\int x(i)x(i')(\mu_G^{(K)}T_0^{-1})dx
=\int(T_0z)(i)(T_0z)(i')\mu_G^{(K)}(dz)\cr
&=\sum_{k,k'\in K}\alpha_{ik}\alpha_{i'k'}z(k)z(k')\mu_G^{(K)}(dz)
=\sum_{k\in K}\alpha_{ik}\alpha_{i'k}\cr
&=\sum_{k,k'\in K}\alpha_{ik}\alpha_{i'k}\innerprod{v_k}{v_{k'}}
=\innerprod{u_i}{u_{i'}}
=\int x(i)x(i')\mu(dx).\cr}$$

\noindent But this means that $\mu_G^{(K)}T_0^{-1}$ has the same
covariance matrix as $\mu\tilde\pi_J^{-1}$, where
$\tilde\pi_Jx=x\restr J$ for $x\in\BbbR^I$.   Since this also is a
centered Gaussian distribution, the two measures must be equal (456Bb).
We know that $\tilde\pi_J[F]$ has non-zero measure, so there is a
$z_0\in\BbbR^K$ such that $T_0z_0\in\tilde\pi_J[F]$.   Extend $z_0$
arbitrarily to $z_1\in\BbbR^L$.

Set $x_1(i)=\sum_{l\in L}\alpha_{il}z_1(l)$ for $i\in I$.   Then
$\tilde\pi_Jx_1=T_0z_0\in\tilde\pi_J[F]$, so $x_1\in F$, because $F$ is
determined by coordinates in $J$.   If a continuous linear functional
$f:\BbbR^I\to\Bbb R$ is zero a.e., it can be expressed in the form
$f(x)=\sum_{i\in I}\beta_ix(i)$ where $\{i:\beta_i\ne 0\}$ is finite
(4A4Be again).   In this case,

\Centerline{$0=f^{\ssbullet}=\sum_{i\in I}\beta_iu_i
=\sum_{l\in L}\sum_{i\in I}\beta_i\alpha_{il}v_l$}

\noindent in $L^2$.   Since $\family{l}{L}{v_l}$ is linearly
independent, $\sum_{i\in I}\beta_i\alpha_{il}=0$ for every $l\in L$.
But this means that

\Centerline{$f(x_1)=\sum_{i\in I,l\in L}\beta_i\alpha_{il}z_1(l)=0$.}

\noindent As $f$ is arbitrary, $x_1\in Z$ and $Z\cap F\ne\emptyset$.
As $F$ is arbitrary, and $\mu$ is inner regular with respect to the zero
sets, $Z$ has full outer measure.\ \Qed

\medskip

{\bf (c)} $Z$ is self-supporting.   \Prf\ If $W\subseteq\BbbR^I$ is an
open set meeting $Z$, there is an open set $V$, depending on coordinates
in a finite set $J\subseteq I$, such that $V\subseteq W$ and
$V\cap Z\ne\emptyset$.   Write $\tilde\pi_J(x)=x\restr J$ for
$x\in\BbbR^I$, and $\nu_J$ for the image measure $\mu\tilde\pi_J^{-1}$
on $\BbbR^J$;  by 456Ba, this is a centered Gaussian distribution.   By
456Bd, there is a continuous linear operator $T:\BbbR^{\Bbb
N}\to\BbbR^J$ such that $\nu_J=\mu_G^{(\Bbb N)}T^{-1}$.   Since the
support of $\mu_G$ is
$\Bbb R$, the support of $\mu_G^{(\Bbb N)}$ is $\BbbR^{\Bbb N}$
(417E(iv), or otherwise), and the support of $\nu_J$ is
$Z_1=\overline{T[\BbbR^{\Bbb N}]}$ (411Ne).

Write $Q$ for the set of linear functionals $g:\BbbR^J\to\Bbb R$
(necessarily continuous, because $J$ is finite) which
are zero on $Z_1$.   If $g\in Q$, then $gT=0$, so $g=0\,\,\nu_J$-a.e.\
and $g\tilde\pi_J=0\,\,\mu$-a.e.   This means that $g\tilde\pi_J(x)=0$
for every $x\in Z$, that is, $g(y)=0$ for every $y\in\tilde\pi_J[Z]$.
Because $Z_1$ is a linear subspace of $\BbbR^J$, this is enough to show
that $\tilde\pi_J[Z]\subseteq Z_1$.

Now recall that $V\cap Z\ne\emptyset$ so
$Z_1\cap\tilde\pi_J[V]\ne\emptyset$, while
$V=\tilde\pi_J^{-1}[\tilde\pi_J[V]]$.   Since $\tilde\pi_J$ is an open
map (4A2B(f-i)), $\tilde\pi_J[V]$ is open and

\Centerline{$\mu^*(W\cap Z)=\mu W\ge\mu V=\nu_J\tilde\pi_J[V]>0$,}

\noindent because $Z_1$ is the support of $\nu_J$.
\Qed

\medskip

{\bf (d)} If $I$ is countable, $\mu$ is a topological measure so
measures $Z$, and $Z$ is the support of $\mu$.
}%end of proof of 456H

\leader{456I}{Remarks (a)} In the context of 456H, I will call $Z$ the
{\bf support} of the centered Gaussian distribution $\mu$, even though
$\mu$ need not be a topological measure\cmmnt{,
so the definition 411Nb is not immediately applicable}.
\cmmnt{In 456P we shall see that $Z$ really is the support of a canonical extension of $\mu$.}

\spheader 456Ib\cmmnt{ It is worth making one elementary point at once.}   If
$I$ and $J$ are sets, $\mu$ and $\nu$ are centered Gaussian
distributions on $\BbbR^I$ and $\BbbR^J$ respectively with supports $Z$
and $Z'$, and $T:\BbbR^I\to\BbbR^J$ is an \imp\ continuous linear
operator, then $Tz\in Z'$ for every $z\in Z$.   \prooflet{\Prf\ If
$g:\BbbR^J\to\Bbb R$ is a continuous linear functional which is zero
$\nu$-a.e., then $gT:\BbbR^I\to\Bbb R$ is a continuous linear functional
which is zero $\mu$-a.e., so $g(Tz)=(gT)(z)=0$.\ \Qed}

\leader{456J}{Universal Gaussian distributions:  Definition} A centered
Gaussian distribution on $\BbbR^I$ is {\bf universal} if its covariance
matrix $\langle\sigma_{ij}\rangle_{i,j\in I}$ is the inner product for a
Hilbert space structure on $I$.   \cmmnt{(See 456Xe.)}

\leader{456K}{Proposition} Let $I$ be any set, and $\mu$ a centered
Gaussian distribution on $I$.   Then there are a set $J$, a universal
centered Gaussian distribution $\nu$ on $\BbbR^J$, and a continuous
\imp\ linear operator $T:\BbbR^J\to\BbbR^I$.

\proof{{\bf (a)} Set $J=L^2(\mu)$.   Then for any finite
$K\subseteq J$ there is a centered Gaussian distribution $\mu_K$ on
$\BbbR^K$ such that $\int x(u)x(v)\mu(dx)=\innerprod{u}{v}$ for all $u$,
$v\in K$.   \Prf\ If $K=\emptyset$ or $K=\{0\}$ this is trivial, as we
take $\mu$ to be the trivial distribution concentrated at $0$.
Otherwise, let $\ofamily{i}{n}{w_i}$ be an orthonormal basis for the
linear subspace of $J$ generated by $K$.   For each $u\in K$, express
it as $\sum_{i=0}^{n-1}\alpha_{ui}w_i$.   Define $T:\BbbR^n\to\BbbR^K$
by setting $(Tz)(u)=\sum_{i=0}^{n-1}\alpha_{ui}z(i)$ for $z\in\BbbR^n$
and $u\in K$.   Set $\mu_K=\mu_G^{(n)}T^{-1}$.   Then $\mu_K$ is a
centered Gaussian distribution, by 456Ba, and its covariance matrix is
given by

$$\eqalign{\int x(u)x(v)\mu_K(dx)
&=\int(Ty)(u)(Ty)(v)\mu_G^{(n)}(dy)
=\sum_{i=0}^{n-1}\sum_{j=0}^{n-1}\alpha_{ui}\alpha_{vj}
  \int y(i)y(j)\mu_G^{(n)}(dy)\cr
&=\sum_{i=0}^{n-1}\sum_{j=0}^{n-1}\alpha_{ui}\alpha_{vj}
  \innerprod{w_i}{w_j}
=\innerprod{u}{v}\cr}$$

\noindent for all $u$, $v\in K$.   So $\mu_K$ is the distribution we
seek.\ \Qed

\medskip

{\bf (b)} If $K\subseteq J$ is finite and $L\subseteq K$, then
$\mu_L=\mu_K\pi_{KL}^{-1}$, where $\pi_{KL}(x)=x\restr L$ for
$x\in\BbbR^K$.   \Prf\ Since $\mu_K\pi_{KL}^{-1}$ is a centered Gaussian
distribution, all we have to do is to check its covariance matrix.   But
if $u$, $v\in L$ then

$$\eqalign{\int y(u)y(v)(\mu_K\pi_{KL}^{-1})(dy)
&=\int(\pi_{KL}x)(u)(\pi_{KL}x)(v)\mu_K(dx)\cr
&=\int x(u)x(v)\mu_K(dx)
=\innerprod{u}{v}
=\int y(u)y(v)\mu_L(dy).\cr}$$

\noindent By 456Bb, $\mu_L=\mu_K\pi_{KL}^{-1}$.\ \Qed

\medskip

{\bf (c)} By 454G, there is a Baire measure $\nuprime$ on $\BbbR^{J}$
such that $\nuprime\pi_{JK}^{-1}[E]=\mu_KE$ for every finite
$K\subseteq J$ and every Borel set $E\subseteq\BbbR^K$.
Take $\nu$ to be the
completion of $\nuprime$.   Then $\pi_{JK}$ is \imp\ for $\nu$ and
$\mu_K$, for every finite $K\subseteq J$.   If
$f:\BbbR^{J}\to\Bbb R$ is a continuous linear functional, there are a
finite $K\subseteq J$ and a linear functional $g:\BbbR^K\to\Bbb R$
such that $f=\pi_{JK}g$, so that

\Centerline{$\nu\{x:f(x)\le\alpha\}=\mu_K\{x:g(x)\le\alpha\}$}

\noindent for every $\alpha$, and $f$ and $g$ have the same
distribution;  as $g$ is either normal with zero expectation or zero
a.e., so
is $f$.   As $f$ is arbitrary, $\nu$ is a centered Gaussian
distribution.

\medskip

{\bf (d)} $\nu$ is universal.   \Prf\ If $u$, $v\in J$ set
$K=\{u,v\}$.   Then

$$\eqalign{\int x(u)x(v)\nu(dx)
&=\int(\pi_{JK}x)(u)(\pi_{JK}x)(v)\nu(dx)\cr
&=\int y(u)y(v)\mu_K(dy)
=\innerprod{u}{v}.\cr}$$

\noindent Thus the covariance matrix of $\nu$ is just the inner product
of the standard Hilbert space structure of $J$.\ \Qed

\medskip

{\bf (e)} For $i\in I$, let $u_i\in J$ be the equivalence class of the
square-integrable function $x\mapsto x(i):\BbbR^I\to\Bbb R$.   Define
$T:\BbbR^J\to\BbbR^I$ by setting $(Ty)(i)=y(u_i)$ for every
$i\in I$ and $y\in\BbbR^J$.   Then there is a centered Gaussian
distribution $\mu'$ on $\BbbR^I$ such that $T$ is \imp\ for $\nu$ and
$\mu'$.   Now the covariance matrix of $\mu'$ is defined by

$$\eqalign{\int x(i)x(j)\mu'(dx)
&=\int(Ty)(i)(Ty)(j)\nu(dy)
=\int y(u_i)y(u_j)\nu(dy)\cr
&=\innerprod{u_i}{u_j}
=\int x(i)x(j)\mu(dx)\cr}$$

\noindent for all $i$, $j\in I$.   So $\mu$ and $\mu'$ are equal and $T$
is \imp\ for $\nu$ and $\mu$.
}%end of proof of 456K

\leader{456L}{Lemma} Let $\mu$ be a universal centered Gaussian
distribution on $\BbbR^I$;  give $I$ a corresponding Hilbert space
structure such that $\int x(i)x(j)\mu(dx)=\innerprod{i}{j}$ for all $i$,
$j\in I$.   Let $F\in\dom\mu$ be a set determined by coordinates
in $J$, where $J\subseteq I$ is a closed linear subspace for the Hilbert
space structure of $I$.   Let $W$ be the union of all the open subsets
of $\BbbR^I$ which meet $F$ in a negligible set, and $W'$ the union of
the open subsets of $\BbbR^I$ which meet $F$ in a negligible set and are
determined by coordinates in $J$.   If $F\subseteq W$ then
$F\subseteq W'$.

\proof{{\bf (a)} Let $Z$ be the support of $\mu$ in the sense of 456H.
We need to know that $Z$ is just the set of all linear functionals from
$I$ to $\Bbb R$.   \Prf\ If $K\subseteq I$ is finite and
$\family{i}{K}{\alpha_i}\in\BbbR^K$ and
$f(x)=\sum_{i\in I}\alpha_ix(i)$ for $x\in\BbbR^I$, then

\Centerline{$\|f\|_2^2
=\sum_{i,j\in K}\alpha_i\alpha_j\innerprod{i}{j}
=\|\sum_{i\in K}\alpha_ii\|^2$.}

\noindent So, for $x\in\BbbR^I$,

$$\eqalign{x\in Z
&\iff f(x)=0\text{ whenever }f\in(\BbbR^I)^*\text{ and }\|f\|_2=0\cr
&\iff \sum_{i\in K}\alpha_ix(i)=0\text{ whenever }K\subseteq I
  \text{ is finite and }\sum_{i\in K}\alpha_ii=0\text{ in }I\cr
&\iff x:I\to\Bbb R\text{ is linear.  \Qed}\cr}$$

\medskip

{\bf (b)} Let $J^{\perp}$ be the orthogonal complement of $J$ in $I$, so
that $I=J\oplus J^{\perp}$ (4A4Jf).   Give $\BbbR^J$ and
$\BbbR^{J^{\perp}}$ the centered Gaussian
distributions $\mu_J$, $\mu_{J^{\perp}}$ induced by $\mu$ and the
projections
$x\mapsto x\restr J$, $x\mapsto x\restr J^{\perp}$.   Then the product
measure $\lambda$ on $\BbbR^J\times\BbbR^{J^{\perp}}$ is also a centered
Gaussian distribution (456Be).   Define
$T:\BbbR^J\times\BbbR^{J^{\perp}}\to\BbbR^I$ by setting
$T(u,v)(j+k)=u(j)+v(k)$ whenever $j\in J$, $k\in J^{\perp}$,
$u\in\BbbR^J$ and $v\in\BbbR^{J^{\perp}}$.   Then $T$ is \imp\ for
$\lambda$ and $\mu$.   \Prf\ $T$ is a continuous linear operator so we
have a centered Gaussian distribution $\mu'$ on $\BbbR^I$ such that $T$
is \imp\ for $\lambda$ and $\mu'$.   If $j$, $j'\in J$ and $k$,
$k'\in J^{\perp}$,

$$\eqalign{\int x(j+k)x(j'+k')\mu'(dx)
&=\int T(u,v)(j+k)T(u,v)(j'+k')\lambda(d(u,v))\cr
&=\int (u(j)+v(k))(u(j')+v(k'))\lambda(d(u,v))\cr
&=\int u(j)u(j')\mu_J(du)+\int v(k)v(k')\mu_{J^{\perp}}(dv)\cr
&=\int x(j)x(j')\mu(dx)+\int x(k)x(k')\mu(dx)\cr
&=\innerprod{j}{j'}+\innerprod{k}{k'}
=\innerprod{j+k}{j'+k'}\cr
&=\int x(j+k)x(j'+k')\mu(dx).\cr}$$

\noindent Thus $\mu$ and $\mu'$ have the same covariance matrix and are
equal, and $T$ is \imp\ for $\lambda$ and $\mu$.\ \Qed

\medskip

{\bf (c)} Take any $z\in W\cap Z$.   Then there is an open set $V$,
determined by coordinates in a finite set $K_0\subseteq I$, such that
$z\in V$ and $\mu(V\cap F)=0$.   Let $\epsilon>0$ be such that $y\in V$
whenever $x\in\BbbR^I$ and $|x(i)-z(i)|<2\epsilon$ for every $i\in K_0$.
Express each $k\in K_0$ as $k'+k''$
where $k'\in J$ and $k''\in J^{\perp}$.   Set

\Centerline{$V'
=\{x:x\in\BbbR^I,\,|x(k')-z(k')|<\epsilon$ for every $k\in K_0\}$.}

\noindent Then $V'$ is an open set, determined by coordinates in $J$,
and contains $z$.   Also $\mu(V'\cap F)=0$.   \Prf\ Set $V''
=\{x:x\in\BbbR^I$, $|x(k'')-z(k'')|<\epsilon$ for every $k\in K_0\}$.
Then $V\supseteq V'\cap V''\cap Z$.   Since $Z$ has full outer measure
(456H),

\Centerline{$\mu(V'\cap V''\cap F)
=\mu^*(V'\cap V''\cap F\cap Z)\le\mu(V\cap F)=0$.}

Now

$$\eqalignno{0
&=\mu(V'\cap V''\cap F)
=\lambda T^{-1}[V'\cap F\cap V'']\cr
&=\lambda\{(x\restr J,x\restr J^{\perp}):x\in V'\cap F\cap V''\}\cr
\displaycause{because $V'\cap F\cap V''$ is determined by coordinates in
$J\cup J^{\perp}$}
&=\mu_J\{x\restr J:x\in V'\cap F\}
  \cdot\mu_{J^{\perp}}\{x\restr J^{\perp}:x\in V''\}\cr}$$

\noindent because $V'\cap F$ is determined by coordinates in $J$, while
$V''$ is determined by coordinates in $J^{\perp}$.   However,
$z\in V''$, and $z\restr J^{\perp}$ belongs to the support $Z'$ of
$\mu_{J^{\perp}}$, by 456Ib;   since $Z'$ is self-supporting, and
$\{x\restr J^{\perp}:x\in V''\}$ is open,
$\mu_{J^{\perp}}\{x\restr J^{\perp}:x\in V''\}>0$.   We conclude that

\Centerline{$0=\mu_J\{x\restr J:x\in V'\cap F\}=\mu(V'\cap F)$.  \Qed}

\medskip

{\bf (d)} This shows that $z\in W'$.   As $z$ is arbitrary,
$W\cap Z\subseteq W'$.

\Quer\ Suppose, if possible, that there is a point
$z_0\in F\setminus W'$.   If $i$, $j\in J$ and $\alpha\in\Bbb R$, then

\Centerline{$\{x:x(i+j)\ne x(i)+x(j)\}$,
\quad$\{x:x(\alpha i)\ne\alpha x(i)\}$}

\noindent are negligible open sets determined by coordinates in $J$, so
are included in $W'$ and do not contain $z_0$.   Thus
$z_0\restr J:J\to\Bbb R$ is linear.   Let $z:I\to\Bbb R$ be a linear
functional extending $z_0\restr J$.   Then $z\in Z$ and
$z\restr J=z_0\restr J$;  as both $F$ and $W'$ are determined by
coordinates in $J$, $z\in F\setminus W'$.   But this means that
$z\in Z\cap W\setminus W'$, which is impossible.\ \Bang

So $F\subseteq W'$, as claimed.
}%end of proof of 456L

\leader{456M}{Cluster sets:  Lemma} Let $I$ be a countable set, $n\ge 1$
an integer and
$\mu$ a centered Gaussian distribution on $\BbbR^{I\times n}$.
For $\epsilon>0$ set

\Centerline{$I_{\epsilon}
=\{i:i\in I$, $\int|x(i,r)|^2\mu(dx)\le\epsilon^2$ for every $r<n\}$;}

\noindent suppose that no $I_{\epsilon}$ is empty.

(a) There is a closed set $F\subseteq\BbbR^n$ such that

\Centerline{$F=\bigcap_{\epsilon>0}
\overline{\{\ofamily{r}{n}{x(i,r)}:i\in I_{\epsilon}\}}$}

\noindent for almost every $x\in\BbbR^{I\times n}$.

(b) If $z\in F$ and $-1\le\alpha\le 1$, then $\alpha z\in F$.

(c) If $F$ is bounded, then there is some $\epsilon>0$ such that
$\sup_{i\in I_{\epsilon},r<n}|x(i,r)|<\infty$ for almost every
$x\in\BbbR^{I\times n}$.

\proof{{\bf (a)(i)} For $x\in\BbbR^{I\times n}$ and $i\in I$ set
$S_i(x)=\ofamily{r}{n}{x(i,r)}\in\BbbR^n$.   For $x\in\BbbR^{I\times n}$
set

\Centerline{$F_x
=\bigcap_{\epsilon>0}\overline{\{S_ix:i\in I_{\epsilon}\}}$,}

\noindent so that $F_x$ is a closed subset of $\BbbR^n$.   For
$A\subseteq\BbbR^n$ set
$E_A=\{x:x\in\BbbR^{I\times n}$, $A\cap F_x\ne\emptyset\}$.

\medskip

\quad{\bf (ii)} By 456Bd, there is a continuous linear operator
$T:\BbbR^{\Bbb N}\to\BbbR^{I\times n}$ such that
$\mu=\mu_G^{(\Bbb N)}T^{-1}$.   Set $T_i=S_iT$ for $i\in I$;  then
$T_i:\BbbR^{\Bbb N}\to\BbbR^n$ is a continuous linear operator.   For
$y\in\BbbR^{\Bbb N}$, set

\Centerline{$\tilde F_y=F_{T(y)}
=\bigcap_{\epsilon>0}\overline{\{T_i(y):i\in I_{\epsilon}\}}$.}

\noindent For $A\subseteq\BbbR^n$ set

\Centerline{$\tilde E_A=T^{-1}[E_A]
=\{y:y\in\BbbR^{\Bbb N}$, $A\cap\tilde F_y\ne\emptyset\}$.}

\medskip

\quad{\bf (iii)} If $K\subseteq\BbbR^n$ is compact, then
$\tilde E_K$ is a Borel subset of $\BbbR^{\Bbb N}$.   \Prf\
Let $\Cal V$ be a countable base for the topology of $\BbbR^n$, and for
$k\ge 1$ let $\Cal V_k$ be the set of members of $\Cal V$ with diameter
at most $1/k$ which meet $K$.   Then $T_i^{-1}[V]$ is a Borel set for
every $V\in\Cal V$ and $i\in I$, so

\Centerline{$E'
=\bigcap_{k\ge 1}\bigcup_{V\in\Cal V_k}\bigcup_{i\in I_{1/k}}
  T_i^{-1}[V]$}

\noindent is a Borel set.

If $y\in\tilde E_K$, take $k\ge 1$.   There is a
$z\in\tilde F_y\cap K$.   Let $V\in\Cal V$ be such that $z\in V$ and
$\diam V\le 1/k$;  in this case $V\in\Cal V_k$.   Because
$z\in\overline{\{T_i(y):i\in I_{1/k}\}}$, there is an
$i\in I_{1/k}$ such that $T_i(y)\in V$.   As $k$ is arbitrary, this
shows that $y\in E'$;  thus $\tilde E_K\subseteq E'$.

If $y\notin\tilde E_K$, then $K$ is a compact set disjoint from the
closed set $\tilde F_y$.   There is therefore some $\epsilon>0$ such
that $K\cap\overline{\{T_i(y):i\in I_{\epsilon}\}}=\emptyset$ (since
these form a downwards-directed family of compact sets with empty
intersection).   Next, there is a $\delta>0$ such that
$B(z,\delta)\cap\overline{\{T_i(y):i\in I_{\epsilon}\}}=\emptyset$ for
every $z\in K$ (2A2Ed).   Let $k\ge 1$ be such that
$1/k\le\min(\epsilon,\delta)$.   If $V\in\Cal V_k$ and
$i\in I_{1/k}$, there is some $z\in K\cap V$ so $V\subseteq B(z,\delta)$
and $T_i(y)\notin V$.  This shows that $y\notin E'$.   As $y$ is
arbitrary, $E'\subseteq\tilde E_K$.

So $\tilde E_K=E'$ is a Borel set.\ \Qed

It follows at once that $\tilde E_H$ is a Borel set for every
K$_{\sigma}$ set $H$, in particular, for any open or closed set $H$.

\medskip

\quad{\bf (iv)} We need a simple estimate on the coefficients of the
linear operators $T_i$.   Let $\alpha_{irj}$ be such that
$T_i(y)=\ofamily{r}{n}{\sum_{j=0}^{\infty}\alpha_{irj}y(j)}$ for
$i\in I$, $r<n$ and $y\in\BbbR^{\Bbb N}$.   (Of course
$\{j:\alpha_{irj}\ne 0\}$ is finite for each $i$ and $r$.)   Then

\Centerline{$\biggerint x(i,r)^2\mu(dx)
=\int T_i(y)(r)^2\mu_G^{(\Bbb N)}(dy)
=\sum_{j=0}^{\infty}\alpha_{irj}^2$,}

\noindent so $|\alpha_{irj}|\le\epsilon$ whenever $i\in I_{\epsilon}$,
$r<n$ and $j\in\Bbb N$.

\medskip

\quad{\bf (v)} Now suppose that $H\subseteq\BbbR^n$ is open, that
$K\subseteq H$ is compact, and that $\mu_G^{(\Bbb N)}\tilde E_K>0$.
Then $\mu_G^{(\Bbb N)}\tilde E_H=1$.   \Prf\ Let $\epsilon>0$.   Let
$\sequence{j}{\epsilon_j}$ be a sequence of strictly positive real
numbers such that $\sum_{j=0}^{\infty}\epsilon_j
\le\bover12\min(\epsilon,\mu_G^{(\Bbb N)}\tilde E_K)$, and for each
$j\in\Bbb N$ let $\gamma_j\ge 0$ be such that
$\mu_G[-\gamma_j,\gamma_j]\ge 1-\epsilon_j$.   Then

\Centerline{$\tilde E=\{y:y\in\tilde E_K$, $|y(j)|\le\gamma_j$ for every
$j\in\Bbb N\}$}

\noindent has measure at least
$\mu_G^{(\Bbb N)}\tilde E_K-\sum_{j=0}^{\infty}\epsilon_j>0$.   By
254Sb, there are an $m\in\Bbb N$ and a set $\tilde E'$, of measure at
least $1-\bover12\epsilon$, such that for every $y'\in\tilde E'$ there
is a $y\in\tilde E$ such that $y(j)=y'(j)$ whenever
$j>m$.   Set $\tilde E''
=\{y:y\in\tilde E'$, $|y(j)|\le\gamma_j$ for every $j\in\Bbb N\}$, so
that $\mu_G^{(\Bbb N)}\tilde E''\ge 1-\epsilon$.

Let $\delta>0$ be such that $z'\in H$ whenever $z\in K$ and
$\|z-z'\|\le 2\delta$.   Let $\eta>0$ be such that
$2\eta\sqrt{n}\sum_{j=0}^m\gamma_j\le\delta$.   If $y'\in\tilde E''$ and
$i\in I_{\eta}$, there is a $y\in\tilde E$ such that
$y(j)=y'(j)$ for $j>m$.   Also
$|y(j)-y'(j)|\le 2\gamma_j$ for $j\le m$, so

\Centerline{$|T_i(y)(r)-T_i(y')(r)|
\le\sum_{j=0}^{\infty}|\alpha_{irj}||y(j)-y'(j)|
\le\sum_{j=0}^m2\eta\gamma_j
\le\Bover{\delta}{\sqrt n}$}

\noindent for every $r<n$, and $\|T_i(y)-T_i(y')\|\le\delta$.

Now $\tilde F_y\cap K\ne\emptyset$;  take $z\in\tilde F_y\cap K$.   For
every $\zeta>0$, there is an $i\in I_{\min(\eta,\zeta)}$ such that
$\|z-T_i(y)\|\le\delta$, so that $\|z-T_i(y')\|\le 2\delta$.
This means that $B(z,2\delta)\cap\{T_i(y'):i\in I_{\zeta}\}$ is not empty.
As $B(z,2\delta)$ is compact, it must meet $\tilde F_{y'}$.   But this
means that $H\cap\tilde F_{y'}\ne\emptyset$, by the choice of $\delta$.
As $y'$ is arbitrary, $\tilde E''\subseteq\tilde E_H$, while
$\mu_G^{(\Bbb N)}\tilde E''\ge 1-\epsilon$.

This works for every $\epsilon>0$.   So $\tilde E_H$ is conegligible, as
claimed.\ \Qed

\medskip

\quad{\bf (vi)} If $H\subseteq\BbbR^n$ is open and
$\mu_G^{(\Bbb N)}\tilde E_H>0$, then (because $H$ is $\sigma$-compact)
there is a compact set $K\subseteq H$ such that
$\mu_G^{(\Bbb N)}\tilde E_K>0$, and (e) tells us that
$\mu_G^{(\Bbb N)}\tilde E_H=1$.

Set

\Centerline{$\Cal V_0
=\{V:V\in\Cal V$, $\mu_G^{(\Bbb N)}\tilde E_V=0\}
=\{V:V\in\Cal V$, $\mu_G^{(\Bbb N)}\tilde E_V<1\}$.}

\noindent Then we see that

\Centerline{$\Cal V_0=\{V:V\in\Cal V$, $\tilde F_y\cap V=\emptyset\}$}

\noindent for almost every $y\in\BbbR^{\Bbb N}$, that is,

\Centerline{$\Cal V_0=\{V:V\in\Cal V$, $F_x\cap V=\emptyset\}$}

\noindent for almost every $x\in\BbbR^{I\times n}$.   But as every $F_x$
is closed, we have
$F_x=\BbbR^n\setminus\bigcup\Cal V_0$ for almost every $x$.   So we can
set $F=\BbbR^n\setminus\bigcup\Cal V_0$.

\medskip

{\bf (b)(i)} Give $\BbbR^{I\times 2n}\cong(\BbbR^{I\times n})^2$ the
measure $\lambda$ corresponding to the product measure $\mu\times\mu$;
by 456Be, this is a centered Gaussian distribution.   For
$(x_1,x_2)\in(\BbbR^{I\times n})^2$, set

\Centerline{$F'_{x_1x_2}=\bigcap_{\epsilon>0}
  \overline{\{(S_ix_1,S_ix_2):i\in I_{\epsilon}\}}$.}

\noindent By (a), we have a closed set $F'\subseteq\BbbR^{2n}$ such that
$F'=F'_{x_1x_2}$ for almost all $x_1$, $x_2$.   (Of course
$I_{\epsilon}=\{i:\int|x_j(i,r)|^2\lambda(dx)\le\epsilon^2$ for every
$j\in\{1,2\}$, $r<n\}$ whenever $\epsilon>0$.)

\medskip

\quad{\bf (ii)} Now $(z,0)\in F'$.   \Prf\ Take
$x_1\in\BbbR^{I\times n}$ such that $F=F_{x_1}$ and
$E=\{x_2:F'=F'_{x_1x_2}\}$ is conegligible.   (Almost every point of
$\BbbR^{I\times n}$ has these properties.)   For $k\in\Bbb N$,
$z\in\overline{\{S_ix_1:i\in I_{2^{-k}}\}}$;  let
$i_k\in I_{2^{-k}}$ be such that
$\sum_{r<n}|z(r)-x_1(i_k,r)|\le 2^{-k}$.   Next,

\Centerline{$\sum_{k=0}^{\infty}\sum_{r=0}^{n-1}
  $$\biggerint|x(i_k,r)|^2\mu(dx)
\le$$\sum_{k=0}^{\infty}2^{-2k}n<\infty$,}

\noindent so $\sum_{k=0}^{\infty}\sum_{r=0}^{n-1}|x(i_k,r)|^2$ is finite
for almost every $x\in\BbbR^I$, and there must be an
$x_2\in E$ such that
$\sum_{k=0}^{\infty}\sum_{r=0}^{n-1}|x_2(i_k,r)|^2$ is finite.   But in
this case $\lim_{k\to\infty}x_2(i_k,r)=0$ for every $r$, while
$\lim_{k\to\infty}x_1(i_k,r)=z(r)$.   Accordingly
$(z,0)\in F'_{x_1x_2}=F'$.\ \Qed

\medskip

\quad{\bf (iii)} Set $\beta=\sqrt{1-\alpha^2}$ and define
$\tilde T:(\BbbR^{I\times n})^2\to\BbbR^{I\times n}$ by setting
$\tilde T(x_1,x_2)=\alpha x_1+\beta x_2$ for $x_1$,
$x_2\in\BbbR^{I\times n}$.    Then $\tilde T$ is a continuous linear
operator, so the image measure $\lambda\tilde T^{-1}$ is a centered
Gaussian distribution on $\BbbR^{I\times n}$ (456Ba).   Moreover, it has
the same covariance matrix as $\mu$.   \Prf\ If $i$, $j\in I$ then

$$\eqalign{\int x(i)x(j)(\lambda\tilde T^{-1})(dx)
&=\int\tilde T(x_1,x_2)(i)\tilde T(x_1,x_2)(j)\lambda(d(x_1,x_2))\cr
&=\int(\alpha x_1(i)+\beta x_2(i))
  (\alpha x_1(j)+\beta x_2(j))\lambda(d(x_1,x_2))\cr
&=\alpha^2\int x_1(i)x_1(j)\lambda(d(x_1,x_2))
  +\beta^2\int x_2(i)x_2(j)\lambda(d(x_1,x_2))\cr
&=(\alpha^2+\beta^2)\int x(i)x(j)\mu(dx)
=\int x(i)x(j)\mu(dx).  \text{ \Qed}\cr}$$

\noindent So $\lambda\tilde T^{-1}=\mu$ (456Bb).

\medskip

\quad{\bf (iv)} If $x_1$, $x_2\in\BbbR^I$ are such that
$(z,0)\in F'_{x_1x_2}$, then $\alpha z\in F_{\tilde T(x_1,x_2)}$.
\Prf\ For every $\epsilon>0$ there is an $i\in I_{\epsilon}$ such that
$|z(r)-x_1(i,r)|\le\epsilon$ and $|x_2(i,r)|\le\epsilon$ for every
$r<n$.   But now $|\alpha z(r)-\tilde T(x_1,x_2)(r)|\le 2\epsilon$ for
every $r<n$.\ \QeD\   So

\Centerline{$\tilde T^{-1}[\{x:\alpha z\in F_x\}]
=\{(x_1,x_2):\alpha z\in F_{\tilde T(x_1,x_2)}\}
\supseteq\{(x_1,x_2):(z,0)\in F'_{x_1x_2}\}$}

\noindent is $\lambda$-conegligible, and $\alpha z\in F_x$ for
$\mu$-almost every $x$, that is, $\alpha z\in F$, as claimed.

\medskip

{\bf (c)} Suppose now that $F$ is bounded.

\medskip

\quad{\bf (i)} For $L\subseteq I$, $\alpha\ge 0$ set

\Centerline{$Q(L,\alpha)
=\bigcup_{i\in L,r<n}\{x:|x(i,r)|\ge\alpha\}$.}

\noindent By 456G, applied to the image of $\mu$ under the map
$x\mapsto x\restr L\times n:\BbbR^{I\times n}\to\BbbR^{L\times n}$,

\Centerline{$\mu Q(L,\bover12\alpha)
\ge 2\mu Q(L,\alpha)(1-\mu Q(L,\alpha))^3$}

\noindent for every finite $L\subseteq I$ and every $\alpha\ge 0$.

Let $\beta>0$ be such that $\delta=2\beta(1-(n+1)\beta)^3-\beta>0$, and let $\alpha_0>0$ be such that $\alpha_0^2\beta\ge 1$ and $\|z\|<\bover12\alpha_0$ for every $z\in F$, so that
$\mu\{x:|x(i,r)|\ge\alpha_0\}\le\beta$ whenever $i\in I_1$ and $r<n$,
and $\mu Q(\{i\},\alpha_0)\le n\beta$ for every $i\in I_1$.   Set
$K=\{z:z\in\BbbR^n$,
$\bover12\alpha_0\le\max_{r<n}|z(r)|\le\alpha_0\}$, so that $K$ is a
compact set disjoint from $F$.   For almost every $x$,

\Centerline{$\emptyset=K\cap F=K\cap F_x
=\bigcap_{k\ge 1}K\cap\overline{\{S_ix:i\in I_{1/k}\}}$,}

\noindent so there is a $k\ge 1$ such that
$S_ix\notin K$ for every $i\in I_{1/k}$.   Since the sets

\Centerline{$\{x:S_i(x)\in K$ for some $i\in I_{1/k}\}$}

\noindent form a non-increasing sequence of measurable sets with
negligible intersection, there is a $k\ge 1$ such that

\Centerline{$\mu\{x:S_i(x)\in K$ for some $i\in I_{1/k}\}
<\delta$.}

\medskip

\quad{\bf (ii)} \Quer\ Suppose, if possible, that

\Centerline{$\mu Q(I_{1/k},\alpha_0)>\beta$.}

\noindent Let $L\subseteq I_{1/k}$ be a finite set of minimal size such
that $\gamma=\mu Q(L,\alpha_0)\ge\beta$.   Since
$\mu Q(\{i\},\alpha_0)\le n\beta$ for any $i\in L$,
and $L$ is minimal, we must have

\Centerline{$\beta\le\gamma\le(n+1)\beta$.}

\noindent Now this means that

$$\eqalignno{\mu Q(L,\Bover12\alpha_0)
&\ge 2\gamma(1-\gamma)^3\cr
\displaycause{see (i) above}
&\ge 2\gamma(1-(n+1)\beta)^3
=\gamma(1+\Bover{\delta}{\beta})
\ge\gamma+\delta,\cr}$$

\noindent so that
$\mu(Q(L,\bover12\alpha_0)\setminus Q(L,\alpha_0))\ge\delta$.
But if $x\in Q(L,\bover12\alpha_0)\setminus Q(L,\alpha_0)$ there is some
$i\in L$ such that $\max_{r<n}|x(i,r)|\ge\bover12\alpha_0$ while
$\max_{r<n}|x(i,r)|<\alpha_0$, in which case $S_i(x)\in K$.   So we get

$$\eqalign{\delta
&\le\mu(Q(L,\Bover12\alpha_0)\setminus Q(L,\alpha_0))
\le\mu\{x:S_i(x)\in K\text{ for some }i\in L\}\cr
&\le\mu\{x:S_i(x)\in K\text{ for some }i\in I_{1/k}\}
<\delta\cr}$$

\noindent which is absurd.\ \Bang

\medskip

\quad{\bf (iii)} Thus $\mu Q(I_{1/k},\alpha_0)\le\beta$.   For
$\alpha\ge 0$, set $f(\alpha)=\mu Q(I_{1/k},\alpha)$;  then $f$ is
non-increasing.   Also $f(\bover12\alpha)\ge 2f(\alpha)(1-f(\alpha))^3$
for every $\alpha$.   \Prf\Quer\ Otherwise, because

\Centerline{$f(\alpha)
=\sup\{\mu Q(L,\alpha):L\subseteq I_{1/k}$ is finite$\}$,}

\noindent there is a finite $L\subseteq I_{1/k}$ such that
$f(\bover12\alpha)<2\gamma(1-\gamma)^3$, where
$\gamma=\mu Q(L,\alpha)$.   But in this case
$\mu Q(L,\bover12\alpha)\le f(\bover12\alpha)<2\gamma(1-\gamma)^3$,
which is impossible, as remarked in (i).\ \Bang\Qed

Set $\zeta=\lim_{\alpha\to\infty}f(\alpha)$.   Then

\Centerline{$\zeta=\lim_{\alpha\to\infty}f(\bover12\alpha)
\ge 2\zeta(1-\zeta)^3$.}

\noindent But we also know, from (ii), that
$\zeta\le f(\alpha_0)\le\beta$.   So
$(1-\zeta)^3\ge(1-\beta)^3>\bover12$ and $\zeta$ must be $0$.

What this means is that if we set $\epsilon=\bover1k$ then

\Centerline{$\lim_{\alpha\to\infty}
  \mu\{x:\sup_{i\in I_{\epsilon},r<n}|x(i,r)|>\alpha\}
=0$,}

\noindent that is, $\sup_{i\in I_{\epsilon},r<n}|x(i,r)|$ is finite for
almost every $x\in\BbbR^{I\times n}$, as claimed.
}%end of proof of 456M

\leader{456N}{Lemma} Let $J$ be a set and $\mu$ a centered Gaussian
distribution on $\BbbR^J$.   Let $M$ be the linear subspace of
$L^2(\mu)$ generated by $\{\pi_j^{\ssbullet}:j\in J\}$, where
$\pi_j(x)=x(j)$ for $x\in\BbbR^J$ and $j\in J$.   If $M$ is
separable\cmmnt{ (for the norm topology)} then $\mu$ is
$\tau$-additive.

\proof{ Suppose, if possible, otherwise.

\medskip

{\bf (a)} There is an upwards-directed family $\Cal G$ of open Baire
sets in $\BbbR^J$ such that $W_0=\bigcup\Cal G$ is a
Baire set and
$\mu W_0>\sup_{G\in\Cal G}\mu G$.   Let $\Cal G_0\subseteq\Cal G$ be a
countable upwards-directed set such that
$\sup_{G\in\Cal G_0}\mu G=\sup_{G\in\Cal G}\mu G$, and set
$W_1=W_0\setminus\bigcup\Cal G_0$;   then $\mu W_1>0$ and
$\mu(W_1\cap G)=0$ for every $G\in\Cal G$.   Let $W$ be a non-negligible
zero set included in $W_1$.

For each $n\in\Bbb N$, let $\Cal V_n$ be a countable base for the
topology of $\BbbR^n$ consisting of open balls.   Let $\Cal G_n^*$ be
the family of open sets of $\BbbR^J$ of the form $T^{-1}[V]$, where
$T:\BbbR^J\to\BbbR^n$ is a continuous linear operator,
$V\subseteq\BbbR^n$ is open and
$\mu(W\cap T^{-1}[V])=0$.   Of course $\Cal G_0^*=\emptyset$.

\medskip

{\bf (b)} For $n\ge 1$ and $V\in\Cal V_n$, let $\Cal T_{nV}$ be the
family of continuous linear operators $T:\BbbR^J\to\BbbR^n$ such that
$W\cap T^{-1}[V]$ is negligible, but not included in
$\bigcup\Cal G_{n-1}^*$.   Index $\Cal T_{nV}$ as
$\family{i}{I(n,V)}{T_i}$;  it will be convenient to do this in such a
way that all the $I(n,V)$ are disjoint.   Define $f_{ir}$, for
$i\in I(n,V)$ and $r<n$, by saying that
$T_i(x)=\ofamily{r}{n}{f_{ir}(x)}$ for $x\in\BbbR^J$.
Define $\phi_n:\bigcup_{V\in\Cal V_n}I(n,V)\to M^n$ by setting
$\phi_n(i)=\ofamily{r}{n}{f_{ir}^{\ssbullet}}$ for each
$i\in\bigcup_{V\in\Cal V_n}I(n,V)$.   Because $M$ is separable (in its
norm topology), $M^n$ is separable in its product topology (4A2P(a-v)).
Fix a countable set $I'(n,V)\subseteq I(n,V)$ such that
$\{\phi_n(i):i\in I'(n,V)\}$ is dense in $\{\phi_n(i):i\in I(n,V)\}$.
Set $\rho_n(\ofamily{r}{n}{u_r},\ofamily{r}{n}{v_r})
=\max_{r<n}\|u_r-v_r\|_2$ for  $\ofamily{r}{n}{u_r}$,
$\ofamily{r}{n}{v_r}\in M^n$, so that $\rho_n$ is a metric defining the
product topology of $M^n$.

\medskip

{\bf (c)} If $j\in I(n,V)$, then there is a $\delta>0$ such that

\Centerline{$\{T_i(x):i\in I'(n,V)$,
$\rho_n(\phi_n(i),\phi_n(j))\le\delta\}$}

\noindent is bounded for almost every $x\in\BbbR^J$.   \Prf\
Define $S:\BbbR^J\to\BbbR^{I'(n,V)\times n}$ by setting
$(Sx)(i,r)=T_i(x,r)-T_j(x,r)$ for $x\in\BbbR^J$, $i\in I'(n,V)$ and $r<n$.
By 456Ba, the image measure $\lambda=\nu S^{-1}$ is a centered Gaussian
distribution on $\BbbR^{I'(n,V)\times n}$.   For $\delta>0$, set

$$\eqalignno{I'_{\delta}
&=\{i:i\in I'(n,V),\,\int|y(i,r)|^2\lambda(dy)\le\delta^2
  \text{ for every }r<n\}\cr
&=\{i:i\in I'(n,V),\,\int|Sx(i,r)|^2\nu(dx)\le\delta^2
  \text{ for every }r<n\}\cr
&=\{i:i\in I'(n,V),\,\int|T_i(x)(r)-T_j(x)(r)|^2\nu(dx)\le\delta^2
  \text{ for every }r<n\}\cr
&=\{i:i\in I'(n,V),\,\rho_n(\phi_n(i),\phi_n(j))\le\delta\}.\cr}$$

\noindent By 456Ma, there is a closed set $F\subseteq\BbbR^n$ such that
$F=\bigcap_{\delta>0}\overline{\{y(i,r):i\in I'_{\delta}\}}$ for
$\lambda$-almost every $y\in\BbbR^{I'(n,V)\times n}$, so that
$F=\bigcap_{\delta>0}\overline{\{Sx(i,r):i\in I'_{\delta}\}}$ for
$\nu$-almost every $x\in\BbbR^J$.

By 456Mb,
$\alpha z\in F$ whenever $z\in F$ and $|\alpha|\le 1$.   \Quer\ If $F$
is not bounded, then it must include a line $L$ through $0$.   (The sets
$\{\Bover1nz:z\in F$, $\|z\|=n\}$, for $n\ge 1$, form a non-increasing
sequence of non-empty compact sets, so there is a point $z_0$ belonging
to them all;  take $L$ to be the set of multiples of $z_0$.)
Let $D\subseteq L$ be a countable dense set.   For $z\in D$ and
$k\in\Bbb N$ we know that

\inset{for every $i\in I(n,V)$, $T_i(x)\notin V$ for almost every
$x\in W$,}

\inset{for almost every $x\in\BbbR^J$ there is an $i\in I'(n,V)$ such
that $\|T_i(x)-T_j(x)-z\|\le 2^{-k}$}

\noindent and therefore

\inset{for almost every $x\in W$, $T_i(x)\notin V$ for every
$i\in I'(n,V)$, but there is an $i\in I'(n,V)$ such that
$\|T_i(x)-T_j(x)-z\|\le 2^{-k}$}

\noindent so that

\inset{for almost every $x\in W$, the distance from $T_j(x)+z$ to the
closed set $\BbbR^n\setminus V$ is at most $2^{-k}$.}

\noindent This is true for every $k\in\Bbb N$, so we get

\inset{$T_j(x)+z\notin V$ for almost every $x\in W$.}

\noindent And {\it this} is true for every $z\in D$, so we get

\inset{for almost every $x\in W$, $T_j(x)+z\notin V$ for every $z\in D$,
so $T_j(x)\notin V+L$.}

Let $S_0:\BbbR^n\to\BbbR^{n-1}$ be a linear operator with kernel $L$, and
set $V'=S_0[V]$.   Then $V'\subseteq\BbbR^{n-1}$ is open, and
$W\cap(S_0T_j)^{-1}[V']=W\cap T_j^{-1}[V+L]$ is negligible.   But this
means that $T_j^{-1}[V]\subseteq(S_0T_j)^{-1}[V']\in\Cal G_{n-1}^*$ and
$T_j^{-1}[V]$ is included in $\bigcup\Cal G_{n-1}^*$;  which
contradicts the definition of $\Cal T_{nV}$.\ \Bang

So $F$ is bounded.   By 456Mc, there is some $\delta>0$ such that
$\sup_{i\in I'_{\delta},r<n}|y(i,r)|<\infty$ for $\lambda$-almost every
$y\in\BbbR^{I'(n,V)\times n}$, in which case
$\sup_{i\in I'_{\delta},r<n}|Sx(i,r)|<\infty$ for $\nu$-almost every
$x\in\BbbR^J$, that is,
$\{T_i(x)-T_j(x):i\in I'_{\delta}\}$ is bounded for $\nu$-almost every
$x$.   Of course this means that
$\{T_i(x):i\in I(n,V),\,\rho_n(\phi_n(i),\phi_n(j))\le\delta\}$ is
bounded for almost every $x\in\BbbR^J$.\ \Qed

\medskip

{\bf (d)} Accordingly $\phi_n[I(n,V)]$ is covered by the family
$\Cal U_{nV}$ of open sets $U\subseteq M^n$ such that
$\{T_i(x):i\in I'(n,V)$, $\phi_n(i)\in U\}$ is bounded for almost every
$x$.   Because $M^n$ is separable and metrizable, it is
hereditarily Lindel\"of (4A2P(a-iii)), so there is a sequence
$\sequence{k}{U_{nVk}}$ in $\Cal U_{nV}$ covering $\phi_n[I(n,V)]$.
For each $k$, set
$I_{nVk}=I'(n,V)\cap\phi_n^{-1}[U_{nVk}]$.   Then
$\{T_i(x):i\in I_{nVk}\}$
is bounded for almost every $x$.  Because $\phi_n[I'(n,V)]$ is dense in
$\phi_n[I(n,V)]$, $\phi_n[I_{nVk}]=\phi_n[I'(n,V)]\cap U_{nVk}$ is dense
in $\phi_n[I(n,V)]\cap U_{nVk}$.
So for every $i\in I(n,V)$ there is a
$k\in\Bbb N$ such that $\phi_n(i)\in\overline{\phi_n[I_{nVk}]}$.

\woddheader{456N}{4}{2}{2}{30pt}

{\bf (e)} Recall that $W\subseteq\BbbR^J$ is a non-negligible zero set
included in

\Centerline{$\bigcup_{n\ge 1}\bigcup\Cal G_n^*
=\bigcup_{n\ge 1}\bigcup_{V\in\Cal V_n}
  \bigcup_{i\in I(n,V)}T_i^{-1}[V]$.}

\noindent Let $J_0\subseteq J$ be a countable set such that $W$ is
determined by coordinates in $J_0$.

\ifdim\pagewidth>467pt\fontdimen3\tenrm=2pt\fi
\ifdim\pagewidth>467pt\fontdimen4\tenrm=1.67pt\fi
Let $\family{j}{J_0}{\epsilon_j}$ and 
$\langle\epsilon'_{nVk}\rangle_{n\ge 1,V\in\Cal V_n,k\in\Bbb N}$ 
be families of strictly positive real numbers such that
$\sum_{n=1}^{\infty}\sum_{V\in\Cal V_n}
\sum_{k=0}^{\infty}\epsilon'_{nVk}$
and $\sum_{j\in J_0}\epsilon_j$
are both at most $\bover13\mu W$.   Let $\family{j}{J_0}{\gamma_j}$
and $\langle\gamma'_{nVk}\rangle_{n\ge 1,V\in\Cal V_n,k\in\Bbb N}$ be
such that
\fontdimen3\tenrm=1.67pt
\fontdimen4\tenrm=1.11pt

\Centerline{$\mu\{x:x\in\BbbR^J$, $|x(j)|\ge\gamma_j\}\le\epsilon_j$ for
every $j\in J_0$,}

\Centerline{$\mu\{x:x\in\BbbR^J$,
$\sup_{i\in I_{nVk}}\|T_i(x)\|\ge\gamma'_{nVk}\}
\le\epsilon'_{nVk}$ for every $n\ge 1$, $V\in\Cal V_n$, $k\in\Bbb N$.}

\noindent Set
$W'=\{x:x\in W$, $|x(j)|\le\gamma_j$ for every $j\in J_0\}$;  then
$\mu W'\ge\bover23\mu W$ and $W'$ is of the form
$C\times\BbbR^{J\setminus J_0}$, where $C\subseteq\BbbR^{J_0}$ is
compact.   Set

\Centerline{$W''=\{x:x\in W'$, $\|T_i(x)\|\le\gamma'_{nVk}$ whenever
$n\ge 1$, $V\in\Cal V_n$, $k\in\Bbb N$ and $i\in I_{nVk}\}$;}

\noindent then $\mu W''\ge\bover13\mu W$.

\medskip

{\bf (f)} Set $I=\bigcup_{n\ge 1,V\in\Cal V_n}I(n,V)\times n$.   If
$K\subseteq I$ is finite, then

$$\eqalign{W'_K
&=\{x:x\in W',\,\|f_{ir}(x)\|\le\gamma'_{nVk}\text{ whenever }n\ge 1,
  \,V\in\Cal V_n,\,k\in\Bbb N,\cr
&\mskip250mu(i,r)\in K
  \text{ and }\phi_n(i)\in\overline{\phi_n[I_{nVk}]}\}\cr}$$

\noindent has measure at least $\bover13\mu W$.   \Prf\ For each
quintuple $(i,r,n,V,k)$ with $n\ge 1$, $V\in\Cal V_n$, $k\in\Bbb N$,
$(i,r)\in K$ and $\phi_n(i)\in\overline{\phi_n[I_{nVk}]}$, there is a
sequence $\sequence{m}{i_m}$ in $I_{nVk}$ such that
$\rho_n(\phi_n(i),\phi_n(i_m))\le 2^{-m}$ for every $m$;  so that
$\|f_{ir}-f_{i_mr}\|_2\le 2^{-m}$ for every $m$.   But this means that
$f_{ir}(x)=\lim_{m\to\infty}f_{i_mr}(x)$ for almost every $x\in\BbbR^J$.
Accordingly $|f_{ir}(x)|\le\gamma'_{nVk}$ for almost every $x\in W''$.
Since there are only countably many such quintuples $(i,r,n,V,k)$, we
see that $W''\setminus W'_K$ is negligible, so
$\mu W'_K\ge\mu W''\ge\bover13\mu W$.\ \Qed

\medskip

{\bf (g)} For $x\in\BbbR^J$, define $Tx\in\BbbR^I$ by setting
$(Tx)(i,r)=f_{ir}(x)$ for $i\in I(n,V)$ and $r<n$.
Then $T:\BbbR^J\to\BbbR^I$ is a continuous linear operator.
By 4A4H, $T[W']$ is closed.

For finite $K\subseteq I$, let $\Cal H_K$ be the family of open subsets
$H$ of $\BbbR^K$ such that $\mu\{x:x\in W$, $Tx\restr K\in H\}=0$.
Then $\Cal H_K$ is closed under countable unions so has a largest member
$H_K$.   Now there is a $K\in[I]^{<\omega}$ such that
$Tx\restr K\in H_K$ for every $x\in W'_K$.   \Prf\Quer\ Otherwise,
choose for each $K\in[I]^{<\omega}$ an $x_K\in W'_K$ such that
$Tx_K\restr K\notin H_K$.   Let $\Cal F$ be an ultrafilter on
$[I]^{<\omega}$ containing $\{K:L\subseteq K\in[I]^{<\omega}\}$ for
every finite $L\subseteq I$.   If $(i,r)\in I$, there are
$n\ge 1$, $V\in\Cal V_n$ and $k\in\Bbb N$ such that $r<n$ and
$\phi_n(i)\in\overline{\phi_n[I_{nVk}]}$, in which case
$|f_{ir}(x_K)|\le\gamma'_{nVk}$ whenever $K\in[I]^{<\omega}$ contains
$(i,r)$.
This means that $\lim_{K\to\Cal F}f_{ir}(x_K)$ must be defined in
$[-\gamma'_{nVk},\gamma'_{nVk}]$;  consequently
$y^*=\lim_{K\to\Cal F}Tx_K$ is defined in $\BbbR^I$.   Since $x_K\in W'$
for every $K$, $y^*\in\overline{T[W']}=T[W']$.

Let $x^*\in W'$ be such that $Tx^*=y^*$.   Since $x^*\in W$, there are
$n\ge 1$, $V\in\Cal V_n$ and $i\in I(n,V)$ such that $T_i(x^*)\in V$.
Set $L=\{(i,r):r<n\}$,
$H=\{z:z\in\BbbR^L$, $\ofamily{r}{n}{z(i,r)}\in V\}$;  then
$\{x:Tx\restr L\in H\}=T_i^{-1}[V]$.   Since
$y^*\restr L=Tx^*\restr L$ belongs to $H$, and $H$ is open, there must
be a $K\supseteq L$ such that $Tx_K\restr L\in H$.   But in this case
$H'=\{z:z\in\BbbR^K$, $z\restr L\in H\}$ is an open subset of $\BbbR^K$
and

\Centerline{$\{x:Tx\restr K\in H'\}=\{x:Tx\restr L\in H\}
=\{x:T_i(x)\in V\}$}

\noindent meets $W$ in a negligible set, and $H'\subseteq H_K$.   But
this means that $Tx_K\restr K\in H_K$, contrary to the choice of $x_K$.\
\Bang\Qed

\medskip

{\bf (h)} Putting (f) and (g) together, we find ourselves trying to
believe simultaneously that $\mu W'_K>0$ and that $Tx\restr K\in H_K$
for every $x\in W'_K$ and that $W'_K\subseteq W$ and that
$\{x:x\in W$, $Tx\restr K\in H_K\}$ is negligible.   Faced with this we
have to abandon the original supposition that $\mu$ is not
$\tau$-additive.
}%end of proof of 456N

\leader{456O}{}\cmmnt{ We now have all the ideas needed for the main
theorem of this section.

\medskip

\noindent}{\bf Theorem}\cmmnt{ ({\smc Talagrand 81})} Every centered
Gaussian distribution is $\tau$-additive.

\proof{\Quer\ Suppose, if possible, that $\mu$ is a centered Gaussian
distribution on a set $\BbbR^I$ which is not $\tau$-additive.

\medskip

{\bf (a)} By 456K, there are a set $J$ and a universal centered Gaussian
distribution $\nu$ on $\BbbR^J$ and a continuous linear operator
$T:\BbbR^J\to\BbbR^I$ which is \imp\ for $\nu$ and $\mu$.   By
418Ha, $\nu$ is not $\tau$-additive.

\medskip

{\bf (b)} As in part (a) of the proof of 456N, there are a
non-negligible zero set $W\subseteq\BbbR^J$ and a family $\Cal G$ of
open sets, covering $W$, such that $\nu(W\cap G)=0$ for every
$G\in\Cal G$.   Give $J$ a Hilbert space structure such that
$\int x(i)x(j)\nu(dx)=\innerprod{i}{j}$ for all $i$, $j\in J$.   Let
$K_0\subseteq J$ be a countable set such that $W$ is determined by
coordinates in $K_0$, and let $K$ be the closed linear subspace of $J$
generated by $K_0$.
Let $\Cal G'$ be the family of open sets determined by coordinates in
$K$ which meet $W$ in negligible sets.   Then
$W\subseteq\bigcup\Cal G'$, by 456L.

Let $\lambda$ be the centered Gaussian distribution on $\BbbR^K$ for
which the map $\tilde\pi_K=x\mapsto x\restr K:\BbbR^J\to\BbbR^K$ is
\imp.   Then $\tilde\pi_K[W]$ is a zero set in $\BbbR^K$,
$\lambda\tilde\pi_K[W]=\nu W>0$, $\{\tilde\pi_K[G]:G\in\Cal G'\}$ is a
family of open sets in $\BbbR^K$ covering $\tilde\pi_K[W]$, and
$\lambda(\tilde\pi_K[W]\cap\tilde\pi_K[G])=\nu(W\cap G)=0$ for every
$G\in\Cal G'$;  so $\lambda$ is not $\tau$-additive.   However, $K$,
regarded as a normed space, is separable (see 4A4Bg);  and if we set
$\pi_j(y)=y(j)$ for $y\in\BbbR^K$ and $j\in K$, then
$\|\pi_i^{\ssbullet}-\pi_j^{\ssbullet}\|_2=\|i-j\|$ for all $i$,
$j\in K$.   So $\{\pi_j^{\ssbullet}:j\in K\}$ is separable in
$L^2(\lambda)$.   And this is impossible, by 456N.\ \Bang

Thus every centered Gaussian distribution must be $\tau$-additive.
}%end of proof of 456O

\leader{456P}{Corollary} If $\mu$ is a centered Gaussian distribution on
$\BbbR^I$, there is a unique quasi-Radon measure $\tilde\mu$ on
$\BbbR^I$ extending $\mu$.   The support of $\mu$ as defined in 456H is
the support of $\tilde\mu$ as defined in 411N.

\proof{ By 415L, $\mu$ has a unique extension to a quasi-Radon measure
$\tilde\mu$.   Now the support $Z$ of $\mu$ is a closed set, so
$\tilde\mu Z=\mu^*Z$ (415L(i)).   Also $Z$ is self-supporting for $\mu$.
If $G\subseteq\BbbR^I$ is an open set meeting $Z$, then there is a
cozero set $H\subseteq G$ which also meets $Z$, and $\mu^*(Z\cap H)>0$.
It follows that $\mu^*(Z\setminus H)<1$;  as $\tilde\mu$ extends $\mu$,
$\tilde\mu(Z\setminus H)<1$ and $\tilde\mu(Z\cap G)>0$.   This shows
that $Z$ is self-supporting for $\tilde\mu$, so must be the support of
$\tilde\mu$ in the standard sense.
}%end of 456P

\leader{456Q}{Proposition}\dvAnew{2009}
Let $I$ be a set and $R$ the set of functions
$\sigma:I\times I\to\Bbb R$ which are symmetric and positive semi-definite
in the sense of 456C;  give $R$ the subspace topology induced by the usual
topology of $\BbbR^{I\times I}$.   Let $P_{\text{qR}}(\BbbR^I)$
be the space of
quasi-Radon probability measures on $\BbbR^I$ with its narrow 
topology\cmmnt{ (437Jd)}.   For $\sigma\in R$, let $\mu_{\sigma}$ be the
centered Gaussian distribution on $\BbbR^I$ with covariance matrix
$\sigma$\cmmnt{ (456C)}, and $\tilde\mu_{\sigma}$ the quasi-Radon measure 
extending $\mu_{\sigma}$\cmmnt{ (456P)}.   Then $R$ is a closed subset of
$\BbbR^{I\times I}$ and the function
$\sigma\mapsto\tilde\mu_{\sigma}:R\to P_{\text{qR}}(\BbbR^I)$ is
continuous.

\proof{{\bf (a)} From 456C(iv) we see at once that $R$ is closed.   So the
rest of this proof will be devoted to showing that
$\sigma\mapsto\tilde\mu_{\sigma}$ is continuous.

\medskip

{\bf (b)} I had better begin with the one-dimensional case.   If
$I=\{j\}$ is a singleton, and we identify $\BbbR^I$ with $\Bbb R$, then
$\tilde\mu_{\sigma}$ is the ordinary normal distribution with mean $0$ and
variance $\sigma(j,j)$, counting the Dirac measure centered at $0$ as a
normal distribution with zero variance.   If $H\subseteq\Bbb R$ is open
and $\gamma\in\Bbb R$, set

\Centerline{$G=\{\alpha:\alpha>0$,
$\Bover1{\sqrt{2\pi\alpha}}\int_He^{-t^2/\alpha}dt>\gamma\}$;}

\noindent then $G$ is open.   If $0\notin H$, then

\Centerline{$\{\sigma:\tilde\mu_{\sigma}H>\gamma\}
=\{\sigma:\sigma(i,i)\in G\}$}

\noindent is open.   If $0\in H$ and $\gamma\ge 1$, then
$\{\sigma:\tilde\mu_{\sigma}H>\gamma\}$ is empty;  if
$0\in H$ and $\gamma<1$, then

\Centerline{$\{\sigma:\tilde\mu_{\sigma}H>\gamma\}
=\{\sigma:\sigma(i,i)\in G\}\cup\{0\}$}

\noindent is open because there is an $\eta>0$ such that
$[-\eta,\eta]\subseteq H$ and $\alpha\in G$ whenever $\alpha>0$ and

\Centerline{$\Bover1{\sqrt{2\pi}}
  \int_{-\eta/\sqrt{\alpha}}^{\eta/\sqrt{\alpha}}
  e^{-t^2/2}dt>\gamma$.}

\noindent As $H$ is arbitrary, $\sigma\mapsto\tilde\mu_{\sigma}$ is
continuous.

\medskip

{\bf (c)} Now suppose that $I$ is finite.   Let $\sequencen{\sigma_n}$ be a
sequence in $R$ with limit $\sigma\in R$.   Let $\phi_n$, $\phi$ be the
characteristic functions of $\tilde\mu_{\sigma_n}$,
$\tilde\mu_{\sigma}$ respectively
(\S285).   If $y\in\BbbR^I$, set $f(x)=\varinnerprod{x}{y}$ for
$x\in\BbbR^I$;  then

\Centerline{$\phi(y)
=\int e^{i f(x)}\tilde\mu_{\sigma}(dx)
=\int e^{it}(\tilde\mu_{\sigma}f^{-1})(dt)$,}

\noindent writing $\tilde\mu_{\sigma}f^{-1}$ for the image Radon measure on
$\Bbb R$.   Now $\tilde\mu_{\sigma}f^{-1}$ is the
one-dimensional Gaussian distribution
with variance $\sum_{j,k\in I}\sigma(j,k)y(j)y(k)$ (see part (b) of the
proof of 456B).   But since

\Centerline{$\sum_{j,k\in I}\sigma(j,k)y(j)y(k)
=\lim_{n\to\infty}\sum_{j,k\in I}\sigma_n(j,k)y(j)y(k$,}

\noindent (a) tells us that $\tilde\mu_{\sigma}f^{-1}
=\lim_{n\to\infty}\tilde\mu_{\sigma_n}f^{-1}$ for the narrow topology on
$P_{\text{qR}}(\Bbb R)$, therefore also for the vague topology (437L),
and $\phi(y)=\lim_{n\to\infty}\phi_n(y)$.   By 285L,
$\tilde\mu_{\sigma}=\lim_{n\to\infty}\tilde\mu_{\sigma_n}$ for the vague
topology, therefore also for the narrow topology.

Thus $\sigma\mapsto\tilde\mu_{\sigma}$ is sequentially continuous.   As
$I$ is countable, $R$ is metrizable, and
$\sigma\mapsto\tilde\mu_{\sigma}$ is continuous.

\medskip

{\bf (d)} For the general case, suppose that $H\subseteq\BbbR^I$ is an open
set and that $\gamma\in\Bbb R$.   Set
$G_{H\gamma}=\{\sigma:\sigma\in R$, $\tilde\mu_{\sigma}H>\gamma\}$.

\medskip

\quad{\bf (i)} If $H$ is determined by coordinates in a finite set
$J\subseteq I$ then $G_{H\gamma}$ is open in $R$.   \Prf\
Let $R_J$ be the set of symmetric positive semi-definite functions on
$\BbbR^{J\times J}$;  write $h(\sigma)=\sigma\restr J\times J$ for
$\sigma\in R$, and $\tilde\pi_J(x)=x\restr J$ for $x\in\BbbR^I$.
Of course $h(\sigma)\in R_J$ for $\sigma\in R$, and
$h:R\to R_J$ is continuous.   For $\sigma\in R$,
we know that there is a centered Gaussian distribution $\nu$ on $\BbbR^J$
such that $\tilde\pi_J$ is \imp\ for $\mu_{\sigma}$ and $\nu$, by
456Ba;  the covariance matrix of $\nu$
is of course $h(\sigma)$, so we can call it $\mu_{h(\sigma)}$.
Next, there is a quasi-Radon measure $\tilde\nu$
on $\BbbR^J$ such that $\tilde\pi_J$ is \imp\ for $\tilde\mu_{\sigma}$ and
$\tilde\nu$ (418Hb);  as $\tilde\nu$ must extend the Baire measure
$\nu$, it is the unique quasi-Radon measure extending
$\nu$, and we can call it $\tilde\mu_{h(\sigma)}$.

Because $H$ is determined by coordinates in $J$, $H=\tilde\pi_J^{-1}[H']$
where $H'=\tilde\pi_J[H]$ is open in $\BbbR^J$ (4A2B(f-i) again).   So
$G'=\{\tau:\tau\in R_J$, $\tilde\mu_{\tau}H'>\gamma\}$ is open in $R_J$, by
(b), and

\Centerline{$G_{H\gamma}
=\{\sigma:(\tilde\mu_{\sigma}\tilde\pi_J^{-1})(H')>\gamma\}
=\{\sigma:\tilde\mu_{h(\sigma)}H'>\gamma\}
=h^{-1}[G']$}

\noindent is open in $R$.\ \Qed

\medskip

\quad{\bf (ii)} In fact $G_{H\gamma}$ is open in $R$ for any open set
$H\subseteq\BbbR^I$ and $\gamma\in\Bbb R$.   \Prf\ Take any
$\sigma\in G_{H\gamma}$.   Because $\tilde\mu_{\sigma}$ is
$\tau$-additive, and the family

\Centerline{$\Cal V
=\{V:V\subseteq\BbbR^I$ is open and determined by coordinates in a finite
set$\}$}

\noindent is a base for the topology of $\BbbR^I$ closed under finite
unions, there is a $V\in\Cal V$ such that $V\subseteq H$ and
$\tilde\mu_{\sigma}V>\gamma$.   Now
$\sigma\in G_{V\gamma}\subseteq G_{H\gamma}$;  by (i),
$G_{V\gamma}$ is open, so $\sigma\in\interior G_{H\gamma}$;  as $\sigma$ is
arbitrary, $G_{H\gamma}$ is open.\ \QeD\  But this is just what we need to
know to see that $\sigma\mapsto\tilde\mu_{\sigma}$ is continuous for the
narrow topology on $P_{\text{qR}}(\BbbR^I)$, and the proof is complete.
}%end of proof of 456Q

\exercises{\leader{456X}{Basic exercises (a)}%
%\spheader 456Xa
\dvAnew{2009} Let $I$ be any set.
(i) Show that if $y\in\ell^1(I)$ then
$\int\sum_{i\in I}|y(i)x(i)|\mu^{(I)}_G(dx)=\Bover2{\sqrt{2\pi}}\|y\|_1$.
\Hint{start by evaluating $\Expn(|Z|)$ where $Z$ is a standard normal
random variable.}
(ii) Show that if $y\in\ell^2(I)$ then
$\int\sum_{i\in I}|y(i)x(i)|^2\mu^{(I)}_G(dx)=\|y\|_2^2$.
%456B

\spheader 456Xb
Let $n\ge 1$ be an integer.   (i) Show that
$\mu_G^{(n)}T^{-1}=\mu_G^{(n)}$ for any orthogonal linear operator
$T:\BbbR^n\to\BbbR^n$.   (ii) Set $p(x)=\Bover1{\|x\|}x$ for
$x\in\BbbR^n\setminus\{0\}$;  take $p(0)$ to be any point of $S^{n-1}$.
Show that $\mu_G^{(n)}p^{-1}$ is a multiple of $(n-1)$-dimensional
Hausdorff measure on $S^{n-1}$.   \Hint{443U.}
%456B

\spheader 456Xc\dvAnew{2009}
Let $G$ be a group, and $h:G\to\Bbb R$ a real positive definite
function (definition: 445L).   (i) Show that we have a centered Gaussian
distribution $\mu$ on $\BbbR^G$ with covariance matrix
$\langle h(a^{-1}b)\rangle_{a,b\in G}$.
(ii) Show that $\mu$ is invariant under
the left shift action $\action_l$ of $G$ on $\BbbR^G$ (4A5Cc).
%456C

\spheader 456Xd Let $I$ be a countable set, $\mu$ a centered Gaussian
distribution on $\BbbR^I$, and $\gamma\ge 0$.   Set
$\alpha=\mu\{x:\sup_{i\in I}|x(i)|\ge\gamma\}$.   Show that
$\mu\{x:\sup_{i\in I}|x(i)|\ge\bover12\gamma\}\ge 2\alpha(1-\alpha)^3$.
%456G

\spheader 456Xe Let $I$ be a set and
$\langle\sigma_{ij}\rangle_{i,j\in I}$ a family of real numbers.   Show
that there is at most one inner product space structure on $I$ for which
$\sigma_{ij}=\innerprod{i}{j}$ for all $i$, $j\in I$.
%456J

\spheader 456Xf Let $\sequencen{X_n}$ be an independent sequence of
standard normal random variables, and $\sequencen{\alpha_n}$ a
square-summable real sequence.   (i) Show that for any
permutation $\pi:\Bbb N\to\Bbb N$, $X=\sum_{n=0}^{\infty}\alpha_nX_n$ and
$\sum_{n=0}^{\infty}\alpha_{\pi(n)}X_{\pi(n)}$ are finite and equal a.e.
\Hint{273B.}  (ii) Show that $X$ is normal, with mean $0$ and variance
$\sum_{n=0}^{\infty}\alpha_n^2$.
%456K

\sqheader 456Xg For any set $I$, I will say that a
{\bf centered Gaussian quasi-Radon measure} on $\BbbR^I$ is a quasi-Radon
measure $\mu$ on $\BbbR^I$ such that
every continuous linear functional $f:\BbbR^I\to\Bbb R$ is either zero a.e.\ or is normally distributed with zero expectation.   Show that

\quad(i) there is a one-to-one correspondence between centered Gaussian quasi-Radon measures $\mu$ on $\BbbR^I$ and centered Gaussian distributions $\nu$ on $\BbbR^I$
obtained by matching $\mu$ with $\nu$ iff they agree on the zero sets of $\BbbR^I$;

\quad(ii) if $\mu$, $\nu$ are centered Gaussian quasi-Radon measures on $\BbbR^I$ and $\int x(i)x(j)\mu(dx)=\int x(i)x(j)\nu(dx)$ for all $i$, $j\in I$, then $\mu=\nu$;

\quad(iii) the support of a centered Gaussian quasi-Radon measure on $\BbbR^I$ is a linear subspace of $\BbbR^I$;

\quad(iv) if $\family{j}{J}{I_j}$ is a disjoint family of sets with union $I$, and $\mu_j$ is a centered Gaussian quasi-Radon measure on $\BbbR^{I_j}$ for each $j\in J$,
then the quasi-Radon product of $\family{j}{J}{\mu_j}$, regarded as a measure on $\BbbR^I$, is a centered Gaussian quasi-Radon measure.
%456P

\spheader 456Xh\dvAnew{2011}
Let $I$ be a set, and let $H$ be a Hilbert space with
orthonormal basis $\familyiI{e_i}$.   For $i\in I$, $x\in\Bbb R^I$ set
$f_i(x)=x(i)$.    Show that there is a bounded
linear operator $T:H\to L^1(\mu_G^{(I)})$ such that $Te_i=f_i^{\ssbullet}$
for every $i\in I$, and that $\|Tu\|_1=\Bover{2}{\sqrt{2\pi}}\|u\|_2$ 
for every $u\in H$.
%456Xf 456K out of order query

\leader{456Y}{Further exercises (a)}
%\spheader 456Ya
Let $(\Omega,\Sigma,\mu)$ be a probability space with measure algebra
$(\frak A,\bar\mu)$, and $\familyiI{u_i}$ a family in
$L^2(\mu)\cong L^2(\frak A,\bar\mu)$ which is a centered
Gaussian process in the
sense that whenever $X_i\in\eusm L^2(\mu)$ is such that
$X_i^{\ssbullet}=u_i$ for every $i$, then $\familyiI{X_i}$ is a centered
Gaussian process.   Suppose that $\gamma\ge 0$ and that
$\alpha=\bar\mu(\sup_{i\in I}\Bvalue{|u_i|\ge\gamma})$.   Show that
$\bar\mu(\sup_{i\in I}\Bvalue{|u_i|\ge\bover12\gamma})
\ge 2\alpha(1-\alpha)^3$.
%456Xd 456G

\spheader 456Yb Let $U$ be a Hilbert space with an orthonormal basis
$\family{j}{J}{u_j}$, and $\mu$ the universal centered Gaussian
distribution on $\BbbR^U$ with covariance matrix defined by the inner
product of $U$.   Show that there is a function $T:\BbbR^J\to\BbbR^U$,
\imp\ for $\mu_G^{(J)}$ and $\mu$, such that whenever $\sequencen{j_n}$
is a sequence of distinct elements of $J$ and $\sequencen{\alpha_n}$ is
a square-summable sequence in $\Bbb R$, then
$(Tx)(\sum_{n=0}^{\infty}\alpha_nu_{j_n})
=\sum_{n=0}^{\infty}\alpha_nx(j_n)$ for almost every $x\in\BbbR^J$.
%456Xf 456Ea 456K

\spheader 456Yc Let $U$ be an infinite-dimensional Hilbert space and
$\mu$ the universal centered Gaussian distribution on $\BbbR^U$ with
covariance matrix defined by the inner product of $U$.   Show that
$\mu C=0$ for every compact set $C\subseteq\BbbR^U$.
%456K mt45bits

\spheader 456Yd Let $I$ be a set and
$\mu$ be a centered Gaussian distribution on $\BbbR^I$.   Show
that the following are equiveridical:  (i) $\mu$ has countable Maharam
type;  (ii) $L^2(\mu)$ is separable;  (iii) $I$ is separable under the
pseudometric $(i,j)\mapsto\sqrt{\int(x(i)-x(j))^2\mu(dx)}$.
%456N
}%end of exercises

\endnotes{
\Notesheader{456} This section has aimed for a direct route to Talagrand's
theorem 456O, leaving most of the real reasons for studying Gaussian
processes (see {\smc Fernique 97}) to one side.   It should
nevertheless be clear from
such fragments as 252Xi, 456Bb, 456G and the exercises here that
they are one of the many concepts of probability theory which are both
significant and delightful.   Very much the most important Gaussian
processes are those associated with Brownian motion, which will be treated
in \S477 {\it et seq.}

You will of course have observed that the methods used here are entirely
different from those in \S455, even though one of the concerns of
that section was a check for $\tau$-additive distributions and 
corresponding quasi-Radon versions, as in 455K.   However the results of
\S455 were based on the fact that in the most important cases the 
distributions there have extensions to Radon measures (455H).   Gaussian
distributions need not be like this at all, even when they have countable
Maharam type;  see 456Yc.
}%end of notes

\discrversionA{
\bigskip

{\bf question}  if $\mu$ is a centered Gaussian quasi-Radon measure on
$\BbbR^I$ and $T:\BbbR^I\to\BbbR^J$ is a continuous linear operator,
must $\mu T^{-1}$ be a quasi-Radon
measure?   (certainly it is $\tau$-additive, is it inner regular with
respect to the closed sets?)
}{}

\discrpage

