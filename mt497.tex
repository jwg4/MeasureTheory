\frfilename{mt497.tex}
\versiondate{7.12.10}
\copyrightdate{2009}

\def\TaoIG{{\smc Tao 07}}

\def\chaptername{Further topics}
\def\sectionname{Tao's proof of Szemer\'edi's theorem}

\newsection{497\dvAnew{2009}}

Szemer\'edi's celebrated theorem on arithmetic
progressions (497L) is not obviously part of measure theory.
Remarkably, however, it has stimulated significant developments in
the subject.   The first was Furstenberg's multiple recurrence
theorem\cmmnt{ ({\smc Furstenberg 77}, {\smc Furstenberg 81},
{\smc Furstenberg \& Katznelson 85})}.   In this section I will give an
account of an approach due to T.Tao\cmmnt{ (\TaoIG)} which
introduces another
phenomenon of great interest from a measure-theoretic point of view.

\leader{497A}{Definitions (a)} Let $(X,\Sigma,\mu)$ be a probability space,
$\Tau$ a subalgebra of $\Sigma$\cmmnt{ ({\it not} necessarily a
$\sigma$-subalgebra)} and
$\familyiI{\Sigma_i}$ a family of $\sigma$-subalgebras of $\Sigma$.
\cmmnt{I will say that }$\familyiI{\Sigma_i}$ has
{\bf $\Tau$-removable intersections} if whenever $J\subseteq I$ is finite
and not empty, $E_i\in\Sigma_i$ for $i\in J$,
$\mu(\bigcap_{i\in J}E_i)=0$ and $\epsilon>0$, there is a family
$\family{i}{J}{F_i}$ such that $F_i\in\Tau\cap\Sigma_i$ and
$\mu(E_i\setminus F_i)\le\epsilon$ for each $i\in J$, and
$\bigcap_{i\in J}F_i=\emptyset$.
\cmmnt{(This is a stronger version of what \TaoIG\ calls the
`uniform intersection property'.)}

\spheader 497Ab
If $X$ is a set and $\Sigma$, $\Sigma'$ are two $\sigma$-algebras
of subsets of $X$, $\Sigma\vee\Sigma'$ will be the $\sigma$-algebra
generated by $\Sigma\cup\Sigma'$.   If $\familyiI{\Sigma_i}$ is a family of
$\sigma$-algebras of subsets of $X$, I will write
$\bigvee_{i\in I}\Sigma_i$
for the $\sigma$-algebra generated by $\bigcup_{i\in I}\Sigma_i$.

\spheader 497Ac
If $(X,\Sigma,\mu)$ is a probability space and
$\Cal A\subseteq\Cal E\subseteq\Sigma$,\cmmnt{ I will say that} $\Cal A$
is {\bf metrically dense} in $\Cal E$ if for every $E\in\Cal E$ and
$\epsilon>0$
there is an $F\in\Cal A$ such that $\mu(E\symmdiff F)\le\epsilon$\cmmnt{;
that is, if $\{F^{\ssbullet}:F\in A\}$ is dense in
$\{E^{\ssbullet}:E\in\Cal E\}$ for the measure-algebra topology on the
measure algebra of $\mu$ (323A)}.   Note
that a subalgebra of $\Sigma$ is metrically dense
in the $\sigma$-algebra it generates\cmmnt{ (compare 323J)}.

\leader{497B}{Lemma}
Let $(X,\Sigma,\mu)$ be a probability space and $\Tau$ a
subalgebra of $\Sigma$.   Let $\familyiI{\Sigma_i}$ be a family of
$\sigma$-subalgebras of $\Sigma$.

(a) $\familyiI{\Sigma_i}$
has $\Tau$-removable intersections iff $\family{i}{J}{\Sigma_i}$ has
$\Tau$-removable intersections for every finite $J\subseteq I$.

(b) Suppose that $\familyiI{\Sigma_i}$ has $\Tau$-removable intersections
and that $\Tau\cap\Sigma_i$ is metrically dense in $\Sigma_i$ for every
$i$.   Let $J$ be any set and $f:J\to I$ a function.   Then
$\family{j}{J}{\Sigma_{f(j)}}$ has $\Tau$-removable intersections.

(c) Suppose that, for each $i\in I$, we are given a $\sigma$-subalgebra
$\Sigma'_i$ of $\Sigma_i$ such that for every $E\in\Sigma_i$ there is an
$E'\in\Sigma'_i$ such that $E\symmdiff E'$ is negligible.   If
$\familyiI{\Sigma'_i}$ has $\Tau$-removable intersections,
so has $\familyiI{\Sigma_i}$.

\proof{{\bf (a)} is trivial.

\medskip

{\bf (b)} Suppose that $K\subseteq J$ is finite and not empty, that
$\family{j}{K}{E_j}\in\prod_{j\in K}\Sigma_{f(j)}$ is such that
$\mu(\bigcap_{j\in K}E_j)=0$, and $\epsilon>0$.   Set $n=\#(K)$ and
$\eta=\bover{\epsilon}{n+2}>0$.
Set $E'_i=\bigcap_{j\in K,f(j)=i}E_j\in\Sigma_i$ for $i\in f[K]$;  then
$\bigcap_{i\in f[K]}E'_i=\bigcap_{j\in K}E_j$ is negligible, so we have
$F'_i\in\Tau\cap\Sigma_i$, for $i\in f[K]$, such that
$\bigcap_{i\in f[K]}F'_i=\emptyset$ and $\mu(E'_i\setminus F'_i)\le\eta$
for every $i\in f[K]$.   As $\Tau\cap\Sigma_i$ is metrically dense in
$\Sigma_i$ for each $i$, we can find $G_j\in\Tau\cap\Sigma_{f(j)}$
such that $\mu(E_j\symmdiff G_j)\le\eta$ for each $j\in K$.   Set
$G'_i=\bigcap_{j\in K,f(j)=i}G_j$ for $i\in f[K]$.   Then

\Centerline{$\mu(G'_i\setminus F'_i)
\le\mu(G'_i\setminus E'_i)+\mu(E'_i\setminus F'_i)
\le\sum_{j\in K,f(j)=i}\mu(G_j\setminus E_j)+\eta
\le(n+1)\eta$.}

\noindent Note that $G'_i\in\Tau\cap\Sigma_i$ for each $i$.   Now set
$F_j=G_j\setminus(G'_{f(j)}\setminus F'_{f(j)})$ for $j\in K$.
Then $F_j\in\Tau\cap\Sigma_{f(j)}$ and

\Centerline{$\mu(E_j\setminus F_j)
\le\mu(E_j\setminus G_j)+\mu(G'_{f(j)}\setminus F'_{f(j)})
\le(n+2)\eta=\epsilon$.}

\noindent Also

$$\eqalign{\bigcap_{j\in K}F_j
&=\bigcap_{i\in f[K]}\bigcap_{\Atop{j\in K}{f(j)=i}}
    G_j\setminus(G'_i\setminus F'_i)\cr
&=\bigcap_{i\in f[K]}G'_i\setminus(G'_i\setminus F'_i)
\subseteq\bigcap_{i\in f[K]}F'_i
=\emptyset.\cr}$$

\noindent As $\family{j}{K}{E_j}$ and $\epsilon$ are arbitrary,
$\family{j}{J}{\Sigma_{f(j)}}$ has $\Tau$-removable intersections.

\medskip

{\bf (c)} If $J\subseteq I$ is finite and not empty,
$\family{j}{J}{E_j}\in\prod_{j\in J}\Sigma_j$,
$\bigcap_{j\in J}E_j$ is negligible and $\epsilon>0$, then for each
$j\in J$ let $E'_j\in\Sigma'_j$ be such that $E'_j\symmdiff E_j$ is
negligible.   In this case, $\bigcap_{j\in J}E'_j$ is negligible, so
there are $F_j\in\Tau\cap\Sigma'_j$, for $j\in J$, such that
$\mu(E'_j\setminus F_j)\le\epsilon$ for every $j\in J$ and
$\bigcap_{j\in J}F_j$ is empty.   Now $\mu(E_j\setminus F_j)\le\epsilon$
for every $j$.   As $\family{j}{J}{E_j}$ and $\epsilon$ are arbitrary,
$\familyiI{\Sigma_i}$ has $\Tau$-removable intersections.
}%end of proof of 497B

\leader{497C}{Lemma} Let $(X,\Sigma,\mu)$ be a probability space and $\Tau$ a
subalgebra of $\Sigma$.   Let $I$ be a set, $A$ an upwards-directed set,
and $\langle\Sigma_{\alpha i}\rangle_{\alpha\in A,i\in I}$ a family
of $\sigma$-subalgebras of $\Sigma$ such that, setting
$\Sigma_i=\bigvee_{\alpha\in A}\Sigma_{\alpha i}$ for each $i$,

\inset{(i) $\Sigma_{\alpha i}\subseteq\Sigma_{\beta i}$ whenever $i\in I$
and $\alpha\le\beta$ in $A$,

(ii) $\familyiI{\Sigma_{\alpha i}}$ has $\Tau$-removable intersections for
every $\alpha\in A$,

(iii) $\Sigma_i$ and $\bigvee_{j\in I}\Sigma_{\alpha j}$ are relatively
independent over $\Sigma_{\alpha i}$ for every $i\in I$ and $\alpha\in A$.}

\noindent Then $\familyiI{\Sigma_i}$ has $\Tau$-removable intersections.
%Tau 6.13

\proof{ Take a non-empty finite set $J\subseteq I$, a family
$\family{i}{J}{E_i}$ such that $E_i\in\Sigma_i$ for every $i\in J$ and
$\bigcap_{i\in J}E_i$ is negligible, and $\epsilon>0$.   Set
$\delta=\Bover1{\#(J)+1}$,
$\eta=\delta\sqrt{\bover{\epsilon}2}>0$.
For each $i\in J$ there are an $\alpha\in A$ and
an $E'_i\in\Sigma_{\alpha i}$ such that $\mu(E_i\symmdiff E'_i)\le\eta^2$
(because $A$ is upwards-directed, so
$\bigcup_{\alpha\in A}\Sigma_{\alpha i}$ is a subalgebra of $\Sigma$ and
is metrically dense in $\Sigma_i$);
we can suppose that it is the same $\alpha$ for each $i$.   Let
$g_i:X\to[0,1]$ be a $\Sigma_{\alpha i}$-measurable function which is a
conditional expectation of $\chi E_i$ on $\Sigma_{\alpha i}$;   then

\Centerline{$\|\chi E_i-g_i\|_2\le\|\chi E_i-\chi E'_i\|_2\le\eta$}

\noindent(cf.\  244Nb).   Set
$E''_i=\{x:g_i(x)\ge 1-\delta\}\in\Sigma_{\alpha i}$;
then

\Centerline{$\mu(E_i\setminus E''_i)
=\mu\{x:\chi E_i(x)-g_i(x)>\delta\}
\le\Bover{\eta^2}{\delta^2}=\Bover{\epsilon}{2}$.}

Set $E=\bigcap_{i\in J}E''_i$.   Then $\mu E=0$.   \Prf\
Since $\mu(\bigcap_{i\in J}E_i)=0$,
$\mu E\le\sum_{i\in J}\mu(E\setminus E_i)$.
For $i\in J$, set
$H_i=X\cap\bigcap_{j\in J\setminus\{i\}}E''_j$ and let $h_i$ be a
conditional
expectation of $\chi H_i$ on $\Sigma_{\alpha i}$.   Then
$X\setminus E_i\in\Sigma_i$ and
$H_i\in\bigvee_{j\in I}\Sigma_{\alpha j}$ are relatively independent over
$\Sigma_{\alpha i}$, while $\chi X-g_i$ is a
conditional expectation
of $\chi(X\setminus E_i)$ on $\Sigma_{\alpha i}$, so

$$\eqalignno{\mu(E\setminus E_i)
&=\mu((E''_i\setminus E_i)\cap H_i)
=\int_{E''_i}\chi(X\setminus E_i)
  \times\chi H_id\mu
=\int_{E''_i}(\chi X-g_i)\times h_id\mu\cr
\displaycause{by the definition of `relative independence', 458Aa}
&=\int(\chi X-g_i)\times\chi E''_i\times h_id\mu
\le\int\delta\chi E''_i\times h_id\mu\cr
\displaycause{by the definition of $E''_i$}
&=\delta\int_{E''_i}h_id\mu
=\delta\mu(E''_i\cap H_i)
=\delta\mu E.\cr}$$

Summing, we have

\Centerline{$\mu E\le\delta\#(J)\mu E$;}

\noindent but $\delta\#(J)<1$, so $\mu E=0$.\ \Qed

Because $\familyiI{\Sigma_{\alpha i}}$ has $\Tau$-removable intersections, there are $F_i\in\Tau\cap\Sigma_{\alpha i}\subseteq\Tau\cap\Sigma_i$, for $i\in J$,
such that $\bigcap_{i\in I}F_i=\emptyset$ and
$\mu(E''_i\setminus F_i)\le\bover{\epsilon}2$ for each $i$;  in which case
$\mu(E_i\setminus F_i)\le\epsilon$ for each $i$.
As $\family{i}{J}{E_i}$ and $\epsilon$ are arbitrary,
$\familyiI{\Sigma_i}$ has $\Tau$-removable intersections.
}%end of proof of 497C

\leader{497D}{Lemma} Let $(X,\Sigma,\mu)$ be a probability space, $\Tau$ a
subalgebra of $\Sigma$, and $\familyiI{\Sigma_i}$ a finite family of
$\sigma$-subalgebras of $\Sigma$ which has $\Tau$-removable intersections;
suppose that $\Tau\cap\Sigma_i$ is metrically dense in $\Sigma_i$ for each
$i$.   Set $\Sigma^*=\bigvee_{i\in I}\Sigma_i$.
Suppose that we have a finite set $\Gamma$, a function $g:\Gamma\to I$ and a family
$\family{\gamma}{\Gamma}{\Lambda_{\gamma}}$ of $\sigma$-subalgebras of $\Sigma$ such that

\inset{$\family{\gamma}{\Gamma}{\Lambda_{\gamma}}$
is relatively independent over $\Sigma^*$,

for each $\gamma\in\Gamma$, $\Lambda_{\gamma}$ and $\Sigma^*$ are relatively independent
over $\Sigma_{g(\gamma)}$,

for each $\gamma\in\Gamma$, $\Tau\cap\Lambda_{\gamma}$ is metrically dense in $\Lambda_{\gamma}$.}

\noindent Let $A$ be a finite
set and $f:A\to I$, $\phi:A\to\Cal P\Gamma$ functions
such that $\Sigma_{g(\gamma)}\subseteq\Sigma_{f(\alpha)}$ whenever $\alpha\in A$ and
$\gamma\in\phi(\alpha)$.   Suppose that

\inset{for each $\alpha\in A$, $\bigvee_{\gamma\in\phi(\alpha)}\Lambda_{\gamma}$ and
$\Sigma^*\vee\bigvee_{\gamma\in\Gamma\setminus\phi(\alpha)}\Lambda_{\gamma}$
are relatively independent over $\Sigma_{f(\alpha)}$.}

\noindent Set $\tilde\Sigma_{\alpha}
=\Sigma_{f(\alpha)}\vee\bigvee_{\gamma\in\phi(\alpha)}\Lambda_{\gamma}$
for $\alpha\in A$.   Then $\family{\alpha}{A}{\tilde\Sigma_{\alpha}}$ has
$\Tau$-removable intersections.
%compare Tao 6.12

\proof{ Of course we can suppose that $A$ is non-empty, and
that $\Gamma=\bigcup_{\alpha\in A}\phi(\alpha)$.

\medskip

{\bf (a)} To begin with, suppose that every $\Lambda_{\gamma}$ is actually a
finite subalgebra of $\Tau$.

\medskip

\quad{\bf (i)} Take a non-empty set $B\subseteq A$, a family
$\family{\alpha}{B}{E_{\alpha}}\in\prod_{\alpha\in B}\tilde\Sigma_{\alpha}$
such that $\bigcap_{\alpha\in B}E_{\alpha}$ is negligible, and
$\epsilon>0$.   Set $\Delta=\bigcup_{\alpha\in B}\phi(\alpha)$.
Let $\Cal A$ be the set of atoms of $\bigvee_{\gamma\in\Delta}\Lambda_{\gamma}$
and set $\eta=\Bover{\epsilon}{\#(\Cal A)}>0$.

\medskip

\quad{\bf (ii)} For each $H\in\Cal A$ and $\alpha\in B$,
let $C(H,\alpha)$ be the atom of $\bigvee_{\gamma\in\phi(\alpha)}\Lambda_{\gamma}$
including $H$.   Then there is a family
$\family{\alpha}{B}{F_{H\alpha}}$, with empty intersection, such that
$F_{H\alpha}\in\Tau\cap\Sigma_{\alpha}$ and
$\mu(E_{\alpha}\cap C(H,\alpha)\setminus F_{H\alpha})\le\eta$ for each
$\alpha\in B$.   \Prf\ For
each $\gamma\in\Delta$, let $H_{\gamma}$ be the atom of $\Lambda_{\gamma}$ including $H$,
$h_{\gamma}:X\to[0,1]$ a $\Sigma_{g(\gamma)}$-measurable function which is a
conditional
expectation of $\chi H_{\gamma}$ on $\Sigma_{g(\gamma)}$, and $G_{\gamma}=\{x:h_{\gamma}(x)>0\}$.
Note that

\Centerline{$C(H,\alpha)=X\cap\bigcap_{\gamma\in\phi(\alpha)}H_{\gamma}$ for every
$\alpha\in B$,}

\Centerline{$H=X\cap\bigcap_{\gamma\in\Delta}H_{\gamma}=\bigcap_{\alpha\in B}C(H,\alpha)$.}

Because $\Lambda_{\gamma}$ and $\Sigma^*$ are relatively independent over
$\Sigma_{g(\gamma)}$, and $\Sigma_{g(\gamma)}\subseteq\Sigma^*$,
$h_{\gamma}$ is a conditional expectation of $\chi H_{\gamma}$ on $\Sigma^*$ for each
$\gamma$ (458Fb).
Because $\family{\gamma}{\Delta}{\Lambda_{\gamma}}$ is relatively independent over
$\Sigma^*$, $h=\prod_{\gamma\in\Delta}h_{\gamma}$ is a conditional expectation of
$\chi H=\chi(X\cap\bigcap_{\gamma\in\Delta}H_{\gamma})$ on $\Sigma^*$.
(For the trivial case in which $\Delta=\emptyset$, take $h=\chi X$.)
For each $\alpha\in B$ we have
$E_{\alpha}\in\Sigma_{f(\alpha)}\vee\bigvee_{\gamma\in\phi(\alpha)}\Lambda_{\gamma}$,
so there is an an $E'_{\alpha}\in\Sigma_{f(\alpha)}$ such
that $E_{\alpha}\cap C(H,\alpha)=E'_{\alpha}\cap C(H,\alpha)$.
Now $\bigcap_{\alpha\in B}E'_{\alpha}\in\Sigma^*$, so

\Centerline{$\int_{\bigcapop_{\alpha\in B}E'_{\alpha}}h
=\mu(\bigcap_{\alpha\in B}E'_{\alpha}\cap\bigcap_{\gamma\in\Delta}H_{\gamma})
\le\mu(\bigcap_{\alpha\in B}(E_{\alpha}\cap C(H,\alpha)))
=0$;}

\noindent accordingly
$\bigcap_{\alpha\in B}E'_{\alpha}\cap\bigcap_{\gamma\in\Delta}G_{\gamma}$ is negligible.
Set $E''_{\alpha}=E'_{\alpha}\cap\bigcap_{\gamma\in\phi(\alpha)}G_{\gamma}$ for each
$\alpha\in B$;  then $E''_{\alpha}\in\Sigma_{f(\alpha)}$, because
$G_{\gamma}\in\Sigma_{g(\gamma)}\subseteq\Sigma_{f(\alpha)}$ whenever
$\gamma\in\phi(\alpha)$.
Also $\bigcap_{\alpha\in B}E''_{\alpha}$ is negligible.

Because $\familyiI{\Sigma_i}$ has $\Tau$-removable intersections and
$\Tau\cap\Sigma_i$ is metrically dense in $\Sigma_i$ for each $i$,
$\family{\alpha}{B}{\Sigma_{f(\alpha)}}$ has $\Tau$-removable intersections
(497Bb).
So we have $F_{H\alpha}\in\Tau\cap\Sigma_{f(\alpha)}$, for
$\alpha\in B$, such that $\bigcap_{\alpha\in A}F_{H\alpha}=\emptyset$ and
$\mu(E''_{\alpha}\setminus F_{H\alpha})\le\eta$ for every $\alpha$.

If $\alpha\in B$ and $\gamma\in\phi(\alpha)$,

\Centerline{$0=\int_{E'_{\alpha}\setminus G_{\gamma}}h_{\gamma}
=\mu(H_{\gamma}\cap E'_{\alpha}\setminus G_{\gamma})$}

\noindent because $h_{\gamma}$ is a conditional expectation of $\chi H_{\gamma}$ on
$\Sigma^*$.   So if $\alpha\in B$,

$$\eqalignno{\mu(E_{\alpha}\cap C(H,\alpha)\setminus F_{H\alpha})
&=\mu(E'_{\alpha}\cap C(H,\alpha)\setminus F_{H\alpha})\cr
&\le\mu(E''_{\alpha}\setminus F_{H\alpha})
  +\sum_{\gamma\in\phi(\alpha)}\mu(E'_{\alpha}\cap H_{\gamma}\setminus G_{\gamma})\cr
\displaycause{because $C(H,\alpha)=X\cap\bigcap_{\gamma\in\phi(\alpha)}H_{\gamma}$}
&\le\eta,\cr}$$

\noindent as required.\ \Qed

\medskip

\quad{\bf (iii)}
For $\alpha\in B$ let $\Cal A_{\alpha}$ be the set of atoms of
$\bigvee_{\gamma\in\phi(\alpha)}\Lambda_{\gamma}$ and set

\Centerline{$F_{\alpha}
=\bigcup_{G\in\Cal A_{\alpha}}\bigcap_{H\in\Cal A,H\subseteq G}
   F_{H\alpha}$.}

\noindent Then $F_{\alpha}\in\Tau\cap\tilde\Sigma_{\alpha}$ and

$$\eqalign{\mu(E_{\alpha}\setminus F_{\alpha})
&=\sum_{G\in\Cal A_{\alpha}}\mu(E_{\alpha}\cap G\setminus F_{\alpha})\cr
&\le\sum_{G\in\Cal A_{\alpha}}\sum_{\Atop{H\in\Cal A}{H\subseteq G}}
   \mu(E_{\alpha}\cap G\setminus F_{H\alpha})\cr
&=\sum_{H\in\Cal A}
   \mu(E_{\alpha}\cap C(H,\alpha)\setminus F_{H\alpha})
\le\eta\#(\Cal A)=\epsilon.\cr}$$

\noindent If $H\in\Cal A$ then $H\subseteq C(H,\alpha)\in\Cal A_{\alpha}$
and

\Centerline{$H\cap F_{\alpha}
\subseteq F_{\alpha}\cap C(H,\alpha)
\subseteq F_{H\alpha}$,}

\noindent for each $\alpha$.   So $H\cap\bigcap_{\alpha\in A}F_{\alpha}$ is
empty.   But $X=\bigcup\Cal A$ so
$\bigcap_{\alpha\in A}F_{\alpha}=\emptyset$.   As
$\family{\alpha}{B}{E_{\alpha}}$ and $\epsilon$ are arbitrary,
$\family{\alpha}{A}{\tilde\Sigma_{\alpha}}$ has
$\Tau$-removable intersections.

\medskip

{\bf (b)} Next, suppose that each $\Lambda_{\gamma}$ is the
$\sigma$-algebra generated by $\Tau\cap\Lambda_{\gamma}$.

\medskip

\quad{\bf (i)} For $L\in[\Tau]^{<\omega}$, $\gamma\in\Gamma$ and $\alpha\in A$
write $\Lambda_{L\gamma}$ for the algebra $\sigma$-generated by
$\Lambda_{\gamma}\cap L$ and $\tilde\Sigma_{L\alpha}
=\Sigma_{f(\alpha)}\vee\bigvee_{\gamma\in\phi(\alpha)}\Lambda_{L\gamma}$.
Then

\Centerline{$\tilde\Sigma_{\Delta\alpha}\subseteq\tilde\Sigma_{L\alpha}$ whenever
$\alpha\in A$ and $\Delta\subseteq L\in[\Tau]^{<\omega}$,}

\Centerline{$\bigvee_{L\in[\Tau]^{<\omega}}\tilde\Sigma_{L\alpha}
=\Sigma_{f(\alpha)}\vee\bigvee_{\gamma\in\phi(\alpha)}
   \bigvee_{L\in[\Tau]^{<\omega}}\Lambda_{L\gamma}
=\Sigma_{f(\alpha)}\vee\bigvee_{\gamma\in\phi(\alpha)}\Lambda_{\gamma}
=\tilde\Sigma_{\alpha}$}

\noindent because each $\Lambda_{\gamma}$ is the $\sigma$-algebra generated by
$\Tau\cap\Lambda_{\gamma}=\bigcup_{L\in[\Tau]^{<\omega}}\Lambda_{L\gamma}$.
By (a), $\family{\alpha}{A}{\tilde\Sigma_{L\alpha}}$ has $\Tau$-removable
intersections for every $L\in[\Tau]^{<\omega}$.

\medskip

\quad{\bf (ii)} Suppose that $\alpha\in A$ and $L\in[\Tau]^{<\omega}$.
Then $\bigvee_{\gamma\in\phi(\alpha)}\Lambda_{\gamma}$ and
$\Sigma^*\vee\bigvee_{\gamma\in\Gamma\setminus\phi(\alpha)}\Lambda_{\gamma}$ are
relatively independent over $\Sigma_{f(\alpha)}$, by hypothesis.   So
$\tilde\Sigma_{\alpha}
=\Sigma_{f(\alpha)}\vee\bigvee_{\gamma\in\phi(\alpha)}\Lambda_{\gamma}$ and
$\Sigma^*\vee\bigvee_{\gamma\in\Gamma\setminus\phi(\alpha)}\Lambda_{\gamma}$ are
relatively independent over $\Sigma_{f(\alpha)}$ (458Db).
Because
$\Sigma_{f(\alpha)}\subseteq\tilde\Sigma_{L\alpha}
\subseteq\tilde\Sigma_{\alpha}$,
$\tilde\Sigma_{\alpha}$ and
$\Sigma^*\vee\bigvee_{\gamma\in\Gamma\setminus\phi(\alpha)}\Lambda_{\gamma}$ are
relatively independent over $\tilde\Sigma_{L\alpha}$ (458Dc).   Because
$\bigvee_{\gamma\in\phi(\alpha)}\Lambda_{L\gamma}\subseteq\tilde\Sigma_{L\alpha}$,
$\tilde\Sigma_{\alpha}$ and
$\Sigma^*\vee\bigvee_{\gamma\in\Gamma}\Lambda_{L\gamma}$ are relatively independent
over $\tilde\Sigma_{L\alpha}$ (458Db again).
So $\tilde\Sigma_{\alpha}$ and

\Centerline{$\bigvee_{\beta\in A}\tilde\Sigma_{L\beta}
=\bigvee_{\beta\in A}
  (\Sigma_{f(\beta)}\vee\bigvee_{\gamma\in\phi(\beta)}\Lambda_{L\gamma})
\subseteq\Sigma^*\vee\bigvee_{\gamma\in\Gamma}\Lambda_{L\gamma}$}

\noindent are relatively independent over $\tilde\Sigma_{L\alpha}$.



\medskip

\quad{\bf (iii)} With (i), this shows that the family
$\langle\tilde\Sigma_{L\alpha}\rangle_{L\in[\Tau]^{<\omega},\alpha\in A}$
satisfies the conditions of 497C, and
$\family{\alpha}{A}{\tilde\Sigma_{\alpha}}$ has $\Tau$-removable
intersections.

\medskip

{\bf (c)} Finally, for the general case, let $\Lambda'_{\gamma}$ be the
$\sigma$-algebra generated by $\Lambda_{\gamma}\cap\Tau$ for $\gamma\in\Gamma$, and
$\Sigma'_{\alpha}
=\Sigma_{f(\alpha)}\vee\bigvee_{\gamma\in\phi(\alpha)}\Lambda'_{\gamma}$ for
$\alpha\in A$.   If $\gamma\in\Gamma$ and $F\in\Lambda_{\gamma}$, there is an
$F'\in\Lambda'_{\gamma}$ such that $F\symmdiff F'$ is negligible;  so if
$\alpha\in A$ and $E\in\tilde\Sigma_{\alpha}$, there is an
$E'\in\Sigma'_{\alpha}$ such that $E\symmdiff E'$ is negligible.   By
(b), $\family{\alpha}{A}{\Sigma'_{\alpha}}$ has $\Tau$-removable
intersections;  by 497Bc,
$\family{\alpha}{A}{\tilde\Sigma_{\alpha}}$ has $\Tau$-removable
intersections.
}%end of proof of 497D

\leader{497E}{Theorem}\cmmnt{ (\TaoIG)} % Theorem 4.2
Let $(X,\Sigma,\mu)$ be a
probability space, and $\Tau$ a subalgebra of $\Sigma$.   Let $\Gamma$ be a
partially ordered set such that $\gamma\wedge\delta=\inf\{\gamma,\delta\}$
is defined in $\Gamma$ for all
$\gamma$, $\delta\in\Gamma$, and $\family{\gamma}{\Gamma}{\Sigma_{\gamma}}$
a family of $\sigma$-subalgebras of $\Sigma$ such that

\inset{(i) $\Tau\cap\Sigma_{\gamma}$ is metrically dense in
$\Sigma_{\gamma}$ for every $\gamma\in\Gamma$,

(ii) if $\gamma$, $\delta\in\Gamma$ and $\gamma\le\delta$ then
$\Sigma_{\gamma}\subseteq\Sigma_{\delta}$,

(iii) if $\gamma\in\Gamma$ and $\Delta$, $\Delta'$ are
finite subsets of
$\Gamma$ such that $\delta\wedge\gamma\in\Delta'$ for every
$\delta\in\Delta$, then $\Sigma_{\gamma}$ and
$\bigvee_{\delta\in\Delta}\Sigma_{\delta}$ are relatively independent over
$\bigvee_{\delta\in\Delta'}\Sigma_{\delta}$.}

\noindent Then $\family{\gamma}{\Gamma}{\Sigma_{\gamma}}$
has $\Tau$-removable intersections.

\proof{{\bf (a)} To begin with (down to the end of (d) below)
suppose that $\Gamma$ is finite.
In this case, we have a rank function
$r:\Gamma\to\Bbb N$ such that $r(\gamma)
=\min\{n:n\in\Bbb N$, $r(\delta)<n$ for every $\delta<\gamma\}$
for each $\gamma\in\Gamma$.   For $a\subseteq\Gamma$ set
$\tilde\Sigma_a=\bigvee_{\gamma\in a}\Sigma_{\gamma}$;  note that
$\Tau\cap\tilde\Sigma_a$ is always metrically dense in $\tilde\Sigma_a$.

Let $A$ be the family of those sets
$a\subseteq\Gamma$ such that $\gamma\in a$ whenever $\gamma\le\delta\in a$.
For $n\in\Bbb N$ set $\Gamma_n=\{\gamma:r(\gamma)=n\}$ and
$A_n=\{a:a\in A$, $r(\gamma)<n$ for every $\gamma\in a\}$

\medskip

{\bf (b)} Suppose that $a$, $b$, $c$ are subsets of
$\Gamma$ and that $\gamma\wedge\delta\in c$ whenever
$\gamma\in a$ and
$\delta\in b\cup(a\setminus\{\gamma\})$.   Then

\inset{(i) $\family{\gamma}{a}{\Sigma_{\gamma}}$ is relatively
independent over $\tilde\Sigma_c$,

(ii) $\tilde\Sigma_a$ and $\tilde\Sigma_b$ are relatively independent
over $\tilde\Sigma_c$.}

\noindent\Prf\ Induce on $\#(a)$.   If $a=\emptyset$ then
$\tilde\Sigma_a=\{\emptyset,X\}$ and the result is trivial.   For the
inductive step, take $\gamma_0\in a$ and set $a'=a\setminus\{\gamma_0\}$.
Then the inductive hypothesis tells us that
$\family{\gamma}{a'}{\Sigma_{\gamma}}$ is relatively
independent over $\tilde\Sigma_c$ and that
$\tilde\Sigma_{a'}$ and $\tilde\Sigma_b$ are relatively independent
over $\tilde\Sigma_c$.   We also see that $\gamma_0\wedge\delta\in c$
whenever $\delta\in a'$, so that
$\Sigma_{\gamma_0}$ and $\tilde\Sigma_{a'}$
are relatively independent over $\tilde\Sigma_c$, by condition (iii) of
this theorem.  But this means that
$\family{\gamma}{a}{\Sigma_{\gamma}}$ is relatively independent over
$\tilde\Sigma_c$ (458Hb).   Similarly, because in fact
$\gamma_0\wedge\delta\in c$ for every $\delta\in a'\cup b$,
$\Sigma_{\gamma_0}$ and $\tilde\Sigma_{a'}\vee\tilde\Sigma_b$ are
relatively independent over $\tilde\Sigma_c$;  so the triple
$\Sigma_{\gamma_0}$, $\tilde\Sigma_{a'}$ and $\tilde\Sigma_b$ are
relatively independent over $\tilde\Sigma_c$ (458Hb again),
and $\tilde\Sigma_a=\Sigma_{\gamma_0}\vee\tilde\Sigma_{a'}$ and
$\tilde\Sigma_b$ are relatively independent over $\tilde\Sigma_c$
(458Ha).   Thus the induction continues.\ \Qed

\medskip

{\bf (c)} For each $n\in\Bbb N$, $\family{a}{A_n}{\tilde\Sigma_a}$ has
$\Tau$-removable intersections.   \Prf\ Induce on $n$.   If $n=0$ then
$A_n=\{\emptyset\}$ and the result is trivial.   For the inductive step to
$n+1\ge 1$, apply 497D, as follows.   The inductive hypothesis tells
us that $\family{a}{A_n}{\tilde\Sigma_{a}}$ has $\Tau$-removable
intersections, and we know that
$\Tau\cap\tilde\Sigma_{a}$ is always metrically dense in
$\tilde\Sigma_a$.   Set

\Centerline{$\Sigma^*=\bigvee_{a\in A_n}\tilde\Sigma_a=\tilde\Sigma_d$ }

\noindent where $d=\bigcup_{m<n}\Gamma_m$
is the largest member of $A_n$.   Define $g:\Gamma_n\to A_n$ by setting
$g(\gamma)=\{\delta:\delta<\gamma\}$.   Then $\gamma\wedge\delta\in d$ for
all distinct $\gamma$, $\delta\in\Gamma_n$, so
$\family{\gamma}{\Gamma_n}{\Sigma_{\gamma}}$ is relatively independent over
$\Sigma^*$, by (b-i) just above.   If $\gamma\in\Gamma_n$ and
$\delta\in d$, $\gamma\wedge\delta\in g(\gamma)$, so
$\Sigma_{\gamma}=\tilde\Sigma_{\{\gamma\}}$ and
$\Sigma^*=\tilde\Sigma_d$ are relatively independent over
$\tilde\Sigma_{g(\gamma)}$, by (b-ii).
Of course $\Tau\cap\Sigma_{\gamma}$ is
metrically dense in $\Sigma_{\gamma}$ for every $\gamma\in\Gamma_n$.

For $a\in A_{n+1}$, set $\phi(a)=a\cap\Gamma_n$ and

\Centerline{$f(a)=a\setminus\phi(a)=a\cap\bigcup_{m<n}\Gamma_m\in A_n$.}

\noindent If $\gamma\in\phi(a)$
then $g(\gamma)\subseteq a$, by the definition of $A$, so
$g(\gamma)\subseteq f(a)$ and
$\tilde\Sigma_{g(\gamma)}\subseteq\tilde\Sigma_{f(a)}$.   Finally, by
(b-ii), $\bigvee_{\gamma\in\phi(a)}\Sigma_{\gamma}$ and
$\Sigma^*\vee\bigvee_{\gamma\in\Gamma_n\setminus\phi(a)}\Sigma_{\gamma}$
are relatively independent over $\tilde\Sigma_{f(a)}$, because if
$\gamma\in\phi(a)$ and $\delta\in d\cup(\Gamma_n\setminus\phi(a))$ then
$\gamma\wedge\delta\in g(\gamma)\subseteq f(a)$.

So all the hypotheses of 497D are satisfied, and

\Centerline{$\family{a}{A_{n+1}}
{\tilde\Sigma_{f(a)}\vee\bigvee_{\gamma\in\phi(a)}\Sigma_{\gamma}}
=\family{a}{A_{n+1}}{\tilde\Sigma_{a}}$}

\noindent has $\Tau$-removable intersections.   Thus the induction
proceeds.\ \Qed

\medskip

{\bf (d)} Because $\Gamma$ is finite, there is some $n$ such that $A=A_n$.
Now, for each $\gamma\in\Gamma$, set
$e_{\gamma}=\{\delta:\delta\le\gamma\}$;  then $e_{\gamma}\in A$ and
$\Sigma_{\gamma}=\tilde\Sigma_{e_{\gamma}}$.
By 497Bb, or otherwise,
$\family{\gamma}{\Gamma}{\Sigma_{\gamma}}$ has $\Tau$-removable
intersections, as required.

\medskip

{\bf (e)} Thus the theorem is true when $\Gamma$ is finite.   For the
general case, take any finite $\Gamma_0\subseteq\Gamma$ and set
$\Gamma'=\{\inf a:a\subseteq\Gamma_0$ is non-empty$\}$.   Then
$\Gamma'$ is finite and closed under $\wedge$,
and $\family{\gamma}{\Gamma'}{\Sigma_{\gamma}}$ satisfies the conditions of
the theorem.   So $\family{\gamma}{\Gamma'}{\Sigma_{\gamma}}$ and
$\family{\gamma}{\Gamma_0}{\Sigma_{\gamma}}$ have $\Tau$-removable
intersections.   As $\Gamma_0$ is arbitrary,
$\family{\gamma}{\Gamma}{\Sigma_{\gamma}}$ has $\Tau$-removable
intersections (497Ba), and the proof is complete.
}%end of proof of 497E

\leader{497F}{Invariant measures on $\Cal P([I]^{<\omega})$ (a)}
Let $I$ be a set.   Then $\Cal P([I]^{<\omega})$
is a compact Hausdorff space, if we give it its usual topology, generated
by sets of the form $\{R:a\in R\subseteq[I]^{<\omega}$, $b\notin R\}$ for
finite sets $a$, $b\subseteq I$.
\cmmnt{(You should perhaps fix on the case $I=\Bbb N$
for the first reading of this paragraph,
so that $[I]^{<\omega}$ will be a relatively familiar
countable set, and you can remember that $\Cal P([I]^{<\omega})$
is homeomorphic to the Cantor set.)}
Let $G_I$ be the set of permutations of $I$,
and for $\phi\in G_I$, $R\subseteq[I]^{<\omega}$ set

\Centerline{$\phi\action R=\{\phi[a]:a\in R\}
=\{a:a\in[I]^{<\omega}$, $\phi^{-1}[a]\in R\}$,}

\noindent so that $\action$ is an action of $G_I$ on
$\Cal P([I]^{<\omega})$, and $R\mapsto\phi\action R$ is a homeomorphism for
every $\phi\in G_I$.
Let $P_I$ be the set of Radon probability measures on
$\Cal P([I]^{<\omega})$.   Then we have an action of $G_I$ on $P_I$ defined
by saying that

\Centerline{$\phi\action E=\{\phi\action R:R\in E\}$}

\noindent for $\phi\in G_I$ and $E\subseteq\Cal P([I]^{<\omega})$, and

\Centerline{$(\phi\action\mu)(E)=\mu(\phi^{-1}\action E)$}

\noindent for $\phi\in G_I$, $\mu\in P_I$ and Borel sets
$E\subseteq\Cal P([I]^{<\omega})$.
Because $R\mapsto\phi\action R$ is a homeomorphism, the map
$\mu\mapsto\phi\action\mu$ is a homeomorphism when $P_I$ is given its
narrow topology\cmmnt{, corresponding to the weak* topology on
$C(\Cal P([I]^{<\omega}))^*$ (437J, 437Kc)}.

\spheader 497Fb If $\mu\in P_I$,\cmmnt{ I will say that} $\mu$ is
{\bf permutation-invariant} if $\mu=\phi\action\mu$ for every
$\phi\in G_I$.

\spheader 497Fc
For $R\subseteq[I]^{<\omega}$ and $J\subseteq I$\cmmnt{ I} write
$R\lceil J$ for the trace $R\cap\Cal PJ\subseteq[J]^{<\omega}$
of $R$ on $J$.   Let $\Cal V$ be the
family of sets of the form
$\cmmnt{V_{JS}=}\{R:R\subseteq[I]^{<\omega}$, $R\lceil J=S\}$
where $J\subseteq I$ is
finite and $S\subseteq\Cal PJ$.   If $\mu$, $\nu\in P_I$ agree on
$\Cal V$, they are equal.   \prooflet{\Prf\
If $E\subseteq\Cal P([I])^{<\omega}$ is
open-and-closed, it is determined by coordinates in some finite subset
$\Cal K$ of $[I]^{<\omega}$, in the sense that if $R\in E$,
$R'\subseteq[I]^{<\omega}$ and $R\cap\Cal K=R'\cap\Cal K$, then
$R'\in E$.   Let $J\subseteq I$ be a finite set such that
$\Cal K\subseteq[J]^{<\omega}$, and set
$\Cal S=\{R\lceil J:R\in E\}$.   Now $\family{S}{\Cal S}{V_{JS}}$ is a
disjoint family in $\Cal V$ with union $E$, so

\Centerline{$\mu E=\sum_{S\in\Cal S}\mu V_{JS}=\nu E$.}

\noindent As $E$ is arbitrary, $\mu=\nu$ (416Qa).\ \Qed}

%Observe also that if we take $\Cal W$ to be the
%family of sets of the form
%$\{R:R\subseteq[I]^{<\omega}$, $R\lceil J\in\Cal S\}$
%where $J\subseteq I$ is countable and $\Cal S\subseteq\Cal PJ$, then
%$\Cal W$ is a $\sigma$-algebra of subsets of $\Cal P([I]^{<\omega})$
%including $\Cal V$, so contains every Baire set.

\spheader 497Fd
If $I$, $J$ are sets and $f:I\to J$ is a function,\cmmnt{ I} define
$\tilde f:\Cal P([J]^{<\omega})\to\Cal P([I]^{<\omega})$ by setting
$\tilde f(R)=\{a:a\in[I]^{<\omega}$, $f[a]\in R\}$ for
$R\subseteq[J]^{<\omega}$.   \cmmnt{Note that} $\tilde f$ is
continuous\prooflet{, since
$\{R:a\in\tilde f(R)\}=\{R:f[a]\in R\}$ is
a basic open-and-closed set in $\Cal P([J]^{<\omega})$ for every
$a\in[I]^{<\omega}$}.
%if $f:I\to J$ and $g:J\to K$ then $(gf)^{\sim}(R)=\{a:g[f[a]]\in R\}
%=\{a:f[a]\in\tilde g(R)\}=\tilde f(\tilde g(R))$, so
%(f)^{\sim}=\tilde f\tilde g
If $I\subseteq J$ and $f$ is the identity function, then
$\tilde f(R)=R\lceil I$ for every $R\subseteq[J]^{<\omega}$.
Observe that when $\phi\in G_I$ and $R\subseteq[I]^{<\omega}$
then $\tilde\phi(R)=\phi^{-1}\action R$.

\leader{497G}{Theorem}\cmmnt{ (\TaoIG)}
Let $I$ be an infinite set and $\Cal J$ a filter on $I$ not containing
any finite set.   Let $\Tau$ be the algebra of open-and-closed subsets of
$\Cal P([I]^{<\omega})$, and $\mu\in P_I$ a
permutation-invariant measure.   For $J\subseteq I$, write $\Sigma_J$ for
the $\sigma$-algebra of subsets of $\Cal P([I]^{<\omega})$ generated by
sets of the form $E_a=\{R:a\in R\subseteq[I]^{<\omega}\}$ where
$a\in[J]^{<\omega}$.   Then
$\family{J}{\Cal J}{\Sigma_J}$ has $\Tau$-removable intersections with
respect to $\mu$.

\proof{ I seek to apply 497E with $\Gamma=\Cal J$, ordered by $\subseteq$.
If $J\in\Cal J$ and $a\in[J]^{<\omega}$
then $\{R:a\in R\subseteq[I]^{<\omega}\}$ belongs to $\Tau\cap\Sigma_J$;
accordingly $\Sigma_J$ is the
$\sigma$-algebra generated by $\Tau\cap\Sigma_J$ and $\Tau\cap\Sigma_J$
is metrically dense in $\Sigma_J$.
Condition (ii) of 497E is obviously satisfied.
As for condition (iii), we can use 459I, as follows.   Taking
$X=\Cal P([I]^{<\omega})$, we have the action $\action$ of $G_I$ on $X$
described in 497Fa, and $R\mapsto\phi\action R$ is \imp\ for each $\phi$
because $\mu$ is permutation-invariant.   Now we see easily that

\inset{----- for every $J\subseteq I$,
$\bigcup_{K\in[J]^{<\omega}}\Sigma_K$ contains $E_a$ for every
$a\in[J]^{<\omega}$, so $\sigma$-generates $\Sigma_J$;

----- if $a\in[I]^{<\omega}$ and $\phi\in G_I$,

\Centerline{$E_{\phi[a]}
=\{R:\phi[a]\in R\}=\{\phi\action R:\phi[a]\in\phi\action R\}
=\{\phi\action R:a\in R\}=\phi\action E_a$;}

----- if
$J\subseteq I$, then $\{E:\phi\action E\in\Sigma_{\phi[J]}\}$ is a
$\sigma$-algebra of sets containing
$\phi^{-1}\action E_{\phi[a]}=E_a$ whenever
$a\in[J]^{<\omega}$, so it includes $\Sigma_J$, and
$\phi\action E\in\Sigma_{\phi[J]}$ for every $E\in\Sigma_J$;

----- if $J\subseteq I$ and $\phi\in G_I$ is such that
$\phi(i)=i$ for every $i\in J$, then $\{E:\phi\action E=E\}$ is a
$\sigma$-algebra of sets containing $E_a$ for every $a\in[J]^{<\omega}$, so
it includes $\Sigma_J$, and $\phi\action E=E$ for every $E\in\Sigma_J$.}

\noindent Thus the conditions of 459I are satisfied.
So if $J\in\Cal J$ and $\Cal K$, $\Cal K'$ are (finite) subsets of $\Cal J$
such that $J\cap K\in\Cal K'$ for every $K\in\Cal K$, 459I tells us that
$\Sigma_J$ and $\bigvee_{K\in\Cal K}\Sigma_K$ are relatively independent
over $\bigvee_{K\in\Cal K'}\Sigma_K$, as required by (iii) of 497E.

So 497E gives the result we seek.
}%end of proof of 497G

\leader{497H}{}\cmmnt{ I come now to the next essential ingredient of the
proof.

\woddheader{497H}{4}{2}{2}{72pt}

\noindent}{\bf Construction} Suppose we are given a sequence
$\sequencen{(m_n,T_n)}$ and a non-principal
ultrafilter $\Cal F$ on $\Bbb N$ such that

\inset{($\alpha$) $\sequencen{m_n}$ is a sequence in
$\Bbb N\setminus\{0\}$ and $\lim_{n\to\Cal F}m_n=\infty$,

($\beta$) $T_n\subseteq\Cal Pm_n$ for each $n$.}

\noindent Then for any set $I$ there is a permutation-invariant
$\mu\in P_I$ such that

\Centerline{$\mu\{R:R\lceil K=S\}
=\lim_{n\to\Cal F}\Bover1{m_n^{\#(K)}}
\#(\{z:z\in m_n^K$, $\tilde z(T_n)=S\})$}

\noindent whenever $K\subseteq I$ is finite and $S\subseteq\Cal PK$.

\proof{{\bf (a)}
For each $n\in\Bbb N$ let $\nu_n$ be the usual measure
on $m_n^I$, the product of $I$ copies of the uniform probability measure on
the finite set $m_n$.   The function
$w\mapsto\tilde w(T_n):m_n^I\to\Cal P([I]^{<\omega})$ is continuous, since
for any $a\in[I]^{<\omega}$ the set
$\{w:a\in\tilde w(T_n)\}=\{w:w[a]\in T_n\}$ is determined by coordinates in
the finite set $a$.   So we have a corresponding Radon
probability measure $\mu_n$ on
$\Cal P([I]^{<\omega})$ defined by saying that
$\mu_nE=\nu_n\{w:\tilde w(T_n)\in E\}$ for every set
$E\subseteq\Cal P([I]^{<\omega})$ such that $\nu_n$ measures
$\{w:\tilde w(T_n)\in E\}$ (418I).
If $K\subseteq I$ is finite and $w\in m_n^I$, then

\Centerline{$\tilde w(T_n)\lceil K
=\{a:a\subseteq K,\,w[a]\in T_n\}
=\{a:a\subseteq K,\,(w\restr K)[a]\in T_n\}
=(w\restr K)^{\sptilde}(T_n)$.}

\noindent So if $S\subseteq\Cal PK$, then

$$\eqalign{\mu_n\{R:R\lceil K=S\}
&=\nu_n\{w:w\in m_n^I,\,\tilde w(T_n)\lceil K=S\}\cr
&=\nu_n\{w:w\in m_n^I,\,(w\restr K)^{\sptilde}(T_n)=S\}\cr
&=\Bover1{m_n^{\#(K)}}\#(\{z:z\in m_n^K,\,\tilde z(T_n)=S\}).\cr}$$

\noindent Let $\mu$ be the limit $\lim_{n\to\Cal F}\mu_n$ in the narrow
topology on $P_I$;  then

$$\eqalignno{\mu\{R:R\lceil K=S\}
&=\lim_{n\to\Cal F}\mu_n\{R:R\lceil K=S\}\cr
\displaycause{because $\{R:R\lceil K=S\}$ is open-and-closed;  see 437Jf}
&=\lim_{n\to\Cal F}\Bover1{m_n^{\#(K)}}
  \#(\{z:z\in m_n^K,\,\tilde z(T_n)=S\})\cr}$$

\noindent whenever $K\subseteq I$ is finite and $S\subseteq\Cal PK$.

\medskip

{\bf (b)} Now let $\phi:I\to I$ be any permutation and
$\tilde\phi:\Cal P([I]^{<\omega})\to\Cal P([I]^{<\omega})$ the
corresponding permutation.   Then for any finite $K\subseteq I$,

$$\eqalign{\tilde\phi(R)\lceil K
&=\{a:a\subseteq K,\,a\in\tilde\phi(R)\}\cr
&=\{a:a\subseteq K,\,\phi[a]\in R\}
=\{a:a\in[I]^{<\omega},\,\phi[a]\in R\lceil\phi[K]\}.\cr}$$

\noindent Fix $n\in\Bbb N$ for the moment.   If $K\subseteq I$ is finite,
and $S\subseteq\Cal PK$, then

$$\eqalignno{\mu_n\tilde\phi^{-1}\{R:R\lceil K=S\}
&=\mu_n\{R:\tilde\phi(R)\lceil K=S\}\cr
&=\mu_n\{R:S=\{a:a\in[I]^{<\omega},\,\phi[a]\in R\lceil\phi[K]\}\}\cr
&=\mu_n\{R:R\lceil\phi[K]=\{\phi[a]:a\in S\}\}\cr
&=\nu_n\{w:\tilde w(T_n)\lceil\phi[K]=\{\phi[a]:a\in S\}\}\cr
&=\nu_n\{w:\{a:a\subseteq\phi[K],\,w[a]\in T_n\}
  =\{\phi[a]:a\in S\}\}\cr
&=\nu_n\{w:\{\phi[a]:a\subseteq K,\,w[\phi[a]]\in T_n\}
  =\{\phi[a]:a\in S\}\}\cr
&=\nu_n\{w:\{a:a\subseteq K,\,(w\phi)[a]\in T_n\}=S\}\cr
&=\nu_n\{w:\{a:a\subseteq K,\,w[a]\in T_n\}=S\}\cr
\displaycause{because $w\mapsto w\phi:m_n^I\to m_n^I$ is an automorphism
for the measure $\nu_n$}
&=\nu_n\{w:\tilde w(T_n)\lceil K=S\}
=\mu_n\{R:R\lceil K=S\}.\cr}$$

\noindent So $\mu_n$ and $\mu_n\tilde\phi^{-1}$ agree on the family
$\Cal V$ of basic open-and-closed sets described in 497F.
As this is true
for every $n$, we also have $\mu V=\mu\tilde\phi^{-1}[V]$ for every
$V\in\Cal V$, and $\mu=\mu\tilde\phi^{-1}$.   As $\phi$ is
arbitrary, $\mu$ is permutation-invariant.
}%end of proof of 497H

\leader{497I}{Definition} If $I$, $J$ are sets, $R\subseteq\Cal PI$ and
$S\subseteq\Cal PJ$,\cmmnt{ I will say for the purposes of the next two
results that} an
{\bf embedding} of $(I,R)$ in $(J,S)$ is an injective function $f:I\to J$
such that $f[a]\in S$ for every $a\in R$\cmmnt{, that is (when
$S\subseteq[J]^{<\omega}$), $R\subseteq\tilde f(S)$}.

\leader{497J}{Theorem}\cmmnt{ ({\smc Nagle R\"odl \& Schacht 06})}
% Theorem 3)
Let $L$ be a finite set with $r$ members, and
$T\subseteq\Cal PL$.   Then for every $\epsilon>0$
there is a $\delta>0$ such that whenever $I$ is a non-empty finite set,
$R\subseteq\Cal PI$ and the number of embeddings of $(L,T)$ in
$(I,R)$ is at most $\delta\#(I)^r$,
there is an $S\subseteq\Cal PI$
such that $\#(S\cap[I]^k)\le\epsilon\#(I)^k$ for every $k$
and there is no embedding of $(L,T)$ in $(I,R\setminus S)$.

\proof{ (\TaoIG) \Quer\ Suppose, if possible,
otherwise.

\medskip

{\bf (a)} We have a sequence $\sequencen{(m_n,T_n)}$ such that

\inset{\noindent $m_n\in\Bbb N\setminus\{0\}$, $T_n\subseteq\Cal Pm_n$;
the number of embeddings of $(L,T)$ in $(m_n,T_n)$
is at most $2^{-n}m_n^r$;
if $S\subseteq\Cal Pm_n$ and
$\#(S\cap[m_n]^k)\le\epsilon m_n^k$ for every $k$
then there is an embedding of $(L,T)$ in $(m_n,T_n\setminus S)$}

\noindent for every $n\in\Bbb N$.   Of course $(L,T)$ always has at
least one embedding in $(m_n,T_n)$ so $\lim_{n\to\infty}m_n=\infty$.
Let $I$ be an infinite set including $L$ and $\Cal F$ a non-principal
ultrafilter on $\Bbb N$.   Let $\mu\in P_I$ be the
permutation-invariant measure defined from
$\sequencen{(m_n,T_n)}$ and $\Cal F$ by the process of 497H.

\medskip

{\bf (b)} For $c\subseteq L$ set $J_c=c\cup(I\setminus L)$, so that
$\Sigma_{J_c}$, in the notation of 497G, is the $\sigma$-algebra of subsets of
$\Cal P([I]^{<\omega})$ generated by sets of the form
$E_a=\{R:a\in R\subseteq[I]^{<\omega}\}$ where
$a\in[c\cup(I\setminus L)]^{<\omega}$.   Note that every member of
$\Sigma_{J_c}$ is determined by coordinates in $\Cal PJ_c$,
in the sense that if $R\in E\in\Sigma_{J_c}$,
$R'\subseteq\Cal P([I]^{<\omega})$ and
$R\cap\Cal PJ_c)=R'\cap\Cal PJ_c$, then $R'\in E$.

By 497G, applied to the filter $\Cal J$ on $I$
generated by $\{I\setminus L\}$,
$\langle\Sigma_{J_c}\rangle_{c\subseteq L}$
has $\Tau$-removable intersections with respect to $\mu$, where $\Tau$ is
the algebra of open-and-closed subsets of $\Cal P([I]^{<\omega})$.
{\it A fortiori}, $\family{c}{T}{\Sigma_{J_c}}$
has $\Tau$-removable intersections with respect to $\mu$.

\medskip

{\bf (c)} $E_c\in\Sigma_{J_c}$ for every $c\in T$, and

$$\eqalignno{\mu(\bigcap_{c\in T}E_c)
&=\mu\{R:T\subseteq R\}
=\sum_{T\subseteq T'\subseteq\Cal PL}\mu\{R:R\lceil L=T'\}\cr
&=\sum_{T\subseteq T'\subseteq\Cal PL}
   \lim_{n\to\Cal F}\Bover1{m_n^r}
   \#(\{z:z\in m_n^L,\,\tilde z(T_n)=T'\})\cr
&=\lim_{n\to\Cal F}\Bover1{m_n^r}
   \sum_{T\subseteq T'\subseteq\Cal PL}
   \#(\{z:z\in m_n^L,\,\tilde z(T_n)=T'\})\cr
&=\lim_{n\to\Cal F}\Bover1{m_n^r}
   \#(\{z:z\in m_n^L,\,T\subseteq\tilde z(T_n)\})\cr
&=\lim_{n\to\Cal F}\Bover1{m_n^r}
   \#(\{z:z\in m_n^L\text{ is injective},\,T\subseteq\tilde z(T_n)\})\cr
\displaycause{because $\lim_{n\to\infty}\Bover{k_n}{m_n^r}=0$, where
$k_n=m_n^r-\Bover{m_n!}{(m_n-r)!}$ is the number
of non-injective functions from $L$ to $m_n$}
&=\lim_{n\to\Cal F}\Bover1{m_n^r}
   \#(\{z:z\text{ is an embedding of }(L,T)\text{ in }(m_n,T_n)\})
=0.}$$

\medskip

{\bf (d)} Take $\eta>0$ such that $2\eta\#(T)<\epsilon$, and
$\family{c}{T}{F_c}$ such that
$\bigcap_{c\in T}F_c=\emptyset$ and
$F_c\in\Tau\cap\Sigma_{J_c}$ and
$\mu(E_c\setminus F_c)\le\eta$ for every $c\in T$.
Every $F_c$ is open-and-closed, so
there is an $M\in[I]^{<\omega}$ such that $L\subseteq M$ and
every $F_c$ is determined by coordinates in $\Cal PM$.   In this case,
each $F_c$ is determined by coordinates in
$\Cal PM\cap\Cal PJ_c=\Cal P(c\cup(M\setminus L))$.   Setting

\Centerline{$F'_c=\{R\lceil M:R\in F_c\}$,
\quad$E'_c=\{R\lceil M:R\in E_c\}=\{R:c\in R\subseteq\Cal PM\}$,}

\noindent we have

\Centerline{$F_c=\{R:R\subseteq[I]^{<\omega}$, $R\lceil M\in F'_c\}$,
\quad$E_c=\{R:R\subseteq[I]^{<\omega}$, $R\lceil M\in E'_c\}$,}

\noindent while both $E'_c$ and $F'_c$, and therefore
$E'_c\setminus F'_c$, regarded as subsets of $\Cal P(\Cal PM)$,
are determined by coordinates in
$\Cal P(c\cup(M\setminus L))$.   Because
$\bigcap_{c\in T}F_c$ is empty, so is $\bigcap_{c\in T}F'_c$.

\wheader{497J}{4}{2}{2}{48pt}

{\bf (e)} Let $n\ge r$ be such that

$$\eqalign{\Bover1{m_n^{\#(M)}}\#(\{z:z\in m_n^M,\,
   \tilde z(T_n)\in E'_c\setminus F'_c\})
&\le\eta+\mu\{R:R\lceil M\in E'_c\setminus F'_c\}\cr
&=\eta+\mu\{R:R\in E_c\setminus F_c\}
\le 2\eta\cr}$$

\noindent for every $c\in T$.   For $c\in T$ set

\Centerline{$Q_c=\{z:z\in m_n^M$, $\tilde z(T_n)\in E'_c\setminus F'_c\}$,}

\noindent so that $\#(Q_c)\le 2\eta m_n^{\#(M)}$.   Since

$$\eqalign{\sum_{\Atop{c\in T}{w\in m_n^{M\setminus L}}}
   \#(\{z:w\subseteq z\in Q_c\})
&=\sum_{c\in T}\#(Q_c)
\le 2\eta\#(T)m_n^{\#(M)}\cr
&\le\epsilon m_n^{\#(M)}
=\epsilon\#(m_n^{M\setminus L})m_n^r,\cr}$$

\noindent there must be a $w\in m_n^{M\setminus L}$ such that

\Centerline{$\sum_{c\in T,w\in m_n^{M\setminus L}}
   \#(\{z:w\subseteq z\in Q_c\})
\le \epsilon m_n^r$;}

\noindent set

\Centerline{$Q'_c=\{z:w\subseteq z\in Q_c$, $z\restr c$ is injective$\}$}

\noindent for $c\in T$, so that $\sum_{c\in T}\#(Q'_c)\le\epsilon m_n^r$.

If $c\in T$ and $\#(c)=k$, then

\Centerline{$\#(\{z[c]:z\in Q'_c\})=\Bover1{m_n^{r-k}}\#(Q'_c)$.}

\noindent\Prf\ If $z$, $z'\in m_n^M$ and
$z\restr(c\cup(M\setminus L))=z'\restr(c\cup(M\setminus L))$, then
$\tilde z(T_n)\lceil(c\cup(M\setminus L))
=\tilde z'(T_n)\lceil(c\cup(M\setminus L))$, so
$\tilde z(T_n)\in E'_c\setminus F'_c$ iff
$\tilde z'(T_n)\in E'_c\setminus F'_c$, that is, $z\in Q_c$ iff
$z'\in Q_c$.   So if $a=z[c]$ for some $z\in Q'_c$, then

\Centerline{$\{z':z'\in Q'_c$, $z'[c]=a\}
=\{z':z'\in m_n^M$,
$z'\restr(c\cup(M\setminus L))=z\restr(c\cup(M\setminus L))\}$}

\noindent has just $\#(m_n^{L\setminus c})=m_n^{r-k}$ members.\ \Qed

\medskip

{\bf (f)} Consider

\Centerline{$S=\{z[c]:c\in T$, $z\in Q'_c\}$.}

\noindent Then

$$\eqalignno{\#(S\cap[m_n]^k)
&=\#([m_n]^k\cap\{z[c]:c\in T,\,z\in Q'_c\})\cr
&=\#(\{z[c]:c\in T\cap[L]^k,\,z\in Q'_c\})\cr
\displaycause{because every member of $Q'_c$ is injective on $c$}
&\le\sum_{\Atop{c\in T}{\#(c)=k}}\Bover1{m_n^{r-k}}\#(Q'_c)\cr
\displaycause{by the last remark in (e)}
&\le\Bover1{m_n^{r-k}}\epsilon m_n^r
=\epsilon m_n^k\cr}$$

\noindent for every $k$.   So by the choice of $(m_n,T_n)$
there is an embedding $v$ of $(L,T)$ in
$(m_n,T_n\setminus S)$;
take $z=v\cup w$, so that $w\subseteq z\in m_n^M$ and
$z\restr L=v$ is injective and $z[c]\notin S$ for every $c\in T$.
However, there is some $c\in T$ such that
$\tilde z(T_n)\notin F'_c$.   As $c\in\tilde z(T_n)$,
$\tilde z(T_n)\in E'_c$.   But now
$z\in Q'_c$ and $z[c]\in S$.\ \Bang

This contradiction proves the theorem.
}%end of proof of 497J

\leader{497K}{Corollary: the Hypergraph Removal Lemma}
For every $\epsilon>0$ and $r\ge 1$
there is a $\delta>0$ such that whenever $I$ is a finite set,
$R\subseteq[I]^r$ and
$\#(\{J:J\in[I]^{r+1}$, $[J]^r\subseteq R\})\le\delta\#(I)^{r+1}$,
there is an $S\subseteq[I]^r$
such that $\#(S)\le\epsilon\#(I)^r$ and there is no
$J\in[I]^{r+1}$ such that $[J]^r\subseteq R\setminus S$.

\proof{ In 497J, take $L$ to be a set of size $r+1$, and set
$T=[L]^r$ in 497J.   Then there is a $\delta_0>0$ such that whenever
$I$ is a finite set, $R\subseteq[I]^r$ and the number of embeddings of
$(L,[L]^r)$ in $(I,R)$ is at most $\delta_0\#(I)^{r+1}$, there is an
$S\subseteq[I]^r$ such that $\#(S)\le\epsilon\#(I)^r$ and there
is no embedding of $(L,[L^r])$ in $(I,R\setminus S)$.   Try
$\delta=\Bover1{(r+1)!}\delta_0$.   If $I$ is finite, $R\subseteq[I]^r$
and $\Cal J=\{J:J\in[I]^{r+1}$, $[J]^r\subseteq R\}$ has at most
$\delta\#(I)^{r+1}$ members,
then an embedding of $(L,[L]^r)$ in $(I,R)$ is an injective function
$f:L\to I$ such that $f[J]\in R$ for every $J\in[L]^r$, that is,
$f[L]\in\Cal J$.   So the number of such embeddings is
$(r+1)!\#(\Cal J)\le\delta_0\#(I)^{r+1}$.   There is therefore an
$S\subseteq[I]^r$ such that $\#(S)\le\epsilon\#(I)^r$ and there is
no embedding of $(L,[L]^r)$ in $(I,R\setminus S)$, that is, there is no
$J\in[I]^{r+1}$ such that $[J]^r\subseteq R\setminus S$.
}%end of proof of 497K

\leader{497L}{Corollary:  Szemer\'edi's
Theorem}\cmmnt{ ({\smc Szemer\'edi 75})}
For every $\epsilon>0$ and $r\ge 2$ there is an $n_0\in\Bbb N$
such that whenever $n\ge n_0$, $A\subseteq n$ and $\#(A)\ge\epsilon n$
there is an arithmetic progression of length $r+1$ in $A$.

\proof{ ({\smc Frankl \& R\"odl 02})
Set $\eta=\Bover1{r!}(\Bover{\epsilon}{2r!})^r$.
Take $\delta>0$ such that
whenever $I$ is a finite set, $R\subseteq[I]^r$ and
$\#(\{J:J\in[I]^{r+1}$, $[J]^r\subseteq R\})\le\delta\#(I)^{r+1}$,
there is an $S\subseteq[I]^r$
such that $\#(S)\le\Bover{\eta}{2(r+1)^r}\#(I)^r$
and there is
no $J\in[I]^{r+1}$ such that $[J]^r\subseteq R\setminus S$.
Let $n_0$ be such that $\epsilon n\ge 2r\cdot r!$ and
$n(r+1)^{r+1}\delta\ge 1$
whenever $n\ge n_0$.   Take $n\ge n_0$ and
$A\subseteq n$ such that $\#(A)\ge\epsilon n$.

Let $C\subseteq n^r$ be the set

\Centerline{$\{(i_0,i_1,\ldots,i_{r-1}):\sum_{j=0}^{r-1}(j+1)i_j
\in A\}$.}

\noindent Then $\#(C)\ge\eta n^r$.\footnote{For the rest of this proof, and also in
497M and 497N below, I will use the formula $n^r$
both for the set of functions from $r=\{0,\ldots,r-1\}$ to
$n=\{0,\ldots,n-1\}$ and for its
cardinal interpreted as a real number;  I trust that this will not lead to
any confusion.}   \Prf\ For $m<r!$ set
$A_m=\{i:i\in A$, $i\equiv m\mod r!\}$.   Then there is an $m$ such that
$\#(A_m)\ge\Bover{\epsilon n}{r!}$.   Now we have an injection
$\phi:[A_m]^r\to C$ given by saying that if $l_0<\ldots<l_{r-1}$ in $A_m$
then

$$\eqalign{\phi(\{l_0,\ldots,l_{r-1}\})(j)
&=l_0\text{ if }j=0\cr
&=\Bover1{j+1}(l_j-l_{j-1})\text{ if }0<j<r.\cr}$$

\noindent So

\Centerline{$\#(C)\ge\#([A_m]^r)
\ge\Bover1{r!}(\Bover{\epsilon n}{r!}-r)^r
\ge\Bover1{r!}(\Bover{\epsilon n}{2r!})^r
=\eta n^r$.  \Qed}

Let $I$ be $n\times(r+1)$ and for $c=(i_0,\ldots,i_{r-1})\in C$ set

\Centerline{$J_c
=\{(i_j,j):j<r\}\cup\{(\sum_{j=0}^{r-1}i_j,r)\}\in[I]^{r+1}$.}

\noindent Observe that if $c$, $c'\in C$ are distinct, then
$[J_c]^r\cap[J_{c'}]^r=\emptyset$, since given any face of
the $r$-simplex $J_c$ we can read off all but at most
one of the coordinates of $c$ and calculate the last.
Set $R=\bigcup_{c\in C}[J_c]^r\subseteq[I]^r$.

\Quer\ Suppose, if possible, that the only $r$-simplices $J\in[I]^{r+1}$
such that $[J]^r\subseteq R$ are of the form $J_c$ for some $c\in C$.
Then there are at most

\Centerline{$\#(C)\le n^r\le n^r\cdot n(r+1)^{r+1}\delta
=\delta\#(I)^{r+1}$}

\noindent such simplices; by the choice of $\delta$, there is an
$S\subseteq[I]^r$ such that $R\setminus S$ covers no $r$-simplices and

\Centerline{$\#(S)\le\Bover{\eta}{2(r+1)^r}\#(I)^r
=\Bover{\eta}2n^r<\#(C)$.}

\noindent
But every $J_c$ must have a face in $S$, and no two $J_c$ share a face, so
this is impossible.\ \Bang

So we have an $r$-simplex $J\in[I]^{r+1}$, which is not of the form $J_c$
where $c\in C$, such that $[J]^r\subseteq R$.   Now since the only faces
put into $R$ come from the $J_c$, and therefore meet each of the $r+1$
levels $n\times\{k\}$ in at most one point, $J$ must be of the form
$\{(i_j,j):j<r\}\cup\{(l,r)\}$.   Since $\{(i_j,j):j<r\}$ is a face of some
$J_c$, $c=(i_0,\ldots,i_{r-1})\in C$.   Set $l'=i_0+\ldots+i_{r-1}$;  then
$l'\ne l$ because $J\ne J_c$.   For each $k<r$,
$J\setminus\{(i_k,k)\}$ is a face of $J$ and therefore of $J_{c'}$ for some
$c'\in C$;  now $J_{c'}$ must be

\Centerline{$(J\setminus\{(i_k,k)\})\cup\{(l-\sum_{j<r,j\ne k}i_j,k)\}
=(J\setminus\{(i_k,k)\})\cup\{(i_k+l-l',k)\}$}

\noindent and

\Centerline{$\sum_{j=0}^r(j+1)i_j+(k+1)(l-l')$}

\noindent belongs to $A$.   Since this is true for every $k<r$,
and we also have $\sum_{j=0}^r(j+1)i_j\in A$ because $c\in C$,
we have an arithmetic progression in $A$ of length $r+1$, as required.
}%end of proof of 497L

\leader{497M}{}\cmmnt{ For a full-strength version of the multiple
recurrence theorem it seems that the ideas described above are
inadequate;  for an adaptation which goes farther,
see {\smc Austin 10a} and
{\smc Austin 10b}.   However the methods here can reach the following.

\medskip

\noindent}{\bf Lemma} (cf.\ {\smc Solymosi 03})
Suppose that $r\ge 1$ and $n\in\Bbb N$.   For $0\le j$,
$k<r$ set $e_j(k)=1$ if $k=j$, $0$ otherwise.   For
$z\in n^r$ and $C\subseteq n^r$ write

\Centerline{$\Delta(z,C)
=\{k:k\in\Bbb Z$, $z+ke_i\in C$ for every $i<r\}$,
\quad$q(z,C)=\#(\Delta(z,C))$.}

\noindent Then for every $\epsilon>0$ there is a $\delta>0$ such that
$\#(\{z:z\in n^r$, $q(z,C)\ge\delta n\})\ge\delta n^r$
whenever $n\in\Bbb N$, $C\subseteq n^r$ and $\#(C)\ge\epsilon n^r$.

\proof{ We can use some of the same ideas as in 497L.
Let $\epsilon'>0$ be such that $2^rr^r\epsilon'<\epsilon$.
Let $\delta>0$ be such that whenever $I$ is
finite, $R\subseteq[I]^r$ and
$\#(\{J:J\in[I]^{r+1}$, $[J]^r\subseteq R\})\le\delta\#(I)^{r+1}$
there is an $S\subseteq R$ such that
$\#(S)\le\epsilon'\#(I)^r$ and there is no $J\in[I]^{r+1}$ such
that $[J]^r\subseteq R\setminus S$ (497K).

Take $n\in\Bbb N$ and $C\subseteq n^r$ such that $\#(C)\ge\epsilon n^r$.
Set $I=(n\times r)\cup(nr\times\{r\})$, so that $\#(I)=2nr$.
For $c\in C$ set

\Centerline{$J_c
=\{(c(i),i):i<r\}\cup\{(\sum_{i=0}^{r-1}c(i),r)\}\in[I]^{r+1}$;}

\noindent set $R=\bigcup_{c\in C}[J_c]^r$.   Observe that if $c$, $c'\in C$
are distinct then $[J_c]^r$ and $[J_{c'}]^r$ are disjoint.
If $S\subseteq R$ and $\#(S)\le\epsilon'\#(I)^r$, then $\#(S)<\epsilon n^r$
and there must be a $c\in C$ such that $[J_c]^r\cap S=\emptyset$ and
$[J_c]^r\subseteq R\setminus S$.  Consequently

\Centerline{$\Cal K=\{K:K\in[I]^{r+1}$, $[K]^r\subseteq R\}$}

\noindent must have more than
$\delta\#(I)^{r+1}\ge 2\delta n^{r+1}$ members, by the choice of $\delta$.

Next, $\#(\Cal K)=\sum_{z\in n^r}q(z,C)$.   \Prf\
Set $B=\{(z,k):z\in n^r$, $k\in\Bbb Z$, $z+ke_i\in C$ for every $i<r\}$;
then $\#(B)=\sum_{z\in n^r}q(z,C)$.
For any $K\in\Cal K$, there must be a $c_K\in C$ such that
$(c_K(i),i)\in K$ for every $i<r$ while
$(k_K+\sum_{i=0}^{r-1}c_K(i),r)\in K$ for
some $k_K$;  in this case, $c_K+k_Ke_i\in C$ for every $i<r$ and
$(c_K,k_K)\in B$.   Conversely, starting from $(z,k)\in B$,
$\{(z(i),i):i<r\}\cup\{(k+\sum_{i=0}^{r-1}z(i),r)\}$ belongs to $K$.
So $K\mapsto(c_K,k_K)$ is a bijection from $\Cal K$ to $B$ and
$\#(\Cal K)=\#(B)$.\ \Qed

Thus $\sum_{z\in n^r}q(z,C)\ge 2\delta n^{r+1}$.
Of course

\Centerline{$q(z,C)\le\#(\{k:z+ke_0\in n^r\}\le n$}

\noindent for every $z\in n^r$.   So setting
$D=\{z:z\in n^r$, $q(z,C)\ge\delta n\}$, we have

\Centerline{$2\delta n^{r+1}\le n\#(D)+\delta n\cdot n^r
\le n\#(D)+\delta n^{r+1}$}

\noindent and $\#(D)\ge\delta n^r$, as claimed.
}%end of proof of 497M

\leader{497N}{Theorem}\cmmnt{ ({\smc Furstenburg 81})} %7.13
Let $(\frak A,\bar\mu)$ be a probability algebra and
$\ofamily{i}{r}{\pi_i}$
a non-empty finite commuting family of measure-preserving
Boolean homomorphisms from $\frak A$ to itself.
If $a\in\frak A\setminus\{0\}$, there is an $\eta>0$ such that

\Centerline{$\sum_{k=0}^{n-1}\bar\mu(\inf_{i<r}\pi_i^ka)\ge\eta n$}

\noindent for every $n\in\Bbb N$.

\proof{{\bf (a)} To being with, suppose that every $\pi_i$ is an
automorphism.
Doubling $r$ if necessary, we can suppose that for every $i<r$
there is a $j<r$ such that $\pi_j=\pi_i^{-1}$.   Let $(Z,\Sigma,\mu)$ be
the Stone space of $(\frak A,\bar\mu)$ (321K), and set $E=\widehat{a}$, the
open-and-closed subset of $Z$ corresponding to $a\in\frak A$.
For each $i<r$ let
$T_i:Z\to Z$ be the homeomorphism corresponding to
$\pi_i:\frak A\to\frak A$, so that $T_i^{-1}[\widehat{b}]=\widehat{\pi_ib}$
for every $b\in\frak A$ (312Q\formerly{3{}12P});  note that
$T_iT_j$ corresponds to $\pi_j\pi_i$ (312R\formerly{3{}12Q}) and
$T_iT_j=T_jT_i$ (because the representations in 312Q are unique), for all
$i$, $j<r$.

In 497M, set $\epsilon=\bover12\mu E=\bover12\bar\mu a$ and take a
corresponding $\delta>0$;  set $\eta=\bover12\delta^2\epsilon$.
Now, given $n\ge 1$, then for $z\in n^r$ set
$\tilde T_z=\prod_{i<r}T_i^{z(i)}$.   (We can speak of the product
without inhibitions because the $T_i$ commute.)   Consider the set
$W=\{(x,z):z\in n^r$, $\tilde T_z(x)\in E\}$.
Then $W^{-1}[\{z\}]$ has measure $\mu E$ for
every $z$, so if we set $F=\{x:x\in E$, $\#(W[\{x\}])\ge\epsilon n^r\}$
we have

\Centerline{$n^r\mu E\le n^r\mu F+\epsilon n^r$,
\quad$\mu F\ge\epsilon$.}

In the notation of 497M, set

\Centerline{$V=\{(x,z):(x,z)\in W$, $q(z,W[\{x\}])\ge\delta n\}$;}

\noindent then for any $x\in F$ we have $\#(V[\{x\}])\ge\delta n^r$, by
the choice of $\delta$.   There must therefore be a $z\in n^r$ such that
$\mu V^{-1}[\{z\}]\ge\delta\mu F\ge\delta\epsilon$.
Take any $x\in V^{-1}[\{z\}]$ and $k\in\Delta(z,W[\{x\}])$.
Setting $e_i(i)=1$ and $e_i(j)=0$ for $i<r$ and $j\in r\setminus\{i\}$,
$z+ke_i\in W[\{x\}]$, that is, $T_i^k\tilde T_z(x)\in E$, for every
$i<r$.   Also $|k|<n$.   Set $G=\tilde T_z[V^{-1}[\{z\}]]$, so that
$\mu G\ge\delta\epsilon$ and for every $y\in G$ we have
$\#(\{k:|k|<n$, $T_i^k(y)\in E$ for every $i<r\})\ge\delta n$.   But
this means that

$$\eqalignno{\sum_{k=0}^{n-1}\bar\mu(\inf_{i<r}\pi_i^ka)
&=\sum_{k=0}^{n-1}\mu\{x:T_i^k(x)\in E\text{ for every }i<r\}\cr
&\ge\Bover12\sum_{|k|<n}\mu\{x:T_i^k(x)\in E\text{ for every }i<r\}\cr
\displaycause{because if $T_i^k(x)\in E$ for every $i<r$ then
$T_i^{|k|}(x)\in E$ for every $i<r$}
&=\Bover12\int\#(\{k:|k|<n,\,T_i^k(x)\in E
  \text{ for every }i<r\})\mu(dx)\cr
&\ge\Bover12\delta n\mu G
\ge\Bover12\delta^2\epsilon n
=\eta n,\cr}$$

\noindent as required.

\medskip

{\bf (b)} For the general case, 328J tells us that there are
a probability algebra $(\frak C,\bar\lambda)$, a measure-preserving Boolean
homomorphism $\pi:\frak A\to\frak C$ and a commuting family
$\ofamily{i}{r}{\tilde\pi_i}$ of measure-preserving automorphisms of
$\frak C$ such that $\tilde\pi_i\pi=\pi\pi_i$ for every $i<r$.
Now $\pi a\in\frak C\setminus\{0\}$, so there is an $\eta>0$ such that

\Centerline{$\eta n
\le\sum_{k=0}^{n-1}\bar\lambda(\inf_{i<r}\tilde\pi_i^k\pi a)
=\sum_{k=0}^{n-1}\bar\lambda(\inf_{i<r}\pi\pi_i^ka)
=\sum_{k=0}^{n-1}\bar\mu(\inf_{i<r}\pi_i^ka)$}

\noindent for every $n\in\Bbb N$.
}%end of proof of 497N

\exercises{\leader{497X}{Basic exercises (a)}
%\spheader 497Xa
Let $(X,\Sigma,\mu)$ be a probability space,
$\familyiI{\Sigma_i}$ an independent family of $\sigma$-subalgebras of
$\Sigma$, and $\Tau$ a subalgebra of $\Sigma$ such that $\Tau\cap\Sigma_i$
is metrically dense in $\Sigma_i$ for every $i\in I$.   For $J\subseteq I$
set $\tilde\Sigma_J=\bigvee_{i\in J}\Sigma_i$.   Show that
$\langle\tilde\Sigma_J\rangle_{J\subseteq I}$ has $\Tau$-removable
intersections.
%497E

\spheader 497Xb Let $I$ be a set, $G_I$ the
group of permutations of $I$ with its topology of pointwise convergence
(441G, 449Xh), and $\action$ the action of $G_I$ on
$\Cal P([I]^{<\omega})$ described in 497F.   Show that $\action$ is
continuous.
%497F

\spheader 497Xc In 497F, show that
$\{\mu:\mu\in P_I$ is permutation-invariant$\}$ is a
closed subset of $P_I$.
%497F

\leader{497Y}{Further exercises (a)}
%\spheader 497Ya
(i) Show that if $A\subseteq\Bbb N$ has non-zero upper
asymptotic density then there is a translation-invariant additive
functional $\nu:\Cal P\Bbb Z\to[0,1]$ such that $\nu A>0$.
(ii) Consider the statement

\inset{($\dagger$) If $\epsilon>0$ and $A\subseteq\Bbb N$ are such that
$\#(A\cap n)\ge\epsilon n$ for every $n$ then $A$ includes arithmetic
progressions of all finite lengths.}

\noindent Use Theorem 497N to prove ($\dagger$).
(iii) Find a direct proof that ($\dagger$) implies Szemer\'edi's
theorem.
}%end of exercises

\endnotes{\Notesheader{497} I am grateful to T.D.Austin for introducing me
to a preprint of \TaoIG, on which this section is based.

Regarded as a proof of Szemer\'edi's theorem, the argument
above has the virtues of reasonable brevity and
(I hope) of completeness and correctness.   It depends, of course, on
non-trivial ideas from measure theory, which for anyone except a measure
theorist will compromise the claim of `brevity';  and even measure
theorists may find that the proofs here demand close attention.
There are further,
more significant, defects.   The outstanding problem associated with
Szemer\'edi's theorem is the estimation of $n_0$ as a function of $r$ and
$\epsilon$;  and while in a theoretical sense it must be possible to trace
through the arguments above to establish rigorous bounds, the methods are
not well adapted to such an exercise, and one would not expect the bounds
obtained to be good.
There is also the point that I have made uninhibited use of the axiom of
choice.   The ultrafilter in 497J can easily be replaced by an appropriate
sequence, but all standard treatments of measure theory assume at least
the countable axiom of choice, and Szemer\'edi's theorem is clearly
true in significantly weaker theories than ordinary ZF.

The first `measure-theoretic' proof of Szemer\'edi's theorem was due to
{\smc Furstenburg 77}, and relied on a deep analysis of the structure of
measure-preserving transformations.   While the methods described here do
not seem to give us any information on this structure, it is apparently a
folklore result that the hypergraph removal lemma provides a quick
proof of the basic theorem used in Furstenburg's approach (497N, 497Ya).

The value of the work here, therefore, lies less in its applications to the
hypergraph removal lemma and Szemer\'edi's theorem, than in the idea of
`removable intersections', where Theorems 497E and 497G give us two
remarkable results, and useful exercises in the theory of relative
independence from \S458.
We also have an instructive example of a more general
phenomenon.   Given a sequence of finite objects with quantititative
aspects, it is often profitable to seek a measure $\mu$
reflecting the asymptotic behaviour of this sequence;  this is the idea of
the construction in 497H.   The `quantitative aspects' here, as
developed in 497J, are the proportion of functions from $L$ to $m_n$ which
are embeddings of $(L,T)$ in $(m_n,T_n)$, and the proportion of simplices
in $[m_n]^k$ which must be removed from $T_n$
in order to destroy all these embeddings.
The measure $\mu$ is set up to describe the limits of
these proportions as measures of appropriate sets.

Returning to the definition 497Aa, most of its clauses
can be expressed in terms of the
measure algebra of the measure $\mu$;  but the final
`$\bigcap_{i\in J}F_i=\emptyset$' has to be taken literally, and makes
sense only in terms of the measure space itself.   In the key application
(part (d) of the proof of 497J), the original sets $E_c$, with negligible
intersection, already belong to the algebra $\Tau$, but the adjustment
to sets $F_c$ with empty intersection is still non-trivial, because of the
requirement that each $F_c$ must belong to the prescribed $\sigma$-algebra
$\Sigma_{J_c}$.

I said in 497F that you could note that $\Cal P([\Bbb N]^{<\omega})$ is
homeomorphic to the Cantor set, so that $P_{\Bbb N}$ is isomorphic
to the space of Radon probability measures on $\{0,1\}^{\Bbb N}$.
However the point of the construction there
is that we are looking at a particular action of the symmetric group
$G_{\Bbb N}$ on $\Cal P([\Bbb N]^{<\omega})$;  and this has very little
to do with the natural actions of $G_{\Bbb N}$ on $\Cal P\Bbb N$ or
$\{0,1\}^{\Bbb N}$, as studied in 459E and 459H, for instance.
In particular, permutation-invariant measures, in the
sense of 497Fb, will not normally be invariant under the much larger group
derived from all permutations of $[\Bbb N]^{<\omega}$ rather than
just those corresponding to permutations of $\Bbb N$.

I express 497N in terms of measure-preserving
automorphisms of probability algebras in order to connect it with the
treatment of ergodic theory in Chapter 38, but you will observe that the
proof presented
immediately shifts to a more traditional formulation in terms of
probability spaces.   This is only one of many
multiple recurrence theorems, some of them much
stronger (and, it seems, deeper) than 497N or, indeed, 497J.
}%end of notes

\discrpage
