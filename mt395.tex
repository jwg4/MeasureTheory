\frfilename{mt395.tex}
\versiondate{15.6.08}
\copyrightdate{1996}
\def\Gte{$G$-\vthsp$\tau$-\vthsp equidecomposable}
\def\IMPLY#1#2{{\bf (#1)$\Rightarrow$(#2)}}
\def\ptG{\preccurlyeq^{\tau}_G}

\def\chaptername{Measurable algebras}
\def\sectionname{Kawada's theorem}

\newsection{395\dvAformerly{\S3{}94}}

I now describe a completely different characterization
of (homogeneous) measurable algebras, based on the special nature of
their automorphism groups.   The argument depends on the notion of
`non-paradoxical' group of automorphisms;  this is an idea of great
importance in other contexts, and I therefore aim at a fairly thorough
development, with proofs which are adaptable to other circumstances.

\leader{395A}{Definitions} Let $\frak A$ be a Dedekind complete Boolean
algebra, and $G$ a subgroup of $\Aut\frak A$.
For $a$, $b\in\frak A$ I will say that an isomorphism
$\phi:\frak A_a\to\frak A_b$ between the corresponding principal ideals
belongs to
the {\bf full local semigroup generated by $G$} if there are a partition
of unity $\langle a_i\rangle_{i\in I}$ in $\frak A_a$ and a family
$\langle \pi_i\rangle_{i\in I}$ in $G$ such that $\phi c=\pi_ic$
whenever $i\in I$ and $c\Bsubseteq a_i$.   If such an isomorphism exists
I will say that $a$ and $b$ are {\bf $G$-$\tau$-equidecomposable}.

I will write $a\ptG b$ to mean that there is a $b'\Bsubseteq b$ such
that $a$ and $b'$ are \Gte.

For any function $f$ with domain $\frak A$, I will say that $f$ is {\bf
$G$-invariant} if $f(\pi a)=f(a)$ whenever $a\in\frak A$ and $\pi\in G$.

\leader{395B}{}\cmmnt{ The notion of `full local semigroup' is of
course an extension of the idea of `full subgroup' (381Be;  see also
381Yb).   The
word `semigroup' is justified by (c) of the following lemma, and the
word `full' by (e).

\medskip

\noindent}{\bf Lemma} Let $\frak A$ be a Dedekind complete Boolean
algebra and $G$ a subgroup of $\Aut\frak A$.   Write $G^*_{\tau}$ for
the full local semigroup generated by $G$.

(a) Suppose that $a$, $b\in\frak A$ and that
$\phi:\frak A_a\to\frak A_b$ is an isomorphism.   Then the following are
equiveridical:

\quad (i) $\phi\in G^*_{\tau}$;

\quad (ii) for every non-zero
$c_0\Bsubseteq a$ there are a non-zero $c_1\Bsubseteq c_0$ and a
$\pi\in G$ such that $\phi c=\pi c$ for every $c\Bsubseteq c_1$;

\quad (iii) for every non-zero
$c_0\Bsubseteq a$ there are a non-zero $c_1\Bsubseteq c_0$ and a
$\psi\in G^*_{\tau}$ such that $\phi c=\psi c$ for every
$c\Bsubseteq c_1$.

(b) If $a$, $b\in\frak A$ and $\phi:\frak A_a\to\frak A_b$ belongs to
$G^*_{\tau}$, then $\phi^{-1}:\frak A_b\to\frak A_a$ also belongs to
$G^*_{\tau}$.

(c) Suppose that $a$, $b$, $a'$, $b'\in\frak A$ and that $\phi:\frak
A_a\to\frak A_{a'}$, $\psi:\frak A_{b}\to\frak A_{b'}$ belong to
$G^*_{\tau}$.
Then $\psi\phi\in G^*_{\tau}$;
its domain is $\frak A_c$ where $c=\phi^{-1}(b\Bcap a')$, and its
set of values is $\frak A_{c'}$ where $c'=\psi(b\Bcap a')$.

(d) If $a$, $b\in\frak A$ and $\phi:\frak A_a\to\frak A_b$ belongs to
$G^*_{\tau}$, then $\phi\restrp\frak A_c\in G^*_{\tau}$ for any $c\Bsubseteq a$.

(e) Suppose that $a$, $b\in\frak A$ and that
$\psi:\frak A_a\to\frak A_b$ is an isomorphism such that there are a partition of unity $\langle a_i\rangle_{i\in I}$ in $\frak A_a$ and a family $\langle\phi_i\rangle_{i\in I}$ in $G^*_{\tau}$ such that
$\psi c=\phi_ic$ whenever $i\in I$ and $c\Bsubseteq a_i$.   Then
$\psi\in G^*_{\tau}$.

\proof{{\bf (a)} (Compare 381I.)

\medskip

\quad{\bf (i)$\Rightarrow$(iii)} is trivial, since of course
$G\subseteq G^*_{\tau}$.

\medskip

\quad{\bf (iii)$\Rightarrow$(ii)} Suppose that $\phi$ satisfies (iii),
and that $0\ne c_0\Bsubseteq a$.   Then we can find a $\psi\in
G^*_{\tau}$ and a non-zero $c_1\Bsubseteq c_0$ such that $\phi$ agrees
with $\psi$ on $\frak A_{c_1}$.   Suppose that $\dom\psi=\frak A_d$,
where necessarily $d\Bsupseteq c_1$.   Then there are a partition of
unity $\langle d_i\rangle_{i\in I}$ in $\frak A_d$ and a family
$\langle\pi_i\rangle_{i\in I}$ such that $\psi c=\pi_ic$ whenever
$c\Bsubseteq d_i$.   There is some $i\in I$ such that
$c_2=c_1\Bcap d_i\ne 0$, and we see that $\phi c=\psi c=\pi_ic$ for
every $c\Bsubseteq c_2$.   As $c_0$ is arbitrary, $\phi$ satisfies (ii).

\medskip

\quad{\bf (ii)$\Rightarrow$(i)} If $\phi$ satisfies (ii), set

\Centerline{$D=\{d:d\Bsubseteq a$, there is some $\pi\in G$ such that
$\pi c=\phi c$ for every $c\Bsubseteq d\}$.}

\noindent The hypothesis is that $D$ is order-dense in $\frak A$, so
there is a partition of unity $\langle a_i\rangle_{i\in I}$ of
$\frak A_a$ lying within $D$ (313K);  for each $i\in I$ take
$\pi_i\in G$ such that $\phi c=\pi_ic$ for $c\Bsubseteq a_i$;  then
$\langle a_i\rangle_{i\in I}$ and $\langle\pi_i\rangle_{i\in I}$ witness
that $\phi\in G^*_{\tau}$.

\medskip

{\bf (b)} This is elementary;  if $\langle a_i\rangle_{i\in I}$,
$\langle\pi_i\rangle_{i\in I}$ witness that $\phi\in G^*_{\tau}$, then
$\langle\phi a_i\rangle_{i\in I}=\langle\pi_ia_i\rangle_{i\in I}$,
$\langle\pi_i^{-1}\rangle_{i\in I}$ witness that
$\phi^{-1}\in G^*_{\tau}$.

\medskip

{\bf (c)} I ought to start by computing the domain of $\psi\phi$:

$$\eqalign{d\in\dom(\psi\phi)
&\iff d\in\dom\phi,\,\phi d\in\dom\psi\cr
&\iff d\Bsubseteq a,\,\phi d\Bsubseteq b
\iff d\Bsubseteq\phi^{-1}(a'\Bcap b)=c.\cr}$$

\noindent So the domain of $\psi\phi$ is indeed $\frak A_c$;  now
$\phi\restrp\frak A_c$ is an isomorphism between $\frak A_c$ and $\frak
A_{\phi c}$, where $\phi c=a'\Bcap b\in\frak A_b$, so $\psi\phi$ is an
isomorphism between $\frak A_c$ and $\frak A_{\psi\phi c}=\frak A_{c'}$.
Let $\langle a_i\rangle_{i\in I}$, $\langle b_j\rangle_{j\in J}$ be
partitions of unity in $\frak A_a$, $\frak A_b$ respectively, and
$\langle\pi_i\rangle_{i\in I}$, $\langle\theta_j\rangle_{j\in J}$
families in $G$ such that $\phi d=\pi_id$ for $d\Bsubseteq a_i$,
$\psi e=\theta_je$ for $e\Bsubseteq b_j$.   Set
$c_{ij}=a_i\Bcap\pi_i^{-1}b_j$;  then
$\langle c_{ij}\rangle_{i\in I,j\in J}$ is a partition of
unity in $\frak A_c$ and $\psi\phi d=\theta_j\pi_id$ for
$d\Bsubseteq c_{ij}$, so $\psi\phi\in G^*_{\tau}$ (because all the
$\theta_j\pi_i$ belong to $G$).

\medskip

{\bf (d)} This is nearly trivial;  use the definition of $G^*_{\tau}$ or
the criteria of (a), or apply (c) with the identity map on $\frak A_c$
as one of the factors.

\medskip

{\bf (e)} This follows at once from the criterion (a-iii) above, or
otherwise.
}%end of proof of 395B

\leader{395C}{Lemma} Let $\frak A$ be a Dedekind complete Boolean
algebra and $G$ a subgroup of $\Aut\frak A$.   Write $G^*_{\tau}$ for
the full local semigroup generated by $G$.

(a) For $a$, $b\in\frak A$, $a\ptG b$ iff there is a
$\phi\in G^*_{\tau}$ such that $a\in\dom\phi$ and $\phi a\Bsubseteq b$.

(b)(i) $\ptG$ is transitive and reflexive;

\quad(ii) if $a\ptG b$ and $b\ptG a$ then $a$ and $b$ are \Gte.

(c) $G$-$\tau$-equidecomposability is an equivalence relation on
$\frak A$.

(d) If $\langle a_i\rangle_{i\in I}$ and $\langle b_i\rangle_{i\in I}$
are families in $\frak A$, of which $\langle b_i\rangle_{i\in I}$ is
disjoint, and $a_i\ptG b_i$ for every $i\in I$, then
$\sup_{i\in I}a_i\ptG\sup_{i\in I}b_i$.

\proof{{\bf (a)} This is immediate from the definition of `\Gte' and
395Bd.

\medskip

{\bf (b)(i)} $a\ptG a$ because the identity homomorphism belongs to
$G^*_{\tau}$.   If $a\ptG b\ptG c$ there are $\phi$, $\psi\in
G^*_{\tau}$ such that $\phi a\Bsubseteq b$, $\psi b\Bsubseteq c$ so that
$\psi\phi a\Bsubseteq c$;  as $\psi\phi\in G^*_{\tau}$ (395Bc),
$a\ptG c$.

\medskip

\quad{\bf (ii)} (This is of course a Schr\"oder-Bernstein theorem, and
the proof is the usual one.) Take $\phi$, $\psi\in G^*_{\tau}$ such that
$\phi a\Bsubseteq b$, $\psi b\Bsubseteq a$.   Set $a_0=a$, $b_0=b$,
$a_{n+1}=\psi b_n$ and $b_{n+1}=\phi a_n$ for each $n$.   Then
$\sequencen{a_n}$, $\sequencen{b_n}$ are non-increasing sequences;  set
$a_{\infty}=\inf_{n\in\Bbb N}a_n$, $b_{\infty}=\inf_{n\in\Bbb N}b_n$.
For each $n$,

\Centerline{$\phi\restrp\frak A_{a_{2n}\Bsetminus a_{2n+1}}:
\frak A_{a_{2n}\Bsetminus a_{2n+1}}\to\frak A_{b_{2n+1}\Bsetminus
b_{2n+2}}$,}

\Centerline{$\psi\restrp\frak A_{b_{2n}\Bsetminus b_{2n+1}}:
\frak A_{b_{2n}\Bsetminus b_{2n+1}}\to\frak A_{a_{2n+1}\Bsetminus
a_{2n+2}}$}

\noindent are isomorphisms, while

\Centerline{$\phi\restrp\frak A_{a_{\infty}}:\frak A_{a_{\infty}}
\to\frak A_{b_{\infty}}$}

\noindent is another.   So we can define an isomorphism $\theta:\frak
A_a\to\frak A_b$ by setting

$$\eqalign{\theta c&=\phi c\text{ if }c\Bsubseteq
a_{\infty}\Bcup\sup_{n\in\Bbb N}a_{2n}\Bsetminus a_{2n+1},\cr
&=\psi^{-1}c\text{ if }c\Bsubseteq
  \sup_{n\in\Bbb N}a_{2n+1}\Bsetminus a_{2n+2}.\cr}$$

\noindent By 395Be, $\theta\in G^*_{\tau}$, so $a$ and $b$ are \Gte.

\medskip

{\bf (c)} This is easy to prove directly from the results in 395B, but
also follows at once from (b);  any transitive reflexive relation
gives rise to an equivalence relation.

\medskip

{\bf (d)} We may suppose that $I$ is well-ordered by a relation $\le$.
For $i\in I$, set $a'_i=a_i\Bsetminus\sup_{j<i}a_j$.   Set $a=\sup_{i\in
I}a_i=\sup_{i\in I}a'_i$, $b=\sup_{i\in I}b_i$.   For each $i\in I$, we
have a $b'_i\Bsubseteq b_i$ and a $\phi_i\in G^*_{\tau}$ such that
$\phi_ia'_i=b'_i$.   Set $b'=\sup_{i\in I}b'_i\Bsubseteq b$;  then we
have an isomorphism $\psi:\frak A_a\to\frak A_{b'}$ defined by setting
$\psi d=\phi_id$ if $d\Bsubseteq a'_i$, and $\psi\in G^*_{\tau}$, so $a$
and $b'$ are \Gte\ and $a\ptG b$.
}%end of proof of 395C

\leader{395D}{Theorem} Let $\frak A$ be a Dedekind complete Boolean
algebra and $G$ a subgroup of $\Aut\frak A$.   Then the following are
equiveridical:

(i) there is an $a\ne 1$ such that $a$ is $G$-$\tau$-equidecomposable
with $1$;

(ii) there is a disjoint sequence $\sequencen{a_n}$ of non-zero elements
of $\frak A$ which are all $G$-$\tau$-equidecomposable;

(iii) there are non-zero \Gte\ $a$, $b$, $c\in\frak A$ such that $a\Bcap
b=0$ and $a\Bcup b\Bsubseteq c$;

(iv) there are $G$-$\tau$-equidecomposable $a$, $b\in\frak A$ such that
$a\Bsubset b$.

\proof{ Write $G^*_{\tau}$ for the full local semigroup generated by
$G$.

\medskip

\IMPLY{i}{ii} Assume (i).   There is a $\phi\in G^*_{\tau}$ such that
$\phi 1=a$.   Set $a_n=\phi^n(1\Bsetminus a)$ for each $n\in\Bbb N$;
because every $\phi^n$ belongs to $G^*_{\tau}$ (counting $\phi^0$ as the
identity operator on $\frak A$, and using 395Bc), with
$\dom\phi^n=\frak A$, $a_n$ is \Gte\ with $a_0=1\Bsetminus a$ for every
$n$.   Also $a_n=\phi^n1\Bsetminus\phi^{n+1}1$ for each $n$, while
$\sequencen{\phi^n1}$ is non-increasing, so $\sequencen{a_n}$ is
disjoint.   Thus (ii) is true.

\medskip

\IMPLY{ii}{iii} Assume (ii).   Set $a=\sup_{n\in\Bbb N}a_{2n}$,
$b=\sup_{n\in\Bbb N}a_{2n+1}$, $c=\sup_{n\in\Bbb N}a_n$, so that
$a\Bcap b=0$ and $a\Bcup b=c$.   For each $n$ we have a
$\phi_n\in G^*_{\tau}$ such that $\phi_na_0=a_n$.   So if we set

\Centerline{$\psi d
=\sup_{n\in\Bbb N}\phi_n\phi_{2n}^{-1}(d\Bcap a_{2n})$ for
$d\Bsubseteq a$,}

\noindent $\psi$ belongs to $G^*_{\tau}$ (using 395B) and witnesses that
$a$ and $c$ are \Gte.   Similarly, $b$ and $c$ are \Gte, so (iii) is
true.

\medskip

\IMPLY{iii}{iv} is trivial.

\medskip

\IMPLY{iv}{i} Take $\phi\in G^*_{\tau}$ such that $\phi b=a$.   Set

\Centerline{$\psi d=\phi(d\Bcap b)\Bcup(d\Bsetminus b)$}

\noindent for $d\in\frak A$;  then $\psi\in G^*_{\tau}$ witnesses
that $1$ is \Gte\ with $a\Bcup(1\Bsetminus b)\ne 1$.
}%end of proof of 395D

\leader{395E}{Definition} Let $\frak A$ be a Dedekind complete Boolean
algebra and $G$ a subgroup of $\Aut\frak A$.   I will say that $G$ is
{\bf fully non-paradoxical} if the statements of 395D are false;  that
is, if one of the following equiveridical statements is true:

\inset{(i)
if $a$ is $G$-$\tau$-equidecomposable with $1$ then $a=1$;}

\inset{(ii) there is no disjoint sequence $\sequencen{a_n}$ of
non-zero elements of $\frak A$ which are all \Gte;}

\inset{(iii) there are no non-zero \Gte\ $a$, $b$, $c\in\frak A$ such
that $a\Bcap b=0$ and $a\Bcup b\Bsubseteq c$;}

\inset{(iv) if $a\Bsubseteq b\in\frak A$ and $a$, $b$ are \Gte\ then
$a=b$.}

%add: there are $a\in\frak A$, $\phi$ in the full subgroup generated
%by $\pi$ which is not recurrent on $a$.

\noindent Note that if $G$ is fully non-paradoxical, and $H$ is a
subgroup of $G$, then $H$ also is
fully non-paradoxical\prooflet{, because if $a\preccurlyeq^{\tau}_Hb$ then $a\ptG b$, so that $a$ and $b$ are \Gte\ whenever they are
$H$-\vthsp$\tau$-\vthsp equidecomposable}.

\leader{395F}{Proposition} Let $(\frak A,\bar\mu)$ be a totally finite
measure algebra, and $G=\AmuA$ the group of all
measure-preserving automorphisms of $\frak A$.   Then $G$ is fully
non-paradoxical.

\proof{ If $\phi:\frak A\to\frak A_a$ belongs to the full local semigroup
generated by $G$, then we have a partition of unity $\langle
a_i\rangle_{i\in I}$ and a family $\langle\pi_i\rangle_{i\in I}$ in $G$
such that $\phi a_i=\pi_ia_i$ for every $i$;  but this means that

\Centerline{$\bar\mu a=\sum_{i\in I}\bar\mu\phi_ia_i=\sum_{i\in
I}\bar\mu\pi_ia_i
=\sum_{i\in I}\bar\mu a_i=\bar\mu 1$.}

\noindent As $\bar\mu 1<\infty$, we can conclude that $a=1$, so that $G$
satisfies the condition (i) of 395E.
}%end of proof of 395F

\leader{395G}{The fixed-point subalgebra of a group} Let $\frak A$ be a
Boolean algebra and $G$ a subgroup of $\Aut\frak A$.

\spheader 395Ga By the {\bf fixed-point subalgebra} of $G$ I mean

\Centerline{$\frak C=\{c:c\in\frak A,\,\pi c=c$ for every $\pi\in G\}$.}

\noindent\cmmnt{(I looked briefly at this construction in 333R, and
in the special case of a group generated by a single
element it appeared at various points in Chapter 38.)}
This is a subalgebra of $\frak A$, and is
order-closed\cmmnt{, because every $\pi\in G$ is order-continuous}.

\spheader 395Gb Now suppose that $\frak A$ is Dedekind complete.   In
this case $\frak C$ is Dedekind complete\cmmnt{ (314Ea)}, and we have,
for any $a\in\frak A$, an upper envelope $\upr(a,\frak C)$ of $\frak C$,
defined by setting

\Centerline{$\upr(a,\frak C)
=\inf\{c:a\Bsubseteq c\in\frak C\}$\dvro{.}{}}

\noindent\cmmnt{(313S).   }Now
$\upr(a,\frak C)=\sup\{\pi a:\pi\in G\}$.
\prooflet{\Prf\ Set $c_1=\upr(a,\frak C)$, $c_2=\sup\{\pi a:\pi\in G\}$.
(i) Because $a\Bsubseteq c_1\in\frak C$, $\pi a\Bsubseteq \pi c_1=c_1$
for every $\pi\in G$, and $c_2\Bsubseteq c_1$.   (ii) For any
$\phi\in G$,

\Centerline{$\phi c_2=\sup_{\pi\in G}\phi\pi a
=\sup_{\pi\in G}\pi a=c_2$}

\noindent because $G=\{\phi\pi:\pi\in G\}$.   So $c_2\in\frak C$;  since
also $a\Bsubseteq c_2$, $c_1\Bsubseteq c_2$, and $c_1=c_2$, as
claimed.\ \Qed}

\spheader 395Gc Again supposing that $\frak A$ is Dedekind complete,
write $G^*_{\tau}$ for the full local semigroup generated by
$G$.   Then $\phi(a\Bcap c)=\phi a\Bcap c$ whenever $\phi\in
G^*_{\tau}$, $a\in\dom\phi$ and $c\in\frak C$.
\prooflet{\Prf\ We have $\phi a=\sup_{i\in I}\pi_ia_i$, where
$a=\sup_{i\in I}a_i$ and $\pi_i\in G$ for every $i$.   Now

\Centerline{$\phi(a\Bcap c)=\sup_{i\in I}\pi_i(a_i\Bcap c)
=\sup_{i\in I}\pi_ia_i\Bcap c=\phi a\Bcap c$.  \Qed}}

\cmmnt{\noindent Consequently} $\upr(\phi a,\frak C)=\upr(a,\frak C)$
whenever $\phi\in G^*_{\tau}$ and $a\in\dom\phi$.   \prooflet{\Prf\ For
$c\in\frak C$,

\Centerline{$a\Bsubseteq c
\iff a\Bcap c=a
\iff\phi(a\Bcap c)=\phi a
\iff\phi a\Bcap c=\phi a
\iff\phi a\Bsubseteq c$.  \Qed}}

\cmmnt{\noindent It follows
that} $\upr(a,\frak C)\Bsubseteq\upr(b,\frak C)$ whenever $a\ptG b$.

\spheader 395Gd Still supposing that $\frak A$ is Dedekind complete,
we also find that if
$a\ptG b$ and $c\in\frak C$ then $a\Bcap c\ptG b\Bcap c$.
\prooflet{\Prf\ There is a $\phi\in G^*_{\tau}$ such that $\phi
a\Bsubseteq b$;  now $\phi(a\Bcap c)=\phi a\Bcap c\Bsubseteq b\Bcap c$.\
\QeD}   Hence\cmmnt{, or
otherwise,} $a\Bcap c$ and $b\Bcap c$ are \Gte\ whenever $a$ and $b$ are
\Gte\ and $c\in\frak C$.

\spheader 395Ge\dvArevised{2010}\cmmnt{ By analogy with the notion of
`ergodic automorphism',}
I will say that $G$ is {\bf ergodic} if $\sup_{\pi\in G}\pi a=1$
for every non-zero $a\in\frak A$.
\cmmnt{Thus an automorphism $\pi$ is
ergodic in the sense of 372Oa iff the group $\{\pi^n:n\in\Bbb Z\}$ it
generates is ergodic (372Pb).}

\spheader 395Gf\dvAnew{2010} If $G$ is ergodic, then $\frak C=\{0,1\}$.
\prooflet{(If $c\in\frak C\setminus\{0\}$, then
$1=\sup_{\pi\in G}\pi c=c$.)}
If $\frak A$ is Dedekind complete and $\frak C=\{0,1\}$
then $G$ is ergodic.   \prooflet{(If $a\in\frak A\setminus\{0\}$,
then $1=\upr(a,\frak C)=\sup_{\pi\in G}\pi a$, by (b) above.)}
\cmmnt{(Cf.\ 392Sa, 392Sc.)}

\leader{395H}{}\cmmnt{ I now embark on a series of lemmas leading to
the main theorem (395N).

\medskip

\noindent}{\bf Lemma} Let $\frak A$ be a Dedekind complete Boolean
algebra and $G$ a fully non-paradoxical subgroup of $\Aut\frak A$.
Write $\frak C$ for the
fixed-point subalgebra of $G$.   Take any $a$, $b\in\frak A$.   Set
$c_0=\sup\{c:c\in\frak C,\,a\Bcap c\ptG b\}$;  then $a\Bcap c_0\ptG b$
and $b\Bsetminus c_0\ptG a$.

\proof{ Enumerate $G$ as $\langle\pi_{\xi}\rangle_{\xi<\kappa}$, where
$\kappa=\#(G)$.   Define $\langle a_{\xi}\rangle_{\xi<\kappa}$,
$\langle b_{\xi}\rangle_{\xi<\kappa}$ inductively,  setting

\Centerline{$a_{\xi}=(a\Bsetminus\sup_{\eta<\xi}a_{\eta})
\Bcap\pi_{\xi}^{-1}(b\Bsetminus\sup_{\eta<\xi}b_{\eta})$,
\quad$b_{\xi}=\pi_{\xi}a_{\xi}$.}

\noindent Then $\langle a_{\xi}\rangle_{\xi<\kappa}$ is a disjoint
family in $\frak A_a$ and $\langle b_{\xi}\rangle_{\xi<\kappa}$ is a
disjoint family in $\frak A_b$, and $\sup_{\xi<\kappa}a_{\xi}$ is \Gte\
with $\sup_{\xi<\kappa}b_{\xi}$.
Set $a'=a\Bsetminus\sup_{\xi<\kappa}a_{\xi}$,
$b'=b\Bsetminus\sup_{\xi<\kappa}b_{\xi}$,

\Centerline{$\tilde c_0=1\Bsetminus\upr(a',\frak C)
=\sup\{c:c\in\frak C,\,c\Bcap a'=0\}$.}

\noindent Then

\Centerline{$a\Bcap\tilde c_0\Bsubseteq\sup_{\xi<\kappa}a_{\xi}\ptG b$,}

\noindent so $\tilde c_0\Bsubseteq c_0$.

Now $b'\Bsubseteq\tilde c_0$.   \Prf\Quer\ Otherwise, because
$\tilde c_0=1\Bsetminus\sup_{\xi<\kappa}\pi_{\xi}a'$ (395Gb), there must
be a $\xi<\kappa$ such that $\pi_{\xi}a'\Bcap b'\ne 0$.   But in this
case $d=a'\Bcap\pi_{\xi}^{-1}b'\ne 0$, and we have

\Centerline{$d\Bsubseteq(a\Bsetminus\sup_{\eta<\xi}a_{\eta})
\Bcap\pi_{\xi}^{-1}(b\Bsetminus\sup_{\eta<\xi}b_{\eta})$,}

\noindent so that $d\Bsubseteq a_{\xi}$, which is absurd.\ \Bang\QeD\
Consequently

\Centerline{$b\Bsetminus\tilde c_0
\Bsubseteq\sup_{\xi<\kappa}b_{\xi}\ptG a$.}

Now take any $c\in\frak C$ such that $a\Bcap c\ptG b$, and consider
$c'=c\Bsetminus\tilde c_0$.   Then $b'\Bcap c'=0$, that is,
$b\Bcap c'=\sup_{\xi<\kappa}b_{\xi}\Bcap c'$, which is \Gte\ with
$\sup_{\xi<\kappa}a_{\xi}\Bcap c'=(a\Bsetminus a')\Bcap c'$ (395Gd).
But now

\Centerline{$a\Bcap c'=a\Bcap c\Bcap c'
\ptG b\Bcap c'\ptG(a\Bcap c')\Bsetminus(a'\Bcap c')$;}

\noindent because $G$ is fully non-paradoxical, $a'\Bcap c'$ must be
$0$, that is, $c'\Bsubseteq\tilde c_0$ and $c'=0$.   As $c'$ is
arbitrary, $c_0\Bsubseteq\tilde c_0$ and $c_0=\tilde c_0$.
So $c_0$ has the required properties.
}%end of proof of 395H

\cmmnt{\medskip

\noindent{\bf Remark} By analogy with the notation I used in discussing
the Hahn decomposition of countably additive functionals (326S-326T), we
might denote $c_0$ as `$\Bvalue{a\ptG b}$', or perhaps
`$\Bvalue{a\ptG b}_{\frak C}$', `the region (in $\frak C$) where
$a\ptG b$'.   The same
notation would write $\upr(a,\frak C)$ as `$\Bvalue{a\ne 0}_{\frak C}$'.
}%end of comment

\leader{395I}{}\cmmnt{ The construction I wish to use depends
essentially on $L^0$ spaces as described in \S364.   The next step is
the following.

\medskip

\noindent}{\bf Lemma} Let $\frak A$ be a Dedekind complete Boolean
algebra, not $\{0\}$, and $G$ a fully non-paradoxical subgroup of
$\Aut\frak A$.   Let $\frak C$ be the fixed-point subalgebra of $G$.
Suppose that $a$, $b\in\frak A$ and that $\upr(a,\frak C)=1$.   Then
there are $u$, $v\in L^0=L^0(\frak C)$ such that

$$\eqalign{\Bvalue{u\ge n}
&=\max\{c:c\in\frak C,
  \text{ there is a disjoint family }\langle d_i\rangle_{i<n}\cr
&\qquad\qquad\qquad\text{such that }c\Bcap a\ptG d_i\Bsubseteq b
   \text{ for every }i<n\},\cr
\Bvalue{v\le n}
&=\max\{c:c\in\frak C,\text{ there is a family }\langle
d_i\rangle_{i<n}\cr
&\qquad\qquad\qquad\text{such that }d_i\ptG a\text{ for every }i<n
   \text{ and }b\Bcap c\Bsubseteq\sup_{i<n}d_i\}\cr}$$

\noindent for every $n\in\Bbb N$.   Moreover, we can arrange that

\quad(i) $\Bvalue{u\in\Bbb N}=\Bvalue{v\in\Bbb N}=1$,

\quad(ii) $\Bvalue{v>0}=\upr(b,\frak C)$,

\quad(iii) $u\le v\le u+\chi 1$.

\cmmnt{\medskip

\noindent{\bf Remark} By writing `max' in the formulae above, I mean to
imply that the elements $\Bvalue{u\ge n}$, $\Bvalue{v\le n}$ belong to
the sets described.}

\proof{{\bf (a)} Choose $\sequencen{c_n}$, $\sequencen{b_n}$ as follows.
Given $\langle b_i\rangle_{i<n}$, set $b'_n=b\Bsetminus\sup_{i<n}b_i$,

\Centerline{$c_n=\sup\{c:c\in\frak C$, $a\Bcap c\ptG b'_n\}$,}

\noindent so that $a\Bcap c_n\ptG b'_n$ (395H);  choose
$b_n\Bsubseteq b'_n$ such that $a\Bcap c_n$ is \Gte\ with $b_n$, and continue.   Then
$\sequencen{b_n}$ is a disjoint sequence in $\frak A_b$ and
$\sequencen{c_n}$ is a
non-increasing sequence in $\frak C$.

For each $n$, we have
$b'_n\Bsetminus c_n\ptG a$, by 395H;  while $a\Bcap c\not\ptG b'_n$
whenever $c\in\frak C$ and $c\notBsubseteq c_n$.   Note also that,
because $\upr(a,\frak C)=1$,

\Centerline{$c_n=\upr(a\Bcap c_n,\frak C)=\upr(b_n,\frak C)
\Bsubseteq\upr(b'_n,\frak C)$}

\noindent(using 395Gc for the second equality).
\medskip

{\bf (b)} Now $\inf_{n\in\Bbb N}c_n=0$.   \Prf\ Setting
$c_{\infty}=\inf_{n\in\Bbb N}c_n$,
$\sequencen{b_n\Bcap c_{\infty}}$ is a disjoint sequence, all \Gte\ with
$a\Bcap c_{\infty}$, so $a\Bcap c_{\infty}=0$, because $G$ is fully
non-paradoxical;  because
$\upr(a,\frak C)=1$, it follows that $c_{\infty}=0$.\ \Qed\
Accordingly, if we set
$u=\sup_{n\in\Bbb N}(n+1)\chi c_n$, $u\in L^0$ and
$\Bvalue{u\ge n}=c_{n-1}$ for $n\ge 1$.   The construction ensures that
$\Bvalue{u\in\Bbb N}$, as defined in 364G, is equal to $1$.

\medskip

{\bf (c)} Consider next $c'_0=\upr(b,\frak C)$,
$c'_n=c_{n-1}\Bcap\upr(b'_n,\frak C)$ for $n\ge 1$.   Then
$\sequencen{c'_n}$ is a non-increasing sequence with zero infimum, so
again we can define $v\in L^0$ by setting
$v=\sup_{n\in\Bbb N}(n+1)\chi c'_n$.   Once again,
$\Bvalue{v\in\Bbb N}=1$, and $\Bvalue{v\le n}=1\Bsetminus c'_n$ for each
$n$.

Of course $\Bvalue{v>0}=c'_0=\upr(b,\frak C)$.   Because
$c_n\Bsubseteq c'_n\Bsubseteq c_{n-1}$,

\Centerline{$(n+1)\chi c_n\le(n+1)\chi c'_n\le
n\chi c_{n-1}+\chi 1$}

\noindent for each $n\ge 1$, and $u\le v\le u+\chi 1$.

\medskip

{\bf (d)} Now set

$$\eqalign{C_n
&=\{c:c\in\frak C,\text{ there is a disjoint family }
  \langle d_i\rangle_{i<n}\cr
&\qquad\qquad\text{such that }c\Bcap a\ptG d_i\Bsubseteq b
   \text{ for every }i<n\}.\cr}$$

\noindent Then $c_n=\max C_{n+1}$.

\medskip

\Prf\grheada\ Because
$c_n\Bsubseteq c_{n-1}\Bsubseteq\ldots\Bsubseteq c_0$,
$a\Bcap c_n\ptG b_i$ for every $i\le n$, so that
$\langle b_i\rangle_{i\le n}$ witnesses that $c_n\in C_{n+1}$.

\medskip

\quad\grheadb\ Suppose that $c\in C_{n+1}$;  let
$\langle d_i\rangle_{i\le n}$ be a disjoint family such that
$c\Bcap a\ptG d_i\Bsubseteq b$ for every $i$.   Set
$c'=c\Bsetminus c_n$.   For each $i<n$, $b_i\ptG a$, so

\Centerline{$b_i\Bcap c'\ptG a\Bcap c'\ptG d_i\Bcap c'$,}

\noindent while also

\Centerline{$b'_n\Bcap c'\ptG a\Bcap c'\ptG d_n\Bcap c'$.}

\noindent Take $d\Bsubseteq d_n\Bcap c'$ such that $b'_n\Bcap c'$ is
\Gte\ with $d$.   Then

\Centerline{$b\Bcap c'=(b'_n\Bcap c')\Bcup\sup_{i<n}(b_i\Bcap c')
\ptG d\Bcup\sup_{i<n}(d_i\Bcap c')\Bsubseteq b\Bcap c'$.}

\noindent Because $G$ is fully non-paradoxical,
$d\Bcup\sup_{i<n}(d_i\Bcap c')$ must be exactly $b\Bcap c'$, so $d$ must
be the whole of $d_n\Bcap c'$, and

\Centerline{$a\Bcap c'\ptG d_n\Bcap c'=d\ptG b'_n$.}

\noindent But this means that $c'\Bsubseteq c_n$.   Thus $c'=0$ and
$c\Bsubseteq c_n$.   So $c_n=\sup C_{n+1}=\max C_{n+1}$.\ \Qed

Accordingly

\Centerline{$\Bvalue{u\ge n}=c_{n-1}=\max C_n$}

\noindent for $n\ge 1$.   For $n=0$ we have
$\Bvalue{u\ge 0}=1=\max C_0$.   So $\Bvalue{u\ge n}=\max C_n$ for every
$n$, as required.

\medskip

{\bf (e)} Similarly, if we set

$$\eqalign{C'_n
&=\{c:c\in\frak C,\text{ there is a family }\langle d_i\rangle_{i<n}\cr
&\qquad\qquad\text{such that }d_i\ptG a\text{ for every }i<n
   \text{ and }b\Bcap c\Bsubseteq\sup_{i<n}d_i\}\cr}$$

\noindent then $1\Bsetminus c'_n=\max C'_n$ for every $n$.

\medskip

\Prf\grheada\ If $n=0$, then of course (interpreting $\sup\emptyset$ as
$0$) $1\Bsetminus c'_0\in C'_0$ because $b\Bsubseteq c'_0$.
For each $n\in\Bbb N$, set

\Centerline{$\tilde b_n=b_n\Bcup(b'_n\Bsetminus c_n)
=(b_n\Bcap c_n)\Bcup(b'_n\Bsetminus c_n)$.}

\noindent Because $b_n\ptG a$ and $b'_n\Bsetminus c_n\ptG a$, we have
$b_n\Bcap c_n\ptG a\Bcap c_n$ and
$b'_n\Bsetminus c_n\ptG a\Bsetminus c_n$, so $\tilde b_n\ptG a$
(395Cd).   If we look at

\Centerline{$\sup_{i<n}\tilde b_i
\Bsupseteq\sup_{i<n}b_i\Bcup(b'_{n-1}\Bsetminus c_{n-1})$,}

\noindent we see that, for $n\ge 1$,

\Centerline{$b\Bsetminus\sup_{i<n}\tilde b_i
\Bsubseteq b'_n\Bcap c_{n-1}\Bsubseteq c'_n$,}

\noindent so that $b\Bsetminus c'_n\Bsubseteq\sup_{i<n}\tilde b_i$ and
$\{\tilde b_i:i<n\}$ witnesses that $1\Bsetminus c'_n\in C'_n$.

\medskip

\quad\grheadb\ Now take any $c\in C'_n$ and a corresponding family
$\langle d_i\rangle_{i<n}$ such that $d_i\ptG a$ for every $i<n$ and
$b\Bcap c\Bsubseteq\sup_{i<n}d_i$.

Set $c'=c\Bcap c'_n$.   For each $i<n$,

\Centerline{$c'\Bcap d_i\ptG c'\Bcap a\ptG b_i$}

\noindent because $c'\Bsubseteq c_i$.   So (by 395Cd, as usual)

\Centerline{$c'\Bcap b\ptG c'\Bcap\sup_{i<n}b_i\Bsubseteq c'\Bcap b$,}

\noindent and (again because $G$ is fully non-paradoxical)
$c'\Bcap b=c'\Bcap\sup_{i<n}b_i$, that is, $c'\Bcap b'_n=0$.   But
$c'\Bsubseteq c'_n\Bsubseteq\penalty-100\discretionary{}{}{}
\upr(b'_n,\frak C)$, so $c'$ must be $0$,
which means that $c\Bsubseteq 1\Bsetminus c'_n$.   As $c$ is arbitrary,
$1\Bsetminus c'_n=\sup C'_n=\max C'_n$.\ \Qed

Thus $\Bvalue{v\le n}=\max C'_n$, as declared.
}%end of proof of 395I

\leader{395J}{Notation} Observe that the specification of $\Bvalue{u\ge n}$
and $\Bvalue{v\le n}$, together with the declaration that
$\Bvalue{u\in\Bbb N}=\Bvalue{v\in\Bbb N}=1$, determine $u$ and $v$
uniquely\cmmnt{, because $\sequencen{\Bvalue{u=n}}$ and
$\sequencen{\Bvalue{v=n}}$
must be partitions of unity}.   So\cmmnt{, in the context of 395I,} we
can write $\low{b:a}$ for $u$ and $\high{b:a}$ for $v$.

\leader{395K}{Lemma} Let $\frak A$ be a Dedekind complete Boolean
algebra, not $\{0\}$, and $G$ a fully non-paradoxical subgroup of
$\Aut\frak A$ with fixed-point subalgebra $\frak C$.   Suppose that $a$,
$b$, $b_1$, $b_2\in\frak A$ and that $\upr(a,\frak C)=1$.

(a) $\low{0:a}=\high{0:a}=0$, $\low{1:a}\ge\chi 1$ and
$\low{1:1}=\chi 1$.

(b) If $b_1\ptG b_2$ then $\low{b_1:a}\le\low{b_2:a}$ and
$\high{b_1:a}\le\high{b_2:a}$.

(c) $\high{b_1\Bcup b_2:a}\le\high{b_1:a}+\high{b_2:a}$.

(d) If $b_1\Bcap b_2=0$,
$\low{b_1:a}+\low{b_2:a}\le\low{b_1\Bcup b_2:a}$.

(e) If $c\in\frak C$ is such that $a\Bcap c$ is a relative atom over
$\frak C$\cmmnt{ (definition:  331A)}, then
$c\Bsubseteq\Bvalue{\high{b:a}-\low{b:a}=0}$.

\proof{{\bf (a)-(b)} are immediate from the definitions and the basic
properties of $\ptG$, $\high{\ldots}$ and $\low{\ldots}$, as listed in
395C and 395I.

\medskip

{\bf (c)} For $j$, $k\in\Bbb N$, set
$c_{jk}=\Bvalue{\high{b_1:a}=j}\Bcap\Bvalue{\high{b_2:a}=k}$.   Then

\Centerline{$c_{jk}
\Bsubseteq\Bvalue{\high{b_1\Bcup b_2:a}\le j+k}
  \Bcap\Bvalue{\high{b_1:a}+\high{b_2:a}=j+k}$.}

\noindent\Prf\ We may suppose that $c_{jk}\ne 0$.   Of course

\Centerline{$c_{jk}\Bsubseteq\Bvalue{\high{b_1:a}+\high{b_2:a}=j+k}$.}

\noindent Next, there are sets $J$, $J'\subseteq\frak A$ such that
$d\ptG a$ for every $d\in J\cup J'$, $\#(J)\le j$, $\#(J')\le k$,
$\sup J\Bsupseteq b_1\Bcap c_{jk}$ and
$\sup J'\Bsupseteq b_2\Bcap c_{jk}$.
So $\sup(J\cup J')\Bsupseteq(b_1\Bcup b_2)\Bcap c_{jk}$ and $J\cup J'$
witnesses that
$c_{jk}\Bsubseteq\Bvalue{\high{b_1\Bcup b_2:a}\le j+k}$.\ \Qed

Accordingly

\Centerline{$c_{jk}\Bsubseteq
\Bvalue{\high{b_1:a}+\high{b_2:a}-\high{b_1\Bcup b_2:a}\ge 0}$.}

\noindent Now as $\sup_{j,k\in\Bbb N}c_{jk}=1$, we must have
$\high{b_1\Bcup b_2:a}\le\high{b_1:a}+\high{b_2:a}$.

\medskip

{\bf (d)} This time,
set $c_{jk}=\Bvalue{\low{b_1:a}=j}\Bcap\Bvalue{\low{b_2:a}=k}$ for $j$,
$k\in\Bbb N$.   Then

\Centerline{$c_{jk}\Bsubseteq\Bvalue{\low{b_1\Bcup b_2:a}\ge
j+k}\Bcap\Bvalue{\low{b_1:a}+\low{b_2:a}=j+k}$}

\noindent for every $j$, $k\in\Bbb N$.   \Prf\ Once again, we surely
have

\Centerline{$c_{jk}\Bsubseteq\Bvalue{\low{b_1:a}+\low{b_2:a}=j+k}$.}

\noindent Next, we can find a family $\langle d_i\rangle_{i<j+k}$ such
that

\Centerline{$\langle d_i\rangle_{i<j}$ is disjoint, $a\Bcap c_{jk}\ptG
d_i\Bsubseteq b_1$ for every $i<k$,}

\Centerline{$\langle d_i\rangle_{j\le i<j+k}$ is disjoint,
$a\Bcap c_{jk}\ptG d_i\Bsubseteq b_2$ for $j\le i<j+k$.}

\noindent As $b_1\Bcap b_2=0$, the whole family
$\langle d_i\rangle_{i<j+k}$ is disjoint and witnesses that
$c_{jk}\Bsubseteq\Bvalue{\low{b_1\Bcup b_2:a}\ge j+k}$.\ \Qed

So

\Centerline{$c_{jk}
\Bsubseteq\Bvalue{\low{b_1\Bcup b_2:a}-\low{b_1:a}-\low{b_2:a}\ge 0}$}

\noindent Since $\sup_{j,k\in\Bbb N}c_{jk}=1$, as before, we must have
$\low{b_1\Bcup b_2:a}\ge\low{b_1:a}+\low{b_2:a}$.

\medskip

{\bf (e)} \Quer\ Otherwise, there must be some $k\in\Bbb N$ such that

\Centerline{$c_0=c\Bcap\Bvalue{\low{b:a}=k}\Bcap\Bvalue{\high{b:a}>k}
\ne 0$.}

\noindent Let $\ofamily{i}{k}{d_i}$ be a disjoint family in $\frak A_b$
such that $a\Bcap c_0\ptG d_i$ for each $i$;  cutting the $d_i$ down if
necessary, we may suppose that $a\Bcap c_0$ is \Gte\ with $d_i$ for each
$i$.   As $c_0\notBsubseteq\Bvalue{\high{b:a}\le k}$,
$b\Bcap c_0\notBsubseteq\sup_{i<k}d_i$;  set
$d=b\Bcap c_0\Bsetminus\sup_{i<k}d_i\ne 0$.
By 395H, there is a $c_1\in\frak C$ such that $d\Bcap c_1\ptG a$ and
$a\Bsetminus c_1\ptG d$.   Setting $d_k=d$, $\langle d_i\rangle_{i\le k}$
witnesses that $c_0\Bsubseteq c_1\Bsubseteq\Bvalue{\low{b:a}\ge k+1}$,
so $c_0\Bsubseteq c_1$ must be $0$ and $d\Bcap c_0\ptG a$.
There is therefore a non-zero
$\tilde a\Bsubseteq a\Bcap c_0$ such that $\tilde a\ptG d$.   But now
remember that $a\Bcap c$ is supposed to be a relative atom over
$\frak C$, so $\tilde a=a\Bcap\tilde c$ for some $\tilde c\in\frak C$
such that $\tilde c\Bsubseteq c_0$.   In this case,
$a\Bcap\tilde c\ptG d_i$ for every $i<k$ and also
$a\Bcap\tilde c\ptG d$, so
$0\ne\tilde c\Bsubseteq\Bvalue{\low{b:a}\ge k+1}$, which is
absurd.\ \Bang
}%end of proof of 395K

\leader{395L}{Lemma} Let $\frak A$ be a Dedekind complete Boolean
algebra, not $\{0\}$, and $G$ a fully non-paradoxical subgroup of
$\Aut\frak A$ with fixed-point subalgebra $\frak C$.   Suppose that
$a_1$, $a_2$, $b\in\frak A$ and that
$\upr(a_1,\frak C)=\upr(a_2,\frak C)=1$.   Then

\Centerline{$\low{b:a_2}\ge\low{b:a_1}\times\low{a_1:a_2}$,
\quad$\high{b:a_2}\le\high{b:a_1}\times\high{a_1:a_2}$.}

\proof{ I use the same method as in 395K.   As usual, write $G^*_{\tau}$
for the full local semigroup generated by $G$.

\medskip

{\bf (a)} For $j$, $k\in\Bbb N$ set

\Centerline{$c_{j,k}=\Bvalue{\low{b:a_1}=j}
\Bcap\Bvalue{\low{a_1:a_2}=k}$.}

\noindent Then

\Centerline{$c_{j,k}\Bsubseteq\Bvalue{\low{b:a_1}\times\low{a_1:a_2}=jk}
\Bcap\Bvalue{\low{b:a_2}\ge jk}$.}

\noindent\Prf\ Write $c$ for $c_{j,k}$.   As in parts (c) and (d) of the
proof of 395K, the fact that
$c\Bsubseteq\Bvalue{\low{b:a_1}\times\low{a_1:a_2}=jk}$ is elementary;
what we need to check is that $c\Bsubseteq\Bvalue{\low{b:a_2}\ge jk}$.
Again, we may suppose that $c\ne 0$.   There are families
$\langle d_i\rangle_{i<j}$, $\langle d^*_l\rangle_{l<k}$ such that

\Centerline{$\langle d_i\rangle_{i<j}$ is disjoint,
$a_1\Bcap c\ptG d_i\Bsubseteq b$ for every $i<j$,}

\Centerline{$\langle d^*_l\rangle_{l<k}$ is disjoint,
$a_2\Bcap c\ptG d^*_l\Bsubseteq a_1$ for every $l<k$.}

\noindent For each $i<j$, let $\phi_i\in G^*_{\tau}$ be such that
$\phi_i(a_1\Bcap c)\Bsubseteq d_i$.   If $i<j$ and $l<k$, then

\Centerline{$a_2\Bcap c\ptG d^*_l\Bcap c\ptG\phi_i(d^*_l\Bcap c)
\Bsubseteq\phi_i(a_1\Bcap c)\Bsubseteq d_i\Bsubseteq b$.}

\noindent Also $\langle\phi_i(d^*_l\Bcap c)\rangle_{i<j,l<k}$ is
disjoint because $\langle\phi_i(a_1\Bcap c)\rangle_{i<j}$ and
$\langle d^*_l\rangle_{l<k}$ are, so witnesses that
$c\Bsubseteq\Bvalue{\low{b:a_2}\ge jk}$.\ \Qed

Now, just as in 395K, it follows from the fact that
$\sup_{j,k\in\Bbb N}c_{j,k}=1$ that
$\low{b:a_1}\times\low{a_1:a_2}\le\low{b:a_2}$.

\medskip

{\bf (b)} For $j$, $k\in\Bbb N$ set

\Centerline{$c_{j,k}=\Bvalue{\high{b:a_1}=j}
\Bcap\Bvalue{\high{a_1:a_2}=k}$.}

\noindent Then

\Centerline{$c_{j,k}\Bsubseteq
\Bvalue{\high{b:a_1}\times\high{a_1:a_2}=jk}
\Bcap\Bvalue{\high{b:a_2}\le jk}$.}

\noindent\Prf\ Write $c$ for $c_{j,k}$.   Then
$c\Bsubseteq\Bvalue{\high{b:a_1}\times\high{a_1:a_2}=jk}$.    There are
families $\langle d_i\rangle_{i<j}$, $\langle d_l^*\rangle_{l<k}$ such
that $d_i\ptG a_1$ for every $i<j$, $d^*_l\ptG a_2$ for every
$l<k$, $b\Bcap c\Bsubseteq\sup_{i<j}d_i$ and
$a_1\Bcap c\Bsubseteq\sup_{l<k}d^*_l$.
For each $i<j$, let $d_i'\Bsubseteq a_1$ be \Gte\ with $d_i$, and take
$\phi_i\in G^*_{\tau}$ such that $\phi_id_i'=d_i$.   Then

\Centerline{$\phi_i(d'_i\Bcap d^*_l)\ptG d^*_l\ptG a_2$ for every
$i<j$, $l<k$,}

$$\eqalign{\sup_{i<j,l<k}\phi_i(d'_i\Bcap d^*_l)
&=\sup_{i<j}\phi_i(d'_i\Bcap\sup_{l<k}d^*_l)
\Bsupseteq\sup_{i<j}\phi_i(d'_i\Bcap c)\cr
&=\sup_{i<j}d_i\Bcap c
\Bsupseteq b\Bcap c.\cr}$$

\noindent So $\langle\phi_i(d'_i\Bcap d^*_l)\rangle_{i<j,l<k}$ witnesses
that $c\Bsubseteq\Bvalue{\high{b:a_2}\le jk}$.\ \Qed

Once again, it follows easily that
$\high{b:a_1}\times\high{a_1:a_2}\ge\high{b:a_2}$.
}%end of proof of 395L

\vleader{60pt}{395M}{Lemma} Let $\frak A$ be a Dedekind complete Boolean
algebra, not $\{0\}$, and $G$ a subgroup of
$\Aut\frak A$ with fixed-point subalgebra $\frak C$.

(a) For any $a\in\frak A$, there is a $b\Bsubseteq a$ such that
$b\ptG a\Bsetminus b$ and $a\Bsetminus\upr(b,\frak C)$ is a either $0$
or a relative atom over $\frak C$.

(b) Now suppose that $G$ is fully non-paradoxical.   Then for any
$\epsilon>0$ there is an $a\in\frak A$ such that
$\upr(a,\frak C)=1$ and $\high{b:a}\le\low{b:a}+\epsilon\low{1:a}$ for
every $b\in\frak A$.

\proof{{\bf (a)} Set $B=\{d:d\Bsubseteq a,\,d\ptG a\Bsetminus d\}$ and
let $D\subseteq B$ be a maximal subset such that
$\upr(d,\frak C)\Bcap\upr(d',\frak C)\penalty-100\discretionary{}{}{}=0$ for all distinct $d$, $d'\in\frak D$.   Set
$b=\sup D$.   For any $d\in D$, $d\ptG a\Bsetminus d$, so

$$\eqalign{b\Bcap\upr(d,\frak C)
&=\sup_{d'\in D}d'\Bcap\upr(d,\frak C)
=\sup_{d'\in D}d'\Bcap\upr(d',\frak C)\Bcap\upr(d,\frak C)
=d\Bcap\upr(d,\frak C)\cr
&\ptG(a\Bsetminus d)\Bcap\upr(d,\frak C)
=(a\Bsetminus b)\Bcap\upr(d,\frak C)
\Bsubseteq a\Bsetminus b\cr}$$

\noindent by 395Gc.   By 395H,

\Centerline{$b=\sup_{d\in D}b\Bcap\upr(d,\frak C)\ptG a\Bsetminus b$.}

\Quer\ Suppose, if possible, that $a'=a\Bsetminus\upr(b,\frak C)$ is
neither $0$ nor
a relative atom over $\frak C$.   Let $d_0\Bsubseteq a'$ be an element
not expressible as $a'\Bcap c$ for any $c\in\frak C$;  then
$d_0\ne a\Bcap\upr(d_0,\frak C)$ and there must be a $\pi\in G$ such that
$d_1=\pi d_0\Bcap a\Bsetminus d_0$ is non-zero (395Gb).    In this case

\Centerline{$d_1\ptG \pi^{-1}d_1\Bsubseteq d_0
\Bsubseteq a\Bsetminus d_1$,}

\noindent so $d_1\in B$;  but also

\Centerline{$d_1\Bcap\upr(d,\frak C)
\Bsubseteq d_1\Bcap\upr(b,\frak C)=0$,}

\noindent so $\upr(d_1,\frak C)\Bcap\upr(d,\frak C)=0$, for every
$d\in D$, and we ought to have put $d_1$ into $D$.\ \Bang

Thus $b$ has the required properties.

\medskip

{\bf (b)(i)} For every $n\in\Bbb N$ we can find $a_n\in\frak A$ and
$c_n\in\frak C$ such that $\upr(a_n,\frak C)=1$, $a_n\Bsetminus c_n$ is
either $0$ or a
relative atom over $\frak C$, and $\low{1:a_n}\ge 2^n\chi c_n$.   \Prf\
Induce on $n$.   The induction starts with $a_0=c_0=1$, because
$\low{1:1}=\chi 1$.   For the inductive step, having found $a_n$ and
$c_n$, let $d\Bsubseteq a_n\Bcap c_n$ be such that
$d\ptG a_n\Bcap c_n\Bsetminus d$ and
$a_n\Bcap c_n\Bsetminus\upr(d,\frak C)$ is either $0$ or a
relative atom over $\frak C$, as in (a).   Set
$c_{n+1}=\upr(d,\frak C)$, $a_{n+1}=(a_n\Bsetminus c_{n+1})\Bcup d$;
then

$$\eqalign{\upr(a_{n+1},\frak C)
&=\upr(a_n\Bsetminus c_{n+1},\frak C)\Bcup\upr(d,\frak C)\cr
&=(\upr(a_n,\frak C)\Bsetminus c_{n+1})\Bcup c_{n+1}
=(1\Bsetminus c_{n+1})\Bcup c_{n+1}
=1\cr}$$

\noindent by 313Sb-313Sc and the inductive hypothesis.

We have $c_{n+1}\Bcap d\ptG c_{n+1}\Bcap a_n\Bsetminus d$, so

\Centerline{$c_{n+1}\Bcap a_{n+1}=d\Bsubseteq a_n$,
\quad$c_{n+1}\Bcap a_{n+1}\ptG a_n\Bsetminus d$,}

\noindent and $\low{a_n:a_{n+1}}\ge 2\chi c_{n+1}$;  by 395L,

\Centerline{$\low{1:a_{n+1}}\ge\low{1:a_n}\times\low{a_n:a_{n+1}}
\ge 2^n\chi c_n\times 2\chi c_{n+1}=2^{n+1}\chi c_{n+1}$.}

If

\Centerline{$b\Bsubseteq a_{n+1}\Bsetminus c_{n+1}
=(a_n\Bsetminus c_n)\Bcup(a_n\Bcap c_n\Bsetminus c_{n+1})$,}

\noindent then, because both terms on the right are either $0$ or
relative atoms over
$\frak C$, there are $c'$, $c''\in\frak C$ such that

$$\eqalign{b
&=(b\Bcap a_n\Bsetminus c_n)
  \Bcup(b\Bcap a_n\Bcap c_n\Bsetminus c_{n+1})\cr
&=(c'\Bcap a_n\Bsetminus c_n)
   \Bcup(c''\Bcap a_n\Bcap c_n\Bsetminus c_{n+1})
=c\Bcap a_{n+1}\Bsetminus c_{n+1}\cr}$$

\noindent where $c=(c'\Bsetminus c_n)\Bcup(c''\Bcap c_n)$ belongs to
$\frak C$.   So $a_{n+1}\Bsetminus c_{n+1}$ is either $0$ or
a relative atom over $\frak C$.

Thus the induction continues.\ \Qed

\medskip

\quad{\bf (ii)} Now suppose that $\epsilon>0$.   Take $n$ such that
$2^{-n}\le\epsilon$, and consider $a_n$, $c_n$ taken from (i) above.

Let $b\in\frak A$.   Set

\Centerline{$c
  =\Bvalue{\high{b:a_n}-\low{b:a_n}-\epsilon\low{1:a_n}>0}
  \in\frak C$.}

\noindent Since we know that

\Centerline{$\epsilon\low{1:a_n}\ge 2^{-n}2^n\chi c_n=\chi c_n$,
\quad$\high{b:a_n}\le\low{b:a_n}+\chi 1$,}

\noindent we must have $c\Bcap c_n=0$.   But this means that
$a_n\Bcap c$ is either $0$ or a relative atom over $\frak C$.   By 395Ke,
$c$ is included in $\Bvalue{\high{b:a_n}-\low{b:a_n}=0}$;  as also
$\low{1:a_n}\ge\chi 1$ (395Ka), $c$ must be zero, that is,
$\high{b:a_n}\le\low{b:a_n}+\epsilon\low{1:a_n}$.
}%end of proof of 395M

\leader{395N}{}\cmmnt{ We are at last ready for the theorem.

\medskip

\noindent}{\bf Theorem} Let $\frak A$ be a Dedekind complete Boolean
algebra and $G$ a fully non-paradoxical subgroup of $\Aut\frak A$ with
fixed-point subalgebra $\frak C$.   Then there is a unique function
$\theta:\frak A\to L^{\infty}(\frak C)$ such that

(i) $\theta$ is additive, non-negative and order-continuous;

(ii) $\Bvalue{\theta a>0}=\upr(a,\frak C)$ for every $a\in\frak A$;  in
particular, $\theta a=0$ iff $a=0$;

(iii) $\theta 1=\chi 1$;

(iv) $\theta(a\Bcap c)=\theta a\times\chi c$ for every $a\in\frak A$,
$c\in\frak C$;  in particular, $\theta c=\chi c$ for every
$c\in\frak C$;

(v) If $a$, $b\in\frak A$ are \Gte, then $\theta a=\theta b$;  in
particular, $\theta$ is $G$-invariant.

\proof{ If $\frak A=\{0\}$ this is trivial;  so I suppose henceforth
that $\frak A\ne\{0\}$.

\wheader{395N}{4}{2}{2}{36pt}

{\bf (a)} Set $A^*=\{a:a\in\frak A,\,\upr(a,\frak C)=1\}$ and for
$a\in A^*$, $b\in\frak A$ set

$$\theta_a(b)=\bover{\high{b:a}}{\low{1:a}}\in L^0=L^0(\frak C);$$

\noindent the first thing to note is that because $\low{1:a}\ge\chi 1$,
we can always do the divisions to obtain elements $\theta_a(b)$ of
$L^0(\frak A)$ (364N).    Set

\Centerline{$\theta b=\inf_{a\in A^*}\theta_ab$}

\noindent for $b\in\frak A$.
(Note that $L^0(\frak C)$ is Dedekind complete, by 364M, so the
infimum is defined.)

\medskip

{\bf (b)} The formulae of 395K tell us that, for $a\in A^*$ and $b_1$,
$b_2\in\frak A$,

\Centerline{$\theta_a0=0$,
\quad$\theta_ab_1\le\theta_ab_2$ if $b_1\Bsubseteq b_2$,}

\Centerline{$\theta_a(b_1\Bcup b_2)\le\theta_ab_1+\theta_ab_2$,}

\Centerline{$\theta_a1\ge\chi 1$.}

\noindent It follows at once that

\Centerline{$\theta 0=0$,
\quad$\theta b_1\le\theta b_2$ if $b_1\Bsubseteq b_2$,}

\Centerline{$\theta 1\ge\chi 1$.}

\medskip

{\bf (c)} For each $n\in\Bbb N$ there is an $e_n\in A^*$ such that
$\high{b:e_n}\le\low{b:e_n}+2^{-n}\low{1:e_n}$ for every $b\in\frak A$
(395Mb).   Now $\theta_{e_n}b\le\theta_ab+2^{-n}\high{b:a}$ for every
$a\in A^*$, $b\in\frak A$.   \Prf
$\high{a:e_n}\le\low{a:e_n}+2^{-n}\low{1:e_n}$, so

$$\eqalign{\high{a:e_n}\times\low{1:a}
&\le\low{a:e_n}\times\low{1:a}+2^{-n}\low{1:e_n}\times\low{1:a}\cr
&\le\low{1:e_n}+2^{-n}\low{1:e_n}\times\low{1:a}\cr}$$

\noindent (by 395L);  accordingly

$$\eqalignno{\high{b:e_n}\times\low{1:a}
&\le\high{b:a}\times\high{a:e_n}\times\low{1:a}\cr
\noalign{\noindent (by the other half of 395L)}
&\le\high{b:a}\times\low{1:e_n}
    +2^{-n}\high{b:a}\times\low{1:e_n}\times\low{1:a}\cr}$$

\noindent and, dividing by $\low{1:a}\times\low{1:e_n}$, we get
$\theta_{e_n}b\le\theta_ab+2^{-n}\high{b:a}$.\ \Qed

\medskip

{\bf (d)} Now $\theta$ is additive.   \Prf\ Taking $\sequencen{e_n}$
from (c), observe first that

\Centerline{$\inf_{n\in\Bbb N}\theta_{e_n}b
\le\theta_ab+\inf_{n\in\Bbb N}2^{-n}\high{b:a}=\theta_ab$}

\noindent for every $a\in A^*$, $b\in\frak A$, so that $\theta
b=\inf_{n\in\Bbb N}\theta_{e_n}b$ for every $b$.   Now suppose that
$b_1$, $b_2\in\frak A$ and $b_1\Bcap b_2=0$.   Then, for any
$n\in\Bbb N$,

$$\eqalignno{\high{b_1:e_n}+\high{b_2:e_n}
&\le\low{b_1:e_n}+\low{b_2:e_n}+2^{-n+1}\low{1:e_n}\cr
&\le\low{b_1\Bcup b_2:e_n}+2^{-n+1}\low{1:e_n}\cr
\noalign{\noindent (by 395Kd)}
&\le\high{b_1\Bcup b_2:e_n}+2^{-n+1}\low{1:e_n}.\cr}$$

\noindent Dividing by $\low{1:e_n}$, we have

\Centerline{$\theta b_1+\theta b_2
\le\theta_{e_n}b_1+\theta_{e_n}b_2
\le\theta_{e_n}(b_1\Bcup b_2)+2^{-n+1}\chi 1$.}

\noindent Taking the infimum over $n$, we get

\Centerline{$\theta b_1+\theta b_2
\le\theta(b_1\Bcup b_2)$.}

In the other direction, if $a$, $a'\in A^*$ and $n\in\Bbb N$,

$$\eqalign{\theta(b_1\Bcup b_2)
&\le\theta_{e_n}(b_1\Bcup b_2)
\le\theta_{e_n}(b_1)+\theta_{e_n}(b_2)\cr
&\le\theta_a(b_1)+2^{-n}\high{b_1:a}
  +\theta_{a'}(b_2)+2^{-n}\high{b_2:a'}.\cr}$$

\noindent As $n$ is arbitrary,
$\theta(b_1\Bcup b_2)\le\theta_a(b_1)+\theta_{a'}(b_2)$;  as $a$ and
$a'$ are arbitrary, $\theta(b_1\Bcup b_2)\le\theta b_1+\theta b_2$
(using 351Dc).

As $b_1$ and $b_2$ are arbitrary, $\theta$ is additive.\ \Qed

We see also that $\high{1:e_n}\le(1+2^{-n})\low{1:e_n}$, so that
$\theta_{e_n}1\le(1+2^{-n})\chi 1$ for each $n$;  since we already know
that $\theta 1\ge\chi 1$, we have $\theta 1=\chi 1$ exactly.

\medskip

{\bf (e)} If $c\in\frak C$ then

\Centerline{$\Bvalue{\theta c>0}\Bsubseteq\Bvalue{\theta_1 c>0}
\Bsubseteq\Bvalue{\high{c:1}>0}=\upr(c,\frak C)=c$}

\noindent (395I(ii)).   It follows that

\Centerline{$\theta(b\Bcap c)\le\theta b\wedge\theta c\le
\theta b\times\chi c$}

\noindent for any $b\in\frak A$, $c\in\frak C$.   Similarly,
$\theta(b\Bsetminus c)\le\theta b\times\chi(1\Bsetminus c)$;  adding, we
must have equality in both, and $\theta(b\Bcap c)=\theta b\times\chi c$.

Rather late, I point out that

\Centerline{$0\le\theta a\le\theta 1=\chi 1\in
L^{\infty}=L^{\infty}(\frak C)$}

\noindent for every $a\in\frak A$, so that $\theta a\in L^{\infty}$ for
every $a$.

\medskip

{\bf (f)}
If $b\in\frak A\Bsetminus\{0\}$, then

\Centerline{$\Bvalue{\theta b>0}\Bsubseteq\Bvalue{\theta_1b>0}
\Bsubseteq\Bvalue{\high{b:1}>0}=\upr(b,\frak C)$}

\noindent by 395I(ii) again.   \Quer\ Suppose, if possible, that
$\Bvalue{\theta b>0}\ne\upr(b,\frak C)$.   Set
$c_0=\upr(b,\frak C)\Bsetminus\Bvalue{\theta b>0}$,
$a_0=b\Bcup(1\Bsetminus\upr(b,\frak C))\discretionary{}{}{}\in A^*$.
Let $k\ge 1$ be such
that $c_1=c_0\Bcap\Bvalue{\high{1:a_0}\le k}\ne 0$.
Then $a_0\Bcap c_1=b\Bcap c_1$, so

\Centerline{$\theta a_0\times\chi c_1
=\theta(a_0\Bcap c_1)
=\theta(b\Bcap c_1)
=\theta b\times\chi c_1
=0$.}

\noindent By 364L(b-ii), there is an $a\in A^*$ such that
$c_1\notBsubseteq\Bvalue{\theta_aa_0\times\chi c_1\ge\bover1k}$, that
is, $c_2=c_1\Bcap\Bvalue{\theta_aa_0<\bover1k}\ne 0$.   Now

\Centerline{$c_2\Bsubseteq\Bvalue{\low{1:a}-k\high{a_0:a}>0}
\Bsubseteq\Bvalue{\high{1:a_0}\times\high{a_0:a}-k\high{a_0:a}>0}
\Bsubseteq\Bvalue{\high{1:a_0}>k}$,}

\noindent which is impossible, as $c_2\Bsubseteq c_1$.\ \Bang

Thus $\Bvalue{\theta b>0}=\upr(b,\frak C)$.   In particular, $\theta
b=0$ iff $b=0$.

\medskip

{\bf (g)} If $b$, $b'\in\frak A$ and $b\ptG b'$, then
$\theta b\le\theta b'$.   \Prf\ For every $a\in A^*$,
$\high{b:a}\le\high{b':a}$ (395Kb) so
$\theta_ab\le\theta_ab'$.\ \Qed\   So if $b$, $b'\in\frak A$ and
$c=\Bvalue{\theta b-\theta b'>0}$, $b'\Bcap c\ptG b$.
\Prf\Quer\ Otherwise, by 395H, there is a non-zero $c'\Bsubseteq c$ such
that $b\Bcap c'\ptG b'$.   But in this case
$\theta b\times\chi c'=\theta(b\Bcap c')\le\theta b'$ and
$c'\Bsubseteq\Bvalue{\theta b'-\theta b\ge 0}$.\ \Bang\Qed

\medskip

{\bf (h)} If $\langle a_i\rangle_{i\in I}$ is any disjoint family in
$\frak A$ with supremum $a$,
$\theta a=\sum_{i\in I}\theta a_i$, where the sum is to be interpreted
as $\sup_{J\subseteq I\text{ is finite}}\sum_{i\in J}\theta a_i$.
\Prf\ Induce on $\#(I)$.   If $\#(I)$ is finite, this is just finite
additivity ((d) above).   For the inductive step to
$\#(I)=\kappa\ge\omega$, we may suppose that $I$ is actually equal to
the cardinal $\kappa$.   Of course

\Centerline{$\theta a\ge\theta(\sup_{\xi\in J}a_{\xi})
=\sum_{\xi\in J}\theta a_{\xi}$}

\noindent for every finite $J\subseteq\kappa$, so (because
$L^{\infty}(\frak C)$ is Dedekind complete)
$u=\sum_{\xi<\kappa}\theta a_{\xi}$ is defined, and $u\le\theta a$.

For $\zeta<\kappa$, set $b_{\zeta}=\sup_{\xi<\zeta}a_{\xi}$.   By the
inductive hypothesis,

\Centerline{$\theta b_{\zeta}=\sum_{\xi<\zeta}\theta a_{\xi}
=\sup_{J\subseteq\zeta\text{ is finite}}\sum_{\xi\in J}\theta a_{\xi}
\le u$.}

\noindent At the same time, if $J\subseteq\kappa$ is finite, there is
some $\zeta<\kappa$ such that $J\subseteq\zeta$, so that
$\sum_{\xi\in J}\theta a_{\xi}\le\theta b_{\zeta}$;  accordingly
$\sup_{\zeta<\kappa}\theta b_{\zeta}=u$.

\Quer\ Suppose, if possible, that $u<\theta a$;  set $v=\theta a-u$.
Take $\delta>0$ such that $c_0=\Bvalue{v>\delta}\ne 0$.   Let
$\zeta<\kappa$ be such that
$c_1=c_0\Bsetminus\Bvalue{u-\theta b_{\zeta}>\delta}$ is
non-zero (cf.\ 364L(b-ii)).   Now
$v=\theta a-u\le\theta(a\Bsetminus b_{\zeta})$, so

\Centerline{$c_1\Bsubseteq\Bvalue{v>\delta}
\Bsubseteq\Bvalue{\theta(a\Bsetminus b_{\zeta})>0}
=\upr(a\Bsetminus b_{\zeta},\frak C)$,}

\noindent and $c_1\Bcap(a\Bsetminus b_{\zeta})\ne 0$;  there is
therefore an $\eta'\ge\zeta$ such that $d=c_1\Bcap a_{\eta'}\ne 0$.
Since $\theta d\le u-\theta b_{\zeta}$ and
$c_1$ is included in
$\Bvalue{u-\theta b_{\zeta}\le\delta}\Bcap\Bvalue{v>\delta}$,
$\Bvalue{v-\theta d>0}\Bsupseteq c_1$.

Choose $\langle d_{\xi}\rangle_{\xi<\kappa}$ inductively, as follows.
Given that $\langle d_{\eta}\rangle_{\eta<\xi}$ is a disjoint family in
$\frak A_{a\Bsetminus d}$ such that $d_{\eta}$ is \Gte\ with
$a_{\eta}\Bcap c_1$ for every $\eta<\xi$, then
$e_{\xi}=\sup_{\eta<\xi}d_{\eta}$ is \Gte\ with $b_{\xi}\Bcap c_1$, so
that $\theta e_{\xi}\le\theta b_{\xi}$, and

$$\eqalign{\Bvalue{\theta(a\Bsetminus(d\Bcup e_{\xi}))-\theta a_{\xi}>0}
&=\Bvalue{\theta a-\theta d-\theta e_{\xi}-\theta a_{\xi}>0}
\Bsupseteq\Bvalue{\theta a-\theta d-\theta b_{\xi}-\theta a_{\xi}>0}\cr
&=\Bvalue{\theta a-\theta d-\theta b_{\xi+1}>0}
\Bsupseteq\Bvalue{v-\theta d>0}
\Bsupseteq c_1.\cr}$$

\noindent By (g), $a_{\xi}\Bcap c_1\ptG a\Bsetminus(d\Bcup e_{\xi})$;
take $d_{\xi}\Bsubseteq a\Bsetminus(d\Bcup e_{\xi})$ \Gte\ with
$a_{\xi}\Bcap c_1$, and continue.

At the end of this induction, we have a disjoint family $\langle
d_{\xi}\rangle_{\xi<\kappa}$ in $\frak A_{a\Bsetminus d}$ such that
$d_{\xi}$ is \Gte\ with $a_{\xi}\Bcap c_1$ for every $\xi$.   But this
means that $a'=\sup_{\xi<\kappa}d_{\xi}$ is \Gte\ with $a\Bcap c_1$,
while $a'\Bsubseteq (a\Bsetminus d)\Bcap c_1$;  since
$d\Bcap a\Bcap c_1\ne 0$,
$G$ cannot be fully non-paradoxical.\ \Bang

Thus $\theta a=u=\sum_{\xi<\kappa}\theta a_{\xi}$ and the induction
continues.\ \Qed

\medskip

{\bf (i)} It follows that $\theta$ is order-continuous.   \Prf\
($\alpha$) If $B\subseteq\frak A$ is non-empty and upwards-directed and
has supremum $e$, then $\bigcup_{b\in B}\frak A_b$ is order-dense in
$\frak A_e$, so includes a partition of unity $A$ of $\frak A_e$;  now
(h) tells us that

\Centerline{$\theta e=\sum_{a\in A}\theta a\le\sup_{b\in B}\theta b$.}

\noindent   Since of course $\theta b\le\theta e$ for every $b\in B$,
$\theta e=\sup_{b\in B}\theta b$.   ($\beta$) If $B\subseteq\frak A$ is
non-empty and downwards-directed and has infimum $e$, then, using
($\alpha$), we see that

\Centerline{$\theta 1-\theta e
=\theta(1\Bsetminus e)=\sup_{b\in B}\theta(1\Bsetminus b)
=\sup_{b\in B}\theta 1-\theta b$,}

\noindent so that $\theta e=\inf_{b\in B}\theta b$.\ \Qed

\medskip

{\bf (j)} I still have to show that $\theta$ is unique.   Let
$\theta':\frak A\to L^{\infty}$ be any non-negative order-continuous
$G$-invariant additive function such that $\theta'c=\chi c$ for every
$c\in\frak C$.

\medskip

\quad{\bf (i)} Just as in (e) of this proof, but more easily, we
see that $\theta'(b\Bcap c)=\theta'b\times\chi c$ whenever
$b\in\frak A$ and $c\in\frak C$.

\medskip

\quad{\bf (ii)} If $\langle a_i\rangle_{i\in I}$ is a disjoint family in
$\frak A$ with supremum $a$, then
$\langle\sup_{i\in J}a_i\rangle_{J\subseteq I\text{ is finite}}$ is an
upwards-directed family with supremum $a$, so that

\Centerline{$\theta'a
=\sup_{J\subseteq I\text{ is finite}}\theta'
(\sup_{i\in J}a_i)
=\sup_{J\subseteq I\text{ is finite}}\sum_{i\in J}\theta'a_i
=\sum_{i\in I}\theta'a_i$.}

\medskip

\quad{\bf (iii)} $\theta'a=\theta'b$ whenever $a$ and $b$ are \Gte.
\Prf\ Take a partition $\langle a_i\rangle_{i\in I}$ of $a$ and a family
$\langle\pi_i\rangle_{i\in I}$ in $G$ such that
$\langle\pi_ia_i\rangle_{i\in I}$ is a partition of $b$.   Then

\Centerline{$\theta'a=\sum_{i\in I}\theta'a_i
=\sum_{i\in I}\theta'\pi_ia_i=\theta'b$.  \Qed}

\noindent Consequently $\theta'a\le\theta'b$ whenever $a\ptG b$.

\medskip

\quad{\bf (iv)} Take $a\in A^*$, $b\in\frak A$ and for $j$, $k\in\Bbb N$
set $c_{jk}=\Bvalue{\low{1:a}=j}\Bcap\Bvalue{\high{b:a}=k}$.  Then

\Centerline{$\high{b:a}\times\chi
c_{jk}\ge\theta'b\times\low{1:a}\times\chi
c_{jk}$.}

\noindent\Prf\ If $c_{jk}=0$ this is trivial;  suppose $c_{jk}\ne 0$.
Now we have sets $I$, $J$ such that $\#(I)=j$, $\#(J)\le k$, $a\Bcap
c_{jk}\ptG d$ for every $d\in I$, $e\ptG a$ for every $e\in J$, $I$ is
disjoint, and $b\Bcap c_{jk}\Bsubseteq\sup J$.   So

$$\eqalign{\theta'b\times\low{1:a}\times\chi c_{jk}
&=j\theta'b\times\chi c_{jk}
=j\theta'(b\Bcap c_{jk})
\le j\sum_{e\in J}\theta'(e\Bcap c_{jk})\cr
&\le jk\theta'(a\Bcap c_{jk})
\le k\sum_{d\in I}\theta'(d\Bcap c_{jk})
\le k\theta'c_{jk}\cr
&=k\chi c_{jk}
=\high{b:a}\times\chi c_{jk}. \text{ \Qed}\cr}$$

\noindent Summing over $j$ and $k$,
$\high{b:a}\ge\theta'b\times\low{1:a}$, that is, $\theta_ab\ge\theta'b$.
Taking the infimum over $a$, $\theta b\ge\theta'b$.   But also

\Centerline{$\theta b=\chi 1-\theta(1\Bsetminus b)
\le\chi 1-\theta'(1\Bsetminus b)=\theta'b$,}

\noindent so $\theta b=\theta'b$.   As $b$ is arbitrary,
$\theta=\theta'$.   This completes the proof.
}%end of proof of 395N

\leader{395O}{}\cmmnt{ We have reached the summit.   The rest of the
section is a list of easy corollaries.

\medskip

\noindent}{\bf Theorem} Let $\frak A$ be a Dedekind complete Boolean
algebra, not $\{0\}$, and $G$ a fully non-paradoxical subgroup of
$\Aut\frak A$.   Then there is a $G$-invariant additive functional
$\nu:\frak A\to[0,1]$ such that $\nu 1=1$.

\proof{ Let $\frak C$ be the fixed-point subalgebra of $G$, and
$\theta:\frak A\to L^{\infty}(\frak C)$ the function of 395N.   By
311D, there is a ring homomorphism $\nu_0:\frak C\to\{0,1\}$ such
that $\nu_01=1$;  now $\nu_0$ can also be regarded as an additive
functional from $\frak C$ to $\Bbb R$.   Let
$f_0:L^{\infty}(\frak C)\to\Bbb R$ be the
corresponding positive linear functional (363K).  Set $\nu=f_0\theta$.
Then $\nu$ is order-preserving and additive because $f_0$ and $\theta$
are, $\nu 1=f_0(\chi 1)=\nu_01>0$, and $\nu$ is $G$-invariant because
$\theta$ is.

}%end of proof of 395O

\vleader{60pt}{395P}{Theorem} Let $\frak A$ be a Dedekind complete Boolean
algebra and $G$ a fully non-paradoxical subgroup of $\Aut\frak A$ with
fixed-point subalgebra $\frak C$.   Then the following are
equiveridical:

(i) $\frak A$ is a measurable algebra;

(ii) $\frak C$ is a measurable algebra;

(iii) there is a strictly positive $G$-invariant countably additive
real-valued functional on $\frak A$.

\proof{ (iii)$\Rightarrow$(i)$\Rightarrow$(ii) are trivial.   For
(ii)$\Rightarrow$(iii), let $\theta:\frak A\to L^{\infty}(\frak C)$ be
the function of 395N, and $\bar\nu:\frak C\to\Bbb R$ a strictly positive
countably additive functional.
Let $f:L^{\infty}(\frak C)\to\Bbb R$ be the corresponding linear
operator;  then $f$ is sequentially order-continuous (363K again).   Set
$\bar\mu=f\theta$.   Then $\bar\mu$ is additive and order-preserving and
sequentially order-continuous because $f$ and $\theta$ are.   It is also
strictly positive, because if $a\in\frak A\setminus\{0\}$ then
$\theta a>0$ (395N(ii)), that is, there is some $\delta>0$ such that
$\Bvalue{\theta a>\delta}\ne 0$, so that

\Centerline{$\bar\mu a\ge\delta\bar\nu\Bvalue{\theta a>\delta}>0$.}

\noindent Finally, $\bar\mu$ is $G$-invariant because $\theta$ is.
}%end of proof of 395P

\leader{395Q}{Corollary:  Kawada's theorem} Let $\frak A$ be a Dedekind
complete Boolean algebra such that $\Aut\frak A$ has a subgroup which
is ergodic and fully non-paradoxical.   Then $\frak A$ is measurable.

\proof{ By 395Gf, this is the case $\frak C=\{0,1\}$ of 395P.
}%end of proof of 395Q

\leader{395R}{}\cmmnt{ Thus the existence of an ergodic fully
non-paradoxical subgroup is a sufficient condition for a Dedekind
complete Boolean algebra to be measurable.   It is not quite necessary,
because if a measure algebra $\frak A$ is not homogeneous then its
automorphism group is not ergodic.
But for homogeneous algebras the condition is necessary as well as
sufficient, by the following result.

\medskip

\noindent}{\bf Proposition} If $(\frak A,\bar\mu)$ is a homogeneous
totally finite measure algebra, $\AmuA$ is ergodic.

\proof{ If $\frak A=\{0,1\}$ this is trivial.   Otherwise, $\frak A$ is
atomless.   If $a$, $b\in\frak A\setminus\{0,1\}$, set
$\gamma=\min(\bar\mu a,\bar\mu b)$;  then there are $a'\Bsubseteq a$ and
$b'\Bsubseteq b$ such that $\bar\mu a'=\bar\mu b'=\gamma$.
By 383Fb, there is a $\pi\in G$ such
that $\pi a'=b'$, so that $\pi a\Bcap a\ne 0$.   As $b$ is arbitrary,
$\sup_{\pi\in G}\pi a=1$;  as $a$ is arbitrary, $G$ is ergodic.
}%end of proof of 395R

\exercises{
\leader{395X}{Basic exercises (a)}
%\spheader 395Xa
Re-write the section on the assumption that every group $G$ is ergodic,
so that $L^0(\frak C)$ may be
identified with $\Bbb R$, the functions $\high{\ldots}$ and
$\low{\ldots}$ become real-valued, the functionals $\theta_a$ (395N)
become submeasures and $\theta$ becomes a measure.
%-

\spheader 395Xb Let $\frak A$ be a Dedekind complete Boolean algebra and
$G$ a subgroup of $\Aut\frak A$ with fixed-point subalgebra $\frak C$.
Suppose that $\familyiI{c_i}$ is a partition of unity in $\frak C$ and
that $a$, $b\in\frak A$ are such that $a\Bcap c_i\ptG b$ for every
$i\in I$.   Show that $a\ptG b$.
%395G

\spheader 395Xc\dvAformerly{3{}94Xc}
Let $\frak A$ be a Dedekind complete Boolean algebra and
$G$ a subgroup of $\Aut\frak A$ with fixed-point subalgebra $\frak C$.
Show that $\frak A$ is relatively atomless over $\frak C$ iff the full
subgroup generated by $G$ has many involutions (definition:  382O).
%395H

\spheader 395Xd
Let $\frak A$ be a Dedekind complete Boolean algebra and $G$ a fully
non-paradoxical subgroup of $\Aut\frak A$ with fixed-point subalgebra
$\frak C$.   Show that the following are equiveridical:
(i) $\frak A$ is chargeable (definition:  391Bb);
(ii) $\frak C$ is chargeable;  (iii) there is a strictly positive
$G$-invariant real-valued additive functional on $\frak A$.
%395N

\spheader 395Xe
Let $\frak A$ be a Dedekind complete Boolean algebra and $G$ a fully
non-paradoxical subgroup of $\Aut\frak A$ with fixed-point subalgebra
$\frak C$.   Show that the following are equiveridical:
(i) there is a non-zero completely additive functional on $\frak A$;
(ii) there is a non-zero completely additive functional on $\frak C$;
(iii) there is a non-zero $G$-invariant completely additive functional
on $\frak A$.
%395N

\spheader 395Xf Let $\frak A$ be a ccc Dedekind complete Boolean
algebra.   Show that it is a measurable algebra iff there is a fully
non-paradoxical subgroup $G$ of $\Aut\frak A$ such that the fixed-point
subalgebra of $G$ is purely atomic.
%395R

\spheader 395Xg Let $(\frak A,\bar\mu)$ be a localizable measure
algebra.    Show that the following are equiveridical:
(i) $\AmuA$ is ergodic;  (ii) $\frak A$ is quasi-homogeneous
in the sense of 374G.
%395R

\spheader 395Xh Let $(\frak A,\bar\mu)$ be a localizable measure
algebra.   Show that $\AmuA$ is fully non-paradoxical
iff (i) for every infinite cardinal $\kappa$, the Maharam-type-$\kappa$
component of $\frak A$ (definition:  332Gb) has finite measure (ii) for
every $\gamma\in\ooint{0,\infty}$ there are only finitely many atoms of
measure $\gamma$.
%395R

\spheader 395Xi Let $\frak A$ be a Boolean algebra, $G$ a subgroup of
$\Aut\frak A$, and $G^*$ the full subgroup of $\Aut\frak A$ generated by
$G$.   Show that $G^*$ is ergodic iff $G$ is ergodic.
%395G out of order query

\leader{395Y}{Further exercises (a)}
%\spheader 395Ya
Let $\frak A$ be a Dedekind complete Boolean algebra, $G$ a subgroup of
$\Aut\frak A$, and $G^*_{\tau}$ the full local semigroup generated by
$G$.   For $\phi$, $\psi\in G^*_{\tau}$, say that $\phi\le\psi$ if
$\psi$ extends $\phi$.   (i) Show that every member of $G^*_{\tau}$ can
be extended to a maximal member of $G^*_{\tau}$.   (ii) Show that $G$ is
fully non-paradoxical iff every maximal member of $G^*_{\tau}$ is
actually a Boolean automorphism of $\frak A$.
%395E

\spheader 395Yb Let $\frak A$ be a ccc Dedekind complete Boolean algebra
and $G$
a subgroup of $\Aut\frak A$.   Show that $G$ is fully non-paradoxical iff
$\sequencen{\pi_na_n}$ order*-converges to $0$ whenever $\sequencen{a_n}$
is a disjoint sequence in $\frak A$ and $\sequencen{\pi_n}$
is a sequence in $G$.
%395E  easy from 395D(ii) but we don't use order*-cgt much

\spheader 395Yc Let $\frak A$ be a Dedekind complete Boolean algebra and
$G$ a fully non-paradoxical subgroup of $\Aut\frak A$ with fixed-point
subalgebra $\frak C$.   (i)
Show that $\frak A$ is ccc iff $\frak C$ is ccc.
\Hint{if $\frak C$ is ccc, $L^{\infty}(\frak C)$ has the countable sup
property.}
(ii) Show that $\frak A$ is \wsid\ iff $\frak C$ is.
(iii) Show that $\frak A$ is a Maharam algebra iff $\frak C$ is.
%395N   %mt39bits

\spheader 395Yd Let $\frak A$ be a Dedekind complete Boolean algebra,
$G$ an ergodic subgroup of $\Aut\frak A$, and $G^*_{\tau}$ the full local
semigroup generated by $G$.   Suppose that there is a non-zero
$a\in\frak A$ for which there is no $\phi\in G^*_{\tau}$ such that
$\phi a\Bsubset a$.   Show that there is a measure $\bar\mu$ such that
$(\frak A,\bar\mu)$ is a localizable measure algebra.   \Hint{show that
$\frak A_a$ is a measurable algebra.}
%395Q

\spheader 395Ye Show that there are a
semi-finite measure algebra $(\frak A,\bar\mu)$
and a subgroup $G$ of $\AmuA$ such that $G$ is not ergodic but
has fixed-point algebra $\{0,1\}$.
%395G mt39bits out of order query
}%end of exercises

\leader{395Z}{Problem} Suppose that $\frak A$ is a Dedekind complete
Boolean algebra, not $\{0\}$, and $G$ a subgroup of $\Aut\frak A$ such
that whenever $\langle a_i\rangle_{i\le n}$ is a finite partition of
unity in $\frak A$ and we are given $\pi_i$, $\pi'_i\in G$ for every
$i\le n$, then the elements $\pi_0a_0$, $\pi'_0a_0$, $\pi_1a_1$,
$\pi'_1a_1,\ldots,\pi'_na_n$ are not all disjoint.   Must there be a
non-zero non-negative $G$-invariant finitely additive functional
$\theta$ on $\frak A$?

\cmmnt{(See `Tarski's theorem' in the notes below.)}

\endnotes{\Notesheader{395} Regarded as a sufficient condition for
measurability, Kawada's theorem suffers from the obvious defect that it
is going to be rather rarely that we can verify the existence of an
ergodic fully non-paradoxical group of automorphisms without having some
quite different reason for supposing that our algebra is measurable.
If we think of it as a criterion for the existence of a
$G$-invariant measure, rather than as a criterion for measurability in
the abstract, it seems to make better sense.   But if we know from the
start that the algebra $\frak A$ is measurable, the argument
short-circuits, as we shall see in \S396.

I take the trouble to include the `$\tau$' in every `\Gte',
`$G^*_{\tau}$' and `$\ptG$' because there are important variations on
the concept, in which the partitions $\langle a_i\rangle_{i\in I}$ of
395A are required to be finite or countable.   Indeed {\bf Tarski's
theorem} relies on one of these.   I spell it out because it is close to
Kawada's in spirit, though there are significant differences in the
ideas needed in the proof:

\inset{Let $X$ be a set and $G$ a subgroup of
$\Aut\Cal PX$.   Then the following are equiveridical:  (i) there is a
$G$-invariant additive functional $\theta:\Cal PX\to[0,1]$ such that
$\theta A=1$;   (ii) there are no
$A_0,\ldots,A_n$, $\pi_0,\ldots,\pi_n$, $\pi'_0,\ldots,\pi'_n$ such that
$A_0,\ldots,A_n$ are subsets of $X$ covering $X$,
$\pi_0,\ldots,\pi'_n$ all belong to $G$, and $\pi_0[A_0]$, $\pi'_0[A_0]$,
$\pi_1[A_1]$, $\pi'_1[A_1],\ldots,\pi'_n[A_n]$ are all disjoint.}

\noindent For a proof, see 449L in Volume 4;  for an illuminating
discussion of this
theorem, see {\smc Wagon 85}, Chapter 9.   But it seems to be unknown
whether the natural translation of this result is valid in all Dedekind
complete Boolean algebras (395Z).   Note that we are looking for
theorems which do not depend on any special properties of the group $G$
or the Boolean algebra $\frak A$.   For abelian or `amenable' groups, or
\wsid\ algebras, for instance, much more can be done, as described in
396Ya and \S449.

The methods of this section can, however, be used to prove similar
results for {\it countable} groups of automorphisms on Dedekind
$\sigma$-complete Boolean algebras;  I will return to such questions in
\S448.
The
presentation here owes a good deal to {\smc Nadkarni 90} and something
to {\smc Becker \& Kechris 96}.

As noted, Kawada ({\smc Kawada 44}) treated the case in which the group
$G$ of automorphisms is ergodic, that is, the fixed-point subalgebra
$\frak C$ is trivial.   Under this hypothesis the proof is of course
very much
simpler.   (You may find it useful to reconstruct the original version,
as suggested in 395Xa.)   I give the more general argument partly for
the sake of 395O, partly to separate out the steps which really need
ergodicity from those which depend only on
non-paradoxicality, partly to prepare the ground for the countable
version in the next volume, partly to show off the power of the
construction in \S364, and
partly to get you used to `Boolean-valued' arguments.   A bolder use of
language could indeed simplify some formulae slightly by writing (for
instance) $\Bvalue{k\high{a_0:a}<\low{1:a}}$ in place of
$\Bvalue{\low{1:a}-k\high{a_0:a}>0}$ (see part (f) of the proof of
395N).   As in \S388, the differences involved in the extension to
non-ergodic groups are, in a sense, just a matter of
technique;  but this time the technique is more obtrusive.   In \S556 of
Volume 5 I will try to explain a general approach to questions of this
kind, using metamathematical ideas.
}%end of notes

\discrpage

