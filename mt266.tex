\frfilename{mt266.tex}
\versiondate{28.1.09}
\copyrightdate{2004}

\def\chaptername{Change of variable in the integral}
\def\sectionname{The Brunn-Minkowski inequality}

\newsection{*266\dvAnew{2004}}

We now have most of the essential ingredients for a proof of the
Brunn-Minkowski inequality (266C) in a strong form.   I do not at
present expect to use it in this treatise, but it is one of the
basic results of geometric measure theory and from where we now stand is
not difficult, so I include it here.   The preliminary results on
arithmetic and geometric means (266A) and essential closures (266B) are
of great importance for other reasons.

\leader{266A}{Arithmetic and geometric \dvrocolon{means}}\cmmnt{ We
shall need the following standard result.

\medskip

\noindent}{\bf Proposition} If
$u_0,\ldots,u_n,p_0,\ldots,p_n\in\coint{0,\infty}$ and
$\sum_{i=0}^np_i=1$, then $\prod_{i=0}^nu_i^{p_i}\le\sum_{i=0}^np_iu_i$.

\proof{ Induce on $n$.   For $n=0$, $p_0=1$ the result is trivial.   If
$n=1$, then if $u_1=0$ the result is trivial (even if, as is standard in
this book, we interpret $0^0$ as $1$).   Otherwise, set
$t=\Bover{u_0}{u_1}$;  then

\Centerline{$t^{p_0}\le p_0t+1-p_0=p_0t+p_1$}

\noindent(as in part (a) of the proof of 244E), so

\Centerline{$u_0^{p_0}u_1^{p_1}=t^{p_0}u_1\le p_0tu_1+p_1u_1
=p_0u_0+p_1u_1$.}

\noindent For the inductive step to $n\ge 2$, if $p_0=\ldots=p_{n-1}=0$
the result is trivial.   Otherwise, set $q=p_0+\ldots+p_{n-1}=1-p_n$;
then

$$\eqalignno{\prod_{i=0}^nu_i^{p_i}
&=(\prod_{i=0}^{n-1}u_i^{p_i/q})^qu_n^{p_n}
\le(\sum_{i=0}^{n-1}\Bover{p_i}qu_i)^qu_n^{p_n}\cr
\displaycause{by the inductive hypothesis}
&\le q(\sum_{i=0}^{n-1}\Bover{p_i}qu_i)+p_nu_n\cr
\displaycause{by the two-term case just examined}
&=\sum_{i=0}^np_iu_i,\cr}$$

\noindent and the induction continues.
}%end of proof of 266A

\leader{266B}{Proposition} For any set $D\subseteq\BbbR^r$ set

\Centerline{$\clstar D=\{x:\limsup_{\delta\downarrow 0}
\Bover{\mu^*(D\cap B(x,\delta))}{\mu B(x,\delta)}>0\}$,}

\noindent where $\mu$ is Lebesgue measure on $\BbbR^r$.

(a) $D\setminus\clstar D$ is negligible.

(b) $\clstar D\subseteq\overline{D}$.

(c) $\clstar D$ is a Borel set.

(d) $\mu(\clstar D)=\mu^*D$.

(e) If $C\subseteq\Bbb R$ then
$\overline{C}+\clstar D\subseteq\clstar(C+D)$, writing $C+D$ for
$\{x+y:x\in C$, $y\in D\}$.

\proof{{\bf (a)} 261Da.

\medskip

{\bf (b)} If $x\in\BbbR^r\setminus\overline{D}$ then
$D\cap B(x,\delta)=\emptyset$ for all small $\delta$.

\medskip

{\bf (c)} The point is just that
$(x,\delta)\mapsto\mu^*(D\cap B(x,\delta))$ is continuous.   \Prf\ For
any $x$, $y\in\BbbR^r$ and $\delta$, $\eta\ge 0$ we have

$$\eqalignno{|\mu^*(D\cap B(y,\eta))-\mu^*(D\cap B(x,\delta))|
&\le\mu(B(y,\eta)\symmdiff B(x,\delta))\cr
&=2\mu(B(x,\delta)\cup B(y,\eta))-\mu B(x,\delta)-\mu B(y,\eta)\cr
&\le\beta_r\bigl(2(\max(\delta,\eta)+\|x-y\|)^r-\delta^r-\eta^r\bigr)\cr
\displaycause{where $\beta_r=\mu B(\tbf{0},1)$}
&\to 0\cr}$$

\noindent as $(y,\eta)\to(x,\delta)$.\ \QeD\  So

\Centerline{$x\mapsto
\limsup_{\delta\downarrow 0}
  \Bover{\mu^*(D\cap B(x,\delta))}{\mu B(x,\delta)}
=\inf_{\alpha\in\Bbb Q,\alpha>0}\sup_{\beta\in\Bbb Q,0<\beta\le\alpha}
  \Bover1{\beta_r\beta^r}\mu^*(D\cap B(x,\beta))$}

\noindent is Borel measurable, and

\Centerline{$\clstar D=\{x:\limsup_{\delta\downarrow 0}
  \Bover{\mu^*(D\cap B(x,\delta))}{\mu B(x,\delta)}>0\}$}

\noindent is a Borel set.

\medskip

{\bf (d)} By (c), $\mu(\clstar D)$ is defined;  by (a),
$\mu(\clstar D)\ge\mu^*D$.   On the other hand, let $E$ be a measurable
envelope of $D$ (132Ee);  then 261Db tells us that

\Centerline{$\limsup_{\delta\downarrow 0}
  \Bover{\mu^*(D\cap B(x,\delta))}{\mu B(x,\delta)}
\le\limsup_{\delta\downarrow 0}
  \Bover{\mu(E\cap B(x,\delta))}{\mu B(x,\delta)}
=0$}

\noindent for almost every $x\in\BbbR^r\setminus E$, so
$\clstar D\setminus E$ is negligible and

\Centerline{$\mu(\clstar D)\le\mu E=\mu^*D$.}

\medskip

{\bf (e)} If
$x\in\overline{C}$ and $y\in\clstar D$, set

\Centerline{$\gamma=\Bover13\limsup_{\delta\downarrow 0}
  \Bover{\mu^*(D\cap B(y,\delta))}{\mu B(y,\delta)}>0$.}

\noindent For any $\eta>0$, there is a $\delta\in\ocint{0,\eta}$ such
that $\mu^*(D\cap B(y,\delta))\ge 2\gamma\mu B(x,\delta)$.   Let
$\delta_1\in\coint{0,\delta}$ be such that
$\delta^r-\delta_1^r\le\gamma\delta^r$.   Then there is an $x'\in C$
such that $\|x-x'\|\le\delta-\delta_1$.  In this case,

$$\eqalign{\mu^*((C+D)\cap B(x+y,\delta))
&\ge\mu^*((x'+D)\cap B(x'+y,\delta_1))
=\mu^*(D\cap B(y,\delta_1))\cr
&\ge\mu^*(D\cap B(y,\delta))-\mu B(y,\delta)+\mu B(y,\delta_1)\cr
&\ge 2\beta_r\gamma\delta^r-\beta_r\delta^r+\beta_r\delta_1^r
\ge\beta_r\gamma\delta^r.\cr}$$

\noindent As $\eta$ is arbitrary,

\Centerline{$\limsup_{\delta\downarrow 0}
  \Bover{\mu^*((C+D)\cap B(x+y,\delta))}{\mu B(y,\delta)}
\ge\gamma$}

\noindent and $x+y\in\clstar(C+D)$;  as $x$ and $y$ are arbitrary,
$\overline{C}+\clstar D\subseteq\clstar(C+D)$.
}%end of proof of 266B

\cmmnt{\medskip

\noindent{\bf Remark} In this context, $\clstar D$ is called the {\bf
essential closure} of $D$.
}

\leader{266C}{Theorem} Let $A$, $B\subseteq\BbbR^r$ be non-empty sets,
where $r\ge 1$ is an integer.   If $\mu$ is Lebesgue measure on
$\BbbR^r$, and $A+B=\{x+y:x\in A$, $y\in B\}$, then
$\mu^*(A+B)^{1/r}\ge(\mu^*A)^{1/r}+(\mu^*B)^{1/r}$.

\proof{{\bf (a)} Consider first the case in which $A=\coint{a,a'}$ and
$B=\coint{b,b'}$ are half-open intervals.   In this case
$A+B=\coint{a+b,a'+b'}$;  writing $a=(\alpha_1,\ldots,\alpha_r)$, etc.,
as in \S115, set

\Centerline{$u_i
=\Bover{\alpha'_i-\alpha_i}{\alpha'_i+\beta'_i-\alpha_i-\beta_i}$,
\quad$v_i
=\Bover{\beta'_i-\beta_i}{\alpha'_i+\beta'_i-\alpha_i-\beta_i}$}

\noindent for each $i$.   Then we have

$$\eqalignno{(\mu A)^{1/r}+(\mu B)^{1/r}
&=\prod_{i=0}^r(\alpha'_i-\alpha_i)^{1/r}
  +\prod_{i=0}^r(\beta'_i-\beta_i)^{1/r}\cr
&=\mu(A+B)^{1/r}(\prod_{i=1}^ru_i^{1/r}+\prod_{i=1}^rv_i^{1/r})\cr
&\le\mu(A+B)^{1/r}(\Bover1r\sum_{i=1}^ru_i+\Bover1r\sum_{i=1}^rv_i)\cr
\displaycause{266A}
&=\mu(A+B)^{1/r}.\cr}$$

\medskip

{\bf (b)} Now I show by induction on $m+n$ that if
$A=\bigcup_{j=0}^mA_j$ and $B=\bigcup_{j=0}^nB_j$, where
$\langle A_j\rangle_{j\le m}$ and $\langle B_j\rangle_{j\le n}$ are both
disjoint families of non-empty half-open intervals, then
$\mu(A+B)^{1/r}\ge(\mu A)^{1/r}+(\mu B)^{1/r}$.   \Prf\ The induction
starts with the case $m=n=0$, dealt with in (a).   For the inductive
step to $m+n=l\ge 1$, one of $m$, $n$ is non-zero;  the argument is the
same in both cases;  suppose the former.
Since $A_0\cap A_1=\emptyset$, there must be some $j\le r$ and
$\alpha\in\Bbb R$ such that $A_0$ and $A_1$ are separated by the
hyperplane $\{x:\xi_j=\alpha\}$.   Set $A'=\{x:x\in A$, $\xi_j<\alpha\}$
and $A''=\{x:x\in A$, $\xi_j\ge\alpha\}$;  then both $A'$ and $A''$ are
non-empty and can be expressed as the union of at most $m-1$ disjoint
half-open intervals.   Set $\gamma=\Bover{\mu A'}{\mu A}\in\ooint{0,1}$.
The function $\beta\mapsto\mu\{x:x\in B$, $\xi_j<\beta\}$ is continuous,
so there is a $\beta\in\Bbb R$ such that $\mu B'=\gamma\mu B$, where
$B'=\{x:x\in B$, $\xi_j<\beta\}$;  set $B''=B\setminus B$.   Then $B'$
and $B''$ can be expressed as unions of at most $n$ half-open intervals.
By the inductive hypothesis,

\Centerline{$\mu(A'+B')^{1/r}\ge(\mu A')^{1/r}+(\mu B')^{1/r}$,
\quad$\mu(A''+B'')^{1/r}\ge(\mu A'')^{1/r}+(\mu B'')^{1/r}$.}

\noindent Now $A'+B'\subseteq\{x:\xi_j<\alpha+\beta\}$, while
$A''+B''\subseteq\{x:\xi_j\ge\alpha+\beta\}$.   So

$$\eqalign{\mu(A+B)
&\ge\mu(A'+B')+\mu(A''+B'')\cr
&\ge\bigl((\mu A')^{1/r}+(\mu B')^{1/r}\bigr)^r
  +\bigl((\mu A'')^{1/r}+(\mu B'')^{1/r}\bigr)^r\cr
&=\bigl((\gamma\mu A)^{1/r}+(\gamma\mu B)^{1/r}\bigr)^r
  +\bigl(((1-\gamma)\mu A)^{1/r}+((1-\gamma)\mu B)^{1/r}\bigr)^r\cr
&=(\bigl(\mu A)^{1/r}+(\mu B)^{1/r}\bigr)^r.\cr}$$

\noindent Taking $r$th roots,
$\mu(A+B)^{1/r}\ge(\mu A)^{1/r}+(\mu B)^{1/r}$ and the induction
proceeds.\ \Qed

\medskip

{\bf (c)} Now suppose that $A$ and $B$ are compact non-empty subsets of
$\BbbR^r$.   Then $\mu(A+B)^{1/r}\ge(\mu A)^{1/r}+(\mu B)^{1/r}$.
\Prf\ $A+B$ is compact (because $A\times B\subseteq\BbbR^r\times\BbbR^r$
is compact, being closed and bounded, and addition is continuous, so we
can use 2A2Eb).   Let $\epsilon>0$.   Let $G\supseteq A+B$ be an open set
such that $\mu G\le\mu(A+B)+\epsilon$ (134Fa);  then there is a
$\delta>0$ such that $B(x,2\delta)\subseteq G$ for every $x\in A+B$
(2A2Ed).   Let $n\in\Bbb N$ be such that
$2^{-n}\sqrt{r}\le\delta$, and let $A_1$ be the union of all the
half-open
intervals of the form $\coint{2^{-n}z,2^{-n}z+2^{-n}e}$ which meet $A$,
where $z\in\BbbZ^r$ and $e=(1,1,\ldots,1)$.  Then $A_1$ is a finite
disjoint union of half-open intervals, $A\subseteq A_1$ and every point
of $A_1$ is within a distance $\delta$ of some point of $A$.
Similarly, we can find a set $B_1$, a finite disjoint union of half-open
intervals, including $B$ and such that every point of $B_1$ is within
$\delta$ of some point of $B$.   But this means that every point of
$A_1+B_1$ is within a distance $2\delta$ of some point of $A+B$, and
belongs to $G$.   Accordingly

$$\eqalignno{(\mu(A+B)+\epsilon)^{1/r}
&\ge(\mu G)^{1/r}
\ge\mu(A_1+B_1)^{1/r}
\ge(\mu A_1)^{1/r}+(\mu B_1)^{1/r}\cr
\displaycause{by (b)}
&\ge(\mu A)^{1/r}+(\mu B)^{1/r}.\cr}$$

\noindent As $\epsilon$ is arbitrary,
$\mu(A+B)^{1/r}\ge(\mu A)^{1/r}+(\mu B)^{1/r}$.\ \Qed

\medskip

{\bf (d)} Next suppose that $A$, $B\subseteq\BbbR^r$ are Lebesgue
measurable.   Then

$$\eqalignno{(\mu A)^{1/r}+(\mu B)^{1/r}
&=\sup\{(\mu K)^{1/r}+(\mu L)^{1/r}:
  K\subseteq A\text{ and }L\subseteq B\text{ are compact}\}\cr
\displaycause{134Fb}
&\le\sup\{\mu(K+L)^{1/r}:
  K\subseteq A\text{ and }L\subseteq B\text{ are compact}\}\cr
\displaycause{by (c)}
&\le\mu^*(A+B)^{1/r}.\cr}$$

\medskip

{\bf (e)} For the penultimate step, 
suppose that $A$, $B\subseteq\BbbR^r$ have
non-zero outer Lebesgue measure.   Consider $\clstar A$, $\clstar B$ and
$\clstar(A+B)$ as defined in 266B.   Then $\clstar A$ and $\clstar B$ are
non-empty and their sum is included in
$\clstar(A+B)$, by 266Bb and 266Be.   So we have

$$\eqalignno{(\mu^*A)^{1/r}+(\mu^*B)^{1/r}
&=\mu(\clstar A)^{1/r}+\mu(\clstar B)^{1/r}\cr
\displaycause{266Bd}
&\le\mu^*(\clstar A+\clstar B)^{1/r}\cr
\displaycause{by (d) here}
&\le\mu(\clstar(A+B))^{1/r}
=\mu^*(A+B)^{1/r}.\cr}$$

\medskip

{\bf (f)} Finally, for arbitrary non-empty sets $A$,
$B\subseteq\BbbR^r$, note that if (for instance) $A$ is negligible then
we can take any $x\in A$ and see that

\Centerline{$\mu^*(A+B)^{1/r}\ge\mu^*(x+B)^{1/r}=(\mu B)^{1/r}
=(\mu^*A)^{1/r}+(\mu^*B)^{1/r}$,}

\noindent and the result is similarly trivial if $B$ is negligible.   So
all cases are covered.
}%end of proof of 266C

\exercises{\leader{266X}{Basic exercises (a)}
Let $D$, $D'$ be subsets of $\BbbR^r$.   Show that (i)
$\clstar(D\cup D')=\clstar D\cup\clstar D'$ (ii) $\clstar D=\clstar D'$
iff $D$ and $D'$ have a common measurable envelope
(iii) $\clstar D\setminus\clstar(\BbbR^r\setminus D')
\subseteq\clstar(D\cap D')$ (iv) $D$ is Lebesgue measurable iff
$\clstar D\cap\clstar(\BbbR^r\setminus D)$ is Lebesgue negligible (v)
$D\cup\clstar D$ is a measurable envelope of $D$
(vi) $\clstar(\clstar D)=\clstar D$.
%266B

\spheader 266Xb Show that, for a measurable set $E\subseteq\Bbb R$,
$\clstar E$ is just the set of real numbers which are not density points of
$\Bbb R\setminus E$.
%266B

\spheader 266Xc In 266C, show that if $A$ and $B$ are similar convex
sets in the same orientation then $A+B$ is a convex set similar to both
and $\mu(A+B)^{1/r}=(\mu A)^{1/r}+(\mu B)^{1/r}$.
%266C

%\leader{266Y}{Further exercises (a)}

}%end of exercises

\endnotes{
\Notesheader{266}
The proof of 266C is taken from {\smc Federer 69}.
%Is there a Hausdorff-measure version?
There is a slightly specious generality in the form given here.   If the
sets $A$ and $B$ are at all irregular, then $\mu^*(A+B)^{1/r}$ is likely
to be much greater than $(\mu^*A)^{1/r}+(\mu^*B)^{1/r}$.   The critical
case, in which $A$ and $B$ are similar convex sets, is much easier
(266Xc).   The theorem is therefore most useful when $A$ and $B$ are
non-similar convex sets and we get a non-trivial estimate which may be
hard to establish by other means.   For this case we do not need 266B.
Theorem 266C is an instructive example of the way in which the
dimension $r$ enters formulae when we seek results applying to general
Euclidean spaces.   There will be many more when I return to geometric
measure theory in Chapter 47 of Volume 4.
}%end of notes

\discrpage



