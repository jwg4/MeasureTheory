     
 Chapter 35:  Riesz spaces
     
\chapintrosection{26.1.02}{216}{}
     
\section{351}{Partially ordered linear spaces}{16.10.07}{216}{}
{Partially ordered linear spaces;  positive cones;  suprema and
infima;  positive linear operators;  order-continuous linear operators;
Riesz homomorphisms;  quotient spaces;  reduced powers;
representation of p.o.l.ss as subspaces of reduced powers of $\Bbb{R}$;
Archimedean spaces.}
     
\section{352}{Riesz spaces}{22.9.03}{222}{}
{Riesz spaces;  identities;  general distributive laws;  Riesz
homomorphisms;  Riesz subspaces;  order-dense subspaces and
order-continuous operators;   bands;  the algebra of complemented bands;
the algebra of projection bands;  principal bands;  $f$-algebras.}
     
\section{353}{Archimedean and Dedekind complete Riesz
spaces}{26.1.02}{233}{}
{Order-dense subspaces;  bands;  Dedekind ($\sigma$)-complete
spaces;  order-closed Riesz subspaces;  order units;  $f$-algebras.}
     
\section{354}{Banach lattices}{18.8.08}{240}{}
{Riesz norms;  Fatou norms;  the Levi property;  order-continuous
norms;  order-unit norms;  $M$-spaces;  are isomorphic to $C(X)$ for
compact Hausdorff $X$;  $L$-spaces;  uniform integrability in
$L$-spaces.}
     
\section{355}{Spaces of linear operators}{1.12.07}{249}{}{Order-bounded
linear
operators;  the space $\eurm{L}^{\sim}(U;V)$;
order-continuous operators;  extension of order-continuous operators;
the space $\eurm{L}^{\times}(U;V)$;  order-continuous norms.}
     
\section{356}{Dual spaces}{26.1.02}{257}{}
{The spaces $U^{\sim}$, $U^{\times}$, $U^*$;  biduals, embeddings
$U{\to}V^{\times}$ where $V{\subseteq}U^{\sim}$;  perfect Riesz spaces;
$L$- and $M$-spaces;  uniformly integrable sets in the dual of an
$M$-space;  relative weak compactness in $L$-spaces.}
     
\wheader{}{10}{4}{4}{100pt}
     
 Chapter 36:  Function spaces
     
\chapintrosection{2.2.02}{267}{}
     
\section{361}{$S$}{6.2.08}{268}{}
{Additive functions on Boolean rings;  the space $S(\frak{A})$;
universal mapping theorems for linear operators on $S$;  the map
$T_{\pi}:S(\frak{A}){\to}S(\frak{B})$ induced by a ring homomorphism
$\pi:\frak{A}\to\frak{B}$;  projection bands in $S(\frak{A})$;
identifying
$S(\frak{A})$ when $\frak{A}$ is a quotient of an algebra of sets.}
     
\section{362}{$S^{\sim}$}{31.12.10}{279}{}
{Bounded additive functionals on $\frak{A}$ identified with
order-bounded linear functionals on $S(\frak{A})$;   the $L$-space
$S^{\sim}$ and its bands;  countably additive, completely additive,
absolutely continuous and continuous functionals;  uniform integrability
in $S^{\sim}$.}
     
\section{363}{$L^{\infty}$}{4.3.08}{289}{}
{The space $L^{\infty}(\frak{A})$, as an $M$-space and $f$-algebra;
universal mapping theorems for linear operators on $L^{\infty}$;
$T_{\pi}:L^{\infty}(\frak{A}){\to}L^{\infty}(\frak{B})$;  representing
$L^{\infty}$ when $\frak{A}$ is a quotient of an algebra of sets;
integrals with respect to finitely additive functionals;
projection bands in $L^{\infty}$;  $(L^{\infty})^{\sim}$ and its bands;
Dedekind completeness of $\frak{A}$ and $L^{\infty}$;  representing
$\sigma$-complete $M$-spaces;   the generalized Hahn-Banach theorem;
the Banach-Ulam problem.}
     
\section{364}{$L^0$}{16.7.11}{303}{}
{The space $L^0(\frak{A})$;  representing $L^0$ when $\frak{A}$ is a
quotient of a $\sigma$-algebra of sets;  algebraic operations on
$L^0$;  action of Borel measurable functions on $L^0$;  identifying
$L^0(\frak{A})$ with $L^0(\mu)$ when $\frak{A}$ is a measure algebra;
embedding $S$ and $L^{\infty}$ in $L^0$;  suprema and infima in $L^0$;
Dedekind completeness in $\frak{A}$ and $L^0$;  multiplication in $L^0$;
projection bands;  $T_{\pi}:L^0(\frak{A}){\to}L^0(\frak{B})$;  
when $\pi$ represented by a $(\Tau,\Sigma)$-measurable function;  simple
products;  *regular open algebras;  *the space $C^{\infty}(X)$.}
     
\section{365}{$L^1$}{26.5.03}{323}{}
{The space $L^1(\frak{A},\bar\mu)$;  identification with
$L^1(\mu)$;  $\int_au$;  the Radon-Nikod\'ym theorem again;
${\int}w{\times}T_{\pi}u\,d\bar\nu={\int}u\,d\bar\mu$;  additive functions on $\frak{A}$ and linear operators
on $L^1$;  the duality between $L^1$ and $L^{\infty}$;  $T_{\pi}:L^1(\frak{A},\bar\mu){\to}L^1(\frak{B},\bar\nu)$ and
$P_{\pi}:L^1(\frak{B},\bar\nu){\to}L^1(\frak{A},\bar\mu)$;  conditional
expectations;  bands in $L^1$;  varying $\bar\mu$.}
     
\section{366}{$L^p$}{10.11.08}{339}{}
{The spaces $L^p(\frak{A},\bar\mu)$;  identification with
$L^p(\mu)$;  $L^q$ as the dual of $L^p$;  the spaces $M^0$ and
$M^{1,0}$;  $T_{\pi}:M^0(\frak{A},\bar\mu){\to}M^0(\frak{B},\bar\nu)$
and
$P_{\pi}:M^{1,0}(\frak{B},\bar\nu){\to}M^{1,0}(\frak{A},\bar\mu)$;
conditional expectations;  the case $p=2$;  spaces 
$L^p_{\Bbb{C}}(\frak{A},\bar\mu)$.}
     
\section{367}{Convergence in measure}{18.9.03}{349}{}
{Order*-convergence of sequences in lattices;  in Riesz spaces;
in Banach lattices;  in quotients of spaces of measurable functions;  in
$C(X)$;   Lebesgue's Dominated Convergence Theorem and Doob's
Martingale Theorem;  convergence in measure in $L^0(\frak{A})$;  and
pointwise convergence;  defined by the Riesz space structure;  positive
linear operators on $L^0$;  convergence in measure and the canonical
projection $(L^1)^{**}{\to}L^1$;  the set of independent families of random variables.}
     
\section{368}{Embedding Riesz spaces in $L^0$}{16.9.09}{365}{}
{Extension of order-continuous Riesz homomorphisms into $L^0$;
representation of Archimedean Riesz spaces as subspaces of $L^0$;
Dedekind completion of Riesz spaces;  characterizing $L^0$ spaces as
Riesz spaces;  \wsid\ Riesz spaces.}
     
\section{369}{Banach function spaces}{5.11.03}{375}{}
{Riesz spaces separated by their order-continuous duals;
representing $U^{\times}$ when $U{\subseteq}L^0$;  Kakutani's
representation of $L$-spaces as $L^1$ spaces;  extended Fatou norms;
associate norms;  $L^{\tau'}\cong(L^{\tau})^{\times}$;  Fatou norms and
convergence in measure;  $M^{\infty,1}$ and $M^{1,\infty}$,
$\|\,\|_{\infty,1}$ and $\|\,\|_{1,\infty}$;  $L^{\tau_1}+L^{\tau_2}$.}
     
\wheader{}{10}{4}{4}{100pt}
     
 Chapter 37:  Linear operators between function spaces
     
\chapintrosection{10.2.02}{390}{}
     
\section{371}{The Chacon-Krengel theorem}{13.12.06}{390}{}
{$\eurm{L}^{\sim}(U;V)=\eurm{L}^{\times}(U;V)=\eurm{B}(U;V)$ for
$L$-spaces
$U$ and $V$;  the class $\Cal{T}^{(0)}_{\bar\mu,\bar\nu}$ of
$\|\,\|_1$-decreasing, $\|\,\|_{\infty}$-decreasing linear operators
from $M^{1,0}(\frak{A},\bar\mu)$ to $M^{1,0}(\frak{B},\bar\nu)$.}
     
\section{372}{The ergodic theorem}{7.12.08}{394}{}
{The Maximal Ergodic Theorem and the Ergodic Theorem for operators
in $\Cal{T}^{(0)}_{\bar\mu,\bar\mu}$;  for \imp\ functions
$\phi:X{\to}X$;
limit operators as conditional expectations;  applications to continued
fractions;  mixing and ergodic transformations.}
     
\section{373}{Decreasing rearrangements}{10.2.02}{410}{}{The classes
$\Cal{T}$, $\Cal{T}^{\times}$;  the space
$M^{0,\infty}$;  decreasing rearrangements $u^*$;  $\|u^*\|_p=\|u\|_p$;
$\int|Tu{\times}v|\le{\int}u^*{\times}v^*$ if $T\in\Cal{T}$;  the very
weak
operator topology and compactness of $\Cal{T}$;  $v$ is expressible as
$Tu$, where $T\in\Cal{T}$, iff $\int_0^tv^*\le\int_0^tu^*$ for every
$t$;   finding $T$ such that ${\int}Tu{\times}v={\int}u^*{\times}v^*$;
the adjoint operator from $\Cal{T}^{(0)}_{\bar\mu,\bar\nu}$ to
$\Cal{T}^{(0)}_{\bar\nu,\bar\mu}$.}
     
\section{374}{Rearrangement-invariant spaces}{15.6.09}{427}{}
{$\Cal{T}$-invariant subspaces of $M^{1,\infty}$, and
$\Cal{T}$-invariant extended Fatou norms;  relating $\Cal{T}$-invariant
norms on different spaces;  rearrangement-invariant sets and norms;
when rearrangement-invariance implies $\Cal{T}$-invariance.}
     
\section{375}{Kwapien's theorem}{30.1.10}{437}{}
{Linear operators on $L^0$ spaces;  if $\frak{B}$ is measurable, a
positive linear operator from $L^0(\frak{A})$ to $L^0(\frak{B})$ can be
assembled from Riesz homomorphisms.}
     
\section{376}{Integral operators}{8.4.10}{443}{}
{Kernel operators;  free products of measure algebras and tensor
products of $L^0$ spaces;  tensor products of $L^1$ spaces;  abstract
integral operators (i) as a band in $\eurm{L}^{\times}(U;V)$ (ii)
represented by kernels belonging to $L^0(\frak{A}\tensorhat\frak{B})$
(iii) as operators converting weakly convergent sequences into
order*-convergent sequences;  operators into $M$-spaces or
out of $L$-spaces.}
     
\section{377}{Function spaces of reduced products}{30.12.09}{459}{}
{Measure-bounded Boolean homomorphisms on products of probability algebras;
associated maps on subspaces of $\prod_{i\in{I}}L^0(\frak{A}_i)$ and     
$\prod_{i\in{I}}L^p(\frak{A}_i)$;  reduced powers;
universal mapping theorems for
function spaces on projective and inductive limits of probability
algebras.}

\wheader{}{10}{4}{4}{100pt}
     
 Chapter 38:  Automorphisms
     
\chapintrosection{15.8.08}{472}{}
     
\section{381}{Automorphisms of Boolean algebras}{19.7.06}{472}{}
{Assembling an automorphism;  elements supporting
an automorphism;  periodic and aperiodic parts;  
full and countably full subgroups;  recurrence;  
induced automorphisms of principal ideals;  Stone spaces;  
cyclic automorphisms.}

\section{382}{Factorization of automorphisms}{15.8.06}{484}{}
{Separators and transversals;  Frol\'{\i}k's theorem;  
exchanging involutions;  expressing an automorphism as the product of 
three involutions;  
subgroups of $\Aut\frak{A}$ with many involutions;  normal
subgroups of full groups with many involutions;  simple 
automorphism groups.}
     
\section{383}{Automorphism groups of measure algebras}{24.10.03}{498}{}
{Measure-preserving automorphisms as products of involutions;
normal subgroups of $\Aut\frak{A}$ and $\Aut_{\bar\mu}\frak{A}$;
conjugacy in $\Aut\frak{A}$ and $\Aut_{\bar\mu}\frak{A}$.}
     
\section{384}{Outer automorphisms}{10.6.03}{502}{}
{If $G\le\Aut\frak{A}$, $H\le\Aut\frak{B}$ have many involutions,
any isomorphism between $G$ and $H$ arises from an isomorphism between
$\frak{A}$ and $\frak{B}$;  if $\frak{A}$ is nowhere rigid,
$\Aut\frak{A}$
has no outer automorphisms;  applications to localizable measure
algebras.}

\section{385}{Entropy}{21.10.03}{511}{}
{Entropy of a partition of unity in a probability algebra;
conditional entropy;  entropy of a measure-preserving homomorphism;
calculation of entropy (Kolmogorov-Sina\v{\i} theorem);  Bernoulli
shifts;  isomorphic homomorphisms and conjugacy classes in
$\Aut_{\bar\mu}\frak{A}$;  almost isomorphic \imp\ functions.}
     
\section{386}{More about entropy}{17.11.03}{524}{}
{The Halmos-Rokhlin-Kakutani lemma;   
the Shannon-McMillan-Breiman theorem;  various lemmas.}
     
\section{387}{Ornstein's theorem}{18.12.03}{534}{}
{Bernoulli partitions;  finding Bernoulli partitions with elements
of given measure (Sina\v{\i}'s theorem);  adjusting Bernoulli
partitions;
Ornstein's theorem (Bernoulli shifts of the same finite entropy are
isomorphic);  Ornstein's and Sina\v{\i}'s theorems in the case of
infinite entropy.}
     
\section{388}{Dye's theorem}{14.1.04}{555}{}
{Orbits of \imp\ functions;  von Neumann transformations;
von Neumann transformations generating a given full subgroup;
classification of full subgroups generated by a single automorphism.}
     
\wheader{}{10}{4}{4}{100pt}
     
 Chapter 39:  Measurable algebras
     
\chapintrosection{13.8.98}{568}{}
     
\section{391}{Kelley's theorem}{5.9.07}{568}{}
{Measurable algebras;  strictly positive additive functionals and
weak $(\sigma,\infty)$-distributivity;  additive functionals
subordinate to or dominating
a given functional;  intersection numbers;   existence of strictly
positive additive functionals.}
     
\section{392}{Submeasures}{11.2.08}{573}{}
{Submeasures;  exhaustive, uniformly exhaustive and Maharam
submeasures;  the Kalton-Roberts
theorem (a strictly positive uniformly exhaustive submeasure
provides a strictly positive additive functional);  strictly positive
submeasures, associated metrics and metric completions of algebras;
products of submeasures.}
     
\section{393}{Maharam algebras}{11.5.08}{580}{}
{Maharam submeasures;  Maharam algebras;  topologies on Boolean algebras;
order-sequential topologies;  characterizations of Maharam algebras.}

\section{394}{Talagrand's example}{13.6.11}{590}{}
{An exhaustive submeasure which is not uniformly exhaustive;  a 
non-measurable Maharam algebra;  control measures.}
     
\section{395}{Kawada's theorem}{15.6.08}{601}{}{Full local semigroups;
$\tau$-equidecomposability;  fully
non-paradoxical subgroups of $\Aut\frak{A}$;  \hbox{$\low{b:a}$} and
$\high{b:a}$;  invariant additive functions from $\frak{A}$ to
$L^{\infty}(\frak{C})$, where $\frak{C}$ is the fixed-point subalgebra of 
a group;  invariant additive functionals and measures;  ergodic fully
non-paradoxical groups.}
     
\section{396}{The Hajian-Ito theorem}{15.8.08}{614}{}
{Invariant measures on measurable algebras;  weakly wandering
elements.}
     
\wheader{}{10}{4}{4}{100pt}
     
 Appendix to Volume 3
     
\chapintrosection{2.3.02}{617}{}
     
\section{3A1}{Set theory}{31.10.07}{617}{}
{Calculation of cardinalities;   cofinal sets,
cofinalities;  notes on the use of Zorn's Lemma;  the natural numbers as
finite ordinals;  lattice homomorphisms;  the Marriage Lemma.}
     
\section{3A2}{Rings}{22.11.07}{619}{}
{Rings;  subrings, ideals, homomorphisms, quotient rings, the
First Isomorphism Theorem;  products.}
     
\section{3A3}{General topology}{14.12.07}{622}{}
{Hausdorff, regular, completely regular, zero-dimensional,
extremally disconnected, compact and locally compact spaces;  continuous
functions;  dense subsets;  meager sets;  Baire's theorem for locally
compact spaces;  products;  Tychonoff's theorem;  the usual topologies
on $\{0,1\}^I$, $\BbbR^I$;  cluster points of filters;  topology bases;
uniform convergence of sequences of functions;  one-point
compactifications;  topologies defined from sequential convergences.}
     
\section{3A4}{Uniformities}{30.1.08}{626}{}
{Uniform spaces;  and pseudometrics;  uniform continuity;
subspaces;  product uniformities;  Cauchy filters and completeness;
extending uniformly continuous functions;  completions.}
     
\section{3A5}{Normed spaces}{22.5.11}{628}{}
{The Hahn-Banach theorem in analytic and geometric forms;  cones
and convex sets;  weak and weak* topologies;  reflexive spaces;
Uniform Boundedness Theorem;  strong operator topologies;
completions;  normed algebras;  compact
linear operators;  Hilbert spaces;  bounded sets in linear topological
spaces.}
     
\section{3A6}{Groups}{6.8.08}{631}{}
{Involutions;  inner and outer automorphisms;  normal subgroups.}
     
\medskip
     
% Concordance to part II \pagereference{424}{}
     
\medskip
     
References for Volume 3 \vtmpb{2.3.02}\pagereference{633}{}
     
     
% \medskip
     
% Index to Volumes 1-3
     
% \qquad Principal topics and results \pagereference{637}{}
     
% \qquad General index \pagereference{644}{}
     
%468 pp
