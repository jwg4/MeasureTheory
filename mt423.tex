\frfilename{mt423.tex}
\versiondate{13.8.05}
\copyrightdate{2002}

\def\NN{\BbbN^{\Bbb N}}

\def\chaptername{Descriptive set theory}
\def\sectionname{Analytic spaces}

\newsection{423}

We come now to the original class of K-analytic spaces, the `analytic'
spaces.   I define these as continuous images of $\NN$ (423A), but move
as quickly as possible to their characterization as K-analytic spaces
with countable networks (423C), so that many other fundamental facts
(423E-423G) can be regarded as simple corollaries of results in \S422.
I give two versions of Lusin's theorem on injective images of Borel sets
(423I), and a form of the von Neumann-Jankow measurable selection
theorem (423N).   I end with notes on constituents of coanalytic sets
(423P-423Q).

\leader{423A}{Definition} A Hausdorff space is {\bf
analytic} or {\bf Souslin} if it is either empty or a continuous image
of $\NN$.

\leader{423B}{Proposition} (a) A Polish space is analytic.

(b) A Hausdorff continuous image of an analytic Hausdorff space is
analytic.

(c) A product of countably many analytic Hausdorff spaces is analytic.

(d) A closed subset of an analytic Hausdorff space is analytic.

(e) An analytic Hausdorff space has a countable network consisting of
analytic sets.

\proof{{\bf (a)} Let $X$ be a Polish space.   If $X=\emptyset$, we can
stop.   Otherwise, let $\rho$ be a metric on $X$, inducing its topology,
under which $X$ is complete.   For
$\sigma\in S=\bigcup_{n\in\Bbb N}\BbbN^n$ choose
$X_{\sigma}\subseteq X$ as follows.
$X_{\emptyset}=X$. Given that $X_{\sigma}$ is a closed non-empty subset
of $X$, where $\sigma\in S$, then $X_{\sigma}$ is
separable, because $X$ is separable and metrizable
(4A2P(a-iv)), and we can choose a sequence $\sequence{i}{x_{\sigma i}}$
in $X_{\sigma}$ such that $\{x_{\sigma i}:i\in\Bbb N\}$ is dense in
$X_{\sigma}$.   Set
$X_{\sigma^{\smallfrown}\fraction{i}}=X_{\sigma}\cap B(x_{\sigma i},2^{-n})$ for
each $i\in\Bbb N$, where
$B(x,\delta)=\{y:\rho(y,x)\le\delta\}$, and continue.   Note that
because $\{x_{\sigma i}:i\in\Bbb N\}$ is dense in $X_{\sigma}$,
$X_{\sigma}=\bigcup_{i\in\Bbb N}X_{\sigma^{\smallfrown}\fraction{i}}$, for every
$\sigma\in S$.

For each $\phi\in\NN$, $\sequence{n}{X_{\phi\restr n}}$ is a
non-increasing sequence of non-empty closed sets, and
$\diam(X_{\phi\restr n+1})\penalty-100
\le 2^{-n+1}$ for every $n$.   Because $X$ is
complete under
$\rho$, $\bigcap_{n\in\Bbb N}X_{\phi\restr n}$ is a singleton
$\{f(\phi)\}$ say. ($f(\phi)$ is the limit of the Cauchy sequence
$\sequencen{x_{\phi\restr n,\phi(n)}}$.)   Thus we have a function
$f:\NN\to X$.   $f$ is
continuous because $\rho(f(\psi),f(\phi))\le 2^{-n+1}$ whenever
$\phi\restr n+1=\psi\restr n+1$ (since in this case both $f(\psi)$ and
$f(\phi)$ belong to $X_{\phi\restr n+1}$).   $f$ is surjective because,
given $x\in X$, we can choose $\sequence{i}{\phi(i)}$ inductively so
that $x\in X_{\phi\restr n}$ for every $n$;  at the inductive step, we
have $x\in X_{\phi\restr n}
=\bigcup_{i\in\Bbb N}X_{(\phi\restr n)^{\smallfrown}\fraction{i}}$, so we can
take $\phi(n)$ such that
$x\in X_{(\phi\restr n)^{\smallfrown}\fraction{\phi(n)}}
=X_{\phi\restr n+1}$.

Thus $X$ is a continuous image of $\NN$, as claimed.

\medskip

{\bf (b)} If $X$ is an analytic Hausdorff space and $Y$ is a Hausdorff
continuous image of
$X$, then either $X$ is a continuous image of $\NN$ and $Y$ is a
continuous image of $\NN$, or $X=\emptyset$ and $Y=\emptyset$.

\medskip

{\bf (c)} Let $\familyiI{X_i}$ be a countable family of analytic
Hausdorff spaces,
with product $X$.   Then $X$ is Hausdorff (3A3Id).   If $I=\emptyset$
then $X=\{\emptyset\}$ is a
continuous image of $\NN$, therefore analytic.   If there is some $i\in
I$ such that $X_i=\emptyset$, then $X=\emptyset$ is analytic.
Otherwise, we have for
each $i\in I$ a continuous surjection $f_i:\NN\to X_i$.   Setting
$f(\phi)=\familyiI{f_i(\phi(i))}$ for $\phi\in(\NN)^I$, $f:(\NN)^I\to X$
is a continuous surjection.   But $(\NN)^I\cong\BbbN^{\Bbb N\times I}$
is homeomorphic to $\NN$, so
$X$ is analytic.

\medskip

{\bf (d)} Let $X$ be an analytic Hausdorff space and $F$ a closed subset
of $X$.   Then $F$ is Hausdorff in its subspace topology (4A2F(a-i)).
If $X=\emptyset$ then $F=\emptyset$ is analytic.   Otherwise, there is a
continuous surjection $f:\NN\to X$.   Now $H=f^{-1}[F]$ is a closed
subset of the Polish space $\NN$, therefore Polish in its induced
topology (4A2Qd).   By (a), $H$ is analytic, so its continuous
image $F=f[H]$ also is analytic, by (b).

\medskip

{\bf (e)} Let $X$ be an analytic Hausdorff space.   If it is empty then
of course it has a countable network consisting of analytic sets.
Otherwise, there is a continuous surjection $f:\NN\to X$.   For
$\sigma\in S$ set $I_{\sigma}=\{\phi:\sigma\subseteq\phi\in\NN\}$;
then $\{I_{\sigma}:\sigma\in S\}$ is a base for the topology of $\NN$,
so $\{f[I_{\sigma}]:\sigma\in S\}$ is a network for the topology of
$X$ (see the proof of 4A2Nd).   But $I_{\sigma}$ is homeomorphic to
$\NN$, so $f[I_{\sigma}]$ is analytic, for every $\sigma\in S$, and
$\{f[I_{\sigma}]:\sigma\in S\}$ is a countable network consisting of
analytic sets.
}%end of proof of 423B

\leader{423C}{Theorem} A Hausdorff space is analytic iff it is
K-analytic and has a countable network.

\proof{{\bf (a)} Let $X$ be an analytic Hausdorff space.   By 423Be, it
has a countable network.   If $X=\emptyset$ then
surely it is K-analytic.   Otherwise, $X$ is
a continuous image of $\NN$.   But $\NN$ is K-analytic (422Gb),  so
$X$ also is K-analytic, by 422Gd.

\medskip

{\bf (b)} Now suppose that $X$ is a K-analytic Hausdorff space and has a
countable network.

\medskip

\quad{\bf (i)} If $X\subseteq\NN$ then $X$ is analytic.   \Prf\ Let
$R\subseteq\NN\times X$ be an usco-compact relation such that
$R[\NN]=X$.   Then $R$ is still usco-compact when regarded as a subset
of $\NN\times\NN$ (422Dg), so is closed in $\NN\times\NN$ (422Da).   But
$\NN\times\NN\cong\NN$ is analytic, so $R$ is in itself an analytic
space (423Bd), and its continuous image $X$ is analytic, by 423Bb.\ \Qed

\medskip

\quad{\bf (ii)} Now suppose that $X$ is regular.   By 4A2Ng, $X$ has a
countable network $\Cal E$ consisting of closed sets.   Adding
$\emptyset$ to $\Cal E$ if need be, we may suppose that $\Cal
E\ne\emptyset$.   Let
$\sequencen{E_n}$ be a sequence running over $\Cal E$.   For each
$n\in\Bbb N$, let $\sequence{i}{F_{ni}}$ be a sequence running over
$\{E_n\}\cup\{E:E\in\Cal E,\,E\cap E_n=\emptyset\}$.   Now consider the
relation

\Centerline{$R=\{(\phi,x):\phi\in\NN,\,x\in\bigcap_{n\in\Bbb
N}F_{n,\phi(n)}\}\subseteq\NN\times X$.}

\medskip

\qquad\grheada\ $R$ is closed in $\NN\times X$.   \Prf\ Because every
$F_{ni}$ is closed,

\Centerline{$(\NN\times X)\setminus R
=\bigcup_{i,n\in\Bbb N}\{(\phi,x)$:  $\phi(n)=i$ and $x\notin F_{ni}\}$}

\noindent is open.\ \Qed

\medskip

\qquad\grheadb\ $R[\NN]=X$.   \Prf\  For every $n\in\Bbb N$,

\Centerline{$X\setminus E_n=\bigcup\{E:E\in\Cal E,\,E\subseteq
X\setminus E_n\}$}

\noindent because $\Cal E$ is a network and $E_n$ is closed, so
$\bigcup_{i\in\Bbb N}F_{ni}=X$.   So, given $x\in X$, we can find for
each $n$ a $\phi(n)$ such that $x\in F_{n,\phi(n)}$, and $(\phi,x)\in
R$.\ \Qed

\medskip

\qquad\grheadc\ $R$ is the graph of a function.   \Prf\Quer\ Suppose
that we have $(\phi,x)$ and $(\phi,y)$ in $R$ where $x\ne y$.   Because
the topology of $X$ is Hausdorff, there is an $n\in\Bbb N$ such that
$x\in E_n$ and $y\notin E_n$.   But in this case $x\in E_n\cap
F_{n,\phi(n)}$, so $F_{n,\phi(n)}=E_n$, while $y\in
F_{n,\phi(n)}\setminus E_n$, so
$F_{n,\phi(n)}\ne E_n$;  which is absurd.\ \Bang\Qed

\medskip

\qquad\grheadd\ Set $A=R^{-1}[X]$, so that $R$ is the graph of a
function from $A$ to $X$;  in recognition of its new status, give it a
new name $f$.   Then $f$ is continuous.   \Prf\ Suppose that $\phi\in A$
and that $x=f(\phi)\in G$, where $G\subseteq X$ is open.   Then there is
an $n\in\Bbb N$ such that $x\in E_n\subseteq G$.   In this case $x\in
F_{n,\phi(n)}$, because $(\phi,x)\in R$, so $F_{n,\phi(n)}=E_n$.   Now
if $\psi\in A$ and $\psi(n)=\phi(n)$, we must have

\Centerline{$f(\psi)\in F_{n,\psi(n)}=E_n\subseteq G$.}

\noindent Thus $f^{-1}[G]$ includes a neighbourhood of $\phi$ in $A$.
As $\phi$ and $G$ are arbitrary, $f$ is continuous.\ \Qed

\medskip

\qquad\grheade\ At this point recall that $X$ is K-analytic.   It
follows that $\NN\times X$ is K-analytic (422Ge), so that its closed
subset $R$ is K-analytic (422Gf) and $A$, which is a continuous image of
$R$, is K-analytic (422Gd).   But now $A$ is a K-analytic subset of
$\NN$, so is analytic, by (i) just above.   And, finally, $X$ is a
continuous image of $A$, so is analytic.

\medskip

\quad{\bf (iii)} Thus any regular K-analytic space with a countable
network is analytic.  Now suppose that $X$ is an arbitrary K-analytic
Hausdorff space with a countable network.   Let $R\subseteq\NN\times X$
be an usco-compact relation such that $R[\NN]=X$.   Then $R$ is a closed
subset of $\NN\times X$, so is itself a K-analytic space with a
countable network.   But it is also regular, by 422E.   So $R$ is
analytic and its continuous image $X$ is analytic.

This completes the proof.
}%end of proof of 423C

\leader{423D}{Corollary} (a) An analytic Hausdorff space is
hereditarily Lindel\"of.

(b) In a regular analytic Hausdorff space, closed sets are zero sets
and the Baire and Borel $\sigma$-algebras coincide.
%4@35

(c) A compact subset of an analytic Hausdorff space is metrizable.

(d) A metrizable space is analytic iff it is K-analytic.

\proof{{\bf (a)-(c)} These are true just because there is a countable
network (4A2Nb, 4A3Kb, 4A2Na, 4A2Qh).

\medskip

{\bf (d)} Let $X$ be a metrizable space.   If $X$ is analytic, of course
it is K-analytic.   If $X$ is K-analytic, it is Lindel\"of (422Gg)
therefore separable (4A2Pd) and has a countable network (4A2P(a-iii)),
so is analytic.
}%end of proof of 423D

\leader{423E}{Theorem} (a) For any Hausdorff space $X$, the family of
subsets of $X$ which are analytic in their subspace topologies is closed
under Souslin's operation.

(b) Let $(X,\frak T)$ be an analytic Hausdorff space.   For a subset $A$
of $X$, the following are equiveridical:

\inset{(i) $A$ is analytic;

(ii) $A$ is K-analytic;

(iii) $A$ is Souslin-F;

(iv) $A$ can be obtained by Souslin's operation from the family
of Borel subsets of $X$.}

\noindent In particular, all Borel sets in $X$ are analytic.

\proof{{\bf (a)} Let $X$ be a Hausdorff space and $\Cal A$ the family of
analytic subsets of $X$.   Let
$\family{\sigma}{S^*}{A_{\sigma}}$ be a Souslin scheme in $\Cal A$ with
kernel $A$.   Then every $A_{\sigma}$ is
K-analytic, so $A$ is K-analytic, by 422Hc.   Also every $A_{\sigma}$
has a countable network, so $A'=\bigcup_{\sigma\in S^*}A_{\sigma}$ has a
countable network (4A2Nc);  as $A\subseteq A'$, $A$ also has a countable
network (4A2Na) and is analytic.

\medskip

{\bf (b)} Because $X$ has a countable network, so does $A$.   So 423C
tells us at once that (i)$\Leftrightarrow$(ii).   In particular, $X$ is
K-analytic, so 422Hb tells us that (ii)$\Leftrightarrow$(iii).   Of course
(iii)$\Rightarrow$(iv).

Now suppose that $G\subseteq X$ is open.   Then $G\in\Cal A$.   \Prf\
If $X=\emptyset$ then $G=\emptyset$ is open.   Otherwise, there is a
continuous surjection $f:\NN\to X$.   Set $H=f^{-1}[G]$, so that
$H\subseteq\NN$ is open and $G=f[H]$.   Being an open set in a metric
space, $H$ is F$_{\sigma}$ (4A2Lc), so, in particular, is
Souslin-F;  but $\NN$ is analytic, so $H$ is analytic and its continuous
image $G$ is analytic.\ \Qed

We have already seen that closed subsets of $X$ belong to $\Cal A$
(423Bd).   Because $\Cal A$ is closed under Souslin's operation, it
contains every Borel set, by 421F.   It therefore contains every set
obtainable by Souslin's operation from Borel sets, and
(iv)$\Rightarrow$(i).
}%end of proof of 423E

\cmmnt{\medskip

\noindent{\bf Remark} See also 423Yb below.}

\leader{423F}{Proposition} Let $(X,\frak T)$ be an analytic Hausdorff
space.

(a) A set $E\subseteq X$ is Borel iff both $E$ and $X\setminus E$ are
analytic.

(b) If $\frak S$ is a coarser\cmmnt{ (= smaller)} Hausdorff topology on
$X$, then $\frak S$ and $\frak T$ have the same Borel sets.

\proof{{\bf (a)} If $E$ is Borel, then $E$ and $X\setminus E$ are
analytic, by 423Eb.   If $E$ and $X\setminus E$ are analytic, they are
K-analytic (423Eb) and disjoint, so there is a Borel set $F\supseteq E$
which is disjoint from $X\setminus E$ (422J);   but now of course
$F=E$, so $E$ must be Borel.

\medskip

{\bf (b)} Because the identity map from $(X,\frak T)$ to $(X,\frak S)$
is continuous, $\frak S$ is an analytic topology (423Bb) and every
$\frak S$-Borel set is $\frak T$-Borel.   If $E\subseteq X$ is
$\frak T$-Borel, then it and its complement are $\frak T$-analytic,
therefore $\frak S$-analytic (423Bb), and $E$ is $\frak S$-Borel by (a).
}%end of proof of 423F

\vleader{60pt}{423G}{Lemma} Let $X$ and $Y$ be analytic Hausdorff spaces and
$f:X\to Y$ a Borel measurable function.

(a)\cmmnt{ (The graph of)} $f$ is an analytic set.

(b) $f[A]$ is an analytic set in $Y$ for any analytic set\cmmnt{ (in
particular, any Borel set)} $A\subseteq X$.

(c) $f^{-1}[B]$ is an analytic set in $X$ for any analytic
set\cmmnt{ (in particular, any Borel set)} $B\subseteq Y$.

\proof{{\bf (a)} Let $\Cal E$ be a countable network for the topology of
$Y$.   Set

\Centerline{$R=\bigcap_{E\in\Cal E}(X\times\overline
E)\cup((X\setminus f^{-1}[\overline{E}])\times Y)$.}

\noindent Then $R$ is a Borel set in $X\times Y$.   But also $R$ is the
graph of $f$.   \Prf\ If $f(x)=y$, then surely $y\in\overline{E}$
whenever $x\in f^{-1}[\overline{E}]$, so $(x,y)\in R$.   On the other
hand, if $x\in X$, $y\in Y$ and $f(x)\ne y$, there are disjoint open
sets $G$, $H\subseteq Y$ such that $f(x)\in G$ and $y\in H$;  now there
is an $E\in\Cal E$ such that $f(x)\in E\subseteq G$, so that
$f(x)\in\overline{E}$ but
$y\notin\overline{E}$, and $(x,y)\notin R$.\ \Qed

Because $X\times Y$ is analytic (423Bc), $R$ is analytic (423Eb).

\medskip

{\bf (b)} If $A\subseteq X$ is analytic, then $A\times Y$ and
$R\cap(A\times Y)$ are analytic (423Ea), so $f[A]=R[A]$, which is a
continuous image of $R\cap(A\times Y)$, is analytic.

\medskip

{\bf (c)} Similarly, if $B\subseteq Y$ is analytic, then $f^{-1}[B]$ is
a continuous image of $R\cap(X\times B)$, so is analytic.
}%end of proof of 423G

\leader{423H}{Lemma} Let $(X,\frak T)$ be an analytic Hausdorff space,
and $\sequencen{E_n}$ any sequence of Borel sets in $X$.   Then the
topology $\frak T'$ generated by $\frak T\cup\{E_n:n\in\Bbb N\}$ is
analytic.

\proof{ If $X=\emptyset$ this is trivial.   Otherwise, there is a
continuous surjection $f:\NN\to X$.   Set $F_n=f^{-1}[E_n]$ for each
$n$;  then $F_n$ is a Borel subset of $\NN$, so there is a Polish
topology $\frak S'$ on $\NN$, finer than the usual topology, for which
every $F_n$ is open, by 4A3I.   But now $f$
is continuous for $\frak S'$ and $\frak T'$, so $\frak T'$ is analytic, by 423Ba and 423Bb.   (Of course $\frak T'$
is Hausdorff, because it is finer than $\frak T$.)
}%end of proof of 423H

\leader{423I}{Theorem} Let $X$ be a Polish space, $E\subseteq X$ a Borel
set, $Y$ a Hausdorff space and $f:E\to Y$ an injective function.

(a) If $f$ is continuous, then $f[E]$ is Borel.

(b) If $Y$ has a countable network\cmmnt{ (e.g., is an analytic space
or a separable metrizable space),} and $f$ is Borel measurable, then
$f[E]$ is Borel.

\proof{{\bf (a)(i)} Since there is a finer Polish topology on $X$ for
which $E$ is closed (4A3I), therefore Polish in the subspace topology
(4A2Qd), and $f$ will still be continuous for this topology, we may
suppose that $E=X$.

\medskip

\quad{\bf (ii)} Let $\sequencen{U_n}$ run over a base for the topology
of $X$ (4A2P(a-i)).   For each pair $m$, $n\in\Bbb N$ such that
$U_m\cap U_n$
is empty, $f[U_m]$ and $f[U_n]$ are analytic sets in $Y$ (423Eb, 423Bb)
and are disjoint (because $f$ is injective), so there is a Borel set
$H_{mn}$ including $f[U_m]$ and disjoint from $f[U_n]$ (422J).
Set

\Centerline{$E_n=\overline{f[U_n]}\cap\bigcap\{H_{nm}\setminus
H_{mn}:m\in\Bbb N,\,U_m\cap U_n=\emptyset\}$}

\noindent for each $n\in\Bbb N$;  then $E_n$ is a Borel set in $Y$
including $f[U_n]$.   Note that if $U_m\cap U_n$ is empty, then
$E_m\cap E_n\subseteq(H_{mn}\setminus H_{nm})
\cap(H_{nm}\setminus H_{mn})$ is also empty.

Fix a metric $\rho$ on $X$, inducing its topology, for which $X$ is
complete, and for $k\in\Bbb N$ set

\Centerline{$F_k=\bigcup\{E_n:n\in\Bbb N,\,\diam U_n\le 2^{-k}\}$,}

\noindent so that $F_k$ is Borel.   Let $F=\bigcap_{k\in\Bbb N}F_k$;
then $F$ also is a Borel subset of $Y$.

The point is that $F=f[X]$.   \Prf\ (i) If $x\in X$, then for every
$k\in\Bbb N$ there is an $n\in\Bbb N$ such that

\Centerline{$x\in U_n\subseteq\{y:\rho(y,x)\le 2^{-k-1}\}$;}

\noindent now $\diam U_n\le 2^{-k}$, so

\Centerline{$f(x)\in f[U_n]\subseteq E_n\subseteq F_k$.}

\noindent As $k$ is arbitrary, $f(x)\in F$;  as $x$ is arbitrary,
$f[X]\subseteq F$.   (ii) If $y\in F$, then for each $k\in\Bbb N$ we can
find an $n(k)$ such that $y\in E_{n(k)}$ and $\diam U_{n(k)}\le 2^{-k}$.
Since $\overline{f[U_{n(k)}]}\supseteq E_{n(k)}$ is not empty, nor is
$U_{n(k)}$, and we can choose $x_k\in U_{n(k)}$.   Indeed, for any $j$,
$k\in\Bbb N$, $E_{n(j)}\cap E_{n(k)}$ contains $y$, so is not empty, and
$U_{n(j)}\cap U_{n(k)}$ cannot be empty;  but this means that there is
some $x$ in the intersection, and

\Centerline{$\rho(x_j,x_k)\le\rho(x_j,x)+\rho(x,x_k)\le\diam
U_{n(j)}+\diam U_{n(k)}\le 2^{-j}+2^{-k}$.}

\noindent This means that $\sequence{k}{x_k}$ is a Cauchy sequence.
But $X$ is supposed to be complete, so $\sequence{k}{x_k}$ has a limit
$x$ say.

\Quer\ If $f(x)\ne y$, then (because $Y$ is Hausdorff) there is an open
set $H$ containing $f(x)$ such that $y\notin\overline{H}$.   Now $f$ is
continuous, so there is a $\delta>0$ such that $f(x')\in H$ whenever
$\rho(x',x)\le\delta$.   There is a $k\in\Bbb N$ such that
$2^{-k}+\rho(x_k,x)\le\delta$.   If $x'\in U_{n(k)}$, then

\Centerline{$\rho(x',x)\le\rho(x',x_k)+\rho(x_k,x)\le\delta$;}

\noindent thus $f[U_{n(k)}]\subseteq H$, and

\Centerline{$E_{n(k)}\subseteq\overline{f[U_{n(k)}]}
\subseteq\overline{H}$.}

\noindent But $y\in E_{n(k)}\setminus\overline{H}$.\ \Bang

Thus $y=f(x)$ belongs to $f[X]$;  as $y$ is arbitrary, $F\subseteq
f[X]$.\ \QeD\  Accordingly $f[X]=F$ is a Borel subset of $Y$, as
claimed.

\medskip

{\bf (b)} By 4A2Nf, there is a countable family $\Cal V$ of open sets in
$Y$ such that whenever $y$, $y'$ are distinct points of $Y$ there are
disjoint $V$, $V'\in\Cal V$ such that $y\in V$ and $y'\in V'$.   Let
$\frak S'$ be the topology generated by $\Cal V$;  then $\frak S'$ is
Hausdorff.   For each $V\in\Cal V$, $f^{-1}[V]$ is a Borel set in $X$,
so there is a Polish topology $\frak T'$ on $X$, finer than the original
topology, for which every $f^{-1}[V]$ is open (4A3I again).
Now $f$ is continuous for $\frak T'$ and
$\frak S'$ (4A2B(a-ii)), and $E$ is $\frak T'$-Borel, so $f[E]$ is a
$\frak S'$-Borel set in $Y$, by (a).   Since $\frak S'$ is coarser than
the original topology $\frak S$ on $Y$, $f[E]$ is also $\frak S$-Borel.
}%end of proof of 423I

\leader{423J}{Lemma} If $X$ is an uncountable analytic Hausdorff space,
it has subsets homeomorphic to $\{0,1\}^{\Bbb N}$ and $\NN$.

\proof{{\bf (a)} Let $f:\NN\to X$ be a continuous surjection.   Write
$S=\bigcup_{n\in\Bbb N}\BbbN^n$,
$I_{\sigma}=\{\phi:\sigma\subseteq\phi\in\NN\}$ for $\sigma\in S$,

\Centerline{$T=\{\sigma:\sigma\in S,\,f[I_{\sigma}]$ is
uncountable$\}$.}

\noindent Then if $\sigma\in T$ there are $\tau$, $\tau'\in T$, both
extending $\sigma$, such that $f[I_{\tau}]\cap f[I_{\tau'}]=\emptyset$.
\Prf\ Set

\Centerline{$A=\bigcup\{f[I_{\tau}]:\tau\in S\setminus T\}$.}

\noindent Then $A$ is a countable union of countable sets, so is
countable.   There must therefore be distinct points $x$, $y$ of
$f[I_{\sigma}]\setminus A$;  express $x$ as $f(\phi)$ and $y$ as
$f(\psi)$ where $\phi$ and $\psi$ belong to $I_{\sigma}$.   Because $X$
is Hausdorff, there are disjoint open sets $G$, $H$ such that $x\in G$
and $y\in H$.   Because $f$ is continuous, there are $m$, $n\in\Bbb N$
such that $I_{\phi\restr m}\subseteq f^{-1}[G]$ and $I_{\psi\restr
n}\subseteq f^{-1}[H]$.   Of course both $\tau=\phi\restr m$ and
$\tau'=\psi\restr n$ must extend $\sigma$, and they belong to $T$
because $x\in f[I_{\tau}]\setminus A$ and $y\in f[I_{\tau'}]\setminus A$.\
\Qed

\medskip

{\bf (b)} We can therefore choose inductively a family
$\family{\upsilon}{S_2}{\tau(\upsilon)}$ in $T$, where
$S_2=\bigcup_{n\in\Bbb N}\{0,1\}^n$, such that

\Centerline{$\tau(\emptyset)=\emptyset$,}

\Centerline{$\tau(\upsilon^{\smallfrown}\fraction{i})\supseteq\tau(\upsilon)$
whenever $\upsilon\in S_2$, $i\in\{0,1\}$,}

\Centerline{$f[I_{\tau(\upsilon^{\smallfrown}\fraction{0}})]
  \cap f[I_{\tau(\upsilon^{\smallfrown}\fraction{1}}]=\emptyset$ for every
$\upsilon\in S_2$.}

\noindent Note that $\#(\tau(\upsilon))\ge\#(\upsilon)$ for every
$\upsilon\in S_2$.   For each $z\in\{0,1\}^{\Bbb N}$,
$\sequencen{\tau(z\restr n)}$ is a sequence in $S$ in which each term
strictly extends its predecessor, so there is a unique $g(z)\in\NN$ such
that $\tau(z\restr n)\subseteq g(z)$ for every $n$.    Now
$g(z')\restr n=g(z)\restr n$ whenever $z\restr n=z'\restr n$, so $g$ and
$fg:\{0,1\}^{\Bbb N}\to X$ are continuous.   If $w$, $z$ are distinct
points of $\{0,1\}^{\Bbb N}$, there is a first $n$ such that
$w(n)\ne z(n)$, in which case
$fg(w)\in f[I_{\tau(w\restr n)^{\smallfrown}\fraction{w(n)}}]$
and $fg(z)\in f[I_{\tau(w\restr n)^{\smallfrown}\fraction{z(n)}}]$ are distinct.
So $fg:\{0,1\}^{\Bbb N}\to X$ is a continuous injection, therefore a
homeomorphism between $\{0,1\}^{\Bbb N}$ and its image, because
$\{0,1\}^{\Bbb N}$ is compact (3A3Dd).

\medskip

{\bf (c)} Thus $X$ has a subspace homeomorphic to $\{0,1\}^{\Bbb N}$.
Now $\{0,1\}^{\Bbb N}$ has a subspace homeomorphic to $\Bbb N$.   \Prf\
For instance, setting $d_n(n)=1$, $d_n(i)=0$ for $i\ne n$,
$D=\{d_n:n\in\Bbb N\}$ is homeomorphic to $\Bbb N$.\ \QeD\  Now $D^{\Bbb
N}$ is homeomorphic to $\NN$ and is a subspace of $(\{0,1\}^{\Bbb
N})^{\Bbb N}\cong\{0,1\}^{\Bbb N\times\Bbb N}\cong\{0,1\}^{\Bbb N}$, so
$\{0,1\}^{\Bbb N}$ has a subspace homeomorphic to $\NN$.   Accordingly
$X$ also has a subspace homeomorphic to $\NN$.
}%end of proof of 423J

\leader{423K}{Corollary} Any uncountable Borel set in any analytic 
Hausdorff space has cardinal $\frak c$.

\proof{ If $X$ is an analytic space and $E\subseteq X$ is an uncountable
Borel set, then $E$ is analytic (423E), so includes a copy of
$\{0,1\}^{\Bbb N}$ and must have cardinal at least
$\#(\{0,1\}^{\Bbb N})=\frak c$.   On the other hand, $E$ is also a
continuous image of $\NN$, so has cardinal at most $\#(\NN)=\frak c$.
}%end of proof of 423K

\leader{423L}{Proposition} Let $X$ be an uncountable analytic Hausdorff
space.   Then it has a non-Borel analytic subset.

\proof{{\bf (a)} I show first that there is an analytic set
$A\subseteq\NN\times\NN$ such that every
analytic subset of $\NN$ is a vertical section of $A$.
\Prf\ Let $\Cal U$ be a countable base for the topology of $\BbbN^{\Bbb
N}\times\NN$, containing $\emptyset$, and $\sequencen{U_n}$
an enumeration of $\Cal U$.   Write

\Centerline{$M=(\NN\times\NN\times\NN)
\setminus\bigcup_{m,n\in\Bbb N}(\{x:x(m)=n\}\times U_n)$.}

\noindent Then $M$ is a closed subset of $(\NN\times\Bbb
N^{\Bbb N})\times\NN$, therefore analytic (423Ba, 423Bd), so
its continuous image

\Centerline{$A=\{(x,z):$ there is some $y$ such that $(x,y,z)\in M\}$}

\noindent is analytic (423Bb).

Now let $E$ be any analytic subset of $\NN$.   By 423E, $E$
is Souslin-F;  by 421J, there is a closed
set $F\subseteq\NN\times\NN$ such that
$E=\{z:\exists\,y,\,(y,z)\in F\}$.   Let $\sequence{m}{x(m)}$ be a
sequence running over $\{n:n\in\Bbb N,\,U_n\cap F=\emptyset\}$, so that

\Centerline{$F=(\NN\times\BbbN^{\Bbb
N})\setminus\bigcup_{m\in\Bbb N}U_{x(m)}=\{(y,z):(x,y,z)\in M\}$.}

\noindent Now

$$\eqalign{\{z:(x,z)\in A\}
&=\{z:\text{ there is some }y\text{ such that }(x,y,z)\in M\}\cr
&=\{z:\text{ there is some }y\text{ such that }(y,z)\in F\}
=E,\cr}$$

\noindent and $E$ is a vertical section of $A$, as required.\ \Qed

\medskip

{\bf (b)} It follows that there is a non-Borel analytic set
$B\subseteq\NN$.   \Prf\ Take $A$ from (a) above, and try

\Centerline{$B=\{x:(x,x)\in A\}$.}

\noindent Because $B$ is the inverse image of $A$ under the continuous
map $x\mapsto(x,x)$, it is analytic (423Gc).   \Quer\ If
$B$ were a Borel set, then $B'=\NN\setminus B$ would also be
Borel, therefore analytic (423E), and there would be an $x\in\Bbb
N^{\Bbb N}$ such that $B'=\{y:(x,y)\in A\}$.   But in this case

\Centerline{$x\in B\iff (x,x)\in A\iff x\in B'$,}

\noindent which is a difficulty you may have met before.\ \Bang\Qed

\medskip

{\bf (c)} Now return to our arbitrary uncountable analytic Hausdorff
space $X$.   By 423J, $X$ has a subset $Z$ homeomorphic to $\NN$.   By
(b), $Z$ has an analytic subset $A$ which is not Borel in $Z$, therefore
cannot be a Borel subset of $X$.
}%end of proof of 423L

\leader{423M}{}\cmmnt{ I devote a few paragraphs to an important
method of constructing selectors.

\medskip

\noindent}{\bf Theorem} Let $X$ be an analytic Hausdorff space,  $Y$ a
set, and $\Cal C\subseteq\Cal PY$.   Write $\Tau$ for the
$\sigma$-algebra of subsets of $Y$ generated by $\Cal S(\Cal C)$, where
$\Cal S$ is Souslin's operation, and $\Cal V$ for
$\Cal S(\{F\times C:F\subseteq X$ is closed, $C\in\Cal C\})$.   If
$W\in\Cal V$, then $W[X]\in\Cal S(\Cal C)$ and there is a $\Tau$-measurable
function $f:W[X]\to X$ such that $(f(y),y)\in W$ for every $y\in W[X]$.

\proof{ Write $\Cal F$ for
$\{F\times C:F\subseteq X$ is closed, $C\in\Cal C\}$.

\medskip

{\bf (a)} Consider first the case in which $X=\NN$ and all the
horizontal sections $W^{-1}[\{y\}]$ of $W$ are closed.   Let
$\Cal E$ be the family of closed subsets of $Y$.   For
$\sigma\in S=\bigcup_{n\in\Bbb N}\BbbN^n$ set
$I_{\sigma}=\{\phi:\sigma\subseteq\phi\in\NN\}$.   Then
$W\cap(I_{\sigma}\times Y)\in\Cal V$.   \Prf\
Because Souslin's operation is idempotent (421D),
$\Cal S(\Cal V)=\Cal V$.   The set
$\{V:V\cap(I_{\sigma}\times Y)\in\Cal V\}$ is therefore closed under
Souslin's operation (apply 421Cc to the identity map from
$I_{\sigma}\times Y$ to $X\times Y$, or otherwise);  since it includes
$\Cal F$, it is the whole of $\Cal V$, and contains $W$.\ \Qed

By 421G, $W[I_{\sigma}]=(W\cap(I_{\sigma}\times Y))[\NN]$ belongs to
$\Cal S(\Cal C)\subseteq\Tau$ for every $\sigma$.   In particular,
$W[\NN]=W[I_{\emptyset}]\in\Cal S(\Cal C)$.   Define
$\family{\sigma}{S}{Y_{\sigma}}$ in $\Tau$
inductively, as follows.   $Y_{\emptyset}=W[\NN]$.   Given that
$Y_{\sigma}\in\Tau$ and that $Y_{\sigma}\subseteq W[I_{\sigma}]$, set

\Centerline{$Y_{\sigma^{\smallfrown}\fraction{j}}
=Y_{\sigma}\cap W[I_{\sigma^{\smallfrown}\fraction{j}}]
\setminus\bigcup_{i<j}W[I_{\sigma^{\smallfrown}\fraction{i}}]$}

\noindent for every $j\in\Bbb N$.   Continue.

At the end of the induction, we have

\Centerline{$\bigcup_{j\in\Bbb N}Y_{\sigma^{\smallfrown}\fraction{j}}
=Y_{\sigma}\cap\bigcup_{j\in\Bbb N}W[I_{\sigma^{\smallfrown}\fraction{j}}]
=Y_{\sigma}\cap W[I_{\sigma}]=Y_{\sigma}$}

\noindent for every $\sigma\in S$, while
$\sequence{j}{Y_{\sigma^{\smallfrown}\fraction{j}}}$ is always disjoint.   
So for
each $y\in Y_{\emptyset}=W[\NN]$ we have a unique $f(y)\in\NN$ such that
$y\in Y_{f(y)\restr n}$ for every $n$.   Since
$f^{-1}[I_{\sigma}]=Y_{\sigma}\in\Tau$ for every $\sigma\in S$, $f$ is
$\Tau$-measurable (4A3Db).   Also $(f(y),y)\in W$ for every
$y\in W[\NN]$.   \Prf\ For each $n\in\Bbb N$,
$y\in Y_{f(y)\restr n}=W[I_{f(y)\restr n}]$, so there is an $x_n\in\NN$
such that $x_n\restr n=f(y)\restr n$ and $(x_n,y)\in W$.   But this
means that $f(y)=\lim_{n\to\infty}x_n$;  since we are supposing that the
horizontal sections of $W$ are closed, $(f(y),y)\in W$.\ \Qed

Thus the theorem is true if $X=\NN$ and $W$ has closed horizontal
sections.

\medskip

{\bf (b)} Now suppose that $X=\NN$ and that $W\subseteq\NN\times Y$ is
any set in $\Cal V$.   Then there is a Souslin scheme
$\family{\sigma}{S^*}{F_{\sigma}\times C_{\sigma}}$ in $\Cal F$ with
kernel $W$;  of course I mean you to suppose that
$F_{\sigma}\subseteq\NN$ is closed and $C_{\sigma}\in\Cal C$ for every
$\sigma$.   Set

\Centerline{$\tilde W
=\bigcap_{k\ge 1}\bigcup_{\sigma\in\BbbN^k}
  I_{\sigma}\times F_{\sigma}\times C_{\sigma}
\subseteq\NN\times\NN\times Y$.}

\noindent Then $W$ is the projection of $\tilde W$ onto the last two
coordinates, by 421Ce.   If $y\in Y$, then

\Centerline{$\{(\phi,\psi):(\phi,\psi,y)\in\tilde W\}
=\bigcap_{k\ge 1}\bigcup\{I_{\sigma}\times F_{\sigma}:
  \sigma\in\BbbN^k,\,y\in C_{\sigma}\}$}

\noindent is closed in $\NN\times\NN$.   (If $J$ is any subset of
$\BbbN^k$, then

\Centerline{$(\NN\times\NN)
  \setminus\bigcup_{\sigma\in J}I_{\sigma}\times F_{\sigma}
=\bigcup_{\sigma\in J}I_{\sigma}\times(\NN\setminus F_{\sigma})
   \cup\bigcup_{\sigma\in\BbbN^k\setminus J}I_{\sigma}\times\NN$}

\noindent is open.)   Also $I_{\sigma}\times F_{\sigma}$ is a closed
subset of $\NN\times\NN$ for every $\sigma$, and $\NN\times\NN$ is
homeomorphic to $\NN$.   We can therefore apply (a) to $\tilde W$,
regarded as a subset of $(\NN\times\NN)\times Y$, to see that
$W[\NN]=\tilde W[\NN\times\NN]\in\Cal S(\Cal C)$ and that there is a
$\Tau$-measurable function $h=(g,f):W[\NN]\to\NN\times\NN$ such that
$(g(y),f(y),y)\in\tilde W$ for every $y\in W[\NN]$.   Now, of course,
$f:W[\NN]\to\NN$ is $\Tau$-measurable and $(f(y),y)\in W$ for every
$y\in W[\NN]$.

\medskip

{\bf (c)} Finally, suppose only that $X$ is an analytic Hausdorff space
and that $W\in\Cal V$.   If $X$ is empty, so is
$Y$, and the result is trivial.   Otherwise, there is a continuous
surjection $h:\NN\to X$.   Set $\tilde h(\phi,y)=(h(\phi),y)$ for
$\phi\in\NN$ and
$y\in Y$;  then $\tilde h:\NN\times Y\to X\times Y$ is a continuous
surjection, and $\tilde W=\tilde h^{-1}[W]$ is the kernel of a Souslin
scheme in

\Centerline{$\{\tilde h^{-1}[F\times C]:
  F\subseteq\NN\text{ is closed}, C\in\Cal C\}
=\{h^{-1}[F]\times C:F\subseteq\NN\text{ is closed}, C\in\Cal C\}$}

\noindent by 421Cb.   So we can apply (b) to see that
$W[X]=\tilde W[\NN]\in\Cal S(\Cal C)$ and there is a
$\Tau$-measurable $g:W[X]\to\NN$ such that $(g(y),y)\in\tilde W$ for
every $y\in Y$.   Finally $f=hg:W[X]\to X$ is $\Tau$-measurable and
$(f(y),y)\in W$ for every $y\in Y$.   This completes the proof.
}%end of proof of 423M

\leader{423N}{}\cmmnt{ The expression

\Centerline{$\Cal V
=\Cal S(\{F\times C:F\subseteq X$ is closed, $C\in\Cal C\})$}

\noindent in 423M is a new formulation, and I had better describe one
of the basic cases in which we can use the result.

\medskip

\noindent}{\bf Corollary} Let $X$ be an analytic Hausdorff space and $Y$
any topological space.   Let $\Tau$ be the $\sigma$-algebra of subsets
of $Y$ generated by $\Cal S(\Cal B(Y))$, where $\Cal B(Y)$ is the Borel
$\sigma$-algebra of $Y$.   If $W\in\Cal S(\Cal B(X\times Y))$, then
$W[X]\in\Tau$ and there is a
$\Tau$-measurable function $f:W[X]\to X$ such that $(f(y),y)\in W$ for
every $y\in W[X]$.

\proof{{\bf (a)} Suppose to begin with that $X=\NN$.   In 423M, set
$\Cal C=\Cal B(Y)$.   Then every open subset and every closed subset of
$X\times Y$ belongs to $\Cal V$ as defined in 423M.   \Prf\ For
$\sigma\in S=\bigcup_{k\in\Bbb N}\BbbN^k$, set
$I_{\sigma}=\{\phi:\sigma\subseteq\phi\in\NN\}$.   If
$V\subseteq X\times Y$ is open, set

\Centerline{$H_{\sigma}=\bigcup\{H:H\subseteq Y$ is open,
$I_{\sigma}\times H_{\sigma}\subseteq V\}$}

\noindent for each $\sigma\in S$.   Because
$\{I_{\sigma}:\sigma\in S\}$ is a base for the topology of $\NN$,
$V=\bigcup_{\sigma\in S}I_{\sigma}\times H_{\sigma}\in\Cal V$.

As for the complement of $V$, we have

$$\eqalign{(\NN\times Y)\setminus V
&=\bigcap_{\Atop{k\in\Bbb N}{\sigma\in\BbbN^k}}
  (\NN\times Y)\setminus(I_{\sigma}\times H_{\sigma})\cr
&=\bigcap_{\Atop{k\in\Bbb N}{\sigma\in\BbbN^k}}
  ((\NN\setminus I_{\sigma})\times Y)
  \cup(\NN\times(Y\setminus H_{\sigma}))\cr
&=\bigcap_{\Atop{k\in\Bbb N}{\sigma\in\BbbN^k}}
  \bigcup_{\Atop{\tau\in\BbbN^k}{\tau\ne\sigma}}
  (I_{\tau}\times Y)
  \cup(\NN\times(Y\setminus H_{\sigma}))\cr}$$

\noindent which again belongs to $\Cal V$, because $\Cal V$ is closed under
countable unions and intersections and contains $I_{\tau}\times Y$ and
$\NN\times(Y\setminus H_{\sigma})$ for all $\sigma$, $\tau\in S$.\
\Qed

By 421F, $\Cal V$ contains every Borel subset of $\NN\times Y$, so
includes $\Cal S(\Cal B(\NN\times Y))$.   So in this case we can apply
423M directly to get the result.

\medskip

{\bf (b)} Now suppose that $X$ is any analytic space.   If $X$ is empty,
the result is trivial.   Otherwise, let
$h:\NN\to X$ be a continuous surjection.   Set
$\tilde h(\phi,y)=(h(\phi),y)$ for $\phi\in\NN$ and $y\in Y$, so that
$\tilde h:\NN\times Y\to X\times Y$ is continuous.   Set
$\tilde W=\tilde h^{-1}[W]$.   If
$V\in\Cal B(X\times Y)$ then $\tilde h^{-1}[V]\in\Cal B(\NN\times Y)$
(4A3Cd), so  $\tilde W\in\Cal S(\Cal B(\NN\times Y))$ (421Cc).
By (a), $\tilde W[\NN]\in\Tau$ and there is a $\Tau$-measurable function
$g:\tilde W[\NN]\to\NN$ such that $(g(y),y)\in\tilde W$ for every
$y\in\tilde W[\NN]$.   It is now easy to check that
$W[X]=\tilde W[\NN]\in\Tau$ (this is where we need to know that $h$ is
surjective), that $f=hg:W[X]\to X$ is $\Tau$-measurable, and that
$(f(y),y)\in W$ for every $y\in W[X]$, as required.
}%end of proof of 423N

\cmmnt{\medskip

\noindent{\bf Remark} This is a version of the {\bf von Neumann-Jankow
selection theorem}.
}%end of comment

\leader{423O}{Corollary}\dvArevised{2011} Let $X$ and $Y$ be analytic
Hausdorff spaces, $A$ an analytic subset of $X$ and $f:A\to Y$ a Borel 
measurable function.   Let
$\Tau$ be the $\sigma$-algebra of subsets of $Y$ generated by the
Souslin-F subsets of $Y$.   
Then $f[A]\in\Tau$ and there is a $\Tau$-measurable function 
$g:f[A]\to A$ such that $fg$ is the identity on $f[A]$.

\proof{ In this context, every Borel subset of $Y$ is Souslin-F (423Eb),
so every member of $\Cal S(\Cal B(Y))$ is Souslin-F (421D) and
$\Tau=\Cal S(\Cal B(Y))$.
If we think of $f$ as a subset of $X\times Y$, it is analytic
(423Ga), therefore Souslin-F in
$X\times Y$;  now we can use 423N to find
a $\Tau$-measurable function $g:f[A]\to A$ such that
$(g(y),y)\in f$, that is, $f(g(y))=y$, for every $y\in f[A]$.
}%end of proof of 423O

\leader{*423P}{Constituents of coanalytic sets:  Theorem} Let $X$ be a
Hausdorff space, and $A\subseteq X$ an analytic subset of $X$.   Then
there is a non-decreasing family $\ofamily{\xi}{\omega_1}{E_{\xi}}$ of
Borel subsets of $X$, with union $X\setminus A$, such that every
analytic subset of $X\setminus A$ is included in some $E_{\xi}$.

\proof{ Put 422K(iii) and 423C together.
}%end of proof of 423P

\leader{*423Q}{Remarks (a)} Let $A$ be an analytic set in an analytic
space $X$ and $\ofamily{\xi}{\omega_1}{E_{\xi}}$ a family of Borel sets
as in 423P.   There is nothing unique about the $E_{\xi}$.   But if
$\ofamily{\xi}{\omega_1}{E'_{\xi}}$ is another such family, then
\dvro{there is a cofinal closed set $C$ in $\omega_1$ such that}{every
$E'_{\xi}$
is an analytic subset of $X\setminus A$, by 423E, so is included in
some $E_{\eta}$;  and, similarly, every $E_{\xi}$ is included in some
$E'_{\eta}$.   We therefore have a function $f:\omega_1\to\omega_1$ such
that $E'_{\xi}\subseteq E_{f(\xi)}$ and $E_{\xi}\subseteq E'_{f(\xi)}$
for every $\xi<\omega$.   If we set
$C=\{\xi:\xi<\omega_1,\,f(\eta)<\xi$ for every $\eta<\xi\}$, then $C$ is
a closed cofinal set in $\omega_1$ (4A1Bc), and}
$\bigcup_{\eta<\xi}E_{\eta}=\bigcup_{\eta<\xi}E'_{\eta}$ for every
$\xi\in C$.   \cmmnt{If $X\setminus A$ is itself analytic, that is, if
$A$ is a Borel set, then we shall have to have
$X\setminus A=E_{\xi}=E'_{\xi}$ for some $\xi<\omega_1$.}

Another way of expressing the result in 423P is to say that if we write
$\Cal I=\{B:B\subseteq X\setminus A$ is analytic$\}$, then
$\{E:E\in\Cal I,\,E$ is Borel$\}$ is cofinal with $\Cal I$\cmmnt{
(this is the First Separation Theorem)} and $\cf\Cal I\le\omega_1$.

\spstheader 423Qb It is a remarkable fact that, in some models of set theory, we can have
non-Borel coanalytic sets in Polish spaces such that all their
constituents are
countable\cmmnt{ ({\smc Jech 78}, p.\ 529, Cor.\ 2)}.
\cmmnt{(Note that, by (a), this is the same thing as
saying that $X\setminus A$ is uncountable but all its Borel subsets are
countable.)}   But in `ordinary' cases we shall have, \cmmnt{for every
Borel subset $E$ of $X\setminus A$, an uncountable Borel subset of
$(X\setminus A)\setminus E$;  so that} for any family
$\ofamily{\xi}{\omega_1}{G_{\xi}}$ of Borel constituents of
$X\setminus A$, \cmmnt{there must be uncountably many} uncountable
$G_{\xi}$.
\prooflet{To see that this happens at least sometimes, take any
non-Borel analytic subset $A_0$ of $\NN$ (423L), and consider
$A=A_0\times\NN\subseteq(\NN)^2$.
Then $A$ is analytic (423B).   If $E\subseteq(\NN)^2\setminus A$ is
Borel, then $\pi_1[E]=\{x:(x,y)\in E\}$ is an analytic subset of
$\NN\setminus A$, so is not the whole of $\NN\setminus A_0$ (by 423Fa).
Taking any $x\in(\NN\setminus A_0)\setminus\pi_1[E]$, $\{x\}\times\NN$
is an uncountable Borel subset of $((\NN)^2\setminus A)\setminus E$.
For an alternative construction, see 423Ye.}

\leader{*423R}{Coanalytic and PCA sets}\dvAnew{2010} 
\cmmnt{ In the case of Polish spaces, we can go a great deal 
farther.   I mention only some fragments which will be used in
Volume 5.}   Let $X$ be a Polish space.

\spheader 423Ra A subset $A$ of $X$ is
{\bf coanalytic}\cmmnt{ or $\pmb{\Pi}^1_1$} if $X\setminus A$ is 
analytic, and
{\bf PCA}\cmmnt{ or $\pmb{\Sigma}^1_2$} if there is a coanalytic set
$R\subseteq\BbbN^{\Bbb N}\times X$ such that $R[\BbbN^{\Bbb N}]=A$.
\cmmnt{Generally, for $n\ge 1$, $A\subseteq X$ is 
$\pmb{\Pi}^1_n$ if $X\setminus A$ is $\pmb{\Sigma}^1_n$, and
$\pmb{\Sigma}^1_{n+1}$ if there is a $\pmb{\Pi}^1_n$ set
$R\subseteq\BbbN^{\Bbb N}\times X$ such that $R[\BbbN^{\Bbb N}]=A$.
If $A$ is both $\pmb{\Sigma}^1_n$ and $\pmb{\Pi}^1_n$, we say that $A$ is
$\pmb{\Delta}^1_n$.}

\spheader 423Rb\dvro{ Every}{ Analytic subsets of $X$ are Souslin-F
(423Eb).   Applying 421P to the Borel $\sigma$-algebra 
of $X$, we see that a subset of
$X$ which is either analytic or coanalytic can be expressed as the union of
at most $\omega_1$ Borel sets.   It follows that every} PCA set 
$A\subseteq X$ can be expressed as the union of at most $\omega_1$ Borel 
sets.   \prooflet{\Prf\ Let
$R\subseteq\BbbN^{\Bbb N}\times X$ be a coanalytic set such that
$A=R[\NN]$.   Express $R$ as $\bigcup_{\xi<\omega_1}R_{\xi}$ where every
$R_{\xi}$ is a Borel subset of the Polish space $\NN\times X$.   
For each $\xi<\omega_1$,
$R_{\xi}$ is analytic (423Eb) so $A_{\xi}=R_{\xi}[\NN]$ is analytic (423Bb)
and can be expressed as $\bigcup_{\eta<\omega_1}E_{\xi\eta}$ where every
$E_{\xi\eta}$ is a Borel subset of $X$.   Now
$A=\bigcup_{\xi,\eta<\omega_1}E_{\xi\eta}$ is the union of at most
$\omega_1$ Borel sets.\ \Qed}

\spheader 423Rc A subset of $X$ is Borel iff it 
is\cmmnt{ $\pmb{\Delta}^1_1$, that is, is} both analytic and
coanalytic\prooflet{ (423Eb, 423Fa)}.
\dvro{ The}{ Since the intersection and union of a
sequence of analytic subsets of $X$ are analytic (423Ea), the} union 
and intersection of
a sequence of coanalytic subsets of $X$ are coanalytic.   If $Y$ is
another Polish space and $h:X\to Y$ is Borel measurable, 
then\cmmnt{ $h^{-1}[A]$ is analytic for every analytic $A\subseteq Y$ 
(423Gc), so} $h^{-1}[B]$ is coanalytic in $X$ for every coanalytic 
$B\subseteq Y$.   If $Y$ is a G$_{\delta}$ subset of $X$, and
$B\subseteq Y$ is coanalytic in $Y$\cmmnt{ (remember that $Y$, with its 
subspace topology, is Polish, by 4A2Qd),} then $B$ is coanalytic in 
$X$\prooflet{, because
$X\setminus B=(X\setminus Y)\cup(Y\setminus B)$ is the union of two
analytic sets}.

\spheader 423Rd If $X$ and $Y$ are Polish spaces, $A\subseteq Y$ is PCA and
$f:X\to Y$ is Borel measurable, then $f^{-1}[A]$ is a PCA set in $X$.
\prooflet{\Prf\ Let $R\subseteq\NN\times X$ be a coanalytic set such that
$R[\NN]=A$.   Set $g(\phi,x)=(\phi,f(x))$ for $\phi\in\NN$ and $x\in X$;
writing $\Cal B(X)$ for the Borel $\sigma$-algebra of $X$, etc., 
$g$ is $(\Cal B(\NN)\tensorhat\Cal B(X),
\Cal B(\NN)\tensorhat\Cal B(Y))$-measurable (use 4A3Bc).   But
$\Cal B(\NN)\tensorhat\Cal B(X)=\Cal B(\NN\times X)$ and
$\Cal B(\NN)\tensorhat\Cal B(Y)=\Cal B(\NN\times Y)$ (4A3Ga), so
$g$ is Borel measurable.   By (c), $S=g^{-1}[R]$ is a coanalytic set in
$\NN\times X$.   Now 

\Centerline{$S[\NN]
=\{x:\,\Exists\phi,\,(\phi,x)\in g^{-1}[R]\}
=\{x:\,\Exists\phi,\,(\phi,f(x))\in R\}=f^{-1}[A]$,}

\noindent so $f^{-1}[A]$ is PCA.\ \Qed}

\leaveitout{\spheader 423R?
If $X$ and $Y$ are Polish spaces, $A\subseteq X$ is PCA and
$f:X\to Y$ is a Borel measurable function, then $f[A]$ is a PCA subset of
$Y$.   \prooflet{\Prf\ If $X$ is empty, so is $f[A]$, and we can stop.
Otherwise, let $R\subseteq\NN\times X$ be a coanalytic set such that
$R[\NN]=A$, and $g:\NN\to X$ a continuous surjection.
Consider

\Centerline{$S
=\{((\phi,\psi),y):\phi$, $\psi\in\NN$, $y\in Y$, $(\phi,g(\psi))\in R$,
$f(g(\psi))=y\}$.}

\noindent The complement $((\NN\times\NN)\times Y)\setminus S$ is equal to
$C\cup D$ where

\Centerline{$C=\{((\phi,\psi),y):(\phi,g(\psi))\notin R\}$,
\quad$D=\{((\phi,\psi),y):f(g(\psi))\ne y\}$.}

\noindent Setting $\hat g((\phi,\psi),y)=(\phi,g(\psi))$, 
$\hat g:(\NN\times\NN)\times Y\to\NN\times X$ is continuous, so
$C=\hat g^{-1}[(\NN\times X)\setminus R]$ is analytic (423Gc).
On the other hand, setting $\bar g((\phi,\psi),y)=(g(\psi),y)$,
$\bar g:(\NN\times\NN)\times Y\to X\times Y$ is continuous, while
$D_0=\{(x,y):f(x)\ne y\}$ is a Borel subset of $X\times Y$ (4A3Gb),
so $D=\bar g^{-1}[D_0]$ is Borel (4A3Cd), therefore analytic (423Eb
again).   Thus $C\cup D$ is analytic (423Ea) and $S$ is coanalytic, while
$S[\NN\times\NN]=f[A]$.   Since $\NN\times\NN$ is homeomorphic to $\NN$,
this is enough to show that $f[A]$ is PCA.\ \Qed}
}

\cmmnt{\spheader 423Re For a fuller account of this material, see
{\smc Kechris 95} or {\smc Kuratowski 66}.}

\leader{423S}{Proposition}\dvAnew{2011}
Let $(X,\frak T)$ be an analytic Hausdorff space, and
$\Sigma$ a countably generated $\sigma$-subalgebra of the Borel
$\sigma$-algebra $\Cal B(X,\frak T)$
which separates the points of $X$.   Then $\Sigma=\Cal B(X,\frak T)$.

\proof{ Let $\sequencen{E_n}$ be a sequence in $\Sigma$ which
$\sigma$-generates $\Sigma$.   If $x$, $y\in X$ are distinct, the set
$\{E:E\subseteq X$, $x\in E\iff y\in E\}$ is a $\sigma$-algebra of subsets
of $X$ not including $\Sigma$, so there is an $n\in\Bbb N$ such that just
one of $x$, $y$ belongs to $E_n$, and the topology
$\frak S$ generated by 
$\{E_n:n\in\Bbb N\}\cup\{X\setminus E_n:n\in\Bbb N\}$ is Hausdorff.
Write 
$\frak T\vee\frak S$ for the topology generated by $\frak T\cup\frak S$,
that is, the topology generated by
$\frak T\cup\{E_n:n\in\Bbb N\}\cup\{X\setminus E_n:n\in\Bbb N\}$.
By 423H, $\frak T\vee\frak S$ is analytic;  because both $\frak T$ and
$\frak S$ are Hausdorff topologies coarser than $\frak T\vee\frak S$, 
the Borel $\sigma$-algebras $\Cal B(X,\frak T)$, 
$\Cal B(X,\frak T\vee\frak S)$ and $\Cal B(X,\frak S)$ are all the same
(423Fb).
Next, $\frak S$ is second-countable, therefore hereditarily Lindel\"of
(4A2O), with a subbase included in $\Sigma$, so 
$\Cal B(X,\frak S)\subseteq\Sigma$ (4A3Da) and $\Cal B(X,\frak T)$ must be
equal to $\Sigma$.
}%end of proof of 423S
%out of order query;  better after 423I?
 
\exercises{\leader{423X}{Basic exercises $\pmb{>}$(a)}
%\spheader 423Xa
For a Hausdorff space $X$, show that the following are
equiveridical:  (i) $X$ is analytic;  (ii) $X$ is a continuous image of
a Polish space;  (iii) $X$ is a continuous image of a closed subset of
$\NN$.
%423B

\spheader 423Xb Write out a direct proof of 423Ea, not quoting 423C or
421D.
%423E

\sqheader 423Xc Let $X$ be a set, $\frak S$ a
Hausdorff topology on $X$ and $\frak T$ an analytic topology on $X$ such
that $\frak S\subseteq\frak T$.   Show that $\frak S$ and $\frak T$ have
the same analytic sets.   \Hint{423F.}
%423F

\spheader 423Xd Let $X$ be an analytic Hausdorff space.   (i) Show that
its Borel $\sigma$-algebra $\Cal B(X)$ is countably generated as
$\sigma$-algebra.  \Hint{use 4A2Nf
%countable network Hausdorff > coarser second countable Hausdorff topy
and 423Fb.}   (ii) Show that there is an analytic subset $Y$ of
$\Bbb R$ such that $(X,\Cal A_X)$ is isomorphic to $(Y,\Cal A_Y)$, where
$\Cal A_X$, $\Cal A_Y$ are the families of Souslin-F subsets of $X$, $Y$
respectively.   \Hint{show
that there is an injective Borel measurable function from $X$ to
$\Bbb R$ (cf.\ 343E), and use 423G.}   (iii) Show that $(X,\Cal B(X))$
is isomorphic to $(Y,\Cal B(Y))$, where $\Cal B(Y)$ is the Borel
$\sigma$-algebra of
$Y$.   \Hint{423Fa.}   (iv) Let $\Tau_X$, $\Tau_Y$ be the
$\sigma$-algebras generated by $\Cal A_X$, $\Cal A_Y$ respectively.
Show that $(X,\Tau_X)$ and $(Y,\Tau_Y)$ are isomorphic.
%423F 423G 423S

\spheader 423Xe Let $\frak S$ be the right-facing Sorgenfrey topology on
$\Bbb R$ (415Xc).   Show that $\frak S$ has the same Borel sets as the
usual topology $\frak T$ on $\Bbb R$.   \Hint{$\frak S$ is hereditarily
Lindel\"of (419Xf) and has a base consisting of $\frak T$-Borel sets.}
Show that $\frak S$ is not analytic.
%423F

\spheader 423Xf Let $X$ be an analytic Hausdorff space and $Y$
any topological space.   Let $\Tau$ be the $\sigma$-algebra of subsets
of $Y$ generated by the Souslin-F sets.   Show that if
$W\subseteq X\times Y$ is Souslin-F, then $W[X]\in\Tau$ and there is a
$\Tau$-measurable function $f:W[X]\to Y$ such that $(f(y),y)\in W$ for
every $y\in W[X]$.   \Hint{start with $X=\NN$, as in 423N.}
%423N

\spheader 423Xg Let $X$ and $Y$ be Hausdorff spaces, and
$R\subseteq X\times Y$ an analytic set such that $R^{-1}[Y]=X$.   Show
that there is
a function $g:X\to Y$, measurable with respect to the $\sigma$-algebra
generated by the Souslin-F subsets of $X$, such that $(x,g(x))\in R$ for
every $x\in X$.
%423O

\sqheader 423Xh(i) Show that the family of analytic subsets of $[0,1]$
has cardinal $\frak c$.   \Hint{421Xc.}   (ii) Show that the
$\sigma$-algebra $\Tau$ of subsets of $[0,1]$ generated by the analytic
sets has cardinal $\frak c$.  \Hint{421Xh.}   (iii) Show that
there is a set $A\subseteq[0,1]$ which does not belong to $\Tau$.
%423O

\spheader 423Xi Let $X=Y=[0,1]$.   Give $Y$ the usual topology, and give
$X$ the topology corresponding to the one-point compactification of the
discrete topology on $\coint{0,1}$, that is, a set $G\subseteq X$ is
open if either $1\notin G$ or $G$ is cofinite.   Show that the identity
map $f:X\to Y$ is a Borel measurable bijection, but that $f^{-1}$ is not
measurable for the $\sigma$-algebra of subsets of $Y$ generated by the
Souslin-F sets.
%423O %423Xh

\leader{423Y}{Further exercises (a)}
%\spheader 423Ya
Show that a space with a countable network is {\bf hereditarily
separable} (that is, every subset is separable), therefore countably
tight.

\spheader 423Yb
Show that if $X$ is a Hausdorff space with a countable network, then
every analytic subset of $X$ is
obtainable by Souslin's operation from the open subsets of $X$.
%423E %mt42bits

\spheader 423Yc Let $(X,\frak T)$ and $(Y,\frak S)$ be analytic
Hausdorff
spaces and $f:X\to Y$ a Borel measurable function.   (i) Show that there
is a zero-dimensional separable metrizable topology $\frak S'$ on $Y$
with the same Borel sets and the same analytic sets as $\frak S$.
\Hint{423Xd.}  (ii) Show that there is a zero-dimensional
separable metrizable topology $\frak T'$ on $X$, with the same Borel
sets and the same analytic sets as $\frak T$, such that $f$ is
continuous for the topologies $\frak T'$ and
$\frak S'$.   (iii) Explain how to elaborate these ideas
to deal with any countable family of analytic spaces and Borel
measurable functions between them.
%423G, 423Xd

\spheader 423Yd Let $X$ be an analytic Hausdorff space, $Y$ a Hausdorff
space with a countable network, and $f:X\to Y$ a Borel measurable
surjection.   Let
$\Tau$ be the $\sigma$-algebra of subsets of $Y$ generated by the
Souslin-F sets
in $Y$.    Show that there is a $\Tau$-measurable function $g:Y\to X$
such that $fg$ is the identity on $Y$.
%423O

\spheader 423Ye Set $S^*=\bigcup_{n\ge 1}\BbbN^n$ and consider 
$\Cal P(S^*)$
with its usual topology.   Let $\Cal T\subseteq\Cal PS^*$ be the set of
trees (421N);  show that $\Cal T$ is closed, therefore a compact
metrizable space.   Set $F_{\sigma}=\{T:\sigma\in T\in\Cal T\}$ for
$\sigma\in S^*$, and let $A$ be the kernel of the Souslin scheme
$\family{\sigma}{S^*}{F_{\sigma}}$.   Show that the constituents of
$\Cal T\setminus A$ for this scheme are just the sets
$G_{\xi}=\{T:r(T)=\xi\}$, where $r$ is the rank function of 421Ne.
Show by induction on $\xi$ that all the $G_{\xi}$ is non-empty, so that
$A$ is not a Borel set.   Show that $\#(G_{\xi})=\frak c$ for
$1\le\xi<\omega_1$.   Show that if $X$ is any topological space and
$B\subseteq X$ is a Souslin-F set, there is a Borel measurable function
$f:X\to\Cal T$ such that $B=f^{-1}[A]$.
%423Q
}%end of exercises

\endnotes{
\Notesheader{423}
We have been dealing, in this section and the last, with three classes
of topological space:  the class of analytic spaces, the class of
K-analytic spaces and the class of spaces with countable networks.   The
first is more important than the other two put together, and I am sure
many people would find it more comfortable, if more time-consuming, to
learn the theory of
analytic spaces thoroughly first, before proceeding to the others.
This was indeed my own route into the subject.   But I think that the
theory of K-analytic spaces has now matured to the point that it can
stand on its own, without constant reference to its origin as an
extension of descriptive set theory on the real line;  and that our
understanding of analytic spaces is usefully advanced by seeing how
easily their properties
can be deduced from the fact that they are K-analytic spaces with
countable networks.

As in \S422, I have made no attempt to cover the general theory of
analytic spaces, nor even to give a balanced introduction.   I have
tried instead to give a condensed account of the principal methods for
showing that spaces are analytic, with some of the ideas which can
be applied to make them more accessible to the imagination (423J,
423Xc-423Xd, 423Yb-423Yc).   Lusin's theorem 423I does not
mention `analytic' sets in its statement, but it depends essentially on
the separation theorem 422J, so cannot really be put with the other
results on Polish spaces in 4A2Q.   You must of course know that not all
analytic sets are Borel (423L) and that not all sets are analytic
(423Xh).   For further information about this fascinating subject, see
{\smc Rogers 80}, {\smc Kechris 95} and {\smc Moschovakis 80}.

`Selection theorems' appear everywhere in mathematics.   The axiom of
choice is a selection theorem;  it says that whenever
$R\subseteq X\times Y$ is a relation and $R[X]=Y$, there is a function
$f:Y\to X$ such that $(f(y),y)\in R$ for every $y\in Y$.   The Lifting
Theorem (\S341) asks for a selector which is a Boolean homomorphism.   In
general topology we look for continuous selectors.   In measure theory,
naturally, we are interested in measurable selectors, as in
423M-423O.  %423M 423N 423O
Any selection theorem will have expressions either as a theorem on right
inverses of functions, as in 423O, or as a theorem on selectors for
relations, as in 423M-423N.  In the language here, however, we get
better theorems by examining relations, because the essence of the
method is that we can find measurable functions into analytic spaces,
and the relations of 423M can be very far from being analytic, even when
there is a natural topology on the space $Y$.   The value of these
results will become clearer in \S433, when we shall see that the
$\sigma$-algebras $\Tau$ of 423M-423O %423M 423N 423O
are often included in familiar $\sigma$-algebras.   Typical applications
are in 433F-433G below.
}%end of notes

\leaveitout{The irrational slope topology ({\smc Steen \& Seebach 78},
\S75) is a countable, second-countable connected Hausdorff space;  of
course it is analytic.
}%end of leaveitout

\discrpage

