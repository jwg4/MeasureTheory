\frfilename{mt2a5.tex}
\versiondate{13.11.07}
\copyrightdate{1996}

\def\chaptername{Appendix}
\def\sectionname{Linear topological spaces}

\newsection{2A5}

The principal objective of \S2A3 is in fact the study of certain
topologies on the linear spaces of Chapter 24.   I give some fragments
of the general theory.

\leader{2A5A}{Linear space \dvrocolon{topologies}}\cmmnt{ Something
which is not covered in
detail by every introduction to functional analysis is the general
concept of `linear topological space'.   The ideas needed for the
work of \S245 are reasonably briefly expressed.

\medskip

\noindent}{\bf Definition} A {\bf linear topological space} or
{\bf topological vector space} over $\RoverC$ is a linear
space $U$ over $\RoverC$ together with a topology $\frak T$ such that
the maps

\Centerline{$(u,v)\mapsto u+v:U\times U\to U$,}

\Centerline{$(\alpha,u)\mapsto \alpha u:\RoverC\times U\to U$}

\noindent are both continuous, where the product spaces $U\times U$ and
$\RoverC\times U$ are given their product topologies\cmmnt{ (2A3T)}.
Given a linear space $U$, a topology on $U$ satisfying the conditions
above is a {\bf linear space topology}.   Note that

\Centerline{$(u,v)\mapsto u-v=u+(-1)v:U\times U\to U$}

\noindent will also be continuous.

\leader{2A5B}{}\cmmnt{ All the linear topological spaces we need
turn out to be readily presentable in the following terms.

\medskip

\noindent}{\bf Proposition} Suppose that $U$ is a linear space over
$\RoverC$, and $\Tau$
is a family of functionals $\tau:U\to\coint{0,\infty}$ such that

\quad (i) $\tau(u+v)\le\tau(u)+\tau(v)$ for all $u$, $v\in U$,
$\tau\in\Tau$;

\quad (ii) $\tau(\alpha u)\le\tau(u)$ if $u\in U$, $|\alpha|\le 1$,
$\tau\in\Tau$;

\quad (iii) $\lim_{\alpha\to 0}\tau(\alpha u)=0$ for every $u\in U$,
$\tau\in \Tau$.

\noindent For $\tau\in\Tau$, define
$\rho_{\tau}:U\times U\to\coint{0,\infty}$ by setting
$\rho_{\tau}(u,v)=\tau(u-v)$ for all $u$, $v\in U$.   Then each
$\rho_{\tau}$
is a pseudometric on $U$, and the topology defined by
$\Rho=\{\rho_{\tau}:\tau\in\Tau\}$ renders $U$ a linear topological
space.

\proof{{\bf (a)} It is worth noting immediately that

\Centerline{$\tau(0)=\lim_{\alpha\to 0}\tau(\alpha 0)=0$}

\noindent for every $\tau\in\Tau$.

\medskip

{\bf (b)} To see that every $\rho_{\tau}$ is a pseudometric, argue as
follows.

\medskip

\quad{\bf (i)} $\rho_{\tau}$ takes values
in $\coint{0,\infty}$ because $\tau$ does.

\medskip

\quad{\bf (ii)} If $u$, $v$, $w\in U$ then

$$\eqalign{\rho_{\tau}(u,w)
&=\tau(u-w)
=\tau((u-v)+(v-w))\cr
&\le\tau(u-v)+\tau(v-w)
=\rho_{\tau}(u,v)+\rho_{\tau}(v,w).\cr}$$

\medskip

\quad{\bf (iii)} If $u$, $v\in U$, then

\Centerline{$\rho(v,u)=\tau(v-u)
=\tau(-1(u-v))\le\tau(u,v)=\rho_{\tau}(u,v)$,}

\noindent and similarly $\rho_{\tau}(u,v)\le\rho_{\tau}(v,u)$, so the
two are equal.

\medskip

\quad{\bf (iv)} If $u\in U$ then $\rho_{\tau}(u,u)=\tau(0)=0$.

\medskip

{\bf (c)} Let $\frak T$ be the topology on $U$ defined by
$\{\rho_{\tau}:\tau\in\Tau\}$ (2A3F).

\medskip

\quad{\bf (i)} Addition is continuous because, given $\tau\in\Tau$, we
have

$$\eqalign{\rho_{\tau}(u'+v',u+v)
&=\tau((u'+v')-(u+v))\cr
&\le\tau(u'-u)+\tau(v'-v)
\le\rho_{\tau}(u',u)+\rho_{\tau}(v',v)\cr}$$

\noindent for all $u$, $v$, $u'$, $v'\in U$.   This means that, given
$\epsilon>0$ and $(u,v)\in U\times U$, we shall have

\Centerline{$\rho_{\tau}(u'+v',u+v)\le\epsilon$ whenever
$(u',v')\in
U((u,v);\tilde\rho_{\tau},\bar\rho_{\tau};\Bover{\epsilon}2)$,}

\noindent using the language of 2A3Tb.   Because $\tilde\rho_{\tau}$,
$\bar\rho_{\tau}$ are two of the pseudometrics defining the product
topology of $U\times U$ (2A3Tb), $(u,v)\mapsto u+v$ is continuous, by
the criterion of 2A3H.

\medskip

\quad{\bf (ii)} Scalar multiplication is continuous because if $u\in U$
and $n\in\Bbb N$
then $\tau(nu)\le n\tau(u)$ for every $\tau\in\Tau$ (induce on $n$).
Consequently, if $\tau\in\Tau$,

\Centerline{$\tau(\alpha u)\le n\tau(\Bover{\alpha}{n}u)\le n\tau(u)$}

\noindent whenever $|\alpha|<n\in\Bbb N$ and $\tau\in\Tau$.  Now, given
$(\alpha,u)\in \RoverC\times U$ and $\epsilon>0$, take $n>|\alpha|$ and
$\delta>0$ such that $\delta\le\min(n-|\alpha|,\bover{\epsilon}{2n})$
and $\tau(\gamma u)\le\bover{\epsilon}2$ whenever $|\gamma|\le\delta$;
then

$$\eqalign{\rho_{\tau}(\alpha'u',\alpha u)
&=\tau(\alpha'u'-\alpha u)
\le\tau(\alpha'(u'-u))+\tau((\alpha'-\alpha)u)\cr
&\le n\tau(u'-u)+\tau((\alpha'-\alpha)u)\cr}$$

\noindent whenever  $u'\in U$ and  $\alpha'\in\RoverC$
and $|\alpha'|<n\in\Bbb N$.   Accordingly, setting
$\theta(\alpha',\alpha)=|\alpha'-\alpha|$ for $\alpha'$,
$\alpha\in\RoverC$,

\Centerline{$\rho_{\tau}(\alpha'u',\alpha u)
\le n\delta+\Bover{\epsilon}2\le\epsilon$}

\noindent whenever

\Centerline{$(\alpha',u')\in
U((\alpha,u);\tilde\theta,\bar\rho_{\tau};\delta)$.}

\noindent Because $\tilde\theta$ and $\bar\rho_{\tau}$ are among the
pseudometrics defining the topology of $\RoverC\times U$, the map
$(\alpha,u)\mapsto\alpha u$ satisfies the criterion of 2A3H and is
continuous.

Thus $\frak T$ is a linear space topology on $U$.
}%end of proof of  2A5B

\medskip

\noindent{\bf Remark}\dvAnew{2013} 
Functionals satisfying the conditions
(i)-(iii) above are called {\bf F-seminorms};  an F-seminorm $\tau$ such
that $\tau(u)\ne 0$ for every non-zero $u$ is an {\bf F-norm}.
 
\leader{*2A5C}{}\cmmnt{ We do not need it for Chapter 24, but the
following is worth knowing.

\medskip

\noindent}{\bf Theorem} Let $U$ be a linear space and $\frak T$ a linear
space topology on $U$.

(a) There is a family $\Tau$ of F-seminorms defining $\frak T$ as in
2A5B.

(b) If $\frak T$ is metrizable, we can take $\Tau$ to consist of a
single functional.

\proof{{\bf (a)} {\smc Kelley \& Namioka 76}, p.\ 50.

\medskip

{\bf (b)} {\smc K\"othe 69}, \S15.11.
}%end of proof of 2A5C

\vleader{72pt}{2A5D}{Definition} Let $U$ be a linear space over $\RoverC$.
Then a {\bf seminorm} on $U$ is a functional $\tau:U\to\coint{0,\infty}$
such that

\quad (i) $\tau(u+v)\le\tau(u)+\tau(v)$ for all $u$, $v\in U$;

\quad (ii) $\tau(\alpha u)=|\alpha|\tau(u)$ if $u\in U$,
$\alpha\in\RoverC$.

\cmmnt{\noindent Observe that a norm is always a seminorm, and that a
seminorm is always an F-seminorm.   In
particular, the association of a metric with a norm (2A4Bb) is a special
case of 2A5B.}

\leader{2A5E}{Convex sets (a)} Let $U$ be a linear space over $\RoverC$.
A subset $C$ of $U$ is {\bf convex} if
$\alpha u+(1-\alpha)v\in C$ whenever $u$, $v\in C$ and $\alpha\in[0,1]$.
The intersection of any family of convex sets is convex, so for every
set $A\subseteq U$ there is a smallest convex set including $A$;
this is\cmmnt{ just} the set of vectors expressible as
$\sum_{i=0}^n\alpha_iu_i$
where $u_0,\ldots,u_n\in A$, $\alpha_0,\ldots,\alpha_n\in[0,1]$ and
$\sum_{i=0}^n\alpha_i=1$\prooflet{ ({\smc Bourbaki 87}, II.2.3)};  it
is the {\bf convex hull} of $A$.   If $C$, $C'\subseteq U$ 
are convex, and $\alpha\in\RoverC$, 
then $\alpha C$ and $C+C'$ are convex\dvAnew{2012}.

\spheader 2A5Eb If $U$ is a linear topological space, the closure of
any convex set is convex\prooflet{ ({\smc Bourbaki 87}, II.2.6)}.   It
follows that, for any $A\subseteq U$, the closure of the convex hull of
$A$ is the smallest closed convex set including $A$;  this is the
{\bf closed convex hull} of $A$.

\spheader 2A5Ec I note for future reference that in a linear topological
space, the closure of any linear subspace is a linear subspace.
\prooflet{({\smc Bourbaki 87}, I.1.3; {\smc K\"othe 69}, \S15.2.   Compare
2A4Cb.)}

\leader{2A5F}{Completeness in linear topological
\dvrocolon{spaces}}\cmmnt{ In normed spaces,
completeness can be described in terms of Cauchy sequences (2A4D).   In
general linear topological spaces this is inadequate.   The true theory
of `completeness' demands the concept of `uniform space' (see \S3A4 in
the next volume, or
{\smc Kelley 55}, chap.\ 6;  {\smc Engelking 89}, \S8.1:
{\smc Bourbaki 66}, chap.\ II);  I shall not
describe this here, but will give a version adapted
to linear spaces.   I mention this only because you will I hope some day
come to the general theory (in Volume 3 of this treatise, if not
before), and you should be aware that the special
case described here gives a misleading emphasis at some points.

\medskip

\noindent}{\bf Definitions} Let $U$ be a linear space over $\RoverC$,
and $\frak T$ a
linear space topology on $U$.   A filter $\Cal F$ on $U$ is {\bf Cauchy}
if for every open set $G$ in $U$ containing $0$ there is an $F\in\Cal F$
such that $F-F=\{u-v:u,\,v\in F\}$ is included in $G$.
$U$ is {\bf complete} if every Cauchy filter on $U$ is convergent.

\leader{2A5G}{}\cmmnt{ Cauchy filters have a simple description when a
linear space topology is defined by the method of 2A5B.

\medskip

\noindent}{\bf Lemma} Let $U$ be a linear space over $\RoverC$, and let
$\Tau$ be a family of F-seminorms defining a linear space topology on
$U$, as in 2A5B.   Then a filter $\Cal F$ on $U$ is
Cauchy iff for every $\tau\in\Tau$ and $\epsilon>0$ there is an
$F\in\Cal F$ such that $\tau(u-v)\le\epsilon$ for all $u$, $v\in F$.

\proof{{\bf (a)} Suppose that $\Cal F$ is Cauchy, $\tau\in\Tau$ and
$\epsilon>0$.   Then
$G=U(0;\rho_{\tau};\epsilon)$ is open (using the language of 2A3F-2A3G),
so there is an $F\in\Cal F$ such that $F-F\subseteq G$;  but this just
means that $\tau(u-v)<\epsilon$ for all $u$, $v\in F$.

\medskip

{\bf (b)} Suppose that
$\Cal F$ satisfies the criterion, and that $G$ is an open set containing
$0$.   Then there are $\tau_0,\ldots,\tau_n\in\Tau$ and $\epsilon>0$
such that $U(0;\rho_{\tau_0},\ldots,\rho_{\tau_n};\epsilon)\subseteq G$.
For each $i\le n$ there is an $F_i\in\Cal F$ such that
$\tau_i(u,v)<\bover{\epsilon}2$ for all $u$, $v\in F_i$;  now
$F=\bigcap_{i\le n}F_i\in\Cal F$ and $u-v\in G$ for all $u$, $v\in F$.
}%end of proof of 2A5G

\leader{2A5H}{Normed spaces and sequential
\dvrocolon{completeness}}\cmmnt{ I had
better point out that for normed spaces the definition of 2A5F agrees
with that of 2A4D.

\medskip

\noindent}{\bf Proposition} Let $(U,\|\,\|)$ be a normed space over
$\RoverC$, and let $\frak T$ be the linear space topology on $U$ defined
by the method of 2A5B from the set $\Tau=\{\|\,\|\}$.   Then $U$ is
complete in the sense of 2A5F iff it is complete in the sense of 2A4D.

\proof{{\bf (a)} Suppose first that $U$ is complete in the sense of
2A5F.   Let $\sequencen{u_n}$ be a sequence in $U$ which is Cauchy in
the sense of 2A4Da.   Set

\Centerline{$\Cal F
=\{F:F\subseteq U,\,\{n:u_n\notin F\}\text{ is finite}\}$.}

\noindent Then it is easy to check that $\Cal F$ is a filter on $U$, the
image of the Fr\'echet filter under the map $n\mapsto u_n:\Bbb N\to U$.
If $\epsilon>0$, take $m\in\Bbb N$ such that $\|u_j-u_k\|\le\epsilon$
whenever $j$, $k\ge m$;  then $F=\{u_j:j\ge m\}$ belongs to $\Cal F$,
and $\|u-v\|\le\epsilon$ for all $u$, $v\in F$.   So $\Cal F$ is Cauchy
in the sense of 2A5F, and has a
limit $u$ say.   Now, for any $\epsilon>0$, the set
$\{v:\|v-u\|<\epsilon\}=U(u;\rho_{\|\,\|};\epsilon)$ is an open set
containing $u$, so belongs to $\Cal F$, and $\{n:\|u_n-u\|\ge\epsilon\}$
is finite, that is, there is an $m\in\Bbb N$ such that
$\|u_m-u\|<\epsilon$ whenever $n\ge m$.   As $\epsilon$ is arbitrary,
$u=\lim_{n\to\infty}u_n$ in the sense of 2A3M.   As $\sequencen{u_n}$
is arbitrary, $U$ is complete in the sense of 2A4D.

\medskip

{\bf (b)} Now suppose that $U$ is complete in the sense of 2A4D.
Let $\Cal F$ be a Cauchy filter on $U$.   For each $n\in\Bbb N$, choose
a set $F_n\in\Cal F$ such that $\|u-v\|\le 2^{-n}$ for all $u$,
$v\in F_n$.   For each $n\in\Bbb N$, $F'_n=\bigcap_{i\le n}F_i$ belongs
to $\Cal F$, so
is not empty;  choose $u_n\in F'_n$.   If $m\in\Bbb N$ and $j$,
$k\ge m$, then both $u_j$ and $u_k$ belong to $F_m$, so
$\|u_j-u_k\|\le 2^{-m}$;  thus $\sequencen{u_n}$ is a Cauchy sequence in
the sense of 2A4Da, and has a limit $u$ say.   Now take
any $\epsilon>0$ and $m\in\Bbb N$ such that $2^{-m+1}\le\epsilon$.
There is surely a $k\ge m$ such that $\|u_k-u\|\le 2^{-m}$;  now
$u_k\in F_m$, so

\Centerline{$F_m\subseteq\{v:\|v-u_k\|\le 2^{-m}\}
\subseteq\{v:\|v-u\|\le 2^{-m+1}\}
\subseteq\{v:\rho_{\|\,\|}(v,u)\le\epsilon\}$,}

\noindent and $\{v:\rho_{\|\,\|}(v,u)\le\epsilon\}\in\Cal F$.
As $\epsilon$ is arbitrary, $\Cal F$ converges to $u$, by 2A3Sd.   As
$\Cal F$ is arbitrary, $U$ is complete.

\medskip

{\bf (c)}
Thus the two definitions coincide, provided at least that we allow the
countably many simultaneous choices of the $u_n$ in part (b) of the
proof.
}%end of proof of 2A5H

\leader{2A5I}{Weak topologies}\cmmnt{ I come now to brief notes on
`weak topologies' on normed spaces;
from the point of view of this volume, these
are in fact the primary examples of linear space topologies.}
Let $U$ be a normed linear space over $\RoverC$.

\spheader 2A5Ia  Write $U^*$ for its dual
$\eurm B(U;\RoverC)$\cmmnt{ (2A4H)}.   \cmmnt{If $h\in U^*$, then
$|h|:U\to\coint{0,\infty}$ is a seminorm, so}
$\Tau=\{|h|:h\in U^*\}$ defines a linear space topology on $U$\cmmnt{, 
by 2A5B};  
this is\cmmnt{ called} the {\bf weak topology} of $U$.

\spheader 2A5Ib A filter $\Cal F$ on $U$ converges to $u\in U$ for the
weak topology of $U$ iff\cmmnt{ $\lim_{v\to\Cal F}\rho_{|h|}(v,u)=0$
for every
$h\in U^*$ (2A3Sd), that is, iff $\lim_{v\to\Cal F}|h(v-u)|=0$ for every
$h\in U^*$, that is,
iff} $\lim_{v\to\Cal F}h(v)=h(u)$ for every $h\in U^*$.

\spheader 2A5Ic A set $C\subseteq U$ is called {\bf weakly compact} if
it is compact for the weak topology of $U$.   So\cmmnt{ (subject to
the axiom of choice)} a set $C\subseteq U$ is weakly compact iff for
every ultrafilter
$\Cal F$ on $U$ containing $C$ there is a $u\in C$ such that
$\lim_{v\to\Cal F}h(v)=h(u)$ for every $h\in U^*$\cmmnt{ (put 2A3R
together with (b) above)}.

\spheader 2A5Id A subset $A$ of $U$ is called {\bf relatively weakly
compact} if it is a subset of some weakly compact subset of $U$.

\spheader 2A5Ie If $h\in U^*$, then $h:U\to\RoverC$ is continuous for
the weak
topology on $U$ and the usual topology of $\RoverC$\cmmnt{;  this is
obvious if we apply the criterion of 2A3H}.   So if
$A\subseteq U$ is relatively weakly compact, $h[A]$ must be bounded in
$\RoverC$.   \prooflet{\Prf\ Let $C\supseteq A$ be a weakly compact set.
Then $h[C]$ is compact in $\RoverC$, by 2A3Nb, so is bounded, by 2A2F
(noting that if the underlying field is $\Bbb C$, then it can be
identified, as metric space, with $\BbbR^2$).
Accordingly $h[A]$ also is bounded.\ \Qed}

\spheader 2A5If If $V$ is another normed space and $T:U\to V$ is a
bounded linear operator, then $T$ is continuous for the respective weak
topologies.   \prooflet{\Prf\ If $h\in V^*$ then the composition $hT$
belongs to $U^*$.   Now, for any $u$, $v\in U$,

\Centerline{$\rho_{|h|}(Tu,Tv)=|h(Tu-Tv)|=|hT(u-v)|=\rho_{|hT|}(u,v)$,}

\noindent taking $\rho_{|h|}$, $\rho_{|hT|}$ to be the pseudometrics on
$V$, $U$ respectively defined by the formula of 2A5B.   By 2A3H, $T$ is
continuous.\ \Qed}

\spheader 2A5Ig Corresponding to the weak topology on a normed space
$U$, we have the {\bf weak*} or {\bf w*-}topology on its dual
$U^*$, defined by the set $\Tau=\{|\hat u|:u\in U\}$, where I write
$\hat u(f)=f(u)$ for every $f\in U^*$, $u\in U$.   As in (a), this is
a linear space topology on $U^*$.   \cmmnt{(It is essential to
distinguish between the `weak*' topology and the `weak' topology on
$U^*$.   The former depends only on the action of $U$ on $U^*$, the
latter on the action of $U^{**}=(U^*)^*$.   You will have no difficulty
in checking that $\hat u\in U^{**}$ for every $u\in U$, but the point is
that there may be members of $U^{**}$ not representable in this way,
leading to open sets for the weak topology which are not open for the
weak* topology.)}

\leader{*2A5J}{Angelic spaces}\cmmnt{ I do not rely on the
following ideas, but they may throw light on some results in
\S\S246-247.}   First, a topological space $X$ is {\bf regular} if
whenever $G\subseteq X$ is open and $x\in G$ then there is an open set
$H$ such that $x\in H\subseteq\overline{H}\subseteq G$.   Next, a
regular Hausdorff space $X$ is {\bf angelic} if whenever $A\subseteq X$
is such that every sequence in $A$ has a cluster point in $X$, then
$\overline{A}$ is compact and every point of $\overline{A}$ is the limit
of a sequence in $A$.\cmmnt{  What this means is that compactness in
$X$, and
the topologies of compact subsets of $X$, can be effectively described
in terms of sequences.}   Now\cmmnt{ the theorem (due to Eberlein and
\v Smulian) is that} any normed space is
angelic in its weak topology.\cmmnt{   (462D in Volume 4;
{\smc K\"othe 69},
\S24;  {\smc Dunford \& Schwartz 57}, V.6.1.)   In
particular, this is true of $L^1$ spaces, which makes it less surprising
that there should be criteria for weak compactness in $L^1$ spaces which
deal only with sequences.
}%end of comment

\discrpage

