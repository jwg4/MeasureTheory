\frfilename{mt35.tex} 
\versiondate{4.9.09} 
\copyrightdate{1995} 
 
\def\chaptername{Riesz spaces} 
\def\sectionname{Introduction} 
 
\newchapter{35} 
 
The next three chapters are devoted to an abstract description of the 
`function spaces' described in Chapter 24, this time concentrating on 
their internal structure and relationships with their associated 
measure algebras.   I find that any convincing account of these must 
involve a substantial amount of general theory concerning partially 
ordered linear spaces, and in particular various types of Riesz space or 
vector lattice.   I therefore provide an introduction to this theory, a 
kind of appendix built into the middle of the volume.   The relation of 
this chapter to the next two is very like the relation of Chapter 31 to 
Chapter 32.   As with Chapter 31, it is not really meant to be read for 
its own sake;  those with a particular interest in Riesz spaces might be 
better served by {\smc Luxemburg \& Zaanen 71}, {\smc Schaefer 74}, 
{\smc Zaanen 83} or my own book {\smc Fremlin 74a}. 
 
I begin with three sections in an easy gradation towards the particular 
class of spaces which we need to understand:  partially ordered linear 
spaces (\S351), general Riesz spaces (\S352) and Archimedean Riesz 
spaces (\S353);  the last includes notes on Dedekind 
($\sigma$-\nobreak)\penalty-100complete 
spaces.   These sections cover the fragments of the algebraic theory of 
Riesz spaces which I will use.   In the second half of the chapter, I deal 
with normed Riesz spaces (in particular, $L$- and $M$-spaces)(\S354), 
spaces of linear operators (\S355) and dual Riesz spaces (\S356). 
 
\discrpage 
 
 
