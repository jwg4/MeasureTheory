\frfilename{mt529.tex}
\versiondate{26.5.11}
\copyrightdate{2003}

\def\chaptername{Cardinal functions of measure theory}
\def\sectionname{Further partially ordered sets of measure theory}

\def\Fn{\mathop{\text{Fn}}\nolimits}

\newsection{529}

I end the chapter with notes on some more structures which can be
approached by the methods used earlier.   The Banach lattices of Chapter
36 are of course partially ordered sets, and many of them can easily be
assigned places in the Tukey classification
(529C, 529D\cmmnt{, 529Xa}).   More surprising is the fact that the
Nov\'ak numbers of $\{0,1\}^I$, for large $I$, are supported by the
additivity of Lebesgue measure (529F);   this is associated with an
interesting property of the localization poset from the last section
(529E).   There is a similarly unexpected connexion between the covering
number of Lebesgue measure and `reaping numbers'
$\frak r(\omega_1,\lambda)$ for large $\lambda$ (529H).

\leader{529A}{Notation} \cmmnt{As in previous sections,} I will write
$\Cal N(\mu)$ for the null ideal of $\mu$ in a measure space
$(X,\Sigma,\mu)$, and $\Cal N$ for the null ideal of Lebesgue measure on
$\Bbb R$.

\leader{529B}{Proposition} Let $(\frak A,\bar\mu)$ be a semi-finite
measure algebra.

(a) For $p\in\coint{1,\infty}$, give
$L^p=L^p(\frak A,\bar\mu)$\cmmnt{ (definition: 366A)} its norm
topology.   Then its topological density is

$$\eqalign{d(L^p)
&=1\text{ if }\frak A=\{0\},\cr
&=\omega\text{ if }0<\#(\frak A)<\omega,\cr
&=\max(c(\frak A),\tau(\frak A))
  \text{ if }\frak A\text{ is infinite.}\cr}$$

(b) Give $L^0=L^0(\frak A)$ its topology of convergence in
measure\cmmnt{ (367L)}.   Then

$$\eqalign{d(L^0)
&=1\text{ if }\frak A=\{0\},\cr
&=\omega\text{ if }0<\#(\frak A)<\omega,\cr
&=\tau(\frak A)\text{ if }\frak A\text{ is infinite.}\cr}$$

\proof{{\bf (a)(i)} The case in which $\frak A$ is finite is elementary,
since in this case $L^p\cong\BbbR^n$, where $n$ is the
number of atoms of $\frak A$.   So henceforth let us suppose that
$\frak A$ is infinite.

\medskip

\quad{\bf (ii)} If $\frak A^f$ is the set of elements of $\frak A$ of
finite
measure, we have a natural injection $a\mapsto\chi a:\frak A^f\to L^p$, and
$\|\chi a-\chi b\|_p=\mu(a\Bsymmdiff b)^{1/p}$, so $\chi$ is a
homeomorphism
for the measure metric on $\frak A^f$ and the norm metric on $L^p$.
It follows that the density $d(A)$ of $A=\{\chi a:a\in\frak A^f\}$ for the
norm topology is equal to the density of $\frak A^f$ for the strong
measure-algebra topology, which is $\max(c(\frak A),\tau(\frak A))$, by
521Eb.   So

\Centerline{$\max(c(\frak A),\tau(\frak A))=d(A)\le d(L^p)$}

\noindent by 5A4B(h-ii).
In the other direction, if $A_0$ is a dense subset of $A$
of size $d(A)$ and $D$ is the set of rational
linear combinations of members of $A_0$, $\overline{D}\supseteq S(\frak A^f)$
is dense in $L^p$ (366C), so

\Centerline{$d(L^p)\le\#(D)\le\max(\omega,\#(A_0))
=\max(c(\frak A),\tau(\frak A))$.}

\medskip

{\bf (b)} Again, the case of finite $\frak A$ is trivial, so we need
consider only infinite $\frak A$.   In this case, $\tau(\frak A)$ is
equal to the topological density $d_{\frak T}(\frak A)$
of $\frak A$ with its measure-algebra topology $\frak T$ (521Ea).

\medskip

\quad{\bf (i)} Let $A\subseteq\frak A$ be a topologically dense set of
cardinal $\tau(\frak A)$.   Set

\Centerline{$D=\{\sum_{i=0}^nq_i\chi a_i:q_0,\ldots,q_n\in\Bbb Q$,
$a_0,\ldots,a_n\in\frak A\}$,}

\noindent so that $D\subseteq L^0$ has cardinal $\tau(\frak A)$.
Because $a\mapsto\chi a:\frak A\to L^0$ is continuous (367R), the
closure $\overline{D}$ of $D$ includes $\{\chi a:a\in\frak A\}$.
Because $\overline{D}$ is a linear subspace of $L^0$, it includes
$S(\frak A)$.   Because $S(\frak A)$ is dense in $L^0$
(367Nc),
$\overline{D}=L^0$ and $d(L^0)\le\#(D)=\tau(\frak A)$.

\medskip

\quad{\bf (ii)} Let $B\subseteq L^0$ be a dense set with cardinal
$d(L^0)$.   Set

\Centerline{$A=\{\Bvalue{u>\bover12}:u\in B\}$,}

\noindent so that $A\subseteq\frak A$ and $\#(A)\le d(L^0)$.   Then $A$
is topologically dense in $\frak A$.   \Prf\ If $c\in\frak A$,
$a\in\frak A^f$ and $\epsilon>0$, there is a $u\in B$ such that
$\int|u-\chi c|\wedge\chi a\le\bover12\epsilon$.   But in this case,
setting $b=\Bvalue{u>\bover12}$, $|\chi(b\Bsymmdiff c)|\le 2|u-\chi c|$,
so

\Centerline{$\bar\mu(a\Bcap(b\symmdiff c))
\le 2\int|u-\chi c|\wedge\chi a\le\epsilon$.}

\noindent As $c$, $a$ and $\epsilon$ are arbitrary, $A$ is topologically
dense in $\frak A$.\ \Qed

Accordingly

\Centerline{$\tau(\frak A)=d_{\frak T}(\frak A)\le\#(A)\le d(L^0)$}

\noindent and $d(L^0)=\tau(\frak A)$, as claimed.
}%end of proof of 529B

\leader{529C}{Theorem}\cmmnt{ ({\smc Fremlin 91})} Let $U$ be an
$L$-space.   Then $U\equivT\ell^1(\kappa)$, where $\kappa=\dim U$ if $U$
is finite-dimensional, and otherwise is the topological density of $U$.

\proof{{\bf (a)} The finite-dimensional case is trivial, since in this
case $U$ and $\ell^1(\kappa)$ are isomorphic as Banach lattices.   So
henceforth let us suppose that $U$ is infinite-dimensional.   Now
$\vee:U\times U\to U$ is uniformly continuous.
\Prf\ We have only to observe that $u\vee v=\bover12(u+v+|u-v|)$ in any
Riesz space, that addition and subtraction are uniformly continuous in
any linear topological space, and that $u\mapsto|u|$ is uniformly
continuous just because $||u|-|v||\le|u-v|$ (see 354B).\ \QeD\
So 524C, with $Q=P=U$, tells us that $U\prT\ell^1(\kappa)$.

The rest of the proof will therefore be devoted to showing that
$\ell^1(\kappa)\prT U$.

\medskip

{\bf (b)} By Kakutani's theorem (369E), there is a localizable measure
algebra $(\frak A,\bar\mu)$ such that $U$ is isomorphic, as Banach
lattice, to $L^1(\frak A,\bar\mu)$.   Let $\familyiI{a_i}$ be a
partition of unity in $\frak A$ such that $0<\bar\mu a_i<\infty$ and the
principal ideal $\frak A_{a_i}$ is homogeneous for each $i$.
Set $\kappa_i=\tau(\frak A_{a_i})$, so that $\kappa_i$ is either $0$ or
infinite for every $i$, and $\kappa=\max(\#(I),\sup_{i\in I}\kappa_i)$
by 529Ba.

It will simplify the calculations to follow if we arrange that all the
$a_i$ have measure $1$.   To do this, set
$\bar\nu a=\sum_{i\in I}\Bover{\bar\mu(a\Bcap a_i)}{\bar\mu a_i}$ for
$a\in\frak A$;  that is, $\bar\nu a=\int_aw\,d\bar\mu$, where
$w=\sup_{i\in I}\Bover1{\bar\mu a_i}\chi a_i$ in $L^0(\frak A)$.   In this
case, $\int v\,d\bar\nu=\int v\times w\,d\bar\nu$ for every
$v\in L^1(\frak A,\bar\nu)$, while
$\int u\,d\bar\mu=\int\Bover1w\times u\,d\bar\nu$ for every
$u\in L^1(\frak A,\bar\mu)$ (365S\formerly{3{}65T}).   But this means that
$u\mapsto u\times\Bover1w$ is a Banach lattice isomorphism between
$L^1(\frak A,\bar\mu)$ and $L^1(\frak A,\bar\nu)$, and $U$ is
isomorphic, as $L$-space, to $L^1=L^1(\frak A,\bar\nu)$;  while
$\bar\nu a_i=1$ for every $i$.

\medskip

{\bf (c)} There are a set $J$, with cardinal $\kappa$, and a family
$\family{j}{J}{u_j}$ in $L^1$ such that $\#(J)=\kappa$, $\|u_j\|\le 2$
for every $j\in J$ and
$\|\sup_{j\in K}u_j\|\ge\bover12\sqrt{\#(K)}$ for every finite
$K\subseteq J$.   \Prf\ Set

\Centerline{$J=\{(i,0):i\in I$, $\kappa_i=0\}
  \cup\{(i,\xi):i\in I$, $\xi<\kappa_i\}$.}

\noindent Then $\#(J)=\kappa$.   If $i\in I$ and $\kappa_i=0$, set
$u_{(i,0)}=\chi a_i$.   If $i\in I$ and $\kappa_i>0$, then
$(\frak A_{a_i},\bar\nu\restrp\frak A_{a_i})$ is a homogeneous
probability algebra with Maharam type $\kappa_i\ge\omega$, so is
isomorphic to the measure algebra $(\frak C_i,\bar\lambda_i)$ of
$\ocint{0,1}^{\kappa_i}$ with its usual measure $\lambda_i$, the product
of Lebesgue measure on each copy of $\ocint{0,1}$ (334E).   For
$\xi<\kappa_i$, set

\Centerline{$h_{i\xi}(t)=\Bover1{\sqrt{t(\xi)}}$ for
$t\in\ocint{0,1}^{\kappa_i}$,}

\noindent and let $u_{(i,\xi)}\in L^1$ correspond to
$h_{i\xi}^{\ssbullet}\in L^1(\lambda_i)
\cong L^1(\frak C_i,\bar\lambda_i)$ (365B).   Of course

$$\eqalignno{\|u_{(i,\xi)}\|
&=\int h_{i\xi}(t)\lambda_i(dt)
=\int_0^1\Bover1{\sqrt{\alpha}}d\alpha\cr
\displaycause{because the coordinate map $t\mapsto t(\xi)$ is \imp}
&=2.\cr}$$

If $L\subseteq\kappa_i$ is finite and not empty, then
$\|\sup_{\xi\in L}u_{(i,\xi)}\|=\int g\,d\lambda_i$ where
$g=\sup_{\xi\in L}h_{i\xi}$, that is,
$g(t)=\sup_{\xi\in L}\Bover1{\sqrt{t(\xi)}}$ for
$t\in\ocint{0,1}^{\kappa_i}$.
Now, for any $\alpha\ge 1$,

$$\eqalignno{\lambda_i\{t:g(t)\le\alpha\}
&=\lambda_i\{t:\alpha^2 t(\xi)\ge 1\text{ for every }\xi\in L\}\cr
&=(1-\Bover1{\alpha^2})^{\#(L)}
\le\max(\Bover12,1-\Bover1{2\alpha^2}\#(L))\cr
\displaycause{induce on $\#(L)$}
&=1-\Bover1{2\alpha^2}\#(L)\text{ if }\alpha\ge\sqrt{\#(L)}.\cr}$$

\noindent So

$$\eqalignno{\|\sup_{\xi\in L}u_{(i,\xi)}\|
&=\int g\,d\lambda_i
=\int_0^{\infty}\lambda_i\{t:g(t)>\alpha\}d\alpha\cr
\displaycause{252O}
&\ge\int_{\sqrt{\#(L)}}^{\infty}\lambda_i\{t:g(t)>\alpha\}d\alpha\cr
&\ge\int_{\sqrt{\#(L)}}^{\infty}\Bover1{2\alpha^2}\#(L)d\alpha
=\Bover12\sqrt{\#(L)}.\cr}$$

What this means is that if $K\subseteq J$ is finite and all the first
coordinates of members of $K$ are the same, then
$\|\sup_{j\in K}u_j\|\ge\bover12\sqrt{\#(K)}$.   In general, if
$K\subseteq J$ is finite, then for each $i\in I$ set
$L_i=\{\xi:(i,\xi)\in K\}$.   Set $v_i=0$ if $L_i$ is empty,
$\sup_{\xi\in L_i}u_{(i,\xi)}$ otherwise, so that
$\|v_i\|\ge\bover12\sqrt{\#(L_i)}$;  now
$\sup_{j\in K}u_j=\sum_{i\in I}v_i$, so

\Centerline{$\|\sup_{j\in K}u_j\|=\sum_{i\in I}\|v_i\|
\ge\Bover12\sum_{i\in I}\sqrt{\#(L_i)}
\ge\Bover12\sqrt{\sumop_{i\in I}\#(L_i)}=\Bover12\sqrt{\#(K)}$,}

\noindent as required.\ \Qed

\medskip

{\bf (d)} We can now apply the idea of the proof of 524C, as follows.
The density of $\ell^1(\kappa)$ is of course $\kappa$, by 529Ba applied
to counting measure on $\kappa$, or otherwise.
Index a dense subset of $\ell^1(\kappa)$ as
$\family{j}{J}{y_j}$.   For each $x\in\ell^1$, choose a sequence
$\sequencen{k(x,n)}$ in $J$ such that

\Centerline{$\|x-\sum_{m=0}^ny_{k(x,m)}\|\le 8^{-n}$}

\noindent for every $n$.   Note that

\Centerline{$\|y_{k(x,n)}\|
\le\|x-\sum_{m=0}^ny_{k(x,m)}\|+\|x-\sum_{m=0}^{n-1}y_{k(x,m)}\|
\le 9\cdot 8^{-n}$}

\noindent for each $n$.
Choose $f(x)\in L^1$ such that $\|f(x)\|\ge\|x\|$ and
$f(x)\ge\sum_{n=0}^{\infty}2^{-n}u_{k(x,n)}$;
this is possible because $\{u_{k(x,n)}:n\in\Bbb N\}$ is bounded.

\medskip

{\bf (e)} $f$ is a Tukey function.   \Prf\ Take $v\in L^1$ and set

\Centerline{$A=\{x:f(x)\le v\}$,
\quad$K_n=\{k(x,n):x\in A\}$}

\noindent for $n\in\Bbb N$.   Fix $n$ for the moment.   If $j\in K_n$,
then there is an $x\in A$ such that $j=k(x,n)$ and

\Centerline{$u_j=u_{k(x,n)}\le 2^nf(x)\le 2^nv$,}

\noindent while $\|y_j\|=\|y_{k(x,n)}\|\le 9\cdot 8^{-n}$.   If
$K\subseteq K_n$ is finite, $\|2^nv\|\ge\bover12\sqrt{\#(K)}$, by (c);
so $\#(K_n)\le 2^{2n+2}\|v\|^2$.   This means that if we set
$z_n=\sum_{j\in K_n}|y_j|$ we shall have
$\|z_n\|\le 9\cdot 8^{-n}\#(K_n)\le 36\cdot 2^{-n}\|v\|^2$, while
$y_{k(x,n)}\le z_n$ for every $x\in A$.

Now $z=\sum_{n=0}^{\infty}z_n$ is defined in $\ell^1(\kappa)$, and if
$x\in A$ then $\sum_{m=0}^ny_{k(x,m)}\le z$ for every $n\in\Bbb N$, so
that $x\le z$.   Thus $A$ is bounded above in $\ell^1(\kappa)$.   As $v$
is arbitrary, $f$ is a Tukey function.\ \Qed

\medskip

{\bf (f)} Accordingly $\ell^1(\kappa)\prT L^1\cong U$, and the proof is
complete.
}%end of proof of 529C

\leader{529D}{Theorem}\cmmnt{ ({\smc Fremlin 91})} Let $\frak A$ be a
homogeneous measurable algebra with Maharam type $\kappa\ge\omega$.  Then
$L^0(\frak A)\equivT\ell^1(\kappa)$.

\proof{{\bf (a)} Let $\bar\mu$ be such that $(\frak A,\bar\mu)$ is a
probability algebra.   If we give $L^0=L^0(\frak A)$ its topology of
convergence in measure, its density is $\kappa$, by 529Bb.   Moreover,
this topology is defined by the metric
$(u,v)\mapsto\int|u-v|\wedge\chi 1$, under which the lattice operation
$\vee$ is uniformly continuous.   \Prf\ Just as in part (a) of the proof
of 529C, we have $u\vee v=\bover12(u+v+|u-v|)$ for all $u$ and $v$,
addition and subtraction are uniformly continuous, and $u\mapsto|u|$ is
uniformly continuous.\ \QeD\   So, just as in
529C, we can use 524C to see that $L^0\prT\ell^1(\kappa)$.

\medskip

{\bf (b)} For the reverse connection, I repeat ideas from the proof
of 529C.   $(\frak A,\bar\mu)$ is isomorphic to the measure algebra
$(\frak C,\bar\lambda)$ of $\ocint{0,1}^{\kappa}$ with its usual measure
$\lambda$.    For $\xi<\kappa$ and $t\in\ocint{0,1}^{\kappa}$ set
$h_{\xi}(t)=\Bover1{\sqrt{t(\xi)}}$, and set
$u_{\xi}=h_{\xi}^{\ssbullet}$ in $L^0(\lambda)\cong L^0(\frak C)$
(364Ic).
This time, observe that if $x\in\ell^1(\kappa)^+$ and
$\alpha\ge\sqrt{\|x\|}$ then

$$\eqalignno{\bar\lambda\Bvalue{\sup_{\xi<\kappa}\sqrt{x(\xi)}u_{\xi}
  \le\alpha}
&=\bar\lambda(\inf_{\xi<\kappa}\Bvalue{\sqrt{x(\xi)}u_{\xi}\le\alpha})
=\prod_{\xi<\kappa}\bar\lambda\Bvalue{\sqrt{x(\xi)}u_{\xi}\le\alpha}\cr
&=\prod_{\xi<\kappa}\lambda\{t:\sqrt{\Bover{x(\xi)}{t(\xi)}}\le\alpha\}
=\prod_{\xi<\kappa}(1-\Bover{x(\xi)}{\alpha^2})\cr
&\ge 1-\Bover1{\alpha^2}\sum_{\xi<\kappa}x(\xi)
\to 0\cr}$$

\noindent as $\alpha\to\infty$.   This means that
$\sup_{\xi<\kappa}\sqrt{x(\xi)}u_{\xi}$ is defined in
$L^0(\lambda)$ (364La).   So we can define
$f:\ell^1(\kappa)\to L^0(\lambda)$ by saying that
$f(x)=\sup_{\xi<\kappa}\sqrt{\max(0,x(\xi))}u_{\xi}$ for every
$x\in\ell^1(\kappa)$.

\medskip

{\bf (c)} $f$ is a Tukey function.   \Prf\ Take $v\in L^0(\lambda)^+$,
and set $A=\{x:f(x)\le v\}$.    Note that
$f(x\vee x')=f(x)\vee f(x')$ for all $x$, $x'\in\ell^1(\kappa)$, so $A$
is upwards-directed.   Take $\alpha>0$ such that
$\bar\lambda\Bvalue{v\le\alpha}=\beta>\bover12$.
If $x\in A$ and $x\ge 0$
then $f(x)\ge\sqrt{x(\xi)}\chi 1$ so $x(\xi)\le\alpha$ for every
$\xi$.   Now the calculation in (b) tells us that

$$\eqalignno{\beta
&\le\bar\lambda\Bvalue{\sup_{\xi<\kappa}\sqrt{x(\xi)}u_{\xi}
  \le\alpha}
=\prod_{\xi<\kappa}(1-\Bover1{\alpha^2}x(\xi))\cr
&\le\max(\Bover12,1-\Bover1{2\alpha^2}\sum_{\xi<\kappa}x(\xi))
=\max(\Bover12,1-\Bover1{2\alpha^2}\|x\|),\cr}$$

\noindent so $\|x\|\le 2\alpha^2(1-\beta)$.   As $A$ is
upwards-directed and norm-bounded and contains $0$, it is bounded above in
$\ell^1(\kappa)$ (354N).   As $v$ is arbitrary, $f$ is a Tukey
function.\ \Qed

\medskip

{\bf (d)} Accordingly
$\ell^1(\kappa)\prT L^0(\lambda)\cong L^0(\frak A)$ and
$\ell^1(\kappa)$ and $L^0(\frak A)$ are Tukey equivalent.
}%end of proof of 529D

\leader{529E}{Proposition} Let $\Cal S^{\infty}$ be the
$\Bbb N$-localization
poset\cmmnt{ (528I)}.   Then $\RO(\{0,1\}^{\frakc})$ can be
regularly embedded in $\RO^{\uparrow}(\Cal S^{\infty})$.

\woddheader{529E}{0}{0}{0}{60pt}

\proof{{\bf (a)} Let $\ofamily{\xi}{\frakc}{h_{\xi}}$ be a family of
eventually-different functions in $\NN$ (5A1Mc).   Set

$$\eqalign{W_0
&=\bigcup_{n\in\Bbb N\text{ is even}}
  \{(h,p):h\in\NN,\,p\in\Cal S^{\infty},\,\#(p[\{n\}])=2^n,\cr
&\mskip180mu (n,h(n))\notin p,\,(i,h(i))\in p
  \text{ for every }i>n\}\cr
&\mskip90mu \cup\{(h,p):h\in\NN,\,p\in\Cal S^{\infty},\,(i,h(i))\in p
  \text{ for every }i\in\Bbb N\},\cr
W_1
&=\bigcup_{n\in\Bbb N\text{ is odd}}\{(h,p):h\in\NN,
\,p\in\Cal S^{\infty},\,\#(p[\{n\}])=2^n,\cr
&\mskip180mu (n,h(n))\notin p,\,(i,h(i))\in p
  \text{ for every }i>n\}.\cr}$$

\noindent Observe that (i) $W_0\cap W_1=\emptyset$ (ii) if
$(h,p)\in W_j$, where $j=0$ or $j=1$, and
$p\subseteq q\in\Cal S^{\infty}$ then
$(h,q)\in W_j$ (iii) if $p\in\Cal S^{\infty}$ then

\Centerline{$\#(\{\xi:(h_{\xi},p)\in W_0\cup W_1\})
\le\|p\|$}

\noindent is finite.

\medskip

{\bf (b)} Set $Q=\Fn_{<\omega}(\frak c,\{0,1\})$, the set of functions from
finite subsets of $\frak c$ to $\{0,1\}$, ordered
by extension of functions, so that $(Q,\subseteq)$ is isomorphic to
$(\Cal U,\supseteq)$ where
$\Cal U$ is the usual base of the topology of $\{0,1\}^{\frakc}$, and
$\RO^{\uparrow}(Q)\cong\RO^{\downarrow}(\Cal U)$
can be identified with the regular open algebra of $\{0,1\}^{\frak c}$
(514Sd).   Define $f:\Cal S^{\infty}\to Q$ by setting $f(p)(\xi)=j$ if
$(h_{\xi},p)\in W_j$.   Then $f$ is order-preserving.

\medskip

{\bf (c)} For $p\in\Cal S^{\infty}$, set

\Centerline{$A_0(p)
=\{\xi:\xi<\frak c$, $\{n:n$ is even, $(n,h_{\xi}(n))\notin p\}$ is
finite$\}$,}

\Centerline{$A_1(p)
=\{\xi:\xi<\frak c$, $\{n:n$ is odd, $(n,h_{\xi}(n))\notin p\}$ is
finite$\}$,}

\Centerline{$A(p)=A_0(p)\cup A_1(p)$,}

\noindent so that $A(p)$ is finite and $\dom f(p)\subseteq A(p)$.   Now
$P_1=\{p:p\in\Cal S^{\infty}$, $A(p)=\dom f(p)\}$ is cofinal with $\Cal
S^{\infty}$.   \Prf\
Take $p\in\Cal S^{\infty}$.   Let $m$ be such that

\Centerline{$2^m\ge\|p\|+\#(A(p))$,}

\Centerline{$(n,h_{\xi}(n))\in p$ whenever
$\xi\in A_0(p)$ and $n>m$ is even,}

\Centerline{$(n,h_{\xi}(n))\in p$ whenever
$\xi\in A_1(p)$ and $n>m$ is odd.}

\noindent Let $p'\in\Cal S^{\infty}$ be such that

\Centerline{for $n\le m$, $p'[\{n\}]\supseteq p[\{n\}]$ and
$\#(p'[\{n\}])=2^n$,}

\Centerline{for $n>m$,
$p'[\{n\}]=p[\{n\}]\cup\{h_{\xi}(n):\xi\in A(p)\}$.}

\noindent Then $p\le p'$ and $A(p')=A(p)$.   Also
$A(p)=\dom f(p')$, because if $\xi\in A(p)$ then either
$(n,h_{\xi}(n))\in p'$ for every $n$ and $(h_{\xi},p')\in W_0$, or there
is a largest $n$ such that $(n,h_{\xi}(n))\notin p'$, in which case
$n\le m$ and $\#(p'[\{n\}])=2^n$, so $(h_{\xi},p')$ belongs to $W_0$ if
$n$ is even and $W_1$ otherwise.\ \Qed

\medskip

{\bf (d)} If $p\in P_1$ and $q\in Q$ extends $f(p)$, there is a
$p_1\in\Cal S^{\infty}$ such that $p_1\supseteq p$ and $f(p_1)=q$.
\Prf\ Let $m$ be such that
$2^m\ge\|p\|+\#(\dom q)$ and $h_{\xi}(n)\ne h_{\eta}(n)$ whenever $\xi$,
$\eta\in\dom q$ are distinct and $n\ge m$.
For each $\xi\in\dom q\setminus\dom f(p)=\dom q\setminus A(p)$,
$\{n:n$ is even, $(n,h_{\xi}(n))\notin p\}$
and $\{n:n$ is odd, $(n,h_{\xi}(n))\notin p\}$ are both infinite.
So we can find $m'\ge m$ such that all these sets meet
$m'\setminus m$.   Set

$$\eqalign{p'
&=p\cup\{(n,h_{\xi}(n)):n\in m'\setminus m
  \text{ is odd},\,q(\xi)=0\}\cr
&\mskip200mu
\cup\{(n,h_{\xi}(n)):n\in m'\setminus m\text{ is even},\,q(\xi)=1\}\cr
&\mskip200mu
\cup\{(n,h_{\xi}(n)):n\in\Bbb N\setminus m',\,\xi\in\dom q\},
\cr}$$

\noindent so that $p\subseteq p'\in\Cal S^{\infty}$.   Let
$p_1\in\Cal S^{\infty}$ be such that $p_1\supseteq p'$,
$p_1\setminus p'$ is finite, $\#(p_1[\{n\}])=2^n$ for every $n<m'$ and
$(n,h_{\xi}(n))\notin p_1\setminus p'$ whenever $n\in\Bbb N$ and
$\xi\in\dom q$.   Now $f(p_1)=q$, while $p\subseteq p_1$.\ \Qed

\medskip

{\bf (e)} Putting (c) and (d) together, we see that $f^{-1}[Q_0]$ must
be cofinal with $\Cal S^{\infty}$ for every cofinal $Q_0\subseteq Q$;
moreover, since $\emptyset\in P_1$ and $f(\emptyset)$ is the empty
function, $f[\Cal S^{\infty}]=Q$.   So $f$ satisfies the conditions of
514O, and $\RO^{\uparrow}(Q)\cong\RO(\{0,1\}^{\frakc})$ can be regularly
embedded in $\RO^{\uparrow}(\Cal S^{\infty})$.
}%end of proof of 529E

%questions:  can \frak B_{\frak c} be regularly embedded in
%\RO^{\uparrow}(\Cal S^{\infty})?
%can \RO(\Bbb R) be regularly embedded in
%\AM(\frak B_{\omega},\bar\nu_{\omega}\bover12)?

\leader{529F}{Corollary}\cmmnt{ ({\smc Brendle 00}, 2.3.10;
{\smc Brendle 06}, Theorem 1)}
$n(\{0,1\}^I)\ge\add\Cal N$ for every set $I$.

\proof{ If $I$ is finite, this is trivial.   Otherwise, write
$\lambda=n(\{0,1\}^I)$.   Then $\lambda\ge n(\{0,1\}^{\frakc})$.   \Prf\
If $J\subseteq I$ is a countably infinite set, then
$\{\{x:x\restr J=z\}:z\in\{0,1\}^J\}$ is a cover of $\{0,1\}^I$ by
continuum many nowhere dense sets, so $\lambda\le\frak c$.   Let
$\ofamily{\xi}{\lambda}{E_{\xi}}$ be a cover of $\{0,1\}^{\kappa}$ by
nowhere dense sets.   Then each $E_{\xi}$ is included in a nowhere dense
closed set $F_{\xi}$ determined by coordinates in a countable set
$K_{\xi}\subseteq I$ (4A2E(b-iii)).   Set
$K=\bigcup_{\xi<\lambda}K_{\xi}$, so that $\#(K)\le\frak c$.   Then all
the projections $F'_{\xi}=\{x\restr K:x\in F_{\xi}\}$ are nowhere dense
in $\{0,1\}^K$ (apply 4A2B(f-ii) to the continuous open surjections
$x\mapsto x\restr K_{\xi}:\{0,1\}^I\to\{0,1\}^{K_{\xi}}$ and
$y\mapsto y\restr K_{\xi}:\{0,1\}^K\to\{0,1\}^K_{\xi}$),
and they cover $\{0,1\}^K$.
Next, we have an injection $\phi:K\to\frak c$, and the sets
$F''_{\xi}=\{x:x\phi\in F'_{\xi}\}$ form a cover of $\{0,1\}^{\frakc}$
by nowhere dense sets;  so $n(\{0,1\}^{\frakc})\le\lambda$.\ \Qed

Because every non-empty open set $\{0,1\}^{\frakc}$ includes an open set
homeomorphic to $\{0,1\}^{\frakc}$,

$$\eqalignno{n(\{0,1\}^{\frakc})
&=\min\{n(H):H\subseteq\{0,1\}^{\frakc}\text{ is open and not empty}\}\cr
&=\frak m(\RO(\{0,1\}^{\frak c}))\cr
\displaycause{517J}
&\ge\frak m(\RO^{\uparrow}(\Cal S^{\infty}))\cr
\displaycause{where $\Cal S^{\infty}$ is the $\Bbb N$-localization poset,
by 529E and 517Ia}
&=\add\Cal N\cr}$$

\noindent by 528N.
}%end of proof of 529F

\leader{529G}{Reaping numbers}\cmmnt{ (following
{\smc Brendle 00})} For cardinals $\theta\le\lambda$ let
$\frak r(\theta,\lambda)$ be the smallest cardinal of any set
$\Cal A\subseteq[\lambda]^{\theta}$ such that for every
$B\subseteq\lambda$ there is an $A\in\Cal A$ such that either
$A\subseteq B$ or $A\cap B=\emptyset$.

\leader{529H}{Proposition}\cmmnt{ ({\smc Brendle 00}, 2.7;
{\smc Brendle 06}, Theorem 5)}
$\frak r(\omega_1,\lambda)\ge\cov\Cal N$ for all uncountable
$\lambda$.

\proof{ Let $\ofamily{\xi}{\kappa}{A_{\xi}}$ be a family in
$[\lambda]^{\omega_1}$, where $\kappa<\cov(\Cal N)$.
I seek a $B\subseteq\lambda$ such that $A_{\xi}\cap B$ and
$A_{\xi}\setminus B$ are non-empty for every $\xi<\kappa$.

\medskip

{\bf (a)} If $\kappa\le\omega_1$, then choose
$\ofamily{\xi}{\kappa}{\alpha_{\xi}}$ and
$\ofamily{\xi}{\kappa}{\beta_{\xi}}$ inductively so that

\Centerline{$\alpha_{\xi}\in A_{\xi}\setminus\{\beta_{\eta}:\eta<\xi\}$,
\quad$\beta_{\xi}\in A_{\xi}\setminus\{\alpha_{\eta}:\eta\le\xi\}$}

\noindent for every $\xi<\kappa$, and set
$B=\{\beta_{\xi}:\xi<\kappa\}$;  this serves.   So henceforth let us
suppose that $\kappa>\omega_1$.

\medskip

{\bf (b)} For each $\xi<\kappa$ let $A'_{\xi}\subseteq A_{\xi}$ be a set
of order type $\omega_1$.   For each $n$, let $X_n$ be a set of size
$n!$ with its discrete topology and the uniform probability measure
which gives measure $\Bover1{n!}$ to every singleton.   Give
$X=\prod_{n\in\Bbb N}X_n$ its product measure $\mu$ and its product
topology.   Because $X$ is a compact metrizable space and $\mu$ is a
Radon measure (416U), $\cov\Cal N(\mu)=\cov\Cal N$ (522Wa).
We can therefore choose a family $\ofamily{\xi}{\kappa}{x_{\xi}}$ in $X$
in such a way that each $x_{\zeta}$ is random over its predecessors in
the sense that

\Centerline{whenever $\xi<\kappa$ and
$\overline{\{x_{\eta}:\eta\in A'_{\xi}\cap\zeta\}}$
is negligible, it does not contain $x_{\zeta}$.}

\noindent For distinct $x$, $y\in X$, set
$\Delta(x,y)=\min\{i:x(i)\ne y(i)\}$.   For $x\in X$,
set $B(x)=\{\eta:x_{\eta}\ne x$, $\Delta(x_{\eta},x)$ is even$\}$.

\medskip

{\bf (c)} For every $\xi<\kappa$,
$\{x:x\in X$, $A_{\xi}\subseteq B(x)\}$ and
$\{x:x\in X$, $A_{\xi}\cap B(x)=\emptyset\}$ are
negligible.   \Prf\ There is a $\zeta<\kappa$ such that
$\zeta\in A'_{\xi}$, $A'_{\xi}\cap\zeta$ is countable and
$D=\{x_{\eta}:\eta\in A'_{\xi}\cap\zeta\}$ is dense in
$\{x_{\eta}:\eta\in A'_{\xi}\}$.   Since
$x_{\zeta}\in\overline{D}$, $\overline{D}$ has measure greater than $0$.
By 275I, applied to the sequence $\sequencen{\Sigma_n}$ where $\Sigma_n$ is
the finite algebra of subsets of $X$ determined by coordinates less than
$n$, $\overline{D}$ has a point $w$ which is a density point in the sense
that

\Centerline{$\lim_{n\to\infty}
  \Bover{\mu\{y:y\restr n=w\restr n,\,y\in\overline{D}\}}
{\mu\{y:y\restr n=w\restr n\}}=1$.}

\noindent Consequently, setting

\Centerline{$J_n=\{y(n):y\in D\}
=\{y(n):y\in\overline{D}\}
\supseteq\{y(n):y\in\overline{D}$, $y\restr n=w\restr n\}$,}

\noindent

\Centerline{$\Bover{\#(J_n)}{n!}
\ge\Bover{\mu\{y:y\restr n=w\restr n,\,y\in\overline{D}\}}
{\mu\{y:y\restr n=w\restr n\}}\to 1$}

\noindent as $n\to\infty$.

Next note that, for any $y\in X$ and $n\in\Bbb N$,

\Centerline{$\mu\{x:\,\Exists i>n,\,x(i)=y(i)\}
\le\sum_{i=n+1}^{\infty}\Bover1{i!}\le\Bover{n+2}{(n+1)(n+1)!}$.}

\noindent So

$$\eqalign{\mu\{x:\,\Exists y\in D,\,&x(n)=y(n),\,
  x(i)\ne y(i)\text{ for every }i>n\}\cr
&\mskip100mu\ge\Bover{\#(J_n)}{n!}\bigl(1-\Bover{n+2}{(n+1)(n+1)!}\bigr)
\to 1\cr}$$

\noindent as $n\to\infty$, and

\Centerline{$\mu\{x:\,\Exists y\in D\setminus\{x\},\,
\Delta(x,y)$ is even$\}
=\mu\{x:\,\Exists y\in D\setminus\{x\},\,\Delta(x,y)$ is odd$\}
=1$.}

\noindent But if $y\in D\setminus\{x\}$ and $\Delta(x,y)$ is even, we have an
$\eta\in A_{\xi}$ such that $\Delta(x,x_{\eta})$ is even, and
$\eta\in A_{\xi}\cap B(x)$;  similarly, if there is a
$y\in D\setminus\{x\}$ such that
$\Delta(x,y)$ is odd, there is an $\eta\in A_{\xi}\setminus B(x)$.   So
$\{x:A_{\xi}\subseteq B(x)\}$ and $\{x:A_{\xi}\cap B(x)=\emptyset\}$ are
both negligible.\ \Qed

\medskip

{\bf (d)} Since $\cov\Cal N(\mu)=\cov\Cal N>\kappa$, there is an
$x\in X$ such that both
$A_{\xi}\cap B(x)$ and $A_{\xi}\setminus B(x)$ are non-empty for every
$\xi<\kappa$.   So in this case also we have a suitable set $B$.
}%end of proof of 529H

\exercises{\leader{529X}{Basic exercises (a)}
%\spheader 529Xa
Let $(X,\Sigma,\mu)$ be a measure space, and $p\in\coint{1,\infty}$.
Show that $L^p(\mu)\equivT\ell^1(\kappa)$, where $\kappa=\dim L^p(\mu)$
if this is finite, $d(L^p(\mu))$ otherwise.
%529C

\spheader 529Xb Let $U$ be an $L$-space.   (i) Show that $\add U=\infty$
if $U=\{0\}$, $\omega$ otherwise.   (ii) Show that
$\add_{\omega}U=\infty$ if $U$ is finite-dimensional, $\add\Cal N$ if
$U$ is separable and infinite-dimensional, $\omega_1$ otherwise.
(iii) Show that $\cf U=1$ if $U=\{0\}$, $\omega$ if $0<\dim U<\omega$,
$\max(\cf\Cal N,\cff[d(U)]^{\le\omega})$ otherwise.   (iv) Show that
$\link^{\uparrow}_{<\kappa}(U)=1$ if $\kappa\le\omega$, $\cf U$
otherwise.
%529C

\spheader 529Xc Let $U$ be a separable Banach lattice.   Suppose that
$\ofamily{\xi}{\kappa}{u_{\xi}}$ is a family in $U$, where
$\kappa<\add\Cal N$.   Show that there is a family
$\ofamily{\xi}{\kappa}{\epsilon_{\xi}}$ of strictly positive real numbers
such that $\{\epsilon_{\xi}u_{\xi}:\xi<\kappa\}$ is order-bounded in $U$.
%529C 529Xb out of order query

\sqheader 529Xd Let $U$ be a separable Banach lattice, and $D\subseteq U$
a dense set.   Let
$A\subseteq U$ be a set of size less than $\add\Cal N$.   Show that there
is a $w\in U$ such that for every $u\in A$ and every $\epsilon>0$ there is
a $v\in D$ such that $|u-v|\le\epsilon w$.
%529C out of order query

\spheader 529Xe\dvAformerly{5{}29Xc}
Let $\Cal I$ be the ideal of subsets $I$ of $\Bbb N$
such that $\sum_{n\in I}\Bover1{n+1}$ is finite.  (See 419A.)   Show
that $\ell^1\equivT\Cal I$, so that $\add_{\omega}\Cal I=\add\Cal N$ and
$\cf\Cal I=\cf\Cal N$.
%529+ mt52bits
%note:  analytic P-ideal enough

\spheader 529Xf\dvAnew{2011} Show that if
$\theta\le\theta'\le\lambda'\le\lambda$ are cardinals, then
$\frak r(\theta,\lambda)\le\frak r(\theta',\lambda')$.
%529G

\spheader 529Xg\dvAformerly{5{}29Xd;  revised 2011}(i) Show that
$\frak r(\omega,\omega)\ge\cov\Cal E\ge\max(\cov\Cal N,\frakmctbl)$, where
$\Cal E$ is the ideal of subsets of $\Bbb R$ with Lebesgue negligible
closures.
(ii) Show that if $\lambda$ is an infinite cardinal then
$\frak r(\omega,\lambda)\ge\max(\add\Cal N,\cov\Cal N_{\lambda})$, where
$\Cal N_{\lambda}$ is the null ideal of the usual measure on
$\{0,1\}^{\lambda}$.   \Hint{529F.}
%529H

\leader{529Y}{Further exercises (a)}
%\spheader 529Ya
Let $X$ be a Polish space and $\Cal K_{\sigma}$ the
family of K$_{\sigma}$ subsets of $X$.    Show that, defining $\le^*$ as in
522C,
$(\Cal K_{\sigma},\subseteq)\preccurlyeq_{\text{T}}(\NN,\le^*)$.
%529B

\spheader 529Yb Let $X$ be a topological space with a countable network,
and $c:\Cal PX\to[0,\infty]$ an outer regular submodular Choquet
capacity (definitions:  432J).   Show that if $\Cal A$ is an
upwards-directed family of subsets of $X$ such that
$\#(\Cal A)<\frak m_{\sigma\text{-linked}}$,
then $c(\bigcup\Cal A)=\sup_{A\in\Cal A}c(A)$.
%529B+

\spheader 529Yc Let $r\ge 3$ be an integer.   (i) Let
$c:\Cal P\BbbR^r\to[0,\infty]$ be Choquet-Newton capacity
(\S479).   Show that if $\Cal A$ is an
upwards-directed family of subsets of $\BbbR^r$ such that
$\#(\Cal A)<\add\Cal N$, then $c(\bigcup\Cal A)=\sup_{A\in\Cal A}c(A)$.
\Hint{479Xi.}
(ii) Let $\Cal I$ be the ideal of polar sets in $\BbbR^r$.
Show that $\add\Cal I=\add\Cal N$.
%look at  A\times[0,1]^{r-2} ;  non-zero capacity iff  \mu_1^*A>0
%529D

\spheader 529Yd
Show that, for any infinite set $I$, the regular
open algebra $\RO(\{0,1\}^I)$ of $\{0,1\}^I$ is homogeneous, so that
$\frak m(\RO(\{0,1\}^I))=n(\{0,1\})^I$.
%529E

\spheader 529Ye Show that
$\frak b\le\frak r(\omega,\omega)\le\pi(\Cal P\Bbb N/[\Bbb N]^{<\omega})$.
%529H

}%end of exercises

\endnotes{
\Notesheader{529}
Many of the ideas of the last two chapters were first embodied in
forcing arguments.   In 529E this becomes particularly transparent.
If we have an upwards-directed set $R\subseteq\Cal S^{\infty}$ which is
`generic' in the sense that it meets all the cofinal subsets of
$\Cal S^{\infty}$ definable in a language $\Cal L$ with terms for all
the functions $h_{\xi}$, as well as such obvious ones as
$\{p:\#(p[\{n\}])=2^n\}$ for each $n$, and we set $S=\bigcup\Cal R$,
then $S$ will belong to the set $\Cal S=\Cal S_{\Bbb N}$ of 522K,
and we shall have
$h_{\xi}\subseteq^*S$ for every $\xi$;  so that we have a corresponding
function $\tilde f(S)=\bigcup_{p\in R}f(p)\in\{0,1\}^{\frakc}$
defined by setting

\Centerline{$\tilde f(S)(\xi)
\equiv\sup\{i:(i,h_{\xi}(i))\notin S\}\mod 2$.}

\noindent Next, if $G\subseteq\{0,1\}^{\frakc}$ is a dense open set with
a definition in $\Cal L$, then $\tilde f(S)\in G$;  for,
setting $U_q=\{\phi:q\subseteq\phi\in\{0,1\}^{\frakc}\}$ when
$q\in Q=\Fn_{<\omega}(\frak c,\{0,1\})$, $\{q:U_q\subseteq G\}$ is
cofinal with $Q$, so $\{p:U_{f(p)}\subseteq G\}$ is cofinal with
$\Cal S^{\infty}$ (part (d) of the proof of 529E) and meets $R$.   Thus
$\tilde f(S)$ is `generic' in the sense that it belongs to every dense
open set with a name in $\Cal L$;  and it is a commonplace in the theory
of forcing that a function which transforms generic objects in one
forcing extension into generic objects in another extension corresponds
to a regular embedding of the corresponding regular open algebras.
}%end of notes

\discrpage

