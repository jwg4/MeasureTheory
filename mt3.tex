\frfilename{mt3.tex}
\versiondate{21.7.01/31.12.01}
\copyrightdate{1995}
     
\def\volumename{Measure algebras}
     
\newvolume{3}
     
One of the first things one learns, as a student of measure theory, is
that sets of measure zero are frequently `negligible' in the
straightforward sense that they can safely be ignored.   This is not
quite a universal principle, and one of my purposes in writing this
treatise is to call attention to the exceptional cases in which
negligible sets are important.   But very large parts of the theory,
including some of the topics already treated in Volume 2, can be
expressed in an appropriately abstract language in which negligible sets
have been factored out.   This is what the present volume is about.   A
`measure algebra' is a quotient of an algebra of measurable sets by a null ideal;  that is, the elements of the measure algebra
are equivalence classes of measurable sets.   At the cost of an extra
layer of abstraction, we obtain a language which can give concise and
elegant expression to a substantial proportion of the ideas of measure
theory, and which offers insights almost everywhere in the subject.
     
It is here that I embark wholeheartedly on `pure' measure theory.   I
think it is fair to say that the applications of measure theory to other
branches of mathematics are more often through measure {\it spaces}
rather than measure {\it algebras}.   Certainly there will be in this
volume many theorems of wide importance outside measure theory;  but
typically their usefulness will be in forms translated back into the
language of the first two volumes.   But it is also fair to say that the
language of measure algebras is the only reasonable way to discuss large
parts of a
subject which, as pure mathematics, can bear comparison with any.
     
In the structure of this volume I can distinguish seven `working' and
two `accessory' chapters.   The `accessory' chapters are 31 and
35.   In these I develop the theories of Boolean algebras
and Riesz spaces (= vector lattices) which are needed later.   As in
Volume 2 you have a certain amount of choice in the order in which you
take the material.   Everything except Chapter 35 depends on Chapter 31,
and everything except Chapters 31 and 35 depends on Chapter 32.
Chapters 33, 34 and 36 can be taken in any order, but Chapter 36 relies
on Chapter 35.   (I do not mean that Chapter 33 is never referred to in
Chapter 34, nor even that the later chapters do not rely on results from Chapter 33.   What I mean is that their most important ideas are
accessible without learning the material of Chapter 33 properly.)
Chapter 37 depends on Chapters 35 and 36.   Chapter 38 would be difficult to make sense of without
some notion of what has been done in Chapter 33.   Chapter 39 uses
fragments of Chapters 35 and 36.
     
The first third of the volume follows almost the only line permitted by
the structure of the subject.   If we are going to study measure
algebras at all, we must know the relevant facts about Boolean algebras
(Chapter 31) and how to translate what we know about measure spaces into
the new language (Chapter 32).   Then we must get a proper grip on the
two most important theorems:  Maharam's theorem on the classification of
measure algebras (Chapter 33) and the von Neumann-Maharam lifting
theorem (Chapter 34).   Since I am now writing for readers who are
committed -- I hope, happily committed -- to learning as much as they
can about the subject, I take the space to push these ideas as far as
they can easily go, giving a full classification of closed subalgebras
of probability algebras, for instance (\S333), and investigating special
types of lifting (\S\S345-346).   I mention here three sections
interpolated into Chapter 34 (\S\S342-344) which attack a subtle and
important question:  when can we expect homomorphisms between measure
algebras to be realizable in terms of transformations between measure
spaces, as discussed briefly in \S235 and elsewhere.
     
Chapters 36 and 37 are devoted to re-working the ideas of Chapter 24 on
`function spaces' in the more abstract context now available, and
relating them to the general Riesz spaces of Chapter 35.   I am
concerned here not to develop new structures, nor even to prove striking
new theorems, but rather to offer new ways of looking at the old ones.
Only in the Ergodic Theorem (\S372) do I come to a really
important new result.   Chapter 38 looks at two questions, both obvious
ones to ask if you have been trained in twentieth-century pure
mathematics:  what does the automorphism group of a measure algebra look
like, and inside such an automorphism group, what do the conjugacy
classes look like?   (The second question is a fancy way of asking how
to decide,
given two automorphisms of one of the structures considered in this
volume, whether they are really different, or just copies
of each other obtained by looking at the structure a different way up.)
Finally, in Chapter 39, I discuss what is known about the question of
which Boolean algebras can appear as measure algebras.
     
Concerning the prerequisites for this volume, we certainly do not need
everything in Volume 2.   The important chapters there are 21, 23, 24,
25 and 27.   If you are approaching this volume without having read the
earlier parts of this treatise, you will need the Radon-Nikod\'ym
theorem and product measures (of arbitrary families of probability
spaces), for Maharam's theorem;  a simple version of the martingale
theorem, for the lifting theorem;  and an acquaintance with $L^p$ spaces
(particularly, with $L^0$ spaces) for Chapter 36.   But I would
recommend the results-only versions of Volumes 1 and 2 in case some
reference is totally obscure.   Outside measure theory, I call on quite
a lot of terms from general topology, but none of the ideas needed are
difficult (Baire's and Tychonoff's theorems are the deepest);  they are
sketched in
\S\S3A3 and 3A4.   We do need some functional analysis for Chapters 36
and 39, but very little more than was already used in Volume 2, except
that I now call on versions of the Hahn-Banach theorem (\S3A5).
     
In this volume I assume that readers have substantial experience in both
real and abstract analysis, and I make few concessions which would not
be appropriate when addressing active researchers, except that perhaps I
am a little gentler when calling on ideas from set theory and general
topology than I should be with my own colleagues, and I continue to
include all the easiest exercises I can think of.   I do maintain my
practice of giving proofs in very full detail, not so much because I am
trying to make them easier, but because one of my purposes here is to
provide a complete account of the ideas of the subject.   I hope that
the result will be accessible to most doctoral students who are studying
topics in, or depending on, measure theory.
     
\frnewpage

