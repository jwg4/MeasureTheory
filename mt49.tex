\frfilename{mt49.tex}
\versiondate{31.8.09}
\copyrightdate{2009}

\def\chaptername{Further topics}
\newchapter{49}

I conclude the volume with notes on six almost unconnected special
topics.   In \S491 I look at equidistributed sequences
and the ideal $\Cal Z$ of sets with asymptotic density zero.
I give the principal theorems on the existence of equidistributed
sequences in abstract topological measure spaces,
and examine the way in which an equidistributed sequence can induce an
embedding of a measure algebra in the quotient algebra
$\Cal P\Bbb N/\Cal Z$.
The next three sections are linked.
In \S492 I present some forms of `concentration of measure'
which echo ideas from \S476 in combinatorial, rather than geometric,
contexts, with theorems of Talagrand and
Maurey on product measures and the Haar measure of a permutation group.
In \S493 I show how the ideas
of \S\S449, 476 and 492 can be put together in the theory of
`extremely amenable' topological groups.
Some of the important examples of extremely amenable groups are full groups
of measure-preserving automorphisms of measure algebras, previously
treated in \S383;   these are the
subject of \S494\dvAnew{2009}, where I look also at some striking algebraic
properties of these groups.
In \S495, I move on to Poisson point processes, with notes on
disintegrations and some special cases in which they can be represented
by Radon measures.
In \S496\dvAnew{2008}, I revisit the Maharam submeasures of Chapter 39,
showing that various ideas from the present
volume can be applied in this more general context.
In \S497\dvAnew{2009}, I give a version of Tao's proof of
Szemer\'edi's theorem on arithmetic progressions, based on a deep analysis
of relative independence, as introduced in \S458.
Finally, in \S498\dvAformerly{\S4{}94} I give a pair of simple, but
perhaps surprising, results on subsets of
sets of positive measure in product spaces.

\discrpage

%496 much less weighty than 497
