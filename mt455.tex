\frfilename{mt455.tex}
\versiondate{18.1.09/24.7.13}
\copyrightdate{2007}

\def\mucdlg{\mu_{\text{cdlg}}}
\def\rti{right-{\vthsp}translation-{\vthsp}invariant}

\def\chaptername{Perfect measures, disintegrations and processes}
\def\sectionname{Markov and L\'evy processes}

\newsection{455}

For a `Markov' process, in which the evolution of the system after a
time $t$ depends only on the state at time $t$, the general theory of
\S454 leads
to a straightforward existence theorem (at least for random variables
taking values in standard Borel spaces) dependent only on a natural
consistency condition on the transitional probabilities (455A, 455E).
The formulation leads naturally to descriptions of the `Markov property'
(for stopping times taking only countably many values) in
terms of disintegrations and conditional expectations (455C, 455Ec).
With appropriate continuity conditions, we find that the process can be
represented either by a Radon measure (455H) or
by a measure on the set of \cadlag\ paths (455Gc) for which we
have a formulation of the strong Markov property (for general
stopping times) in terms of disintegrations
(455O).   These conditions are satisfied by L\'evy processes
(455P-455R).  %455P 455Q 455R
For these, we have an alternative expression of the strong Markov property
in terms of \imp\ functions (455U).
By far the most important example of a continuous-time Markov process is
Brownian motion, but I defer discussion of this to \S477.

\vleader{72pt}{455A}{Theorem} Let $T$ be a totally ordered set with least
element $t^*$, and for each $t\in T$ let $\Omega_t$ be a
non-empty set and $\Tau_t$ a
$\sigma$-algebra of subsets of $\Omega_t$ containing all singleton subsets
of $\Omega_t$.   Set $\Omega=\prod_{t\in T}\Omega_t$ and for $t\in T$,
$\omega\in\Omega$ set $X_t(\omega)=\omega(t)$.
Fix $x^*\in\Omega_{t^*}$.   Suppose that we are given, for each
pair $s<t$ in $T$, a family $\family{x}{\Omega_s}{\nu^{(s,t)}_x}$ of
perfect probability measures on $\Omega_t$, all with domain $\Tau_t$,
and suppose that

\inset{($\dagger$) whenever $s<t<u$ in $T$ and $x\in\Omega_s$,
then $\family{y}{\Omega_t}{\nu^{(t,u)}_y}$ is a disintegration of
$\nu^{(s,u)}_x$ over $\nu^{(s,t)}_x$.}

\noindent For $J\subseteq T$ write $\pi_J$ for the canonical map from
$\Omega$ onto $Z_J=\prod_{t\in J}\Omega_t$.
Then there is a unique probability measure $\mu$ on $\Omega$,
with domain $\Tensorhat_{t\in T}\Tau_t$, such that, writing
$\lambda_J$ for the image measure $\mu\pi_J^{-1}$,

$$\eqalign{\int fd\lambda_J
&=\int f(\omega(t^*),\omega(t_1),\ldots,\omega(t_n))\mu(d\omega)\cr
&=\int\ldots\iint f(x^*,x_1,\ldots,x_n)
   \nu_{x_{n-1}}^{(t_{n-1},t_n)}(dx_n)\cr
&\qquad\qquad\qquad\qquad
  \nu_{x_{n-2}}^{(t_{n-2},t_{n-1})}(dx_{n-1})
  \ldots\nu_{x^*}^{(t^*,t_1)}(dx_1)\cr}$$

\noindent whenever $t^*<t_1<\ldots<t_n$, $J=\{t^*,t_1,\ldots,t_n\}$ and
$f$ is $\lambda_J$-integrable.   $\mu$ is perfect, and the marginal measure
$\mu_t=\mu X_t^{-1}$ is equal to $\nu_{x^*}^{(t^*,t)}$, if $t>t^*$,
while $\mu_{t^*}\{x^*\}=1$.

\proof{{\bf (a)} For $I\subseteq T$, write
$\Tau_I=\Tensorhat_{t\in I}\Tau_t$.   If $I=\{t_0,t_1,\ldots,t_n\}$ is
a finite subset of $T$ with $t^*=t_0<t_1<\ldots<t_n$, then we have a
probability measure $\lambda_I$ on $Z_I$ with domain $\Tau_I$ such that

$$\eqalign{\int fd\lambda_I
&=\int\ldots\iint f(x^*,x_1,\ldots,x_n)
   \nu_{x_{n-1}}^{(t_{n-1},t_n)}(dx_n)\cr
&\qquad\qquad\qquad\qquad
  \nu_{x_{n-2}}^{(t_{n-2},t_{n-1})}(dx_{n-1})
  \ldots\nu_{x^*}^{(t^*,t_1)}(dx_1)\cr}$$

\noindent for every $\lambda_I$-integrable function $f$.   \Prf\ Use
454H on the finite sequence $(\Omega_{t_0},\ldots,\Omega_{t_n})$.
The measures
$\nu_z$ required by 454H must be constructed by the rule

\Centerline{$\nu_z=\nu_{z(t_m)}^{(t_m,t_{m+1})}$}

\noindent for $m<n$, $z\in\prod_{i\le m}\Omega_{t_i}$, while of course
$\nu_{\emptyset}\{x^*\}=1$.   (Having a finite sequence rather than an
infinite one clearly makes things easier;  we can stop the argument at
the end of part (b) of the proof of 454H.)\ \Qed

When $I=\{t^*\}$, so that $Z_I$ can be identified with $\Omega_{t^*}$,
I mean to interpret these formulae in such a way that
$\lambda_I\{x^*\}=1$.   When $J=\{t^*,t\}$, with $t^*<t$,
and $E\in\Tau_t$, then we can apply the formula above to the function
$z\mapsto\chi E(z(t))$ to get
$\lambda_J\{z:z(t)\in E\}=\nu_{x^*}^{(t^*,t)}(E)$.

\medskip

{\bf (b)} Of course the point of this is that these measures $\lambda_I$
form a consistent family;  if $t^*\in I\subseteq J\in[T]^{<\omega}$, then
the canonical projection $\pi_{IJ}:Z_J\to Z_I$ is \imp.   \Prf\ It is
enough to consider the case in which $J$ has just one more point than
$I$, since then we can induce on $\#(J\setminus I)$.   In this case,
express $J$ as $\{t_0,\ldots,t_n\}$ where $t^*=t_0<\ldots<t_n$, and
suppose that $I=J\setminus\{t_m\}$.   If $W\in\Tau_I$, then

$$\eqalignno{\lambda_J\pi_{IJ}^{-1}[W]
&=\int\ldots\iint\ldots\int\chi W(x^*,x_1,\ldots,x_{m-1},x_{m+1},\ldots,x_n)
   \nu_{x_{n-1}}^{(t_{n-1},t_n)}(dx_n)\cr
&\qquad\qquad\qquad\qquad
   \ldots\nu_{x_m}^{(t_m,t_{m+1})}(dx_{m+1})
   \nu_{x_{m-1}}^{(t_{m-1},t_m)}(dx_m)
   \ldots\nu^{(t^*,t_1)}_{x^*}(dx_1)\cr
&=\int\ldots\iint g_{(x_1,\ldots,x_{m-1})}(x_{m+1})
    \nu_{x_m}^{(t_m,t_{m+1})}(dx_{m+1})&(^*)\cr
&\qquad\qquad\qquad\qquad\qquad\qquad
  \nu_{x_{m-1}}^{(t_{m-1},t_m)}(dx_m)\ldots\nu_{x^*}^{(t^*,t_1)}(dx_1)\cr}$$

\noindent where

$$\eqalign{g_{(x_1,\ldots,x_{m-1})}(x_{m+1})
&=\int\ldots\int\chi W(x^*,x_1\ldots,x_{m-1},x_{m+1},\ldots,x_n)\cr
&\qquad\qquad\qquad\qquad
   \nu_{x_{n-1}}^{(t_{n-1},t_n)}(dx_n)
   \ldots\nu_{x_{m+1}}^{(t_{m+1},t_{m+2})}(dx_{m+2}).\cr}$$

\noindent Here, of course, we use the hypothesis ($\dagger$);  since
$\family{y}{\Omega_{t_m}}{\nu_y^{(t_m,t_{m+1})}}$ is a disintegration of
$\nu_{x_{m-1}}^{(t_{m-1},t_{m+1})}$ over $\nu_{x_{m-1}}^{(t_{m-1},t_m)}$,
and $g_{(x_1,\ldots,x_{m-1})}$ is bounded and
$\nu^{(t_{m-1},t_{m+1})}_{x_{m-1}}$-integrable (by 454H),

$$\eqalign{\int g_{(x_1,\ldots,x_{m-1})}&(x_{m+1})
  \nu_{x_{m-1}}^{(t_{m-1},t_{m+1})}(dx_{m+1})\cr
&=\iint g_{(x_1,\ldots,x_{m-1})}(x_{m+1})
  \nu_{x_m}^{(t_m,t_{m+1})}(dx_{m+1})
  \nu_{x_{m-1}}^{(t_{m-1},t_m)}(dx_m)\cr}$$

\noindent (452F).   Substituting this into (*) above,

$$\eqalignno{\lambda_J\pi_{IJ}^{-1}[W]
&=\int\ldots\iint g_{(x_1,\ldots,x_{m-1})}(x_{m+1})\cr
&\qquad\qquad\qquad\qquad
  \nu_{x_m}^{(t_m,t_{m+1})}(dx_{m+1})
  \nu_{x_{m-1}}^{(t_{m-1},t_m)}(dx_m)\ldots\nu_{x^*}^{(t^*,t_1)}(dx_1)\cr
&=\int\ldots\int g_{(x_1,\ldots,x_{m-1})}(x_{m+1})
  \nu_{x_{m-1}}^{(t_{m-1},t_{m+1})}(dx_{m+1})\ldots
  \nu_{x^*}^{(t^*,t_1)}(dx_1)\cr
&=\int\ldots\int\ldots\int
      \chi W(x^*,x_1,\ldots,x_{m-1},x_{m+1},\ldots,x_n)\cr
&\qquad\qquad\qquad\qquad
  \nu_{x_{n-1}}^{(t_{n-1},t_n)}(dx_n)\ldots
  \nu_{x_{m-1}}^{(t_{m-1},t_{m+1})}(dx_{m+1})\ldots
  \nu_{x^*}^{(t^*,t_1)}(dx_1)\cr
&=\lambda_IW,\cr}$$

\noindent applying the formula in (a) again.\ \Qed

(Some of the formulae here are inappropriate if $m=n>1$.   In this case,
of course,

$$\eqalignno{\lambda_J\pi_{IJ}^{-1}[W]
&=\int\ldots\int\chi W(x^*,x_1,\ldots,x_{n-1})
   \nu_{x_{n-1}}^{(t_{n-1},t_n)}(dx_n)\ldots\nu_{x^*}^{(t^*,t_1)}(dx_1)\cr
&=\int\ldots\int\chi W(x^*,x_1,\ldots,x_{n-1})
  \nu_{x_{n-2}}^{(t_{n-2},t_{n-1})}(dx_{n-1})\ldots\nu_{x^*}^{(t^*,t_1)}(dx_1)
 =\lambda_IW.\cr}$$

\noindent If $m=1<n$, there is a collapse of a different kind;  we must
look at

$$\eqalign{\lambda_J\pi_{IJ}^{-1}[W]
&=\int\ldots\iint\chi W(x^*,x_2,\ldots,x_n)
  \nu_{x_{n-1}}^{(t_{n-1},t_n)}(dx_n)
     \ldots\nu_{x_1}^{(t_1,t_2)}(dx_2)\nu_{x^*}^{(t^*,t_1)}(dx_1)\cr
&=\int\ldots\iint\chi W(x^*,x_2,\ldots,x_n)
  \nu_{x_{n-1}}^{(t_{n-1},t_n)}(dx_n)
     \ldots\nu_{x^*}^{(t^*,t_2)}(dx_2)
=\lambda_IW.\cr}$$

\noindent If $m=n=1$ then

\Centerline{$\lambda_J\pi_{IJ}^{-1}[W]
=\int\chi W(x^*)\nu_{x^*}^{(t^*,t_1)}(dx_1)
=\chi W(x^*)
=\lambda_IW$).}

\medskip

{\bf (c)} Part (b) tells us that we have a consistent family of measures
on the finite products $Z_J$, and therefore have a functional $\lambda$
on $\bigotimes_{t\in T}\Tau_t$ defined by setting
$\lambda\pi_J^{-1}[W]=\lambda_JW$ for every finite $J\subseteq T$
containing $t^*$ and $W\in\bigotimes_{t\in J}\Tau_t$.
$\lambda$ is finitely
additive, and its images $\mu_t=\lambda X_t^{-1}$ are all countably
additive and perfect because $\mu_t=\nu_{x^*}^{(t^*,t)}$ for $t>t^*$,
while $\mu_{t^*}$ is concentrated at $\{x^*\}$.

By 454D, we have a perfect
measure $\mu$ extending $\lambda$.   We have to
check that each $\lambda_J$ is the image
measure $\mu\pi_J^{-1}$;  but this is true because they agree on
$\bigotimes_{t\in J}\Tau_t$ (using the Monotone Class Theorem in the
form 136C, as always).   So the integral formula sought for
$\lambda_J$ is just that obtained in part (a).
By the last remark in (a), we have the declared formulae for the marginal
measures $\mu_t$.
}%end of proof of 455A

\leader{455B}{Lemma} Suppose that $T$, $t^*$,
$\family{t}{T}{(\Omega_t,\Tau_t)}$, $\Omega$, $x^*$ and
$\langle\nu^{(s,t)}_x\rangle_{s<t,x\in\Omega_s}$ are as in
455A.

(a) Suppose that $\mu$ is constructed from $x^*$
and $\langle\nu^{(s,t)}_x\rangle_{s<t,x\in\Omega_s}$ as in 455A.
If $F\in\Tensorhat_{t\in T}\Tau_t$ is determined by
coordinates in $[t^*,t_0]$ and
$H^*=\{\omega:\omega(t_i)\in E_i$ for $1\le i\le n\}$ where
$t_0<t_1\ldots<t_n$ and $E_i\in\Tau_{t_i}$ for $1\le i\le n$, then

$$\eqalignno{\mu(H^*\cap F)
&=\int_F\int\ldots\int\chi H(y_1,\ldots,y_n)
   \nu_{y_{n-1}}^{(t_{n-1},t_n)}(dy_n)
   \ldots\nu_{\omega(t_0)}^{(t_0,t_1)}(dy_1)\mu(d\omega)&(*)}$$

\noindent where $H=\prod_{1\le i\le n}E_i$.

(b) Suppose that $\omega\in\Omega$ and $a\in T\cup\{\infty\}$, where
$\infty$ is taken to be greater than every element of $T$.   For
$s<t$ in $T$ and $x\in\Omega_s$ set

$$\eqalign{\nu^{(s,t)}_{\omega ax}
&=\nu_x^{(s,t)}\text{ if }a<s,\cr
&=\nu_{\omega(a)}^{(a,t)}\text{ if }s\le a<t,\cr
&=\delta^{(t)}_{\omega(t)}\text{ if }t\le a,\cr}$$

\noindent here writing $\delta^{(t)}_x$ for the
probability measure with domain $\Tau_t$ such that
$\delta^{(t)}_x(\{x\})=1$.

\quad(i) $\nu^{(s,t)}_{\omega ax}$ is always a
perfect probability measure with domain $\Tau_t$, and
$\family{y}{\Omega_t}{\nu^{(t,u)}_{\omega ay}}$ is a disintegration of
$\nu^{(s,u)}_{\omega ax}$ over $\nu^{(s,t)}_{\omega ax}$ whenever $s<t<u$
in $T$ and $x\in\Omega_s$.

\quad(ii) Taking $\mu_{\omega a}$ to be the measure on $\Omega$ defined from
$\omega(t^*)$ and $\langle\nu^{(s,t)}_{\omega ax}\rangle_{s<t,x\in\Omega_t}$
by the method of 455A, then
$\{\omega':\omega'\in\Omega$, $\omega'\restr D=\omega\restr D\}$ is
$\mu_{\omega a}$-conegligible for every countable
$D\subseteq T\cap[t^*,a]$.

\quad(iii) If $\omega$, $\omega'\in\Omega$ and
$\omega\restr[t^*,a]=\omega'\restr[t^*,a]$ then
$\mu_{\omega a}=\mu_{\omega'a}$.

\proof{{\bf (a)(i)} Suppose first that $F$ is of the form
$\{\omega:\omega(s_i)\in F_i$ for $i\le m\}$ where
$t^*=s_0<\ldots<s_m=t_0$.   For $x\in\Omega_{t_0}$ set

\Centerline{$f(x)=\int\ldots\int\chi H(y_1,\ldots,y_n)
   \nu_{y_{n-1}}^{(t_{n-1},t_n)}(dy_n)
   \ldots\nu_x^{(t_0,t_1)}(dy_1)$.}

\noindent Writing $G=\prod_{i\le m}F_i$, we have

$$\eqalignno{\mu(H^*\cap F)
&=\int\ldots\iint\ldots\int\chi G(x^*,x_1,\ldots,x_m)
   \chi H(y_1,\ldots,y_n)\nu_{y_{n-1}}^{(t_{n-1},t_n)}(dy_n)\cr
&\mskip100mu
   \ldots\nu_{x_m}^{(s_m,t_1)}(dy_1)
   \nu_{x_{m-1}}^{(s_{m-1},s_m)}(dx_m)
   \ldots\nu_{x^*}^{(t^*,s_1)}(dx_1)\cr
&=\int\ldots\int\chi G(x^*,x_1,\ldots,x_m)f(x_m)
   \nu_{x_{m-1}}^{(s_{m-1},s_m)}(dx_m)
   \ldots\nu_{x^*}^{(t^*,s_1)}(dx_1)\cr
&=\int g\,d\lambda_J\cr
\displaycause{where $J=\{t^*,s_1,\ldots,s_m\}$,
$g(z)=\chi G(z(t^*),\ldots,z(s_m))f(z(s_m))$ for
$z\in\prod_{s\in J}\Omega_s$, and $\lambda_J$ is defined as in 455A}
&=\int g\pi_Jd\mu
=\int_Ff(\omega(t_0))\mu(d\omega)\cr
\displaycause{because $g\pi_J(\omega)=f(\omega(s_m))=f(\omega(t_0))$
if $\omega\in F$, $0$ otherwise}
&=\int_F\int\ldots\int\chi H(y_1,\ldots,y_n)
   \nu_{y_{n-1}}^{(t_{n-1},t_n)}(dy_n)
   \ldots\nu_{\omega(t_0)}^{(t_0,t_1)}(dy_1)\mu(d\omega).\cr}$$

\medskip

\quad{\bf (ii)} Let $\Cal I$ be the family of sets $F$ of the type dealt with in
(a).   Since the intersection of two members of $\Cal I$ belongs to
$\Cal I$, the Monotone Class Theorem tells us
that (*) is true for all sets in the $\sigma$-algebra $\Tau$
generated by $\Cal I$.   But any member of
$\Tensorhat_{t\in T}\Tau_t$ determined by coordinates in $[t^*,t_0]$
belongs to $\Tau$.
\Prf\ Fix $v\in\prod_{s\in T\setminus[t^*,t_0]}\Omega_s$.   For
$\omega\in\Omega$ define $f(\omega)\in\Omega$ by setting

$$\eqalign{f(\omega)(s)
&=\omega(s)\text{ if }s\le t_0,\cr
&=v(s)\text{ if }s>t_0.\cr}$$

\noindent Then $\Tau'=\{F:F\subseteq\Omega$, $f^{-1}[F]\in\Tau\}$ is a
$\sigma$-algebra of
subsets of $\Omega$ containing $\{\omega:\omega(t)\in E\}$ whenever
$t\in T$ and $E\in\Tau_t$, so includes $\Tensorhat_{t\in T}\Tau_t$.
If $F\in\Tensorhat_{t\in T}\Tau_t$ and $F$ is determined by coordinates
in $[t^*,t_0]$, then $F=f^{-1}[F]\in\Tau$.\ \Qed

So (*) is true of every $F\in\Tensorhat_{t\in T}\Tau_t$, as claimed.

\woddheader{455B}{4}{2}{2}{108pt}

{\bf (b)(i)} Of course every $\nu^{(s,t)}_{\omega ax}$ is a perfect
probability measure with domain $\Tau_t$.
If $s<t<u$ and $E\in\Tau_u$, then

$$\eqalign{
\int_{\Omega_t}\nu^{(t,u)}_{\omega ay}(E)\nu^{(s,t)}_{\omega ax}(dy)
&=\int_{\Omega_t}\nu^{(t,u)}_y(E)\nu^{(s,t)}_x(dy)
   =\nu_x^{(s,u)}(E)
   =\nu_{\omega ax}^{(s,u)}(E)\cr
&\mskip250mu   \text{ if }a<s,\cr
&=\int_{\Omega_t}\nu^{(t,u)}_y(E)
   \nu^{(a,t)}_{\omega(a)}(dy)
   =\nu_{\omega(a)}^{(a,u)}(E)
   =\nu_{\omega ax}^{(s,u)}(E)\cr
&\mskip250mu
   \text{ if }s\le a<t,\cr
&=\int_{\Omega_t}\nu^{(t,u)}_{\omega(t)}(E)\delta^{(t)}_{\omega(t)}(dy)
   =\nu^{(t,u)}_{\omega(t)}(E)
   =\nu_{\omega ax}^{(s,u)}(E)\cr
&\mskip250mu
   \text{ if }a=t,\cr
&=\int_{\Omega_t}\nu^{(a,u)}_{\omega(a)}(E)
   \delta^{(t)}_{\omega(t)}(dy)
   =\nu^{(a,u)}_{\omega(a)}(E)
   =\nu_{\omega ax}^{(s,u)}(E)\cr
&\mskip250mu
   \text{ if }t<a<u,\cr
&=\int_{\Omega_t}\delta^{(u)}_{\omega(u)}(E)
   \delta^{(t)}_{\omega(t)}(dy)
   =\delta^{(u)}_{\omega(u)}(E)
   =\nu_{\omega ax}^{(s,u)}(E)\cr
&\mskip250mu
   \text{ if }u\le a.\cr}$$

\medskip

\quad{\bf (ii)}
Consider first the case $D=\{t\}$, where $t^*<t\le a$.   Then

\Centerline{$\mu_{\omega a}\{\omega':\omega'(t)=\omega(t)\}
=\nu^{(t^*,t)}_{\omega,a,\omega(t^*)}\{\omega(t)\}
=\delta^{(t)}_{\omega(t)}\{\omega(t)\}
=1$.}

\noindent As for $D=\{t^*\}$, $\mu_{\omega a}$ starts at $\omega(t^*)$, so
(as noted in the last clause of the statement of 455A)
$\mu_{\omega a}\{\omega':\omega'(t^*)=\omega(t^*)\}=1$.

For general $D$, we have an intersection of countably many sets
of these types, which will be $\mu_{\omega a}$-conegligible.

\medskip

\quad{\bf (iii)} Looking at the definition, we see that
$\nu^{(s,t)}_{\omega' ax}=\nu^{(s,t)}_{\omega ax}$
for all $s$, $t$ and $x$, and of course $\omega'(t^*)=\omega(t^*)$, so
$\mu_{\omega'a}=\mu_{\omega a}$.
}%end of proof of 455B

\leader{455C}{Theorem} Suppose that $T$, $t^*$,
$\family{t}{T}{(\Omega_t,\Tau_t)}$, $\Omega$, $x^*$,
$\langle\nu^{(s,t)}_x\rangle_{s<t,x\in\Omega_s}$ and $\mu$ are as in
455A.   Adjoin a point $\infty$ to $T$ above any point of $T$,
and let $\tau:\Omega\to T\cup\{\infty\}$ be a
function taking countably many
values and such that $\{\omega:\tau(\omega)\le s\}$
belongs to $\Tensorhat_{t\in T}\Tau_t$ and is determined by
coordinates in $[t^*,s]$ for every $s\in T$.

(a) For $\omega\in\Omega$ define $\nu^{(s,t)}_{\omega,\tau(\omega),x}$, for
$s<t$ and $x\in\Omega_s$, as in 455Bb, and let $\mu_{\omega,\tau(\omega)}$
be the corresponding measure on $\Omega$.   Then
$\family{\omega}{\Omega}{\mu_{\omega,\tau(\omega)}}$ is a disintegration of
$\mu$ over itself.

(b) Let $\Sigma_{\tau}$ be the set of those
$E\in\Tensorhat_{t\in T}\Tau_t$ such that
$E\cap\{\omega:\tau(\omega)\le t\}$ is determined by
coordinates in $[t^*,t]$ for every $t\in T$.
Then $\Sigma_{\tau}$ is a $\sigma$-subalgebra of
$\Tensorhat_{t\in T}\Tau_t$.   If
$f$ is any $\mu$-integrable real-valued function, and we set
$g_f(\omega)=\int fd\mu_{\omega,\tau(\omega)}$ when this is defined
in $\Bbb R$, then $g_f$ is a
conditional expectation of $f$ on $\Sigma_{\tau}$.

\proof{{\bf (a)(i)}
Set $F_t=\{\omega:\omega\in\Omega$, $\tau(\omega)=t\}$ for
$t\in T\cup\{\infty\}$;  note that $F_t\in\Tensorhat_{t\in T}\Tau_t$ for
every $t\in T\cup\{\infty\}$, and that $F_t$ is determined by coordinates
in $[t^*,t]$ for $t\in T$.

\medskip

\quad{\bf (ii)} Consider first the case in which $\tau$ takes only
finitely many values.   Suppose that
$J\subseteq T$ is a finite set including
$\{t^*\}\cup(T\cap\tau[\Omega])$.
Enumerate $J$ as $\langle t_i\rangle_{i\le n}$.
Suppose that $E_i\in\Tau_{t_i}$ for $i\le n$ and set
$H^*=\{\omega:\omega(t_i)\in E_i$ for every $i\le n\}$.
We need to calculate
$\int_{\Omega}\mu_{\omega,\tau(\omega)}(H^*)\mu(d\omega)$.

Set $H=\prod_{i\le n}E_i$,

\Centerline{$H_j=\prod_{j<i\le n}E_j$,
\quad$H^*_j=\{\omega:\omega\in\Omega$,
$\omega(t_i)\in E_i$ for $j<i\le n\}$,}

\Centerline{$G^*_j=\{\omega:\omega(t_i)\in E_i$ for $i\le j\}$}

\noindent for $j\le n$.
If $i<n$, $j\le n$, $\omega\in F_{t_j}$ and $x\in\Omega_{t_i}$, then

$$\eqalign{\nu_{\omega,\tau(\omega),x}^{(t_i,t_{i+1})}
&=\nu_x^{(t_i,t_{i+1})}\text{ if }i>j,\cr
&=\nu^{(t_j,t_{j+1})}_{\omega(t_j)}\text{ if }i=j,\cr
&=\delta^{(t_{i+1})}_{\omega(t_{j+1})}\text{ if }i<j.\cr}$$

So if $j\le n$ and $\omega\in F_{t_j}$,

$$\eqalign{\mu_{\omega,\tau(\omega)}(H^*)
&=\int\ldots\int\chi H(\omega(t^*),x_1,\ldots,x_n)
   \nu_{\omega,\tau(\omega),x_{n-1}}^{(t_{n-1},t_n)}(dx_n)
   \ldots\nu_{\omega,\tau(\omega),\omega(t^*)}^{(t^*,t_1)}(dx_1)\cr
&=\iint\ldots\int\chi H(x_0,\ldots,x_n)
   \nu_{\omega,\tau(\omega),x_{n-1}}^{(t_{n-1},t_n)}(dx_n)
   \ldots\nu_{\omega,\tau(\omega),x_0}^{(t^*,t_1)}(dx_1)
   \delta^{(t^*)}_{\omega(t^*)}(dx_0)\cr
&=\iint\ldots\iiint\ldots\int\chi H(x_0,\ldots,x_n)
   \nu_{\omega,\tau(\omega),x_{n-1}}^{(t_{n-1},t_n)}(dx_n)\cr
&\mskip100mu
   \ldots\nu_{\omega,\tau(\omega),x_{j+1}}^{(t_{j+1},t_{j+2})}(dx_{j+2})
   \nu_{\omega,\tau(\omega),x_j}^{(t_j,t_{j+1})}(dx_{j+1})
   \nu_{\omega,\tau(\omega),x_{j-1}}^{(t_{j-1},t_j)}(dx_j)\cr
&\mskip100mu  \ldots\delta^{(t_1)}_{\omega(t_1)}(dx_1)
    \delta^{(t^*)}_{\omega(t^*)}(dx_0)\cr
&=\int\ldots\iiint\ldots\int\chi H(x_0,\ldots,x_n)
   \nu_{x_{n-1}}^{(t_{n-1},t_n)}(dx_n)\cr
&\mskip100mu   \ldots\nu_{x_j}^{(t_j,t_{j+1})}(dx_{j+1})
   \nu_{\omega(t_j)}^{(t_j,t_{j+1})}(dx_{j+1})
   \delta^{(t_j)}_{\omega(t_j)}(dx_j)
   \ldots
   \delta^{(t^*)}_{\omega(t^*)}(dx_0)\cr
&=\int\ldots\int\chi H(\omega(t^*),\ldots,\omega(t_j),x_{j+1},\ldots,x_n)
   \nu_{x_{n-1}}^{(t_{n-1},t_n)}(dx_n)\cr
&\mskip100mu
   \ldots\nu_{\omega(t_j)}^{(t_j,t_{j+1})}(dx_{j+1})\cr
&=\int\ldots\int\chi H_j(x_{j+1},\ldots,x_n)
   \nu_{x_{n-1}}^{(t_{n-1},t_n)}(dx_n)
   \ldots\nu_{\omega(t_j)}^{(t_j,t_{j+1})}(dx_{j+1})\cr
&\mskip200mu\text{ if }\omega\in G^*_j,\cr
&=0\text{ otherwise}.\cr}$$

\noindent As noted in 455B(b-ii),
$\mu_{\omega,\tau(\omega)}(H^*)=\chi H^*(\omega)$ if $\tau(\omega)=\infty$.

Now

$$\eqalignno{\int_{\Omega}\mu_{\omega,\tau(\omega)}(H^*)\mu(d\omega)
&=\sum_{j=0}^n\int_{F_{t_j}}\mu_{\omega,\tau(\omega)}(H^*)\mu(d\omega)
  +\int_{F_{\infty}}\mu_{\omega,\tau(\omega)}(H^*)\mu(d\omega)\cr
&=\sum_{j=0}^n\int_{F_{t_j}\cap G^*_j}\int\ldots\int
\chi H_j(x_{j+1},\ldots,x_n)
   \nu_{x_{n-1}}^{(t_{n-1},t_n)}(dx_n)\cr
&\mskip100mu
   \ldots\nu_{\omega(t_j)}^{(t_j,t_{j+1})}(dx_{j+1})\mu(d\omega)
   +\mu(F_{\infty}\cap H^*)\cr
&=\sum_{j=0}^n\mu(F_{t_j}\cap G^*_j\cap H^*_j)
   +\mu(H^*\cap F_{\infty})\cr
\displaycause{by 455Ba}
&=\sum_{j=0}^n\mu(F_{t_j}\cap H^*)
   +\mu(F_{\infty}\cap H^*)
=\mu H^*.\cr}$$

Thus we have the formula we need when $E$ is of the special form
$\{\omega:\omega(t)\in E_t$ for every $t\in J\}$, $J\subseteq T$
being a finite set and $E_t$ being a member of $\Tau_t$ for every
$t\in J$.   By the Monotone Class Theorem (136B), we shall have
$\int\mu_{\omega,\tau(\omega)}(E)\mu(d\omega)=\mu E$ for every
$E\in\Tensorhat_{t\in T}\Tau_t$, so that
$\family{\omega}{\Omega}{\mu_{\omega,\tau(\omega)}}$ is a disintegration of $\mu$ over
itself.

\medskip

\quad{\bf (iii)} If $\tau$ takes infinitely many values, enumerate them as
$\sequencen{t_n}$, and for $n\in\Bbb N$ define
$\tau_n:T\to T\cup\{\infty\}$ by setting

$$\eqalign{\tau_n(\omega)
&=t_i\text{ if }i\le n\text{ and }\tau(\omega)=t_i,\cr
&=\infty\text{ if }\tau(\omega)\notin\{t_i:i\le n\}.\cr}$$

\noindent Then $\tau_n$ takes only finitely many values, and
$\{\omega:\tau_n(\omega)\le t\}\in\Tensorhat_{t\in T}\Tau_t$
is determined by coordinates in $[0,t]$ for every $t\in T$.
So we shall have

\Centerline{$\int\mu_{\omega,\tau_n(\omega)}(E)\mu(d\omega)=\mu E$}

\noindent for every $E\in\Tensorhat_{t\in T}\Tau_t$.   Now observe that
$\mu_{\omega,\tau_n(\omega)}=\mu_{\omega,\tau(\omega)}$ whenever
$\tau(\omega)=\tau_n(\omega)$.
So, for each $\omega$,
$\mu_{\omega,\tau_n(\omega)}=\mu_{\omega,\tau(\omega)}$ for all but
finitely many $n$.   This means that, for every
$E\in\Tensorhat_{t\in T}\Tau_t$,

\Centerline{$\mu_{\omega,\tau(\omega)}(E)
=\lim_{n\to\infty}\mu_{\omega,\tau_n(\omega)}(E)$}

\noindent for every $\omega\in\Omega$, and

\Centerline{$\int\mu_{\omega,\tau(\omega)}(E)\mu(d\omega)
=\lim_{n\to\infty}\int\mu_{\omega,\tau_n(\omega)}(E)\mu(d\omega)=\mu E$,}

\noindent as required.

\medskip

{\bf (b)(i)} Since $\{\omega:\tau(\omega)\le t\}$ is determined by
coordinates in $[t^*,t]$ for every $t\in T$, $\Omega\in\Sigma_{\tau}$, and it
is now elementary to confirm that $\Sigma_{\tau}$ is a $\sigma$-algebra.

\medskip

\quad{\bf (ii)} I had better note that $g_f$ is defined almost everywhere;
this is because, by (a) above and 452F,

\Centerline{$\int g_fd\mu=\iint fd\mu_{\omega,\tau(\omega)}\mu(d\omega)
=\int fd\mu$.}

\medskip

\quad{\bf (iii)} If $\omega$, $\omega'\in\Omega$ and
$\omega'\restr[t^*,\tau(\omega)]=\omega\restr[t^*,\tau(\omega)]$, then
$g_f(\omega)=g_f(\omega')$ if either is defined.   \Prf\ Since
$F_{\tau(\omega)}$ is determined by coordinates in $[t^*,\tau(\omega)]$,
$\tau(\omega')=\tau(\omega)$.   By 455B(b-ii),
$\mu_{\omega',\tau(\omega')}=\mu_{\omega,\tau(\omega)}$, so
$g_f(\omega)=g_f(\omega')$ if either is defined.\ \Qed

\medskip

\quad{\bf (iv)} If $F\in\Sigma_{\tau}$ and $\omega\in\Omega$, then
$\mu_{\omega,\tau(\omega)}F=1$ if $\omega\in F$, $0$ otherwise.   \Prf\
Setting $b=\tau(\omega)$, $F\cap F_b$ and $F_b\setminus F$
are determined by coordinates
in a countable subset of $T\cap[t^*,b]$, so by 455B(b-ii) we have
$\mu_{\omega b}F=1$ if $\omega\in F_b\cap F$ and
$\mu_{\omega b}(F_b\setminus F)=1$ if $\omega\in F_b\setminus F$.\ \Qed

It follows that if $f$ is $\mu$-integrable and $F\in\Sigma_{\tau}$, then
$g_{f\times\chi F}=g_f\times\chi F$.   \Prf\ If $\omega\in F$, then
$\mu_{\omega,\tau(\omega)}F=1$ and

\Centerline{$g_{f\times\chi F}(\omega)
=\int_Ffd\mu_{\omega,\tau(\omega)}=\int fd\mu_{\omega,\tau(\omega)}$;}

\noindent if $\omega\notin F$ then $\mu_{\omega,\tau(\omega)}F=0$ and

\Centerline{$g_{f\times\chi F}(\omega)
=\int_Ffd\mu_{\omega,\tau(\omega)}=0$.   \Qed}

\medskip

\quad{\bf (v)} Now let $f$ be any $\mu$-integrable real-valued function.
Then there
is a $\Sigma_{\tau}$-measurable function
$g'_f:\Omega\to\ocint{-\infty,\infty}$
such that $g'_f\eae g_f$, $g'_f(\omega)\le g_f(\omega)$ for every
$\omega\in\dom g_f$, and $g'_f(\omega)=-\infty$ for every
$\omega\in\Omega\setminus\dom g_f$.   \Prf\ For $q\in\Bbb Q$, set
$W_q=\{\omega:\omega\in\dom g_f$, $g_f(\omega)\ge q\}$.
For $q\in\Bbb Q$ and $b\in\tau[\Omega]$, consider $W_{bq}=W_q\cap F_b$.
$W_{bq}$ is measured by the completion $\hat\mu$ of $\mu$,
and is determined by coordinates in $T\cap[t^*,b]$, by (iii).
By 451K(b-ii) there is a
$W'_{bq}\in\Tensorhat_{t\in T}\Tau_t$ such that
$W'_{bq}\subseteq W_{bq}$, $W_{bq}\setminus W'_{bq}$ is negligible and
$W'_{bq}$ is determined by coordinates in $T\cap [t^*,b]$.

Having defined the family
$\langle W'_{bq}\rangle_{b\in\tau[\Omega],q\in\Bbb Q}$, set
$W'_q=\bigcup_{b\in\tau[\Omega]}W'_{bq}$ for $q\in\Bbb Q$.   Then
$W'_q\in\Tensorhat_{t\in T}\Tau_t$ and
$W'_q\cap F_b=W'_{bq}$ is determined by
coordinates in $T\cap[t^*,b]$ for every $b\in\tau[\Omega]$, so
$W'_q\in\Sigma_{\tau}$.   Also
$W'_q\subseteq W_q$ and $W_q\setminus W'_q$ is negligible.

Set

\Centerline{$g'_f(\omega)
=\sup\{q:q\in\Bbb Q$, $\omega\in W'_q\}$}

\noindent for $\omega\in\Omega$, counting $\sup\emptyset$ as
$-\infty$.   Then $g'_f$ is $\Sigma_{\tau}$-measurable,
$g'_f(\omega)=-\infty$ for $\omega\notin\dom g_f$,
$g'_f(\omega)\le g_f(\omega)$ for $\omega\in\dom g_f$, and
$g'_f=g_f$ on $\dom g_f\setminus\bigcup_{q\in\Bbb Q}W_q\setminus W'_q$, so
$g'_f\eae g_f$.\ \Qed

\medskip

\quad{\bf (vi)} Continuing from (v), we find that $g'_f$ is a conditional
expectation of $f$ on $\Sigma_{\tau}$.\footnote{The definition of
`conditional expectation' in 233D was directed towards real-valued
functions, and $g'_f$ is permitted to take the values $\pm\infty$.
So what I really mean here
is that the restriction of $g'_f$ to the set on which
it is finite is a conditional expectation of $f$.}
\Prf\ I have already shown that
$g'_f$ is $\Sigma_{\tau}$-measurable.   If $F\in\Sigma_{\tau}$ then

$$\eqalignno{\int_Fg'_fd\mu
&=\int g'_f\times\chi F\,d\mu
=\int g_f\times\chi F\,d\mu\cr
\displaycause{because $g_f\eae g'_f$}
&=\int g_{f\times\chi F}d\mu\cr
\displaycause{by (iv)}
&=\iint f\times\chi F\,d\mu_{\omega,\tau(\omega)}\mu(d\omega)
=\int f\times\chi F\,d\mu\cr
\displaycause{452F once again}
&=\int_Ffd\mu.  \text{ \Qed}\cr}$$

\medskip

\quad{\bf (vii)} Similarly, or applying the arguments of (v)-(vi) to $-f$,
we see that for any $\mu$-integrable function $f$ there is a conditional
expectation $g''_f$ of $f$ on $\Sigma_{\tau}$ such that
$g''_f(\omega)\ge g_f(\omega)$ when $\omega\in\dom g_f$ and
$g''_f(\omega)=\infty$ when $g_f(\omega)$ is undefined.   Now
$g'_f\eae g''_f$ and both are $\Sigma_{\tau}$-measurable.   It follows that
$g_f$ is defined, and equal to both $g'_f$ and $g''_f$,
$(\mu\restr\Sigma_{\tau})$-a.e.;  so that $g_f$ itself is also a
conditional expectation of $f$ on $\Sigma_{\tau}$.
}%end of proof of 455C

\cmmnt{
\leader{455D}{Remarks (a)}
The idea of the construction in 455A
is that $\family{t}{T}{X_t}$ is a family of random
variables, and that we start from the assurance that `history is
irrelevant';  if, at time $b$, we wish to make guesses about the
behaviour of $X_t$, the state of the system at a future time $t$, then
we expect that it will be useful to look at the current state $X_b$, but
once we know the value
of $X_b$ then any further information about $X_s$ for $s<b$ will
tell us nothing more about $X_t$.   We are given the {\bf transitional
probabilities}
$\nu_x^{(s,t)}$, which can be thought of as the conditional
distributions
of $X_t$ given that $X_s=x$.   The condition ($\dagger$) of 455A
is plainly
necessary if the system is going to make sense at all;  the content of
the theorem is that it is also sufficient, at least when all the
conditional expectations are perfect measures, to ensure
that the system as a whole can indeed be represented as a family of
random variables, in Kolmogorov's sense, on a suitable probability
space.

\spheader 455Db The statement
`$\family{\omega}{\Omega}{\mu_{\omega,\tau(\omega)}}$
is a disintegration of $\mu$ over itself' in 455Ca
is not obviously a target worth
working very hard for.   But the point of this particular family is that
not only does $\mu_{\omega,\tau(\omega)}$ follow $\omega$ up to
and including time $\tau(\omega)$ (455B(b-ii)), but also
$\mu_{\omega,\tau(\omega)}=\mu_{\omega',\tau(\omega')}$ whenever
$\omega'\restr[t^*,\tau(\omega)]=\omega\restr[t^*,\tau(\omega)]$, as noted
in (b-iii) of the proof of 455B.

If we take $\tau$ in 455C to be constant, with value $b\in T$, then we get
a precise description of what it means for `history to be irrelevant'.
In this case, we can take the measures $\mu_{\omega b}$, and
project them
onto $\prod_{t\ge b}\Omega_t$;  let
$\lambda^{(\omega)}_{\coint{b,\infty}}$
be the image measure.   Then it is easy to check that
$\lambda^{(\omega)}_{\coint{b,\infty}}$ is the measure defined from the
point $\omega(b)$ and the family
$\langle\nu^{(s,t)}_x\rangle_{b\le s<t,x\in\Omega_s}$ by the method of
455A;  so that
$\lambda^{(\omega)}_{\coint{b,\infty}}
=\lambda^{(\omega')}_{\coint{b,\infty}}$ whenever $\omega(b)=\omega'(b)$.

\spheader 455Dc I have called 455C a `theorem', and there are certainly
enough ideas in it to warrant the title.   But the restriction to stopping
times taking only countably many values means that we are a large step away
from a result which is really useful in continuous time.
The calculations with sets
$\{\omega:\tau(\omega)=b\}$ in the proofs of 455C and 455E
are a clear sign that we are not yet
ready for continuous stopping times, in which
$\{\omega:\tau(\omega)=b\}$ will usually be negligible for every $b$,
except perhaps $b=\infty$.   Of course we can
use 455C with $T=\Bbb N$;  but it must be obvious that there are better
and cleaner expressions of the result in this case.   In the work below,
455C is going to function as a lemma, the first stage in much stronger
results (starting with 455O)
which depend on special properties of the measures $\nu^{(s,t)}_x$.

\spheader 455Dd In the context of 455A, it seemed to involve
fewer explanations to take a fixed $\sigma$-algebra $\Tau_t$ for each $t$
and to define $\mu$ on
$\Tensorhat_{t\in T}\Tau_t$.   As you know, I ordinarily have a strong
prejudice in favour of completing measures.   In the situations most
important to us,
this is perfectly straightforward, if a touch laborious;  I present a
version in the next theorem.
}%end of comment

\vleader{72pt}{455E}{Theorem} Let $T$ be a totally ordered set with least
element $t^*$.   Let $\family{t}{T}{\Omega_t}$ be a family of Hausdorff
spaces;  suppose that we are given an $x^*\in\Omega_{t^*}$
and, for each
pair $s<t$ in $T$, a family $\family{x}{\Omega_s}{\nu^{(s,t)}_x}$ of
Radon probability measures on $\Omega_t$ such that

\inset{$\family{y}{\Omega_s}{\nu_y^{(t,u)}}$ is a disintegration of
$\nu_x^{(s,u)}$ over $\nu_x^{(s,t)}$
whenever $s<t<u$ in $T$ and $x\in\Omega_s$.}

\noindent Write $\Omega=\prod_{t\in T}\Omega_t$;
for $t\in T$ let
$\Cal B(\Omega_t)$ be the Borel $\sigma$-algebra of $\Omega_t$, and
$X_t:\Omega\to\Omega_t$ the canonical map;
for $J\subseteq T$ write $\pi_J$ for the canonical map from
$\Omega$ onto $\prod_{t\in J}\Omega_t$.   For $t\in T$ and $x\in\Omega_t$
let $\delta^{(t)}_x$ be the Dirac measure on $\Omega_t$ concentrated at
$x$.

(a) There is a unique complete probability measure
$\hat\mu$ on $\Omega$, inner regular with respect to
$\Tensorhat_{t\in T}\Cal B(\Omega_t)$,
such that, writing $\hat\lambda_J$ for the image measure
$\hat\mu\pi_J^{-1}$,

$$\eqalign{\int fd\hat\lambda_J
&=\int f(\omega(t^*),\omega(t_1),\ldots,\omega(t_n))\hat\mu(d\omega)\cr
&=\int\ldots\iint f(x^*,x_1,\ldots,x_n)
   \nu_{x_{n-1}}^{(t_{n-1},t_n)}(dx_n)\cr
&\qquad\qquad\qquad\qquad
  \nu_{x_{n-2}}^{(t_{n-2},t_{n-1})}(dx_{n-1})
  \ldots\nu_{x^*}^{(t^*,t_1)}(dx_1)\cr}$$

\noindent
whenever $t^*<t_1<\ldots< t_n$ in $T$, $J=\{t^*,t_1,\ldots,t_n\}$ and
$f$ is $\hat\lambda_J$-integrable.
In particular, the image measure $\hat\mu X_t^{-1}$ is equal
to $\nu_{x^*}^{(t^*,t)}$ if $t>t^*$, and to $\delta^{(t^*)}_{x^*}$ if
$t=t^*$.

(b)(i) For $\omega\in\Omega$ and $a\in T\cup\{\infty\}$ define
$\langle\nu^{(s,t)}_{\omega ax}\rangle_{s<t,x\in\Omega_s}$
by setting

$$\eqalign{\nu^{(s,t)}_{\omega ax}
&=\nu_x^{(s,t)}\text{ if }a<s,\cr
&=\nu_{\omega(a)}^{(a,t)}\text{ if }s\le a<t,\cr
&=\delta^{(t)}_{\omega(t)}\text{ if }t\le a.\cr}$$

\noindent The family
$\langle\nu^{(s,t)}_{\omega ax}\rangle_{s<t,x\in\Omega_s}$,
together with the point $\omega(t^*)\in\Omega_{t^*}$,
satisfy the conditions of (a), so can
be used to define a complete measure $\hat\mu_{\omega a}$ on $\Omega$.

\quad(ii) If $\omega\in\Omega$ and $D\subseteq T\cap[t^*,a]$
is countable, then
$\hat\mu_{\omega a}\{\omega':\omega'\restr D=\omega\restr D\}=1$.

\quad(iii) If $\omega$, $\omega'\in\Omega$ and
$\omega'\restr[t^*,a]=\omega\restr[t^*,a]$, then
$\hat\mu_{\omega' a}=\hat\mu_{\omega a}$.

(c) Let $\Sigma$ be the domain of $\hat\mu$.
Suppose that $\tau:\Omega\to T\cup\{\infty\}$ is a
function taking countably many
values and such that $\{\omega:\tau(\omega)\le t\}$ belongs to
$\Sigma$ and is determined by
coordinates in $[t^*,t]$ for every $t\in T$.

\quad(i) $\family{\omega}{\Omega}{\hat\mu_{\omega,\tau(\omega)}}$
is a disintegration of $\hat\mu$ over itself.

\quad(ii) Let $\Sigma_{\tau}$ be the set

\Centerline{$\{E:E\in\Sigma$,
$E\cap\{\omega:\tau(\omega)\le t\}$ is determined by
coordinates in $[t^*,t]$ for every $t\in T\}$.}

\noindent Then $\Sigma_{\tau}$ is a $\sigma$-subalgebra of
$\Sigma$.   If
$f$ is any $\hat\mu$-integrable real-valued function, and we set
$g_f(\omega)=\int fd\hat\mu_{\omega,\tau(\omega)}$ when this is defined
in $\Bbb R$, then $g_f$ is a
conditional expectation of $f$ on $\Sigma_{\tau}$.

\proof{ My aim is to apply 455A-455C %455A 455B 455C
to the Borel measures
$\grave\nu_x^{(s,t)}=\nu_x^{(s,t)}\restr\Cal B(\Omega_t)$, and
take $\hat\mu$ to be the completion of the Baire measure
$\mu$ produced by the
method of 455A.   The essential discipline is to check carefully that
almost every measure $\zeta$ is
the completion of an appropriate measure $\grave\zeta$.

\medskip

{\bf (a)} At the start, every Radon probability measure is the completion
of the corresponding Borel measure, so that the
$\nu^{(s,t)}_x$ are indeed the completions of the
$\grave\nu^{(s,t)}_x$ defined from them.    Since completing a measure does not
affect the associated integration (212Fb), the condition

\inset{whenever $s<t<u$ in $T$, $x\in\Omega_s$ and
$E\subseteq\Omega_u$ is a Borel
set, then $\grave\nu_s^{(s,u)}(E)=\int\grave\nu_y^{(t,u)}(E)\grave\nu_x^{(s,t)}(dy)$}

\noindent follows at once from

\inset{whenever $s<t<u$ in $T$ and $x\in\Omega_s$,
then $\family{y}{\Omega_s}{\nu_y^{(t,u)}}$ is a disintegration of
$\nu_x^{(s,u)}$ over $\nu_x^{(s,t)}$.}

\noindent Also the $\grave\nu_x^{(s,t)}$, being tight Borel measures,
are all perfect (342L/451C).
So we can indeed form a measure $\mu$ on
$\Omega$ with domain $\Tensorhat_{t\in T}\Cal B(\Omega_t)$
by the process in 455A, and complete it.

The next step has a little more content in it:  I need to show that for any
$J\subseteq T$, the image measure $\hat\mu\pi_J^{-1}$ on
$\prod_{t\in J}\Omega_t$ is the completion of the image measure
$\mu\pi_J^{-1}$.   But here we just have to recall that $\mu$ is perfect
(454D), so that we can use 451Kb.   For finite $J\subseteq T$
we can therefore write
$\hat\lambda_J$ indifferently for the completion of
$\lambda_J=\mu\pi_J^{-1}$ and for $\hat\mu\pi_J^{-1}$,
and the formula for $\int fd\hat\lambda_J$ can be read off
from 455A, since it deals only with integrals, which are unaffected by
completions.

\medskip

{\bf (b)} This follows 455Bb.   This time we must start by noting that
every $\nu^{(s,t)}_{\omega ax}$ is a Radon probability measure.

\medskip

\quad{\bf (i)} The formulae of part (i) of the proof of 455Bb can still be
applied to show that

\Centerline{$\int_{\Omega_t}\nu^{(t,u)}_{\omega ay}(E)
   \nu^{(s,t)}_{\omega ax}(dy)
=\nu_{\omega ax}^{(s,u)}(E)$}

\noindent whenever $s<t<u$, $x\in\Omega_s$ and $E\in\Cal B(\Omega_u)$.
Since any set measured by $\nu_{\omega ax}^{(s,u)}$ can be
approximated internally and externally by Borel sets, we see that
$\family{y}{\Omega_t}{\nu^{(t,u)}_{\omega ay}}$ is a
disintegration of $\nu^{(s,u)}_{\omega ax}$ over
$\nu^{(s,t)}_{\omega ax}$.   (Cf.\ 452Xg.)

\medskip

\quad{\bf (ii)} Similarly, the argument of part (ii) of the proof of 455Bb
can still be used to show that whenever $\omega\in\Omega$ and
$D\subseteq T\cap[t^*,a]$ is countable, then
$\omega'\restr D=\omega\restr D$ for
$\hat\mu_{\omega a}$-almost every $\omega'\in\Omega$.

\medskip

\quad{\bf (iii)} Once again, we can use the argument
from 455B;  if $\omega'\restr[t^*,a]=\omega\restr[t^*,a]$, then
$\nu^{(s,t)}_{\omega' ax}=\nu^{(s,t)}_{\omega ax}$ for all
$x$, $s$ and $t$, and $\hat\mu_{\omega'a}=\hat\mu_{\omega a}$.

\medskip

{\bf (c)(i)}\grheada\
The key step here is to observe that there is a function
$\grave\tau:\Omega\to T\cup\{\infty\}$ which satisfies the properties
required in 455C and is equal $\hat\mu$-almost everywhere to $\tau$.
\Prf\
For each $a\in T\cap\tau[\Omega]$, $F_a=\tau^{-1}[\{a\}]$ belongs to
$\Sigma$ and is determined by coordinates in $[t^*,a]$.
By 451K(b-ii) again, there is an
$F'_a\in\Tensorhat_{t\in T}\Cal B(\Omega_t)$ such that
$F'_a\subseteq F_a$, $F'_a$ is determined by coordinates in $[t^*,a]$ and
$\hat\mu(F_a\setminus F'_a)=0$.   Define $\grave\tau$ by setting

$$\eqalign{\grave\tau(\omega)
&=a\text{ if }a\in T\cap\tau[\Omega]\text{ and }\omega\in F'_a,\cr
&=\infty
  \text{ if }\omega\in\Omega\setminus\bigcup_{a\in T\cap\tau[\Omega]}F'_a.
\cr}$$

\noindent It is easy to check that this $\grave\tau$ will serve.\ \Qed

\medskip

\qquad\grheadb\ For $\omega\in\Omega$ and $a\in T$, define
$\langle\grave\nu^{(s,t)}_{\omega ax}
\rangle_{s<t,x\in\Omega_t}$ and
$\family{\omega}{\Omega}{\mu_{\omega a}}$
from $\langle\grave\nu^{(s,t)}_x\rangle_{s<t,x\in\Omega_t}$ and
$\grave\tau$ as in 455Bb.   If $\tau(\omega)=\grave\tau(\omega)$ then
$\grave\nu^{(s,t)}_{\omega,\grave\tau(\omega),x}
=\nu^{(s,t)}_{\omega,\tau(\omega),x}\restr\Cal B(\Omega_t)$
for all $s$, $t$ and $x$, so that
$\hat\mu_{\omega,\tau(\omega)}$ is the completion of
$\mu_{\omega,\grave\tau(\omega)}$.
This is true for almost all $\omega$.   Now we know from 455Ca that
$\family{\omega}{\Omega}{\mu_{\omega,\grave\tau(\omega)}}$
is a disintegration of $\mu$ over itself,
and therefore also over $\hat\mu$.   It follows that
$\family{\omega}{\Omega}{\hat\mu_{\omega,\grave\tau(\omega)}}$ is a
disintegration of $\hat\mu$ over $\hat\mu$, by 452B(ii).   But
$\hat\mu_{\omega,\grave\tau(\omega)}=\hat\mu_{\omega,\tau(\omega)}$ for
$\hat\mu$-almost every
$\omega$, so $\family{\omega}{\Omega}{\hat\mu_{\omega,\tau(\omega)}}$
also is a disintegration of $\hat\mu$ over itself.

\medskip

\quad{\bf (ii)}\grheada\ Just as in part (b-i) of the proof of 455C,
$\Sigma_{\tau}$ is a $\sigma$-algebra because it contains $\Omega$.

\medskip

\qquad\grheadb\ Recall the $F_a$, $F'_a$ in (i-$\alpha$) above.
Set $F_{\infty}=\tau^{-1}[\{\infty\}]$, and take
$F'_{\infty}\in\Tensorhat_{t\in T}\Cal B(\Omega_t)$ such that
$F'_{\infty}\subseteq F_{\infty}$ and $F_{\infty}\setminus F'_{\infty}$ is
negligible.   Then $F^*=\bigcup_{a\in\tau[\Omega]}F'_a$ is conegligible in
$\Omega$.   Write $\grave\Sigma_{\grave\tau}$ for the
set of those $F\in\Tensorhat_{t\in T}\Cal B(\Omega_t)$ such that
$F\cap F'_a$ is determined by coordinates in $[t^*,a]$ for every
$a\in T\cap\grave\tau[\Omega]$.   Then
$F^*\in\grave\Sigma_{\grave\tau}\cap\Sigma_{\tau}$ because
$F^*\cap F'_a=F^*\cap F_a=F'_a$ for every $a\in\tau[\Omega]$.
In fact we have more.   First, $\tau\restr F^*=\grave\tau\restr F^*$.
Next, if $F\subseteq F^*$ and $F\in\grave\Sigma_{\grave\tau}$, then
$F\in\Sigma_{\tau}$.   \Prf\ For any $a\in T\cap\tau[\Omega]$,
$F\cap F_a=F\cap F'_a$ belongs to
$\Tensorhat_{t\in T}\Cal B(\Omega_t)\subseteq\Sigma$
and is determined by coordinates in $[t^*,a]$.\ \QeD\  And thirdly, if
$F\in\Sigma_{\tau}$, there is a
$G\in\grave\Sigma_{\grave\tau}$ such that $G\subseteq F$ and
$F\setminus G$ is negligible.   \Prf\ As in (iv-$\alpha$), we can find for
each $a\in\tau[\Omega]$ a set $G_a\in\Tensorhat_{t\in T}\Cal B(\Omega_t)$,
determined by coordinates in $T\cap[t^*,a]$, such that
$G_a\subseteq F\cap F_a$ and
$(F\cap F_a)\setminus G_a$ is negligible.   Set
$G=\bigcup_{a\in\tau[\Omega]}G_a$.\ \Qed

\medskip

\qquad\grheadc\ Now take a $\hat\mu$-integrable function $f$.   Then it is
$\mu$-integrable.   By 455Cb,
$\grave g_f$ is a conditional expectation of $f$ on
$\grave\Sigma_{\grave\tau}$, where

\Centerline{$\grave g_f(\omega)
=\int fd\mu_{\omega,\grave\tau(\omega)}
=\int fd\hat\mu_{\omega,\grave\tau(\omega)}$}

\noindent whenever the integral is defined in $\Bbb R$.
We know that there is a $\grave\Sigma_{\grave\tau}$-measurable
function $g':\Omega\to\Bbb R$ equal to
$\grave g_f$ except perhaps on a negligible set $H$ belonging to
$\grave\Sigma_{\grave\tau}$.    Replacing $g'$ by $g'\times\chi F^*$
and $H$ by $H\cup(\Omega\setminus F^*)$ if
necessary, we can suppose that $g'$ is zero outside $F^*$
and that $\Omega\setminus H\subseteq F^*$.   In this case,
$g'$ is $\Sigma_{\tau}$-measurable.   \Prf\ For any $\alpha\in\Bbb R$,

$$\eqalign{\{\omega:g'(\omega)\ge\alpha\}
&=\{\omega:\omega\in F^*,\,g'(\omega)\ge\alpha\}\cup(\Omega\setminus F^*)
   \text{ if }\alpha\le 0,\cr
&=\{\omega:\omega\in F^*,\,g'(\omega)\ge\alpha\}
   \text{ if }\alpha>0,\cr}$$

\noindent and in either case belongs to $\Sigma_{\tau}$, by
($\beta$).\ \QeD\  At the same time, we note that $H\in\Sigma_{\tau}$.

If $\omega\in\Omega\setminus H$, then $\omega\in F^*$,
$\tau(\omega)=\grave\tau(\omega)$,
$\hat\mu_{\omega,\tau(\omega)}=\hat\mu_{\omega,\grave\tau(\omega)}$ and

\Centerline{$g_f(\omega)=\int fd\hat\mu_{\omega,\tau(\omega)}
=\int fd\hat\mu_{\omega,\grave\tau(\omega)}
=\int fd\mu_{\omega,\grave\tau(\omega)}
=\grave g_f(\omega)=g'(\omega)$.}

\noindent So $g_f$ is defined and equal to $g'$ and $\grave g_f$
except perhaps on the
negligible set $H$ belonging to $\Sigma_{\tau}$;  consequently $g_f$ is
defined $(\hat\mu\restr\Sigma_{\tau})$-a.e.\ and is
$(\hat\mu\restr\Sigma_{\tau})$-virtually measurable.

If $F\in\Sigma_{\tau}$,
there is a $G\in\grave\Sigma_{\grave\tau}$ such
that $G\subseteq F$ and $F\setminus G$ is negligible,
by the last remark in ($\beta$).   So

$$\eqalignno{\int_Ffd\hat\mu
&=\int_Gfd\mu
=\int_G\grave g_fd\mu\cr
\displaycause{because $\grave g_f$ is a conditional expectation of $f$ on
$\grave\Sigma_{\grave\tau}$}
&=\int_F\grave g_fd\hat\mu
=\int_Fg_fd\hat\mu
=\int_Fg_fd(\hat\mu\restr\Sigma_{\tau}).\cr}$$

\noindent As $F$ is arbitrary, $g_f$ is a conditional expectation of
$f$ on $\Sigma_{\tau}$, and the proof is complete.
}%end of proof of 455E

\leader{455F}{}\cmmnt{ Of course the leading example for the work above
is the case in which $T=\coint{0,\infty}$ and $\Omega_t=\Bbb R$ for every
$t\ge 0$.   Moving towards this, a natural intermediate stage is when
$T=\coint{0,\infty}$ and all the $\Omega_t$ are the same, so that we can
regard an element of $\prod_{t\in T}\Omega_t$ as the path of a moving
point.   In this case we can begin to think about paths which are more or
less continuous.   The next theorem gives a widely applicable condition for
existence of many paths which are one-sidedly continuous.
It depends on a fairly strong continuity property for the transitional
probabilities.

\medskip

\noindent}{\bf Definitions (a)} Let $U$ be a Hausdorff space and
$\langle\nu^{(s,t)}_x\rangle_{0\le s<t,x\in U}$ a family of Radon
probability measures on $U$.   I will say that
$\langle\nu^{(s,t)}_x\rangle_{0\le s<t,x\in U}$ is
{\bf narrowly continuous} if
it is continuous, as a function from $\{(s,t):0\le s<t\}\times U$
to the set of
Radon probability measures on $U$, when the latter is given its narrow
topology\cmmnt{ (437Jd)}.

\cmmnt{\medskip

\noindent{\bf Remark} I speak of the `narrow' topology here partly
because, in the present treatise, this has become the standard topology on
spaces of Radon measures, and partly because the phrase `vaguely
continuous' seems inappropriate.   But, as will appear, all the results
below will rely on the fact that the vague topology (437Jc)
is coarser than the narrow topology.   In the present context, in which we
have Radon measures on a completely regular Hausdorff space, the two
topologies actually coincide (437L).   So
$\langle\nu^{(s,t)}_x\rangle_{0\le s<t,x\in U}$ is narrowly continuous iff
$(s,t,x)\mapsto\int fd\nu^{(s,t)}_x$ is continuous for every bounded
continuous $f:\Omega\to\Bbb R$.}

\medskip

{\bf (b)} Let $(U,\rho)$ be a metric
space, and $\langle\nu^{(s,t)}_x\rangle_{0\le s<t,x\in U}$ a family of
Radon probability measures on $U$.   I will say that
$\langle\nu^{(s,t)}_x\rangle_{0\le s<t,x\in U}$ is
{\bf uniformly time-continuous on the right} if for every $\epsilon>0$
there is a
$\delta>0$ such that $\nu^{(s,t)}_xB(x,\epsilon)\ge 1-\epsilon$ whenever
$x\in U$ and $0\le s<t\le s+\delta$.

%slightly stronger than "Feller process", which means I think "uniformly
%time-cts on the right on cpct sets",
%but used only for locally compact spaces

\leader{455G}{Theorem} Let $(U,\rho)$ be a complete metric
space and $\langle\nu^{(s,t)}_x\rangle_{0\le s<t,x\in U}$ a family of
Radon probability measures on $U$, uniformly time-continuous on the right,
such that $\family{y}{U}{\nu_y^{(t,u)}}$ is a disintegration of
$\nu_x^{(s,u)}$ over $\nu_x^{(s,t)}$ whenever $0\le s<t<u$ and $x\in U$.
Take a point $\tilde\omega$ in $\Omega=U^{\coint{0,\infty}}$, and
$a\in[0,\infty]$.   Let $\hat\mu_{\tilde\omega a}$ be the completed
probability measure on $\Omega$ defined from
$\langle\nu^{(s,t)}_x\rangle_{0\le s<t,x\in U}$, $\tilde\omega$ and $a$
as in 455Eb.

(a) For $\hat\mu_{\tilde\omega a}$-almost every $\omega\in\Omega$,
$\lim_{q\in\Bbb Q,q\downarrow t}\omega(q)$ and
$\lim_{q\in\Bbb Q,q\uparrow t}\omega(q)$ are defined in $U$ for every
$t>a$.

(b)(i) If $a\le t<\infty$, then
$\omega(t)=\lim_{q\in\Bbb Q,q\downarrow t}\omega(q)$
for $\hat\mu_{\tilde\omega a}$-almost every $\omega\in\Omega$.

\quad(ii) If $a<t<\infty$, then
$\omega(t)=\lim_{q\in\Bbb Q,q\uparrow t}\omega(q)$
for $\hat\mu_{\tilde\omega a}$-almost every $\omega\in\Omega$.

(c)(i) Let $\Clll$ be the set of \callal\ functions from
$\coint{0,\infty}$ to $U$\cmmnt{ (438S)}.
If $\tilde\omega\in\Clll$,
$\Clll$ has full outer measure for $\hat\mu_{\tilde\omega a}$.

\quad(ii) Let $\Cdlg$ be the set of \cadlag\ functions from
$\coint{0,\infty}$ to $U$.   If $\tilde\omega\in\Cdlg$,
$\Cdlg$ has full outer measure for $\hat\mu_{\tilde\omega a}$.

\cmmnt{\medskip

\noindent{\bf Remark} In this result and the ones
to follow, I have not spelt
out separately what it means if $a=0$;  but of course this is the case in
which we are starting the process at time $t^*=0$ and value 
$x^*=\tilde\omega(0)$,
just as in the original construction 455A.
}

\proof{{\bf (a)} Of course we can assume in this part of the proof that $a$
is finite.

\medskip

\quad{\bf (i)} Suppose that $\eta\in\ooint{0,1}$ and $\epsilon$, $\delta>0$ are
such that $\nu^{(s,t)}_xB(x,\epsilon)\ge 1-\eta$ whenever
$x\in U$ and $0\le s<t\le s+\delta$.   Then

\Centerline{$\hat\mu_{\tilde\omega a}\{\omega:\omega\in\Omega,\,
\diam\omega[D]\le 4\epsilon\}\ge\Bover{1-2\eta}{1-\eta}$}

\noindent whenever $D\subseteq\coint{a,\infty}$ is a
countable set of diameter at most $\delta$.

\medskip

\Prf\ \grheada\ For finite $D$, I seek to induce on $\#(D)$.
If $\#(D)\le 1$ then of course
$\diam\omega[D]\le 4\epsilon$ for every $\omega$ and we can stop.   So
suppose that $D=\{t_0,\ldots,t_n\}$ where $n\ge 1$ and
$a\le t_0<\ldots<t_n$.    To begin with, I go through the formulae when
$t_0>0$.

For $k\le n$ set

\Centerline{$E_k=\{\omega:\rho(\omega(t_k),\omega(t_0))>2\epsilon$,
$\rho(\omega(t_i),\omega(t_0))\le 2\epsilon$ for $i<k\}$,}

\Centerline{$F_k=\{\omega:\omega\in E_k$,
$\rho(\omega(t_n),\omega(t_k))\le\epsilon\}$,}

\Centerline{$G_k=\{(x_0,\ldots,x_k):\rho(x_k,x_0)>2\epsilon$,
$\rho(x_i,x_0)\le 2\epsilon$ for $i<k\}\subseteq U^{k+1}$.}

\noindent If $1\le k<n$ then

$$\eqalignno{\hat\mu_{\tilde\omega a}F_k
&=\lambda_{\{0,t_0,\ldots,t_k,t_n\}}\{(x,x_0,\ldots,x_k,x_n):
\rho(x_i,x_0)\le 2\epsilon\text{ for }i<k,\cr
&\mskip300mu
  \rho(x_0,x_k)>2\epsilon,\,\rho(x_k,x_n)\le\epsilon\}\cr
\displaycause{defining $\lambda_J$ as the image measure of
$\hat\mu_{\tilde\omega a}$ on $U^J$, as in
455E}
&=\int\ldots\int\chi G_k(x_0,\ldots,x_k)\chi B(x_k,\epsilon)(x_n)
\nu^{(t_k,t_n)}_{\tilde\omega ax_k}(dx_n)
\ldots\nu^{(0,t_0)}_{\tilde\omega(0)}(dx_0)\cr
\displaycause{were $\nu^{(s,t)}_{\tilde\omega ax}$ is defined as in 455Eb}
&=\int\ldots\int\chi G_k(x_0,\ldots,x_k)\chi B(x_k,\epsilon)(x_n)
\nu^{(t_k,t_n)}_{x_k}(dx_n)
\ldots\nu^{(0,t_0)}_{\tilde\omega(0)}(dx_0)\cr
\displaycause{because $a\le t_0<\ldots<t_n$}
&=\int\ldots\int\chi G_k(x_0,\ldots,x_k)
\nu^{(t_k,t_n)}_{x_k}(B(x_k,\epsilon))\nu^{(t_{k-1},t_k)}_{x_{k-1}}(dx_k)
\ldots\nu^{(0,t_0)}_{\tilde\omega(0)}(dx_0)\cr
&\ge\int\ldots\int(1-\eta)\chi G_k(x_0,\ldots,x_k)
\nu^{(t_{k-1},t_k)}_{x_{k-1}}(dx_k)
\ldots\nu^{(0,t_0)}_{\tilde\omega(0)}(dx_0)\cr
\displaycause{because $t_k<t_n\le t_k+\delta$,
so $\nu^{(t_k,t_n)}_xB(x,\epsilon)\ge 1-\eta$ for every $x$}
&=(1-\eta)\lambda_{\{0,t_0,\ldots,t_k\}}
\{(x,x_0,\ldots,x_k,x_n):
\rho(x_i,x_0)\le 2\epsilon\text{ for }i<k,\cr
&\mskip420mu \rho(x_0,x_k)>2\epsilon\}\cr
&=(1-\eta)\hat\mu_{\tilde\omega a} E_k.\cr}$$

\noindent If $k=n$, then of course $F_k=E_k$, so again
$\hat\mu_{\tilde\omega a} F_k\ge(1-\eta)\hat\mu_{\tilde\omega a} E_k$.
Accordingly

$$\eqalign{(1-\eta)\sum_{k=1}^n\hat\mu_{\tilde\omega a} E_k
&\le\sum_{k=1}^n\hat\mu_{\tilde\omega a} F_k
\le\hat\mu_{\tilde\omega a}\{\omega:\rho(\omega(t_n),\omega(t_0))>\epsilon\}\cr
&=\lambda_{\{0,t_0,t_n\}}\{(x,x_0,x_n):\rho(x_0,x_n)>\epsilon\}\cr
&=\int\nu^{(t_0,t_n)}_{x_0}(U\setminus B(x_0,\epsilon))
   \nu^{(0,t_0)}_{\tilde\omega(0)}(dx_0)
\le\eta\cr}$$

\noindent because $t_n-t_0\le\delta$ so
$\nu^{(t_0,t_n)}_x(U\setminus B(x,\epsilon))\le\eta$ for every $x$.

But now we have

$$\eqalign{\hat\mu_{\tilde\omega a}\{\omega:\omega\in\Omega,\,\diam\omega[D]\le 4\epsilon\}
&\ge\hat\mu_{\tilde\omega a}\{\omega:\rho(\omega(t_k),\omega(t_0))\le 2\epsilon
 \text{ for }1\le k\le n\}\cr
&=\hat\mu_{\tilde\omega a}(\Omega\setminus\bigcup_{1\le k\le n}E_k)
\ge 1-\Bover{\eta}{1-\eta}
=\Bover{1-2\eta}{1-\eta}\cr}$$

\noindent as required.

\medskip

\qquad\grheadb\ If $t_0=a=0$ the formulae simplify slightly, but the ideas
are the same.   We have $\omega(0)=\tilde\omega(0)$ for
$\hat\mu_{\tilde\omega 0}$-almost every $\omega$, so

$$\eqalign{\hat\mu_{\tilde\omega 0}F_k
&=\lambda_{0,t_1,\ldots,t_k,t_n}
   \{(\tilde\omega(0),x_1,\ldots,x_k,x_n):
      \rho(x_i,\tilde\omega(0))\le 2\epsilon\text{ for }i<k,\cr
&\mskip300mu
      \rho(\tilde\omega(0),x_k)>2\epsilon,\,\rho(x_k,x_n)\le\epsilon\}\cr
&=\int\ldots\int\chi G_k(\tilde\omega(0),x_1,\ldots,x_k)
   \chi B(x_k,\epsilon)(x_n)\nu^{(t_k,t_n)}_{\tilde\omega ax_k}(dx_n)
   \ldots\nu^{(0,t_1)}_{\tilde\omega(0)}(dx_1)\cr
&\ge(1-\eta)\int\ldots\int\chi G_k(\tilde\omega(0),x_1,\ldots,x_k)
   \nu^{(t_{k-1},t_k)}_{\tilde\omega ax_{k-1}}(dx_k)
   \ldots\nu^{(0,t_1)}_{\tilde\omega(0)}(dx_1)\cr
&=(1-\eta)\hat\mu_{\tilde\omega 0}E_k\cr}$$

\noindent for $1\le k<n$,

$$\eqalign{(1-\eta)\sum_{k=1}^n\hat\mu_{\tilde\omega 0}E_k
&\le\lambda_{\{0,t_n\}}\{(\tilde\omega(0),x_n):
   \rho(\tilde\omega(0),x_n)>\epsilon\}\cr
&=\int\nu^{(t_0,t_n)}_{\tilde\omega(0)}
   (U\setminus B(\tilde\omega(0),\epsilon))
\le\eta,\cr}$$

\noindent and the final calculation is unchanged.

\medskip

\qquad\grheadc\ For countably infinite $D$, let $\sequencen{I_n}$ be a
non-decreasing sequence of finite sets with union $D$;  then
$\sequencen{\{\omega:\omega\in\Omega,\,\diam\omega[I_n]\le 4\epsilon\}}$
is a non-increasing sequence with intersection
$\{\omega:\omega\in\Omega,\,\diam\omega[D]\le 4\epsilon\}$, so the measure
of the limit is the limit of the measures, and is at most
$\Bover{1-2\eta}{1-\eta}$.\ \Qed

\wheader{455G}{4}{2}{2}{36pt}

\quad{\bf (ii)}
For $m\in\Bbb N$, $\epsilon>0$ and $A\subseteq\coint{0,\infty}$
let $G(A,\epsilon,m)$ be

$$\eqalign{\{\omega:\omega\in\Omega,
  \text{ there are }&s_0<s'_0\le s_1<s'_1\le\ldots\le s_m\le s'_m
\text{ in }A\cr
&\text{ such that }\rho(\omega(s'_i),\omega(s_i))>4\epsilon
\text{ for every }i\le m\}.\cr}$$

\noindent Let $\delta>0$ be such that
$\nu_x^{(s,t)}B(x,\epsilon)\ge\bover45$
whenever $x\in U$ and $s<t\le s+\delta$.
Then $\hat\mu_{\tilde\omega a} G(D,\epsilon,m)\le 2^{-m}$
whenever $m\in\Bbb N$ and
$D\subseteq\coint{a,\infty}$ is a countable
set of diameter at most $\delta$.

\medskip

\Prf\ \grheada\ As in (i), first consider finite $D$.
For these, we can induce on $m$.   If $m=0$ then
$G(D,\epsilon,0)=\{\omega:\diam\omega[D]>4\epsilon\}$ so
(i), with $\eta=\bover15$, tells us that
$\hat\mu_{\tilde\omega a}G(D,\epsilon,0)\le\Bover{2\eta}{1-\eta}
=\Bover12$.   For the
inductive step to $m+1$, define $\tau:\Omega\to[0,\infty]$ by setting

$$\eqalign{\tau(\omega)
&=\min\{t:t\in D,\,\omega\in G(D\cap[a,t],\epsilon,m)\}
\text{ if }\omega\in G(D,\epsilon,m),\cr
&=\infty\text{ otherwise}.\cr}$$

\noindent Then $\tau$ takes only finitely many values, all strictly
greater than $a$, and $\{\omega:\tau(\omega)=t\}$ belongs to
$\Tensorhat_{\coint{0,\infty}}\Cal B(U)
=\Tensorhat_{t\in\coint{0,\infty}}\Cal B(U)$
and is determined by coordinates in $[0,t]$ for every $t\ge 0$.
We can therefore apply 455E(b)-(c).

For each $\omega\in\Omega$, define
$\langle\nu^{(s,t)}_{\omega,\tau(\omega),x}\rangle_{s<t,x\in U}$
from $\langle\nu^{(s,t)}_x\rangle_{s<t,x\in U}$ as in 455Eb;  let
$\langle\tilde\nu^{(s,t)}_{\omega,\tau(\omega),x}\rangle_{s<t,x\in U}$
be the family defined in the same way from
$\langle\nu^{(s,t)}_{\tilde\omega ax}\rangle_{s<t,x\in U}$.
Let $\hat\mu_{\omega,\tau(\omega)}$ be defined from
$\omega(0)$ and
$\langle\nu^{(s,t)}_{\omega,\tau(\omega),x}\rangle_{s<t,x\in U}$,
and $\hat\mu'_{\omega,\tau(\omega)}$ from
$\omega(0)$ and
$\langle\tilde\nu^{(s,t)}_{\omega,\tau(\omega),x}\rangle_{s<t,x\in U}$,
again as in 455Eb.   Then 455E(c-i) tells us that
$\family{\omega}{\Omega}{\hat\mu'_{\omega,\tau(\omega)}}$ is a
disintegration of $\hat\mu_{\tilde\omega a}$ over itself.   But now
observe that, for any $\omega\in\Omega$ and $x\in U$,

$$\eqalignno{\nu^{(s,t)}_{\omega,\tau(\omega),x}
&=\nu^{(s,t)}_x=\nu^{(s,t)}_{\tilde\omega ax}
   =\tilde\nu^{(s,t)}_{\omega,\tau(\omega),x}
   \text{ if }\tau(\omega)<s<t,\cr
\displaycause{because $a<\tau(\omega)$}
&=\nu^{(\tau(\omega),t)}_{\omega(\tau(\omega))}
  =\nu^{(\tau(\omega),t)}_{\tilde\omega,a,\omega(\tau(\omega))}
  =\tilde\nu^{(s,t)}_{\omega,\tau(\omega),x}
  \text{ if }s\le\tau(\omega)<t,\cr
&=\delta^{(t)}_{\omega(t)}
   =\tilde\nu^{(s,t)}_{\omega,\tau(\omega),x}
  \text{ if }s<t\le\tau(\omega),\cr}$$

\noindent so
$\hat\mu_{\omega,\tau(\omega)}=\hat\mu'_{\omega,\tau(\omega)}$.
Accordingly
$\family{\omega}{\Omega}{\hat\mu_{\omega,\tau(\omega)}}$ is a
disintegration of $\hat\mu_{\tilde\omega a}$ over itself.

Now, for $\omega\in\Omega$, consider

\Centerline{$H_{\omega}=\{\omega':\omega'\in G(D,\epsilon,m+1)$,
  $\omega'\restr D\cap[0,\tau(\omega)]
  =\omega\restr D\cap[0,\tau(\omega)]\}$.}

\noindent If $\omega\notin G(D,\epsilon,m)$ then $\tau(\omega)=\infty$
and $H_{\omega}=\emptyset$, because
$G(D,\epsilon,m+1)\subseteq G(D,\epsilon,m)$ are determined by coordinates
in $D$.   If
$\omega\in G(D,\epsilon,m)$ and $\tau(\omega)=b$, then

\Centerline{$H_{\omega}
=\{\omega':\omega'\restr D\cap[0,b]=\omega\restr D\cap[0,b]
  \text{ and }\diam(\omega'[D\cap\coint{b,\infty}\,])>4\epsilon\}$,}

\noindent so that $\hat\mu_{\omega,\tau(\omega)}H_{\omega}\le\bover12$ by
(i), again with $\eta=\bover15$.

So

$$\eqalignno{\hat\mu_{\tilde\omega a} G(D,\epsilon,m+1)
&=\int\hat\mu_{\omega,\tau(\omega)}G(D,\epsilon,m+1)
   \hat\mu_{\tilde\omega a}(d\omega)
=\int\hat\mu_{\omega,\tau(\omega)}H_{\omega}
   \hat\mu_{\tilde\omega a}(d\omega)\cr
\displaycause{using 455E(b-ii)}
&=\int_{G(D,\epsilon,m)}\hat\mu_{\omega,\tau(\omega)}H_{\omega}
   \hat\mu_{\tilde\omega a}(d\omega)
\le\Bover12\hat\mu_{\tilde\omega a}G(D,\epsilon,m)
\le 2^{-m-1}\cr}$$

\noindent by the inductive hypothesis.   Thus the induction proceeds.

\medskip

\qquad\grheadb\ Now, for countably infinite $D$, again express $D$ as the
union of a non-decreasing sequence $\sequencen{I_n}$ of finite sets, and
observe that $\sequencen{G(I_n,\epsilon,m)}$ is a non-decreasing sequence
with union $G(D,\epsilon,m)$;  so

\Centerline{$\hat\mu_{\tilde\omega a} G(D,\epsilon,m)
=\lim_{n\to\infty}\hat\mu_{\tilde\omega a} G(I_n,\epsilon,m)\le 2^{-m}$}

\noindent for every $m\in\Bbb N$.\ \Qed

\medskip

\quad{\bf (iii)} For $n\in\Bbb N$, let $\delta_n>0$ be such that
$\nu_x^{(s,t)}B(x,2^{-n})\ge\bover45$ whenever $x\in U$ and
$s<t\le s+\delta_n$.   Consider the set

\Centerline{$E=\bigcup_{n,k\in\Bbb N}\bigcap_{m\in\Bbb N}
G(\Bbb Q\cap[a+k\delta_n,a+(k+1)\delta_n],2^{-n+2},m)$.}

\noindent Then $\hat\mu_{\tilde\omega a}E=0$.   Suppose that
$\omega\in\Omega\setminus E$ and $t>a$.
\Quer\ If $\lim_{q\in\Bbb Q,q\uparrow t}\omega(q)$ is undefined,
then (because $U$ is
complete under $\rho$) there must be an $n\in\Bbb N$ and a strictly
increasing sequence $\sequence{i}{q_i}$ in $\Bbb Q$, with supremum
$t$, such that $\rho(\omega(q_{i+1}),\omega(q_i))\ge 2^{-n+2}$ for every
$i\in\Bbb N$.   Let $k\in\Bbb N$ be such
that $t\in\ocint{a+k\delta_n,a+(k+1)\delta_n}$;  let $l\in\Bbb N$ be such
that $q_l\ge a+k\delta_n$.   Then, for every $m\in\Bbb N$,
$(q_l,q_{l+1},q_{l+1},q_{l+2},\ldots,q_{m-1},q_m)$ witnesses that
$\omega\in G(\Bbb Q\cap[a+k\delta_n,a+(k+1)\delta_n],2^{-n+2},m)$;
which is
impossible.\ \BanG\   So $\lim_{q\in\Bbb Q,q\uparrow t}\omega(q)$ is
defined;  similarly, $\lim_{q\in\Bbb Q,q\downarrow t}\omega(q)$ is defined.

As $E$ is $\hat\mu_{\tilde\omega a}$-negligible, this proves (a).

\medskip

{\bf (b)(i)} This is actually easier.
Consider part (a-i) of the proof above.   Given $n\in\Bbb N$, we
see that there is a $\delta_n>0$ such that
$\hat\mu_{\tilde\omega a}\{\omega:\diam\omega[D]\le 2^{-n}\}
\ge 1-2^{-n}$ whenever $D\subseteq\coint{a,\infty}$ is a countable set of
diameter at most $\delta_n$.   Set

\Centerline{$D_n=\{t\}\cup(\Bbb Q\cap[t,t+\delta_n])$,
\quad$E_n=\{\omega:\diam\omega[D_n]\le 2^{-n}\}$}

\noindent for each $n\in\Bbb N$, and
$E=\bigcup_{n\in\Bbb N}\bigcap_{m\ge n}E_m$.   Then
$\hat\mu_{\tilde\omega a}E=1$, and for $\omega\in E$ we have an
$n\in\Bbb N$ such that
$\rho(\omega(t),\omega(q))\le 2^{-m}$ whenever $m\ge n$ and
$q\in\Bbb Q\cap[t,t+\delta_m]$, so that
$\omega(t)=\lim_{q\in\Bbb Q,q\downarrow t}\omega(q)$.

\medskip

\quad{\bf (ii)} If $t>a$, the same argument applies on the other side of
$t$, taking $D_n=\{t\}\cup(\Bbb Q\cap[\max(a,t-\delta_n),t])$, to see that
$\omega(t)=\lim_{q\in\Bbb Q,q\uparrow t}\omega(q)$ for
$\hat\mu_{\tilde\omega a}$-almost every $\omega$.

\medskip

{\bf (c)}\grheada\ Suppose that $E\subseteq\Omega$ and
$\hat\mu_{\tilde\omega a}E>0$.   Then there is an $\omega^*\in E$ such that

\Centerline{$\omega^*(t)=\tilde\omega(t)$ for every $t\le a$,}

\Centerline{$\omega^*(t)=\lim_{s\downarrow t}\omega^*(s)$ for every
$t\ge a$,}

\Centerline{$\lim_{s\uparrow t}\omega^*(t)$ is defined for every $t>a$.}

\noindent\Prf\ Let $E'\in\Tensorhat_{\coint{0,\infty}}\Cal B(U)$ be such
that $E'\subseteq E$ and $\hat\mu_{\tilde\omega a}E'>0$.   Let
$D\subseteq\coint{0,\infty}$ be a countable set such that $E'$ is
determined by coordinates in $D$;  we can suppose that $a\in D$ if $a$ is
finite.   Let $F$ be the set of those $\omega\in\Omega$ such that

\Centerline{$\lim_{q\in\Bbb Q,q\downarrow t}\omega(q)$ and
$\lim_{q\in\Bbb Q,q\uparrow t}\omega(q)$ are defined in $U$ for every
$t>a$,}

\Centerline{$\omega(t)=\lim_{q\in\Bbb Q,q\downarrow t}\omega(t)$
for every $t\in D\cap\coint{a,\infty}$,}

\Centerline{$\omega(t)=\tilde\omega(t)$ for every
$t\in D\cap[0,a]$.}

\noindent Then (a) and (b), with 455E(b-ii), tell us that $F$ is
$\hat\mu_{\tilde\omega a}$-conegligible.   So there is an
$\omega\in E\cap F$.    Define $\omega^*\in\Omega$ by setting

$$\eqalign{\omega^*(t)&=\tilde\omega(t)\text{ if }t\le a,\cr
&=\lim_{q\in\Bbb Q,q\downarrow t}\omega(t)\text{ if }t\ge a;\cr}$$

\noindent note that the definitions of $\omega^*(a)$ are consistent if $a$
is finite, and that $\omega^*\restr D=\omega\restr D$, so that
$\omega^*\in E'\subseteq E$.

If $t\le a$, then of course $\omega^*(t)=\tilde\omega(t)$.   If $t\ge a$
and $\epsilon>0$, there is a $\delta>0$ such that
$\rho(\omega(q),\omega^*(t))\le\epsilon$ whenever
$q\in\Bbb Q\cap\ocint{t,t+\delta}$;  in which case
$\rho(\omega^*(s),\omega^*(t))\le\epsilon$ whenever
$s\in\coint{t,t+\delta}$;  as $\epsilon$ is arbitrary,
$\omega^*(t)=\lim_{s\downarrow t}\omega^*(s)$.
If $t>a$ and $\epsilon>0$, there is a $\delta>0$ such that
$\rho(\omega(q),\omega(q'))\le\epsilon$ whenever
$q\in\Bbb Q\cap\coint{t-\delta,t}$;  in which case
$\rho(\omega^*(s),\omega^*(s'))\le\epsilon$ whenever
$s\in\coint{t-\delta,t}$;  as $\epsilon$ is arbitrary and $U$ is complete,
$\lim_{s\uparrow t}\omega^*(s)$ is defined in $U$.   So we have an
appropriate $\omega^*$.\ \Qed

\medskip

\quad\grheadb\ Suppose, in ($\alpha$), that $\tilde\omega\in\Clll$.
Then $\omega^*\in\Clll$.   \Prf\

\Centerline{$\lim_{s\uparrow t}\omega^*(s)
=\lim_{s\uparrow t}\tilde\omega(s)$ is defined whenever $0<t\le a$,}

\Centerline{$\lim_{s\downarrow t}\omega^*(s)
=\lim_{s\downarrow t}\tilde\omega(s)$ is defined whenever $0\le t<a$,}

\Centerline{if $a>0$,
$\lim_{s\downarrow 0}\omega^*(s)
=\lim_{s\downarrow 0}\tilde\omega(s)=\tilde\omega(0)=\omega^*(0)$,}

\inset{\noindent
if $0<t<a$, then $\omega^*(t)=\tilde\omega(t)$ is equal to at least one of
$\lim_{s\uparrow t}\omega^*(s)=\lim_{s\uparrow t}\tilde\omega(s)$,
$\lim_{s\downarrow t}\omega^*(s)=\lim_{s\downarrow t}\tilde\omega(s)$.}

\noindent Since we already know that

\Centerline{$\omega^*(t)=\lim_{s\downarrow t}\omega^*(s)$ for every
$t\ge a$,}

\Centerline{$\lim_{s\uparrow t}\omega^*(t)$ is defined for every $t>a$,}

\noindent $\omega^*$ is \callal.\ \Qed

As $E$ is arbitrary,
it follows that if $\tilde\omega\in\Clll$ then $\Clll$ meets every
non-negligible
$\hat\mu_{\tilde\omega a}$-measurable set, so that
$\hat\mu_{\tilde\omega a}^*\Clll=1$, as required by (i).

\medskip

\quad\grheadc\ Similarly, if $\tilde\omega\in\Cdlg$, then any $\omega^*$
with the properties described in ($\alpha$) also belongs to $\Cdlg$.
\Prf\ This time, we have

\Centerline{if $0\le t<a$, then $\omega^*(t)=\tilde\omega(t)
=\lim_{s\downarrow t}\omega^*(s)=\lim_{s\downarrow t}\tilde\omega(s)$,}

\noindent which with the other properties listed is enough to ensure that
$\omega^*\in\Cdlg$.\ \QeD\  Since $E$ is arbitrary,
$\hat\mu_{\tilde\omega a}^*\Cdlg=1$.

This completes the proof of part (c).
}%end of proof of 455G

\leader{455H}{Corollary} Let $(U,\rho)$ be a complete metric
space and $\langle\nu^{(s,t)}_x\rangle_{0\le s<t,x\in U}$ a family of
Radon probability measures on $U$, uniformly time-continuous on the right,
such that $\family{y}{U}{\nu_y^{(t,u)}}$ is a disintegration of
$\nu_x^{(s,u)}$ over $\nu_x^{(s,t)}$
whenever $0\le s<t<u$ and $x\in U$.   Let $\Clll(U)$ be the set of
\callal\ functions from
$\coint{0,\infty}$ to $U$.   Suppose that $\tilde\omega\in\Clll(U)$, and
$a\in[0,\infty]$;  let $\hat\mu_{\tilde\omega a}$ be the completed
probability measure on
$\Omega=U^{\coint{0,\infty}}$ defined from
$\tilde\omega$, $a$ and $\langle\nu^{(s,t)}_x\rangle_{0\le s<t,x\in U}$ as in
455Eb.   Then $\hat\mu_{\tilde\omega a}$ has a unique
extension to a Radon measure $\tilde\mu_{\tilde\omega a}$ on $\Omega$, and
$\tilde\mu_{\tilde\omega a}\Clll(U)=1$.

\proof{{\bf (a)} In the language of 455E(b-i),
$\nu^{(0,t)}_{\tilde\omega ax}$ is a Radon measure
whenever $t>0$ and $x\in U$, so
the image measure defined from $\hat\mu_{\tilde\omega a}$ and the map
$\omega\mapsto\omega(t)$ is always a Radon measure on $U$, and there
there is a $\sigma$-compact set $H_t\subseteq U$
such that $\omega(t)\in H_t$ for
$\hat\mu_{\tilde\omega a}$-almost every $\omega$.
Set $U_0=\overline{\bigcup_{q\in\Bbb Q\cap\coint{0,\infty}}H_q}$;
then $U_0$ is separable and $\hat\mu_{\tilde\omega a}E=1$, where
$E=\{\omega:\omega(q)\in U_0$ for every $q\in\Bbb Q\cap\coint{0,\infty}\}$.
By 455G(c-i), $E\cap\Clll(U)$ has full outer measure;
and if $\omega\in E\cap\Clll(U)$, then
$\omega(t)\in U_0$ for every $t\ge 0$.

\medskip

{\bf (b)}
Thus $E\cap\Clll(U)$ is included in $\Clll(U_0)$, the set of \callal\
functions from $\coint{0,\infty}$ to the Polish space $U_0$.
So $\hat\mu_{\tilde\omega a}^*\Clll(U_0)=1$.
Let $\hat\mu_C$ be the subspace probability measure on $\Clll(U_0)$.

Since $\hat\mu_{\tilde\omega a}$ is inner regular with respect to
$\Tensorhat_{\coint{0,\infty}}\Cal B(U)$,
$\hat\mu_C$ is inner regular with respect to the $\sigma$-algebra
$\Sigma=\{E\cap\Clll(U_0):E\in\Tensorhat_{\coint{0,\infty}}\Cal B(U)\}$
(412Ob).   But $\Sigma$ is just the $\sigma$-algebra generated by the maps
$\omega\mapsto\omega(t):\Clll(U_0)\to U_0$ for $t\ge 0$, which is the Baire
$\sigma$-algebra of $\Clll(U_0)$ (4A3Nd).  Accordingly
$\hat\mu_C\restr\Sigma$ is a Baire measure and is inner regular with
respect to the closed sets (412D);  it follows that its completion
$\hat\mu_C$ is inner regular with respect to the closed sets
(412Ab).

At this point, recall that $\Clll(U_0)$ is K-analytic (438Sc).
So $\hat\mu_C$ has an extension to a Radon measure $\tilde\mu_C$
on $\Clll(U_0)$ (432D).   Now $\tilde\mu_C$ has an extension to a Radon
probability measure $\tilde\mu_{\tilde\omega a}$ on $\Omega$ such that
$\tilde\mu_{\tilde\omega a}\Clll(U)=\tilde\mu_{\tilde\omega a}\Clll(U_0)=1$.
And if $\hat\mu_{\tilde\omega a}$ measures $E$, then

\Centerline{$\tilde\mu_{\tilde\omega a} E
=\tilde\mu_C(E\cap\Clll(U_0))
=\hat\mu_C(E\cap\Clll(U_0))
=\hat\mu_{\tilde\omega a}^*(E\cap\Clll(U_0))
=\hat\mu_{\tilde\omega a}E$,}

\noindent so $\tilde\mu_{\tilde\omega a}$ extends
$\hat\mu_{\tilde\omega a}$.

\medskip

{\bf (c)} As for uniqueness, observe that
$\dom\hat\mu_{\tilde\omega a}$ includes a base for the topology
of $\Omega$, so by 415H there can be at most one Radon measure extending
$\hat\mu_{\tilde\omega a}$.
}%end of proof of 455H

\leader{455I}{}\cmmnt{ In fact we can go farther;  the Radon measure
$\tilde\mu$ is much more closely related to the completed Baire measure
it extends than one might expect.

\medskip

\noindent}{\bf Lemma} Let $(U,\rho)$ be a complete
separable metric space
and $\langle\nu^{(s,t)}_x\rangle_{0\le s<t,x\in U}$ a family of
Radon probability measures on $U$, uniformly time-continuous on the right,
such that $\family{y}{U}{\nu_y^{(t,u)}}$ is a disintegration of
$\nu_x^{(s,u)}$ over $\nu_x^{(s,t)}$ whenever $0\le s<t<u$ and $x\in U$.
Suppose that $\tilde\omega\in\Omega$, and
$a\in[0,\infty]$;   let $\hat\mu_{\tilde\omega a}$ be the completed
probability measure on $\Omega=U^{\coint{0,\infty}}$ defined from
$\langle\nu^{(s,t)}_x\rangle_{0\le s<t,x\in U}$, $\tilde\omega$ and $a$
as in 455Eb.

(a) Suppose that $0\le q_0<q_1$ and $\epsilon>0$.
For $\omega\in\Omega$, I will say that
$\ooint{q_0,q_1}$ is an {\bf$\epsilon$-shift interval}
of $\omega$ with {\bf$(q_0,q_1,\epsilon)$-shift point} $t$ if
$\rho(\omega(q_0),\omega(q_1))>2\epsilon$ and

$$\eqalign{t
&=\sup\{q:q\in\Bbb Q\cap\ooint{q_0,q_1},\,
\rho(\omega(q),\omega(q_0))\le\epsilon\}\cr
&=\inf\{q:q\in\Bbb Q\cap\ooint{q_0,q_1},\,
\rho(\omega(q),\omega(q_1))\le\epsilon\}.\cr}$$

\noindent Let $E$ be the set of such $\omega$.

\quad(i) $E\in\CalBa(\Omega)=\Tensorhat_{\coint{0,\infty}}\Cal B(U)$.

\quad(ii) The function
$f:E\to\ooint{q_0,q_1}$ which takes each $\omega\in E$ to its
$(q_0,q_1,\epsilon)$-shift point is $\CalBa(\Omega)$-measurable.

\quad(iii) If $q_0\ge a$,
the set $\{\omega:\omega\in E$, $f(\omega)=t\}$ is
$\hat\mu_{\tilde\omega a}$-negligible for every $t\in\ooint{q_0,q_1}$.

\quad(iv) If $q_0$, $q_1\in\Bbb Q$, $\omega\in E$, $\omega'\in\Omega$ and
$\omega'\restr\Bbb Q=\omega\restr\Bbb Q$, then $\omega'\in E$ and
$f(\omega')=f(\omega)$.

(b) Suppose that $\langle q_i\rangle_{i\le n}$,
$\langle q'_i\rangle_{i\le n}$, $\langle\le_i\rangle_{i\le n}$,
$\epsilon>0$, $E\in\CalBa(\Omega)$
and $\langle f_i\rangle_{i\le n}$ are such that, for every $i\le n$,

\Centerline{$q_i$, $q'_i\in\Bbb Q$,
\quad$q_i<q'_i$,
\quad$\le_i$ is either $\le$ or $\ge$,}

\Centerline{$\ooint{q_i,q'_i}$ is an $\epsilon$-shift interval of $\omega$
with
$(q_i,q'_i,\epsilon)$-shift point $f_i(\omega)$, for every $\omega\in E$,}

\noindent and also

\Centerline{$a\le q_0$,
\quad$q'_i\le q_{i+1}$ for every $i<n$,}

\Centerline{whenever $\omega$, $\omega'\in E$ there is an $i\le n$ such
that $f_i(\omega')\le_if_i(\omega)$.}

\noindent Then $E$ is $\hat\mu_{\tilde\omega a}$-negligible.

(c) Suppose that $\langle q_i\rangle_{i\le n}$,
$\langle q'_i\rangle_{i\le n}$, $\langle\le_i\rangle_{i\le n}$,
$\epsilon>0$, $E\in\CalBa(\Omega)$
and $\langle f_i\rangle_{i\le n}$ are such that, for every $i\le n$,

\Centerline{$q_i$, $q'_i\in\Bbb Q$,
\quad$q_i<q'_i$,
\quad$\le_i$ is either $\le$ or $\ge$,}

\Centerline{$\ooint{q_i,q'_i}$
is an $\epsilon$-shift interval of $\omega$ with
$(q_i,q'_i,\epsilon)$-shift point $f_i(\omega)$, for every $\omega\in E$,}

\noindent and also

\Centerline{$a\le q_0$,
\quad$q'_i\le q_{i+1}$ for every $i<n$.}

\noindent Then for $\hat\mu_{\tilde\omega a}$-almost every $\omega\in E$
there is an $\omega'\in E$ such that $f_i(\omega')<_if_i(\omega)$ for every
$i\le n$.

\proof{{\bf (a)(i)} Note that by 4A3Na we can identify
$\Tensorhat_{\coint{0,\infty}}\Cal B(U)$ with the Baire $\sigma$-algebra
$\CalBa(\Omega)$ of $\Omega$.   If $s$, $t\ge 0$, then
$\omega\mapsto(\omega(s),\omega(t)):\Omega\to U^2$ is
$\CalBa(\Omega)$-measurable, by 418Bb;  so
$\omega\mapsto\rho(\omega(s),\omega(t))$ is $\CalBa(\Omega)$-measurable.
For $\omega\in\Omega$, $\omega\in E$ iff
($\alpha$) $\rho(\omega(q_0),\omega(q_1))>2\epsilon$
($\beta$) whenever $q$, $q'\in\Bbb Q\cap\ooint{q_0,q_1}$,
$\rho(\omega(q),\omega(q_0))\le\epsilon$ and
$\rho(\omega(q'),\omega(q_1))\le\epsilon$ then $q\le q'$ ($\gamma$) for
every $n\in\Bbb N$ there are $q$, $q'\in\Bbb Q\cap\ooint{q_0,q_1}$
such that $\rho(\omega(q),\omega(q_0))\le\epsilon$,
$\rho(\omega(q'),\omega(q'_0))\le\epsilon$ and $q'\le q+2^{-n}$.
So $E\in\CalBa(\Omega)$.

\medskip

\quad{\bf (ii)} Now, for any $t$,

\Centerline{$\{\omega:\omega\in E$, $f(\omega)>t\}
=\bigcup_{q\in\Bbb Q\cap\ooint{t,q_1}}\{\omega:\omega\in E$,
   $\rho(\omega(q),\omega(q_0))\le\epsilon\}$}

\noindent belongs to $\CalBa(\Omega)$, so $f$ is
$\CalBa(\Omega)$-measurable.

\medskip

\quad{\bf (iii)} Consider the set $E'$ of those $\omega\in\Omega$ such that

\Centerline{$\lim_{q\in\Bbb Q,q\uparrow t}\omega(q)
  =\omega(t)=\lim_{q\in\Bbb Q,q\downarrow t}\omega(q)$.}

\noindent If $\omega\in E\cap E'$,
at least one of $\rho(\omega(t),\omega(q_0))$,
$\rho(\omega(t),\omega(q_1))$ must be greater than $\epsilon$;  in the
first case, $t$ cannot be
$\sup\{q:q\in\Bbb Q\cap\ooint{q_0,q_1}$,
$\rho(\omega(q),\omega(q_0))\le\epsilon\}$;
in the second case, $t$ cannot be
$\inf\{q:q\in\Bbb Q\cap\ooint{q_0,q_1}$,
$\rho(\omega(q),\omega(q_1))\le\epsilon\}$;  so in either case $f(\omega)$
cannot be equal to $t$.   Now $\hat\mu_{\tilde\omega a}E'=1$, by 455Gb, so
$\{\omega:\omega\in E$, $f(\omega)=t\}\subseteq\Omega\setminus E'$ is
$\hat\mu_{\tilde\omega a}$-negligible.

\medskip

\quad{\bf (iv)} Immediate from the definitions.

\medskip

{\bf (b)} Induce on $n$.   Of course we need consider only the case
$E\ne\emptyset$.

\medskip

\quad{\bf (i)} If $n=0$, $f_0$ must be constant on $E$, so
$E$ must be negligible, by (a-iii).

\medskip

\quad{\bf (ii)} For the inductive step to $n\ge 1$, set
$E_t=\{\omega:\omega\in E$, $f_0(\omega)=t\}$ for
$t\in\ooint{q_0,q'_0}$;  by (a-ii),
$E_t\in\CalBa(\Omega)$.

\medskip

\qquad\grheada\ There is a countable
set $J\subseteq\ooint{q_0,q'_0}$ such that whenever
$t\in[q_0,q'_0]\setminus J$ and $\omega$, $\omega'\in E_t$ then
there is an $i$ such that $1\le i\le n$ and $f_i(\omega)\le_if_i(\omega')$.
\Prf\ Let $\Cal W$ be a countable base for the topology of
$\prod_{1\le i\le n}\ooint{q_i,q'_i}$.   For $\omega\in E$ set
$g(\omega)=\langle f_i(\omega)\rangle_{1\le i\le n}$;  note that
$g:E\to\prod_{1\le i\le n}\ooint{q_i,q'_i}$ is
$\CalBa(\Omega)$-measurable.   For $W\in\Cal W$, set

\Centerline{$A_W=\{t:t\in\ooint{q_0,q'_0}$
and there is an $\omega\in E_t$ such that $g(\omega)\in W\}$.}

\noindent Set

\Centerline{$J
=\{t:t\in\ooint{q_0,q'_0}$, $t$ is either $\inf A_W$ or $\sup A_W$
for some $W\in\Cal W\}$.}

\noindent Then $J$ is a countable subset of $\ooint{q_0,q'_0}$.
\Quer\ Suppose that $t\in\ooint{q_0,q'_0}\setminus J$ and
$\omega$, $\omega'\in E_t$ are such that $f_i(\omega')<_if_i(\omega)$ for
$1\le i\le n$.   Let $W\in\Cal W$ be such that
$g(\omega')\in W$ and $z(i)<_if_i(\omega)$
whenever $1\le i\le n$ and $z\in W$.   Then $\omega'$ witnesses that
$t\in A_W$;  since $t$ is neither the greatest nor the least element of
$A_W$, there is a $t'\in A_W$ such that $t'<_0t$;
take $\omega''\in E_{t'}$ such that $g(\omega'')\in W$.   Then

\Centerline{$f_0(\omega'')=t'<_0t=f_0(\omega)$,}

\Centerline{$f_i(\omega'')=g(\omega'')(i)<_if_i(\omega)$
for $1\le i\le n$,}

\noindent which is impossible.\ \BanG\  Thus
$J$ has the required property.\ \Qed

\medskip

\qquad\grheadb\  Now consider the family
$\family{\omega}{\Omega}{\hat\mu_{\omega q_1}}$.   Because $q_1>a$,
this is a disintegration of $\hat\mu_{\tilde\omega a}$ over itself.
\Prf\ As in part (a-ii-$\alpha$) of the proof of 455G,
we can think of each $\hat\mu_{\omega q_1}$ as defined either from
$\langle\nu^{(s,t)}_x\rangle_{0\le s<t,x\in U}$ or from
$\langle\nu^{(s,t)}_{\tilde\omega ax}\rangle_{a\le s<t,x\in U}$;
and in the latter form we can apply 455E(c-i).\ \Qed

Consider $\mu_{\omega q_1}(E)$ for $\omega\in\Omega$.   This time, note
that
$\{\omega':\omega'\restr[0,q_1]\cap\Bbb Q=\omega\restr[0,q_1]\cap\Bbb Q\}$
is $\mu_{\omega q_1}$-conegligible.
In particular, $\mu_{\omega q_1}(E)=0$ unless $\ooint{q_0,q'_0}$ is an
$\epsilon$-shift interval of $\omega$.   Next,

\Centerline{$\{\omega:\ooint{q_0,q'_0}$
is an $\epsilon$-shift interval of $\omega$ with
$(q_0,q'_0,\epsilon)$-shift point in $J\}$}

\noindent is $\hat\mu_{\tilde\omega a}$-negligible, by (a-iii) again.
Finally, suppose that
$\omega\in\Omega$ is such that $\ooint{q_0,q'_0}$ is an $\epsilon$-shift
interval of $\omega$ with $(q_0,q'_0,\epsilon)$-shift
point $t\in\ooint{q_0,q'_0}\setminus J$.   Then

$$\eqalign{\mu_{\omega q_1}E
&=\mu_{\omega q_1}\{\omega':\omega'\in E,\,
   \omega'\restr\Bbb Q\cap[0,q_1]=\omega\restr\Bbb Q\cap[0,q_1]\}\cr
&=\mu_{\omega q_1}\{\omega':\omega'\in E_t\}.
\cr}$$

\noindent But the choice of $J$ in ($\alpha$) ensured that
$E_t$ would be a set of the same type as $E$, one level down, determined by
intervals starting from $q_1$,
so that $\hat\mu_{\omega q_1}E_t=0$, by the inductive hypothesis applied
to $\omega$ and $q_1$ in place of $\tilde\omega$ and $a$.

\medskip

\qquad\grheadc\ So we see that $\hat\mu_{\omega q_1}E=0$ for
$\hat\mu_{\tilde\omega a}$-almost every
$\omega$, and $\hat\mu_{\tilde\omega a}E=0$.   Thus the induction proceeds.

\medskip

{\bf (c)} Let $F$ be the set of those $\omega\in E$ for which there
is no $\omega'\in E$ such that $f_i(\omega')<_if_i(\omega)$
for every $i\le n$.   Then $F\in\CalBa(\Omega)$.
\Prf\ For each $\omega\in E$ set
$f(\omega)=\langle f_i(\omega)\rangle_{i\le n}$ and

\Centerline{$W_{\omega}
=\{z:z\in\prod_{i\le n}\ooint{q_i,q'_i}$, $f_i(\omega)<_iz(i)$
for every $i\le n\}$,}

\noindent so that $W_{\omega}$ is open in
$\prod_{i\le n}\ooint{q_i,q'_i}$.   Set
$W=\bigcup_{\omega\in E}W_{\omega}$.   Then $W$ is open and
$F=\{\omega:\omega\in E$, $f(\omega)\notin W\}$ belongs to
$\CalBa(\Omega)$.\ \Qed

If $\omega$, $\omega'\in F$ then there is surely some $i\le n$ such that
$f_i(\omega')\le_if_i(\omega)$.   By (b), $\hat\mu_{\tilde\omega a}F=0$.
}%end of proof of 455I

\leader{455J}{Theorem} Let $(U,\rho)$ be a complete separable metric space
and $\langle\nu^{(s,t)}_x\rangle_{0\le s<t,x\in U}$ a family of
Radon probability measures on $U$, uniformly time-continuous on the right,
such that $\family{y}{U}{\nu_y^{(t,u)}}$ is a disintegration of
$\nu_x^{(s,u)}$ over $\nu_x^{(s,t)}$ whenever $0\le s<t<u$ and $x\in U$.
Write $\Clll$ for the set of \callal\ functions from
$\coint{0,\infty}$ to $U$.
Suppose that $\tilde\omega\in\Clll$, and
$a\in[0,\infty]$;   let $\hat\mu_{\tilde\omega a}$ be the completed
probability measure on $\Omega=U^{\coint{0,\infty}}$ defined from
$\langle\nu^{(s,t)}_x\rangle_{0\le s<t,x\in U}$, $\tilde\omega$ and $a$
as in 455Eb, and $\tilde\mu_{\tilde\omega a}$ its extension to a Radon
measure on $\Omega$, as in 455H.   Then $\tilde\mu_{\tilde\omega a}$ is
inner regular with
respect to sets of the form $F\cap\Clll$ where $F\subseteq\Omega$ is a zero
set.

\proof{{\bf (a)} As in 455I, $\Tensorhat_{\coint{0,\infty}}\Cal B(U)$
is the Baire $\sigma$-algebra of $\Omega$.
Let $D$ be
$((\{a\}\cup\Bbb Q)\cap\coint{0,\infty})
\cup\{t:t\ge 0$, $\tilde\omega$ is not continuous at $t\}$;
then $D$ is countable (438S(a-i)).   Let $E^*$ be
$\{\omega:\omega\in\Omega$,
$\omega\restr D\cap[0,a]=\tilde\omega\restr D\cap[0,a]\}$;  then
$E^*$ is $\hat\mu_{\tilde\omega a}$-conegligible, by 455E(b-ii) once again.

\medskip

{\bf (b)} Let $\Cal G$ be a countable base for the topology of
$U$.   Let $\Cal W$ be the family of open subsets of $\Omega$ of the form
$\{\omega:\omega(q)\in G_q$ for every $q\in J\}$ where $J\subseteq D$
is finite and $G_q\in\Cal G$ for every $q\in J$.
Let $\Theta$ be the set of all strings

\Centerline{$\theta=(q_0,q'_0,\ldots,q_n,q'_n,
\le_0,\ldots,\le_n,k,W)$}

\noindent such that

\Centerline{$q_0,\ldots,q'_n\in\Bbb Q$,
\quad$a\le q_0<q'_0\le q_1<q'_1\le\ldots\le q_n<q'_n$,}

\Centerline{for each $i\le n$, $\le_i$ is either $\le$ or $\ge$,}

\Centerline{$k\in\Bbb N$,
\quad$W\in\Cal W$;}

\noindent then $\Theta$ is countable.

\medskip

{\bf (c)} Let $K\subseteq E^*\cap\Clll$ be compact.
Set $L=\pi_D^{-1}[\pi_D[K]]$,
where $\pi_D(\omega)=\omega\restr D$ for $\omega\in\Omega$;
then $L$ is a Baire subset of $\Omega$, because $\pi_D[K]$ is a
compact subset of the metrizable space $U^D$.

\medskip

\quad{\bf (i)} For

\Centerline{$\theta=(q_0,q'_0,\ldots,q_n,q'_n,
\le_0,\ldots,\le_n,k,W)\in\Theta$}

\noindent let $E_{\theta}$ be the set of those
$\omega\in L\cap W$ such that, for each $i\le n$,
$\ooint{q_i,q'_i}$ is a $2^{-k}$-shift interval of $\omega$
(definition:  455Ia).   For $\omega\in E_{\theta}$ and
$i\le n$ let $f_i(\theta,\omega)$ be the $(q_i,q'_i,2^{-k})$-shift
point of $\omega$.   By
455Ia, $E_{\theta}$ is a Baire subset of $\Omega$ and
$\omega\mapsto f_i(\theta,\omega)$ is Baire measurable.   Let
$F_{\theta}$ be the set of those $\omega\in E_{\theta}$ such that
there is no $\omega'\in E_{\theta}$ with
$f_i(\theta,\omega')<_if_i(\theta,\omega)$ for every $i\le n$;  by
455Ic, $F_{\theta}$ is $\hat\mu_{\tilde\omega a}$-negligible.
So $F^*=\bigcup_{\theta\in\Theta}F_{\theta}$ is
$\hat\mu_{\tilde\omega a}$-negligible.

\medskip

\quad{\bf (ii)} Suppose that $\omega\in K\setminus F^*$.
Let $A$ be the set of points in $\ooint{a,\infty}$
at which $\omega$ is discontinuous.   If $J\subseteq A$ is finite and
$\epsilon_t\in\{-1,1\}$ for each $t\in J$,
there is an $\omega'\in K$ such that
$\omega'\restr D=\omega\restr D$ and
$\omega'$ is continuous on the right
at every point $t$ of $J$ such that $\epsilon_t=1$, while
$\omega'$ is continuous on the left
at every point $t$ of $J$ such that $\epsilon_t=-1$.
\Prf\ This is trivial if
$J$ is empty.   Otherwise, enumerate $J$ in ascending order as
$t_0<t_1<\ldots<t_n$.   Set $x_i=\lim_{t\uparrow t_i}\omega(t)$,
$y_i=\lim_{t\downarrow t_i}\omega(t)$;  because $\omega\in\Clll$ these are
defined, and because $\omega$ is not continuous at $t$ they are different.

Let $k\in\Bbb N$ be such that $\rho(x_i,y_i)>2^{-k+1}$ for each
$i\le n$.   For $i\le n$, let
let $q_i$, $q'_i\in\Bbb Q$ be such that
$q_i<t_i<q'_i$, $\rho(\omega(t),x_i)\le 2^{-k-1}$ for
$t\in\coint{q_i,t_i}$,
and $\rho(\omega(t),y_i)\le 2^{-k-1}$ for $t\in\ocint{t_i,q'_i}$.
Of course we
can suppose that $a\le q_0$ and that $q'_i\le q_{i+1}$ for $i<n$.   Observe
that this will ensure that every $\ooint{q_i,q'_i}$ is a
$2^{-k}$-shift interval of $\omega$ with
$(q_i,q'_i,2^{-k})$-shift point $t_i$.

Let $\le_i$ be $\le$ if $\epsilon_{t_i}=1$, $\ge$ if $\epsilon_{t_i}=-1$.
For each $W\in\Cal W$ containing $\omega$,
let $\theta_W\in\Theta$ be
$(q_0,\ldots,q'_n,\le_0,\ldots,\le_n,k,W)$.   Then
$\omega\in E_{\theta_W}$, and $f_i(\theta_W,\omega)=t_i$ for each
$i\le n$.   Because $\omega\notin F_{\theta_W}$, there is
an $\omega_W\in E_{\theta_W}$ such that
$f_i(\theta_W,\omega_W)<_if_i(\theta_W,\omega)=t_i$ for every $i\le n$.
Let $\omega'_W\in K$ be such that $\omega'_W\restr D=\omega_W\restr D$;
then $\omega'_W\in E_{\theta_W}$ and

\Centerline{$f_i(\theta_W,\omega'_W)=f_i(\theta_W,\omega_W)
<_it_i$}

\noindent for every $i\le n$ (455I(a-iv)).

If $i\le n$ and $\epsilon_{t_i}=1$,

\Centerline{$\rho(\omega'_W(q),\omega'_W(q'_i))\le 2^{-k}$}

\noindent for every rational $q\in\ocint{f_i(\theta_W,\omega'_W),q'_i}$;
because $\omega'_W\in\Clll$ and
$t_i\in\ocint{f_i(\theta_W,\omega'_W),q'_i}$
$\rho(\omega'_W(t_i),\omega'_W(q'_i))\le 2^{-k}$.   Similarly, if
$\epsilon_{t_i}=-1$, $\rho(\omega'_W(t_i),\omega'_W(q_i))\le 2^{-k}$.

Let $\Cal F$ be an ultrafilter on $\Cal W$ containing all sets of the form
$\{W:\omega\in W\subseteq W_0\}$ where $\omega\in W_0\in\Cal W$,
and set $\omega'=\lim_{W\to\Cal F}\omega'_W\in K$.
Then $\omega'\restr D=\omega\restr D$, because $\omega_W$ and $\omega'_W$
belong to $W$ whenever $\omega\in W\in\Cal W$.
If $i\le n$ and $\epsilon_{t_i}=1$, then

\Centerline{$\rho(\omega'(t_i),\omega'(q'_i))
=\lim_{W\to\Cal F}\rho(\omega'_W(t_i),\omega'_W(q'_i))\le 2^{-k}$,}

\Centerline{$\rho(\omega'(q_i),\omega'(q'_i))
=\rho(\omega(q_i),\omega(q'_i))>2^{-k+1}$,}

\noindent so $\rho(\omega'(q_i),\omega'(t_i))>2^{-k}$.
On the other hand,

\Centerline{$\rho(\omega'(q_i),\omega'(q))=\rho(\omega(q_i),\omega(q))
\le 2^{-k}$}

\noindent for every rational $q\in\coint{q_i,t_i}$.   So $\omega'$ cannot
be continuous on the left at $t_i$;  because $\omega'\in\Clll$, it
must be continuous on the right at $t_i$.   Similarly, if $i\le n$ and
$\epsilon_{t_i}=-1$, $\omega'$ cannot be continuous on the right at $t_i$
and must be continuous on the left at $t_i$.
But this is what we need to know.\ \Qed

\medskip

\quad{\bf (iii)} Suppose that $\omega\in K\setminus F^*$,
$\omega'\in\Clll$ and $\omega\restr D=\omega'\restr D$.
Then $\omega'\in K$.   \Prf\ Let $A$ be the set of points in
$\ooint{a,\infty}$ where $\omega$ is
not continuous, and for $t\in A$ let $\epsilon_t$ be $1$ if $\omega'$ is
continuous on the right at $t$, $-1$ if $\omega'$ is continuous on the left
at $t$.   For each finite $J\subseteq A$, (ii) tells us that there is an
$\omega_J\in K$ such that
$\omega_J\restr D=\omega\restr D=\omega'\restr D$ and,
for $t\in J$, $\omega_J$ is continuous on the right at $t$ if
$\epsilon_t=1$, and continuous on the left at $t$ if $\epsilon_t=-1$.
As both $\omega_J$ and $\omega'$ are \callal,
this means that $\omega_J(t)=\omega'(t)$ for $t\in J$.   Taking a
cluster point $\omega^*\in K$ of $\omega_J$ as $J$ increases through the
finite subsets of $A$, we see that
$\omega^*\restr(A\cup D)=\omega'\restr(A\cup D)$.

Now recall that $\omega\in E^*$, so that

\Centerline{$\omega'\restr D\cap[0,a]
=\omega\restr D\cap[0,a]=\tilde\omega\restr D\cap[0,a]$.}

\noindent Since both $\omega'$ and $\tilde\omega$ are \callal,
$\tilde\omega$ is discontinuous at any point
of $\coint{0,a}$ at which $\omega'$ is discontinuous.   Since I arranged
that $a$ (if finite) would be in $D$, $D\cup A$ contains every point at
which $\omega'$ is discontinuous.   But this means that $\omega^*=\omega'$
(438S(a-ii)).   So $\omega'\in K$.\ \Qed

\medskip

\quad{\bf (iv)} Suppose that $\tilde\mu_{\tilde\omega a}K>\gamma\ge 0$.
Then there is a
zero set $F\subseteq\Omega$ such that $F\cap\Clll\subseteq K$ and
$\tilde\mu_{\tilde\omega a}(F\cap\Clll)\ge\gamma$.   \Prf\ Because
$\tilde\mu_{\tilde\omega a}F^*=\hat\mu_{\tilde\omega a}F^*=0$,
there is a compact
$K'\subseteq K\setminus F^*$ such that
$\tilde\mu_{\tilde\omega a}K'\ge\gamma$.
Set $F=\pi_D^{-1}[\pi_D[K']]$;  $F$ is a zero set in $\Omega$ because
$\pi_D[K']$ is a zero set in $U^D$.   By (iii),
$F\cap\Clll\subseteq K$;  and

\Centerline{$\tilde\mu_{\tilde\omega a}(F\cap\Clll)
\ge\tilde\mu_{\tilde\omega a} K'\ge\gamma$.\ \Qed}

\medskip

{\bf (c)} Since $E^*$ and $\Clll$ are
$\tilde\mu_{\tilde\omega a}$-conegligible, the Radon measure
$\tilde\mu_{\tilde\omega a}$ is certainly inner regular with respect to the
compact subsets of $E^*\cap\Clll$;   by (b-iv), $\tilde\mu_{\tilde\omega a}$
is inner regular with respect to the
intersections of $\Clll$ with zero sets.
}%end of proof of 455J

\leader{455K}{Corollary} Suppose, in 455J, that $\tilde\omega\in\Cdlg$,
the space of \cadlag\ functions from $\coint{0,\infty}$ to $U$.
Then the subspace measure $\ddot\mu_{\tilde\omega a}$
on $\Cdlg$ induced by $\hat\mu_{\tilde\omega a}$ is a completion regular
quasi-Radon measure.

\proof{ The point is that the outer measures
$\tilde\mu_{\tilde\omega a}^*$ and $\hat\mu^*_{\tilde\omega a}$
agree on subsets of $\Cdlg$.   \Prf\ Since $\tilde\mu_{\tilde\omega a}$
extends $\hat\mu_{\tilde\omega a}$,
$\tilde\mu_{\tilde\omega a}^*A\le\hat\mu_{\tilde\omega a}^*A$
for every $A\subseteq\Omega$.   On the other hand, if $A\subseteq\Cdlg$ and
$\tilde\mu_{\tilde\omega a}^*A<\gamma$, there is an $E\supseteq A$ such that
$\tilde\mu_{\tilde\omega a} E<\gamma$.   By 455J, there is a zero set
$F\subseteq\Omega$ such that $E\cap F\cap\Clll=\emptyset$ and
$\tilde\mu_{\tilde\omega a}(F\cap\Clll)\ge 1-\gamma$.   Now

$$\eqalignno{\hat\mu_{\tilde\omega a}^*A
&\le\hat\mu_{\tilde\omega a}^*(\Cdlg\setminus F)
=\hat\mu_{\tilde\omega a}(\Omega\setminus F)\cr
\displaycause{because $\hat\mu_{\tilde\omega a}^*\Cdlg=1$, by 455G(c-ii)}
&=\tilde\mu_{\tilde\omega a}(\Omega\setminus F)
=\tilde\mu_{\tilde\omega a}(\Clll\setminus F)
\le\gamma.\cr}$$

\noindent As $\gamma$ is arbitrary,
$\hat\mu_{\tilde\omega a}^*A\le\tilde\mu_{\tilde\omega a}^*A$.\ \Qed

Write $\ddot{\tilde\mu}_{\tilde\omega a}$
for the subspace measure on $\Cdlg$ induced by
$\tilde\mu_{\tilde\omega a}$.   By 214Cd, the outer measures
$\ddot{\tilde\mu}\vthsp_{\tilde\omega a}^*
=\tilde\mu_{\tilde\omega a}^*\restr\Cal P\Cdlg$ and
$\ddot\mu_{\tilde\omega a}^*$ are the same.   Because
$\ddot{\tilde\mu}_{\tilde\omega a}$ and
$\ddot\mu_{\tilde\omega a}$ are both complete
probability measures, they must be identical (213C).
Because $\tilde\mu_{\tilde\omega a}$ is a Radon measure,
$\ddot{\tilde\mu}_{\tilde\omega a}=\ddot\mu_{\tilde\omega a}$
is quasi-Radon (415B).   Because
$\hat\mu_{\tilde\omega a}$ is the completion of a Baire measure, therefore
inner regular with respect to the zero sets in $\Omega$ (412D, 412Ha),
$\ddot\mu_{\tilde\omega a}$ is inner regular
with respect to the zero sets in $\Cdlg$, by 412Pd,
and is completion regular.
}%end of proof of 455K

\leader{455L}{Stopping times}\cmmnt{ We need the continuous-time
version of the concept of `stopping time' introduced in \S275.}
Let $\Omega$ be a set, $\Sigma$ a
$\sigma$-algebra of subsets of $\Omega$ and
$\langle\Sigma_t\rangle_{t\ge 0}$ a non-decreasing family of
$\sigma$-subalgebras of $\Sigma$.   (Such a family is called a
{\bf filtration}.)  For $t\ge 0$, set
$\Sigma^+_t=\bigcap_{s>t}\Sigma_s$, so that
$\langle\Sigma^+_t\rangle_{t\ge 0}$ also is a non-decreasing family of
$\sigma$-algebras.   Of course $\Sigma^+_t=\bigcap_{s>t}\Sigma^+_s$ for
every $t\ge 0$.

\spheader 455La A function
$\tau:\Omega\to[0,\infty]$ is a {\bf stopping time adapted to
$\langle\Sigma_t\rangle_{t\ge 0}$} if
$\{\omega:\omega\in\Omega$, $\tau(\omega)\le t\}$ belongs to $\Sigma_t$
for every $t\ge 0$.

\cmmnt{Note that in this case }$\tau$ will be $\Sigma$-measurable.

\spheader 455Lb A function $\tau:\Omega\to[0,\infty]$ is a stopping time
adapted to $\langle\Sigma^+_t\rangle_{t\ge 0}$ iff
$\{\omega:\tau(\omega)<t\}\in\Sigma_t$ for every $t\ge 0$.
\prooflet{\Prf\ (i) If $\tau$ is adapted to
$\langle\Sigma^+_t\rangle_{t\ge 0}$ and $t\ge 0$, then
$\{\omega:\tau(\omega)\le q\}\in\Sigma^+_q\subseteq\Sigma_t$ whenever
$0\le q<t$, so

\Centerline{$\{\omega:\tau(\omega)<t\}
=\bigcup_{q\in\Bbb Q\cap\coint{0,t}}\{\omega:\tau(\omega)\le q\}
\in\Sigma_t$.}

\noindent Thus $\tau$ satisfies the condition.   (ii) If $\tau$ satisfies
the condition and $t\ge 0$, set $t_n=t+2^{-n}$ for each $n$.
Then

\Centerline{$\{\omega:\tau(\omega)<t_n\}
\in\Sigma_{t_n}\subseteq\Sigma_{t_m}$}

\noindent whenever $m\le n$, so

\Centerline{$\{\omega:\tau(\omega)\le t\}
=\bigcap_{n\ge m}\{\omega:\tau(\omega)<t_n\}\in\Sigma_{t_m}$}

\noindent for every $m$, and

\Centerline{$\{\omega:\tau(\omega)\le t\}
\in\bigcap_{m\in\Bbb N}\Sigma_{t_m}
=\Sigma^+_t$.}

\noindent As $t$ is arbitrary, $\tau$ is adapted to
$\langle\Sigma^+_t\rangle_{t\ge 0}$.\ \Qed}

\spheader 455Lc{\bf (i)}
Constant functions on $\Omega$ are\cmmnt{ of course} stopping
times.

\medskip

\quad{\bf (ii)} If $\tau$ and $\tau'$ are stopping times adapted to
$\langle\Sigma_t\rangle_{t\ge 0}$, so is
$\tau+\tau'$.   \prooflet{\Prf\

\Centerline{$\{\omega:\tau(\omega)+\tau'(\omega)\le t\}
=\bigcap_{q\in\Bbb Q\cap[0,t]}
  \{\omega:\tau(\omega)\le q\}\cup\{\omega:\tau'(\omega)\le t-q\}
\in\Sigma_t$}

\noindent for every $t\ge 0$.\ \Qed}

\medskip

\quad{\bf (iii)}\cmmnt{ (Compare 455Cb and 455E(c-ii)).} If $\tau$ is a
stopping time adapted to $\langle\Sigma_t\rangle_{t\ge 0}$, then

\Centerline{$\Sigma_{\tau}
=\{E:E\in\Sigma$, $E\cap\{\omega:\tau(\omega)\le t\}\in\Sigma_t$ for every
$t\ge 0\}$}

\noindent is a $\sigma$-subalgebra of $\Sigma$.   \prooflet{(The check is
elementary.)}

\medskip

\quad{\bf (iv)} If $\familyiI{\tau_i}$ is a countable
family of stopping times adapted to $\langle\Sigma_t\rangle_{t\ge 0}$,
then $\tau=\sup_{i\in I}\tau_i$ is adapted to
$\langle\Sigma_t\rangle_{t\ge 0}$.   \prooflet{\Prf\ For any $t\ge 0$,

\Centerline{$\{\omega:\tau(\omega)\le t\}
=\bigcap_{i\in I}\{\omega:\tau_i(\omega)\le t\}
\in\Sigma_t$.  \Qed}}

\medskip

\quad{\bf (v)} If $\familyiI{\tau_i}$ is a countable family of stopping
times adapted to $\langle\Sigma^+_t\rangle_{t\ge 0}$,
then $\tau=\inf_{i\in I}\tau_i$ is adapted to
$\langle\Sigma^+_t\rangle_{t\ge 0}$\prooflet{, because

\Centerline{$\{\omega:\tau(\omega)<t\}
=\bigcup_{i\in I}\{\omega:\tau_i(\omega)<t\}\in\Sigma_t$}

\noindent for every $\tau\ge 0$}.

\spheader 455Ld Now suppose that $Y$ is a topological space and
we have a family
$\langle X_t\rangle_{t\ge 0}$ of functions from $\Omega$ to $Y$, and that
$\tau:\Omega\to[0,\infty]$ is any $\Sigma$-measurable function.
Set $X_{\tau}(\omega)=X_{\tau(\omega)}(\omega)$ when $\tau(\omega)<\infty$.
If $(t,\omega)\mapsto X_t(\omega):\coint{0,\infty}\times\Omega\to Y$
is $\Cal B(\coint{0,\infty})\tensorhat\Sigma$-measurable, where
$\Cal B(\coint{0,\infty})$ is the Borel $\sigma$-algebra of
$\coint{0,\infty}$, then $X_{\tau}:\{\omega:\tau(\omega)<\infty\}\to Y$
is $\Sigma$-measurable.   \prooflet{\Prf\ Setting
$\Omega_0=\{\omega:\tau(\omega)<\infty\}$, the map
$\omega\mapsto(\tau(\omega),\omega):
\Omega_0\to\coint{0,\infty}\times\Omega$
is $(\Sigma,\Cal B(\coint{0,\infty})\tensorhat\Sigma)$-measurable (4A3Bc),
so $X_{\tau}$ is the composition of a
$(\Sigma,\Cal B(\coint{0,\infty})\tensorhat\Sigma)$-measurable function
with a $\Cal B(\coint{0,\infty})\tensorhat\Sigma$-measurable function and
is $\Sigma$-measurable, by 4A3Bb.\ \Qed}

%\def\spheader#1#2#3#4#5{\header{#1#2#3#4#5}{\bf (#5)}}
\header{455Le}{\bf *(e)} Again take a topological space $Y$, a family
$\langle X_t\rangle_{t\ge 0}$ of functions from $\Omega$ to $Y$, and a
stopping time $\tau:\Omega\to[0,\infty]$ adapted to
$\langle\Sigma_t\rangle_{t\ge 0}$.   This time, suppose that
$\langle X_t\rangle_{t\ge 0}$ is {\bf progressively measurable}, that is,
that $(s,\omega)\mapsto X_s(\omega):[0,t]\times\Omega\to Y$ is
$\Cal B([0,t])\tensorhat\Sigma_t$-measurable for every $t\ge 0$, and
moreover that $\Sigma_t$ is closed under Souslin's operation\cmmnt{ (421B)}
for every $t$.   Then $X_{\tau}$, as defined in (d), will be
$\Sigma_{\tau}$-measurable.
\prooflet{\Prf\ Suppose that $H\subseteq Y$ is open, and set
$E=\{\omega:\omega\in\dom X_{\tau}$, $X_{\tau}(\omega)\in H\}$.
Of course $\langle X_t\rangle_{t\ge 0}$ satisfies the condition of (d), so
$E\in\Sigma$.   Take any $t\ge 0$.   Then

$$\eqalign{\{(s,\omega):0\le s\le t,\,\tau(\omega)=s\}
&=\bigcap_{n\in\Bbb N}
  \bigcup_{i\in\Bbb N}\{s:2^{-n}(i-1)<s\le\min(t,2^{-n}i)\}\cr
&\mskip80mu  \times\{\omega:2^{-n}(i-1)<\tau(\omega)
   \le\min(t,2^{-n}i)\}\cr
&\in\Cal B([0,t])\tensorhat\Sigma_t,\cr}$$

\Centerline{$\{(s,\omega):s\le t$, $X_s(\omega)\in H\}
\in\Cal B([0,t])\tensorhat\Sigma_t$,}

\noindent so

\Centerline{$W=\{(s,\omega):s\le t$, $\tau(\omega)=s$,
$X_s(\omega)\in H\}$}

\noindent also belongs to $\Cal B([0,t])\tensorhat\Sigma_t$.
Consequently the projection of $W$ onto $\Omega$ belongs to
$\Cal S(\Sigma_t)=\Sigma_t$ (423M).   But this is just

\Centerline{$\{\omega:\tau(\omega)\le t$, $X_{\tau(\omega)}\in H\}
=E\cap\{\omega:\tau(\omega)\le t\}$.}

\noindent As $t$ is arbitrary, $E\in\Sigma_{\tau}$;  as $H$ is arbitrary,
$X_{\tau}$ is $\Sigma_{\tau}$-measurable.\ \Qed}

\header{455Le}{\bf *(f)}\dvAnew{2013}\cmmnt{ There are some technical
points
concerning stopping times which are perhaps worth noting here.

\medskip

\quad}{\bf (i)} Suppose
that $\mu$ is a probability measure with domain $\Sigma$ and null ideal
$\Cal N(\mu)$.   Then we can
form the completion $\hat\mu$ with domain $\hat\Sigma$.   If we now
set $\hat\Sigma_t=\{E\symmdiff A:E\in\Sigma_t$, $A\in\Cal N(\mu)\}$,
$\langle\hat\Sigma_t\rangle_{t\ge 0}$ and
$\langle\hat\Sigma_t^+\rangle_{t\ge 0}$ are filtrations, where
$\hat\Sigma_t^+=\bigcap_{s>t}\hat\Sigma_s$ for $t\ge 0$.

\medskip

\quad{\bf (ii)} We find that
$\hat\Sigma_t^+=\{E\symmdiff A:E\in\Sigma_t^+$, $A\in\Cal N(\mu)\}$ for
every $t\ge 0$.   \prooflet{\Prf\ Of course

\Centerline{$\{E\symmdiff A:E\in\Sigma_t^+$, $A\in\Cal N(\mu)\}
\subseteq\bigcap_{s>t}
   \{E\symmdiff A:E\in\Sigma_s$, $A\in\Cal N(\mu)\}
=\hat\Sigma_t^+$.}

\noindent If $F\in\hat\Sigma_t^+$, then for every $q\in\Bbb Q$ such that
$q>t$ there is an $E_q\in\Sigma_q$ such that $F\symmdiff E_q$ is
negligible.   Set

\Centerline{$E
=\bigcup_{q\in\Bbb Q,q>t}\bigcap_{q'\in\Bbb Q,t<q'\le q}E_{q'}$,
\quad$A=F\symmdiff E$;}

\noindent then $E\in\Sigma_t^+$, $A\in\Cal N(\mu)$ and $F=E\symmdiff A$.\
\Qed}

\medskip

\quad{\bf (iii)} Of course every stopping time adapted to
$\langle\Sigma_t^+\rangle_{t\ge 0}$ is adapted to
$\langle\hat\Sigma_t^+\rangle_{t\ge 0}$.   Conversely, if
$\tau:\Omega\to[0,\infty]$ is a stopping time adapted to
$\langle\hat\Sigma_t^+\rangle_{t\ge 0}$, there is a stopping time
$\tau'$, adapted to $\langle\Sigma_t^+\rangle_{t\ge 0}$, such that
$\tau\eae\tau'$.   \prooflet{\Prf\
For each $q\in\Bbb Q\cap\coint{0,\infty}$, set
$F_q=\{\omega:\tau(\omega)<q\}$;  by (b), $F_q\in\hat\Sigma_q$ and
there is an $E_q\in\Sigma_q$ such that $F_q\symmdiff E_q$ is negligible.
For $\omega\in\Omega$,
set $\tau'(\omega)=\inf\{q:q\in\Bbb Q\cap\coint{0,\infty}$,
$\omega\in E_q\}$, counting $\inf\emptyset$ as $\infty$.   Then
$\{\omega:\tau'(\omega)<t\}
=\bigcup_{q\in\Bbb Q\cap\coint{0,t}}E_q$ belongs to $\Sigma_t$ for
every $t$, so $\tau'$ is adapted to $\langle\Sigma_t^+\rangle_{t\ge 0}$.
And $\{\omega:\tau'(\omega)\ne\tau(\omega)\}
\subseteq\bigcup_{q\in\Bbb Q\cap\coint{0,\infty}}E_q\symmdiff F_q$ is
negligible.\ \Qed}

\medskip

\quad{\bf (iv)} Continuing from (iii) just above, we find that, defining
$\hat\Sigma_{\tau}^+$ from $\langle\hat\Sigma_t^+\rangle_{t\ge 0}$ and
$\tau$ and $\Sigma_{\tau'}^+$ from
$\langle\Sigma_t^+\rangle_{t\ge 0}$ and
$\tau'$ by the formula in (c-iii), then
$\hat\Sigma_{\tau}^+=\{F\symmdiff A:F\in\Sigma_{\tau'}^+$,
$A\in\Cal N(\mu)\}$.   \prooflet{\Prf\ Let $A_0$ be the negligible set
$\{\omega:\tau(\omega)\ne\tau'(\omega)$.   ($\alpha$) If
$E\in\Sigma_{\tau'}^+$, then for every $t\ge 0$ we have

\Centerline{$E\cap\{\omega:\tau'(\omega)\le t\}
\in\Sigma^+_t$,}

\Centerline{$(E\cap\{\omega:\tau(\omega)\le t\})\symmdiff
   (E\cap\{\omega:\tau'(\omega)\le t\})\subseteq A_0\in\Cal N(\mu)$,}

\noindent so (using (ii))
$E\cap\{\omega:\tau(\omega)\le t\}\in\hat\Sigma^+_t$;  as $t$ is arbitrary,
$E\in\hat\Sigma_{\tau}^+$.   ($\beta$) If $F\in\hat\Sigma_{\tau}^+$, then
for every $q\in\Bbb Q\cap\coint{0,\infty}$ the sets
$F\cap\{\omega:\tau(\omega)\le q\}$ and
$F\cap\{\omega:\tau'(\omega)\le q\}$ belong to $\hat\Sigma^+_q$, so there
is an $E_q\in\Sigma^+_q$ such that
$E_q\symmdiff(F\cap\{\omega:\tau'(\omega)\le q\})$ is negligible.
Set $E'_q=\bigcup_{r\in\Bbb Q\cap[0,q]}E_r$ for
$q\in\Bbb Q\cap\coint{0,\infty}$;  then $E'_q\in\Sigma^+_q$ and
$E'_q\symmdiff(F\cap\{\omega:\tau'(\omega)\le q\})$ is negligible for
each $q$, while $E'_q\subseteq E'_r$ if $q\le r$ in
$\Bbb Q\cap\coint{0,\infty}$.   It follows that

\Centerline{$\bigcap_{q\in\Bbb Q\cap\coint{t,\infty}}E'_q
=\bigcap_{q\in\Bbb Q\cap[t,s]}E'_q\in\Sigma^+_s$}

\noindent whenever $t<s$ in $\coint{0,\infty}$, so that
$\bigcap_{q\in\Bbb Q\cap\coint{t,\infty}}E'_q\in\Sigma^+_t$ for every $t$.
Set

\Centerline{$E=\bigcap_{q\in\Bbb Q\cap\coint{0,\infty}}
   E'_q\cup\{\omega:\tau'(\omega)>q\}$.}

\noindent Then

$$\eqalign{&E\cap\{\omega:\tau'(\omega)\le t\}\cr
&\mskip50mu
=\{\omega:\tau'(\omega)\le t\}\cap\bigcap_{q\in\Bbb Q\cap\coint{0,\infty}}
   (E'_q\cup\{\omega:\tau'(\omega)>q\})\cr
&\mskip50mu
=\{\omega:\tau'(\omega)\le t\}\cap\bigcap_{q\in\Bbb Q\cap\coint{0,t}}
   (E'_q\cup\{\omega:\tau'(\omega)>q\})
   \cap\bigcap_{q\in\Bbb Q\cap\coint{t,\infty}}E'_q\cr
&\mskip50mu\in\Sigma^+_t\cr}$$

\noindent for any $t\ge 0$, so $E\in\Sigma^+_{\tau'}$.
If we look at $(E\symmdiff F)\cap\{\omega:\tau'(\omega)<\infty\}$,
we see that this is included in the negligible set

\Centerline{$\bigcup_{q\in\Bbb Q\cap\coint{0,\infty}}
   E'_q\symmdiff(F\cap\{\omega:\tau'(\omega)\le q\})$}

\noindent because $\{\omega:\tau'(\omega)<\infty\}\cap F$ is just

\Centerline{$\{\omega:\tau'(\omega)<\infty\}
  \cap\bigcap_{q\in\Bbb Q\cap\coint{0,\infty}}
      (F\cap\{\omega:\tau'(\omega)\le q\})\cup\{\omega:\tau'(\omega)>q\}$.}

\noindent As for the set $H=\{\omega:\tau'(\omega)=\infty\}$, this belongs
to $\Sigma$, and every subset of $H$ belonging to $\Sigma$ also belongs to
$\Sigma^+_{\tau'}$.   Let $H'\in\Sigma$ be such that $H'\subseteq H$ and
$H'\symmdiff(F\cap H)$ is negligible;  then $E'=H'\cup(E\setminus H)$
belongs to $\Sigma^+_{\tau'}$ and differs from $F$ by a negligible set.\
\Qed}

\leader{455M}{Hitting \dvrocolon{times}}\cmmnt{ I mention a class of
stopping times which is particularly important in applications, and also
very helpful in giving an idea of the concept.

\medskip

\noindent}{\bf Proposition}  Let $U$ be a Polish
space and $\Cdlg$ the set of \cadlag\ functions from $\coint{0,\infty}$ to
$U$.   Let $A\subseteq U$ be an analytic set, and define
$\tau:\Cdlg\to[0,\infty]$ by setting

\Centerline{$\tau(\omega)=\inf\{t:\omega(t)\in A\}$}

\noindent for $\omega\in\Cdlg$, counting $\inf\emptyset$ as $\infty$.

(a) Let $\Sigma$ be a $\sigma$-algebra of subsets of $\Cdlg$ closed
under Souslin's operation and including the algebra generated by the
functionals $\omega\mapsto\omega(t)$ for $t\ge 0$.   Then $\tau$ is
$\Sigma$-measurable.

(b) For $t\ge 0$ let $\Sigma_t$ be

\Centerline{$\{F:F\in\Sigma,\,\omega'\in F$ whenever $\omega$,
$\omega'\in\Cdlg$, $\omega\in F$ and
$\omega\restr[0,t]=\omega'\restr[0,t]\}$,}

\noindent and $\Sigma_t^+=\bigcap_{s>t}\Sigma_t$.   Then $\tau$ is a
stopping time adapted to $\langle\Sigma^+_t\rangle_{t\ge 0}$.

(c) If $A$ is closed, then $\tau$ is adapted to
$\langle\Sigma_t\rangle_{t\ge 0}$.
%this bit seems to need hitting time to be  \inf_{t\ge 0}  rather than
% \inf_{t>0}

\proof{{\bf (a)(i)} It will help to recall from 4A3W that there is a Polish
topology $\frak S$ on $\Cdlg$ such that the Borel
$\sigma$-algebra $\Cal B(\Cdlg)$
is just the $\sigma$-algebra generated by the coordinate functions
$\omega\mapsto\omega(t)$.   In this case, every $\frak S$-analytic set,
being $\frak S$-Souslin-F (423Eb), belongs to $\Sigma$.

\medskip

\quad{\bf (ii)} The set

\Centerline{$W
=\{(\omega,t,x):\omega\in\Cdlg$, $t\ge 0$, $x\in U$, $\omega(t)=x\}$}

\noindent is a Borel subset of $\Cdlg\times\coint{0,\infty}\times U$.
\Prf\ If $\rho$ is a metric on $U$ inducing its topology and $D$ is a
countable dense subset of $U$,

$$\eqalign{W
&=\bigcap_{n\in\Bbb N}\bigcup_{q\in\Bbb Q,q\ge 0}\bigcup_{y\in D}
    \{(\omega,t,x):t\in[q-2^{-n},q],\cr
&\mskip200mu  \rho(\omega(q),y)\le 2^{-n},\,
    \rho(x,y)\le 2^{-n}\}.  \text{ \Qed}\cr}$$

\medskip

\quad{\bf (iii)} Since $\Cdlg$, $\coint{0,\infty}$ and $U$ are all
Polish, $W$ is an analytic set.   Now, for any $t\ge 0$,

\Centerline{$W'
=\{(\omega,s,x):s\in\coint{0,t}$, $x\in A$, $(\omega,s,x)\in W\}$}

\noindent is analytic and its projection

\Centerline{$\{\omega:\tau(\omega)<t\}
=\{\omega:$ there are $s$, $x$ such that $(\omega,s,x)\in W'\}$}

\noindent is analytic and belongs to $\Sigma$.   As $t$ is arbitrary,
$\tau$ is $\Sigma$-measurable.

\medskip

{\bf (b)} Now, given $t\ge 0$, $F=\{\omega:\tau(\omega)<t\}$ belongs to
$\Sigma$, and if $\omega\in F$, $\omega'\in\Cdlg$ are such that
$\omega'\restr[0,t]=\omega\restr[0,t]$, there is an $s<t$ such that
$\omega'(s)=\omega(s)\in A$, so $\tau(\omega')<t$ and $\omega'\in F$.
Thus $F\in\Sigma_t$.   As $t$ is arbitrary, $\tau$ is a stopping time
adapted to $\langle\Sigma_t^+\rangle_{t\ge 0}$, by 455Lb.

\medskip

{\bf (c)} As $A$ is closed and every member of $\Cdlg$ is continuous on the
right, $\omega(\tau(\omega))\in A$ whenever $\tau(\omega)<\infty$.   So
if $\omega$, $\omega'\in\Cdlg$, $\tau(\omega)\le t$ and
$\omega'\restr[0,t]=\omega\restr[0,t]$, then
$\omega'(\tau(\omega))\in A$ and
$\tau(\omega')\le t$.   Thus $\{\omega:\tau(\omega)\le t\}\in\Sigma_t$ for
every $t$, and $\tau$ is adapted to $\langle\Sigma_t\rangle_{t\ge 0}$.
}%end of proof of 455M

\leader{455N}{}\cmmnt{ We need an elementary fact about narrow (more
properly, vague) convergence.

\medskip

\noindent}{\bf Lemma} Let $(U,\rho)$ be a metric space, $n\in\Bbb N$
and $f:U^{n+1}\to\Bbb R$ a bounded uniformly continuous function.   Let
$\langle\nu^{(k)}_x\rangle_{k<n,x\in U}$ be a family of topological
probability measures on $U$ such that $x\mapsto\nu^{(k)}_x$ is continuous
for the narrow topology for every $k<n$.   Then

\Centerline{$y
\mapsto\iint\ldots\int f(y,x_1,\ldots,x_n)\nu^{(n-1)}_{x_{n-1}}(dx_n)
    \ldots\nu_{x_1}^{(1)}(dx_2)\nu_y^{(0)}(dx_1)$}

\noindent is defined everywhere on $U$ and continuous.

\proof{ Induce on $n$.   If $n=0$ the formula is just
$y\mapsto f(y)$, so the result is trivial.
For the inductive step to $n\ge 1$, set

\Centerline{$g(y,x_1)
=\int\ldots\int f(y,x_1,\ldots,x_n)
  \nu^{(n-1)}_{x_{n-1}}(dx_n)\ldots\nu_{x_1}^{(1)}(dx_2)$}

\noindent for $y$, $x_1\in U$;  by the inductive hypothesis this is
well-defined and $x_1\mapsto g(y,x_1)$ is continuous.   Note that
$g$ is bounded because $f$ is.   It follows that
$h(y)=\int g(y,x_1)\nu_y^{(0)}(dx_1)$ is defined for every
$y$.   I need to show that $h$ is continuous.   Take any
$y\in U$ and $\epsilon>0$.   Then there is a $\delta_0>0$ such that
$|f(y',x_1,\ldots,x_n)-f(y,x_1,\ldots,x_n)|\le\epsilon$ whenever
$\rho(y',y)\le\delta_0$ and $x_1,\ldots,x_n\in U$;  so that
$|g(y',x_1)-g(y,x_1)|\le\epsilon$ whenever $\rho(y',y)\le\delta_0$ and
$x_1\in U$.   Next, because $x\mapsto\nu_x^{(0)}$ is narrowly continuous,
$x\mapsto\int g(y,x_1)\nu_x^{(0)}(dx_1)$ is continuous (437Jf/437Kb),
and there is a $\delta\in\ocint{0,\delta_0}$ such that
$|\int g(y,x_1)\nu_{y'}^{(0)}(dx_1)
-\int g(y,x_1)\nu_y^{(0)}(dx_1)|\le\epsilon$ whenever
$\rho(y',y)\le\delta$.   So if $\rho(y',y)\le\delta$,

$$\eqalign{|h(y')-h(y)|
&\le|\int g(y',x_1)\nu_{y'}^{(0)}(dx_1)-\int g(y,x_1)\nu_{y'}^{(0)}(dx_1)|
  \cr
&\mskip100mu
 +|\int g(y,x_1)\nu_{y'}^{(0)}(dx_1)-\int g(y,x_1)\nu_{y}^{(0)}(dx_1)|\cr
&\le\int|g(y',x_1)-g(y,x_1)|\nu_{y'}^{(0)}(dx_1)
  +\epsilon
\le 2\epsilon.\cr}$$

\noindent As $y$ and $\epsilon$ are arbitrary, $h$ is continuous and the
induction proceeds.
}%end of proof of 455N

\leader{455O}{}\cmmnt{ If both the continuity conditions in 455F are
satisfied, we have a version of 455C/455Eb which is much more to the point.

\medskip

\noindent}{\bf Theorem} Suppose that $(U,\rho)$ is a complete metric space,
$x^*$ is a point of $U$,
$\langle\nu^{(s,t)}_x\rangle_{0\le s<t,x\in U}$ is a family of Radon
probability measures on $U$ which is both narrowly continuous and uniformly
time-continuous on the right, and
that $\family{y}{U}{\nu^{(t,u)}_y}$ is a disintegration of $\nu^{(s,u)}_x$
over $\nu^{(s,t)}_x$ whenever $x\in U$ and $s<t<u$.
Let $\hat\mu$ be the
corresponding completed measure on $\Omega=U^{\coint{0,\infty}}$,
as in 455E.
Let $\Cdlg$ be the set of \cadlag\ functions from $\coint{0,\infty}$ to
$U$, $\ddot\mu$ the subspace measure on
$\Cdlg$, and $\ddot\Sigma$ its domain.   For $t\ge 0$, let $\ddot\Sigma_t$
be

\Centerline{$\{F:F\in\ddot\Sigma$,
$\omega'\in F$ whenever $\omega\in F$, $\omega'\in\Cdlg$ and
$\omega'\restr[0,t]=\omega\restr[0,t]\}$,}

\noindent and $\ddot\Sigma^+_t=\bigcap_{s>t}\ddot\Sigma_t$.

For $\omega\in\Omega$ and $a\ge 0$ let $\hat\mu_{\omega a}$ be the
completed measure on $\Omega$ built from $\omega(0)$ and
$\langle\nu^{(s,t)}_{\omega ax}\rangle_{0\le s<t,x\in U}$
as in 455Eb;  let $\ddot\mu_{\omega a}$ be the subspace measure on $\Cdlg$.
Let $\tau:\Cdlg\to[0,\infty]$ be a stopping time adapted to
$\langle\ddot\Sigma^+_t\rangle_{t\ge 0}$.

(a) $\family{\omega}{\Cdlg}{\ddot\mu_{\omega,\tau(\omega)}}$
is a disintegration of $\ddot\mu$ over itself.

(b) Set

\Centerline{$\ddot\Sigma^+_{\tau}
=\{F:F\in\ddot\Sigma,\,F\cap\{\omega:\tau(\omega)\le t\}
   \in\ddot\Sigma^+_t\text{ for every }t\ge 0\}$.}

\noindent Then $\ddot\Sigma^+_{\tau}$ is a $\sigma$-algebra of subsets of
$\Cdlg$.   For a $\ddot\mu$-integrable function
$f$ on $\Cdlg$, write
$\ddot g_f(\omega)=\int_{\Cdlg}fd\ddot\mu_{\omega,\tau(\omega)}$
when this is defined in $\Bbb R$.
Then $\ddot g_f$ is a conditional expectation of $f$ on
$\ddot\Sigma^+_{\tau}$.

(c) If $\tau$ is adapted to $\langle\ddot\Sigma_t\rangle_{t\ge 0}$, set

\Centerline{$\ddot\Sigma_{\tau}
=\{F:F\in\ddot\Sigma,\,F\cap\{\omega:\tau(\omega)\le t\}
   \in\ddot\Sigma_t\text{ for every }t\ge 0\}$.}

\noindent Then $\ddot\Sigma_{\tau}$ is a $\sigma$-algebra of subsets of
$\Cdlg$, and $\ddot g_f$ is a conditional expectation of $f$ on
$\ddot\Sigma_{\tau}$, for every $f\in\eusm L^1(\ddot\mu)$.

\proof{{\bf (a)(i)} I had better begin by checking that the ground is
clear.   By 455G, $\hat\mu^*\Cdlg=\hat\mu^*_{\omega a}\Cdlg=1$
for every $\omega\in\Cdlg$ and $a\ge 0$, so that $\ddot\mu$
and $\ddot\mu_{\omega a}$ (for $\omega\in\Cdlg$) are all
probability measures.
\leaveitout{Because $\hat\mu$ is the completion of its restriction to
$\Tensorhat_{\coint{0,\infty}}\Cal B(U)$, $\ddot\mu$ is the completion of
its restriction to the subspace $\sigma$-algebra $\Tau$ induced by
$\Tensorhat_{\coint{0,\infty}}\Cal B(U)$ on $\Cdlg$
(214Ib\formerly{2{}14Xb});
now this is the $\sigma$-algebra generated by the functionals
$\omega\mapsto\omega(t)$ for $t\ge 0$, so that $(\Cdlg,\Tau)$ is a standard
Borel space (4A3N, 4A3W).}

Of course $\langle\ddot\Sigma_t\rangle_{t\ge 0}$ is a non-decreasing family of
$\sigma$-subalgebras of $\ddot\Sigma$, so that
$\langle\ddot\Sigma^+_t\rangle_{t\ge 0}$ is another such family, and
we are in the territory explored in 455L.

\medskip

\quad{\bf (ii)} Write $\Sigma$ for the domain of $\hat\mu$, and for
$t\ge 0$ set

\Centerline{$\Sigma_t
=\{E:E\in\Sigma$, $E$ is determined by coordinates in
$[0,t]\}$.}

\noindent Then $\ddot\Sigma_t=\{E\cap\Cdlg:E\in\Sigma_t\}$.
\Prf\ If $E\in\Sigma_t$, then $E\cap\Cdlg\in\ddot\Sigma$ and clearly
$E\cap\Cdlg\in\ddot\Sigma_t$.  If $F\in\ddot\Sigma_t$,
let $E\in\Sigma$ be such that $E\cap\Cdlg=F$.
Applying 455Ec to the stopping time with constant value $t$, we have

\Centerline{$\hat\mu E
=\int_{\Omega}\hat\mu_{\omega t}(E)\hat\mu(d\omega)$.}

\noindent Set

\Centerline{$E^*=\{\omega:\omega\in\Omega$, $\hat\mu_{\omega t}(E)$ is
defined$\}$,}

\Centerline{$E_0=\{\omega:\omega\in E^*$, $\hat\mu_{\omega t}(E)=0\}$,
\quad$E_1=\{\omega:\omega\in E^*$, $\hat\mu_{\omega t}(E)=1\}$.}

\noindent Then $E^*$, $E_0$ and $E_1$ are measured by $\hat\mu$ and are
determined by coordinates in $[0,t]$ (by 455E(b-iii)),
and $\hat\mu E^*=1$.

If $\omega\in E^*\cap\Cdlg$, then $\hat\mu_{\omega t}^*\Cdlg=1$, so

\Centerline{$\ddot\mu_{\omega t}(F)
=\hat\mu^*_{\omega t}(E\cap\Cdlg)
=\hat\mu_{\omega t}(E)$.}

\noindent If $\omega\in\Cdlg$, let $D$ be a countable dense subset of
$[0,t]$ containing $t$;  then

$$\eqalign{1
&=\hat\mu_{\omega t}\{\omega':\omega'\in\Omega,\,
   \omega'\restr D=\omega\restr D\}
=\ddot\mu_{\omega t}\{\omega':\omega'\in\Cdlg,\,
   \omega'\restr D=\omega\restr D\}\cr
&=\ddot\mu_{\omega t}\{\omega':\omega'\in\Cdlg,\,
   \omega'\restr[0,t]=\omega\restr[0,t]\}.\cr}$$

\noindent So if $\omega\in E^*\cap\Cdlg$,

\Centerline{$\hat\mu_{\omega t}(E)
=\ddot\mu_{\omega t}(F)
=\ddot\mu_{\omega t}\{\omega':\omega'\in F,\,
  \omega'\restr[0,t]=\omega\restr[0,t]\}
=\chi F(\omega)\in\{0,1\}$}

\noindent because $F$ is determined (relative to $\Cdlg$)
by coordinates in $[0,t]$.   This means
that $E_1\cap\Cdlg\subseteq F$ and $E_0\cap F=\emptyset$, while
$E_0\cup E_1$ is $\hat\mu$-conegligible.   So if we take

\Centerline{$E'
=E_1\cup\{\omega:\omega\Omega\setminus E_1$
and there is an $\omega'\in F$ such that
$\omega'\restr[0,t]=\omega\restr[0,t]\}$,}

\noindent $E'\cap\Cdlg=F$, $E'$ is determined by coordinates in $[0,t]$,
$E_1\subseteq E'\subseteq\Omega\setminus E_0$, $\hat\mu$ measures $E'$ and
$E'\in\Sigma_t$.   Thus $\ddot\Sigma_t=\{E\cap\Cdlg:E\in\Sigma_t\}$,
as claimed.\ \Qed

\medskip

\quad{\bf (iii)} Take $n\in\Bbb N$, and set $D_n=\{2^{-n}i:i\in\Bbb N\}$.
Suppose that $\tau:\Cdlg\to D_n\cup\{\infty\}$ is a
stopping time adapted to $\langle\ddot\Sigma_t\rangle_{t\ge 0}$.   Then
$\family{\omega}{\Cdlg}{\hat\mu_{\omega,\tau(\omega)}}$
is a disintegration of $\hat\mu$ over $\ddot\mu$.   \Prf\ For each
$i\in\Bbb N$, $F_i=\tau^{-1}[\{2^{-n}i\}]$ belongs to
$\ddot\Sigma_{2^{-n}i}$, so there is an $E_i\in\Sigma$,
determined by coordinates in
$[0,2^{-n}i]$, such that $F_i=E_i\cap\Cdlg$.   For $\omega\in\Omega$, set

\Centerline{$\grave\tau(\omega)
=\inf\{2^{-n}i:i\in\Bbb N,\,\omega\in E_i\}$,}

\noindent counting $\inf\emptyset$ as $\infty$.   Then
$\grave\tau\restr\Cdlg=\tau$.  Also
$\grave\tau[\Omega]\subseteq D_n\cup\{\infty\}$ is
countable, and $\grave\tau^{-1}[\{b\}]\in\Sigma$ is determined by
coordinates in $[0,b]$ for every $b\in D_n$.   By 455Ec,
$\family{\omega}{\Omega}{\hat\mu_{\omega,\grave\tau(\omega)}}$ is a
disintegration of $\hat\mu$ over itself.

Now take any $E\in\Sigma$.   Then

$$\eqalignno{\hat\mu E
&=\int_{\Omega}\hat\mu_{\omega,\grave\tau(\omega)}(E)\hat\mu(d\omega)
=\int_{\Cdlg}\hat\mu_{\omega,\grave\tau(\omega)}(E)\ddot\mu(d\omega)\cr
\displaycause{214F}
&=\int_{\Cdlg}\hat\mu_{\omega,\tau(\omega)}(E)\ddot\mu(d\omega).
\text{ \Qed}\cr}$$

It follows that
$\family{\omega}{\Cdlg}{\ddot\mu_{\omega,\tau(\omega)}}$
is a disintegration of $\ddot\mu$ over itself.   \Prf\ If
$F\in\ddot\Sigma$, there is an $E\in\Sigma$ such that $F=E\cap\Cdlg$.
Now

$$\eqalign{\ddot\mu F
&=\hat\mu E
=\int_{\Cdlg}\hat\mu_{\omega,\tau(\omega)}(E)\ddot\mu(d\omega)\cr
&=\int_{\Cdlg}\ddot\mu_{\omega,\tau(\omega)}(E\cap\Cdlg)\ddot\mu(d\omega)
=\int_{\Cdlg}\ddot\mu_{\omega,\tau(\omega)}(F)\ddot\mu(d\omega).
\text{ \Qed}\cr}$$

\medskip

\quad{\bf (iv)} Now let $\tau:\Cdlg\to[0,\infty]$ be any stopping time adapted to
$\langle\ddot\Sigma^+_t\rangle_{t\ge 0}$.   For each $n\in\Bbb N$, define
$\tau_n:\Cdlg\to D_n\cup\{\infty\}$ by setting

$$\eqalign{\tau_n(\omega)
&=2^{-n}(i+1)\text{ if }i\in\Bbb N\text{ and }
   2^{-n}i\le\tau(\omega)<2^{-n}(i+1),\cr
&=\infty\text{ if }\tau(\omega)=\infty.\cr}$$

\noindent By 455Lb, $\{\omega:\tau_n(\omega)=t\}\in\ddot\Sigma_t$ for every
$t\in D_n$.   So (iii) tells us that
$\langle\ddot\mu_{\omega,\tau_n(\omega)}\rangle_{\omega\in\Cdlg}$ is a
disintegration of $\ddot\mu$ over itself.

\medskip

\quad{\bf (v)} Suppose that $k\in\Bbb N$, $0=t_0<t_1<\ldots<t_k$,
$h:U^{k+1}\to\Bbb R$ is bounded and uniformly continuous,
and $\omega\in\Cdlg$.   Then

\Centerline{$\int_{\Omega}h(\omega'(t_0),\ldots,\omega'(t_k))
   \hat\mu_{\omega,\tau(\omega)}(d\omega')
=\lim_{n\to\infty}\int_{\Omega}h(\omega'(t_0),\ldots,\omega'(t_k))
   \hat\mu_{\omega,\tau_n(\omega)}(d\omega')$.}

\noindent\Prf\ Recall from 455E that

$$\eqalign{\int_{\Omega}h(\omega'(t_0),\ldots\,&,\omega'(t_k))
   \hat\mu_{\omega,\tau(\omega)}(d\omega')\cr
&=\int_U\ldots\int_Uh(\omega(0),x_1,\ldots,x_k)
   \nu_{\omega,\tau(\omega),x_{k-1}}^{(t_{k-1},t_k)}(dx_k)
   \ldots\nu^{(0,t_1)}_{\omega(0)}(dx_1),\cr}$$

\noindent and similarly for each $\tau_n$.
If $\tau(\omega)\ge t_k$, then

\Centerline{$\nu^{(t_{i-1},t_i)}_{\omega,\tau_n(\omega),x}
=\delta_{\omega(t_i)}=\nu^{(t_{i-1},t_i)}_{\omega,\tau(\omega),x}$}

\noindent for $1\le i\le k$, $n\in\Bbb N$ and $x\in U$, so the result is
trivial.   If $j\le k$ is such that $t_{j-1}\le\tau(\omega)<t_j$, then

$$\eqalign{\nu^{(t_{i-1},t_i)}_{\omega,\tau(\omega),x}
&=\delta_{\omega(t_i)}\text{ if }i<j,\cr
&=\nu^{(\tau(\omega),t_j)}_{\omega(\tau(\omega))}
   \text{ if }i=j,\cr
&=\nu_x^{(t_{i-1},t_i)}\text{ if }j<i<k.\cr}$$

\noindent So

$$\eqalign{\int_{\Omega}h(\omega'(t_0),\ldots\,&,\omega'(t_k))
   \hat\mu_{\omega,\tau(\omega)}(d\omega')\cr
&=\int_U\int_U\ldots\int_Uh(\omega(0),\ldots,\omega(j-1),x_j,\ldots,x_k)\cr
&\mskip150mu
   \nu_{x_{k-1}}^{(t_{k-1},t_k)}(dx_k)
   \ldots
   \nu_{x_j}^{(t_j,t_{j+1})}(dx_{j+1})
   \nu_{\omega(\tau(\omega))}^{(\tau(\omega),t_j)}(dx_j).\cr}$$

\noindent Moreover, there is some $n_0$ such that $\tau_n(\omega)<t_j$ for
every $n\ge n_0$, so that we can use this formula for all such $n$.
Setting

\Centerline{$g(x)
=\int_U\ldots\int_Uh(\omega(0),\ldots,\omega(j-1),x,x_{j+1},\ldots,x_k)
   \nu_{x_{k-1}}^{(t_{k-1},t_k)}(dx_k)
   \ldots
   \nu_x^{(t_j,t_{j+1})}(dx_{j+1})$}

\noindent for $x\in U$, we see from 455N that $g$ is continuous, while of
course it is also bounded, because $h$ is bounded.   At this point, recall
that $\omega$ is supposed to be continuous on the right, while
the system of transitional probabilities is jointly continuous, so that

\Centerline{$\nu_{\omega(\tau(\omega))}^{(\tau(\omega),t_j)}
=\lim_{n\to\infty}\nu_{\omega(\tau_n(\omega))}^{(\tau_n(\omega),t_j)}$}

\noindent for the narrow topology, and

$$\eqalign{\lim_{n\to\infty}
   \int_{\Omega}h(\omega'(t_0),\ldots\,&,\omega'(t_k))
   \hat\mu_{\omega,\tau_n(\omega)}(d\omega')\cr
&=\lim_{n\to\infty}
  \int_U\int_U\ldots\int_Uh(\omega(0),\ldots,\omega(j-1),x_j,\ldots,x_k)\cr
&\mskip100mu
   \nu_{x_{k-1}}^{(t_{k-1},t_k)}(dx_k)
   \ldots
   \nu_{x_j}^{(t_j,t_{j+1})}(dx_{j+1})
   \nu_{\omega(\tau_n(\omega))}^{(\tau_n(\omega),t_j)}(dx_j)\cr
&=\lim_{n\to\infty}\int_Ug(x_j)
   \nu_{\omega(\tau_n(\omega))}^{(\tau_n(\omega),t_j)}(dx_j)\cr
&=\int_Ug(x_j)
   \nu_{\omega(\tau(\omega))}^{(\tau(\omega),t_j)}(dx_j)\cr
&=\int_{\Omega}h(\omega'(t_0),\ldots,\omega'(t_k))
   \hat\mu_{\omega,\tau(\omega)}(d\omega'),\cr}$$

\noindent as claimed.\ \Qed

\medskip

\quad{\bf (vi)} Again suppose that $0=t_0<t_1<\ldots<t_k$.   If
$h:U^{k+1}\to\Bbb R$ is bounded and uniformly continuous, then

\Centerline{$\int_{\Cdlg}\int_{\Omega}h(\omega'(t_0),\ldots,\omega'(t_k))
   \hat\mu_{\omega,\tau(\omega)}(d\omega')\ddot\mu(d\omega)
=\int_{\Omega}h(\omega(t_0),\ldots,\omega(t_k))\hat\mu(d\omega)$.}

\noindent\Prf\ The point here is that

$$\eqalign{\int_{\Cdlg}\int_{\Omega}h(\omega'(t_0),\ldots,\omega'(t_k))
   \hat\mu_{\omega,\tau_n(\omega)}(d\omega')\ddot\mu(d\omega)
&=\int_{\Omega}h(\omega(t_0),\ldots,\omega(t_k))
   \hat\mu(d\omega)\cr}$$

\noindent is defined for every $n\in\Bbb N$, by (iii) and 452F, as usual.
Now the integrands

\Centerline{$\omega
\mapsto\int_{\Omega}h(\omega'(t_0),\ldots,\omega'(t_k))
   \hat\mu_{\omega,\tau_n(\omega)}(d\omega')$}

\noindent converge at every point of $\Cdlg$, by (v), and are uniformly
bounded, because $h$ is, so that

$$\eqalign{\int_{\Cdlg}\int_{\Omega}h(\omega'(t_0),\ldots\,&,\omega'(t_k))
   \hat\mu_{\omega,\tau(\omega)}(d\omega')\ddot\mu(d\omega)\cr
&=\lim_{n\to\infty}\int_{\Cdlg}\int_{\Omega}
    h(\omega'(t_0),\ldots,\omega'(t_k))
   \hat\mu_{\omega,\tau_n(\omega)}(d\omega')\ddot\mu(d\omega)\cr
&=\int_{\Omega}h(\omega(t_0),\ldots,\omega(t_k))(d\omega).
\text{  \Qed}\cr}$$

If $G\subseteq U^{k+1}$ is open, there is a non-decreasing sequence
$\sequence{m}{h_m}$ of uniformly continuous functions from $U^{k+1}$ to
$[0,1]$ such that $\chi G=\sup_{m\in\Bbb N}h_m$, in which case

$$\eqalign{\int_{\Cdlg}
  \hat\mu_{\omega,\tau(\omega)}
  \{\omega':(\omega'(t_0)&,\ldots,\omega'(t_k))\in G\}
 \ddot\mu(d\omega)\cr
&=\lim_{m\to\infty}
  \int_{\Cdlg}\int_{\Omega}h_m(\omega'(t_0),\ldots,\omega'(t_k))
  \hat\mu_{\omega,\tau(\omega)}(d\omega')\ddot\mu(d\omega)\cr
&=\lim_{m\to\infty}
  \int_{\Omega}h_m(\omega(t_0),\ldots,\omega(t_k))\hat\mu(d\omega)\cr
&=\hat\mu\{\omega:(\omega(t_0),\ldots,\omega(t_k))\in G\}.\cr}$$

\noindent By the Monotone Class Theorem, we get

$$\eqalign{\int_{\Cdlg}
  \hat\mu_{\omega,\tau(\omega)}
  \{\omega':(\omega'(t_0)&,\ldots,\omega'(t_k))\in E\}
 \ddot\mu(d\omega)\cr
&=\hat\mu\{\omega:(\omega(t_0),\ldots,\omega(t_k))\in E\}\cr}$$

\noindent for every Borel set $E\subseteq U^{k+1}$.   Now recall that
$t_0,\ldots,t_k$ were any strictly increasing sequence starting at $0$, so
we can use the Monotone Class Theorem yet again to see that

\Centerline{$\int_{\Cdlg}
  \hat\mu_{\omega,\tau(\omega)}(E)\ddot\mu(d\omega)
=\hat\mu(E)$}

\noindent for every $E\in\Tensorhat_{\coint{0,\infty}}\Cal B(U)$
and therefore for every $E\in\Sigma$.

\medskip

\quad{\bf (vii)} Finally, if $F\in\ddot\Sigma$,
there is an $E\in\Sigma$ such that $F=E\cap\Cdlg$, so that

\Centerline{$\ddot\mu(F)=\hat\mu(E)
=\int_{\Cdlg}\hat\mu_{\omega,\tau(\omega)}(E)\ddot\mu(d\omega)
=\int_{\Cdlg}\ddot\mu_{\omega,\tau(\omega)}(F)\ddot\mu(d\omega)$;}

\noindent which is what we set out to prove.

\medskip

{\bf (b)(i)} By 455L(c-iii), $\ddot\Sigma^+_{\tau}$ is a $\sigma$-algebra.
If $f$ is a $\ddot\mu$-integrable real-valued function, then
$\int_{\Cdlg}\ddot g_f\ddot\mu=\int_{\Cdlg}f\ddot\mu$, by (a) and 452F.
For $\alpha\in\Bbb R$ set

\Centerline{$E(f,\alpha)
=\{\omega:\omega\in\Cdlg$, $\ddot g_f(\omega)$ is defined in $\Bbb R$ and
$\ddot g_f(\omega)\le\alpha\}$,}

\noindent so that $E(f,\alpha)\in\ddot\Sigma$.   For $t\ge 0$, set

\Centerline{$H_t=\{\omega:\omega\in\Cdlg$, $\tau(\omega)\le t\}$,
\quad$H'_t=\{\omega:\omega\in\Cdlg$, $\tau(\omega)<t\}$,}

\noindent so that $H_t\in\ddot\Sigma_t^+$ and $H'_t\in\ddot\Sigma_t$
(455Lb).

\medskip

\quad{\bf (ii)} If $\omega$, $\omega'\in\Cdlg$ and $s>\tau(\omega)$
are such that
$\omega'\restr[0,s]=\omega\restr[0,s]$, then
$\tau(\omega')=\tau(\omega)$.   \Prf\ $H_{\tau(\omega)}$,
$H'_{\tau(\omega)}$ and their difference belong to $\ddot\Sigma_s$,
so are determined (relative to $\Cdlg$) by coordinates in $[0,s]$;
since $H_{\tau(\omega)}\setminus H'_{\tau(\omega)}$ contains $\omega$, it
also contains $\omega'$, and $\tau(\omega')=\tau(\omega)$.\ \Qed

\medskip

\quad{\bf (iii)} If $f$ is $\ddot\mu$-integrable, $\alpha\in\Bbb R$ and
$s>0$, then $E(f,\alpha)\cap H'_s\in\ddot\Sigma_s$.
\Prf\ Certainly $E(f,\alpha)\cap H'_s\in\ddot\Sigma$.   If $\omega$,
$\omega'\in\Cdlg$ and $\omega\restr[0,s]=\omega'\restr[0,s]$,
then

$$\eqalignno{\omega\in E(f,\alpha)\cap H'_s
&\Longrightarrow\tau(\omega)<s\text{ and }
  \int_{\Cdlg}fd\ddot\mu_{\omega,\tau(\omega)}\le\alpha\cr
&\Longrightarrow\tau(\omega')=\tau(\omega)<s\text{ and }
  \int_{\Cdlg}fd\ddot\mu_{\omega,\tau(\omega')}\le\alpha\cr
\displaycause{by (ii)}
&\Longrightarrow\omega'\in E(f,\alpha)\cap H'_s.\cr}$$

\noindent So $E(f,\alpha)\cap H'_s$ is determined (relative to $\Cdlg$) by
coordinates in $[0,s]$ and belongs to $\ddot\Sigma_s$.\ \Qed

Consequently $E(f,\alpha)\cap H_t\in\ddot\Sigma^+_t$ for every $t\ge 0$.
\Prf\ $H_t=\bigcap_{n\in\Bbb N}H'_{t_n}$ where $t_n=t+2^{-n}$ for each $n$,
so

\Centerline{$E(f,\alpha)\cap H_t
=\bigcap_{n\ge m}E(f,\alpha)\cap H'_{t_n}$}

\noindent belongs to $\ddot\Sigma_{t_m}$ for every $m\in\Bbb N$, and
$E(f,\alpha)\cap H_t\in\ddot\Sigma^+_t$.\ \Qed

Thus $E(f,\alpha)\in\ddot\Sigma^+_{\tau}$ for every $\alpha$.   As $\alpha$ is
arbitrary, $\dom\ddot g_f\in\ddot\Sigma^+_{\tau}$ and $\ddot g_f$ is
$\ddot\Sigma^+_{\tau}$-measurable.

\medskip

\quad{\bf (iv)} Define $\sequencen{\tau_n}$ as in (a-iv) above, so that
each $\tau_n$ is a stopping time adapted to
$\langle\ddot\Sigma_t\rangle_{t\ge 0}$ and $\sequencen{\tau_n(\omega)}$ is a
non-increasing sequence with limit $\tau(\omega)$ for every $\omega$.
For a $\ddot\mu$-integrable real-valued function $f$ on $\Cdlg$,
$\omega\in\Cdlg$ and $n\in\Bbb N$, set

\Centerline{$\ddot g^{(n)}_f(\omega)
=\int_{\Cdlg}fd\ddot\mu_{\omega,\tau_n(\omega)}$}

\noindent whenever the right-hand side is defined in $\Bbb R$.
By (a), $\int_{\Cdlg}\ddot g^{(n)}_fd\ddot\mu=\int_{\Cdlg}fd\ddot\mu$.
We have seen also, in
(a-iii), that each $\tau_n$ has an extension $\grave\tau_n$ which is a
stopping time on $\Omega$ of the type considered in 455Ec.   So if we take
a $\hat\mu$-integrable function $\tilde f$ extending $f$, and set

\Centerline{$g^{(n)}_{\tilde f}(\omega)
=\int_{\Omega}\tilde fd\hat\mu_{\omega,\grave\tau_n(\omega)}$}

\noindent whenever $\omega\in\Omega$ is such that the integral is defined
in $\Bbb R$, $g^{(n)}_{\tilde f}$ will be a conditional expectation
of $\tilde f$ on $\Sigma_{\grave\tau_n}$, the algebra of sets
$E\in\Sigma$ such that
$E\cap\{\omega:\grave\tau_n(\omega)\le t\}$ is determined by coordinates in
$[0,t]$ for every $t\ge 0$.

If $\omega\in\Cdlg$, then $\Cdlg$ has full outer measure for
$\hat\mu_{\omega,\tau_n(\omega)}=\hat\mu_{\omega,\grave\tau_n(\omega)}$, so

\Centerline{$g^{(n)}_{\tilde f}(\omega)
=\int_{\Omega}\tilde fd\hat\mu_{\omega,\grave\tau_n(\omega)}
=\int_{\Cdlg} fd\ddot\mu_{\omega,\tau_n(\omega)}
=\ddot g^{(n)}_f(\omega)$}

\noindent whenever either is defined.

\medskip

\quad{\bf (v)} Set

\Centerline{$\ddot\Sigma_{\tau_n}
=\{F:F\in\ddot\Sigma$, $F\cap\{\omega:\tau_n(\omega)\le t\}\in\ddot\Sigma_t$
for every $t\ge 0\}$.}

\noindent Then every $F\in\ddot\Sigma_{\tau_n}$ is of the form
$\tilde F\cap\Cdlg$ where $\tilde F\in\Sigma_{\grave\tau_n}$.
\Prf\ Recall that $\tau_n$ and $\grave\tau_n$ take values in
$D_n\cup\{\infty\}$, where $D_n=\{2^{-n}i:i\in\Bbb N\}$.   For each
$i\in\Bbb N$, set $F_i=\{\omega:\omega\in F$, $\tau_n(\omega)=2^{-n}i\}$;
then $F_i\in\ddot\Sigma_{2^{-n}i}$, so there is an
$E_i\in\Sigma_{2^{-n}i}$ such that $F_i=E_i\cap\Cdlg$ (a-ii).
Let $E_{\infty}\in\Sigma$ be such that
$E_{\infty}\cap\Cdlg=\{\omega:\tau_n(\omega)=\infty\}$, and try

\Centerline{$\tilde F
=\bigcup_{i\in\Bbb N}(E_i\cap\grave\tau_n^{-1}[\{2^{-n}i\}])
   \cup(E_{\infty}\cap\grave\tau_n^{-1}[\{\infty\}])$.}

\noindent Then $\tilde F\cap\Cdlg=F$ (because $\grave\tau_n$ extends
$\tau_n$) and $\tilde F\in\Sigma_{\grave\tau_n}$ (because

\Centerline{$\tilde F\cap\grave\tau_n^{-1}[\{2^{-n}i\}]
=E_i\cap\grave\tau_n^{-1}[\{2^{-n}i\}]\in\Sigma_{2^{-n}i}$}

\noindent for every $i$).\ \Qed

\medskip

\quad{\bf (vi)} If $f$ is $\ddot\mu$-integrable, then $\ddot g^{(n)}_f$ is a
conditional expectation of $f$ on $\ddot\Sigma_{\tau_n}$ for every $n$.
\Prf\ Take $F\in\ddot\Sigma_{\tau_n}$.   Then there are an
$\tilde F\in\Sigma_{\grave\tau_n}$ such that
$F=\tilde F\cap\Cdlg$, and a $\hat\mu$-integrable $\tilde f$ such that
$f=\tilde f\restr\Cdlg$.   So

$$\eqalignno{\int_Ffd\ddot\mu
&=\int_{\tilde F}\tilde fd\hat\mu
=\int_{\tilde F}g^{(n)}_{\tilde f}d\hat\mu\cr
\displaycause{455E(c-ii)}
&=\int_{\tilde F}\int_{\Omega}
  \tilde fd\hat\mu_{\omega,\grave\tau_n(\omega)}
  \hat\mu(d\omega)
=\int_F\int_{\Omega}
  \tilde fd\hat\mu_{\omega,\tau_n(\omega)}\ddot\mu(d\omega)\cr
\displaycause{because $\hat\mu^*\Cdlg=1$, $F=\tilde F\cap\Cdlg$ and
$\tau_n=\grave\tau_n\restr\Cdlg$}
&=\int_F\int_{\Cdlg}fd\ddot\mu_{\omega,\tau_n(\omega)}\ddot\mu(d\omega)
=\int_F\ddot g^{(n)}_fd\ddot\mu.  \text{ \Qed}\cr}$$

\medskip

\quad{\bf (vii)} Let $\Phi$ be the set of those $\ddot\mu$-integrable
real-valued functions $f$ such that
\penalty-100$\lim_{n\to\infty}
\int_{\Cdlg}|\ddot g_f-\ddot g^{(n)}_f|d\ddot\mu=0$.
For $J\subseteq\coint{0,\infty}$ let $\pi_J:\Omega\to U^J$ be the
restriction map.
By (a-v), $f=h\pi_J\restr\Cdlg$ belongs to $\Phi$ whenever
$J\subseteq\coint{0,\infty}$ is finite and
$h:U^J\to\Bbb R$ is bounded and uniformly continuous, since in this case

$$\eqalign{\ddot g_f(\omega)
&=\int_{\Cdlg}h\pi_Jd\ddot\mu_{\omega,\tau(\omega)}
=\int_{\Omega}h\pi_Jd\hat\mu_{\omega,\tau(\omega)}
=\lim_{n\to\infty}\int_{\Omega}h\pi_Jd\hat\mu_{\omega,\tau_n(\omega)}\cr
&=\lim_{n\to\infty}\int_{\Cdlg}h\pi_Jd\ddot\mu_{\omega,\tau_n(\omega)}
=\lim_{n\to\infty}\ddot g^{(n)}_f(\omega)\cr}$$

\noindent for every
$\omega\in\Cdlg$.   Next, $\ddot g_{\alpha f}\eae\alpha\ddot g_f$,
$\ddot g_{f+f'}\eae\ddot g_f+\ddot g_{f'}$ and

\Centerline{$\int_{\Cdlg}|\ddot g_f-\ddot g_{f'}|d\ddot\mu
=\int_{\Cdlg}|\ddot g_{f-f'}|d\ddot\mu
\le\int_{\Cdlg}\ddot g_{|f-f'|}d\ddot\mu
=\int_{\Cdlg}|f-f'|d\ddot\mu$}

\noindent for all $f$, $f'\in\eusm L^1(\ddot\mu)$ and $\alpha\in\Bbb R$;
and we have similar expressions for every $\ddot g^{(n)}_f$.   So $f+f'\in\Phi$
and $\alpha f\in\Phi$ whenever $f$, $f'\in\Phi$, and moreover
$f\in\Phi$ whenever $f\in\eusm L^1(\ddot\mu)$ and there is a sequence
$\sequence{f}{f_k}$ in $\Phi$ such that
$\lim_{k\to\infty}\int_{\Cdlg}|f-f_k|d\ddot\mu=0$.

If $J\subseteq\coint{0,\infty}$ is finite and $G\subseteq U^J$,
then $(\chi G)\pi_J\restr\Cdlg\in\Phi$.   \Prf\ There is a
non-decreasing sequence $\sequence{k}{h_k}$ of bounded uniformly continuous
functions on $U^J$ with limit $\chi G$;  now $h_k\pi_J\restr\Cdlg\in\Phi$ and
$(\chi G)\pi_J\restr\Cdlg=\lim_{k\to\infty}h_k\pi_J\restr\Cdlg$.\ \QeD\  By the Monotone Class
Theorem, $(\chi E)\pi_J\restr\Cdlg\in\Phi$
whenever $J\subseteq\coint{0,\infty}$ is
finite and $E\in\Cal B(U^J)$.   By the Monotone Class Theorem again,
$\chi(E\cap\Cdlg)\in\Phi$ whenever
$E\in\Tensorhat_{\coint{0,\infty}}\Cal B(U)$.   Since we surely have
$f'\in\Phi$ whenever $f\in\Phi$ and $f'=f\,\,\ddot\mu$-a.e.,
$\chi(E\cap\Cdlg)\in\Phi$ whenever $E\in\Sigma$, that is,
$\chi E\in\Phi$ for every $E\in\ddot\Sigma$.   It follows at once that
$\Phi=\eusm L^1(\ddot\mu)$.

\medskip

\quad{\bf (viii)} We are nearly home.   Suppose that
$f\in\eusm L^1(\ddot\mu)$ and $F\in\ddot\Sigma^+_{\tau}$.   If $n\in\Bbb N$,
then $F\in\ddot\Sigma_{\tau_n}$.   \Prf\ For any $t>0$,

\Centerline{$F\cap\{\omega:\tau(\omega)<t\}
=\bigcup_{q\in\Bbb Q,q<t}F\cap\{\omega:\tau(\omega)\le q\}
\in\ddot\Sigma_t$.}

\noindent So, for any $i\in\Bbb N$,

\Centerline{$F\cap\{\omega:\tau_n(\omega)\le 2^{-n}i\}
=F\cap\{\omega:\tau(\omega)<2^{-n}i\}
\in\ddot\Sigma_{2^{-n}i}$.  \text{\Qed}}

\noindent So $\int_F\ddot g^{(n)}_fd\ddot\mu=\int_Ffd\ddot\mu$.
But $f\in\Phi$, so

\Centerline{$\int_F\ddot g_fd\ddot\mu=\lim_{n\to\infty}\int_F\ddot g^{(n)}_fd\ddot\mu
=\int_Ffd\ddot\mu$.}

\noindent Since we already know, from (iii) above,
that $\dom\ddot g_f\in\ddot\Sigma^+_{\tau}$ and
$\ddot g_f$ is $\ddot\Sigma^+_{\tau}$-measurable, 
$\ddot g_f$ is a conditional expectation
of $f$ on $\ddot\Sigma^+_{\tau}$, as claimed.

\medskip

{\bf (c)(i)} By 455L(c-iii) again,
$\ddot\Sigma_{\tau}$ is a $\sigma$-algebra.

\medskip

\quad{\bf (ii)} If $\omega$, $\omega'\in\Cdlg$ and
$\omega'\restr[0,\tau(\omega)]=\omega\restr[0,\tau(\omega)]$ then
$\ddot\mu_{\omega',\tau(\omega')}=\ddot\mu_{\omega,\tau(\omega)}$.   \Prf\
Set $t=\tau(\omega)$.   This time, $H_t$ and $H'_t$, defined as in
(b-i), belong to $\ddot\Sigma_t$, so their difference belongs to
$\ddot\Sigma_t$ and
is determined (relative to $\Cdlg$) by coordinates in $[0,t]$;  so
$\omega'\in H_t\setminus H'_t$ and $\tau(\omega')=t$.  Now, reading off the
definition in 455Eb, $\nu^{(s,u)}_{\omega'tx}=\nu^{(s,u)}_{\omega tx}$ for
all $s$, $u$ and $x$, so $\ddot\mu_{\omega't}=\ddot\mu_{\omega t}$.\ \Qed

\medskip

\quad{\bf (iii)} It follows that if $f\in\eusm L^1(\ddot\mu)$
and $\alpha\in\Bbb R$ then
$F=\{\omega:\omega\in\Cdlg$, $\ddot g_f(\omega)$ is defined and at most
$\alpha\}$ belongs to $\ddot\Sigma_{\tau}$.
\Prf\ We know from (b-iii) that $F\in\ddot\Sigma$.   If $t\ge 0$,
$\omega\in F$, $\omega'\in\Cdlg$,
$\tau(\omega)\le t$ and $\omega'\restr[0,t]=\omega\restr[0,t]$ then
$\ddot\mu_{\omega',\tau(\omega')}=\ddot\mu_{\omega,\tau(\omega)}$, so
$\ddot g_f(\omega')=\ddot g_f(\omega)$ and $\omega'\in F$.   Thus
$F\cap\{\omega:\tau(\omega)\le t\}$ is determined (relative to $\Cdlg$)
by coordinates in $[0,t]$
and belongs to $\ddot\Sigma_t$.\ \Qed

\medskip

\quad{\bf (iv)} Thus $\dom\ddot g_f\in\ddot\Sigma_{\tau}$ and $\ddot g_f$ is
$\ddot\Sigma_{\tau}$-measurable.   As we already know that it is a
conditional
expectation of $f$ on $\ddot\Sigma_{\tau}^+\supseteq\ddot\Sigma_{\tau}$, it is a
conditional expectation of $f$ on $\ddot\Sigma_{\tau}$.
}%end of proof of 455O

\leader{455P}{}\cmmnt{ The eventual objective of this section is to
provide a foundation for study of the original,
and still by far the most important, example of a
continuous-time Markov process, Brownian motion.   In the language
developed above, we shall
have $U=\Bbb R$ (or, when we come to the applications
in \S\S477-479, $U=\BbbR^r$), and all the transitional probabilities
$\nu_x^{(s,t)}$ will be Gaussian.   But the techniques so far developed
can tell us a great deal about much more general processes with
some of the same features.

\medskip

\noindent}{\bf Theorem} Let $U$ be a metrizable
topological group which is complete
under a \rti\ metric $\rho$ inducing its topology.
Let $\langle\lambda_t\rangle_{t>0}$
be a family of Radon probability measures
on $U$ such that $\lambda_s*\lambda_t=\lambda_{s+t}$ for all $s$, $t>0$.
Suppose that $\lim_{t\downarrow 0}\lambda_tG=1$ for every open
neighbourhood $G$ of the identity in $U$.
For $x\in U$ and $0\le s<t$, let $\nu^{(s,t)}_x$ be the Radon probability
measure on $U$ defined by saying that
$\nu^{(s,t)}_x(E)=\lambda_{t-s}(Ex^{-1})$ whenever $\lambda_{t-s}$ measures
$Ex^{-1}$.

(a) $\family{y}{U}{\nu^{(t,u)}_y}$ is a disintegration of $\nu^{(s,u)}_x$
over $\nu^{(s,t)}_x$ whenever $x\in U$ and $0\le s<t<u$.

(b) $\langle\nu^{(s,t)}_x\rangle_{0\le s<t,x\in U}$ is narrowly continuous
and uniformly time-continuous on the right.

(c)(i) For any $x^*\in U$, we can define a complete measure
$\hat\mu$ on $U^{\coint{0,\infty}}$ by the method of 455E applied to
$x^*$ and $\langle\nu^{(s,t)}_x\rangle_{0\le s<t,x\in U}$.

\quad(ii) If $\Cdlg$ is the space of \cadlag\ functions from
$\coint{0,\infty}$ to $U$, then $\hat\mu^*\Cdlg=1$, and the subspace
measure $\ddot\mu$ on $\Cdlg$ will have the properties described in 455O.

\quad(iii) $\hat\mu$ has a unique extension to a Radon measure $\tilde\mu$
on $U^{\coint{0,\infty}}$.

\proof{{\bf (a)} Note first that $y\mapsto yx$ is \imp\ for $\lambda_{t-s}$
and $\nu^{(s,t)}_x$, so that
$\int f(y)\nu_x^{(s,t)}(dy)=\int f(yx)\lambda_{t-s}(dy)$
for any real-valued function on $U$ for which either is defined
(235Gb).   If $E\subseteq U$ is measured by $\nu^{(s,u)}_x$, then

$$\eqalignno{\nu^{(s,u)}_x(E)
&=\lambda_{u-s}(Ex^{-1})
=(\lambda_{u-t}*\lambda_{t-s})(Ex^{-1})
=\int\lambda_{u-t}(Ex^{-1}y^{-1})\lambda_{t-s}(dy)\cr
\displaycause{444A}
&=\int\nu_{yx}^{(t,u)}(E)\lambda_{t-s}(dy)
=\int\nu_y^{(t,u)}(E)\nu_x^{(s,t)}(dy);\cr}$$

\noindent as $E$ is arbitrary,
$\family{y}{U}{\nu^{(t,u)}_y}$ is a disintegration of $\nu^{(s,u)}_x$
over $\nu^{(s,t)}_x$.

\medskip

{\bf (b)(i)}\grheada\
Suppose that $x\in U$ and $0\le s<t$;  set $u=t-s$.
Let $f:U\to\Bbb R$ be
a bounded continuous function and set $M=\|f\|_{\infty}$.   Take
$\epsilon\in\ooint{0,1}$.   Let $K\subseteq U$ be a compact set such that
$\lambda_uK\ge 1-\epsilon$.   Then there is a
symmetric open neighbourhood $V$
of the identity $e$ of $U$ such that
$|f(wyx^{-1})-f(wx^{-1})|\le 2\epsilon$
whenever $w\in K$ and $y\in V^2$.   \Prf\ For each $w\in K$ there is a
neighbourhood $W_w$ of $e$ such that $|f(wyx^{-1})-f(wx^{-1})|\le\epsilon$
whenever $y\in W_w^2$.   Because $K$ is compact, there are
$w_0,\ldots,w_n\in K$ such that $K\subseteq\bigcup_{i\le n}w_iW_{w_i}$;
set $W=\bigcap_{i\le n}W_{w_i}$.   If $w\in K$ and $y\in W$, there is an
$i\le n$ such that $w\in w_iW_{w_i}$, in which case

$$\eqalign{|f(wyx^{-1})-f(wx^{-1})|
&\le|f(w_i(w_i^{-1}wy)x^{-1})-f(w_ix^{-1})|\cr
&\mskip200mu
     +|f(w_ix^{-1})-f(w_i(w_i^{-1}w)x^{-1})|\cr
&\le 2\epsilon\cr}$$

\noindent because both $w_i^{-1}wy$ and $w_i^{-1}w$ belong to $W_{w_i}^2$.
So if we take a symmetric open neighbourhood $V$ of $e$ such that
$V^2\subseteq W$, this will serve.\ \Qed

\medskip

\qquad\grheadb\
Let $\delta>0$ be such that $\lambda_vV\ge 1-\epsilon$ whenever
$0<v\le 2\delta$.   It will be worth noting that
$\lambda_v(KV)\ge 1-2\epsilon$ whenever $0<v<u$ and $u-v\le 2\delta$.
\Prf\ In this case, $\lambda_u=\lambda_v*\lambda_{u-v}$.   Now
$U\setminus K\supseteq(U\setminus KV)V^{-1}$.   So

\Centerline{$\epsilon
\ge\lambda_u(U\setminus K)
\ge\lambda_v(U\setminus KV)\lambda_{u-v}(V^{-1})
\ge(1-\epsilon)\lambda_v(U\setminus KV)$}

\noindent and

\Centerline{$\lambda_v(KV)\ge 1-\Bover{\epsilon}{1-\epsilon}
\ge 1-2\epsilon$.  \Qed}

\medskip

\qquad\grheadc\
Suppose that $0\le s'<t'$ and $y\in U$ are such that
$y^{-1}x\in W$, $|s'-s|\le\delta$ and $|t'-t|\le\delta$.   Then
$|\int fd\nu_y^{(s',t')}-\int fd\nu_x^{(s,t)}|\le(6M+4)\epsilon$.
\Prf\ Set $u'=t'-s'$, so that $|u-u'|\le 2\delta$.   We have

\Centerline{$|\int fd\nu_y^{(s',t')}-\int fd\nu_x^{(s,t)}|
=|\int f(wy^{-1})\lambda_{u'}(dw)-\int f(wx^{-1})\lambda_u(dw)|$.}

\medskip

{\bf case 1} Suppose that $u'<u$.   Then
$\lambda_u=\lambda_{u'}*\lambda_{u-u'}$, so

$$\eqalignno{|\int fd\nu_y^{(s',t')}-\int fd\nu_x^{(s,t)}|
&=|\int f(wy^{-1})\lambda_{u'}(dw)
   -\int f(wx^{-1})(\lambda_{u'}*\lambda_{u-u'})(dw)|\cr
&=|\int f(wy^{-1})\lambda_{u'}(dw)
   -\iint f(wzx^{-1})\lambda_{u-u'}(dz)\lambda_{u'}(dw)|\cr
\displaycause{444C}
&\le\int|f(wy^{-1})-\int f(wzx^{-1})\lambda_{u-u'}(dz)|\lambda_{u'}(dw)\cr
&\le 4M\epsilon
   +\sup_{w\in KV}|f(wy^{-1})-\int f(wzx^{-1})\lambda_{u-u'}(dz)|\cr
\displaycause{because $\lambda_{u'}(U\setminus KV)\le 2\epsilon$, by
($\beta$), and
$|f(wy^{-1})-\int f(wzx^{-1})\lambda_{u-u'}(dz)|\le 2M$ for every $w$}
&\le 4M\epsilon
   +\sup_{w\in KV}\int|f(wy^{-1})-\int f(wzx^{-1})|\lambda_{u-u'}(dz)\cr
&\le 6M\epsilon
   +\sup_{w\in KV,z\in V}|f(wy^{-1})-\int f(wzx^{-1})|\cr
\displaycause{because $\lambda_{u-u'}V\ge 1-\epsilon$}
&\le 6M\epsilon
   +\sup_{w\in K,v\in V,z\in V}|f(wvy^{-1}xx^{-1})-f(wvzx^{-1})|\cr
&\le 6M\epsilon
   +\sup_{w\in K,v\in V,z\in V}\bigl(|f(wvy^{-1}xx^{-1})-f(wx^{-1})|\cr
&\mskip220mu
         +|f(wvzx^{-1})-f(wx^{-1})|\bigr)\cr
&\le 6M\epsilon+4\epsilon\cr}$$

\noindent by the choice of $V$, because $y^{-1}x\in V$.

\medskip

{\bf case 2} Suppose that $u'=u$.   Then

$$\eqalign{|\int fd\nu_y^{(s',t')}-\int fd\nu_x^{(s,t)}|
&\le\int|f(wy^{-1})-f(wx^{-1})|\lambda_u(dw)\cr
&\le 2M\epsilon+\sup_{w\in K}|f(wy^{-1}xx^{-1})-f(wx^{-1})|\cr
&\le 2M\epsilon+\epsilon.\cr}$$

\medskip

{\bf case 3} Suppose that $u'>u$.   Then
$\lambda_{u'}=\lambda_u*\lambda_{u'-u}$, so

$$\eqalignno{|\int fd\nu_y^{(s',t')}-\int fd\nu_x^{(s,t)}|
&=|\iint f(wzy^{-1})\lambda_{u'-u}(dz)\lambda_u(dw)
   -\int f(wx^{-1})\lambda_u(dw)|\cr
&\le\int|\int f(wzy^{-1})\lambda_{u'-u}(dz)-f(wx^{-1})|\lambda_u(dw)\cr
&\le 2M\epsilon+\sup_{w\in K}
   |\int f(wzy^{-1})\lambda_{u'-u}(dz)-f(wx^{-1})|\cr
&\le 2M\epsilon+2M\epsilon
   +\sup_{w\in K,z\in V}|f(wzy^{-1})-f(wx^{-1})|\cr
&\le 4M\epsilon+2\epsilon.\cr}$$

\noindent So we have the result in all cases.\ \Qed

\medskip

\qquad\grheadd\ As $s$, $t$, $x$, $\epsilon$ and $f$ are arbitrary,
$\langle\nu^{(s,t)}_x\rangle_{0\le s<t,x\in U}$ is narrowly (= vaguely)
continuous.

\medskip

\quad{\bf (ii)} Given $\epsilon>0$, there is a $\delta>0$ such that
$\lambda_t\{x:\rho(x,e)<\epsilon\}\ge 1-\epsilon$ whenever $0<t\le\delta$.
Now suppose that $x\in U$ and $0\le s<t\le s+\delta$.   Then

$$\eqalignno{\nu^{(s,t)}_xB(x,\epsilon)
&=\lambda_{t-s}(B(x,\epsilon)x^{-1})
=\lambda_{t-s}B(e,\epsilon)\cr
\displaycause{because $\rho$ is \rti}
&\ge 1-\epsilon.\cr}$$

\noindent As $\epsilon$ is arbitrary,
$\langle\nu^{(s,t)}_x\rangle_{0\le s<t,x\in U}$ is uniformly
time-continuous on the right.

\medskip

{\bf (c)} This is now just a matter of putting 455O and 455H
together.
}%end of proof of 455P

\leader{455Q}{L\'evy processes}
\cmmnt{If we approach as
probabilists, without prejudices in favour of any particular realization,
the processes in 455P manifest themselves as follows.}
Let $U$ be a separable metrizable
topological group with identity $e$, and consider the following list of
properties of a family
$\langle X_t\rangle_{t\ge 0}$ of $U$-valued random variables:

\inset{$X_0=e$ almost everywhere,}

\inset{$\Pr(X_tX_s^{-1}\in F)=\Pr(X_{t-s}\in F)$ whenever $0\le s<t$ and
$F\subseteq U$ is Borel}

\noindent (the process is {\bf stationary}),

\inset{whenever $0\le t_0<t_1<\ldots<t_n$, then
$X_{t_1}X_{t_0}^{-1},X_{t_2}X_{t_1}^{-1},\ldots,X_{t_n}X_{t_{n-1}}^{-1}$
are independent}

\noindent\cmmnt{ in the sense of 418U }(the process has
{\bf independent increments}),

\inset{$X_t\to e$ in measure as $t\downarrow 0$}

\noindent (that is, $\lim_{t\downarrow 0}\Pr(X_t\in G)=1$ for every
neighbourhood $G$ of the identity).
\cmmnt{I say here that $U$ should be
separable and metrizable
in order to ensure that all the functions $X_tX_s^{-1}$
should be measurable (of course it will be enough if $U$ is metrizable and
of measure-free weight, as in 438E).}   Such a family I will call a
{\bf L\'evy process}.

\leader{455R}{Theorem} Let $U$ be a Polish group
with identity $e$
which is complete under a \rti\ metric inducing its topology.
A family $\langle X_t\rangle_{t\ge 0}$
of $U$-valued random variables is a L\'evy process
iff there is a family $\langle\lambda_t\rangle_{t>0}$ of Radon probability
measures on
$U$, satisfying the conditions of 455P, such that if we start from $x^*=e$
and build the measure $\hat\mu$ on $U^{\coint{0,\infty}}$ as in 455P, then

\Centerline{$\Pr(X_{t_i}\in F_i$ for every $i\le n)
=\hat\mu\{\omega:\omega(t_i)\in F_i$ for every $i\le n\}$}

\noindent whenever $t_0,\ldots,t_n\in\coint{0,\infty}$ and $F_i\subseteq U$
is a Borel set for every $i\le n$.

\proof{{\bf (a)} Suppose we have a family
$\langle\lambda_t\rangle_{t>0}$ of Radon probability measures on $U$
such that $\lambda_s*\lambda_t=\lambda_{s+t}$ for all $s$, $t>0$ and
$\lim_{t\downarrow 0}\lambda_tG=1$ for every open
neighbourhood $G$ of $e$ in $U$.   Define
$\langle\nu_x^{(s,t)}\rangle_{0\le s<t,x\in U}$ as in 455P,
and let $\hat\mu$ be the corresponding completed measure on
$\Omega=U^{\coint{0,\infty}}$ as in 455Pc, starting from
$x^*=e$.   Set $X_t(\omega)=\omega(t)$ for $t\ge 0$ and $\omega\in\Omega$.
Then $X_0=e$ a.e.\ (455Ea) and

\Centerline{$\Pr(X_t\in F)=\hat\mu X_t^{-1}[F]=\nu^{(0,t)}_0F=\lambda_tF$}

\noindent for $t>0$ and $F\in\Cal B(U)$ (455Ea again).   In particular,

\Centerline{$\lim_{t\downarrow 0}\Pr(X_t\in G)
=\lim_{t\downarrow 0}\lambda_tG=1$}

\noindent for every neighbourhood $G$ of the identity.
If $0<s<t$ and $F\in\Cal B(U)$, set
$H=\{(e,x,y):yx^{-1}\in F\}\subseteq U^3$.   Then

$$\eqalignno{\Pr(X_tX_s^{-1}\in F)
&=\hat\mu\{\omega:(\omega(0),\omega(s),\omega(t))\in H\}\cr
&=\iint\chi H(e,x,y)\nu_x^{(s,t)}(dy)\nu_e^{(0,s)}(dx)\cr
\displaycause{455E}
&=\iint\chi H(e,x,yx)\lambda_{t-s}(dy)\lambda_s(dx)\cr
&=\iint\chi F(y)\lambda_{t-s}(dy)\lambda_s(dx)
=\lambda_{t-s}(F)
=\Pr(X_{t-s}\in F).\cr}$$

\noindent If $0=s<t$ then $X_tX_s^{-1}\eae X_t=X_{t-s}$, of course.
If $0=t_0<t_1<\ldots<t_n$ and
$F_0,\ldots,F_{n-1}\in\Cal B(U)$, set

\Centerline{$E_k
=\{\omega:\omega\in\Omega$, $\omega(t_{i+1})\omega(t_i)^{-1}\in F_i$
  for every $i<k\}$,}

\Centerline{$H_k=\{(x_0,\ldots,x_k):x_{i+1}x_i^{-1}\in F_i$ for every
$i<k\}\subseteq U^{k+1}$}

\noindent for $k\le n$.   Then

\Centerline{$\hat\mu E_1
=\hat\mu\{\omega:\omega(t_1)\omega(0)^{-1}\in F_0\}
=\hat\mu\{\omega:\omega(t_1)\in F_0\}
=\nu_e^{(0,t_1)}F_0
=\lambda_{t_1}F_0$,}

\noindent and for $k\ge 2$

$$\eqalignno{\Pr(X_{t_{i+1}}X_{t_i}^{-1}&\in F_i\text{ for every }i<k)\cr
&=\hat\mu E_k
=\int\ldots\int\chi H_k(e,x_1,\ldots,x_k)
    \nu_{x_{k-1}}^{(t_{k-1},t_k)}(dx_k)
    \ldots\nu_e^{(0,t_1)}(dx_1)\cr
\displaycause{455E}
&=\int\ldots\iint\chi H_k(e,x_1,\ldots,x_{k-1},x_kx_{k-1})
    \lambda_{t_k-t_{k-1}}(dx_k)\cr
&\mskip100mu
    \nu_{x_{k-2}}^{(t_{k-2},t_{k-1})}(dx_{k-1})
    \ldots\nu_e^{(0,t_1)}(dx_1)\cr
&=\int\ldots\iint\chi H_{k-1}(e,x_1,\ldots,x_{k-1})
    \chi F_{k-1}(x_k)
    \lambda_{t_k-t_{k-1}}(dx_k)\cr
&\mskip100mu
    \nu_{x_{k-2}}^{(t_{k-2},t_{k-1})}(dx_{k-1})
    \ldots\nu_e^{(0,t_1)}(dx_1)\cr
&=\lambda_{t_k-t_{k-1}}(F_{k-1})
    \int\ldots\int\chi H_{k-1}(e,x_1,\ldots,x_{k-1})\cr
&\mskip100mu
    \nu_{x_{k-2}}^{(t_{k-2},t_{k-1})}(dx_{k-1})
    \ldots\nu_e^{(0,t_1)}(dx_1)\cr
&=\lambda_{t_k-t_{k-1}}(F_{k-1})\cdot\hat\mu E_{k-1}.\cr}$$

\noindent So

$$\eqalign{\Pr(X_{t_{i+1}}X_{t_i}^{-1}&\in F_i\text{ for every }i<n)
=\hat\mu E_n
=\prod_{i=0}^{n-1}\lambda_{t_i-t_{i-1}}F_i\cr
&=\prod_{i=0}^{n-1}\Pr(X_{t_i-t_{i-1}}\in F_i)
=\prod_{i=0}^{n-1}\Pr(X_{t_i}X_{t_{i-1}}^{-1}\in F_i).\cr}$$

\noindent As $F_0,\ldots,F_{n-1}$ are arbitrary,
$X_{t_1}X_{t_0}^{-1},X_{t_2}X_{t_1}^{-1},\ldots,X_{t_n}X_{t_{n-1}}^{-1}$
are independent.   Thus all the conditions of 455Q are satisfied.

\medskip

{\bf (b)(i)} In the other direction, given a family
$\langle X_t\rangle_{t\ge 0}$ with the properties listed in 455Q, then for
each $t>0$ there is a Radon measure
$\lambda_t$ on $U$ such that $\lambda_tF=\Pr(X_t\in F)$
for every $F\in\Cal B(U)$, for each $t>0$.   \Prf\ $U$ is Polish, therefore
analytic, and we can apply 433Cb to the Borel measure
$F\mapsto\Pr(X_t\in F)$.\ \QeD\   If $s$, $t>0$, then
the distribution of $X_{s+t}X_s^{-1}$ is the same as the distribution of
$X_t$, so is $\lambda_t$.

If $s$, $t>0$ then $\lambda_{s+t}=\lambda_s*\lambda_t$.   \Prf\ If
$F_1$, $F_2\in\Cal B(U)$ then

$$\eqalignno{\Pr((X_t,X_{s+t}X_t^{-1})\in F_1\times F_2)
&=\Pr(X_t\in F_1,X_{s+t}X_t^{-1}\in F_2)\cr
&=\Pr(X_t\in F_1)\Pr(X_{s+t}X_t^{-1}\in F_2)\cr
\displaycause{because $X_t$ and $X_{s+t}X_t^{-1}$ are independent}
&=\lambda_tF_1\cdot\lambda_sF_2
=(\lambda_t\times\lambda_s)(F_1\times F_2).\cr}$$

\noindent By the Monotone Class Theorem, or otherwise,

\Centerline{$\Pr((X_t,X_{s+t}X_t^{-1})\in H)
=(\lambda_t\times\lambda_s)H$}

\noindent for every $H\in\Cal B(U)\tensorhat\Cal B(U)=\Cal B(U^2)$.   So if
$F\in\Cal B(U)$ we shall have

$$\eqalign{(\lambda_s*\lambda_t)(F)
&=(\lambda_s\times\lambda_t)\{(x,y):xy\in F\}
=(\lambda_t\times\lambda_s)\{(y,x):xy\in F\}\cr
&=\Pr(X_{s+t}X_t^{-1}X_t\in F)
=\lambda_{s+t}F,\cr}$$

\noindent and $\lambda_s*\lambda_t=\lambda_{s+t}$.  (Cf.\
272T\formerly{2{}72S}.)\ \Qed

Next, for any neighbourhood $G$ of $e$,

\Centerline{$\lim_{t\downarrow 0}\lambda_tG
=\lim_{t\downarrow 0}\Pr(X_t\in G)
=1$.}

\noindent So $\langle\lambda_t\rangle_{t>0}$ satisfies the conditions of
455P.   Let $\hat\mu$ be the corresponding completed measure on
$U^{\coint{0,\infty}}$ as in 455Pc.

\medskip

\quad{\bf (ii)} $\Pr(X_{t_i}\in F_i$ for every $i\le k)
=\hat\mu\{\omega:\omega(t_i)\in F_i$ for every $i\le k\}$
whenever $t_0,\ldots,t_n\in\coint{0,\infty}$ and $F_i\in\Cal B(U)$
for every $i\le k$.   \Prf\ It is enough to consider the case
$0=t_0<t_1<\ldots<t_n$.   In this case, whenever
$E_0,\ldots,E_n\in\Cal B(U)$,

$$\eqalignno{
\Pr((X_{t_0},&X_{t_1}X_{t_0}^{-1},\ldots,X_{t_n}X_{t_{n-1}}^{-1})\in
  E_0\times\ldots\times E_n)\cr
&=\Pr(X_{t_0}\in E_0)\Pr(X_{t_1}X_{t_0}^{-1}\in E_1)\ldots
  \Pr(X_{t_n}X_{t_{n-1}}^{-1})\in E_n)\cr
&=\delta_e(E_0)\lambda_{t_1-t_0}(E_1)\ldots\lambda_{t_n-t_{n-1}}(E_n)\cr
\displaycause{where $\delta_e$ is the Dirac measure concentrated at $e$}
&=\hat\mu\{\omega:\omega(t_0)\in E_0\}
  \hat\mu\{\omega:\omega(t_1)\omega(t_0)^{-1}\in E_1\}\ldots\cr
&\mskip200mu
  \hat\mu\{\omega:\omega(t_n)\omega(t_{n-1})^{-1}\in E_n\}\cr
&=\hat\mu\{\omega:\omega(t_0)\in E_0,\,
  \omega(t_1)\omega(t_0)^{-1}\in E_1,\,
  \ldots\omega(t_n)\omega(t_{n-1})^{-1}\in E_n\}\cr
\displaycause{by (a) above}
&=\hat\mu\{\omega:(\omega(t_0),\omega(t_1)\omega(t_0)^{-1},\ldots,
  \omega(t_n)\omega(t_{n-1})^{-1})\in E_0\times\ldots\times E_n\}.\cr}$$

\noindent So in fact

$$\eqalign{\Pr((X_{t_0},X_{t_1}X_{t_0}^{-1},
  &\ldots,X_{t_n}X_{t_{n-1}}^{-1})\in H)\cr
&=\hat\mu\{\omega:(\omega(t_0),\omega(t_1)\omega(t_0)^{-1},\ldots,
  \omega(t_n)\omega(t_{n-1})^{-1})\in H\}\cr}$$

\noindent for every Borel set $H\subseteq U^{n+1}$.

Set

\Centerline{$\phi(x_0,\ldots,x_n)
=(x_0,x_1x_0,x_2x_1x_0,\ldots,x_nx_{n-1}\ldots x_1x_0)$}

\noindent for $x_0,\ldots,x_n\in U$, so that $\phi:U^{n+1}\to U^{n+1}$ is
continuous.   If $H\in\Cal B(U^{n+1})$, then

$$\eqalign{\Pr(X_{t_0},\ldots,X_{t_n}\in H)
&=\Pr(\phi(X_{t_0},X_{t_1}X_{t_0}^{-1},\ldots,X_{t_n}X_{t_{n-1}}^{-1})
  \in H)\cr
&=\Pr((X_{t_0},X_{t_1}X_{t_0}^{-1},\ldots,X_{t_n}X_{t_{n-1}}^{-1})
  \in\phi^{-1}[H])\cr
&=\hat\mu\{\omega:((\omega(t_0),\omega(t_1)\omega(t_0)^{-1},\ldots,
  \omega(t_n)\omega(t_{n-1})^{-1})\in\phi^{-1}[H]\}\cr
&=\hat\mu\{\omega:(\omega(t_0),\ldots,\omega(t_n))\in H\}.\cr}$$

\noindent Taking $H=F_0\times\ldots\times F_n$ we have the result.\ \Qed
}%end of proof of 455R

\leader{455S}{Lemma}\dvAnew{2013} Let $U$ be a metrizable
topological group which is complete
under a \rti\ metric inducing its topology.
Let $\langle\lambda_t\rangle_{t>0}$
be a family of Radon probability measures
on $U$ such that $\lambda_s*\lambda_t=\lambda_{s+t}$ for all $s$, $t>0$
and $\lim_{t\downarrow 0}\lambda_tG=1$ for every open
neighbourhood $G$ of the identity $e$ in $U$.   For $x\in U$ and
$0\le s<t$, let $\nu_x^{(s,t)}$ be the Radon probability measure on $U$
defined by saying that $\nu_x^{(s,t)}(E)=\lambda_{t-s}(Ex^{-1})$ whenever
$\lambda_{t-s}$ measures $Ex^{-1}$.

(a) If $0\le t_0<t_1<\ldots<t_n$, $z\in U$ and $f:\BbbR^J\to\Bbb R$ is a
bounded Borel measurable function, where $J=\{t_0,\ldots,t_n\}$, then

$$\eqalign{
&\iint\ldots\int f(z,x_1,\ldots,x_n)\nu_{x_{n-1}}^{(t_{n-1},t_n)}(dx_n)
\ldots\nu_{x_1}^{(t_1,t_2)}(dx_2)\nu_z^{(t_0,t_1)}(dx_1)\cr
&\mskip50mu
=\iint\ldots\int f(z,x_1z,\ldots,x_nz)\nu_{x_{n-1}}^{(t_{n-1},t_n)}(dx_n)\cr
&\mskip200mu
\ldots\nu_{x_1}^{(t_1,t_2)}(dx_2)\nu_e^{(t_0,t_1)}(dx_1).\cr}$$

(b) Take $\omega\in U^{\coint{0,\infty}}$ and $a\ge 0$.
Let $\hat\mu$ and $\hat\mu_{\omega a}$
be the measures on $U^{\coint{0,\infty}}$ defined
from $\langle\nu^{(s,t)}_x\rangle_{s<t,x\in U}$ by the method of 455E,
starting from $x^*=e$.   Define
$\phi_{\omega a}:U^{\coint{0,\infty}}\to U^{\coint{0,\infty}}$ by setting

$$\eqalign{\phi_{\omega a}(\omega')(t)
&=\omega(t)\text{ if }t<a,\cr
&=\omega'(t-a)\omega(a)\text{ if }t\ge a.\cr}$$

\noindent Then $\hat\mu_{\omega a}$ is the image measure
$\hat\mu\phi_{\omega a}^{-1}$.

(c) In (b), suppose that $\omega$ belongs to the set $\Cdlg$ of \cadlag\
functions from $\coint{0,\infty}$ to $U$.
Then $\phi_{\omega a}(\omega')\in\Cdlg$ for every
$\omega'\in\Cdlg$, and $\phi_{\omega a}:\Cdlg\to\Cdlg$ is \imp\ for
the subspace measures $\ddot\mu$ and $\ddot\mu_{\omega a}$ on $\Cdlg$.

\proof{{\bf (a)(i)} If $x\in U$ and $0\le s<t$, then
$\nu_x^{(s,t)}(E)=\nu_e^{(s,t)}(Ex^{-1})$ for any $E\subseteq U$ such that
either is defined;  so
$\int f(y)\nu_x^{(s,t)}(dy)=\int f(yx)\nu_e^{(s,t)}(dy)$ for any function
$f:U\to\Bbb R$ for which either is defined.   More generally,

\Centerline{$\int f(y)\nu_{xz}^{(s,t)}(dy)=\int f(yxz)\nu_e^{(s,t)}(dy)
=\int f(yz)\nu_x^{(s,t)}(dy)$}

\noindent whenever $f$ is such that any of the three integrals is defined.

\medskip

\quad{\bf (ii)} Now induce on $n$.   For the case $n=0$, the natural
interpretation of both sides of the formula presented
is $f(z)$.   For the inductive step to $n+1$,
we have

$$\eqalignno{
&\int\ldots\int f(z,x_1,x_2,\ldots,x_{n+1})
   \nu_{x_n}^{(t_n,t_{n+1})}(dx_{n+1})\ldots\nu_z^{(t_0,t_1)}(dx_1)\cr
&\mskip50mu
=\int\ldots\int f(z,x_1z,\ldots,x_nz,x_{n+1})
   \nu_{x_nz}^{(t_n,t_{n+1})}(dx_{n+1})
%\cr&\mskip200mu
\ldots\nu_e^{(t_0,t_1)}(dx_1)\cr
\displaycause{by the inductive hypothesis applied to
$(x_0,x_1,\ldots,x_n)
\mapsto\int f(x_0,\ldots,x_n,x_{n+1})\nu_{x_n}^{(t_n,t_{n+1})}(dx_{n+1})$}
&\mskip50mu
=\int\ldots\int f(z,x_1z,\ldots,x_nz,x_{n+1}z)
   \nu_{x_n}^{(t_n,t_{n+1})}(dx_{n+1})
%\cr&\mskip200mu
\ldots\nu_e^{(t_0,t_1)}(dx_1)\cr}$$

\noindent by (i) applied to the functions
$y\mapsto f(z,x_1z,\ldots,x_nz,y)$ for each $x_1,\ldots,x_n$.

\medskip

{\bf (b)(i)} Suppose that $J\subseteq\coint{0,\infty}$ is a finite set
containing both $0$ and $a$, enumerated in increasing order as
$(t_0,\ldots,t_n)$ with $a=t_j$.   Set $z=\omega(a)$.
Let $f:\BbbR^J\to\Bbb R$ be a function.
Then $f\pi_J\phi_{\omega a}=g\pi_K$ where $K=\{0,t_{j+1}-a,\ldots,t_n-a\}$
and $g(x_j,\ldots,x_n)
=f(\omega(0),\omega(t_1),\ldots,\omega(t_{j-1}),x_jz,\ldots,x_nz)$ for
$x_j,\ldots,x_n\in U$.
\Prf\ For $\omega'\in U^{\coint{0,\infty}}$,

$$\eqalign{f\pi_J\phi_{\omega a}(\omega')
&=(f(\phi_{\omega a}(\omega')(t_0)),\ldots,f(\phi_{\omega a}(\omega')(t_n))
   \cr
&=(f(\omega(0)),\ldots,f(\omega(t_{j-1})),f(\omega'(0)z),
   \ldots f(\omega'(t_n-a)z))\cr
&=g(\omega'(0),\ldots,\omega'(t_n-a))
=g\pi_K(\omega').  \text{\Qed}\cr}$$

\medskip

{\bf (ii)} Again suppose that $J\subseteq\coint{0,\infty}$ is a finite set
containing both $0$ and $a$, enumerated in increasing order as
$(t_0,\ldots,t_n)$ with $a=t_j$, and set $z=\omega(a)$.
This time, let $f:\BbbR^J\to\Bbb R$ be a bounded Borel measurable function.
Then

$$\eqalignno{\int f\pi_Jd\hat\mu_{\omega a}
&=\int\ldots\int
    f(e,x_1,\ldots,x_n)\nu_{\omega ax_{n-1}}^{(t_{n-1},t_n)}(dx_n)
  \ldots\nu_{\omega a0}^{(0,t_1)}(dx_1)\cr
&=\int\ldots\iint\ldots\int
    f(e,x_1,\ldots,x_{j-1},x_j,x_{j+1},\ldots,x_n)\cr
&\mskip100mu
    \nu_{x_{n-1}}^{(t_{n-1},t_n)}(dx_n)
      \ldots\nu_z^{(a,t_{j+1})}(dx_{j+1})
      \delta_z(dx_j)\ldots\delta_{\omega(t_1)}(dx_1)\cr
\displaycause{reading from the formulae in 455E;  here each $\delta_x$ is a
Dirac measure on $U$}
&=\int\ldots\int
    f(e,\omega(t_1),\ldots,z,x_{j+1},\ldots,x_n)\cr
&\mskip200mu
    \nu_{x_{n-1}}^{(t_{n-1},t_n)}(dx_n)
  \ldots\nu_z^{(a,t_{j+1})}(dx_{j+1})\cr
&=\int\ldots\int
    f(e,\omega(t_1),\ldots,z,x_{j+1}z,\ldots,x_nz)\cr
&\mskip200mu
    \nu_{x_{n-1}}^{(t_{n-1},t_n)}(dx_n)
  \ldots\nu_e^{(a,t_{j+1})}(dx_{j+1})\cr
\displaycause{applying (a) to the function $(y_0,\ldots,y_{n-j})
\mapsto f(e,\ldots,\omega(t_{j-1}),y_0,\ldots,y_{n-j})$}
&=\int\ldots\int
    g(e,x_{j+1},\ldots,x_n)
    \nu_{x_{n-1}}^{(t_{n-1},t_n)}(dx_n)
  \ldots\nu_e^{(a,t_{j+1})}(dx_{j+1})\cr
\displaycause{where $g(x_j,\ldots,x_n)
=f(\omega(0),\omega(t_1),\ldots,\omega(t_{j-1}),x_jz,\ldots,x_nz)$ for
$x_j,\ldots,x_n\in U$}
&=\int\ldots\int g(e,x_{j+1},\ldots,x_n)
    \nu_{x_{n-1}}^{(t_{n-1}-a,t_n-a)}(dx_n)
  \ldots\nu_e^{(0,t_{j+1}-a)}(dx_{j+1})\cr
\displaycause{because $\nu_x^{(s-a,t-a)}E=\lambda_{t-s}(Ex^{-1})
=\nu_x^{(s,t)}E$ whenever $E\subseteq\BbbR^K$ is Borel, $x\in U$ and
$a\le s<t$}
&=\int g\pi_Kd\hat\mu\cr
\displaycause{where $K=\{0,t_{j+1}-a,\ldots,t_n-a\}$}
&=\int f\pi_J\phi_{\omega a}d\hat\mu
=\int f\pi_Jd(\hat\mu\phi_{\omega a}^{-1}).\cr}$$

\noindent As $f$ and $J$ are arbitrary, $\hat\mu_{\omega a}$ and
$\hat\mu\phi_{\omega a}^{-1}$ agree on the algebra
$\bigotimes_{\coint{0,\infty}}\Cal B(U)$ generated by sets of the form
$\{\omega:\omega(t)\in E\}$ for $t\ge 0$ and Borel sets $E\subseteq U$.
By the Monotone Class Theorem, the measures agree on the $\sigma$-algebra
$\Tensorhat_{\coint{0,\infty}}\Cal B(U)$
generated by $\bigotimes_{\coint{0,\infty}}\Cal B(U)$;  because they are
both defined as complete measures inner regular with respect to this
$\sigma$-algebra, they are identical.

\medskip

{\bf (c)} The defining formula for $\phi_{\omega a}$ makes it plain that
$\phi_{\omega a}(\omega')$ is \cadlag\ whenever $\omega$, $\omega'$ are
\cadlag;  and in fact that, if $\omega$ is \cadlag, then
$\phi_{\omega a}(\omega')$ is \cadlag\ iff $\omega'$ is \cadlag.
If $W$ is measured by $\ddot\mu_{\omega a}$, there is a
$W'\in\dom\hat\mu_{\omega a}$ such that $W=W'\cap\Cdlg$.   In this case,
$\phi_{\omega a}^{-1}[W]=\phi_{\omega a}^{-1}[W']\cap\Cdlg$
while $\phi_{\omega a}^{-1}[W']\in\dom\hat\mu$;  so
$\ddot\mu\phi_{\omega a}^{-1}[W]$ is defined and equal to

\Centerline{$\hat\mu\phi_{\omega a}^{-1}[W']
=\hat\mu_{\omega a}W'=\ddot\mu_{\omega a}W$.}
}%end of proof of 455S

\leader{455T}{Corollary}\dvAnew{2013} Let $U$ be a metrizable
topological group which is complete
under a \rti\ metric inducing its topology.
Let $\langle\lambda_t\rangle_{t>0}$
be a family of Radon probability measures
on $U$ such that $\lambda_s*\lambda_t=\lambda_{s+t}$ for all $s$, $t>0$
and $\lim_{t\downarrow 0}\lambda_tG=1$ for every open
neighbourhood $G$ of the identity $e$ in $U$;
let $\hat\mu$ be the measure on $U^{\coint{0,\infty}}$ defined
from $\langle\lambda_t\rangle_{t>0}$ by the method of 455P, starting from
$x^*=e$.   Let $\Cdlg$ be the set of \cadlag\ functions from
$\coint{0,\infty}$ to $U$, $\ddot\mu$ the subspace measure on $\Cdlg$
and $\ddot\Sigma$ its domain.   For $t\ge 0$, let $\ddot\Sigma_t$ be

\Centerline{$\{F:F\in\ddot\Sigma$, $\omega'\in F$ whenever $\omega\in F$,
   $\omega'\in\Cdlg$ and $\omega'\restr[0,t]=\omega\restr[0,t]\}$}

\noindent and
$\hat{\ddot\Sigma}_t=\{F\symmdiff A:F\in\ddot\Sigma_t$, $\ddot\mu A=0\}$.
Then $\hat{\ddot\Sigma}_t=\bigcap_{s>t}\hat{\ddot\Sigma}_s$ includes
$\ddot\Sigma^+_t=\bigcap_{s>t}\ddot\Sigma_s$.

\proof{{\bf (a)} I show first that
$\ddot\Sigma^+_t\subseteq\hat{\ddot\Sigma}_t$.   
\Prf\ Take $E\in\ddot\Sigma^+_t$.
Let $\tau:\Cdlg\to[0,\infty]$
be the constant stopping time with value $t$,
and $f$ the characteristic function $\chi E$.   Set
$g(\omega)=\int_{\Cdlg}fd\ddot\mu_{\omega t}$ when this is defined in
$\Bbb R$, where $\ddot\mu_{\omega t}$ is defined as in 455O;  then
$g$ is a conditional expectation of $f$ on $\ddot\Sigma_{\tau}$
(455Ob).   Since

$$\eqalign{\ddot\Sigma_{\tau}^+
&=\{H:H\in\ddot\Sigma,\,H\cap\{\omega:\tau(\omega)\le s\}\in\ddot\Sigma_s^+
\text{ for every }s\ge 0\}\cr
&=\{H:H\in\ddot\Sigma,\,H\in\ddot\Sigma_s^+
\text{ for every }s\ge t\}
=\ddot\Sigma_t^+\cr}$$

\noindent contains $E$, $g\eae\chi E$.
Setting $F=\{\omega:\omega\in\dom g$, $g(\omega)=1\}$, $F\in\ddot\Sigma$
(remember that $\ddot\mu$ is complete), and $E\symmdiff F$ is negligible.

Now 455Sc, with 235Gb, tells us that

\Centerline{$g(\omega)=\int_{\Cdlg}fd\ddot\mu_{\omega t}
=\int_{\Cdlg}f\phi_{\omega t}d\ddot\mu$}

\noindent whenever either integral is defined in $\Bbb R$, where

$$\eqalign{\phi_{\omega t}(\omega')(s)
&=\omega(s)\text{ if }s<t,\cr
&=\omega'(s-t)\omega(t)\text{ if }s\ge t.\cr}$$

\noindent If $\omega_0$, $\omega_1\in\Cdlg$ and
$\omega_0\restr[0,t]=\omega_1\restr[0,t]$, then
$\phi_{\omega_0t}=\phi_{\omega_1t}$ so $g(\omega_0)=g(\omega_1)$ if either
is defined.   It follows that $\omega_0\in F$ iff $\omega_1\in F$.
As $\omega_0$ and $\omega_1$ are arbitrary, $F\in\ddot\Sigma_t$ and
$E\in\hat{\ddot\Sigma}_t$.

\medskip

{\bf (b)} Of course $\hat{\ddot\Sigma}_t\subseteq\hat{\ddot\Sigma}_s$ 
whenever $s>t$.   Putting (a) and 455L(f-ii) together,

\Centerline{$\bigcap_{s>t}\hat{\ddot\Sigma}_s
=\{E\symmdiff A:E\in\ddot\Sigma_t^+$, $\ddot\mu A=0\}
\subseteq\{E\symmdiff A:E\in\hat{\ddot\Sigma}_t$, $\ddot\mu A=0\}
=\hat{\ddot\Sigma}_t$}

\noindent and we have equality.
}%end of proof of 455T

\leader{455U}{Theorem} Let $U$ be a metrizable
topological group which is complete
under a \rti\ metric inducing its topology.
Let $\langle\lambda_t\rangle_{t>0}$
be a family of Radon probability measures
on $U$ such that $\lambda_s*\lambda_t=\lambda_{s+t}$ for all $s$, $t>0$
and $\lim_{t\downarrow 0}\lambda_tG=1$ for every open
neighbourhood $G$ of the identity $e$ in $U$;
let $\hat\mu$ be the measure on $U^{\coint{0,\infty}}$ defined
from $\langle\lambda_t\rangle_{t>0}$ by the method of 455P, starting from
$x^*=e$.   Let $\Cdlg$ be the set of \cadlag\ functions from
$\coint{0,\infty}$ to $U$, $\ddot\mu$ the subspace measure on $\Cdlg$
and $\ddot\Sigma$ its domain.   For $t\ge 0$, let $\ddot\Sigma_t$ be

\Centerline{$\{F:F\in\ddot\Sigma$, $\omega'\in F$ whenever $\omega\in F$,
   $\omega'\in\Cdlg$ and $\omega'\restr[0,t]=\omega\restr[0,t]\}$,}

\noindent and $\ddot\Sigma^+_t=\bigcap_{s>t}\ddot\Sigma_s$;  let
$\tau:\Cdlg\to[0,\infty]$ be a stopping time adapted to
$\langle\ddot\Sigma^+_t\rangle_{t\ge 0}$.   Define
$\phi_{\tau}:\Cdlg\times\Cdlg\to\Cdlg$ by setting

$$\eqalign{\phi_{\tau}(\omega,\omega')(t)
&=\omega'(t-\tau(\omega))\omega(\tau(\omega))
   \text{ if }t\ge\tau(\omega),\cr
&=\omega(t)\text{ otherwise}.\cr}$$

\noindent Then $\phi_{\tau}$ is \imp\ for the product measure
$\ddot\mu\times\ddot\mu$ on $\Cdlg\times\Cdlg$ and $\ddot\mu$ on $\Cdlg$.

\proof{{\bf (a)} First, I ought to remark that we know from 455P
that the conditions of 455O are satisfied.   I aim to apply 455Oa, using
455S to give a description of the measures
$\ddot\mu_{\omega,\tau(\omega)}$.
Now if $f$ is $\ddot\mu$-integrable, we have, in the notation of 455O and
455S,

$$\eqalignno{\int_{\Cdlg}fd\ddot\mu
&=\int_{\Cdlg}\int_{\Cdlg}fd\ddot\mu_{\omega,\tau(\omega)}\ddot\mu(d\omega)\cr
\displaycause{455Oa}
&=\int_{\Cdlg}\int_{\Cdlg}f\phi_{\omega,\tau(\omega)}(\omega')
  \ddot\mu(d\omega')\ddot\mu(d\omega)\cr
\displaycause{455Sc}
&=\int_{\Cdlg}\int_{\Cdlg}f\phi_{\tau}(\omega,\omega')
  \ddot\mu(d\omega')\ddot\mu(d\omega)\cr.}$$

\medskip

{\bf (b)} To convert the repeated integral into the product measure,
we have still to check for measurability.    The point is that, writing
$\Lambda$ for the domain of the product measure $\ddot\mu\times\ddot\mu$,
$\phi_{\tau}$ is
$(\Lambda,\Tensorhat_{\coint{0,\infty}}\Cal B(U))$-measurable.

\medskip

\Prf{\bf (i)} Consider first the case in which $U$ is separable.
Take $t\ge 0$.   Then
$E_t=\{\omega:\omega\in\Cdlg$, $\tau(\omega)\le t\}$ belongs to
$\ddot\Sigma$.   The function

\Centerline{$\omega\mapsto t-\tau(\omega):E_t\to\coint{0,\infty}$}

\noindent is $(\ddot\Sigma,\Cal B(\coint{0,\infty}))$-measurable;
the function

\Centerline{$(\omega',s)\mapsto\omega'(s):\Cdlg\times\coint{0,\infty}
\to U$}

\noindent is
$(\ddot\Sigma\tensorhat\Cal B(\coint{0,\infty}),\Cal B(U))$-measurable, by
4A3Wc, because $U$ is Polish;   so the function

\Centerline{$(\omega,\omega')\mapsto\omega'(t-\tau(\omega)):
   E_t\times\Cdlg\to U$}

\noindent is $(\ddot\Sigma\tensorhat\ddot\Sigma,\Cal B(U))$-measurable.   Next,
similarly,

\Centerline{$\omega\mapsto\omega(\tau(\omega)):E_t\to U$}

\noindent is $(\ddot\Sigma,\Cal B(U))$-measurable, while

\Centerline{$(y,z)\mapsto yz:U\times U\to U$}

\noindent is $(\Cal B(U)\tensorhat\Cal B(U),\Cal B(U))$-measurable,
because $U$ is a second-countable topological group.   But this means
that

\Centerline{$(\omega,\omega')
  \mapsto\omega'(t-\tau(\omega))\omega(\tau(\omega)):
  E_t\times\Cdlg\to U$}

\noindent is $(\ddot\Sigma\tensorhat\ddot\Sigma,\Cal B(U))$-measurable.
On the other hand, of course,

\Centerline{$(\omega,\omega')\mapsto\omega(t):
   (\Cdlg\setminus E_t)\times\Cdlg\to U$}

\noindent is $(\ddot\Sigma\tensorhat\ddot\Sigma,\Cal B(U))$-measurable.
Putting these together,

\Centerline{$(\omega,\omega')\mapsto\phi_{\tau}(\omega,\omega')(t):
   \Cdlg\times\Cdlg\to U$}

\noindent is $(\ddot\Sigma\tensorhat\ddot\Sigma,\Cal B(U))$-measurable.
This is true for every $t\ge 0$, so $\phi_{\tau}\restr\Cdlg\times\Cdlg$ is
$(\ddot\Sigma\tensorhat\ddot\Sigma,
  \Tensorhat_{\coint{0,\infty}}\Cal B(U))$-measurable.

\medskip

\quad{\bf (ii)} For the general case, we can use the trick in 455H.
There is a separable subset $U'$ of $U$ such that
$\nu_e^{(0,q)}U'=1$ for every rational $q\ge 0$.
We can suppose that $U'$ is
a closed subgroup of $U$.  Because $U'$ is closed,

$$\eqalign{\Cdlg'
&=\{\omega:\omega\in\Cdlg,\,\omega(t)\in U'\text{ for every }t\ge 0\}\cr
&=\{\omega:\omega\in\Cdlg,\,\omega(q)\in U'
   \text{ for every }q\in\Bbb Q\cap\coint{0,\infty}\}\cr}$$

\noindent is $\ddot\mu$-conegligible in $\Cdlg$, and
because $U'$ is a subgroup,
$\phi_{\tau}(\omega,\omega')\in\Cdlg'$ for all $\omega$,
$\omega'\in\Cdlg'$.   Now the argument of (i) shows that
$\phi_{\tau}\restr\Cdlg'\times\Cdlg'$ is
$(\ddot\Sigma\tensorhat\ddot\Sigma,
   \Tensorhat_{\coint{0,\infty}}\Cal B(U'))$-measurable, therefore
$(\ddot\Sigma\tensorhat\ddot\Sigma,
   \Tensorhat_{\coint{0,\infty}}\Cal B(U))$-measurable.
Since $\Cdlg'\times\Cdlg'$ is
$(\ddot\mu\times\ddot\mu)$-conegligible,
$\phi_{\tau}$ is
$(\Lambda,\Tensorhat_{\coint{0,\infty}}\Cal B(U))$-measurable.\ \Qed

\medskip

{\bf (c)} It follows that if $E\in\Tensorhat_{\coint{0,\infty}}\Cal B(U)$
then, setting $f=\chi(E\cap\Cdlg)$ in (a),

\Centerline{$(\ddot\mu\times\ddot\mu)\phi_{\tau}^{-1}[E\cap\Cdlg]
=\int f(\phi_{\tau}(\omega,\omega'))\ddot\mu(d\omega')\ddot\mu(d\omega)
=\int fd\ddot\mu
=\ddot\mu(E\cap\Cdlg)$}

\noindent by Fubini's theorem.
But $\ddot\mu$ is the subspace measure generated by the
completion of a
measure with domain $\Tensorhat_{\coint{0,\infty}}\Cal B(U)$,
so is inner regular with respect to sets of the form $E\cap\Cdlg$ with
$E\in\Tensorhat_{\coint{0,\infty}}\Cal B(U)$;  by 412K,
$\phi_{\tau}$ is \imp.
}%end of proof of 455U

\exercises{\leader{455X}{Basic exercises (a)}
%\spheader 455Xa
Let $\langle A_n\rangle_{n\ge 1}$ be a non-increasing sequence of subsets
of $[0,1]$, all with Lebesgue outer measure $1$, and with empty
intersection.   Set $T=\{0\}\cup\{\bover1n:n\ge 1\}$, $\Omega_0=\{0\}$,
$\Omega_{1/n}=A_n$ for $n\ge 1$;  for $t\in T$ let $\Tau_t$ be the Borel
$\sigma$-algebra of $\Omega_t$.   For $s<t$ in $T$ and $x\in\Omega_s$
define a Borel measure $\nu^{(s,t)}_x$ on $\Omega_t$ by saying that

\inset{if $n\ge 1$,
then $\nu^{(0,1/n)}_0(E\cap A_n)$ is the Lebesgue measure
of $E$ for every Borel set $E\subseteq[0,1]$,

if $0<s<t$, then $\nu^{(s,t)}_x\{x\}=1$.}

\noindent Show that $\family{y}{\Omega_t}{\nu^{(t,u)}_y}$ is a
disintegration of $\nu^{(s,u)}_x$ over $\nu^{(s,t)}_x$ whenever
$s<t<u$ in $T$ and $x\in\Omega_s$.   Taking $x^*=0$,
show that there is no measure $\mu$ on
$\prod_{t\in T}\Omega_t$ with the properties listed in 455A.
%455A

\spheader 455Xb Let $T$, $t^*$,
$\family{t}{T}{(\Omega_t,\Tau_t)}$, $x^*$,
$\langle\nu^{(s,t)}_x\rangle_{s<t,x\in\Omega_s}$ and $\mu$ be as in 455A.
Suppose that we are given a family
$\family{t}{T}{(\Omega'_t,\Tau'_t,\pi_t)}$ such
that ($\alpha$) $\Omega'_t$ is a set, $\Tau'_t$ is a $\sigma$-algebra of
subsets of $\Omega'_t$ and $\pi_t:\Omega_t\to\Omega'_t$ is a surjective
$(\Tau_t,\Tau'_t)$-measurable function for every $t\in T$
($\beta$) whenever $s<t$ in $T$ and $x$, $x'\in\Omega_s$ are such that
$\pi_s(x)=\pi_s(x')$, then $\nu^{(s,t)}_x$ and $\nu^{(s,t)}_{x'}$ agree on
$\{\pi_t^{-1}[F]:F\in\Tau'_t\}$.   (i) Show that
if we set $\acute\nu^{(s,t)}_w(F)=\nu^{(s,t)}_x\pi_t^{-1}[F]$
whenever $s<t$ in $T$, $x\in\Omega_s$, $w=\pi_s(x)$ and $F\in\Tau'_t$,
then every $\acute\nu^{(s,t)}_w$ is a
perfect probability measure, and
$\family{z}{\Omega'_t}{\acute\nu^{(t,u)}_z}$ is a disintegration of
$\acute\nu^{(s,u)}_w$ over $\acute\nu^{(s,t)}_w$ whenever $s<t<u$ in $T$
and $w\in\Omega'_s$.   (ii) Let $\mu'$ be the measure on
$\Omega'=\prod_{t\in T}\Omega'_t$ defined by the method of 455A from
$\pi_{t^*}(x^*)$ and
$\langle\acute\nu^{(s,t)}_w\rangle_{s<t,w\in\Omega'_s}$.   Show that
$\pi:\Omega\to\Omega$ is \imp\ for $\mu$ and $\mu'$, where
$\pi(\omega)(t)=\pi_t(\omega(t))$ for $\omega\in\Omega$ and $t\in T$.
%455A

\spheader 455Xc
In 455E, set $T=\{-1\}\cup\coint{0,\infty}$, let each
$\Omega_t$ be $\Bbb R$, and for $x\in\Bbb R$,
$0\le s<t$ let $\nu_x^{(s,t)}$ be the Dirac measure on $\Bbb R$
concentrated at $\psi(x,t-s)$ on $\Bbb R$, where

$$\eqalign{\psi(x,t)
&=\Bover{x}{1-xt}\text{ if }xt\ne 1\text{ and }x\ne 0,\cr
&=0\text{ if }xt=1,\cr
&=-\Bover1t\text{ if }x=0.\cr}$$

\noindent Let $\nu$ be any atomless Radon probability measure on
$\Bbb R$, and complete the definition by setting
$\nu_x^{(-1,t)}(E)=\nu\{y:\psi(y,t)\in E\}$ whenever $t\ge 0$ and this is
defined;  set $x^*=0$.
Show that the conditions of 455E are satisfied, that the
measure $\hat\mu$ constructed in 455E is a distribution on $\BbbR^T$, and
that $\hat\mu$ is not $\tau$-additive.   \Hint{setting $\phi(y)(-1)=0$,
$\phi(y)(t)=\psi(y,t)$ for $t\ge 0$ and $y\in\Bbb R$,
show that $\phi:\Bbb R\to\BbbR^T$ is \imp\ for
$\nu_0$ and $\hat\mu$, and that every point of
$\BbbR^T$ has a neighbourhood of zero measure.}
%455E

\spheader 455Xd Let $T$, $t^*$, $\family{t}{T}{\Omega_t}$, $x^*$,
$\langle\nu^{(s,t)}_x\rangle_{s<t,x\in\Omega_s}$ and
$\hat\mu$ be as in 455E.   Suppose that we are given a family
$\family{t}{T}{(\Omega'_t,\pi_t)}$ such
that ($\alpha$) $\Omega'_t$ is a Hausdorff space and
$\pi_t:\Omega_t\to\Omega'_t$ is a continuous surjective
function for every $t\in T$
($\beta$) whenever $s<t$ in $T$ and $x$, $x'\in\Omega_s$ are such that
$\pi_s(x)=\pi_s(x')$, then the image measures $\nu^{(s,t)}_x\pi_t^{-1}$
and $\nu^{(s,t)}_{x'}\pi_t^{-1}$ on $\Omega'_t$ are the same.
(i) Show that if we set $\acute\nu^{(s,t)}_w=\nu^{(s,t)}_x\pi_t^{-1}$
whenever $s<t$ in $T$, $x\in\Omega_s$ and $w=\pi_s(x)$,
then every $\acute\nu^{(s,t)}_w$ is a Radon probability measure, and
$\family{z}{\Omega'_t}{\acute\nu^{(t,u)}_z}$ is a disintegration of
$\acute\nu^{(s,u)}_w$ over $\acute\nu^{(s,t)}_w$ whenever $s<t<u$ in $T$
and $w\in\Omega'_s$.   (ii) Let $\hat\mu'$ be the measure on
$\Omega'=\prod_{t\in T}\Omega'_t$ defined by the method of 455E from
$\pi_{t^*}(x^*)$ and
$\langle\acute\nu^{(s,t)}_w\rangle_{s<t,w\in\Omega'_s}$.   Show that
$\pi:\Omega\to\Omega$ is \imp\ for $\hat\mu$ and $\hat\mu'$, where
$\pi(\omega)(t)=\pi_t(\omega(t))$ for $\omega\in\Omega$ and $t\in T$.
%455E

\spheader 455Xe Let $U$ be a locally compact metrizable group and $\nu$
any
Radon probability measure on $U$.   For $t>0$ let $\lambda_t$ be the Radon
probability measure

\Centerline{$e^{-t}(\delta_e+t\nu+\Bover{t^2}{2!}\nu*\nu
+\Bover{t^3}{3!}\nu*\nu*\nu+\ldots)$,}

\noindent where $\delta_e$ is the Dirac measure on $U$ concentrated at the
identity $e$ of $U$, and the sum is defined as in 234G\formerly{1{}12Ya}.
Show that $\langle\lambda_t\rangle_{t>0}$ satisfies the
conditions of 455P (with respect to an appropriate metric on $U$).
\Hint{4A5Mb, 4A5Q(iv).}
%455P

\spheader 455Xf Let $U$, $\langle\lambda_t\rangle_{t>0}$ and $\hat\mu$ be
as in 455P.   Let $V$ be a Hausdorff space, $z^*$ a point of $V$ and
$\action$ a continuous action of $U$ on $V$;  set
$\pi(x)=x\action z^*$ for $x\in U$.   Define
$\langle\nu^{(s,t)}_x\rangle_{0\le s<t,x\in U}$ from
$\langle\lambda_t\rangle_{t>0}$ as in 455P.   (i) Show that if $x$,
$x'\in U$ and $\pi(x)=\pi(x')$, then the image measures
$\nu^{(s,t)}_x\pi^{-1}$, $\nu^{(s,t)}_{x'}\pi^{-1}$ on $V$ are equal
whenever $s<t$.   (ii) Let $\hat\mu'$ be the measure on
$V^{\coint{0,\infty}}$ defined as in 455Xd.   Show that if we define
$\tilde\pi:U^{\coint{0,\infty}}\to V^{\coint{0,\infty}}$ by setting
$\tilde\pi(\omega)(t)=\omega(t)\action z^*$ for every
$\omega\in U^{\coint{0,\infty}}$ and $t\ge 0$, $\tilde\pi$ is \imp\ for
$\hat\mu$ and $\hat\mu'$ and also for the Radon measures extending them;
moreover, that the restriction of $\tilde\pi$ to $\Cdlg(U)$, the space
of \cadlag\ functions from $\coint{0,\infty}$ to $U$, is \imp\ for the
subspace measures on $\Cdlg(U)$ and $\Cdlg(V)$.
%455P 455Xd

\sqheader 455Xg For $t>0$, let $\lambda_t$ be the normal distribution on
$\Bbb R$ with expectation $0$ and variance $t$.
Show that $\langle\lambda_t\rangle_{t>0}$ satisfies the conditions of
455Q.
%455Q

\sqheader 455Xh For $t>0$, let $\lambda_t$ be the Poisson
distribution with expectation $t$, that is,
$\lambda_t(E)=e^{-t}\sum_{m\in E\cap\Bbb N}t^m/m!$ for $E\subseteq\Bbb R$
(cf.\ 285Q, 285Xo).
(i) Show that $\langle\lambda_t\rangle_{t>0}$ satisfies the conditions of
455Q.
(ii) Show that if $\tilde\mu$ is the Radon measure defined from
$\langle\lambda_t\rangle_{t>0}$ as in 455Pc,
then $\omega$ is non-decreasing
and $\omega[\,\coint{0,\infty}\,]=\Bbb N$ for $\tilde\mu$-almost every
$\omega\in\BbbR^{\coint{0,\infty}}$.
%455Q

\sqheader 455Xi For $t>0$, let $\lambda_t$ be the Cauchy
distribution with centre $0$ and scale parameter $t$, that is, the
distribution with probability density function
$x\mapsto\Bover{t}{\pi(x^2+t^2)}$ (285Xm).
(i) Show that $\langle\lambda_t\rangle_{t>0}$ satisfies the conditions of
455Q.   (ii) Show that if $\hat\mu$ is the corresponding distribution on
$\BbbR^{\coint{0,\infty}}$, then $C(\coint{0,\infty})$ is
$\hat\mu$-negligible.
\Hint{estimate $\Pr(|X_{(i+1)/n}-X_{i/n}|\le\epsilon$ for every $i<n)$.}
(iii) Suppose that $\alpha>0$.   Define
$T_{\alpha}:\BbbR^{\coint{0,\infty}}\to\BbbR^{\coint{0,\infty}}$ by setting
$(T_{\alpha}\omega)(t)=\Bover1{\alpha}\omega(\alpha t)$ for $t\ge 0$ and
$\omega\in\BbbR^{\coint{0,\infty}}$.   Show that $T_{\alpha}$ is \imp\ for
$\hat\mu$.   (iv) For $\omega\in\BbbR^{\coint{0,\infty}}$ set
$(R\omega)(t)=t^2\omega(\bover1t)$ if $t>0$, $\omega(0)$ if $t=0$.
Show that $R:\BbbR^{\coint{0,\infty}}\to\BbbR^{\coint{0,\infty}}$ is \imp\
for $\hat\mu$.   (Compare 477E below.)
%455Q

\sqheader 455Xj(i) The {\bf standard gamma distribution} with
parameter $t$ is the probability distribution $\lambda_t$ on
$\Bbb R$ with probability density function
$x\mapsto\Bover1{\Gamma(t)}x^{t-1}e^{-x}$ for $x>0$.   Show
that its expectation is $t$.  \Hint{225Xj(iv).}   Show that its
variance is $t$.
(ii) Show that $\langle\lambda_t\rangle_{t>0}$ satisfies the conditions of
455Q.    \Hint{272U\formerly{2{}72T}, 252Yf.}   (iii) Show that
$\lim_{t\downarrow 0}t\Gamma(t)=1$, so that
$\lim_{t\downarrow 0}\bover1t\lambda_t\coint{1,\infty}
=\int_1^{\infty}\bover1xte^{-x}dx>0$.
(iv) Show that if $\tilde\mu$ is the Radon measure on
$\BbbR^{\coint{0,\infty}}$ defined from $\langle\lambda_t\rangle_{t>0}$ as
in 455Pc, then $\{\omega:\omega$ is strictly increasing and not
continuous$\}$ is $\tilde\mu$-conegligible.
%455Q

\spheader 455Xk Let $U$ be an abelian Hausdorff topological group.   Let
$\langle\lambda'_t\rangle_{t>0}$, $\langle\lambda''_t\rangle_{t>0}$ be two
families of Radon probability measures on $U$ and set
$\lambda_t=\lambda'_t*\lambda''_t$ for $t>0$.   (i) Show that if
$\lambda'_{s+t}=\lambda'_s*\lambda'_t$ and
$\lambda''_{s+t}=\lambda''_s*\lambda''_t$ for all $s$, $t>0$, then
$\lambda_{s+t}=\lambda_s*\lambda_t$ for all $s$, $t>0$.   (ii) Show that if
$\lim_{t\downarrow 0}\lambda'_tG=\lim_{t\downarrow 0}\lambda''_tG=1$ for
every open set containing the identity $e$ of $U$, then
$\lim_{t\downarrow 0}\lambda_tG=1$ for every open set $G$ containing $e$.
(iii) Now suppose that $U$ is metrizable and complete under a \rti\ metric
inducing its topology.   Let $\hat\mu'$, $\hat\mu''$ and $\hat\mu$ be the
measures on $U^{\coint{0,\infty}}$ defined from
$\langle\lambda'_t\rangle_{t>0}$, $\langle\lambda''_t\rangle_{t>0}$ and
$\langle\lambda_t\rangle_{t>0}$ as in 455P.   Set
$\theta(\omega,\omega')(t)=\omega(t)\omega'(t)$ for $\omega$,
$\omega'\in U^{\coint{0,\infty}}$ and $t\ge 0$.   Show that $\theta:
U^{\coint{0,\infty}}\times U^{\coint{0,\infty}}\to U^{\coint{0,\infty}}$ 
is \imp\ for $\hat\mu'\times\hat\mu''$ and $\hat\mu$.
(iv) Repeat (iii) for the subspace measures on the space of
\cadlag\ functions from $\coint{0,\infty}$ to $U$.
%455U

\leader{455Y}{Further exercises (a)}
%\spheader 455Ya
Let $\family{n}{\Bbb Z}{X_n}$ be a double-ended sequence
of real-valued random variables such that (i) for each $n\in\Bbb Z$,
$Y_n=X_{n+1}-X_n$ is independent of $\{X_i:i\le n\}$ (ii)
$\family{n}{\Bbb Z}{Y_n}$ is identically distributed.   Show that the
$Y_n$ are essentially constant.   \Hint{285Yc.}
%455B

\spheader 455Yb For $0\le s<t$ and $x\in\Bbb R$ define a Radon
probability measure
$\nu^{(s,t)}_x$ on $\Bbb R$ by saying that

$$\eqalign{\nu^{(s,t)}_x
&=\Bover{1-t}{1-s}\delta_0+\Bover{t-s}{1-s}\lambda_{[s,t]}
\text{ if }x=0\text{ and }t\le 1,\cr
&=\Bover{t-1}{1-s}\delta_0+\Bover{2-t-s}{1-s}\lambda_{[s,2-t]}
\text{ if }x=0\text{ and }1\le t<2-s,\cr
%&=\delta_0\text{ if }x=0\text{ and }2-s\le t\cr
&=\delta_0\text{ if }0<x<1\text{ and }2-x\le t,\cr
&=\delta_x\text{ otherwise},\cr}$$

\noindent writing $\delta_x$ for the Dirac measure on $\Bbb R$ concentrated
at $x$, and $\lambda_{[s,t]}$ for the uniform distribution based on the
interval $[s,t]$.   (i) Show that
$\langle\nu^{(s,t)}_x\rangle_{s<t,x\in\Bbb R}$ satisfies the conditions of
455E.   (ii) Starting from $x^*=0$,
let $\hat\mu$ be the corresponding measure on
$\BbbR^{\coint{0,\infty}}$.   Show that $\hat\mu^*\Cdlg=1$, where $\Cdlg$
is the space of \cadlag\ functions from $\coint{0,\infty}$ to $\Bbb R$.
(iii) Show that $\hat\mu$ has a unique extension to a Radon measure
$\tilde\mu$ on $\BbbR^{\coint{0,\infty}}$.   (iv) Show that
$\tilde\mu\Cdlg=0$.   (v) Show that the subspace measure on $\Cdlg$ induced
by $\hat\mu$ is not $\tau$-additive.
%455K
%the intention is to put the measure on $\omega_{\alpha}$, for
%$0\le\alpha\le 1$, where $\omega_{\alpha}(t)=\alpha$ for
%$\alpha\le t<2-\alpha$, $0$ elsewhere.

\spheader 455Yc A probability distribution $\lambda$ on $\Bbb R$ is {\bf
infinitely divisible} if for every $n\ge 1$ it is expressible as a
convolution $\nu*\ldots*\nu$ of $n$ copies of a probability distribution.
Let $\phi$ be the characteristic function (\S285) of an infinitely
divisible distribution $\lambda$.   (i) Show that for each $n\ge 1$
there is a
characteristic function $\phi_n$ such that $\phi_n^n=\phi$.   (ii) Show
that if $\delta>0$ is such that $\phi(y)\ne 0$ for $|y|\le\delta$, then
$\lim_{n\to\infty}\phi_n(y)=1$ for $|y|\le\delta$.   (iii) Show that
$\lim_{n\to\infty}\phi_n(y)=1$ for every $y\in\Bbb R$.   \Hint{for any
characteristic function $\psi$, $4\Real\psi(y)\le 3+\Real\psi(2y)$ for
every $y$.}   (iv) Show that $\phi$ is never zero, and that there is a
unique family $\langle\lambda_t\rangle_{t>0}$ of distributions satisfying
the conditions in 455P and such that $\lambda_1=\lambda$.   (v) Show that
if $\lambda$ has finite expectation, then $\Expn(\lambda_t)$ is defined and
equal to $t\Expn(\lambda)$ for every $t>0$.   
%272Xi 285Ga
(vi) Show that if
$\lambda$ has finite variance, then $\Var(\lambda_t)$ is defined and equal
to $t\Var(\lambda)$ for every $t>0$.
%455P

\spheader 455Yd Let $U$ be  a Hausdorff topological group and
$\langle\lambda_t\rangle_{t>0}$ a family of Radon probability measures on
$U$ such that $\lambda_{s+t}=\lambda_s*\lambda_t$ whenever $s$, $t>0$.
(i) Show that we can define a family
$\langle\nu_x^{(s,t)}\rangle_{0\le s<t,x\in U}$ as in 455P, and that
$\family{y}{U}{\nu_y^{(t,u)}}$ is a disintegration of $\nu_x^{(s,u)}$ over
$\nu_x^{(s,t)}$ whenever $x\in U$ and $s<t<u$, so that, starting from
$x^*=e$ the identity of $U$, we can apply 455E to obtain a measure
$\hat\mu$ on $U^{\coint{0,\infty}}$.   (ii) Now suppose that $U=\BbbR^r$
where $r\ge 1$.
For $t>0$, $E\subseteq U$ set $\Reverse{\lambda}_tE=\lambda_t(-E)$
whenever  $\lambda_t$ measures $-E$;  now set
$\lambda^{\#}_t=\lambda_t*\Reverse{\lambda}_t$.   Show that
$\lambda^{\#}_{s+t}=\lambda^{\#}_s*\lambda^{\#}_t$ for all $s$, $t>0$, and
that $\lim_{t\downarrow 0}\lambda^{\#}_tG=1$ for every
open neighbourhood $G$ of $0$.   Show that $\hat\mu$ has an extension to a
Radon measure on $(\BbbR^r)^{\coint{0,\infty}}$.
%455P

\spheader 455Ye Let $Y$ be a metrizable
space and $\Cdlg$ the set of \cadlag\ functions from $\coint{0,\infty}$ to
$Y$.   For $\omega\in\Cdlg$ and $t\ge 0$ set $X_t(\omega)=\omega(t)$.
Let $\Sigma$ be a $\sigma$-algebra of subsets of $\Cdlg$ such that
$X_t:\Cdlg\to Y$ is measurable for every $t\ge 0$.
For $t\ge 0$ let $\Sigma_t$ be

\Centerline{$\{F:F\in\Sigma,\,\omega'\in F$ whenever $\omega$,
$\omega'\in\Cdlg$, $\omega\in F$ and
$\omega\restr[0,t]=\omega'\restr[0,t]\}$.}

\noindent Show that
$\langle X_t\rangle_{t\ge 0}$ is progressively measurable with respect
to $\langle\Sigma_t\rangle_{t\ge 0}$.
%455M out of order query
}%end of exercises

\endnotes{
\Notesheader{455}
This section has grown into the longest in this treatise.
There are some big theorems here.
I am trying to do two rather different things:  sketch the
fundamental properties of Markov processes, and work through
the details of particular realizations of them.   I remarked in the
introduction to Chapter 27 that probability theory is not really about
measure spaces and measurable functions.   It is much more about
distributions, and by `distribution' here I do not really mean a Radon
probability measure on $\BbbR^r$, let alone a completed Baire measure on
$\BbbR^I$, as in 454K.   I mean rather the family of probabilities of
the type $\Pr(X_i\le\alpha_i\Forall i\le n)$;  everything else is formal
structure, offering proofs and (I hope) some kinds of deeper understanding,
but essentially secondary.   The appalling formulae above
($\nu^{(t_j,t_{j+1})}_{\omega,\tau(\omega),x_j}(dx_{j+1})$,
$\ddot\Sigma\tensorhat\ddot\Sigma$ and so on)
arise from my attempts to distinguish
clearly among the host of probability spaces which present themselves to us
as relevant.

However one of the messages of this section is that for many stochastic
processes it is possible to identify semi-canonical realizations.   We
already have a crude one in 454J;  starting from any family
$\familyiI{X_i}$ of real-valued
random variables on any probability space, we can move to a measure
on $\BbbR^I$ which is in some sense unique and carries the probabilistic
content of the original family.   I noted in \S454 that when
this measure is $\tau$-additive we have a canonical extension to a
quasi-Radon measure, just as good regarded as a realization of the abstract
process, and possibly with useful further properties.   In 455H we find
that many of the most important processes can be represented by Radon
measures;  I do not think these Radon measures have been much studied,
except, of course, in the case of Brownian motion.
But 455O and 455U show that for some purposes we are better off
with quasi-Radon measures on the set of \cadlag\ functions.
The most important stopping times are the hitting times of 455M, which are
adapted to families of the form $\langle\Sigma_t^+\rangle_{t\ge 0}$;  and
for such a stopping time to be approximated by discrete stopping times, as
in parts (a-vi) and (b-vii) of the proof of 455O, we need to know that
our paths are continuous on the right.

It is of course true
that when the complete metric space $U$, in 455O or later,
is separable, then we have a standard Borel structure on the space $\Cdlg$
of \cadlag\ functions (4A3Wb), so that the measures $\ddot\mu$ are Radon
measures for appropriate Polish topologies on $\Cdlg$.
\leaveitout{{\smc Billingsley 99};
 http://eom.springer.de/S/s110200.htm;
 http://eom.springer.de/s/s130370.htm.
 No apparent hope of good topological behaviour of the functions
 $\phi_{\tau}$ -- examples 455Xi, 455Xj where we have atomless marginals
 $\nu^{(0,t)}_{x^*}$ but discontinuous processes;  in this case we can have
 continuous stopping times but $\omega\mapsto\phi_{\tau}(\omega,\omega')$
 seriously discontinuous.}

Returning to the detailed exposition, 455A is an attempt at a
continuous-time version of 454H.   I
use the letters $t$, $T$ to suggest the probabilistic intuitions behind
these results;  we think of the spaces $\Omega_t$ in 455A as being the sets
of possible states of a system at `time' $t$, so that the measures
$\nu_x^{(s,t)}$ are descriptions of how we believe the system is likely
to evolve between times $s$ and $t$, having observed that
it is in state $x$ at
time $s$.   In the case of `discrete time', when we observe the system
only at clearly separated moments, it is easy to handle non-Markov
processes, in which evolution between times $n-1$ and $n$ can depend on
the whole history up to time $n-1$;  thus in 454H the measures
$\nu_z=\nu_z^{(n-1,n)}$ are defined for every $z\in\prod_{i<n}X_i$, but
we make no attempt to describe measures $\nu_z^{(n-1,m)}$ for any $m>n$.
In `continuous time' we do need to say something about arbitrary time
steps, and it is hard to formulate a consistency condition to fill the
place of ($\dagger$) in 455A without limiting the kind of process being
examined.   At the cost of an appalling increase in complexity, of
course, the formulae of 455A can
sometimes be adapted to general processes, if we
replace the `current' state space $\Omega_t$ by the `historical' state
space $\prod_{t^*\le s\le t}\Omega_s$.
(For we can hope that
$(\prod_{t^*\le s\le t}\Omega_s,\Tensorhat_{t^*\le s\le t}\Tau_s)$
will have the
`perfect measure property' of 454Xd.)   We should finish up with a
measure on $\prod_{t\in T}(\prod_{t^*\le s\le t}\Omega_s)$.
But the important
applications, even when not Markov, are open to more economical and more
enlightening approaches.
We really do need a least element $t^*$ of $T$;  see 455Ya.

I have not yet come to the reason why this section is such hard work.
This is in its attempt to analyze the `Markov property' of the
distributions being examined here.   The point about the families
$\langle\nu^{(s,t)}_x\rangle_{s<t,x\in\Omega_s}$ of transitional
probabilities is that they not only give us stochastic processes, as in
455A, but also recipes for conditional expectations, derived from the
truncated families $\langle\nu^{(s,t)}_x\rangle_{a\le s<t,x\in\Omega_s}$.
These lead to measures $\mu'_{ax}$ on $\prod_{t\ge a}\Omega_t$
which can be thought of as distributions of future paths given
that we have reached the point $x$ at time $a$.   It is no surprise that
these should
provide straightforward descriptions of conditional expectations on
algebras of the form

\Centerline{$\{F:F$ is determined by coordinates in $[t^*,t]\}$.}

\noindent Without much more trouble,
we can extend this to suitable algebras
defined from simple `stopping times', as in 455C.
The arguments there have
some technical features which you may find annoying (and I invite you to
find your own way past the complications),
but are essentially elementary, as they have to be in such a
general context.   It is interesting that we can move to stopping times
taking countably many values without further difficulty.

However, we are still only seven pages into the section, and not everything
to come is as straightforward as the completion processes described in
455E.   An essential aspect of continuous-time Markov processes is the
possibility of stopping times which take a continuum of values, as is
typically the case in the examples provided by 455M.   These are
much harder to deal with, and we have to restrict sharply the
class of processes we examine.
The particular restriction I have chosen is
described by the definitions in 455F.   I should of course say that these,
particularly 455Fb (`uniformly time-continuous on the right') are more
limiting than is strictly necessary;  in `Feller processes'
({\smc Rogers \& Williams 94}, III.6) we have a slightly different approach
to the same intuitive target.   The aim is to find sufficient conditions
for the `strong Markov property', in which we can find disintegrations and
conditional expectations associated with general stopping times, as in
455O.   To do this, we have to abandon the set
$\Omega=U^{\coint{0,\infty}}$ and move to the correct
set of full outer measure, the set $\Cdlg$ of `\cadlag' functions, which
dominates the central part of this section.   The first thing the
definition 455Fb must do is to ensure that $\Cdlg$ has full outer
measure not only for the distribution on $\Omega$ but also for the
conditional distributions we shall be using (455G).
If $U$ is a Polish space, $\Cdlg$ has a standard Borel structure (4A3Wb),
which is comforting.

I hope that you are becoming resigned to the view that the
notational complexities of this section are not
solely due to an inconsiderate disregard for the
reader's eyesight.   The original probability measures $\nu^{(s,t)}_x$
of 455A really do form a three-parameter family, the conversion of these
into finite-dimensional distributions $\lambda_J$ really is a multiple
repeated integral, the derived probabilities $\nu^{(s,t)}_{\omega ax}$
in 455B are
a five-parameter system.   Without wishing to insist on my use of
grave accents in the proof of 455E, it is surely safer to have a way of
distinguishing between completed and uncompleted measures, and while the
result may be `obvious', I think there are some twists on the way which not
everyone would foresee.   Again, if you wish to dispense with
the double-dotted symbols from 455O on, you will have to find some other
way of reminding yourself that we are looking at a new representation of
the process on a new probability space.

This treatise as a whole is theory-heavy and example-light.   I assure you
that all the theory here is in fact example-driven.   You should start with
the four examples of L\'evy processes in
455Xg-455Xj.   %455Xg 455Xh 455Xi 455Xj
Of these, 455Xg is {\bf Brownian motion}, the starting point
of the whole theory;  I will return to this in \S477.   A problem with the
formalization in 455A is that we have to start with an exact description of
the transitional probabilities $\nu^{(s,t)}_x$.   It does not help at all
in establishing the existence of such families matching
some probabilistic intuition.   Only in rather special cases do we have
elegant formulae for these systems.   In 455Xb, 455Xd and 455Xf I try to
show how the general theory gives us methods of using one
system to build others.

I suppose that 455O is the summit;  from here on the going is easier.
In 455P I introduce `L\'evy processes', a particularly interesting class
intermediate in generality between the continuous processes of 455O and
Brownian motion.   These have of course mostly been considered in the case
$U=\Bbb R$, but the extension to
Banach spaces $U$ is an obvious one, and we
can even manage non-abelian groups if we are careful.   (For an elementary
example of a process which can really exploit a non-abelian group,
see 455Xe.)   The `Poisson process' in 455Xh is by some way the most
important example after Brownian motion itself.
L\'evy processes on $\Bbb R$ are well understood;  the family
$\langle\lambda_t\rangle_{t>0}$ is determined by $\lambda_1$, any
`infinitely divisible' distribution can be taken for $\lambda_1$
(455Yc), and
a complete description of infinitely divisible distributions is provided by
the L\'evy-Khintchine representation theorem ({\smc Fristedt \& Gray 97},
16.3).
%Khintchine the spelling used in his Roman-alphabet papers? yes it seems
As a final result in the general theory, I give an alternative version of
the strong Markov property in 455U.   For L\'evy processes, we can
re-start, following any of the usual stopping times,
with an exact copy of the process,
and this corresponds to a true \imp\ function from $\Cdlg^2$ to $\Cdlg$.

A comment on 455T.   The idea behind the $\sigma$-algebras $\Sigma_t$,
$\ddot\Sigma_t$  of 455M, 455O and later is that they consist of events
`observable at time $t$', that is, determined by the path taken up to and
including time $t$.   We quickly find ourselves forced to consider
augmented algebras $\Sigma_t^+=\bigcap_{s>t}\Sigma_s$, where somehow we are
allowed infinitesimal intuitions into the immediate future.   (A typical
situation is that of 455Mb when the set $A$ is open, so that if
$\omega(t)\in\overline{A}$ we can expect that there will be paths which
continue immediately into $A$, and others which do not, and it may not be
obvious which, if either, should be regarded as typical.)   The question
is, whether $\Sigma_t$ is really different from $\Sigma_t^+$.
The claim of 455T is that
$\ddot\Sigma_t^+$ is included in a kind of completion $\hat{\ddot\Sigma_t}$
of $\ddot\Sigma_t$.
Of course the completion is in terms of the measure $\ddot\mu$ on the
whole space $\Cdlg$ of \cadlag\ paths;  we need advance knowledge of which
subsets of $\Cdlg$ are negligible.   But if we are interested in the
measure algebra $\frak A$ of $\ddot\mu$ and its closed subalgebras
$\frak A_t=\{E^{\ssbullet}:E\in\ddot\Sigma_t\}$, 455T tells us that (in the
context of L\'evy processes) we can expect to have
$\frak A_t=\bigcap_{s>t}\frak A_s$.   Turning to the definition of
$\ddot\mu$ in 455O as a subspace measure, we see that $\frak A$ can be
regarded as the measure algebra of the measure $\hat\mu$ on
$U^{\coint{0,\infty}}$ defined by the formulae of 455E;  and even that
$\frak A_t$ can be identified with

\Centerline{$\{E^{\ssbullet}:E\in\dom\hat\mu$, $E$ is determined by
coordinates in $[0,t]\}$}

\noindent (see part (a-ii) of the proof of 455O).   But I think that this
last step will not usually be helpful, because (as noted above) $\ddot\mu$
will commonly be a Radon measure for an appropriate topology, while
$\hat\mu$ is likely at best to be the completion of a Baire measure.

I have cast the second half of the section in terms of measures on
$\Cdlg$, because it is reasonably well adapted to L\'evy processes in
general.   When we come to look at particular processes, we often find
that there is a smaller class of functions
(e.g., continuous functions in the case of
Brownian motion, or non-decreasing $\Bbb N$-valued functions in the case of
the Poisson process) which is fully adequate and easier to focus on.
For the detailed study of such processes, as in \S477 below, I think it
will usually be helpful to make the shift.   But there may be rival
conegligible subsets of $\Cdlg$ with different virtues, as in 477Ef.
}%end of notes

\discrpage

