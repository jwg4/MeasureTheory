\frfilename{mt48.tex} 
\versiondate{9.5.11} 
\copyrightdate{2001} 
 
\def\chaptername{Gauge integrals} 
\def\sectionname{Introduction} 
 
\newchapter{48} 
 
For the penultimate chapter of this volume I turn to a completely different 
approach to integration which has been developed in the last fifty 
years, following {\smc Kurzweil 57} and {\smc Henstock 63}.   This 
depends for its inspiration on a formulation of the Riemann 
integral (see 481Xe), and leads in particular to some remarkable 
extensions of the Lebesgue integral (\S\S483-484).   While (in my view) 
it remains peripheral to the most important parts of measure theory, it 
has deservedly attracted a good deal of interest in recent years, and is 
entitled to a place here. 
 
From the very beginning, in the definitions of \S122, I have presented 
the Lebesgue integral in terms of almost-everywhere approximations by 
simple functions.   Because the integral $\int\lim_{n\to\infty}f_n$ of a 
limit is {\it not} always the limit $\lim_{n\to\infty}\int f_n$ of the 
integrals, we are forced, from the start, to constrain ourselves by 
the ordering, and to work with monotone or dominated sequences.   This 
almost automatically leads us to an `absolute' integral, in which $|f|$ 
is integrable whenever $f$ is, whether we start from measures (as in 
Chapter 11) or from linear functionals (as in \S436).   For four volumes 
now I have been happily developing the concepts and intuitions 
appropriate to such integrals.   But if we return to one of the 
foundation stones of Lebesgue's theory, the Fundamental Theorem of 
Calculus, we find that it is easy to construct a differentiable 
function $f$ such that the absolute value $|f'|$ of its derivative is 
not integrable (483Xd).   It was observed very early ({\smc Perron 1914}) 
that the Lebesgue integral can be extended to integrate the derivative 
of any function which is differentiable everywhere.   The achievement of 
{\smc Henstock 63} was to find a formulation of this extension which was 
conceptually transparent enough to lend itself to a general theory, 
fragments of which I will present here. 
 
The first step is to set out the essential structures on which the 
theory depends (\S481), with a first attempt at a classification scheme. 
(One of the most interesting features of the Kurzweil-Henstock approach 
is that we have an extraordinary degree of freedom in describing our 
integrals, and apart from the Henstock integral itself it is not clear 
that we have yet found the right canonical forms to use.)   In \S482 I 
give a handful of general theorems showing what kinds of result can be 
expected and what difficulties arise.   In \S483, I work through the 
principal properties of the Henstock integral on the real line, showing, 
in particular, that it coincides with the Perron and special Denjoy 
integrals.   Finally, in \S484, I look at a very striking 
integral on $\BbbR^r$, due to W.F.Pfeffer. 
 
\discrpage 
 
