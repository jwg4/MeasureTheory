\frfilename{mt50I.tex}
\versiondate{15.3.08}
%\def\vtmp#1{#1}
%\def\vtmpb#1{(#1)}
%\def\vtmp#1{}
     
\Loadfourteens
\hyphenation{Dede-kind}
     
\gdef\topparagraph{}
\gdef\bottomparagraph{}
     
\vbox{\vskip 2truein
     
\centerline{\fourteenbf MEASURE THEORY}
}%end of vbox
     
\bigskip
     
\centerline{\fourteenbf Volume 5}
     
\bigskip

% \centerline{\fourteenbf Part I}
     
\vskip 2truein
     
\centerline{\fourteenrm D.H.Fremlin}
     
\vskip 2truein
     
\largelogofalse
\input mtlogo
     
\vfill\eject
     
\vbox{\vskip 2truein
     
\noindent By the same author:
     
{\it Topological Riesz Spaces and Measure Theory}, Cambridge University
Press, 1974.
     
{\it Consequences of Martin's Axiom}, Cambridge University Press, 1982.
}%end of vbox
     
\bigskip
     
\noindent Companions to the present volume:
     
{\it Measure Theory}, vol.\ 1, Torres Fremlin, 2000.
     
{\it Measure Theory}, vol.\ 2, Torres Fremlin, 2001.
     
{\it Measure Theory}, vol.\ 3, Torres Fremlin, 2002.
     
{\it Measure Theory}, vol.\ 4, Torres Fremlin, 2003.
     
\vfill
     
\Centerline{\it First printing 2008}
     
\Centerline{\it Second printing 2015}
     
\vfill\eject
\def\Loadtwenties{
      \font\twentyrm=cmr10 scaled \magstep4
      \font\twentybf=cmbx10 scaled \magstep4
      \font\twentyit=cmsl10 scaled \magstep4}
     
\Loadtwenties
     
\vbox{\vskip 1.5truein
     
\centerline{\twentybf MEASURE THEORY}
}%end of vbox
     
\bigskip\bigskip
     
\centerline{\twentybf Volume 5}
     
\bigskip\bigskip
     
\centerline{\twentyrm Set-theoretic Measure Theory}
     
\bigskip\bigskip

% \centerline{\twentyrm Part I}
     
\vskip 1truein
\vfill
     
\centerline{\twentyrm D.H.Fremlin}
     
\bigskip\bigskip
     
\centerline{\fourteenit Emeritus Professor in Mathematics, 
University of Essex}
     
\vskip 1truein
\vfill
     
\largelogotrue
\input mtlogo
     
\eject
     
\iflulu

\vbox{\vskip 6truecm
     
\Centerline{\fourteenit Dedicated by the Author}
     
\Centerline{\fourteenit to the Publisher}
}%end of vbox
          
\vfill
     
\Centerline{This book may be ordered from the printers,
{\tt http://www.lulu.com}}
     
\vfill
     
\bigskip
     
\noindent First published in 2008
     
by Torres Fremlin, 25 Ireton Road, Colchester CO3 3AT, England
     
\medskip
     
\noindent\copyright\ D.H.Fremlin 2008
     
\noindent The right of D.H.Fremlin to be identified as author of this
work has been asserted in accordance with the Copyright, Designs and
Patents Act 1988.
This work is issued under the terms of the Design Science License as 
published in {\tt http://www.gnu.org/licenses/dsl.html}.
For the source files see {\tt http://www.essex.\discretionary{}
{}{}ac.uk/maths/people{\bsp}fremlin{\bsp}mt5.2015/index.htm}.
     
\medskip
     
\noindent Library of Congress classification QA312.F72
     
\medskip
     
\noindent AMS 2010 classification 28-01
     
\medskip
%\noindent British Library Cataloguing in Publication data
     
     
%\medskip
     
\noindent ISBN 978-0-9538129-5-0
     
\medskip
     
\noindent Typeset by \AmSTeX
     
\medskip
     
\noindent Printed by Lulu.com

\else\vbox{\vskip 5truecm
     
\Centerline{\fourteenit Dedicated by the Author}
     
\Centerline{\fourteenit to the Publisher}
}%end of vbox
     
\vfill
     
\vbox{\narrower\narrower%\narrower%\narrower
\noindent This book may be ordered from the
publisher at the address below.   For price and means of payment see the
author's Web page
{\tt http://www.essex.ac.uk/maths/people/fremlin{\penalty-50}/mtsales.htm}, or enquire from {\tt fremdh\@essex.ac.uk}.
}%end of vbox
     
\vfill
     
\bigskip
     
\noindent First published in 2008
     
by Torres Fremlin, 25 Ireton Road, Colchester CO3 3AT, England
     
\medskip
     
\noindent\copyright\ D.H.Fremlin 2008
     
\noindent The right of D.H.Fremlin to be identified as author of this
work has been asserted in accordance with the Copyright, Designs and
Patents Act 1988.
This work is issued under the terms of the Design Science License as 
published in {\tt http://www.gnu.org/licenses/dsl.html}.
For the source files see {\tt http://www.essex.\discretionary{}
{}{}ac.uk/maths/people{\bsp}fremlin/mt5.2015/index.htm}.
     
%No commercial use may be made of any part of this work without
%permission in writing from the publisher.
     
\medskip
     
\noindent Library of Congress classification QA312.F72
     
\medskip
     
\noindent AMS 2010 classification 28-01
     
\medskip
%\noindent British Library Cataloguing in Publication data
     
     
%\medskip
     
\noindent ISBN 0-9538129-5-2
     
\medskip
     
\noindent Typeset by \AmSTeX
     
\medskip
     
\noindent Printed in England by Digital Books Logistics, Peterborough\query
\fi
     
\eject
     
\pageno=5
     
\vbox{\ifresultsonly\vskip1truein\fi
     
\centerline{\bf Contents}}
     
\bigskip
     
General Introduction \pagereference{9}{9}
     
\medskip
     
\input mt05I
     
\wheader{}{10}{10}{4}{120pt} 
 
% \centerline{\bf Part II}

\input mt05II
     
\vfill\eject
     
\gdef\topparagraph{}
\gdef\newparagraph{{\it General introduction}}
\gdef\bottomparagraph{{\it General introduction}}
     
\def\volumename{Set-theoretic Measure Theory}
     
\bigskip
     
\noindent{\bf General introduction} In this treatise I aim to give a
comprehensive description of modern abstract measure theory, with some
indication of its principal applications.   The first two volumes are
set at an introductory level;  they are intended for students with a
solid grounding in the concepts of real analysis, but possibly with
rather limited detailed knowledge.    As the book proceeds, the
level of sophistication and expertise demanded will increase;  thus for
the volume on topological measure spaces, familiarity with general
topology will be assumed.
The emphasis throughout is on the mathematical ideas involved, which in
this subject are mostly to be found in the details of the proofs.
     
My intention is that the book should be usable both as a first
introduction to the subject and as a reference work.   For the sake of
the first aim, I try to limit the ideas of the early volumes to those
which are really essential to the development of the basic theorems.
For the sake of the second aim, I try to express these ideas in their
full natural generality, and in particular I take care to avoid
suggesting any unnecessary restrictions in their applicability.   Of
course these principles are to to some extent contradictory.
Nevertheless, I find that most of the time they are very nearly
reconcilable, {\it provided} that I indulge in a certain degree of
repetition.   For instance, right at the beginning, the puzzle arises:
should one develop Lebesgue measure first on the real line, and then in
spaces of higher dimension, or should one go straight to the
multidimensional case?   I believe that there is no single correct
answer to this question.   Most students will find the one-dimensional
case easier, and it therefore seems more appropriate for a first
introduction, since even in that case the technical problems can be
daunting.   But certainly every student of measure theory must at a
fairly early stage
come to terms with Lebesgue area and volume as well as length;
and with the correct formulations, the multidimensional case differs
from the one-dimensional case only in a definition and a (substantial)
lemma.   So what I have done is to write them both out (\S\S114-115).
In the same spirit, I have been uninhibited, when setting out exercises,
by the fact that many of the results I invite students to look for will
appear in later chapters;  I believe that throughout mathematics one has
a better chance of understanding a theorem if one has previously
attempted something similar alone.
     
The plan of the work is as follows:
     
\medskip
     
\qquad\qquad Volume 1:  The Irreducible Minimum
     
\qquad\qquad Volume 2:  Broad Foundations
     
\qquad\qquad Volume 3:  Measure Algebras
     
\qquad\qquad Volume 4:  Topological Measure Spaces
     
\qquad\qquad Volume 5:  Set-theoretic Measure Theory.
     
\medskip
     
\noindent Volume 1 is intended for those with no prior knowledge of
measure theory, but competent in the elementary techniques of real
analysis.  I hope that it will be found useful by undergraduates meeting
Lebesgue measure for the first time.    Volume 2 aims to lay out some of
the fundamental results of pure measure theory (the
Radon-Nikod\'ym theorem, Fubini's theorem), but also gives short
introductions to some of the most important applications of measure
theory (probability theory, Fourier analysis).   While I should like to
believe that most of it is written at a level accessible to anyone who
has mastered the contents of Volume 1, I should not myself have the
courage to try to cover it in an undergraduate course, though I would
certainly attempt to include some parts of it.   Volumes 3 and 4 are
set at a rather higher level, suitable to postgraduate courses;  while
Volume 5 assumes a wide-ranging competence over large parts of
analysis and set theory.
     
There is a disclaimer which I ought to make in a place where you might
see it in time to avoid paying for this book.   I make no real attempt to
describe the history of the subject.   This is not because I think the
history uninteresting or unimportant;  rather, it is because I have no
confidence of saying anything which would not be seriously misleading.
Indeed I have very little confidence in anything I have ever read
concerning the history of ideas.   So while I am happy to honour the
names of Lebesgue and Kolmogorov and Maharam in more or less appropriate
places, and I try to include in the bibliographies the works which I
have myself consulted, I leave any consideration of the details to those
bolder and better qualified than myself.
     
For the time being, at
least, printing will be in short runs.   I hope that readers will be
energetic in commenting on errors and omissions, since it should be
possible to correct these relatively promptly.   An inevitable
consequence of this is that paragraph references may go out of date
rather quickly.   I shall be most flattered if anyone chooses to rely on
this book as a source for basic material;  and I am willing to attempt
to maintain a concordance to such references, indicating where migratory
results have come to rest for the moment, if authors will supply me with
copies of papers which use them.
     
I mention some minor points concerning the layout of the material.
Most sections conclude with lists of `basic exercises' and
`further exercises', which I hope will be generally instructive and
occasionally entertaining.   How many of these you should attempt must
be for you and your teacher, if any, to decide, as no two students will
have quite the same needs.   I mark with a $\pmb{>}$ those which seem to
me to be particularly important.   But while you may not need to write
out solutions to all the `basic exercises', if you are in any doubt
as to your capacity to do so you should take this as a warning to slow
down a bit.   The `further exercises' are unbounded in difficulty,
and are unified only by a presumption that each has at least one
solution based
on ideas already introduced.   Occasionally I add a final `problem',
a question to which I do not know the answer and which seems to arise
naturally in the course of the work.
     
The impulse to write this book is in large part a desire to present a
unified account of the subject.   Cross-references are correspondingly
abundant and wide-ranging.  In order to be able to refer freely across
the whole
text, I have chosen a reference system which gives the same code name to
a paragraph wherever it is being called from.   Thus 132E is the fifth
paragraph in the second section of the
third chapter of Volume 1, and is referred to by that name throughout.
Let me emphasize that cross-references are supposed to help the reader,
not distract her.   Do not take the interpolation `(121A)' as an
instruction, or even a recommendation, to lift Volume 1 off the shelf
and hunt for \S121.   If you are happy with an argument as it stands,
independently of the reference, then carry on.   If, however, I seem to
have made rather a large jump, or the notation has suddenly become
opaque, local cross-references may help you to fill in the gaps.
     
Each volume will have an appendix of `useful facts', in which I set
out material which is called on somewhere in that volume, and which I do
not feel I can take for granted.   Typically the arrangement of material
in these appendices is directed very narrowly at the particular
applications I have in mind, and is unlikely to be a satisfactory
substitute for conventional treatments of the topics touched on.
Moreover, the ideas may well be needed only on rare and isolated
occasions.   So as a rule I
recommend you to ignore the appendices until you have some
direct reason to suppose that a fragment may be useful to you.
     
During the extended gestation of this project I have been helped by many
people, and I hope that my friends and colleagues will be pleased when
they recognise their ideas scattered through the pages below.   But I am
especially grateful to those who have taken the trouble to read through
earlier drafts and comment on obscurities and errors.

There is a particular debt which may not be obvious from the text,
and which I ought to acknowledge.   
From 1984 to 2006 the biennial CARTEMI conferences, organized by the
Department of Mathematics of the University Federico II of Naples, were the
principal meeting place of European measure theorists, and a clearing house
for ideas from all over the world.   I had the good fortune to attend
nearly all the meetings from 1988 onwards.   I do not think it is a
coincidence that I should have started work on this book in 1992;  and from
then on every meeting has contributed something to its content.   It would
have been very different, probably shorter, but certainly duller, without
this regular stimulation.   Now the CARTEMI conferences, while of course
dependent on the energies and talents of many people, were essentially the
creation of one man, whose vision and determination maintained a consistent
level of quality and variety.   So while the dedication on the title page
must remain to my wife, without whose support and forbearance the project
would have been simply impossible, I should like to offer a second
dedication here, to my friend Paolo de Lucia.


\wheader{{\it Introduction to Volume 5}}{18}{6}{6}{48pt}
     
\input mt5
     
