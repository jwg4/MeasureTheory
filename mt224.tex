\frfilename{mt224.tex}
\versiondate{29.9.04}
\copyrightdate{1997}

\def\chaptername{The Fundamental Theorem of Calculus}
\def\sectionname{Functions of bounded variation}

\newsection{224}

I turn now to the second of the two problems to which this chapter is
devoted:  the identification of those real functions which are
indefinite integrals.   I take the opportunity to offer a brief
introduction to the theory of functions of bounded variation, which are
interesting in themselves and will be important in Chapter 28.   I give
the
basic characterization of these functions as differences of monotonic
functions (224D), with a representative sample of their elementary
properties.

\leader{224A}{Definition} Let $f$ be a
real-valued function  and  $D$ a subset of $\Bbb R$.   I define
$\Var_D(f)$, the {\bf (total) variation of $f$ on $D$}, as follows.   If
$D\cap\dom f=\emptyset$, $\Var_D(f)=0$.   Otherwise, $\Var_D(f)$ is

\Centerline{$\sup\{\sum_{i=1}^n|f(a_i)-f(a_{i-1})|:
  a_0,a_1,\ldots,a_n\in D\cap\dom f,\,a_0\le a_1\le\ldots\le a_n\}$,}

\noindent allowing $\Var_{D}(f)=\infty$.
If $\Var_{D}(f)$ is finite, we say that $f$ is {\bf of bounded
variation} on $D$.   \cmmnt{If the context seems clear,} I may write
$\Var f$ for $\Var_{\dom f}(f)$, and say that $f$ is simply `{\bf of
bounded variation}' if this is finite.

\leader{224B}{Remarks}\cmmnt{ {\bf (a)} In the present chapter, we
shall virtually exclusively be concerned with the case in which $D$ is a
bounded closed interval included in $\dom f$.   The general formulation
will be useful for some technical questions arising in Chapter 28;  but
if it makes
you more comfortable, you will lose nothing by supposing for the moment
that $D$ is an interval.

\spheader 224Bb Clearly}

\Centerline{$\Var_D(f)=\Var_{D\cap\dom f}(f)=\Var(f\restr D)$}

\noindent for all $D$, $f$.

\leader{224C}{Proposition} (a) If $f$, $g$ are two real-valued functions
and $D\subseteq\Bbb R$, then

\Centerline{$\Var_{D}(f+g)\le\Var_{D}(f)+\Var_{D}(g)$.}

(b) If $f$ is a real-valued function, $D\subseteq\Bbb R$ and
$c\in\Bbb R$ then $\Var_{D}(cf)=|c|\Var_{D}(f)$.

(c) If $f$ is a real-valued function, $D\subseteq\Bbb R$ and
$x\in \Bbb R$ then

\Centerline{$\Var_{D}(f)
\ge\Var_{D\cap\ocint{-\infty,x}}(f)+\Var_{D\cap\coint{x,\infty}}(f)$,}

\noindent with equality if $x\in D\cap\dom f$.

(d) If $f$ is a real-valued function and $D\subseteq D'\subseteq\Bbb R$
then $\Var_D(f)\le\Var_{D'}(f)$.

(e) If $f$ is a real-valued function and $D\subseteq\Bbb R$, then
$|f(x)-f(y)|\le\Var_D(f)$ for all $x$, $y\in D\cap\dom f$;  so if $f$ is
of bounded variation on $D$ then $f$ is bounded on $D\cap\dom f$ and (if
$D\cap\dom f\ne\emptyset$)

\Centerline{$\sup_{y\in D\cap\dom f}|f(y)|\le |f(x)|+\Var_D(f)$}

\noindent for every $x\in D\cap\dom f$.

(f) If $f$ is a monotonic real-valued function and $D\subseteq\Bbb R$
meets $\dom f$, then
\discrcenter{468pt}
{$\Var_D(f)=\sup_{x\in D\cap\dom f}f(x)-\inf_{x\in D\cap\dom f}f(x)$.}

\proof{{\bf (a)}  If $D\cap\dom(f+g)=\emptyset$ this is trivial,
because $\Var_D(f)$ and $\Var_D(g)$  are surely non-negative.
Otherwise, if
$a_0\le\ldots\le a_n$ in $D\cap\dom(f+g)$, then

$$\eqalign{\sum_{i=1}^n|(f+g)(a_i)-(f+g)(a_{i-1})|
&\le \sum_{i=1}^n|f(a_i)-f(a_{i-1})|
+\sum_{i=1}^n|g(a_i)-g(a_{i-1})|\cr
&\le\Var_{D}(f)+\Var_{D}(g);\cr}$$

\noindent as $a_0,\ldots,a_n$ are arbitrary,
$\Var_{D}(f+g)\le\Var_{D}(f)+\Var_{D}(g)$.

\medskip

{\bf (b)}

\Centerline{$\sum_{i=1}^n|(cf)(a_i)-(cf)(a_{i-1})|
=|c|\sum_{i=1}^n|f(a_i)-f(a_{i-1})|$}

\noindent whenever
$a_0\le\ldots\le a_n$ in $D\cap\dom f$.

\medskip

{\bf (c)(i)} If either $D\cap\ocint{-\infty,x}\cap\dom f$ or
$D\cap\coint{x,\infty}\cap\dom f$ is empty, this is trivial.   If
$a_0\le\ldots\le a_m$ in $D\cap\ocint{-\infty,x}\cap\dom f$,
$b_0\le\ldots\le b_n$ in $D\cap\coint{x,\infty}\cap\dom f$, then

$$\eqalign{\sum_{i=1}^m|f(a_i)-f(a_{i-1})|
+\sum_{j=1}^n|f(b_i)-f(b_{i-1})|
&\le\sum_{i=1}^{m+n+1}|f(a_i)-f(a_{i-1})|\cr
&\le\Var_{[a,b]}(f),\cr}$$

\noindent if we write $a_i=b_{i-m-1}$ for $m+1\le i\le m+n+1$.
So

\Centerline{$\Var_{D\cap\ocint{-\infty,x}}(f)
+\Var_{D\cap\coint{x,\infty}}(f)
\le\Var_D(f)$.}

\medskip

\quad{\bf (ii)} Now suppose that $x\in D\cap\dom f$.
If $a_0\le\ldots\le a_n$ in $D\cap\dom f$, and $a_0\le x\le a_n$, let
$k$ be such that $x\in [a_{k-1},a_k]$;
then

$$\eqalign{\sum_{i=1}^n|f(a_i)-f(a_{i-1})|
&\le\sum_{i=1}^{k-1}|f(a_i)-f(a_{i-1})|
+|f(x)-f(a_{k-1})|\cr
&\mskip150mu +|f(a_k)-f(x)|+\sum_{i=k+1}^n|f(a_i)-f(a_{i-1})|\cr
&\le\Var_{D\cap\ocint{-\infty,x}}(f)
+\Var_{D\cap\coint{x,\infty}}(f)
\cr}$$

\noindent (counting empty sums $\sum_{i=1}^0$,
$\sum_{i=n+1}^n$ as $0$).   If $x\le a_0$ then
$\sum_{i=1}^n|f(a_i)-f(a_{i-1})|\le\Var_{D\cap\coint{x,\infty}}(f)$;
if $x\ge a_n$ then
$\sum_{i=1}^n|f(a_i)-f(a_{i-1})|\le\Var_{D\cap\ocint{-\infty,x}}(f)$.
Thus

\Centerline{$\sum_{i=1}^n|f(a_i)-f(a_{i-1})|
\le\Var_{D\cap\ocint{-\infty,x}}(f)
+\Var_{D\cap\coint{x,\infty}}(f)
$}

\noindent in all cases;  as $a_0,\ldots,a_n$ are arbitrary,

\Centerline{$\Var_D(f)
\le\Var_{D\cap\ocint{-\infty,x}}(f)
+\Var_{D\cap\coint{x,\infty}}(f)$.}

\noindent So the two sides are equal.

\medskip

{\bf (d)} is trivial.

\medskip

{\bf (e)} If $x$, $y\in D\cap\dom f$ and $x\le y$ then

\Centerline{$|f(x)-f(y)|=|f(y)-f(x)|\le \Var_D(f)$}

\noindent by the definition of $\Var_D$;  and the same is true if
$y\le x$.   So of course $|f(y)|\le|f(x)|+\Var_D(f)$.

\medskip

{\bf (f)} If $f$ is non-decreasing, then

$$\eqalign{\Var_D(f)
&=\sup\{\sum_{i=1}^n|f(a_i)-f(a_{i-1})|:a_0,a_1,\ldots,a_n\in
D\cap\dom f,\,a_0\le
a_1\le\ldots\le a_n\}\cr
&=\sup\{\sum_{i=1}^nf(a_i)-f(a_{i-1}):a_0,a_1,\ldots,a_n\in
D\cap\dom f,\,a_0\le
a_1\le\ldots\le a_n\}\cr
&=\sup\{f(b)-f(a):a,\,b\in D\cap\dom f,\,a\le b\}\cr
&=\sup_{b\in D\cap\dom f}f(b)-\inf_{a\in D\cap\dom f}f(a).\cr}$$

\noindent If $f$ is non-increasing then

$$\eqalign{\Var_D(f)
&=\sup\{\sum_{i=1}^n|f(a_i)-f(a_{i-1})|:a_0,a_1,\ldots,a_n\in
D\cap\dom f,\,a_0\le
a_1\le\ldots\le a_n\}\cr
&=\sup\{\sum_{i=1}^nf(a_{i-1})-f(a_i):a_0,a_1,\ldots,a_n\in
D\cap\dom f,\,a_0\le
a_1\le\ldots\le a_n\}\cr
&=\sup\{f(a)-f(b):a,\,b\in D\cap\dom f,\,a\le b\}\cr
&=\sup_{a\in D\cap\dom f}f(a)-\inf_{b\in D\cap\dom f}f(b).\cr}$$
}%end of proof of 224C

\leader{224D}{Theorem} For any real-valued function $f$ and any set
$D\subseteq \Bbb R$, the following are equiveridical:

\quad (i) there are two bounded
non-decreasing functions $f_1$, $f_2:\Bbb R\to\Bbb R$ such that
$f=f_1-f_2$ on $D\cap\dom f$;

\quad (ii) $f$ is of bounded variation on $D$;

\quad (iii) there are bounded
non-decreasing functions $f_1$, $f_2:\Bbb R\to\Bbb R$ such that
$f=f_1-f_2$ on $D\cap\dom f$ and $\Var_D(f)=\Var f_1+\Var f_2$.

\proof{{\bf (i)$\Rightarrow$(ii)} If $f:\Bbb R\to\Bbb R$ is
bounded and non-decreasing, then
$\Var f=\sup_{x\in\Bbb R}f(x)-\inf_{x\in\Bbb R}f(x)$ is
finite.   So if $f$ agrees on $D\cap\dom f$ with $f_1-f_2$
where $f_1$ and $f_2$ are bounded and non-decreasing, then

$$\eqalign{\Var_D(f)
&=\Var_{D\cap\dom f}(f)
\le\Var_{D\cap\dom f}(f_1)+\Var_{D\cap\dom f}(f_2)\cr
&\le\Var f_1+\Var f_2
<\infty,}$$

\noindent using (a), (b) and (d) of 224C.

\medskip

{\bf(ii)$\Rightarrow$(iii)} Suppose that $f$ is of bounded variation on
$D$.    Set $D'=D\cap\dom f$.   If $D'=\emptyset$ we can take both $f_j$
to be the zero function, so henceforth suppose that $D'\ne\emptyset$.
Write

\Centerline{$g(x)=\Var_{D\cap\ocint{-\infty,x}}(f)$}

\noindent for $x\in D'$.   Then $g_1=g+f$ and $g_2=g-f$ are both
non-decreasing.   \Prf\ If $a$, $b\in D'$ and $a\le b$, then

\Centerline{$g(b)=g(a)+\Var_{D\cap[a,b]}(f)\ge g(a)+|f(b)-f(a)|$.}

\noindent So

\Centerline{$g_1(b)-g_1(a)=g(b)-g(a)+f(b)-f(a)$,
\quad$g_2(b)-g_2(a)=g(b)-g(a)-f(b)+f(a)$}

\noindent are both non-negative.\ \Qed

Now there are non-decreasing functions $h_1$, $h_2:\Bbb R\to\Bbb R$,
extending $g_1$, $g_2$ respectively, such that $\Var h_j=\Var g_j$ for
both $j$.   \Prf\  $f$
is bounded on $D$, by 224Ce, and $g$ is bounded just because
$\Var_D(f)<\infty$, so that $g_j$ is bounded.   Set
$c_j=\inf_{x\in D'}g_j(x)$ and

\Centerline{$h_j(x)=\sup(\{c_j\}\cup\{g_j(y):y\in D',\,y\le x\})$}

\noindent for every $x\in\Bbb R$;  this works.\ \QeD\   Observe that for
$x\in D'$,

\Centerline{$h_1(x)+h_2(x)=g_1(x)+g_2(x)=g(x)+f(x)+g(x)-f(x)=2g(x)$,}

\Centerline{$h_1(x)-h_2(x)=2f(x)$.}

Now, because $g_1$ and $g_2$ are non-decreasing,

\Centerline{$\sup_{x\in D'}g_1(x)+\sup_{x\in D'}g_2(x)=\sup_{x\in
D'}g_1(x)+g_2(x)=2\sup_{x\in D'}g(x)$,}

\Centerline{$\inf_{x\in D'}g_1(x)+\inf_{x\in D'}g_2(x)=\inf_{x\in
D'}g_1(x)+g_2(x)=2\inf_{x\in D'}g(x)\ge 0$.}

\noindent But this means that

\Centerline{$\Var h_1+\Var h_2
=\Var g_1+\Var g_2
=2\Var g
\le 2\Var_D(f)$,}

\noindent using 224Cf three times.   So if we set
$f_j(x)=\bover12h_j(x)$
for $j\in\{1,2\}$ and $x\in\Bbb R$, we shall have non-decreasing functions
such that

\Centerline{$f_1(x)-f_2(x)=f(x)$ for $x\in D'$,
\quad$\Var f_1+\Var f_2=\Bover12\Var h_1+\Bover12\Var h_2\le\Var_D(f)$.}

\noindent Since we surely also have

\Centerline{$\Var_D(f)\le\Var_D(f_1)+\Var_D(f_2)\le\Var f_1+\Var f_2$,}

\noindent we see that $\Var_D(f)=\Var f_1+\Var f_2$, and (iii) is true.

\medskip

{\bf (iii)$\Rightarrow$(i)} is trivial.
}%end of proof of 224D

\leader{224E}{Corollary} Let $f$ be a real-valued function and $D$ any
subset of $\Bbb R$.   If $f$ is of bounded variation on $D$, then

\Centerline{$\lim_{x\downarrow a}\Var_{D\cap\ocint{a,x}}(f)
=\lim_{x\uparrow a}\Var_{D\cap\coint{x,a}}(f)=0$}
\noindent for every $a\in\Bbb R$, and

\Centerline{$\lim_{a\to-\infty}\Var_{D\cap\ocint{-\infty,a}}(f)
=\lim_{a\to\infty}\Var_{D\cap\coint{a,\infty}}(f)=0$.}

\proof{{\bf (a)} Consider first the case in which
$D=\dom f=\Bbb R$ and $f$ is a bounded non-decreasing function.   Then

\Centerline{$\Var_{D\cap\ocint{a,x}}(f)
=\sup_{y\in\ocint{a,x}}f(x)-f(y)
=f(x)-\inf_{y>a}f(y)
=f(x)-\lim_{y\downarrow a}f(y)$,}

\noindent so of course

\Centerline{$\lim_{x\downarrow a}\Var_{D\cap\ocint{a,x}}(f)
=\lim_{x\downarrow a}f(x)-\lim_{y\downarrow a}f(y)=0$.}

\noindent In the same way

\Centerline{$\lim_{x\uparrow a}\Var_{D\cap\coint{x,a}}(f)
=\lim_{y\uparrow a}f(y)-\lim_{x\uparrow a}f(x)=0$,}

\Centerline{$\lim_{a\to-\infty}\Var_{D\cap\ocint{-\infty,a}}(f)
=\lim_{a\to-\infty}f(a)-\lim_{y\to-\infty}f(y)=0$,}

\Centerline{$\lim_{a\to\infty}\Var_{D\cap\coint{a,\infty}}(f)
=\lim_{y\to\infty}f(y)-\lim_{a\to\infty}f(a)=0$.}

\medskip

{\bf (b)} For the general case, define $f_1$, $f_2$ from $f$ and $D$ as
in 224D.   Then for every interval $I$ we have

\Centerline{$\Var_{D\cap I}(f)\le\Var_I(f_1)+\Var_I(f_2)$,}

\noindent so the results for $f$ follow from those for $f_1$ and $f_2$
as established in part (a) of the proof.
}%end of proof of 224E

\leader{224F}{Corollary} Let $f$ be a real-valued function of bounded
variation on $[a,b]$, where $a<b$.   If $\dom f$ meets every interval
$\ocint{a,a+\delta}$ with $\delta>0$,   then

\Centerline{$\lim_{t\in\dom f,t\downarrow a}f(t)$}

\noindent is defined in $\Bbb R$.   If $\dom f$ meets
$\coint{b-\delta,b}$ for every $\delta>0$, then

\Centerline{$\lim_{t\in\dom f,t\uparrow b}f(t)$}

\noindent is defined in $\Bbb R$.

\wheader{224F}{0}{0}{0}{36pt}

\proof{ Let $f_1$, $f_2:\Bbb R\to\Bbb R$ be non-decreasing
functions such that $f=f_1-f_2$ on $[a,b]\cap\dom f$.   Then

\Centerline{$\lim_{t\in\dom f,t\downarrow a}f(t)
=\lim_{t\downarrow a}f_1(t)-\lim_{t\downarrow a}f_2(t)
=\inf_{t>a}f_1(t)-\inf_{t>a}f_2(t)$,}

\Centerline{$\lim_{t\in\dom f,t\uparrow b}f(t)
=\lim_{t\uparrow b}f_1(t)-\lim_{t\uparrow b}f_2(t)
=\sup_{t<b}f_1(t)-\sup_{t<b}f_2(t)$.}
}%end of proof of 224F

\leader{224G}{Corollary} Let $f$, $g$ be real functions and $D$ a subset
of $\Bbb R$.   If $f$ and $g$ are of bounded variation on $D$, so is
$f\times g$.

\proof{{\bf (a)} The point is that there are {\it non-negative}
bounded non-decreasing functions $f_1$, $f_2:\Bbb R\to\Bbb R$ such that
$f=f_1-f_2$ on $D\cap\dom f$.   \Prf\ We know that there are bounded
non-decreasing $h_1$, $h_2$ such that $f=h_1-h_2$ on $D\cap\dom f$.
Set $\gamma_i=\inf_{x\in\Bbb R}h_i(x)$ for $i=1$, $2$,

\Centerline{$\beta_1=\max(\gamma_1-\gamma_2,0)$,
\quad$\beta_2=\max(\gamma_2-\gamma_1,0)$,}

\Centerline{$f_1=h_1-\gamma_1+\beta_1$,
\quad$f_2=h_1-\gamma_2+\beta_2$;}

\noindent this works.\ \Qed

\medskip

{\bf (b)} Now taking similar functions $g_1$, $g_2$ such that
$g=g_1-g_2$ on $D\cap\dom g$, we have

\Centerline{$f\times g=f_1\times g_1-f_2\times g_1-f_1\times g_2
+f_2\times g_2$}

\noindent everywhere in $D\cap\dom(f\times g)=D\cap\dom f\cap\dom g$;
but all the $f_i\times g_j$ are bounded non-decreasing functions, so of
bounded variation, and $f\times g$ must be of bounded variation on $D$.
}%end of proof of 224G

\leader{224H}{Proposition} Let $f:D\to\Bbb R$ be a function of bounded
variation, where $D\subseteq\Bbb R$.   Then $f$ is continuous at all
except countably many points of $D$.

\proof{ For $n\ge 1$ set

$$\eqalign{A_n
&=\{x:x\in D,\text{ for every }\delta>0\text{ there is a }y\in
D\cap[x-\delta,x+\delta]\cr
&\qquad\qquad\qquad\qquad\text{ such that }|f(y)-f(x)|
  \ge\Bover1n\}.\cr}$$

\noindent Then  $\#(A_n)\le n\Var f$.   \Prf\Quer\ Otherwise, we can
find distinct $x_0,\ldots,x_k\in A_n$ with $k+1>n\Var f$.   Order these
so that $x_0<x_1<\ldots<x_k$.   Set
$\delta=\bover12\min_{1\le i\le k}x_i-x_{i-1}>0$.   For each $i$, there
is a $y_i\in D\cap[x_i-\delta,x_i+\delta]$ such that
$|f(y_i)-f(x_i)|\ge\bover1n$.   Take $x'_i$, $y'_i$ to be $x_i$, $y_i$
in order, so that $x'_i<y'_i$.   Now

\Centerline{$x'_0\le y'_0\le x'_1\le y'_1\le\ldots\le x'_k\le y'_k$,}
\noindent and

\Centerline{$\Var f
\ge\sum_{i=0}^k|f(y'_i)-f(x'_i)|
=\sum_{i=0}^k|f(y_i)-f(x_i)|
\ge \Bover1n(k+1)
>\Var f$,}

\noindent which is impossible.\ \Bang\Qed

It follows that $A=\bigcup_{n\in\Bbb N}A_n$ is countable, being a
countable
union of finite sets.   But $A$ is exactly the set of points of $D$ at
which $f$ is not continuous.
}%end of proof of 224H

\leader{224I}{Theorem} Let $I\subseteq\Bbb R$ be an interval, and
$f:I\to\Bbb R$ a function of bounded variation.   Then $f$ is
differentiable almost everywhere in $I$, and $f'$ is integrable over
$I$, with

\Centerline{$\int_I|f'|\le\Var_I(f)$.}

\proof{{\bf (a)} Let $f_1$ and $f_2$ be non-decreasing functions such
that $f=f_1-f_2$ everywhere in $I$ (224D).   Then $f_1$ and $f_2$ are
differentiable almost everywhere (222A).   At any point of $I$ except
possibly its endpoints, if any, $f$ will be differentiable if $f_1$ and
$f_2$ are, so $f'(x)$ is defined for almost every $x\in I$.

\medskip

{\bf (b)} Set $F(x)=\Var_{I\cap\ocint{-\infty,x}}f$ for $x\in\Bbb R$.
If $x$, $y\in I$ and $x\le y$, then

\Centerline{$F(y)-F(x)=\Var_{[x,y]}f\ge|f(y)-f(x)|$,}

\noindent by 224Cc;  so $F'(x)\ge|f'(x)|$ whenever $x$ is an interior
point of $I$ and both derivatives exist, which is almost everywhere.
So $\int_I|f'|\le\int_IF'$.   But if $a$, $b\in I$ and $a\le b$,

\Centerline{$\int_a^bF'\le F(b)-F(a)\le F(b)\le\Var f$.}

\noindent Now $I$ is expressible as $\bigcup_{n\in\Bbb N}[a_n,b_n]$
where $a_{n+1}\le a_n\le b_n\le b_{n+1}$ for every $n$.   So

$$\eqalignno{\int_I|f'|
&\le\int_IF'
=\int F'\times\chi I\cr
&=\int\sup_{n\in\Bbb N}F'\times\chi[a_n,b_n]
=\sup_{n\in\Bbb N}\int F'\times\chi[a_n,b_n]\cr
\noalign{\noindent (by B.Levi's theorem)}
&=\sup_{n\in\Bbb N}\int_{a_n}^{b_n}F'
\le\Var_I(f).\cr}$$
}%end of proof of 224I

\leader{224J}{}\cmmnt{ The next result is not needed in this chapter,
but is one of the most useful properties of functions of bounded
variation, and will be used repeatedly in Chapter 28.

\medskip

\noindent}{\bf Proposition} Let $f$, $g$ be real-valued
functions defined on subsets of $\Bbb R$, and suppose that $g$ is
integrable over an interval $[a,b]$, where $a<b$, and $f$ is of bounded
variation on $\ooint{a,b}$ and defined almost everywhere in
$\ooint{a,b}$.   Then $f\times g$ is integrable over $[a,b]$, and

$$\bigl|\int_a^bf\times g\bigr|
\le\bigl(\lim_{x\in\dom f,x\uparrow b}|f(x)|+\Var_{\ooint{a,b}}(f)\bigr)
  \sup_{c\in[a,b]}\bigl|\int_a^cg\bigr|.$$

\proof{{\bf (a)} By 224F, $l=\lim_{x\in\dom f,x\uparrow b}f(x)$ is
defined.   Write $M=|l|+\Var_{\ooint{a,b}}(f)$.   Note that if $y$ is
any point of $\dom f\cap\ooint{a,b}$,

\Centerline{$|f(y)|\le|f(x)|+|f(x)-f(y)|
\le|f(x)|+\Var_{\ooint{a,b}}(f)\to M$}

\noindent as $x\uparrow b$ in $\dom f$, so
$|f(y)|\le M$.   Moreover, $f$ is measurable
on $\ooint{a,b}$, because there are bounded monotonic functions
$f_1$, $f_2:\Bbb R\to\Bbb R$ such that $f=f_1-f_2$
everywhere in $\ooint{a,b}\cap\dom f$.   So $f\times g$ is measurable
and dominated by $M|g|$, and is integrable over $\ooint{a,b}$ or
$[a,b]$.

\medskip

{\bf (b)} For $n\in\Bbb N$, $k\le 2^n$ set
$a_{nk}=a+2^{-n}k(b-a)$, and for $1\le k\le 2^n$ choose
$x_{nk}\in\dom f\cap\ocint{a_{n,k-1},a_{nk}}$.    Define
$f_n:\ocint{a,b}\to\Bbb R$ by
setting $f_n(x)=f(x_{nk})$ if $1\le k\le 2^n$ and
$x\in\ocint{a_{n,k-1},a_{nk}}$.   Then $f(x)=\lim_{n\to\infty}f_n(x)$
whenever $x\in\ooint{a,b}\cap\dom f$ and $f$ is continuous at $x$, which
must be almost everywhere (224H).   Note next that all the $f_n$ are
measurable, and that they are uniformly bounded, in
modulus, by $M$.    So $\{f_n\times g:n\in\Bbb N\}$ is dominated by the
integrable function $M|g|$, and Lebesgue's
Dominated Convergence Theorem tells us that

\Centerline{$\int_a^bf\times g=\lim_{n\to\infty}\int_a^bf_n\times g$.}

\medskip

{\bf (c)} Fix $n\in\Bbb N$ for the moment.    Set
$K=\sup_{c\in[a,b]}|\int_a^cg|$.   (Note that $K$ is finite because
$c\mapsto\int_a^cg$ is continuous.)  Then

$$\eqalignno{\bigl|\int_a^bf_n\times g\bigr|
&=\bigl|\sum_{k=1}^{2^n}\int_{a_{n,k-1}}^{a_{nk}}f_n\times g\bigr|\cr
&=\bigl|\sum_{k=1}^{2^n}
    f(x_{nk})(\int_a^{a_{nk}}g-\int_a^{a_{n,k-1}}g)\bigr|\cr
&=\bigl|\sum_{k=1}^{2^n-1}
   (f(x_{nk})-f(x_{n,k+1}))\int_a^{a_{nk}}g
   +f(x_{n,2^n})\int_a^bg\bigr|\cr
&\le\bigl|f(x_{n,2^n})\bigr|\bigl|\int_a^bg\bigr|+\sum_{k=1}^{2^n-1}
   \bigl|f(x_{n,k+1})-f(x_{nk})\bigr|\bigl|\int_a^{a_{nk}}g\bigr|\cr
&\le(|f(x_{n,2^n})|+\Var_{\ooint{a,b}}(f))K
\to MK\cr}$$

\noindent as $n\to\infty$.

\medskip

{\bf (d)} Now

\Centerline{$|\int_a^bf\times g|
=\lim_{n\to\infty}|\int_a^bf_n\times g|\le MK$,}

\noindent as required.
}%end of proof of 224J

\leader{224K}{Complex-valued functions}\cmmnt{ So far I have taken all
functions to be real-valued.   This is adequate for the needs of the
present chapter, but in Chapter 28 we shall need to look at
complex-valued functions of bounded variation, and I should perhaps
spell out the (elementary) adaptations involved in the extension to the
complex case.
}

\spheader 224Ka Let $D$ be a subset of $\Bbb R$ and $f$ a
complex-valued function.   The
{\bf variation} of $f$ on $D$, $\Var_D(f)$, is zero if
$D\cap\dom f=\emptyset$, and otherwise is

\Centerline{$\sup\{\sum_{j=1}^n|f(a_j)-f(a_{j-1})|:
 a_0\le a_1\le\ldots\le a_n$ in $D\cap\dom f\}$,}

\noindent allowing $\infty$.
If $\Var_{D}(f)$ is finite, we say that $f$ is {\bf of bounded
variation} on $D$.

\spheader 224Kb \dvro{A}{Just as in the real case, a} complex-valued
function of bounded variation must be bounded, and
\Centerline{$\Var_{D}(f+g)\le\Var_{D}(f)+\Var_{D}(g)$,}

\Centerline{$\Var_{D}(cf)=|c|\Var_{D}(f)$,}

\Centerline{$\Var_{D}(f)\ge\Var_{D\cap\ocint{-\infty,x}}(f)
+\Var_{D\cap\coint{x,\infty}}(f)$}

\noindent for every
$x\in\Bbb R$, with equality if $x\in D\cap\dom f$,

\Centerline{$\Var_D(f)\le\Var_{D'}(f)$ whenever $D\subseteq
D'$\dvro{.}{;}}

\cmmnt{\noindent the arguments of 224C go through unchanged.}

\spheader 224Kc A complex-valued function is of bounded
variation iff its real and imaginary parts are both of bounded
variation\dvro{. }{ (because

\Centerline{$\max(\Var_{D}(\Real f),\Var_{D}(\Imag f))
\le\Var_{D}(f)\le\Var_{D}(\Real f)+\Var_{D}(\Imag f)$.)}

\noindent}So a complex-valued function $f$ is of bounded variation on
$D$ iff there are bounded non-decreasing functions
$f_1,\ldots,f_4:\Bbb R\to\Bbb R$ such that
$f=f_1-f_2+if_3-if_4$ on $D$\cmmnt{ (224D)}.

\spheader 224Kd Let $f$ be a complex-valued function and $D$ any
subset of $\Bbb R$.   If $f$ is of bounded variation on $D$, then

\Centerline{$\lim_{x\downarrow a}\Var_{D\cap\ocint{a,x}}(f)
=\lim_{x\uparrow a}\Var_{D\cap\coint{x,a}}(f)=0$}

\noindent for every $a\in\Bbb R$, and

\Centerline{$\lim_{a\to-\infty}\Var_{D\cap\ocint{-\infty,a}}(f)
=\lim_{a\to\infty}\Var_{D\cap\coint{a,\infty}}(f)=0$.}

\prooflet{\noindent (Apply 224E to the real and imaginary parts of
$f$.)}

\spheader 224Ke Let $f$ be a complex-valued function of bounded
variation on $[a,b]$, where $a<b$.   If $\dom f$ meets every interval
$\ocint{a,a+\delta}$ with $\delta>0$,   then
 $\lim_{t\in\dom f,t\downarrow a}f(t)$
is defined in $\Bbb C$.   If $\dom f$ meets
$\coint{b-\delta,b}$ for every $\delta>0$, then
$\lim_{t\in\dom f,t\uparrow b}f(t)$
is defined in $\Bbb C$.   \prooflet{(Apply 224F to the real
and imaginary parts of $f$.)}

\spheader 224Kf Let $f$, $g$ be complex functions and $D$ a subset
of $\Bbb R$.   If $f$ and $g$ are of bounded variation on $D$, so is
$f\times g$.   \prooflet{(For $f\times g$ is expressible as a linear
combination of the four products
$\Real f\times\Real g,\ldots,\Imag f\times\Imag g$,
to each of which we can apply 224G.)}

\spheader 224Kg Let $I\subseteq\Bbb R$ be an interval, and
$f:I\to\Bbb C$ a function of bounded variation.   Then $f$ is
differentiable almost everywhere in $I$, and $\int_I|f'|\le\Var_I(f)$.
\prooflet{(As 224I.)}

\spheader 224Kh Let $f$ and $g$ be complex-valued
functions defined on subsets of $\Bbb R$, and suppose that $g$ is
integrable over an interval $[a,b]$, where $a<b$, and $f$ is of bounded
variation on $\ooint{a,b}$ and defined almost everywhere in
$\ooint{a,b}$.   Then $f\times g$ is integrable over $[a,b]$, and

$$\bigl|\int_a^bf\times g\bigr|
\le\bigl(\lim_{x\in\dom f,x\uparrow b}|f(x)|+\Var_{\ooint{a,b}}(f)\bigr)
  \sup_{c\in[a,b]}\bigl|\int_a^cg\bigr|.$$

\prooflet{\noindent (The argument of 224J applies virtually unchanged.)}

\exercises{
\leader{224X}{Basic exercises $\pmb{>}$(a)}
%\spheader 224Xa
Set $f(x)=x^2\sin\Bover1{x^2}$ for $x\ne 0$, $f(0)=0$.   Show that
$f:\Bbb R\to\Bbb R$ is differentiable everywhere and uniformly
continuous, but is not of bounded variation on any non-trivial interval
containing $0$.
%224A

\spheader 224Xb Give an example of a non-negative function
$g:[0,1]\to[0,1]$, of bounded variation, such that $\sqrt{g}$ is not of
bounded variation.
%224A

\spheader 224Xc Show that if $f$ is any real-valued function
defined on a subset of $\Bbb R$, there is a function
$\tilde f:\Bbb R\to\Bbb R$, extending $f$, such that
$\Var\tilde f=\Var f$.   Under what circumstances is $\tilde f$ unique?
%224D

\spheader 224Xd Let $f:D\to\Bbb R$ be a function of bounded variation,
where $D\subseteq\Bbb R$ is a non-empty set.   Show that if $\inf_{x\in
D}|f(x)|>0$ then $1/f$ is of bounded variation.
%224C

\spheader 224Xe Let $f:[a,b]\to\Bbb R$ be a continuous function, where
$a\le b$ in $\Bbb R$.   Show that if $c<\Var f$ then there is a
$\delta>0$
such that $\sum_{i=1}^n|f(a_i)-f(a_{i-1})|\ge c$ whenever
$a=a_0\le a_1\le\ldots\le a_n=b$ and
$\max_{1\le i\le n}a_i-a_{i-1}\le\delta$.
%224C

\spheader 224Xf Let $\sequencen{f_n}$ be a sequence of real functions,
and set $f(x)=\lim_{n\to\infty}f_n(x)$ whenever the limit is defined.
Show that $\Var f\le\liminf_{n\to\infty}\Var f_n$.
%224A

\spheader 224Xg Let $f$ be a real-valued function which is integrable
over an interval $[a,b]\subseteq\Bbb R$.   Set $F(x)=\int_a^xf$ for
$x\in[a,b]$.
Show that $\Var F=\int_a^b|f|$.   ({\it Hint\/}: start by
checking that $\Var F\le\int|f|$;  for the reverse inequality, consider
the case $f\ge 0$ first.)
%224D

\spheader 224Xh Show that if $f$ is a real-valued function defined on a
non-empty set $D\subseteq\Bbb R$, then

\Centerline{$\Var f=\sup\{|\sum_{i=1}^n(-1)^i(f(a_i)-f(a_{i-1}))|:
  a_0\le a_1\le\ldots\le a_n$ in $D\}$.}
%224A

\spheader 224Xi Let $f$ be a real-valued function which is integrable
over a bounded interval $[a,b]\subseteq\Bbb R$.   Show that

\Centerline{$\int_a^b|f|
=$$\sup\{|\sum_{i=1}^n(-1)^i\int_{a_{i-1}}^{a_i}f|:
  a=a_0\le a_1\le a_2\le\ldots\le a_n=b\}$.}

\noindent({\it Hint\/}: put 224Xg and 224Xh together.)
%224Xg, 224Xh, 224D

\spheader 224Xj Let $f$ and $g$ be real-valued
functions defined on subsets of $\Bbb R$, and suppose that $g$ is
integrable over an interval $[a,b]$, where $a<b$, and $f$ is of bounded
variation on $\ooint{a,b}$ and defined almost everywhere in
$\ooint{a,b}$.   Show that

\Centerline{$|\int_a^bf\times g|
\le(\lim_{x\in\dom f,x\downarrow a}|f(x)|
  +\Var_{\ooint{a,b}}(f))\sup_{c\in[a,b]}|\int_c^bg|$.}
%224J

\spheader 224Xk\dvAnew{2012} Suppose that $D\subseteq\Bbb R$ and
$f:D\to\Bbb R$ is a function.   Show that $f$ is
expressible as a difference of non-decreasing functions iff
$\Var_{D\cap[a,b]}(f)$ is finite whenever $a\le b$ in $D$.
%224D out of order query

\spheader 224Xl\dvAnew{2012}
Suppose that $D\subseteq\Bbb R$ and that $f:D\to\Bbb R$ is a continuous
function of bounded variation.   Show that $f$ is expressible as the
difference of two continuous non-decreasing functions.
%224E out of order query

\spheader 224Xm\dvAnew{2013}
Suppose that $D\subseteq\Bbb R$ and that $f:D\to\Bbb R$ is a
function of bounded variation which is continuous on the right, that is,
whenever $x\in D$ and $\epsilon>0$ there is a $\delta>0$ such that
$|f(t)-f(x)|\le\epsilon$ for every $t\in D\cap[x,x+\delta]$.
Show that $f$ is expressible as the
difference of two non-decreasing functions which are continuous on the
right.
%224E out of order query

\leader{224Y}{Further exercises (a)}
%\spheader 224Ya
Show that if $f$ is any complex-valued function
defined on a subset of $\Bbb R$, there is a function
$\tilde f:\Bbb R\to\Bbb C$, extending $f$, such that
$\Var\tilde f=\Var f$.   Under what circumstances is $\tilde f$ unique?

\spheader 224Yb Let $D$ be any non-empty subset of $\Bbb R$, and
let $\eusm V$ be the space of functions $f:D\to\Bbb R$ of bounded
variation.   For $f\in\eusm V$ set

\Centerline{$\|f\|=\sup\{|f(t_0)|+\sum_{i=1}^n|f(t_i)-f(t_{i-1})|:
t_0\le\ldots\le t_n\in D\}$.}

\noindent Show that (i) $\|\,\|$ is a norm on $\eusm V$ (ii) $\eusm V$
is complete under $\|\,\|$ (iii) $\|f\times g\|\le\|f\|\|g\|$ for all
$f$, $g\in\eusm V$, so that $\eusm V$ is a Banach algebra.

\spheader 224Yc Let $f:\Bbb R\to\Bbb R$ be a function of bounded
variation.
Show that there is a sequence $\sequencen{f_n}$ of differentiable
functions
such that $\lim_{n\to\infty}f_n(x)=f(x)$ for every $x\in\Bbb R$,
$\lim_{n\to\infty}\int|f_n-f|=0$, and $\Var(f_n)\le\Var(f)$ for every
$n\in\Bbb N$.   \Hint{start with non-decreasing $f$.}

\spheader 224Yd For any partially ordered set $X$ and any
function $f:X\to\Bbb R$, say that $\Var_X(f)=0$ if $X=\emptyset$ and
otherwise

\Centerline{$\Var_X(f)
=\sup\{\sum_{i=1}^n|f(a_i)-f(a_{i-1})|:a_0,a_1,\ldots,a_n\in X,\,a_0\le
a_1\le\ldots\le a_n\}$.}

\noindent State and prove results in this framework generalizing 224D
and 224Yb.   ({\it Hints\/}:  $f$ will be `non-decreasing' if
$f(x)\le f(y)$ whenever $x\le y$;  interpret $\ocint{-\infty,x}$ as
$\{y:y\le x\}$.)

\spheader 224Ye Let $(X,\rho)$ be a metric space and $f:[a,b]\to X$ a
function, where $a\le b$ in $\Bbb R$.   Set
\ifnum\stylenumber=12

\Centerline{$\Var_{[a,b]}(f)
=\sup\{\sum_{i=1}^n\rho(f(a_i),f(a_{i-1})):
a\le a_0\le\ldots\le a_n\le b\}$.}

\noindent\else
$\Var_{[a,b]}(f)
=\sup\{\sum_{i=1}^n\rho(f(a_i),f(a_{i-1})):
a\le a_0\le\ldots\le a_n\le b\}$. \fi
(i) Show that
$\Var_{[a,b]}(f)=\Var_{[a,c]}(f)+\Var_{[c,b]}(f)$ for every $c\in[a,b]$.
(ii) Show that if $\Var_{[a,b]}(f)$ is finite
then $f$ is continuous at all but countably many points of $[a,b]$.
(iii) Show that if $X$ is complete and $\Var_{[a,b]}(f)<\infty$ then
$\lim_{t\uparrow x}f(t)$ is defined for every $x\in\ocint{a,b}$.
(iv) Show that if $X$ is complete then
$\Var_{[a,b]}(f)$ is finite iff $f$ is expressible as a
composition $gh$, where $h:[a,b]\to\Bbb R$ is non-decreasing and
$g:\Bbb R\to X$ is $1$-Lipschitz, that is,
$\rho(g(c),g(d))\le|c-d|$ for all $c$, $d\in\Bbb R$.

\spheader 224Yf Let $U$ be a normed space and $a\le b$ in $\Bbb R$.
For functions $f:[a,b]\to U$ define $\Var_{[a,b]}(f)$ as in 224Ye, using
the standard metric $\rho(x,y)=\|x-y\|$ for $x$, $y\in U$.    (i) Show
that $\Var_{[a,b]}(f+g)\le\Var_{[a,b]}(f)+\Var_{[a,b]}(g)$,
$\Var_{[a,b]}(cf)=|c|\Var_{[a,b]}(f)$ for all $f$, $g:[a,b]\to U$ and
all $c\in\Bbb R$.   (ii) Show that if $V$ is another normed space and
$T:U\to V$ is a bounded linear operator then
$\Var_{[a,b]}(Tf)\le\|T\|\Var_{[a,b]}(f)$ for every $f:[a,b]\to U$.

\spheader 224Yg Let $f:[0,1]\to\Bbb R$ be a continuous function.   For
$y\in\Bbb R$ set $h(y)=\#(f^{-1}[\{y\}])$ if this is finite, $\infty$
otherwise.   Show that (if we allow $\infty$ as a value of the integral)
$\Var_{[0,1]}(f)=\int h$.   ({\it Hint\/}:  for $n\in\Bbb N$, $i<2^n$
set $c_{ni}=\sup\{f(x)-f(y):x,y\in[2^{-n}i,2^{-n}(i+1)]\}$,
$h_{ni}(y)=1$ if $y\in f[\,\coint{2^{-n}i,2^{-n}(i+1)}\,]$, $0$
otherwise.   Show that $c_{ni}=\int h_{ni}$,
$\lim_{n\to\infty}\sum_{i=0}^{2^n-1}c_{ni}=\Var f$,
$\lim_{n\to\infty}\sum_{i=0}^{2^n-1}h_{ni}=h$.)
(See also 226Yc.)

\spheader 224Yh Let $\nu$ be any Lebesgue-Stieltjes measure on $\Bbb R$,
$I\subseteq \Bbb R$ an interval (which may be either open or closed,
bounded or unbounded), and $D\subseteq I$ a non-empty
set.  Let $\eusm V$ be the space of functions of bounded variation from
$D$ to $\Bbb R$, and $\|\,\|$ the norm of 224Yb on $\eusm V$.   Let
$g:D\to\Bbb R$ be a function such that $\int_{[a,b]\cap D}g\,d\nu$
exists whenever $a\le b$ in $I$, and
$K=\sup_{a,b\in I,a\le  b}|\int_{[a,b]\cap D}g\,d\nu|$.
Show that $|\int_Df\times g\,d\nu|\le K\|f\|$ for every $f\in\eusm V$.
%224K

\spheader 224Yi Explain how to apply 224Yh with $D=\Bbb N$ to
obtain Abel's theorem that the product of a monotonic sequence
converging to $0$ with a series which has bounded partial sums is
summable.
%224Yh 224K

\spheader 224Yj Suppose that $I\subseteq\Bbb R$ is an interval, and
that $\sequencen{A_n}$ is a sequence of sets covering $I$.
Let $f:I\to\Bbb R$ be continuous.   Show that
$\Var f\le\sum_{n=0}^{\infty}\Var_{A_n}f$.
\Hint{reduce to the case of closed sets $A_n$;
use Baire's theorem (4A2Ma).}
}%end of exercises

\endnotes{
\Notesheader{224} I have taken the ideas above rather farther than we
need immediately;  for the present chapter, it is enough to consider the
case in which $D=\dom f=[a,b]$ for some interval $[a,b]\subseteq\Bbb R$.
The extension to functions
with irregular domains will be useful in Chapter 28, and the extension
to irregular sets $D$, while not important to us here, is of some
interest -- for instance, taking $D=\Bbb N$, we obtain the notion of
`sequence of bounded variation', which is surely relevant to problems of
convergence and summability.

The central result of the section is of course the fact that a function
of bounded variation can be expressed as the difference of monotonic
functions (224D);  indeed, one of the objects of the concept is to
characterize the
linear span of the monotonic functions.   Nearly everything else here
can be derived as easy consequences of this, as in 224E-224G.   In 224I
and 224Xg we go a little deeper, and indeed some measure theory appears;
this is where the ideas here begin to connect with the real business of
this  chapter, to be continued in the next section.
Another result which is
easy enough in itself, but contains the germs of important ideas, is
224Yg.

In 224Yb I mention a natural development in functional analysis, and in
224Yd-224Yf %224Yd 224Ye 224Yf
I suggest further wide-ranging generalizations.
}%end of notes

\discrpage
