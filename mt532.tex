\frfilename{mt532.tex}
\versiondate{1.6.13}
\copyrightdate{2004}

\def\chaptername{Topologies and measures III}
\def\sectionname{Completion regular measures on $\{0,1\}^I$}


\newsection{532}
\def\headlinesectionname{Completion regular measures on
$\{0,1{\delimiter"5267309}^I$}

As I remarked in the introduction to \S434, the trouble with topological
measure theory is that there are too many questions to ask.   In \S531 I
looked at the problem of determining the possible Maharam types of Radon
measures on a Hausdorff space $X$.   But we can ask the same
question for any of the other classes of topological measures listed in
\S411.   It turns out that the very narrowly focused topic of completion
regular Radon measures on powers of $\{0,1\}$ already leads us to some
interesting arguments.

I define the classes $\MahcrR(X)$, corresponding to the $\MahR(X)$
examined in \S531, in 532A.   They are less accessible, and I almost
immediately specialize to the relation
$\lambda\in\MahcrR(\{0,1\}^{\kappa})$.   This at least is more or less
convex (532G, 532K), and can be characterized in terms of the measure
algebras $\frak B_{\lambda}$ (532I).   On the way it is helpful to extend
the treatment of completion regular measures given in \S434 (532D, 532E,
532H).   For fixed infinite $\lambda$, there is a critical cardinal
$\kappa_0\le(2^{\lambda})^+$ such that
$\lambda\in\MahcrR(\{0,1\}^{\kappa})$ iff $\lambda\le\kappa<\kappa_0$;
under certain conditions, when $\lambda=\omega$,
we can locate $\kappa_0$ in terms of the
cardinals of Cicho\'n's diagram (532P, 532Q).   This depends on facts
about the Lebesgue measure algebra (532M, 532O) which are of independent
interest.    Finally, for other $\lambda$ of countable cofinality, the
square principle and Chang's transfer principle are relevant (532R-532S).

\leader{532A}{Definition} If $X$ is a topological space, I
write $\Mahcr(X)$ for the set of Maharam types of \Mth\ completion
regular topological probability measures on $X$.
If $X$ is a Hausdorff space, I write
$\MahcrR(X)$ for the set of Maharam types of \Mth\ completion regular
Radon probability measures on $X$.

\leader{532B}{Proposition} Let $X$ be a Hausdorff space.   Then a
probability algebra $(\frak A,\bar\mu)$ is isomorphic to the measure
algebra of a completion regular Radon probability measure on $X$ iff
($\alpha$) $\tau(\frak A_a)\in\MahcrR(X)$ whenever $\frak A_a$ is a
non-zero homogeneous principal ideal of $\frak A$ ($\beta$) the number
of atoms of $\frak A$ is not greater than the number of points $x\in X$
such that $\{x\}$ is a zero set.

\proof{{\bf (a)} Suppose that $\mu$ is a completion regular Radon
probability measure on $X$ and $\frak A_a$ is a non-zero homogeneous
principal ideal of its measure algebra $\frak A$.   Let $F$ be such that
$F^{\ssbullet}=a$ and $\nu$ the indefinite-integral measure over $\mu$
defined by the function $\Bover1{\mu F}\chi F$.   Then $\nu$ is a Radon
measure (416S), inner regular with respect to the zero sets (412Q);  and
its measure algebra is isomorphic, up to a scalar multiple, to
$\frak A_a$, so is homogeneous with Maharam type
$\tau(\frak A_a)$.   So $\nu$ witnesses that
$\tau(\frak A_a)\in\MahcrR(X)$.    This shows that $\frak A$ satisfies
condition ($\alpha$).

As for condition ($\beta$), each atom of $\frak A$ is of the form
$\{x\}^{\ssbullet}$ for some $x\in X$ such that $\mu\{x\}>0$ (414G, or
otherwise).   In this case, because $\mu$ is completion regular, $\{x\}$
must be a zero set.   So we have at least as many singleton zero sets as
we have atoms in $\frak A$.

\medskip

{\bf (b)} Now suppose that $(\frak A,\bar\mu)$ satisfies the conditions.
I copy the argment of 531F.   Express $(\frak A,\bar\mu)$ as the simple
product of a countable family $\familyiI{(\frak A_i,\bar\mu'_i)}$ of
non-zero homogeneous measure algebras.   For $i\in I$, set
$\kappa_i=\tau(\frak A_i)$ and $\gamma_i=\bar\mu'_i1_{\frak A_i}$.   Set
$J=\{i:i\in I$, $\kappa_i\ge\omega\}$.   $(\beta$) tells us that
$\#(I\setminus J)$ is less than or equal to the number of singleton zero
sets in $X$;  let $\family{i}{I\setminus J}{x_i}$ be a
family of distinct elements of $X$ such that every $\{x_i\}$ is a zero
set.

For each $i\in J$,
($\alpha$) tells us that there is a completion regular \Mth\ Radon
probability measure
$\mu_i$ on $X$ with Maharam type $\kappa_i$.   Now there is a disjoint
family $\family{i}{J}{E_i}$ of Baire subsets of $X$ such that $\mu_iE_i>0$
for every $i\in J$.   \Prf\ We may suppose that $J\subseteq\Bbb N$.
Choose $\sequence{i}{E_i}$, $\sequence{i}{F_i}$ inductively, as follows.
$F_0=X\setminus\{x_i:i\in I\setminus J\}$.   Given that $F_i$ is a Baire
set and $\mu_jF_i>0$ for every $j\in J\setminus i$, then if $i\notin J$
set $E_i=\emptyset$
and $F_{i+1}=F_i$;  otherwise, because $\mu_i$ is atomless and
completion regular, we can find,
for each $j\in J$ such that $j>i$, a Baire set $G_{ij}\subseteq F_i$
such that $\mu_iG_{ij}<2^{-j}\mu_iF_i$ and $\mu_jG_{ij}>0$;  set
$F_{i+1}=\bigcup_{j\in J,j>i}G_{ij}$ and $E_i=F_i\setminus F_{i+1}$;
continue.\ \QeD\   Now set

\Centerline{$\mu E
=\sum_{i\in I\setminus J,x_i\in E}\gamma_i
  +\sum_{i\in J}(\mu_iE_i)^{-1}\gamma_i\mu_i(E\cap E_i)$}

\noindent whenever $E\subseteq X$ is such that $\mu_i$ measures
$E\cap E_i$ for every $i\in J$.   Of course $\mu$ is a probability
measure.   Because every $\mu_i$ is a topological measure, so is $\mu$;
because every $\mu_i$ is inner regular with respect to the compact sets,
so is $\mu$;  because every $\mu_i$ is complete, so is $\mu$;  so $\mu$
is a Radon measure.   Because every subspace measure
$(\mu_i)_{E_i}$ is \Mth\ with Maharam type $\kappa_i$, the measure
algebra of $\mu$ is isomorphic to $(\frak A,\bar\mu)$.
Because all the $\{x_i\}$ are zero sets and all the $\mu_i$ are
completion regular, $\mu$ is completion regular.
}%end of proof of 532B

\leader{532C}{Remarks}\cmmnt{ Nearly the whole of this section will be
devoted to the usual measures on powers of $\{0,1\}$.   Accordingly the
following notation will be useful, as previously in this volume.}
If $I$ is any set, $\nu_I$ will be the usual measure on $\{0,1\}^I$,
$\frak B_I$ its measure algebra and $\Cal N_I$ its null ideal.   
\cmmnt{In this context, 
$\familyiI{e_i}$ will be the standard generating family in
$\frak B_I$ (525A), and for $J\subseteq I$, $\frak C_J$ will be the closed
subalgebra of $\frak B_I$ generated by $\{e_i:i\in J\}$.}

\cmmnt{If $X$
is a topological space, $\Cal B(X)$ will be its Borel $\sigma$-algebra.}

Let $\kappa$ be an infinite cardinal.   Then\cmmnt{ $\nu_{\kappa}$ is
a completion regular Radon probability measure (416U), and
$\frak B_{\kappa}$ is homogeneous with Maharam type $\kappa$.   So}
$\kappa\in\MahcrR(\{0,1\}^{\kappa})$.
\cmmnt{Next, any Radon measure on $\{0,1\}^{\kappa}$ can have Maharam
type at most $w(\{0,1\}^{\kappa})$ (531Aa), so }$\lambda\le\kappa$ for
every $\lambda\in\MahcrR(\{0,1\}^{\kappa})$.   \cmmnt{At the bottom
end,} $0\in\MahcrR(\{0,1\}^{\kappa})$ iff\cmmnt{ $\{0,1\}^{\kappa}$
has a singleton G$_{\delta}$ set, that is, iff} $\kappa=\omega$.

\cmmnt{From this we see already that we do not have direct
equivalents of any of the results
531Eb-531Ef. %531Eb 531Ec 531Ed 531Ee 531Ef
However the class
$\{(\lambda,\kappa):\lambda\in\MahcrR(\{0,1\}^{\kappa})\}$ is convex in
two senses (532G, 532K).   For the first of these, it will be useful to
have a result left over from \S434.}

\leader{532D}{Theorem}\cmmnt{ ({\smc Fremlin \& Grekas 95})} Let
$(X,\mu_1)$ and $(Y,\mu_2)$ be
effectively locally finite topological measure spaces of which
$X$ is quasi-dyadic\cmmnt{ (definition:  434O)}, $\mu_1$ is completion
regular and $\mu_2$ is $\tau$-additive.
Let $\mu$ be the c.l.d.\ product measure on
$X\times Y$\cmmnt{ as defined in \S251}.   Then $\mu$ is a
$\tau$-additive topological measure.

\proof{{\bf (a)} To begin with (down to the end of (e)) let us suppose
that $\mu_1$ and $\mu_2$ are complete and totally finite and
inner regular with respect to
the Borel sets.   Let $\familyiI{X_i}$ be a family of separable
metrizable spaces such that there is a continuous surjection
$f:\prod_{i\in I}X_i\to X$.   For each $i\in I$, let $\Cal U_i$ be a
countable base for the topology of $X_i$ not containing $\emptyset$;
for $J\subseteq I$, let
$\Cal C_J$ be the family of cylinder sets expressible in the form
$\{z:z\in\prod_{i\in I}X_i$, $z(i)\in U_i$ for every $i\in K\}$ where
$K\subseteq J$ is finite and $U_i\in\Cal U_i$ for each $i\in K$.

\medskip

{\bf (b)} \Quer\ Suppose, if possible, that $\mu$ is not a topological
measure.   Let $W\subseteq X\times Y$ be a closed set which is not
measured by $\mu$.   By 434Q, $\mu_1$ is $\tau$-additive;
by 417C, there is a $\tau$-additive topological
measure $\tilde\mu$ extending $\mu$, and $\mu^*W=\tilde\mu W$ (apply
417C(iv) to the complement of $W$).

\medskip

{\bf (c)} If $J\subseteq I$ is countable, there are sets $H$, $V$, $V'$
such that $H\subseteq Y$ is open, $V\in\Cal C_J$,
$V'\in\Cal C_{I\setminus J}$, $f[V\cap V']\times H$ is disjoint from
$W$, and $\mu^*(W\cap(f[V]\times H))>0$.   \Prf\ For $V\in\Cal C_J$, set

\Centerline{$\Cal H_V
=\bigcup_{V'\in\Cal C_{I\setminus J}}
  \{H:H\subseteq Y$ is open, $W\cap(f[V\cap V']\times H)=\emptyset\}$,}

\Centerline{$H_V=\bigcup\Cal H_V$,}

\noindent and choose a measurable envelope $F_V$ of $f[V]$.   As
$\Cal C_J$ is countable,

\Centerline{$W_1
=(X\times Y)\setminus\bigcup_{V\in\Cal C_J}F_V\times H_V$}

\noindent is measured by $\mu$;  also $W_1\subseteq W$ because

\Centerline{$\{f[V\cap V']\times H:
V\in\Cal C_J$, $V'\in\Cal C_{I\setminus J}$, $H\subseteq Y$ is open$\}$}

\noindent is a network for the topology of $X\times Y$.   So

\Centerline{$\tilde\mu W_1=\mu W_1\le\mu_*W<\mu^*W=\tilde\mu W$}

\noindent and $\tilde\mu(W\setminus W_1)>0$.   There must therefore be a
$V\in\Cal C_J$ such that $\tilde\mu(W\cap(F_V\times H_V))>0$.   Next,
because $\mu_2$ is $\tau$-additive, there is a countable
$\Cal H\subseteq\Cal H_V$ such that
$\mu_2(H_V\setminus\bigcup\Cal H)=0$, and now
$\tilde\mu(W\cap(F_V\times\bigcup\Cal H))
=\tilde\mu(W\cap(F_V\times H_V))$ is non-zero.   Accordingly there is an
$H\in\Cal H$ such that $\tilde\mu(W\cap(F_V\times H))>0$.   By 417H,

\Centerline{$\int_{F_V}\mu_2(W[\{x\}]\cap H)\mu_1(dx)
=\tilde\mu(W\cap(F_V\times H))$}

\noindent is greater than $0$.   But this means that
$\mu_1\{x:x\in F_V$, $\mu_2(W[\{x\}]\cap H)>0\}>0$.   (Recall that we
are supposing that $\mu_1$ is complete.)   So
$\{x:x\in f[V]$, $\mu_2(W[\{x\}]\cap H)>0\}$ is not $\mu_1$-negligible,
and $W\cap(f[V]\times H)$ is not $\mu$-negligible.
Finally, because $H\in\Cal H_V$, there is a $V'\in\Cal C_{I\setminus J}$
such that $W\cap(f[V\cap V']\times H)=\emptyset$.\ \Qed

\medskip

{\bf (d)} We may therefore choose inductively families
$\ofamily{\xi}{\omega_1}{J_{\xi}}$,
$\ofamily{\xi}{\omega_1}{H_{\xi}}$,
$\ofamily{\xi}{\omega_1}{V_{\xi}}$,
$\ofamily{\xi}{\omega_1}{V'_{\xi}}$ in such a way that, for every
$\xi<\omega_1$,

\inset{$J_{\xi}$ is a countable subset of $I$,

$H_{\xi}$ is an open subset of $Y$,

$V_{\xi}\in\Cal C_{J_{\xi}}$, $V'_{\xi}\in\Cal C_{I\setminus J_{\xi}}$,

$W\cap(f[V_{\xi}\cap V'_{\xi}]\times H_{\xi})=\emptyset$,

$\mu^*(W\cap(f[V_{\xi}]\times H_{\xi}))>0$,

$\bigcup_{\eta<\xi}J_{\eta}\subseteq J_{\xi}$,

$V_{\xi}$, $V'_{\xi}\in\Cal C_{J_{\xi+1}}$.}

\noindent For each $\xi<\omega_1$, let $K_{\xi}$ be a finite subset of
$J_{\xi+1}$ such that $V_{\xi}$ and $V'_{\xi}$ are determined by
coordinates in $K_{\xi}$.   By the $\Delta$-system Lemma (4A1Db), there
is an uncountable set $A\subseteq\omega_1$ such that
$\family{\xi}{A}{K_{\xi}}$ is a $\Delta$-system with root $K$ say.   Set
$\zeta_0=\min A$.    Express each $V_{\xi}$ as
$\tilde V_{\xi}\cap\hat V_{\xi}$ where  $\tilde V_{\xi}\in\Cal C_K$ and
$\hat V_{\xi}\in\Cal C_{K_{\xi}\setminus K}$;   because $\Cal C_K$ is
countable, there is a $\tilde V$ such that
$B=\{\xi:\xi\in A$, $\xi>\zeta_0$, $\tilde V_{\xi}=\tilde V\}$ is
uncountable.   Note that $\mu_1^*f[\tilde V]>0$, because
$\mu_1^*f[\tilde V]\ge\mu^*(W\cap(f[V_{\xi}]\times H_{\xi}))$ for any
$\xi\in B$.   Also

\Centerline{$K\subseteq K_{\zeta_0}\subseteq J_{\zeta_0+1}
\subseteq J_{\xi}$,}

\noindent so $V'_{\xi}$ is determined by coordinates in
$K_{\xi}\setminus J_{\xi}\subseteq K_{\xi}\setminus K$, for every
$\xi\in B$.

\medskip

{\bf (e)} Set $H'_{\xi}=\bigcup_{\eta\in B\setminus\xi}H_{\eta}$ for
each $\xi<\omega_1$.   Then $\ofamily{\xi}{\omega_1}{H'_{\xi}}$ is
non-increasing, so there is a $\zeta<\omega_1$ such that
$\mu_2H'_{\xi}=\mu_2H'_{\zeta}$ whenever $\xi\ge\zeta$.   Now consider
$F=\{x:\mu_2(W[\{x\}]\cap H'_{\zeta})>0\}$.   Applying 417H to
the indicator function of $W\cap(X\times H'_{\zeta})$, and
recalling once more that $\mu_1$ is complete, we see that $\mu_1$
measures $F$.   Also $\mu_1^*(F\cap f[\tilde V])>0$.   \Prf\ Take any
$\xi\in B\setminus\zeta$.   Then

\Centerline{$F\cap f[\tilde V]
\supseteq\{x:x\in f[\tilde V_{\xi}]$, $\mu_2(W[\{x\}]\cap H_{\xi})>0\}$}

\noindent must be non-$\mu_1$-negligible because
$W\cap(f[\tilde V_{\xi}]\times H_{\xi})$ is not $\mu$-negligible.\ \Qed

At this point, recall that we are supposing that $\mu_1$ is completion
regular.   So there is a zero set $Z\subseteq F$ such that
$\mu_1Z>\mu_1F-\mu_1^*(F\cap f[\tilde V])$, and
$Z\cap f[\tilde V]\ne\emptyset$, that is, $\tilde V\cap f^{-1}[Z]$ is
not empty.   $f^{-1}[Z]$ is a zero set (4A2C(b-iv)), so there is a
countable set $J\subseteq I$ such that $f^{-1}[Z]$ is determined by
coordinates in $J$ (4A3Nc);  we may suppose that $K\subseteq J$.
Because $\family{\eta}{A}{K_{\eta}\setminus K}$ is disjoint, there is a
$\xi\ge\zeta$ such that $J\cap K_{\eta}=K$ for every
$\eta\in A\setminus\xi$.

Take any $w\in\tilde V\cap f^{-1}[Z]$ and modify it to produce
$w'\in\prod_{i\in I}X_i$ such that $w'\restr J=w\restr J$ and
$w'\in\hat V_{\eta}\cap V'_{\eta}$ for every $\eta\in B\setminus\xi$;
this is possible because $\hat V_{\eta}\cap V'_{\eta}$ is determined by
coordinates in $K_{\eta}\setminus K$ for each $\eta$, and $J$ and the
$K_{\eta}\setminus K$ are disjoint.
Set $x=f(w')$;  then $x\in Z\subseteq F$, so
$\mu_2(W[\{x\}]\cap H'_{\zeta})>0$.

$w'\in\tilde V$, because $w\in\tilde V$
and $\tilde V$ is determined by coordinates in $K\subseteq J$;  so
$w'\in\tilde V\cap\hat V_{\eta}\cap V'_{\eta}=V_{\eta}\cap V'_{\eta}$ for
every $\eta\in B\setminus\xi$.
Accordingly $x\in f[V_{\eta}\cap V'_{\eta}]$;  as
$W\cap(f[V_{\eta}\cap V'_{\eta}]\times H_{\eta})=\emptyset$,
$W[\{x\}]$ does not meet $H_{\eta}$.   As $\eta$ is arbitrary,
$W[\{x\}]$ does not meet $H'_{\xi}$ and $W[\{x\}]\cap H'_{\zeta}$ is
$\mu_2$-negligible.   But this is impossible.\ \Bang

\medskip

{\bf (f)} This contradiction shows that $\mu$ will be a topological
measure, at least if $\mu_1$ and $\mu_2$ are complete, totally finite
and inner regular
with respect to the Borel sets.   Now suppose just that $\mu_1$ and
$\mu_2$ are totally finite.   Let $\mu'_1$ and $\mu'_2$ be the completions
of the Borel measures $\mu_1\restr\Cal B(X)$ and $\mu_2\restr\Cal B(Y)$,
and $\mu'$ their c.l.d.\
product.   Then $\mu_1\restr\Cal B(X)$ and $\mu'_1$ are
completion regular topological measures, while $\mu_2\restr\Cal B(Y)$ and
$\mu'_2$ are $\tau$-additive.   So (a)-(e) tell us that $\mu'$ measures
every open set.   Now the completions $\hat\mu_1$, $\hat\mu_2$ extend
$\mu'_1$ and $\mu'_2$, and $\mu$ is the c.l.d.\ product of $\hat\mu_1$ and
$\hat\mu_2$ (251T), so $\mu$ extends $\mu'$
(251L).   Thus we again have a topological product measure $\mu$.

\medskip

{\bf (g)} In the general case, let $W\subseteq X\times Y$ be an open set,
$E\subseteq X$ a zero set of finite measure, and $F\subseteq Y$ any set of
finite measure.   Then $\mu$ measures $W\cap(E\times F)$.   \Prf\ Let
$(\mu_1)_E$ and $(\mu_2)_F$ be the subspace measures.   Then both are
totally finite topological measures, $(\mu_1)_E$ is inner regular with
respect to the zero sets (412Pd), $E$ is
quasi-dyadic (434Pc), and $(\mu_2)_F$ is $\tau$-additive (414K).   So
the product $(\mu_1)_E\times(\mu_2)_F$ is a topological measure and
measures $W\cap(E\times F)$.    By 251Q, $\mu$ measures
$W\cap(E\times F)$.\ \Qed

Let $\Cal K$ be the family of zero sets of finite measure in $X$, $\Cal L$
the family of Borel sets of finite measure in $Y$, and $\Cal M$ the family
of sets $M\subseteq X\times Y$ such that $\mu$ measures $W\cap M$.
Because $\mu_1$ is inner regular with respect to $\Cal K$,
$\mu_2$ is inner regular with respect to $\Cal L$,
$E\times F\in\Cal M$ for every $E\in\Cal K$ and $F\in\Cal L$, and
$\Cal M$ is a $\sigma$-algebra of sets, 412R tells us
that $\mu$ is inner regular with respect to $\Cal M$.   As $\mu$ is
complete and locally determined, it must measure $W$ (412Ja).   As $W$ is
arbitrary, $\mu$ is a topological measure.

\medskip

{\bf (h)} Finally, as noted in (b), $\mu_1$ is $\tau$-additive and
there is a $\tau$-additive topological measure $\tilde\mu$ on $X\times Y$
extending $\mu$.   (434Q and 417C still apply.)
So $\mu$ too must be $\tau$-additive.
}%end of proof of 532D

\leader{532E}{Corollary} Let $\familyiI{X_i}$ be a family of regular
spaces with countable networks, and $Y$ any topological space.   Suppose
that we are given a strictly positive topological probability measure
$\mu_i$ on each $X_i$, and a $\tau$-additive topological probability
measure $\nu$ on $Y$.   Let $\mu$ be the ordinary product measure on
$Z=\prod_{i\in I}X_i\times Y$.

(a) $\mu$ is a topological measure.

(b) $\mu$ is $\tau$-additive.

(c) If $\nu$ is completion regular, and every $\mu_i$ is inner regular with
respect to the Borel sets, then $\mu$ is completion regular.

\proof{{\bf (a)} For each $i$, $X_i$ is hereditarily Lindel\"of (4A2Nb), so
$\mu_i$ is $\tau$-additive (414O);  let $\mu'_i$ be the completion of the
Borel measure $\mu_i\restr\Cal B(X_i)$.   Then $\mu'_i$ is a 
quasi-Radon measure (415C).   
By 4A2Nb, $X_i$ is perfectly normal, so $\mu'_i$ is completion regular.
By 434Pb-434Pc, $\prod_{i\in I}X_i$ is
quasi-dyadic.   The product $\nu_1$ of the $\mu'_i$ is a topological
measure (453I) and inner regular with respect to the zero sets (412Ub);
so the product $\mu'$ of $\nu_1$ and $\nu$ is a topological measure, by
532D.   Now $\mu'$ is also the product of the measures
$\mu_i\restr\Cal B(X_i)$ and $\nu$ (254I, 254N), 
so $\mu$ extends $\mu'$ (254H) and $\mu$ also is a topological measure.

\medskip

{\bf (b)} Because every $\mu_i$ is $\tau$-additive, as is $\nu$,
417E tells us that
there is a $\tau$-additive measure extending $\mu$, so $\mu$ itself must
be $\tau$-additive.

\medskip

{\bf (c)} For any $i\in I$, we know from (a) that 
$\mu'_i$ is inner regular with respect to
the zero sets.   Now every non-$\mu_i$-negligible set includes a
non-$\mu_i$-negligible Borel set, which includes a non-$\mu_i$-negligible
zero set;  accordingly $\mu_i$ is completion regular.
By 412Ub again, $\mu$ is inner regular with respect to the
zero sets, so is completion regular.
}%end of proof of 532E

\leader{532F}{Corollary} Let $\familyiI{(X_i,\mu_i)}$ be a family of
quasi-dyadic compact Hausdorff spaces with strictly positive completion
regular Radon measures.   Then the ordinary product measure $\mu$ on
$\prod_{i\in I}X_i$ is a completion regular Radon measure.

\proof{ By 532D, the ordinary product measure on
$\prod_{i\in J}X_i$ is a topological measure, for every finite
$J\subseteq I$.   By 417Sc, $\mu$ is the $\tau$-additive product
measure on $\prod_{i\in I}X_i$,
which by 417Q is a Radon measure.   By 412Ub once more, $\mu$
is completion regular.
}%end of proof of 532F

\leader{532G}{Proposition} Suppose that $\lambda$, $\lambda'$ and
$\kappa$ are cardinals such that
$\max(\omega,\lambda)\le\lambda'\le\kappa$
and $\lambda\in\MahcrR(\{0,1\}^{\kappa})$.   Then
$\lambda'\in\MahcrR(\{0,1\}^{\kappa})$.

\proof{ Let $\nu$ be a completion regular \Mth\ Radon probability
measure on $\{0,1\}^{\kappa}$ with Maharam type $\lambda$, and consider
the ordinary product measure $\mu$ of $\nu_{\lambda'}$ and $\nu$ on
$X=\{0,1\}^{\lambda'}\times\{0,1\}^{\kappa}$.   Applying 532E with
$Y=\{0,1\}^{\kappa}$ and $X_{\xi}=\{0,1\}$ for $\xi<\lambda'$, we see
that $\mu$ is a completion regular topological probability measure on a
compact Hausdorff space, therefore (being complete) a
Radon measure.   By 334A, the Maharam type of $\mu$ is at most
$\max(\omega,\lambda',\lambda)=\lambda'$, so the measure algebra
$(\frak A,\bar\mu)$ of $\mu$ can be embedded in $\frak B_{\lambda'}$.
At the same time, the \imp\ projection from $X$ onto
$\{0,1\}^{\lambda'}$ induces a measure-preserving embedding of
$\frak B_{\lambda'}$ into $\frak A$.   By 332Q, $(\frak A,\bar\mu)$ and
$(\frak B_{\lambda'},\bar\nu_{\lambda'})$ are isomorphic, that is, $\mu$
is \Mth\ with Maharam type $\lambda'$.   So $\mu$ witnesses that
$\lambda'\in\MahcrR(X)=\MahcrR(\{0,1\}^{\kappa})$.
}%end of proof of 532G

\vleader{48pt}{532H}{Lemma} Let $\familyiI{X_i}$ be a family of separable
metrizable spaces, and $\mu$ a totally finite completion regular
topological measure on $X=\prod_{i\in I}X_i$.   Then

(a) the support of $\mu$ is a zero set;

(b) $\mu$ is inner regular with respect to the self-supporting zero
sets.

\proof{{\bf (a)} Recall from 434Q that $\mu$ is $\tau$-additive, so has
a support $Z$.   Let $\sequencen{K_n}$ be a sequence of zero sets such
that $K_n\subseteq Z$ and $\mu K_n\ge\mu X-2^{-n}$ for each $n$.   Then
there is a countable set $J\subseteq I$ such that every $K_n$ is
determined by coordinates in $J$ (4A3Nc again).   
So $\bigcup_{n\in\Bbb N}K_n$ and
$Z'=\overline{\bigcup_{n\in\Bbb N}K_n}$ are determined by coordinates in
$J$ (4A2B(g-i)), and $Z'$ is a zero set, by 4A3Nc in the other direction.
But $Z'\subseteq Z$ and $\mu Z'=\mu Z$ so
$Z=Z'$ is a zero set.

\medskip

{\bf (b)} If $\mu E>\gamma$ then there is a zero set $K\subseteq E$ such
that $\mu K\ge\gamma$.   Now $\mu\LLcorner K$ (234M) is a totally finite
topological measure on $X$ which is completion regular (412Q),
so its support $Z$ is a
zero set, by (a);  and $Z\subseteq K\subseteq E$ 
is self-supporting for $\mu$ with $\mu Z\ge\gamma$.
}%end of proof of 532H

\leader{532I}{}\cmmnt{ There is a useful general characterization of
the sets $\MahcrR(\{0,1\}^{\kappa})$ in terms of the measure algebras
$\frak B_{\lambda}$.   At the same time, we can check that other
products of separable metrizable spaces follow powers of $\{0,1\}$, as
follows.

\medskip

\noindent}{\bf Theorem}\cmmnt{ ({\smc Choksi \& Fremlin 79})} Let
$\lambda\le\kappa$ be infinite cardinals.   Then the following are
equiveridical:

(i) $\lambda\in\MahcrR(\{0,1\}^{\kappa})$;

(ii) there is a family $\ofamily{\xi}{\kappa}{X_{\xi}}$ of non-singleton
separable metrizable spaces such that
$\lambda\in\Mahcr(\prod_{\xi<\kappa}X_{\xi})$;

(iii) there is a Boolean-independent family
$\ofamily{\xi}{\kappa}{b_{\xi}}$ in
$\frak B_{\lambda}$ with the following property:   for every
$a\in\frak B_{\lambda}$
there is a countable set $J\subseteq\kappa$ such that the subalgebras
generated by $\{a\}\Bcup\{b_{\xi}:\xi\in J\}$ and
$\{b_{\eta}:\eta\in\kappa\setminus J\}$ are Boolean-independent.

\proof{ If $\kappa=\omega$ then $\lambda=\omega$ and (i)-(iii) are all
true.   So we may assume that $\kappa$ is uncountable.

\medskip

{\bf (i)$\Rightarrow$(ii)} is trivial.

\medskip

{\bf (ii)$\Rightarrow$(iii)}\grheada\
If $\lambda\in\Mahcr(X)$, where every $X_{\xi}$ is a
non-trivial separable metrizable space and
$X=\prod_{\xi<\kappa}X_{\xi}$, let $\mu$ be a \Mth\ completion regular
topological probability measure on $X$ with Maharam type $\lambda$.   By
532Ha and 4A3Nc,
the support $Z$ of $\mu$ is determined by coordinates in a
countable subset $L$ of $\kappa$.

\medskip

\quad\grheadb\
Let $\frak A$ be the measure algebra of $\mu$.   For each $\xi<\kappa$,
let $f_{\xi}:X_{\xi}\to[0,1]$ be a continuous function taking both values
$0$ and $1$;  let $t_{\xi}\in\ooint{0,1}$ be such that
$\mu\{x:x\in X$, $f_{\xi}(x(\xi))=t_{\xi}\}=0$.   Set
$U_{\xi}=\{x:f_{\xi}(x(\xi))<t_{\xi}\}$,
$V_{\xi}=\{x:f_{\xi}(x)>t_{\xi}\}$;  then $U_{\xi}$ and $V_{\xi}$ are
disjoint non-empty open sets in $X$, both determined by coordinates in
$\{\xi\}$, and $\mu(U_{\xi}\cup V_{\xi})=1$.   Set
$b_{\xi}=U_{\xi}^{\ssbullet}$ in $\frak A$.   Then
$\ofamily{\xi}{\kappa\setminus L}{b_{\xi}}$ is Boolean-independent.
\Prf\ If $I$, $I'\subseteq\kappa\setminus L$ are disjoint finite sets,
then $H=X\cap\bigcap_{\xi\in I}U_{\xi}\cap\bigcap_{\xi\in I'}V_{\xi}$ is
a non-empty open set in $X$.   As $H$ is determined by coordinates in
$I\cup I'$ and $Z$ is determined by coordinates in $L$, $Z\cap H$ is
non-empty and therefore non-negligible;  so $\mu H>0$ and
$\inf_{\xi\in I}b_{\xi}\Bsetminus\sup_{\xi\in I'}b_{\xi}$ is non-zero in
$\frak A$.\ \Qed

\medskip

\quad\grheadc\
If $a\in\frak A$ let $E$ be such that $E^{\ssbullet}=a$.   By 532Hb, we
can choose for each $n\in\Bbb N$ self-supporting zero sets
$K_n\subseteq E$, $\tilde K_n\subseteq X\setminus E$ such that
$\mu K_n+\mu\tilde K_n\ge 1-2^{-n}$.   Let $J\subseteq\kappa\setminus L$
be a countable set such that every $K_n$ and every $\tilde K_n$ is
determined by coordinates in $J\cup L$.
Now the subalgebras $\frak D_1$, $\frak D_2$ generated by
$\{a\}\cup\{b_{\xi}:\xi\in J\}$ and
$\{b_{\xi}:\xi\in(\kappa\setminus L)\setminus J\}$ are
Boolean-independent.   \Prf\ Take non-zero $d_1\in\frak D_1$ and
$d_2\in\frak D_2$.   Suppose for the moment that
$d_1\Bcap a\ne 0$.   As in ($\beta$),
there is an open set $G$, determined by coordinates in
$J$, such that $0\ne a\Bcap G^{\ssbullet}\Bsubseteq d_1$.
There is also an open set $H$, determined by coordinates in
$\kappa\setminus(J\cup L)$, such that
$0\ne H^{\ssbullet}\Bsubseteq d_2$.   Next, as
$a=\sup_{n\in\Bbb N}K_n^{\ssbullet}$, there is an $n\in\Bbb N$ such that
$0\ne K_n^{\ssbullet}\Bcap G^{\ssbullet}$, that is,
$K_n\cap G\ne\emptyset$.   As $K_n\cap G$ is determined by coordinates
in $J\cup L$ and $H$ is determined by coordinates in
$\kappa\setminus(J\cup L)$,
$K_n\cap G\cap V\ne\emptyset$;  as $K_n$ is self-supporting,

\Centerline{$0\ne(K_n\cap G\cap H)^{\ssbullet}\Bsubseteq d_1\Bcap d_2$.}

\noindent In the same way, using $K'_n$ in place of $K_n$, we see that
$d_1\Bcap d_2\ne 0$ if $d_1\Bsetminus a\ne 0$.   As $d_1$ and $d_2$ are
arbitrary, $\frak D_1$ and $\frak D_2$ are Boolean-independent.\ \Qed

\medskip

\quad\grheadd\
As $\#(\kappa\setminus L)=\kappa$ and $\frak A\cong\frak B_{\lambda}$,
$\family{\xi}{\kappa\setminus L}{b_{\xi}}$, suitably reinterpreted,
witnesses that (iii) is satisfied.

\medskip

{\bf (iii)$\Rightarrow$(i)} Now suppose that the conditions of (iii) are
satisfied.   Let $(Z,\nu)$ be the Stone space of
$(\frak B_{\lambda},\bar\nu_{\lambda})$.   (See 411P
for a summary of the properties of these spaces.)   For
$b\in\frak B_{\lambda}$ write $\widehat{b}$ for the corresponding
open-and-closed subset of $Z$.    Define $\phi:Z\to\{0,1\}^{\kappa}$ by
setting $\phi(z)=\ofamily{\xi}{\kappa}{\chi\widehat{b}_{\xi}(z)}$ for
$z\in Z$.   Then $\phi$ is continuous;  let $\mu=\nu\phi^{-1}$ be the
image Radon measure on $\{0,1\}^{\kappa}$ (418I).   Now $\mu$ is
completion regular.   \Prf\ Suppose that $K\subseteq\{0,1\}^{\kappa}$ is
compact and self-supporting.   Identifying
$\frak B_{\lambda}$ with the measure algebra of $\nu$, we have a Boolean
homomorphism $\psi:\dom\mu\to\frak B_{\lambda}$ defined by setting
$\psi E=(\phi^{-1}[E])^{\ssbullet}$ whenever $\mu$ measures $E$, and
$\bar\nu_{\lambda}\psi E=\nu\phi^{-1}[E]=\mu E$ for every $E$;  setting
$E_{\xi}=\{x:x\in\{0,1\}^{\kappa}$, $x(\xi)=1\}$,
$\psi E_{\xi}=b_{\xi}$.   Set $a=\psi K$.   Let $J\subseteq\kappa$ be a
countable set such that
the subalgebras $\frak D_1$, $\frak D_2$ generated by
$\{a\}\Bcup\{b_{\xi}:\xi\in J\}$ and
$\{b_{\eta}:\eta\in\kappa\setminus J\}$ are Boolean-independent.
\Quer\ If $x\in K$, $y\in\{0,1\}^{\kappa}\setminus K$ and
$x\restr J=y\restr J$, let $U$ be an open cylinder containing $y$
and disjoint from $K$.   Express $U$ as $U'\cap U''$ where $U'$ is
determined by coordinates in $J$ and $U''$ by coordinates in
$\kappa\setminus J$.   Then $\psi U'\in\frak D_1$ and
$\psi U''\in\frak D_2$.   As $\ofamily{\xi}{\kappa}{b_{\xi}}$ is
Boolean-independent,
$\psi U''\ne 0$.   Now $K$ is self-supporting and $x\in K\cap U'$, so
$\mu(K\cap U')>0$ and $\psi(K\cap U')=a\Bcap\psi U'$ is non-zero;  also
$a\Bcap\psi U'\in\frak D_1$;  because $\frak D_1$ and $\frak D_2$ are
Boolean-independent, $\psi(K\cap U)=a\Bcap\psi U'\Bcap\psi U''\ne 0$ and
$K\cap U$ cannot be empty, contrary to the choice of $U$.\ \Bang

This shows that $K$ is determined by coordinates in $J$ and is a zero
set (4A3Nc, in the other direction).
As $K$ is arbitrary, we see that all self-supporting compact sets
are zero sets.   But as $\mu$ is a Radon measure, it is inner regular
with respect to the self-supporting compact sets, therefore with respect
to the zero sets, and is completion regular.\ \Qed

The \imp\ function $\phi$ (and, of course, the Boolean homomorphism
$\psi$) correspond to an embedding of the measure algebra
of $\mu$ into
$\frak B_{\lambda}$.   So the Maharam type of $\mu$ is at most
$\lambda$.
There is therefore a $\lambda'\in\MahcrR(\{0,1\}^{\kappa})$ such
that $\lambda'\le\lambda$ (532B).   By 532G,
$\lambda\in\MahcrR(\{0,1\}^{\kappa})$.
}%end of proof of 532I

\leader{532J}{Corollary}
(a) Suppose that $\lambda$, $\kappa$ are infinite cardinals and
$\lambda\in\MahcrR(\{0,1\}^{\kappa})$.   Then
$\kappa$ is at most the cardinal power $\lambda^{\omega}$.

(b) If $\kappa$ is an infinite cardinal such that
$\lambda^{\omega}<\kappa$ for every $\lambda<\kappa$\cmmnt{ (e.g.,
$\kappa=\frak c^+$)}, then $\MahcrR(\{0,1\}^{\kappa})=\{\kappa\}$.

\proof{{\bf (a)} By 532I, $\kappa\le\#(\Cal B_{\lambda})$;  by
524Ma, $\#(\Cal B_{\lambda})\le\lambda^{\omega}$.

\medskip

{\bf (b)} By (a), no infinite cardinal less than $\kappa$ can belong to
$\MahcrR(\{0,1\}^{\kappa})$.   Also $\kappa$ is uncountable, so the
remarks in 532C tell us the rest of what we need.
}%end of proof of 532J

\leader{532K}{Corollary} If $\omega\le\lambda\le\kappa'\le\kappa$ and
$\lambda\in\MahcrR(\{0,1\}^{\kappa})$ then
$\lambda\in\MahcrR(\{0,1\}^{\kappa'})$.

\proof{ If $\ofamily{\xi}{\kappa}{b_{\xi}}$ witnesses the truth of
532I(iii) for $\lambda$ and $\kappa$, then its subfamily
$\ofamily{\xi}{\kappa'}{b_{\xi}}$ witnesses the truth of 532I(iii) for
$\lambda$ and $\kappa'$.   \Prf\ Of course
$\ofamily{\xi}{\kappa'}{b_{\xi}}$ is Boolean-independent.   If
$a\in\frak B_{\lambda}$, there is a countable set $J\subseteq\kappa$ such
that the subalgebras generated by $\{a\}\cup\{b_{\xi}:\xi\in J\}$ and
$\{b_{\eta}:\eta\in\kappa\setminus J\}$ are Boolean-independent.
Now $J'=J\cap\kappa'$ is a countable subset of $\kappa'$ and
the subalgebras generated by $\{a\}\cup\{b_{\xi}:\xi\in J'\}$ and
$\{b_{\eta}:\eta\in\kappa'\setminus J'\}$ are Boolean-independent.\ \Qed
}%end of proof of 532K

\leader{532L}{Corollary} If $\omega\le\lambda\le\lambda'$ and
$\cff[\lambda']^{\le\lambda}<\cf\kappa$ and
$\lambda'\in\MahcrR(\{0,1\}^{\kappa})$, then
$\lambda\in\MahcrR(\{0,1\}^{\kappa})$.

\proof{ Let $\ofamily{\xi}{\kappa}{b_{\xi}}$ be a family in
$\frak B_{\lambda'}$ satisfying (iii) of 532I.   Let
$\ofamily{\eta}{\lambda'}{e_{\eta}}$ be the standard generating
family in $\frak B_{\lambda'}$, and
$\Cal J$ a cofinal subset of $[\lambda']^{\lambda}$ with cardinal less
than $\cf\kappa$.   For each $\xi<\kappa$, there are a countable set
$L\subseteq\lambda'$ such that $b_{\xi}$ belongs to the closed
subalgebra $\frak C_L$ of $\frak B_{\lambda'}$ generated by
$\{e_{\eta}:\eta\in L\}$, and a $J_{\xi}\in\Cal J$ such that
$L\subseteq J_{\xi}$.   Because
$\#(J)<\cf\kappa$, there is a $J\in\Cal J$ such that
$A=\{\xi:\xi<\kappa$, $J_{\xi}=J\}$ has cardinal $\kappa$.   Now the
closed subalgebra $\frak C_J$ of $\frak B_{\lambda'}$ generated by
$\{e_{\eta}:\eta\in J\}$ is isomorphic to $\frak B_{\lambda}$, and the
Boolean-independent $\family{\xi}{A}{b_{\xi}}$ in $\frak C_J$
witnesses that 532I(iii) is true of $\lambda$
and $\kappa$, as in the proof of 532K.
}%end of proof of 532L

\leader{532M}{}\cmmnt{ I turn now to the question of identifying those
$\kappa$ for which $\omega\in\MahcrR(\{0,1\}^{\kappa})$.   We know from
532C and 532Ja that
they all lie between $\omega$ and $\frak c$.   To go farther we need to
look at some of the cardinals from \S522.

\medskip

\noindent}{\bf Proposition} If
$A\subseteq\frak B_{\omega}\setminus\{0\}$ and
$\#(A)<\frak d\cmmnt{\mskip5mu=\cf(\BbbN^{\Bbb N})}$, then there is
a $c\in\frak B_{\omega}$ such that neither $c$ nor $1\Bsetminus c$
includes any member of $A$.

\proof{ Let $\sequencen{e_n}$ be the standard generating family in
$\frak B_{\omega}=\frak B_{\Bbb N}$.   
For $a\in\frak A$ and $n\in\Bbb N$ let
$f_a(n)\in\Bbb N$ be such that there is a $b$ in the subalgebra
$\frak C_{f_a(n)^2}$ generated by $\{e_i:i<f_a(n)^2\}$ such that
$\bar\nu_{\omega}(b\Bsymmdiff a)<2^{-n-3}\bar\mu a$.   Because
$\#(A)<\frak d$, there is an $f\in\BbbN^{\Bbb N}$ such that
$f\not\le f_a$ for every $a\in\frak A$;  we may suppose that $f$ is
strictly increasing and $f(0)>0$.   Note that

\Centerline{$f(n)^2+n+1<f(n)^2+2f(n)+1\le f(n+1)^2$}

\noindent for every $n$.   For
each $n\in\Bbb N$, set 

\Centerline{$I_n=f(n)^2\subseteq\Bbb N$,
\quad$I'_n=I_{n+1}\setminus I_n$,}

\Centerline{$c'_n=\inf_{f(n)^2\le i\le f(n)^2+n+1}e_i
\in\frak C_{I'_n}$;}

\noindent then $\bar\nu_{\omega}c'_n=2^{-n-2}$ for each
$n$.   Define $c_n\in\frak C_{I_n}$, 
for $n\in\Bbb N$, by setting $c_0=0$ and
$c_{n+1}=c_n\Bsymmdiff c'_n$ for each $n$.   Then 
$\bar\nu_{\omega}(c_{n+1}\Bsymmdiff c_n)=2^{-n-2}$ for every $n$, so
$\sequencen{c_n}$ is a Cauchy sequence for the measure metric on
$\frak B_{\omega}$, and has a limit $c$.   Note that

\Centerline{$\sum_{i=n}^{m-1}2^{-i-3}
\le\bar\nu_{\omega}(c_m\Bsymmdiff c_n)
\le\sum_{i=n}^{m-1}2^{-i-2}$}

\noindent whenever $m\ge n$.   \Prf\ Induce on $m$.   For $m=n$ the
result is trivial (interpreting $\sum_{i=n}^{n-1}$ as zero).   For the
inductive step to $m+1$, $c'_m\in\frak C_{I'_m}$ is stochastically
independent of $c_m\Bsymmdiff c_n\in\frak C_{I_m}$, so

$$\eqalignno{\bar\nu_{\omega}(c_{m+1}\Bsymmdiff c_n)
&=\bar\nu_{\omega}(c'_m\Bsymmdiff c_m\Bsymmdiff c_n)\cr
&=\bar\nu_{\omega}c'_m+\bar\nu_{\omega}(c_m\Bsymmdiff c_n)
   -2\bar\nu_{\omega}(c'_m\Bcap(c_m\Bsymmdiff c_n))\cr
&=2^{-m-2}+(1-2^{-m-1})\bar\nu_{\omega}(c_m\Bsymmdiff c_n)\cr
&\ge 2^{-m-2}+(1-2^{-m-1})\sum_{i=n}^{m-1}2^{-i-3}\cr
\displaycause{by the inductive hypothesis}
&=\sum_{i=n}^{m-1}2^{-i-3}+2^{-m-3}(2-4\sum_{i=n}^{m-1}2^{-i-3})
\ge\sum_{i=n}^m2^{-i-3};\cr}$$

\noindent on the other hand,

\Centerline{$\bar\nu_{\omega}(c_{m+1}\Bsymmdiff c_n)
\le 2^{-m-2}+\bar\nu_{\omega}(c_m\Bsymmdiff c_n)
\le\sum_{i=n}^m2^{-i-2}$.}

\noindent So the induction proceeds.\ \QeD\  Taking the limit
as $m\to\infty$, we see
that $2^{-n-2}\le\bar\nu_{\omega}(c\Bsymmdiff c_n)\le 2^{-n-1}$ for
every $n\in\Bbb N$.

Take any $a\in A$.   Let $n\in\Bbb N$ be such that $f_a(n)<f(n)$.   Then
there is a $b\in\frak C_{I_n}$ such that
$\bar\nu_{\omega}(a\Bsymmdiff b)<2^{-n-3}\bar\mu a$.   Now
$c\Bsymmdiff c_n\in\frak C_{\Bbb N\setminus I_n}$ 
is stochastically independent of both
$b\Bsetminus c_n$ and $b\Bcap c_n$, so

$$\eqalign{\bar\nu_{\omega}(b\Bsetminus c)
&=\bar\nu_{\omega}(((b\Bsetminus c_n)\Bsetminus(c\Bsymmdiff c_n))
 \Bcup((b\Bcap c_n)\Bcap(c\Bsymmdiff c_n)))
\cr
&=\bar\nu_{\omega}(b\Bsetminus c_n)
  (1-\bar\nu_{\omega}(c\Bsymmdiff c_n))
 +\bar\nu_{\omega}(b\Bcap c_n)\cdot\bar\nu_{\omega}(c\Bsymmdiff c_n)
\cr
&\ge\bar\nu_{\omega}(b\Bsetminus c_n)(1-2^{-n-1})
  +2^{-n-2}\bar\nu_{\omega}(b\Bcap c_n)
\ge 2^{-n-2}\bar\nu_{\omega}b
\ge 2^{-n-3}\bar\nu_{\omega}a.\cr}$$

\noindent So

\Centerline{$\bar\nu_{\omega}(a\Bsetminus c)
\ge 2^{n-3}\bar\nu_{\omega}a-\bar\nu_{\omega}(b\Bsetminus a)
>0$,}

\noindent and $a\notBsubseteq c$.   Similarly,

$$\eqalign{\bar\nu_{\omega}(b\Bcap c)
&=\bar\nu_{\omega}(b\cap c_n)
  (1-\bar\nu_{\omega}(c\Bsymmdiff c_n))
 +\bar\nu_{\omega}(b\Bsetminus c_n)
  \cdot\bar\nu_{\omega}(c\Bsymmdiff c_n)\cr
&\ge\bar\nu_{\omega}(b\Bcap c_n)(1-2^{-n-1})
  +2^{-n-2}\bar\nu_{\omega}(b\Bsetminus c_n)
\ge 2^{-n-2}\bar\nu_{\omega}b,\cr}$$

\noindent and $\bar\nu_{\omega}(a\Bcap c)>0$.

As $a$ is arbitrary, we have found an appropriate $c$.
}%end of proof of 532M

\leader{532N}{}\cmmnt{ It will be useful to have a classic example
relevant to a question already examined in 325F.

\medskip

\noindent}{\bf Lemma} There is a Borel set
$W\subseteq\{0,1\}^{\Bbb N}\times\{0,1\}^{\Bbb N}$ such that whenever
$E$, $F\subseteq\{0,1\}^{\Bbb N}$ have positive measure for
$\nu_{\omega}$ then neither $(E\times F)\cap W$ nor
$(E\times F)\setminus W$ is negligible for the product measure
$\nu_{\omega}^2$ on  $\{0,1\}^{\Bbb N}\times\{0,1\}^{\Bbb N}$.

\proof{{\bf (a)} (Cf.\ 134Jb.)   There is a Borel set
$H\subseteq\{0,1\}^{\Bbb N}$ such that both $H$ and its complement meet
every non-empty open set in a set of non-zero measure.   \Prf\ For
$x\in\{0,1\}^{\Bbb N}$ set $I_x=\{n:x(i)=0$ for $2^n\le i<2^{n+1}\}$.   Set
$H=\{x:I_x$ is finite and not empty and $\max I_x$ is even$\}$.\ \Qed

\medskip

{\bf (b)} Let $+$ be the usual group operation on
$\{0,1\}^{\Bbb N}\cong\Bbb Z_2^{\Bbb N}$.   In this group, addition and
subtraction are identical, as $x+x=0$ for every $x$;  but the formulae
may be easier to read if I use the symbol $-$ when it seems appropriate.
Set $W=\{(x,y):x$, $y\in\{0,1\}^{\Bbb N}$, $x-y\in H\}$.

Let $E$, $F\subseteq\{0,1\}^{\Bbb N}$ be sets of positive measure.   Then
$\{z:z\in\{0,1\}^{\Bbb N}$, $\nu_{\omega}(E\cap(F+z))>0\}$ is open
(443C) and not empty (443Da), so meets $H$ in a set of positive measure.
Now

$$\eqalignno{\nu_{\omega}^2((E\times F)\cap W)
&=\nu_{\omega}^2\{(x,y):x\in E,\,y\in F,\,x-y\in H\}\cr
&=\nu_{\omega}^2\{(x,z):x\in E,\,x-z\in F,\,z\in H\}\cr
\displaycause{because $(x,y)\mapsto(x,x-y)$ is a measure space automorphism
for $\nu_{\omega}^2$, as in 255Ae or 443Xa}
&=\nu_{\omega}^2\{(x,z):x\in E,\,x\in F+z,\,z\in H\}\cr
&=\int_H\nu_{\omega}(E\cap(F+z))\nu_{\omega}(dz)
>0.\cr}$$

\noindent Applying the same argument with $\{0,1\}^{\Bbb N}\setminus H$
in the
place of $H$, we see that the same is true of $(E\times F)\setminus W$.
}%end of proof of 532N

\leader{532O}{Proposition} If
$A\subseteq\frak B_{\omega}\setminus\{0\}$ and
$\#(A)<\cov\Cal N_{\omega}$, then there is a $c\in\frak B_{\omega}$
such that neither $c$ nor $1\Bsetminus c$ includes any member of $A$.

\proof{ Take $W\subseteq\{0,1\}^{\Bbb N}\times\{0,1\}^{\Bbb N}$ as in
532N.   For $x\in\{0,1\}^{\Bbb N}$, set $c_x=W[\{x\}]^{\ssbullet}$ in
$\frak B_{\omega}$.
If $a\in A$, then $\{x:a\Bsubseteq c_x\}\in\Cal N_{\omega}$.   \Prf\ Let
$F\in\Tau_{\omega}$ be such that $F^{\ssbullet}=a$, and set
$E=\{x:a\Bsubseteq c_x\}$.   Because
$x\mapsto c_x$ is measurable when $\frak B_{\omega}$ is given its
measure-algebra topology (418Ta), $E\in\Tau_{\omega}$.   For every
$x\in E$, $F\setminus W[\{x\}]$ is negligible, so
$(E\times F)\setminus W$ is negligible, by Fubini's theorem (252D).   
But this means that at least
one of $E$, $F$ must be negligible;  since $F^{\ssbullet}=a\ne 0$,
$\nu_{\omega}E=0$, as required.\ \Qed

Similarly, $\{x:a\Bcap c_x=0\}$ is negligible.   Since
$\{0,1\}^{\Bbb N}$ cannot be covered by $\#(A)$ negligible sets, there
is an $x\in\{0,1\}^{\Bbb N}$ such that $c_x$ neither includes, nor is
disjoint from, any member of $A$.
}%end of proof of 532O

\leader{532P}{Proposition} Set
$\kappa=\max(\frak d,\cov\Cal N_{\omega})$.   If
$\FN(\Cal P\Bbb N)=\omega_1$, then
$\omega\in\MahcrR(\{0,1\}^{\kappa})$.   In particular, if
$\frak c=\omega_1$ then $\omega\in\MahcrR(\{0,1\}^{\omega_1})$.

\proof{{\bf (a)} By 524O(b-ii), $\FN(\frak B_{\omega})=\omega_1$;  let
$f:\frak B_{\omega}\to[\frak B_{\omega}]^{\le\omega}$ be a
Freese-Nation function.   By 532M (if $\kappa=\frak d$) or 532O (if
$\kappa=\cov\Cal N_{\omega}$), we can choose inductively a family
$\ofamily{\xi}{\kappa}{b_{\xi}}$ in $\frak B_{\omega}$ such that neither
$b_{\xi}$ nor $1\Bsetminus b_{\xi}$ includes any nonzero member of
$\frak D_{\xi}$, where
$\frak D_{\xi}$ is the smallest subalgebra of $\frak B_{\omega}$
including $\{b_{\eta}:\eta<\xi\}$ and such that
$f(d)\subseteq\frak D_{\xi}$ for every $d\in\frak D_{\xi}$.
Of course this
implies that $\ofamily{\xi}{\kappa}{b_{\xi}}$ is Boolean-independent.

\medskip

{\bf (b)} For $K$, $L\subseteq\kappa$ set
$d_{KL}=\inf_{\xi\in K}b_{\xi}\Bsetminus\sup_{\xi\in L}b_{\xi}$.
For $a\in\frak B_{\omega}$, set
$Q_a=\{(K,L):K$, $L\in[\kappa]^{<\omega}$ are disjoint,
$d_{KL}\Bsubseteq a\}$, and let $Q'_a$ be the set of minimal members of
$Q_a$, taking $(K,L)\le(K',L')$ if $K\subseteq K'$ and $L\subseteq L'$.
Of course $Q_a$ is well-founded so $Q'_a$ is coinitial with $Q_a$.
Now $R_{an}=\{(K,L):(K,L)\in Q'_a$, $\#(K\cup L)=n\}$ is countable for
every $n\in\Bbb N$ and $a\in\frak B_{\omega}$.   \Prf\ Induce on $n$.
If $n=0$ this is trivial.   For the inductive step to $n+1$, set
$R'_{\zeta}
=\{(K,L):K\cup L\subseteq\zeta$, $(K\cup\{\zeta\},L)\in R_{a,n+1}\}$ for
each $\zeta<\kappa$.   For $(K,L)\in R'_{\zeta}$,
$b_{\zeta}\cap d_{KL}=d_{K\cup\{\zeta\},L}$ is included in $a$,
so there is a
$c_{KL\zeta}\in f(d_{K\cup\{\zeta\},L})\cap f(a)$ such that
$d_{K\cup\{\zeta\},L}\Bsubseteq c_{KL\zeta}\Bsubseteq a$, in which case
$b_{\zeta}\Bsubseteq c_{KL\zeta}\Bcup(1\Bsetminus d_{KL})$.   If
$\zeta<\zeta'<\kappa$, $(K,L)\in R'_{\zeta}$ and
$(K',L')\in R'_{\zeta'}$, then $d_{K'L'}\notBsubseteq a$ (because
$(K'\cup\{\zeta'\},L')$ is a minimal member of $Q_a$), so
$c_{KL\zeta}\Bcup(1\Bsetminus d_{K'L'})\ne 1$;  as $c_{KL\zeta}$ and
$d_{K'L'}$ both belong to $\frak D_{\zeta'}$,
$b_{\zeta'}\notBsubseteq c_{KL\zeta}\Bcup(1\Bsetminus d_{K'L'})$ and
$c_{KL\zeta}\ne c_{K'L'\zeta'}$.   As $f(a)$ is countable,
$A=\{\zeta:R'_{\zeta}\ne\emptyset\}$ is countable.   Next, for any
$\zeta\in A$ and $(K,L)\in R'_{\zeta}$, we see that
$d_{KL}\subseteq a\Bcup(1\Bsetminus b_{\zeta})$, and indeed that
$(K,L)\in Q'_{a\Bcup(1\Bsetminus b_{\zeta})}$, so that
$(K,L)\in R_{a\Bcup(1\Bsetminus b_{\zeta}),n}$.   By the inductive
hypothesis, $R'_{\zeta}$ is countable.

This shows that
$\{(K,L,\zeta):K\cup L\subseteq\zeta$,
$(K\cup\{\zeta\},L)\in R_{a,n+1}\}$ is countable.    In the same way,
applying the ideas above to
$1\Bsetminus b_{\zeta}$ in place of $b_{\zeta}$,
$\{(K,L,\zeta):K\cup L\subseteq\zeta$,
$(K,L\cup\{\zeta\})\in R_{a,n+1}\}$ is countable;  so $R_{a,n+1}$ is
countable and the induction proceeds.\ \Qed

It follows that $Q'_a$ is countable for every $a\in\frak B_{\omega}$.

\medskip

{\bf (c)} Now take any $a\in\frak B_{\omega}$ and let
$J\subseteq\kappa$ be a countable set such that $K\cup L\subseteq J$
whenever $(K,L)\in Q'_a\cup Q'_{1\Bsetminus a}$.   \Quer\ Suppose, if
possible, that the algebras $\frak E_1$, $\frak E_2$ generated by
$\{a\}\Bcup\{b_{\xi}:\xi\in J\}$ and
$\{b_{\eta}:\eta\in\kappa\setminus J\}$ are not Boolean-independent.
Then there must be finite subsets $K$, $L$, $K'$ and $L'$ of $\kappa$
such that $K\cup L\subseteq J$,
$K'\cup L'\subseteq\kappa\setminus J$, $d_{K'L'}\ne 0$, and either

\Centerline{$d_{KL}\Bcap a\ne 0$,
$d_{K'L'}\Bcap d_{KL}\Bcap a=0$}

\noindent or

\Centerline{$d_{KL}\Bsetminus a\ne 0$,
$d_{K'L'}\Bcap d_{KL}\Bsetminus a=0$.}

\noindent Suppose the former.   Then
$(K\cup K',L\cup L')\in Q_{1\Bsetminus a}$ so there is a
$(K'',L'')\in Q'_{1\Bsetminus a}$ such that $K''\subseteq K\cup K'$ and
$L''\subseteq L\cup L'$;  in which case $K''\cup L''\subseteq J$ so in
fact $K''\subseteq K$, $L''\subseteq L$ and
$d_{KL}\Bcap a\Bsubseteq d_{K''L''}\Bcap a=0$, which is impossible.
Replacing $a$ by $1\Bsetminus a$ we get a similar contradiction in the
second case.\ \BanG\  So $\frak E_1$ and $\frak E_2$ are
Boolean-independent.

\medskip

{\bf (d)} As $a$ is arbitrary, (c) shows that
$\ofamily{\xi}{\kappa}{b_{\xi}}$ satisfies the conditions of 532I(iii),
so that $\omega$ belongs to $\MahcrR(\{0,1\}^{\kappa})$, as claimed.
}%end of proof of 532P

\leader{532Q}{Proposition} Suppose that $\add\Cal N_{\omega}>\omega_1$.

(a) $\lambda\notin\MahcrR(\{0,1\}^{\kappa})$ whenever $\lambda\ge\omega$
and $\max(\omega,\cff[\lambda]^{\le\omega})<\kappa$.

(b) If $\omega_1\le\kappa\le\omega_{\omega}$ then
$\MahcrR(\{0,1\}^{\kappa})=\{\kappa\}$.

\proof{{\bf (a)} \Quer\ If $\lambda\in\MahcrR(\{0,1\}^{\kappa})$, set
$\kappa'=(\max(\omega,\cff[\lambda]^{\le\omega}))^+$;  then
$\lambda<\kappa'$ so
$\lambda\in\MahcrR(\{0,1\}^{\kappa'})$ (532K).   As
$\cff[\lambda]^{\le\omega}<\cf\kappa'$, $\omega$ belongs to
$\MahcrR(\{0,1\}^{\lambda^+})$ (532L) and therefore to
$\MahcrR(\{0,1\}^{\omega_1})$ (532K again).

Let $\ofamily{\xi}{\omega_1}{b_{\xi}}$ be a
family in $\frak B_{\omega}$ satisfying the conditions of 532I(iii).
By 524Mb, $\omega_1<\wdistr(\frak B_{\omega})$;  by 514K, there is a
countable $C\subseteq\frak B_{\omega}\setminus\{0\}$ such that for every
$\xi<\omega_1$ there is a $c\in C$ such that
$c\Bsubseteq b_{\xi}$.   Let $a\in C$ be such that
$\{\xi:\xi<\omega_1$, $a\Bsubseteq b_{\xi}\}$ is uncountable.   
There is
supposed to be a countable $J\subseteq\omega_1$ such that the
subalgebras generated by $\{a\}$ and
$\{b_{\xi}:\xi\in\omega_1\setminus J\}$ are
Boolean-independent;  but then $\{\xi:a\Bsubseteq b_{\xi}\}\subseteq J$,
which is impossible.\ \Bang

This shows that (a) is true.

\medskip

{\bf (b)} If $\omega\le\lambda<\kappa\le\omega_{\omega}$, then
$\cff[\lambda]^{\le\omega}\le\lambda<\kappa$ (5A1E(e-iv)),
so (a) tells us that
$\lambda\notin\MahcrR(\{0,1\}^{\kappa})$.   From 532C we see that
$\MahcrR(\{0,1\}^{\kappa})$ must be $\{\kappa\}$ exactly.
}%end of proof of 532Q

\leader{532R}{}\cmmnt{ Two combinatorial principles already used in
524O are relevant to the questions treated here.

\medskip

\noindent}{\bf Proposition} Suppose that $\lambda$ is an uncountable
cardinal with countable cofinality such that
$\square_{\lambda}$\cmmnt{ (definition:  5A6D(a-ii))} is
true.   Set $\kappa=\lambda^+$.   Then
$\lambda\in\MahcrR(\{0,1\}^{\kappa})$.

\proof{{\bf (a)} Let $\ofamily{\xi}{\kappa}{I_{\xi}}$ be a family of
countably infinite subsets of $\lambda$ as in 5A6E.   For each
$\xi<\kappa$, let $\sequencen{I_{\xi n}}$, $\sequencen{\alpha_{\xi n}}$
be such that $\sequencen{I_{\xi n}}$ is a disjoint sequence of subsets
of $I_{\xi}$ with $\#(I_{\xi n})=n$ for each $n$ and
$\sequencen{\alpha_{\xi n}}$ is a sequence of distinct points in
$I_{\xi}\setminus\bigcup_{n\in\Bbb N}I_{\xi n}$.   Set

\Centerline{$U_{\xi n}
=\{x:x\in\{0,1\}^{\lambda}$, $x(\eta)=0$ for every $\eta\in I_{\xi n}\}$,}

\Centerline{$V_{\xi n}
=\{x:x\in U_{\xi n}\setminus\bigcup_{m>n}U_{\xi m}$,
$x(\alpha_{\xi n})=1\}$,}

\Centerline{$\tilde V_{\xi n}
=\{x:x\in U_{\xi n}\setminus\bigcup_{m>n}U_{\xi m}$,
$x(\alpha_{\xi n})=0\}$}

\noindent for $n\in\Bbb N$.   Note that as
$\nu_{\kappa}U_{\xi m}=2^{-m}$ for each $n$, $V_{\xi n}$ and
$\tilde V_{\xi n}$ are non-negligible, while both are determined by
coordinates in
$\{\alpha_{\xi n}\}\cup\bigcup_{m\ge n}I_{\xi m}\subseteq I_{\xi}$.   Set

\Centerline{$F_{\xi}=\bigcup_{n\in\Bbb N}V_{\xi n}$,
\quad$b_{\xi}=F_{\xi}^{\ssbullet}\in\frak B_{\lambda}$.}

\noindent Note that $F_{\xi}\cap\tilde V_{\xi n}=\emptyset$ for every $n$.

\medskip

{\bf (b)} Take any $a\in\frak B_{\lambda}$.   Then we can express $a$ as
$E^{\ssbullet}$ where $E\subseteq\{0,1\}^{\lambda}$ is a Baire set;  let
$I\subseteq\lambda$ be a countable set such that $E$ is determined by
coordinates in $I$.   By the choice of $\ofamily{\xi}{\kappa}{I_{\xi}}$
there is a countable set $J\subseteq\kappa$ such that $I\cap I_{\xi}$ is
finite for every $\xi\in\kappa\setminus J$.   Let $\frak D_1$,
$\frak D_2$ be the subalgebras of $\frak B_{\lambda}$ generated by
$\{a\}\cup\{b_{\xi}:\xi\in J\}$ and
$\{b_{\xi}:\xi\in\kappa\setminus J\}$ respectively.   Then $\frak D_1$
and $\frak D_2$ are Boolean-independent.   \Prf\ If $d_1\in\frak D_1$
and $d_2\in\frak D_2$ are non-zero, we can express $d_1$ as
$H_1^{\ssbullet}$ where $H_1\subseteq\{0,1\}^{\lambda}$ is a Baire set
determined by coordinates in $L=I\cup\bigcup_{\xi\in K}I_{\xi}$ for some
finite $K\subseteq J$.   Next, we can find disjoint finite sets $K'$,
$K''\subseteq\kappa\setminus J$ such that
$d_2\Bsupseteq\inf_{\xi\in K'}b_{\xi}\Bsetminus\sup_{\xi\in K''}b_{\xi}$.
Because all the sets $I_{\xi}\cap I_{\eta}$, for distinct $\xi$,
$\eta<\kappa$, and also the sets $I\cap I_{\xi}$, for
$\xi\in\kappa\setminus J$, are finite, there is an $m\in\Bbb N$ such
that all the sets
$J_{\xi}=\{\alpha_{\xi m}\}\cup\bigcup_{n\ge m}I_{\xi n}$, for
$\xi\in K'\cup K''$, are disjoint from each other and from $I$.   Look
at the sets $V_{\xi m}$, for $\xi\in K'$, and $\tilde V_{\xi m}$,
for $\xi\in K''$.   Set
$H_2=\{0,1\}^{\lambda}\cap\bigcap_{\xi\in K'}V_{\xi m}
\cap\bigcap_{\xi\in K''}\tilde V_{\xi m}$.   Then
$H_2^{\ssbullet}\Bsubseteq d_2$.   But observe now that all the
$V_{\xi m}$ and $\tilde V_{\xi m}$ are non-negligible and that $V_{\xi m}$,
$\tilde V_{\xi m}$ are determined by coordinates in $J_{\xi}$ for each
$\xi\in K'\cup K''$.   So the sets $H_1$, $V_{\xi m}$ (for $\xi\in K'$)
and $\tilde V_{\xi m}$ (for $\xi\in K''$) are stochastically
independent, and

\Centerline{$\bar\nu_{\lambda}(d_1\Bcap d_2)
\ge\nu_{\lambda}(H_1\cap H_2)
=\nu_{\lambda} H_1\cdot\prod_{\xi\in K'}\nu_{\lambda} V_{\xi m}
  \cdot\prod_{\xi\in K''}\nu_{\lambda}\tilde V_{\xi m}
>0$.}

\noindent Thus $d_1\Bcap d_2\ne 0$;  as $d_1$ and $d_2$ are arbitrary,
$\frak D_1$ and $\frak D_2$ are stochastically independent.\ \Qed

\medskip

{\bf (c)} The argument of (b) works equally well with $I=\emptyset$
and $J$ an arbitrary
finite subset of $\kappa$ to show that $\ofamily{\xi}{\kappa}{b_{\xi}}$
is Boolean-independent.   So the conditions of 532I(iii) are satisfied
and $\kappa\in\MahcrR(\lambda)$, as claimed.
}%end of proof of 532R

\leader{532S}{Proposition} Suppose that $\add\Cal N_{\omega}>\omega_1$
and that $\lambda$ is an infinite cardinal such that
CTP$(\lambda^+,\lambda)$\cmmnt{ (definition:  5A6Fa)}
is true.   Then $\lambda\notin\MahcrR(\{0,1\}^{\kappa})$ for any
$\kappa>\lambda$.

\proof{ By 532K, it is enough to consider the case $\kappa=\lambda^+$.
\Quer\ Suppose, if possible, that there is a family
$\ofamily{\xi}{\kappa}{b_{\xi}}$ in $\frak B_{\lambda}$ satisfying the
conditions of 532I(iii).   Let
$\ofamily{\eta}{\lambda}{e_{\eta}}$ be the standard generating
family in $\frak B_{\lambda}$.   Then for
each $\xi<\kappa$ we have a countable set $I_{\xi}\subseteq\lambda$ such
that $b_{\xi}$ belongs to the closed subalgebra of $\frak B_{\lambda}$
generated by $\{e_{\eta}:\eta\in I_{\xi}\}$.   Because
CTP$(\kappa,\lambda)$ is true, there is an uncountable set
$A\subseteq\kappa$ such that $J=\bigcup_{\xi\in A}I_{\xi}$ is
countable (5A6F(b-ii)).   Now the closed subalgebra $\frak C_J$ generated
by $\{e_{\eta}:\eta\in J\}$ is isomorphic to $\frak B_{\omega}$, so
$\family{\xi}{A}{b_{\xi}}$ witnesses that
$\omega\in\MahcrR(\{0,1\}^{\omega_1})$;  but this contradicts 532Qa.\ \Bang
}%end of proof of 532S

\exercises{
\leader{532X}{Basic exercises (a)}
%\spheader 532Xa
Let $X$ be a normal Hausdorff space and $Y\subseteq X$ a
zero set.   Show that $\MahcrR(Y)\subseteq\MahcrR(X)$.
%532A

\spheader 532Xb Let $\beta\Bbb N$ be the Stone-\v{C}ech
compactification of $\Bbb N$.   (i) Show that
$\MahcrR(\beta\Bbb N)=\{0\}$.   \Hint{non-empty zero sets in 
$\beta\Bbb N\setminus\Bbb N$ are never ccc.}
(ii) Give an example of a non-empty
compact Hausdorff space $X$ such that $\MahcrR(X)=\emptyset$.
%532A

\spheader 532Xc Let $X$ and $Y$ be compact Hausdorff spaces.   Show that
$\MahcrR(X\times Y)\subseteq\MahcrR(X)\cup\MahcrR(Y)$.  
\Hint{454T.}
%532B

\spheader 532Xd Let $\lambda$ and $\kappa$ be infinite cardinals such that
$\lambda\in\MahcrR(\{0,1\}^{\kappa})$.   (i) Show that there is a
strictly positive \Mth\ completion regular Radon probability measure on
$\{0,1\}^{\kappa}$ with Maharam type $\lambda$.   (ii) Suppose that
$\lambda$ is uncountable and that $H\subseteq\{0,1\}^{\kappa}$ is a
non-empty G$_{\delta}$ set.   Show that $\lambda\in\MahcrR(H)$.
%532E

\spheader 532Xe Find a proof of 532E which does not rely on 532D.
\Hint{415E.}
%532E

\spheader 532Xf Let $\familyiI{(X_i,\mu_i)}$ be a family of
quasi-dyadic
spaces with strictly positive completion regular topological probability
measures.   Show that the ordinary product measure on
$\prod_{i\in I}X_i$ is a strictly positive completion regular
$\tau$-additive topological probability measure.
%532F
%each \mu_i \tau-additive 434Q
%so finite products \tau-add topological measures 532D
%so product topological measure \tau-add top measure 417Sc
%completion regular by 412Ub
%strictly positive 411Xk

\leader{532Y}{Further exercises (a)}
%\spheader 532Ya
Let $Z$ be the Stone space of $\frak B_{\lambda}$, where
$\lambda\ge\omega$.   (i) Show that if $F\subseteq Z$ is a
non-empty nowhere dense zero set then it is not ccc.   (ii) Show that
$\MahcrR(Z)=\{\lambda\}$.
(iii) Show that $\MahcrR(Z\times Z)=\emptyset$.
%532Xb 532Xc 532B mt53bits

\spheader 532Yb Let $\familyiI{X_i}$ be a family of topological spaces
with countable networks, and $Y$ any topological space.   Suppose that
we are given a strictly positive topological probability measure $\mu_i$
on each $X_i$, and a $\tau$-additive topological probability measure
$\nu$ on $Y$.   Show that the ordinary product measure on
$\prod_{i\in I}X_i\times Y$ is a topological measure.
%532E 
%reduce to case \supp\nu=Y , use 417Sc

\spheader 532Yc Suppose that $\FN(\Cal P\Bbb N)=\omega_1$.   Show that
there are a Hausdorff space $X$ and a completion regular Radon measure
$\mu$ on $X$ such that the Maharam type of $\mu$ is $\omega$, but the
Maharam type of $\mu\restr\Cal B(X)$ is $\omega_1$.   \Hint{419C.}
%532P
%\omega\in\MahcrR(\{0,1\}^{\omega_1\setminus\xi}) 532G 
% construct  X  as in 419C
% witnessed by \mu_{\xi} , \mu  direct sum of  \mu_{\xi}
% if we have open sets G_n\subseteq X, there is a cofinal closed set
% V  such that  \mu_{\xi}(G_n\cap X_{\xi})\in\{0,1\}  for every
% \xi\in V ;  so  \mu\restr\Cal B  has Maharam type > \omega
}%end of exercises

\leader{532Z}{Problems (a)} In 532P, can we take 
$\kappa=\cf\Cal N_{\omega}$?
%532P

\spheader 532Zb We have $\omega\in\MahcrR(\{0,1\}^{\omega_1})$ if
$\FN(\Cal P\Bbb N)=\omega_1$\cmmnt{ (532P, 532K)} and not if
$\add\Cal N_{\omega}>\omega_1$\cmmnt{ (532Q)}.   Can we narrow the
gap?
%532Q

\spheader 532Zc For a Hausdorff space $X$ let $\MahspcrR(X)$ be the set of
Maharam types of strictly positive Maharam homogeneous completion regular
Radon measures on $X$.
Describe the sets $\Gamma$ of cardinals for which there
are compact Hausdorff spaces $X$ such that $\MahspcrR(X)=\Gamma$.

\endnotes{
\Notesheader{532}
I have spent a good many pages on a rather specialized topic.   But I
think the patterns here are instructive.   When looking at $\MahR(X)$,
as in \S531, we quickly come to feel that it is a measure of a
certain kind of complexity;  the richer the space $X$, the larger
$\MahR(X)$ will be.   531Eb and 531Ed are direct manifestations of this,
and 531G develops the theme.   $\MahcrR(X)$ can sometimes tell us more
about $X$;  knowing $\MahcrR(X)$ we may have a lower bound on the
complexity
of $X$ as well as an upper bound.   (On the other hand, $\MahcrR(X)$ can
evaporate for non-trivial reasons, as in 532Xb and 532Ya, and leave us
with very
little idea of what $X$ might be like.)   In place of the
straightforward facts in 531E, we have the relatively complex and
partial results in 532G and 532K.   As soon as we leave the constrained
context of powers of $\{0,1\}$, the most natural questions seem to be
obscure (532Zc).

However, if we follow the paths which are open, rather than those we
might otherwise have chosen, we come to some interesting ideas, starting
with 532I.   Here, as happened in \S531, we see that a proper
understanding of the
measure algebras $\frak B_{\lambda}$ will take us a long way;  and once
again we find that this understanding has to be conditional on the model
of set theory we are working in.   Even to decide which powers of
$\{0,1\}$ carry completion regular Radon measures with countable Maharam
type we need to examine some new aspects of the Lebesgue measure algebra
(532M-532O).  %532M 532N 532O
Moreover, as well as the familiar cardinals of Cicho\'n's diagram, we
have to look at the Freese-Nation number of $\Cal P\Bbb N$ (532P).
For larger Maharam types, in a way that we are becoming accustomed to,
other combinatorial principles become relevant (532R, 532S).
}%end of notes

\discrpage

