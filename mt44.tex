\frfilename{mt44.tex}
\versiondate{18.4.08}
\copyrightdate{1998}

\def\chaptername{Topological groups}
\def\sectionname{Introduction}

\newchapter{44}

Measure theory begins on the real line, which is of course a group;  and
one of the most fundamental properties of Lebesgue measure is its
translation-invariance\cmmnt{ (134A)}.   Later we come to the standard
measure on the unit circle\cmmnt{ (255M)}, and counting measure on the
integers is also translation-invariant, if we care to notice;  moreover,
Fourier series and transforms clearly depend utterly
on the fact that shift operators don't disturb the measure-theoretic
structures we are building.   Yet another example appears in the usual
measure on $\{0,1\}^I$, which is translation-invariant if we identify
$\{0,1\}^I$ with the group $\Bbb Z_2^I$\cmmnt{ (345Ab)}.   Each of
these examples is special in many other ways.   But it turns out that
a particular combination of properties which they share,
all being locally
compact Hausdorff spaces with group operations for which multiplication
and inversion are continuous, is the basis of an extraordinarily
powerful theory of invariant measures.

As usual, I have no choice but to move rather briskly through a wealth
of ideas.   The first step is to set out a suitably general existence
theorem, assuring us that every locally compact Hausdorff topological
group has non-trivial invariant Radon measures, that is, `Haar
measures'\cmmnt{ (441E)}.   As remarkable as the existence of
Haar measures is their (essential) uniqueness\cmmnt{ (442B)};  the
algebra, topology and measure theory of a topological group are linked
in so many ways that they form a peculiarly solid structure.
I investigate a miscellany of facts about this structure in \S443,
including the basic theory of the modular functions linking left-invariant
measures with right-invariant measures.

I have already mentioned that Fourier analysis depends on the
translation-invariance of Lebesgue measure.   It turns out that
substantial parts of the abstract theory of Fourier series and
transforms can be generalized to arbitrary locally compact groups.   In
particular, convolutions\cmmnt{ (\S255)} appear again, even in
non-abelian groups\cmmnt{ (\S444)}.   But for the central part of the
theory, a transform relating functions on a group $X$ to functions on
its `dual' group $\Cal X$, we do need the group to be abelian.
Actually I give only the foundation of this theory:  if $X$ is an
abelian locally compact Hausdorff group, it is the dual of its
dual\cmmnt{ (445U)}.   (In `ordinary' Fourier theory, where we are
dealing with the cases $X=\Cal X=\Bbb R$ and $X=S^1$, $\Cal X=\Bbb Z$,
this duality is so straightforward that one hardly notices it.)   But on
the way to the duality theorem we necessarily see many of the themes of
Chapter 28 in more abstract guises.

A further remarkable
fact is that any Haar measure has a translation-invariant
lifting\cmmnt{ (447J)}.   The proof demands a union between the ideas
of the ordinary Lifting Theorem\cmmnt{ (\S341)} and some of the
elaborate structure theory which has been developed for locally compact
groups\cmmnt{ (\S446)}.

For the last two sections of the chapter, I look at groups which are not
locally compact, and their actions on appropriate spaces.
For a particularly important class of group actions, Borel measurable
actions of Polish groups on Polish spaces,
we have a natural necessary and
sufficient condition for the existence of an invariant
measure\cmmnt{ (448P)}, complementing the result for locally compact
spaces in 441C.
In a slightly different direction, we can look at those groups, the
`amenable' groups, for which all actions (on compact Hausdorff spaces)
have invariant measures.   This again leads to some very remarkable
ideas, which I sketch in \S449.

\discrpage


