\frfilename{mt411.tex}
\versiondate{31.12.08}
\copyrightdate{2002}

\def\chaptername{Topologies and measures I}
\def\sectionname{Definitions}

\newsection{411}

In something of the spirit of \S211, but this time without apologising,
I start this volume with a list of definitions.   The rest of Chapter 41
will be devoted to discussing these definitions and relationships
between them, and integrating the new ideas into the concepts and
constructions
of earlier volumes;  I hope that by presenting the terminology now
I can give you a sense of the directions the following sections will
take.   I ought to remark immediately that there are many cases in which
the exact phrasing of a definition is important in ways which may
not be immediately apparent.

\leader{411A}{}\cmmnt{ I begin with a phrase which will be a useful
shorthand for the context in which most, but not all, of the theory
here will be developed.

\medskip

\noindent}{\bf Definition} A {\bf topological measure space} is a
quadruple $(X,\frak T,\Sigma,\mu)$ where $(X,\Sigma,\mu)$ is a measure
space and $\frak T$ is a topology on $X$ such that
$\frak T\subseteq\Sigma$\cmmnt{, that is, every open set (and
therefore every Borel set) is measurable}.

\leader{411B}{}\cmmnt{ Now I come to what are in my view the two most
important concepts to master;  jointly they will dominate the chapter.

\medskip

\noindent}{\bf Definition} Let $(X,\Sigma,\mu)$ be a measure space and
$\Cal K$ a family of sets.   I say that $\mu$ is {\bf inner
regular with respect to $\Cal K$} if

\Centerline{$\mu E=\sup\{\mu K:K\in\Sigma\cap\Cal K,\,K\subseteq E\}$}

\noindent for every $E\in\Sigma$.   \cmmnt{(Cf.\ 256Ac, 342Aa.)}

\cmmnt{\medskip

\noindent{\bf Remark} Note that in this definition I do not assume that
$\Cal K\subseteq\Sigma$, nor even that $\Cal K\subseteq\Cal PX$.   But
of course $\mu$ will be inner regular with
respect to $\Cal K$ iff it is inner regular with respect to
$\Cal K\cap\Sigma$.

It is convenient in this context to interpret $\sup\emptyset$ as $0$, so
that we have to check the definition only when $\mu E>0$, and need not
insist that $\emptyset\in\Cal K$.
}%end of comment

\leader{411C}{Definition} Let $(X,\Sigma,\mu)$ be a measure space and
$\frak T$ a topology on $X$.   I say that $\mu$ is
{\bf $\tau$-additive} (the phrase {\bf $\tau$-regular} has also been used)
if whenever $\Cal G$ is a non-empty upwards-directed family of open sets
such that $\Cal G\subseteq\Sigma$ and $\bigcup\Cal G\in\Sigma$ then
$\mu(\bigcup\Cal G)=\sup_{G\in\Cal G}\mu G$.

\cmmnt{\medskip

\noindent{\bf Remark} Note that in this definition I do not assume that
every open set is measurable.   Consequently we cannot take it for
granted that an extension of a $\tau$-additive measure will be
$\tau$-additive;  on
the other hand, the restriction of a $\tau$-additive measure to any
$\sigma$-subalgebra will be $\tau$-additive.

}%end of comment

\leader{411D}{}\cmmnt{ Complementary to 411B we have the following.

\medskip

\noindent}{\bf Definition} Let $(X,\Sigma,\mu)$ be a measure space and
$\Cal H$ a family of subsets of $X$.   Then $\mu$ is
{\bf outer regular with respect to $\Cal H$} if

\Centerline{$\mu E=\inf\{\mu H:H\in\Sigma\cap\Cal H,\,H\supseteq E\}$}

\noindent for every $E\in\Sigma$.

Note that a totally finite measure on a topological space is
inner regular with respect to the family of
closed sets iff it is outer regular with
respect to the family of open sets.

\leader{411E}{}\cmmnt{ I delay discussion of most of the relationships
between the concepts here to later in the chapter.   But it will be
useful to have a basic fact set out immediately.

\medskip

\noindent}{\bf Proposition} Let $(X,\Sigma,\mu)$ be a measure space and
$\frak T$ a topology on $X$.   If $\mu$ is inner regular with respect to
the compact sets, it is $\tau$-additive.

\proof{ Let $\Cal G$ be a non-empty upwards-directed family of
measurable open sets such that $H=\bigcup\Cal G\in\Sigma$.   If
$\gamma<\mu H$, there is a compact set $K\subseteq H$ such that
$\mu K\ge\gamma$;  now there must be a $G\in\Cal G$ which includes $K$,
so that $\mu G\ge\gamma$.  As $\gamma$ is arbitrary,
$\sup_{G\in\Cal G}\mu G=\mu H$.
}%end of proof of 411E

\leader{411F}{}\cmmnt{ In order to deal efficiently with measures
which are not totally finite, I think we need the following ideas.

\medskip

\noindent}{\bf Definitions} Let $(X,\Sigma,\mu)$ be a measure space
and $\frak T$ a topology on $X$.

\spheader 411Fa I say that $\mu$ is {\bf locally finite} if every point
of $X$ has a neighbourhood of finite measure\cmmnt{, that is, if the
open sets of finite outer measure cover $X$}.

\spheader 411Fb I say that $\mu$ is {\bf effectively locally finite} if
for every non-negligible measurable set $E\subseteq X$ there is
a measurable open set $G\subseteq X$ such that $\mu G<\infty$ and
$E\cap G$ is not negligible.

\cmmnt{Note that an effectively locally finite measure
must measure many open sets, while a locally finite measure need not.
}%end of comment

\spheader 411Fc \cmmnt{This seems a convenient moment at which to
introduce the following term.}   A real-valued function $f$ defined on a
subset of $X$ is {\bf locally integrable} if for every $x\in X$ there
is an open set $G$ containing $x$ such that $\int_Gf$ is
defined\cmmnt{ (in the sense of 214D)} and finite.

\leader{411G}{Elementary facts (a)}
%\spheader 411Ga
If $\mu$ is a locally finite measure on a topological space $X$, then
$\mu^*K<\infty$ for every compact set $K\subseteq X$.   \prooflet{\Prf\
The family $\Cal G$ of open sets of finite outer measure is
upwards-directed and covers $X$,
so there must be some $G\in\Cal G$ including
$K$, in which case $\mu^*K\le\mu^*G$ is finite.\ \Qed}

\spheader 411Gb A measure $\mu$ on $\BbbR^r$ is locally finite
iff every bounded set has finite outer measure\cmmnt{ (cf.\ 256Ab)}.
\prooflet{\Prf\ (i) If every bounded set has finite outer measure then,
in particular, every open ball has finite outer measure, so that $\mu$
is locally finite.   (ii) If $\mu$ is locally finite and
$A\subseteq\Bbb R^r$ is bounded, then its closure $\overline{A}$ is
compact (2A2F), so
that $\mu^*A\le\mu^*\overline{A}$ is finite, by (a) above.\ \Qed}

\cmmnt{\spheader 411Gc I should perhaps remark immediately that a
locally finite topological measure need not be effectively locally
finite (419A), and an effectively locally finite measure need not be
locally finite (411P).
}%end of comment

\spheader 411Gd An effectively locally finite measure must be
semi-finite.

\spheader 411Ge A locally finite measure on a Lindel\"of space
$X$ is $\sigma$-finite.
\prooflet{\Prf\ Let $\Cal G$ be the family of open sets of finite
outer measure.   Because $\mu$ is locally finite, $\Cal G$ is a cover
of $X$.   Because $X$
is Lindel\"of, there is a sequence $\sequencen{G_n}$ in $\Cal G$
covering $X$.   For each $n\in\Bbb N$, there is a measurable set
$E_n\supseteq G_n$
of finite measure, and now $\sequencen{E_n}$ is a sequence of sets of
finite measure covering $X$.\ \Qed}

\spheader 411Gf Let $(X,\frak T,\Sigma,\mu)$ be a topological measure
space such that $\mu$ is locally
finite and inner regular with respect to the compact sets.   Then
$\mu$ is effectively locally finite.   \prooflet{\Prf\ Suppose that
$\mu E>0$.   Then there is a measurable compact set $K\subseteq E$ such
that $\mu K>0$.   As in the argument for (a) above, there is an open set
$G$ of finite measure including $K$, so that $\mu(E\cap G)>0$.\ \Qed}

\spheader 411Gg \cmmnt{Corresponding to (a) above, we have the
following fact.}   If $\mu$ is a measure on a topological space and
$f\in\eusm L^0(\mu)$ is locally
integrable, then $\int_Kfd\mu$ is finite for every compact
$K\subseteq X$\prooflet{, because $K$ can be covered by a finite family
of open sets $G$ such that $\int_G|f|d\mu<\infty$}.

\spheader 411Gh If $\mu$ is a locally finite measure on
a topological space $X$, and $f\in\eusm L^p(\mu)$ for some
$p\in[1,\infty]$, then $f$ is locally integrable\cmmnt{;  this is
because $\int_G|f|\le\int_E|f|\le\|f\|_p\|\chi E\|_q$ is finite whenever
$G\subseteq E$ and $\mu E<\infty$, where $\bover1p+\bover1q=1$, by
H\"older's inequality (244Eb)}.

\spheader 411Gi If $(X,\frak T)$ is a completely regular space and $\mu$
is a locally finite topological measure on $X$, then the collection of open
sets with negligible boundaries is a base for $\frak T$.
\prooflet{\Prf\ If $x\in G\in\frak T$, let $H\subseteq G$ be an open set
of finite measure containing $x$, and $f:X\to[0,1]$ a continuous
function such that $f(x)=1$ and $f(y)=0$ for $y\in X\setminus H$.   Then
$\{f^{-1}[\{\alpha\}]:0<\alpha<1\}$ is an uncountable disjoint family of
measurable subsets of $H$, so there must be some $\alpha\in\ooint{0,1}$
such that $f^{-1}[\{\alpha\}]$ is negligible.   Set
$U=\{y:f(y)>\alpha\}$;  then $U$ is an open neighbourhood of $x$
included in $G$ and its boundary
$\partial U\subseteq f^{-1}[\{\alpha\}]$ is negligible.\ \Qed}

\leader{411H}{}\cmmnt{ Two particularly important combinations of the
properties above are the following.

\medskip

\noindent}{\bf Definitions (a)} A {\bf quasi-Radon measure space} is a
topological measure space $(X,\frak T,\Sigma,\mu)$ such that (i)
$(X,\Sigma,\mu)$ is complete and locally determined (ii) $\mu$ is
$\tau$-additive, inner regular with respect to the closed sets and
effectively locally finite.

\spheader 411Hb  A {\bf Radon measure space} is a
topological measure space $(X,\frak T,\Sigma,\mu)$ such that (i)
$(X,\Sigma,\mu)$ is
complete and locally determined (ii) $\frak T$ is Hausdorff (iii) $\mu$
is locally finite and inner regular with respect to the compact sets.

\leader{411I}{Remarks}\cmmnt{ {\bf (a)} You may like to seek your own
proof that a Radon measure space is always quasi-Radon, before looking
it up in \S416 below.

\medskip

{\bf (b)}} Note that a measure on Euclidean space $\BbbR^r$ is a
Radon measure on the definition above iff it is a Radon measure as
described in 256Ad.   \prooflet{\Prf\ In 256Ad, I said that a measure
$\mu$ on $\BbbR^r$ is `Radon' if it is a locally finite complete
topological
measure, inner regular with respect to the compact sets.   (The
definition of `locally
finite' in 256A was not the same as the one above, but I have already
covered this point in 411Gb.)   So the only thing to add is that $\mu$
is necessarily locally determined, because it is $\sigma$-finite
(256Ba).\ \Qed
}%end of prooflet

\leader{411J}{}\cmmnt{ The following special types of inner regularity
are of sufficient importance to have earned separate names.

\medskip

\noindent}{\bf Definitions (a)} If $(X,\frak T)$ is a topological space,
I will say that a measure $\mu$ on $X$ is {\bf tight} if it is inner
regular with respect to the closed compact sets.

\spheader 411Jb If $(X,\frak T,\Sigma,\mu)$ is a topological
measure space, I will say that $\mu$ is {\bf completion regular} if it
is inner regular with respect to the zero sets\cmmnt{ (definition:
3A3Qa)}.

\leader{411K}{Borel and Baire measures}
If $(X,\frak T)$ is a topological space, I will call a measure with
domain\cmmnt{ (exactly)} the Borel $\sigma$-algebra of
$X$\cmmnt{ (4A3A)} a {\bf Borel measure}
on $X$, and a measure with domain\cmmnt{ (exactly)} the Baire
$\sigma$-algebra of $X$\cmmnt{ (4A3K)} a {\bf Baire measure} on $X$.

\cmmnt{Of course a Borel measure is a topological measure in the
sense of 411A.   On a metrizable space,
the Borel and Baire measures coincide (4A3Kb).
The most important measures in this chapter will be c.l.d.\ versions
of Borel measures.}

\leader{411L}{}\cmmnt{ When we come to look at functions defined on
a topological measure space, we shall have to relate ideas of continuity
and measurability.   Two basic concepts are the following.

\medskip

\noindent}{\bf Definition} Let $X$ be a set, $\Sigma$ a $\sigma$-algebra
of subsets of $X$ and $(Y,\frak S)$ a topological space.   I will say
that a function $f:X\to Y$ is {\bf measurable} if $f^{-1}[G]\in\Sigma$
for every open set $G\subseteq Y$.

\cmmnt{\medskip

\noindent{\bf Remarks (a)} Note that a function $f:X\to\Bbb R$ is
measurable on this definition (when $\Bbb R$ is given its usual
topology) iff it is measurable according to the familiar definition in
121C, which
asks only that sets of the form $\{x:f(x)<\alpha\}$ should be measurable
(121Ef).

\medskip

{\bf (b)} For any topological space $(Y,\frak S)$, a function $f:X\to Y$
is measurable iff $f$ is $(\Sigma,\Cal B(Y))$-measurable,
where $\Cal B(Y)$ is the Borel $\sigma$-algebra of $Y$ (4A3Cb).
}%end of comment

\leader{411M}{Definition} Let $(X,\Sigma,\mu)$ be a measure space,
$\frak T$ a topology on $X$, and $(Y,\frak S)$ another topological
space.   I will say that a function $f:X\to Y$ is
{\bf almost continuous}\cmmnt{ or {\bf Lusin measurable}} if $\mu$
is inner regular with respect to the
family of subsets $A$ of $X$ such that $f\restr A$ is continuous.

\leader{411N}{}\cmmnt{ Finally, I introduce some terminology to
describe ways in which (sometimes) measures can be located in one part
of a topological space rather than another.

\medskip

\noindent}{\bf Definitions} Let $(X,\Sigma,\mu)$ be a measure space
and $\frak T$ a topology on $X$.

\spheader 411Na I will call a set $A\subseteq X$ {\bf self-supporting}
if $\mu^*(A\cap G)>0$ for every open set $G$ such that $A\cap G$ is
non-empty.   \cmmnt{(Such sets are sometimes called {\bf of positive
measure everywhere}.)}

\spheader 411Nb A {\bf support} of $\mu$ is a closed self-supporting
set $F$ such that $X\setminus F$ is negligible.

\spheader 411Nc\cmmnt{ Note that} $\mu$ can have at most one support.
\prooflet{\Prf\ If $F_1$, $F_2$ are supports then
$\mu^*(F_1\setminus F_2)\le\mu^*(X\setminus F_2)=0$ so $F_1\setminus F_2$
must be empty.
Similarly, $F_2\setminus F_1=\emptyset$, so $F_1=F_2$.\ \Qed}

\spheader 411Nd If $\mu$ is a $\tau$-additive topological measure it
has a support.   \prooflet{\Prf\ Let $\Cal G$ be the family
of negligible open sets, and $F$ the closed set
$X\setminus\bigcup\Cal G$.   Then $\Cal G$ is an upwards-directed family
in $\frak T\cap\Sigma$ and $\bigcup\Cal G\in\frak T\cap\Sigma$, so

\Centerline{$\mu(X\setminus F)=\mu(\bigcup\Cal G)
=\sup_{G\in\Cal G}\mu G=0$.}

\noindent If $G$ is open and $G\cap F\ne\emptyset$ then $G\notin\Cal G$
so $\mu^*(G\cap F)=\mu(G\cap F)=\mu G>0$;  thus $F$ is self-supporting
and is the support of $\mu$.\ \Qed}

\spheader 411Ne Let $X$ and $Y$ be topological spaces with topological
measures $\mu$, $\nu$ respectively and a continuous \imp\ function
$f:X\to Y$.   Suppose that $\mu$ has a support $E$.   Then
$\overline{f[E]}$ is the support of $\nu$.   \prooflet{\Prf\ We have
only to observe that for an open set $H\subseteq Y$

$$\eqalign{\nu H>0
&\iff\mu f^{-1}[H]>0
\iff f^{-1}[H]\cap E\ne\emptyset\cr
&\iff H\cap f[E]\ne\emptyset
\iff H\cap\overline{f[E]}\ne\emptyset. \text{ \Qed}\cr}$$}

\spheader 411Nf $\mu$ is {\bf strictly positive}\cmmnt{ (with
respect
to $\frak T$)} if $\mu^*G>0$ for every non-empty open set
$G\subseteq X$\cmmnt{, that is, $X$ itself is the support of $\mu$}.

\header{*411Ng}{\bf *(g)} If $(X,\frak T)$ is a topological space, and
$\mu$ is a strictly positive $\sigma$-finite measure on $X$ such that
the domain\cmmnt{ $\Sigma$} of $\mu$ includes a
$\pi$-base\cmmnt{ $\Cal U$} for $\frak T$, then $X$ is ccc.
\prooflet{\Prf\ Let $\sequencen{E_n}$
be a sequence of sets of finite measure covering $X$.   Let $\Cal G$ be
a disjoint family of non-empty open sets.   For each $G\in G$, take
$U_G\in\Cal U\setminus\{\emptyset\}$ such that $U_G\subseteq G$;  then
$\mu U_G>0$, so there is an $n(G)$ such that $\mu(E_{n(G)}\cap U_G)>0$.
Now $\sum_{G\in G,n(G)=k}\mu(E_k\cap U_G)\le\mu E_k$ is finite for every
$k$, so $\{G:n(G)=k\}$ must be countable and $\Cal G$ is countable.\
\Qed}

\leader{411O}{Example} Lebesgue measure on $\BbbR^r$ is a Radon
measure\cmmnt{ (256Ha)};  \cmmnt{in particular,} it is locally
finite and tight.   It is\cmmnt{ therefore} $\tau$-additive and
effectively locally finite\cmmnt{ (411E, 411Gf)}.   It is completion
regular\cmmnt{ (because every compact set is a zero set, see
4A2Lc)}, outer regular with respect to the open
sets\cmmnt{ (134Fa)} and strictly positive.

\leader{411P}{Example:  Stone spaces (a)} Let $(Z,\frak T,\Sigma,\mu)$
be the Stone space of a semi-finite measure algebra
$(\frak A,\bar\mu)$, so that $(Z,\frak T)$ is a
zero-dimensional compact Hausdorff space, $(Z,\Sigma,\mu)$ is complete
and semi-finite, the open-and-closed sets are measurable, the negligible
sets are the nowhere dense sets, and every measurable set differs by a
nowhere dense set from an open-and-closed set\cmmnt{ (311I, 321K,
322Bd, 322Ra\formerly{3{}22Qa})}.
%311I:  X compact 0-dim
%321K:  measurable iff (open&closed\symmdiff meager),
%       negligible iff meager, \mu complete
%322Bd:  semi-finite
%322Ra:  meager iff nwd

\spheader 411Pb $\mu$ is inner regular with respect to the
open-and-closed sets\cmmnt{ (322Ra)};  \cmmnt{in particular,} it
is completion regular and tight.   Consequently it is
$\tau$-additive\cmmnt{ (411E)}.

\spheader 411Pc $\mu$ is strictly positive\cmmnt{, because the
open-and-closed sets form a base for $\frak T$ (311I) and a non-empty
open-and-closed set has non-zero measure}.   $\mu$ is effectively
locally finite.   \prooflet{\Prf\ Suppose that $E\in\Sigma$ is not
negligible.
There is a measurable set $F\subseteq E$ such that $0<\mu F<\infty$;
now there is a non-empty open-and-closed
set $G$ included in $F$, in which case $\mu G<\infty$ and
$\mu(E\cap G)>0$.\ \Qed}

\wheader{411Pd}{0}{0}{0}{60pt}

\spheader 411Pd The following are equiveridical, that is, if one is true
so are the others:

\quad (i) $(\frak A,\bar\mu)$ is localizable;

\quad(ii) $\mu$ is strictly localizable;

\quad(iii) $\mu$ is locally determined;

\quad(iv) $\mu$ is a quasi-Radon measure.

\prooflet{\noindent\Prf\ The equivalence of (i)-(iii) is Theorem
322O\formerly{3{}22N}.
(iv)$\Rightarrow$(iii) is trivial.   If one, therefore all, of (i)-(iii)
are true, then $\mu$ is a topological measure, because if
$G\subseteq Z$ is open, then $\overline{G}$ is open-and-closed, by 314S,
therefore measurable, and $\overline{G}\setminus G$ is nowhere dense,
therefore also measurable.   We know already that $\mu$ is complete,
effectively locally finite and $\tau$-additive, so that if it is also
locally determined it is a quasi-Radon measure.\ \Qed}

\spheader 411Pe The following are equiveridical:

\quad(i) $\mu$ is a Radon measure;

\quad(ii) $\mu$ is totally finite;

\quad (iii) $\mu$ is locally finite;

\quad (iv) $\mu$ is outer regular with respect to the open sets.

\prooflet{\noindent\Prf\ (ii)$\Rightarrow$(iv) If $\mu$ is totally
finite and $E\in\Sigma$, then for any $\epsilon>0$ there is a closed
set $F\subseteq Z\setminus E$ such that
$\mu F\ge\mu(Z\setminus E)-\epsilon$, and now
$G=Z\setminus F$ is an open set including $E$ with
$\mu G\le\mu E+\epsilon$.   (iv)$\Rightarrow$(iii) Suppose that $\mu$
is outer regular with respect to the
open sets, and $z\in Z$.   Because $Z$ is Hausdorff, $\{z\}$ is
closed.   If it is open it is
measurable, and because $\mu$ is semi-finite it must have finite
measure.   Otherwise it is nowhere dense, therefore negligible, and
must be included
in open sets of arbitrarily small measure.   Thus in both cases $z$
belongs to an open set of finite measure;  as $z$ is arbitrary, $\mu$
is locally finite.  (iii)$\Rightarrow$(ii) Because $Z$ is compact, this
is a consequence of 411Ga.   (i)$\Rightarrow$(iii)
is part of the definition of `Radon measure'.   Finally,
(ii)$\Rightarrow$(i), again directly from the definition and the
facts set out in (a)-(b) above.\ \Qed}

\spheader 411Pf Let $W\subseteq Z$ be the union of all the open
subsets of $Z$ with finite measure.   \cmmnt{Because $\mu$ is
effectively locally finite,} $W$ has full outer measure, so
$(\frak A,\bar\mu)$ can be identified with the measure algebra of the
subspace measure $\mu_W$\cmmnt{ (322Jb)}.   \cmmnt{By the
definition of $W$,} $\mu_W$ is locally finite.   If
$(\frak A,\bar\mu)$ is localizable, then $\mu_W$ is a Radon measure.
\prooflet{\Prf\ Every open subset of $W$ belongs to $\Sigma$, by (d),
and therefore to the domain of $\mu_W$, and $\mu_W$ is a topological
measure.   By 214Ka, $\mu_W$ is complete and locally determined.
Because $\mu$ is inner regular with respect to the compact sets, so
is $\mu_W$.\ \Qed}


\leader{411Q}{Example:  Dieudonn\'e's measure} Recall that a set
$E\subseteq\omega_1$ is a Borel set iff either $E$ or its complement
includes a cofinal closed set\cmmnt{ (4A3J)}.   So we may define
a Borel measure $\mu$ on $\omega_1$ by saying that $\mu E=1$ if $E$
includes a cofinal closed set and $\mu E=0$ if $E$ is disjoint from a
cofinal closed set.   \cmmnt{If $E$ is disjoint from some
cofinal closed set, so is any subset of $E$, so} $\mu$ is complete.
\cmmnt{Since} $\mu$\cmmnt{ takes only the values $0$ and $1$, it}
is a purely atomic probability measure.

$\mu$ is a topological measure;  \cmmnt{being totally finite,} it
is\cmmnt{ surely} locally finite and effectively locally finite.
It is inner regular with respect to
the closed sets\cmmnt{ (because if $\mu E>0$, there is a cofinal
closed set $F\subseteq E$, and now $F$ is a closed set with
$\mu F=\mu E$)}, therefore outer regular with respect to the open sets.
It is not $\tau$-additive\cmmnt{ (because $\xi=\coint{0,\xi}$ is an open
set of zero measure for every $\xi<\omega_1$, and the union of these
sets is a measurable open set of measure $1$)}.

$\mu$ is not completion regular\cmmnt{, because the set of countable
limit ordinals is a closed set (4A1Bb) which does not include
any uncountable zero set (see 411R below)}.

The only self-supporting subset of $\omega_1$ is the empty
set\cmmnt{ (because there is a cover of $\omega_1$ by negligible
open sets)}.   \cmmnt{In particular,} $\mu$ does not have a support.

\cmmnt{\medskip

\noindent{\bf Remark} There is a measure of this type on any ordinal
of uncountable cofinality;  see 411Xj.
}%end of comment

\leader{411R}{Example:  The Baire $\sigma$-algebra of $\omega_1$}
The Baire $\sigma$-algebra\cmmnt{ $\CalBa(\omega_1)$} of $\omega_1$ is the
countable-cocountable algebra\cmmnt{ (4A3P)}.   The
countable-cocountable measure $\mu$ on $\omega_1$
is\cmmnt{ therefore} a Baire measure\cmmnt{ on the definition of
411K}.   \cmmnt{Since all sets of
the form $\ooint{\xi,\omega_1}$ are zero sets,} $\mu$ is inner regular
with respect to the zero sets and outer regular with respect to the
cozero sets.   \cmmnt{Since sets of
the form $\coint{0,\xi}$ ($=\xi$) form a cover of $\omega_1$ by
measurable open sets of zero measure,} $\mu$ is not $\tau$-additive.

\exercises{
\leader{411X}{Basic exercises $\pmb{>}$(a)}
%\spheader 411Xa
Let $(X,\Sigma,\mu)$ be a totally finite measure space and $\frak T$ a
topology on $X$.   Show that $\mu$ is inner regular with respect to the
closed sets iff it is outer regular with respect to the open sets, and
is inner regular with respect to the zero sets iff it is outer regular
with respect to the cozero sets.
%411D used in 457M

\spheader 411Xb Let $\mu$ be a Radon measure on $\BbbR^r$, where
$r\ge 1$, and $f\in\eusm L^0(\mu)$.   Show that $f$ is locally
integrable in the sense of 411Fc iff it is locally integrable in the
sense of 256E, that is, $\int_Efd\nu<\infty$ for every bounded set
$E\subseteq\BbbR^r$.
%411F

\spheader 411Xc Let $\mu$ be a measure on a topological space, $\hat\mu$
its completion and $\tilde\mu$ its c.l.d.\ version.   Show that $\mu$
is locally finite iff $\hat\mu$ is locally finite, and in this case
$\tilde\mu$ is locally finite.
%411G

\sqheader 411Xd Let $\mu$ be an effectively locally finite measure on
a topological space $X$.   (i) Show that the completion and c.l.d.\
version of $\mu$ are effectively locally finite.   (ii) Show that if
$\mu$ is complete and locally determined, then the union of the
measurable open sets of finite measure is conegligible.   (iii) Show
that if $X$ is hereditarily Lindel\"of then $\mu$ must be
$\sigma$-finite.
%411G

\spheader 411Xe Let $X$ be a topological space and $\mu$ a measure on
$X$.   Let $U\subseteq L^0(\mu)$ be the set of equivalence classes of
locally integrable functions in $\eusm L^0(\mu)$.   Show that $U$ is a
solid linear subspace of $L^0(\mu)$.   Show that if $\mu$ is locally
finite then $U$ includes $L^p(\mu)$ for every $p\in[1,\infty]$.
%411G

\spheader 411Xf Let $X$ be a topological space.   (i) Let $\mu$, $\nu$
be two totally finite Borel measures which agree on the closed sets.
Show that they are equal.   \Hint{136C.}   (ii) Let $\mu$, $\nu$ be
two totally finite Baire measures which agree on the zero sets.   Show
that they are equal.
%411K

\spheader 411Xg Let $(X,\frak T)$ be a topological space, $\mu$ a
measure on $X$, and $Y$ a subset of $X$;  let $\frak T_Y$, $\mu_Y$ be
the subspace
topology and measure.    Show that if $\mu$ is a topological measure,
or locally finite, or a Borel measure, so is $\mu_Y$.

\spheader 411Xh Let $\langle (X_i,\Sigma_i,\mu_i)\rangle_{i\in I}$ be
a family of measure spaces, with direct sum $(X,\Sigma,\mu)$;  suppose
that we are given a topology $\frak T_i$ on each $X_i$, and let
$\frak T$ be the disjoint union topology on $X$ (definition:  4A2A).
Show that $\mu$ is a topological
measure, or locally finite, or effectively locally finite, or a Borel
measure, or a Baire measure, or strictly positive, iff every $\mu_i$ is.

\spheader 411Xi Let $(X,\Sigma,\mu)$ and $(Y,\Tau,\nu)$ be two measure
spaces, with c.l.d.\ product measure $\lambda$ on $X\times Y$.   Suppose
we are given topologies $\frak T$, $\frak S$ on $X$, $Y$ respectively,
and give $X\times Y$ the product topology.   Show that $\lambda$ is
locally finite, or effectively locally finite, if $\mu$ and $\nu$ are.

\spheader 411Xj Let $\alpha$ be any ordinal of uncountable cofinality
with its order topology (definitions:  3A1Fb, 4A2A).   Show that there is a complete topological
probability measure $\mu$ on $\alpha$ defined by saying that $\mu E=1$
if $E$ includes a cofinal closed set in $\alpha$, $0$ if $E$ is
disjoint from some
cofinal closed set.   Show that $\mu$ is inner regular with respect to
the closed sets but is not completion regular.
%411Q

\spheader 411Xk\dvAnew{2013} Let $\familyiI{(X_i,\frak T_i)}$ be a family
of topological spaces, and $\mu_i$ a strictly positive probability measure
on $X_i$ for each $i$.   Show that the product measure on 
$\prod_{i\in I}X_i$ is strictly positive.
%411N out of order query

\leader{411Y}{Further exercises (a)}
%\spheader 411Ya
Let $r$, $s\ge 1$ be integers.
Show that a function $f:\BbbR^r\to\BbbR^s$ is measurable
iff it is almost continuous (where $\BbbR^r$ is endowed with Lebesgue
measure and its usual topology, of course).   \Hint{256F.}
%411M

\spheader 411Yb Let $(X,\rho)$ be a metric space, $r\ge 0$, and write
$\mu_{Hr}$ for $r$-dimensional Hausdorff measure on $X$ (264K,
471A).   (i) Show that $\mu_{Hr}$ is a topological measure, outer
regular with respect to the
Borel sets.  (ii) Show that the c.l.d.\ version $\tilde\mu_{Hr}$ of
$\mu_{Hr}$ is inner regular with respect to the closed totally bounded
sets.   (iii) Show that $\tilde\mu_{Hr}$ is completion regular.
(iv) Show that if $X$ is complete then $\tilde\mu_{Hr}$ is tight.
%411J

\spheader 411Yc Let $(X,\frak T,\Sigma,\mu)$ be a topological measure
space.   Set $\Cal E=\{E:E\subseteq X,\,\mu(\partial E)=0\}$, where
$\partial E$ is the boundary of $A$.   (i) Show that $\Cal E$ is a
subalgebra of $\Cal PX$, and that every member of $\Cal E$ is measured
by the completion of $\mu$.   ($\Cal E$ is sometimes called the {\bf
Jordan algebra} of $(X,\frak T,\Sigma,\mu)$.   Do not confuse with the
`Jordan algebras' of abstract algebra.)   (ii) Suppose that $\mu$
is totally finite and inner regular with respect to the
closed sets, and that $\frak T$ is normal.   Show that
$\{E^{\ssbullet}:E\in\Cal E\cap\Sigma\}$ is dense in the measure algebra of
$\mu$ endowed with its usual topology.
(iii) Suppose that $\mu$ is a quasi-Radon measure
and $\frak T$ is completely regular.
Show that $\{E^{\ssbullet}:E\in\Cal E\}$ is dense in the measure
algebra of $\mu$.   \Hint{414Aa.}
%411G 411H

\spheader 411Yd\dvAnew{2011}
Let $(X,\frak T,\Sigma,\mu)$ be a second-countable
atomless topological probability space with a strictly positive measure, 
$\Cal E$ the
Jordan algebra of $\mu$ as defined in 411Yc, $(\frak A,\bar\mu)$ the
measure algebra of $\mu$ and $\frak E$ the image
$\{E^{\ssbullet}:E\in\Cal E\}\subseteq\frak A$.   Let $\frak B$ be a
Boolean algebra and $\nu:\frak B\to[0,1]$ a finitely additive functional.
Show that $(\frak B,\nu)\cong(\frak E,\bar\mu\restrp\frak E)$ iff
($\alpha$) 
$\nu$ is strictly positive and properly atomless in the sense of
326F\formerly{3{}26Ya}, and $\nu 1=1$ 
($\beta$) there is a countable subalgebra $\frak B_0$ of $\frak B$
such that $\nu b=\sup\{\nu c:c\in\frak B_0$, $c\Bsubseteq b\}$ for every
$b\in\frak B$ 
($\gamma$) whenever $A$, $B\subseteq\frak B$ are upwards-directed
sets such that $a\Bcap b=0$ for every $a\in A$ and $b\in B$ and
$\sup\{\nu(a\Bcup b):a\in A$, $b\in B\}=1$, then $\sup A$ is defined in
$\frak B$.
%411Yc 411H

}%end of exercises

\cmmnt{\Notesheader{411} Of course the list above can give only a
rough idea of the ways in which topologies and measures can interact.
In particular I have rather arbitrarily given a sort of priority to
three particular relationships between the domain $\Sigma$ of a measure
and the topology:  `topological measure space' (in which $\Sigma$
includes the
Borel $\sigma$-algebra), `Borel measure' (in which $\Sigma$ is precisely
the Borel $\sigma$-algebra) and `Baire measure' (in which $\Sigma$ is
the Baire $\sigma$-algebra).

Abstract topological measure theory is a relatively new subject, and
there are many technical questions on which different authors take
different views.   For instance, the phrase `Radon measure' is commonly
used to mean what I would call a `tight locally finite Borel measure'
(cf.\ 416F);  and some writers
enlarge the definition of `topological measure' to include Baire
measures as defined above.

I give very few examples at this stage, two drawn from the constructions
of Volumes 1-3 (Lebesgue measure and Stone spaces, 411O-411P) and one
new one (`Dieudonn\'e's measure', 411Q), with a glance at the
countable-cocountable measure of $\omega_1$ (411R).   The most glaring
omission is that of the product measures on $\{0,1\}^I$ and $[0,1]^I$.
I pass these by at the moment because a proper study of them requires
rather more preparation than can be slipped into a parenthesis.   (I
return to them in 416U.)   I have also omitted
any discussion of `measurable' and `almost continuous' functions, except
for a reference to a theorem in Volume 2 (411Ya), which will have
to be repeated and amplified later on (418K).
There is an obvious complementarity between the notions of `inner' and
`outer' regularity (411B, 411D), but it works well only for totally
finite spaces (411Xa);  in other cases it may not be obvious what will
happen (411O, 411Pe, 412W).

}%end of notes

\discrpage
