\frfilename{mt546.tex}
\versiondate{10.5.14}
\copyrightdate{2005}

\def\chaptername{Real-valued-measurable cardinals}
\def\sectionname{Power set $\sigma$-quotient algebras}

\def\kaqmc{\kappa_{\text{aqmc}}}
\def\pssqa{power set $\sigma$-quotient algebra}
\def\sphaT{\sphat\mskip5mu}

\newsection{546}

One way of interpreting the Gitik-Shelah theorem (543E) is to say that it
shows that `simple' atomless probability algebras cannot be of the form
$\Cal PX/\Cal N(\mu)$.
Similarly, the results of \S541-\S542 show that any ccc Boolean
algebra expressible as the quotient of a power set by a non-trivial
$\sigma$-ideal involves us
in dramatic complexities, though it is not clear that these must appear in the
quotient algebra itself.   In this section I give two further results of
M.Gitik and S.Shelah (546G, 546P)
showing that certain algebras cannot appear in this way.   I try
to present
the ideas in a form which leads naturally to some outstanding questions
(546Z).

\leader{546A}{(a) Definition} A {\bf\pssqa} is a Boolean algebra which is
isomorphic
to an algebra of the form $\Cal PT/\Cal J$ where $T$ is a set and
$\Cal J$ is a $\sigma$-ideal of subsets of $T$.

\spheader 546Ab\cmmnt{ I recall some notation which I will use in this
section.}  If $X$ is a
topological space, $\Cal B(X)$ will be its Borel $\sigma$-algebra,
$\CalBa(X)$ its
Baire $\sigma$-algebra, $\widehat{\Cal B}(X)$ its Baire-property algebra,
and $\Cal M(X)$ the $\sigma$-ideal of meager sets in $X$.
If $(X,\Sigma,\mu)$ is a measure space, $\Cal N(\mu)$ will be the null
ideal of $\mu$.   I will write $\non\Cal M$ for the uniformity of the
meager ideal of $\Bbb R$.

\leader{546B}{Lemma} (a) Any \pssqa\ is Dedekind $\sigma$-complete.

(b) If $\frak A$ is a \pssqa, $\frak B$ is a Boolean algebra and
$\pi:\frak A\to\frak B$ is a surjective sequentially order-continuous
Boolean homomorphism, then $\frak B$ is a \pssqa.
In particular, any principal ideal of a \pssqa\ is a \pssqa.

(c) The simple product of any family of \pssqa{s} is a \pssqa.

(d) If a non-zero atomless measurable algebra $\frak A$ is a \pssqa,
there is an \am\
cardinal $\kappa$ such that
$\tau(\frak A)\ge\min(\kappa^{(+\omega)},2^{\kappa})$.

\proof{{\bf (a)} 314C.

\medskip

{\bf (b)} Observe that by 313Qb a Boolean algebra $\frak A$ is a \pssqa\
iff there are
a set $T$ and a surjective sequentially order-continuous Boolean
homomorphism from $\Cal PT$ onto $\Cal A$.   It follows immediately
that an image of such
an algebra under a sequentially order-continuous homomorphism is again a
\pssqa.
And of course a principal ideal of $\frak A$ is an image of $\frak A$
under a
homomorphism $a\mapsto a\Bcap c$ which is actually order-continuous.

\medskip

{\bf (c)} If $\familyiI{\frak A_i}$ is a family of \pssqa{s}, then for
each $i\in I$
we have a set $T_i$ and a surjective sequentially order-continuous Boolean
homomorphism $\phi_i:\Cal PT_i\to\frak A_i$.
We can arrange that the $T_i$ are disjoint;  now
$A\mapsto\familyiI{\phi_i(A\cap T_i)}:\Cal PT\to\prod_{i\in I}\frak A_i$
is a surjective sequentially order-continuous Boolean homomorphism, so
$\prod_{i\in I}\frak A_i$ is a \pssqa.

\medskip

{\bf (d)} Let $\bar\mu$ be a measure on $\frak A$ such that
$(\frak A,\bar\mu)$ is a
probability measure.   Let $T$ be a set and $\phi:\Cal PT\to\frak A$ a
sequentially
order-continuous Boolean homomorphism.   Then we have a probability measure
$\nu=\bar\mu\phi$ with domain $\Cal PT$, and the measure algebra of $\nu$
is isomorphic to $\frak A$.   As $\nu$ is atomless, $\nu\{t\}=0$ for every
$t\in T$.   So $\kappa=\add\nu$ is \rvm\ (543Ba);  since
$\add\nu\le\cov\Cal N(\nu)\le\frak c$ (521I), $\kappa$ is \am\ (543Bc).
And $\tau(\frak A)\ge\max(\kappa^{(+\omega)},2^{\kappa})$ by 543F.
}%end of proof of 546B

\leader{546C}{}\cmmnt{ An elementary construction will be used more than once
below, and it may be less distracting if it is spelt out here.

\medskip

\noindent}{\bf Lemma} Suppose that $I$ is a set and $\Cal I$ a
$\sigma$-ideal of
subsets of $X=\{0,1\}^I$ which is generated by $\Cal I\cap\CalBa(X)$.
Suppose that $\Sigma$ is a $\sigma$-algebra of subsets of $X$ such that

\Centerline{$\CalBa(X)\subseteq\Sigma
\subseteq\{E\symmdiff A:E\in\CalBa(X)$, $A\in\Cal I\}$,}

\noindent and set $\frak A=\Sigma/\Sigma\cap\Cal I$.

(a) If $T$ is a set,
$\Cal J$ a $\sigma$-algebra of subsets of $T$ and
$\phi:\frak A\to\Cal PT/\Cal J$ is
a sequentially order-continuous Boolean homomorphism,
there is a function $f:T\to X$
such that $f^{-1}[E]^{\ssbullet}=\phi(E^{\ssbullet})$ for every
$E\in\Sigma$.

(b) Now suppose that $\phi$ is injective.   Set
$\Cal I^*=\{A:A\subseteq X$, $f^{-1}[A]\in\Cal J\}$.  Then $\Cal I^*$ is a
$\sigma$-ideal of subsets of $X$ including $\Cal I$,
and $\Sigma\cap\Cal I^*=\Sigma\cap\Cal I$.

(c) If $\phi$ is an isomorphism then for every $A\subseteq X$ there is an
$E\in\Sigma$ such that $A\symmdiff E\in\Cal I^*$.   So
we have an isomorphism between
$\frak A$ and $\Cal PX/\Cal I^*$ obtained by mapping $E^{\ssbullet}$
(interpreted in $\Sigma/\Sigma\cap\Cal I$) into $E^{\ssbullet}$
(interpreted in $\Cal PX/\Cal I^*)$ for every $E\in\Sigma$.

(d) If $\phi$ is an isomorphism and $X=\bigcup\Cal I\notin\Cal I$,
set $\kappa=\add\Cal I^*$.
Then there is a $\kappa$-additive ideal $\Cal J^*$
of subsets of $\kappa$, containing singletons,
such that $\Cal P\kappa/\Cal J^*$ is isomorphic to a
$\sigma$-subalgebra of a non-zero principal ideal of $\frak A$.

(e) If\cmmnt{, moreover,} $\frak A$ is atomless and ccc,
then $\kappa\le\frak c$ and $\Cal P\kappa/\Cal J^*$ is atomless;
and we can arrange that $\Cal J^*$ should be a normal ideal.

\proof{ Recall that $\CalBa(X)$ is just the $\sigma$-algebra generated by
$\{E_i:i\in I\}$ where $E_i=\{x:x\in X$, $x(i)=1\}$ for each $i\in I$
(4A3Na).

\medskip

{\bf (a)} For each $i\in I$, let $F_i\subseteq T$
be such that $F_i^{\ssbullet}=\phi E_i^{\ssbullet}$ in $\Cal PT/\Cal J$.
Define $f:T\to X$ by setting $f(t)(i)=1$ if $t\in F_i$, $0$ if
$t\in X\setminus F_i$.   Then
$\Sigma_0=\{E:E\in\Sigma$, $f^{-1}[E]^{\ssbullet}=\phi E^{\ssbullet}\}$
is a $\sigma$-ideal of subsets of $X$ containing every $E_i$
and therefore including $\CalBa(X)$.   If $A\in\Cal I$, there is an
$E\in\CalBa(X)\cap\Cal I$ including $A$, so

\Centerline{$f^{-1}[A]^{\ssbullet}
\subseteq f^{-1}[E]^{\ssbullet}=\phi E^{\ssbullet}=0$}

\noindent and $f^{-1}[A]\in\Cal J$.   If $E$ is any member of $\Sigma$, there is an
$E_0\in\CalBa(X)$ such that $E\symmdiff E_0\in\Cal I$, so that
$f^{-1}[E]\symmdiff f^{-1}[E_0]\in\Cal J$ and

\Centerline{$f^{-1}[E]^{\ssbullet}=f^{-1}[E_0]^{\ssbullet}
=\phi E_0^{\ssbullet}=\phi E^{\ssbullet}$.}

\medskip

{\bf (b)} Of course $\Cal I^*$ is a $\sigma$-ideal of subsets of $X$, and I have
already noted that $\Cal I\subseteq\Cal I^*$.
If $\phi$ is injective then, for $E\in\Sigma$,

$$\eqalign{E\in\Cal I^*
&\iff f^{-1}[E]\in\Cal J
\iff f^{-1}[E]^{\ssbullet}=0\cr
&\iff \phi E^{\ssbullet}=0
\iff E^{\ssbullet}=0
\iff E\in\Cal I.\cr}$$

\noindent So $\Sigma\cap\Cal I^*=\Sigma\cap\Cal I$.

\medskip

{\bf (c)} If $A\subseteq X$ is any set,
consider $a=\phi^{-1}(f^{-1}[A]^{\ssbullet})\in\frak A$.   Let $E\in\Sigma$ be such
that $E^{\ssbullet}$ (interpreted in $\Sigma/\Sigma\cap\Cal I$) is equal to $a$.
Then $f^{-1}[E]^{\ssbullet}=f^{-1}[A]^{\ssbullet}$, that is,
$f^{-1}[E\symmdiff A]\in\Cal J$ and $E\symmdiff A\in\Cal I^*$.

Because $\Sigma\cap\Cal I^*=\Sigma\cap\Cal I$, the proposed assignment
gives us an injective Boolean
homomorphism $\psi:\frak A\to\Cal PX/\Cal I^*$;  and we have just seen that every
subset of $X$ is equivalent, mod $\Cal I^*$, so some member of $\Sigma$.
This shows that $\psi$ is
surjective, therefore an isomorphism, as claimed.

\medskip

{\bf (d)} If $X=\bigcup\Cal I$ then $X=\bigcup\Cal I^*$ because
$\Cal I\subseteq\Cal I^*$;  if $X\notin\Cal I$ then $X\notin\Cal I^*$
because $X\in\Sigma$.   So $\kappa=\add\Cal I^*$ is not $\infty$.   Let
$\ofamily{\xi}{\kappa}{A_{\xi}}$ be a family in $\Cal I^*$ with union
$A\notin\Cal I^*$.   Define $g:A\to\kappa$ by setting $g(x)=\min\{\xi:x\in A_{\xi}\}$
for $x\in A$.   Set

\Centerline{$\Cal J^*=\{B:B\subseteq\kappa$, $g^{-1}[B]\in\Cal I^*\}
=\{B:B\subseteq\kappa$, $(gf)^{-1}[B]\in\Cal J\}$.}

\noindent
Then $\Cal J^*$ is a proper $\kappa$-additive ideal of subsets of $\kappa$ containing
singletons.
The map $B\mapsto g^{-1}[B]:\Cal P\kappa\to\Cal PA$ induces an injective
sequentially
order-continuous Boolean homomorphism from $\Cal P\kappa/\Cal J^*$ to
the principal ideal of $\Cal PX/\Cal I^*\cong\frak A$ generated by
$A^{\ssbullet}$, so we have a sequentially order-continuous embedding of
$\Cal P\kappa/\Cal J^*$ into a
principal ideal of $\frak A$, necessarily non-zero.   By 314F(b-i),
the image of
$\Cal P\kappa/\Cal J^*$ is a $\sigma$-subalgebra of the principal ideal.

\medskip

{\bf (e)} If $\frak A\cong\Cal PX/\Cal I^*$ is atomless and ccc,
then 541O tells us
that $\kappa\le\frak c$.   So $\Cal P\kappa/\Cal J^*$ will be atomless,
by 541P.

Returning to the argument of (d), if $\frak A$ is ccc we have the option
of using 541J to give us a function $g:A\to\kappa$ such that
$\Cal J^*=\{B:B\subseteq\kappa$, $g^{-1}[B]\in\Cal I^*\}$
is a normal ideal, and then proceeding as before.
}%end of proof of 546C

\leader{546D}{}\cmmnt{ In 527M I introduced `harmless' algebras.
Here we need to know a little about harmless \pssqa{s}.

\medskip

\noindent}{\bf Lemma} Suppose that $\kappa$ is a regular uncountable
cardinal and $\Cal I$ is a $\kappa$-additive
ideal of subsets of $\kappa$ such that $\Cal P\kappa/\Cal I$ is
harmless.   Then
$\cmmnt{\frak A=\mskip5mu}\Cal P(\kappa\times\kappa)/\Cal I\ltimes\Cal I$\cmmnt{ (definition:
527Ba)} is harmless.

\proof{{\bf (a)} If
$\kappa=\omega_1$ then $\Cal I$ is of the form $\{I:A\cap I=\emptyset\}$ for some
countable $A\subseteq\kappa$.   \Prf\ Set
$A=\{\xi:\xi<\omega_1$, $\{\xi\}\notin\Cal I\}$.
Because $\Cal P\omega_1/\Cal I$ is harmless, it is ccc, so $A$ is countable.
Set $\Cal J=\Cal I\cap\Cal P(\omega_1\setminus A)$, so that $\Cal J$ is an
$\sigma$-ideal of subsets of $\omega_1\setminus A$ containing
singletons, and $\Cal P(\omega_1\setminus A)/\Cal J$ can be identified with the
principal ideal of $\Cal P\omega_1/\Cal I$ generated by
$(\omega_1\setminus A)^{\ssbullet}$, so is ccc.
Since $\omega_1$ is certainly
not weakly inaccessible, $\Cal J$ is not a proper ideal, by 541L, and
$\omega_1\setminus A\in\Cal I$.   It follows that
$\Cal I=\Cal P(\omega_1\setminus A)$, as stated.\ \Qed

In this case, $\frak A\cong\Cal P(A\times A)$ has a countable $\pi$-base
and is harmless, by 527Nd.
So let us suppose from now on that $\kappa>\omega_1$.

\medskip

{\bf (b)} By 527Bb, $\Cal I\ltimes\Cal I$ is $\kappa$-additive;  in
particular,
it is a $\sigma$-ideal.   Next, it is $\omega_1$-saturated.   \Prf\
Let $\ofamily{\alpha}{\omega_1}{V_{\alpha}}$ be a disjoint family in
$\Cal P(\kappa\times\kappa)$.   For each $\xi<\kappa$,
there is an $\alpha_{\xi}<\omega_1$ such that
$V_{\alpha}[\{\xi\}]\in\Cal I$ for every
$\alpha\ge\alpha_{\xi}$.   Because $\Cal I$ is $\omega_2$-additive and
$\omega_1$-saturated, there is an $\alpha^*<\omega_1$ such that
$\{\xi:\alpha_{\xi}>\alpha^*\}\in\Cal I$ (541E).   But now
$V_{\alpha}\in\Cal I\ltimes\Cal I$ for every $\alpha>\alpha^*$.\ \Qed

Consequently $\frak A$ is Dedekind complete (541B).

\medskip

{\bf (c)} Let $\frak C$ be an order-closed subalgebra of
$\frak A$ with countable Maharam type;  let $\sequencen{C_n}$ be a sequence of
subsets of $\kappa\times\kappa$ such that
$\frak C$ is the order-closed subalgebra of
$\frak A$ generated by $\{C_n^{\ssbullet}:n\in\Bbb N\}$.
For each $\xi<\kappa$, let $\frak B_{\xi}$ be the
order-closed subalgebra of $\Cal P\kappa/\Cal I$ generated by
$\{C_n[\{\xi\}]^{\ssbullet}:n\in\Bbb N\}$.   By 527Nb, $\frak B_{\xi}$ has
countable $\pi$-weight;  let $\Cal E_{\xi}$ be a countable subset
of $\Cal P\kappa$ such that $\{E^{\ssbullet}:E\in\Cal E_{\xi}\}$ is an order-dense
subset of $\frak B_{\xi}$;  let $\sequencen{E_{\xi n}}$ run over $\Cal E_{\xi}$.
We can of course suppose that $C_n[\{\xi\}]\in\Cal E_{\xi}$ and that
$E_{\xi,2n}=C_n[\{\xi\}]$ for each $n$.   For $n\in\Bbb N$ set
$E_n=\{(\xi,\eta):\xi<\kappa$, $\eta\in E_{\xi n}\}$;
we have $E_{2n}=C_n$ for each $n$.   Let $\frak B$ be the order-closed subalgebra of
$\Cal P\kappa/\Cal I$ generated by
$\{\{\xi:E_{\xi m}\cap E_{\xi n}\in\Cal I\}^{\ssbullet}:m$, $n\in\Bbb N\}$, and
$\Cal F$ a countable subset of $\Cal P\kappa$, containing $\kappa$,
such that $\{F^{\ssbullet}:F\in\Cal F\}$ is an order-dense set in $\frak B$.
Let $\Cal A$ be the countable set
$\{\emptyset\}\cup\{E_m\cap(F\times\kappa):m\in\Bbb N$, $F\in\Cal F\}$.

\medskip

{\bf (d)} If $\sequencen{A_n}$ is any sequence in $\Cal A$, there is a sequence
$\sequencen{\hat A_n}$ in $\Cal A$ such that
$\sup_{n\in\Bbb N}\hat A_n^{\ssbullet}$
is the complement of $\sup_{n\in\Bbb N}A_n^{\ssbullet}$ in $\frak A$.
\Prf\ Express
$A_n$ as $E_{m_n}\cap(F_n\times\kappa)$ where $m_n\in\Bbb N$ and
$F_n\in\Cal F$ for
each $n$.   Set $W=\bigcup_{n\in\Bbb N}A_n$.   If $\xi<\kappa$ then
$W[\{\xi\}]=\bigcup_{n\in\Bbb N,\xi\in F_n}E_{\xi,m_n}$, so
$W[\{\xi\}]^{\ssbullet}\in\frak B_{\xi}$;  set
$J_{\xi}=\{j:W[\{\xi\}]\cap E_{\xi j}\in\Cal I\}$.   If
$G_j=\{\xi:\xi<\kappa$, $j\in J_{\xi}\}$ for each $j$ and
$\hat W=\bigcup_{j\in\Bbb N}E_j\cap(G_j\times\kappa)$,
$W[\{\xi\}]^{\ssbullet}$ and $\hat W[\{\xi\}]^{\ssbullet}$ are
complementary elements of
$\frak B_{\xi}$ for each $\xi$, so $W^{\ssbullet}$ and
$\hat W^{\ssbullet}$ are complementary elements of $\frak A$.

Now

\Centerline{$G_j=\{\xi:W[\{\xi\}]\cap E_{\xi j}\in\Cal I\}
=\bigcap_{n\in\Bbb N}\{\xi:\xi\notin F_n$ or
$E_{\xi m_n}\cap E_{\xi j}\in\Cal I\}$,}

\noindent so $G_j^{\ssbullet}\in\frak B$ and there is an
$\Cal F_j\subseteq\Cal F$
such that $G_j^{\ssbullet}=\sup_{F\in\Cal F_j}F^{\ssbullet}$.   Taking
$\sequencen{\hat A_n}$ to run over
$\{\emptyset\}\cup\{E_j\cap(F\times\kappa):j\in\Bbb N$, $F\in\Cal F_j\}$,
we get a sequence in $\Cal A$ such that
$\sup_{n\in\Bbb N}\hat A_n^{\ssbullet}=\hat W^{\ssbullet}$, as required.\
\Qed

\medskip

{\bf (e)} It follows that if we take $\frak D$ to be the set of those
$a\in\frak A$
expressible in the form $\sup_{n\in\Bbb N}A_n^{\ssbullet}$ for some
sequence in
$\Cal A$, the complement of an element of $\frak D$ belongs to $\frak D$;  as
$\frak D$ is certainly closed under countable suprema,
it is a $\sigma$-subalgebra of $\frak A$, therefore order-closed, because
$\frak A$ is ccc.
And $\{A_n^{\ssbullet}:n\in\Bbb N\}$ witnesses that
$\pi(\frak D)\le\omega$.

As $C_n=E_{2n}\cap(\kappa\times\kappa)\in\Cal A$ for each $n$,
$\frak C\subseteq\frak D$.   So
$\omega\ge\pi(\frak D)\ge\pi(\frak C)$, by 514Eb.

As $\frak C$ is an arbitrary countably generated order-closed subalgebra
of $\frak A$, $\frak A$ is harmless, by 527Nb in the other direction.
}%end of proof of 546D

\leader{546E}{}\cmmnt{ I wish to follow the lines of the argument in
543C-543E to
prove a similar theorem in which `measure' is replaced by `category'.
The lemma just
proved corresponds to the definition of $\tilde\nu$ in part (c) of the
proof of 543D.
The next result will play the role previously taken by 543C.

\medskip

\noindent}{\bf Proposition} Suppose that $\kappa$ is an uncountable
regular cardinal
and $\Cal I$ is a $\kappa$-additive ideal of subsets of $\kappa$ such that
$\Cal P\kappa/\Cal I$ is harmless.   Let $X$ be a
ccc topological space of $\pi$-weight less than $\kappa$.   Then
$\Cal M(X)\rtimes\Cal I\subseteq\Cal M(X)\ltimes\Cal I$.

\proof{{\bf (a)} Take $C\in\Cal M(X)\rtimes\Cal I$.
Set $A=\{\xi:\xi<\kappa$, $C^{-1}[\{\xi\}]\notin\Cal M(X)\}$ and
$B=\{x:x\in X$, $C[\{x\}]\notin\Cal I\}$, so that
$A\in\Cal I$ and I need to show that $B\in\Cal M(X)$.
For $\xi\in\kappa\setminus A$, let $\sequencen{F_{\xi n}}$ be a
sequence of nowhere dense sets in $X$ with union $C^{-1}[\{\xi\}]$;
for $\xi\in A$ set $F_{\xi n}=\emptyset$ for every $n$.   For each $n$,
set $C_n=\{(x,\xi):\xi<\kappa$, $x\in F_{\xi n}\}$ and
$B_n=\{x:C_n[\{x\}]\notin\Cal I\}$, so that $B=\bigcup_{n\in\Bbb N}B_n$ and it
will be enough to show that every $B_n$ is meager.   Fix $n\in\Bbb N$.

\medskip

{\bf (b)} Let $\ofamily{\alpha}{\pi(X)}{G_{\alpha}}$ enumerate a
$\pi$-base for the topology of $X$, and for $\alpha<\pi(X)$ let $D_{\alpha}$ be the
set of those $\xi<\kappa$ such that $G_{\alpha}\cap F_{\xi n}=\emptyset$.   Then
$W=\bigcup_{\alpha<\pi(X)}G_{\alpha}\times D_{\alpha}$ is disjoint from
$C_n$.   For each $\xi<\kappa$, set
$I_{\xi}=\{\alpha:\alpha<\pi(X)$, $\xi\in D_{\alpha}\}$;
then $\bigcup_{\alpha\in I_{\xi}}G_{\alpha}$ is dense in $X$.
Because $X$ is ccc,
there is a countable $J_{\xi}\subseteq I_{\xi}$ such that
$\bigcup_{\alpha\in J_{\xi}}G_{\alpha}$ is dense (5A4Bd).   Now $\Cal I$ is
$\omega_1$-saturated and $\pi(X)<\add\Cal I$, so there is a countable
$I\subseteq\pi(X)$
such that $A'=\{\xi:\xi<\kappa$, $J_{\xi}\not\subseteq I\}$ belongs to
$\Cal I$ (541D).

\medskip

{\bf (c)} Let $\frak B$ be the order-closed subalgebra of
$\Cal P\kappa/\Cal I$ generated by
$\{D_{\alpha}^{\ssbullet}:\alpha\in I\}$.
Because $\Cal P\kappa/\Cal I$ is harmless, $\pi(\frak B)\le\omega$
(527Nb);  let $\sequence{i}{F_i}$ be a sequence in $\Cal P\kappa$ such that
$\{F_i^{\ssbullet}:i\in\Bbb N\}$ is order-dense in $\frak B$.
Let $\Cal E$ be the countable subalgebra of $\Cal P\kappa$ generated by
$\{F_i:i\in\Bbb N\}\cup\{D_{\alpha}:\alpha\in I\}$, and set
$V=\bigcup(\Cal E\cap\Cal I)$, so that $V\in\Cal I$.
Give $Y=\kappa\setminus V$ the
second-countable topology generated by $\{E\setminus V:E\in\Cal E\}$.

If $H$ is a dense open set in $Y$, then $\kappa\setminus H\in\Cal I$.
\Prf\ Setting $\Cal E'=\{E:E\in\Cal E$, $E\setminus V\subseteq H\}$,
$H=(\bigcup\Cal E')\setminus V$, so
$H^{\ssbullet}=\sup_{E\in\Cal E'}E^{\ssbullet}$ in
$\Cal P\kappa/\Cal I$, and $H^{\ssbullet}\in\frak B$.   \Quer\ If
$\kappa\setminus H\notin\Cal I$, then $H^{\ssbullet}\ne 1$ and there is an
$i\in\Bbb N$ such that $F_i^{\ssbullet}$ is non-zero and disjoint from
$H^{\ssbullet}$.   In this case, $F_i\cap E\in\Cal I$ for every $E\in\Cal E'$, so
$F_i\setminus V$ is disjoint from $H$;  but $F_i\setminus V$ is a non-empty open
subset of $Y$.\ \Bang\Qed

Consequently $\Cal M(Y)\subseteq\Cal I$.

\medskip

{\bf (d)}
Set $W=\bigcup_{\alpha\in I}G_{\alpha}\times(D_{\alpha}\setminus V)$, so that $W$ is
an open set in $X\times Y$.   Then $W$ is dense.
\Prf\Quer\ Otherwise, we have a non-empty open set $G\subseteq X$ and a non-empty
open set $U\subseteq Y$ such that $I=I'\cup I''$, where
$I'=\{\alpha:\alpha\in I$, $G\cap G_{\alpha}=\emptyset\}$ and
$I''=\{\alpha:\alpha\in I$, $U\cap D_{\alpha}=\emptyset\}$.   As $U$ includes some
non-empty set of the form $E\setminus V$ where $E\in\Cal E$, $U\notin\Cal I$.
So there must be a $\xi\in U\setminus A'$.   In this case, $J_{\xi}\subseteq I$
while
$\bigcup_{\alpha\in J_{\xi}}G_{\alpha}$ is dense and meets $G$;  there is therefore
an $\alpha\in J_{\xi}\setminus I'$.   But now $\alpha\in I_{\xi}$ so
$\xi\in D_{\alpha}$, while also $\alpha\in I\setminus I'=I''$, so $U\cap D_{\alpha}=\emptyset$ and $\xi\notin D_{\alpha}$.\
\Bang\Qed

\medskip

{\bf (e)} $W'=(X\times Y)\setminus W$ is therefore meager in $X\times Y$
and belongs to $\Cal M(X)\ltimes\Cal M(Y)$, by 527Db.
If $x\in B_n$, then $C_n[\{x\}]\notin\Cal I$;  but
$C_n[\{x\}]\setminus V\subseteq W'[\{x\}]$,
so $W'[\{x\}]\notin\Cal I$ and $W'[\{x\}]\notin\Cal M(Y)$.
Accordingly

\Centerline{$B_n\subseteq\{x:W'[\{x\}]\notin\Cal M(Y)\}\in\Cal M(X)$,}

\noindent as required.
}%end of proof of 546E

\leader{546F}{Corollary} Suppose that $X$, $\kappa$ and $\Cal I$ are as
in 546E, and that moreover $\bigcup\Cal I=\kappa\notin\Cal I$.

(a) Suppose that $\ofamily{\xi}{\kappa}{A_{\xi}}$ is a non-decreasing
family of subsets of $X$ with union $A$.   Then there is a
$\theta<\kappa$ such that $E\cap A_{\theta}$
is non-meager whenever $E\subseteq X$ is a set
with the Baire property and $E\cap A$ is not meager.

(b) If $\ofamily{\xi}{\kappa}{A_{\xi}}$ is a family in
$\Cal PX\setminus\Cal M(X)$ such that
$\#(\bigcup_{\xi<\kappa}A_{\xi})<\kappa$, then there are distinct $\xi$,
$\eta<\kappa$ such that $A_{\xi}\cap A_{\eta}\notin\Cal M(X)$.

(c) If we have a family $\ofamily{\xi}{\kappa}{h_{\xi}}$ of functions such that
$\dom h_{\xi}$ is a non-meager subset of $X$ for each $\xi$ and
$\#(\bigcup_{\xi<\kappa}h_{\xi})<\kappa$\cmmnt{ (identifying
each $h_{\xi}$ with its graph)}, then there are distinct
$\xi$, $\eta<\kappa$ such that
$\{x:h_{\xi}(x)$ and $h_{\eta}(x)$ are defined and equal$\}$ is non-meager.

\proof{{\bf (a)} Let $\frak G=\widehat{\Cal B}(X)/\Cal M(X)$
be the category algebra of $X$;  for $B\subseteq X$
set $\psi(B)=\inf\{E^{\ssbullet}:B\subseteq E\in\widehat{\Cal B}(X)\}$,
as in 514Ie.
Because $\frak G$ is ccc (514Ja), there is a sequence $\sequencen{\xi_n}$ in
$\kappa$ such that
$\sup_{n\in\Bbb N}\psi(A_{\xi_n})=\sup_{\xi<\kappa}\psi(A_{\xi})$;  setting
$\theta=\sup_{n\in\Bbb N}\xi_n$, we see that $\theta<\kappa$ (because
$\cf\kappa>\omega$) and that $\psi(A_{\xi})\Bsubseteq\psi(A_{\theta})$ for every
$\xi<\kappa$.

\Quer\ Suppose, if possible, that there is a set $E\in\widehat{\Cal B}(X)$ such that
$E\cap A_{\theta}$ is meager but $E\cap A$ is non-meager.   Replacing
$E$ by $E\setminus A_{\theta}$ if
necessary, we may suppose that $E\cap A_{\theta}$ is empty.   If
$\xi<\kappa$, then

\Centerline{$\psi(E\cap A_{\xi})
\Bsubseteq E^{\ssbullet}\cap\psi(A_{\theta})
\Bsubseteq E^{\ssbullet}\cap(X\setminus E)^{\ssbullet}=0$,}

\noindent so $E\cap A_{\xi}$ is meager.

Define $f:E\cap A\to\kappa$ by setting $f(x)=\min\{\xi:x\in A_{\xi}\}$ for $x\in E$.
Consider the set

\Centerline{$C=\{(x,\xi):f(x)\le\xi<\kappa\}\subseteq(E\cap A)\times\kappa$.}

\noindent If $\xi<\kappa$, then

\Centerline{$C^{-1}[\{\xi\}]=\{x:x\in E$, $f(x)\le\xi\}
\subseteq E\cap A_{\xi}\in\Cal M(X)$;}

\noindent thus $C\in\Cal M(X)\rtimes\Cal I$.   By 546E,
$C\in\Cal M(X)\ltimes\Cal I$.   As $E\cap A$ is not meager, there is an $x\in E\cap A$
such that $C[\{x\}]\in\Cal I$.   But
$C[\{x\}]=\{\xi:f(x)\le\xi<\kappa\}\notin\Cal I$.\ \Bang

So $\theta$ has the required property.

\medskip

{\bf (b)} Write $\Cal J=\Cal I\ltimes\Cal I\normalsubgroup\Cal P(\kappa\times\kappa)$.
By 546D, $\Cal P(\kappa\times\kappa)/\Cal J$ is harmless.   Set

\Centerline{$W=\{(x,\xi,\eta):\xi$, $\eta<\kappa$, $\xi\ne\eta$,
$x\in A_{\xi}\cap A_{\eta}\}$.}

\noindent\Quer\ If $A_{\xi}\cap A_{\eta}\in\Cal M(X)$ for all distinct $\xi$,
$\eta<\kappa$, then $W$, regarded as a subset of $X\times(\kappa\times\kappa)$,
belongs to $\Cal M(X)\rtimes\Cal J$;  by 546E, $W\in\Cal M(X)\ltimes\Cal J$.
For $x\in X$ set $C_x=\{\xi:\xi<\kappa$, $x\in A_{\xi}\}$.   Then
$W[\{x\}]=C_x^2\setminus\Delta$, where $\Delta=\{(\xi,\xi):\xi<\kappa\}$.   So
$W[\{x\}]\in\Cal J$ iff $C_x\in\Cal I$, and
$E=\{x:C_x\notin\Cal I\}$ is meager.   Next, $A=\bigcup_{\xi<\kappa}A_{\xi}$ is
supposed to have cardinal less than $\kappa$, so
$\bigcup_{x\in A\setminus E}C_x\in\Cal I$ and there is some
$\zeta\in\kappa\setminus\bigcup_{x\in A\setminus E}C_x$.   But in this case
$A_{\zeta}\subseteq E$ is meager.\ \BanG\  So we have the result.

\medskip

{\bf (c)(i)} For each $\xi<\kappa$,
set $A_{\xi}=\dom h_{\xi}$ and let $H_{\xi}$ be
the regular open set in $X$ such that $A_{\xi}\setminus H_{\xi}$
is meager and $G\cap H_{\xi}$ is empty whenever $G$ is open and
$G\cap A_{\xi}$
is meager (4A3Ra).   Set $h'_{\xi}=h_{\xi}\restr H_{\xi}$ and
$Y=\bigcup_{\xi<\kappa}h'_{\xi}$;  let
$\pi_1:Y\to X$ be the first-coordinate projection.   Give $Y$ the topology
$\frak S=\{\pi_1^{-1}[G]:G\in\frak T\}$, where $\frak T$ is the topology
of $X$.

\medskip

\quad{\bf (ii)} If $\Cal U$ is any $\pi$-base for $\frak T$, then
$\Cal V=\{\pi_1^{-1}[U]:U\in\Cal U\}$ is a $\pi$-base for $\frak S$.
\Prf\ If $H\subseteq Y$ is open and not empty, take $G\in\frak T$ such that
$H=\pi_1^{-1}[G]$ and a $\xi<\kappa$ such that $H\cap h'_{\xi}\ne\emptyset$.
Then $G\cap H_{\xi}\cap A_{\xi}=G\cap\dom h'_{\xi}$ is non-empty;
by the choice of
$H_{\xi}$, $G\cap H_{\xi}\cap A_{\xi}$ is non-meager.   Set
$\Cal U'=\{U:U\in\Cal U$, $U\cap G\cap H_{\xi}\cap A_{\xi}=\emptyset\}$.   Then
$\bigcup\Cal U'$ cannot be dense and there is a non-empty $U\in\Cal U$
disjoint from
$\bigcup\Cal U'$.   But now $U\cap G\ne\emptyset$, so there is a non-empty
$U'\in\Cal U$ with $U'\subseteq U\cap G$;  in which case
$V=\pi_1^{-1}[U']$ belongs to
$\Cal V$, is included in $H$ and meets $h'_{\xi}$, so is not empty.
As $H$ is arbitrary, $\Cal V$ is a $\pi$-base for $\frak S$.\ \QeD\

\medskip

\quad{\bf (iii)}
It follows at once that $\pi(Y)\le\pi (X)<\kappa$.   We see also that if
$A\subseteq X$ is nowhere dense, then $\{G:G\in\frak T$, $G\cap A=\emptyset\}$ is a
$\pi$-base for $\frak T$,

\Centerline{$\{\pi_1^{-1}[G]:G\in\frak T$, $G\cap A=\emptyset\}
=\{H:H\in\frak S$, $H\cap\pi_1^{-1}[A]=\emptyset\}$}

\noindent is a $\pi$-base for $\frak S$ and $\pi_1^{-1}[A]$ is nowhere dense in $Y$.
Accordingly $\pi_1^{-1}[A]\in\Cal M(Y)$ for every $A\in\Cal M(X)$.

\medskip

\quad{\bf (iv)}
If $B\subseteq Y$ is nowhere dense in $Y$ then $\pi_1[B]$ is nowhere dense in $X$.
\Prf\ If $G\subseteq X$ is a non-empty open set, then either $\pi_1^{-1}[G]$ is empty
and $G\cap\pi_1[B]=\emptyset$, or $\pi_1^{-1}[G]$ is a non-empty open subset of $Y$.
In the latter case, $\pi_1^{-1}[G]\setminus\overline{B}$ must be of the form
$\pi_1^{-1}[G']$ for some open set $G'\subseteq X$, and $G'\cap G$ is a non-empty open
subset of $G$ disjoint from $\pi_1[B]$.\ \QeD\
It follows at once that $\pi_1[B]\in\Cal M(X)$ whenever $B\in\Cal M(Y)$.

\medskip

\quad{\bf (v)}
Since $\pi_1[h'_{\xi}]=A_{\xi}\cap H_{\xi}$ is non-meager in $X$, $h'_{\xi}$ is
non-meager in $Y$, for every $\xi$.   So (b) here tells us that there are distinct
$\xi$, $\eta<\kappa$ such that $h'_{\xi}\cap h'_{\eta}$ is non-meager in $Y$.
In this case, setting $A=\{x:h_{\xi}(x)$ and $h_{\xi}(y)$ are defined and equal$\}$,
$\pi_1^{-1}[A]$ includes $h'_{\xi}\cap h'_{\eta}$ so is non-meager, and $A$ is
non-meager, by (iii).
}%end of proof of 546F

\leader{546G}{The Gitik-Shelah theorem for
Cohen \dvrocolon{algebras}}\cmmnt{ I come
now to a companion result to the Gitik-Shelah Theorem in 543E.
I follow the proof I gave before as closely as I can.

\medskip

\noindent}{\bf Theorem}\cmmnt{ ({\smc Gitik \& Shelah
89}, {\smc Gitik \& Shelah 93})} Let $\kappa$ be a regular uncountable
cardinal and
$\Cal I$ a $\kappa$-additive ideal of subsets of $\kappa$ such that
$\cmmnt{\frak A=\mskip5mu}\Cal P\kappa/\Cal I$
is isomorphic to the category algebra\cmmnt{ $\frak G_{\lambda}$} of
$\cmmnt{X=\mskip5mu}\{0,1\}^{\lambda}$ for some infinite
cardinal $\lambda$.   Then $\lambda\ge\min(\kappa^{(+\omega)},2^{\kappa})$.

\proof{{\bf (a)} Let $\phi:\frak G_{\lambda}\to\frak A$
be an isomorphism.
As $X$ is completely regular and ccc, $\Cal M(X)$ is generated by
$\CalBa(X)\cap\Cal M(X)$ (5A4E(d-ii)), and
$\frak G_{\lambda}=\{H^{\ssbullet}:H\in\CalBa(X)\}$.
By 546C, we have a function
$f:\kappa\to X$ such that $f^{-1}[E]^{\ssbullet}=\phi E^{\ssbullet}$ in
$\frak A$ for every $E\in\widehat{\Cal B}(X)$.

\medskip

{\bf (b)}\Quer\ Suppose, if possible, that
$\lambda<\min(\kappa^{(+\omega)},2^{\kappa})$.

Set $\zeta=\max(\lambda^+,\kappa^+)$.   Then we have an infinite cardinal
$\delta<\kappa$, a stationary set $S\subseteq\zeta$, and a family
$\family{\alpha}{S}{g_{\alpha}}$ of functions from $\kappa$ to
$2^{\delta}$ such that $g_{\alpha}[\kappa]\subseteq\alpha$ for every
$\alpha\in S$ and $\#(g_{\alpha}\cap g_{\beta})<\kappa$ for distinct
$\alpha$, $\beta\in S$.   Moreover,

----- if $\lambda<\Tr(\kappa)$, then
$g_{\alpha}[\kappa]\subseteq\kappa$ for every $\alpha\in S$;

----- if $\lambda\ge\Tr(\kappa)$, then
$g_{\alpha}\restr\gamma=g_{\beta}\restr\gamma$ whenever $\gamma<\kappa$
is a limit ordinal and $\alpha$, $\beta\in S$ are such that
$g_{\alpha}(\gamma)=g_{\beta}(\gamma)$.

\medskip

\Prf\ Copy the argument from part (b) of the proof of 543E.\ \Qed

\medskip

{\bf (c)} Fix an injective function $h:2^{\delta}\to\{0,1\}^{\delta}$.
For $\alpha\in S$ and $\iota<\delta$ set

\Centerline{$U_{\alpha\iota}
=\{\xi:\xi<\kappa$, $(hg_{\alpha}(\xi))(\iota)=1\}$,}

\noindent and let
$H_{\alpha\iota}\in\CalBa(X)$ be such that
$H_{\alpha\iota}^{\ssbullet}=\phi^{-1}(U^{\ssbullet}_{\alpha\iota})$ in
$\frak G_{\lambda}$;  then
$U_{\alpha\iota}\symmdiff f^{-1}[H_{\alpha\iota}]\in\Cal I$.   Define
$\tilde g_{\alpha}:X\to\{0,1\}^{\delta}$ by setting

$$\eqalign{(\tilde g_{\alpha}(x))(\iota)
&=1\text{ if }x\in H_{\alpha\iota},\cr
&=0\text{ otherwise}.\cr}$$

\noindent Then

$$\eqalign{\{\xi:
  \xi<\kappa,\,\tilde g_{\alpha}f(\xi)\ne hg_{\alpha}(\xi)\}
&=\bigcup_{\iota<\delta}\{\xi:
  (\tilde g_{\alpha}f(\xi))(\iota)\ne(hg_{\alpha}(\xi))(\iota)\}\cr
&=\bigcup_{\iota<\delta}
  U_{\alpha\iota}\symmdiff f^{-1}[H_{\alpha\iota}]
\in\Cal I\cr}$$

\noindent because $\delta<\kappa=\add\Cal I$.
Set $V_{\alpha}=\{\xi:\tilde g_{\alpha}f(\xi)=hg_{\alpha}(\xi)\}$, so
that $\kappa\setminus V_{\alpha}\in\Cal I$, for each $\alpha\in S$.

\medskip

{\bf (d)} Because every $H_{\alpha\iota}$ is determined by coordinates
in a countable
set, there is for each $\alpha\in S$ a set $I_{\alpha}\subseteq\lambda$
such that $\#(I_{\alpha})\le\delta$ and $H_{\alpha\iota}$ is determined
by coordinates in $I_{\alpha}$ for every $\iota<\delta$.
By 5A1J there is an $M\subseteq\lambda$ such that
$S_1=\{\alpha:\alpha\in S$, $I_{\alpha}\subseteq M\}$
is stationary in $\zeta$ and $\cf(\#(M))\le\delta$;  because
$\lambda<\kappa^{(+\omega)}$ and $\cf(\kappa)=\kappa>\delta$,
$\#(M)<\kappa$.   Set $\pi_M(z)=z\restr M$ for $z\in X$,
and $f_M=\pi_Mf:\kappa\to\{0,1\}^M$.

If $E\subseteq\{0,1\}^M$ has the Baire property,
then $\pi_M^{-1}[E]$ has the Baire
property in $X$, and $\pi_M^{-1}[E]$ is meager iff $E$ is
(5A4E(c-iii), applied to $\{0,1\}^M\times\{0,1\}^{\lambda\setminus M}$).
So $f_M^{-1}[E]\in\Cal I$ iff $E$ is meager.

\medskip

{\bf (e)} For each $\alpha\in S_1$, there is a $\theta_{\alpha}<\kappa$
such that $f_M[V_{\alpha}\cap\theta_{\alpha}]$ meets every non-empty open subset of
$\{0,1\}^M$ in a non-meager set.
\Prf\ Apply 546Fa to $\ofamily{\xi}{\kappa}{f_M[V_{\alpha}\cap\xi]}$.
$\bigcup\Cal I=\kappa$ because $\frak A$ is atomless, and
$\kappa\notin\Cal I$ because $\frak A\ne\{0\}$;
while $\{0,1\}^M$ is certainly ccc, and has $\pi$-weight at most
$\max(\omega,\#(M))<\kappa$.   There is therefore
a $\theta_{\alpha}<\kappa$ such that $E\cap f_M[V_{\alpha}\cap\theta]$ is
non-meager whenever $E\subseteq\{0,1\}^M$ has the Baire property
and $E\cap f_M[V_{\alpha}]$ is non-meager.
If $G\subseteq\{0,1\}^M$ is a non-empty open set, then

\Centerline{$f_M^{-1}[G\setminus f_M[V_{\alpha}]]
\subseteq\kappa\setminus V_{\alpha}\in\Cal I$,}

\noindent so either
$G\setminus f_M[V_{\alpha}]$ is meager or it does not have the
Baire property;  in
either case, $G\cap f_M[V_{\alpha}]$ is non-meager so
$G\cap f_M[V_{\alpha}\cap\theta_{\alpha}]$ is non-meager.\ \Qed

Evidently we may take it that every $\theta_{\alpha}$ is a non-zero
limit ordinal.

\medskip

{\bf (f)} Because $\zeta=\cf\zeta>\kappa$, there is a $\theta<\kappa$ such that
$S_2=\{\alpha:\alpha\in S_1$, $\theta_{\alpha}=\theta\}$
is stationary in $\zeta$.   Now there is a
$Y\in[2^{\delta}]^{<\kappa}$ such that
$S_3=\{\alpha:\alpha\in S_2$, $g_{\alpha}[\theta]\subseteq Y\}$ is
stationary in $\zeta$.
\Prf\ Use the argument of part (f) of the proof of 543E.\ \Qed

\medskip

{\bf (g)} For each $\alpha\in S_3$, set

\Centerline{$Q_{\alpha}=f_M[V_{\alpha}\cap\theta]
=f_M[V_{\alpha}\cap\theta_{\alpha}]$,}

\noindent so that $Q_{\alpha}$ meets every non-empty open subset of
$\{0,1\}^M$ in a non-meager set.  Now every $H_{\alpha\iota}$ is
determined by coordinates in $I_{\alpha}\subseteq M$, so
we can express $\tilde g_{\alpha}$ as
$g^*_{\alpha}\pi_M$,
where $g^*_{\alpha}:\{0,1\}^M\to\{0,1\}^{\delta}$ is
Baire measurable in each coordinate.   If $y\in Q_{\alpha}$, take
$\xi\in V_{\alpha}\cap\theta$ such that $f_M(\xi)=y$;  then

\Centerline{$g_{\alpha}^*(y)=g_{\alpha}^*\pi_Mf(\xi)
=\tilde g_{\alpha}f(\xi)=hg_{\alpha}(\xi)\in h[Y]$.}

\noindent Thus
$g_{\alpha}^*\restr Q_{\alpha}\subseteq f_M[\theta]\times h[Y]$ for
every $\alpha\in S_3$, and we may apply 546Fc to $\{0,1\}^M$ and the family
$\langle g^*_{\alpha}\restr Q_{\alpha}\rangle_{\alpha\in S'}$,
where $S'\subseteq S_3$ is a set with cardinal $\kappa$,
to see that there are distinct $\alpha$, $\beta\in S_3$ such that
$\{y:y\in Q_{\alpha}\cap Q_{\beta}$, $g^*_{\alpha}(y)=g^*_{\beta}(y)\}$ is non-meager.
Now, however, consider

\Centerline{$E=\{y:y\in\{0,1\}^M$, $g^*_{\alpha}(y)=g^*_{\beta}(y)\}$.}

\noindent Then $E=\bigcap_{\iota<\delta}E_{\iota}$, where

\Centerline{$E_{\iota}
=\{y:y\in\{0,1\}^M$, $g^*_{\alpha}(y)(\iota)=g^*_{\beta}(y)(\iota)\}$}

\noindent is a Baire subset of $\{0,1\}^M$ for each $\iota<\delta$.
Because $\delta<\kappa$ and $\Cal I$ is $\kappa$-additive and $\omega_1$-saturated,

$$\eqalign{f_M^{-1}[E]^{\ssbullet}
&=(\bigcap_{\iota<\delta}f_M^{-1}[E_{\iota}])^{\ssbullet}
=\inf_{\iota<\delta}f_M^{-1}[E_{\iota}]^{\ssbullet}\cr
&=\inf_{\iota\in K}f_M^{-1}[E_{\iota}]^{\ssbullet}
=f_M^{-1}[\bigcap_{\iota\in K}E_{\iota}]^{\ssbullet}\cr}$$

\noindent for some countable $K\subseteq\delta$.   In this case,
$E'=\bigcap_{\iota\in K}E_{\iota}$ is a Baire set including $E$, and
$f_M^{-1}[E'\setminus E]\in\Cal I$;  since $E'$ includes the non-meager set
$\{y:y\in Q_{\alpha}\cap Q_{\beta}$, $g^*_{\alpha}(y)=g^*_{\beta}(y)\}$, $E'$ is
non-meager and $f_M^{-1}[E']\notin\Cal I$, by (d) above;
accordingly $f_M^{-1}[E]\notin\Cal I$.

Consequently

$$\eqalign{\{\xi:g_{\alpha}(\xi)=g_{\beta}(\xi)\}^{\ssbullet}
&=\{\xi:hg_{\alpha}(\xi)=hg_{\beta}(\xi)\}^{\ssbullet}\cr
&=\{\xi:\xi\in V_{\alpha}\cap V_{\beta},
  \,\tilde g_{\alpha}f(\xi)=\tilde g_{\beta}f(\xi)\}^{\ssbullet}\cr
&=\{\xi:g^*_{\alpha}\pi_M f(\xi)=g^*_{\beta}\pi_M f(\xi)\}^{\ssbullet}
=f^{-1}_M[E]^{\ssbullet}
\ne 0\cr}$$

\noindent in $\frak A$.   But this is
absurd, because in (b) above we chose
$g_{\alpha}$, $g_{\beta}$ in such a way that
$\{\xi:g_{\alpha}(\xi)=g_{\beta}(\xi)\}$ would be bounded in $\kappa$.\
\BanG\

Thus we have the required contradiction, and the theorem is true.
}%end of proof of 546G

\leader{546H}{}\cmmnt{ For the next step we need an elementary basic
fact.

\medskip

\noindent}{\bf Lemma} (a) A Boolean algebra\cmmnt{ $\frak A$} is
isomorphic to the category algebra $\frak G_{\omega}$
of $\{0,1\}^{\omega}$ iff it is Dedekind complete, atomless, has
countable $\pi$-weight and is not $\{0\}$.

(b) $\frak G_{\omega}$ is homogeneous.

(c) Every atomless order-closed subalgebra of $\frak G_{\omega}$ is
isomorphic to $\frak G_{\omega}$.

\proof{{\bf (a)(i)} All category algebras are Dedekind complete.
The algebra
$\Cal E$ of open-and-closed subsets of $\{0,1\}^{\omega}$ is countable and atomless
and isomorphic to an order-dense subalgebra of $\frak G_{\omega}$,
so $\frak G_{\omega}$ is atomless and has countable $\pi$-weight.

\medskip

\quad{\bf (ii)} If $\frak A$ satisfies the conditions,
let $B$ be a countable order-dense
subset of $\frak A$ and $\frak B$ the subalgebra of $\frak A$ generated
by $B$.   Then $\frak B$ is countable, atomless and not $\{0\}$,
so is isomorphic to $\Cal E$
(316M).   Now any isomorphism between
$\frak B$ and $\Cal E$ extends to an isomorphism
between their completions, which by 314Ub can be identified with
$\frak A$ and $\frak G_{\omega}$ respectively.

\medskip

{\bf (b)-(c)} All we have to observe is that any non-zero principal ideal
of $\frak G_{\omega}$,
and any atomless order-closed subalgebra of $\frak G_{\omega}$,
satisfy the conditions of (a) (see 514E);  or use 316P.
}%end of proof of 546H

\leader{546I}{Corollary} The category algebra $\frak G_{\omega}$
of $\{0,1\}^{\omega}$ is not a \pssqa.

\proof{ \Quer\ Suppose otherwise.   By 546C there are an uncountable
regular
cardinal $\kappa$ and a normal ideal $\Cal J^*$ on $\kappa$ such that
$\Cal P\kappa/\Cal J^*$ is isomorphic to an atomless $\sigma$-subalgebra
$\frak D$
of a principal ideal $(\frak G_{\omega})_c$ of $\frak G_{\omega}$.
As $\frak G_{\omega}$ is ccc and Dedekind complete, $\frak D$
is order-closed in $(\frak G_{\omega})_c$, and is itself Dedekind
complete.   Also

\Centerline{$\pi(\frak D)\le\pi((\frak G_{\omega})_c)
\le\pi(\frak G_{\omega})=\omega$}

\noindent (514Eb).   So $\frak D\cong\frak G_{\omega}$;
but by 546G this is impossible.\ \Bang
}%end of proof of 546I

\leader{546J}{}\cmmnt{ I now embark on an investigation of algebras of
the form $\frak A\tensorhat\frak G$, where $\frak A$ is a measure algebra,
$\frak G$ is a category algebra and $\frak A\tensorhat\frak G$ is the
Dedekind completion of the free product $\frak A\otimes\frak G$.   The
ideas here are based on {\smc Burke n96}, itself drawn from
{\smc Gitik \& Shelah 01}.
We have to begin with two lemmas which really refer to measure and
category in $[0,1]^2$.

\medskip

\noindent}{\bf Lemma} Let $X$ be a set, $\Sigma$ a $\sigma$-algebra of subsets of
$X$, and $Y$ a Polish space.
If $V\in\Sigma\tensorhat\Cal B(Y)$ and $V[\{x\}]$ is
non-meager for every $x\in X$, there is a measurable
function $f:X\to Y$ such that $(x,f(x))\in V$ for every $x\in X$.

\proof{ Let $\Cal W$ be the family of sets
expressible in the form $\bigcup_{n\in\Bbb N}E_n\times G_n$ where
$E_n\in\Sigma$ and
$G_n\subseteq Y$ is open for each $n$;  let $\Cal W^*$ be the set of those
$W\in\Cal W$ such that $W[\{x\}]$ is dense for every $x\in X$.   Note that
$W\cap W'\in\Cal W$ for all $W$,
$W'\in\Cal W$.   By 527I, there are $W$ and $\sequencen{W_n}$ such that

\inset{$W\in\Cal W$, $W_n\in\Cal W^*$ for every $n$,

$(V\symmdiff W)\cap\bigcap_{n\in\Bbb N}W_n=\emptyset$.}

\noindent Since every vertical section of $V$ is non-meager, every
vertical section
of $W$ is non-empty, while $W\cap\bigcap_{n\in\Bbb N}W_n\subseteq V$.

If $Y$ is empty, so is $X$, and the result is trivial.   Otherwise,
proceed as
follows.   Fix a complete metric $\rho$ on $Y$ defining its topology.
Set $V_0=W$.   Given that $V_n\in\Cal W$ and $V_n[\{x\}]$ is non-empty for
every $x\in X$, express $V_n$ as $\bigcup_{i\in\Bbb N}E_{ni}\times G_{ni}$ where
$E_{ni}\in\Sigma$ and $G_{ni}\subseteq Y$ is a non-empty open set for each
$i\in\Bbb N$.   Set $F_{ni}=E_{ni}\setminus\bigcup_{j<i}E_{nj}$ for each
$i$, so that $\sequence{i}{F_{ni}}$ is disjoint and
$\bigcup_{i\in\Bbb N}F_{ni}=\bigcup_{i\in\Bbb N}E_{ni}=X$.   Next, for each
$i\in\Bbb N$, let $H_{ni}$ be a non-empty open set of diameter at most
$2^{-n}$ such
that $\overline{H}_{ni}\subseteq G_{ni}$, and $y_{ni}$ a point of
$H_{ni}$;  then
$V'_n=\bigcup_{i\in\Bbb N}F_{ni}\times H_{ni}$ belongs to $\Cal W$, and
all its
vertical sections are non-empty.   Define $f_n:X\to Y$ by saying that
$f_n(x)=y_{ni}$ if $x\in F_{ni}$;  then $f_n$ is measurable.   If we now
take $V_{n+1}=V'_n\cap W_n$, $V_{n+1}$
again belongs to $\Cal W$, and its vertical sections are non-empty because the
vertical sections of $W_n$ are dense.   Continue.

At the end of the induction, observe that, for each $x\in X$ and
$m\le n\in\Bbb N$,
$f_n(x)\in V_n[\{x\}]\subseteq V'_m[\{x\}]$, so
$\rho(f_n(x),f_m(x))\le\diam V'_m[\{x\}]\le 2^{-m}$.   Accordingly
$\sequencen{f_n(x)}$ is a Cauchy sequence and converges in $Y$ to $f(x)$
say.   Moreover,

\Centerline{$f(x)\in\overline{V'_n[\{x\}]}
\subseteq V_n[\{x\}]\subseteq W_n[\{x\}]\cap W[\{x\}]$}

\noindent for each $n$,
so $(x,f(x))\in W\cap\bigcap_{n\in\Bbb N}W_n\subseteq V$.
By 418Ba, $f:X\to Y$ is measurable, so we have a suitable function.
}%end of proof of 546J

\leader{546K}{Lemma} Let $(X,\Sigma,\mu)$ be a probability space, $Y$ a
separable
metrizable space and $\nu$ a topological probability measure on $Y$.
Suppose we have a double sequence $\langle f_{kj}\rangle_{k,j\in\Bbb N}$ of
measurable functions from $X$ to $Y$
such that $\{x:f_{ki}(x)=f_{kj}(x)\}$ is negligible whenever $k\in\Bbb N$
and $i\ne j$.   Then for any $\epsilon>0$ there is a Borel set
$F\subseteq Y$ such that $\nu(Y\setminus F)\le\epsilon$ and
$\bigcap_{j\in\Bbb N}f_{kj}^{-1}[F]$ is negligible for every $k\in\Bbb N$.

\proof{{\bf (a)} Suppose that $\delta>0$ and $k\in\Bbb N$.
Then there is a Borel set $H\subseteq Y$ such that

\Centerline{$\nu(Y\setminus H)\le\delta$,
\quad$\mu(\bigcap_{i\in\Bbb N}f_{ki}^{-1}[H])\le\delta$.}

\noindent\Prf\ Set $\eta=\bover13\delta$ and take $m\in\Bbb N$ such
that $(1-\eta)^m\le\eta$.
Let $\sequencen{U_n}$ be a sequence running over a base for the topology of
$Y$.   For each $n$ let $\Cal F_n$ be the set of atoms in the
algebra of subsets of $Y$ generated by $\{U_i:i\le n\}$, and let $E_n$ be the set of
those $x\in X$ such that there are distinct $i$, $j<m$ such that
$f_{ki}(x)$ and $f_{kj}(x)$ belong to the same member of $\Cal F_n$.
Then $\sequencen{E_n}$ is a non-increasing sequence of measurable sets.
If $x\in\bigcap_{n\in\Bbb N}E_n$, then there must be $i<j<m$ such that
$\{n:f_{ki}(x)$ and $f_{kj}(x)$ belong to the same member of $\Cal F_n\}$ is infinite,
in which case $f_{ki}(x)=f_{kj}(x)$;  thus

\Centerline{$\bigcap_{n\in\Bbb N}E_n
\subseteq\bigcup_{i<j<m}\{x:f_{ki}(x)=f_{kj}(x)\}$}

\noindent is negligible.   There is therefore an $n\in\Bbb N$ such that
$\mu E_n\le\eta$.

Give $\{0,1\}^{\Cal F_n}$ the product measure
$\lambda$ in which each copy of $\{0,1\}$ is given
the measure assigning measure $\eta$ to $\{0\}$ and $1-\eta$ to $\{1\}$.   For
$w\in\{0,1\}^{\Cal F_n}$ set

\Centerline{$H_w=\bigcup\{F:F\in\Cal F_n$, $w(F)=1\}$,}

\Centerline{$g(w)=\nu(Y\setminus H_w)$,
\quad$h(w)=\mu(\bigcap_{i<m}f_{ki}^{-1}[H_w])$.}

\noindent Then

\Centerline{$\int g\,d\lambda=1-\sum_{F\in\Cal F_n}\nu F\cdot\lambda\{w:w(F)=1\}
=1-(1-\eta)\sum_{F\in\Cal F_n}\nu F=\eta$.}

\noindent Next, for each $x\in X\setminus E_n$,

$$\eqalign{\lambda\{w:x\in\bigcap_{i<m}f_{ki}^{-1}[H_w]\}
&=\lambda\{w:w(F)=1\text{ whenever }i<m\text{ and }f_{ki}(x)\in F\in\Cal F_n\}\cr
&=(1-\eta)^m\cr}$$

\noindent because, by the definition of $E_n$, all the $f_{ki}(x)$, for different
$i<m$, belong to different members of $\Cal F_n$.   Since the set

\Centerline{$\{(x,w):w\in\{0,1\}^{\Cal F_n},\,x\in\bigcap_{i<m}f_{ki}^{-1}[H_w]\}
=\bigcap_{i<m}\bigcup_{F\in\Cal F_n}(f_{ki}^{-1}[F]\times\{w:w(F)=1\})$}

\noindent belongs to $\Sigma\tensorhat\Cal B(\{0,1\}^{\Cal F_n})$,
Fubini's theorem tells us that

\Centerline{$\int h\,d\lambda
=\int\lambda\{w:x\in\bigcap_{i<m}f_{ki}^{-1}[H_w]\}\mu(dx)
\le\mu E_n+(1-\eta)^m\le 2\eta$.}

Accordingly $\int g+h\,d\lambda\le 3\eta$ and there is a
$w\in\{0,1\}^{\Cal F_n}$ such
that $g(w)+h(w)\le 3\eta=\delta$.   Looking back at the definitions, we see that
$H_w$ will serve for $H$, with something to spare.\ \Qed

\medskip

{\bf (b)} Now all we need to do is take a double sequence
$\langle\delta_{kl}\rangle_{k,l\in\Bbb N}$ of strictly positive numbers
such that $\sum_{k,l\in\Bbb N}\delta_{kl}\le\epsilon$
and a corresponding family $\langle H_{kl}\rangle_{k,l\in\Bbb N}$ of
Borel subsets of $Y$ such that, for each $k$ and $l$,

\Centerline{$\nu(Y\setminus H_{kl})\le\delta_{kl}$,
\quad$\mu(\bigcap_{i\in\Bbb N}f_{ki}^{-1}[H_{kl}])\le\delta_{kl}$,}

\noindent and set $F=\bigcap_{k,l\in\Bbb N}H_{kl}$.
}%end of proof of 546K

\leader{546L}{}\cmmnt{ We come now to the central step, in which the results of the
last two lemmas are adapted to give a key to the structure of the algebras
$\frak A\tensorhat\frak G$ we are investigating.

\medskip

\noindent}{\bf Lemma} Let $(\frak A,\bar\mu)$ be a probability
algebra and $\frak B$ a Dedekind complete Boolean
algebra with countable $\pi$-weight, not $\{0\}$.   Let
$\frak C$ be the Dedekind completion of $\frak A\otimes\frak B$, and write
$\varepsilon_1:\frak A\to\frak A\otimes\frak B\subseteq\frak C$,
$\varepsilon_2:\frak B\to\frak A\otimes\frak B\subseteq\frak C$ for the canonical
order-continuous Boolean homomorphisms.   For $c\in\frak C$ write
${\upr}_1c=\inf\{a:a\in\frak A$, $c\Bsubseteq\varepsilon_1(a)\}$.

Let $\frak D$ be an
order-closed subalgebra of $\frak C$ such that there is
no non-zero principal ideal of
$\frak D$ which is measurable with Maharam type at most $\tau(\frak A)$.

(a) If $a\in\frak A$ and $b\in\frak B$ are non-zero, there is a
$d\in\frak D$ such that

\Centerline{$a\Bcap{\upr}_1(\varepsilon_2(b)\Bcap d)
\Bcap{\upr}_1(\varepsilon_2(b)\Bsetminus d)\ne 0$.}

(b) There is a countable set $D\subseteq\frak D$ such that

\Centerline{$\sup_{d\in D}({\upr}_1(\varepsilon_2(b)\Bcap d)
\Bcap{\upr}_1(\varepsilon_2(b)\Bsetminus d))=1$}

\noindent in $\frak A$, for every $b\in\frak B\setminus\{0\}$.

(c) For any $d\in\frak D$, $b_0,\ldots,b_m\in\frak B$ and $\epsilon>0$
there is a $d'\in\frak D$ such that $d'\Bsubseteq d$ and

\Centerline{$\bar\mu({\upr}_1(\varepsilon_2(b_k)\Bcap d'))
\ge\bar\mu({\upr}_1(\varepsilon_2(b_k)\Bcap d))-\epsilon$,}

\Centerline{$\bar\mu({\upr}_1(\varepsilon_2(b_k)\Bcap d\Bsetminus d'))
\ge\bar\mu({\upr}_1(\varepsilon_2(b_k)\Bcap d))-\epsilon$}

\noindent for every $k\le m$.

(d) $\frak D$ has a non-zero principal ideal with an atomless order-closed subalgebra
with countable $\pi$-weight.

\proof{{\bf (a)} \Quer\ Otherwise, set
$\pi d=a\Bcap{\upr}_1(\varepsilon_2(b)\Bcap d)$ for $d\in\frak D$.   Then
$\pi$ is a
function from $\frak D$ to the principal ideal $\frak A_a$ generated by
$a$.
Certainly $\pi(\sup D)=\sup\pi[D]$ for every non-empty $D\subseteq\frak D$
(cf.\ 313Sb), and $\pi 1=1$.
The counter-hypothesis is that $\pi d\Bcap\pi(1\Bsetminus d)=0$ for
every $d\in\frak D$;  but this is enough to make $\pi$ an order-continuous Boolean
homomorphism (312H(iv), 313L(b-iv)).   Setting

\Centerline{$d_0=\inf\{d:a\otimes b\Bsubseteq d\in\frak D\}
=1\Bsetminus\sup\{d:\pi d=0\}$,}

\noindent $\pi\restrp\frak D_{d_0}$ is an injective
order-continuous Boolean homomorphism from the principal ideal
$\frak D_{d_0}$ into $\frak A_a$, so is an isomorphism between
$\frak D_{d_0}$ and a closed subalgebra of $\frak A_a$ (314F(a-i)).
But $\frak A_a$ and all its
closed subalgebras are measurable algebras with Maharam types at most
$\tau(\frak A)$ (331Hc, 332Tb), so $\frak D_{d_0}$ is a measurable
algebra with Maharam type at most $\tau(\frak A)$, which is supposed to be
impossible.\ \Bang

\medskip

{\bf (b)} For each non-zero $b\in\frak B$, (a) tells us that

\Centerline{$\sup_{d\in\frak D}
  {\upr}_1(\varepsilon_2(b)\Bcap d)\Bcap{\upr}_1(\varepsilon_2(b)\Bsetminus d)=1$.}

\noindent Because
$\frak A$ is ccc, there is a countable $D_b\subseteq\frak D$ such that

\Centerline{$\sup_{d\in D_b}
  {\upr}_1(\varepsilon_2(b)\Bcap d)\Bcap{\upr}_1(\varepsilon_2(b)\Bsetminus d)=1$.}

\noindent Let $B$ be a countable order-dense subset of
$\frak B\setminus\{0\}$ and set
$D=\bigcup_{b\in B}D_b$.   If $b\in\frak B\setminus\{0\}$ there is a $b'\in B$ such
that $b'\Bsubseteq b$, and in this case

$$\eqalign{\sup_{d\in D}{\upr}_1(\varepsilon_2(b)\Bcap d)
  &\Bcap{\upr}_1(\varepsilon_2(b)\Bsetminus d)\cr
&\Bsupseteq\sup_{d\in D_{b'}}
  {\upr}_1(\varepsilon_2(b')\Bcap d)\Bcap{\upr}_1(\varepsilon_2(b')\Bsetminus d)=1.\cr}$$

\wheader{546L}{6}{2}{2}{48pt}

{\bf (c)} (The hard part.)

\medskip

\quad{\bf (i)} Let $(X,\Sigma,\mu)$ be a probability space with measure
algebra isomorphic to $(\frak A,\bar\mu)$ (321J);
for $E\in\Sigma$ write $E^{\ssbullet}$ for
the corresponding element of $\frak A$.
Next, let $\frak B_0$ be a countable order-dense subalgebra of $\frak B$,
and $Y$ the Stone space of $\frak B_0$, so that
$Y$ is a compact metrizable space,
and we can identify $\frak B_0$ with the algebra of
open-and-closed subsets of $Y$.
In this case, $\frak B$ is isomorphic to the
regular open algebra of $Y$ (see 314T)
and can be identified with the category algebra
of $Y$ (514If).   For $H\in\Cal B(Y)$ let $H^{\ssbullet}$ be the
corresponding member of $\frak B$.

\medskip

\quad{\bf (ii)} Write $\Cal L$ for
$(\Sigma\tensorhat\Cal B(Y))\cap(\Cal N(\mu)\ltimes\Cal M(Y))$.
By 527O, we can identify $\frak C$ with
$(\Sigma\tensorhat\Cal B(Y))/\Cal L$,
the canonical embeddings $\varepsilon_1$ and
$\varepsilon_2$ corresponding to the maps
$E^{\ssbullet}\mapsto(E\times Y)^{\ssbullet}$ and
$F^{\ssbullet}\mapsto(X\times F)^{\ssbullet}$.
Let $\Cal W$ be the family of subsets of $X\times Y$ expressible in the
form $\bigcup_{n\in\Bbb N}E_n\times G_n$ where $E_n\in\Sigma$ and
$G_n\subseteq Y$ is open for every $n\in\Bbb N$.
By 527I, every member of $\frak C$ can be expressed as
$W^{\ssbullet}$ for some $W\in\Cal W$.

If $W\in\Cal W$, then ${\upr}_1(W^{\ssbullet})=\pi_1[W]^{\ssbullet}$, where
$\pi_1(x,y)=x$ for $x\in X$ and $y\in Y$.
\Prf\ If $E$, $F\in\Sigma$ and $H\subseteq Y$ is a non-meager Borel set, then

$$\eqalign{(E\times H)^{\ssbullet}\Bsubseteq\varepsilon_1(F^{\ssbullet})
&\iff(E\times H)\setminus(F\times Y)\in\Cal L
\iff(E\setminus F)\times H\in\Cal L\cr
&\iff E\setminus F\text{ is negligible}
\iff E^{\ssbullet}\Bsubseteq F^{\ssbullet}.\cr}$$

\noindent So ${\upr}_1((E\times H)^{\ssbullet})=E^{\ssbullet}$ in $\frak A$.   Now if
$W=\bigcup_{n\in\Bbb N}E_n\times G_n$, where $E_n\in\Sigma$ and
$G_n$ is open for each $n$, then

$$\eqalign{{\upr}_1(W^{\ssbullet})
&={\upr}_1(\sup_{n\in\Bbb N}(E_n\times G_n)^{\ssbullet}))
=\sup_{n\in\Bbb N}{\upr}_1((E_n\times G_n)^{\ssbullet}))\cr
&=\sup_{n\in\Bbb N,G_n\ne\emptyset}E_n^{\ssbullet}
=\pi_1[W]^{\ssbullet}. \text{ \Qed}\cr}$$

\medskip

\quad{\bf (iii)} Take a countable set $D\subseteq\frak D$ as in (b) above, and
let $\sequencen{d_n}$ run over $D$.
For each $n\in\Bbb N$ choose $V_n$, $W_n\in\Cal W$ such
that $V_n^{\ssbullet}=d_n$ and $W_n^{\ssbullet}=1\Bsetminus d_n$.   Then

$$\eqalign{E^*_0
&=\bigcup_{n\in\Bbb N}\bigl(
  \{x:x\in X,\,V_n[\{x\}]\cap W_n[\{x\}]\ne\emptyset\}\cr
&\mskip150mu\cup\{x:x\in X,\,V_n[\{x\}]\cup W_n[\{x\}]
  \text{ is not dense}\}\bigr)\cr
&=\bigcup_{n\in\Bbb N}
  \bigl(\{x:x\in X,\,V_n[\{x\}]\cap W_n[\{x\}]\text{ is not meager}\}\cr
&\mskip150mu\cup\{x:x\in X,\,V_n[\{x\}]\cup W_n[\{x\}]
  \text{ is not comeager}\}\bigr)\cr}$$

\noindent is negligible.   If $G\subseteq Y$ is a non-empty open set, then

\Centerline{$\bigcup_{n\in\Bbb N}
  \bigl(\pi_1[V_n\cap(X\times G)]\cap\pi_1[W_n\cap(X\times G)]\bigr)$}

\noindent is measurable and conegligible.   \Prf\ Set
$b=G^{\ssbullet}\in\frak B$.   Then

$$\eqalign{&\bigl(\bigcup_{n\in\Bbb N}
  (\pi_1[V_n\cap(X\times G)]
  \cap\pi_1[W_n\cap(X\times G)])\bigr)^{\ssbullet}\cr
&\mskip200mu=\sup_{n\in\Bbb N}
  ({\upr}_1(\varepsilon_2(b)\Bcap d_n)
    \Bcap{\upr}_1(\varepsilon_2(b)\Bsetminus d_n))
=1.\text{ \Qed}\cr}$$

\noindent In fact

\Centerline{$X_0
=\bigcap_{G\subseteq Y\text{ is open and not empty}}
\bigcup_{n\in\Bbb N}
  \bigl(\pi_1[V_n\cap(X\times G)]\cap\pi_1[W_n\cap(X\times G)]\bigr)$}

\noindent is measurable and conegligible.
\Prf\ Let $\sequence{m}{U_m}$ run over a base for the topology
of $Y$ consisting of non-empty sets.   If $G\subseteq Y$ is open and not empty, then
there is an $m\in\Bbb N$ such that $G\supseteq U_m$, so that

$$\eqalign{\bigcup_{n\in\Bbb N}
  \bigl(\pi_1[V_n\cap(X\times G)]&\cap\pi_1[W_n\cap(X\times G)]\bigr)\cr
&\supseteq\bigcup_{n\in\Bbb N}
  \bigl(\pi_1[V_n\cap(X\times U_m)]\cap\pi_1[W_n\cap(X\times U_m)]\bigr).\cr}$$

\noindent This means that

\Centerline{$X_0=\bigcap_{m\in\Bbb N}
  \bigcup_{n\in\Bbb N}\bigl(\pi_1[V_n\cap(X\times U_m)]
    \cap\pi_1[W_n\cap(X\times U_m)]\bigr)$}

\noindent is measurable and conegligible.\ \Qed

Set $X_1=X_0\setminus E_0^*$, so that $X_1$ also is measurable and conegligible.

\medskip

\quad{\bf (iv)} Write $Z$ for $\{0,1\}^{\Bbb N}$.
Define $h:X\times Y\to Z$ by
setting $h(x,y)(n)=1$ if $(x,y)\in V_n$, $0$ otherwise.   If
$H\in\Cal B(Z)$, then
$h^{-1}[H]$ belongs to the $\sigma$-algebra generated by
$\{V_n:n\in\Bbb N\}$, so
$h^{-1}[H]^{\ssbullet}\in\frak D$.
If $W\in\Cal W$ is such that $\pi_1[W]\supseteq X_1$, and $f:X\to Y$ is any
measurable
function, there is a $W'\in\Cal W$ such that $W'\subseteq W$, $\pi_1[W]\supseteq X_1$
and $h(x,y)\ne h(x,f(x))$ whenever $x\in X_1$ and $(x,y)\in W'$.   \Prf\ Set

\Centerline{$E_n=\pi_1[W\cap V_n]\cap\pi_1[W\cap W_n]$}

\noindent for each $n\in\Bbb N$.   Then $E_n$ is measurable (because $W\cap V_n$ and
$W\cap W_n$ belong to $\Cal W$) and for every
$x\in X_1$ there must be some $n$ such that

\Centerline{$x
\in\pi_1[V_n\cap(X\times W[\{x\}])]\cap\pi_1[W_n\cap(X\times W[\{x\}])]$,}

\noindent in which case $x\in E_n$.   Accordingly we can find a partition
$\sequencen{E'_n}$ of $X_1$ into measurable sets such that
$E'_n\subseteq E_n$ for each $n$.   Next, set

\Centerline{$F_n=\{x:x\in E_n$, $(x,f(x))\in W_n\}$ for each $n\in\Bbb N$,}

\Centerline{$\tilde W
=\bigcup_{n\in\Bbb N}(((E'_n\cap F_n)\times Y)\cap V_n)
   \cup(((E'_n\setminus F_n)\times Y)\cap W_n)$,}

\Centerline{$W'=W\cap\tilde W$.}

\noindent Then $W'\in\Cal W$.
If $x\in X_1$, there is an $n\in\Bbb N$ such that $x\in E'_n$.   In this case,
$\tilde W[\{x\}]=V_n[\{x\}]\cup W_n[\{x\}]$ is dense, so meets $W[\{x\}]$, and
$W'[\{x\}]$ is non-empty.   If now $(x,y)\in W'$, then,
because $V_n[\{x\}]\cap W_n[\{x\}]$ is empty,

$$\eqalign{h(x,f(x))(n)=1
&\iff (x,f(x))\in V_n\iff x\notin F_n\cr
&\iff (x,y)\in W_n\iff h(x,y)(n)=0,\cr}$$

\noindent so $h(x,f(x))\ne h(x,y)$.   So $W'$ serves.\ \Qed

\medskip

\quad{\bf (v)} If $f:X\to Y$ is measurable, then
$x\mapsto(x,f(x)):X\to X\times Y$ is
$(\Sigma,\Sigma\tensorhat\Cal B(Y))$-measurable, because
$\{x:(x,f(x))\in E\times H\}$ belongs to $\Sigma$ whenever $E\in\Sigma$ and $H\in\Cal B(Y)$.
So if we set $\tilde f(x)=h(x,f(x))$
for each $x$, $\tilde f:X\to Z$ is again measurable.

At this point, take the $d$ and $b_0,\ldots,b_m$ and $\epsilon$
of the statement of this part of the lemma.
Let $W\in\Cal W$ be such that $W^{\ssbullet}=d$.
Write $R$ for the family of measurable functions $f:X\to Y$ such
that $(x,f(x))\in\bigcap_{n\in\Bbb N}V_n\cup W_n$ for every $x\in X_1$.
For each $k\le m$, let $G_k\subseteq Y$ be the regular open set such that
$G_k^{\ssbullet}=b_k$ in $\frak B$, and write $R_k$
for the set of functions $f\in R$ such that $f(x)\in G_k$ and
$(x,f(x))\in W$ whenever $x\in\pi_1[W\cap(X_1\times G_k)]$.

\medskip

\quad{\bf (vi)} Now, for each $k\le m$,
there is a sequence $\sequence{i}{f_{ki}}$ in $R_k$
such that $\tilde f_{ki}(x)\ne\tilde f_{kj}(x)$ whenever $i<j$ and
$x\in X_1$.   \Prf\ Choose the sequence inductively.   Given $f_{ki}$ for
$i<j$, set

\Centerline{$W'_0
=(W\cap(X\times G_k))\cup((X\setminus\pi_1[W\cap(X\times G_k)])\times Y)$.}

\noindent Then $W'_0$ belongs to $\Cal W$ and its vertical sections are non-empty.
By (iv), we can choose $W'_1,\ldots,W'_j$ such that, for each $i<j$,
$W'_{i+1}\in\Cal W$, $W'_{i+1}\subseteq W'_i$,
$\pi_1[W'_{i+1}]\supseteq X_1$ and
$h(x,y)\ne h(x,f_{ki}(x))$ whenever $x\in X_1$ and $(x,y)\in W'_{i+1}$.
By 546J, applied to
$W'_j\cup((X\setminus X_1)\times Y)\in\Sigma\tensorhat\Cal B(Y)$,
we can find a measurable function $f_{kj}:X\to Y$ such that
$(x,f_{kj}(x))\in W'_j\cap\bigcap_{n\in\Bbb N}(V_n\cup W_n)$
for every $x\in X_1$.   Now, if $x\in X_1$ and $i<j$,
$(x,f_{kj}(x))\in W'_{i+1}$ so

\Centerline{$\tilde f_{kj}(x)=h(x,f_{kj}(x))\ne h(x,f_{ki}(x))=\tilde f_{ki}(x)$.}

\noindent So this sequence works.\ \QeD\
For completeness in the next step,
set $f_{ki}=f_{0i}$ if $k>m$ and $i\in\Bbb N$.

\wheader{546L}{4}{2}{2}{30pt}

\quad{\bf (vii)} Let $\nu$ be the Borel probability measure on $Z$
defined by saying

\Centerline{$\nu H=\Bover1{m+1}\sum_{k=0}^m\mu\tilde f_{k0}^{-1}[H]$}

\noindent for every $H\in\Cal B(Z)$.   By 546K there is a Borel set
$H\subseteq Z$ such that

\Centerline{$\nu(Z\setminus H)<\Bover{\epsilon}{m+1}$,
\quad$\mu(\bigcap_{i\in\Bbb N}\tilde f_{ki}^{-1}[H])=0$ for every $k\le m$.}

\noindent Being a Borel probability measure on a compact metrizable space,
$\nu$ must be inner
regular with respect to the compact sets (433Ca);
let $K\subseteq H$ be a compact
set such that $\nu(Z\setminus K)\le\Bover{\epsilon}{m+1}$.   In this case,
$\mu\tilde f_{k0}^{-1}[K]\ge 1-\epsilon$ for every $k\le m$.
Next, there is a non-increasing sequence $\sequencen{H_n}$ of open-and-closed sets
in $Z$ with intersection $K$, so that
$\bigcap_{n\in\Bbb N}\bigcap_{i\in\Bbb N}\tilde f_{ki}^{-1}[H_n]$ is negligible for
each $k$, and there is a $p\in\Bbb N$ such that
$\mu(\bigcap_{i\in\Bbb N}\tilde f_{ki}^{-1}[H_p])\le\epsilon$ for every $k\le m$.
Let $r\in\Bbb N$ be such that $H_p$ is determined by coordinates less than $r$.
Set

\Centerline{$V=(X_1\times Y)\cap\bigcap_{n\le r}(V_n\cup W_n)\cap h^{-1}[H_p]$,}

\Centerline{$\tilde V=(X_1\times Y)\cap\bigcap_{n\le r}(V_n\cup W_n)\setminus h^{-1}[H_p]$;}

\noindent then $V$ and $\tilde V$ both belong to $\Cal W$, $V\cap\tilde V=\emptyset$,
$V^{\ssbullet}$ and $\tilde V^{\ssbullet}$ both belong to $\frak D$, and
$V^{\ssbullet}\Bcup\tilde V^{\ssbullet}=1$ in $\frak D$.

Set $d'=(W\cap V)^{\ssbullet}$, so that $d'\in\frak D$ and $d'\Bsubseteq d$.

\medskip

\quad{\bf (viii)} For each $k\le m$,

\Centerline{$(X\times G_k)\cap W\in\Cal W$,
\quad$\varepsilon_2(b_k)\Bcap d=((X\times G_k)\cap W)^{\ssbullet}$,}

\Centerline{$\bar\mu({\upr}_1(\varepsilon_2(b_k)\Bcap d))=\mu\pi_1[W\cap(X\times G_k)]
=\mu\pi_1[W\cap(X_1\times G_k)]$,}

\Centerline{$\bar\mu({\upr}_1(\varepsilon_2(b_k)\Bcap d'))
=\mu\pi_1[W\cap V\cap(X_1\times G_k)]$}

\noindent ((ii) above),

\Centerline{$\bar\mu({\upr}_1(\varepsilon_2(b_k)\Bcap d\Bsetminus d'))
=\mu\pi_1[W\cap\tilde V\cap(X\times G_k)]$.}

$\pi_1[W\cap(X_1\times G_k)]\setminus\pi_1[W\cap V\cap(X_1\times G_k)]$ does not meet $\tilde f_{k0}^{-1}[K]$.
\Prf\ If $x\in\pi_1[W\cap(X_1\times G_k)]\setminus\pi_1[W\cap V\cap(X_1\times G_k)]$,
then $(x,f_{k0}(x))\notin W\cap V\cap(X_1\times G_k)$;  as $f_{k0}\in R_k$,
$(x,f_{k0}(x))\in W\cap\bigcap_{n\in\Bbb N}(V_n\cup W_n)\cap(X_1\times G_k)$, so $(x,f_{k0}(x))\notin h^{-1}[H_p]$
and $\tilde f_{k0}(x)\notin H_p$.   As $K\subseteq H_p$, $x\notin\tilde f_{k0}^{-1}[K]$.\ \QeD\  It follows that

$$\eqalign{\bar\mu(\upr_1(\epsilon_2(b_k)\Bcap d))
&=\mu\pi_1[W\cap(X_1\times G_k)]\cr
&\le(1-\mu\tilde f_{k0}^{-1}[K])+\mu\pi_1[W\cap V\cap(X_1\times G_k)]\cr
&\le\bar\mu(\upr_1(\epsilon_2(b_k)\Bcap d'))+\epsilon.\cr}$$

On the other side,
$\pi_1[W\cap(X_1\times G_k)]\subseteq\pi_1[W\cap\tilde V\cap(X_1\times G_k)]\cup\bigcap_{i\in\Bbb N}\tilde f_{ki}^{-1}[H_p]$.
\Prf\ This time, if $x\in\pi_1[W\cap(X_1\times G_k)]\setminus\bigcap_{i\in\Bbb N}\tilde f_{ki}^{-1}[H_p]$,
let $i\in\Bbb N$ be such that $h(x,f_{ki}(x))\notin H_p$.   Since $f_{ki}\in R_k$,
$(x,f_{ki}(x))\in W\cap(X_1\times Y)\cap\bigcap_{n\in\Bbb N}(V_n\cup W_n)\setminus h^{-1}[H_p]$ and
$x\in\pi_1[W\cap\tilde V\cap(X_1\times G_k)]$.\ \QeD\  So we get

$$\eqalign{\bar\mu(\upr_1(\epsilon_2(b_k)\Bcap d))
&\le\mu(\bigcap_{i\in\Bbb N}\tilde f_{ki}^{-1}[H_p])+\mu\pi_1[W\cap\tilde V\cap(X_1\times G_k)]\cr
&\le\bar\mu(\upr_1(\epsilon_2(b_k)\Bcap d\Bsetminus d'))+\epsilon.\cr}$$

\noindent This is true for every $k\le m$, so we have found an appropriate $d'$.

\medskip

{\bf (d)} Let $\sequencen{b_n}$ be a sequence running over an order-dense subset of
$\frak B$, starting with $b_0=1$.
Choose $d_{\sigma}$, for $\sigma\in S_2=\bigcup_{n\in\Bbb N}\{0,1\}^n$, as
follows.   $d_{\emptyset}=1$.   Given $d_{\sigma}\in\frak D$, where
$\sigma\in\{0,1\}^n$, use (c) to find $d_{\sigma}'\in\frak D$ such that
$d_{\sigma}'\Bsubseteq d_{\sigma}$ and

\Centerline{$\bar\mu\bigl({\upr}_1(d_{\sigma}\cap\varepsilon_2(b_k))
  \Bsetminus({\upr}_1(d_{\sigma}'\Bcap\varepsilon_2(b_k))
  \Bcap{\upr}_1(d_{\sigma}\Bcap\varepsilon_2(b_k)\Bsetminus d_{\sigma}'))\bigr)
\le 4^{-n-2}$}

\noindent whenever $k\le n$.   Now, writing $\sigma^{\smallfrown}\fraction{0}$ and
$\sigma^{\smallfrown}\fraction{1}$ for the two members of $\{0,1\}^{n+1}$ extending $\sigma$,
set $d_{\sigma^{\smallfrown}\fraction{0}}=d'_{\sigma}$ and
$d_{\sigma^{\smallfrown}\fraction{1}}=d_{\sigma}\Bsetminus d'_{\sigma}$, and continue.

At the end of the construction,

\Centerline{$a_0=\sup_{\sigma\in S_2,k\le\#(\sigma)}
  \upr_1(d_{\sigma}\cap\varepsilon_2(b_k))
  \Bsetminus(\upr_1(d_{\sigma^{\smallfrown}\fraction{0}}\Bcap\varepsilon_2(b_k))
    \Bcap\upr_1(d_{\sigma^{\smallfrown}\fraction{1}}\Bcap\varepsilon_2(b_k)))$}

\noindent has measure at most

\Centerline{$\sum_{n=0}^{\infty}2^n\,\sum_{k=0}^n4^{-n-2}
=\sum_{n=0}^{\infty}(n+1)2^{-n-4}=\Bover14$,}

\noindent so $a=1\Bsetminus a_0$ is non-zero.   Now an easy induction on $n$ shows
that $a\Bcap{\upr}_1(d_{\sigma}\Bcap\varepsilon_2(b_k))
=a\Bcap{\upr}_1(d_{\tau}\Bcap\varepsilon_2(b_k))$ whenever $k\le m\le n$,
$\sigma\in\{0,1\}^m$, $\tau\in\{0,1\}^n$ and $\sigma\Bsubseteq\tau$.
In particular,
$a\Bsubseteq{\upr}_1(d_{\sigma})$ for every $\sigma\in S_2$.

Set $d^*=\inf\{d:d\in\frak D,\,d\Bsupseteq\varepsilon_1(a)\}$.   Then
$a\Bcap{\upr}_1(d^*\Bcap d)=a\Bcap{\upr}_1(d)$ for every $d\in\frak D$;
in particular,
$a\Bsubseteq\penalty-100{\upr}_1(d^*\Bcap d_{\sigma})$ and
$d^*\Bcap d_{\sigma}\ne 0$ for every
$\sigma\in S_2$.   Writing $\Cal E$ for the algebra of open-and-closed subsets of $Z$,
and $U_{\sigma}=\{z:\sigma\subseteq z\in Z\}$ for $\sigma\in S_2$, we
have an injective Boolean homomorphism $\phi$ from $\Cal E$ to the
principal ideal $\frak D_{d^*}$ defined by
setting $\phi U_{\sigma}=d_{\sigma}\Bcap d^*$ for every $\sigma\in S_2$.
Now $\phi$ is order-continuous.   \Prf\Quer\ Otherwise, there is a set
$T\subseteq S_2$ such that $\sup_{\tau\in T}U_{\tau}=1$ in $\Cal E$, that is,
$\bigcup_{\tau\in T}U_{\tau}$ is dense in $Z$, but
$d^*\notBsubseteq\sup_{\tau\in T}d_{\tau}$.   In this case
$\varepsilon_1(a)\notBsubseteq\sup_{\tau\in T}d_{\tau}$, so there must be a non-zero
$a'\Bsubseteq a$ and a $k\in\Bbb N$ such that
$\varepsilon_1(a')\Bcap\varepsilon_2(b_k)\Bcap d_{\tau}=0$ for every $\tau\in T$.
Now, however,
there is surely a $\sigma\in\{0,1\}^k$ such that
$\varepsilon_1(a')\Bcap\varepsilon_2(b_k)\Bcap d_{\sigma}\ne 0$, in which case there is a
$\tau\in T$ such that $\tau\supseteq\sigma$, so that

$$\eqalign{0
&=a'\Bcap{\upr}_1(d_{\tau}\Bcap\varepsilon_2(b_k))
=a'\Bcap a\Bcap{\upr}_1(d_{\tau}\Bcap\varepsilon_2(b_k))\cr
&=a'\Bcap a\Bcap{\upr}_1(d_{\sigma}\Bcap\varepsilon_2(b_k))
=a'\Bcap{\upr}_1(d_{\sigma}\Bcap\varepsilon_2(b_k))
\ne 0,\cr}$$

\noindent which is absurd.\ \Bang\Qed

Accordingly $\phi$ has an extension to an order-continuous embedding
of the Dedekind
completion $\frak G$ of $\Cal E$ into $\frak D_{d^*}$ (314Tb),
and the image of
$\frak G$ in $\frak D_{d^*}$ is an atomless order-closed subalgebra of
$\frak D_{d^*}$ with countable $\pi$-weight.
}%end of proof of 546L

\vleader{48pt}{546M}{Theorem}
Let $\frak A$ be a measurable algebra and $\frak B$ a
Dedekind complete Boolean algebra with countable $\pi$-weight.   Let
$\frak C$ be the Dedekind completion of $\frak A\otimes\frak B$ and
$\frak D$ an order-closed subalgebra of $\frak C$.   Then there is a
$d^*\in\frak D$ such that $\frak D_{d^*}$ is a measurable algebra,
with Maharam type at
most $\max(\omega,\tau(\frak A))$, and $\frak D_{1\Bsetminus d^*}$ has
an atomless order-closed subalgebra of countable $\pi$-weight.

\proof{{\bf (a)} Let $D_0$ be the set of those $d\in\frak D$ such that
$\frak D_d$ is
measurable with Maharam type at most $\tau(\frak A)$, and $D_1$ the set of those
$d\in\frak D$ such that $\frak D_d$ has an atomless order-closed subalgebra of
countable $\pi$-weight.
Then $D_0\cup D_1$ is order-dense in $\frak D$.   \Prf\ Take
$d_0\in\frak D\setminus\{0\}$.   If there is a is non-zero $d\in D_0$ such that
$d\Bsubseteq d_0$, we can stop.   Otherwise, there are non-zero $a\in\frak A$,
$b\in\frak B$ such that $c=a\otimes b\Bsubseteq d_0$.   Now the principal ideal
$(\frak A\otimes\frak B)_c$ is isomorphic to the
free product $\frak A_a\otimes\frak B_b$, so the principal ideal
$\frak C_c$ can be
identified with the Dedekind completion of $\frak A_a\otimes\frak B_b$.
Consider
$\frak D_c=\{d\Bcap c:d\in\frak D\}$;  this is an order-closed subalgebra of
$\frak C_c$, and is isomorphic to the principal ideal of $\frak D$ generated by
$\upr(c,\frak D)=\inf\{d:c\Bsubseteq d\in\frak D\}\Bsubseteq d_0$.
In particular, none of the non-zero principal
ideals of $\frak D_c$ can be measurable algebras with Maharam type less than or equal to
$\tau(\frak A)$.

Since $\frak A_a$ is a non-zero measurable algebra, there is a functional $\bar\mu$
such that $(\frak A_a,\bar\mu)$ is a probability algebra;
while $\frak B_b$ is Dedekind complete and has countable $\pi$-weight.
So 546Ld tells us that
there is a non-zero principal ideal $(\frak D_c)_e$ of
$\frak D_c$ with an atomless order-closed subalgebra of countable $\pi$-weight.
Copying this into $\frak D_{\upr(c,\frak D)}$, we get a non-zero
$d_1\Bsubseteq d_0$ such that $\frak D_{d_1}$ has an
atomless order-closed subalgebra of countable $\pi$-weight, that is,
$d_1\in D_1$.\ \Qed

\medskip

{\bf (b)} Accordingly there is a partition of unity
$\langle d_i\rangle_{i\in I}$
in $\frak D$ such that every $d_i$ belongs to $D_0\cup D_1$.   Set

\Centerline{$J=\{i:i\in I$, $d_i\in D_0\}$,
\quad$K=I\setminus J$,
\quad$d^*=\sup_{i\in J}d_i$.}

\noindent Because $\frak C$ and $\frak D$ are ccc (525Tb, 516U, 514Ea),
$I$, $J$ and $K$ are
countable.   Now $\frak D_{d^*}\cong\prod_{i\in J}\frak D_{d_i}$ is
measurable (391Ca, or otherwise), and its Maharam type is at most
$\max(\omega,\sup_{i\in J}\tau(\frak D_{d_i}))\le\max(\omega,\tau(\frak A))$.
If $K=\emptyset$ then $d^*=1$ and $\frak D_{1\Bsetminus d^*}=\{0\}$ is itself atomless and of countable $\pi$-weight.
Otherwise, for each
$i\in K$, let $\frak E_i$ be an atomless order-closed subalgebra of
$\frak D_{d_i}$ with countable $\pi$-weight.   Then
$\frak E=\prod_{i\in K}\frak E_i$
is atomless and has countable $\pi$-weight and is an order-closed subalgebra
of $\prod_{i\in K}\frak D_{d_i}\cong\frak D_{1\Bsetminus d^*}$.
So we have a decomposition as envisaged in the statement of the theorem.
}%end of proof of 546M

\leader{546N}{Lemma} Let $(X,\frak T,\Sigma,\mu)$ be a zero-dimensional
quasi-Radon probability space and $(Y,\rho)$ a complete separable
metric space.   On the space $C(X;Y)$ of
continuous functions from $X$ to $Y$ we have a metric $\tilde\rho$
defined by saying that
$\tilde\rho(f,g)=\sup_{x\in X}\min(1,\rho(f(x),g(x)))$ for $f$,
$g\in C(X;Y)$.   Now

\Centerline{$\{g:g\in C(X;Y),\,\{x:(x,g(x))\in V\}$ is negligible$\}$}

\noindent is comeager in $C(X;Y)$ for every
$V\in\Cal N(\mu)\ltimes_{\Sigma\tensorhat\Cal B(Y)}\Cal M(Y)$.

\proof{{\bf (a)} Let $\Cal W$ be the family of sets
expressible in the form $\bigcup_{n\in\Bbb N}E_n\times G_n$ where $E_n\in\Sigma$ and
$G_n\subseteq Y$ is open for each $n$, and $\Cal W^*$ the set of those $W\in\Cal W$
such that all the vertical sections of $W$ are dense in $Y$.
If $W\in\Cal W^*$ and $\epsilon>0$ then

\Centerline{$Q=\{g:g\in C(X;Y),\,\mu\{x:x\in X,\,(x,g(x))\in W\}
\ge 1-\epsilon\}$}

\noindent has dense interior in $C(X;Y)$.   \Prf\ Express $W$ as
$\bigcup_{n\in\Bbb N}E_n\times G_n$ where $E_n\in\Sigma$ and
$G_n\subseteq Y$ is open for each $n$.   Take any $g_0\in C(X;Y)$
and $\delta>0$, and let $\sequence{m}{U_m}$ run over a base for the
topology of $Y$ consisting of sets of diameter at most $\bover12\delta$
(4A2Ob).   For $m$,
$n\in\Bbb N$ set $E_{nm}=E_n\cap g_0^{-1}[U_m]$, $G_{nm}=G_n\cap U_m$.
If $x\in X$, there is an $m\in\Bbb N$ such that $g_0(x)\in U_m$;  now
$W[\{x\}]\cap U_m\ne\emptyset$, so there is an $n\in\Bbb N$ such that $x\in E_n$ and
$G_n\cap U_m\ne\emptyset$;  in which case $x\in E_{nm}$ and $G_n\cap U_m$ is
non-empty.   Thus $\bigcup_{(n,m)\in K}E_{nm}=X$, where
$K=\{(n,m):G_n\cap U_m\ne\emptyset\}$.   Of course every $E_{nm}$ is measurable,
because $g_0$ is continuous and $\frak T\subseteq\Sigma$.
Let $\langle E'_{nm}\rangle_{(n,m)\in K}$
be a disjoint family of measurable sets such that $E'_{nm}\subseteq E_{nm}$ for all
$(n,m)\in K$ and $\bigcup_{(n,m)\in K}E'_{nm}=X$;  let $J\subseteq K$ be a finite set
such that $\mu(\bigcup_{(n,m)\in J}E'_{nm})\ge 1-\bover13\epsilon$.   Because $\mu$ is
inner regular with respect to the closed sets, we can now find closed sets
$F_{nm}\subseteq E'_{nm}$, for $(n,m)\in J$, such that
$\mu(\bigcup_{(n,m)\in J}F_{nm})\ge 1-\bover23\epsilon$.

For $(n,m)\in J$, set
$B_{nm}=g_0^{-1}[U_m]\setminus\bigcup_{(i,j)\in J\setminus\{(n,m)\}}F_{ij}$, so that
$B_{nm}$ is open and includes $F_{nm}$.   Each point of $B_{nm}$ belongs to an
open-and-closed set included in $B_{nm}$, because $X$ is zero-dimensional.
Because $\mu$ is $\tau$-additive,
there is an open-and-closed set $C_{nm}\subseteq B_{nm}$ such that
$\mu(B_{nm}\setminus C_{nm})\le\Bover{\epsilon}{3\#(J)+1}$.   So if we set
$C'_{nm}=C_{nm}\setminus\bigcup_{(i,j)\in J\setminus\{(n,m)\}}C_{ij}$,
we shall again have an open-and-closed set such that
$C'_{nm}\subseteq g_0^{-1}[U_m]$
and

\Centerline{$\mu(F_{nm}\setminus C'_{nm})
=\mu(F_{nm}\setminus C_{nm})\le\Bover{\epsilon}{3\#(J)+1}$}

\noindent for $(n,m)\in J$; and
$\mu(\bigcup_{(n,m)\in J}F_{nm}\cap C'_{nm})\ge 1-\epsilon$.

Now, for each $(n,m)\in J$, $G_n\cap U_m$ is non-empty;  take
$y_{nm}\in G_n\cap U_m$.
Because $\langle C'_{nm}\rangle_{(n,m)\in J}$ is a finite disjoint family of
open-and-closed subsets of $X$, we have a continuous $g_1:X\to Y$ defined by saying
that $g_1(x)=y_{nm}$ whenever $(n,m)\in J$ and $x\in C'_{nm}$, while $g_1(x)=g_0(x)$
for $x\in X\setminus\bigcup_{(n,m)\in J}C'_{nm}$.   Note that if $(n,m)\in J$ and
$x\in C'_{nm}$, then both $g_0(x)$ and $y_{nm}$ belong to $U_m$, so
$\rho(g_0(x),g_1(x))\le\bover12\delta$;  accordingly
$\tilde\rho(g_0,g_1)\le\bover12\delta$.

Let $\eta\in\ooint{0,\min(1,\bover12\delta)}$ be such that
$y\in G_n$ whenever
$(n,m)\in J$ and $\rho(y,y_{nm})\le\eta$.   Consider the non-empty
open set $Q'=\{g:g\in C(X;Y)$, $\tilde\rho(g,g_1)<\eta\}$.
Then $\tilde\rho(g,g_0)<\delta$ for every $g\in Q'$.
If $g\in Q'$, $(n,m)\in J$ and $x\in F_{nm}\cap C'_{nm}$, then
$\rho(g(x),y_{nm})<\eta$, so $g(x)\in G_n$;  as $x\in E_n$, $(x,g(x))\in W$.
So

\Centerline{$\{x:(x,g(x))\in W\}
=\bigcup_{n\in\Bbb N}E_n\cap g^{-1}[G_n]
\supseteq\bigcup_{(n,m)\in J}F_{nm}\cap C'_{nm}$}

\noindent has measure at least $1-\epsilon$, and $g\in Q$.

This shows that $\interior Q$ includes $Q'$ and meets
$\{g:\tilde\rho(g,g_0)<\delta\}$.   As $g_0$ and $\delta$ are arbitrary, $\interior Q$
is dense.\ \Qed

\medskip

{\bf (b)} Recall from 527I that every member of
$\Sigma\tensorhat\Cal B(Y)$ can be
expressed in the form $W\symmdiff V'$ where $W\in\Cal W$ and
$V'\cap\bigcap_{n\in\Bbb N}W_n$ is empty for some sequence
$\sequencen{W_n}$ in $\Cal W^*$.   If
$V\in\Cal N(\mu)\ltimes_{\Sigma\tensorhat\Cal B(Y)}\Cal M(Y)$, we must have
a $\tilde V\in(\Cal N(\mu)\ltimes\Cal M(Y))\cap(\Sigma\tensorhat\Cal B(Y))$
including $V$, a $W\in\Cal W$ and a sequence $\sequencen{W_n}$ in
$\Cal W^*$ such that
$W\symmdiff\tilde V$ is disjoint from $\bigcap_{n\in\Bbb N}W_n$.   For
almost
every $x\in X$, $W[\{x\}]$ is open and $\tilde V[\{x\}]$ is meager, so
$W[\{x\}]$ has meager intersection with the comeager set
$\bigcap_{n\in\Bbb N}W_n[\{x\}]$;  because $Y$ is a Baire space,
$W[\{x\}]=\emptyset$ and
$V[\{x\}]\cap\bigcap_{n\in\Bbb N}W_n[\{x\}]=\emptyset$.   Now

\Centerline{$Q_{mn}
=\{g:g\in C(X;Y)$, $\mu\{x:x\in X$, $(x,g(x))\in W_m\}\ge 1-2^{-n}\}$}

\noindent is comeager for all $m$, $n\in\Bbb N$, by (a), while
$\{g:g\in C(X;Y)$, $\mu^*\{x:x\in X,\,(x,g(x))\in V\}=0\}$ includes
$\bigcap_{m,n\in\Bbb N}Q_{mn}$, so is also comeager.
}%end of proof of 546N

\bigskip

\leader{546O}{Lemma} Suppose that $\Gamma$ is an infinite set and $\Cal I$
is a $\sigma$-ideal of subsets of $X\times Z$, where $X=\{0,1\}^{\Gamma}$
and $Z=\{0,1\}^{\Bbb N}$, such that

\quad(i) writing $\nu_{\Gamma}$ for the usual measure on $X$,
$\Cal I
\supseteq(\Cal N(\nu_{\Gamma})\ltimes\Cal M(Z))\cap\CalBa(X\times Z)$,

\quad(ii) for every $W\subseteq X\times Z$ there is a
$V\in\CalBa(X\times Z)$ such that $W\symmdiff V\in\Cal I$,

\quad(iii) $\add\Cal I=\non\Cal M$.

\noindent Then $\cff[\Gamma]^{\le\omega}>\non\Cal M$.
%2K

\proof{ \Quer\ Suppose, if possible, otherwise.   Write $\kappa$ for
$\non\Cal M$.

\medskip

{\bf (a)} Give $C(Z;Z)$ its topology of uniform convergence, so that it is
a Polish space without isolated points (5A4H), and
$\kappa=\non\Cal M(C(Z;Z))$
(see 522Wb).   Let $\langle g_{\xi}\rangle_{\xi<\kappa}$ enumerate a
non-meager set in $C(Z;Z)$.   Let $\langle I_{\xi}\rangle_{\xi<\kappa}$
run over a cofinal subset of $[\Gamma]^{\omega}$.   For each $\xi<\kappa$, let
$\theta_{\xi}:\omega\to I_{\xi}$ be a bijection, so that
$x\mapsto x\theta_{\xi}$
is a continuous surjection from $X$ onto $Z$, and for $\xi$, $\eta<\kappa$ write

\Centerline{$h_{\xi\eta}(x)=g_{\eta}(x\theta_{\xi})$}

\noindent for $x\in X$;  this makes $h_{\xi\eta}$ a
continuous function from $X$ to $Z$.

For $\alpha<\kappa$ and $x\in X$,

\Centerline{$M_{x\alpha}=\{h_{\xi\eta}(x):\xi$, $\eta<\alpha\}$}

\noindent is meager in $Z$, because $\kappa=\non\Cal M(Z)$.
Let $\sequence{i}{U_i}$ enumerate a base for the topology of $Z$.
For $x\in X$ and $\alpha<\kappa$ choose a sequence
$\sequencen{K(x,\alpha,n)}$ of subsets of $\Bbb N$ such that
$\bigcup_{i\in K(x,\alpha,n)}U_i$ is dense in $Z$ for each $n$ and
$M_{x\alpha}\cap\bigcap_{n\in\Bbb N}\bigcup_{i\in K(x,\alpha,n)}U_i=\emptyset$.

\medskip

{\bf (b)} Because $\add\Cal I=\kappa$ there is a set $Y\subseteq X\times Z$ such that
$Y\notin\Cal I$ but there is a function $\psi:Y\to\kappa$ such that
$\psi^{-1}[\alpha]\in\Cal I$ for every $\alpha<\kappa$.   Write

\Centerline{$Q=\{(x,y,h_{\xi\eta}(x)):(x,y)\in Y$, $\xi$, $\eta<\psi(x,y)\}
\subseteq X\times Z\times Z$.}

\noindent Then $Q\in\Cal I\ltimes\Cal M(Z)$.   Now there are
$R\in\CalBa(X\times Z\times Z)$ and $A_0\in\Cal I$ such that
$R[\{(x,y)\}]\in\Cal M(Z)$ for every $x\in X$ and $y\in Z$, while
$Q\subseteq R\cup(A_0\times Z)$.   \Prf\ For $n$, $i\in\Bbb N$ set

\Centerline{$C_{ni}
=\{(x,y):(x,y)\in Y$, $i\in K(x,\psi(x,y),n)\}\subseteq X\times Z$.}

\noindent Then

\Centerline{$Q\cap\bigcap_{n\in\Bbb N}\bigcup_{i\in\Bbb N}C_{ni}\times U_i=\emptyset$.}

\noindent Take $W_{ni}\in\CalBa(X\times Z)$ such that
$C_{ni}\symmdiff W_{ni}\in\Cal I$ for $n$, $i\in\Bbb N$, and set

\Centerline{$R_0=(X\times Z\times Z)
  \setminus\bigcap_{n\in\Bbb N}\bigcup_{i\in\Bbb N}W_{ni}\times U_i$,}

\Centerline{$A_0=\bigcup_{n,i\in\Bbb N}C_{ni}\symmdiff W_{ni}$,}

\Centerline{$V=\{(x,y):x\in X,\,y\in Z,\,R_0[\{(x,y)\}]\notin\Cal M(Z)\}$,}

\Centerline{$R=R_0\setminus(V\times Z)$.}

\noindent Then $R_0\in\CalBa(X\times Z\times Z)$ and $A_0\in\Cal I$.
Next, $V\in\CalBa(X\times Z)$ (4A3Sa), so $R\in\CalBa(X\times Z\times Z)$, while
$R[\{(x,y)\}]$ is meager for every $(x,y)\in X\times Z$.
If $(x,y)\in Y\setminus A_0$, then

$$\eqalign{R_0[\{(x,y)\}]
&=Z\setminus\bigcap_{n\in\Bbb N}\bigcup\{U_i:i\in\Bbb N,\,(x,y)\in W_{ni}\}\cr
&=Z\setminus\bigcap_{n\in\Bbb N}
   \bigcup\{U_i:i\in\Bbb N,\,(x,y)\in C_{ni}\}\cr
&=Z\setminus\bigcap_{n\in\Bbb N}\bigcup\{U_i:i\in K(x,\psi(x,y),n)\}\cr}$$

\noindent is meager and includes $Q[\{(x,y)\}]$.   So $Y\setminus A_0$ does not meet
$V$ and $Q\subseteq R\cup(A_0\times Z)$.\ \Qed

\medskip

{\bf (c)} For any $x\in X$, consider the set $E_x=\{(y,z):(x,y,z)\in R\}$.
Then $E_x$ is a Borel subset of $Z\times Z$ and $E_x[\{y\}]$ is meager for every
$y\in Z$.   By the Kuratowski-Ulam theorem (527E),
$\{z:z\in Z$, $E_x^{-1}[\{z\}]$ is not meager$\}$ is meager.
But this means that if we write

\Centerline{$V'=\{(x,z):\{y:(x,y,z)\in R\}$ is not meager$\}$,}

\noindent then $\{z:(x,z)\in V'\}$ is meager for every $x\in X$, while
$V'\in\CalBa(X\times Z)$ (4A3Sa again).

\medskip

{\bf (d)} At this point, note that because $R\in\CalBa(X\times Z\times Z)$
there is some countable subset $I$ of $\Gamma$ such
that $R$ belongs to $\Sigma_I\tensorhat\Cal B(Z)\tensorhat\Cal B(Z)$, where
$\Sigma_I$ is the $\sigma$-algebra generated by the sets
$\{x:x(\gamma)=1\}$ as $\gamma$ runs over $I$.   Let $\xi<\kappa$ be
such that $I\subseteq I_{\xi}$.   Now $V'$ is of the form

\Centerline{$\{(x,z):(\tilde x,z)\in\tilde V'\}$,}

\noindent where for $x\in X$ I define $\tilde x\in Z$ by setting
$\tilde x=x\theta_{\xi}$;
$\tilde V'$ is a Borel set in $Z\times Z$.   For $w\in Z\times Z$,
$\{z:(w,z)\in\tilde V'\}=\{z:(x,z)\in V'\}$ whenever $w=\tilde x$, so
is meager;  thus $\tilde V'\in\Cal N(\nu_{\Bbb N})\ltimes\Cal M(Z)$, where
$\nu_{\Bbb N}$ is the usual measure on $Z$.

Next, recall that $\{g_{\eta}:\eta<\kappa\}$ is not meager in
$C(Z;Z)$.   So by 546N
there is an $\eta<\kappa$ such that $\{w:(w,g_{\eta}(w))\in\tilde V'\}$
is $\nu_{\Bbb N}$-negligible.   Accordingly

\Centerline{$\{x:(x,h_{\xi\eta}(x))\in V'\}
=\{x:(x,g_{\eta}(\tilde x))\in V'\}
=\{x:(\tilde x,g_{\eta}(\tilde x))\in\tilde V'\}$}

\noindent is $\nu_{\Gamma}$-negligible.   Set

\Centerline{$A_1=\{(x,y):(x,y,h_{\xi\eta}(x))\in R\}$.}

\noindent Because $h_{\xi\eta}$ is continuous and
$R\in\CalBa(X\times Z\times Z)$, $A_1\in\CalBa(X\times Z)$. Also

\Centerline{$\{x:\{y:(x,y)\in A_1\}$ is not meager$\}
=\{x:(x,h_{\xi\eta}(x))\in V'\}\in\Cal N(\nu_{\Gamma})$,}

\noindent so $A_1\in\Cal N(\nu_{\Gamma})\ltimes\Cal M(Z)$ and $A_1\in\Cal I$.
There is therefore some
$(x,y)\in Y\setminus(A_0\cup A_1\cup\psi^{-1}[\alpha])$, where
$\alpha=\max(\xi,\eta)+1$.   In this case, however,
$\psi(x,y)\ge\alpha$, so $(x,y,h_{\xi\eta}(x))\in Q$;  as $(x,y)\notin A_0$,
$(x,y,h_{\xi\eta}(x))\in R$ and $(x,y)\in A_1$, which is absurd.\ \Bang
}%end of proof of 546O

\leader{546P}{Theorem} Let $\frak A$ be a measurable algebra, not $\{0\}$,
and $\frak G_{\omega}$ the
category algebra of $\cmmnt{Z=\mskip5mu}\{0,1\}^{\omega}$.
Suppose that the Dedekind
completion of $\frak A\otimes\frak G_{\omega}$ is a \pssqa.
Then there is a \qm\ cardinal less than $\tau(\frak A)$.

\proof{{\bf (a)} To begin with (down to the end of (c) below),
suppose that $\frak A$ is homogeneous
with infinite Maharam type, so that there is a set $\Gamma$ with cardinal
$\tau(\frak A)$ such that $\frak A$ is isomorphic to the measure algebra of the usual
measure $\nu_{\Gamma}$ on $X=\{0,1\}^{\Gamma}$.   Note that
$\nu_{\Gamma}$ is completion regular (416U), so $\frak A$ can be
regarded as the measure algebra of $\nu_{\Gamma}\restr\CalBa(X)$.
Of course $\frak G_{\omega}$ is harmless (527Nc).   So we can apply
527O to see that the Dedekind completion of the
free product $\frak A\otimes\frak G_{\omega}$ is isomorphic to
$\frak C=(\CalBa(X)\tensorhat\Cal B(Z))/\Cal L$, where $\Cal L
=(\CalBa(X)\tensorhat\Cal B(Z))\cap(\Cal N(\nu_{\Gamma})\ltimes\Cal M(Z))$;
also $\frak C$ is ccc.

Observe that $\CalBa(X)\tensorhat\Cal B(Z)=\CalBa(X\times Z)$ (4A3N).
So we can apply 546C to see that we have a $\sigma$-ideal $\Cal I^*$
of subsets of $X\times Z$ such that

\inset{($\alpha$) $\CalBa(X\times Z)\cap\Cal I^*=\Cal L$,

($\beta$) for every $A\subseteq X\times Z$ there is an
$E\in\CalBa(X\times Z)$ such that $A\symmdiff E\in\Cal I^*$,

($\gamma$) setting $\kappa=\add\Cal I^*$, there is a
normal ideal $\Cal J^*$ on $\kappa$ such that $\Cal P\kappa/\Cal J^*$ is
isomorphic to an atomless order-closed subalgebra of a principal ideal of
$\frak C$.}

\noindent In particular, $\kappa$ is \qm.
But $\frak A\otimes\frak G_{\omega}$ and $\frak C$ are homogeneous,
by 546Hb and 316P-316Q,
so $\Cal P\kappa/\Cal J^*$ is isomorphic to an atomless order-closed
subalgebra $\frak D$ of $\frak C$ itself.

Now $\kappa\le\non\Cal M$.   \Prf\ $X\times Z\notin\Cal L$ so
$X\times Z\notin\Cal I^*$ and

$$\eqalignno{\kappa
&\le\cov\Cal I^*\le\cov\Cal L\le\cov\Cal N(\nu_{\Bbb N})
\Displaycause{527B(b-iii)}
\le\non\Cal M
\cr}$$

\noindent by 522G (using 522Wa to move between Lebesgue measure on
$\Bbb R$ and $\nu_{\Bbb N}$ on $Z$).\ \Qed

\medskip

{\bf (b)} Suppose that $\frak D$ has a non-zero principal ideal
$\frak D_d$ which is a measurable algebra with Maharam type at most
$\tau(\frak A)$.    Then $\frak D_d$ is a \pssqa\ (546Bb),
and there is a \qm\ cardinal (in fact, an \am\ cardinal) less than
$\tau(\frak D_d)\le\tau(\frak A)$, by 546Bd.
So in this case we can stop.

\medskip

{\bf (c)} Otherwise, 546M tells us that $\frak D$ has
an atomless order-closed
subalgebra $\frak E$ of countable $\pi$-weight;  by 546Ha,
$\frak E\cong\frak G_{\omega}$.   In this case,
$\non\Cal M\le\kappa$.   \Prf\ We have an injective
order-continuous Boolean homomorphism
from $\Cal B(Z)/\Cal B(Z)\cap\Cal M(Z)$ to $\Cal P\kappa/\Cal J^*$.
By 546C, there is an $h:\kappa\to Z$ such that, for $E\in\Cal B(Z)$,
$h^{-1}[E]\in\Cal J^*$ iff $E\in\Cal M(Z)$.   But this means that no meager subset of
$Z$ can include $h[\kappa]$, and $h[\kappa]$ witnesses that
$\kappa\ge\non\Cal M(Z)=\non\Cal M$.\ \Qed

Putting this together with (a), we see that $\kappa$ is precisely
$\non\Cal M$, and that all the conditions of 546O are satisfied.
So we must have $\cff[\Gamma]^{\le\omega}>\kappa$.
But this implies that $\#(\Gamma)>\kappa$ (542Ia).
So in this case also we have $\tau(\frak A)=\#(\Gamma)$ greater than some
\qm\ cardinal.

\medskip

{\bf (d)} All this has been done on the assumption that $\frak A$ is
homogeneous and
atomless.   In general, $\frak A$ has a non-zero homogeneous principal ideal
$\frak A_a$.   Now the Dedekind completion of
$\frak A_a\otimes\frak G_{\omega}$ can be
identified with a principal ideal of $\frak C$, so is also
a \pssqa\ (546Bb).   Since $\frak G_{\omega}$ itself is {\it not} a
\pssqa\ (546I), $\frak A_a\ne\{0,a\}$ and
$\frak A_a$ is atomless.   So (a)-(c) tell us that there is a \qm\ cardinal strictly
less than $\tau(\frak A_a)\le\tau(\frak A)$, and the proof is complete.
}%end of proof of 546P

\leader{546Q}{Corollary} If $\mu$ is Lebesgue measure on $\Bbb R$, then
$\frak C
=\Cal B(\BbbR^2)/\Cal B(\BbbR^2)\cap(\Cal N(\mu)\ltimes\Cal M(\Bbb R))$
is not a \pssqa.

\proof{ Let $\nu$ be a probability measure with the same domain and the
same null sets as $\mu$, and $\frak A$ its measure algebra.
Then $\frak C$ can be identified with the Dedekind completion of
$\frak A\otimes\frak G$, where $\frak G$ is the category
algebra of $\Bbb R$ (527O).   Also $\frak G$ is isomorphic to the
category algebra of
$\{0,1\}^{\omega}$ (522Wb), while $\tau(\frak A)=\omega$.   So 546P tells us that
$\frak C$ cannot be a \pssqa.
}%end of proof of 546Q

\exercises{\leader{546X}{Basic exercises (a)}
%\spheader 546Xa
Let $\frak A$ be any Dedekind complete ccc Boolean algebra.   Show that it is
isomorphic to an order-closed subalgebra of a \pssqa.   \Hint{314M.}
%546B

\spheader 546Xb Show that a \pssqa\ with countable $\pi$-weight is purely
atomic.
%546G

\leader{546Y}{Further exercises (a)}
%\spheader 546Ya
For cardinals $\kappa$, write $\frak G_{\kappa}$
for the category algebra of $\{0,1\}^{\kappa}$.
Show that any order-closed subalgebra of $\frak G_{\omega_1}$ has a
principal ideal isomorphic to one of $\{0,1\}$, $\frak G_{\omega}$ or
$\frak G_{\omega_1}$, and is therefore a \pssqa\ iff it is purely atomic.
%546I mt54bits

\spheader 546Yb Let $\frak A$ be a measurable algebra,
$\frak B$ a Boolean algebra, $\frak C$ the Dedekind completion of
$\frak A\otimes\frak B$,
and $\frak D$ an order-closed subalgebra of $\frak C$.
Show that $\tau(\frak D)\le\max(\omega,\tau(\frak A),\pi(\frak B))$.
%546M
}%end of exercises

\leader{546Z}{Problems (a)}
Can the category algebra of $\{0,1\}^{\omega_2}$ be a \pssqa?

\spheader 546Zb Let $\frak G_{\omega}$ be the category algebra of
$\{0,1\}^{\omega}$.
Can there be any non-zero measurable algebra $\frak A$ such that the Dedekind
completion of $\frak A\otimes\frak G_{\omega}$ is a \pssqa?   \cmmnt{546P
tells us only that such an algebra $\frak A$ must have large Maharam type.}

\spheader 546Zc Let $\frak A$ be the measure algebra of Lebesgue measure
on $\Bbb R$, and $\frak G_{\omega_1}$ the category algebra of
$\{0,1\}^{\omega_1}$.
Can the Dedekind completion of $\frak A\otimes\frak G_{\omega_1}$ be
a \pssqa?   \cmmnt{Many of the arguments in 546L-546M can be extended to
$\frak G_{\omega_1}$ and beyond, but the final steps elude me.}

\spheader 546Zd Can there be an atomless \pssqa\ $\frak A$ such that
$c(\frak A)=\omega$ and $\pi(\frak A)=\omega_1$?
\cmmnt{Note that there seems to be no obstacle to an atomless
ccc \pssqa\ having Maharam type $\omega$ (555K).}

\endnotes{
\Notesheader{546}
The most notable feature of this section is just how hard we have to work
to get results for two `standard' algebras:
the category algebra $\frak G_{\omega}$ of $\{0,1\}^{\omega}$
and the Dedekind completion of $\frak B_{\omega}\otimes\frak G_{\omega}$,
where
$\frak B_{\omega}$ is the measure algebra of the usual measure on
$\{0,1\}^{\omega}$.   As usual, I give the arguments in general forms,
but as far as I know most of the steps have to be gone through
for the special cases.   The same thing happened, of course, in
\S543;  most of the ideas of 543E, other than
some of the infinitary combinatorics from
\S5A1, are needed to show that $\frak B_{\omega}$
is not a \pssqa.   The byways we have to trace are very interesting
(see 546D, for instance), and some of them involve non-trivial facts about
`standard' spaces (546J-546K).
Part of the extraordinary nature of the topic
lies in the way we have to couple ideas based on Borel sets in $\BbbR^2$
with properties of large cardinals -- `large' in the sense of infinitary
combinatorics, that is;  in this
section everything is bounded by $\frak c$.

The general message of the work here seems to be that `standard' algebras
cannot be \pssqa{s}.   The arguments as I have presented them use
restrictions on $\pi$-weight or Maharam type;  but I note that the
most important cases dealt with here (546Bd, 546G, 546Q)
can be expressed as quotients $\Cal B(\{0,1\}^{\omega})/\Cal I$,
and perhaps we ought rather to look at
the descriptive set theory of the ideals $\Cal I$.
Of course the algebras we
are interested in can generally be expressed as
order-closed subalgebras of \pssqa{s} (546Xa).

All the results here depend on a move from a \pssqa\ to an order-closed
subalgebra with the stronger property that it is of the
form $\Cal P\kappa/\Cal J^*$ where $\Cal J^*$ is $\kappa$-additive,
as in 546Cd.
If we start from a measurable algebra, the subalgebra will again be
measurable, with Maharam type no greater than that of
the original algebra;  this is why we can move from 543E
to 543F.   For category algebras this doesn't work in the same way,
because an order-closed
subalgebra of the category algebra of $\{0,1\}^{\omega_2}$, for instance,
can be very different in character.   (For examples see
{\smc Koppelberg \& Shelah 96} and {\smc Balcar Jech \& Zapletal 97}.)
So 546G gives us only 546I and 546Ya.
The discussion of completed free products in 546L-546M depends on a careful
analysis of their order-closed subalgebras which seems not to have simple
generalizations.

}%end of notes

\discrpage

