\frfilename{mt28.tex} 
\versiondate{17.1.15} 
\copyrightdate{1994} 
 
\def\chaptername{Fourier analysis} 
\def\sectionname{Introduction} 
 
\newchapter{28} 
 
For the last chapter of this volume, I attempt a brief account of one of 
the most important topics in analysis.   This is a bold enterprise, and 
I cannot hope to satisfy the reasonable demands of anyone who knows and 
loves the subject as it deserves.   But I also cannot pass it by without 
being false to my own subject, since problems contributed by the 
study of Fourier series and transforms have led measure theory 
throughout its history.   What I will try to do, therefore, is to give 
versions of those results which everyone ought to know in language 
unifying them with the rest of this treatise, aiming to open up a 
channel for the transfer of intuitions and techniques between the 
abstract general study of measure spaces, which is the centre of our 
work, and this particular family of applications of the theory of 
integration. 
 
I have divided the material of this chapter, conventionally enough, into 
three parts:  Fourier series, Fourier transforms and the characteristic 
functions of probability theory.   While it will be obvious that many 
ideas are common to all three, I do not think it useful, at this stage, 
to try to formulate an explicit generalization to unify them;  that 
belongs to a more general theory of harmonic analysis on groups, which 
must wait until Volume 4.   I begin however with a section on the 
Stone-Weierstrass theorem (\S281), which is one of the basic tools of 
functional analysis, as well as being useful for this chapter.   The 
final section (\S286), a proof of Carleson's theorem, is at a rather 
different level from the rest. 
 
\discrpage 
 
