\frfilename{mt544.tex}
\versiondate{31.12.13}
\copyrightdate{2004}


\def\chaptername{Real-valued-measurable cardinals}
\def\sectionname{Measure theory with an \am\ cardinal}

\def\CalUn{\Cal{U}\eurm{n}}

\newsection{544}

As is to be expected, a witnessing measure on a \rvm\ cardinal has some
striking properties, especially if it is normal.   What is less obvious is
that the mere existence of such a cardinal can have implications for
apparently unrelated questions in analysis.   In 544J, for instance, we see
that if there is any \am\ cardinal then we have a version of Fubini's
theorem, $\iint f(x,y)dxdy=\iint f(x,y)dydx$, for many functions $f$
on $\BbbR^2$ which are not jointly measurable.
In this section I explore results of this
kind.   We find that, in the presence of an \am\ cardinal, the
covering number of the Lebesgue null ideal is large (544B) while
its uniformity is small (544G-544H).
There is a second inequality on repeated integrals (544C) to add to the
one already given in 543C, and which tells us
something about measure-precalibers (544D);  I add a couple of variations
(544I-544J).    Next, I give a pair of
theorems (544E-544F) on a measure-combinatorial property of the filter of
conegligible sets of a normal witnessing measure.   Revisiting the theory
of Borel measures on metrizable spaces, discussed in \S438 on the
assumption that no \rvm\ cardinal was present, we find that there are some
non-trivial arguments applicable to spaces with non-measure-free
weight (544K-544L).

In \S541 I briefly mentioned `weakly compact' cardinals.
Two-valued-measurable cardinals
are always weakly compact;  \am\ cardinals never are;  but \am\ cardinals
may or may not have a significant combinatorial property which can be
regarded as a form of weak compactness (544M, 544Yc).   Finally,
I summarise what is known about the location of an \am\ cardinal on
Cicho\'n's diagram (544N).

\leader{544A}{Notation} I repeat some of my notational conventions.
For a measure $\mu$, $\Cal N(\mu)$ will be its null ideal.
For any set $I$, $\nu_I$ will be the usual measure on $\{0,1\}^I$,
$\Cal N_I=\Cal N(\nu_I)$ its null ideal and $(\frak B_I,\bar\nu_I)$ its
measure algebra.

\leader{544B}{Proposition} Let $\kappa$ be an \am\ cardinal.
If $(X,\Sigma,\mu)$ is any locally compact\cmmnt{ (definition:  342Ad)}
semi-finite measure space with $\mu X>0$, then
$\cov\Cal N(\mu)\penalty-100\ge\kappa$.

\proof{ By 521Lb, applied to any \Mth\ subspace of $X$ with non-zero finite
measure, it is enough to show that
$\cov\Cal N_{\lambda}\ge\kappa$ for every $\lambda$;  by
523F, we need look only at the case $\lambda=\kappa$.
Fix on an atomless $\kappa$-additive
probability $\nu$ with domain $\Cal P\kappa$.   By 543G there is an
\imp\ function $f:\kappa\to\{0,1\}^{\kappa}$.   So
$\cov\Cal N_{\kappa}\ge\cov\Cal N(\nu)=\kappa$, by 521Ha.
}%end of proof of 544B

\leader{544C}{Theorem}\cmmnt{ ({\smc Kunen n70})}
Let $\kappa$ be a \rvm\ cardinal
and $\nu$ a normal witnessing probability on $\kappa$;
let $(X,\mu)$ be a compact probability space and
$f:X\times\kappa\to\coint{0,\infty}$ any function.   Then

\Centerline{$\underlineint\bigr(\int f(x,\xi)\nu(d\xi)\bigl)\mu(dx)
\le\int\bigl(\overlineint f(x,\xi)\mu(dx)\bigr)\nu(d\xi)$.}

\proof{\Quer Suppose, if possible, otherwise.

\medskip

{\bf (a)} We are supposing that there is a $\mu$-integrable function
$g:X\to\Bbb R$ such that $0\le g(x)\le\int f(x,\xi)\nu(d\xi)$
for every $x\in X$ and

\Centerline{$\int g(x)\,\mu(dx)>\int\overlineint f(x,\xi)\mu(dx)
\nu(d\xi)$.}

\noindent We can suppose that $g$ is a simple function;  express it
as $\sum_{i=0}^nt_i\chi F_i$ where $(F_0,\ldots,F_n)$ is a partition of $X$
into measurable sets.   For any $\xi<\kappa$,

\Centerline{$\overlineint f(x,\xi)\mu(dx)
=\sum_{i=0}^n\overlineint f(x,\xi)\chi F_i(x)\mu(dx)$}

\noindent (133L).    So there must be some $i\le n$ such that

\Centerline{$t_i\mu F_i
>\int\overlineint f(x,\xi)\chi F_i(x)\mu(dx)\nu(d\xi)$.}

\noindent Set $Y=F_i$, $\mu_1=(\mu F_i)^{-1}\mu\restr\Cal PF_i$,
$t=t_i$;  then  $(Y,\mu_1)$ is a compact probability space (451Da) and

$$\eqalignno{\int\overline{\int}f(y,\xi)\mu_1(dy)\nu(d\xi)
&=\Bover1{\mu F_i}\int\overline{\int}f(x,\xi)\chi F_i(x)\mu(dx)\nu(d\xi)\cr
\displaycause{135Id}
&<t
\le\inf_{y\in Y}\int f(y,\xi)\nu(d\xi).\cr}$$

\medskip

{\bf (b)} Let $(\frak A,\bar\mu_1)$ be the measure algebra of $(Y,\mu_1)$.
Then there is a cardinal $\lambda\ge\kappa$ such that $(\frak A,\bar\mu_1)$
can be embedded in $(\frak B_{\lambda},\bar\nu_{\lambda})$, the measure
algebra of $\nu_{\lambda}$.
Because $\mu_1$ is compact, there is an \imp\
function $\phi:\{0,1\}^{\lambda}\to Y$ (343B).
By 235A, $\overline{\int}f(\phi(z),\xi)\nu_{\lambda}(dz)
\le\overline{\int}f(y,\xi)\mu_1(dy)$ for every $\xi$, so
$\int\overline{\int}f(\phi(z),\xi)\nu_{\lambda}(dz)\nu(d\xi)<t$.

For each $\xi<\kappa$ choose a Baire measurable
function $h_{\xi}:\{0,1\}^{\lambda}\to\Bbb R$ such that
$f(\phi(z),\xi)\le h_{\xi}(z)$
for every $z\in\{0,1\}^{\lambda}$ and
$\int h_{\xi}(z)\nu_{\lambda}(dz)=
\overline{\int}f(\phi(z),\xi)\nu_{\lambda}(dz)$;  we can do this because
$\nu_{\lambda}$ is the completion of its restriction to the Baire
$\sigma$-algebra $\CalBa(\{0,1\}^{\lambda})$ (see 4A3Of), so we can apply
133J(a-i) to the Baire measure
$\nu_{\lambda}\restr\CalBa(\{0,1\}^{\lambda})$.
For each $\xi$, there is a countable set
$I_{\xi}\subseteq\lambda$ such that $h_{\xi}$ is determined by coordinates
in $I_{\xi}$, in the sense that $h_{\xi}(z)=h_{\xi}(z')$ whenever
$z\restr I_{\xi}=z'\restr I_{\xi}$.

By 541Rb, there are $\Gamma\subseteq\kappa$ and a countable set
$J\subseteq\lambda$ such
that $\nu\Gamma=1$ and $I_{\xi}\cap I_{\eta}\subseteq J$
whenever $\xi$, $\eta$ are distinct members of $\Gamma$.

\medskip

{\bf (c)} For $u\in\{0,1\}^J$ and $u'\in\{0,1\}^{\lambda\setminus J}$ write
$u\cup u'$ for their common extension to a member of
$\{0,1\}^\lambda$.   Set

\Centerline{$f_1(u,\xi)
=\int h_{\xi}(u\cup u')\nu_{\lambda\setminus J}(du')$}

\noindent for $u\in\{0,1\}^J$ and $\xi<\kappa$.   Then, applying
Fubini's theorem to
$\{0,1\}^{\lambda}\cong\{0,1\}^J\times\{0,1\}^{\lambda\setminus J}$,
we have

\Centerline{$\int f_1(u,\xi)\nu_J(du)
=\int h_{\xi}(z)\nu_{\lambda}(dz)$,}

\noindent  so that

\Centerline{$\iint f_1(u,\xi)\nu_J(du)\nu(d\xi)
=\int\overlineint f(\phi(z),\xi)\nu_{\lambda}(dz)\nu(d\xi)
<t,$}

\noindent and

\Centerline{$\overlineint\int f_1(u,\xi)\nu(d\xi)\nu_J(du)<t$}

\noindent by Theorem 543C.   Accordingly there is a $u\in\{0,1\}^J$
such that $\int f_1(u,\xi)\nu(d\xi)<t$.

\medskip

{\bf (d)} For each $\xi\in\Gamma$ take
$v_{\xi}\in\{0,1\}^{\lambda\setminus J}$ such that
$h_{\xi}(u\cup v_{\xi})\le f_1(u,\xi)$.
Let $w\in\{0,1\}^{\lambda}$ be such that

\Centerline{$w\restr J=u$,
\quad$w\restr I_{\xi}\setminus J=v_{\xi}\restr I_{\xi}\setminus J$
for every $\xi\in\Gamma$;}

\noindent such a $w$ exists because if $\xi$, $\eta\in\Gamma$
and $\xi\ne\eta$ then $I_{\xi}\cap I_{\eta}\subseteq J$.   Now

\Centerline{$f(\phi(w),\xi)
\le h_{\xi}(w)=h_{\xi}(u\cup v_{\xi})
\le f_1(u,\xi)$}

\noindent for every $\xi\in\Gamma$, so

\Centerline{$\int f(\phi(w),\xi)\nu(d\xi)\le\int f_1(u,\xi)\nu(d\xi)
<t$,}

\noindent contradicting the last sentence of (a) above.\ \Bang

This completes the proof.
}%end of proof of 544C

\leader{544D}{Corollary} If $\kappa$ is an \am\ cardinal and
$\omega\le\lambda\le\kappa$, then $\lambda$ is a measure-precaliber
of every probability algebra.

\proof{ If $\lambda<\kappa$ this is a corollary of 544B and 525J.
If $\lambda=\kappa$, we can use 544C and 525C.   For let
$\langle E_{\xi}\rangle_{\xi<\kappa}$ be
a non-decreasing family in $\Cal N_{\kappa}$ with union $E$.
Let $\nu$ be a normal witnessing probability on $\kappa$.   Set

\Centerline{$C=\{(x,\xi):\xi<\kappa,\,x\in E_{\xi}\}
\subseteq\{0,1\}^{\kappa}\times\kappa$.}

\noindent Then

\Centerline{$\underlineint \nu C[\{x\}]\nu_{\kappa}(dx)\ge\mu_*E$,
\quad$\int\nu_{\kappa}^*C^{-1}[\{\xi\}]\nu(d\xi)=0$,}

\noindent so 544C, applied to the indicator function of $C$, tells us
that $\mu_*E=0$;  now 525Cc tells us that
$\kappa$ is a precaliber of $\frak B_{\kappa}$, and therefore of every
probability algebra, by 525Ia, while 525Ea tells us that
$\kappa$ will therefore be a measure-precaliber of every probability
algebra.
}%end of proof of 544D

\leader{544E}{Theorem}\cmmnt{ ({\smc Kunen n70})} %RVMC 6D
Let $\kappa$ be a \rvm\ cardinal and $\nu$ a
normal witnessing probability on $\kappa$.   If $(X,\mu)$ is a
quasi-Radon probability space of weight strictly less than $\kappa$,
and $f:[\kappa]^{<\omega}\to\Cal N(\mu)$ is any function, then

\Centerline{$\bigcap_{V\subseteq\kappa,\nu V=1}
  \bigcup_{I\in[V]^{<\omega}}f(I)\in\Cal N(\mu)$.}

%would \Mah(\mu)<\kappa be enough?   Try $X$ the Stone space of
%[0,1].   Also in 543C.

\proof{ Let $\Cal F$ be the filter of
$\nu$-conegligible subsets of $\kappa$.

\medskip

{\bf (a)} I show by induction on $n\in\Bbb N$ that if
$g:[\kappa]^{\le n}\to\Cal N(\mu)$ is any function, then

\Centerline{$E(g)=\bigcap_{V\in\Cal F}\bigcup_{I\in[V]^{\le n}}g(I)
\in\Cal N(\mu)$.}

\noindent \Prf\ For $n=0$ this is
trivial;  $E(g)=g(\emptyset)\in\Cal N(\mu)$.   For the
inductive step to $n+1$, given
$g:[\kappa]^{\le n+1}\to\Cal N(\mu)$, then for each $\xi<\kappa$
define $g_{\xi}:[\kappa]^{\le n}\to\Cal N(\mu)$ by setting
$g_{\xi}(I)=g(I\cup\{\xi\})$ for each $I\in[\kappa]^{\le n}$.
By the inductive hypothesis,  $E(g_{\xi})\in\Cal N(\mu)$.   Set

\Centerline{$C=\{(x,\xi):x\in E(g_{\xi})\}\subseteq X\times\kappa$.}

\noindent Then

\Centerline{$\int\mu^*C^{-1}[\{\xi\}]\nu(d\xi)
=\int\mu^*E(g_{\xi})\nu(d\xi)=0$,}

\noindent so by 543C

\Centerline{$\overlineint\nu C[\{x\}]\mu(dx)=0$,}

\noindent and
$\mu D=0$, where $D=g(\emptyset)\cup\{x:\nu C[\{x\}]>0\}$.

Take any $x\in X\setminus D$ and set
$W=\kappa\setminus C[\{x\}]\in\Cal F$.
For each $\xi\in W$, $x\notin E(g_{\xi})$, so there is a
$V_{\xi}\in\Cal F$ such that $\nu V_{\xi}=1$ and $x\notin g_{\xi}(I)$
for every $I\in[V_{\xi}]^{\le n}$.   Set

\Centerline{$V=W\cap\{\xi:\xi\in V_{\eta}$ for every
$\eta<\xi\}$.}

\noindent Then $V\in\Cal F$.   If $I\in[V]^{\le n+1}$, either
$I=\emptyset$ and $x\notin g(I)$, or there is a least element $\xi$ of $I$;
in the latter case,
$\xi\in W$ and $J=I\setminus\{\xi\}\subseteq V_{\xi}$ and
$x\notin g_{\xi}(J)=g(I)$.
So $x\notin\bigcup\{g(I):I\in[V]^{\le n+1}\}$.   As $x$ is
arbitrary, $E(g)\subseteq D\in\Cal N(\mu)$ and the
induction proceeeds.\ \Qed

\medskip

{\bf (b)} Now consider

\Centerline{$G=\bigcup_{n\in\Bbb N}E(f\restr[\kappa]^{\le n})
\in\Cal N(\mu)$.}

\noindent If $x\in X\setminus G$ then for each $n\in\Bbb N$ there
is a $V_n\in\Cal F$ such that
$x\notin\bigcup\{f(I):I\in[V_n]^{\le n}\}$.   Set
$V=\bigcap_{n\in\Bbb N}V_n\in\Cal F$;
then $x\notin\bigcup\{f(I):I\in[V]^{<\omega}\}$.   As $x$
is arbitrary,

\Centerline{$\bigcap_{V\subseteq\kappa,\nu V=1}
  \bigcup_{I\in[V]^{<\omega}}f(I)\subseteq G\in\Cal N(\mu)$,}

\noindent as required.
}%end of proof of 544E

\leader{544F}{Theorem}\cmmnt{ ({\smc Kunen n70})}
Let $\kappa$ be a \rvm\ cardinal with a
normal witnessing probability $\nu$.   If $(X,\mu)$ is a
locally compact semi-finite measure space with $\mu X>0$
and $f:[\kappa]^{<\omega}\to\Cal N(\mu)$ is a function,
then there is a $\nu$-conegligible $V\subseteq\kappa$ such that
$\bigcup\{f(I):I\in[V]^{<\omega}\}\ne X$.

\proof{{\bf (a)} Consider first the case
$(X,\mu)=(\{0,1\}^{\kappa},\nu_{\kappa})$.   For any
$L\subseteq\kappa$ let $\pi_L:\{0,1\}^{\kappa}\to\{0,1\}^L$
be the restriction map.
Let $\Cal F$ be the conegligible filter on $\kappa$.

\medskip

\quad{\bf (i)}
For each $I\in[\kappa]^{<\omega}$, there is a countable set
$g(I)\subseteq\kappa$ such that
$\nu_{g(I)}(\pi_{g(I)}[f(I)])=0$ (254Od);   enlarging $f(I)$ if
necessary, we may suppose
that $f(I)=\pi_{g(I)}^{-1}[\pi_{g(I)}[f(I)]]$.
By 541Q there are a set $C\in\Cal F$ and a function
$h:[\kappa]^{<\omega}\to[\kappa]^{\le\omega}$ such that
$g(I)\cap\eta\subseteq h(I\cap\eta)$
whenever $I\in[C]^{<\omega}$ and $\eta<\kappa$.   Set

\Centerline{$\Gamma=\{\gamma:\gamma<\kappa$,
$h(I)\subseteq\gamma$ for every $I\in[\gamma]^{<\omega}\}$;}

\noindent then $\Gamma$ is a closed cofinal set in $\kappa$,
because $\cf(\kappa)>\omega$.
Let $\langle\gamma_{\eta}\rangle_{\eta\le\kappa}$ be the
increasing enumeration of $\Gamma\cup\{0,\kappa\}$.

\medskip

\quad{\bf (ii)}
For $\eta<\kappa$, set $M(\eta)=\kappa\setminus\gamma_{\eta}$
and $L(\eta)=\gamma_{\eta+1}\setminus\gamma_{\eta}$;
then $\nu_{M(\eta)}$ can be identified with the product
measure $\nu_{L(\eta)}\times\nu_{M(\eta+1)}$.   Choose
$u_{\eta}\in\{0,1\}^{\gamma_{\eta}}$, $V_{\eta}\subseteq\kappa$
inductively, as follows.   $u_0\in\{0,1\}^0$ is the empty function.
Given $u_{\eta}$, then for each $I\in[\kappa]^{<\omega}$ set

\Centerline{$f'_{\eta}(I)=\{v:v\in\{0,1\}^{L(\eta)}$,
$\nu_{M(\eta+1)}\{w:{u_{\eta}}\cup v\cup w\in f(I)\}>0\}$,}

\noindent and

$$\eqalign{f_{\eta}(I)
&=f'_{\eta}(I)\text{ if }\nu_{L(\eta)}(f'_{\eta}(I))=0,\cr
&=\emptyset\text{ otherwise}.\cr}$$

\noindent By 544E, applied to
$J\mapsto f_{\eta}(K\cup J)\in\Cal N_{L(\eta)}$,
we can find for each $K\in[\gamma_{\eta+1}]^{<\omega}$
a set $E_{\eta K}\subseteq\{0,1\}^{L(\eta)}$
such that $\nu_{L(\eta)}E_{\eta K}=1$ and for every $v\in E_{\eta K}$
there is a  set $V\in\Cal F$ such that $v\notin f_{\eta}(K\cup J)$ for
any $J\in[V]^{<\omega}$.   Choose
$v_{\eta}\in\bigcap\{E_{\eta K}: K\in[\gamma_{\eta+1}]^{<\omega}\}$
(using 544B);   for
$K\in[\gamma_{\eta+1}]^{<\omega}$ choose $V_{\eta K}\in\Cal F$
such that $v_{\eta}\notin f_{\eta}(K\cup J)$ for any
$J\in[V_{\eta K}]^{<\omega}$.   Set
$V_{\eta}=\bigcap\{V_{\eta K}:K\in[\gamma_{\eta+1}]^{<\omega}\}\in\Cal F$
and
$u_{\eta+1}={u_{\eta}}\cup v_{\eta}\in\{0,1\}^{\gamma_{\eta+1}}$.

At limit ordinals $\eta$ with $0<\eta\le\kappa$, set
$u_{\eta}=\bigcup_{\xi<\eta}u_{\xi}\in\{0,1\}^{\gamma_{\eta}}$.

\medskip

\quad{\bf (iii)}  Now consider $u=u_{\kappa}\in\{0,1\}^{\kappa}$ and

\Centerline{$V=\{\xi:\xi\in C$, $\xi\in V_{\eta}$ for every $\eta<\xi\}
\in\Cal F$.}

\noindent If $I\in[V]^{<\omega}$ then

\Centerline{$\nu_{M(\eta)}\{w:{u_{\eta}}\cup w\in f(I)\}=0$}

\noindent for every $\eta<\kappa$.  \Prf\ Induce on $\eta$.  For
$\eta=0$ this says just that
$\nu_{\kappa}f(I)=0$, which was our hypothesis on $f$.
For the inductive step to $\eta+1$, we have

\Centerline{$\nu_{M(\eta)}\{w:{u_{\eta}}\cup w\in f(I)\}=0$}

\noindent by the inductive hypothesis, so Fubini's theorem tells
us that

\Centerline{$\nu_{L(\eta)}\{v:
\nu_{M(\eta+1)}\{w:u_{\eta}\cup v\cup w\in f(I)\}>0\}=0$,}

\noindent that is, $\nu_{L(\eta)}f'_{\eta}(I)=0$, so that
$f_{\eta}(I)=f'_{\eta}(I)$.   Now setting
$K=I\cap\gamma_{\eta+1}$ and $J=I\setminus\gamma_{\eta+1}$, we see that
$J\subseteq V_{\eta}$ (because of course $\eta<\gamma_{\eta+1}$),
therefore $J\subseteq V_{\eta K}$ and
$v_{\eta}\notin f_{\eta}(K\cup J)=f'_{\eta}(I)$;  but this says just that

\Centerline{$\nu_{M(\eta+1)}\{w:
  u_{\eta}\cup v_{\eta}\cup w\in f(I)\}=0$,}

\noindent that is, that

\Centerline{$\nu_{M(\eta+1)}\{w:
  {u_{\eta+1}}\cup w\in f(I)\}=0$,}

\noindent so that the induction continues.

For the inductive step to a non-zero limit ordinal
$\eta\le\kappa$, there is a non-zero $\zeta<\eta$ such that
$I\cap\gamma_{\eta}\subseteq\gamma_{\zeta}$.   Now

\Centerline{$g(I)\cap\gamma_{\eta}
\subseteq h(I\cap\gamma_{\eta})=h(I\cap\gamma_{\zeta})
\subseteq\gamma_{\zeta}$,}

\noindent by the choice of $\Gamma$.   As $f(I)$ is determined by
coordinates in $g(I)$, this means that

\Centerline{$\{w:w\in\{0,1\}^{M(\zeta)}$,
   ${u_{\zeta}}\cup w\in f(I)\}
=\{0,1\}^{\gamma_{\eta}\setminus\gamma_{\zeta}}
  \times\{w:w\in\{0,1\}^{M(\eta)}$, ${u_{\eta}}\cup w\in f(I)\}$.}

\noindent By the inductive hypothesis,

\Centerline{$\nu_{M(\zeta)}\{w:{u_{\zeta}}\cup w\in f(I)\}=0$,}

\noindent so that

\Centerline{$\nu_{M(\eta)}\{w:{u_{\eta}}\cup w\in f(I)\}=0$}

\noindent and the induction continues.\ \Qed

\medskip

\quad{\bf (iv)}
But now, given $I\in[V]^{<\omega}$,  there is surely some $\eta<\kappa$
such that $g(I)\subseteq\gamma_{\eta}$, and in this case
$\{w:{u_{\eta}}\cup w\in f(I)\}$ is either $\emptyset$ or
$\{0,1\}^{M(\eta)}$.   As it is $\nu_{M(\eta)}$-negligible it must be
empty, and $u\notin f(I)$.

Thus we have a point $u\notin\bigcup\{f(I):I\in[V]^{<\omega}\}$,
as required.

\medskip

{\bf (b)} If $(X,\mu)$ is a compact probability space,
we have a $\lambda\ge\kappa$ and an \imp\ function
$\phi:\{0,1\}^{\lambda}\to X$.   For each $I\in[\kappa]^{<\omega}$
let $J_I\in[\lambda]^{\le\omega}$ be such that
$\nu_{J_I}\pi_{J_I}[\phi^{-1}[f(I)]]=0$, where here
$\pi_{J_I}$ is interpreted as a
map from $\{0,1\}^{\lambda}$ to $\{0,1\}^{J_I}$;  set
$J=\kappa\cup\bigcup_{I\in[\kappa]^{<\omega}}J_I$.
Let $q:\{0,1\}^J\to\{0,1\}^{\lambda}$ be any function such that
$\pi_Jq$ is the identity on $\{0,1\}^J$, and set $\psi=\phi q$.
For any $I\in[\kappa]^{<\omega}$,

\Centerline{$\psi^{-1}[f(I)]=q^{-1}[\phi^{-1}[f(I)]]
\subseteq\pi_J[\phi^{-1}[f(I)]]
\subseteq\pi_J[\pi_{J_I}^{-1}[\pi_{J_I}[\phi^{-1}[f(I)]]]]$}

\noindent is $\nu_J$-negligible because
$\pi_{J_I}^{-1}[\pi_{J_I}[\phi^{-1}[f(I)]]]$ is $\nu_{\lambda}$-neglgible
and determined by coordinates in $J$.

Because $\#(J)=\kappa$, (a) tells us that
there are $u\in\{0,1\}^J$ and a conegligible $V\subseteq\kappa$ such that
$u\notin\psi^{-1}[f(I)]$ for every
$I\in[V]^{<\omega}$;  in which case $\psi(u)\notin f(I)$ for every
$I\in[V]^{<\omega}$ and $\bigcup\{f(I):I\in[V]^{<\omega}\}\ne X$.

\medskip

{\bf (c)} For the general case, take a subset $E$ of $X$ with non-zero
finite measure, and apply (b) to the function $I\mapsto E\cap f(I)$ and
the normalized subspace measure $\Bover1{\mu E}\nu_E$.
}%end of proof of 544F

\leader{544G}{Proposition} Let $\kappa$ be an \am\ cardinal and
$\omega_1\le\lambda<\kappa$.
If $(X,\mu)$ is an atomless locally compact  semi-finite measure space of
Maharam type less than $\kappa$, and $\mu X>0$,
then there is a Sierpi\'nski set $A\subseteq X$ with cardinal $\lambda$.

%this has to work for strongly Sierpi\'nski query

\proof{{\bf (a)} To begin with, suppose that $X=\{0,1\}^{\theta}$ and
$\mu=\nu_{\theta}$ where $\theta<\kappa$.
Let $\nu$ be an atomless $\kappa$-additive
probability defined on $\Cal P\kappa$.   By 543G
there is a function $f:\kappa\to(\{0,1\}^{\theta})^{\lambda}$ which
is inverse-measure-preserving for $\nu$ and the usual measure
$\nu_{\theta}^{\lambda}$ of
$(\{0,1\}^{\theta})^{\lambda}$, which we may think of either as the
power of $\nu_{\theta}$, or as the Radon power of $\nu_{\theta}$, or as a
copy of $\nu_{\theta\times\lambda}$.   For $\xi<\kappa$, set

\Centerline{$A_{\xi}
=\{f(\xi)(\eta):\eta<\lambda\}\subseteq\{0,1\}^{\theta}$.}

\Quer Suppose, if possible, that for every
$\xi<\kappa$ there is a set $J_{\xi}\subseteq\lambda$ such that
$\#(J_{\xi})=\omega_1$ but $E_{\xi}=f(\xi)[J_{\xi}]$ is
$\nu_{\theta}$-negligible.
For each $\xi$ choose a countable set $I_{\xi}\subseteq\theta$
such that $E'_{\xi}=\pi_{I_{\xi}}^{-1}[\pi_{I_{\xi}}[E_{\xi}]$
is $\nu_{\theta}$-negligible, writing $\pi_{I_{\xi}}(x)=x\restr I_{\xi}$
for $x\in\{0,1\}^{\theta}$.   By 541D, there is a countable
$I\subseteq\theta$ such that
$V=\{\xi:I_{\xi}\subseteq I\}$ is $\nu$-conegligible.   For $\xi\in V$ set
$E^*_{\xi}=\pi_I[E_{\xi}]\subseteq\{0,1\}^I$,
so that $\nu_I E^*_{\xi}=0$.  Fix a sequence
$\sequence{m}{U_m}$ running over the open-and-closed
subsets of $\{0,1\}^I$, and for each $\xi\in V$,
$n\in\Bbb N$ choose an open set
$G_{n\xi}\subseteq\{0,1\}^I$ such that $E^*_{\xi}\subseteq G_{n\xi}$ and
$\nu_I(G_{n\xi})\le 2^{-n}$.   For $m, n\in\Bbb N$ set

\Centerline{$D_{nm}=\{\xi:\xi\in V$, $U_m\subseteq G_{n\xi}\}$.}

\noindent For each $\alpha<\lambda$, set
$f_{\alpha}(\xi)=\pi_I(f(\xi)(\alpha))\in\{0,1\}^I$ for $\xi<\kappa$;
then the functions $f_{\alpha}$ are all stochastically independent, in the
sense that the $\sigma$-algebras
$\Sigma_{\alpha}=\{f_{\alpha}^{-1}[H]:H\subseteq\{0,1\}^I$ is Borel$\}$ are
independent.   \Prf\ Suppose that
$\alpha_0,\ldots,\alpha_n<\lambda$ are distinct and
$H_0,\ldots,H_n$ are Borel subsets of $\{0,1\}^I$.   For each $i$, set

\Centerline{$W_i=\{u:u\in(\{0,1\}^{\theta})^{\lambda}$,
$u(\alpha_i)\in\pi_I^{-1}[H_i]\}$.}

\noindent Then

$$\eqalign{\nu(\bigcap_{i\le n}f_{\alpha_i}^{-1}[H_i])
&=\nu f^{-1}[\bigcap_{i\le n}W_i]
=\nu_{\theta}^{\lambda}(\bigcap_{i\le n}W_i)\cr
&=\prod_{i\le n}\nu_{\theta}^{\lambda}W_i
=\prod_{i\le n}\nu f_{\alpha_i}^{-1}[H_i].  \text{ \Qed}\cr}$$

\noindent By 272Q, there is for each $\xi<\kappa$ an
$\alpha(\xi)\in J_{\xi}$ such that $\Sigma_{\alpha(\xi)}$ is
stochastically independent of the $\sigma$-algebra $\Tau$ generated by
$\{D_{nm}:n$, $m\in\Bbb N\}$.   Because $\lambda<\kappa$ and $\nu$ is
$\kappa$-additive, there is a $\gamma<\lambda$ such that
$B=\{\xi:\alpha(\xi)=\gamma\}$ has $\nu B>0$.
Take $n\in\Bbb N$ such that $\nu(B)>2^{-n}$, and examine

\Centerline{$C=\bigcup_{m\in\Bbb N}(D_{nm}\cap f_{\gamma}^{-1}[U_{m}])$.}

\noindent Then $\nu C=(\nu\times\nu_I)(\tilde C)$ where

\Centerline{$\tilde C
=\bigcup_{m\in\Bbb N}(D_{nm}\times U_m)\subseteq\kappa\times\{0,1\}^I$}

\noindent and $\nu\times\nu_I$ is the c.l.d.\ product measure on
$\kappa\times\{0,1\}^I$.   \Prf\ Because $\Tau$ and $\Sigma_{\gamma}$ are
independent, and $\nu_I$ is the completion of its restriction to the Borel
(or the Baire) $\sigma$-algebra of $\{0,1\}^I$, the map
$\xi\mapsto(\xi,f_{\gamma}(\xi)):\kappa\to\kappa\times\{0,1\}^I$ is \imp\
for $\nu$ and $(\nu\restr\Tau)\times\nu_I$ (cf.\ 272J).
The inverse of $\tilde C$ under this map is just $C$, so

$$\eqalign{\nu C
&=((\nu\restr\Tau)\times\nu_I)(\tilde C)
=\int(\nu\restr\Tau)\tilde C^{-1}[\{u\}]\nu_I(du)\cr
&=\int\nu\tilde C^{-1}[\{u\}]\nu_I(du)
=(\nu\times\nu_I)(\tilde C). \text{ \Qed}\cr}$$

\noindent But, for each $\xi<\kappa$,
the vertical section $\tilde C[\{\xi\}]$ is just
$\bigcup\{U_m:\xi\in D_{nm}\}=G_{n\xi}$, so

\Centerline{$(\nu\times\nu_I)(\tilde C)
=\int\nu_I(G_{n\xi})\nu(d\xi)\le 2^{-n}$.}

\noindent Accordingly $\nu C\le 2^{-n}<\nu B$ and
there must be a $\xi\in B\cap V\setminus C$.
But in this case $f(\xi)(\gamma)\in E_{\xi}$, because
$\gamma=\alpha(\xi)\in J_{\xi}$, so
$f_{\gamma}(\xi)=\pi_I(f(\xi)(\gamma))\in E^*_{\xi}$.
On the other hand, $f_{\gamma}(\xi)\notin G_{n\xi}$, because
there is no $m$ such that
$f_{\gamma}(\xi)\in U_m\subseteq G_{n\xi}$;
contrary to the choice of $G_{n\xi}$.\ \Bang

So take some $\xi<\kappa$ such that
$\nu_{\theta}^*(f(\xi)[J])>0$ for every uncountable
$J\subseteq\lambda$.   Evidently $f(\xi)$
is countable-to-one, so $A_{\xi}$ must have cardinal
$\lambda$, and will serve for $A$.

\medskip

{\bf (b)} Now suppose that $(X,\mu)$ is an atomless compact probability
space with Maharam type $\theta<\kappa$.   Then we have an \imp\ map
$h:\{0,1\}^{\theta}\to X$.   Let $A\subseteq\{0,1\}^{\theta}$ be a
Sierpi\'nski set of
cardinal $\lambda$;  then $h[A]$ is a Sierpi\'nski set with cardinal
$\lambda$ in $X$, by 537B(b-i).

\medskip

{\bf (c)} Finally, for the general case as stated, we can apply (b) to a
normalized subspace measure, as usual.
}%end of proof of 544G

\leader{544H}{Corollary} Let $\kappa$ be an \am\ cardinal.

(a) $\non\Cal N_{\theta}=\omega_1$ for
$\omega\le\theta<\kappa$.

(b) $\non\Cal N_{\theta}\le\kappa$ for
$\theta\le\min(2^{\kappa},\kappa^{(+\omega)})$.

(c) $\non\Cal N_{\theta}<\theta$ for
$\kappa<\theta<\kappa^{(+\omega)}$.

\proof{{\bf (a)}  Immediate from 544G.

\medskip

{\bf (b)} If $\nu$ is any witnessing probability on $\kappa$ then
we have an \imp\ function $f:\kappa\to\{0,1\}^{\theta}$ (543G);  now
$f[\kappa]$ witnesses that $\non\Cal N_{\theta}\le\kappa$.

\medskip

{\bf (c)} Induce on $\theta$, using 523Ib.
}%end of proof of 544H

\cmmnt{\medskip

\noindent{\bf Remark} There may be more to be said;  see 544Zc.
}%end of comment

\leader{544I}{}\cmmnt{ The following is an elementary corollary of
Theorem 543C.

\medskip

\noindent}{\bf Proposition} Let $(X,\frak T,\Sigma,\mu)$ be a totally
finite
quasi-Radon measure space and $(Y,\Cal PY,\nu)$ a probability space;
suppose that $w(X)<\add\nu$.   Let $f:X\times Y\to\Bbb R$ be a
bounded function such that all the sections
$x\mapsto f(x,y):X\to\Bbb R$ are $\Sigma$-measurable.   Then
the repeated integrals $\iint f(x,y)\nu(dy)\mu(dx)$ and
$\iint f(x,y)\mu(dx)\nu(dy)$ are defined and equal.

\proof{ If $\mu X=0$ this is trivial;  otherwise,
re-scaling $\mu$ if necessary, we may suppose that $\mu X=1$.   By 543C,

\Centerline{$\overlineint\int f(x,y)\nu(dy)\mu(dx)
\le\int\overlineint  f(x,y)\mu(dx)\nu(dy)
=\iint f(x,y)\mu(dx)\nu(dy)$.}

\noindent Similarly

\Centerline{$\overlineint\int (-f(x,y))\nu(dy)\mu(dx)
\le\iint(-f(x,y))\mu(dx)\nu(dy),$}

\noindent so that

\Centerline{$\underlineint \int f(x,y)\nu(dy)\mu(dx)
\ge\iint f(x,y)\mu(dx)\nu(dy)$.}

\noindent Putting these together we have the result.
}%end of proof of 544I

\leader{544J}{Proposition}\cmmnt{ ({\smc Zakrzewski 92})} Let
$\kappa$ be an \am\ cardinal and
$(X,\frak T,\Sigma,\penalty-100\mu)$,
$(Y,\frak S,\Tau,\nu)$  Radon probability spaces both of
weight less than $\kappa$;  let $\mu\times\nu$ be the c.l.d.\ product
measure on $X\times Y$, and $\Lambda$ its domain.
Let $f:X\times Y\to\Bbb R$ be a function
such that all its horizontal and vertical sections

\Centerline{$x\mapsto f(x,y^*):X\to\Bbb R$,
\quad$y\mapsto f(x^*,y):Y\to\Bbb R$}

\noindent are measurable.    Then

(a) if $f$ is bounded, the repeated integrals

\Centerline{$\iint f(x,y)\mu(dx)\nu(dy)$,
\quad$\iint f(x,y)\nu(dy)\mu(dx)$}

\noindent exist and are equal;

(b) in any case, there is a $\Lambda$-measurable function
$g:X\times Y\to\Bbb R$ such that all the
sections $\{x:g(x,y^*)\ne f(x,y^*)\}$, $\{y:g(x^*,y)\ne f(x^*,y)\}$ are
negligible.

\proof{{\bf (a)} By 543H there is a $\kappa$-additive measure
$\tilde\nu$ on $Y$, with domain $\Cal PY$, extending $\nu$.   Now 544I
tells us, among other things, that the function

\Centerline{$x\mapsto\int f(x,y)\nu(dy)=
\int f(x,y)\tilde\nu(dy):X\to\Bbb R$}

\noindent is $\Sigma$-measurable.   Similarly,
$y\mapsto\int f(x,y)\mu(dx)$ is $\Tau$-measurable.
So returning to 544I we get

$$\eqalign{\iint f(x,y)\mu(dx)\nu(dy)
&=\iint f(x,y)\mu(dx)\tilde\nu(dy)\cr
&=\iint f(x,y)\tilde\nu(dy)\mu(dx)
=\iint f(x,y)\nu(dy)\mu(dx).\cr}$$

\medskip

{\bf (b)(i)} Suppose first that $f$ is bounded.   By (a), we can define a
measure $\theta$ on $X\times Y$ by saying that

\Centerline{$\theta G=\int\nu G[\{x\}]\mu(dx)
=\int\mu G^{-1}[\{y\}]\nu(dy)$}

\noindent whenever $G\subseteq X\times Y$ is such that $G[\{x\}]\in\Tau$
for almost every $x\in X$ and $G^{-1}[\{y\}]\in\Sigma$ for almost every
$y\in Y$.   This $\theta$ extends $\mu\times\nu$;
so the Radon-Nikod\'ym theorem (232G) tells us that there is a
$\Lambda$-measurable function $h:X\times Y\to\Bbb R$ such
that $\int_Gf(x,y)\theta(dxdy)=\int_G h(x,y)\theta(dxdy)$ for every
$G\in\Lambda$.

Let $\Cal U$ be a base for the topology $\frak T$, closed under finite
intersections, with
$\#(\Cal U)<\kappa$.   For any $U\in\Cal U$ consider

\Centerline{$V_U=\{y:\int_Uf(x,y)\mu(dx)>\int_Uh(x,y)\mu(dx)\}$.}

\noindent The argument of (a) shows that $y\mapsto\int_Uf(x,y)\mu(dx)$
is $\Tau$-measurable, so $V_U\in\Tau$, and

$$\eqalign{\int_{V_U}\int_Uf(x,y)\mu(dx)\nu(dy)
&=\int_{U\times V_U}f(x,y)\theta(dxdy)\cr
&=\int_{U\times V_U}h(x,y)\theta(dxdy)
=\int_{V_U}\int_U h(x,y)\mu(dx)\nu(dy),\cr}$$

\noindent so $\nu V_U=0$.   Similarly

\Centerline{$\nu\{y:\int_U f(x,y)\mu(dx)<\int_U h(x,y)\mu(dx)\}=0$.}

\noindent Because $\#(\Cal U)<\kappa$, and no
non-negligible measurable set in $Y$ can be covered by fewer than
$\kappa$ negligible sets (544B), we must have

\Centerline{$\nu^*\{y:\int_U f(x,y)\mu(dx)=\int_U h(x,y)\mu(dx)$
for every $U\in\Cal U\}=1$.}

\noindent But because $\Cal U$ is a base for the topology of $X$ closed
under finite intersections, we see that

\Centerline{$\nu^*\{y:f(x,y)=h(x,y)$ for $\mu$-almost every $x\}=1$.}

\noindent (For each $y$ such that
$\int_U f(x,y)\mu(dx)=\int_U h(x,y)\mu(dx)$ for every $U\in\Cal U$,
apply 415H(v) to the indefinite-integral measures
over $\mu$ defined by the functions $x\mapsto f(x,y)$, $x\mapsto h(x,y)$;
these are quasi-Radon by 415Ob.)   Again using (a), we know that the
the repeated integral $\iint|f(x,y)-h(x,y)|\mu(dx)\nu(dy)$ exists, and
it must be $0$.   Thus

\Centerline{$\nu\{y:f(x,y)=h(x,y)$ for $\mu$-almost every $x\}=1$.}

Similarly,

\Centerline{$\mu\{x:f(x,y)=h(x,y)$ for $\nu$-almost every $y\}=1$.}

\noindent But now, changing $h$ on a set of the form
$(E\times Y)\cup(X\times F)$ where $\mu E=\nu F=0$, we can get a function
$g$, still
$\Lambda$-measurable, such that
$\{(x,y):f(x,y)\ne g(x,y)\}$ has all its horizontal  and vertical
sections negligible.

\medskip

\quad{\bf (ii)}
This deals with bounded $f$.   But for general $f$ we can look at the
truncates $(x,y)\mapsto\med(-n,f(x,y),\penalty-100n)$ for each $n$ to get a
sequence $\langle g_n\rangle_{n\in\Bbb N}$ of functions which will
converge at an adequate number of points to provide a suitable $g$.
}%end of proof of 544J

\leader{544K}{Proposition} If $X$ is a metrizable space and $\mu$
is a $\sigma$-finite Borel measure on $X$, then
$\add\Cal N(\mu)\ge\add\Cal N_{\omega}$.

\proof{{\bf (a)} If there is an
atomlessly-measurable cardinal then

\Centerline{$\add\Cal N_{\omega}\le\non\Cal N_{\omega}=\omega_1$}

\noindent (544Ha), so the result is immediate.
So let us henceforth suppose otherwise.

\medskip

{\bf (b)} Because there is a totally finite measure with the same domain
and the same null ideal as $\mu$ (215B(vii)), we can suppose that
$\mu$ itself is totally finite.   Let $(\frak A,\bar\mu)$
be the measure algebra of $\mu$ and
$\Cal U$ a $\sigma$-disjoint base for the topology of $X$
(4A2Lg);  express $\Cal U$ as $\bigcup_{n\in\Bbb N}\Cal U_n$ where each
$\Cal U_n$ is disjoint.   For $\Cal V\subseteq\Cal U_n$ set
$\nu_n\Cal V=\mu(\bigcup\Cal V)$.   Then $\nu_n$ is a
totally finite measure with domain $\Cal P\Cal U_n$.   Because there is no
\am\ cardinal, $\add\nu_n$ is either $\infty$ or a \2vm\ cardinal;  in
either case, $\nu_n$ is $\frak c$-additive and purely atomic (438Ce, 543B).

\medskip

{\bf (c)} $\mu$ has countable Maharam type.
\Prf\ Because $\nu_n$ is purely atomic, there is a sequence
$\sequence{i}{\Cal U_{ni}}$ of subsets of $\Cal U_n$ such that for every
$\Cal V\subseteq\Cal U_n$ there is a $J\subseteq\Bbb N$ such that
$\nu_n(\Cal V\symmdiff\bigcup_{i\in J}\Cal U_{ni})=0$.   Set
$W_{ni}=\bigcup\Cal U_{ni}$ for each $i$.   Let  $\frak B$ be the closed
subalgebra of $\frak A$ generated by
$\{W_{ni}^{\ssbullet}:n$, $i\in\Bbb N\}$.

If $G\subseteq X$ is open, set
$\Cal V_n=\{U:U\in\Cal U_n$, $U\subseteq G\}$ and $G_n=\bigcup\Cal V_n$
for each $n$.   Then we have $J_n\subseteq\Bbb N$ such that

\Centerline{$0
=\nu_n(\Cal V_n\symmdiff\bigcup_{i\in J_n}\Cal U_{ni})
=\mu(G_n\symmdiff\bigcup_{i\in J_n}W_{ni})$,}

\noindent so $G_n^{\ssbullet}=\sup_{i\in J_n}W_{ni}^{\ssbullet}\in\frak B$.
Now $G=\bigcup_{n\in\Bbb N}G_n$ so
$G^{\ssbullet}=\sup_{n\in\Bbb N}G_n^{\ssbullet}$ belongs to $\frak B$.

The set $\Sigma=\{E:E\subseteq X$ is Borel, $E^{\ssbullet}\in\frak B\}$
is a $\sigma$-algebra of subsets of $X$, and we have just seen that it
contains every open set;  so $\Sigma$ is the whole Borel $\sigma$-algebra
and $\frak A=\frak B$ has countable Maharam type.\ \Qed

\medskip

{\bf (d)} Next, if $\ofamily{\xi}{\kappa}{G_{\xi}}$ is a family of
open sets where $\kappa<\frak c$, and $G=\bigcup_{\xi<\kappa}G_{\xi}$,
then $G^{\ssbullet}=\sup_{\xi<\kappa}G_{\xi}^{\ssbullet}$ in $\frak A$.
\Prf\ Look at
$\Cal V_{n\xi}=\{U:U\in\Cal U_n$, $U\subseteq G_{\xi}\}$,
$\Cal V_n=\bigcup_{\xi<\kappa}\Cal V_{n\xi}$ for each $n$.
Because $\nu_n$ is $\frak c$-additive,

\Centerline{$\mu(\bigcup\Cal V_n)=\nu\Cal V_n
=\sup_{J\subseteq\kappa\text{ is finite}}
   \nu(\bigcup_{\xi\in J}\Cal V_{n\xi})$}

\noindent and there is a countable set $J_n\subseteq\kappa$ such that
$\mu(\bigcup\Cal V_n)=\mu(\bigcup_{\xi\in J_n}\bigcup\Cal V_{n\xi})$.
Now

\Centerline{$G^{\ssbullet}
=\sup_{n\in\Bbb N}(\bigcup\Cal V_n)^{\ssbullet}
=\sup_{n\in\Bbb N}\sup_{\xi\in J_n}(\bigcup\Cal V_{n\xi})^{\ssbullet}
\Bsubseteq\sup_{\xi<\kappa}G_{\xi}^{\ssbullet}
\Bsubseteq G^{\ssbullet}$.  \Qed}

\medskip

{\bf (e)} Let $\ofamily{\xi}{\kappa}{E_{\xi}}$ be a
family in $\Cal N(\mu)$ where $\kappa<\add\Cal N_{\omega}$.
Because $\mu$ is inner regular with respect to the closed sets (412D),
we can find closed sets $F_{\xi n}\subseteq X\setminus E_{\xi}$
such that $\mu F_{\xi n}\ge\mu X-2^{-n}$ for $\xi<\kappa$ and $n\in\Bbb N$.
By 524Mb and (c) above, $\wdistr(\frak A)\ge\add\Cal N_{\omega}$,
so there is a countable $C\subseteq\frak A$ such that
$F_{\xi n}^{\ssbullet}
=\sup\{c:c\in C$, $c\subseteq F_{\xi n}^{\ssbullet}\}$
for every $n\in\Bbb N$ and
$\xi<\kappa$ (514K).   Again because $\mu$ is inner
regular with respect to the closed sets, there is a
sequence$\sequence{m}{F_m}$ of closed sets such that whenever
$C'\subseteq C$ is finite then $\bar\mu(\sup C')
=\sup\{\mu F_m:m\in\Bbb N$, $F_m^{\ssbullet}\Bsubseteq\sup C'\}$.
Consequently

\Centerline{$\mu F_{\xi n}
=\sup\{\mu F_m:m\in\Bbb N$,
  $F_m^{\ssbullet}\Bsubseteq F_{\xi n}^{\ssbullet}\}$}

\noindent for every $\xi<\kappa$ and $n\in\Bbb N$.   Set

\Centerline{$H_m
=X\cap\bigcap\{F_{\xi n}:n\in\Bbb N$, $\xi<\kappa$,
$F_m^{\ssbullet}\subseteq F_{\xi n}^{\ssbullet}\}$.}

\noindent Applying (d) to
$\{X\setminus F_{\xi n}:F_m^{\ssbullet}\subseteq F_{\xi n}^{\ssbullet}\}$,
we see that $H_m^{\ssbullet}\Bsupseteq F_m^{\ssbullet}$, that is,
$F_m\setminus H_m$ is negligible, for each $m$.

Each $H_m$ is closed;  let $f_m:X\to[0,1]$ be a continuous function
such that $H_m=f_m^{-1}[\{0\}]$.   Set
$f(x)=\sequence{m}{f_m(x)}\in[0,1]^{\Bbb N}$ for $x\in X$, and let $\nu$ be
the restriction of the image measure $\mu f^{-1}$ to the Borel
$\sigma$-algebra of $[0,1]^{\Bbb N}$.   Then
$\add\Cal N(\nu)\ge\add\Cal N_{\omega}$ (apply 522W(a-i) to the atomless
part of $\nu$).   For each $\xi<\kappa$ and $n\in\Bbb N$, there is an
$m\in\Bbb N$ such that $F_m^{\ssbullet}\Bsubseteq F_{\xi n}^{\ssbullet}$
and $\mu F_m\ge\mu F_{\xi n}-2^{-n}\ge\mu X-2^{-n+1}$;  now
$H_m\subseteq F_{\xi n}$ is disjoint from $E_{\xi}$ and
$\mu H_m\ge\mu X-2^{-n+1}$.
So $\{z:z\in[0,1]^{\Bbb N}$, $z(m)>0\}$ includes
$f[E_{\xi}]$ and has measure at most $2^{-n+1}$.   Accordingly $f[E_{\xi}]$
is $\nu$-negligible.

As $\add\Cal N(\nu)>\kappa$, $\bigcup_{\xi<\kappa}f[E_{\xi}]$ is
$\nu$-negligible;  as $f$ is \imp, $\bigcup_{\xi<\kappa}E_{\xi}$ is
$\mu$-negligible;  as $\ofamily{\xi}{\kappa}{E_{\xi}}$ is arbitrary,
$\add\Cal N(\mu)\ge\add\Cal N_{\omega}$.
}%end of proof of 544K

\leader{544L}{Corollary}  Let $X$ be a metrizable space.

(a) If $\CalUn$ is the $\sigma$-ideal of universally negligible
subsets of $X$, then $\add\CalUn\ge\add\Cal N_{\omega}$.

(b) If $\Sigma_{\text{um}}$ is the $\sigma$-algebra of universally
measurable subsets of $X$, then $\bigcup\Cal E\in\Sigma_{\text{um}}$
whenever $\Cal E\subseteq\Sigma_{\text{um}}$ and
$\#(\Cal E)<\add\Cal N_{\omega}$.

\proof{{\bf (a)} Let $\Cal E\subseteq\CalUn$ be a set with cardinal less
than $\add\Cal N_{\omega}$, and $\mu$ a Borel probability measure on $X$
such that $\mu\{x\}=0$ for every $x\in X$.   Then
$\Cal E\subseteq\Cal N(\mu)$;  by 544K, $\bigcup\Cal E\in\Cal N(\mu)$;
as $\mu$ is arbitrary, $\bigcup\Cal E\in\CalUn$;  as $\Cal E$ is arbitrary,
$\add\CalUn\ge\add\Cal N_{\omega}$.

\medskip

{\bf (b)} Let $\mu$ be a totally finite Borel measure on $X$ and $\hat\mu$
its completion.   By 521Ad and 544K,

\Centerline{$\add\hat\mu
=\add\Cal N(\hat\mu)=\add\Cal N(\mu)\ge\add\Cal N_{\omega}
>\#(\Cal E)$.}

\noindent Since $\hat\mu$ measures every member of $\Cal E$, it
also measures $\bigcup\Cal E$ (521Aa);  as $\mu$ is arbitrary,
$\bigcup\Cal E\in\Sigma_{\text{um}}$.
}%end of proof of 544L

\leader{544M}{Theorem} Let $\kappa$ be an \am\ cardinal.
Then the following are equiveridical:

(i) for every family $\ofamily{\xi}{\kappa}{f_{\xi}}$ of regressive
functions defined on $\kappa\setminus\{0\}$ there is a family
$\ofamily{\xi}{\kappa}{\alpha_{\xi}}$ in $\kappa$ such that

\Centerline{$\{\kappa\setminus\zeta:\zeta<\kappa\}
  \cup\{f_{\xi}^{-1}[\{\alpha_{\xi}\}]:\xi<\kappa\}$}

\noindent has the finite intersection property;

(ii) for every family $\ofamily{\xi}{\kappa}{f_{\xi}}$ in
$\BbbN^{\kappa}$ there is a family
$\ofamily{\xi}{\kappa}{m_{\xi}}$ in $\Bbb N$ such that

\Centerline{$\{\kappa\setminus\zeta:\zeta<\kappa\}
  \cup\{f_{\xi}^{-1}[\{m_{\xi}\}]:\xi<\kappa\}$}

\noindent has the finite intersection property;

(iii) $\cov\Cal N_{\kappa}>\kappa$;

(iv) $\cov\Cal N(\mu)>\kappa$ whenever $(X,\mu)$ is a
locally compact semi-finite measure space and $\mu X>0$.

\wheader{544M}{0}{0}{0}{72pt}
\proof{  Let $\nu$ be a normal witnessing probability on $\kappa$.

\medskip

{\bf (i)$\Rightarrow$(ii)} Given a family $\ofamily{\xi}{\kappa}{f_{\xi}}$
as in (ii), apply (i) to $\ofamily{\xi}{\kappa}{f'_{\xi}}$ where
$f'_{\xi}(\eta)=0$ if $0<\eta<\omega$, $f_{\xi}(\eta)$ if
$\omega\le\eta<\kappa$.

\medskip

{\bf (ii)$\Rightarrow$(iii)} Let $\ofamily{\alpha}{\kappa}{A_{\alpha}}$
be a family in $\Cal N_{\kappa}$.
For each $\alpha<\kappa$ let $\langle F_{\alpha n}\rangle_{n\in\Bbb N}$
be a disjoint sequence of compact subsets of
$\{0,1\}^{\kappa}\setminus A_{\alpha}$
such that $\nu_{\kappa}(\bigcup_{n\in\Bbb N}F_{\alpha n})=1$.
By 543G there is a function
$h:\kappa\to\{0,1\}^{\kappa}$ which is \imp\ for $\nu$ and
$\nu_{\kappa}$.
Set $H_{\alpha}=h^{-1}(\bigcup_{n\in\Bbb N}F_{\alpha n})$;  then
$\nu H_{\alpha}=1$.
Let $H$ be the diagonal intersection of
$\ofamily{\alpha}{\kappa}{H_{\alpha}}$, so that $\nu H=1$.
Let $\langle\gamma_{\xi}\rangle_{\xi<\kappa}$
be the increasing enumeration of $H$.

For $\alpha$, $\xi<\kappa$ set

$$\eqalign{f_{\alpha}(\xi)
&=n\text{ if }h(\gamma_{\xi})\in F_{\alpha n},\cr
&=0\text{ if }\gamma_{\xi}\notin H_{\alpha}.\cr}$$

\noindent Then
there is a family $\ofamily{\alpha}{\kappa}{m_{\alpha}}$ in $\Bbb N$
such that
$\Cal E=\{\kappa\setminus\zeta:\zeta<\kappa\}
  \cup\{f_{\alpha}^{-1}[\{m_{\alpha}\}]:\alpha<\kappa\}$
has the finite intersection property.   Let $\Cal F\supseteq\Cal E$ be an
ultrafilter.
For any $\alpha<\kappa$ we have
$H\setminus H_{\alpha}\subseteq\alpha+1$, so that
$\{\xi:\gamma_{\xi}\notin H_{\alpha}\}$ is bounded above in $\kappa$ and
cannot belong to $\Cal F$.   Consequently
$\{\xi:h(\gamma_{\xi})\in F_{\alpha,m_{\alpha}}\}\in\Cal F$.   But this
implies at once that
$\langle F_{\alpha,m_{\alpha}}\rangle_{\alpha<\kappa}$ has the finite
intersection
property;  because all the  $F_{\alpha n}$ are compact, there is a
$y\in\bigcap_{\alpha<\kappa}F_{\alpha,m_{\alpha}}$, and now
$y\notin\bigcup_{\alpha<\kappa}A_{\alpha}$.

Because $\langle A_{\alpha}\rangle_{\alpha<\kappa}$ was arbitrary,
$\cov\Cal N_{\kappa}>\kappa$.

\medskip

{\bf(iii)$\Rightarrow$(iv)} As in 544B, this follows from 523F and 521Lb.

\medskip

{\bf (iv)$\Rightarrow$(i)} Let $(Z,\tilde\nu)$ be the Stone space of the
measure algebra $\frak A$ of $\nu$;
for $A\subseteq\kappa$ let $A^*$ be the
open-and-closed subset of $Z$ corresponding to
the image $A^{\ssbullet}$ of $A$ in $\frak{A}$.

Now let $\langle f_{\xi}\rangle_{\xi<\kappa}$ be a family of regressive
functions defined on $\kappa\setminus\{0\}$.
Because $\Cal N(\nu)$ is normal and $\omega_1$-saturated
and $f_{\xi}$ is regressive, there is for
each $\xi<\kappa$ a countable set $D_{\xi}\subseteq\kappa$ such that
$\nu f_{\xi}^{-1}[D_{\xi}]=1$ (541Ka).   For $\xi$, $\eta<\kappa$
set $A_{\xi\eta}=f_{\xi}^{-1}[\{\eta\}]$;  then
$\nu(\bigcup_{\eta\in D_{\xi}}A_{\xi\eta})=1$ so (because $D_{\xi}$
is countable) $\sup_{\eta\in D_{\xi}}A^{\ssbullet}_{\xi\eta}=1$ in
$\frak A$ and $\tilde\nu(\bigcup_{\eta\in D_{\xi}}A^*_{\xi\eta})=1$.
Set $E_{\xi}
=Z\setminus\bigcup_{\eta\in D_{\xi}}A^*_{\xi\eta}\in\Cal N(\tilde\nu)$.
By (iv), $Z\ne\bigcup_{\xi<\kappa}E_{\xi}$;
take $z\in Z\setminus\bigcup_{\xi<\kappa}E_{\xi}$.
Then for every $\xi<\kappa$ there must be an $\alpha_{\xi}\in D_{\xi}$
such that $z\in A^*_{\xi,\alpha_{\xi}}$.   But this implies that
$\{A^*_{\xi,\alpha_{\xi}}:\xi<\kappa\}$
is  a centered family of open subsets of $Z$.   It follows that
$\{A^{\ssbullet}_{\xi,\alpha_{\xi}}:\xi<\kappa\}$
is centered in $\frak{A}$.  Since $\nu\zeta=0$ for every $\zeta<\kappa$,
$\{A_{\xi,\alpha_{\xi}}:\xi<\kappa\}
\cup\{\kappa\setminus\zeta:\zeta<\kappa\}$ must have the finite
intersection property, as required.
}%end of proof of 544M

\leader{544N}{Cicho\'n's diagram and other cardinals (a)} Returning to the
concerns of Chapter 52,\cmmnt{ we see from the results above that}
any \am\
cardinal $\kappa$ is necessarily connected with the structures there.
\cmmnt{By 544B,} $\kappa\le\cov\Cal N_{\lambda}$ for every $\lambda$;
\cmmnt{by 544G, $\non\Cal N_{\omega}=\omega_1$, so}
all the cardinals on the bottom line of Cicho\'n's
diagram\cmmnt{ (522B)}, and\cmmnt{ therefore} the Martin numbers $\frak m$,
$\frak p$ etc.\cmmnt{\ (522T)}, must be $\omega_1$, while all the
cardinals on the top line must be at least $\kappa$.
\cmmnt{From 522Ub we see
also that} $\FN(\Cal P\Bbb N)$ must be at least $\kappa$.
\cmmnt{Concerning $\frak b$
and $\frak d$, the position is more complicated.}

\spheader 544Nb If $\kappa$ is an \am\ cardinal, then $\frak b<\kappa$.
\prooflet{\Prf\Quer\ Otherwise, we can choose inductively a family
$\ofamily{\xi}{\kappa}{f_{\xi}}$ in $\BbbN^{\Bbb N}$ such that
$\{n:f_{\xi}(n)\le f_{\eta}(n)\}$ is finite whenever $\eta<\xi<\kappa$.
Let $\nu$ be a witnessing probability measure on $\kappa$.   For $m$,
$i\in\Bbb N$ set $D_{mi}=\{\xi:\xi<\kappa$, $f_{\xi}(m)=i\}$.   Then

$$\eqalign{W
&=\{(\xi,\eta):\eta<\xi<\kappa\}
=\bigcup_{n\in\Bbb N}\bigcap_{m\ge n}
  \{(\xi,\eta):f_{\eta}(m)<f_{\xi}(m)\}\cr
&=\bigcup_{n\in\Bbb N}\bigcap_{m\ge n}
  \bigcup_{i<j}D_{mi}\times D_{mj}\cr}$$

\noindent belongs to $\Cal P\kappa\tensorhat\Cal P\kappa$.   But also

\Centerline{$\int\nu W[\{\xi\}]\nu(d\xi)=0<1
=\int\nu W^{-1}[\{\eta\}]\nu(d\eta)$,}

\noindent so this contradicts Fubini's theorem.\ \Bang\Qed
}%end of prooflet

\spheader 544Nc If $\kappa$ is an \am\ cardinal, then
$\cf\frak d\ne\kappa$.   \prooflet{\Prf\Quer\ Otherwise, let
$A\subseteq\BbbN^{\Bbb N}$ be a cofinal set with cardinal $\frak d$, and
express $A$ as $\bigcup_{\xi<\kappa}A_{\xi}$ where
$\ofamily{\xi}{\kappa}{A_{\xi}}$ is non-decreasing and
$\#(A_{\xi})<\frak d$
for every $\xi<\kappa$.   For each $\xi<\kappa$, we have an
$f_{\xi}\in\BbbN^{\Bbb N}$ such that $f_{\xi}\not\le g$ for any
$g\in A_{\xi}$.   Let $\nu$ be a witnessing probability on $\kappa$.   Then
for each $n\in\Bbb N$ we have an $h(n)\in\Bbb N$ such that
$\nu\{\xi:f_{\xi}(n)\ge h(n)\}\le 2^{-n-2}$.   This defines a function
$h\in\BbbN^{\Bbb N}$.   There must be a $g\in A$
such that $h\le g$;  let $\zeta<\kappa$ be such that $g\in A_{\zeta}$.
The set $\{\xi:f_{\xi}\le h\}$ has measure at least $\bover12$, so there is
some $\xi\ge\zeta$ such that

\Centerline{$f_{\xi}\le h\le g\in A_{\zeta}\subseteq A_{\xi}$,}

\noindent contrary to the choice of $f_{\xi}$.\ \Bang\Qed
}%end of prooflet

\spheader 544Nd\cmmnt{ As for the cardinals studied in \S523,
I have already noted
that $\cov\Cal N_{\lambda}\ge\kappa$ for any \am\ cardinal $\kappa$
and any $\lambda$, and we can say something about the possibility that
$\cov\Cal N_{\lambda}=\kappa$ (544M).
Recall that $\cff[\kappa]^{\le\omega}=\kappa$ (542Ia),
so that }$\cf\Cal N_{\kappa}=\max(\kappa,\cf\Cal N_{\omega})$ for any \qm\
cardinal $\kappa$.


\exercises{\leader{544X}{Basic exercises (a)}
%\spheader 544Xa
Let $\kappa$ be an \am\ cardinal, and $\nu$ a witnessing
probability on $\kappa$.   Show that there is a
set $C\subseteq\{0,1\}^{\kappa}\times\kappa$ such that
$\nu_{\kappa}C^{-1}[\{\xi\}]=0$ for every $\xi<\kappa$, but
$\nu_{\kappa}^*\{x:\nu C[\{x\}]=1\}=1$.
%543C

\spheader 544Xb Suppose that $\kappa$ is an \am\ cardinal.   Show that
$\BbbR^{\lambda}$ is measure-compact for every $\lambda<\kappa$.
\Hint{533J.}
%544B

\spheader 544Xc Let $\kappa$ be a \2vm\ cardinal, $\Cal I$ a normal maximal
ideal of $\Cal P\kappa$, $(X,\mu)$ a
quasi-Radon probability space of weight strictly less than $\kappa$,
and $f:[\kappa]^{<\omega}\to\Cal N(\mu)$ a function.   Show that there is a
$V\in\Cal I$ such that
$\bigcup\{f(I):I\in[\kappa\setminus V]^{<\omega}\}$ is $\mu$-negligible.
\Hint{541Xf.}
%544E  in fact  f  is constant on  \kappa\setminus V]^{<\omega}

\spheader 544Xd In 544F, show that if the magnitude of $\mu$ is less than
$\kappa$ and the augmented shrinking number $\shr^+(\mu)$ is at most
$\kappa$ then there is a $\nu$-conegligible $V\subseteq\kappa$ such that
$\mu_*(\bigcup_{I\in[V]^{<\omega}}f(I))=0$.
%544F

\spheader 544Xe Suppose that there is an \am\ cardinal.   Show that every
Radon measure on a first-countable compact Hausdorff space is uniformly
regular.   \Hint{533Hb.}
%544Ga 533Hb

\spheader 544Xf Suppose that $\kappa$ is an \am\ cardinal and that
$2^{\kappa}=\kappa^{(+n+1)}$.   Show that
$\non\Cal N_{2^{(\kappa^+)}}\le\kappa^+$.   \Hint{523I(a-iii).}
%544H

\spheader 544Xg Suppose that $\kappa$ is an \am\ cardinal and that
$(X,\rho)$ is a metric space.   Show that no subset of $X$ with
strong measure zero can have cardinal $\kappa$.
%544N

\spheader 544Xh Let $(X,\Sigma,\mu)$ be a $\sigma$-finite measure space
such that every subset of $X^2$ is measured by the c.l.d.\
product measure $\mu\times\mu$.   Show that there is a countable subset of
$X$ with full outer measure.   \Hint{if singletons are negligible, consider
a well-ordering of $X$ as a subset of $X^2$.}
%?

\spheader 544Xi Let $\kappa$ be an \am\ cardinal, and $G$ a
group of permutations of $\kappa$  such that $\#(G)<\kappa$.
Show that there is a non-zero
strictly localizable atomless $G$-invariant measure with domain
$\Cal P\kappa$ and magnitude at most $\#(G)$.   \Hint{start with $G$
countable.}
%?
%   (Put a measure $\nu$ on a selector $S$ for the orbits
%of $G$, and set $\mu A=\sum_{g\in G}\nu(S\cap g^{-1}[A])$.)

\leader{544Y}{Further exercises (a)}
%\spheader 544Ya
Let $\kappa$ be a \rvm\ cardinal with witnessing
probability $\nu$.   Give $\kappa$ its discrete topology, so that $\nu$ is
a Borel measure and $\kappa^{\Bbb N}$ is metrizable.
Let $\lambda$ be the
Borel measure on $\kappa^{\Bbb N}$ constructed from $\nu$ by the method
of 434Ym.   (i) Show that if $\kappa$ is \am\ then
$\add\Cal N(\lambda)=\omega_1$.   (ii) Show that if $\kappa$ is \2vm\ then
$\add\Cal N(\lambda)=\kappa$.
%544K 54bits

\spheader 544Yb
Show that $\frak c$ does not have the property of 544M(ii).
%544M

\spheader 544Yc Show that a cardinal $\kappa$ is weakly compact iff it is
strongly inaccessible and has the property (i) of 544M.
%544M \query ref?  try Jech 78, Jech 03, Kanamori 03
}%end of exercises

\leader{544Z}{Problems (a)}
%\spheader 544Za
In 543C, can we replace `$w(X)<\add\nu$' with
`$\tau(\mu)<\add\nu$'?   More concretely, suppose that $(Z,\lambda)$ is the
Stone space of $(\frak B_{\omega},\bar\nu_{\omega})$, $\kappa$ is an \am\
cardinal and $\nu$ a
normal witnessing probability on $\kappa$\dvro{. }{, so that

\Centerline{$\tau(\lambda)=\omega<\add\nu\le\frak c=w(Z)$.}

\noindent}Let
$C\subseteq\kappa\times Z$ be such that $\lambda C[\{\xi\}]=0$ for every
$\xi<\kappa$.   \cmmnt{By 544C, we know that}
$\{z:\nu C^{-1}[\{z\}]>0\}$ has inner
measure zero.   But does it have to be negligible?
%544C

\spheader 544Zb Suppose that $\kappa$ is an \am\ cardinal.
Must there be a Sierpi\'nski set $A\subseteq\{0,1\}^{\omega}$
with cardinal $\kappa$?   (See 552E.)
%544G

\spheader 544Zc Suppose that $\kappa$ is an \am\ cardinal.   Can
$\non\Cal N_{\kappa}$ be greater than $\omega_1$?   What if
$\kappa=\frak c$?   (See 552H.)
%544H

\spheader 544Zd Can there be an \am\ cardinal less than $\frak d$?
(See the notes to \S555.)
%544N

\spheader 544Ze Can there be an \am\ cardinal less than or equal to
$\shr\Cal N_{\omega}$?   (See 555Yd.)
%544N

%to lift $\shr\Cal N_{\omega}$, start with a \2vm\ $\kappa$,
%force with ccc forcing to get $\frak m=\lambda$, now add $\kappa$ random
%reals;  the reals of the intermediate model give a set $A$ with cardinal
%$\lambda$ of full outer measure witnessing that
%$\shr\Cal N_{\omega}\ge\lambda$ (because every subset of $A$ in the final
%model is included in a countable union of subsets of $A$ in the
%intermediate model of the same cardinality).

\spheader 544Zf Suppose that there is an \am\ cardinal.   Does it follow
that $\cov\Cal N_{\omega}=\frak c$?   \cmmnt{(See 552Gc.)}

\endnotes{
\Notesheader{544} The vocabulary of this section (`locally compact
semi-finite measure space', `quasi-Radon probability space of weight less
than $\kappa$', `compact probability space with Maharam type less than
$\kappa$') makes significant demands on the reader, especially the reader
who really wants to know only what happens to Lebesgue measure.   But the
formulations I have chosen are not there
just on the off-chance that someone may
wish to apply the results in unexpected contexts.
I have tried to use the concepts established earlier in this treatise to
signal the nature of the arguments used at each stage.   Thus in 543C we
had an argument which depended on topological ideas, and could work only on
a space with a base which was
small compared with the \am\ cardinal in hand;  in 544C, the argument
depends on an \imp\ function from some power $\{0,1\}^{\lambda}$, so
requires a compact measure, but then finds a $\Delta$-nebula
with a countable root-cover $J$, so that 543C
can be applied to the usual measure on $\{0,1\}^J$, irrespective of the
size of $\lambda$.   Similar, but to my mind rather deeper, ideas
lead from 544E to 544F.   In both cases, there is a price to be paid for
moving to spaces $X$ of arbitrary complexity;  in one, an inequality
$\overline{\int}\int\le\int\overline{\int}$ becomes the weaker
$\underline{\int}\int\le\int\overline{\int}$;  in the other, a negligible
set turns into a set of inner measure zero (544Xd).

Another way to classify the results here is to ask which of them depend on
the Gitik-Shelah theorem.   The formulae in 544H betray such a dependence;
but it seems that the Gitik-Shelah theorem is also needed for the full
strength of 544B, 544G, 544J and 544M as written.
Historically this is significant,
because the idea behind 544G was worked out by K.Prikry and R.M.Solovay
before it was known for sure
that a witnessing measure on an \am\ cardinal could not
have countable Maharam type.   However 544B and 544J, for instance,
can be proved for Lebesgue measure without using the Gitik-Shelah theorem.

In the next chapter I will present a description of measure
theory in random real models.   Those already familiar with random real
forcing may recognise some of the theorems of this section (544G,
544N) as versions of characteristic results from this theory (552E, 552C).

544M is something different.
It was recognised in the 1960s that some of the ways in which \2vm\
cardinals are astonishing is that they are
`weakly $\Pi^1_1$-indescribable' (and, moreover, have many
weakly $\Pi^1_1$-indescribable cardinals below them;  see
{\smc Fremlin 93}, 4K).   I do
not give the `proper' definition of weak $\Pi^1_1$-indescribability, which
relies on concepts from model theory;  you may find it in {\smc Levy 71},
{\smc Baumgartner Taylor \& Wagon 77} or {\smc Fremlin 93};  for our
purposes here, the equivalent combinatorial definition in 544M(i) will
I think suffice.   For strongly inaccessible cardinals, it is the same
thing as weak compactness (544Yc).   Here I mention it only because it
turns out to be related to one of the standard questions I have been asking
in this volume (544M(iii)).   Of course the arguments above beg the
question, whether an \am\ cardinal can be weakly $\Pi^1_1$-indescribable,
especially in view of 544Yb;  see {\smc Fremlin 93}, 4R.

In 544K-544L I look at a question which seems to belong in Chapter 52,
or perhaps with the corresponding result in Hausdorff measures (534Bb).
But unless I am missing something,
the facts here depend on the Gitik-Shelah theorem via 544Ha.

This section has a longer list of problems than most.   In the last four
sections I have tried
to show something of the richness of the structures associated with any
\am\ cardinal;  I remain quite uncertain how much more we can hope to glean
from the combinatorial and measure-theoretic arguments available.
The problems of this chapter mostly have a special status.   They are
of course vacuous unless we suppose that there is an \am\ cardinal;  but
there is something else.   There is a well-understood process,
`Solovay's method', for building
models of set theory with \am\ cardinals from models with
\2vm\ cardinals (\S555).
In most cases, the problems have been solved for such models, and perhaps
they should be regarded as challenges to develop new forcing techniques.
}%end of notes

\discrpage




