\frfilename{mt536.tex}
\versiondate{20.2.12}
\copyrightdate{2003}

\def\chaptername{Topologies and measures III}
\def\sectionname{Alexandra Bellow's problem}

\newsection{536}

In 463Za I mentioned a curious problem concerning pointwise compact sets
of continuous functions.   This problem is known to be soluble if we are
allowed to assume the continuum hypothesis, for instance.   Here I
present the relevant arguments.

\leader{536A}{The problem}\cmmnt{ I recall some ideas from \S463.}
Let $(X,\Sigma,\mu)$ be a measure space, and
$\eusm L^0\cmmnt{\mskip5mu =\eusm L^0(\Sigma)}$ the space of all
$\Sigma$-measurable functions from $X$ to $\Bbb R$\cmmnt{, so that
$\eusm L^0$ is a linear subspace of $\BbbR^X$}.   On $\eusm L^0$ we have
the linear space topologies $\frak T_p$ and $\frak T_m$ of pointwise
convergence and convergence in measure\cmmnt{ (462Ab, 245Ab)}.
$\frak T_p$ is Hausdorff and locally convex;  if $\mu$ is $\sigma$-finite,
$\frak T_m$ is pseudometrizable.
The question\cmmnt{, already asked in 463Za,} is this:  suppose that
$K\subseteq\eusm L^0$ is compact for $\frak T_p$,
and that $\frak T_m$ is Hausdorff on $K$.   Does it follow that
$\frak T_p$ and $\frak T_m$ agree on $K$?

\leader{536B}{Known cases} Let $(X,\Sigma,\mu)$ be a $\sigma$-finite
measure space.   Given that $K\subseteq\eusm L^0$ is compact
for $\frak T_p$, and $\frak T_m$ is Hausdorff on $K$, and

\inset{{\it either} $K$ is sequentially compact for $\frak T_p$

{\it or} $K$ is countably tight for $\frak T_p$

{\it or} $K$ is convex

{\it or} $X$ has a topology for which $K\subseteq C(X)$,
$\mu$ is a strictly positive topological measure, and every function
$h\in\BbbR^X$ which is continuous on every relatively countably compact
set is continuous

{\it or} $\mu$ is perfect,}

\noindent then $K$ is metrizable for $\frak T_p$, and $\frak T_p$ and
$\frak T_m$ agree on $K$\cmmnt{ (463Cd, 463F, 463G, 463H,
463Lc)}.

\cmmnt{Now for the new results.}

\leader{536C}{Proposition}\dvAnew{2013}\cmmnt{ (see {\smc Talagrand 84},
9-3-3.)}   Let $(X,\Sigma,\mu)$ be a probability space
such that the $\pi$-weight $\pi(\mu)$ of $\mu$ is at most $\frak p$.   If
$K\subseteq\eusm L^0$ is $\frak T_p$-compact then it is
$\frak T_m$-compact.

\proof{{\bf (a)} For the time being (down to the end of (d) below),
suppose that $|f|\le\chi X$ for every $f\in K$.   Let $\sequence{i}{f_i}$
be any sequence in $K$.

\medskip

{\bf (b)} For $I\in[\Bbb N]^{\omega}$, write
$\limsup_{i\to I}f_i$ for $\inf_{n\in\Bbb N}\sup_{i\in I\setminus n}f_i$
and
$\liminf_{i\to I}f_i$ for $\sup_{n\in\Bbb N}\inf_{i\in I\setminus n}f_i$.
Then there is an $I\in[\Bbb N]^{\omega}$ such
that $\liminf_{i\to J}f_i\eae\liminf_{i\to I}f_i$ and
$\limsup_{i\to J}f_i\eae\limsup_{i\to I}f_i$ for every $J\in[I]^{\omega}$.
\Prf\ (See the proof of 463D.)   For $I$, $J\in[\Bbb N]^{\omega}$ set
$\Delta(I)=\int\limsup_{i\to I}f_i-\liminf_{i\to I}f_i$ and say that
$J\preceq I$ if either $J\subseteq I$ or
$J\setminus I$ is finite and $I\setminus J$
is infinite.   Then $\Delta(J)\le\Delta(I)$ whenever $J\preceq I$, and
any non-increasing sequence in $[\Bbb N]^{\omega}$ has a $\preceq$-lower
bound
in $[\Bbb N]^{\omega}$.   By 513P, inverted, there is an
$I\in[\Bbb N]^{\omega}$ such
that $\Delta(J)=\Delta(I)$ whenever $J\preceq I$, and this $I$ will serve.\
\Qed

Set $g=\liminf_{i\to I}f_i$ and $h=\limsup_{i\to I}f_i$.

\medskip

{\bf (c)} \Quer\ Suppose, if possible, that $E=\{x:g(x)<h(x)\}$ is not
negligible.   Let $\Cal H$ be a coinitial subset of
$\Sigma\setminus\Cal N(\mu)$, where $\Cal N(\mu)$ is the null ideal of
$\mu$, with cardinal $\pi(\mu)\le\frak p$, and
$\ofamily{\xi}{\frak p}{H_{\xi}}$ a family running over
$\{H:H\in\Cal H$, $H\subseteq E\}$.
Choose $\ofamily{\xi}{\frak p}{I_{\xi}}$,
$\ofamily{\xi}{\frak p}{x_{\xi}}$ and
$\ofamily{\xi}{\frak p}{y_{\xi}}$ inductively, as follows.
The inductive hypothesis will be that, for any $\xi<\frak p$,
$\ofamily{\eta}{\xi}{I_{\eta}}$ is a family of infinite subsets of
$\Bbb N$ such that $I_{\eta}\setminus I_{\zeta}$ is finite whenever
$\zeta\le\eta<\xi$.   Start with
$I_0=I$.   For the inductive step to $\xi+1$, where $\xi<\frak p$,
since $g\eae\liminf_{i\to I_{\xi}}f_i$, there must be
an $x_{\xi}\in H_{\xi}\cap E$ such that
$g(x_{\xi})=\liminf_{i\to I_{\xi}}f_i(x)$.
Let $J\in[I_{\xi}]^{\omega}$ be such that
$\lim_{i\to J}f_i(x_{\xi})=g(x_{\xi})$.
Now $\limsup_{i\to J}f_i\eae h$, so we can find a
$y_{\xi}\in E\cap H_{\xi}$ such that
$\limsup_{i\to J}f_i(y_{\xi})=h(y_{\xi})$ and an
$I_{\xi+1}\in[J]^{<\omega}$ such that
$\lim_{i\to I_{\xi+1}}f_i(y_{\xi})=h(y_{\xi})$.

For non-zero limit ordinals $\xi<\frak p$, let $I_{\xi}$ be an infinite
subset of $I$ such that $I_{\xi}\setminus I_{\eta}$ is finite for every
$\eta<\xi$.

At the end of the induction, there will be a non-principal ultrafilter
$\Cal F$ on $\Bbb N$ containing $I_{\xi}$ for every $\xi<\frak p$.
Set $f=\lim_{i\to\Cal F}f_i$.   Because $K$ is $\frak T_p$-compact,
$f\in K\subseteq\eusm L^0$.   So at least one of the measurable sets
$E'=\{x:x\in E$, $g(x)<f(x)\}$ and $E''=\{x:x\in E$, $f(x)<h(x)\}$ is
non-negligible and contains $H_{\xi}$ for some $\xi<\frak p$.   Now
$I_{\xi+1}\in\Cal F$, so
$f(x_{\xi})=\lim_{i\to I_{\xi+1}}f_i(x_{\xi})=g(x_{\xi})$ and
$f(y_{\xi})=h(y_{\xi})$.   But this means that
$x_{\xi}\in H_{\xi}\setminus E''$ and $y_{\xi}\in H_{\xi}\setminus E'$, so
$H_{\xi}$ cannot be included in either $E'$ or $E''$.\ \Bang

\medskip

{\bf (d)} So $g\eae h$ and $\{x:g(x)=\lim_{i\to I}f_i(x)\}$ includes the
conegligible set $\{x:g(x)=h(x)\}$.
We also have a $g_0\in K$ which is a $\frak T_p$-cluster point of
$\langle f_i\rangle_{i\in I}$.   Of course $g\le g_0\le h$, and
all three must be equal $\mu$-a.e.
But this means that $\langle f_i\rangle_{i\in I}$ converges
almost everywhere to $g_0$, and
therefore converges in measure to $g_0$ (245Ec).
Now recall that
$\sequence{i}{f_i}$ was an arbitrary sequence in $K$.   So we see that
every sequence in $K$ has a subsequence which is $\frak T_m$-convergent to
a point of $K$.   As $\frak T_m$ is pseudometrizable, $K$ is
$\frak T_m$-compact (4A2Le).

\medskip

{\bf (e)} This concludes the proof when $|f|\le\chi X$ for every $f\in K$.
For the general case, let $\phi:\Bbb R\to\ooint{-1,1}$ be a homeomorphism,
and consider $K'=\{\phi f:f\in K\}$.   Since
$f\mapsto\phi f$ is a $\frak T_p$-continuous function from $\eusm L^0$
to itself, $K'$ is $\frak T_p$-compact,
therefore $\frak T_m$-compact, by (a)-(c).
Next, $f\mapsto\phi^{-1}f:K'\to K$
is $\frak T_m$-continuous.  \Prf\ If $\sequencen{f_n}$ is a sequence in
$K'$ which is $\frak T_m$-convergent to $f\in K'$, and
$\sequencen{g_n}$ is a subsequence of $\sequencen{f_n}$, then
$\sequencen{g_n}$ has a sub-subsequence $\sequencen{h_n}$ converging a.e.\
to $f$ (245Ka);
now $\phi^{-1}h_n$ converges a.e.\ to $\phi^{-1}f\in K$, so
converges in measure to $\phi^{-1}f$.   As $\sequencen{g_n}$ is arbitrary,
$\sequencen{\phi^{-1}f_n}$ converges in measure to $\phi^{-1}f$.
Thus $f\mapsto\phi^{-1}f$ is sequentially continuous for $\frak T_m$,
therefore continuous (4A2Ld).\ \QeD\  So $K=\{\phi^{-1}f:f\in K'\}$ is
$\frak T_m$-compact, as claimed.
}%end of proof of 536C

\leader{536D}{Theorem}\dvAformerly{5{}36C}
Let $(X,\Sigma,\mu)$ be a probability space, and
$\eusm L^0$ the space of $\Sigma$-measurable real-valued functions on
$X$.   Write $\frak T_p$, $\frak T_m$ for the topologies of pointwise
convergence and convergence in measure on $\eusm L^0$.   Suppose that
$K\subseteq\eusm L^0$ is $\frak T_p$-compact and that
$\mu\{x:f(x)\ne g(x)\}>0$ for any distinct $f$, $g\in K$, but that $K$
is not $\frak T_p$-metrizable.

(a) Every infinite Hausdorff space which is a continuous image of a
closed subset of $K$ has a non-trivial convergent sequence.

(b) There is a continuous surjection from a closed subset of $K$ onto
$\{0,1\}^{\omega_1}$.

(c) Every infinite compact Hausdorff space of weight at most $\omega_1$
has a non-trivial convergent sequence.

(d) $\frak c>\omega_1$.

(e) The Maharam type of $\mu$ is at least $2^{\omega_1}$.

(f) There is a non-negligible measurable set in $\Sigma$ which can be
covered by $\omega_1$ negligible sets.

(g)\dvAnew{2013} $\pi(\mu)>\frak p$.

(h) $\frakmctbl=\omega_1$.

\proof{ For $f$, $g\in\eusm L^0$ set
$\rho(f,g)=\int\min(1,|f-g|)$;  then $\rho$ is a pseudometric on
$\eusm L^0$ defining $\frak T_m$, and
$\rho\restr K\times K$ is a metric on $K$.
Set $\Delta(\emptyset)=0$, and for non-empty $A\subseteq\eusm L^0$ set
$\Delta(A)=\sup\{\rho(\inf L,\sup L):\emptyset\ne L\in[A]^{<\omega}\}$.
Note that if $A\subseteq K$ has more than one member then $\Delta(A)>0$,
and that $\Delta(A)\le\Delta(B)$ whenever $A\subseteq B$.

\medskip

{\bf (a)(i)} \Quer\ Suppose, if possible, that $Z$ is an infinite
Hausdorff space, $K_0\subseteq K$ is closed, $\phi:K_0\to Z$ is a
continuous surjection and there is no non-trivial
convergent sequence in $Z$.   Write $\Cal L$ for the family of closed
subsets $L$ of $K_0$ such that $\phi[L]$ is infinite.   Then
$L=\bigcap_{n\in\Bbb N}L_n$ belongs to
$\Cal L$ for every non-increasing sequence $\sequencen{L_n}$ in
$\Cal L$.   \Prf\ $\sequencen{\phi[L_n]}$ is a non-increasing sequence of
infinite closed
subsets of $Z$;  because $Z$ is supposed to have no non-trivial
convergent sequence, $M=\bigcap_{n\in\Bbb N}\phi[L_n]$ is infinite
(4A2G(h-i)).   Since $\phi[L]=M$ (5A4Cf), $L\in\Cal L$.\ \QeD\
By 513P again, there is a
$K_1\in\Cal L$ such that $\Delta(L)=\Delta(K_1)$ for every $L\in\Cal L$
such that $L\subseteq K_1$.

\medskip

\quad{\bf (ii)} Now there is no non-trivial convergent sequence in
$\phi[K_1]$, so $\phi[K_1]$ cannot be scattered (4A2G(h-ii)), and there is
a continuous surjection $\psi:\phi[K_1]\to[0,1]$ (4A2G(j-iv)).   Let
$M\subseteq\phi[K_1]$ be a closed set such that $\psi[M]=[0,1]$ and
$\psi\restr M$ is irreducible (4A2G(i-i)).   Then $M$ is infinite,
has a countable $\pi$-base and no isolated points (4A2G(i-ii)).
Let $K_2\subseteq\phi^{-1}[M]$ be a closed set such that $\phi[K_2]=M$
and $\phi\restr K_2$ is irreducible.   Then $K_2$ has a countable
$\pi$-base, and $\phi[K_2]$ is infinite, so $\Delta[K_2]=\Delta[K_1]$.

Let $\Cal V$ be a countable $\pi$-base for the topology of $K_2$, not
containing $\emptyset$.   For each $V\in\Cal V$, choose
$h_V\in V$.    Set $g_0=\inf_{V\in\Cal V}h_V$,
$g_1=\sup_{V\in\Cal V}h_V$ in $\BbbR^X$.   Then $g_0$ and $g_1$
are measurable, and

\Centerline{$\int g_1-g_0\ge\Delta(K_2)=\Delta(K_1)>0$.}

\noindent Set $g(x)=\max(\bover12(g_0(x)+g_1(x)),g_1(x)-\bover12)$ for
$x\in X$, and

\Centerline{$E=\{x:g_0(x)<g_1(x)\}=\{x:g(x)<g_1(x)\}
=\{x:g_0(x)<g(x)\}$,}

\noindent so that
$\mu E>0$.   For $x\in E$, the set $F_x=\{f:f\in K_2,\,f(x)\le g(x)\}$
is a proper closed subset of $K_2$, so there is
some $V\in\Cal V$ such that $V\cap F_x=\emptyset$.   Because
$\Cal V$ is countable, there is a $V\in\Cal V$ such that
$D=\{x:x\in E,\,V\cap F_x=\emptyset\}$ is non-negligible.
But now observe that
$f(x)>g(x)$ whenever $f\in V$ and $x\in D$, so
$h_U(x)>g(x)$
whenever $U\in\Cal V$, $U\subseteq V$ and $x\in D$.   Set
$\Cal V'=\{U:U\in\Cal V$, $U\subseteq V\}$,
$g'_0=\inf_{U\in\Cal V'}h_U$ and
$L=\{f:f\in K_2$, $g'_0\le f\le g_1\}$.   Then $g\le g'_0$ and

\Centerline{$\{x:x\in X$, $g_1(x)-g'_0(x)<\min(1,g_1(x)-g_0(x))\}
\supseteq D$}

\noindent is non-negligible, so

\Centerline{$\Delta(L)\le\int\min(1,g_1-g'_0)
<\int\min(1,g_1-g_0)=\Delta(K_1)$.}

\noindent On the other hand, $L$ meets every member of $\Cal V'$,
so $L\cap V$ is dense in $V$ and $L$ includes $V$.   Because
$\phi\restr K_2$ is irreducible, $\phi[K_2\setminus V]\ne M$ and
$\phi[L]$ includes the non-empty open subset
$M\setminus\phi[K_2\setminus V]$ of $M$, which is infinite because
$M$ has no isolated points.   So $\Delta(L)$ ought to be equal to
$\Delta(K_1)$, by the choice of $K_1$.\ \Bang

Thus (a) is true.

\medskip

{\bf (b)} If $\sequencen{f_n}$ is a sequence in $K$ which converges
at almost every point of $X$, then any two $\frak T_p$-cluster points of
$\sequencen{f_n}$ must be equal a.e.\ and therefore equal, so
$\sequencen{f_n}$ is $\frak T_p$-convergent (5A4Ce).

\Quer\ Suppose, if possible, that there is no continuous
surjection from a closed subset of $K$ onto $\{0,1\}^{\omega_1}$.   Then
463D tells us that every sequence in $K$ has a subsequence
which is convergent almost everywhere, therefore convergent.
So $K$ is sequentially compact, which is impossible, as noted in 536B.\
\Bang

\medskip

{\bf (c)} Since $[0,1]$ is a continuous image of $\{0,1\}^{\Bbb N}$,
$[0,1]^{\omega_1}$ is a continuous image of
$\{0,1\}^{\omega_1\times\Bbb N}\cong\{0,1\}^{\omega_1}$ and therefore of
a closed subset of $K$.   If $Z$ is an infinite compact Hausdorff space
of weight at most $\omega_1$, it is homeomorphic to a closed subset of
$[0,1]^{\omega_1}$ (5A4Cc) and therefore to a continuous image of a
closed subset of $K$.   By (a), $Z$ must have a non-trivial convergent
sequence.

\medskip

{\bf (d)} Since $\beta\Bbb N$ has weight $\frak c$ (5A4Ia), is infinite,
but has no non-trivial convergent subsequence (4A2I(b-v)), we must have
$\omega_1<\frak c$.

\medskip

{\bf (e)(i)} If $F_1$, $F_2$ are disjoint non-empty $\frak T_p$-closed
subsets of $K$, then $\rho(F_1,F_2)>0$.   \Prf\Quer\ Otherwise, there
are sequences $\sequencen{f_n}$ in $F_1$, $\sequencen{g_n}$ in $F_2$
such that $\rho(f_n,g_n)\le 2^{-n}$ for every $n\in\Bbb N$.   Let
$\Cal F$ be any non-principal ultrafilter on
$\Bbb N$ and set $f=\lim_{n\to\Cal F}f_n$, $g=\lim_{n\to\Cal F}g_n$,
taking the limits in $K$ for the topology $\frak T_p$.   Then, for any
$n\in\Bbb N$,

\Centerline{$\{x:|f(x)-g(x)|>2^{-n}\}
\subseteq\bigcup_{i\ge 2n}\{x:|f_i(x)-g_i(x)|>2^{-n}\}$}

\noindent has measure at most $\sum_{i=2n}^{\infty}2^{-i+n}=2^{-n+1}$,
so $f\eae g$ and $f=g$;  but $f\in F_1$ and $g\in F_2$, so this is
impossible.\ \Bang\Qed

\medskip

\quad{\bf (ii)} By (b), there are a closed subset $K_0$ of $K$ and a
continuous surjection $\psi:K_0\to\{0,1\}^{\omega_1}$.   For
$\xi<\omega_1$, set $F_{\xi}=\{f:f\in K_0$, $\psi(f)(\xi)=0\}$,
$F'_{\xi}=\{f:f\in K_0$, $\psi(f)(\xi)=1\}$;  then
$\rho(F_{\xi},F'_{\xi})>0$.   There must therefore be a $\delta>0$ such
that $C=\{\xi:\rho(F_{\xi},F'_{\xi})\ge\delta\}$ is uncountable.   For
each $D\subseteq C$, choose $h_D\in K_0$ such that
$\psi(h_D)\restr C=\chi D$.   Then $\rho(h_D,h_{D'})\ge\delta$ for all
distinct $D$, $D'\subseteq C$.   Thus
$A=\{h_D^{\ssbullet}:D\subseteq C\}$ is a subset of $L^0=L^0(\mu)$, of
cardinal $2^{\omega_1}$, such that any two members of $A$ are distance
at least $\delta$ apart for the metric on $L^0$ corresponding to $\rho$.
Accordingly the cellularity and topological density of $L^0$ are at
least $2^{\omega_1}$;  by 529Bb, the Maharam type of $\mu$ is at least
$2^{\omega_1}$.

\medskip

{\bf (f)(i)} By (b), there is a continuous surjection
$\psi:K_0\to\{0,1\}^{\omega_1}$ where $K_0\subseteq K$ is closed.
Let $Q$ be the set of pairs $(F,C)$ such that $F\subseteq K_0$ is
closed, $C\subseteq\omega_1$ is closed and cofinal and
$\{\psi(f)\restr C:f\in F\}=\{0,1\}^C$.   If $\sequencen{(F_n,C_n)}$ is
a non-increasing sequence in $Q$, then it has a lower bound in $Q$.
\Prf\ Set $F=\bigcap_{n\in\Bbb N}F_n$ and $C=\bigcap_{n\in\Bbb N}C_n$.
Then for any $z\in\{0,1\}^C$ and $n\in\Bbb N$ there is an $f_n\in F_n$
such that $\psi(f_n)\restr C=z$;  now take a $\frak T_p$-cluster point
$f$ of $\sequencen{f_n}$, and see that $f\in F$ and that
$\psi(f)\restr C=z$.  As $z$ is arbitrary, $(F,C)\in Q$.\ \QeD\
By 513P once more, there is a member $(K_1,C^*)$ of $Q$
such that $\Delta(F)=\Delta(K_1)$ whenever $(F,C)\in Q$,
$F\subseteq K_1$ and $C\subseteq C^*$.   Now $C^*$ is order-isomorphic
to $\omega_1$ and its order topology agrees with the subspace topology
induced by the order topology of $\omega_1$ (4A2Rm).   Let
$\theta:\omega_1\to C^*$ be an order-isomorphism and set
$\psi_1(f)=\psi(f)\theta$ for $f\in K_1$.   Then
$\psi_1:K_1\to\{0,1\}^{\omega_1}$ is a continuous surjection, and if
$F\subseteq K_1$ is closed, $C\subseteq\omega_1$ is closed and cofinal
and $\{\psi_1(f)\restr C:f\in F\}=\{0,1\}^C$, then $(F,\theta[C])\in Q$
so $\Delta(F)=\Delta(K_1)$.

\medskip

\quad{\bf (ii)} Let $K_2\subseteq K_1$ be a compact set such that
$\psi_1\restr K_2$ is an irreducible surjection onto
$\{0,1\}^{\omega_1}$ (4A2G(i-i) again).
Because $\{0,1\}^{\omega_1}$ is separable
(4A2B(e-ii)), so is $K_2$ (5A4C(d-i)).   Let $\sequencen{f_n}$ enumerate a
dense subset of $K_2$.   Because $K_2$ is compact in $\Bbb R^X$,
$h_1=\sup_{n\in\Bbb N}f_n$ and $h_0=\inf_{n\in\Bbb N}f_n$ are defined in
$\BbbR^X$, and of course they belong to $\eusm L^0$.   If $f\in K_2$,
then

\Centerline{$f(x)\in\overline{\{f_n(x):n\in\Bbb N\}}
\subseteq[h_0(x),h_1(x)]$}

\noindent for every $x$, and $h_0\le f\le h_1$.   Accordingly we have

\Centerline{$\Delta(K_2)\le\rho(h_0,h_1)
=\sup_{n\in\Bbb N}\rho(\inf_{i\le n}f_i,\sup_{i\le n}f_i)
\le\Delta(K_2)$.}

Let $\Cal U$ be the family of
non-empty cylinder sets in $\{0,1\}^{\omega_1}$.   For $U\in\Cal U$ set
$I_U=\{n:n\in\Bbb N$, $\psi_1(f_n)\in U\}$ and
$g_U=\inf\{f_n:n\in I_U\}$.   Observe that
$F_U=\{f:f\in K_2$, $g_U\le f\le h_1\}$ is a closed subset of $K_1$ and
that $F_U\cap\psi_1^{-1}[U]$ is dense in $\psi_1^{-1}[U]$, so
$U\cap\psi_1[F_U]$ must be dense in $U$ and $U\subseteq\psi_1[F_U]$.
There is a finite set $I\subseteq\omega_1$ such that $U$ is determined
by coordinates in $I$;  in this case, $C=\omega_1\setminus I$ is closed
and cofinal in $\omega_1$, and $\{z\restr C:z\in U\}=\{0,1\}^C$.   By
the choice of $K_1$, $\Delta(F_U)=\Delta(K_1)$.   As
$F_U\subseteq[g_U,h_1]$ in $\eusm L^0$,
$\rho(g_U,h_1)=\Delta(K_1)=\rho(h_0,h_1)$, and
$\min(1,h_1-g_U)\eae\min(1,h_1-h_0)$.

Set $h(x)=\max(\bover12(h_0(x)+h_1(x)),h_1(x)-\bover12)$ for $x\in X$,
and $E=\{x:h_0(x)<h_1(x)\}=\{x:h(x)<h_1(x)\}$, so that $E$ is measurable
and not negligible.   If  $U\in\Cal U$, then

$$\eqalign{E_U
&=\{x:x\in E,\,h(x)\le g_U(x)\}\cr
&\subseteq\{x:x\in E,\,h_1(x)-g_U(x)<\min(1,h_1(x)-h_0(x))\}\cr}$$

\noindent is negligible.

For every $x\in E$,
$F'_x=\{f:f\in K_2$, $f(x)\le h(x)\}$ is a proper closed subset of
$K_2$, so $\psi_1[F'_x]\ne\{0,1\}^{\omega_1}$ and there is some
$U\in\Cal U$ such that $U\cap\psi_1[F'_x]=\emptyset$.   In this case
$f_n\notin F'_x$, that is, $f_n(x)>h(x)$, for every $n\in I_U$, so
$g_U(x)\ge h(x)$.   Thus $E=\bigcup_{U\in\Cal U}E_U$ is a non-negligible
measurable set covered by $\omega_1$ negligible sets.

\medskip

{\bf (g)} This is immediate from 536C.

\medskip

{\bf (h)} Continuing the argument from (f), define
$\phi:X\to\BbbR^{\Bbb N}$ by setting $\phi(x)=\sequencen{f_n(x)}$ for
$x\in X$.   Then $\phi$ is measurable (418Bd), so we have a non-zero
totally finite Borel measure $\nu$ on $\BbbR^{\Bbb N}$ defined by
setting $\nu H=\mu(E\cap\phi^{-1}[H])$ for every Borel set
$H\subseteq\BbbR^{\Bbb N}$.   Note that $\phi[X]\subseteq\ell^{\infty}$
and that $\ell^{\infty}
=\bigcup_{n\in\Bbb N}\bigcap_{i\in\Bbb N}\{w:|w(i)|\le n\}$ is an
F$_{\sigma}$ set in $\BbbR^{\Bbb N}$.   Now set

\Centerline{$h'_1(w)=\sup_{n\in\Bbb N}w(n)$,
\quad$h'_0(w)=\inf_{n\in\Bbb N}w(n)$,}

\Centerline{$h'(w)=\max(\bover12(h'_0(w)+h'_1(w)),h'_1(w)-\bover12)$}

\noindent for $w\in\ell^{\infty}$, so that $h_1=h'_1\phi$,
$h_0=h'_0\phi$ and $h=h'\phi$;  for $U\in\Cal U$, set

\Centerline{$E'_U
=\{w:w\in\ell^{\infty}$, $h'(w)\le\inf_{n\in I_U}w(n)\}$}

\noindent so that $E'_U$ is an F$_{\sigma}$ set and
$E_U=E\cap\phi^{-1}[E'_U]$;  accordingly $\nu E'_U=0$.   Because
$E\subseteq\bigcup_{U\in\Cal U}E_U$,
$\phi[E]\subseteq\bigcup_{U\in\Cal U}E'_U$.

Thus we have a non-negligible subset of $\BbbR^{\Bbb N}$ which is
covered by $\omega_1$ negligible F$_{\sigma}$ sets and therefore by
$\omega_1$ closed negligible sets.   By 526M, $\frakmctbl=\omega_1$.
}%end of proof of 536D

\exercises{\vleader{36pt}{536X}{Basic exercises (a)}
%\spheader 536Xa
Let $(X,\Sigma,\mu)$ be a complete measure space, with null ideal
$\Cal N(\mu)$.   Suppose that $\add\Cal N(\mu)=\cov\Cal N(\mu)$.
Show that there is a $\frak T_p$-compact
$\frak T_m$-compact $K\subseteq\eusm L^0(\Sigma)$
such that the identity map on $K$ is not
$(\frak T_p,\frak T_m)$-continuous.
%463Xh

\spheader 536Xb Let $(X,\Sigma,\mu)$ be a perfect measure space.
For $E\subseteq X$, write $\Cal N(\mu_E)$ for the null ideal of the
subspace measure on $E$.   Suppose that
$\non(E,\Cal N(\mu_E))<\cov(E,\Cal N(\mu_E))$ for every non-negligible
measurable set $E$ of finite measure.   Show that if
$K\subseteq\eusm L^0(\Sigma)$ is $\frak T_p$-compact, then the identity
map on $K$ is $(\frak T_p,\frak T_m)$-continuous.
%536Xa
%463Lb  K  is  \frakT_m-compact

\leader{536Y}{Further exercises (a)}
%\spheader 536Ya
Suppose that the additivity and covering number of the Lebesgue
null ideal are equal.   Find a strictly
localizable perfect measure space $(X,\Sigma,\mu)$ and a
$\frak T_p$-compact $K\subseteq\eusm L^0(\Sigma)$ such that $\frak T_m$
is Hausdorff on $K$ but $K$ is not $\frak T_m$-compact.
%463L mt53bits
}%end of exercises

\endnotes{
\Notesheader{536} The methods here are derived from ideas of
M.Talagrand.   They seem frustratingly close to delivering an answer to
the original question.   But it seems clear that even if a positive
answer -- every $\frak T_p$-compact
$\frak T_m$-separated set is metrizable -- is true in ZFC, some further idea will be needed in the proof.   On the other side, while it may well
be that in some familiar model of set theory there is a negative answer,
parts (c), (d) and (g) of 536D give simple tests to rule out many
candidates.
}%end of notes

\discrpage

