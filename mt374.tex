\frfilename{mt374.tex}
\versiondate{15.6.09}
\copyrightdate{1996}

\def\chaptername{Linear operators between function spaces}
\def\sectionname{Rearrangement-invariant spaces}

\def\ri{rearrangement{\vthsp}-{\vthsp}invariant}

\newsection{374}

As is to be expected, many of the most important function spaces of
analysis are symmetric in various ways;  in particular, they share the
symmetries of the underlying measure algebras.   The natural expression
of this is to say that they are `\ri' (374E).
In fact it turns out that in many cases they have the stronger property
of `$\Cal T$-invariance' (374A).   In this section I give a brief
account of the most important properties of these two kinds of
invariance.   In particular, $\Cal T$-invariance is related to a kind of
transfer mechanism, enabling us to associate function spaces on
different measure algebras (374C-374D).   As for
rearrangement-{\vthsp}invariance, the salient fact is that on the most
important measure algebras many \ri\ spaces are
$\Cal T$-invariant (374K, 374M).

\leader{374A}{$\Cal T$-invariance:  Definitions} Let
$(\frak A,\bar\mu)$ be a measure algebra.   \cmmnt{Recall that I write

\Centerline{$M^{1,\infty}_{\bar\mu}
=L^1_{\bar\mu}+L^{\infty}(\frak A)\subseteq L^0(\frak A)$,}

\Centerline{$M^{\infty,1}_{\bar\mu}
=L^1_{\bar\mu}\cap L^{\infty}(\frak A)$,}

\Centerline{$M^{0,\infty}_{\bar\mu}=\{u:u\in L^0(\frak A),
  \,\inf_{\alpha>0}\bar\mu\Bvalue{|u|>\alpha}<\infty\}$,}

\noindent (369N, 373C).}%end of comment

\spheader 374Aa I will say that a subset $A$ of
$M^{1,\infty}_{\bar\mu}$ is
{\bf $\Cal T$-invariant} if $Tu\in A$ whenever $u\in A$ and
$T\in\Cal T=\Cal T_{\bar\mu,\bar\mu}$\cmmnt{ (definition:  373Aa)}.

\spheader 374Ab An extended Fatou norm $\tau$ on $L^0$
is {\bf $\Cal T$-invariant}\cmmnt{ or {\bf fully symmetric}} if
$\tau(Tu)\le\tau(u)$ whenever $u\in M^{1,\infty}_{\bar\mu}$ and
$T\in\Cal T$.

%`fully symmetric' in Dodds Schl\"uchtermann & Sukochev 01

\spheader 374Ac \cmmnt{As in \S373,} I will write
$(\frak A_L,\bar\mu_L)$ for the measure algebra of Lebesgue measure on
$\coint{0,\infty}$, and $u^*\in M^{0,\infty}_{\bar\mu_L}$ for
the decreasing rearrangement of any $u$ belonging to any
$M^{0,\infty}_{\bar\mu}$\cmmnt{ (373C)}.

\leader{374B}{}\cmmnt{ The first step is to show that the associate of
a $\Cal T$-invariant norm is $\Cal T$-invariant.

\medskip

\noindent}{\bf Theorem} Let $(\frak A,\bar\mu)$ be a semi-finite measure
algebra and $\tau$ a $\Cal T$-invariant extended Fatou norm on
$L^0(\frak A)$.   Let $L^{\tau}$ be the Banach lattice
defined from $\tau$\cmmnt{ (369G)}, and $\tau'$ the associate extended
Fatou norm\cmmnt{ (369H-369I)}.   Then

(i) $M^{\infty,1}_{\bar\mu}\subseteq L^{\tau}
\subseteq M^{1,\infty}_{\bar\mu}$;

(ii)  $\tau'$ is
also $\Cal T$-invariant, and $\int u^*\times v^*\le\tau(u)\tau'(v)$ for
all $u$, $v\in M^{0,\infty}_{\bar\mu}$.

\proof{{\bf (a)} I check first that
$L^{\tau}\subseteq M^{0,\infty}_{\bar\mu}$.
\Prf\ Take any $u\in L^0(\frak A)\setminus M^{0,\infty}_{\bar\mu}$.
There is surely some $w>0$ in $L^{\tau}$, and we can suppose that
$w=\chi a$ for some $a$ of finite measure.   Now, for any $n\in\Bbb N$,

\Centerline{$(|u|\wedge n\chi 1)^*=n\chi 1\ge nw^*$}

\noindent in $L^0(\frak A_L)$, because
$\bar\mu\Bvalue{|u|>n}=\infty$.   So there is a
$T\in\Cal T_{\bar\mu,\bar\mu}$ such that $T(|u|\wedge n\chi 1)=nw$, by
373O, and

\Centerline{$\tau(u)\ge\tau(|u|\wedge n\chi 1)
\ge\tau(T(|u|\wedge n\chi 1))=\tau(nw)=n\tau(w)$.}

\noindent As $n$ is arbitrary, $\tau(u)=\infty$.   As $u$ is arbitrary,
$L^{\tau}\subseteq M^{0,\infty}_{\bar\mu}$.\ \Qed

\medskip

{\bf (b)} Next, $\int u^*\times v^*\le\tau(u)\tau'(v)$ for all $u$,
$v\in M^{0,\infty}_{\bar\mu}$.   \Prf\ If $u\in M^{1,\infty}_{\bar\mu}$,
then

$$\eqalignno{\int u^*\times v^*
&=\sup\{\int|Tu\times v|:T\in\Cal T_{\bar\mu,\bar\mu}\}\cr
\noalign{\noindent (373Q)}
&\le\sup\{\tau(Tu)\tau'(v):T\in\Cal T_{\bar\mu,\bar\mu}\}
=\tau(u)\tau'(v).\cr}$$

\noindent Generally, setting $u_n=|u|\wedge n\chi 1$,
$\sequencen{u_n^*}$ is a non-decreasing sequence with
supremum $u^*$ (373Db, 373Dh), so

\Centerline{$\int u^*\times v^*
=\sup_{n\in\Bbb N}\int u_n^*\times v^*
\le\sup_{n\in\Bbb N}\tau(u_n)\tau'(v)
=\tau(u)\tau'(v)$.  \Qed}

\medskip

{\bf (c)} Consequently, $L^{\tau}\subseteq M^{1,\infty}_{\bar\mu}$.
\Prf\ If $\frak A=\{0\}$, this is trivial.   Otherwise, take
$u\in L^{\tau}$.   There is surely some non-zero $a$ such that
$\tau'(\chi a)<\infty$;  now, setting $v=\chi a$,

\Centerline{$\int_0^{\bar\mu a}u^*=\int u^*\times v^*
\le\tau(u)\tau'(v)<\infty$}

\noindent by (b) above.   But this means that
$u^*\in M^{1,\infty}_{\bar\mu}$, so that $u\in M^{1,\infty}_{\bar\mu}$
(373F(b-ii)).\ \Qed

\medskip

{\bf (d)} Next, $\tau'$ is $\Cal T$-invariant.   \Prf\  Suppose
that $v\in M^{1,\infty}_{\bar\mu}$, $T\in\Cal T_{\bar\mu,\bar\mu}$,
$u\in L^0(\frak A)$ and $\tau(u)\le 1$.   Then
$u\in M^{1,\infty}_{\bar\mu}$, by (c), so

\Centerline{$\int|u\times Tv|\le\int u^*\times v^*
\le\tau(u)\tau'(v)\le\tau'(v)$,}

\noindent using 373J for the first inequality.   Taking the supremum
over $u$, we see that $\tau'(Tv)\le\tau'(v)$;  as $T$ and $v$ are
arbitrary, $\tau'$ is $\Cal T$-invariant.\ \Qed

\medskip

{\bf (e)} Finally, putting (d) and (c) together,
$L^{\tau'}\subseteq M^{1,\infty}_{\bar\mu}$, so that
$L^{\tau}\supseteq M^{\infty,1}_{\bar\mu}$, using 369J and 369O.
}%end of proof of 374B

\leader{374C}{}\cmmnt{ For any $\Cal T$-invariant extended Fatou norm
on $L^0(\frak A_L)$ there are corresponding norms on $L^0(\frak A)$ for
any semi-finite measure algebra, as follows.

\medskip

\noindent}{\bf Theorem} Let $\theta$ be a $\Cal T$-invariant extended
Fatou norm on $L^0(\frak A_L)$, and $(\frak A,\bar\mu)$ a
semi-finite measure algebra.

(a) There is a $\Cal T$-invariant extended Fatou norm $\tau$ on
$L^0(\frak A)$ defined by setting

$$\eqalign{\tau(u)
&=\theta(u^*)\text{ if }u\in M^{0,\infty}_{\bar\mu},\cr
&=\infty\text{ if }u\in L^0(\frak A)\setminus M^{0,\infty}_{\bar\mu}.\cr}$$

(b) Writing $\theta'$, $\tau'$ for the associates of $\theta$ and
$\tau$, we now have

$$\eqalign{\tau'(v)&=\theta'(v^*)
  \text{ if }v\in M^{0,\infty}_{\bar\mu},\cr
&=\infty\text{ if }v\in L^0(\frak A)\setminus M^{0,\infty}_{\bar\mu}.\cr}$$

(c) If $\theta$ is an order-continuous norm on the Banach lattice
$L^{\theta}$, then $\tau$ is an order-continuous norm on $L^{\tau}$.

\wheader{374C}{0}{0}{0}{24pt}

\proof{{\bf (a)(i)} The argument seems to run better if I use a
different formula to define $\tau$:  set

\Centerline{$\tau(u)=\sup\{\int|u\times Tw|:T\in\Cal
T_{\bar\mu_L,\bar\mu},\,w\in L^0(\frak A_L),\,\theta'(w)\le 1\}$}

\noindent for $u\in L^0(\frak A)$.   (By 374B(i),
$w\in M^{1,\infty}_{\bar\mu_L}$ whenever $\theta'(w)\le 1$, so
there is no difficulty in defining $Tw$.)   Now $\tau(u)=\theta(u^*)$
for every $u\in M^{0,\infty}_{\bar\mu}$.   \Prf\
($\alpha$) If $w\in L^0(\frak A_L)$ and $\theta'(w)\le 1$, then
$w\in M^{1,\infty}_{\bar\mu_L}$, so there is
an $S\in\Cal T_{\bar\mu_L,\bar\mu_L}$ such that $Sw=w^*$ (373O).
Accordingly $\theta'(w^*)\le\theta'(w)$ (because $\theta'$ is
$\Cal T$-invariant, by 374B);  now

\Centerline{$\int|u\times Tw|\le\int u^*\times w^*
\le\theta(u^*)\theta'(w^*)\le\theta(u^*)\theta'(w)\le\theta(u^*)$;}

\noindent as $w$ is arbitrary, $\tau(u)\le\theta(u^*)$.  ($\beta$) If
$w\in L^0(\frak A_L)$ and $\theta'(w)\le 1$, then

$$\eqalignno{\int|u^*\times w|
&\le\int(u^*)^*\times w^*\cr
\noalign{\noindent (373E)}
&=\int u^*\times w^*
=\sup\{\int|u\times Tw|:T\in\Cal T_{\bar\mu_L,\bar\mu}\}\cr
\noalign{\noindent (373Q)}
&\le\tau(u).\cr}$$

\noindent But because $\theta$
is the associate of $\theta'$ (369I(ii)), this means that
$\theta(u^*)\le\tau(u)$.\   \Qed

\medskip

\quad{\bf (ii)} Now $\tau$ is an extended Fatou norm on $L^0(\frak A)$.
\Prf\ Of the conditions in 369F, (i)-(iv) are true just because
$\tau(u)=\sup_{v\in B}\int|u\times v|$ for some set $B\subseteq L^0$.
As for (v) and (vi), observe that if $u\in M^{\infty,1}_{\bar\mu}$ then
$u^*\in M^{\infty,1}_{\bar\mu_L}$ (373F(b-iv)), so
that $\tau(u)=\theta(u^*)<\infty$, by 374B(i), while also

\Centerline{$u\ne 0\Longrightarrow u^*\ne 0\Longrightarrow
\tau(u)=\theta(u^*)>0$.}

\noindent As $M^{\infty,1}_{\bar\mu}$ is order-dense in
$L^0(\frak A)$ (this is where I use the hypothesis that
$(\frak A,\bar\mu)$ is semi-finite), 369F(v)-(vi) are satisfied, and
$\tau$ is an extended Fatou norm.\   \Qed

\medskip

\quad{\bf (iii)} $\tau$ is $\Cal T$-invariant.   \Prf\
Take $u\in M^{1,\infty}_{\bar\mu}$ and
$T\in\Cal T_{\bar\mu,\bar\mu}$.    There are
$S_0\in\Cal T_{\bar\mu_L,\bar\mu}$ and
$S_1\in\Cal T_{\bar\mu,\bar\mu_L}$ such that $S_0u^*=u$, $S_1Tu=(Tu)^*$
(373O);  now $S_1TS_0\in\Cal T_{\bar\mu_L,\bar\mu_L}$ (373Be), so

\Centerline{$\tau(Tu)=\theta((Tu)^*)=\theta(S_1TS_0u^*)\le\theta(u^*)
=\tau(u)$}

\noindent because $\theta$ is $\Cal T$-invariant.\ \Qed

\medskip

\quad{\bf (iv)} We can now return to the definition of $\tau$.   I have
already remarked that $\tau(u)=\theta(u^*)$ if $u\in
M^{0,\infty}_{\bar\mu}$.   For other $u$, we must have $\tau(u)=\infty$
just because $\tau$ is a $\Cal T$-invariant extended Fatou norm
(374B(i)).   So the definitions in the statement of the theorem and (i)
above coincide.

\medskip

{\bf (b)} We surely have $\tau'(v)=\infty$ if $v\in L^0(\frak
A)\setminus M^{0,\infty}_{\bar\mu}$, by 374B, because $\tau'$,
like $\tau$, is a $\Cal T$-invariant extended Fatou norm.   So take
$v\in M^{0,\infty}_{\bar\mu}$.

\medskip

\quad{\bf (i)} If $u\in L^0(\frak A)$ and $\tau(u)\le 1$, then

\Centerline{$\int|v\times u|
\le\int v^*\times u^*
\le\theta'(v^*)\theta(u^*)
=\theta'(v^*)\tau(u)
\le\theta'(v^*)$;}

\noindent as $u$ is arbitrary, $\tau'(v)\le\theta'(v^*)$.

\medskip

\quad{\bf (ii)} If $w\in L^0(\frak A_L)$ and $\theta(w)\le 1$, then

$$\eqalignno{\int|v^*\times w|
&\le\int v^*\times w^*
=\sup\{\int|v\times Tw|:T\in\Cal T_{\bar\mu_L,\bar\mu}\}\cr
\noalign{\noindent (373Q)}
&\le\sup\{\tau'(v)\tau(Tw):T\in\Cal T_{\bar\mu_L,\bar\mu}\}
=\sup\{\tau'(v)\theta((Tw)^*):T\in\Cal T_{\bar\mu_L,\bar\mu}\}\cr
&\le\sup\{\tau'(v)\theta(STw):T\in\Cal T_{\bar\mu_L,\bar\mu},
  \,S\in\Cal T_{\bar\mu,\bar\mu_L}\}\cr
\noalign{\noindent (because, given $T$, we can find an $S$ such that
$STw=(Tw)^*$, by 373O)}
&\le\sup\{\tau'(v)\theta(Tw):T\in\Cal T_{\bar\mu_L,\bar\mu_L}\}
\le\tau'(v).\cr}$$

\noindent As $w$ is arbitrary, $\theta'(v^*)\le\tau'(v)$ and the two are
equal.   This completes the proof of (b).

\medskip

{\bf (c)(i)} The first step is to note that
$L^{\tau}\subseteq M^0_{\bar\mu}$.   \Prf\Quer\ Suppose that
$u\in L^{\tau}\setminus M^0_{\bar\mu}$, that is, that
$\bar\mu\Bvalue{|u|>\alpha}=\infty$ for
some $\alpha>0$.   Then $u^*\ge\alpha\chi 1$ in $L^0(\frak A_L)$, so
$L^{\infty}(\frak A_L)\subseteq L^{\theta}$.   For each $n\in\Bbb N$,
set $v_n=\chi\coint{n,\infty}^{\ssbullet}$.   Then $v_n^*=v_0$, so we
can find a $T_n\in\Cal T_{\bar\mu_L,\bar\mu_L}$ such that $T_nv_n=v_0$
(373O), and $\theta(v_n)\ge\theta(v_0)$ for every $n$.   But as
$\sequencen{v_n}$ is a decreasing sequence with infimum $0$, this means
that $\theta$ is not an order-continuous norm.\ \Bang\Qed

\medskip

\quad{\bf (ii)} Now suppose that $A\subseteq L^{\tau}$ is non-empty and
downwards-directed and has infimum $0$.   Then
$\inf_{u\in A}\bar\mu\Bvalue{u>\alpha}=0$ for every $\alpha>0$ (put
364L(b-ii) and 321F
together).   But this means that $B=\{u^*:u\in A\}$ must have infimum
$0$;  since $B$ is surely downwards-directed,
$\inf_{v\in B}\theta(v)=0$, that is, $\inf_{u\in A}\tau(u)=0$.   As $A$
is arbitrary, $\tau$ is an order-continuous norm.
}%end of proof of 374C

\leader{374D}{}\cmmnt{ What is more, every $\Cal T$-invariant extended
Fatou norm can be represented in this way.

\medskip

\noindent}{\bf Theorem} Let $(\frak A,\bar\mu)$ be a semi-finite measure
algebra, and $\tau$ a $\Cal T$-invariant extended Fatou norm on
$L^0(\frak A)$.   Then there is a $\Cal T$-invariant extended Fatou norm
$\theta$ on $L^0(\frak A_L)$ such that $\tau(u)=\theta(u^*)$
for every $u\in M^{0,\infty}_{\bar\mu}$.

\proof{ I use the method of 374C.   If $\frak A=\{0\}$ the result is
trivial;  assume that $\frak A\ne\{0\}$.

\woddheader{374D}{6}{2}{2}{24pt}

{\bf (a)}   Set

\Centerline{$\theta(w)=\sup\{\int|w\times Tv|:
  T\in\Cal T_{\bar\mu,\bar\mu_L},\,v\in L^0(\frak A),\,\tau'(v)\le 1\}$}

\noindent for $w\in L^0(\frak A_L)$. Note that

\Centerline{$\theta(w)
=\sup\{\int w^*\times v^*:v\in L^0(\frak A),\,\tau'(v)\le 1\}$}

\noindent for every $w\in M^{0,\infty}_{\bar\mu_L}$, by 373Q again.

$\theta$ is an extended Fatou norm on $L^0(\frak A_L)$.   \Prf\ As in
374C, the conditions 369F(i)-(iv) are elementary.   If $w>0$ in
$L^0(\frak A_L)$, take any $v\in L^0(\frak A)$ such that
$0<\tau'(v)\le 1$;  then $w^*\times v^*\ne 0$ so
$\theta(w)\ge\int w^*\times v^*>0$.
So 369F(v) is satisfied.   As for 369F(vi), if $w>0$ in
$L^0(\frak A_L)$, take a non-zero $a\in\frak A$ of finite measure such
that $\alpha=\tau(\chi a)<\infty$.   Let $\beta>0$, $b\in\frak A_L$ be
such that $0<\bar\mu_Lb\le\bar\mu a$ and $\beta\chi b\le w$;  then

\Centerline{$\theta(\chi b)
=\sup_{\tau'(v)\le 1}\int(\chi b)^*\times v^*
\le\sup_{\tau'(v)\le 1}\int(\chi a)^*\times v^*
\le\tau(\chi a)<\infty$}

\noindent by 374B(ii).   So $\theta(\beta\chi b)<\infty$ and 369F(vi) is
satisfied.   Thus $\theta$ is an extended Fatou norm.\   \Qed

\medskip

{\bf (b)} $\theta$ is $\Cal T$-invariant.   \Prf\ If
$T\in\Cal T_{\bar\mu_L,\bar\mu_L}$ and $w\in M^{1,\infty}_{\bar\mu_L}$,
then

\Centerline{$\theta(Tw)=\sup_{\tau'(v)\le 1}\int(Tw)^*\times v^*
\le\sup_{\tau'(v)\le 1}\int w^*\times v^*=\theta(w)$}

\noindent by 373G and 373I.\ \Qed

\medskip

{\bf (c)} $\theta(u^*)=\tau(u)$ for every $u\in M^{0,\infty}_{\bar\mu}$.
\Prf\ We have

\Centerline{$\tau(u)=\sup_{\tau'(v)\le 1}\int|u\times v|
\le\sup_{\tau'(v)\le 1}\int u^*\times v^*\le\tau(u)$,}

\noindent using 369I, 373E and 374B.   So

\Centerline{$\theta(u^*)=\sup_{\tau'(v)\le 1}\int u^*\times v^*
=\tau(u)$}

\noindent by the remark in (a) above.\ \Qed
}%end of proof of 374D

\leader{374E}{}\cmmnt{ I turn now to
rearrangement{\vthsp}-invariance.}
Let $(\frak A,\bar\mu)$ be a measure algebra.

\spheader 374Ea I will say that a subset $A$ of $L^0=L^0(\frak A)$ is
{\bf \ri} if $T_{\pi}u\in A$ whenever $u\in A$ and
$\pi:\frak A\to\frak A$ is a measure-preserving Boolean automorphism,
writing $T_{\pi}:L^0\to L^0$ for the isomorphism corresponding to
$\pi$\cmmnt{ (364P)}.

\spheader 374Eb I will say that an extended Fatou norm $\tau$ on $L^0$
is {\bf \ri} if $\tau(T_{\pi}u)=\tau(u)$ whenever
$u\in L^0$ and $\pi:\frak A\to\frak A$ is a measure-preserving
automorphism.

\leader{374F}{Remarks\cmmnt{ (a)}} If $(\frak A,\bar\mu)$ is a
semi-finite measure algebra
and $\pi:\frak A\to\frak A$ is a sequentially order-continuous
measure-preserving Boolean homomorphism, then
$T_{\pi}\restr M^{1,\infty}_{\bar\mu}$ belongs to
$\Cal T_{\bar\mu,\bar\mu}$\cmmnt{;  this is obvious from the
definition of
$M^{1,\infty}=L^1+L^{\infty}$ and the basic properties of $T_{\pi}$
(364P)}.   Accordingly, any $\Cal T$-invariant extended Fatou norm
$\tau$ on $L^0(\frak A)$ must be \ri\cmmnt{, since
(by 374B) we shall have $\tau(u)=\tau(T_{\pi}(u))=\infty$ when
$u\notin M^{1,\infty}_{\bar\mu}$}.   Similarly, any $\Cal T$-invariant
subset of $M^{1,\infty}_{\bar\mu}$ will be \ri.

\cmmnt{\spheader 374Fb I seek to describe cases in which
rearrangement-{\vthsp}invariance implies
$\Cal T$-invariance.   This happens only for certain measure
algebras;  in order to shorten the statements of the main theorems I
introduce a special phrase.}

\leader{374G}{Definition} I say that a measure algebra
$(\frak A,\bar\mu)$ is {\bf quasi-homogeneous} if for any non-zero $a$,
$b\in\frak A$ there is a measure-preserving Boolean automorphism
$\pi:\frak A\to\frak A$ such that $\pi a\Bcap b\ne 0$.

\leader{374H}{Proposition} Let $(\frak A,\bar\mu)$ be a semi-finite
measure algebra.   Then the following are equiveridical:

(i) $(\frak A,\bar\mu)$ is quasi-homogeneous;

(ii) {\it either} $\frak A$ is purely atomic and every atom of $\frak A$
has the same measure {\it or} there is a $\kappa\ge\omega$ such that the
principal ideal $\frak A_a$ is homogeneous, with Maharam type $\kappa$,
for every $a\in\frak A$ of non-zero finite measure.

\proof{{\bf (i)$\Rightarrow$(ii)} Suppose that $(\frak A,\bar\mu)$ is
quasi-homogeneous.

\medskip

\quad\grheada\ Suppose that $\frak A$ has an atom $a$.   In this case,
for any $b\in\frak A\setminus\{0\}$ there is an automorphism $\pi$ of
$(\frak A,\bar\mu)$ such that $\pi a\Bcap b\ne 0$;  now $\pi a$ must be
an atom, so $\pi a=\pi a\Bcap b$ and $\pi a$ is an atom included in $b$.
As $b$ is arbitrary, $\frak A$ is purely atomic;  moreover, if $b$ is an
atom, then it must be equal to $\pi a$ and therefore of the same measure
as $a$, so all atoms of $\frak A$ have the same measure.

\medskip

\quad\grheadb\ Now suppose that $\frak A$ is atomless.   In this case,
if $a\in\frak A$ has finite non-zero measure, $\frak A_a$ is
homogeneous.   \Prf\Quer\ Otherwise, there are non-zero $b$,
$c\Bsubseteq a$ such that the principal ideals $\frak A_b$, $\frak A_c$
are homogeneous and of different Maharam types, by Maharam's theorem
(332B, 332H).   But now there is supposed to be an automorphism $\pi$
such that $\pi b\Bcap c\ne 0$, in which case $\frak A_b$,
$\frak A_{\pi b}$, $\frak A_{\pi b\Bcap c}$ and $\frak A_c$ must all
have the same Maharam type.\   \Bang\Qed

Consequently, if $a$, $b\in\frak A$ are both of non-zero finite measure,
the Maharam types of $\frak A_a$, $\frak A_{a\Bcup b}$ and $\frak A_b$
must all be the same infinite cardinal $\kappa$.

\medskip

{\bf (ii)$\Rightarrow$(i)} Assume (ii), and take $a$,
$b\in\frak A\setminus\{0\}$.   If $a\Bcap b\ne 0$ we can take $\pi$ to
be the identity automorphism and stop.   So let us suppose that
$a\Bcap b=0$.

\medskip

\quad\grheada\ If $\frak A$ is purely atomic and every atom has the same
measure, then there are atoms $a_0\Bsubseteq a$, $b_0\Bsubseteq b$.
Set

$$\eqalign{\pi c
&=c\text{ if }c\Bsupseteq a_0\Bcup b_0
  \text{ or }c\Bcap(a_0\Bcup b_0)=0,\cr
&=c\Bsymmdiff(a_0\Bcup b_0)\text{ otherwise.}\cr}$$

\noindent Then it is easy to check that $\pi$ is a measure-preserving
automorphism of $\frak A$ such that $\pi a_0=b_0$, so that
$\pi a\Bcap b\ne 0$.

\medskip

\quad\grheadb\ If $\frak A_c$ is \Mth\ with the same
infinite Maharam type $\kappa$ for every non-zero $c$ of finite measure,
set $\gamma=\min(1,\bar\mu a,\bar\mu b)>0$.   Because $\frak A$ is
atomless, there are $a_0\Bsubseteq a$, $b_0\Bsubseteq b$ with
$\bar\mu a_0=\bar\mu b_0=\gamma$ (331C).   Now
$\frak A_{a_0}$ and $\frak A_{b_0}$
are homogeneous with the same Maharam type and the same magnitude, so by
Maharam's theorem (331I) there is a measure-preserving isomorphism
$\pi_0:\frak A_{a_0}\to\frak A_{b_0}$.   Define $\pi:\frak A\to\frak A$
by setting

\Centerline{$\pi c
=(c\Bsetminus(a_0\Bcup b_0))\Bcup\pi_0(c\Bcap a_0)
  \Bcup\pi_0^{-1}(c\Bcap b_0)$}

\noindent for $c\in\frak A$;
then it is easy to see that $\pi$ is a measure-preserving
automorphism of $\frak A$ and that $\pi a\Bcap b\ne 0$.
}%end of proof of 374H

\cmmnt{\medskip

\noindent{\bf Remark} We shall return to these ideas in Chapter 38.   In
particular, the construction of $\pi$ from $\pi_0$ in the last part of
the proof will be of great importance;  in the language of 381R,
$\pi=\cycle{a_0\,_{\pi_0}\,b_0}$.
}%end of comment

\leader{374I}{Corollary} Let $(\frak A,\bar\mu)$ be a quasi-homogeneous
semi-finite measure algebra.   Then

(a) whenever $a$, $b\in\frak A$ have the same finite measure, the
principal ideals $\frak A_a$, $\frak A_b$ are isomorphic as measure
algebras;

(b) there is a subgroup $\Gamma$ of the additive group $\Bbb R$ such
that ($\alpha$) $\bar\mu a\in\Gamma$ whenever $a\in\frak A$ and
$\bar\mu a<\infty$ ($\beta$) whenever $a\in\frak A$, $\gamma\in\Gamma$
and $0\le\gamma\le\bar\mu a$ then there is a $c\Bsubseteq a$ such that
$\bar\mu c=\gamma$.

\proof{ If $\frak A$ is purely atomic, with all its atoms of measure
$\gamma_0$, set $\Gamma=\gamma_0\Bbb Z$, and the results are elementary.
If $\frak A$ is atomless, set $\Gamma=\Bbb R$;  then (a) is a
consequence of Maharam's theorem, and (b) is a consequence of 331C,
already used in the proof of 374H.
}%end of proof of 374I

\leader{374J}{Lemma} Let $(\frak A,\bar\mu)$ be a quasi-homogeneous
semi-finite measure algebra and $u$,
$v\in M^{0,\infty}_{\bar\mu}$.   Let $\Aut_{\bar\mu}$ be the group
of measure-preserving automorphisms of $\frak A$.   Then

\Centerline{$\int u^*\times v^*
=\sup_{\pi\in\Aut_{\bar\mu}}\int|u\times T_{\pi}v|$,}

\noindent where $T_{\pi}:L^0(\frak A)\to L^0(\frak A)$ is the
isomorphism corresponding to $\pi$.

\proof{{\bf (a)} Suppose first that $u$, $v$ are non-negative and belong
to $S(\frak A^f)$, where $\frak A^f$ is the ring $\{a:\bar\mu
a<\infty\}$, as usual.   Then they can be expressed as
$u=\sum_{i=0}^m\alpha_i\chi a_i$, $v=\sum_{j=0}^n\beta_j\chi b_j$ where
$\alpha_0\ge\ldots\alpha_m\ge 0$, $\beta_0\ge\ldots\ge\beta_n\ge 0$,
$a_0,\ldots,a_m$ are disjoint and of finite measure, and
$b_0,\ldots,b_n$ are disjoint and of finite measure.   Extending each
list by a final term having a coefficient of $0$, if need be, we may
suppose that $\sup_{i\le m}a_i=\sup_{j\le n}b_j$.

Let $(t_0,\ldots,t_s)$ enumerate in ascending order the set

\Centerline{$\{0\}\cup\{\sum_{i=0}^k\bar\mu a_i:k\le m\}
\cup\{\sum_{j=0}^k\bar\mu b_j:k\le n\}$.}

\noindent Then every $t_r$ belongs to the subgroup $\Gamma$ of 374Ib,
and $t_s=\sum_{i=0}^m\bar\mu a_i=\sum_{j=0}^n\bar\mu b_j$.
For $1\le r\le s$ let $k(r)$, $l(r)$ be minimal subject to the
requirements $t_r\le\sum_{i=0}^{k(r)}\bar\mu a_i$,
$t_r\le\sum_{j=0}^{l(r)}\bar\mu b_j$.   Then
$\bar\mu a_i=\sum_{k(r)=i}t_r-t_{r-1}$, so (using 374Ib) we can find a
disjoint family $\langle c_r\rangle_{1\le r\le s}$ such that
$c_r\Bsubseteq a_{k(r)}$ and $\bar\mu c_r=t_r-t_{r-1}$ for each $r$.
Similarly, there is
a disjoint family $\langle d_r\rangle_{1\le r\le s}$ such that
$d_r\Bsubseteq b_{l(r)}$ and $\bar\mu d_r=t_r-t_{r-1}$ for each $r$.
Now the principal ideals $\frak A_{c_r}$, $\frak A_{d_r}$ are isomorphic
for every $r$, by 374Ia;  let $\pi_r:\frak A_{d_r}\to\frak A_{c_r}$ be
measure-preserving isomorphisms.   Define $\pi:\frak A\to\frak A$ by
setting

\Centerline{$\pi a=(a\Bsetminus\sup_{1\le r\le s}d_r)\Bcup\sup_{1\le
r\le s}\pi_r(a\Bcap d_r)$;}

\noindent because

\Centerline{$\sup_{r\le s}c_r=\sup_{i\le m}a_i=\sup_{j\le n}b_j
=\sup_{r\le s}d_r$,}

\noindent $\pi:\frak A\to\frak A$ is a measure-preserving automorphism.

Now

\Centerline{$u=\sum_{r=1}^s\alpha_{k(r)}\chi c_r$,
\quad$v=\sum_{r=1}^s\beta_{l(r)}\chi d_r$,}

\Centerline{$u^*=\sum_{r=1}^s\alpha_{k(r)}
   \chi\coint{t_{r-1},t_r}^{\ssbullet}$,
\quad$v^*=\sum_{r=1}^s\beta_{l(r)}
   \chi\coint{t_{r-1},t_r}^{\ssbullet}$,}

\noindent so

\Centerline{$\int u\times T_{\pi}v
=\sum_{r=1}^s\alpha_{k(r)}\beta_{l(r)}\bar\mu c_r
=\sum_{r=1}^s\alpha_{k(r)}\beta_{l(r)}(t_r-t_{r-1})
=\int u^*\times v^*$.}

\medskip

{\bf (b)} Now take any $u_0$, $v_0\in M^{0,\infty}_{\bar\mu}$.   Set

\Centerline{$A=\{u:u\in S(\frak A^f),\,0\le u\le|u_0|\}$,
\quad$B=\{v:v\in S(\frak A^f),\,0\le v\le|v_0|\}$.}

\noindent Then $A$ is an upwards-directed set with supremum $|u_0|$,
because $(\frak A,\bar\mu)$ is semi-finite, so $\{u^*:u\in A\}$ is an
upwards-directed set with supremum $|u_0|^*=u_0^*$ (373Db, 373Dh).
Similarly $\{v^*:v\in B\}$ is upwards-directed and has supremum $v_0^*$,
so $\{u^*\times v^*:u\in A,\,v\in B\}$ is upwards-directed and has
supremum $u_0^*\times v_0^*$.

Consequently, if $\gamma<\int u_0^*\times v_0^*$, there are $u\in A$,
$v\in B$ such that $\gamma\le\int u^*\times v^*$.   Now, by (a), there
is a $\pi\in\Aut_{\bar\mu}$ such that

\Centerline{$\gamma\le\int u\times T_{\pi}v
\le\int|u_0|\times T_{\pi}|v_0|=\int|u_0\times T_{\pi}v_0|$}

\noindent because $T_{\pi}$ is a Riesz homomorphism.   As $\gamma$ is
arbitrary,

\Centerline{$\int u_0^*\times v_0^*
\le\sup_{\pi\in\Aut_{\bar\mu}}\int|u_0\times T_{\pi}v_0|$.}

\noindent But the reverse inequality is immediate from 373J.
}%end of proof of 374J

\leader{374K}{Theorem} Let $(\frak A,\bar\mu)$ be a quasi-homogeneous
semi-finite measure algebra, and $\tau$ a \ri\
extended Fatou norm on $L^0=L^0(\frak A)$.   Then $\tau$ is
$\Cal T$-invariant.

\proof{ Write $\tau'$ for the associate of $\tau$.   Then 374J tells us
that for any $u$, $v\in M^{0,\infty}_{\bar\mu}$,

\Centerline{$\int u^*\times v^*
=\sup_{\pi\in\Aut_{\bar\mu}}\int|T_{\pi}u\times v|
\le\sup_{\pi\in\Aut_{\bar\mu}}\tau(T_{\pi}u)\tau'(v)
=\tau(u)\tau'(v)$,}

\noindent writing $u^*$, $v^*$ for the decreasing rearrangements of $u$
and $v$, and $\Aut_{\bar\mu}$ for the group of measure-preserving automorphisms of
$(\frak A,\bar\mu)$.   But now, if $u\in M^{1,\infty}_{\bar\mu}$
and $T\in\Cal T_{\bar\mu,\bar\mu}$,

$$\eqalignno{\tau(Tu)
&=\sup\{\int|Tu\times v|:\tau'(v)\le 1\}\cr
\noalign{\noindent (369I)}
&\le\sup\{\int u^*\times v^*:\tau'(v)\le 1\}\cr
\noalign{\noindent (373J)}
&\le\tau(u).\cr}$$

\noindent As $T$, $u$ are arbitrary, $\tau$ is $\Cal T$-invariant.
}%end of proof of 374K

\leader{374L}{Lemma} Let $(\frak A,\bar\mu)$ be a quasi-homogeneous
semi-finite measure algebra.   Suppose that $u$,
$v\in (M^{0,\infty}_{\bar\mu})^+$
are such that $\int u^*\times v^*=\infty$.   Then there is
a measure-preserving automorphism $\pi:\frak A\to\frak A$ such that
$\int u\times T_{\pi}v=\infty$.

\proof{ I take three cases separately.

\medskip

{\bf (a)} Suppose that $\frak A$ is purely atomic;  then $u$,
$v\in L^{\infty}(\frak A)$ and $u^*$, $v^*\in L^{\infty}(\frak A_L)$, so
neither $u^*$ nor $v^*$ can belong to $L^1_{\bar\mu_L}$ and neither $u$ nor
$v$ can belong to $L^1_{\bar\mu}$.
Let $\gamma$ be the common measure of the atoms of $\frak A$.
For each $n\in\Bbb N$, set

\Centerline{$\alpha_n=\inf\{\alpha:\alpha\ge 0,\,
\bar\mu\Bvalue{u>\alpha}\le 3^n\gamma\}$,
\quad$\tilde a_n=\Bvalue{u>\bover12\alpha_n}$.}

\noindent Then
$\bar\mu\Bvalue{u>\alpha_n}\le 3^n\gamma$;  also
$\alpha_n>0$, since otherwise $u$ would belong to $L^1_{\bar\mu}$, so
$\bar\mu\tilde a_n\ge 3^n\gamma$.  We can therefore choose
$\sequencen{a'_n}$ inductively such that $a'_n\Bsubseteq\tilde a_n$ and
$\bar\mu a'_n=3^n\gamma$ for each $n$  (using 374Ib).
For each $n\ge 1$, set $a''_n=a'_n\Bsetminus\sup_{i<n}a'_i$;  then
$\bar\mu a''_n\ge\Bover12\cdot 3^{-n}\gamma$, so we can choose an
$a_n\Bsubseteq a''_n$ such that $\bar\mu a_n=3^{n-1}\gamma$.

Also, of course,
$\sequencen{\alpha_n}$ is non-increasing.   We now see that

\Centerline{$\langle a_n\rangle_{n\ge 1}$ is disjoint,
\quad$u\ge\Bover12\alpha_n\chi a_n$ for every $n\ge 1$,}

\Centerline{$u^*\le\|u\|_{\infty}\chi\coint{0,\gamma}^{\ssbullet}
   \vee\sup_{n\in\Bbb N}
   \alpha_n\chi\coint{3^n\gamma,3^{n+1}\gamma}^{\ssbullet}$.}

Similarly, there are a non-increasing sequence $\sequencen{\beta_n}$
in $\coint{0,\infty}$ and a disjoint sequence $\langle b_n\rangle_{n\ge 1}$
in $\frak A$ such that

\Centerline{$\bar\mu b_n=3^{n-1}\gamma$,
\quad$v\ge\Bover12\beta_n\chi b_n$ for every $n\ge 1$,}

\Centerline{$v^*\le\|v\|_{\infty}\chi\coint{0,\gamma}^{\ssbullet}
   \vee\sup_{n\in\Bbb N}
   \beta_n\chi\coint{3^n\gamma,3^{n+1}\gamma}^{\ssbullet}$.}

\noindent We are supposing that

$$\eqalign{\infty
&=\int u^*\times v^*
=\gamma\|u\|_{\infty}\|v\|_{\infty}
  +\sum_{n=0}^{\infty}2\cdot 3^n\gamma\alpha_n\beta_n\cr
&=\gamma\|u\|_{\infty}\|v\|_{\infty}+2\gamma\alpha_0\beta_0
  +2\gamma\sum_{n=0}^{\infty}3^{2n+1}(\alpha_{2n+1}\beta_{2n+1}
       +3\alpha_{2n+2}\beta_{2n+2})\cr
&\le\gamma\|u\|_{\infty}\|v\|_{\infty}+2\gamma\alpha_0\beta_0
  +24\sum_{n=0}^{\infty}3^{2n}\gamma\alpha_{2n+1}\beta_{2n+1},\cr}$$

\noindent so $\sum_{n=0}^{\infty}3^{2n}\alpha_{2n+1}\beta_{2n+1}=\infty$.

At this point, recall that we are dealing with a purely atomic algebra
in which every atom has measure $\gamma$.
Let $A_n$, $B_n$ be the sets of
atoms included in $a_n$, $b_n$ for each $n\ge 1$, and
$A=\bigcup_{n\ge 1}A_n\cup B_n$.   Then $\#(A_n)=\#(B_n)=3^{n-1}$ for
each $n\ge 1$.   We
therefore have a permutation $\phi:A\to A$ such that
$\phi[B_{2n+1}]=A_{2n+1}$
for every $n$.   (The point is that
$A\setminus\bigcup_{n\in\Bbb N}A_{2n+1}$ and
$A\setminus\bigcup_{n\in\Bbb N}B_{2n+1}$ are both countably
infinite.)   Define $\pi:\frak A\to\frak A$ by setting

\Centerline{$\pi c
=(c\Bsetminus\sup A)\Bcup\sup_{a\in A,a\Bsubseteq c}\phi a$}

\noindent for $c\in\frak A$.
Then $\pi$ is well-defined (because $A$ is countable), and it
is easy to check that it is a measure-preserving Boolean automorphism
(because it is just a permutation of the atoms);  and
$\pi b_{2n+1}=a_{2n+1}$ for every $n$.   Consequently

\Centerline{$\int u\times T_{\pi}v
\ge\sum_{n=0}^{\infty}\Bover14\alpha_{2n+1}\beta_{2n+1}
   \bar\mu a_{2n+1}
=\Bover14\gamma\sum_{n=0}^{\infty}3^{2n}\alpha_{2n+1}\beta_{2n+1}
=\infty$.}

\noindent So we have found a suitable automorphism.

\medskip

{\bf (b)} Next, consider the case in which $(\frak A,\bar\mu)$ is
atomless and of finite magnitude $\gamma$.   Of course $\gamma>0$.   For
each $n\in\Bbb N$ set

\Centerline{$\alpha_n=\inf\{\alpha:\alpha\ge 0,\,
  \bar\mu\Bvalue{u>\alpha}\le 3^{-n}\gamma\}$,
\quad$\tilde a_n=\Bvalue{u>\Bover12\alpha_n}$.}

\noindent Then $\sequencen{\alpha_n}$ is non-decreasing and

\Centerline{$u^*\le\sup_{n\in\Bbb N}\alpha_{n+1}
\chi\coint{3^{-n-1}\gamma,3^{-n}\gamma}^{\ssbullet}$.}

\noindent This time, $\bar\mu\tilde a_n\ge 3^{-n}\gamma$,
and we are in an atomless measure algebra, so we can choose
$a'_n\Bsubseteq\tilde a_n$ such that $\bar\mu a'_n=3^{-n}\gamma$;
taking $a''_n=a'_n\Bsetminus\sup_{i>n}a'_i$,
$\bar\mu a''_n\ge\Bover12\cdot 3^{-n}\gamma$, and we can choose
$a_n\Bsubseteq a''_n$ such that $\bar\mu a_n=3^{-n-1}\gamma$ for every $n$.
As before, $u\ge\Bover12\alpha_n\chi a_n$ for every $n$, and
$\sequencen{a_n}$ is disjoint.

In the same way, we can find $\sequencen{\beta_n}$, $\sequencen{b_n}$
such that $\sequencen{b_n}$ is disjoint,

\Centerline{$v^*\le\sup_{n\in\Bbb N}\beta_{n+1}
   \chi\coint{3^{-n-1}\gamma,3^{-n}\gamma}^{\ssbullet}$,
\quad$v\ge\sup_{n\in\Bbb N}\Bover12\beta_n\chi b_n$}

\noindent and $\bar\mu b_n=3^{-n-1}\gamma$ for each $n$.   In this case, we
have

\Centerline{$\infty
=\int u^*\times v^*
\le\sum_{n=0}^{\infty}2\cdot 3^{-n-1}\gamma\alpha_{n+1}\beta_{n+1}$,}

\noindent and $\sum_{n=0}^{\infty}3^{-n}\alpha_n\beta_n$ is infinite.

Now all the principal ideals $\frak A_{a_n}$, $\frak A_{b_n}$ are
homogeneous and of the same Maharam type, so there are
measure-preserving isomorphisms $\pi_n:\frak A_{b_n}\to\frak A_{a_n}$;
similarly, setting $\tilde a=1\Bsetminus\sup_{n\in\Bbb N}a_n$ and
$\tilde b=1\Bsetminus\sup_{n\in\Bbb N}b_n$, there is a measure-preserving
isomorphism $\tilde\pi:\frak A_{\tilde b}\to\frak A_{\tilde a}$.
Define $\pi:\frak A\to\frak A$ by setting

\Centerline{$\pi c
=\tilde\pi(c\Bcap\tilde b)\Bcup\sup_{n\in\Bbb N}\pi_n(c\Bcap a_n)$}

\noindent for every $c\in\frak A$;
then $\pi$ is a measure-preserving automorphism
of $\frak A$, and $\pi b_n=a_n$ for each $n$.   In this case,

\Centerline{$\int u\times T_{\pi}v
\ge\Bover14\sum_{n=0}^{\infty}3^{-n-1}\gamma\alpha_n\beta_n=\infty$,}

\noindent and again we have a suitable automorphism.

\medskip

{\bf (c)} Thirdly, consider the case in which $\frak A$ is atomless and
not totally finite;  take $\kappa$ to be the common Maharam type of
all the principal ideals $\frak A_a$ where $0<\bar\mu a<\infty$.   In
this case, set

\Centerline{$\alpha_n
=\inf\{\alpha:\bar\mu\Bvalue{u>\alpha}\le 3^n\}$,
\quad$\beta_n
=\inf\{\alpha:\bar\mu\Bvalue{v>\alpha}\le 3^n\}$
}

\noindent for each $n\in\Bbb Z$.   This time

\Centerline{$u^*
\le\sup_{n\in\Bbb Z}\alpha_n
\chi\coint{3^n,3^{n+1}}^{\ssbullet}$,
\quad$v^*
\le\sup_{n\in\Bbb Z}\beta_n
\chi\coint{3^n,3^{n+1}}^{\ssbullet}$,}

\noindent so

\Centerline{$\infty=\int u^*\times v^*
=2\sum_{n=-\infty}^{\infty}3^n\alpha_n\beta_n
\le 8\sum_{n=-\infty}^{\infty}3^{2n}\alpha_{2n}\beta_{2n}$.}

For each $n\in\Bbb Z$, $3^n\le\bar\mu\Bvalue{u>\bover12\alpha_n}$,
so there is an $a''_n$ such that

\Centerline{$a''_n\Bsubseteq\Bvalue{u>\bover12\alpha_n}$,
\quad$\bar\mu a''_n=3^n$.}

\noindent Set
$a'_n=a''_n\Bsetminus\sup_{-\infty<i<n}a''_i$;  then
$\bar\mu a'_n\ge\Bover12\cdot 3^n$ for each $n$; choose
$a_n\Bsubseteq a'_n$ such that $\bar\mu a_n=3^{n-1}$.   Then
$\sequencen{a_n}$ is disjoint and $u\ge\Bover12\alpha_n\chi a_n$ for each
$n$.

Similarly, there is a disjoint sequence $\sequencen{b_n}$ such that

\Centerline{$\bar\mu b_n=3^{n-1}$,
\quad$v\ge\Bover12\beta_n\chi b_n$}

\noindent for each $n\in\Bbb N$.

Set $d^*=\sup_{n\in\Bbb Z}a_n\Bcup\sup_{n\in\Bbb Z}b_n$.   Then

\Centerline{$\tilde a=d^*\Bsetminus\sup_{n\in\Bbb Z}a_{2n}$,
\quad$\tilde b=d^*\Bsetminus\sup_{n\in\Bbb Z}b_{2n}$}

\noindent both have magnitude $\omega$ and Maharam type $\kappa$.   So
there is a measure-preserving isomorphism
$\tilde\pi:\frak A_{\tilde b}\to\frak A_{\tilde a}$ (332J).
At the same time,
for each $n\in\Bbb Z$ there is a measure-preserving isomorphism
$\pi_n:\frak A_{b_{2n}}\to\frak A_{a_{2n}}$.   So once again we can
assemble these to
form a measure-preserving automorphism $\pi:\frak A\to\frak A$, defined
by the formula

\Centerline{$\pi c
=(c\Bsetminus d^*)\Bcup\tilde\pi(c\Bcap\tilde b)
   \Bcup\sup_{n\in\Bbb Z}\pi_n(c\Bcap b_{2n})$.}

Just as in (a) and (b) above,

\Centerline{$\int u\times T_{\pi}v
\ge\sum_{n=-\infty}^{\infty}\Bover1{4}\cdot 3^{2n-1}\alpha_{2n}\beta_{2n}
=\infty$.}

\noindent
Thus we have a suitable $\pi$ in any of the cases allowed by 374H.
}%end of proof of 374L

\leader{374M}{Proposition} Let $(\frak A,\bar\mu)$ be a
quasi-homogeneous localizable measure algebra, and
$U\subseteq L^0=L^0(\frak A)$ a solid linear subspace which, regarded as
a Riesz space, is perfect.   If $U$ is \ri\ and
$M^{\infty,1}_{\bar\mu}\subseteq U\subseteq M^{1,\infty}_{\bar\mu}$,
then $U$ is $\Cal T$-invariant.

\proof{ Set $V=\{v:u\times v\in L^1$ for every $u\in U\}$, so that $V$
is a solid linear subspace of $L^0$ which can be identified with
$U^{\times}$ (369C),
and $U$ becomes $\{u:u\times v\in L^1$ for every $v\in V\}$;  note that
$M^{\infty,1}_{\bar\mu}\subseteq V\subseteq M^{1,\infty}_{\bar\mu}$
(using 369Q).

If $u\in U^+$, $v\in V^+$ and $\pi:\frak A\to\frak A$ is a
measure-preserving automorphism, then $T_{\pi}u\in U$, so
$\int v\times T_{\pi}u<\infty$;  by 374L, $\int u^*\times v^*$ is
finite.   But this
means that if $u\in U$, $v\in V$ and $T\in\Cal T_{\bar\mu,\bar\mu}$,

\Centerline{$\int|Tu\times v|\le\int u^*\times v^*<\infty$.}

\noindent As $v$ is arbitrary, $Tu\in U$;  as $T$ and $u$ are arbitrary,
$U$ is $\Cal T$-invariant.
}%end of proof of 374M

\exercises{\leader{374X}{Basic exercises $\pmb{>}$(a)}
%\spheader 374Xa
Let $(\frak A,\bar\mu)$ be a measure algebra and
$A\subseteq M^{1,\infty}_{\bar\mu}$ a $\Cal T$-invariant set.   (i) Show
that $A$ is solid.   (ii) Show that if $A$ is a linear subspace and not
$\{0\}$, then it includes $M^{\infty,1}_{\bar\mu}$.   (iii) Show
that if $u\in A$, $v\in M^{0,\infty}_{\bar\mu}$ and
$\int_0^tv^*\le\int_0^tu^*$ for every $t>0$, then $v\in A$.   (iv) Show
that if $(\frak B,\bar\nu)$ is any other
measure algebra, then $B=\{Tu:u\in A,\,T\in\Cal T_{\bar\mu,\bar\nu}\}$
and $C=\{v:v\in M^{1,\infty}_{\bar\nu},\,Tv\in A$ for every
$T\in\Cal T_{\bar\nu,\bar\mu}\}$ are $\Cal T$-invariant subsets of
$M^{1,\infty}_{\bar\nu}$, and that $B\subseteq C$.   Give two
examples in which $B\subset C$.    Show that if
$(\frak A,\bar\mu)=(\frak A_L,\bar\mu_L)$ then $B=C$.
%374A

\sqheader 374Xb Let $(\frak A,\bar\mu)$ be a measure algebra.   Show
that the extended Fatou norm $\|\,\|_p$ on $L^0(\frak A)$ is
$\Cal T$-invariant for every $p\in[1,\infty]$.   \Hint{371Gd.}
%374A

\spheader 374Xc  Let $(\frak A,\bar\mu)$ and $(\frak B,\bar\nu)$ be
semi-finite measure algebras, and $\phi$ a Young's function (369Xc).
Let $\tau_{\phi}$, $\tilde\tau_{\phi}$ be the
corresponding Orlicz norms on $L^0(\frak A)$, $L^0(\frak B)$.   Show
that $\tilde\tau_{\phi}(Tu)\le\tau_{\phi}(u)$ for every
$u\in L^0(\frak A)$, $T\in\Cal T_{\bar\mu,\bar\nu}$.   \Hint{369Xn,
373Xm.}   In particular, $\tau_{\phi}$ is $\Cal T$-invariant.
%374A

\spheader 374Xd Show that if $(\frak A,\bar\mu)$ is a semi-finite
measure algebra and $\tau$ is a $\Cal T$-invariant extended Fatou norm
on $L^0(\frak A)$, then the Banach lattice $L^{\tau}$ defined from
$\tau$ is $\Cal T$-invariant.
%374B

\spheader 374Xe Let $(\frak A,\bar\mu)$ be a semi-finite measure algebra
and $\tau$ a $\Cal T$-invariant extended Fatou norm on $L^0(\frak A)$
which is an order-continuous norm on $L^{\tau}$.   Show that
$L^{\tau}\subseteq M^{1,0}_{\bar\mu}$.
%374B

\spheader 374Xf Let $\theta$ be a $\Cal T$-invariant extended Fatou
norm on $L^0(\frak A_L)$ and $(\frak A,\bar\mu)$, $(\frak B,\bar\nu)$
two semi-finite measure algebras.   Let $\tau_1$, $\tau_2$ be the
extended Fatou norms on $L^0(\frak A)$, $L^0(\frak B)$ defined from
$\theta$ by the method of 374C.   Show that  $\tau_2(Tu)\le\tau_1(u)$
whenever $u\in M^{1,\infty}_{\bar\mu}$ and
$T\in\Cal T_{\bar\mu,\bar\nu}$.
%374C

\sqheader 374Xg Let $(\frak A,\bar\mu)$ be a semi-finite measure
algebra, not $\{0\}$, and set
$\tau(u)=\sup_{0<\bar\mu a<\infty}\Bover1{\sqrt{\bar\mu a}}\int_a|u|$ for
$u\in L^0(\frak A)$.
Show that $\tau$ is a $\Cal T$-invariant extended Fatou norm.   Find
examples of
$(\frak A,\bar\mu)$ for which $\tau$ is, and is not, order-continuous on
$L^{\tau}$.
%374C

\spheader 374Xh Let $(\frak A,\bar\mu)$ and $(\frak B,\bar\nu)$ be
semi-finite measure algebras and $\tau$ a $\Cal T$-invariant extended
Fatou norm on $L^0(\frak A)$.   (i) Show that there is a
$\Cal T$-invariant extended Fatou norm $\theta$ on $L^0(\frak B)$
defined by setting
$\theta(v)=\sup\{\tau(Tv):T\in\Cal T_{\bar\nu,\bar\mu}\}$ for
$v\in M^{1,\infty}_{\bar\nu}$.    (ii) Show that when
$(\frak A,\bar\mu)=(\frak A_L,\bar\mu_L)$ then $\theta(v)=\tau(v^*)$ for
every $v\in M^{0,\infty}_{\bar\nu}$.   (iii) Show that when
$(\frak B,\bar\nu)=(\frak A_L,\bar\mu_L)$ then $\tau(u)=\theta(u^*)$ for
every $u\in M^{0,\infty}_{\bar\mu}$.
%374D

\spheader 374Xi Let $(\frak A,\bar\mu)$ be a semi-finite measure algebra
and $\tau$ an extended Fatou norm on $L^0=L^0(\frak A)$.   Suppose that
$L^{\tau}$ is a $\Cal T$-invariant subset of $L^0$.
Show that there is a $\Cal T$-invariant extended Fatou norm $\tilde\tau$
which is equivalent to $\tau$ in the sense that, for some $M>0$,
$\tilde\tau(u)\le M\tau(u)\le M^2\tilde\tau(u)$ for every $u\in L^0$.
\Hint{show first that $\int u^*\times v^*<\infty$ for every
$u\in L^{\tau}$ and $v\in L^{\tau'}$, then that
$\sup_{\tau(u)\le 1,\tau'(v)\le 1}\int u^*\times v^*<\infty$.}
%374D

\spheader 374Xj Suppose that $\tau$ is a $\Cal T$-invariant extended
Fatou norm on $L^0(\frak A_L)$, and that $0<w=w^*\in
M^{1,\infty}_{\bar\mu_L}$.   Let $(\frak A,\bar\mu)$ be any semi-finite
measure algebra.   Show that the function $u\mapsto\tau(w\times u^*)$
extends to a $\Cal T$-invariant extended Fatou norm $\theta$ on
$L^0(\frak A)$.   \Hint{$\tau(w\times u^*)
=\sup\{\tau(w\times Tu):T\in\Cal T_{\bar\mu,\bar\mu_L}\}$ for
$u\in M^{1,\infty}_{\bar\mu_L}$.}   (When
$\tau=\|\,\|_p$ these norms are called {\bf Lorentz norms};  see {\smc
Lindenstrauss \& Tzafriri 79}, p.\ 121.)
%374D

\spheader 374Xk  Let $(\frak A,\bar\mu)$ be $\Cal P\Bbb N$ with counting
measure.   Identify $L^0(\frak A)$ with $\BbbR^{\Bbb N}$.
Let $U$ be $\{u:u\in\BbbR^{\Bbb N},\,\{n:u(n)\ne 0\}$ is finite$\}$.
Show that $U$ is a perfect Riesz space, and is \ri\ but not
$\Cal T$-invariant.
%374M

\spheader 374Xl Let $(\frak A,\bar\mu)$ be an atomless quasi-homogeneous
localizable measure algebra, and $U\subseteq L^0(\frak A)$ a
\ri\ solid linear subspace which is a perfect Riesz
space.   Show that $U\subseteq M^{1,\infty}_{\bar\mu}$ and that $U$ is
$\Cal T$-invariant.   \Hint{assume $U\ne\{0\}$.   Show that (i)
$\chi a\in U$ whenever $\bar\mu a<\infty$
(ii) $V=\{v:v\times u\in L^1\Forall u\in U\}$ is \ri\ (iii) $U$,
$V\subseteq M^{1,\infty}$.}
%374M

\leader{374Y}{Further exercises (a)}
%\spheader 374Ya
Let $(\frak A,\bar\mu)$ be a localizable measure algebra and
$U\subseteq M^{1,\infty}_{\bar\mu}$ a non-zero
$\Cal T$-invariant Riesz subspace which, regarded as a Riesz space, is
perfect.   (i) Show that $U$ includes $M^{\infty,1}_{\bar\mu}$.   (ii)
Show that its dual
$\{v:v\in L^0(\frak A),\,v\times u\in L^1_{\bar\mu}\Forall u\in U\}$
(which in this exercise I will denote by $U^{\times}$) is also
$\Cal T$-invariant, and is $\{v:v\in M^{0,\infty}_{\bar\mu},\,\int
u^*\times v^*<\infty\Forall u\in U\}$.   (iii) Show that for any
localizable measure algebra $(\frak B,\bar\nu)$ the set
$V=\{v:v\in M^{1,\infty}_{\bar\nu},\,Tv\in U
\Forall T\in\Cal T_{\bar\nu,\bar\mu}\}$ is a perfect Riesz subspace of
$L^0(\frak B)$,
and that $V^{\times}=\{v:v\in M^{1,\infty}_{\bar\nu},\,Tv\in
U^{\times}\Forall T\in\Cal T_{\bar\nu,\bar\mu}\}$.   (iv) Show that
if, in (i)-(iii), $(\frak A,\bar\mu)=(\frak A_L,\bar\mu_L)$, then
$V=\{v:v\in M^{0,\infty},\,v^*\in U\}$.   (v) Show that if, in (iii),
$(\frak B,\bar\nu)=(\frak A_L,\bar\mu_L)$, then
$U=\{u:u\in M^{0,\infty}_{\bar\mu},\,u^*\in V\}$.
%374Xh, 374C, 374D

\spheader 374Yb Let $(\frak A,\bar\mu)$ be a semi-finite measure
algebra, and
suppose that $1\le q\le p<\infty$.   Let $w_{pq}\in L^0(\frak A_L)$ be
the equivalence class of the function $t\mapsto t^{(q-p)/p}$.   (i) Show
that for any $u\in L^0(\frak A)$,

\Centerline{$\int w_{pq}\times(u^*)^q
=p\int_0^{\infty}t^{q-1}(\bar\mu\Bvalue{|u|>t})^{q/p}dt$.}

\noindent (ii) Show that we have an extended Fatou norm $\|\,\|_{p,q}$
on $L^0(\frak A)$ defined by setting

\Centerline{$\|u\|_{p,q}
=\bigl(p\int_0^{\infty}t^{q-1}
(\bar\mu\Bvalue{|u|>t})^{q/p}dt\bigr)^{1/q}$}

\noindent for every $u\in L^0(\frak A)$.   \Hint{use 374Xj with
$w=w_{pq}^{1/q}$, $\|\,\|=\|\,\|_q$.}   (iii) Show that if
$(\frak B,\bar\nu)$ is another semi-finite measure algebra and
$T\in\Cal T_{\bar\mu,\bar\nu}$, then $\|Tu\|_{p,q}\le\|u\|_{p,q}$ for
every $u\in M^{1,\infty}_{\bar\mu}$.   (iv) Show that $\|\,\|_{p,q}$ is
an order-continuous norm on $L^{\|\,\|_{p,q}}$.
%374Xj, 374D

\spheader 374Yc Let $(\frak A,\bar\mu)$ be a homogeneous measure
algebra of uncountable Maharam type, and $u$, $v\ge 0$ in
$M^0_{\bar\mu}$ such that $u^*=v^*$.   Show that there
is a measure-preserving automorphism $\pi$ of $\frak A$ such that
$T_{\pi}u=v$, where $T_{\pi}:L^0(\frak A)\to L^0(\frak A)$ is the
isomorphism corresponding to $\pi$.
%374E

\spheader 374Yd In $L^0(\frak A_L)$ let $u$ be the equivalence class of
the function $f(t)=te^{-t}$.   Show that there is no Boolean
automorphism $\pi$ of $\frak A_L$ such that $T_{\pi}u=u^*$.   \Hint{show
that $\frak A_L$ is $\tau$-generated by
$\{\Bvalue{u^*>\alpha}:\alpha>0\}$.}
%374E

\spheader 374Ye Let $(\frak A,\bar\mu)$ be a quasi-homogeneous
semi-finite measure algebra and $C\subseteq L^0(\frak A)$ a solid convex
order-closed \ri\ set.   Show that $C\cap M^{1,\infty}_{\bar\mu}$ is
$\Cal T$-invariant.
%374K

}%end of exercises

\leaveitout{\leader{374Z}{Problem} In 374D, if we are told that $\tau$
is an order-continuous norm on $L^{\tau}$, can we arrange that $\theta$
should be an order-continuous norm on $L^{\theta}$?
}%end of leaveitout

\endnotes{
\Notesheader{374} I gave this section the title
`\ri\ spaces' because it looks good on the
Contents page, and it follows what has been common practice since {\smc
Luxemburg 67b};  but actually I think that it's $\Cal T$-invariance
which matters, and that \ri\ spaces are significant largely
because the important ones are $\Cal T$-invariant.   The particular
quality of $\Cal T$-invariance which I have tried to bring out here is
its transferability from one measure algebra (or measure space, of
course) to another.   This is what I take at a relatively leisurely pace
in 374B-374D and 374Xf, and then encapsulate in 374Xh and 374Ya.   The
special place of the Lebesgue algebra $(\frak A_L,\bar\mu_L)$ arises
from its being more or less the simplest algebra over which every
$\Cal T$-invariant set can be described;  see 374Xa.

I don't think this work is particularly easy, and (as in \S373) there
are rather a lot of unattractive names in it;  but once one has achieved
a reasonable familiarity with the concepts, the techniques used can be
seen to amount to half a dozen ideas -- non-trivial ideas,
to be sure -- from \S\S369 and 373.
From \S369 I take concepts of duality:  the
symmetric relationship between a perfect Riesz space $U\subseteq L^0$
and the representation of its dual (369C-369D), and the notion of
associate extended Fatou norms (369H-369K). %369H 369I 369J 369K
From \S373 I take the idea
of `decreasing rearrangement' and theorems guaranteeing the existence
of useful members of $\Cal T_{\bar\mu,\bar\nu}$
(373O-373Q). %373O 373P 373Q
The results of the present section all depend on repeated use of these
facts, assembled in a variety of patterns.

There is one new method here, but an easy one:  the construction of
measure-preserving automorphisms by joining isomorphisms together, as in
the proofs of 374H and 374J.   I shall return to this idea, in greater
generality and more systematically investigated, in \S381.   I hope that
the special cases here will give no difficulty.

While $\Cal T$-invariance is a similar phenomenon for both extended
Fatou norms and perfect Riesz spaces (see 374Xh, 374Ya), the former seem
easier to deal with.  The essential difference is I think in 374B(i);
with a $\Cal T$-invariant extended Fatou norm, we are necessarily
confined to $M^{1,\infty}$, the natural domain of the methods used here.
For perfect Riesz spaces we have examples like $\BbbR^{\Bbb N}\cong
L^0(\Cal P\Bbb N)$ and its dual, the space of eventually-zero sequences
(374Xk);  these are \ri\ but not $\Cal T$-invariant,
as I have defined it.   This problem does not
arise over atomless algebras (374Xl).

I think it is obvious that for algebras which are not
quasi-homogeneous (374G)
rearrange\discretionary{-}{}{}ment-{\vthsp}invariance is going to be
of limited interest;  there will be regions between which there is no
communication by means of
measure-preserving automorphisms, and the best we can hope for is a
discussion of quasi-homogeneous components, if they exist, corresponding
to the partition of unity used in the proof of 332J.
There is a special difficulty concerning
rearrangement-{\vthsp}invariance
in $L^0(\frak A_L)$:  two elements can have the same decreasing
rearrangement without being rearrangements of each other in the strict
sense (373Ya, 374Yd).   The phenomenon of 373Ya is specific to algebras
of countable Maharam type (374Yc).   You will see that some of the
labour of 374L is because we have to make room for the pieces to move
in.   374J is easier just because in that context we can settle for a
supremum, rather than
an actual infinity, so the rearrangement needed (part (a) of the proof)
can be based on a region of finite measure.
}%end of comment

\discrpage


