\frfilename{mt515.tex}
\versiondate{29.8.14}
\copyrightdate{2001}

\def\chaptername{Cardinal functions}
\def\sectionname{The Balcar-Fran\v{e}k theorem}

\def\ind{\mathop{\text{ind}}}

\newsection{515}

I interpolate a section to give two basic results on Dedekind complete
Boolean algebras:  the Balcar-Fran\v{e}k theorem (515H) on independent
sets and the Pierce-Koppelberg theorem (515L) on cardinalities.

\leader{515A}{Definition} Let $\frak A$ be a Boolean algebra, not
$\{0\}$.

(a) I say that a family $\familyiI{\frak B_i}$ of subalgebras of
$\frak A$ is {\bf Boolean-independent} if $\inf_{i\in J}b_i\ne 0$
whenever $J\subseteq I$ is finite and
$b_i\in\frak B_i^+=\frak B_i\setminus\{0\}$ for every $i\in J$.

(b) I say that a family $\familyiI{a_i}$ in $\frak A$ is
{\bf Boolean-independent} if
$\inf_{j\in J}a_j\Bsetminus\sup_{k\in K}a_k$ is non-zero whenever $J$,
$K\subseteq I$ are disjoint finite sets.   Similarly, a set
$B\subseteq\frak A$ is {\bf Boolean-independent} if
$\inf J\Bsetminus\sup K\ne 0$ for any disjoint finite sets $J$,
$K\subseteq B$.

(c) I say that a family $\familyiI{D_i}$ of partitions of unity in
$\frak A$ is {\bf Boolean-independent} if $\inf_{i\in J}d_i\ne 0$
whenever $J\subseteq I$ is finite and $d_i\in D_i$ for every $i\in J$.

\cmmnt{(Many authors write `independent' rather than
`Boolean-independent'.   But in this book it is more often natural to
read `independent' as `stochastically independent', as in 458L and 525H.)}

\leader{515B}{Lemma}\cmmnt{ (Compare 272D.)} Let $\frak A$ be a Boolean
algebra, not $\{0\}$.

(a) A family $\familyiI{a_i}$ in $\frak A$ is Boolean-independent iff no
$a_i$ is $0$ or $1$ and $\familyiI{\{0,a_i,1\Bsetminus a_i,1\}}$ is a
Boolean-independent family of subalgebras of $\frak A$.

(b) Let $\familyiI{\frak B_i}$ be a family of subalgebras of $\frak A$.
Let $\frak B$ be the free product of $\familyiI{\frak B_i}$, and
$\varepsilon_i:\frak B_i\to\frak B$ the canonical homomorphism for each
$i\in I$\cmmnt{ (315I)}.   Then we have a unique Boolean homomorphism
$\phi:\frak B\to\frak A$ such that $\phi\varepsilon_i(b)=b$ whenever
$i\in I$ and $b\in\frak B_i$, and $\familyiI{\frak B_i}$ is
Boolean-independent iff $\phi$ is injective;  in which case $\frak B$ is
isomorphic to the subalgebra of $\frak A$ generated by
$\bigcup_{i\in I}\frak B_i$.

(c) If $\familyiI{\frak B_i}$ is a Boolean-independent family of
subalgebras of $\frak A$, $\family{j}{J}{I_j}$ is a disjoint family
of subsets of $I$, and $\frak C_j$ is the subalgebra of $\frak A$
generated by $\bigcup_{i\in I_j}\frak B_i$ for each $j$, then
$\family{j}{J}{\frak C_j}$ is Boolean-independent.

(d) Suppose that $B\subseteq\frak A$ is a Boolean-independent set and
that $\family{j}{J}{C_j}$ is a disjoint family of subsets of $B$.   For
$j\in J$ write $\frak C_j$ for the subalgebra of $\frak A$ generated by
$C_j$.   Then $\family{j}{J}{\frak C_j}$ is Boolean-independent.

(e) Suppose that $\familyiI{\frak B_i}$ is a Boolean-independent family
of subalgebras of $\frak A$, and that for each $i\in I$ we have a
Boolean-independent subset $B_i$ of $\frak B_i$.   Then $\familyiI{B_i}$
is disjoint and $\bigcup_{i\in I}B_i$ is Boolean-independent.

(f) Let $\familyiI{D_i}$ be a family of partitions of unity in
$\frak A$, none containing $0$.
For each $i\in I$ let $\frak B_i$ be the order-closed subalgebra of
$\frak A$ generated by $D_i$.   Then
$\familyiI{\frak B_i}$ is Boolean-independent iff $\familyiI{D_i}$ is
Boolean-independent.

\proof{{\bf (a)} The point is just that in 515Aa we need consider only
$b_i\in\frak B_i\setminus\{0,1\}$, while in 515Ab we have

\Centerline{$\inf_{j\in J}a_j\Bsetminus\sup_{k\in K}a_k
=\inf_{j\in J}a_j\Bcap\inf_{i\in K}1\Bsetminus a_k$.}

\medskip

{\bf (b)} 315Jb, applied to the identity maps from the $\frak B_i$ to
$\frak A$, assures us that there is a unique Boolean homomorphism
$\phi:\frak B\to\frak A$ such that $\phi\varepsilon_i(b)=b$ for every
$i\in I$ and $b\in\frak B_i$.

\medskip

\quad{\bf (i)} By 315K(e-ii), $\familyiI{\varepsilon_i[\frak B_i]}$ is a
Boolean-independent family of subalgebras of $\frak B$.   So if $\phi$
is injective,
$\familyiI{\frak B_i}=\familyiI{\pi_i[\varepsilon_i[\frak B_i]]}$ is
Boolean-independent in $\phi[\frak B]$ and therefore in $\frak A$.
In this case, because $\frak B$ is the subalgebra of itself generated by
$\bigcup_{i\in I}\varepsilon_i[\frak B_i]$ (315Ka), the subalgebra of
$\frak A$ generated by $\bigcup_{i\in I}\frak B_i$ is $\phi[\frak B]$
and is isomorphic to $\frak B$.

\medskip

\quad{\bf (ii)} If $\familyiI{\frak B_i}$ is Boolean-independent and
$b\in\frak B^+$, there are a finite $J\subseteq I$ and a
family $\family{j}{J}{b_j}\in\prod_{j\in J}\frak B_j^+$ such
that $b\Bsupseteq\inf_{j\in J}\varepsilon_j(b_j)$ (315Kb).   Now
$\phi(b)\Bsupseteq\inf_{j\in J}b_j$ is non-zero;  as $b$ is arbitrary,
$\phi$ is injective.

\medskip

{\bf (c)} Let $L\subseteq J$ be a finite set and suppose that
$c_j\in\frak C_j^+$ for each $j\in L$.   As observed in (b),
the embeddings $\frak B_i\embedsinto\frak C_j$ identify $\frak C_j$ with
the free product of $\family{i}{I_j}{\frak B_i}$, so 315Kb tells us that
there must be a finite set $K_j\subseteq I_j$ and elements
$b_i\in\frak B_i^+$, for $i\in K_j$, such that
$\inf_{i\in K_j}b_i\Bsubseteq c_j$.   Now
$\inf_{j\in L}c_j\Bsupseteq\inf\{b_i:i\in\bigcup_{j\in L}K_j\}$ is
non-zero.   As $\family{j}{L}{c_j}$ is arbitrary,
$\family{j}{J}{\frak C_j}$ is Boolean-independent.  (Compare 315L.)

\medskip

{\bf (d)} Set $\frak B_b=\{0,b,1\Bsetminus b,1\}$ for $b\in B$, so that
$\family{b}{B}{\frak B_b}$ is Boolean-independent, by (a);  now apply
(c).

\medskip

{\bf (e)(i)} If $i$, $j\in I$ are distinct, $b\in B_i$ and $b'\in B_j$,
then $b\in\frak B_i^+$ and
$1\Bsetminus b'\in\frak B_j^+$, so $b\Bsetminus b'\ne 0$ and
$b\ne b'$.

\medskip

\quad{\bf (ii)} If $J$, $K$ are disjoint finite subsets of
$\bigcup_{i\in I}B_i$, then $J\cap B_i$ and $K\cap B_i$ are disjoint
finite subsets of $B_i$, so that

\Centerline{$b_i=\inf(J\cap B_i)\Bsetminus\sup(K\cap B_i)
\in\frak B_i^+$}

\noindent for each $i\in I$.   Let $L\subseteq I$ be a finite set such
that $J\cup K\subseteq\bigcup_{i\in L}B_i$;  then

\Centerline{$\inf J\Bsetminus\sup K=\inf_{i\in L}b_i\ne 0$.}

\noindent As $J$ and $K$ are arbitrary, $\bigcup_{i\in I}B_i$ is
Boolean-independent.

\medskip

{\bf (f)} Since $D_i\subseteq\frak B_i$, $\familyiI{D_i}$ must be
Boolean-independent if $\familyiI{\frak B_i}$ is.

On the other hand, each $D_i$ is order-dense in $\frak B_i$.   \Prf\ For
$d\in D_i$, the set $\{b:d\Bsubseteq b$ or $d\Bcap b=0\}$ is an
order-closed subalgebra of $\frak A$ including $D_i$, so includes
$\frak B_i$.
If  $b\in\frak B_i^+$, then (because $\sup D_i=1$ in $\frak A$) there
must be a $d\in D_i$ such that $b\Bcap d\ne 0$, in which case
$0\ne d\Bsubseteq b$.   As $b$ is arbitrary, $D_i$ is order-dense in
$\frak B_i$.\ \Qed

Now suppose that $\familyiI{D_i}$ is Boolean-independent, $J\subseteq I$
is finite and $b_i\in\frak B_i^+$ for each $i\in J$.   Then we have
non-zero $d_i\in D_i$ such that $d_i\Bsubseteq b_i$ for each $i$.   So
$\inf_{i\in J}b_i\Bsupseteq\inf_{i\in J}d_i$ is non-zero.   As
$\family{i}{J}{b_i}$ is arbitrary, $\familyiI{\frak B_i}$ is
Boolean-independent.
}%end of proof of 515B

\leader{515C}{Proposition} Let $\frak A$ be a Boolean algebra, not
$\{0\}$, and $\kappa$ a cardinal.

(a) There is a Boolean-independent subset of $\frak A$ with cardinal
$\kappa$ iff there is a subalgebra of $\frak A$ which is isomorphic to
the algebra of open-and-closed subsets of $\{0,1\}^{\kappa}$.

(b) If $\frak A$ is Dedekind complete,
there is a Boolean-independent subset of $\frak A$ with cardinal
$\kappa$ iff there is a subalgebra of $\frak A$ which is isomorphic to
the regular open algebra of $\{0,1\}^{\kappa}$.

\proof{ Set $Z=\{0,1\}^{\kappa}$;  write $\Cal E$ for the algebra of
open-and-closed subsets of $Z$ and $\frak G$ for the regular open
algebra of $Z$.

\medskip

{\bf (a)(i)} Suppose that $\frak A$ has a Boolean-independent subset of
cardinal $\kappa$, enumerated as $\ofamily{\xi}{\kappa}{a_{\xi}}$.
Setting
$\frak A_{\xi}=\{0,a_{\xi},1\Bsetminus a_{\xi},1\}$ for each $\xi$,
$\ofamily{\xi}{\kappa}{\frak A_{\xi}}$ is a Boolean-independent family
of subalgebras of $\frak A$, and the subalgebra $\frak C$ of $\frak A$
generated by $\bigcup_{\xi<\kappa}\frak A_{\xi}$ can be identified with
the free product of $\ofamily{\xi}{\kappa}{\frak A_{\xi}}$ (515Bb).
But since the Stone space of each $\frak A_{\xi}$ has just two points,
the construction of 315I makes it plain that the Stone space of
$\frak C$ is homeomorphic to $Z$, so that $\frak C$ is
isomorphic to $\Cal E$.

\medskip

\quad{\bf (ii)} In the other direction, the sets
$E_{\xi}=\{z:z\in Z,\,z(\xi)=1\}$ are Boolean-independent in $\Cal E$,
so if $\Cal E$ can be embedded in $\frak A$ there must be a
Boolean-independent subset of $\frak A$ with cardinal $\kappa$.

\medskip

{\bf (b)} Now suppose that $\frak A$ is Dedekind complete.   $\Cal E$ is
an order-dense subalgebra of $\frak G$ (314T).   So if $\frak A$ has a
subalgebra isomorphic to $\frak G$ it certainly has one isomorphic to
$\Cal E$.   On the other hand, if $\frak A$ has a subalgebra isomorphic
to $\Cal E$, so that there is an injective Boolean homomorphism
$\pi:\Cal E\to\frak A$, then (because $\frak A$ is Dedekind complete)
$\pi$ has an extension to a Boolean homomorphism
$\pi_1:\frak G\to\frak A$ (314K);
because $\Cal E$ is order-dense in $\frak G$ and $\pi$ is injective,
$\pi_1$ is injective, so that $\pi_1[\frak G]$ is a
subalgebra of $\frak A$ isomorphic to $\frak G$.

Putting this together with (a), we see that $\frak A$ has a
Boolean-independent subset with cardinal $\kappa$ iff it has a subalgebra
isomorphic to $\frak G$.
}%end of proof of 515C

\leader{515D}{Lemma} Let $\frak A$ be a Dedekind complete Boolean
algebra, not $\{0\}$, and $\frak B$ an order-closed subalgebra of
$\frak A$ such that
$\frak A$ is relatively atomless over $\frak B$.   Then there is an
$a^*\in\frak A\setminus\{0,1\}$ such that
$\frak B$ and $\{0,a^*,1\Bsetminus a^*,1\}$ are Boolean-independent
subalgebras of $\frak A$.

\cmmnt{\medskip

\noindent{\bf Remark} Recall from 331A that a Boolean algebra $\frak A$
is `relatively atomless' over an order-closed subalgebra $\frak B$ if
for every $a\in\frak A^+$ there is a $c\Bsubseteq a$ which is not of the
form $a\Bcap b$ for any $b\in\frak B$.
}%end of comment

\proof{ Set

\Centerline{$C
=\{c:c\in\frak A,\,c\ne 0,\,\frak B\cap\frak A_c=\{0\}\}$,}

\noindent where $\frak A_c$ is the principal ideal of $\frak A$
generated by $c$.   Then $C$ is order-dense in $\frak A$.   \Prf\ If
$a\in\frak A^+$, there is a
$c\in\frak A_a\setminus\{a\Bcap b:b\in\frak B\}$.
Set $b_0=\sup\{b:b\in\frak B,\,b\Bsubseteq c\}$;
then $c\Bsetminus b_0\Bsubseteq a$ and $c\Bsetminus b_0\in C$.\ \Qed

For $a\in\frak A$ set
$\upr(a,\frak B)=\inf\{b:a\Bsubseteq b\in\frak B\}$, as in 313S.   Set
$E=\{\upr(c,\frak B):c\in C\}$.   Then $E$ is order-dense in $\frak B$,
because if $b\in\frak B^+$ there is a $c\in C$ such that
$c\Bsubseteq b$, and now $\upr(c,\frak B)$ belongs to
$E$ and is included in $b$.
So there is a partition $D$ of unity in $\frak B$ included in $E$
(313K).   For each $d\in D$ choose $c_d\in C$ such that
$d=\upr(c_d,\frak B)$, and set $a^*=\sup\{c_d:d\in D\}$.   If
$b\in\frak B^+$, there is a $d\in D$ such that
$b\Bcap d\ne 0$, that is, $d\notBsubseteq 1\Bsetminus b$, so
$c_d\notBsubseteq 1\Bsetminus b$ and $b\Bcap c_d\ne 0$;  so
$b\Bcap a^*\ne 0$.   Also, because $c_d\in C$,
$b\Bcap d\notBsubseteq c_d=a^*\Bcap d$, so $b\notBsubseteq a^*$ and
$b\Bcap(1\Bsetminus a^*)\ne 0$.   As $b$ is arbitrary, $\frak B$ and
$\{0,a^*,1\Bsetminus a^*,1\}$ are Boolean-independent.
}%end of proof of 515D

\leader{515E}{Lemma}\cmmnt{ ({\smc Balcar \& Vojt\'a\v{s} 77})} Let
$\frak A$ be a Boolean algebra.   Suppose that
$C\subseteq\frak A^+$ and that $\#(C)<c(\frak A_c)$ for
every $c\in C$\cmmnt{, where $\frak A_c$ is the
principal ideal of $\frak A$ generated by $c$}.   Then there is a
partition $D$ of unity in $\frak A$ such that every member of $C$
includes a non-zero member of $D$.

\proof{ Enumerate $C$ as $\ofamily{\xi}{\kappa}{c_{\xi}}$.   For each
$\xi<\kappa$, let $B_{\xi}$ be a disjoint set in
$\frak A_{c_{\xi}}^+$ of size $\kappa^+$, and set

\Centerline{$A_{\xi}
=\{\eta:\eta<\kappa,\,\#(\{b:b\in B_{\xi},\,b\Bcap c_{\eta}\ne 0\})
  \le\kappa\}$,}

\Centerline{$B'_{\xi}
=B_{\xi}\setminus\bigcup_{\eta\in A_{\xi}}\{b:b\in B_{\xi},
  \,b\Bcap c_{\eta}\ne 0\}$.}

\noindent Then $B'_{\xi}$ is a disjoint set in $\frak A_{c_{\xi}}$,
$\#(B'_{\xi})=\kappa^+$, and
$\{b:b\in B'_{\xi},\,b\Bcap c_{\eta}\ne 0\}$ is empty if
$\eta\in A_{\xi}$ and has cardinal $\kappa^+$ otherwise.   Now define
$A\subseteq\kappa$ inductively by saying that $\xi\in A$ iff
$\xi\in A_{\eta}$ whenever $\eta\in A\cap\xi$, and set
$B=\bigcup_{\xi\in A}B'_{\xi}$.

$B$ is disjoint.   \Prf\ If $\eta$, $\xi\in A$, $\eta\le\xi$,
$b\in B'_{\eta}$, $b'\in B'_{\xi}$ and $b\ne b'$, then either
$\eta=\xi$ and
$b\Bcap b'=0$ because $B'_{\xi}$ is disjoint, or $\eta<\xi$ and
$\xi\in A_{\eta}$ and $b\Bcap b'\Bsubseteq b\Bcap c_{\xi}=0$.\ \QeD\
Also $D_{\xi}=\{b:b\in B,\,b\Bcap c_{\xi}\ne 0\}$ has cardinal
$\kappa^+$ for every $\xi<\kappa$.   \Prf\ If $\xi\in A$, then
$D_{\xi}\supseteq B'_{\xi}$ has cardinal $\kappa^+$.   If $\xi\notin A$
there is some $\eta\in A\cap\xi$ such that $\xi\notin A_{\eta}$ and
$D_{\xi}\supseteq\{b:b\in B'_{\eta},\,b\Bcap c_{\xi}\ne 0\}$ has
cardinal $\kappa^+$.\ \Qed

We can therefore find an injection $\xi\mapsto b_{\xi}:\kappa\to B$ such
that $c_{\xi}\Bcap b_{\xi}\ne 0$ for every $\xi$.   Let $D$ be any
partition of unity including $\{c_{\xi}\Bcap b_{\xi}:\xi<\kappa\}$;
this works.
}%end of proof of 515E

\leader{515F}{Lemma} Let $\frak A$ be a Dedekind complete Boolean
algebra such that $c(\frak A)=\sat(\frak A)$ and $\frak A$ is
cellularity-homogeneous.   Then there is a Boolean-independent family
$\familyiI{D_i}$ of partitions of unity in $\frak A$ such that
$\#(I)=\sup_{i\in I}\#(D_i)=c(\frak A)$.

\proof{ Write $\kappa$ for $c(\frak A)=\sat(\frak A)$.   Choose
$\ofamily{\xi}{\kappa}{D_{\xi}}$ inductively, as follows.
Given $D_{\eta}$ for $\eta<\xi$, let $\frak C_{\xi}$ be the subalgebra
of $\frak A$ generated by $\bigcup_{\eta<\xi}D_{\eta}$;  then
$\#(\frak C_{\xi})<\kappa$.   (Recall from 513Bb that
$\kappa=\sat^{\downarrow}(\frak A^+)$ must be a regular
uncountable cardinal, while of course $\#(D_{\eta})<\kappa$ for every
$\eta$.)   By 515E we have a partition $D$ of unity in $\frak A$, not
containing $\{0\}$, such that every non-zero element of $\frak C_{\xi}$
includes an element of $D$.   For each $d\in D$ the principal ideal of
$\frak A$ generated by $d$ has cellularity $\kappa>\#(\xi)$ so
there is a disjoint family
$\langle b_{d\eta}\rangle_{\eta\le\xi}$ of non-zero elements with supremum
$d$.   Set $b_{\eta}=\sup_{d\in D}b_{d\eta}$ for $\eta\le\xi$, and
$D_{\xi}=\{b_{\eta}:\eta\le\xi\}$;  then $D_{\xi}$ is a partition of
unity in $\frak A$.

The construction ensures that whenever $d\in D_{\xi}$ and
$c\in\frak C_{\xi}^+$ then $d\Bcap c\ne 0$.   It follows
that $\ofamily{\xi}{\kappa}{D_{\xi}}$ is Boolean-independent.   \Prf\ I
show by induction on $\#(J)$ that if $J\subseteq\kappa$ is finite and
$d_{\xi}\in D_{\xi}$ for each $\xi\in J$, then
$\inf_{\xi\in J}d_{\xi}\ne 0$.   If $J$ is empty this is trivial.   For
the inductive step to $\#(J)=n+1$, set $\xi=\max J$ and $J'=\xi\cap J$.
By the inductive hypothesis, $c=\inf_{\eta\in J'}d_{\eta}$ is non-zero;
but $c\in\frak C_{\xi}$, so, by the construction of
$D_{\xi}$, $c\Bcap d_{\xi}=\inf_{\eta\in J}d_{\eta}$ is non-empty.
So the induction proceeds.\ \Qed
}%end of proof of 515F

\leader{515G}{Lemma} Let $\familyiI{\frak A_i}$ be a non-empty family of
Boolean algebras with simple product $\frak A$.   Suppose that
for each $i\in I$ the algebra $\frak A_i$ has a Boolean-independent set
with cardinal $\kappa_i\ge\omega$.   Then $\frak A$ has a
Boolean-independent set with cardinal $\kappa=\#(\prod_{i\in I}\kappa_i)$.

\proof{ For each $i\in I$ let $B_i$ be a Boolean-independent set in
$\frak A_i$ with cardinal $\kappa_i$.
Let $\ofamily{\xi}{\kappa}{a_{\xi}}$ be a family in
$\prod_{i\in I}B_i\subseteq\frak A$ such that for every finite
$J\subseteq\kappa$ there is an $i\in I$ such that
$a_{\xi}(i)\ne a_{\eta}(i)$ whenever $\xi$,
$\eta\in J$ are distinct (5A1K).    Now $\ofamily{\xi}{\kappa}{a_{\xi}}$
is Boolean-independent in $\frak A$.   \Prf\ Suppose that $J$,
$K\subseteq\kappa$ are finite and disjoint.   Then there is an $i\in I$
such that $a_{\xi}(i)\ne a_{\eta}(i)$ whenever $\xi$, $\eta\in J\cup K$
are distinct.   But this means that $\family{\xi}{J\cup K}{a_{\xi}(i)}$
is Boolean-independent in $\frak A_i$, so that, setting
$a=\inf_{\xi\in J}a_{\xi}\Bsetminus\sup_{\xi\in K}a_{\xi}$,

\Centerline{$a(i)
=\inf_{\xi\in J}a_{\xi}(i)\Bsetminus\sup_{\xi\in K}a_{\xi}(i)\ne 0$,}

\noindent and $a\ne 0$.   As $J$ and $K$ are arbitrary,
$\ofamily{\xi}{\kappa}{a_{\xi}}$ is a Boolean-independent family.\ \Qed

Accordingly $\{a_{\xi}:\xi<\kappa\}$ is a Boolean-independent set of
size $\kappa$.
}%end of proof of 515G

\leader{515H}{The Balcar-Fran\v{e}k theorem}\cmmnt{ ({\smc Balcar \&
Fran\v{e}k 82})} Let $\frak A$ be an infinite
Dedekind complete Boolean algebra.   Then there is a Boolean-independent
set $A\subseteq\frak A$ such that $\#(A)=\#(\frak A)$.

\proof{ Set $\kappa=\#(\frak A)$.   For $a\in\frak A$ write $\frak A_a$
for the principal ideal of $\frak A$ generated by $a$.

\medskip

{\bf (a)} Suppose that $\frak A$ is purely atomic.   Then $\frak A$ has
an independent set of size $\kappa$.   \Prf\ Let $B$ be the set of its
atoms;  because
$\frak A$ is infinite, so is $B$;  set $\lambda=\#(B)$, so that

\Centerline{$\frak A\cong\prod_{b\in B}\frak A_b\cong\Cal P\lambda$}

\noindent (315F(iii)), and $\kappa=2^{\lambda}$.   There is a dense
subset $D$ of $\{0,1\}^{\kappa}$ with $\#(D)=\lambda$ (5A4Be);  let
$f:B\to D$ be a surjection.   For $\xi<\kappa$ set

\Centerline{$a_{\xi}=\sup\{b:b\in B,\,f(b)(\xi)=1\}$.}

\noindent If $J$, $K\subseteq\kappa$ are disjoint finite sets, the set

\Centerline{$G=\{x:x\in\{0,1\}^{\kappa},\,x(\xi)=1\Forall\xi\in J,
  \,x(\eta)=0\Forall\eta\in K\}$}

\noindent is a non-empty open set, so there is a $b\in B$ such that
$f(b)\in G$;  but this means that
$\inf_{\xi\in J}a_{\xi}\Bsetminus\sup_{\eta\in K}a_{\eta}\Bsupseteq b$
is non-zero.   As $J$ and $K$ are arbitrary, $\{a_{\xi}:\xi<\kappa\}$ is
an independent subset of $\frak A$ with cardinal $\kappa$.\ \Qed

\medskip


{\bf (b)} Suppose that $\frak A$ is \Mth, and that $\frak B$ is an
order-closed subalgebra of $\frak A$ with Maharam type less than
$\tau(\frak A)$.   Then there is a subalgebra $\frak C$ of $\frak A$,
Boolean-independent of $\frak B$, such that $\frak C$ has a
Boolean-independent subset with cardinal $\tau(\frak A)$.   \Prf\ Let
$B\subseteq\frak B$ be a set with cardinal less than $\tau(\frak A)$ which
$\tau$-generates $\frak B$.   Choose
$\ofamily{\xi}{\tau(\frak A)}{c_{\xi}}$ inductively, as follows.   Given
$\ofamily{\eta}{\xi}{c_{\eta}}$, where $\xi<\tau(\frak A)$, let
$\frak B_{\xi}$ be the order-closed subalgebra of $\frak A$ generated by
$B\cup\{c_{\eta}:\eta<\xi\}$.   If $a\in\frak A^+$, the
order-closed subalgebra
$\frak D=\{a\Bcap b:b\in\frak B_{\xi}\}$ of $\frak A_a$ is
$\tau$-generated by
$\{a\Bcap c_{\eta}:\eta<\xi\}\cup\{a\Bcap b:b\in B\}$ (314Hb), so
$\tau(\frak D)<\tau(\frak A)=\tau(\frak A_a)$ and
$\frak D\ne\frak A_a$.   Thus $\frak A$ is relatively atomless over
$\frak B_{\xi}$;  by 515D, there is a
$c_{\xi}\in\frak A\setminus\{0,1\}$ such that $\frak B_{\xi}$ and
$\{0,c_{\xi},1\Bsetminus c_{\xi},1\}$ are Boolean-independent.
Continue.   Now an easy induction on $\#(J\cup K)$ (as in the last part
of the proof of 515F) shows that if $J$,
$K$ are disjoint finite subsets of $\tau(\frak A)$, and $b\in\frak B$ is
non-zero,
$b\Bcap\inf_{\xi\in J}c_{\xi}\Bsetminus\sup_{\eta\in K}c_{\eta}\ne 0$.
So if we take $\frak C$ to be the subalgebra of $\frak A$ generated by
$C=\{c_{\xi}:\xi<\tau(\frak A)\}$, $\frak C$ and $\frak B$ are
Boolean-independent and $C\subseteq\frak C$ is a Boolean-independent set
with cardinal $\tau(\frak A)$.\ \Qed

\medskip

{\bf (c)} Suppose that $\frak A$ is \Mth\ and that
$c(\frak A)<\sat(\frak A)$.   Then $\frak A$ has a Boolean-independent
subset with cardinal $\kappa$.   \Prf\ Because $\frak A$ is infinite,
$c(\frak A)$ is infinite.   Let $D\subseteq\frak A^+$ be a
disjoint set with cardinal $c(\frak A)$;  adding $1\Bsetminus\sup D$ if
necessary, we may suppose that $D$ is a partition of unity.   For each
$d\in D$, $\frak A_d$ has a Boolean-independent set with cardinal
$\tau(\frak A_d)=\tau(\frak A)$ (apply (b) above to $\frak A_d$, with
$\frak D=\{0,d\}$).   By
315F(iii) again, $\frak A\cong\prod_{d\in D}\frak A_d$;  by 515G,
$\prod_{d\in D}\frak A_d$ has a Boolean-independent subset of size the
cardinal power
$\tau(\frak A)^{\#(D)}=\tau(\frak A)^{c(\frak A)}$, so $\frak A$ also
has.   But

\Centerline{$\kappa
\le\sup_{\lambda<\sat(\frak A)}\tau(\frak A)^{\lambda}
=\tau(\frak A)^{c(\frak A)}$}

\noindent by 514De, so $\frak A$ has a Boolean-independent set of
cardinal $\kappa$.\ \Qed

\medskip

{\bf (d)} Suppose that $\frak A$ is cellularity-homogeneous and
\Mth\ and
$c(\frak A)=\sat(\frak A)$.   Then $\frak A$ has a Boolean-independent
subset with cardinal $\kappa$.

\medskip

\Prf\ {\bf (i)} By 515F, we can find a
Boolean-independent family $\familyiI{D_i}$ of partitions of unity in
$\frak A$ such that $\#(I)=\sup_{i\in I}\#(D_i)=\sat(\frak A)$.   We
know that $\sat(\frak A)\ge\omega_1$, so we can suppose that all the
$D_i$ are infinite.   For each $i\in I$, let $\frak D_i$ be the
order-closed subalgebra of $\frak A$ generated by $D_i$.   By (a) above,
$\frak D_i$ has a Boolean-independent subset $B_i$ with cardinal
$2^{\#(D_i)}$, so that $B=\bigcup_{i\in I}B_i$ has cardinal
$\sup_{\lambda<\sat(\frak A)}2^{\lambda}$.
By 515Df, $\familyiI{\frak D_i}$ is Boolean-independent.   By 515De, $B$
is Boolean-independent.

\medskip

\quad{\bf (ii)} If $\#(B)=\kappa$, we can stop.   Otherwise, let $\frak D$
be the order-closed subalgebra of $\frak A$ generated by
$D=\bigcup_{i\in I}D_i$.   Because

$$\eqalignno{\sup_{\lambda<\sat(\frak A)}\#(B)^{\lambda}
&=\#(B)
\Displaycause{5A1Ef}
<\kappa
\le\sup_{\lambda<\sat(\frak A)}\tau(\frak A)^{\lambda}
\cr}$$

\noindent (514Be), we must have

\Centerline{$\tau(\frak A)>\#(B)=\sup_{\lambda<\sat(\frak A)}2^{\lambda}
\ge\sat(\frak A)=\#(D)\ge\tau(\frak D)$.}

\noindent By (b), we have a subalgebra $\frak C$ of $\frak A$,
Boolean-independent of $\frak D$, such that $\frak C$ has a
Boolean-independent subset $C$ with cardinal $\tau(\frak A)$.
Let $\familyiI{C_i}$ be a disjoint family of subsets of $C$ all with
cardinal $\tau(\frak A)$.

For $i\in I$, let $\frak C_{i0}$ be the subalgebra of $\frak A$ generated
by $D_i$ and $\frak C_{i1}$ the subalgebra generated by $C_i$.
Let $\frak E_i$ be the subalgebra generated by
$\frak C_{i0}\cup\frak C_{i1}$ and $\widehat{\frak E}_i$ its Dedekind
completion (314T-314U).   In $\widehat{\frak E}_i$ we have the partition
of unity $D_i$ and the Boolean-independent set $C_i$ with cardinal
$\tau(\frak A)$.
For each $b\in D_i$, the principal ideal $(\widehat{\frak E}_i)_b$ of
$\widehat{\frak E}_i$ generated by $b$ has a Boolean-independent set
$\{b\Bcap c:c\in C_i\}$ with cardinal $\tau(\frak A)$.   Because
$\widehat{\frak E}_i$ is Dedekind complete, it is isomorphic to
$\prod_{b\in D_i}(\widehat{\frak E}_i)_b$, and has a Boolean-independent
subset with cardinal $\tau(\frak A)^{\#(D_i)}$ (515G again).

Because $\frak A$ is Dedekind complete, the
embedding $\frak E_i\embedsinto\frak A$ extends to a Boolean homomorphism
$\pi_i:\widehat{\frak E}_i\to\frak A$ (314K).
Because $\frak E_i$ is order-dense in $\widehat{\frak E}_i$, 
$\pi_i$ is injective.
So $\frak E_i^*=\pi_i[\widehat{\frak E}_i]$ is a
subalgebra of $\frak A$ isomorphic to $\widehat{\frak E}_i$, and has a
Boolean-independent subset $E_i$ with cardinal $\tau(\frak A)^{\#(D_i)}$.

\medskip

\quad{\bf (iii)} By 515Df and 515Dd, $\familyiI{\frak C_{i0}}$ and
$\familyiI{\frak C_{i1}}$ are both Boolean-independent families;  because
$\frak C_{i0}\subseteq\frak D$ and $\frak C_{j1}\subseteq\frak C$ whenever
$i$, $j\in I$, and $\frak D$ and $\frak C$ are Boolean-independent,
$\langle\frak C_{ij}\rangle_{i\in I,j\in\{0,1\}}$ is Boolean-independent,
so $\familyiI{\frak E_i}$ is Boolean-independent (515Dc).   If
$J\subseteq I$ is finite, and $e_i\in(\frak E_i^*)^+$ for each
$i\in J$, then there are $e'_i\in\frak E_i$ such that
$0\ne e'_i\Bsubseteq e_i$ for each $i$.   Now
$\inf_{i\in J}e_i\Bsupseteq\inf_{i\in J}e'_i\ne 0$.   As
$\family{i}{J}{e_i}$ is arbitrary, $\familyiI{\frak E^*_i}$ is
Boolean-independent.   But this means that $E=\bigcup_{i\in I}E_i$ is
Boolean-independent (515De), while

\Centerline{$\#(E)\ge\sup_{i\in I}\tau(\frak A)^{\#(D_i)}
=\sup_{\lambda<\sat(\frak A)}\tau(\frak A)^{\lambda}\ge\kappa$.}

\noindent Of course $\#(E)\le\#(\frak A)=\kappa$, so
we have a Boolean-independent set with cardinal $\kappa$ in this case
also.\ \Qed

\medskip

{\bf (e)} If $\frak A$ is atomless it has a Boolean-independent subset
with cardinal $\kappa$.   \Prf\ Because Maharam type and cellularity are
both order-preserving cardinal functions (514Ed), $\frak A$ is
isomorphic to the product of a family $\familyiI{\frak A_i}$ of \Mth\
cellularity-homogeneous algebras, none of them $\{0\}$ (514Gc).
Now, for each $i$, $\frak A_i$ is an atomless (therefore infinite) \Mth\
cellularity-homogeneous Dedekind complete Boolean algebra, so by (c)-(d)
above has a Boolean-independent set with cardinal $\#(\frak A_i)$.   By
515G once more, $\frak A$ has a Boolean-independent set with cardinal
$\#(\prod_{i\in I}\frak A_i)=\kappa$.\ \Qed

\medskip

{\bf (f)} Finally, for the general case, let $A$ be the set of atoms of
$A$ and set $c=\sup A$, so that the principal ideal $\frak A_c$ is
purely atomic and the principal ideal $\frak A_{1\Bsetminus c}$ is
atomless.   Because $\frak A\cong\frak A_c\times\frak A_{1\Bsetminus c}$
is infinite, one of $\frak A_c$, $\frak A_{1\Bsetminus c}$ has cardinal
$\kappa$, and therefore (by (a) or (f)) has a Boolean-independent subset
with cardinal $\kappa$;  which is now a Boolean independent subset of
$\frak A$ with cardinal $\kappa$.

This completes the proof.
}%end of proof of 515H

\leader{515I}{Corollary} If $\frak A$ is an infinite Dedekind complete
Boolean algebra and $\kappa\le\#(\frak A)$,
$\frak A$ has a subalgebra isomorphic to the regular open algebra of
$\{0,1\}^{\kappa}$.

\proof{ By 515H, $\frak A$ has a
Boolean-independent family $\ofamily{\xi}{\kappa}{a_{\xi}}$.
By 515Cb, $\frak A$ has a subalgebra isomorphic to the regular open
algebra of $\{0,1\}^{\kappa}$.
}%end of proof of 515I

\leader{515J}{Corollary} If $\frak A$ is an infinite Dedekind complete
Boolean algebra with Stone space $Z$, then $\#(Z)=2^{\#(\frak A)}$.

\proof{ Since $Z$ may be identified with the set of uniferent ring
homomorphisms from $\frak A$ to $\Bbb Z_2$ (311E),
$\#(Z)\le 2^{\#(\frak A)}$.   On the other hand, writing
$W=\{0,1\}^{\#(\frak A)}$, we have a subalgebra of $\frak A$ isomorphic to
the algebra $\Cal E$ of open-and-closed subsets of $W$ (515I).
If $\pi:\Cal E\to\frak A$ is an injective Boolean homomorphism,
it corresponds to a surjective continuous function $\psi:Z\to W$
(312Sa), so that $\#(Z)\ge\#(W)=2^{\#(\frak A)}$.
}%end of proof of 515J

\leader{515K}{}\cmmnt{ I extract part of the proof of the next theorem
as a lemma.

\medskip

\noindent}{\bf Lemma} Let $\frak A$ be an infinite Boolean algebra with
the $\sigma$-interpolation property.

(a) Let $\sequencen{a_n}$ be a disjoint sequence in $\frak A$.   Then
$\#(\frak A)\ge\#(\prod_{n\in\Bbb N}\frak A_{a_n})$\cmmnt{, writing
$\frak A_d$ for the principal ideal of $\frak A$ generated by $d$, as
usual}.

(b) Set $\kappa=\#(\frak A)$, and let $I$ be the set of those
$a\in\frak A$ such that $\#(\frak A_a)<\kappa$.   Then $I$ is an ideal
of $\frak A$, and

\quad{\it either} $\frak A/I$ is infinite,

\quad{\it or} there is a set $J\subseteq I$ with cardinal $\kappa$
such that every sequence in $J$ has an upper bound in $J$,

\quad{\it or} $\#(\prod_{n\in\Bbb N}\frak A_{a_n})=\kappa$ for
some sequence $\sequencen{a_n}$ in $I$.

\cmmnt{\medskip

\noindent{\bf Remark} Recall from 466G that $\frak A$ has the
`$\sigma$-interpolation property' if whenever $A$, $B\subseteq\frak A$
are countable and $a\Bsubseteq b$ for every $a\in A$ and $b\in B$, then
there is a $c\in\frak A$ such that $a\Bsubseteq c\Bsubseteq b$ for every
$a\in A$ and $b\in B$.   See also 514Yf above.
}%end of comment

\proof{{\bf (a)} The point is that the map
$a\mapsto\sequencen{a\Bcap a_n}:
\frak A\to\prod_{n\in\Bbb N}\frak A_{a_n}$ is surjective.
\Prf\ If $\sequencen{b_n}\in\prod_{n\in\Bbb N}\frak A_{a_n}$, there must
be an $a\in\frak A$ such that
$b_n\Bsubseteq a\Bsubseteq 1\Bsetminus(a_n\Bsetminus b_n)$ for every
$n$, so that $a\Bcap a_n=b_n$ for every $n$.\ \QeD\  The result follows
at once.

\medskip

{\bf (b)} If $a$, $b\in I$ then $(c,d)\mapsto c\Bcup d$ is a surjection
from $\frak A_a\times\frak A_b$ onto $\frak A_{a\Bcup b}$, so
$a\Bcup b\in I$;  of course $b\in I$ whenever $b\Bsubseteq a\in I$, so
$I$ is an ideal of $\frak A$.

\Quer\ Suppose, if possible, that all three alternatives are false.
Then $\frak A/I$ is finite;  let $v_0,\ldots,v_m$ be its atoms.
Let  $c_0,\ldots,c_m\in\frak A$ be such that $c_i^{\ssbullet}=v_i$ for
every $i$.   Observe that
$\frak A$ is the union of finitely many sets of size $\#(I)$,
so $I$ itself must
have cardinal $\kappa$, and there is a sequence $\sequencen{b'_n}$ in $I$
with no upper bound in $I$;  setting $b_n=b'_n\Bsetminus\sup_{m<n}b'_m$
for each $n$, we get a disjoint sequence $\sequencen{b_n}$ in $I$ with
no upper bound in $I$.
Now there is some $k\le m$ such that
$\sequencen{b_n\Bcap c_k}$ has no upper bound in $I$.   Set
$K=\{d:d\Bsubseteq c_k,\,d\Bcap b_n=0$ for every $n\in\Bbb N\}$.
Then $K\normalsubgroup\frak A_{c_k}$.   If $d\in K$, $c_k\Bsetminus d$
is an upper bound for
$\{b_n\cap c_k:n\in\Bbb N\}$, so does not belong to $I$;  as
$c_k^{\ssbullet}$ is an atom in $\frak A/I$, $d$ must belong to $I$.
Thus $K\subseteq I$.
The function $d\mapsto\sequencen{d\Bcap b_n}:
\frak A_{c_k}\to\prod_{n\in\Bbb N}\frak A_{c_k\Bcap b_n}$ is a Boolean
homomomorphism with kernel $K$, so

\Centerline{$\#(\frak A_{c_k}/K)
\le\#(\prod_{n\in\Bbb N}\frak A_{c_k\Bcap b_n})<\kappa$}

\noindent (since the third alternative is false, and
$\#(\prod_{n\in\Bbb N}\frak A_{c_k\Bcap b_n})\le\kappa$ by (a));  as
$\#(\frak A_{c_k})=\kappa$, $\#(K)=\kappa$.
There is therefore a sequence $\sequencen{d_n}$ in $K$ with no upper
bound in $K$.   But there is a $d\in\frak A$ such that
$d_n\Bsubseteq d\Bsubseteq 1\Bsetminus b_n$ for every $n\in\Bbb N$,
because $\frak A$ has the $\sigma$-interpolation property;   so that
$d\Bcap c_k\in K$ is an upper bound for $\{d_n:n\in\Bbb N\}$.\ \Bang
}%end of proof of 515K

\leader{515L}{Theorem}\cmmnt{ ({\smc Koppelberg 75})} If
$\frak A$ is an infinite Boolean algebra with the $\sigma$-interpolation
property, then $\#(\frak A)$ is equal to the cardinal power
$\#(\frak A)^{\omega}$.

%Pierce 58 does the Dedekind complete case

\proof{ Induce on $\kappa=\#(\frak A)$.

\medskip

{\bf (a)} If $\kappa\le\frak c$, then (because $\frak A$ is infinite)
there is a disjoint sequence $\sequencen{a_n}$ in $\frak A^+$, so that

\Centerline{$\frak c\le\#(\prod_{n\in\Bbb N}\frak A_{a_n})
\le\#(\frak A)$}

\noindent by 515Ka, and $\kappa=\frak c$.  So
$\kappa^{\omega}=(2^{\omega})^{\omega}=\kappa$.

\medskip

{\bf (b)} For the inductive step to $\kappa>\frak c$, set
$I=\{a:a\in\frak A,\,\#(\frak A_a)<\kappa\}$, as in 515Kb.
It is easy to see that every principal ideal of $\frak A$ has the
$\sigma$-interpolation property, so that
$\#(\frak A_a)^{\omega}\le\max(\frak c,\#(\frak A_a))$ for every
$a\in I$.   Now consider the three possibilities of 515Kb.

\medskip

\quad{\bf case 1} If the quotient algebra $\frak A/I$ is infinite, then
$\kappa^{\omega}=\kappa$.   \Prf\ There is a disjoint sequence
$\sequencen{u_n}$ in $\frak A/I$.   For each $n\in\Bbb N$ take
$a_n\in\frak A$ such that $a_n^{\ssbullet}=u_n$;  now setting
$a'_n=a_n\Bsetminus\sup_{i<n}a_i$ for each $n$, $\sequencen{a'_n}$ is a
disjoint sequence in $\frak A\setminus I$.   So

\Centerline{$\kappa\le\kappa^{\omega}
=\#(\prod_{n\in\Bbb N}\frak A_{a'_n})\le\kappa$}

\noindent by 515Ka again.\ \Qed

\medskip

\quad{\bf case 2} Suppose that there is a set $J\subseteq I$ such that
$\#(J)=\kappa$ and every sequence in $J$ has an upper bound in $J$.
Then $\kappa^{\omega}=\kappa$.   \Prf\

$$\eqalignno{\kappa^{\omega}
&=\#(J^{\Bbb N})
\le\#(\bigcup_{a\in J}\frak A_a^{\Bbb N})\cr
\displaycause{because every sequence in $J$ is included in $\frak A_a$
for some $a\in J$}
&\le\max(\omega,\#(J),\sup_{a\in I}\#(\frak A_a^{\Bbb N}))
\le\max(\kappa,\sup_{a\in J}\#(\frak A_a))
=\kappa\le\kappa^{\omega}. \text{ \Qed}\cr}$$

\medskip

\quad{\bf case 3} Suppose there is a sequence $\sequencen{a_n}$ in $I$
such that $\#(\prod_{n\in\Bbb N}\frak A_{a_n})=\kappa$.   Then
$\kappa^{\omega}=\kappa$.   \Prf\ Set
$L=\{n:n\in\Bbb N,\,\frak A_{a_n}$ is infinite$\}$.   Then

$$\eqalign{\kappa^{\omega}
&=\#((\prod_{n\in\Bbb N}\frak A_{a_n})^{\Bbb N})
=\#(\prod_{n\in\Bbb N\setminus L}\frak A_{a_n}^{\Bbb N}
  \times\prod_{n\in L}\frak A_{a_n}^{\Bbb N})\cr
&\le\#(\frak c\times\prod_{n\in L}\frak A_{a_n})
\le\max(\frak c,\kappa)
=\kappa.  \text{ \Qed}\cr}$$

\noindent Thus in all three cases we have $\kappa^{\omega}=\kappa$, and
the induction proceeds.
}%end of proof of 515L

\leader{515M}{Corollary} (a) If $\frak A$ is an infinite ccc Dedekind
$\sigma$-complete Boolean algebra then $\#(\frak A)=\tau(\frak A)^{\omega}$.

(b) If $\frak A$ is any infinite Dedekind $\sigma$-complete Boolean
algebra, then $\#(L^0(\frak A))=\#(L^{\infty}(\frak A))=\#(\frak A)$.

\proof{{\bf (a)} Of course $\frak A$, being Dedekind $\sigma$-complete, has the
$\sigma$-interpolation property, as noted in 466G.   So by 515L and
514De,

\Centerline{$\tau(\frak A)^{\omega}\le\#(\frak A)^{\omega}
=\#(\frak A)\le\tau(\frak A)^{\omega}$.}

\medskip

{\bf (b)} $a\mapsto\chi a:\frak A\to L^{\infty}(\frak A)$ and
$u\mapsto\family{q}{\Bbb Q}{\Bvalue{u>q}}:L^0(\frak A)\to\frak A^{\Bbb Q}$
are injective, so

\Centerline{$\#(\frak A)\le\#(L^{\infty}(\frak A))
\le\#(L^0(\frak A))\le\#(\frak A)^{\omega}=\#(\frak A)$.}
}%end of proof of 515M

\leader{515N}{}\cmmnt{ It will be convenient at one point later to
know a little more about the regular open algebras of powers of $\{0,1\}$.

\medskip

\noindent}{\bf Proposition} Let $\frak A$ be a Boolean algebra, and
$\kappa$ a cardinal.   Then $\frak A$ is isomorphic to the regular open
algebra $\RO(\{0,1\}^{\kappa})$ iff it is Dedekind complete and
there is a Boolean-independent family
$\ofamily{\xi}{\kappa}{e_{\xi}}$ in $\frak A$ such that the subalgebra
generated by $\{e_{\xi}:\xi<\kappa\}$ is order-dense in $\frak A$.

\proof{ Write $\Cal E$ for the algebra of open-and-closed subsets of
$\{0,1\}^{\kappa}$.

\medskip

{\bf (a)} $\RO(\{0,1\}^{\kappa})$ is Dedekind
complete (314P) and $\Cal E$ is order-dense in $\RO(\{0,1\}^{\kappa})$,
because $\{0,1\}^{\kappa}$ is zero-{\vthsp}dimensional.   Setting
$e_{\xi}=\{x:x\in\{0,1\}^{\kappa}$, $x(\xi)=1\}$,
$\ofamily{\xi}{\kappa}{e_{\xi}}$ is Boolean-independent and generates
$\Cal E$.   So $\RO(\{0,1\}^{\kappa})$ has the declared properties.

\medskip

{\bf (b)} If $\frak A$ satisfies the conditions, then, as in (a-i) of the
proof of 515C, the subalgebra of $\frak A$ generated by
$\{e_{\xi}:\xi<\kappa\}$ is isomorphic to $\Cal E$.   Now both $\frak A$
and $\RO(\{0,1\}^{\kappa})$ are Dedekind completions of $\Cal E$, so they
are isomorphic (314U).
}%end of proof of 515N

\exercises{\leader{515X}{Basic exercises (a)}
%\spheader 515Xa
Let $\frak A$ be a Boolean algebra, not $\{0\}$, and $\familyiI{D_i}$
a family of partitions of unity in $\frak A$, none containing $0$.
Show that the following
are equiveridical:  (i) $\familyiI{D_i}$ is Boolean-independent;  (ii)
$\familyiI{\frak B_i}$ is Boolean-independent, where $\frak B_i$ is the
subalgebra of $\frak A$ generated by $D_i$ for each $i\in I$.
%515B

\spheader 515Xb\dvAnew{2014}
Give an example of a Boolean algebra $\frak A$ with
Boolean-independent subalgebras $\frak B$, $\frak C$ such that the
order-closed subalgebras generated by $\frak B$ and $\frak C$ are not
Boolean-independent.
%515B

\spheader 515Xc For a Boolean algebra $\frak A$, not $\{0\}$,
write $\ind(\frak A)$
for $\sup\{\#(A):A\subseteq\frak A$ is Boolean-independent$\}$.   (If
$\frak A=\{0\}$, say $\ind(\frak A)=0$.)   (i)
Show that if $\frak B$ is either a subalgebra or a principal ideal or a
homomorphic image of $\frak A$ then $\ind(\frak B)\le\ind(\frak A)$.
(ii) Show that
$\frak A$ is infinite iff $\ind(\frak A)$ is infinite.   (iii) Show that
if $\frak A$ is finite and not $\{0\}$ then $\ind(\frak A)$ is the
largest $n$ such that
$2^{2^n}\le\#(\frak A)$.   (iv) Show that if $\frak A$ is the
finite-cofinite algebra of subsets of an infinite set $X$, then
$\ind(\frak A)=\omega$ but $\frak A$ has no infinite Boolean-independent
set.   (v) Show that if $\frak A$ and $\frak B$ are Boolean algebras
then
$\ind(\frak A\times\frak B)$ is at most the cardinal sum
$\ind(\frak A)+\ind(\frak B)$.
(vi) Show that if $\frak A$ is infinite and has the
$\sigma$-interpolation property then $\ind(\frak A)\ge\frak c$.
%515B

\spheader 515Xd Let $Z$ be an infinite extremally disconnected
compact Hausdorff space.   Show that there is a
continuous surjection from $Z$ onto $\{0,1\}^{w(Z)}$.
%515H

\spheader 515Xe Let $\frak A$ be a Boolean algebra with the
$\sigma$-interpolation property.   Show that any homomorphic image of
$\frak A$ has the $\sigma$-interpolation property.
%515K

\spheader 515Xf Let $\kappa$ be an infinite cardinal.   Show that the
following are equiveridical:  (i) there is a measure algebra with
cardinal $\kappa$;  (ii) there is a measurable algebra with cardinal
$\kappa$;  (iii) $\kappa^{\omega}=\kappa$.
%515M

\leader{515Y}{Further exercises (a)}(i)
%spheader 515Ya
Show that if $\frak A$ is any Boolean algebra, other than $\{0\}$,
with cardinal at most $\omega_1$,
it is isomorphic to a subalgebra of $\Cal P\Bbb N/[\Bbb N]^{<\omega}$.
%514Yf(i)  $\Cal P\Bbb N/[\Bbb N]^{<\omega}$ has $\sigma$-interp propy
%do purely atomic and atomless separately
(ii) Show that an atomless Boolean algebra
with cardinal $\omega_1$ and the $\sigma$-interpolation property is
isomorphic to $\Cal P\Bbb N/[\Bbb N]^{<\omega}$.   (This is a version of
{\bf Parovi\v{c}enko's theorem}.)
%Koppelberg 5.30

\spheader 515Yb Let $\frak A$ be a Dedekind complete Boolean algebra, and
$\kappa\le\#(\frak A)$ a regular uncountable cardinal.   Show
that there is a strictly increasing family
$\ofamily{\xi}{\kappa}{\frak A_{\xi}}$ of subalgebras of $\frak A$ with
union $\frak A$.   (Compare 494Yk.)
%find suitable homomorphism from \frak A to \frak B_{\kappa}
%query:  what about other algebras?
}%end of exercises

\endnotes{
\Notesheader{515}
The material of this section is taken from {\smc Koppelberg 89}, where
you can find a good deal more.   I have picked out the results which are
essential to a proper understanding of measure algebras.   Of course
there are short cuts, using Maharam's theorem (332B), if we know that we
are dealing with a localizable measure algebra;  but I should not like
to leave you with the impression that the theorems here are restricted
to measure algebras.

Any theorem about Boolean algebras is also a theorem about
zero-dimensional compact Hausdorff spaces;  thus 515H and 515Xd have an
equal right to be called the Balcar-Fran\v{e}k theorem.
515D and part (b) of the proof of 515H may be regarded as a simple form
of some of the ideas of \S331.

Clearly some of the ideas of this section can be expressed in terms of
the independence number $\ind(\frak A)$ (515Xc).   But the expression is
complicated by the fact that (like cellularity) the independence number
may not be attained (see 515Xc(iv)), while the theorems here mostly need
actual independent families.   Since $\ind(\frak A)=\#(\frak A)$ for
infinite Dedekind complete Boolean algebras (515H), we shall not have to
grapple with these difficulties.
}%end of notes

\discrpage

