\frfilename{mt438.tex}
\versiondate{13.12.06/10.10.07}
\copyrightdate{2007}

\def\Marik{Ma\v r\'\i k}

\def\chaptername{Topologies and measures II}
\def\sectionname{Measure-free cardinals}

\newsection{438}

At several points in \S418, and again in \S434, we had theorems about
separable metrizable spaces in which the proofs undoubtedly needed some
special property of these spaces (e.g., the fact that they are
Lindel\"of), but left it unclear whether something more general could be
said.   When we come to investigate further, asking (for instance)
whether complete metric spaces in general are Radon (438H), we find
ourselves once again approaching the
Banach-Ulam problem, already mentioned at several points in previous
volumes\cmmnt{, and in particular in 363S}.
It seems to be undecidable, in
ordinary set theory with the axiom of choice, whether or not every
discrete space is Radon in the sense of 434C.
On the other hand it
is known that discrete spaces of cardinal at most $\omega_{\omega_1}$ (for
instance) are indeed always Radon.   While as a rule I am deferring
questions of this type to Volume 5, this particular phenomenon is so
pervasive that I think it is worth taking a section now to clarify it.

The central definition is that of `measure-free cardinal' (438A), and
the basic results are 438B-438D.   In particular, `small' infinite
cardinals are measure-free (438C).   From the point of view of
measure theory, a metrizable space whose weight is measure-free is
almost separable, and most of the results in \S418 concerning separable
metrizable spaces can be extended (438E-438G).   In fact `measure-free
weight' exactly determines whether a metrizable space is
measure-compact (438J\cmmnt{, 438Xm}) and whether a complete metric
space is Radon (438H).   If $\frak c$ is measure-free, some
interesting spaces of functions are Radon (438T).   I approach these
last spaces through the concept of `hereditary weak
$\theta$-refinability' (438K), which enables us to do most of the work
without invoking any special axiom.

\leader{438A}{Measure-free cardinals:  Definition} A cardinal
$\kappa$ is {\bf measure-free} or {\bf of measure zero} if whenever
$\mu$ is a
probability measure with domain $\Cal P\kappa$ then there is a
$\xi<\kappa$ such that $\mu\{\xi\}>0$.
\cmmnt{In 363S I discussed some statements equiveridical with the
assertion `every cardinal is measure-free'.}%end of comment

\leader{438B}{}\cmmnt{ It is worth getting some basic facts out
into the open immediately.

\medskip

\noindent}{\bf Lemma} Let $(X,\Sigma,\mu)$ be a semi-finite measure
space and $\familyiI{E_i}$ a point-finite family of subsets of $X$ such
that $\#(I)$ is measure-free and
$\bigcup_{i\in J}E_i\in\Sigma$ for every $J\subseteq I$.   Set
$E=\bigcup_{i\in I}E_i$.

(a) $\mu E
=\sup_{J\subseteq I\text{ is finite}}\mu(\bigcup_{i\in J}E_i)$.

(b) If $\familyiI{E_i}$ is disjoint, then
$\mu E=\sum_{i\in I}\mu E_i$.   In particular, if $\Sigma=\Cal PX$ and
$A\subseteq X$ has measure-free cardinal, then
$\mu A=\sum_{x\in A}\mu\{x\}$.

(c) If $\mu$ is $\sigma$-finite, then $L=\{i:i\in I,\,\mu E_i>0\}$ is
countable and $\bigcup_{i\in I\setminus L}E_i$ is negligible.

\proof{{\bf (a)(i)} The first step is to show, by induction on $n$, that
the result is true if $\mu X<\infty$ and every $E_i$ is negligible and
$\#(\{i:i\in I,\,x\in E_i\})\le n$ for every $x\in X$.   If $n=0$ this
is trivial, since every $E_i$ must be empty.   For the inductive
step to $n\ge 1$, define $\nu:\Cal PI\to\coint{0,\infty}$ by setting
$\nu J=\mu(\bigcup_{i\in J}E_i)$ for every $J\subseteq I$.   Then $\nu$
is a measure on $I$.   \Prf\ Write $F_J=\bigcup_{i\in J}E_j$ for
$J\subseteq I$.  ($\alpha$) If $J$, $K\subseteq I$ are disjoint, then
for $i\in I$ set $E'_i=E_i\cap F_K$ for $i\in J$, $\emptyset$ for
$i\in I\setminus J$.   In this case,
$\family{i}{I}{E'_i}$ is a family of negligible subsets of $X$,
$\bigcup_{i\in J'}E'_i=F_{J'\cap J}\cap F_K$ is measurable for every
$J'\subseteq I$, and $\#(\{i:x\in E'_i\})\le n-1$ for every $x\in X$;
so the inductive hypothesis tells us that

\Centerline{$\mu(\bigcup_{i\in I}E'_i)
=\sup_{J'\subseteq I\text{ is finite}}\mu(\bigcup_{i\in J'}\mu E'_i)
=0$,}

\noindent that is, $F_J\cap F_K$ is negligible.   But this means that

\Centerline{$\nu(J\cup K)=\mu F_{J\cup K}=\mu(F_J\cup F_K)
=\mu F_J+\mu F_K=\nu J+\nu K$.}

\noindent As $J$ and $K$ are arbitrary, $\nu$ is additive.   ($\beta$)
If $\sequencen{J_n}$ is a disjoint sequence in $\Cal PI$, then

$$\eqalign{\nu(\bigcup_{n\in\Bbb N}J_n)
&=\mu(\bigcup_{n\in\Bbb N}F_{J_n})
=\lim_{n\to\infty}\mu(\bigcup_{m\le n}F_{J_m})\cr
&=\lim_{n\to\infty}\sum_{m=0}^n\nu J_m
=\sum_{n=0}^{\infty}\nu J_n,\cr}$$

\noindent so $\nu$ is countably additive and is a measure.\ \Qed

At the same time, $\nu\{i\}=\mu E_i=0$ for every $i$.   Because $\#(I)$
is measure-free, $\nu I=0$.   \Prf\Quer\ Otherwise, let
$f:I\to\kappa=\#(I)$ be any bijection and set
$\lambda A=\Bover1{\nu I}\nu f^{-1}[A]$ for every $A\subseteq\kappa$;
then $\lambda$ is a
probability measure with domain $\Cal P\kappa$ which is zero on
singletons, and $\kappa$ is not measure-free.\ \Bang\QeD\   But this
means just that $\mu(\bigcup_{i\in I}E_i)=0$.
Thus the induction proceeds.

\medskip

\quad{\bf (ii)} \Quer\ Now suppose, if possible, that the general result
is false.   For finite sets $J\subseteq I$ set
$F_J=\bigcup_{i\in J}E_i$, as before, and consider
$\Cal E=\{F_J:J\in[\kappa]^{<\omega}\}$
(see 3A1J for this notation).   Then $\Cal E$ is closed under finite unions
and $\gamma=\sup_{H\in\Cal E}\mu H$ is finite, because it is less than
$\mu E$;  let $\sequencen{H_n}$ be a non-decreasing sequence in
$\Cal E$ such that $\mu(H\setminus H^*)=0$ for every $H\in\Cal E$, where
$H^*=\bigcup_{n\in\Bbb N}H_n$ and $\mu H^*=\gamma$ (215Ab).

Because $\mu$ is semi-finite, there is an $F\in\Sigma$ such that
$F\subseteq E$ and $\gamma<\mu F<\infty$.   For each $n\in\Bbb N$, set

\Centerline{$Y_n=\{x:x\in F\setminus H^*,\,\#(\{i:x\in E_i\})\le n\}$.}

\noindent Then there is some $n\in\Bbb N$ such that $\mu^*Y_n>0$.   Let
$\nu$ be the subspace measure on $Y_n$, so that $\nu$ is non-zero and
totally finite.   Now
$\familyiI{E_i\cap Y_n}$ is a family of negligible subsets of $Y_n$,
$\bigcup_{i\in J}E_i\cap Y_n=Y_n\cap\bigcup_{i\in J}E_i$ is measured by
$\nu$ for every $J\subseteq I$, and $\#(\{i:x\in E_i\cap Y_n\})\le n$
for every $x\in Y_n$.   But this contradicts (i) above.\ \Bang

This proves (a).

\medskip

{\bf (b)} If $\familyiI{E_i}$ is disjoint, then

\Centerline{$\sup_{J\in[I]^{<\omega}}\mu(\bigcup_{i\in J}E_i)
=\sup_{J\in[I]^{<\omega}}\sum_{i\in J}\mu E_i
=\sum_{i\in I}\mu E_i$.}

\noindent Setting $I=A$, $E_x=\{x\}$ for $x\in A$, we get the special
case.

\medskip

{\bf (c)} Let
$\sequencen{X_n}$ be a non-decreasing sequence of measurable sets of
finite measure covering $X$.  For each $n$, set
$L_n=\{i:i\in I,\,\mu(E_i\cap X_n)\ge 2^{-n}\}$.   \Quer\ If $L_n$ is
infinite, take a
sequence $\sequence{k}{i_k}$ of distinct elements in $L_n$, and consider
$G_m=\bigcup_{k\ge m}E_{i_k}$ for $m\in\Bbb N$;  then every $G_m$ has
measure at least $2^{-n}$, $G_0$ has finite measure, $\sequence{m}{G_m}$
is non-increasing, and $\bigcap_{m\in\Bbb N}G_m$ is empty, because
$\sequence{k}{E_{i_k}}$ is point-finite.   But this is impossible.\
\Bang

Thus every $L_n$ is finite and $L=\bigcup_{n\in\Bbb N}L_n$ is
countable.

Now (a), applied to $\familyiI{E'_i}$ where $E'_i=E_i$ if
$i\in I\setminus L$, $\emptyset$ if $i\in L$, tells us that
$\bigcup_{i\in I\setminus L}E_i$ is negligible.
}%end of proof of 438B

\leader{438C}{}\cmmnt{ I do not think we are ready for the most
interesting set-theoretic results concerning measure-free cardinals.
But the following facts may help to make sense of the general pattern.

\medskip

\noindent}{\bf Theorem}\cmmnt{ ({\smc Ulam 30})} (a) $\omega$ is
measure-free.

(b) If $\kappa$ is a measure-free cardinal and $\kappa'\le\kappa$ is
a smaller cardinal, then $\kappa'$ is measure-free.

(c) If $\ofamily{\xi}{\lambda}{\kappa_{\xi}}$ is a family of
measure-free cardinals, and $\lambda$ also is measure-free, then
$\kappa=\sup_{\xi<\lambda}\kappa_{\xi}$ is measure-free.

(d) If $\kappa$ is a measure-free cardinal so is $\kappa^+$.

(e) The following are equiveridical:

\quad(i) $\frak c$ is not measure-free;

\quad(ii) there is a semi-finite measure space $(X,\Cal PX,\mu)$ which
is not purely atomic;

\quad(iii) there is a measure $\mu$ on $[0,1]$ extending Lebesgue
measure and measuring every subset of $[0,1]$.

(f) If $\kappa\ge\frak c$ is a measure-free cardinal then
$2^{\kappa}$ is measure-free.

\proof{{\bf (a)} This is trivial.

\medskip

{\bf (b)} If $\mu$ is a probability measure with domain $\Cal P\kappa'$,
set $\nu A=\mu(\kappa'\cap A)$ for every $A\subseteq\kappa$.   Then
$\nu$ is a probability measure with domain $\Cal P\kappa$, so there is a
$\xi<\kappa$ such that $\nu\{\xi\}>0$;  evidently $\xi<\kappa'$ and
$\mu\{\xi\}>0$.

\medskip

{\bf (c)} Let $\mu$ be a probability measure on $\kappa$ with domain
$\Cal P\kappa$.   Define $f:\kappa\to\lambda$ by setting
$f(\alpha)=\min\{\xi:\alpha<\kappa_{\xi}\}$ for $\alpha<\kappa$.
Then the image measure $\mu f^{-1}$ is a probability measure on
$\lambda$ with domain $\Cal P\lambda$, so there is a $\xi<\lambda$ such
that $\mu f^{-1}[\{\xi\}]>0$.   Now $\mu\kappa_{\xi}>0$.   Applying
438Bb to $A=\kappa_{\xi}$, we see that there is an
$\alpha<\kappa_{\xi}$ such that $\mu\{\alpha\}>0$.   As $\mu$ is
arbitrary, $\kappa$ is measure-free.

\medskip

{\bf (d)} By (a) and (b), we need consider only the case
$\kappa\ge\omega$.   \Quer\
Suppose, if possible, that $\mu$ is a probability measure with domain
$\Cal P\kappa^+$ such that $\mu\{\alpha\}=0$ for every
$\alpha<\kappa^+$.   For each $\alpha<\kappa^+$, choose an injection
$f_{\alpha}:\alpha\to\kappa$.   For $\beta<\kappa^+$,
$\xi<\kappa$ set
$A(\beta,\xi)=\{\alpha:\beta<\alpha<\kappa^+,\,f_{\alpha}(\beta)=\xi\}$.
Then $\kappa^+\setminus\bigcup_{\xi<\kappa}A(\beta,\xi)=\beta+1$ has
cardinal at most $\kappa$, which is measure-free, so $\mu(\beta+1)=0$
and $\mu(\bigcup_{\xi<\kappa}A(\beta,\xi))>0$.   Also
$\ofamily{\xi}{\kappa}{A(\beta,\xi)}$ is disjoint.   There is therefore
a $\xi_{\beta}<\kappa$ such that $\mu A(\beta,\xi_{\beta})>0$, by
438Bb.   Now $\kappa^+>\max(\omega,\kappa)$, so there must be an
$\eta<\kappa$ such that $B=\{\beta:\xi_{\beta}=\eta\}$ is uncountable.
In this case, however, $\family{\beta}{B}{A(\beta,\eta)}$ is an
uncountable family of sets of measure greater than zero, and cannot be
disjoint, because $\mu$ is totally finite (215B(iii));  but if
$\alpha\in A(\beta,\eta)\cap A(\beta',\eta)$, where $\beta\ne\beta'$,
then $f_{\alpha}(\beta)=f_{\alpha}(\beta')=\eta$, which is impossible,
because $f_{\alpha}$ is supposed to be injective.\ \Bang

So there is no such measure $\mu$, and $\kappa^+$ is measure-free.

\medskip

{\bf (e)(i)$\Rightarrow$(ii)} Suppose that $\frak c$ is not
measure-free;  let $\mu$ be a probability measure with domain
$\Cal P\frak c$
such that $\mu\{\xi\}=0$ for every $\xi<\frak c$.   Then $\mu$ is
atomless.   \Prf\Quer\ Suppose, if possible, that $A\subseteq\frak c$ is
an atom for $\mu$.   Let $f:\frak c\to\Cal P\Bbb N$ be a bijection.
For each $n\in\Bbb N$, set $E_n=\{\xi:n\in f(\xi)\}$.   Set
$D=\{n:\mu(A\cap E_n)=\mu A\}$.   Because $A$ is an atom,
$\mu(A\cap E_n)=0$ for every $n\in\Bbb N\setminus D$.   This means
that $B=\bigcap_{n\in D}E_n\setminus\bigcup_{n\in\Bbb N\setminus D}E_n$
has measure $\mu A>0$;  but $f(\xi)=D$ for every $\xi\in B$, so
$\#(B)\le 1$, and $\mu\{\xi\}>0$ for some $\xi$, contrary to
hypothesis.\ \Bang\Qed

So (ii) is true.

\medskip

\quad{\bf (ii)$\Rightarrow$(iii)} Suppose that there is a semi-finite
measure space $(X,\Cal PX,\mu)$ which is not purely atomic.   Then there
is a non-negligible set $E\subseteq X$ which does not include any atom;
let $F\subseteq E$ be a set of non-zero finite measure.   If we take
$\nu$ to be $\Bover1{\mu F}\mu_F$, where $\mu_F$ is the subspace measure
on $F$, then $\nu$ is an atomless probability measure with domain
$\Cal PF$.
Consequently there is a function $g:F\to[0,1]$ which is \imp\ for $\nu$
and Lebesgue measure (343Cb).   But this means that the image measure
$\nu g^{-1}$ is a measure defined on every subset of $[0,1]$ which
extends Lebesgue measure.

\medskip

\quad{\bf not-(i)$\Rightarrow$not-(iii)} Conversely, if $\frak c$ is
measure-free, then any probability measure on $[0,1]$ measuring every
subset must give positive measure to some singleton, and cannot extend
Lebesgue measure.

\medskip

{\bf (f)} We are supposing that $\kappa\ge\frak c$ is measure-free,
so, in particular, $\frak c$ is measure-free.
Let $\mu$ be a probability measure with domain
$\Cal P(2^{\kappa})$.   By (e), it cannot be atomless;  let
$E\subseteq 2^{\kappa}$ be an atom.   Let $f:2^{\kappa}\to\Cal P\kappa$
be a bijection, and for $\xi<\kappa$ set
$E_{\xi}=\{\alpha:\alpha<2^{\kappa},\,\xi\in f(\alpha)\}$;  set
$D=\{\xi:\xi<\kappa,\,\mu(E\cap E_{\xi})=\mu E\}$.   Note that
$\mu(E\cap E_{\xi})$ must be zero for every $\xi\in\kappa\setminus D$,
so that $E\cap\{\alpha:\xi\in D\symmdiff f(\alpha)\}$ is always
negligible.   Consider

\Centerline{$A_{\xi}
=\{\alpha:\alpha\in E,\,\xi=\min(D\symmdiff f(\alpha))\}$}

\noindent for $\xi<\kappa$.   Then $\ofamily{\xi}{\kappa}{A_{\xi}}$ is a
disjoint family of negligible sets, so its union $A$ is negligible, by
438Bb, because $\kappa$ is measure-free.   But
$E\setminus A\subseteq f^{-1}[\{D\}]$ has at most one element, and is
not negligible;  so $\mu\{\alpha\}>0$ for some $\alpha$.   As $\mu$ is
arbitrary, $2^{\kappa}$ is measure-free.
}%end of proof of 438C

\cmmnt{\medskip

\noindent{\bf Remark} This extends the result of 419G, which used a
different approach to show that $\omega_1$ is measure-free.

We see from (d) above that $\omega_2$, $\omega_3,\ldots$ are all
measure-free;  so, by (c), $\omega_{\omega}$ also is;  generally, if
$\kappa$ is any measure-free cardinal, so is $\omega_{\kappa}$
(438Xa).   I ought to point out that there are more powerful
arguments showing that any cardinal which is not measure-free must be
enormous (see 541L in Volume 5).   In this context,
however, $\frak c=2^{\omega}$ can be `large', at least in the absence of
an axiom like the continuum hypothesis to locate it in the hierarchy
$\family{\xi}{\On}{\omega_{\xi}}$;  it is generally believed that it is
consistent to suppose that $\frak c$ is not measure-free.
}%end of comment

\leader{438D}{}\cmmnt{ I turn now to the contexts in which
measure-free cardinals behave as if they were `small'.

\medskip

\noindent}{\bf Proposition} Let $(X,\Sigma,\mu)$ be a $\sigma$-finite
measure space, $Y$ a metrizable space with measure-free weight, and
$f:X\to Y$ a measurable function.   Then there is a closed separable set
$Y_0\subseteq Y$ such that
$f^{-1}[Y_0]$ is conegligible;  that is, there is a conegligible
measurable set $X_0\subseteq X$ such that $f[X_0]$ is separable.

\proof{ Let $\Cal U$ be a $\sigma$-disjoint base for the topology of $Y$
(4A2L(g-ii));  express it as $\bigcup_{n\in\Bbb N}\Cal U_n$ where each
$\Cal U_n$ is a disjoint family of open sets.   If $n\in\Bbb N$,
$\#(\Cal U_n)\le w(Y)$ (4A2Db) is a measure-free cardinal (438Cb),
and $\family{U}{U_n}{f^{-1}[U]}$ is a disjoint family in $\Sigma$ such
that $\bigcup_{u\in\Cal V}f^{-1}[U]=f^{-1}[\bigcup\Cal V]$ is measurable
for every $\Cal V\subseteq\Cal U_n$;  so 438Bc tells us that there is a
countable set $\Cal V_n\subseteq\Cal U_n$ such that

\Centerline{$f^{-1}[\bigcup(\Cal U_n\setminus\Cal V_n)]
=\bigcup_{U\in\Cal U_n\setminus\Cal V_n}f^{-1}[U]$}

\noindent is negligible.    Set

\Centerline{$Y_0
=Y\setminus\bigcup_{n\in\Bbb N}\bigcup(\Cal U_n\setminus\Cal V_n)$.}

\noindent Then $f^{-1}[Y\setminus Y_0]
=\bigcup_{n\in\Bbb N}f^{-1}[\bigcup(\Cal U_n\setminus\Cal V_n)]$ is
negligible.   On the other hand,

\Centerline{$\{U\cap Y_0:U\in\Cal U\}
\subseteq\{\emptyset\}
  \cup\{V\cap Y_0:V\in\bigcup_{n\in\Bbb N}\Cal V_n\}$}

\noindent is countable, and is a base for the subspace topology of $Y_0$
(4A2B(a-vi));  so $Y_0$ is second-countable and must be separable
(4A2Oc).

Thus we have an appropriate $Y_0$.   Now $X_0=f^{-1}[Y_0]$ is
conegligible and measurable and $f[X_0]\subseteq Y_0$ is separable
(4A2P(a-iv)).
}%end of proof of 438D

\vleader{36pt}{438E}{Proposition}\cmmnt{ (cf.\ 418B)}
Let $(X,\Sigma,\mu)$ be a complete locally determined measure space.

(a) If $Y$ is a topological space, $Z$ is a
metrizable space, $w(Z)$ is measure-free, and $f:X\to Y$, $g:X\to Z$
are measurable functions, then $x\mapsto (f(x),g(x)):X\to Y\times Z$ is
measurable.

(b) If $\sequencen{Y_n}$ is a sequence of metrizable
spaces, with product $Y$, $w(Y_n)$ is measure-free for every
$n\in\Bbb N$, and $f_n:X\to Y_n$ is measurable for every $n\in\Bbb N$,
then $x\mapsto f(x)=\sequencen{f_n(x)}:X\to\prod_{n\in\Bbb N}Y_n$ is
measurable.

\proof{{\bf (a)(i)} Consider first the case in which $\mu$ is totally
finite.   Then there is a conegligible set $X_0\subseteq X$ such that
$g[X_0]$ is separable (438D).   Applying 418Bb to $f\restr X_0$ and
$g\restr X_0$, we see that $x\mapsto(f(x),g(x)):X_0\to Y\times g[Z_0]$
is measurable.   As $\mu$ is complete, it follows that
$x\mapsto(f(x),g(x)):X\to Y\times Z$ is measurable.

\medskip

\quad{\bf (ii)} In the general case, take any open set
$W\subseteq Y\times Z$ and any measurable set $F\subseteq X$ of finite
measure.
Set $Q=\{x:(f(x),g(x))\in W\}$.   By (i), applied to $f\restr F$ and
$g\restr F$, $F\cap Q\in\Sigma$;  as $F$ is arbitrary and $\mu$ is
locally determined, $Q\in\Sigma$;  as $W$ is arbitrary,
$x\mapsto(f(x),g(x))$ is measurable.

\medskip

{\bf (b)} As in (a), it is enough to consider the case in which $\mu$ is
totally finite.    In this case, we have for each $n\in\Bbb N$ a
conegligible set $X_n$ such that $f_n[X_n]$ is separable.   Set
$X'=\bigcap_{n\in\Bbb N}X_n$;  then 418Bd tells us that $f\restr X'$ is
measurable, so that $f$ is measurable.
}%end of proof of 438E

\leader{438F}{Proposition}\cmmnt{ (cf.\ 418J)} Let $(X,\Sigma,\mu)$ be
a semi-finite measure
space and $\frak T$ a topology on $X$ such that $\mu$ is inner regular
with respect to the closed sets.   Suppose that $Y$ is a
metrizable space, $w(Y)$ is measure-free and $f:X\to Y$ is
measurable.   Then $f$ is almost continuous.

\proof{ Take $E\in\Sigma$ and $\gamma<\mu E$.   Then there is a
measurable set $F\subseteq E$ such that $\gamma<\mu F<\infty$.   Let
$F_0\subseteq F$ be a measurable set such that $F\setminus F_0$ is
negligible and
$f[F_0]$ is separable (438D).   By 412Pc, the subspace measure on $F_0$
is still inner regular with respect to the closed sets, so $f\restr F_0$
is almost continuous (418J), and there is a measurable set
$H\subseteq F_0$, of measure at least $\gamma$, such that $f\restr H$ is
continuous.   As $E$ and $\gamma$ are arbitrary, $f$ is almost
continuous.
}%end of proof of 438F

\leader{438G}{Corollary}\cmmnt{ (cf.\ 418K)} Let
$(X,\frak T,\Sigma,\mu)$ be a quasi-Radon
measure space and $Y$ a metrizable space such that $w(Y)$ is
measure-free.   Then a function
$f:X\to Y$ is measurable iff it is almost continuous.

\leaveitout{438F}

\leader{438H}{}\cmmnt{ Now let us turn to questions which arose in
\S434.

\medskip

\noindent}{\bf Proposition} A complete metric space is Radon iff its
weight is measure-free.

\proof{ Let $(X,\rho)$ be a complete metric space, and $\kappa=w(X)$ its
weight.

\medskip

{\bf (a)} If $\kappa$ is measure-free, let $\mu$ be any totally
finite Borel measure on $X$.   Applying 438D to the identity map from
$X$ to itself, we see that there is a closed separable conegligible
subspace $X_0$.   Now $X_0$ is complete, so is a Polish space, and by
434Kb is a Radon space.   The subspace measure $\mu_{X_0}$ is
therefore tight (that is, inner regular with respect to the compact sets);  as $X_0$ is
conegligible, it follows at once that $\mu$ also is.   As $\mu$ is
arbitrary, $X$ is a Radon space.

\medskip

{\bf (b)} If $\kappa$ is not measure-free, take any $\sigma$-disjoint
base $\Cal U$ for the topology of $X$.   Express $\Cal U$ as
$\bigcup_{n\in\Bbb N}\Cal U_n$ where every $\Cal U_n$ is disjoint.
Then $\kappa\le\#(\Cal U)$
and there is a probability measure $\nu$ on $\Cal U$, with domain
$\Cal P\Cal U$, such that $\nu\{U\}=0$ for every $U\in\Cal U$.
Let $n\in\Bbb N$ be such that $\nu\Cal U_n>0$.
For each $U\in\Cal U_n$ choose
$x_U\in U$. For Borel sets $E\subseteq X$ set
$\mu E=\nu\{U:U\in\Cal U_n,\,x_U\in E\}$;
then $\mu$ is a Borel measure on $X$ and
$\mu(\bigcup\Cal U_n)=\nu\Cal U_n>0$, while
$\mu(\bigcup\Cal V)=\nu\Cal V=0$ for every finite
$\Cal V\subseteq\Cal U_n$.
Thus $\mu$ is not $\tau$-additive and cannot be tight,
and $X$ is not a Radon space.
}%end of proof of 438H

\leader{438I}{Proposition} Let $X$ be a metrizable space and
$\ofamily{\xi}{\kappa}{F_{\xi}}$ a non-decreasing family of closed
subsets of $X$, where $\kappa$ is a measure-free cardinal.   Then

\Centerline{$\mu(\bigcup_{\xi<\kappa}F_{\xi})
=\sum_{\xi<\kappa}\mu(F_{\xi}\setminus\bigcup_{\eta<\xi}F_{\eta})$}

\noindent for every semi-finite Borel measure $\mu$ on $X$.

\proof{{\bf (a)} I had better begin by remarking that
$H_{\xi}=\bigcup_{\eta<\xi}F_{\eta}$ is an F$_{\sigma}$ set for every
ordinal $\xi\le\kappa$, by 4A2Ld and 4A2Ka.   So, setting
$E_{\xi}=F_{\xi}\setminus H_{\xi}$, $\sum_{\xi<\kappa}\mu E_{\xi}$ is
defined.

\medskip

{\bf (b)} I show by induction on $\zeta$ that
$\mu H_{\zeta}=\sum_{\xi<\zeta}\mu E_{\xi}$ for every $\zeta\le\kappa$.
The induction starts trivially with $\mu H_0=0$.   The inductive step to
a successor ordinal $\zeta+1$ is also immediate, as
$H_{\zeta+1}=H_{\zeta}\cup E_{\zeta}$.   For the inductive step to a
limit ordinal $\zeta$ of countable cofinality, let $\sequencen{\zeta_n}$
be a non-decreasing sequence in $\zeta$ with supremum $\zeta$;  then

\Centerline{$\mu H_{\zeta}
=\sup_{n\in\Bbb N}\mu H_{\zeta_n}
=\sup_{n\in\Bbb N}\sum_{\xi<\zeta_n}\mu E_{\xi}
=\sum_{\xi<\zeta}\mu E_{\xi}$,}

\noindent as required.

\medskip

{\bf (c)} So we are left with the case in which $\zeta$ is a limit
ordinal of uncountable cofinality.   In this case,
$\mu(E\cap H_{\zeta})\le\sum_{\xi<\zeta}\mu E_{\xi}$ whenever $\mu E$ is
finite.   \Prf\ Let $\Cal U$ be a $\sigma$-disjoint base for the
topology of $X$ (4A2L(g-ii)), and express $\Cal U$ as
$\bigcup_{n\in\Bbb N}\Cal U_n$ where each $\Cal U_n$ is disjoint.   For
$n\in\Bbb N$, $\xi\le\zeta$ set

\Centerline{$\Cal V_{n\xi}
=\{U:U\in\Cal U_n,\,U\cap H_{\xi}\ne\emptyset\}$,
\quad$V_{n\xi}=\bigcup\Cal V_{n\xi}$.}

\noindent Define $\phi_n:V_{n\zeta}\to\zeta$ by saying that
$\phi_n(x)=\min(\{\xi:\xi<\zeta,\,x\in V_{n\xi}\})$.   Then, for any
$D\subseteq\zeta$,

\Centerline{$\phi_n^{-1}[D]
=\bigcup_{\xi\in D}
  \bigcup(\Cal V_{n\xi}\setminus\bigcup_{\eta<\xi}\Cal V_{n\eta})$}

\noindent is a union of members of $\Cal U_n$, so is open.   We
therefore have a measure $\nu_n$ on $\Cal P\zeta$ defined by saying that
$\nu_nD=\mu(E\cap\phi_n^{-1}[D])$ for every $D\subseteq\zeta$.   At this
point, recall that we are supposing that $\kappa$ is measure-free, so
$\#(\zeta)$ also is measure-free (438Cb) and
$\nu_n\zeta=\sum_{\xi<\zeta}\nu_n\{\xi\}=\sup_{\xi<\zeta}\nu_n\xi$
(438Bb).   Interpreting this in $X$, we have
$\mu(E\cap V_{n\zeta})=\sup_{\xi<\zeta}\mu(E\cap V_{n\xi})$.

This is true for every $n\in\Bbb N$.   So there is a countable set
$C\subseteq\zeta$ such that
$\mu(E\cap V_{n\zeta})=\sup_{\xi\in C}\mu(E\cap V_{n\xi})$ for every
$n\in\Bbb N$.   Because $\cf\zeta>\omega$, there is an $\alpha<\zeta$
such that $C\subseteq\alpha$, and
$\mu(E\cap V_{n\zeta})=\mu(E\cap V_{n\alpha})$, that is,
$E\cap V_{n\zeta}\setminus V_{n\alpha}$ is
negligible, for every $n\in\Bbb N$.

Now note that $F_{\alpha}$ is closed.   So

$$\eqalign{H_{\zeta}\setminus F_{\alpha}
&\subseteq\bigcup\{U:U\in\Cal U,\,H_{\zeta}\cap U\ne\emptyset,\,
   U\cap F_{\alpha}=\emptyset\}\cr
&=\bigcup_{n\in\Bbb N}V_{n\zeta}\setminus V_{n,\alpha+1},\cr}$$

\noindent and $E\cap H_{\zeta}\setminus F_{\alpha}$ is negligible.
Accordingly, using the inductive hypothesis,

\Centerline{$\mu(E\cap H_{\zeta})
\le\mu F_{\alpha}
=\mu H_{\alpha+1}\le\sum_{\xi\le\alpha}\mu E_{\xi}
\le\sum_{\xi<\zeta}\mu E_{\xi}$,}

\noindent as claimed.\ \Qed

Because $\mu$ is semi-finite, and $E$ is arbitrary,
$\mu H_{\zeta}\le\sum_{\xi<\zeta}E_{\xi}$;  but the reverse inequality
is trivial, so we have equality, and the induction proceeds in this case
also.

\medskip

{\bf (d)} At the end of the induction we have
$\mu H_{\kappa}=\sum_{\xi<\kappa}\mu E_{\xi}$, as stated.
}%end of proof of 438I

\leader{438J}{}\cmmnt{ So far we have been looking at metrizable
spaces, the obvious first step.   But it turns out that the concept of
`metacompactness' leads to generalizations of some of the results above.

\medskip

\noindent}{\bf Proposition}\cmmnt{ ({\smc Moran 70}, {\smc Haydon 74})}
Let $X$ be a metacompact space with measure-free weight.

(a) $X$ is Borel-measure-compact.

(b) If $X$ is normal, it is measure-compact.

(c) If $X$ is perfectly normal\cmmnt{ (for instance, if it is
metrizable)}, it is Borel-measure-complete.

\proof{{\bf (a)} \Quer\ If $X$ is not Borel-measure-compact, there are
a non-zero totally finite Borel measure $\mu$ on $X$ and a cover
$\Cal G$ of $X$ by negligible open sets (434H(a-v)).   Let $\Cal H$ be
a point-finite open cover of $X$ refining $\Cal G$.   By 4A2Dc,
$\#(\Cal H)$ is at most $\max(\omega,w(X))$, so is measure-free, by 438C.
Because $\mu$ is a Borel measure,
$\bigcup\Cal H'$ is
measurable for every $\Cal H'\subseteq\Cal H$;  $\mu H=0$ for every
$H\in\Cal H$;  while $\mu(\bigcup\Cal H)=\mu X>0$.   But this
contradicts 438Ba.\ \Bang

\medskip

{\bf (b)} Now suppose that $X$ is normal, and that $\mu$ is a totally
finite Baire measure on $X$.   Because a
normal metacompact space is countably paracompact (4A2F(g-iii)), $\mu$
has an extension to a Borel measure $\mu_1$ which is inner regular with
respect to the closed sets, by \Marik's theorem (435C).   Now $\mu_1$ is
$\tau$-additive, by (a) above, so $\mu$ also is (411C).   As $\mu$ is
arbitrary, $X$ is measure-compact.

\medskip

{\bf (c)} Since on a perfectly normal space the Baire and Borel measures are the same, $X$ is Borel-measure-complete iff it is measure-compact, and we can use (b).
}%end of proof of 438J

\cmmnt{\medskip

\noindent{\bf Remark} The arguments here can be adapted in various
ways, and in particular the hypotheses can be weakened;  see
438Yd-438Yf. %438Yd 438Ye 438Yf
}%end of comment

\leader{438K}{Hereditarily weakly $\theta$-refinable spaces} A
topological space $X$ is {\bf \hwtr}\cmmnt{ (also called {\bf
hereditarily $\sigma$-relatively metacompact},
{\bf hereditarily weakly submetacompact})} if for every family $\Cal G$
of open subsets of $X$ there is a
$\sigma$-isolated family $\Cal A$ of subsets of $X$, refining
$\Cal G$, such that $\bigcup\Cal A=\bigcup\Cal G$.

\vleader{48pt}{438L}{Lemma} (a) Any subspace of a \hwtr\ topological
space is \hwtr.

(b) A hereditarily metacompact
space\cmmnt{ (e.g., any metrizable space, see 4A2Lb)} is \hwtr.

(c) A hereditarily Lindel\"of space is \hwtr.

(d) A topological space with a $\sigma$-isolated network is
\hwtr.
%in particular, any space with a development is \hwtr

\proof{{\bf (a)} If $X$ is \hwtr, $Y$ is a subspace of $X$, and $\Cal H$
is a family of open subsets of $Y$, set
$\Cal G=\{G:G\subseteq X$ is open, $G\cap Y\in\Cal H\}$.   Then there is
a $\sigma$-isolated family $\Cal A$, refining $\Cal G$, with union
$\bigcup\Cal G$;  and $\{A\cap Y:A\in\Cal A\}$ is $\sigma$-isolated
(4A2B(a-viii)), refines $\Cal H$, and has union $\bigcup\Cal H$.   As
$\Cal H$ is arbitrary, $Y$ is \hwtr.

\medskip

{\bf (b)} If $X$ is hereditarily metacompact, and $\Cal G$ is any
family of open sets in $X$ with union $W$, then $\Cal G$ is an open
cover of the metacompact space $W$, so has a point-finite open
refinement $\Cal H$ with the same union.   For each $x\in W$, set
$\Cal H_x=\{H:x\in H\in\Cal H\}$, $V_x=\bigcap\Cal H_x$, so that
$\Cal H_x$ is
a non-empty finite set and $V_x$ is an open set containing $x$.   For
$n\ge 1$, set
$A_n=\{x:x\in W,\,\#(\Cal H_x)=n\}$;  then for any distinct $x$,
$y\in A_n$, either $\Cal H_x=\Cal H_y$ and $V_x=V_y$, or
$\#(\Cal H_x\cup\Cal H_y)>n$ and $V_x\cap V_y\cap A_n=\emptyset$.   This
means that $\Cal A_n=\{V_x\cap A_n:x\in A_n\}$ is a partition of
$A_n$ into relatively open sets, and is an isolated family.   Also,
$\Cal A_n$ is a refinement of $\Cal H$ and therefore of $\Cal G$;  so
$\bigcup_{n\ge 1}\Cal A_n$ is a $\sigma$-isolated refinement of
$\Cal G$, and its union is $\bigcup_{n\ge 1}A_n=W$.   As $\Cal G$ is
arbitrary, $X$ is \hwtr.

\medskip

{\bf (c)} If $X$ is hereditarily Lindel\"of and $\Cal G$ is a
family of open subsets of $X$, there is a countable
$\Cal G_0\subseteq\Cal G$ with the same union;  now $\Cal G_0$, being
countable, is $\sigma$-isolated.   As $\Cal G$ is arbitrary, $X$
is \hwtr.

\medskip

{\bf (d)} If $X$ has a $\sigma$-isolated network $\Cal A$, and
$\Cal G$ is any family of open subsets of $X$, then

\Centerline{$\Cal E=\{A:A\in\Cal A$, $A$ is included in some member of
$\Cal G\}$}

\noindent is a $\sigma$-isolated family (4A2B(a-viii) again), refining
$\Cal G$, with union $\bigcup\Cal G$.
}%end of proof of 438L

\leader{438M}{Proposition}\cmmnt{ ({\smc Gardner 75})}
If $X$ is a \hwtr\ topological space with measure-free weight, it is
Borel-measure-complete.

%Gardner & Pfeffer 10.2;  Gardner 75

\proof{ Let $\mu$ be a Borel probability measure on $X$, and $\Cal G$
the family of $\mu$-negligible open sets.   Let $\Cal A$ be a
$\sigma$-isolated family refining $\Cal G$ with union $\bigcup\Cal G$.
Express $\Cal A$ as $\bigcup_{n\in\Bbb N}\Cal A_n$ where each $\Cal A_n$
is an isolated family;  for each $n\in\Bbb N$, set
$X_n=\bigcup\Cal A_n$ and let
$\mu_n$ be the subspace measure on $X_n$.   Then $\Cal A_n$ is a
disjoint family of relatively open $\mu_n$-negligible sets;  as
$\#(\Cal A_n)\le w(X_n)\le w(X)$ (4A2D) is measure-free, and $\mu_n$
is a totally finite Borel measure on $X_n$,

\Centerline{$\mu^*X_n=\mu_nX_n=\mu_n(\bigcup\Cal A_n)=0$,}

\noindent by 438Bb.   Now
$\mu(\bigcup\Cal G)=\mu^*(\bigcup_{n\in\Bbb N}X_n)=0$.   As $\mu$ is
arbitrary, $X$ is Borel-measure-complete (434I(a-iv)).
}%end of proof of 438M

%[is a normal weakly $\theta$-refinable space countably paracompact?]

\leader{438N}{}\cmmnt{ For the next few paragraphs, I will use the
following notation.}   Let
$X$ be a topological space and $\Cal G$ a family of subsets of $X$.
Then $\Cal J(\Cal G)$ will be the family of subsets of $X$ expressible
as $\bigcup\Cal A$ for some $\sigma$-isolated family $\Cal A$
refining $\Cal G$.   \cmmnt{Observe that} $X$ is \hwtr\ iff
$\bigcup\Cal G$
belongs to $\Cal J(\Cal G)$ for every family $\Cal G$ of open subsets of
$X$.

\spheader 438Na $\Cal J(\Cal G)$ is always a $\sigma$-ideal of subsets
of $X$.
\prooflet{\Prf\ If $\Cal A$ is a $\sigma$-isolated family of
subsets of $X$, refining $\Cal G$, and $B$ is any set, then
$\{B\cap A:A\in\Cal A\}$ is still
$\sigma$-isolated and still refines $\Cal G$;  so any subset of a
member of $\Cal J(\Cal G)$ belongs to $\Cal J(\Cal G)$.   If
$\sequencen{\Cal A_n}$ is a sequence of $\sigma$-isolated
families refining $\Cal G$, then $\bigcup_{n\in\Bbb N}\Cal A_n$ is
$\sigma$-isolated and refines $\Cal G$;  so the union of
any sequence in $\Cal J(\Cal G)$ belongs to $\Cal J(\Cal G)$.\ \Qed}

\spheader 438Nb If $\Cal H$ refines $\Cal G$, then
$\Cal J(\Cal H)\subseteq\Cal J(\Cal G)$.  \prooflet{\Prf\ All we need to
remember is that any family refining $\Cal H$ also refines $\Cal G$.\
\Qed}

\spheader 438Nc If $X$ and $Y$ are topological spaces, $A\subseteq X$,
$f:A\to Y$ is
continuous, and $\Cal H$ is a family of subsets of $Y$, set
$\Cal G=\{f^{-1}[H]:H\in\Cal H\}$.   Then
$\Cal J(\Cal G)\supseteq\{f^{-1}[B]:B\in\Cal J(\Cal H)\}$.
\prooflet{\Prf\ If $B\in\Cal J(\Cal H)$, there is a $\sigma$-isolated
family $\Cal D$ of subsets of $Y$, refining $\Cal H$, and with union
$B$.   Now
$\Cal A=\{f^{-1}[D]:D\in\Cal D\}$ refines $\Cal G$ and has union
$f^{-1}[B]$.   We can express $\Cal D$ as
$\bigcup_{n\in\Bbb N}\Cal D_n$, where each $\Cal D_n$ is an isolated
family;  set
$\Cal A_n=\{f^{-1}[D]:D\in\Cal D_n\}$, so that
$\Cal A=\bigcup_{n\in\Bbb N}\Cal A_n$.   For each $n$, $\Cal A_n$ is
disjoint, because $\Cal D_n$ is.   Moreover, if $D\in\Cal D_n$, then
$D=H\cap\bigcup\Cal D_n$ for some open set $H\subseteq Y$, so that
$f^{-1}[D]=f^{-1}[H]\cap\bigcup\Cal A_n$ is relatively open in
$\bigcup\Cal A_n$;  this shows that $\Cal A_n$ is an isolated
family.   Accordingly $\Cal A$ is $\sigma$-isolated and witnesses
that $f^{-1}[B]\in\Cal J(\Cal G)$.   As $B$ is arbitrary, we have the
result.\ \Qed}

\spheader 438Nd If $X$ is a topological space, $\Cal G$ is a family of
subsets of $X$,
and $\familyiI{D_i}$ is an isolated family in $\Cal J(\Cal G)$,
then $\bigcup_{i\in I}D_i\in\Cal J(\Cal G)$.   \prooflet{\Prf\ For each
$i\in I$, let $\sequencen{\Cal A_{ni}}$ be a sequence of isolated
families, all refining $\Cal G$, such that
$D_i=\bigcup_{n\in\Bbb N}\bigcup\Cal A_{in}$.   Set
$\Cal A_n=\bigcup_{i\in I}\Cal A_{in}$ for each $n$.   Then $\Cal A_n$
refines $\Cal G$, and
$\bigcup_{n\in\Bbb N}\bigcup\Cal A_n=\bigcup_{i\in I}D_i$.   It is easy
to check that every $\Cal A_n$ is isolated, so that
$\bigcup_{n\in\Bbb N}\Cal A_n$ witnesses that $\bigcup_{i\in I}D_i$
belongs to $\Cal J(\Cal G)$.\ \Qed}

\leader{438O}{Lemma} Give
$\Bbb R$ the topology $\frak S$ generated by the closed intervals
$\ocint{-\infty,t}$ for $t\in\Bbb R$, and let $r\ge 1$.   Then
$\BbbR^r$, with the product topology corresponding to $\frak S$, is
\hwtr.

\proof{ Induce on $r$.   Write $\le$ for the usual partial order of
$\BbbR^r$, and $\ocint{-\infty,x}$ for $\{y:y\le x\}$;  set
$V_A=\bigcup_{x\in A}\ocint{-\infty,x}$ for $A\subseteq\BbbR^r$.   The
sets $\ocint{-\infty,x}$, as $x$ runs over $\BbbR^r$, form a base for
the topology of $\BbbR^r$.

The induction starts easily because $\frak S$ itself is hereditarily
Lindel\"of.   \Prf\ If $\Cal G\subseteq\frak S$, set

\Centerline{$A
=\{x:x\in\Bbb R$, there is some $G\in\Cal G$ such that
$\ocint{-\infty,x}\subseteq G\}$.}

\noindent Then $A$ has a
countable cofinal set $D$, so that there is a corresponding countable
subset of $\Cal G$ with the same union as $\Cal G$.\ \QeD\  By
438Lc, $\frak S$ is \hwtr.

For the inductive step to $r+1$, where $r\ge 1$, let $\Cal G$ be a
family of open subsets of $\BbbR^{r+1}$, and set

\Centerline{$A
=\{x:x\in\BbbR^r$, there is some $G\in\Cal G$ such that
$\ocint{-\infty,x}\subseteq G\}$.}

\noindent For each $k\le r$, $q\in\Bbb Q$ set
$K_k=(r+1)\setminus\{k\}$ and let $B_{kq}\subseteq\BbbR^{K_k}$ be the
set $\{z:z^{\smallfrown}\fraction{q}\in V_A\}$,
writing $z^{\smallfrown}\fraction{q}$ for that
member $x$ of $\BbbR^{r+1}$ such that $x\restr K_k=z$ and $x(k)=q$.   (I
am thinking of members of $\BbbR^{r+1}$ as functions from
$r+1=\{0,\ldots,r\}$ to $\Bbb R$.)   Set
$A_{kq}=\{x:x\in V_A,\,x(k)=q\}$,
$\Cal G_{kq}=\{\ocint{-\infty,x}:x\in A_{kq}\}$.   Then, in the notation
of 438N, $V_{A_{kq}}\in\Cal J(\Cal G)$.   \Prf\ Set $f(x)=x\restr K_k$
for each $x\in V_{A_{kq}}$, so that $f:V_{A_{kq}}\to\BbbR^{K_k}$ is
continuous.   For $x\in A_{kq}$,
$\ocint{-\infty,x}=f^{-1}[\,\ocint{-\infty,f(x)}\,]$.   Now
$\BbbR^{K_k}$ is \hwtr, by the inductive hypothesis, so if we set
$\Cal H_{kq}=\{\ocint{-\infty,f(x)}:x\in A_{kq}\}$,
$\bigcup\Cal H_{kq}\in\Cal J(\Cal H_{kq})$ and (by 438Nc)

\Centerline{$V_{A_{kq}}=f^{-1}[\bigcup\Cal H_{kq}]
\in\Cal J(\Cal G_{kq})\subseteq\Cal J(\Cal G)$.  \Qed}

\noindent Accordingly $W\in\Cal J(\Cal G)$, where
$W=\bigcup_{k\le r,q\in\Bbb Q}V_{A_{kq}}$, by 438Na.

Now consider $V_A\setminus W$.   If $x$, $x'\in V_A\setminus W$ and
$x\le x'$ then $x=x'$.   \Prf\Quer\ Otherwise, there are a $k\le n$ and
a $q\in\Bbb Q$ such that $x(k)\le q\le x'(k)$.   In this case, setting
$y\restr K_k=x\restr K_k$ and $y(k)=q$, we have $y\in A_{kq}$ and
$x\in V_{A_{kq}}$.\ \Bang\Qed

But this means that the subspace topology of $V_A\setminus W$ is
discrete, so that $\{\{x\}:x\in V_A\setminus W\}$ is an isolated
family covering $V_A\setminus W$ and refining $\Cal G$;  thus
$V_A\setminus W\in\Cal J(\Cal G)$ and $\bigcup\Cal G=V_A$ belongs to
$\Cal J(\Cal G)$.   As $\Cal G$ is arbitrary, $\BbbR^{r+1}$ is \hwtr\
and the induction proceeds.
}%end of proof of 438O

\leader{438P}{Lemma} Let $X$ be a Polish space, and
$\tildeClll=\tildeClll(X)$ the family of functions
$\omega:\Bbb R\to X$ such that $\lim_{s\uparrow t}\omega(s)$ and
$\lim_{s\downarrow t}\omega(s)$ are defined in $X$ for every
$t\in\Bbb R$.

(a) For $A\subseteq B\subseteq\Bbb R$ and $f\in X^B$, set

$$\eqalign{\jump_A(f,\epsilon)
&=\sup\{n:\text{ there is an }I\in[A]^n
\text{ such that }\rho(f(s),f(t))>\epsilon\cr
&\mskip180mu
\text{ whenever }s<t\text{ are successive elements of }I\}.\cr}$$

\noindent Now a function $\omega\in X^{\Bbb R}$ belongs to $\tildeClll$ iff
$\jump_{[-n,n]}(\omega,\epsilon)$ is finite for every $n\in\Bbb N$ and
$\epsilon>0$.

(b) If $\omega\in\tildeClll$ then $\omega$ is continuous at all but
countably many points of $\Bbb R$.

(c) If $\omega\in\tildeClll$ then $\omega[\,[-n,n]\,]$ is relatively
compact in $X$ for every $n\in\Bbb N$.

\proof{Fix on a complete metric $\rho$ inducing the topology of $X$.

\medskip

{\bf (a)(i)} Suppose that $\omega\in\tildeClll$, $n\in\Bbb N$ and
$\epsilon>0$.
For every $t\in[-n,n]$, there is a $\delta_t>0$ such that
$\rho(\omega(s),\omega(s'))\le\epsilon$ whenever either
$t<s\le s'\le t+\delta_t$
or $t-\delta_t\le s\le s'<t$.   Now there is an $m\ge 1$ such that
whenever $s$, $s'\in[-n,n]$ and $|s-s'|\le\Bover{2n}{m}$ there is a
$t\in[-n,n]$ such that both $s$ and $s'$ belong to
$[t-\delta_t,t+\delta_t]$.   Suppose now that
$-n\le t_0<t_1<\ldots<t_{3m}\le n$.   Then there must be
an $i<m$ such that $t_{3i+3}-t_{3i}\le\Bover{2n}m$.    Let $t$ be such
that both $t_{3i}$ and $t_{3i+3}$ belong to $[t-\delta_t,t+\delta_t]$.
There is at least one $j$ such that $3i\le j\le 3i+2$ and
$t\notin[t_j,t_{j+1}]$;  in which case
$\rho(\omega(t_j),\omega(t_{j+1}))\le\epsilon$.
So $\jump_{[-n,n]}(\omega,\epsilon)\le 3m$.   As $n$ and $\epsilon$ are
arbitrary, the condition is satisfied.

\medskip

\quad{\bf (ii)} Suppose that $\omega$ satisfies the condition.
If $t\in\Bbb R$ and $\epsilon>0$, take $n\ge|t|+1$ and
$m=\jump_{[-n,n]}(\omega,\epsilon)$;  then there must be a $\delta>0$
such that $\diam\{\omega(s):t<s\le t+\delta\}\le 2\epsilon$, since
otherwise we should be able to find $t_0>t_1>\ldots>t_m>t$ such that
$t_0=t+1$ and
$\rho(\omega(t_{i+1}),\omega(t_i))>\epsilon$ for $i<m$.   Because $X$ is
$\rho$-complete, $\lim_{s\downarrow t}\omega(s)$ is defined.   Similarly,
$\lim_{s\uparrow t}\omega(s)$ is defined;  as $t$ is arbitrary,
$\omega\in\tildeClll$.

\medskip

{\bf (b)} For $k\in\Bbb N$, set
set $A_k=\{t:t\in\Bbb R$,
$\limsup_{s\to t}\rho(\omega(s),\omega(t))>2^{-k+1}\}$.
Then $\#(A_k\cap\coint{-n,n})\le\jump_{[-n,n]}(\omega,2^{-k})$ for every
$n\in\Bbb N$.   \Prf\ If $t_0,\ldots,t_m\in A_k$ and
$-n\le t_0<\ldots<t_m<n$, then we can choose
$s_0,\ldots,s_m$ such that $s_0=t_0$,

\Centerline{$s_{i-1}<s_i<t_{i+1}$,
\quad $\rho(\omega(s_i),\omega(s_{i-1}))>2^{-k}$}

\noindent whenever $1\le i\le m$, interpreting $t_{m+1}$ as $n$.
Now $\{s_i:i\le m\}$ witnesses that
$\jump_{[-n,n]}(\omega,2^{-k})>m$.\ \Qed

By (a), $A_k\cap\coint{-n,n}$ is finite for every $n$,
and

\Centerline{$\{t:\omega$ is discontinuous at $t\}=\bigcup_{k\in\Bbb N}A_k$}

\noindent is countable.

\medskip

{\bf (c)} If $\sequence{k}{t_k}$ is any monotonic sequence in $\Bbb R$ with
limit $t$,
$\sequence{k}{\omega(t_k)}$ is convergent to one of
$\lim_{s\uparrow t}\omega(s)$, $\lim_{s\downarrow t}\omega(s)$.
But this means that if $\sequence{k}{t_k}$ is any sequence in
$[-n,n]$, $\sequence{k}{\omega(t_k)}$ has a subsequence which is convergent
in $X$;  by 4A2Le, $\omega[\,[-n,n]\,]$ is relatively compact in $X$.
}%end of proof of 438P

\leader{438Q}{Theorem} Let $X$ be a Polish space, and
$\tildeClll=\tildeClll(X)$ the family of functions
$\omega:\Bbb R\to X$ such that $\lim_{s\uparrow t}\omega(s)$ and
$\lim_{s\downarrow t}\omega(s)$ are defined in $X$ for every
$t\in\Bbb R$.

(a) $\tildeClll$,
with its topology of pointwise convergence
inherited from the product topology of $X^{\Bbb R}$, is K-analytic.

(b) $\tildeClll$ is \hwtr.

\proof{ Fix on a complete metric $\rho$ inducing the topology of $X$.

\medskip

{\bf (a)} By 4A2Qg, $X$ can be regarded as a G$_{\delta}$ set in a compact
metrizable space $Z$.

\medskip

\quad{\bf (i)}
Give the space $\Cal C=\Cal C(Z)$ of closed subsets of $Z$ its
Fell topology;  then $\Cal C$ is compact and metrizable (4A2T(b-iii),
4A2Tf).    Let $\Cal K$ be the family of compact subsets of $X$, that is,
the set of those $K\in\Cal C$ which are included in $X$.   Then $\Cal K$ is
a G$_{\delta}$ set in $\Cal C$.   \Prf\ $Z\setminus X$ is an F$_{\sigma}$
set in $Z$, so is expressible as the union of a sequence $\sequencen{L_n}$
of compact sets;  now
$\Cal K=\bigcap_{n\in\Bbb N}\{K:K\in\Cal C$, $K\cap L_n=\emptyset\}$ is
G$_{\delta}$, by the definition of the Fell topology (4A2T(a-ii)).\ \Qed

\medskip

\quad{\bf (ii)} For $n\in\Bbb N$, set

\Centerline{$Q_n
=\{\omega:\omega\in Z^{\Bbb R}$, $\omega[\,[-n,n]\,]$ is a
relatively compact subset of $X\}$.}

\noindent Then $Q_n$ is K-analytic.  \Prf\ Set

\Centerline{$R_n
=\{(K,\omega):K\in\Cal K$, $\omega\in Z^{\Bbb R}$,
$\omega[\,[-n,n]\,]\subseteq K\}$.}

\noindent Then

\Centerline{$R_n
=\bigcap_{t\in[-n,n]}\{(K,\omega):K\in\Cal K$, $\omega\in Z^{\Bbb R}$,
$\omega(t)\in K\}$}

\noindent is a closed set in $\Cal K\times Z^{\Bbb R}$, by 4A2T(e-i).
Since $\Cal K$ is Polish and $Z^{\Bbb R}$ is compact,
$R_n$ is K-analytic (423Ba, 423C, 422Ge, 422Gf).   But $Q_n$ is the
projection of $R_n$ onto the second coordinate, so it too is K-analytic
(422Gd).\ \Qed

\medskip

\quad{\bf (iii)} Next, for $m$, $k\in\Bbb N$, defining the function
$\jump_{[-n,n]}$ from $\rho$ as in 438P,

\Centerline{$V_{mk}=\{\omega:\omega\in Q_n$,
$\jump_{[-n,n]}(\omega,2^{-k})\le m\}$}

\noindent is relatively closed in $Q_n$, therefore K-analytic, and

\Centerline{$Q'_n=Q_n\cap\bigcap_{k\in\Bbb N}\bigcup_{m\in\Bbb N}V_{mk}$}

\noindent is K-analytic (422Hc again).

\medskip

\quad{\bf (iv)} Consequently $Q=\bigcap_{n\in\Bbb N}Q'_n$ is K-analytic.
But $Q=\tildeClll$.
\Prf\ For $n$, $k\in\Bbb N$, $\tildeClll\subseteq Q_n$ by 438Pc,
and $\tildeClll\subseteq\bigcup_{m\in\Bbb N}V_{mk}$ by 438Pa.
So $\tildeClll\subseteq Q$.
Conversely, if $\omega\in Q$, then surely
$\omega(t)\in X$ for every $t\in\Bbb R$, and
$\jump_{[-n,n]}(\omega,2^{-k})$ is finite for all $n$, $k\in\Bbb N$;  so
$\omega\in\tildeClll$ by 438Pa in the other direction.\ \Qed

Accordingly $\tildeClll$ is K-analytic, as claimed.

\medskip

{\bf (b)} Let $\Cal G$ be a family of open sets in $\tildeClll$.

\medskip

\quad{\bf (i)} In the
notation of 438N, I seek to show that $\bigcup\Cal G$ belongs to
$\Cal J(\Cal G)$.   Of course it will be enough to consider the case in
which $\bigcup\Cal G$ is non-empty.   The following elementary remarks
will be useful.

\medskip

\qquad\grheada\
If $D\subseteq\bigcup\Cal G$, and $\Cal E$ is a partition of
$D$ into relatively open sets such that $\Cal E$ refines $\Cal G$, then
$D\in\Cal J(\Cal G)$.

\medskip

\qquad\grheadb\ If $\familyiI{D_i}$ is any family in $\Cal J(\Cal G)$, and
$\familyiI{H_i}$ is a  family of open sets, and
$D\subseteq\{\omega:\#(\{i:\omega\in H_i\})=1\}$, then
$D\cap\bigcup_{i\in I}H_i\cap D_i$ belongs to
$\Cal J(\Cal G)$.   \Prf\ $\familyiI{D\cap H_i\cap D_i}$ is an isolated
family in $\Cal J(\Cal G)$;  use 438Nd.\ \Qed

\medskip

\quad{\bf (ii)} Let $\Cal I$ be the family of non-empty open intervals in
$\Bbb R$ with rational endpoints, and $\Cal U$ a countable base for the
topology of $X$.   Write $Q$ for the family of all
finite sequences

\Centerline{$\pmb{q}=
(I_0,U_0,V_0,W_0,I_1,U_1,V_1,W_1,\ldots,I_n,U_n,V_n,W_n)$}

\noindent where $I_0,I_1,\ldots,I_n$ are disjoint members of $\Cal I$,
all the $U_i$, $V_i$, $W_i$ belong to $\Cal U$, and, for each $i\le n$,
any pair of $U_i$, $V_i$, $W_i$ are either equal or disjoint.   Fix
$\pmb{q}=(I_0,\ldots,W_n)\in Q$ for the moment.

\medskip

\quad{\bf (iii)} Set $T_{\pmb{q}}=\prod_{i\le n}I_i$.   For
$\tau\in T_{\pmb{q}}$, write $F_{\pmb{q}\tau}$ for the set of those
$\omega\in\tildeClll$ such that, for every $i\le n$,
$\omega(s)\in U_i$ for $s\in I_i\cap\ooint{-\infty,\tau(i)}$,
$\omega(\tau(i))\in V_i$ and $\omega(s)\in W_i$ for
$s\in I_i\cap\ooint{\tau(i),\infty}$.   Set
$\Omega_{\pmb{q}}=\bigcup\{F_{\pmb{q}\tau}:\tau\in T_{\pmb{q}}\}$,
and for $\tau\in T_{\pmb{q}}$ set
$H_{\pmb{q}\tau}
=\{\omega:\omega\in \Omega_{\pmb{q}},\,\omega(\tau(i))\in V_i$
for every $i\le n\}$.
Finally, set

\Centerline{$S_{\pmb{q}}
=\{\tau:\tau\in T_{\pmb{q}}$ and $H_{\pmb{q}\tau}$ is included in some
member of $\Cal G\}$.}

\medskip

\quad{\bf (iv)} If $T\subseteq S_{\pmb{q}}$ then
$H=\bigcup_{\tau\in T}H_{\pmb{q}\tau}$ belongs to $\Cal J(\Cal G)$.
\Prf\ Induce on $\#(L(T))$, where

\Centerline{$L(T)=\{i:i\le n$, there are $\tau$, $\tau'\in T$ such that
$\tau(i)\ne\tau'(i)\}$.}

\noindent If $L(T)=\emptyset$, then $\#(T)\le 1$, so $H$ is either empty
or included in some member of $\Cal G$, and the induction
starts.   For the inductive step to $\#(L(T))=k\ge 1$, consider three
cases.

\medskip

\qquad{\bf case 1} Suppose there is a $j\in L(T)$ such that
$U_j=V_j=W_j$.   Then $\omega(t)\in V_j$ whenever
$\omega\in\Omega_{\pmb{q}}$ and
$t\in I_j$.   Fix any $t^*\in I_j$ and for $\tau\in T_{\pmb{q}}$ define
$\tau^*\in T_{\pmb{q}}$ by setting $\tau^*(j)=t^*$, $\tau^*(i)=\tau(i)$
for $i\ne j$;  then $H_{\pmb{q}\tau^*}=H_{\pmb{q}\tau}$.   Set
$T^*=\{\tau^*:\tau\in T\}$;  then $L(T^*)=L(T)\setminus\{j\}$, so
$\#(L(T^*))<\#(L(T))$, while $T^*\subseteq S_{\pmb{q}}$.   By the
inductive hypothesis, $H=\bigcup_{\tau\in T^*}H_{\pmb{q}\tau}$ belongs to
$\Cal J(\Cal G)$.

\medskip

\qquad{\bf case 2} Suppose there is a $j\in L(T)$ such that $U_j\ne V_j$
and $V_j\ne W_j$.   Then $V_j\cap(U_j\cup W_j)=\emptyset$.   For
$s\in I_j$ set $T^*_s=\{\tau:\tau\in T,\,\tau(j)=s\}$.   Then
$\#(L(T^*_s))<\#(L(T))$ so, by the inductive hypothesis,
$H^*_s\in\Cal J(\Cal G)$,
where $H^*_s=\bigcup_{\tau\in T^*_s}H_{\pmb{q}\tau}$.   But, for
$\tau\in T$ and $\omega\in H_{\pmb{q}\tau}$, $\omega(s)\in V_j$ iff
$s=\tau(j)$;
so $H^*_s=\{\omega:\omega\in H,\,\omega(s)\in V_j\}$ and
$\family{s}{I_j}{H^*_s}$ is a
partition of $H$ into relatively open sets.   By (i-$\beta$),
$H\in\Cal J(\Cal G)$.

\medskip

\qquad{\bf case 3} Otherwise, $L(T)=J\cup J'$ where
$J=\{i:i\in L(T),\,U_i=V_i\}$ and $J'=\{i:i\in L(T),\,V_i=W_i\}$ are
disjoint.   For $\omega\in H$ and $i\in J$ we see that there is a
largest $t\in I_i$ such that $\omega(t)\in V_i$;  set $\phi_i(\omega)=-t$.
Similarly, if $\omega\in H$, $i\in J'$ there is a smallest
$t\in I_i$ such that $\omega(t)\in V_i$;  in this case, set
$\phi_i(\omega)=t$.   Observe that, for $i\in J$ and $s\in\Bbb R$,

$$\eqalign{\{\omega:\omega\in H,\,\phi_i(\omega)\le s\}
&=\emptyset\text{ if }s<-t\text{ for every }t\in I_i,\cr
&=H\text{ if }-t<s\text{ for every }t\in I_i,\cr
&=\{\omega:\omega\in H,\,\omega(-s)\in V_i\}\text{ if }-s\in I_i,\cr}$$

\noindent so is always relatively open in $H$, and $\phi_i:H\to\Bbb R$ is
continuous if $\Bbb R$ is given the left-facing
topology $\frak S$ of Lemma 438O.   Similarly, for $i\in J'$, $s\in\Bbb R$,

$$\eqalign{\{\omega:\omega\in H,\,\phi_i(\omega)\le s\}
&=\emptyset\text{ if }s<t\text{ for every }t\in I_i,\cr
&=H\text{ if }t<s\text{ for every }t\in I_i,\cr
&=\{\omega:\omega\in H,\,\omega(s)\in V_i\}\text{ if }s\in I_i.\cr}$$

\noindent So in this case also $\phi_i$ is continuous.

Accordingly, giving $\BbbR^{L(T)}$ the product topology corresponding to
$\frak S$, we have a continuous map $\phi:H\to\BbbR^{L(T)}$
defined by setting $\phi(\omega)=\family{i}{L(T)}{\phi_i(\omega)}$ for
$\omega\in H$.   For $\tau\in T$, set $\tilde\tau(i)=-\tau(i)$ if $i\in J$,
$\tau(i)$ if $i\in J'$, and
$\tilde H_{\tau}=\ocint{-\infty,\tilde\tau}\subseteq\BbbR^{L(T)}$.  Then

$$\eqalignno{H_{\pmb{q}\tau}
&=\{\omega:\omega\in H,\,\omega(\tau(i))\in V_i
  \text{ for every }i\le n\}\cr
&=\{\omega:\omega\in H,\,\omega(\tau(i))\in V_i
  \text{ for every }i\in L(T)\}\cr
\displaycause{because if $\omega\in H$, $i\le n$ and $i\notin L(T)$ then
there is some $\tau'\in T$ such that $\omega\in H_{\pmb{q}\tau'}$, so that
$\omega(\tau'(i))\in V_i$ and therefore $\omega(\tau(i))\in V_i$}
&=\{\omega:\omega\in H,\,\phi_i(\omega)\le\tilde\tau(i)
  \text{ for every }i\in L(T)\}
=\phi^{-1}[\tilde H_{\tau}].\cr}$$

Set $\tilde\Cal G=\{\tilde H_{\tau}:\tau\in T\}$.   Because
$\BbbR^{L(T)}$ is \hwtr\ (438O), and $\tilde\Cal G$ is a family of open
subsets of $\BbbR^{L(T)}$,
$\bigcup\tilde\Cal G\in\Cal J(\tilde\Cal G)$.   By 438Nc,
$H=\bigcup_{\tau\in T}H_{\pmb{q}\tau}=\phi^{-1}[\bigcup\tilde\Cal G]$
belongs to $\Cal J(\{H_{\pmb{q}\tau}:\tau\in T\}$ and therefore to
$\Cal J(\Cal G)$, by 438Nb.

Thus in all three cases the induction proceeds.\ \Qed

\medskip

\quad{\bf (v)} This means that, for any $\pmb{q}\in Q$,
$Y_{\pmb{q}}=\bigcup\{H_{\pmb{q}\tau}:\tau\in S_{\pmb{q}}\}$ belongs to
$\Cal J(\Cal G)$.   Since $Q$ is countable, $Y\in\Cal J(\Cal G)$, where
$Y=\bigcup_{\pmb{q}\in Q}Y_{\pmb{q}}$.   But $\bigcup\Cal G\subseteq Y$.
\Prf\ If $\omega\in G\in\Cal G$, there are $t_0<\ldots<t_n$ and
$V'_i\in\Cal I$, for $i\le n$, such that

\Centerline{$\omega
\in\{\omega':\omega'\in\tildeClll,\,\omega'(t_i)\in V'_i$ for every
  $i\le n\}\subseteq G$.}

\noindent Set $x_i=\omega(t_i)$, $x^-_i=\lim_{s\uparrow t_i}\omega(s)$,
$x^+_i=\lim_{s\downarrow t_i}\omega(s)$ for $i\le n$;  let $U_i$, $V_i$,
$W_i\in\Cal U$ be such that $x^-_i\in U_i$, $x_i\in V_i\subseteq V'_i$,
$x^+_i\in W_i$ and any pair of $U_i$, $V_i$, $W_i$ are either equal or
disjoint;  and let $I_0,\ldots,I_n\in\Cal I$ be disjoint and such that
$t_i\in I_i$, $\omega(s)\in U_i$ for $s\in I_i\cap\ooint{-\infty,t_i}$ and
$\omega(s)\in W_i$ for $s\in I_i\cap\ooint{t_i,\infty}$ for each $i\le n$.
Then, setting $\pmb{q}=(I_0,\ldots,W_n)$ and $\tau(i)=t_i$ for $i\le n$,

\Centerline{$\omega\in F_{\pmb{q}\tau}
\subseteq H_{\pmb{q}\tau}\subseteq G$,}

\noindent so that $\tau\in S_{\pmb{q}}$ and
$\omega\in Y_{\pmb{q}}\subseteq Y$.\ \Qed

As $\Cal G$ is arbitrary, $\tildeClll $ is \hwtr.
}%end of proof of 438Q

\leader{438R}{Corollary} (a) Let $I^{\|}$ be the split
interval\cmmnt{ (419L)}.   Then any countable power of $I^{\|}$ is a
\hwtr\ compact Hausdorff space.

(b) Let $Y$ be the `Helly space', the space of non-decreasing functions
from $[0,1]$ to itself with the topology of pointwise convergence
inherited from the product topology on
$[0,1]^{[0,1]}$\cmmnt{ ({\smc Kelley 55}, Ex.\ 5M)}.
Then $Y$ is a \hwtr\ compact Hausdorff space.

\medskip

\proof{ These are both (homeomorphic to) subspaces of the
space $\tildeClll$ of Proposition 438Q, if we take $X$ there to be
$\Bbb R$.   To see this, argue as follows.   For
(a), observe that we have a function $f:I^{\|}\to\tildeClll$
defined by setting
$f(t^-)(s)=f(t^+)(s)=1$ if $s<t$, $f(t^-)(s)=f(t^+)(s)=0$ if $s>t$, and
$f(t^-)(t)=0$, $f(t^+)(t)=1$, and that $f$ is a homeomorphism between
$I^{\|}$ and its image.   Next, for any $L\subseteq\Bbb N$, we can
define $g:(I^{\|})^L\to\tildeClll$ by setting

$$\eqalign{g(\tbf{t})(s)
&=f(t_n)(s-2n)\text{ if }n\in L\text{ and }2n\le s\le 2n+1,\cr
&=0\text{ if }s\in\Bbb R\setminus\bigcup_{n\in L}[2n,2n+1]\cr}$$

\noindent for $\tbf{t}=\family{n}{L}{t_n}\in (I^{\|})^L$;  it is
easy to check that $g$ is a homeomorphism between $(I^{\|})^L$
and its image in $\tildeClll$.
As for (b), if we take $g(y)$ to be the extension of the function
$y:[0,1]\to[0,1]$ to the function which is constant on each of the
intervals $\ocint{-\infty,0}$ and $\coint{1,\infty}$, then
$g:Y\to\tildeClll$ is
a homeomorphism between $Y$ and its image $g[Y]$.

Since both $(I^{\|})^L$ and $Y$ are compact, they are
homeomorphic to closed subsets of $\tildeClll$, and are \hwtr\ (438La).
}%end of proof of 438R

\leader{*438S}{C\`all\`al \dvrocolon{functions}}\cmmnt{ To support some
of the theory of L\'evy processes which I will present in \S455, I give a
further consequence of 438Q.

\medskip

\noindent}{\bf Proposition} Let $X$ be a
Polish space.   Let $\Clll=\Clll(X)$ be the set of \callal\
functions\cmmnt{ (definition:  4A2A)} from $\coint{0,\infty}$ to $X$,
with its topology of
pointwise convergence inherited from the product topology of
$X^{\coint{0,\infty}}$.

(a)(i) If $\omega\in\Clll$, then $\omega$ is continuous at all but
countably many points of $\coint{0,\infty}$.

\quad(ii) If $\omega$, $\omega'\in\Clll$, $D$ is a dense subset of
$\coint{0,\infty}$ containing every point at which $\omega$ is
discontinuous, and $\omega'\restr D=\omega\restr D$, then $\omega'=\omega$.

(b) $\Clll$ is \hwtr.

(c) $\Clll$ is K-analytic.

\proof{ Fix a complete metric $\rho$ on $X$ defining its topology.
Let $\tildeClll\subseteq X^{\Bbb R}$ be the space of 438P-438Q.

\medskip

{\bf (a)} If $X=\emptyset$ the results are trivial.
Otherwise, fix $x_0\in X$, and for $\omega\in X^{\coint{0,\infty}}$ define
$\tilde\omega\in X^{\Bbb R}$ to be that extension of $\omega$
which takes the value $x_0$ everywhere on $\ooint{-\infty,0}$.

\medskip

\quad{\bf (i)}
If $\omega\in\Clll$, then $\tilde\omega\in\tildeClll$;
so the result follows from 438Pb.

\medskip

\quad{\bf (ii)} If $t\in D$, $\omega'(t)$ is certainly equal to
$\omega(t)$.   Next,

\Centerline{$\omega'(0)
=\lim_{s\downarrow 0}\omega'(s)
=\lim_{s\in D,s\downarrow 0}\omega'(s)
=\lim_{s\in D,s\downarrow 0}\omega(s)
=\omega(0)$.}

\noindent If $t\in\ooint{0,\infty}\setminus D$, then $\omega$ is continuous
at $t$, so

\Centerline{$\lim_{s\uparrow t}\omega'(s)
=\lim_{s\in D,s\uparrow t}\omega'(s)
=\lim_{s\in D,s\uparrow 0}\omega(s)
=\omega(t)$,}

\Centerline{$\lim_{s\downarrow t}\omega'(s)
=\lim_{s\in D,s\downarrow t}\omega'(s)
=\lim_{s\in D,s\downarrow 0}\omega(s)
=\omega(t)$.}

\noindent Since $\omega'(t)$ must be either
$\lim_{s\uparrow t}\omega'(s)$ or $\lim_{s\downarrow t}\omega'(s)$, it is
again equal to $\omega(t)$.   So $\omega'=\omega$.

\medskip

{\bf (b)} Since $\Clll$ is homeomorphic to a subspace of $\tildeClll$, it
is \hwtr\ (438Qb, 438La).

\medskip

{\bf (c)} Set $\tilde Q=\{\tilde\omega:\omega\in\Clll\}$;  then
$\tilde Q\subseteq\tildeClll$ is homeomorphic to $\Clll$.   But
$\tilde Q$ is a Souslin-F set in $\tildeClll$.   \Prf\ If $\omega\in\Clll$
then it belongs to $\tilde Q$ iff $\omega(t)=x_0$ for every $t<0$.
$\lim_{t\downarrow 0}\omega(t)=\omega(0)$, and
$\omega(t)\in\{\lim_{s\uparrow t}\omega(s),\lim_{s\downarrow t}\omega(s)\}$
for every $t>0$.   Now

\Centerline{$\{\omega:\omega(t)=x_0$ for every $t<0\}$}

\noindent is closed, while

$$\eqalignno{\{\omega:\omega\in\tildeClll,\,
  \omega(0)=\lim_{t\downarrow 0}\omega(t)\}
&=\{\omega:\omega\in\tildeClll,\,
  \omega(0)=\lim_{i\to\infty}\omega(2^{-i})\}\cr
\displaycause{because $\lim_{t\downarrow 0}\omega(t)$ is defined for every
$\omega\in\tildeClll$}
&=\bigcap_{k\in\Bbb N}\bigcup_{m\in\Bbb N}\bigcap_{i\ge m}
 \{\omega:\omega\in\tildeClll,\,\rho(\omega(2^{-i}),\omega(0))\le 2^{-k}\}
\cr}$$

\noindent is Souslin-F.   As for the other condition, note that for $t>0$
and $\omega\in\tildeClll$,
$\omega(t)$ belongs to
$\{\lim_{s\uparrow t}\omega(s),\lim_{s\downarrow t}\omega(s)\}$
iff for every $\epsilon>0$ there are distinct rational
numbers $q$, $q'\in[t-\epsilon,t+\epsilon]$ such that
$\omega(q)$ and $\omega(q')$ belong to $B(\omega(t),\epsilon)$.
Let $\Cal U$ be a countable base for the topology of $X$ and
$\Cal I$ a countable base for the topology of $\ooint{0,\infty}$ not
containing $\emptyset$;  then for $\omega\in\tildeClll$,
$\omega(t)\in\{\lim_{s\uparrow t}\omega(s),\lim_{s\downarrow t}\omega(s)\}$
for every $t>0$ if and only if

\inset{\noindent for every $U\in\Cal U$ and $I\in\Cal I$
either $I\cap\omega^{-1}[U]=\emptyset$ or
$I\cap\Bbb Q\cap\omega^{-1}[\overline{U}]$ has at least two members.}

\noindent Since, for $U\in\Cal U$ and $I\in\Cal I$,

\Centerline{$\{\omega:\omega\in\tildeClll$,
$I\cap\omega^{-1}[U]=\emptyset\}
=\bigcap_{t\in I}\{\omega:\omega\in\tildeClll$, $\omega(t)\notin U\}$}

\noindent is closed in $\tildeClll$, and

$$\eqalign{\{\omega:\omega\in\tildeClll,\,
I\cap\Bbb Q\cap\omega^{-1}&[\overline{U}]\text{ has at least two members}\}
\cr
&=\bigcup_{\Atop{q,q'\in I\cap\Bbb Q}{q<q'}}
   \{\omega:\omega(q),\,\omega(q')\in\overline{U}\}\cr}$$

\noindent is F$_{\sigma}$ in $\tildeClll$, while $\Cal I$ and $\Cal U$ are
countable,

\Centerline{$\{\omega:\omega\in\tildeClll$,
$\omega(t)\in\{\lim_{s\uparrow t}\omega(s),\lim_{s\downarrow t}\omega(s)\}$
for every $t>0\}$}

\noindent is Souslin-F in $\tildeClll$.   Taking the intersection, we see
that $\tilde Q$ is Souslin-F.\  \Qed

Accordingly $\tilde Q$ and $\Clll$ are K-analytic (422Hb).
}%end of proof of 438S

\leader{438T}{Proposition} Assume that $\frak c$ is measure-free.
Then $(I^{\|})^{\Bbb N}$, the Helly
space\cmmnt{ (438Rb)} and the spaces $\tildeClll(X)$, $\Clll(X)$ of 438Q
and 438S, for any Polish space $X$, are all Radon spaces.

\proof{ By 438Q-438S, %438Q 438R 438S
they are K-analytic and \hwtr, also they have weight at most
$w(X^{\Bbb R})\le\frak c$.   They are therefore
pre-Radon (434Jf), Borel-measure-complete (438M) and Radon
(434Ka).
}%end of proof of 438T

\leader{438U}{}\cmmnt{ In 434R I described a construction of
product measures.   In accordance with my general practice of
examining the measure algebra of any new measure, I give the
following result.

\medskip

\noindent}{\bf Proposition} Let $X$ and $Y$ be topological spaces with
$\sigma$-finite Borel measures $\mu$, $\nu$ respectively.   Suppose that
{\it either} $X$ is first-countable {\it or} $\nu$ is $\tau$-additive
and effectively locally finite.   Write $\lambda$ for the Borel measure
on $X\times Y$ defined by the formula

\Centerline{$\lambda W=\int\nu W[\{x\}]\mu(dx)$
for every Borel set $W\subseteq X\times Y$}

\noindent as in 434R(ii).   If {\it either} the weight of $X$ {\it or}
the Maharam type of $\nu$ is a measure-free cardinal, then for every
Borel set $W\subseteq X\times Y$ there is a set
$W'\in\Cal B(X)\tensorhat\Cal B(Y)$ such that
$\lambda(W\symmdiff W')=0$;  consequently, the measure algebra of
$\lambda$ can be identified with
the localizable measure algebra free product of the measure algebras of
$\mu$ and $\nu$.

\proof{{\bf (a)} Write $(\frak B,\bar\nu)$ for the measure algebra of
$\nu$.   With its measure-algebra topology, this is metrizable (323Gb).
Let $\sequencen{Y_n}$ be
a non-decreasing sequence of Borel sets of finite measure in $Y$ with
union $Y$.

\medskip

{\bf (b)} For the moment (down to the end of (e) below) fix on an open
set $W\subseteq X\times Y$.   For $x\in X$, set
$f(x)=W[\{x\}]^{\ssbullet}$ in $\frak B$.   Then $f:X\to\frak B$ is
Borel measurable.

\medskip

\Prf\ {\bf (i)} Let $H\subseteq\frak B$ be an open set.   For $k$,
$n\in\Bbb N$ set

\Centerline{$E_{nk}
=\{x:x\in X,\,2^{-n}k\le\nu(Y_n\cap W[\{x\}])<2^{-n}(k+1)\}$.}

\noindent Just as in part (a) of the proof of 434R, the function
$x\mapsto\nu(Y_n\cap W[\{x\}])$ is lower semi-continuous, so $E_{nk}$ is
a Borel set.   Set

\Centerline{$G_{nk}
=\bigcup\{G:G\subseteq X$ is open, $G\cap E_{nk}\subseteq f^{-1}[H]\}$;}

\noindent then $E=\bigcup_{n,k\in\Bbb N}(G_{nk}\cap E_{nk})$ is a Borel
set included in $f^{-1}[H]$.

\medskip

\quad{\bf (ii)} The point is that $E=f^{-1}[H]$.   To see this, take any
$x$ such that $f(x)\in H$.   Then there are $b\in\frak B$, $\epsilon>0$
such that $\bar\nu b<\infty$ and $c\in H$ whenever $c\in\frak B$ and
$\bar\nu(b\Bcap(c\Bsymmdiff f(x)))\le 5\epsilon$.   Since
$b=\sup_{n\in\Bbb N}b\Bcap Y_n^{\ssbullet}$, there is an $n\in\Bbb N$
such that $\bar\nu(b\Bsetminus Y_n^{\ssbullet})\le\epsilon$ and
$2^{-n}\le\epsilon$.   In this case $c\in H$ whenever $c\in\frak B$ and
$\bar\nu(Y_n^{\ssbullet}\Bcap(c\Bsymmdiff f(x)))\le 4\epsilon$;  thus

\Centerline{$\{x':\nu(Y_n\cap(W[\{x'\}]\symmdiff W[\{x\}]))\le
4\epsilon\}\subseteq f^{-1}[H]$.}

Let $k\in\Bbb N$ be such that
$2^{-n}k\le\nu(Y_n\cap W[\{x\}])<2^{-n}(k+1)$, that is, $x\in E_{nk}$.
Again using the ideas of part (a) of the proof of 434R, there are an
open set $G$ containing $x$ and an open set $V\subseteq Y$ such that
$G\times V\subseteq W$ and $\nu(Y_n\cap V)\ge 2^{-n}(k-1)$.   Now if
$x'\in G\cap E_{nk}$, $V\subseteq W[\{x'\}]\cap W[\{x\}]$, so

$$\eqalign{\nu(Y_n\cap(W[\{x'\}]\symmdiff W[\{x\}]))
&\le\nu(Y_n\cap W[\{x'\}])+\nu(Y_n\cap W[\{x\}])-2\nu(Y_n\cap V)\cr
&\le 2^{-n}((k+1)+(k+1)-2(k-1))
=4\cdot 2^{-n}\le 4\epsilon.\cr}$$

\noindent But this means that $G\cap E_{nk}\subseteq f^{-1}[H]$, so
$G\subseteq G_{nk}$ and $x\in G\cap E_{nk}\subseteq E$.   As $x$ is
arbitrary, $f^{-1}[H]\subseteq E$ and $E=f^{-1}[H]$.

\medskip

\quad{\bf (iii)} Thus $f^{-1}[H]$ is a Borel set.   As $H$ is arbitrary,
$f$ is Borel measurable.\ \Qed

\medskip

{\bf (c)} We need to know also that if
$\Cal H$ is a disjoint family of open subsets of $\frak B$ all meeting
$f[X]$, then

\Centerline{$\#(\Cal H)\le\max(\omega,\min(w(X),\tau(\frak B)))$}

\noindent\Prf\ Repeat the ideas of (b) above, setting

\Centerline{$G^{(H)}_{nk}=\bigcup\{G:G\subseteq X$ is open, $G\cap
E_{nk}\subseteq f^{-1}[H]\}$}

\noindent for $H\in\Cal H$ and $k$, $n\in\Bbb N$, so that
$f^{-1}[H]=\bigcup_{n,k\in\Bbb N}G^{(H)}_{nk}\cap E_{nk}$.   For fixed
$n$ and $k$ the family $\family{H}{\Cal H}{G^{(H)}_{nk}\cap E_{nk}}$ is
disjoint, so can have at most $w(E_{nk})\le w(X)$ non-empty members
(4A2D again).   But this means that

\Centerline{$\Cal H
=\bigcup_{n,k\in\Bbb N}\{H:G^{(H)}_{nk}\cap E_{nk}\ne\emptyset\}$}

\noindent has cardinal at most $\max(\omega,w(X))$.

On the other hand, there is a set $B\subseteq\frak B$, of cardinal
$\tau(\frak B)$, which $\tau$-generates $\frak B$.   The algebra
$\frak B_0$ generated by $B$ has cardinal at most $\max(\omega,\#(B))$
(331Gc), and $\frak B_0$ is topologically dense in $\frak B$ (323J), so
every member of $\Cal H$ meets $\frak B_0$, and

\Centerline{$\#(\Cal H)\le\#(\frak B_0)\le\max(\omega,\tau(\frak B))$.}

\noindent Putting these together, we have the result.\ \Qed

In particular, under the hypotheses above, $\#(\Cal H)$ is measure-free
whenever $\Cal H$ is a disjoint family of open subsets of $\frak B$
all meeting $f[X]$.

\medskip

{\bf (d)} The next step is to observe that there is a conegligible Borel
set $Z\subseteq X$ such that $f[Z]$ is separable.   \Prf\ Let $\Cal H$
be a $\sigma$-disjoint base for the topology of $\frak B$;
express it as $\bigcup_{n\in\Bbb N}\Cal H_n$ where each $\Cal H_n$ is
disjoint.   Let $\sequence{m}{X_m}$ be a cover of $X$ by Borel sets of
finite measure.   For $n\in\Bbb N$ consider
$\Cal H'_n=\{H:H\in\Cal H_n,\,H\cap f[X]\ne\emptyset\}$.   For
$m\in\Bbb N$, we have a totally
finite measure $\nu_{nm}$ with domain $\Cal P\Cal H'_n$ defined by
saying

\Centerline{$\nu_{nm}\Cal E=\mu(X_m\cap f^{-1}(\bigcup\Cal E))$}

\noindent for every $\Cal E\subseteq\Cal H'_n$.   Since $\Cal H'_n$ has
measure-free cardinal, by (c), there must be a countable set
$\Cal E_{nm}\subseteq\Cal H'_n$ such that
$\nu_{nm}(\Cal H'_n\setminus\Cal E_{nm})=0$.   Set

\Centerline{$Z=X\setminus\bigcup_{m,n\in\Bbb N}
  (X_m\cap f^{-1}[\bigcup(\Cal H'_n\setminus\Cal E_{nm})])$;}

\noindent then $Z$ is conegligible.   If $x\in Z$ and
$f(x)\in H\in\Cal H_n$, then there is some $m\in\Bbb N$ such that
$x\in X_m$, while
$H\in\Cal H'_n$, so $H$ must belong to $\Cal E_{nm}$.   But this means
that $\{f[Z]\cap H:H\in\Cal H\}$, which is a base for the topology of
$f[Z]$, is just $\{f[Z]\cap H:H\in\bigcup_{m,n\in\Bbb N}\Cal E_{mn}\}$,
and is countable.   So $f[Z]$ is separable (4A2Oc), as required.\
\Qed

\medskip

{\bf (e)} 418T(a-ii) now tells us that there is a set
$W'\in\Cal B(X)\tensorhat\Cal B(Y)$ such that
$f(x)=W'[\{x\}]^{\ssbullet}$ for every $x\in Z$, so that
$\nu(W[\{x\}]\symmdiff W'[\{x\}])=0$ for almost every $x$, that is,
$\lambda(W\symmdiff W')=0$.
And this is true for every open set $W\subseteq X\times Y$.

\medskip

{\bf (f)} Now let $\Cal W$ be the family of those Borel sets
$W\subseteq X\times Y$ for which there is a
$W'\in\Cal B(X)\tensorhat\Cal B(Y)$ such
that $\lambda(W\symmdiff W')=0$.   This is a $\sigma$-algebra containing
every open set, so is the whole Borel $\sigma$-algebra, as required.

Since the c.l.d.\ product measure $\lambda_0$ on $X\times Y$ is just the
completion of its restriction to $\Cal B(X)\tensorhat\Cal B(Y)$
(251K), and $\lambda_0$ and $\lambda$ agree on
$\Cal B(X)\tensorhat\Cal B(Y)$ (by Fubini's theorem), the embedding
$\Cal B(X)\tensorhat\Cal B(Y)\embedsinto\Cal B(X\times Y)$ induces an
isomorphism
between the measure algebras of $\lambda$ and $\lambda_0$.   As remarked
in 325Eb, because $\mu$ and $\nu$ are strictly localizable, the latter
may be
identified with `the' localizable measure algebra free product of the
measure algebras of $\mu$ and $\nu$.
}%end of proof of 438U

\cmmnt{\medskip

\noindent{\bf Remark} The hypothesis on the weight of $X$ can be
slightly weakened;  see 438Yg.   439L below shows that some restriction on
$(X,\mu)$ and $(Y,\nu)$ is necessary.
}%end of comment

\exercises{\leader{438X}{Basic exercises (a)}
%\spheader 438Xa
Show that a cardinal $\kappa$ is measure-free iff
$M_{\sigma}=M_{\tau}$, where $M_{\sigma}$, $M_{\tau}$ are the spaces of
countably additive and completely additive functionals on the algebra
$\Cal P\kappa$ (362B).
%438A

\spheader 438Xb Let $(X,\Sigma,\mu)$ be a localizable measure space with
magnitude (definition:  332Ga) which is either finite or a measure-free
cardinal.   Show that any absolutely continuous countably additive
functional $\nu:\Sigma\to\Bbb R$ is truly continuous.   \Hint{363S.}
%438A, 363S

\spheader 438Xc Let $\frak A$ be a Dedekind complete Boolean algebra
with measure-free cellularity.   Show that any countably additive
functional $\nu:\frak A\to\Bbb R$ is completely additive.
%438A, 363S

\spheader 438Xd Let $U$ be a Dedekind complete Riesz space such that any
disjoint order-bounded family in $U^+$ has measure-free cardinal.
Show that $U^{\sim}_c=U^{\times}$.
%438A, 363S

\spheader 438Xe Let $(X,\Sigma,\mu)$ be a complete locally determined
measure space, $(Y,\Tau,\nu)$ a strictly localizable measure space, and
$f:X\to Y$ an \imp\ function.   Suppose that the magnitude of $\nu$ is
either finite or a measure-free cardinal.   Show that $\mu$ is strictly
localizable.
%438B query out of order

\spheader 438Xf Let $(X_1,\Sigma_1,\mu_1)$, $(X_2,\Sigma_2,\mu_2)$,
$(Y_1,\Tau_1,\nu_1)$ and $(Y_2,\Tau_2,\nu_2)$ be measure spaces, and
$\lambda_1$, $\lambda_2$ the c.l.d.\ product measures on $X_1\times Y_1$,
$X_2\times Y_2$ respectively;   suppose that $f:X_1\to X_2$ and
$g:Y_1\to Y_2$ are \imp\ functions, and that
$h(x,y)=(f(x),g(y))$ for $x\in X_1$, $y\in Y_1$.   Show that if $\mu_2$ and
$\nu_2$ are both strictly localizable, with
magnitudes which are either finite or measure-free
cardinals, then $h$ is \imp.   (Compare 251L.)
%438Xe 438B

\sqheader 438Xg Show that if $\kappa$ is a measure-free cardinal, so
is $\omega_{\kappa}$.   \Hint{show by induction on ordinals $\xi$ that
if $\#(\xi)$ is measure-free, then so is $\omega_{\xi}$.}
%438C

\sqheader 438Xh
Let $(X,\Sigma,\mu)$ be a complete locally determined measure space,
$(Y,\rho)$ a complete metric space with measure-free weight, and
$\sequencen{f_n}$ a sequence of
measurable functions from $X$ to $Y$.   Show that
$\{x:\lim_{n\to\infty}f_n(x)$ is defined in $Y\}$ is measurable.
(Cf.\ 418C.)
%438E+

\spheader 438Xi Let $(X,\Sigma,\mu)$ and $(Y,\Tau,\nu)$ be
$\sigma$-finite measure spaces with c.l.d.\
product\discretionary{}{}{}
$(X\times Y,\discretionary{}{}{}\Lambda,\lambda)$.   Give $L^0(\nu)$ the
topology of convergence in measure.   Suppose that $f:X\to L^0(\nu)$ is
measurable and there is a conegligible set
$X_0\subseteq X$ such that $w(f[X_0])$ is measure-free.   Show that
there is an $h\in\eusm L^0(\lambda)$ such that $f(x)=h_x^{\ssbullet}$
for every $x\in X$, where $h_x(y)=h(x,y)$ for $(x,y)\in\dom h$.
(Cf.\ 418S.)
%438G+

\spheader 438Xj Let $(Y,\Tau,\nu)$ be a
$\sigma$-finite measure space, and $(\frak B,\bar\nu)$ its measure
algebra,  with its usual topology;  assume that the Maharam type of
$\frak B$ is measure-free.
Let $(X,\Sigma,\mu)$ be a $\sigma$-finite measure space and
$\Lambda$ the domain of the c.l.d.\ product measure $\lambda$ on
$X\times Y$.   Show that if $f:X\to\frak B$ is measurable, then there is
a $W\in\Lambda$ such that $f(x)=W[\{x\}]^{\ssbullet}$ for every
$x\in X$.   (Cf.\ 418T(b-ii).)
%438G+

\spheader 438Xk Let $(X,\Sigma,\mu)$ be a complete locally determined
measure space and $V$ a normed space such that $w(V)$ is measure-free.
(i) Show that the space $\eusm L$ of measurable functions from
$X$ to $V$ is a linear space, setting $(f+g)(x)=f(x)+g(x)$, etc.   (ii)
Show that if $V$ is a Riesz space with a Riesz norm then $\eusm L$ is a
Riesz space under the natural operations.
%438G+

\sqheader 438Xl Let $X$ be a topological space and $\Cal G$ a
point-finite open cover of $X$ such that $\#(\Cal G)$ is measure-free.
Suppose that $E\subseteq X$ is such that $E\cap G$ is
universally measurable for every $G\in\Cal G$.   Show that $E$ is
universally measurable.    (Compare 434Xe(iv).)
%438J-
%query:  do we need to know that \#(\Cal G) is measure-free?
% or that \Cal G is point-finite?  for the latter, yes - try
% X=\omega_1

\sqheader 438Xm Show
that for a metrizable space $X$, the following are equiveridical:
(i) $X$ is Borel-measure-compact;  (ii) $X$ is
Borel-measure-complete;  (iii) $X$ is measure-compact;
(iv) $w(X)$ is measure-free.
%438J

\spheader 438Xn Let $X$ be a topological space and $\Cal G$ a family of
open subsets of $X$.   Show that the following are equiveridical:  (i)
there is a $\sigma$-isolated family $\Cal A$ of sets, refining $\Cal G$,
such that $\bigcup\Cal A=\bigcup\Cal G$;  (ii) there is a sequence
$\Cal H_n$ of families of open sets, all refining $\Cal G$,
such that for every $x\in\bigcup\Cal G$ there is an $n\in\Bbb N$ such
that $\{H:x\in H\in\Cal H_n\}$ is finite and not empty.
%438K

\spheader 438Xo Let $Y$ be the Helly space.  (i) Show that $Y$ is a compact convex subset of $\BbbR^{[0,1]}$ with its usual topology.   (ii) Show that there is
a natural one-to-one correspondence between the split interval $I^{\|}$ and the set of extreme points of $Y$, matching $t^-\in I^{\|}$ with the function
$\chi\coint{0,t}$ and $t^+$ with $\chi[0,t]$.
(iii) Let $P_{\text{R}}$ be the set of Radon probability measures on
$I^{\|}$ with its narrow topology (437J).   Show that
there is a natural homeomorphism $\phi:P_{\text{R}}\to Y$
defined by setting $\phi(\mu)(t)=\mu[0^-,t^-]$ for
$\mu\in P_{\text{R}}$, $t\in[0,1]$.
%438R

\spheader 438Xp Give $\Bbb R$ the right-facing Sorgenfrey topology
(415Xc).   Show that any countable power of $\Bbb R$ is \hwtr.
%438R

\sqheader 438Xq Let $I^{\|}$ be the split interval.   Show that
$I^{\|}\times I^{\|}$ is a Radon space iff $\frak c$ is measure-free.
\Hint{$\{(\alpha^+,(1-\alpha)^+):\alpha\in[0,1]\}$ is a discrete Borel
subset of cardinal $\frak c$.}
%438T

\spheader 438Xr Give
$\Bbb R$ the right-facing Sorgenfrey topology.   Show that the
following are equiveridical:  (i) $\frak c$ is measure-free;  (ii)
$\BbbR^{\Bbb N}$, with the corresponding product topology, is
Borel-measure-complete;  (iii) $\BbbR^2$, with the product topology, is
Borel-measure-compact.   (Compare 439Q.)
%438T, 438Xq

\spheader 438Xs Suppose that $\frak c$ is measure-free.   Let
$X\subseteq\BbbR^{\Bbb R}$ be the set of functions of bounded variation
on $\Bbb R$, with the topology of pointwise convergence inherited from
the product
topology of $\BbbR^{\Bbb R}$.   Show that $X$ is a Radon space.
\Hint{$X$ is an F$_{\sigma}$ subset of the space $\tildeClll$ of 438Q.}
%438T

\leader{438Y}{Further exercises (a)}
%\spheader 438Ya
Let $(X,\Sigma,\mu)$ be a probability space and $\familyiI{E_i}$ a
point-finite family of measurable sets such that
$\nu J=\mu(\bigcup_{i\in J}E_i)$ is defined for every $J\subseteq I$.
Show directly that $\nu$ is a uniformly exhaustive Maharam submeasure on
$\Cal PI$, and use the Kalton-Roberts theorem to prove 438Ba.
%438B

\spheader 438Yb Suppose that $\frak c$ is measure-free, but that
$\kappa>\frak c$ is not measure-free.   Show that there is a
non-principal $\omega_1$-complete ultrafilter on $\kappa$.   \Hint{part
(b) of the proof of 451Q.}

\spheader 438Yc Show that if $X$ is a metrizable space and
$\min(\frak c,w(X))$ is measure-free, then every $\sigma$-finite
Borel measure on $X$ has countable Maharam type.
%438H

\spheader 438Yd Let $X$ be a metacompact T$_1$ space.   Show that $X$ is
Borel-measure-compact iff every closed discrete
subspace has measure-free cardinal.
%438J

\spheader 438Ye Let $X$ be a topological space such that every subspace
of $X$ is metacompact and has measure-free cellularity.   Show that
$X$ is Borel-measure-complete.
%438J

\spheader 438Yf Let $X$ be a normal metacompact Hausdorff space.   Show
that it is measure-compact iff every closed discrete subspace has
measure-free cardinal.
%438J

\spheader 438Yg In 438U, show that it would be enough to suppose that
every discrete subset of $X$ has measure-free cardinal.
%438U

\spheader 438Yh Suppose that $\frak c$ is measure-free.   Let $D$ be
any subset of $\Bbb R$ and $X\subseteq\BbbR^D$ the set of functions of
bounded variation on $D$, with the topology of pointwise convergence
inherited from the
product topology of $\BbbR^D$.   Show that $X$ is a Radon space.
%438T, 438Xs

\spheader 438Yi Let $X$ be a totally ordered set with its order
topology.   Show that if $c(X)$ is measure-free then every
$\sigma$-finite Borel measure on $X$ has countable Maharam type.
(Cf.\ 434Yo.)
%+

\spheader 438Yj Suppose that $X$ is a normal metacompact Hausdorff space
which is not realcompact.   Show that there are a closed discrete subset
$D$ of $X$ and a non-principal $\omega_1$-complete ultrafilter on $D$.
\Hint{in 435C, if we start with a $\{0,1\}$-valued Baire measure we obtain
a $\{0,1\}$-valued Borel measure;  in the proof of 438Ba,
if $\mu$ is $\{0,1\}$-valued then $\nu$ is $\{0,1\}$-valued.}

\spheader 438Yk Let $Z$ be a regular
Hausdorff space, $T$ a Dedekind complete totally
ordered space with least and greatest elements $a$, $b$,
and $x:T\to Z$ a function such that
$\lim_{s\uparrow t}x(s)$ and $\lim_{s\downarrow t}x(s)$ are defined in
$Z$ for every $t\in T$ (except $t=a$ in the first case and $t=b$ in the
second).   Show that $x[T]$ is relatively compact in $Z$.
%438Q out of order query

\spheader 438Yl\dvAnew{2009} Let $(X,\rho)$ be a metric space, and
$P_{\text{Bor}}$ the set of Borel probability measures on $X$.
For $\mu$, $\nu\in P_{\text{Bor}}$ set
$\bar\rho_{\text{KR}}(\mu,\nu)
=\sup\{|\int u\,d\mu-\int u\,d\nu|:u:X\to[-1,1]$ is $1$-Lipschitz$\}$.
(i) Show that $\bar\rho_{\text{KR}}$ is a metric on
$P_{\text{Bor}}$.   (ii) Let $\frak T_{\text{KR}}$ be the topology it
induces on $P_{\text{Bor}}$.   Show that $\frak T_{\text{KR}}$ is finer
than the narrow topology on $P_{\text{Bor}}$.   (iii) Show that the
following are equiveridical:  ($\alpha$) the narrow topology on
$P_{\text{Bor}}$ is metrizable;  ($\beta$) $\frak T_{\text{KR}}$ is the
narrow topology on $P_{\text{Bor}}$;  ($\gamma$) $w(X)$ is measure-free.
(Cf.\ 437Rg, 437Yp.)
}%end of exercises

\endnotes{
\Notesheader{438} Since the axiom `every cardinal is measure-free' is
admissible -- that is, will not lead to a paradox unless one is already
latent in the Zermelo-Fraenkel axioms for set theory -- it is tempting,
in the context of this section, to assume it;  so that `every complete
metric space is Radon' becomes a theorem, along with `every measurable
function from a quasi-Radon measure space to a metrizable space is
almost continuous' (438G), `$U^{\sim}_c=U^{\times}$ for every Dedekind
complete Riesz space $U$' (438Xd), `metacompact spaces are
Borel-measure-compact' (438J), `the sum of two measurable functions
from a complete probability space to a normed space is measurable'
(438Xk) and `the Helly space is Radon' (438T).   Undoubtedly the
consequent mathematical universe is tidier.   In my view, the tidiness
is the tidiness of poverty.   Apart
from anything else, it leads us to neglect such questions as `is every
measurable function from a Radon measure space to a metrizable space
almost continuous?', which have answers in ZFC (451T).

From the point of view of measure theory, the really interesting
question is whether $\frak c$ is measure-free.
It is not quite clear from the results above why this should be so;
438T is a very small part of the story.   There is a larger hint
in 438Ce-438Cf:  if $\frak c$ is measure-free, but $\kappa>\frak c$
is not measure-free, then the witnessing measures will be purely
atomic.   I will return to this point in \S543 of Volume 5.   For a general
exploration of universes in which $\frak c$ is {\it not} measure-free,
see \S544 and {\smc Fremlin 93}.   For fragments of what happens if we
suppose that we have an atom for a measure which witnesses that $\kappa$
is not measure-free, see 438Yb and the notes on normal filters in
4A1I-4A1L.  %4A1I 4A1J 4A1K 4A1L

There are many further applications of 438Q besides those in 438R and
438Xp-438Xs.  %438Xp 438Xr 438Xs
But the most obvious candidate, the space $C(\Bbb R)$ of
continuous real-valued functions on $\Bbb R$, although indeed it is a
Borel subset of the potentially Radon space of 438Q, is in fact Radon
whether or not $\frak c$ is measure-free (454Sa).   As soon as we
start using any such special axiom as `$\frak c=\omega_1$' or `$\frak c$
is measure-free', we must make a determined effort to check, through
such examples as 438Xq, that our new theorems do indeed depend on
something more than ZFC.
}%end of notes

\discrpage

