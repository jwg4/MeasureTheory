\frfilename{mt321.tex} 
\versiondate{3.1.11} 
\copyrightdate{1995} 
 
\def\chaptername{Measure algebras} 
\def\sectionname{Measure algebras} 
 
\newsection{321} 
 
I begin by defining `measure algebra' and relating this concept to the 
work of Chapter 31 and to the elementary properties of measure 
spaces. 
 
\leader{321A}{Definition} A {\bf measure algebra} is a pair  
$(\frak A,\bar\mu)$, where $\frak A$ is a Dedekind 
$\sigma$-complete Boolean algebra and $\bar\mu:\frak A\to[0,\infty]$ is 
a function such that 
 
\qquad $\bar\mu 0=0$; 
 
\qquad whenever $\sequencen{a_n}$ is a disjoint sequence in $\frak A$, 
$\bar\mu(\sup_{n\in\Bbb N}a_n)=\sum_{n=0}^{\infty}\bar\mu a_n$; 
 
\qquad $\bar\mu a>0$ whenever $a\in\frak A$ and $a\ne 0$. 
 
\leader{321B}{Elementary properties of measure algebras} 
\cmmnt{Corresponding 
to the most elementary properties of measure spaces (112C in Volume 1), 
we have the following basic properties of measure algebras.} 
Let $(\frak A,\bar\mu)$ be a measure algebra. 
 
\header{321Ba}{\bf (a)} If $a$, $b\in\frak A$ and $a\Bcap b=0$ then 
$\bar\mu(a\Bcup b)=\bar\mu a+\bar\mu b$.   \prooflet{\Prf\ Set $a_0=a$, 
$a_1=b$, 
$a_n=0$ for $n\ge 2$;  then 
 
\Centerline{$\bar\mu(a\Bcup b)=\bar\mu(\sup_{n\in\Bbb N}a_n) 
=\sum_{n=0}^{\infty}\bar\mu a_n=\bar\mu a+\bar\mu b$.   \Qed}} 
 
\header{321Bb}{\bf (b)} If $a$, $b\in\frak A$ and $a\Bsubseteq b$ then 
$\bar\mu a\le\bar\mu b$.   \prooflet{\Prf\ 
 
\Centerline{$\bar\mu a\le\bar\mu a+\bar\mu(b\Bsetminus a)=\bar\mu b$. 
\Qed}} 
 
\header{321Bc}{\bf (c)} For any $a$, $b\in\frak A$,  
$\bar\mu(a\Bcup b)\le\bar\mu a+\bar\mu b$.   \prooflet{\Prf\ 
 
\Centerline{$\bar\mu(a\Bcup b)=\bar\mu a+\bar\mu(b\Bsetminus 
a)\le\bar\mu a+\bar\mu b$. 
\Qed}} 
 
\header{321Bd}{\bf (d)} If $\sequencen{a_n}$ is any sequence in  
$\frak A$, then  
$\bar\mu(\sup_{n\in\Bbb N}a_n)\le\sum_{n=0}^{\infty}\bar\mu a_n$. 
\prooflet{\Prf\ For each $n$, set $b_n=a_n\Bsetminus\sup_{i<n}a_i$. 
Inducing on $n$, we see that $\sup_{i\le n}a_i=\sup_{i\le n}b_i$ for 
each $n$, so $\sup_{n\in\Bbb N}a_n=\sup_{n\in\Bbb N}b_n$ and 
 
\Centerline{$\bar\mu(\sup_{n\in\Bbb N}a_n)=\bar\mu(\sup_{n\in\Bbb N}b_n) 
=\sum_{n=0}^{\infty}\bar\mu b_n\le\sum_{n=0}^{\infty}\bar\mu a_n$} 
 
\noindent because $\sequencen{b_n}$ is disjoint.   \Qed} 
 
\header{321Be}{\bf (e)} If $\sequencen{a_n}$ is a non-decreasing 
sequence in $\frak A$, then $\bar\mu(\sup_{n\in\Bbb 
N}a_n)=\lim_{n\to\infty}\bar\mu a_n$.   \prooflet{\Prf\ Set $b_0=a_0$, 
$b_n=a_n\Bsetminus a_{n-1}$ for $n\ge 1$.   Then 
 
$$\eqalign{\bar\mu(\sup_{n\in\Bbb N}a_n) 
&=\bar\mu(\sup_{n\in\Bbb N}b_n) 
=\sum_{n=0}^{\infty}\bar\mu b_n\cr 
&=\lim_{n\to\infty}\sum_{i=0}^n\bar\mu b_i 
=\lim_{n\to\infty}\bar\mu(\sup_{i\le n}b_i) 
=\lim_{n\to\infty}\bar\mu a_n.\text{  \Qed}\cr}$$} 
 
\spheader 321Bf If $\sequencen{a_n}$ is a non-increasing sequence in 
$\frak A$ and $\inf_{n\in\Bbb N}\bar\mu a_n<\infty$, then  
$\bar\mu(\inf_{n\in\Bbb N}a_n)=\lim_{n\to\infty}\bar\mu a_n$.   
\prooflet{\Prf\ (Cf.\  
112Cf.)   Set $a=\inf_{n\in\Bbb N}a_n$.   Take $k\in\Bbb N$ such that  
$\bar\mu a_k<\infty$.   Set $b_n=a_k\Bsetminus a_n$ for $n\in\Bbb N$;  then  
$\sequencen{b_n}$ is  
non-decreasing and $\sup_{n\in\Bbb N}b_n=a_k\Bsetminus a$ (313Ab).   Because  
$\bar\mu a_k$ is finite, 
 
$$\eqalignno{\bar\mu a 
&=\bar\mu a_k-\bar\mu(a_k\Bsetminus a) 
=\bar\mu a_k-\lim_{n\to\infty}\bar\mu b_n\cr 
\displaycause{by (e) above} 
&=\lim_{n\to\infty}\bar\mu(a_k\Bsetminus b_n) 
=\lim_{n\to\infty}\bar\mu a_n.  \text{ \Qed}\cr}$$
} 
 
\leader{321C}{Proposition} Let $(\frak A,\bar\mu)$ be a measure algebra, 
and $A\subseteq\frak A$ a non-empty upwards-directed set.   If 
$\sup_{a\in A}\bar\mu a<\infty$, then $\sup A$ is defined in $\frak A$ 
and $\bar\mu(\sup A)=\sup_{a\in A}\bar\mu a$. 
 
\proof{ (Compare 215A.)   Set $\gamma=\sup_{a\in A}\bar\mu a$, and for 
each $n\in\Bbb N$ 
choose $a_n\in A$ such that $\bar\mu a_n\ge\gamma-2^{-n}$.   Next, 
choose $\sequencen{b_n}$ in $A$ such that 
$b_{n+1}\Bsupseteq b_n\Bcup a_n$ for 
each $n$, and set $b=\sup_{n\in\Bbb N}b_n$.   Then 
 
\Centerline{$\bar\mu b=\lim_{n\to\infty}\bar\mu b_n\le\gamma$, 
\quad$\bar\mu a_n\le\bar\mu b$ for every $n\in\Bbb N$,} 
 
\noindent so $\bar\mu b=\gamma$. 
 
If $a\in A$, then for every $n\in\Bbb N$ there is an $a_n'\in A$ such 
that $a\Bcup a_n\Bsubseteq a_n'$, so that 
 
\Centerline{$\bar\mu(a\Bsetminus b)\le\bar\mu(a\Bsetminus a_n) 
\le\bar\mu(a_n'\Bsetminus a_n) 
=\bar\mu a'_n-\bar\mu a_n\le\gamma-\bar\mu a_n\le 2^{-n}$.} 
 
\noindent This means that $\bar\mu(a\setminus b)=0$, so $a\Bsetminus 
b=0$ 
and $a\Bsubseteq b$.   Accordingly $b$ is an upper bound of $A$, and is 
therefore $\sup A$;  since we already know that $\bar\mu b=\gamma$, the 
proof is complete. 
}%end of proof of 321C 
 
\leader{321D}{Corollary} Let $(\frak A,\bar\mu)$ be a measure algebra 
and $A\subseteq\frak A$ a non-empty upwards-directed set.   If $\sup A$ 
is defined in $\frak A$, then $\bar\mu(\sup A)=\sup_{a\in A}\bar\mu a$. 
 
\proof{ If $\sup_{a\in A}\bar\mu a=\infty$, this is trivial;  otherwise 
it follows from 321C. 
}%end of proof of 321D 
 
\leader{321E}{Corollary} Let $(\frak A,\bar\mu)$ be a measure algebra 
and $A\subseteq\frak A$ a disjoint set.   If $\sup A$ is defined in 
$\frak A$, then $\bar\mu(\sup A)=\sum_{a\in A}\bar\mu a$. 
 
\proof{ If $A=\emptyset$ then $\sup A=0$ and the result is trivial. 
Otherwise, set $B=\{a_0\Bcup\ldots\Bcup a_n:a_0,\ldots,a_n\in A$ are 
distinct$\}$.   Then $B$ is upwards-directed, and 
$\sup_{b\in B}\bar\mu b=\sum_{a\in A}\bar\mu a$ because $A$ is disjoint. 
Also $B$ has the same upper bounds as $A$, so $\sup B=\sup A$ and 
 
\Centerline{$\bar\mu(\sup A)=\bar\mu(\sup B) 
=\sup_{b\in B}\bar\mu b=\sum_{a\in A}\bar\mu a$.} 
 
}%end of proof of 321E 
 
\leader{321F}{Corollary} Let $(\frak A,\bar\mu)$ be a measure algebra 
and $A\subseteq\frak A$ a non-empty downwards-directed set.   If 
$\inf_{a\in A}\bar\mu a<\infty$, then $\inf A$ is defined in $\frak A$ 
and $\bar\mu(\inf A)=\inf_{a\in A}\bar\mu a$. 
 
\proof{ Take $a_0\in A$ with $\bar\mu a_0<\infty$, and set 
$B=\{a_0\Bsetminus a:a\in A\}$.   Then $B$ is upwards-directed, and 
$\sup_{b\in B}\bar\mu b\le\bar\mu a_0<\infty$, so $\sup B$ is defined. 
Accordingly $\inf A=a_0\setminus\sup B$ is defined (313Aa), and 
 
$$\eqalign{\bar\mu(\inf A) 
&=\bar\mu a_0-\bar\mu(\sup B) 
=\bar\mu a_0-\sup_{b\in B}\bar\mu b\cr 
&=\inf_{b\in B}\bar\mu(a_0\Bsetminus b) 
=\inf_{a\in A}\bar\mu(a_0\Bcap a) 
=\inf_{a\in A}\bar\mu a.\cr}$$ 
}%end of proof of 321F 
 
\leader{321G}{Subalgebras} If $(\frak A,\bar\mu)$ is a measure algebra, 
and $\frak B$ is a $\sigma$-subalgebra of $\frak A$, then 
$(\frak B,\bar\mu\restrp\frak B)$ is a measure algebra. 
\prooflet{\Prf\ 
As remarked in 314Eb, $\frak B$ is Dedekind $\sigma$-complete.   If 
$\sequencen{b_n}$ is a disjoint sequence in $\frak B$, then the supremum 
$b=\sup_{n\in\Bbb N}b_n$ is the same whether taken in $\frak B$ or 
$\frak A$, so that we have $\bar\mu b=\sum_{n=0}^{\infty}\bar\mu b_n$. 
\Qed} 
 
\leader{321H}{The measure algebra of a measure 
\dvrocolon{space}}\cmmnt{ I introduce the 
abstract notion of `measure algebra' because I believe that this is the 
right language in which to formulate the questions addressed in this 
volume.   However it is very directly linked with the idea of  
`measure space', as the next two results show. 
 
\wheader{321H}{4}{2}{2}{72pt} 
 
\noindent}{\bf Theorem} Let $(X,\Sigma,\mu)$ be a measure space, and 
$\Cal N$ the null ideal of $\mu$.   Let $\frak A$ be the 
Boolean algebra quotient $\Sigma/\Sigma\cap\Cal N$.   Then we have a 
functional $\bar\mu:\frak A\to[0,\infty]$ defined by setting 
 
\Centerline{$\bar\mu E^{\ssbullet}=\mu E$ for every $E\in\Sigma$,} 
 
\noindent and $(\frak A,\bar\mu)$ is a measure algebra.  The canonical 
map $E\mapsto E^{\ssbullet}:\Sigma\to\frak A$ is sequentially 
order-continuous. 
 
\proof{{\bf (a)} By 314C, $\frak A$ is a Dedekind $\sigma$-complete 
Boolean algebra.   By 313Qb, $E\mapsto E^{\ssbullet}$ is sequentially 
order-continuous, because $\Sigma\cap\Cal N$ is a $\sigma$-ideal of 
$\Sigma$. 
 
\medskip 
 
{\bf (b)} If $E$, $F\in\Sigma$ and $E^{\ssbullet}=F^{\ssbullet}$ in 
$\frak A$, then $E\symmdiff F\in\Cal N$, so 
 
\Centerline{$\mu E\le\mu F+\mu(E\setminus F)=\mu F 
\le\mu E+\mu(F\setminus E)=\mu E$} 
 
\noindent and $\mu E=\mu F$.   Accordingly the given formula does indeed 
define a function $\bar\mu:\frak A\to[0,\infty]$. 
 
\medskip 
 
{\bf (c)} Now 
 
\Centerline{$\bar\mu 0=\bar\mu\emptyset^{\ssbullet}=\mu\emptyset=0$.} 
 
\noindent If $\sequencen{a_n}$ is a disjoint sequence in $\frak A$, 
choose for each $n\in\Bbb N$ an $E_n\in\Sigma$ such that 
$E_n^{\ssbullet}=a_n$.   Set $F_n=E_n\setminus\bigcup_{i<n}E_i$;  then 
 
\Centerline{$F_n^{\ssbullet} 
=E_n^{\ssbullet}\Bsetminus\sup_{i<n}E_i^{\ssbullet} 
=a_n\Bsetminus\sup_{i<n}a_i=a_n$} 
 
\noindent for each $n$, so $\bar\mu a_n=\mu F_n$ for each $n$. 
Now set $E=\bigcup_{n\in\Bbb N}E_n=\bigcup_{n\in\Bbb N}F_n$;  then 
$E^{\ssbullet}=\sup_{n\in\Bbb N}F_n^{\ssbullet}=\sup_{n\in\Bbb N}a_n$. 
So 
 
\Centerline{$\bar\mu(\sup_{n\in\Bbb N}a_n) 
=\mu E 
=\sum_{n=0}^{\infty}\mu F_n 
=\sum_{n=0}^{\infty}\bar\mu a_n$.} 
 
\noindent Finally, if $a\ne 0$, then there is an $E\in\Sigma$ such that 
$E^{\ssbullet}=a$, and $E\notin\Cal N$, so $\bar\mu a=\mu E>0$. 
Thus $(\frak A,\bar\mu)$ is a measure algebra. 
}%end of proof of 321H 
 
\leader{321I}{Definition} For any measure space $(X,\Sigma,\mu)$ I will 
call $(\frak A,\bar\mu)$, as constructed above, the {\bf measure 
algebra of} $(X,\Sigma,\mu)$. 
 
\leader{321J}{The Stone representation of a 
measure \dvrocolon{algebra}}\cmmnt{ Just as 
with Dedekind $\sigma$-complete Boolean algebras (314N), every measure 
algebra is obtainable from the construction above. 
 
\medskip 
 
\noindent}{\bf Theorem} Let $(\frak A,\bar\mu)$ be any measure algebra. 
Then it is isomorphic, as measure algebra, to the measure algebra of 
some measure space. 
 
\proof{{\bf (a)} 
We know from 314M that $\frak A$ is isomorphic, as Boolean algebra, to a 
quotient algebra $\Sigma/\Cal M$ where $\Sigma$ is a 
$\sigma$-algebra of subsets of the Stone space $Z$ of $\frak A$, and 
$\Cal M$ is the ideal of meager subsets of $Z$.   Let 
$\pi:\Sigma/\Cal M\to\frak A$ be the canonical isomorphism, and set 
$\theta E=\pi E^{\ssbullet}$ for each $E\in\Sigma$;  then 
$\theta:\Sigma\to\frak A$ is a sequentially 
order-continuous surjective Boolean homomorphism with kernel $\Cal M$. 
 
\medskip 
 
{\bf (b)} For $E\in\Sigma$, set 
 
\Centerline{$\nu E=\bar\mu(\theta E)$.} 
 
\noindent Then $(Z,\Sigma,\nu)$ is a measure space.   \Prf\ (i) We know 
already that $\Sigma$ is a $\sigma$-algebra of subsets of $Z$.   (ii) 
 
\Centerline{$\nu\emptyset=\bar\mu(\theta\emptyset)=\bar\mu 0=0$.} 
 
\noindent (iii) If $\sequencen{E_n}$ is a disjoint sequence in $\Sigma$, 
then (because $\theta$ is a Boolean homomorphism) 
$\sequencen{\theta E_n}$ 
is a disjoint sequence in $\frak A$ and (because $\theta$ is 
sequentially order-continuous) 
$\theta(\bigcup_{n\in\Bbb N}E_n)=\sup_{n\in\Bbb N}\theta E_n$;  so 
 
\Centerline{$\nu(\bigcup_{n\in\Bbb N}E_n) 
=\bar\mu(\sup_{n\in\Bbb N}\theta E_n) 
=\sum_{n=0}^{\infty}\bar\mu(\theta E_n) 
=\sum_{n=0}^{\infty}\nu E_n$.  \Qed} 
 
\medskip 
 
{\bf (c)} For $E\in\Sigma$, 
 
\Centerline{$\nu E=0 
\iff\bar\mu(\theta E)=0 
\iff\theta E=0 
\iff E\in\Cal M$.} 
 
\noindent So the measure algebra of $(Z,\Sigma,\nu)$ is just 
$\Sigma/\Cal M$, with 
 
\Centerline{$\bar\nu E^{\ssbullet}=\nu E 
=\bar\mu(\theta E)=\bar\mu(\pi E^{\ssbullet})$} 
 
\noindent for every $E\in\Sigma$.   Thus the Boolean algebra isomorphism 
$\pi$ is also an isomorphism between the measure algebras 
$(\Sigma/\Cal M,\bar\nu)$ and $(\frak A,\bar\mu)$, and 
$(\frak A,\bar\mu)$ is represented in the required form. 
}%end of proof of 321J 
 
\leader{321K}{Definition} I will call the measure space $(Z,\Sigma,\nu)$ 
constructed in the proof of 321J 
the {\bf Stone space} of the measure algebra $(\frak A,\bar\mu)$. 
 
\cmmnt{For later reference, I repeat the description of this space as 
developed in 311E, 311I, 314M and 321J.}   $Z$ 
is a compact Hausdorff space, being the Stone space of $\frak A$. 
$\frak A$ can be identified with the algebra of open-and-closed sets in 
$Z$.   The null ideal of $\nu$ coincides with the ideal of 
meager subsets of $Z$;  \cmmnt{in particular,} $\nu$ is complete. 
The measurable sets are precisely those 
expressible in the form $E=\widehat{a}\symmdiff M$ where $a\in\frak A$, 
$\widehat{a}\subseteq Z$ is the corresponding open-and-closed set, and 
$M$ is meager;  in this case $\nu E=\bar\mu a$ and 
$a\cmmnt{\mskip5mu=\theta E}$ is the 
member of $\frak A$ corresponding to $E$. 
 
\cmmnt{For the most important classes of measure algebras, more can be 
said;  see 322O {\it et seq.} below.} 
 
\exercises{ 
\leader{321X}{Basic exercises (a)} Let $(\frak A,\bar\mu)$ be a measure 
algebra, and $a\in\frak A$;  write $\frak A_a$ for the principal ideal of 
$\frak A$ generated by $a$.   Show that 
$(\frak A_a,\bar\mu\restrp\frak A_a)$ is a measure algebra. 
 
\spheader 321Xb Let $(X,\Sigma,\bar\mu)$ be a measure space, and 
$\frak A$ its measure algebra.   (i) Show that if $\Tau$ is a 
$\sigma$-subalgebra of $\Sigma$, then $\{E^{\ssbullet}:E\in\Tau\}$ is a 
$\sigma$-subalgebra of $\frak A$.   (ii) Show that if 
$\frak B$ is a $\sigma$-subalgebra of $\frak A$ then 
$\{E:E\in\Sigma,\,E^{\ssbullet}\in\frak B\}$ is a 
$\sigma$-subalgebra of $\Sigma$. 
 
\leader{321Y}{Further exercises (a)}  
%\spheader 321Ya 
Let $(\frak A,\bar\mu)$ be a 
measure algebra, and $I\normalsubgroup\frak A$ a $\sigma$-ideal.   For 
$u\in\frak A/I$ set  
$\bar\nu u=\inf\{\bar\mu a:a\in\frak A,\,a^{\ssbullet}=u\}$.   
(i) Show that the infimum is always attained.
(ii) Show that $(\frak A/I,\bar\nu)$ is a measure algebra. 
}%end of exercises 
 
\cmmnt{ 
\Notesheader{321} The idea behind taking the quotient $\Sigma/\Cal N$, 
where $\Sigma$ is the algebra of measurable sets and $\Cal N$ is the null ideal, is just that if negligible sets can be ignored 
-- as is the case for a very large proportion of the results of measure 
theory -- then two measurable sets can be counted as virtually the same 
if they differ by a negligible set, that is, if they represent the same 
member of the measure algebra.   The definition in 321A is designed to 
be 
an exact characterization of these quotient algebras, taking into 
account the measures with which they are endowed.   In the course of the 
present chapter I will work through many of the basic ideas dealt with 
in Volumes 1 and 2 to show how they can be translated into theorems 
about measure algebras, as I have done in 321B-321F.   It is worth 
checking these correspondences carefully, because some of the ideas 
mutate significantly in translation.   In measure algebras, it becomes 
sensible to take seriously the suprema and infima of uncountable sets 
(see 321C-321F). 
 
I should perhaps remark that while the Stone representation (321J-321K) 
is significant, it is not the most important method of representing 
measure algebras, which is surely Maharam's theorem, to be dealt with in 
the next chapter.   Nevertheless, the Stone representation is a 
canonical one, and will 
appear at each point that we meet a new construction involving measure 
algebras, just as the ordinary Stone representation of Boolean algebras 
can be expected to throw 
light on any aspect of Boolean algebra. 
}%end of notes 
 
\discrpage 
 
