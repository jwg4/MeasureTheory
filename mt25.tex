\frfilename{mt25.tex}
\versiondate{31.5.03}

\def\chaptername{Product Measures}
\newchapter{25}
\def\chaptername{Product measures}

I come now to another chapter on `pure' measure theory, discussing a
fundamental construction -- or, as you may prefer to consider it, two
constructions, since the problems involved in forming the product of
two arbitrary measure spaces (\S251) are rather different
from those arising in the product of arbitrarily many probability spaces
($\S$254).   This work is going to stretch our technique to the utmost,
for while the fundamental theorems to which we are moving are natural
aims, the proofs are lengthy and there are many
pitfalls beside the true paths.

\allowmorestretch{500}{
The central idea is that of `repeated integration'.   You have
probably already seen formulae of the type\penalty-100 
`$\iint f(x,y)dxdy$' used
to calculate the integral of a function of two real variables over a
region in the plane.   One of the basic techniques of advanced calculus
is reversing the order of integration;  for instance, we expect
$\int_0^1(\int_y^1f(x,y)dx)dy$ to be equal to
$\int_0^1(\int_0^xf(x,y)dy)dx$.   As I have developed the subject, we
already have a third calculation to compare with these two:  $\int_Df$,
where $D=\{(x,y):0\le y\le x\le 1\}$ and the integral is taken with
respect to Lebesgue measure on the plane.   The first two sections of
this chapter are devoted to an analysis of the relationship between
one- and two-dimensional Lebesgue measure which makes these operations
valid -- some of the time;  part of the work has to be devoted to a
careful description of the exact conditions which must be imposed on $f$
and $D$ if we are to be safe.
}%end of allowmorestretch

Repeated integration, in one form or another, appears everywhere in
measure theory, and it is therefore necessary sooner or later to develop
the most general possible expression of the idea.   The standard method
is through the theory of products of general measure spaces.   Given
measure spaces $(X,\Sigma,\mu)$ and $(Y,\Tau,\nu)$, the aim is to find a
measure $\lambda$ on $X\times Y$ which will, at least, give the right
measure $\mu E\cdot\nu F$ to a `rectangle' $E\times F$ where
$E\in\Sigma$
and $F\in\Tau$.   It turns out that there are already difficulties in
deciding what `the' product measure is, and to do the job properly I
find I need, even at this stage, to describe two related but
distinguishable constructions.   These constructions and their
elementary properties take up the whole of \S251.   In \S252 I turn to
integration over the product, with Fubini's and Tonelli's theorems
relating $\int fd\lambda$ with $\iint f(x,y)\mu(dx)\nu(dy)$.   Because
the construction of $\lambda$ is symmetric between the two factors, this
automatically provides theorems relating $\iint f(x,y)\mu(dx)\nu(dy)$
with $\iint f(x,y)\nu(dy)\mu(dx)$.   \S253 looks at the space
$L^1(\lambda)$ and its relationship with $L^1(\mu)$ and $L^1(\nu)$.

For general measure spaces, there are obstacles in the way of forming an
infinite product;  to start with, if $\sequencen{(X_n,\mu_n)}$ is a
sequence of measure spaces, then a product measure $\lambda$ on
$X=\prod_{n\in\Bbb N}X_n$ ought to set $\lambda
X=\prod_{n=0}^{\infty}\mu_nX_n$, and there is no guarantee that the
product will converge, or behave well when it does.   But for
probability spaces, when $\mu_nX_n=1$ for every $n$, this problem at
least evaporates.   It is possible to define the product of any family
of probability spaces;  this is the burden of \S254.

I end the chapter with three sections which are a preparation for
Chapters
27 and 28, but are also important in their own right as an investigation
of the way in which the group structure of $\BbbR^r$ interacts with
Lebesgue and other measures.   \S255 deals with the `convolution'
$f*g$ of two functions, where $(f*g)(x)=\int f(y)g(x-y)dy$ (the
integration being with respect to Lebesgue measure).   In \S257 I show
that some of the same ideas, suitably transformed, can be used to
describe a convolution $\nu_1*\nu_2$ of two measures on $\BbbR^r$;  in
preparation for this I include a section on Radon measures on $\BbbR^r$
(\S256).

\discrpage

