\frfilename{mt255.tex}
\versiondate{3.7.08}
\copyrightdate{2000}

\def\chaptername{Product measures}
\def\sectionname{Convolutions of functions}

\newsection{255}

I devote a section to a construction which is of great importance -- and
will in particular be very useful in Chapters 27 and 28 -- and may also
be regarded as a series of exercises on the work so far.

I find it difficult to know how much repetition to indulge in in this
section, because the natural unified expression of the ideas is in the
theory of topological
groups, and I do not think we are yet ready for the general theory (I
will come to it in Chapter 44 in Volume 4).   The groups we need for this volume are

\qquad $\Bbb R$;

\qquad $\BbbR^r$, for $r\ge 2$;

\qquad $S^1=\{z:z\in\Bbb C,\,|z|=1\}$, the `circle group';

\qquad $\Bbb Z$, the group of integers.

\noindent All the ideas already appear in the theory of convolutions on
$\Bbb R$, and I will therefore present this material in relatively
detailed form, before sketching the forms appropriate to the groups
$\BbbR^r$ and $S^1$ (or $\ocint{-\pi,\pi}$);  $\Bbb Z$ can I think be
safely left to the exercises.

\leader{255A}{}\cmmnt{ This being a book on measure theory, it is
perhaps appropriate for me to emphasize, as the basis of the theory of
convolutions, certain measure space isomorphisms.

\medskip

\noindent}{\bf Theorem} Let $\mu$ be Lebesgue measure on $\Bbb R$ and
$\mu_2$ Lebesgue measure on $\BbbR^2$;  write $\Sigma$, $\Sigma_2$ for
their domains.

(a) For any $a\in\Bbb R$, the map $x\mapsto a+x:\Bbb R\to\Bbb R$ is a
measure space automorphism of $(\Bbb R,\Sigma,\mu)$.

(b) The map $x\mapsto -x:\Bbb R\to\Bbb R$ is a measure space
automorphism of $(\Bbb R,\Sigma,\mu)$.

(c) For any $a\in\Bbb R$, the map $x\mapsto a-x:\Bbb R\to\Bbb R$ is a
measure space automorphism of $(\Bbb R,\Sigma,\mu)$.

(d) The map $(x,y)\mapsto(x+y,y):\BbbR^2\to\BbbR^2$ is a measure space
automorphism of $(\BbbR^2,\Sigma_2,\mu_2)$.

(e) The map $(x,y)\mapsto(x-y,y):\BbbR^2\to\BbbR^2$ is a measure space
automorphism of $(\BbbR^2,\Sigma_2,\mu_2)$.

\cmmnt{\medskip

\noindent{\bf Remark} I ought to remark that (b), (d) and (e) may be
regarded as simple
special cases of Theorem 263A in the next chapter.   I nevertheless feel
that it is worth writing out separate proofs here, partly because the
general case of linear operators dealt with in 263A requires some extra
machinery not needed here, but more because the result here has nothing
to do with the {\it linear} structure of $\Bbb R$ and $\BbbR^2$;  it is
exclusively dependent on the {\it group} structure of $\Bbb R$, together
with the links between its topology and measure, and the arguments I
give now are adaptable to the proper generalizations to abelian
topological groups.

}%end of comment

\proof{{\bf (a)} This is just the translation-invariance of Lebesgue
measure, dealt with in \S134.   There I showed that if $E\in\Sigma$
then $E+a\in\Sigma$ and $\mu(E+a)=\mu E$ (134Ab);  that is, writing
$\phi(x)=x+a$, $\mu(\phi[E])$ exists and is equal to $\mu E$ for every
$E\in\Sigma$.   But of course we also have

\Centerline{$\mu(\phi^{-1}[E])=\mu(E+(-a))=\mu E$}

\noindent for every $E\in\Sigma$, so $\phi$ is an automorphism.

\medskip

{\bf (b)} The point is that $\mu^*(A)=\mu^*(-A)$ for every
$A\subseteq\Bbb R$.   \Prf\ (I follow the definitions of Volume 1.)   If
$\epsilon>0$, there is a sequence $\sequencen{I_n}$ of half-open
intervals covering $A$ with $\sum_{n=0}^{\infty}\mu
I_n\le\mu^*A+\epsilon$.   Now $-A\subseteq\bigcup_{n\in\Bbb N}(-I_n)$.
But if $I_n=\coint{a_n,b_n}$ then $-I_n=\ocint{-b_n,a_n}$, so

\Centerline{$\mu^*(-A)\le\sum_{n=0}^{\infty}\mu(-I_n)
=\sum_{n=0}^{\infty}\max(0,-a_n-(-b_n))=\sum_{n=0}^{\infty}\mu I_n
\le\mu^*A+\epsilon$.}

\noindent As $\epsilon$ is arbitrary, $\mu^*(-A)\le\mu^*A$.   Also of
course $\mu^*A\le\mu^*(-(-A))=\mu^*A$, so $\mu^*(-A)=\mu^*A$.   \Qed

This means that, setting $\phi(x)=-x$ this time, $\phi$ is an
automorphism of the structure $(\Bbb R,\mu^*)$.   But since $\mu$ is
defined from $\mu^*$ by the abstract procedure of \Caratheodory's
method, $\phi$ must also be an automorphism of the structure $(\Bbb
R,\Sigma,\mu)$.

\medskip

{\bf (c)} Put (a) and (b) together;  $x\mapsto a-x$ is the composition
of the automorphisms $x\mapsto -x$ and $x\mapsto a+x$, and the
composition of automorphisms is surely an automorphism.

\medskip

{\bf (d)(i)} Write $\Tau$ for the set
$\{E:E\in\Sigma_2,\,\phi[E]\in\Sigma_2\}$, where this time
$\phi(x,y)=(x+y,y)$ for $x$, $y\in\Bbb R$, so that 
$\phi:\BbbR^2\to\Bbb R^2$ is a permutation.   
Then $\Tau$ is a $\sigma$-algebra, being the
intersection of the $\sigma$-algebras $\Sigma_2$ and
$\{E:\phi[E]\in\Sigma_2\}=\{\phi^{-1}[F]:F\in\Sigma_2\}$.   Moreover,
$\mu_2E=\mu_2(\phi[E])$ for every $E\in\Tau$.   \Prf\  By 252D,
we have

\Centerline{$\mu_2E=\int\mu\{x:(x,y)\in E\}\mu(dy)$.}

\noindent But applying the same result to $\phi[E]$ we have

$$\eqalignno{\mu_2\phi[E]
&=\int\mu\{x:(x,y)\in\phi[E]\}\mu(dy)
=\int\mu\{x:(x-y,y)\in E\}\mu(dy)\cr
&=\int\mu(E^{-1}[\{y\}]+y)\mu(dy)
=\int\mu E^{-1}[\{y\}]\mu(dy)\cr
\noalign{\noindent (because Lebesgue measure is translation-invariant)}
&=\mu_2E. \text{  \Qed}\cr}$$

\medskip

\quad{\bf (ii)} Now $\phi$ and $\phi^{-1}$ are clearly continuous, so
that $\phi[G]$ is open, and therefore measurable, for every open $G$;
consequently all open sets must belong to $\Tau$.   Because $\Tau$ is a
$\sigma$-algebra, it contains all Borel sets.   Now let $E$ be any
measurable set.   Then there are Borel sets $H_1$, $H_2$ such that
$H_1\subseteq E\subseteq H_2$ and $\mu_2(H_2\setminus H_1)=0$ (134Fb).
We have $\phi[H_1]\subseteq\phi[E]\subseteq\phi[H_2]$ and

\Centerline{$\mu(\phi[H_2]\setminus\phi[H_1])
=\mu\phi[H_2\setminus H_1]=\mu(H_2\setminus H_1)=0$.}

\noindent Thus $\phi[E]\setminus\phi[H_1]$ must be negligible, therefore
measurable, and $\phi[E]=\phi[H_1]\cup(\phi[E]\setminus\phi[H_1])$ is
measurable.   This shows that $\phi[E]$ is measurable whenever $E$ is.

\medskip

\quad{\bf (iii)} Repeating the same arguments with $-y$ in the place of
$y$, we see that $\phi^{-1}[E]$ is measurable, and
$\mu_2\phi^{-1}[E]=\mu_2E$, for every $E\in\Sigma_2$.
So $\phi$ is an automorphism of the structure $(\BbbR^2,\Sigma_2,\mu_2)$.

\medskip

{\bf (e)} Of course this is an immediate corollary either of the proof
of (d) or of (d) itself as stated, since $(x,y)\mapsto(x-y,y)$ is just
the inverse of $(x,y)\mapsto(x+y,y)$.
}%end of proof of 255A

\leader{255B}{Corollary} (a) If $a\in\Bbb R$, then for any
complex-valued function $f$ defined on a subset of $\Bbb R$

\Centerline{$\int f(x)dx
=\int f(a+x)dx=\int f(-x)dx=\int f(a-x)dx$}

\noindent in the sense that if one of the integrals exists so do the
others, and they are then all equal.

(b) If $f$ is a complex-valued function defined on a subset of
$\BbbR^2$, then

\Centerline{$\int f(x+y,y)d(x,y)=\int f(x-y,y)d(x,y)
=\int f(x,y)d(x,y)$}

\noindent in the sense that if one of the integrals exists and is finite
so does the other, and they are then equal.

\cmmnt{
\leader{255C}{Remarks (a)}  I am not sure whether it ought to be
`obvious' that if $(X,\Sigma,\mu)$, $(Y,\Tau,\nu)$ are measure spaces
and $\phi:X\to Y$ is an isomorphism, then for any function
$f$ defined on a subset of $Y$

\Centerline{$\int f(\phi(x))\mu(dx)=\int f(y)\nu(dy)$}

\noindent in the sense that if one is defined so is the other, and they
are then equal.   If it is obvious then the obviousness must be
contingent on the nature of the definition of integration:
integrability with respect to the measure $\mu$ is something which
depends on the structure $(X,\Sigma,\mu)$ and on no other properties of
$X$.   If it is not obvious then it is an easy deduction from Theorem
235A above, applied in turn to $\phi$ and $\phi^{-1}$ and to the real
and imaginary parts of $f$.   In any case the isomorphisms of 255A are
just those needed to prove 255B.

\header{255Cb}{\bf (b)} Note that in 255Bb I write $\int f(x,y)d(x,y)$
to emphasize that I am considering the integral of $f$ with respect to
two-dimensional Lebesgue measure.   The fact that

\Centerline{$\int\bigl(\int f(x,y)dx\bigr)dy
=\int\bigl(\int f(x+y,y)dx\bigr)dy=\int\bigl(\int f(x-y,y)dx\bigr)dy$}

\noindent is actually easier, being an immediate consequence of the
equality $\int f(a+x)dx=\int f(x)dx$.   But applications of this result
often depend essentially on the fact that the functions
$(x,y)\mapsto f(x+y,y)$, $(x,y)\mapsto f(x-y,y)$ are measurable as
functions of two variables.

\header{255Cc}{\bf (c)} I have moved directly to complex-valued
functions because these are necessary for the applications in Chapter
28.   If however
they give you any discomfort, either technically or aesthetically, all
the measure-theoretic ideas of this section are already to be found in
the real case, and you may wish at first to read it as if only real
numbers were involved.
}%end of comment

\leader{255D}{}\cmmnt{ A further corollary of 255A will be useful.

\medskip

\noindent}{\bf Corollary} Let $f$ be a complex-valued function defined
on a subset of $\Bbb R$.

(a) If $f$ is measurable, then the functions $(x,y)\mapsto f(x+y)$,
$(x,y)\mapsto f(x-y)$ are measurable.

(b) If $f$ is defined almost everywhere in $\Bbb R$, then the functions
$(x,y)\mapsto f(x+y)$, $(x,y)\mapsto f(x-y)$ are defined almost
everywhere in $\BbbR^2$.

\proof{ Writing $g_1(x,y)=f(x+y)$, $g_2(x,y)=f(x-y)$ whenever these are
defined, we have

\Centerline{$g_1(x,y)=(f\otimes\chi\Bbb R)(\phi(x,y))$,
\quad $g_2(x,y)=(f\otimes\chi\Bbb R)(\phi^{-1}(x,y))$,}

\noindent writing $\phi(x,y)=(x+y,y)$ as in 255B(d-e), and
$(f\otimes\chi\Bbb R)(x,y)=f(x)$, following the notation of 253B.   By
253C, $f\otimes\chi\Bbb R$ is measurable if $f$ is, and defined
almost everywhere if $f$ is.   Because $\phi$ is a measure space
automorphism, $(f\otimes\chi\Bbb R)\phi=g_1$ and
$(f\otimes\chi\Bbb R)\phi^{-1}=g_2$ are measurable, or defined almost
everywhere, if $f$ is.
}%end of proof of 255D

\vleader{48pt}{255E}{The basic formula} Let $f$ and $g$ be measurable
complex-valued
functions defined almost everywhere in $\Bbb R$.   Write $f*g$ for the
function defined by the formula

\Centerline{$(f*g)(x)=\int f(x-y)g(y)dy$}

\noindent whenever the integral exists\cmmnt{ (with respect to
Lebesgue measure, naturally)} as a complex number.   Then $f*g$ is the
{\bf convolution} of the functions $f$ and $g$.
\cmmnt{

Observe that} $\dom(|f|*|g|)=\dom(f*g)$, and\cmmnt{ that}
$|f*g|\le |f|*|g|$ everywhere on their common domain, for all $f$ and
$g$.

\cmmnt{\medskip

\noindent{\bf Remark} Note that I am here prepared to contemplate the
convolution of $f$ and $g$ for arbitrary members of
$\eusm L^0_{\Bbb C}$, the space of
almost-everywhere-defined measurable complex-valued functions, even
though the domain of $f*g$ may be empty.   }%end of comment

\leader{255F}{Elementary properties (a)} Because integration is linear,
we surely have


\Centerline{$((f_1+f_2)*g)(x)=(f_1*g)(x)+(f_2*g)(x)$,}

\Centerline{$(f*(g_1+g_2))(x)=(f*g_1)(x)+(f*g_2)(x)$,}

\Centerline{$(cf*g)(x)=(f*cg)(x)=c(f*g)(x)$}

\noindent whenever the right-hand sides of the formulae are defined.

\spheader 255Fb If $f$ and $g$ are measurable complex-valued
functions defined almost everywhere in $\Bbb R$, then
$f*g=g*f$\cmmnt{, in the
strict sense that they have the same domain and the same value at each
point of that common domain}.

\prooflet{\Prf\ Take $x\in\Bbb R$ and apply 255Ba to see that

$$\eqalign{(f*g)(x)
&=\int f(x-y)g(y)dy
=\int f(x-(x-y))g(x-y)dy\cr
&=\int f(y)g(x-y)dy
=(g*f)(x)\cr}$$

\noindent if either is defined.   \Qed}

\header{255Fc}{\bf (c)} If $f_1$, $f_2$, $g_1$, $g_2$ are measurable
complex-valued functions defined almost everywhere in $\Bbb R$,
$f_1\eae f_2$ and $g_1\eae g_2$, then $f_1*g_1=f_2*g_2$.
\prooflet{\Prf\ For every $x\in\Bbb R$ we
shall have $f_1(x-y)=f_2(x-y)$ for almost every $y\in\Bbb R$, by 255Ac.
Consequently $f_1(x-y)g_1(y)=f_2(x-y)g_2(y)$ for almost every $y$,
and $(f_1*g_1)(x)=(f_2*g_2)(x)$ in the sense that if one of these is
defined so is the other, and they are then equal.\ \Qed}

It follows that if $u$, $v\in L^0_{\Bbb C}$, then we have a function
$\theta(u,v)$
which is equal to $f*g$ whenever $f$, $g\in\eusm L^0_{\Bbb C}$
are such that $f^{\ssbullet}=u$ and $g^{\ssbullet}=u$.
\cmmnt{Observe that $\theta(u,v)=\theta(v,u)$, and
that $\theta(u_1+u_2,v)$ extends $\theta(u_1,v)+\theta(u_2,v)$,
$\theta(cu,v)$ extends $c\theta(u,v)$ for all $u$, $u_1$, $u_2$,
$v\in L^0_{\Bbb C}$ and $c\in\Bbb C$.}

\leader{255G}{}\cmmnt{ I have grouped 255Fa-255Fc together because
they depend
only on ideas up to and including 255Ac and 255Ba.   Using the second
halves of 255A and 255B we get much deeper.   I begin with what seems to
be the fundamental result.

\medskip

\noindent}{\bf Theorem} Let $f$, $g$ and $h$ be measurable
complex-valued functions defined almost everywhere in $\Bbb R$.

(a) Suppose that $\int h(x+y)f(x)g(y)d(x,y)$ exists in $\Bbb C$.
Then

$$\eqalign{\int h(x)(f*g)(x)dx
&=\int h(x+y)f(x)g(y)d(x,y)\cr
&=\iint h(x+y)f(x)g(y)dxdy
=\iint h(x+y)f(x)g(y)dydx\cr}$$

\noindent provided
that in the expression $h(x)(f*g)(x)$ we interpret the product as $0$ if
$h(x)=0$ and $(f*g)(x)$ is undefined.

(b) If, on a similar interpretation of $|h(x)|(|f|*|g|)(x)$, the integral
$\int |h(x)|(|f|*|g|)(x)dx$ is finite, then
$\int h(x+y)f(x)g(y)d(x,y)$ exists in $\Bbb C$.

\proof{ Consider the functions

\Centerline{$k_1(x,y)=h(x)f(x-y)g(y)$,\quad $k_2(x,y)=h(x+y)f(x)g(y)$}

\noindent wherever these are defined.   255D tells us that $k_1$ and
$k_2$ are measurable and defined almost everywhere.   Now setting
$\phi(x,y)=(x+y,y)$, we have $k_2=k_1\phi$, so that

\Centerline{$\int k_1(x,y)d(x,y)=\int k_2(x,y)d(x,y)$}

\noindent if either exists, by 255Bb.

If

\Centerline{$\int h(x+y)f(x)g(y)d(x,y)=\int k_2$}

\noindent exists, then by Fubini's theorem we have

\Centerline{$\int k_2=\int k_1(x,y)d(x,y)
=\int(\int h(x)f(x-y)g(y)dy)dx$}

\noindent so $\int h(x)f(x-y)g(y)dy$ exists almost everywhere, that is,
$(f*g)(x)$ exists for almost every $x$ such that $h(x)\ne 0$;  on the
interpretation I am using here, $h(x)(f*g)(x)$ exists almost everywhere,
and

$$\eqalign{\int h(x)(f*g)(x)dx
&=\int\bigl(\int h(x)f(x-y)g(y)dy\bigr)dx
=\int k_1\cr
&=\int k_2
=\int h(x+y)f(x)g(y)d(x,y)\cr
&=\iint h(x+y)f(x)g(y)dxdy
=\iint h(x+y)f(x)g(y)dydx\cr}$$

\noindent by Fubini's theorem again.

If (on the same interpretation) $|h|\times(|f|*|g|)$ is integrable,

\Centerline{$|k_1(x,y)|=|h(x)||f(x-y)||g(y)|$}

\noindent is measurable, and

\Centerline{$\iint|h(x)||f(x-y)||g(y)|dydx
=\int|h(x)|(|f|*|g|)(x)dx$}

\noindent is finite, so by Tonelli's theorem (252G, 252H) $k_1$ and
$k_2$ are integrable.
}%end of proof of 255G

\leader{255H}{}\cmmnt{ Certain standard results are now easy.

\wheader{255H}{4}{2}{2}{24pt}

\noindent}{\bf Corollary} If $f$, $g$ are complex-valued
functions which are integrable over $\Bbb R$, then $f*g$ is integrable,
with

\Centerline{$\int f*g =\int f\int g$,
\quad$\int|f*g|\le\int|f|\int|g|$.}

\proof{ In 255G, set $h(x)=1$ for every $x\in\Bbb R$;  then

\Centerline{$\int h(x+y)f(x)g(y)d(x,y)
=\int f(x)g(y)d(x,y)=\int f\int g$}

\noindent by 253D, so

\Centerline{$\int f*g
=\int h(x)(f*g)(x)dx=\int h(x+y)f(x)g(y)d(x,y)=\int f\int g$,}

\noindent as claimed.   Now

\Centerline{$\int|f*g|\le\int|f|*|g|=\int|f|\int|g|$.}
}%\end of proof of 255H

\leader{255I}{Corollary} For any measurable complex-valued functions
$f$, $g$ defined almost everywhere in $\Bbb R$, $f*g$ is measurable and
has measurable domain.

\proof{ Set $f_n(x)=f(x)$ if $x\in\dom f$, $|x|\le n$ and $|f(x)|\le n$,
and $0$ elsewhere in $\Bbb R$;  define $g_n$ similarly from $g$.   Then
$f_n$ and $g_n$ are integrable, $|f_n|\le|f|$ and $|g_n|\le|g|$ almost
everywhere, $f\eae\lim_{n\to\infty}f_n$ and $g\eae\lim_{n\to\infty}g_n$.
Consequently, by Lebesgue's Dominated Convergence Theorem,

$$\eqalign{(f*g)(x)
&=\int f(x-y)g(y)dy
=\int \lim_{n\to\infty}f_n(x-y)g_n(y)dy\cr
&=\lim_{n\to\infty}\int f_n(x-y)g_n(y)dy
=\lim_{n\to\infty}(f_n*g_n)(x)\cr}$$

\noindent for every $x\in\dom f*g$.   But $f_n*g_n$ is integrable,
therefore measurable, for every $n$, so that $f*g$ must be measurable.

As for the domain of $f*g$,

$$\eqalign{x\in\dom(f*g)
&\iff\int f(x-y)g(y)dy\text{ is defined in }\Bbb C\cr
&\iff\int|f(x-y)||g(y)|dy\text{ is defined in }\Bbb R\cr
&\iff\int|f_n(x-y)||g_n(y)|dy\text{ is defined in }\Bbb R
\text{ for every }n\cr
&\qquad\qquad\text{and }
\sup_{n\in\Bbb N}\int|f_n(x-y)||g_n(y)|dy<\infty.\cr
}$$

\noindent Because every $|f_n|*|g_n|$ is integrable, therefore
measurable and with measurable domain,

\Centerline{$\dom(f*g)=\{x:x\in\bigcap_{n\in\Bbb N}\dom(|f_n|*|g_n|),\,
\sup_{n\in\Bbb N}(|f_n|*|g_n|)(x)<\infty\}$}

\noindent is measurable.
}%end of proof of 255I

\leader{255J}{Theorem} Let $f$, $g$ and $h$ be complex-valued measurable
functions, defined almost everywhere in $\Bbb R$, such that $f*g$ and $g*h$
are defined a.e.   Suppose that
$x\in\Bbb R$ is such that one of $(|f|*(|g|*|h|))(x)$,
$((|f|*|g|)*|h|)(x)$ is defined in $\Bbb R$.   Then $f*(g*h)$ and
$(f*g)*h$ are defined and equal at $x$.

\proof{ Set $k(y)=f(x-y)$ when this is defined, so that $k$ is
measurable and defined almost everywhere (255D).

\medskip

{\bf (a)} If $(|f|*(|g|*|h|))(x)$ is defined, this is
$\int|k(y)|(|g|*|h|)(y)dy$, so by 255G we have

\Centerline{$\int k(y)(g*h)(y)dy=\int k(y+z)g(y)h(z)d(y,z)$,}

\noindent that is,

$$\eqalign{(f*(g*h))(x)
&=\int f(x-y)(g*h)(y)dy
=\int k(y)(g*h)(y)dy\cr
&=\int k(y+z)g(y)h(z)d(y,z)
=\iint k(y+z)g(y)h(z)dydz\cr
&=\iint f(x-y-z)g(y)h(z)dydz
=\int (f*g)(x-z)h(z)dz\cr
&=((f*g)*h)(x).\cr}$$

\medskip

{\bf (b)} If $((|f|*|g|)*|h|)(x)$ is defined, this is

$$\eqalign{\int(|f|*|g|)(x-z)|h(z)|dz
&=\iint|f(x-z-y)||g(y)||h(z)|dydz\cr
&=\iint|k(y+z)||g(y)||h(z)|dydz.\cr}$$

\noindent By 255D again, $(y,z)\mapsto k(y+z)$ is measurable, so we can
apply Tonelli's theorem to see that $\int k(y+z)g(y)h(z)d(y,z)$ is defined,
and is equal to $\int k(y)(g*h)(y)dy=(f*(g*h))(x)$ by 255Ga.   On the other
side, by the last two lines of the proof of (a),
$\int k(y+z)g(y)h(z)d(y,z)$ is also equal to $((f*g)*h)(x)$.
}%end of proof of  255J

\leader{255K}{}\cmmnt{ I do not think we shall need an exhaustive
discussion of
the question of just when $(f*g)(x)$ is defined;  this seems to be
complicated.   However there is a fundamental case in which we can be
sure that $(f*g)(x)$ is defined everywhere.

\medskip

\noindent}{\bf Proposition} Suppose that $f$, $g$ are measurable
complex-valued
functions defined almost everywhere in $\Bbb R$, and that
$f\in\eusm L^p_{\Bbb C}$,
$g\in\eusm L^q_{\Bbb C}$ where $p$, $q\in[1,\infty]$ and
$\bover1p+\bover1q=1$\cmmnt{ (writing $\bover{1}{\infty}=0$ as
usual)}.   Then $f*g$ is defined everywhere in $\Bbb R$, is uniformly
continuous, and

$$\eqalign{\sup_{x\in\Bbb R}|(f*g)(x)|
&\le\|f\|_p\|g\|_q\text{ if }1<p<\infty,\,1<q<\infty,\cr
&\le\|f\|_1\esssup|g|\text{ if }p=1,\,q=\infty,\cr
&\le\esssup|f|\cdot\|g\|_1\text{ if }p=\infty,\,q=1.\cr}$$

\proof{{\bf (a)} (For an introduction to $\eusm L^p$ spaces, see \S244.)
For any $x\in\Bbb R$, the function $f_x$, defined by setting
$f_x(y)=f(x-y)$ whenever $x-y\in\dom f$, must also belong to
$\eusm L^p$, because $f_x=f\phi$ for an automorphism $\phi$ of the
measure
space.   Consequently $(f*g)(x)=\int f_x\times g$ is defined, and of
modulus at most $\|f\|_p\|g\|_q$ or $\|f\|_1\esssup|g|$ or
$\esssup|f|\cdot\|g\|_1$, by 244Eb/244Pb and 243Fa/243K.

\medskip

{\bf (b)} To see that $f*g$ is uniformly continuous, argue as follows.
Suppose first that $p<\infty$.   Let $\epsilon>0$.   Let $\eta>0$ be
such that $(2+2^{1/p})\|g\|_q\eta\le\epsilon$.
Then there is a bounded continuous function $h:\Bbb R\to\Bbb C$ such
that $\{x:h(x)\ne 0\}$ is bounded and $\|f-h\|_p\le\eta$ (244Hb/244Pb);
let $M\ge 1$ be such that $h(x)=0$ whenever $|x|\ge M-1$.   Next,
$h$ is uniformly continuous, so there is a $\delta\in\ocint{0,1}$ such
that $|h(x)-h(x')|\le M^{-1/p}\eta$ whenever $|x-x'|\le\delta$.

Suppose that $|x-x'|\le\delta$.   Defining $h_x(y)=h(x-y)$, as before,
we have

$$\eqalignno{\int|h_x-h_{x'}|^p
&=\int|h(x-y)-h(x'-y)|^pdy
=\int|h(t)-h(x'-x+t)|^pdt\cr
\displaycause{substituting $t=x-y$}
&=\int_{-M}^M|h(t)-h(x'-x+t)|^pdt\cr
\displaycause{because $h(t)=h(x'-x+t)=0$ if $|t|\ge M$}
&\le 2M(M^{-1/p}\eta)^p\cr
\displaycause{because $|h(t)-h(x'-x+t)|\le M^{-1/p}\eta$ for every $t$}
&=2\eta^p.\cr}$$

\noindent So $\|h_x-h_{x'}\|_p\le 2^{1/p}\eta$.   On the other hand,

\Centerline{$\int|h_{x}-f_{x}|^p
=\int|h(x-y)-f(x-y)|^pdy
=\int|h(y)-f(y)|^pdy$,}

\noindent so $\|h_x-f_x\|_p=\|h-f\|_p\le\eta$, and similarly
$\|h_{x'}-f_{x'}\|_p\le\eta$.   So

\Centerline{$\|f_x-f_{x'}\|_p
\le\|f_x-h_x\|_p+|h_x-h_{x'}\|_p+\|h_{x'}-f_{x'}\|_p
\le(2+2^{1/p})\eta$.}

\noindent This means that

$$\eqalign{|(f*g)(x)-(f*g)(x')|
&=|\int f_x\times g-\int f_{x'}\times g|
=|\int(f_x-f_{x'})\times g|\cr
&\le\|f_x-f_{x'}|_p\|g\|_q
\le(2+2^{1/p})\|g\|_q\eta
\le\epsilon.\cr}$$

\noindent As $\epsilon$ is arbitrary, $f*g$ is uniformly continuous.

The argument here supposes that $p$ is finite.   But if $p=\infty$ then
$q=1$ is finite, so we can apply the method with $g$ in place of $f$ to
show that $g*f$ is uniformly continuous, and $f*g=g*f$  by 255Fb.
}%end of proof of 255K

\leader{255L}{The $r$-dimensional case} I have written 255A-255K out as
theorems about Lebesgue measure on $\Bbb R$.   However they all
apply\cmmnt{ equally well} to Lebesgue measure on $\BbbR^r$ for any
$r\ge 1$\cmmnt{, and the
modifications required are so small that I think I need do no more than
ask you to read through the arguments again, turning every $\Bbb R$ into
an $\BbbR^r$, and every $\BbbR^2$ into an $(\BbbR^r)^2$.   In 255A
and elsewhere, the measure $\mu_2$ should be read either as Lebesgue
measure on $\BbbR^{2r}$ or as the product measure on $(\BbbR^r)^2$;
by 251N the two may be identified.   There is a trivial modification
required in part (b) of the proof;  if $I_n=\coint{a_n,b_n}$ then

\Centerline{$\mu I_n=\mu(-I_n)
=\prod_{i=1}^r\max(0,\beta_{ni}-\alpha_{ni})$,}

\noindent writing $a_n=(\alpha_{n1},\ldots,\alpha_{nr})$.   In the proof
of 255I, the functions $f_n$ should be defined by saying that
$f_n(x)=f(x)$ if $|f(x)|\le n$ and $\|x\|\le n$, $0$ otherwise.

In quoting these results, therefore, I shall be uninhibited in referring
to the paragraphs 255A-255K as if they were actually written out for
general $r\ge 1$}.

\leader{255M}{The case of $\ocint{-\pi,\pi}$}\dvro{ {\bf (a)}}{ The same
ideas also apply
to the circle group $S^1$ and to the interval $\ocint{-\pi,\pi}$, but
here perhaps rather more explanation is in order.

\spheader 255Ma The first thing to establish is the appropriate
group operation.}   If we think of $S^1$ as the set 
$\{z:z\in\Bbb C,\,|z|=1\}$, then the group operation is complex 
multiplication, and in
the formulae above $x+y$ must be rendered as $xy$, while $x-y$ must be
rendered as $xy^{-1}$.   On the interval $\ocint{-\pi,\pi}$, the group
operation is $+_{2\pi}$, where for $x$, $y\in\ocint{-\pi,\pi}$ I write
$x+_{2\pi}y$ for whichever of $x+y$, $x+y+2\pi$, $x+y-2\pi$ belongs to
$\ocint{-\pi,\pi}$.   \cmmnt{To see that this is indeed a group
operation, one
method is to note that it corresponds to multiplication on $S^1$ if we
use the canonical bijection
$x\mapsto e^{ix}:\ocint{-\pi,\pi}\to S^1$;  another, to note that it
corresponds to the operation on the quotient group $\Bbb R/2\pi\Bbb Z$.
Thus in this interpretation of the ideas of 255A-255K, we shall wish to
replace $x+y$ by $x+_{2\pi}y$, $-x$ by $-_{2\pi}x$, and $x-y$ by
$x-_{2\pi}y$, where

\Centerline{$-_{2\pi}x=-x$ if $x\in\ooint{-\pi,\pi}$,
\quad$-_{2\pi}\pi=\pi$,}

\noindent and $x-_{2\pi}y$ is whichever of $x-y$, $x-y+2\pi$, $x-y-2\pi$
belongs to $\ocint{-\pi,\pi}$.
}%end of comment

\header{255Mb}{\bf (b)} As for the measure, the measure to use on
$\ocint{-\pi,\pi}$ is just Lebesgue measure.   \cmmnt{Note that
because
$\ocint{-\pi,\pi}$ is Lebesgue measurable, there will be no confusion
concerning the meaning of `measurable subset', as the relatively
measurable subsets of $\ocint{-\pi,\pi}$ are actually measured by
Lebesgue measure on $\Bbb R$.   Also we can identify the product measure
on $\ocint{-\pi,\pi}\times\ocint{-\pi,\pi}$ with the subspace measure
induced by Lebesgue measure on $\BbbR^2$ (251R).}

On $S^1$, we need the corresponding measure induced by the canonical
bijection between $S^1$ and $\ocint{-\pi,\pi}$\cmmnt{, which indeed is
often called `Lebesgue measure on $S^1$'}.   \cmmnt{(We shall see in
265E that it is
also equal to Hausdorff one-dimensional measure on $S^1$.)   We are very
close to the level at which it
would become reasonable to move to $S^1$ and this measure (or its
normalized version, in which it is reduced by a factor of $2\pi$, so as
to make $S^1$ a probability space).   However, the elementary theory of
Fourier series, which will be the principal application of this work in
the present volume, is generally done on intervals in $\Bbb R$, so that
formulae based on $\ocint{-\pi,\pi}$ are closer to the standard
expressions.   Henceforth, therefore, I will express the work in
terms of $\ocint{-\pi,\pi}$.}


\cmmnt{
\header{255Mc}{\bf (c)} The result corresponding to 255A now takes a
slightly different form, so I spell it out.
}

\leader{255N}{Theorem} Let $\mu$ be Lebesgue measure on
$\ocint{-\pi,\pi}$ and
$\mu_2$ Lebesgue measure on $\ocint{-\pi,\pi}\times\ocint{-\pi,\pi}$;
write $\Sigma$, $\Sigma_2$ for their domains.

(a) For any $a\in\ocint{-\pi,\pi}$, the map $x\mapsto
a+_{2\pi}x:\ocint{-\pi,\pi}\to\ocint{-\pi,\pi}$ is a
measure space automorphism of $(\ocint{-\pi,\pi},\Sigma,\mu)$.

(b) The map $x\mapsto -_{2\pi}x:\ocint{-\pi,\pi}\to\ocint{-\pi,\pi}$ is
a measure space automorphism of $(\ocint{-\pi,\pi},\Sigma,\mu)$.

(c) For any $a\in\ocint{-\pi,\pi}$, the map
$x\mapsto a-_{2\pi}x:\ocint{-\pi,\pi}\to\ocint{-\pi,\pi}$ is a
measure space automorphism of $(\ocint{-\pi,\pi},\Sigma,\mu)$.

(d) The map $(x,y)\mapsto(x+_{2\pi}y,y):\ocint{-\pi,\pi}^2\to
\ocint{-\pi,\pi}^2$ is a measure space automorphism of
$(\ocint{-\pi,\pi}^2,
\ifdim\pagewidth>467pt\penalty-50\fi
\Sigma_2,
\ifdim\pagewidth>467pt\penalty-10\fi
\mu_2)$.

(e) The map $(x,y)\mapsto(x-_{2\pi}y,y):\ocint{-\pi,\pi}^2\to
\ocint{-\pi,\pi}^2$ is a measure space automorphism of
$(\ocint{-\pi,\pi}^2,
\ifdim\pagewidth>467pt\penalty-50\fi
\Sigma_2,
\ifdim\pagewidth>467pt\penalty-10\fi
\mu_2)$.

\proof{{\bf (a)} Set $\phi(x)=a+_{2\pi}x$.   Then for any
$E\subseteq\ocint{-\pi,\pi}$,

\Centerline{$\phi[E]
=((E+a)\cap\ocint{-\pi,\pi})
\cup(((E+a)\cap\ocint{\pi,3\pi})-2\pi)
\cup(((E+a)\cap\ocint{-3\pi,-\pi})+2\pi)$,}

\noindent and these three sets are disjoint, so that

$$\eqalignno{\mu\phi[E]
&=\mu((E+a)\cap\ocint{-\pi,\pi})
+\mu(((E+a)\cap\ocint{\pi,3\pi})-2\pi)\cr
&\qquad\qquad\qquad\qquad\qquad\qquad
  +\mu(((E+a)\cap\ocint{-3\pi,-\pi})+2\pi)\cr
&=\mu_L((E+a)\cap\ocint{-\pi,\pi})
+\mu_L(((E+a)\cap\ocint{\pi,3\pi})-2\pi)\cr
&\qquad\qquad\qquad\qquad\qquad\qquad
  +\mu_L(((E+a)\cap\ocint{-3\pi,-\pi})+2\pi)\cr
\noalign{\noindent (writing $\mu_L$ for Lebesgue measure on $\Bbb R$)}
&=\mu_L((E+a)\cap\ocint{-\pi,\pi})
+\mu_L((E+a)\cap\ocint{\pi,3\pi})
+\mu_L((E+a)\cap\ocint{-3\pi,-\pi})\cr
&=\mu_L(E+a)
=\mu_LE
=\mu E.\cr}$$

\noindent Similarly, $\mu\phi^{-1}[E]$ is defined and equal to $\mu E$
for every $E\in\Sigma$, so that $\phi$ is an automorphism of
$(\ocint{-\pi,\pi},
\ifdim\pagewidth>467pt\penalty-50\fi
\Sigma,
\ifdim\pagewidth>467pt\penalty-10\fi
\mu)$.

\medskip

{\bf (b)} Of course this is quicker.   Setting $\phi(x)=-_{2\pi}x$ for
$x\in\ocint{-\pi,\pi}$,  we have

$$\eqalign{\mu(\phi[E])
&=\mu(\phi[E]\cap\ooint{-\pi,\pi})
=\mu(-(E\cap\ooint{-\pi,\pi})\cr
&=\mu_L(-(E\cap\ooint{-\pi,\pi}))
=\mu_L(E\cap\ooint{-\pi,\pi})\cr
&=\mu(E\cap\ooint{-\pi,\pi})
=\mu E\cr}$$

\noindent for every $E\in\Sigma$.

\medskip

{\bf (c)} This is just a matter of putting (a) and (b) together, as in
255A.

\medskip

{\bf (d)} We can argue as in (a), but with a little more elaboration.
If $E\in\Sigma_2$, and $\phi(x,y)=(x+_{2\pi}y,y)$ for $x$,
$y\in\ocint{-\pi,\pi}$, set $\psi(x,y)=(x+y,y)$ for $x$, $y\in\Bbb R$,
and write $c=(2\pi,0)\in\BbbR^2$, $H=\ocint{-\pi,\pi}^2$, $H'=H+c$,
$H''=H-c$.   Then for any $E\in\Sigma_2$,

\Centerline{$\phi[E]=(\psi[E]\cap H)\cup((\psi[E]\cap H')-c)
\cup((\psi[E]\cap H'')+c)$,}

\noindent so

$$\eqalignno{\mu_2\phi[E]
&=\mu_2(\psi[E]\cap H)
+\mu_2((\psi[E]\cap H')-c)
+\mu_2((\psi[E]\cap H'')+c)\cr
&=\mu_L(\psi[E]\cap H)
+\mu_L((\psi[E]\cap H')-c)
+\mu_L((\psi[E]\cap H'')+c)\cr
\noalign{\noindent (this time writing $\mu_L$ for Lebesgue measure on
$\BbbR^2$)}
&=\mu_L(\psi[E]\cap H)
+\mu_L(\psi[E]\cap H')
+\mu_L(\psi[E]\cap H'')\cr
&=\mu_L\psi[E]
=\mu_LE
=\mu_2 E.\cr}$$

\noindent In the same way, $\mu_2(\phi^{-1}[E])=\mu_2E$ for every
$E\in\Sigma_2$, so $\phi$ is an automorphism of
$(\ocint{-\pi,\pi}^2,\Sigma_2,\mu_2)$, as required.

\medskip

{\bf (e)} Finally, (e) is just a restatement of (d), as before.
}%end of proof of 255N

\leader{255O}{Convolutions on $\ocint{-\pi,\pi}$}\cmmnt{ With the
fundamental
result established, the same arguments as in 255B-255K now yield the
following.}   Write $\mu$ for Lebesgue measure on $\ocint{-\pi,\pi}$.


\header{255Oa}{\bf (a)} Let $f$ and $g$ be measurable
complex-valued
functions defined almost everywhere in $\ocint{-\pi,\pi}$.   Write $f*g$
for the function defined by the formula

\Centerline{$(f*g)(x)=\int_{-\pi}^{\pi}f(x-_{2\pi}y)g(y)dy$}

\noindent whenever $x\in\ocint{-\pi,\pi}$ and the integral exists as a
complex number.   Then $f*g$ is the {\bf
convolution} of the functions $f$ and $g$.

\header{255Ob}{\bf (b)} If $f$ and $g$ are measurable complex-valued
functions defined almost everywhere in $\ocint{-\pi,\pi}$, then
$f*g=g*f$.

\vspheader{60pt}255Oc Let $f$, $g$ and $h$ be measurable complex-valued
functions defined almost everywhere in $\ocint{-\pi,\pi}$.   Then

\quad{(i)}

\Centerline{$\int_{-\pi}^{\pi}h(x)(f*g)(x)dx
=\int_{\ocint{-\pi,\pi}^2}h(x+_{2\pi}y)f(x)g(y)d(x,y)$}

\noindent whenever the right-hand side exists and is finite, provided
that in the expression $h(x)(f*g)(x)$ we interpret the product as $0$ if
$h(x)=0$ and $(f*g)(x)$ is undefined.

\quad{(ii)} If, on the same interpretation of $|h(x)|(|f|*|g|)(x)$,
the integral  $\int_{-\pi}^{\pi}|h(x)|(|f|*|g|)(x)dx$ is
finite, then $\int_{\ocint{-\pi,\pi}^2}h(x+_{2\pi}y)f(x)g(y)d(x,y)$ exists in
$\Bbb C$, so again we shall have

\Centerline{$\int_{-\pi}^{\pi}h(x)(f*g)(x)dx
=\int_{\ocint{-\pi,\pi}^2}h(x+_{2\pi}y)f(x)g(y)d(x,y)$.}

\spheader 255Od If $f$, $g$ are complex-valued
functions which are integrable over $\ocint{-\pi,\pi}$, then $f*g$ is
integrable, with

\Centerline{$\int_{-\pi}^{\pi}f*g
=\int_{-\pi}^{\pi}f\int_{-\pi}^{\pi}g$,
\quad$\int_{-\pi}^{\pi}|f*g|
\le\int_{-\pi}^{\pi}|f|\int_{-\pi}^{\pi}|g|$.}

\spheader 255Oe Let $f$, $g$, $h$ be complex-valued measurable
functions defined almost everywhere in $\ocint{-\pi,\pi}$, such that $f*g$
and $g*h$ are also defined almost everywhere.   Suppose
that $x\in\ocint{-\pi,\pi}$ is such that one of $(|f|*(|g|*|h|))(x)$,
$((|f|*|g|)*|h|)(x)$ is defined in $\Bbb R$.   Then $f*(g*h)$ and
$(f*g)*h$ are defined and equal at $x$.

\spheader 255Of Suppose that $f\in\eusm L^p_{\Bbb C}(\mu)$,
$g\in\eusm L^{q}_{\Bbb C}(\mu)$ where $p$, $q\in[1,\infty]$ and
$\bover1p+\bover1{q}=1$.   Then $f*g$ is defined
everywhere in $\ocint{-\pi,\pi}$, and
$\sup_{x\in\ocint{-\pi,\pi}}|(f*g)(x)|\le\|f\|_p\|g\|_q$, interpreting
$\|\,\,\|_{\infty}$ as $\esssup |\,\,|$\cmmnt{, as in 255K}.

\exercises{
\leader{255X}{Basic exercises $\pmb{>}$(a)}
%\sqheader 255Xa
Let $f$, $g$ be complex-valued
functions defined almost everywhere in $\Bbb R$.   Show that for any
$x\in\Bbb R$, $(f*g)(x)=\int f(x+y)g(-y)dy$ if either is defined.
%255E

\sqheader 255Xb Let $f$ and $g$ be complex-valued functions defined
almost everywhere in $\Bbb R$.   (i) Show that if $f$ and $g$ are even
functions, so is $f*g$.   (ii) Show that if $f$ is even and $g$ is odd
then $f*g$ is odd.   (iii) Show that if $f$ and $g$ are odd then $f*g$
is even.
%255E

\spheader 255Xc Suppose that $f$, $g$ are real-valued measurable
functions defined almost everywhere in $\BbbR^r$ and such that $f>0$
a.e., $g\ge 0$ a.e.\ and $\{x:g(x)>0\}$ is not negligible.   Show that
$f*g>0$ everywhere in $\dom(f*g)$.
%255F

\sqheader 255Xd Suppose that $f:\Bbb R\to\Bbb C$ is a bounded
differentiable function and that $f'$ is bounded.   Show that for any
integrable
complex-valued function $g$ on $\Bbb R$, $f*g$ is differentiable and
$(f*g)'=f'*g$ everywhere.   \Hint{123D.}
%255F

\spheader 255Xe A complex-valued function $g$ defined almost everywhere
in $\Bbb R$ is {\bf locally integrable} if $\int_a^bg$ is defined in
$\Bbb C$ whenever $a<b$ in $\Bbb R$.   Suppose that $g$ is such a
function and that $f:\Bbb R\to\Bbb C$ is a differentiable function, with
continuous derivative, such that $\{x:f(x)\ne 0\}$ is bounded.   Show
that $(f*g)'=f'*g$ everywhere.
%255F 255Xd

\sqheader 255Xf Set $\phi_{\delta}(x)=\exp(-\bover1{\delta^2-x^2})$ if
$|x|<\delta$, $0$ if $|x|\ge\delta$, as in 242Xi.   Set
$\alpha_{\delta}=\int\phi_{\delta}$,
$\psi_{\delta}=\alpha_{\delta}^{-1}\phi_{\delta}$.   Let $f$ be a
locally integrable
complex-valued function on $\Bbb R$.   (i) Show that $f*\psi_{\delta}$
is a smooth function defined everywhere on $\Bbb R$ for every
$\delta>0$.   (ii) Show that
$\lim_{\delta\downarrow 0}(f*\psi_{\delta})(x)=f(x)$ for almost every
$x\in\Bbb R$.   \Hint{223Yg.}   (iii) Show that if $f$ is integrable
then $\lim_{\delta\downarrow 0}\int|f-f*\psi_{\delta}|=0$.   \Hint{use
(ii) and 245H(a-ii) {\it or} look first at the case $f=\chi[a,b]$ and
use 242O, noting that $\int|f*\psi_{\delta}|\le\int|f|$.}   (iv) Show
that if $f$ is uniformly continuous and defined everywhere in $\Bbb R$
then
$\lim_{\delta\downarrow 0}\sup_{x\in\Bbb R}|f(x)-(f*\psi_{\delta})(x)|=0$.
%255F 255Xe

\sqheader 255Xg For $\alpha>0$, set
$g_{\alpha}(t)=\Bover1{\Gamma(\alpha)}t^{\alpha-1}$ for $t>0$, $0$ for
$t\le 0$.   Show that $g_{\alpha}*g_{\beta}=g_{\alpha+\beta}$ for all
$\alpha$, $\beta>0$.   \Hint{252Yf.}
%255F

\sqheader 255Xh Let $\mu$ be Lebesgue measure on $\Bbb R$.
For $u$, $v$,
$w\in L^0_{\Bbb C}=L^0_{\Bbb C}(\mu)$, say that
$u*v=w$ if $f*g$ is defined almost everywhere and $(f*g)^{\ssbullet}=w$
whenever $f$, $g\in\eusm L^0_{\Bbb C}(\mu)$, $f^{\ssbullet}=u$ and
$g^{\ssbullet}=w$.   (i) Show that $(u_1+u_2)*v=u_1*v+u_2*v$ whenever
$u_1$, $u_2$, $v\in L^0_{\Bbb C}$ and $u_1*v$ and $u_2*v$ are defined in
this sense.   (ii) Show that $u*v=v*u$ whenever $u$, $v\in L^0(\Bbb C)$ and
either $u*v$ or $v*u$ is defined.   (iii) Show that if $u$, $v$,
$w\in L^0_{\Bbb C}$, $u*v$ and $v*w$ are defined,
and either $|u|*(|v|*|w|)$ or $(|u|*|v|)*|w|$ is
defined, then then $u*(v*w)=(u*v)*w$ are defined and equal.
%255I

\sqheader 255Xi Let $\mu$ be Lebesgue measure on $\Bbb R$.
(i) Show that $u*v$, as defined in 255Xh, belongs to $L^1_{\Bbb C}(\mu)$
whenever $u$, $v\in L^1_{\Bbb C}(\mu)$.
(ii) Show that $L^1_{\Bbb C}$ is a
commutative Banach algebra under $*$ (definition: 2A4J).
%255Xh 255I

\spheader 255Xj(i) Show that if $h$ is an integrable
function on $\BbbR^2$, then   $(Th)(x)=\int h(x-y,y)dy$ exists for
almost every $x\in\Bbb R$, and that $\int (Th)(x)dx=\int h(x,y)d(x,y)$.
(ii) Write $\mu_2$ for Lebesgue measure on $\BbbR^2$, $\mu$ for
Lebesgue measure on $\Bbb R$.   Show that there is a linear operator
$\tilde T:L^1(\mu_2)\to L^1(\mu)$ defined by setting
$\tilde T(h^{\ssbullet})=(Th)^{\ssbullet}$ for every integrable function
$h$ on $\BbbR^2$.   (iii) Show that in the language of 253E and 255Xh,
$\tilde T(u\otimes v)=u*v$ for all $u$, $v\in L^1(\mu)$.
%255Xh 255I 255Xi

\sqheader 255Xk  For $\pmb{a}$, $\pmb{b}\in\Bbb C^{\Bbb Z}$ set
$(\pmb{a}*\pmb{b})(n)=\sum_{i\in\Bbb Z}\pmb{a}(n-i)\pmb{b}(i)$ whenever
$\sum_{i\in\Bbb Z}|\pmb{a}(n-i)\pmb{b}(i)|<\infty$.   Show that

\quad (i) $\pmb{a}*\pmb{b}=\pmb{b}*\pmb{a}$;

\quad (ii) $\sum_{i\in\Bbb Z}\pmb{c}(i)(\pmb{a}*\pmb{b})(i)
=\sum_{i,j\in\Bbb Z}\pmb{c}(i+j)\pmb{a}(i)\pmb{b}(j)$ if
$\sum_{i,j\in\Bbb Z}|\pmb{c}(i+j)\pmb{a}(i)\pmb{b}(j)|<\infty$;

\quad (iii) if $\pmb{a}$, $\pmb{b}\in\ell^1(\Bbb Z)$ then
$\pmb{a}*\pmb{b}\in\ell^1(\Bbb Z)$ and
$\|\pmb{a}*\pmb{b}\|_1\le\|\pmb{a}\|_1\|\pmb{b}\|_1$;

\quad (iv) If $\pmb{a}$, $\pmb{b}\in\ell^2(\Bbb Z)$ then
$\pmb{a}*\pmb{b}\in\ell^{\infty}(\Bbb Z)$ and
$\|\pmb{a}*\pmb{b}\|_{\infty}\le\|\pmb{a}\|_2\|\pmb{b}\|_2$;

\quad (v) if $\pmb{a}$, $\pmb{b}$, $\pmb{c}\in\Bbb C^{\Bbb Z}$ and
$(|\pmb{a}|*(|\pmb{b}|*|\pmb{c}|))(n)$ is well-defined, then
$(\pmb{a}*(\pmb{b}*\pmb{c}))(n)=((\pmb{a}*\pmb{b})*\pmb{c})(n)$.
%255K

\leader{255Y}{Further exercises (a)}
%\spheader 255Ya
Let $f$ be a complex-valued function which is integrable
over $\Bbb R$.   (i) Let $x$ be any point of the Lebesgue set of $f$.
Show that for any $\epsilon>0$ there is a $\delta>0$ such that
$|f(x)-(f*g)(x)|\le\epsilon$ whenever $g:\Bbb R\to\coint{0,\infty}$ is a
function which is non-decreasing on $\ocint{-\infty,0}$, non-decreasing
on $\coint{0,\infty}$, and has $\int g=1$ and
$\int_{-\delta}^{\delta}g\ge 1-\delta$.   (ii) Show that for any
$\epsilon>0$ there is a $\delta>0$ such that $\|f-f*g\|_1\le\epsilon$
whenever $g:\Bbb R\to\coint{0,\infty}$ is a function which is
non-decreasing on $\ocint{-\infty,0}$, non-decreasing on
$\coint{0,\infty}$, and has $\int g=1$ and
$\int_{-\delta}^{\delta}g\ge 1-\delta$.
%255Xf 255F

\spheader 255Yb Let $f$ be a complex-valued function which is integrable
over $\Bbb R$.   Show that, for almost every $x\in\Bbb R$,

\Centerline{$\lim_{a\to\infty}
  \Bover{a}{\pi}\int_{-\infty}^{\infty}
  \Bover{f(y)}{1+a^2(x-y)^2}dy$,
\quad$\lim_{a\to\infty}\Bover1a\int_x^{\infty}f(y)e^{-a(y-x)}dy$,}

\Centerline{$\lim_{\sigma\downarrow 0}\Bover1{\sigma\sqrt{2\pi}}
  \int_{-\infty}^{\infty}f(y)e^{-(y-x)^2/2\sigma^2}dy$}

\noindent all exist and are equal to $f(x)$.   \Hint{263G.}
%255Ya, 255Xf 255F

\spheader 255Yc
Set $f(x)=1$ for all $x\in\Bbb R$, $g(x)=\Bover{x}{|x|}$
for $0<|x|\le 1$ and $0$ otherwise, $h(x)=\tanh x$ for all $x\in\Bbb R$.
Show that $f*(g*h)$ and $(f*g)*h$ are both defined (and constant)
everywhere, and are different.
%255J

\spheader 255Yd Discuss what can happen if, in the context of 255J, we
know that $(|f|*(|g|*|h|))(x)$ is defined, but have no information on
the domain of $f*g$.
%255J

\spheader 255Ye Suppose that $p\in\coint{1,\infty}$
and that $f\in\eusm L^p_{\Bbb C}(\mu)$, where $\mu$ is Lebesgue measure
on $\BbbR^r$.   For $a\in\BbbR^r$ set $(S_af)(x)=f(a+x)$ whenever
$a+x\in\dom f$.   Show that $S_af\in\eusm L^p_{\Bbb C}(\mu)$, and that
for every $\epsilon>0$ there is a $\delta>0$ such that
$\|S_af-f\|_p\le\epsilon$ whenever $|a|\le\delta$.
%255K

\spheader 255Yf Suppose that $p$, $q\in\ooint{1,\infty}$ and
$\bover1p+\bover1q=1$.   Take $f\in\eusm L^p_{\Bbb C}(\mu)$ and
$g\in\eusm L^q_{\Bbb C}(\mu)$, where $\mu$ is Lebesgue measure on
$\BbbR^r$.   Show that $\lim_{\|x\|\to\infty}(f*g)(x)=0$.
\Hint{use 244Hb.}
%255K

\spheader 255Yg Repeat 255Ye and 255K, this time taking $\mu$ to
be Lebesgue measure on $\ocint{-\pi,\pi}$, and setting
$(S_af)(x)=f(a+_{2\pi}x)$ for $a\in\ocint{-\pi,\pi}$;  show that in the
new version of 255K,
$(f*g)(\pi)=\lim_{x\downarrow -\pi}(f*g)(x)$.
%255K, 255Ye

\spheader 255Yh Let $\mu$ be Lebesgue measure on $\Bbb R$.   For
$a\in\Bbb R$, $f\in\eusm L^0=\eusm L^0(\mu)$ set $(S_af)(x)=f(a+x)$
whenever $a+x\in\dom f$.

\quad (i) Show that $S_af\in\eusm L^0$ for every $f\in\eusm L^0$.

\quad  (ii) Show that we have a map $\tilde S_a:L^0\to L^0$ defined by
setting $\tilde S_a(f^{\ssbullet})=(S_af)^{\ssbullet}$ for every
$f\in\eusm L^0$.

\quad  (iii) Show that $\tilde S_a$ is a Riesz space isomorphism and is
a homeomorphism for the topology of convergence in measure;  moreover,
that $\tilde S_a(u\times v)=\tilde S_au\times\tilde S_av$ for all $u$,
$v\in L^0$.

\quad  (iv) Show that $\tilde S_{a+b}=\tilde S_a\tilde S_b$ for all $a$,
$b\in\Bbb R$.

\quad  (v) Show that $\lim_{a\to 0}\tilde S_au=u$ for the topology of
convergence in measure, for every $u\in L^0$.

\quad  (vi) Show that if $1\le p\le\infty$ then $\tilde S_a\restr L^p$
is an isometric isomorphism of the Banach lattice $L^p$.

\quad  (vii) Show that if $p\in\coint{1,\infty}$ then
$\lim_{a\to 0}\|\tilde S_au-u\|_p=0$ for every $u\in L^p$.

\quad  (viii) Show that if $A\subseteq L^1$ is uniformly integrable and
$M\ge 0$, then $\{\tilde S_au:u\in A,\,|a|\le M\}$ is uniformly
integrable.

\quad (ix) Suppose that $u$, $v\in L^0$ are such that $u*v$ is defined
in $L^0$ in the sense of 255Xh.   Show that
$\tilde S_a(u*v)=(\tilde S_au)*v=u*(\tilde S_av)$ for
every $a\in\Bbb R$.
%255Ye, 255K

\spheader 255Yi Prove 255Nd from 255Na by the method used to
prove 255Ad from 255Aa, rather than by quoting 255Ad.
%255N

\spheader 255Yj Let $\mu$ be Lebesgue measure on $\Bbb R$, and
$\phi:\Bbb R\to\Bbb R$ a convex function;  let
$\bar\phi:L^0\to L^0=L^0(\mu)$ be the associated operator (see 241I).
Show that if $u\in L^1=L^1(\mu)$, $v\in L^0$ are such that $u\ge 0$,
$\int u=1$ and $u*v$, $u*\bar\phi(v)$ are both defined in the sense of
255Xh, then $\bar\phi(u*v)\le u*\bar\phi(v)$.
\Hint{233I.}
%255Xh 255I

\spheader 255Yk Let $\mu$ be Lebesgue measure on $\Bbb R$, and
$p\in[1,\infty]$.   Let $f\in\eusm L^1_{\Bbb C}(\mu)$,
$g\in\eusm L^p_{\Bbb C}(\mu)$.   Show that
$f*g\in\eusm L^p_{\Bbb C}(\mu)$ and that
$\|f*g\|_p\le\|f\|_1\|g\|_p$.   \Hint{argue from 255Yj, as in 244M.}
%255Yj 255K 255I

\spheader 255Yl Suppose that $p$, $q$, $r\in\ooint{1,\infty}$ and that
$\bover1p+\bover1q=1+\bover1r$.   Let $\mu$ be Lebesgue measure on
$\Bbb R$.   (i) Show that

\Centerline{$\int f\times g
\le\|f\|_p^{1-p/r}\|g\|_q^{1-q/r}(\int f^p\times g^q)^{1/r}$}

\noindent whenever $f$, $g\ge 0$ and $f\in\eusm L^p(\mu)$,
$g\in\eusm L^q(\mu)$.   \Hint{set $p'=p/(p-1)$, etc.;  $f_1=f^{p/q'}$,
$g_1=g^{q/p'}$, $h=(f^p\times g^q)^{1/r}$.   Use 244Xc to see that
$\|f_1\times g_1\|_{r'}\le\|f_1\|_{q'}\|g_1\|_{p'}$, so that
$\int f_1\times g_1\times h\le\|f_1\|_{q'}\|g_1\|_{p'}\|h\|_r$.}   (ii)
Show that $f*g$ is defined a.e.\ and that
$\|f*g\|_r\le\|f\|_p\|g\|_q$ for all $f\in\eusm L^p(\mu)$,
$g\in\eusm L^q(\mu)$.   \Hint{take $f$, $g\ge 0$.   Use (i) to see that
$(f*g)(x)^r\le\|f\|_p^{r-p}\|g\|_q^{r-q}\int f(y)^pg(x-y)^qdy$, so that
$\|f*g\|_r^r\le\|f\|_p^{r-p}\|g\|_q^{r-q}\int f(y)^p\|g\|_q^qdy$.}
(This is {\bf Young's inequality}.)
%244Xc 255K

\spheader 255Ym Repeat the results of this section for the group
$(S^1)^r$, where $r\ge 2$, given its product measure.
%+

\spheader 255Yn Let $G$ be a group and $\mu$ a
$\sigma$-finite measure on $G$ such that ($\alpha$) for every $a\in G$,
the map $x\mapsto ax$ is an automorphism of $(G,\mu)$ ($\beta$) the map
$(x,y)\mapsto (x,xy)$ is an automorphism of $(G^2,\mu_2)$, where $\mu_2$
is the c.l.d.\ product measure on $G\times G$.   For $f$,
$g\in\eusm L^0_{\Bbb C}(\mu)$ write $(f*g)(x)=\int f(y)g(y^{-1}x)dy$
whenever this is defined.   Show that
%+

\quad (i) if $f$, $g$, $h\in\eusm L^0_{\Bbb C}(\mu)$ and $\int
h(xy)f(x)g(y)d(x,y)$ is defined in $\Bbb C$, then $\int h(x)(f*g)(x)dx$
exists and is equal to $\int h(xy)f(x)g(y)d(x,y)$, provided
that in the expression $h(x)(f*g)(x)$ we interpret the product as $0$ if
$h(x)=0$ and $(f*g)(x)$ is undefined;

\quad (ii) if $f$, $g\in\eusm L^1_{\Bbb C}(\mu)$ then $f*g\in\eusm
L^1_{\Bbb C}(\mu)$ and $\int f*g=\int f\int g$,
$\|f*g\|_1\le\|f\|_1\|g\|_1$;

\quad (iii) if $f$, $g$, $h\in\eusm L^1_{\Bbb C}(\mu)$ then
$f*(g*h)=(f*g)*h$.

\noindent (See {\smc Halmos 50}, \S59.)

\spheader 255Yo Repeat 255Yn for counting measure on any group $G$.
%+ 255Yn
}%end of exercises

\endnotes{
\Notesheader{255} I have tried to set this section out in such a way
that it will be clear that the basis of all the work here is 255A, and
the crucial application is 255G.   I hope that if and when you come to
look at general topological groups (for instance, in Chapter 44), you
will find it easy to trace
through the ideas in any abelian topological group for which
you can prove a version of 255A.   For non-abelian groups, of
course, rather more care is necessary, especially as in some important
examples we no longer have $\mu\{x^{-1}:x\in E\}=\mu E$ for every $E$;
see 255Yn-255Yo for a little of what can be done without using
topological ideas.

The critical point in 255A is the move from the one-dimensional results
in 255Aa-255Ac, which are just the translation- and
reflection-invariance of
Lebesgue measure, to the two-dimensional results in 255Ac-255Ad.   And
the living centre of the argument, as I present it, is the fact that the
shear transformation $\phi$ is an automorphism of the structure
$(\BbbR^2,\Sigma_2)$.   The actual calculation of $\mu_2\phi[E]$,
assuming that it is measurable, is an easy application of Fubini's and
Tonelli's theorems and the translation-invariance of $\mu$.   It is for
this step
that we absolutely need the topological properties of Lebesgue measure.
I should perhaps remind you that the fact that $\phi$ is a homeomorphism
is not sufficient;  in 134I I described a homeomorphism of the unit
interval which does not preserve measurability, and it is easy to adapt
this to produce a homeomorphism $\psi:\BbbR^2\to\BbbR^2$ such that
$\psi[E]$ is not always measurable for measurable $E$.   The argument of
255A is dependent on the special relationships between all three of the
measure, topology and group structure of $\Bbb R$.

I have already indulged in a few remarks on what ought, or ought not, to
be `obvious' (255C).   But perhaps I can add that such results as
255B and the later claim, in the proof of 255K, that a reflected version
of a function in $\eusm L^p$ is also in $\eusm L^p$, can only be trivial
consequences of results like 255A if every step in the construction of
the integral is done in the abstract context of general measure spaces.
Even though we are here working exclusively with the Lebesgue integral,
the argument will become untrustworthy if we have at any stage in the
definition of the integral even mentioned that we are thinking of
Lebesgue measure.   I advance this as a solid reason for defining
`integration' on abstract measure spaces from the beginning, as I did in
Volume 1.   Indeed, I suggest that generally in pure mathematics there
are good reasons for casting arguments into the forms appropriate to the
arguments themselves.

I am writing this book for readers who are interested in proofs, and as
elsewhere I have written the proofs of this section out in detail.   But
most of us find it useful to go through some
material in `advanced calculus' mode, by which I mean starting with a
formula such as

\Centerline{$(f*g)(x)=\int f(x-y)g(y)dy$,}

\noindent and then working out consequences by formal manipulations, for
instance

\Centerline{$\int h(x)(f*g)(x)dx=\iint h(x)f(x-y)g(y)dydx
=\iint h(x+y)f(x)g(y)dydx$,}

\noindent without troubling about the precise applicability of the
formulae to begin with.   In some ways this formula-driven approach can
be more truthful to the structure of the subject than the careful
analysis I habitually present.   The exact hypotheses necessary to make
the theorems strictly true are surely secondary, in such contexts as
this section, to the pattern formed by the ensemble of the theorems,
which can be adequately and elegantly expressed in straightforward
formulae.   Of course I do still insist that we cannot properly
appreciate the structure, nor safely use it, without mastering the ideas
of the proofs -- and as I have said elsewhere, I believe that mastery of
ideas necessarily includes mastery of the formal details, at least in
the sense of being able to reconstruct them fairly fluently
on demand.

Throughout the main exposition of this section, I have worked with
functions rather than equivalence classes of functions.    But all the
results here have interpretations of great importance for the theory of
the `function spaces' of Chapter 24.   In 255Xh and the succeeding
exercises, I have pointed to a definition of convolution as an operator
from a subset of $L^0\times L^0$ to $L^0$.
It is an interesting point
that if $u$, $v\in L^0$ then $u*v$ can be interpreted as a
{\it function}, not as a member of $L^0$ (255Fc).
Thus 255H can be regarded as saying that $u*v\in\eusm L^1$
for $u$, $v\in L^1$.   We cannot quite say that convolution is a
bilinear operator from $L^1\times L^1$ to $\eusm L^1$, because $\eusm L^1$,
as I define it,
is not strictly speaking a linear space.   If we want a bilinear
operator, then we have to regard convolution as a function from
$L^1\times L^1$ to $L^1$.   But when we look at convolution as a
function on $L^2\times L^2$, for instance, then our functions $u*v$ are
defined everywhere (255K), and indeed are continuous functions vanishing
at $\infty$ (255Ye-255Yf).   So in this case it seems more appropriate
to regard convolution as a bilinear operator from $L^2\times L^2$ to
some space of continuous functions, and not as an operator from
$L^2\times L^2$ to $L^{\infty}$.   For an example of an interesting
convolution which is not naturally representable in terms of an operator
on $L^p$ spaces, see 255Xg.

Because convolution acts as a continuous bilinear operator from
$L^1(\mu)\times L^1(\mu)$ to $L^1(\mu)$, where $\mu$ is Lebesgue measure
on $\Bbb R$, Theorem 253F tells us that it must correspond to a linear
operator from $L^1(\mu_2)$ to $L^1(\mu)$, where $\mu_2$ is Lebesgue
measure on $\BbbR^2$.  This is the operator $\tilde T$ of 255Xj.

So far in these notes I have written as though we were concerned only
with Lebesgue measure on $\Bbb R$.   However many applications of the
ideas involve $\BbbR^r$ or $\ocint{-\pi,\pi}$ or $S^1$.   The move to
$\BbbR^r$ should be elementary.   The move to $S^1$ does require a
re-formulation of the basic result 255A/255N.   It should also be clear
that there will be no new difficulties in moving to $\ocint{-\pi,\pi}^r$
or $(S^1)^r$.   Moreover, we can also go through the whole theory for
the groups $\Bbb Z$ and $\Bbb Z^r$, where the appropriate measure is now
counting measure, so that $L^0_{\Bbb C}$ becomes identified with
$\Bbb C^{\Bbb Z}$ or $\Bbb C^{\Bbb Z^r}$ (255Xk, 255Yo).


}%end of notes

\discrpage

