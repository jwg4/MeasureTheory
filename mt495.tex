\frfilename{mt495.tex}
\versiondate{20.12.08}
\copyrightdate{2003}

\def\chaptername{Further topics}
\def\sectionname{Poisson point processes}

\newsection{495}

A classical challenge in probability theory is to formulate a consistent
notion of `random set'.   Simple geometric considerations lead us to a
variety of measures which are both interesting and important.
All these are manifestly special constructions.   Even in the most
concrete structures, we have to make choices which come to seem
arbitrary as soon as we are conscious of the many alternatives.   There
is however one construction which has a claim to pre-eminence because it
is both robust under the transformations of abstract measure theory and
has striking properties when applied to familiar measures (to the point,
indeed, that it is relevant to questions in physics and chemistry).
This gives the `Poisson point processes' of 495D-495E.   In this section
I give a brief introduction to the measure-theoretic aspects of this
construction.

\leader{495A}{Poisson distributions}\cmmnt{ We need a little of the
elementary theory of Poisson distributions.

\medskip

}{\bf (a)} The {\bf Poisson distribution} with parameter $\gamma>0$
is the point-supported Radon probability measure $\nu_{\gamma}$ on
$\Bbb R$ such that $\nu_{\gamma}\{n\}=\Bover{\gamma^n}{n!}e^{-\gamma}$
for every $n\in\Bbb N$.   \cmmnt{(See 285Q and 285Xo.)}   Its
expectation is
$\cmmnt{\sum_{n=1}^{\infty}\Bover{\gamma^n}{(n-1)!}e^{-\gamma}
=\mskip5mu}\gamma$.
\cmmnt{Since
$\nu_{\gamma}\Bbb N=1$, }$\nu_{\gamma}$ can be identified with the
corresponding subspace measure on $\Bbb N$.
It will be convenient to allow $\gamma=0$, so that the Dirac
measure on $\Bbb R$ or $\Bbb N$ concentrated at $0$ becomes a `Poisson
distribution with expectation $0$'.

\spheader 495Ab The convolution of two Poisson distributions is a
Poisson distribution.   \prooflet{\Prf\ If $\alpha$, $\beta>0$ then

$$\eqalignno{(\nu_{\alpha}*\nu_{\beta})(\{n\})
&=\int\nu_{\beta}(\{n\}-t)\nu_{\alpha}(dt)\cr
\displaycause{444A}
&=\sum_{i=0}^n\Bover{\beta^{n-i}}{(n-i)!}e^{-\beta}
  \cdot\Bover{\alpha^i}{i!}e^{-\alpha}\cr
&=e^{-\alpha-\beta}\Bover1{n!}
   \sum_{i=0}^n\Bover{n!}{i!(n-i)!}\alpha^i\beta^{n-i}
=\Bover{(\alpha+\beta)^n}{n!}e^{-\alpha-\beta}\cr}$$

\noindent for every $n\in\Bbb N$, so
$\nu_{\alpha}*\nu_{\beta}=\nu_{\alpha+\beta}$.\ \QeD}
So if $f$ and $g$ are independent random variables with Poisson
distributions then $f+g$ has a Poisson distribution\cmmnt{
(272T\formerly{2{}72S})}.

\spheader 495Ac If $\familyiI{f_i}$ is a countable independent family of
random variables with Poisson distributions, and
$\alpha=\sum_{i\in I}\Expn(f_i)$ is finite, then $f=\sum_{i\in I}f_i$ is
defined a.e.\ and has a Poisson distribution with expectation $\alpha$.
\prooflet{\Prf\ For finite $I$ we can induce on $\#(I)$, using (b) (and
272L) for the inductive step.   For the infinite case we can suppose
that $I=\Bbb N$.   In this case $f_i\ge 0$ a.e.\ for each $i$ so
$f=\sum_{i=0}^{\infty}f_i$ is defined a.e.\ and has expectation
$\alpha$, by B.Levi's theorem.   Setting $g_n=\sum_{i=0}^nf_i$ for each
$n$, so that $g_n$ has a Poisson distribution with expectation
$\beta_n=\sum_{i=0}^n\alpha_i$, we have

$$\Pr(f\le\gamma)
=\lim_{n\to\infty}\Pr(g_n\le\gamma)
=\lim_{n\to\infty}\sum_{i=0}^{\lfloor\gamma\rfloor}
  \Bover{\beta_n^i}{i!}e^{-\beta_n}
=\sum_{i=0}^{\lfloor\gamma\rfloor}\Bover{\alpha^i}{i!}e^{-\alpha}$$

\noindent for every $\gamma\ge 0$, so $f$ has a Poisson distribution
with expectation $\alpha$.\ \Qed}

\spheader 495Ad
\cmmnt{I find myself repeatedly calling on the simple fact that}
$1-e^{-\gamma}(1+\gamma)=\nu_{\gamma}(\Bbb N\setminus\{0,1\})$ is at most
$\bover12\gamma^2$ for every $\gamma\ge 0$\prooflet{;  this is because
$\bover{d}{dt}(\bover12t^2+e^{-t}(1+t))=t(1-e^{-t})\ge 0$ for $t\ge 0$}.

\leader{495B}{Theorem} Let $(X,\Sigma,\mu)$ be a measure
space.   Set $\Sigma^f=\{E:E\in\Sigma$, $\mu E<\infty\}$.   Then for any
$\gamma>0$ there
are a probability space $(\Omega,\Lambda,\lambda)$ and a family
$\family{E}{\Sigma^f}{g_E}$ of random variables on $\Omega$ such that

\inset{(i) for every $E\in\Sigma^f$, $g_E$ has a Poisson distribution
with expectation $\gamma\mu E$;

(ii) whenever $\familyiI{E_i}$ is a disjoint family in $\Sigma^f$, then
$\familyiI{g_{E_i}}$ is stochastically independent;

(iii) whenever $\sequence{i}{E_i}$ is a disjoint sequence in $\Sigma^f$
with union $E\in\Sigma^f$, then $g_E\eae\sum_{i=0}^{\infty}g_{E_i}$.}

\proof{{\bf (a)} Let
$\Cal H\subseteq\{H:H\in\Sigma$, $0<\mu H<\infty\}$ be a maximal family
such that $H\cap H'$ is negligible for all distinct $H$, $H'\in\Cal H$.
For $H\in\Cal H$, let $\mu'_H$ be the normalized subspace measure
defined by setting $\mu'_HE=\mu E/\mu H$ for $E\in\Sigma\cap\Cal PH$,
and $\lambda_H$ the corresponding product probability measure on
$H^{\Bbb N}$.   Next, for
$H\in\Cal H$, let $\nu_H$ be the Poisson distribution with expectation
$\gamma\mu H$,
regarded as a probability measure on $\Bbb N$.   Let $\lambda$ be the
product measure on
$\Omega=\prod_{H\in\Cal H}(\Bbb N\times H^{\Bbb N})$, giving each
$\Bbb N\times H^{\Bbb N}$ the product measure $\nu_H\times\lambda_H$.
For $\omega\in\Omega$, write $m_H(\omega)$, $x_{Hj}(\omega)$ for its
coordinates, so that
$\omega=\family{H}{\Cal H}{(m_H(\omega),\sequence{j}{x_{Hj}(\omega)})}$.

\medskip

{\bf (b)} For $H\in\Cal H$ and $E\in\Sigma$, set
$g_{HE}(\omega)=\#(\{j:j<m_H(\omega)$, $x_{Hj}(\omega)\in E\})$
when this is finite.
Then $g_{HE}$ is measurable and has a Poisson distribution with
expectation
$\gamma\mu(H\cap E)$;  moreover, if $E_0,\ldots,E_r\in\Sigma$ are
disjoint,
then $g_{HE_0},\ldots,g_{HE_r}$ are independent.   \Prf\ It is enough to
examine the case in which the $E_i$ cover $X$.   Then for any
$n_0,\ldots,n_r\in\Bbb N$ with sum $n$,

$$\eqalignno{\lambda\{\omega:g_{HE_i}&(\omega)=n_i
   \text{ for every }i\le r\}\cr
&=\lambda\{\omega:\#(\{j:j<m_H(\omega),\,x_{Hj}\in E_i\})=n_i
   \text{ for every }i\le r\}\cr
&=\lambda\{\omega:m_H(\omega)=n,\,\#(\{j:j<n,\,x_{Hj}\in E_i\})=n_i
   \text{ for every }i\le r\}\cr
&=\sum_{\Atop{J_0,\ldots,J_r\text{ partition }n}
   {\#(J_i)=n_i\text{ for each }i\le r}}
   \lambda\{\omega:m_H(\omega)=n,\,
   x_{Hj}\in E_i\text{ whenever }i\le r,\,j\in J_i\}\cr
&=\sum_{\Atop{J_0,\ldots,J_r\text{ partition }n}
   {\#(J_i)=n_i\text{ for each }i\le r}}
   \Bover{(\gamma\mu H)^n}{n!}e^{-\gamma\mu H}
     \prod_{i=0}^r\bigl(\Bover{\mu(H\cap E_i)}{\mu H}\bigr)^{n_i}\cr
&=\Bover{n!}{n_0!\ldots n_r!}\Bover1{n!}e^{-\gamma\mu H}
     \prod_{i=0}^r(\gamma\mu(H\cap E_i))^{n_i}\cr
&=\prod_{i=0}^r\Bover{(\gamma\mu(H\cap E_i))^{n_i}}{n_i!}
     e^{-\gamma\mu(H\cap E_i)},\cr}$$

\noindent which is just what we wanted to know.\ \Qed

Obviously $g_{HE}=\sum_{i=0}^{\infty}g_{HE_i}$ whenever
$\sequence{i}{E_i}$ is a disjoint sequence in $\Sigma$ with union
$E$, and $g_{HE}=0$ a.e.\ if $\mu(H\cap E)=0$.

\medskip

{\bf (c)} Suppose that $H_0,\ldots,H_m\in\Cal H$ are distinct and
$E_0,\ldots,E_r\in\Sigma$ are disjoint.   Then the random variables
$g_{H_jE_i}$ are independent.   \Prf\ For each $j\le m$, $g_{H_jE_i}$ is
$\Lambda_{H_j}$-measurable, where $\Lambda_{H_j}$ is the
$\sigma$-algebra of subsets of $\Omega$ which are measured by $\lambda$
and determined by the single coordinate $H_j$ in the product
$\prod_{H\in\Cal H}(\Bbb N\times H^{\Bbb N})$.   Now the
$\sigma$-algebras $\Lambda_{H_j}$ are independent (272Ma).   So if we
have any family $\langle n_{ij}\rangle_{i\le r,j\le m}$ in $\Bbb N$,

$$\eqalign{\lambda\{\omega:g_{H_jE_i}(\omega)=n_{ij}
&  \text{ for every }i\le r,j\le m\}\cr
&=\prod_{j=0}^m\lambda\{\omega:g_{H_jE_i}(\omega)=n_{ij}
  \text{ for every }i\le r\}\cr
&=\prod_{j=0}^m\prod_{i=0}^r
  \lambda\{\omega:g_{H_jE_i}(\omega)=n_{ij}\}\cr}$$

\noindent by (b);  and this is what we need to know.\ \Qed

\medskip

{\bf (d)} For $E\in\Sigma^f$, set
$\Cal H_E=\{H:H\in\Cal H$, $\mu(E\cap H)>0\}$;  then $\Cal H_E$ is
countable, because $\Cal H$ is almost disjoint, and
$\mu E=\sum_{H\in\Cal H_E}\mu(H\cap E)$, because $\Cal H$ is maximal.
Set $g_E(\omega)=\sum_{H\in\Cal H_E}g_{HE}(\omega)$ when this is finite.
Then $g_E$ is defined a.e.\ and has a Poisson distribution with
expectation
$\gamma\mu E$ (495Ac).   Also $\familyiI{g_{E_i}}$ are independent
whenever $\familyiI{E_i}$ is a disjoint family in $\Sigma^f$.   \Prf\ It
is enough to deal with the case of finite $I$ (272Bb).    Set
$\Cal H^*=\bigcup_{i\in I}\Cal H_{E_i}$, so that $\Cal H^*$ is
countable, and for $i\in I$ set $g'_i=\sum_{H\in\Cal H^*}g_{HE_i}$.
Then each $g'_i$ is equal almost everywhere to the corresponding
$g_{E_i}$, and
$\familyiI{g'_i}$ is independent, by 272K.   (The point is that each
$g'_i$ is $\Lambda^*_i$-measurable, where $\Lambda^*_i$ is the
$\sigma$-algebra generated by $\{g_{HE_i}:H\in\Cal H\}$, and 272K, with (c)
above,
assures us that the $\Lambda^*_i$ are independent.)   It follows at once
that $\familyiI{g_{E_i}}$ is independent (272H).\ \QeD\  This proves (i)
and (ii).

\medskip

{\bf (e)} Similarly, if $\sequence{i}{E_i}$ is a disjoint sequence in
$\Sigma^f$ with union $E\in\Sigma^f$, set
$\Cal H^*=\Cal H_E\cup\bigcup_{i\in\Bbb N}\Cal H_{E_i}$.   For each
$i\in\Bbb N$, set $g'_i=\sum_{H\in\Cal H^*}g_{HE_i}$;  then
$g'_i\eae g_{E_i}$.   Now

$$\sum_{i=0}^{\infty}g_{E_i}
=_{\text{a.e.}}\sum_{i=0}^{\infty}g'_i
=\sum_{H\in\Cal H^*}\sum_{i=0}^{\infty}g_{HE_i}
=\sum_{H\in\Cal H^*}g_{HE}
=_{\text{a.e.}}g_E,$$

\noindent as required by (iii).
}%end of proof of 495B

\vleader{48pt}{495C}{Lemma} Let $X$ be a set and $\Cal E$ a subring of
the Boolean algebra $\Cal PX$.   Let $\Cal H$ be the family of sets
of the form

\Centerline{$\{S:S\subseteq X$, $\#(S\cap E_i)=n_i$ for every
$i\in I\}$}

\noindent where $\familyiI{E_i}$ is a finite disjoint family in $\Cal E$
and $n_i\in I$ for every $i\in I$.   Then the Dynkin class
$\Tau\subseteq\Cal P(\Cal PX)$
generated by $\Cal H$ is the $\sigma$-algebra of subsets of $\Cal PX$
generated by $\Cal H$.

\proof{ Let $Q$ be the set of functions $q$ from finite subsets
of $\Cal E$ to $\Bbb N$, and for $q\in Q$ set

\Centerline{$H_q=\{S:S\subseteq X$, $\#(S\cap E)=q(E)$ for every
$E\in\dom q\}$.}

\noindent Our family $\Cal H$ is just
$\{H_q:q\in Q$, $\dom q$ is disjoint$\}$.

If $q\in Q$ and $\dom q$ is a subring of $\Cal E$, then $H_q\in\Tau$.
\Prf\ Being a finite Boolean ring, $\dom q$ is a Boolean algebra;  let
$\Cal A$ be the set of its atoms.   Then $H_q$ is either empty or equal
to $H_{q\restr\Cal A}$;  in either case it belongs to $\Tau$.\ \Qed

If $q$ is any member of $Q$, then $H_q\in\Tau$.   \Prf\ Let $\Cal E$ be
the subring of $\Cal PX$ generated by $\dom q$.   Then
$H_q=\bigcup_{q\subseteq q'\in Q,\dom q'=\Cal E}H_{q'}$ is the union of
a finite disjoint family in $\Tau$, so belongs to $\Tau$.\ \Qed

Now observe that $\Cal H_1=\{H_q:q\in Q\}\cup\{\emptyset\}$ is a subset
of $\Tau$ closed under finite intersections, so by the Monotone Class
Theorem (136B) $\Tau$ includes the $\sigma$-algebra generated by
$\Cal H_1$, and must be precisely the $\sigma$-algebra generated by
$\Cal H$.
}%end of proof of 495C

\leader{495D}{Theorem} Let $(X,\Sigma,\mu)$ be an atomless measure
space.   Set $\Sigma^f=\{E:E\in\Sigma$, $\mu E<\infty\}$;  for
$E\in\Sigma^f$, set $f_E(S)=\#(S\cap E)$ when $S\subseteq X$ meets $E$
in a finite set.   Let $\Tau$ be the $\sigma$-algebra of subsets of
$\Cal PX$ generated by sets of the form $\{S:f_E(S)=n\}$ where
$E\in\Sigma^f$ and $n\in\Bbb N$.   Then for any $\gamma>0$ there
is a unique probability measure $\nu$ with domain $\Tau$ such that

(i) for every $E\in\Sigma^f$, $f_E$ is
measurable and has a Poisson distribution with expectation
$\gamma\mu E$;

(ii) whenever $\familyiI{E_i}$ is a disjoint family in $\Sigma^f$, then
$\familyiI{f_{E_i}}$ is stochastically independent.

\proof{{\bf (a)} Let $\Cal H$, $\family{H}{\Cal H}{\nu_H}$,
$\family{H}{\Cal H}{\mu_H}$,
$\family{H}{\Cal H}{\mu'_H}$,
$\family{H}{\Cal H}{\lambda_H}$, $\Omega$, $\lambda$,
$\family{E}{\Sigma^f}{\Cal H_E}$ and $\family{E}{\Sigma^f}{g_E}$ be as
in the proof of 495B.   Note that all the $\mu'_H$ are atomless
(234Nf\formerly{2{}34F}).
Define $\phi:\Omega\to\Cal PX$ by setting

\Centerline{$\phi(\omega)
=\{x_{Hj}(\omega):H\in\Cal H$, $j<m_H(\omega)\}$}

\noindent for $\omega\in\Omega$.

\medskip

{\bf (b)} For $E\in\Sigma^f$, let $A_E$ be the set of those
$\omega\in\Omega$ such that

\inset{{\it either} there are $H\in\Cal H\setminus\Cal H_E$,
$j\in\Bbb N$ such that $x_{Hj}(\omega)\in E$

{\it or} there are distinct $H$, $H'\in\Cal H_E$ and $j\in\Bbb N$ such
that $x_{Hj}(\omega)\in H'$

{\it or} there is an $H\in\Cal H$ such that the
$x_{Hj}(\omega)$, for $j\in\Bbb N$, are not all distinct.}

\noindent Then for any sequence $\sequence{i}{E_i}$ in $\Sigma^f$,
$\lambda_*(\bigcup_{i\in\Bbb N}A_{E_i})=0$.    \Prf\ Set
$\Cal H^*=\bigcup_{i\in\Bbb N}\Cal H_{E_i}$, so that $\Cal H^*$ is a
countable subset of $\Cal H$.   For $H\in\Cal H$, set

\Centerline{$F_H=H\setminus(\bigcup\{E_i:i\in\Bbb N$, $H\cap E_i$ is
negligible$\}\cup\bigcup\{H':H'\in\Cal H^*$, $H'\ne H\})$,}

\Centerline{$W_H=\{\pmb{x}:\pmb{x}\in H^{\Bbb N}$ is injective$\}$,}

\noindent so that $F_H$ is $\mu'_H$-conegligible and $W_H$ is
$\lambda_H$-conegligible (because $\mu'_H$ is atomless, see 254V).
Now

\Centerline{$\Omega\setminus\bigcup_{k\in\Bbb N}A_{E_k}
\supseteq\prod_{H\in\Cal H}
  (\Bbb N\times(W_H\cap F_H^{\Bbb N}))$}

\noindent has full outer measure in $\Omega$, by 254Lb, and its
complement has zero inner measure (413Ec).\ \Qed

It follows that there is a probability measure $\tilde\lambda$ on
$\Omega$, extending $\lambda$, such that $\tilde\lambda A_E=0$ for every
$E\in\Sigma^f$ (417A).
Let $\nu_0$ be the image measure $\tilde\lambda\phi^{-1}$.

\medskip

{\bf (c)} If $E\in\Sigma^f$ and $\omega\in\Omega\setminus A_E$,
then $f_E(\phi(\omega))=g_E(\omega)$ if either is defined.   \Prf\
If $H\in\Cal H$, then all the $x_{Hj}(\omega)$ are distinct;  if
$H\in\Cal H\setminus\Cal H_E$, no $x_{Hj}(\omega)$ can belong to $E$;
if $H$, $H'\in\Cal H_E$ are distinct, then no $x_{Hj}(\omega)$ can
belong to $H'$.   So all the $x_{Hj}(\omega)$, $x_{H'k}(\omega)$ for
$H$, $H'\in\Cal H_E$ and $j$, $k\in\Bbb N$ must be distinct, and

$$\eqalign{f_E(\phi(\omega))
&=\#(\{x_{Hj}(\omega):H\in\Cal H,\,j<m_H(\omega),\,
   x_{Hj}(\omega)\in E\})\cr
&=\#(\{(H,j):H\in\Cal H_E,\,j<m_H(\omega),\,
   x_{Hj}(\omega)\in E\})\cr
&=\sum_{H\in\Cal H_E}g_{HE}(\omega)
=g_E(\omega)}$$

\noindent if any of these is finite.\ \QeD\   It follows at once that if
$E_0,\ldots,E_r\in\Sigma^f$ are disjoint, then
$\{\omega:f_{E_i}(\phi(\omega))=g_{E_i}(\omega)$ for every $i\le r\}$ is
$\tilde\lambda$-conegligible, so that if $n_0,\ldots,n_r\in\Bbb N$ then

$$\eqalign{\nu_0\{S:f_{E_i}(S)=n_i\text{ for every }i\le r\}
&=\tilde\lambda\{\omega:f_{E_i}(\phi(\omega))=n_i
  \text{ for every }i\le r\}\cr
&=\tilde\lambda\{\omega:g_{E_i}(\omega)=n_i\text{ for every }i\le r\}\cr
&=\lambda\{\omega:g_{E_i}(\omega)=n_i\text{ for every }i\le r\}\cr
&=\prod_{i=0}^r\Bover{(\gamma\mu E_i)^{n_i}}{n_i!}
  e^{-\gamma\mu E_i}.\cr}$$

\noindent Thus every $f_{E_i}$ is finite $\nu_0$-a.e., belongs to
$\eusm L^0(\nu_0)$ and has a Poisson
distribution with the appropriate expectation, and they are independent.

\medskip

{\bf (d)} As $\Tau$ is defined to be the $\sigma$-algebra generated by
the family $\{f_E:E\in\Sigma^f\}$, it is included in the domain of
$\nu_0$.   Set $\nu=\nu_0\restr\Tau$;  then $\nu$ has the properties (i)
and (ii).   To see that it is unique, observe that if $\nuprime$ also
has these properties,
then $\{A:\nu A=\nuprime A\}$ is a Dynkin class containing every set of
the form

\Centerline{$\{S:f_{E_i}(S)=n_i$ for $i\le r\}$}

\noindent where $E_0,\ldots,E_r\in\Sigma^f$ are disjoint and
$n_0,\ldots,n_r\in\Bbb N$.   By 495C it contains the $\sigma$-algebra
generated by this family, which is $\Tau$.   So $\nu$ and $\nuprime$
agree on $\Tau$, and are equal.
}%end of proof of 495D

\leader{495E}{Definition} In the context of 495D, I will call the
completion of $\nu$ the
{\bf Poisson point process} on $X$ with density $\gamma$.

\cmmnt{Note that the Poisson point process on $(X,\mu)$ with density
$\gamma>0$ is identical with the Poisson point process on
$(X,\gamma\mu)$ with density $1$.   There would therefore be no real
loss of generality in the main theorems of this section if I spoke only
of point processes with density $1$.   I retain the extra parameter
because applications frequently demand it, and the formulae will be more
useful with the
$\gamma$s in their proper places;  moreover, there are important ideas
associated with variations in $\gamma$, as in 495Xe.
}%end of comment

\leader{495F}{Proposition} Let $(X,\Sigma,\mu)$ be a perfect atomless
measure space, and $\gamma>0$.   Then the Poisson point process on $X$
with density $\gamma$ is a perfect probability measure.

\proof{ I refer to the construction in 495B-495D.   In (b) of the proof
of 495D, use
the construction set out in the proof of 417A, so that the domain
$\tilde\Lambda$ of $\tilde\lambda$ is precisely the family of sets of
the form
$W\symmdiff A$ where $W$ belongs to the domain $\Lambda$ of the product
measure $\lambda$ and $A$ belongs to the $\sigma$-ideal $\Cal A^*$
generated by $\{A_E:E\in\Sigma^f\}$.   Then $\tilde\lambda$ is perfect.
\Prf\ Let $h:\Omega\to\Bbb R$ be a $\tilde\Lambda$-measurable function
and $W\in\tilde\Lambda$ a set of non-zero measure.   Then there are a
$W'\in\Lambda$ and an $A\in\Cal A^*$ such that
$W\symmdiff W'\subseteq A$ and $h\restr\Omega\setminus A$ is
$\Lambda$-measurable;  let $\sequencen{E_n}$ be a sequence in $\Sigma^f$
such that $A\subseteq\bigcup_{n\in\Bbb N}A_{E_n}$, and
$h_1:\Omega\to\Bbb R$ a $\Lambda$-measurable function agreeing with $h$
on $\Omega\setminus A$.   Set
$\Cal H^*=\bigcup_{n\in\Bbb N}\Cal H_{E_n}$, so that $\Cal H^*$ is
countable.   As in the proof of 495D, set

\Centerline{$F_H=H\setminus(\bigcup\{E_n:n\in\Bbb N$, $H\cap E_n$ is
negligible$\}\cup\bigcup\{H':H'\in\Cal H^*$, $H'\ne H\})$,}

\Centerline{$W_H=\{\pmb{x}:\pmb{x}\in H^{\Bbb N}$ is injective$\}$,}

\noindent for $H\in\Cal H$, so that
$W'_H=W_H\cap F_H^{\Bbb N}$ is $\lambda_H$-conegligible.
Set $\Omega'=\prod_{H\in\Cal H}(\Bbb N\times W'_H)$.   This is disjoint
from every $A_{E_n}$ (as in 495D) and therefore from $A$.   The subspace
measure $\lambda_{\Omega'}$ on $\Omega'$ induced by $\lambda$ is just
the product of the
measures on $\Bbb N\times W'_H$ (254La).   All of these are perfect
(451Jc, 451Dc), so $\lambda_{\Omega'}$ also is perfect (451Jc again).
Now

\Centerline{$\lambda_{\Omega'}(W\cap\Omega')
=\lambda_{\Omega'}(W'\cap\Omega')=\lambda W'=\tilde\lambda W>0$.}

\noindent It follows that there is a compact set
$K\subseteq h_1[W\cap\Omega']$ such that
$\lambda_{\Omega'}(h_1^{-1}[K]\cap\Omega')>0$.   As $h$ and $h_1$ agree
on $\Omega'$, $K\subseteq h[W]$, while

$$\eqalignno{\tilde\lambda h^{-1}[K]
&=\tilde\lambda h_1^{-1}[K]\cr
\displaycause{because $\tilde\lambda A=0$}
&=\lambda h_1^{-1}[K]
=\lambda_{\Omega'}(h^{-1}[K]\cap\Omega')
>0.\cr}$$

\noindent As $W$ and $h$ are arbitrary, $\tilde\lambda$ is perfect.\
\Qed

It follows at once that the image measure $\tilde\lambda\phi^{-1}$
and its restriction to $\Tau$ are perfect (451Ea);  finally, the
completion is perfect, by 451Gc.
}%end of proof of 495F

\leader{495G}{Proposition} Let $(X_1,\Sigma_1,\mu_1)$ and
$(X_2,\Sigma_2,\mu_2)$ be
atomless measure spaces, and $f:X_1\to X_2$ an \imp\ function.   Let
$\gamma>0$, and let $\nu_1$, $\nu_2$ be the Poisson point processes on
$X_1$, $X_2$ respectively with density $\gamma$.   Then
$S\mapsto f[S]:\Cal PX_1\to\Cal PX_2$ is \imp\ for $\nu_1$ and $\nu_2$;
in particular, $\Cal PA$ has full outer measure for $\nu_2$ whenever
$A\subseteq X_2$ has full outer measure for $\mu_2$.

\proof{ Set $\psi(S)=f[S]$ for $S\subseteq X_1$.

\medskip

{\bf (a)} If $F\in\Sigma_2^f$, then
$\{S:S\subseteq X_1$, $f\restr f^{-1}[F]\cap S$ is not injective$\}$ is
$\nu_1$-negligible.   \Prf\ Let $n\in\Bbb N$.   Set
$\alpha=\bover1{n+1}\mu_2F$.   Because $\mu_2$ is atomless, we can find
a partition of $F$ into sets $F_0,\ldots,F_n$ of measure $\alpha$.
Now

\Centerline{$\{S:f\restr f^{-1}[F]\cap S$
  is not injective$\}
\subseteq\bigcup_{i\le n}\{S:\#(S\cap f^{-1}[F_i])>1\}$}

\noindent has $\nu_1$-outer measure at most

\Centerline{$(n+1)(1-e^{-\gamma\alpha}(1+\gamma\alpha))
\le\Bover12(n+1)\alpha^2\gamma^2=\Bover1{2(n+1)}(\gamma\mu_2F)^2$.}

\noindent As $n$ is arbitrary,
$\{S:f\restr f^{-1}[F]\cap S$ is not injective$\}$ is negligible.\ \Qed

\medskip

{\bf (b)} It follows that, for any $F\in\Sigma_2^f$ and $n\in\Bbb N$,

\Centerline{$\{S:\#(f[S]\cap F)=n\}
\symmdiff\{S:\#(S\cap f^{-1}[F])=n\}$}

\noindent is $\nu_1$-negligible and $\{S:\#(f[S]\cap F)=n\}$ is measured
by $\nu_1$.   So if $\Tau_2$ is the
$\sigma$-algebra of subsets of
$\Cal PX_2$ generated by sets of the form $\{T:\#(F\cap T)=n\}$ for
$F\in\Sigma_2^f$ and $n\in\Bbb N$, then $\nu_1$ measures
$\psi^{-1}[H]$ for every $H\in\Tau_2$.   Next,
if $\familyiI{F_i}$ is a finite disjoint family in $\Sigma_2^f$ and
$n_i\in\Bbb N$ for $i\in I$,

$$\eqalignno{\nu_1\{S:\#(f[S]\cap F_i)=n_i&\text{ for every }i\in I\}\cr
&=\nu_1\{S:\#(S\cap f^{-1}[F_i])=n_i\text{ for every }i\in I\}\cr
&=\prod_{i\in I}\Bover{(\gamma\mu_1f^{-1}[F_i])^n}{n!}
  e^{-\gamma\mu_1f^{-1}[F_i]}\cr
\displaycause{because $\familyiI{f^{-1}[F_i]}$ is a disjoint family in
$\Sigma_1^f$}
&=\prod_{i\in I}\Bover{(\gamma\mu_2F_i)^n}{n!}e^{-\gamma\mu_2F_i}.\cr}$$

\noindent So the image measure $\nu_1\psi^{-1}$ satisfies (i) and (ii)
of 495D, and must agree with $\nu_2$ on $\Tau_2$;  that is, $\psi$ is
\imp\ for $\nu_1$ and $\nu_2\restr\Tau_2$.   As $\nu_1$ is complete,
$\psi$ is \imp\ for $\nu_1$ and $\nu_2$ (234Ba\formerly{2{}35Hc}).

\medskip

{\bf (c)} If $A\subseteq X_2$ has full outer measure, then we can take
$\mu_1$ to be the subspace measure on $X_1=A$ and $f(x)=x$ for $x\in A$.
In this case, $\Cal PA=\psi[\Cal PA]$ must have full outer measure for
$\nu_2$.
}%end of proof of 495G

\leader{495H}{Lemma} Let $(\tilde X,\tilde\Sigma,\tilde\mu)$ be an
atomless $\sigma$-finite measure space, and $\gamma>0$.   Write $\mu_L$
for Lebesgue measure on $[0,1]$, $\mu'$ for the product measure on
$X'=\tilde X\times[0,1]$, and $\lambda'$ for the product measure on
$\Omega'=[0,1]^{\tilde X}$.   Let $\tilde\nu$, $\nuprime$ be the Poisson
point processes on $\tilde X$, $X'$ respectively with density $\gamma$.
For $T\subseteq\tilde X$ define $\psi_T:\Omega'\to\Cal PX'$ by setting
$\psi_T(z)=\{(t,z(t)):t\in T\}$ for $z\in\Omega'$;  let $\nuprime_T$ be
the
image measure $\lambda'\psi_T^{-1}$ on $\Cal PX'$.   Then
$\langle\nuprime_T\rangle_{T\subseteq\tilde X}$ is a disintegration of
$\nuprime$ over $\tilde\nu$\cmmnt{ (definition:  452E)}.

\proof{{\bf (a)} Let $E\subseteq X'$ be a measurable set with finite
measure, and write $H_E=\{S:S\cap E\ne\emptyset\}$.   Then
$\nuprime H_E=1-e^{-\gamma\mu'E}\le\gamma\mu'E$;  but
also $\overline{\int}\nuprime_T(H_E)\tilde\nu(dT)\le 2\gamma\mu'E$.
\Prf\ We know that $\int\mu_LE[\{t\}]\tilde\mu(dt)=\mu'E$ (252D).   Let
$Y\subseteq\tilde X$ be a conegligible set such that $E[\{t\}]$ is
measurable for every $t\in Y$ and $t\mapsto\mu_LE[\{t\}]:Y\to[0,1]$ is
measurable.   Set $F_i=\{t:t\in Y$, $2^{-i-1}<\mu_LE[\{t\}]\le 2^{-i}\}$
for each $i\in\Bbb N$;  let $\sequence{i}{F'_i}$ be a sequence of sets
of finite measure with union $\tilde X\setminus\bigcup_{i\in\Bbb N}F_i$.
Let $W$ be the set of those $T\subseteq\tilde X$ such that
$T\cap(\tilde X\setminus Y)$ is empty and $T\cap F_i$, $T\cap F'_i$ are
finite for every $i\in\Bbb N$;  then $W$ is $\tilde\nu$-conegligible.

For any $T\in W$,

$$\eqalign{\psi_T^{-1}[H_E]
&=\{z:\psi_T(z)\cap E\ne\emptyset\}\cr
&=\bigcup_{i\in\Bbb N}\bigcup_{t\in T\cap F_i}\{z:z(t)\in E[\{t\}]\}
   \cup\bigcup_{i\in\Bbb N}\bigcup_{t\in T\cap F'_i}
   \{z:z(t)\in E[\{t\}]\}\cr}$$

\noindent is measured by $\lambda'$ and has measure at most
$\sum_{i=0}^{\infty}2^{-i}\#(T\cap F_i)$, because $\mu_LE[\{t\}]$ has
measure at most $2^{-i}$ if $t\in T\cap F_i$, and is zero if
$t\in T\cap F'_i$.   So

$$\eqalignno{\overline{\int}\nuprime_T(H_E)\tilde\nu(dT)
&=\overline{\int}\lambda'\psi_T^{-1}[H_E]\tilde\nu(dT)
\le\int\sum_{i=0}^{\infty}2^{-i}\#(T\cap F_i)\tilde\nu(dT)\cr
&=\sum_{i=0}^{\infty}2^{-i}\int\#(T\cap F_i)\tilde\nu(dT)
=\sum_{i=0}^{\infty}2^{-i}\gamma\tilde\mu F_i\cr
\displaycause{because $T\mapsto\#(T\cap F_i)$ has expectation
$\gamma\tilde\mu F_i$}
&\le 2\gamma\int\mu_LE[\{t\}]\tilde\nu(dt)
=2\gamma\mu'E.  \text{ \Qed}\cr}$$

\medskip

{\bf (b)} Suppose that $\ofamily{j}{s}{F_j}$,
$\langle C_{ij}\rangle_{i<r,j<s}$ and
$\langle n_{ij}\rangle_{i<r,j<s}$ are such that

\inset{$r$, $s\in\Bbb N$,

$n_{ij}\in\Bbb N$ for $i<r$, $j<s$,

$\ofamily{j}{s}{F_j}$ is a disjoint family of subsets of $\tilde X$ with
finite measure,

for each $j<s$, $\ofamily{i}{r}{C_{ij}}$ is a disjoint family of
measurable subsets of $[0,1]$.}

\noindent Set $E_{ij}=F_j\times C_{ij}$ for $i<r$ and $j<s$, and
$H=\{S:S\subseteq X'$,
$\#(S\cap E_{ij})=n_{ij}$ for every $i<r$, $j<s\}$.
Then $\int\nuprime_T(H)\tilde\nu(dT)=\nuprime H$.

\medskip

\Prf\ {\bf (i)} To begin with, suppose that
$\bigcup_{i<r}C_{ij}=[0,1]$ for every $j$.   Set
$n_j=\sum_{i=0}^{r-1}n_{ij}$ for each $j$, and let $W$
be the set of those $T\subseteq\tilde X$ such that $\#(T\cap F_j)=n_j$
for every $j$.   Then
$\tilde\nu W=\prod_{j=0}^{s-1}
\Bover{(\gamma\tilde\mu F_j)^{n_j}}{n_j!}e^{-\gamma\tilde\mu F_j}$.
If $T\subseteq\tilde X$ and $z\in\psi_T^{-1}[H]$, then for each $j<s$ we
must have

$$\eqalign{\#(T\cap F_j)
&=\#(\{t:t\in T\cap F_j,\,z(t)\in\bigcup_{i<r}C_{ij}\})\cr
&=\sum_{i=0}^{r-1}\#(\{t:t\in T\cap F_j,\,z(t)\in C_{ij}\})\cr
&=\sum_{i=0}^{r-1}\#(\psi_T(z)\cap E_{ij})
=\sum_{i=0}^{r-1}n_{ij}
=n_j.\cr}$$

\noindent Turning this round, we see that if $T\notin W$ then
$\psi_T^{-1}[H]=\emptyset$ and $\nuprime_TH=0$.

If $T\in W$, let $Q$ be the set of all
$q=\langle q(i,j)\rangle_{i<r,j<s}$ such that $\ofamily{i}{r}{q(i,j)}$
is a disjoint family of subsets of $T\cap F_j$ for each $j$ and
$\#(q(i,j))=n_{ij}$ for all $i$ and $j$.   Then
$\#(Q)=\prod_{j=0}^{s-1}\Bover{n_j!}{\prod_{i=0}^{r-1}n_{ij}!}$.
Accordingly

$$\eqalignno{\nuprime_TH
&=\lambda'\{z:\psi_T(z)\in H\}\cr
&=\lambda'\{z:\#(\{t:t\in T\cap F_j,\,z(t)\in C_{ij}\})=n_{ij}
   \text{ for all }i,\,j\}\cr
&=\sum_{q\in Q}\lambda\{z:z(t)\in C_{ij}
   \text{ whenever }i<r,\,j<s\text{ and }t\in q(i,j)\}\cr
&=\sum_{q\in Q}\prod_{\Atop{i<r,j<s}{t\in q(i,j)}}\mu_LC_{ij}
=\sum_{q\in Q}\prod_{i<r,j<s}(\mu_LC_{ij})^{n_{ij}}
=\prod_{j=0}^{s-1}\bigl(n_j!\prod_{i=0}^{r-1}
   \bover{(\mu_LC_{ij})^{n_{ij}}}{n_{ij}!}\bigr).\cr}$$

\noindent It follows that

$$\eqalignno{\int\nuprime_T(H)\tilde\nu(dT)
&=\tilde\nu W\cdot\prod_{j=0}^{s-1}\bigl(n_j!
  \prod_{i=0}^{r-1}\bover{(\mu_LC_{ij})^{n_{ij}}}{n_{ij}!}\bigr)\cr
&=\prod_{j=0}^{s-1}
  \bover{(\gamma\tilde\mu F_j)^{n_j}}{n_j!}e^{-\gamma\tilde\mu F_j}
  \cdot\prod_{j=0}^{s-1}\bigl(n_j!
  \prod_{i=0}^{r-1}\bover{(\mu_LC_{ij})^{n_{ij}}}{n_{ij}!}\bigr)\cr
&=\prod_{j=0}^{s-1}
  (\gamma\tilde\mu F_j)^{n_j}e^{-\gamma\tilde\mu F_j}
  \prod_{i=0}^{r-1}\bover{(\mu_LC_{ij})^{n_{ij}}}{n_{ij}!}\cr
&=\prod_{j=0}^{s-1}\prod_{i=0}^{r-1}e^{-\gamma\mu'E_{ij}}
  \bover{(\gamma\mu'E_{ij})^{n_{ij}}}{n_{ij}!}
=\nuprime H,\cr}$$

\noindent as required.

\medskip

\quad{\bf (ii)} For the general case, set
$C_{rj}=[0,1]\setminus\bigcup_{i<r}C_{ij}$,
$E_{rj}=F_j\times C_{rj}$ for each $j<s$.
For $\sigma\in\Bbb N^{(r+1)\times s}$, set

\Centerline{$H_{\sigma}=\{S:S\subseteq X'$, $\#(S\cap E_{ij})=\sigma(i,j)$
for every $i\le r$ and $j<s\}$.}

\noindent By (i), we have
$\nuprime H_{\sigma}=\int\nuprime_T(H_{\sigma})\tilde\nu(dT)$ for every
$\sigma\in\Bbb N^{(r+1)\times s}$.

Set

\Centerline{$J=\{\sigma:\sigma\in\Bbb N^{(r+1)\times s}$,
$\sigma(i,j)=n_{ij}$ for $i<r$, $j<s\}$,
\quad$K=\Bbb N^{(r+1)\times s}\setminus J$,}

\Centerline{$H'_1=\bigcup_{\sigma\in J}H_{\sigma}$,
\quad$H'_2=\bigcup_{\sigma\in K}H_{\sigma}$.}

\noindent Then $H'_1\subseteq H$, $H'_2\cap H=\emptyset$ and

\Centerline{$H'_1\cup H'_2
=\{S:S\cap E_{ij}$ is finite for all $i\le r$, $j<s\}$}

\noindent is $\nuprime$-conegligible.   Accordingly we have

$$\eqalignno{\underline{\int}(\nuprime_T)_*(H)\tilde\nu(dT)
&\ge\underline{\int}(\nuprime_T)_*(H'_1)\tilde\nu(dT)
\ge\sum_{\sigma\in J}\int\nuprime_T(H_{\sigma})\tilde\nu(dT)\cr
&=\sum_{\sigma\in J}\nuprime H_{\sigma}
=1-\sum_{\sigma\in K}\nuprime H_{\sigma}\cr
\displaycause{because $\bigcup_{\sigma\in J\cup K}H_{\sigma}$ is
$\nuprime$-conegligible}
&=1-\sum_{\sigma\in K}\int\nuprime_T(H_{\sigma})\tilde\nu(dT)
=\int 1-\sum_{\sigma\in K}\nuprime_T(H_{\sigma})\,\tilde\nu(dT)\cr
&=\int\nuprime_T(\Cal PX'\setminus H'_2)\tilde\nu(dT)
\ge\overline{\int}(\nuprime_T)^*(H)\tilde\nu(dT).\cr}$$

\noindent But this means, first, that $(\nuprime_T)_*(H)=(\nuprime_T)^*(H)$
for $\tilde\nu$-almost every $T$;  since $\nuprime_T$, being an image of
the complete measure $\lambda'$, is always complete, $\nuprime_T(H)$ is
defined for $\tilde\nu$-almost every $T$.   Finally,

\Centerline{$\nuprime H=\nuprime H'_1=\sum_{\sigma\in J}\nuprime H_{\sigma}
=\int\nuprime_T(H)\tilde\nu(dT)$,}

\noindent as required.\ \Qed

\medskip

{\bf (c)} Suppose that $\ofamily{i}{r}{E_i}$ is a disjoint family in
$\tilde\Sigma\otimes\Sigma_L$ such that all the projections of the $E_i$
onto $\tilde X$ have finite measure, and $n_i\in\Bbb N$ for each $i<r$.
Set $H=\{S:S\subseteq X'$,
$\#(S\cap E_i)=n_i$ for every $i<r\}$.
Then $\int\nuprime_T(H)\tilde\nu(dT)=\nuprime H$.

\Prf\ Let $\Cal E$ be a finite subalgebra of $\tilde\Sigma$ such that
every $E_i$ belongs to
$\Cal E\otimes\Sigma_L$, and let $\ofamily{j}{s}{F_j}$ enumerate the
atoms of $\Cal E$ of finite measure;  extend this to an enumeration
$\ofamily{j}{s'}{F_j}$ of all the atoms of $\Cal E$.   Then we can
express each $E_i$ as $\bigcup_{j<s'}F_j\times C_{ij}$ where each
$C_{ij}\in\Sigma_L$;  but as the projection of $E_i$ has finite measure,
$C_{ij}$ must be empty for every $j\ge s$, so
$E_i=\bigcup_{j<s}F_j\times C_{ij}$.   Let $Q$ be the set of all
$q\in\BbbN^{r\times s}$ such that $\sum_{j=0}^{s-1}q(i,j)=n_i$ for every
$i<r$.   For $q\in Q$ set

\Centerline{$H_q=\{S:S\subseteq X'$,
$\#(S\cap(F_j\times C_{ij}))=q(i,j)$ for every $i<r$, $j<s\}$.}

\noindent Then $\family{q}{Q}{H_q}$ is disjoint and has union $H$, so

\Centerline{$\biggerint\nuprime_T(H)\tilde\nu(dT)
=$$\sum_{q\in Q}$$\biggerint\nuprime_T(H_q)\tilde\nu(dT)
=$$\sum_{q\in Q}\nuprime H_q
=\nuprime H$,}

\noindent using (b) for the middle equality.\ \Qed

\medskip

{\bf (d)} Now let $\ofamily{i}{r}{E_i}$ be a finite disjoint family of
subsets of $X'$ of finite measure, and $\ofamily{i}{r}{n_i}$ a family in
$\Bbb N$.   Set $H=\{S:S\subseteq X'$, $\#(S\cap E_i)=n_i$ for every
$i<r\}$.   Then $\int\nuprime_T(H)\tilde\nu(dT)$ is defined and equal to
$\nuprime H$.

\Prf\ Let $\epsilon>0$.   For each $i<r$ we can find an
$E'_i\in\tilde\Sigma\otimes\Sigma_L$ such that
$\mu'(E_i\symmdiff E'_i)\le\epsilon$ (251Ie).   Discarding a negligible
set from $E'_i$ if
necessary, we may suppose that the projection of $E'_i$ on $\tilde X$
has finite measure.   Set
$\hat E_i=E'_i\setminus\bigcup_{k<i}E'_k$ for each $i$, so that
$\ofamily{i}{r}{\hat E_i}$ is a disjoint family in
$\tilde\Sigma\otimes\Sigma_L$, and the projections of the $\hat E_i$ are
still of finite measure.   Set
$\hat H=\{S:S\subseteq X'$, $\#(S\cap\hat E_i)=n_i$ for every $i<r\}$.
Then (c) tells us that
$\int\nuprime_T(\hat H)\tilde\nu(dT)=\nuprime\hat H$.

Set $E=\bigcup_{i<r}(E_i\symmdiff E'_i)$.   Then $\mu'E\le r\epsilon$,
while $E$ includes $E_i\symmdiff\hat E_i$ for every $i$, so

\Centerline{$\hat H\setminus H_E\subseteq H\subseteq\hat H\cup H_E$,}

\noindent where $H_E=\{S:S\cap E\ne\emptyset\}$ as in (a).   Accordingly

$$\eqalignno{\nuprime H-3r\gamma\epsilon
&\le\nuprime\hat H-2r\gamma\epsilon\cr
\displaycause{by (a)}
&=\int\nuprime_T(\hat H)\tilde\nu(dT)-2r\gamma\epsilon\cr
\displaycause{by (c)}
&\le\int\nuprime_T(\hat H)-\nuprime_T(H_E)\tilde\nu(dT)\cr
\displaycause{by the other part of (a)}
&\le\underline{\int}(\nuprime_T)_*(H)\tilde\nu(dT)
\le\overline{\int}(\nuprime_T)^*(H)\tilde\nu(dT)\cr
&\le\int\nuprime_T(\hat H)+\nuprime_T(H_E)\tilde\nu(dT)
\le\nuprime\hat H+2r\gamma\epsilon
\le\nuprime H+3r\gamma\epsilon.\cr}$$

\noindent As $\epsilon$ is arbitrary,

\Centerline{$
\nuprime H=\underline{\intop}(\nuprime_T)_*(H)\tilde\nu(dT)
=\overline{\intop}(\nuprime_T)^*(H)\tilde\nu(dT)$.}

\noindent As in (c-ii) above, it follows that
$\int\nuprime_T(H)\tilde\nu(dT)$ is defined and equal to
$\nuprime H$.\ \Qed

\medskip

{\bf (e)} So if we write $\Cal H$ for the family of subsets $H$ of
$\Cal PX'$ such that $\int\nuprime_T(H)\tilde\nu(dT)$ is defined and
equal
to $\nuprime H$, and $\Cal H_0$ for the family of sets of the form
$H=\{S:S\subseteq X'$, $\#(S\cap E_i)=n_i$ for every $i<r\}$
where $\ofamily{i}{r}{E_i}$ is a disjoint family of sets of finite
measure and $n_i\in\Bbb N$ for $i<r$, we have $\Cal H\supseteq\Cal H_0$.
But $\Cal H$ is a Dynkin class, so includes the $\sigma$-algebra $\Tau'$
generated by $\Cal H_0$, by 495C.   Since every $\nuprime$-negligible
set is
included in a $\nuprime$-negligible member of $\Tau'$, $\Cal H$ contains
every $\nuprime$-negligible set, and therefore every set measured by
$\nuprime$;  which is what we need to know.
}%end of proof of 495H

\leader{495I}{Theorem} Let $(X,\Sigma,\mu)$ and
$(\tilde X,\tilde\Sigma,\tilde\mu)$ be atomless $\sigma$-finite measure
spaces and $\gamma>0$.   Let $\nu$, $\tilde\nu$ be the Poisson point
processes on $X$, $\tilde X$ respectively with density $\gamma$.
Suppose that $f:X\to\tilde X$ is \imp\ and that
$\family{t}{\tilde X}{\mu_t}$ is a disintegration of $\mu$ over
$\tilde\mu$ consistent with $f$\cmmnt{ (definition:  452E)} such that
every $\mu_t$ is a probability measure.   Write $\lambda$ for the
product measure $\prod_{t\in\tilde X}\mu_t$ on
$\Omega=X^{\tilde X}$, and for $T\subseteq\tilde X$ define
$\phi_T:\Omega\to\Cal PX$ by setting $\phi_T(z)=z[T]$ for $z\in\Omega$;
let $\nu_T$ be the image measure $\lambda\phi_T^{-1}$ on $\Cal PX$.
Then $\langle\nu_T\rangle_{T\subseteq\tilde X}$ is a disintegration of
$\nu$ over $\tilde\nu$.   Moreover

(i) setting $\tilde f(S)=f[S]$ for $S\subseteq X$,
$\langle\nu_T\rangle_{T\subseteq\tilde X}$ is consistent with
$\tilde f:\Cal PX\to\Cal P\tilde X$;

(ii) if $\family{t}{\tilde X}{\mu_t}$ is strongly consistent with $f$, then
$\langle\nu_T\rangle_{T\subseteq\tilde X}$ is strongly consistent with
$\tilde f$.

\proof{{\bf (a)} For $T\subseteq\tilde X$ let $V_T$ be the set of those
$z\in\Omega$ such that $fz\restr T$ is injective.   We need to know that

\Centerline{$W=\{T:T\subseteq\tilde X$, $T$ is countable, $V_T$ is
$\lambda$-conegligible$\}$}

\noindent is $\tilde\nu$-conegligible.   \Prf\ Write
$\tilde\Sigma^f=\{F:F\in\tilde\Sigma$, $\tilde\mu F<\infty\}$.   Because
$\tilde\mu$ is atomless and $\sigma$-finite, there is a countable
subalgebra $\Cal E$ of $\tilde\Sigma$ such that for every
$\epsilon>0$ there is a cover of $\tilde X$ by members of $\Cal E$ of
measure at most $\epsilon$.   Set

\Centerline{$Y=\{t:t\in\tilde X$, $\mu_tf^{-1}[F]=(\chi F)(t)$ for
every $F\in\Cal E\}$,}

\noindent so that $Y$ is $\tilde\mu$-conegligible and $\Cal PY$ is
$\tilde\nu$-conegligible.   For $F\in\Cal E$, let $W_F$ be the set of
those $T\subseteq Y$ such that for every
$t\in T\cap F$ there is an $F'\in\Cal E$ such that $T\cap F'=\{t\}$.
Now, given $F\in\Cal E\cap\tilde\Sigma^f$ and $\epsilon>0$, there is a
partition $\familyiI{F_i}$ of $F$ into members of $\Cal E$ of measure
at most $\epsilon$.   Then

$$\eqalign{\tilde\nu^*(\Cal PY\setminus W_F)
&\le\tilde\nu\{T:\#(T\cap F_i)>1\text{ for some }i\in I\}\cr
&\le\sum_{i\in I}1-e^{-\gamma\tilde\mu F_i}(1+\gamma\tilde\mu F_i)
\le\sum_{i\in I}\Bover12(\gamma\tilde\mu F_i)^2
\le\Bover12\epsilon\gamma^2\sum_{i\in I}\tilde\mu F_i
=\Bover12\epsilon\gamma^2\tilde\mu F.\cr}$$

\noindent As $\epsilon$ is arbitary, $W_F$ is $\tilde\nu$-conegligible;
accordingly $W'=\bigcap\{W_F:F\in\Cal E\cap\tilde\Sigma^f\}$ is
$\tilde\nu$-conegligible.

Now suppose that $T\in W'$.   Because $\tilde X$ is covered by
$\Cal E\cap\tilde\Sigma^f$, we see that for every $t\in T$ there is an
$F\in\Cal E\cap\tilde\Sigma^f$ containing $t$, and now there is an
$F'\in\Cal E$ such that $T\cap F'=\{t\}$.   In particular, $T$ is
countable, so

\Centerline{$U_T
=\{z:z(t)\in f^{-1}[F]$ whenever $F\in\Cal E$ and $t\in T\cap F\}$}

\noindent is $\lambda$-conegligible.   Take $z\in U_T$.   If $t$, $t'$
are distinct points of $T$, there is an $F\in\Cal E$ containing $t$ but
not $t'$, and now $F$ contains $f(z(t))$ but not $f(z(t'))$.   So
$fz\restr T$ is injective.   Thus $U_T\subseteq V_T$ and $V_T$ is
$\lambda$-conegligible.   This is true for every $T\in W'$, so
$W\supseteq W'$ is $\tilde\nu$-conegligible.\ \Qed

\medskip

{\bf (b)} Suppose that $\ofamily{i}{r}{E_i}$ is a disjoint family of
subsets of $X$ with finite measure, and $n_i\in\Bbb N$ for $i<r$.   Set
$H=\{S:S\subseteq X$, $\#(S\cap E_i)=n_i$ for every $i<r\}$.   Then
$\int\nu_T(H)\tilde\nu(dT)$ is defined and equal to $\nu H$.   \Prf\
As in 495H, set $X'=\tilde X\times[0,1]$ with the product measure
$\mu'$, and write $\lambda'$ for the product measure on
$\Omega'=[0,1]^{\tilde X}$.   Let $\nuprime$ be the Poisson point
process on
$X'$ with density $\gamma$.   For $T\subseteq\tilde X$ define
$\psi_T:\Omega'\to\Cal PX'$ by setting $\psi_T(z)=\{(t,z(t)):t\in T\}$
for $z\in\Omega'$ and let $\nuprime_T$ be the image measure
$\lambda'\psi_T^{-1}$ on $\Cal PX'$.   By 495H,
$\langle\nuprime_T\rangle_{T\subseteq\tilde X}$ is a disintegration of
$\nuprime$ over $\tilde\nu$.

For each $i<r$, $\int\mu_t(E_i)\tilde\mu(dt)=\mu E_i$;  set
$Y_1=\{t:\mu_tE_i$ is defined for every $i<r\}$, so that
$Y_1\subseteq\tilde X$ is $\tilde\nu$-conegligible.   Set
$g_i(t)=\sum_{j<i}\mu_tE_j$ for $t\in Y_1$ and $i\le r$, and

\Centerline{$E'_i
=\{(t,\alpha):t\in Y_1$, $g_i(t)\le\alpha<g_{i+1}(t)\}$}

\noindent for $i<r$.   Then
$\mu'E'_i=\int g_{i+1}-g_i\,d\tilde\nu=\mu E_i$ for each $i$, by 252N.

Set

\Centerline{$H'
=\{S:S\subseteq X'$, $\#(S'\cap E_i')=n_i$ for every $i<r\}$,}

\Centerline{$W_1
=\{T:T\in W$, $T\subseteq Y_1$, $\nuprime_TH'$ is defined$\}$,}

\noindent so $H'$ is measured by $\nu'$ and
$W_1$ is $\tilde\nu$-conegligible.   Let $T$ be any
member of $W_1$.   Let $Q$ be the set of partitions
$q=\langle q(i)\rangle_{i\le r}$ of $T$ such that $\#(q(i))=n_i$ for
every $i<r$;  because $T$ is countable, so is $Q$.   Set
$E_r=X\setminus\bigcup_{i<r}E_i$ and
$E'_r=X'\setminus\bigcup_{i<r}E'_i$.   Then

\Centerline{$\mu_tE_r=1-\sum_{i=0}^{r-1}\mu_tE_i
=1-g_r(t)=\mu_LE'_r[\{t\}]$}

\noindent for every $t\in Y_1$.   Now

$$\eqalignno{\nuprime_TH'
&=\lambda'\{z:z\in\Omega',\,\psi_T(z)\in H'\}\cr
&=\lambda'\{z:z\in\Omega',\,\#(\{t:t\in T,\,z(t)\in E'_i[\{t\}]\})=n_i
       \text{ for every }i<r\}\cr
&=\sum_{q\in Q}\lambda'\{z:z\in\Omega',\,z(t)\in E'_i[\{t\}]
       \text{ whenever }i\le r\text{ and }t\in q(i)\}\cr
&=\sum_{q\in Q}\prod_{i=0}^r\prod_{t\in q(i)}\mu_LE'_i[\{t\}]\cr
&=\sum_{q\in Q}\prod_{i=0}^r\prod_{t\in q(i)}\mu_tE_i\cr
&=\sum_{q\in Q}\lambda\{z:z\in\Omega,\,z(t)\in E_i
       \text{ whenever }i\le r\text{ and }t\in q(i)\}\cr
&=\lambda\{z:z\in\Omega,\,\#(\{t:t\in T,\,z(t)\in E_i\})=n_i
       \text{ for every }i<r\}\cr
&=\lambda\{z:z\in V_T,\,\#(\{t:t\in T,\,z(t)\in E_i\})=n_i
       \text{ for every }i<r\}\cr
&=\lambda\{z:z\in V_T,\,\#(z[T]\cap E_i)=n_i
       \text{ for every }i<r\}\cr
&=\lambda\{z:z\in V_T,\,\phi_T(z)\in H\}
=\nu_TH.\cr}$$

\noindent Since this is true for $\tilde\nu$-almost every $T$,

$$\eqalignno{\int\nu_T(H)\tilde\nu(dT)
&=\int\nuprime_T(H')\tilde\nu(dT)
=\nuprime H'\cr
&=\prod_{i<r}\Bover{(\gamma\mu'E'_i)^{n_i}}{n_i!}e^{-\gamma\mu'E'_i}
=\prod_{i<r}\Bover{(\gamma\mu E_i)^{n_i}}{n_i!}e^{-\gamma\mu E_i}
=\nu H.  \text{ \Qed}\cr}$$

\noindent Now, just as in part (e) of the proof of 495H, 495C tells us
that $\int\nu_T(H)\tilde\nu(dT)=\nu H$ whenever $\nu$ measures $H$, so
that $\langle\nu_T\rangle_{T\subseteq\tilde X}$ is a disintegration of
$\nu$ over $\tilde\nu$.

\medskip

{\bf (c)} Let $\ofamily{i}{r}{F_i}$ be a disjoint family in
$\tilde\Sigma^f$, and take $n_i\in\Bbb N$ for $i<r$.   Set

\Centerline{$F_r=\tilde X\setminus\bigcup_{i<r}F_i$,}

\Centerline{$Y_2=\{t:t\in\tilde X$,
$\mu_tf^{-1}[F_i]=(\chi F_i)(t)$ for every $i\le r\}$,}

\Centerline{$W_2=\{T:T\in W$, $T\subseteq Y_2\}$,}

\noindent so that $Y_2$ is $\tilde\mu$-conegligible and $W_2$ is
$\tilde\nu$-conegligible.   Set

\Centerline{$\tilde H
=\{T:T\in W_2$, $\#(T\cap F_i)=n_i$ for every $i<r\}$,}

\Centerline{$H=\tilde f^{-1}[\tilde H]
=\{S:S\subseteq X$, $f[S]\in\tilde H\}$.}

\noindent Then $\nu_TH=\chi\tilde H(T)$ for every
$T\in\tilde H$.   \Prf\ For $i\le r$ and $t\in T\cap F_i$, we
have

\Centerline{$\lambda\{z:z(t)\in f^{-1}[F_i]\}=\mu_tf^{-1}[F_i]=1$}

\noindent because $T\subseteq Y_2$.   So

\Centerline{$V=\{z:z\in V_T$,
$f(z(t))\in F_i$ whenever $i\le r$ and $t\in T\cap F_i\}$}

\noindent is $\lambda$-conegligible.   But if $z\in V$ then $fz\restr T$
is injective, so

\Centerline{$\#(f[z[T]]\cap F_i)=\#(T\cap(fz)^{-1}[F_i])
=\#(T\cap F_i)$}

\noindent for every $i<r$, and $z[T]\in H$ iff $T\in\tilde H$.
Thus

$$\eqalign{\nu_TH
=\lambda\{z:z[T]\in H\}
&=\lambda V=1\text{ if }T\in\tilde H,\cr
&=\lambda(\Omega\setminus V)=0\text{ otherwise.  \Qed}\cr}$$

Setting

\Centerline{$\Cal H=\{\tilde H:\tilde H\subseteq\tilde X$,
$\nu_T\tilde f^{-1}[\tilde H]=\chi\tilde H(T)$ for $\tilde\nu$-almost
every $T\in\tilde H\}$,}

\noindent it is easy to check that $\Cal H$ is a Dynkin class containing
all sets of the form
$\{T:T\subseteq\tilde X$, $\#(T\cap F_i)=n_i$ for every $i<r\}$ where
$\ofamily{i}{r}{F_i}$ is a disjoint family in $\tilde\Sigma^f$, and
therefore including the $\sigma$-algebra generated by such sets, by
495C.   But as $\Cal H$ also contains any subset of a negligible set
belonging to $\Cal H$ (remember that all the $\nu_T$ are complete
probability measures, like $\lambda$), it includes the domain of
$\tilde\nu$, and $\langle\nu_T\rangle_{T\subseteq\tilde X}$ is
consistent with $\tilde f$.

\medskip

{\bf (d)} Now suppose that $\family{t}{\tilde X}{\mu_t}$ is strongly
consistent with $f$.   Set $Y_3=\{t:\mu_tf^{-1}[\{t\}]=1\}$ and
$W_3=\{T:T\in W$, $T\subseteq Y_3\}$.   Then
$\nu_T\tilde f^{-1}[\{T\}]=1$ for every $T\in W_3$.   \Prf\ Set
$V'_T=\{z:z\in\Omega$, $f(z(t))=t$ for every $t\in T\}$.   For
each $t\in T$,

\Centerline{$\lambda\{z:f(z(t))=t\}=\mu_tf^{-1}[\{t\}]=1$,}

\noindent because $T\subseteq W_3$.   As $T$ is countable, $V'_T$ is
$\lambda$-conegligible.   But now

\Centerline{$\nu_T\tilde f^{-1}[\{T\}]=\lambda\{z:f[z[T]]=T\}
\ge\lambda V'_T=1$.  \Qed}

As $W_3$ is $\tilde\nu$-conegligible,
$\langle\nu_T\rangle_{T\subseteq\tilde X}$ is strongly consistent with
$\tilde f$.
}%end of proof of 495I

\vleader{72pt}{495J}{Proposition} 
Let $(\frak A,\bar\mu)$ be a measure algebra,
and $\gamma>0$.   Then there are a probability algebra
$(\frak B,\bar\lambda)$ and a function $\theta:\frak A\to\frak B$ such
that

(i) $\theta(\sup A)=\sup\theta[A]$ for every non-empty
$A\subseteq\frak A$ such that $\sup A$ is defined in $\frak A$;

(ii) $\bar\lambda\theta(a)=1-e^{-\gamma\bar\mu a}$ for every
$a\in\frak A$, interpreting $e^{-\infty}$ as $0$;

(iii) whenever $\familyiI{a_i}$ is a disjoint family in
$\frak A$ and $\frak C_i$ is the closed subalgebra of $\frak B$
generated by $\{\theta(a):a\Bsubseteq a_i\}$ for
each $i$, then $\familyiI{\frak C_i}$ is stochastically independent.

\proof{{\bf (a)} We may suppose that $(\frak A,\bar\mu)$ is the measure
algebra of a measure space $(X,\Sigma,\mu)$ (321J).   Set
$\Sigma^f=\{E:E\in\Sigma$, $\mu E<\infty\}$ and
$\frak A^f=\{a:a\in\frak A$, $\bar\mu a<\infty\}$.   Let
$(\Omega,\Lambda,\lambda)$ and $\family{E}{\Sigma^f}{g_E}$ be as in
495B, and take $(\frak B,\bar\lambda)$ to be the measure algebra of
$(\Omega,\Lambda,\lambda)$.   Note that if $E$, $F\in\Sigma^f$ and
$\mu(E\symmdiff F)=0$, then $g_{E\setminus F}$ and $g_{F\setminus E}$
have Poisson distributions with expectation $0$, so are zero almost
everywhere, while $g_E\eae g_{E\cap F}+g_{E\setminus F}$ and
$g_F\eae g_{E\cap F}+g_{F\setminus E}$;  so that $g_E\eae g_F$.   This
means that we can define
$\theta:\frak A^f\to\frak B$ by setting
$\theta(E^{\ssbullet})=\{\omega:g_E(\omega)\ne 0\}^{\ssbullet}$ whenever
$E\in\Sigma^f$, and we shall have
$\bar\lambda(\theta a)=1-e^{-\gamma\bar\mu a}$ because $g_E$ has a
Poisson distribution with expectation $\bar\mu a$ whenever
$E\in\Sigma^f$ and $E^{\ssbullet}=a$.   For $a\in\frak A\setminus\frak
A^f$ set $\theta(a)=1_{\frak B}$.

\medskip

{\bf (b)} If $a$, $b\in\frak A^f$ are disjoint, they can be represented
as $E^{\ssbullet}$, $F^{\ssbullet}$ where $E$, $F\in\Sigma^f$ are
disjoint.   In this case, $g_{E\cup F}\eae g_E+g_F$, so
$\theta(a\Bcup b)=\theta(a)\Bcup\theta(b)$.   Of course the same is true
if $a$, $b\in\frak A$ are disjoint and either has infinite measure.   It
follows at once that for any $a$, $b\in\frak A$,

\Centerline{$\theta(a\Bcup b)
=\theta(a\Bsetminus b)\Bcup\theta(a\Bcap b)\Bcup\theta(b\Bsetminus a)
=\theta(a)\Bcup\theta(b)$.}

\noindent Consequently $\theta(\sup A)=\sup\theta[A]$ for any finite set
$A\Bsubseteq\frak A$.   If $A\Bsubseteq\frak A$ is an infinite set with
supremum $a^*$, then $A'=\{\sup B:B\in[A]^{<\omega}\}$ is an
upwards-directed set with supremum $a^*$, so there is a non-decreasing
sequence $\sequencen{a_n}$ in $A'$ such that
$\lim_{n\to\infty}\bar\mu a_n=\bar\mu a^*$ (321D).   In this case,
$b^*=\sup_{n\in\Bbb N}\theta(a_n)$ is defined in $\frak B$ and

\Centerline{$\bar\lambda b^*=\lim_{n\to\infty}\bar\lambda\theta(a_n)
=\lim_{n\to\infty}1-e^{-\gamma\bar\mu a_n}
=1-e^{-\gamma\bar\mu a^*}
=\bar\lambda\theta(a^*)$.}

\noindent So $b^*=\theta(a^*)$;  since $\theta(a^*)$ is certainly an
upper bound of $\theta[A']$, it must actually be the supremum of
$\theta[A']$ and therefore (because $\theta$ preserves finite suprema)
of $\theta[A]$.

\medskip

{\bf (c)} Thus $\theta$ satisfies (i) and (ii).   As for (iii), note
first that if
$\familyiI{a_i}$ is a finite disjoint family in $\frak A$, then
$\bar\lambda(\inf_{i\in I}\theta(a_i))
=\prod_{i\in I}\bar\lambda\theta(a_i)$.  \Prf\ Set
$J=\{i:i\in I$, $\bar\mu a_i<\infty\}$.   For $i\in J$, represent $a_i$
as $E_i^{\ssbullet}$ where $\family{i}{J}{E_i}$ is a disjoint family in
$\Sigma^f$.   Then $\family{i}{J}{g_{E_i}}$ is independent, so

$$\eqalign{\bar\lambda(\inf_{i\in I}\theta(a_i))
&=\bar\lambda(\inf_{i\in J}\theta(a_i))
=\lambda(\Omega\cap\bigcap_{i\in J}\{\omega:g_{E_i}(\omega)=0\})\cr
&=\prod_{i\in J}\lambda\{\omega:g_{E_i}(\omega)=0\}
=\prod_{i\in J}\bar\lambda\theta(a_i)
=\prod_{i\in I}\bar\lambda\theta(a_i).  \text{ \Qed}\cr}$$

Now suppose that $\familyiI{a_i}$ is a finite disjoint family in
$\frak A$ and that $\frak D_i$ is the subalgebra of $\frak B$ generated
by $D_i=\{\theta(a):a\Bsubseteq a_i\}$ for each $i$.   We know that each
$D_i$ is closed under $\Bcup$ (by (i)) and that
$\bar\lambda(\inf_{i\in J}d_i)=\prod_{i\in J}\bar\lambda d_i$ whenever
$J\subseteq I$ and $d_i\in D_i$ for each $i\in J$, that is, that
$\familyiI{d_i}$ is stochastically independent whenever $d_i\in D_i$ for
each $i$.   Setting $D'_i=\{1\Bsetminus d:d\in D_i\}\cup\{0\}$, we see
that $D'_i$ is closed under $\Bcap$ and that $\familyiI{d_i}$ is
stochastically independent whenever $d_i\in D'_i$ for each $i$ (as in
272F).   An induction on $\#(J)$, using 313Ga for the inductive step,
shows that if $J\subseteq I$, $d_i\in\frak D_i$ for $i\in J$, and
$d_i\in D'_i$ for $i\in I\setminus J$, then
$\bar\lambda(\inf_{i\in I}d_i)=\prod_{i\in I}\bar\lambda d_i$.   At the
end of the induction, we see that
$\bar\lambda(\inf_{i\in I}d_i)=\prod_{i\in I}\bar\lambda d_i$ whenever
$d_i\in\frak D_i$ for each $i$, and therefore whenever $d_i$ belongs to
the topological closure of $\frak D_i$ for each $i$, where $\frak B$ is
given its measure-algebra topology (\S323).

Finally, suppose that $\familyiI{a_i}$ is any disjoint family in
$\frak A$, and $\frak C_i$ is the closed subalgebra of $\frak B$
generated by $D_i=\{\theta(a):a\Bsubseteq a_i\}$ for each $i$.   Take a
finite set $J\subseteq I$ and $c_i\in\frak C_i$ for each $i\in J$.   By
323J, $\frak C_i$ is the topological closure of the subalgebra
$\frak D_i$ of $\frak B$ generated by $\{\theta(a):a\Bsubseteq a_i\}$;
so $\bar\lambda(\inf_{i\in J}c_i)=\prod_{i\in J}\bar\lambda c_i$.   As
$\family{i}{J}{c_i}$ is arbitrary, $\familyiI{\frak C_i}$ is
independent.
}%end of proof of 495J

\leader{495K}{Proposition} Let $U$ be any $L$-space.
Then there are a probability space $(\Omega,\Lambda,\lambda)$ and a
positive linear operator $T:U\to L^1(\lambda)$ such that
$\|Tu\|_1=\|u\|_1$ whenever $u\in L^1(\mu)^+$ and $\familyiI{Tu_i}$ is
stochastically independent in $L^0(\lambda)$ whenever $\familyiI{u_i}$
is a disjoint family in $L^1(\mu)$.

\medskip

\noindent{\bf Remarks}\cmmnt{ Recall that a family $\familyiI{u_i}$ in
a Riesz space is `disjoint' if $|u_i|\wedge|u_j|=0$ for all distinct
$i$, $j\in I$ (352C).}   \cmmnt{A family} $\familyiI{v_i}$ in
$L^0(\lambda)$ is `independent' if $\familyiI{g_i}$ is an independent
family of random variables whenever $g_i\in\eusm L^0(\lambda)$
represents $v_i$ for each $i$\cmmnt{;  compare 367W}.

\proof{{\bf (a)} By Kakutani's theorem, there is a measure algebra
$(\frak A,\bar\mu)$ such that $U$ is isomorphic, as Banach lattice, to
$L^1(\frak A,\bar\mu)$;  now $(\frak A,\bar\mu)$ can be represented as
the measure algebra of a measure space $(X,\Sigma,\mu)$, and we
can identify $U$ and $L^1(\frak A,\bar\mu)$ with $L^1(\mu)$ (365B).
Set $\Sigma^f=\{E:E\in\Sigma$, $\mu E<\infty\}$ and
$\frak A^f=\{a:a\in\frak A$, $\bar\mu a<\infty\}$ as usual.   Take
$(\Omega,\Lambda,\lambda)$ and $\family{E}{\Sigma^f}{g_E}$ from
495B, with $\gamma=1$.   As in the proof of 495J, we have $g_E\eae g_F$
whenever $E$, $F\in\Sigma^f$ and $\mu(E\symmdiff F)=0$;  consequently we
can define $\psi:\frak A^f\to L^1(\lambda)$ by setting
$\psi a=g_E^{\ssbullet}$ whenever $E\in\Sigma^f$ and $E^{\ssbullet}=a$.
Again as in 495J, $g_{E\cup F}\eae g_E+g_F$ whenever $E$, $F\in\Sigma^f$
are disjoint, so $\psi$ is additive.   Also

\Centerline{$\|\psi a\|_1=\biggerint g_Ed\lambda=\mu E=\bar\mu a$}

\noindent whenever $E\in\Sigma^f$ represents $a\in\frak A^f$.   By 365I,
there is a
unique bounded linear operator $T:L^1(\frak A,\bar\mu)\to L^1(\lambda)$
such that $T(\chi a)=\psi a$ for every $a\in\frak A^f$.   By 365Ka, $T$
is a positive operator.   The set
$\{u:u\in L^1(\frak A,\bar\mu)^+, \|Tu\|_1=\|u\|_1\}$ is closed under
addition, norm-closed and contains $\alpha\chi a$ for every
$a\in\frak A^f$ and $\alpha\ge 0$,
so is the whole of $L^1(\frak A,\bar\mu)^+$, by 365F.

Note that if $\familyiI{a_i}$ is a disjoint family in $\frak A^f$, then
$\familyiI{\psi a_i}$ is stochastically independent, by
495B(ii).

\medskip

{\bf (b)} Now let $\familyiI{u_i}$ be a disjoint family in
$L^1(\frak A,\bar\mu)$.   Then $\familyiI{Tu_i}$ is independent.
\Prf\Quer\ Otherwise, there are a finite set $J\subseteq I$ and a family
$\family{i}{J}{V_i}$ such that $V_i$ is a neighbourhood of $Tu_i$ in the
topology of convergence in measure on $L^0(\mu)$ for each $i\in J$, and
$\family{i}{J}{v_i}$ is not independent whenever $v_i\in V_i$ for each
$i$ (367W).   Because the embedding
$L^1(\lambda)\embedsinto L^0(\lambda)$ is continuous for the norm topology
on $L^1(\lambda)$ and the topology of convergence in measure (245G),
there is a $\delta>0$ such that $Tu'_i\in V_i$ whenever $i\in J$,
$u'_i\in L^1(\frak A,\bar\mu)$ and $\|u'_i-u_i\|_1\le\delta$.   Now we
can find such $u'_i\in S(\frak A^f)$ with $|u'_i|\le|u_i|$ (365F).

Express each $u'_i$ as $\sum_{k=0}^{n_i}\alpha_{ik}\chi a_{ik}$ where
$\langle a_{ik}\rangle_{k\le n_i}$ is a disjoint family in
$\frak A^f$ and no $\alpha_{ik}$ is zero (361Eb).   In this case, all
the $a_{ik}$, for $i\in J$ and $k\le n_i$, are disjoint, so all the
$\psi(a_{ik})$ are independent.   But this means that
$\family{i}{J}{Tu'_i}
=\family{i}{J}{\sum_{k=0}^{n_i}\alpha_{ik}\psi(a_{ik})}$ is independent
(272K);  which is impossible, because $Tu'_i\in V_i$ for every
$i\in J$.\ \Bang\Qed

So $T$, regarded as a function from $U$ to $L^1(\lambda)$, has the
required properties.
}%end of proof of 495K

\leader{495L}{}\cmmnt{ The following is a more concrete expression of
the same ideas.

\medskip

\noindent}{\bf Proposition} Let $(X,\Sigma,\mu)$ be an atomless measure
space, and $\nu$ the Poisson point process on $X$ with density
$\gamma>0$.

(a) If $h\in\eusm L^1(\mu)$, $Q_h(S)=\sum_{x\in S\cap\dom h}h(x)$ is
defined and finite for $\nu$-almost every $S\subseteq X$, and
$\int Q_hd\nu=\gamma\int h\,d\mu$.

(b) We have a positive linear operator $T:L^1(\mu)\to L^1(\nu)$
defined by setting $T(h^{\ssbullet})=Q_h^{\ssbullet}$ for every
$h\in\eusm L^1(\mu)$.

(c) $\|Tu\|_1=\gamma\|u\|_1$ whenever $u\in L^1(\mu)^+$ and
$\familyiI{Tu_i}$ is stochastically independent in $L^0(\lambda)$
whenever $\familyiI{u_i}$ is a disjoint family in $L^1(\mu)$.

\proof{{\bf (a)} In the language of 495D, $Q_{\chi E}=f_E$ for every
$E\in\Sigma^f$.   So $Q_{\chi E}\in\eusm L^1(\nu)$ and
$\int Q_{\chi E}d\nu=\gamma\mu E$ for every $E\in\Sigma^f$.   If
$h=\sum_{i=0}^r\alpha_i\chi E_i$ is a simple function on $X$, then
$Q_h\eae\sum_{i=0}^r\alpha_iQ_{\chi E_i}\in\eusm L^1(\nu)$ and
$\int Q_hd\nu=\int h\,d\mu$.   If $h\in\eusm L^1(\mu)$ is zero a.e.,
then $\{S:S\subseteq h^{-1}[\{0\}]\}$ is $\nu$-conegligible, so
$Q_h=0$ a.e.   It follows that if $h\in\eusm L^1(\mu)$ is non-negative,
and $\sequencen{h_n}$ is a non-decreasing sequence of simple functions
converging to $h$ almost everywhere, then
$Q_h\eae\lim_{n\to\infty}Q_{h_n}$, while $Q_{h_n}\leae Q_{h_{n+1}}$ for
every $n$;  so $Q_h$ is $\nu$-integrable and

\Centerline{$\int Q_hd\nu=\lim_{n\to\infty}\int Q_{h_n}d\nu
=\lim_{n\to\infty}\gamma\int h_nd\mu=\gamma\int hd\mu$.}

\medskip

{\bf (b)} Since $Q_h\eae Q_{h'}$ whenever $h\eae h'$ in
$\eusm L^1(\mu)$, we can define $T:L^1(\mu)\to L^1(\nu)$ by setting
$T(h^{\ssbullet})=(Q_h)^{\ssbullet}$ for every $h\in\eusm L^1(\mu)$;
because $Q_{\alpha h}\eae\alpha Q_h$ and $Q_{h+h'}\eae Q_h+Q_{h'}$
whenever $h$, $h'\in\eusm L^1(\mu)$ and $\alpha\in\Bbb R$, $T$ is
linear;  because $Q_h\ge 0$ a.e.\ whenever $h\ge 0$ a.e., $T$ is positive.

\medskip

{\bf (c)} Because $\int Q_hd\nu=\gamma\int h\,d\mu$ for every
$h\in\eusm L^1(\mu)$, and $T$ is positive, $\|Tu\|_1=\gamma\|u\|_1$ for
every $u\in L^1(\mu)^+$.   Finally, if $\familyiI{u_i}$ is a finite
disjoint family in $L^1(\mu)$, we can find a family $\familyiI{h_i}$ of
measurable functions from $X$ to $\Bbb R$ such that
$h_i^{\ssbullet}=u_i$ for each $i$ and the sets $E_i=\{x:h_i(x)\ne 0\}$
are disjoint.   For each $i\in I$, let $\Tau_i$ be the $\sigma$-algebra
of subsets of $\Cal PX$ generated by sets of the form
$\{S:\#(S\cap E)=n\}$ where $E\subseteq E_i$ has finite measure and
$n\in\Bbb N$.   Then $\familyiI{\Tau_i}$ is independent (as in part (c)
of the proof of 495J), and each $Q_{h_i}$ is $\Tau_i$-measurable, so
$\familyiI{Q_{h_i}}$ is independent and $\familyiI{Tu_i}$ is
independent.
}%end of proof of 495L

\leader{495M}{}\cmmnt{ We can identify the characteristic functions of
the random variables $Q_f$ as defined above.

\medskip

\noindent}{\bf Proposition} Let $(X,\Sigma,\mu)$ be an atomless measure
space, and $\nu$ the Poisson point process on $X$ with density
$\gamma>0$.   For $f\in\eusm L^1(\mu)$ set
$Q_f(S)=\sum_{x\in S\cap\dom f}f(x)$ when $S\subseteq X$ and the sum is
defined in $\Bbb R$.   Then

\Centerline{$\biggerint_{\Cal PX}e^{iyQ_f}d\nu
=\exp\bigl(\gamma\int_X(e^{iyf}-1)d\mu\bigr)$}

\noindent for any $y\in\Bbb R$.

\proof{ Note that $Q_f$ is defined $\nu$-almost everywhere, by 495La.

\medskip

{\bf (a)} Consider first the case in which $f$ is a simple function,
expressed as $\sum_{j=0}^n\alpha_j\chi F_j$ where
$\langle F_j\rangle_{j\le n}$ is a disjoint family of sets of finite
measure and $\alpha_j\in\Bbb R$ for each $j$.   Then
$Q_f(S)=\sum_{j=0}^n\alpha_j\#(S\cap F_j)$ for
$\nu$-almost every $S$, so

$$\eqalignno{\biggerint e^{iyQ_f}d\nu
&=\int\prod_{j=0}^ne^{iy\alpha_j\#(S\cap F_j)}\nu(dS)
=\prod_{j=0}^n\int e^{iy\alpha_j\#(S\cap F_j)}\nu(dS)\cr
\displaycause{because the functions $S\mapsto\#(S\cap F_j)$ are
independent}
&=\prod_{j=0}^n\sum_{k=0}^{\infty}\bover{(\gamma\mu F_j)^k}{k!}
   e^{-\gamma\mu F_j}e^{iy\alpha_jk}
=\prod_{j=0}^ne^{-\gamma\mu F_j}\sum_{k=0}^{\infty}
  \bover{(e^{iy\alpha_j}\gamma\mu F_j)^k}{k!}\cr
&=\prod_{j=0}^n\exp\bigl((e^{iy\alpha_j}-1)\gamma\mu F_j)\bigr)
=\exp\bigl(\gamma\sum_{j=0}^n(e^{iy\alpha_j}-1)\mu F_j\bigr)\cr
&=\exp\bigl(\gamma\int(e^{iyf}-1)d\mu\bigr).\cr}$$

\medskip

{\bf (b)} Now suppose that $f$ is any integrable function.   Then there
is a sequence $\sequencen{f_n}$ of simple functions such that
$|f_n|\leae|f|$ for every $n$ and $\lim_{n\to\infty}f_n\eae f$.   Write
$q_n$, $q$ for $Q_{f_n}$, $Q_f$.   Set

\Centerline{$D=\{x:x\in\dom f$, $|f_n(x)|\le|f(x)|$ for every $n$ and
$\lim_{n\to\infty}f_n(x)=f(x)\}$,}

\noindent so that $D$ is $\mu$-conegligible.   If $S\subseteq D$ and
$Q_{|f|}(S)$ is defined, then $q(S)=\lim_{n\to\infty}q_n(S)$, and this
is true for $\nu$-almost every $S$;  so
$\int e^{iyq}d\nu=\lim_{n\to\infty}e^{iyq_n}d\nu$, by Lebesgue's
Dominated Convergence Theorem.   On the other hand,

\Centerline{$|e^{i\alpha}-1|
=|\biggerint_0^{\alpha}\Bover1ie^{it}dt|\le\alpha$,
\quad$|e^{-i\alpha}-1|
=|\biggerint_0^{\alpha}\Bover1ie^{-it}dt|\le\alpha$}

\noindent for every $\alpha\ge 0$.   So if we set
$g(x)=e^{iyf(x)}-1$, $g_n(x)=e^{iyf_n(x)}-1$ when these are defined, we
have $|g_n|\leae|yf_n|\leae|yf|$ for every $n$.   Accordingly

\Centerline{$\biggerint(e^{iyf}-1)d\mu
=\int\lim_{n\to\infty}(e^{iyf_n}-1)d\mu
=\lim_{n\to\infty}\int(e^{iyf_n}-1)d\mu$}

\noindent by Lebesgue's theorem again.   It follows that

$$\eqalignno{\int e^{iyQ_f}d\nu
&=\lim_{n\to\infty}\int e^{iyq_n}d\nu
=\lim_{n\to\infty}
  \exp\bigl(\gamma\int(e^{iyf_n}-1)d\mu\bigr)\cr
\displaycause{by (a)}
&=\exp\bigl(\gamma\lim_{n\to\infty}
  \int(e^{iyf_n}-1)d\mu\bigr)
=\exp\bigl(\gamma\int(e^{iyf}-1)d\mu\bigr),\cr}$$

\noindent as claimed.
}%end of proof of 495M

\cmmnt{\medskip

\allowmorestretch{468}{
\noindent{\bf Remark} Recall that a Poisson random variable with
expectation $\gamma$ has characteristic function
$y\mapsto\exp(\gamma(e^{iy}-1))$ (part (a) of the proof of 285Q),
corresponding to the case $f=\chi F$ where $\mu F=1$.   The random
variables $Q_f$ have {\bf compound Poisson} distributions.
}%end of allowmorestretch
}%end of comment

\leader{495N}{}\cmmnt{ If our underlying measure is a Radon measure,
we can look for Radon measures on $\Cal PX$ to represent the Poisson
point processes on $X$.   There seem to be difficulties in general, but
I can offer the following.   See also 495Yd.

\medskip

\noindent}{\bf Proposition}\discrversionA{\footnote{Revised 2008.}}{}
Let $(X,\frak T,\Sigma,\mu)$ be a Radon measure
space such that $\mu$ is outer regular with respect to the open sets,
and $\gamma>0$.
Give the space $\Cal C$ of closed subsets of $X$ its
Fell topology\cmmnt{ (4A2T)}.

(a) There is a unique quasi-Radon probability measure
$\tilde\nu$ on $\Cal C$ such that

\Centerline{$\tilde\nu\{C:\#(C\cap E)=\emptyset\}=e^{-\gamma\mu E}$}

\noindent whenever $E\subseteq X$ is a measurable set of
finite measure.

(b) If $E_0,\ldots,E_r$ are disjoint sets of finite measure, none
including any singleton set of non-zero measure, and $n_i\in\Bbb N$ for
$i\le r$, then

\Centerline{$\tilde\nu\{C:\#(C\cap E_i)=n_i$ for every $i\le r\}
=\prod_{i=0}^r\Bover{(\gamma\mu E_i)^{n_i}}{n_i!}e^{-\gamma\mu E_i}$.}

(c) Suppose that $\mu$ is atomless and $\nu$ is the Poisson point process on $X$
with density $\gamma$.

\quad(i) $\Cal C$ has full outer measure for $\nu$, and
$\tilde\nu$ extends the subspace measure $\nu_{\Cal C}$.

\quad(ii) If moreover $\mu$ is $\sigma$-finite, then $\Cal C$ is
$\nu$-conegligible.

(d) If $X$ is locally compact then $\tilde\nu$ is a Radon measure.

(e) If $X$ is second-countable and $\mu$ is atomless
then $\tilde\nu=\nu_{\Cal C}$.
%new 2008

\proof{{\bf (a)(i)} Set $\Sigma^f=\{E:\mu E<\infty\}$.
There is a disjoint family $\Cal H$ of non-empty self-supporting
measurable subsets of $X$ of finite measure such that
$\mu E=\sum_{H\in\Cal H}\mu(E\cap H)$ for every $E\in\Sigma$ (412I);
so if $G\subseteq X$ is an open set of finite measure,
$\{H:H\in\Cal H$, $G\cap H\ne\emptyset\}$ is countable.   If $E$ is any set
of finite measure, it is included in an open set of finite measure, because
$\mu$ is outer regular with respect to the open sets;  so once again
$\{H:H\in\Cal H$, $E\cap H\ne\emptyset\}$ is countable.

Build $\Omega=\prod_{H\in\Cal H}\Bbb N\times H^{\Bbb N}$,
$\langle g_{HE}\rangle_{H\in\Cal H,E\in\Sigma}$ and the product
measure $\lambda$ on $\Omega$ as in the proof of
495B;  as in the proof of 495D, set

\Centerline{$\phi(\omega)=\{x_{Hj}(\omega):H\in\Cal H$, $j<m_H(\omega)\}$}

\noindent for $\omega\in\Omega$.

\medskip

\quad{\bf (ii)} If $E\in\Sigma^f$,
$\lambda\{\omega:E\cap\phi(\omega)=\emptyset\}=e^{-\gamma\mu E}$.
\Prf\ $\Cal H'=\{H:H\in\Cal H$, $E\cap H\ne\emptyset\}$ is countable.
Now

\Centerline{$\{\omega:E\cap\phi(\omega)=\emptyset\}
=\bigcap_{H\in\Cal H'}\{\omega:x_{Hj}\notin E$ for every $j<m_H(\omega)\}$}

\noindent has measure

\Centerline{$\prod_{H\in\Cal H'}\lambda\{\omega:g_{HE}(\omega)=0\}
=\prod_{H\in\Cal H'}e^{-\gamma\mu(H\cap E)}
=e^{-\gamma\mu E}$}

\noindent because $\mu E=\sum_{H\in\Cal H'}\mu(H\cap E)$.\ \Qed

Let $\Tau_0$ be the $\sigma$-algebra of subsets of $\Cal PX$
generated by sets of the form $\{S:S\cap E=\emptyset\}$ where
$E\in\Sigma^f$.   By the Monotone Class Theorem (136B),
$\lambda$ measures $\phi^{-1}[W]$ for every
$W\in\Tau_0$;  set $\nu_0W=\lambda\phi^{-1}[W]$ for $W\in\Tau_0$, so that
if $E\in\Sigma$ then $\nu_0\{S:S\cap E=\emptyset\}=e^{-\gamma\mu E}$.

\medskip

\quad{\bf (iii)} Give $\Cal PX$ the topology $\frak S$ generated by sets of
the form

\Centerline{$\{S:S\cap G\ne\emptyset\}$,
\quad$\{S:S\cap K=\emptyset\}$}

\noindent for open sets $G\subseteq X$ and compact sets $K\subseteq X$.
(Thus the Fell topology on $\Cal C$ is the subspace topology induced by
$\frak S$.)   Then $\Cal PX$ is compact.   \Prf\ Follow the proof of
4A2T(b-iii) word for word, but replacing every $\Cal C$ with $\Cal PX$.\
\Qed

\medskip

\quad{\bf (iv)} $\nu_0$ is inner regular with respect to the
$\frak S$-closed sets.   \Prf\ Write $\Cal L$ for the family of
$\frak S$-closed sets belonging to $\Tau_0$.   Of course $\Cal L$ is closed
under finite unions and countable intersections.

($\alpha$) Suppose that $E\in\Sigma^f$ and
$W=\{S:S\cap E\ne\emptyset\}$.   Let $\epsilon>0$.   Then there is a
compact set $K\subseteq E$ such that $\mu(E\setminus K)\le\epsilon$.
Set $V=\{S:S\cap K\ne\emptyset\}$;  then $V\in\Cal L$, $V\subseteq W$ and

\Centerline{$\nu_0(W\setminus V)
\le\nu_0\{S:S\cap E\setminus K\ne\emptyset\}
\le 1-e^{-\gamma\epsilon}$.}

\noindent As $\epsilon$ is arbitrary,
$\nu_0W=\sup\{\nu_0V:V\in\Cal L$, $V\subseteq W\}$.

($\beta$) Suppose that $E\in\Sigma^f$ and
$W=\{S:S\cap E=\emptyset\}$.   Let $\epsilon>0$.   Then there is an
open set $G\supseteq E$ such that $\mu(G\setminus E)\le\epsilon$.
Set $V=\{S:S\cap G=\emptyset\}$;  then $V\in\Cal L$, $V\subseteq W$ and

\Centerline{$\nu_0(W\setminus V)
\le\nu_0\{S:S\cap G\setminus E\ne\emptyset\}
\le 1-e^{-\gamma\epsilon}$.}

\noindent As $\epsilon$ is arbitrary,
$\nu_0W=\sup\{\nu_0V:V\in\Cal L$, $V\subseteq W\}$.

($\gamma$) By 412C, $\nu_0$ is inner regular with respect to the
$\frak S$-closed sets.\ \Qed

\medskip

\quad{\bf (v)} Since $\frak S$ is a compact topology, the family of
$\frak S$-closed sets is a compact class, so 413O tells us that $\nu_0$ has
an extension to a complete topological measure $\tilde\nu_0$ on
$\Cal PX$, inner regular with respect to the closed sets.
Of course $\tilde\nu_0$, being a probability measure, is
effectively locally finite and locally determined, so it is a quasi-Radon
measure with respect to the topology $\frak S$.   Consequently the subspace
measure $\tilde\nu$ on $\Cal C$ is a quasi-Radon measure for the
Fell topology on $\Cal C$ (415B).

\medskip

\quad{\bf (vi)} $\Cal C$ has full outer measure for $\tilde\nu_0$.
\Prf\Quer\ Otherwise, there is a non-empty closed set
$V\subseteq\Cal PX\setminus\Cal C$.
Consider the family $\Cal U$ of subsets of $\Cal PX$ of the form

\Centerline{$\{S:S\cap K=\emptyset$, $S\cap G_i\ne\emptyset$ for
$i<r\}$}

\noindent where $K\subseteq X$ is compact and $G_i\subseteq X$ is an open
set of finite measure for every $i<r$.
Because $\mu$ is locally finite, this is a base for $\frak S$.   So
$\Cal U'=\{U:U\in\Cal U$, $U\cap W=\emptyset\}$ is a cover of
$\Cal PX\setminus W\supseteq\Cal C$.   Of course $U\cap\Cal C$ is open in
the Fell topology for every $U\in\Cal U$;  because $\Cal C$ is compact,
there are $U_0,\ldots,U_m\in\Cal U'$ covering $\Cal C$.

Express each $U_j$ as
$\{S:S\cap K_j=\emptyset$, $S\cap G_{ji}\ne\emptyset$ for
$i<r_j\}$, where the $K_j$ are all compact and the $G_{ji}$ are all open.
Because $\bigcup_{j\le m}U_j$ is disjoint from $V$, there is an
$S\subseteq\Cal PX$ which does not belong to any $U_j$.   Let $\Cal E$ be
the finite algebra of subsets of $X$ generated by
$\{K_j:j\le m\}\cup\{G_{ji}:j\le m$, $i<r_j\}$;  then there is a finite set
$C\subseteq S$ such that $C\cap E\ne\emptyset$ whenever $E\in\Cal E$ and
$S\cap E\ne\emptyset$.   In this case,
$C\in\Cal C\setminus\bigcup_{j\le m}U_j$;  which is supposed to be
impossible.\ \Bang\Qed

\medskip

\quad{\bf (vii)} Consequently $\tilde\nu$ is a probability measure.
If $E\in\Sigma^f$, then

$$\eqalign{\tilde\nu\{C:C\in\Cal C,\,C\cap E=\emptyset\}
&=\tilde\nu(\Cal C\cap\{S:S\subseteq X,\,S\cap E=\emptyset\})\cr
&=\tilde\nu_0\{S:S\cap E=\emptyset\}
=\nu_0\{S:S\cap E=\emptyset\}
=e^{-\gamma\mu E}.\cr}$$

\medskip

\quad{\bf (viii)} To see that $\tilde\nu$ is uniquely defined, let
$\tilde\nuprime$ be another quasi-Radon probability measure on $\Cal C$
with the same property.

\medskip

\qquad\grheada\ Suppose that $E_0,\ldots,E_r\subseteq X$ are disjoint
measurable sets of finite measure, and

\Centerline{$W
=\{C:C\in\Cal C$, $C\cap E_0=\emptyset$, $C\cap E_i\ne\emptyset$ for
$1\le i\le r\}$.}

\noindent Then $\tilde\nu W=\tilde\nu'W$.   \Prf\ Induce on $r$.   If
$r=0$ the result is immediate.   For the inductive step to $r\ge 1$,
consider
$\{C:C\cap E_0=\emptyset$, $C\cap E_i\ne\emptyset$ for $1\le i<r\}$ and
$\{C:C\cap(E_0\cup E_r)=\emptyset$,
$C\cap E_i\ne\emptyset$ for $1\le i<r\}$.   By the inductive hypothesis,
$\tilde\nu$ and $\tilde\nu'$ agree on these two sets, and therefore on
their difference
$\{C:C\in\Cal C$, $C\cap E_0=\emptyset$, $C\cap E_i\ne\emptyset$ for
$1\le i\le r\}$.\ \Qed

\medskip

\qquad\grheadb\
Suppose that we have a compact set $K\subseteq X$
and open sets $G_i\subseteq X$ of finite measure, for $i<r$, and set

\Centerline{$V=\{C:C\in\Cal C$, $C\cap K=\emptyset$,
$C\cap G_i\ne\emptyset$ for every $i<r\}$.}

\noindent Then $\tilde\nu V=\tilde\nuprime V$.   \Prf\ Let $\Cal E$
be the finite subalgebra of $\Cal PX$ generated
by $\{G_i:i<r\}\cup\{K\}$, and $\Cal A$ the set of atoms of $\Cal E$
included in $K\cup\bigcup_{i<r}G_i$.
For $\Cal I\subseteq\Cal A$ set

\Centerline{$V_{\Cal I}=\{C:C\in\Cal C$, $C\cap E\ne\emptyset$ for
$E\in\Cal I$, $C\cap E=\emptyset$ for $E\in\Cal A\setminus\Cal I\}$.}

\noindent Then $V=\bigcup_{\Cal I\in\frak I}V_{\Cal I}$, where

$$\eqalign{\frak I
&=\{\Cal I:\Cal I\subseteq\Cal A,\,A\cap K=\emptyset
  \text{ for every }A\in\Cal I,\cr
&\mskip100mu\text{ for every }i<r\text{ there is an }A\in\Cal I
  \text{ such that }A\subseteq G_i\}.\cr}$$

\noindent Now ($\alpha$) shows that
$\tilde\nu V_{\Cal I}=\tilde\nuprime V_{\Cal I}$ for every
$\Cal I\subseteq\Cal A$, so that $\tilde\nu V=\tilde\nuprime V$.
Since sets $V$ of the type described form a base for the Fell
topology closed under finite intersections, $\tilde\nu=\tilde\nuprime$
(415H(v)).\ \Qed

This completes the proof of (a).

\medskip

{\bf (b)(i)} In the construction of 495B and (a-i) above, all the
normalized subspace measures $\mu'_H$ are Radon measures (416Rb), while of
course all the Poisson distributions $\nu_H$ are Radon measures, so the
product measure $\lambda$ on
$\Omega=\prod_{H\in\Cal H}\Bbb N\times H^{\Bbb N}$ has an extension to a
Radon measure $\tilde\lambda$ (417Q).
Let $\Cal W$ be the family of those sets $W\subseteq\Cal PX$
such that $\tilde\nu_0W$ and $\tilde\lambda\phi^{-1}[E]$ are defined and
equal.   Then $\Cal W$ is a Dynkin class.   So if
$\Cal W_0\subseteq\Cal W$ is closed under finite intersections,
the $\sigma$-algebra of subsets of $\Cal PX$ generated by $\Cal W_0$ is
included in $\Cal W$.   By (a-ii), $\Tau_0\subseteq\Cal W$.

\medskip

\quad{\bf (ii)} Let $\frak S_0$ be the topology on $\Cal PX$ generated by
sets of the form $\{S:S\cap G\ne\emptyset\}$ where $G\subseteq X$ is open.
(So $\frak S_0$ is coarser than the topology $\frak S$ of (a-iii) above.)
Then $\phi:\Omega\to\Cal PX$ is continuous for the product topology
$\frak U$ on $\Omega$ and $\frak S_0$ on $\Cal PX$.   \Prf\ If
$G\subseteq X$ is open, then

\Centerline{$\phi^{-1}[\{S:S\cap G\ne\emptyset\}]
=\Omega\cap\bigcap_{i<r}\bigcup_{H\in\Cal H,j\in\Bbb N}
  \{\omega:j<m_H(\omega),\,x_{Hj}(\omega)\in G_i\}$}

\noindent is open;  by 4A2B(a-ii), this is enough.\ \Qed

\medskip

\quad{\bf (iii)} $\frak S_0\subseteq\Cal W$.   \Prf\ Because $\mu$ is
locally finite, the family $\Cal U$ of sets of the form

\Centerline{$\{S:S\cap G_i\ne\emptyset$ for $i<r\}$,}

\noindent where $G_i\subseteq X$ is an open set of finite measure for each
$i<r$, is a base for $\frak S_0$;  and
$\Cal U\subseteq\Tau_0\subseteq\Cal W$.   So if $W\in\frak S_0$,
$\Cal V=\{V:V\in\frak S_0\cap\Tau_0$, $V\subseteq W\}$ is an
upwards-directed family of sets with union $W$.   Since $\tilde\nu_0$ and
$\tilde\lambda$ are both $\tau$-additive, and $\phi^{-1}[V]$ is open for
every $V\in\Cal V$,

\Centerline{$\tilde\lambda\phi^{-1}[W]
=\sup_{V\in\Cal V}\tilde\lambda\phi^{-1}[V]
=\sup_{V\in\Cal V}\tilde\nu_0V
=\tilde\nu_0W$,}

\noindent and $W\in\Cal W$.\ \Qed

\medskip

\quad{\bf (iv)} If $G\subseteq X$ is open and $n\in\Bbb N$,
$W=\{S:\#(S\cap G)\ge n\}$ belongs to $\frak S_0$.   \Prf\

$$\eqalign{W
&=\bigcup\{\{S:S\cap G_i\ne\emptyset\text{ for every }i<n\}:
  \ofamily{i}{n}{G_i}\text{ is a disjoint family}\cr
&\mskip100mu
  \text{of open subsets of }G\text{ of finite measure}\}.
  \text{ \Qed}\cr}$$

\medskip

\quad{\bf (v)} If $E_0,\ldots,E_r\subseteq X$ are sets of finite measure,
and $n_0,\ldots,n_r\in\Bbb N$, then

\Centerline{$V=\{S:\#(S\cap E_i)\ge n_i$ for $i\le r\}$}

\noindent belongs to $\Cal W$.   \Prf\ Let $\epsilon>0$.   Then there is a
$\delta>0$ such that $1-e^{-\gamma\delta}\le\epsilon$.   Let
$G_0,\ldots,G_r$ be open sets such that $E_i\subseteq G_i$ for $i\le r$ and
$\sum_{i=0}^r\mu(G_i\setminus E_i)\le\delta$.   Set

\Centerline{$W=\{S:\#(S\cap G_i)\ge n_i$ for $i\le r\}$,
\quad$W_0=\{S:S\cap\bigcup_{i\le r}G_i\setminus E_i\ne\emptyset\}$;}

\noindent then $W\in\frak S_0$ and $W_0\in\Tau_0$, so both belong to
$\Cal W$, while

\Centerline{$\tilde\nu_0W_0=\tilde\lambda\phi^{-1}[W_0]
=1-\exp(-\gamma\mu(\bigcup_{i\le r}G_i\setminus E_i))\le\epsilon$.}

Now

\Centerline{$W\setminus W_0\subseteq V\subseteq W$,
\quad$\phi^{-1}[W]\setminus\phi^{-1}[W_0]\subseteq\phi^{-1}[V]
\subseteq\phi^{-1}[W]$.}

\noindent So

\Centerline{$\tilde\nu_0^*V-(\tilde\nu_0)_*V\le\epsilon$,
\quad$\tilde\lambda^*(\phi^{-1}[V])-\tilde\lambda_*(\phi^{-1}[V])
\le\epsilon$,
\quad$|\tilde\nu_0^*V-\tilde\lambda^*(\phi^{-1}[V])|\le\epsilon$.}

\noindent As $\epsilon$ is arbitrary (and $\tilde\nu_0$, $\tilde\lambda$
are complete), $V$ is measured by $\tilde\nu_0$, $\phi^{-1}[V]$ is measured
by $\tilde\lambda$, and

\Centerline{$|\tilde\nu_0V-\tilde\lambda\phi^{-1}[V]|
=|\tilde\nu_0^*V-\tilde\lambda^*(\phi^{-1}[V])|
=0$.  \Qed}

\medskip

\quad{\bf (vi)} If $E_0,\ldots,E_r\subseteq X$ are sets of finite measure,
$n_0,\ldots,n_r\in\Bbb N$ and $j\le r$, then

\Centerline{$\{S:\#(S\cap E_i)=n_i$ for $i<j$,
$\#(S\cap E_i)\ge n_i$ for $j\le i\le r\}$}

\noindent belongs to $\Cal W$.   \Prf\ Induce on $j$.   For $j=0$ we just
have the case of (v).   For the inductive step to $j+1$, we have

$$\eqalign{\{S:\#&(S\cap E_i)=n_i\text{ for }i\le j,\,
\#(S\cap E_i)\ge n_i\text{ for }j<i\le r\}\cr
&=\{S:\#(S\cap E_i)=n_i\text{ for }i<j,\,
\#(S\cap E_i)\ge n_i\text{ for }j\le i\le r\}\cr
&\mskip50mu
  \setminus\{S:\#(S\cap E_i)=n_i\text{ for }i<j,\,
    \#(S\cap E_j)\ge n_j+1,\cr
&\mskip308mu
    \#(S\cap E_i)\ge n_i\text{ for }j<i\le r\}\cr
&\in\Cal W\cr}$$

\noindent because $\Cal W$ is a Dynkin class.\ \Qed

\medskip

\quad{\bf (vii)}
If $E\in\Sigma$ has finite measure and does not include any
non-negligible singleton, then $\#(E\cap\phi(\omega))=g_E(\omega)$, as
defined in 495B, for
$\lambda$-almost every $\omega\in\Omega$.
\Prf\ Let $A_E$ be the set of those $\omega\in\Omega$ such that

\inset{{\it either} there are an $H\in\Cal H$ and $j\in\Bbb N$ such that
$\mu(H\cap E)=0$ and $x_{Hj}(\omega)\in E$

{\it or} there are an $H\in\Cal H$ and distinct $i$, $j\in\Bbb N$
such that $x_{Hi}(\omega)=x_{Hj}(\omega)\in E$.}

\noindent As observed in (a-i) above,
$\{H:H\in\Cal H$, $H\cap E\ne\emptyset\}$ is countable;  while
for any $H\in\Cal H$ and distinct $i$, $j\in\Bbb N$ the set
$\{\omega:x_{Hi}(\omega)=x_{Hj}(\omega)\in E\}$ is negligible because the
subspace measure on $E$ is atomless (414G/416Xa), so the
diagonal $\{(x,x):x\in E\}$ is negligible in $X^2$.   Consequently
$\lambda A_E=0$.   But $\#(E\cap\phi(\omega))=g_E(\omega)$ for every
$\omega\in\Omega\setminus A_E$.\ \Qed

\medskip

\quad{\bf (viii)} Now suppose that $E_0,\ldots,E_r\subseteq X$ are disjoint
sets of finite measure, none including any non-negligible singleton, and
$n_0,\ldots,n_r\in\Bbb N$.   Then

\Centerline{$V=\{S:S\subseteq X$,
$\#(S\cap E_i)=n_i$ for every $i\le r\}$}

\noindent belongs to $\Cal W$, by (vi).   Next,

\Centerline{$\phi^{-1}[V]=\{\omega:\#(E_i\cap\phi(\omega))=n_i$
for every $i\le r\}$,}

\noindent so

$$\eqalignno{\tilde\nu_0V
&=\tilde\lambda\phi^{-1}[V]
=\tilde\lambda\{\omega:g_{E_i}(\omega)=n_i\text{ for every }i\le r\}\cr
\displaycause{by (vii)}
&=\prod_{i=0}^r\tilde\lambda\{\omega:g_{E_i}(\omega)=n_i\}
=\prod_{i=0}^r\Bover{(\gamma\mu E_i)^{n_i}}{n_i!}e^{-\gamma\mu E_i}.\cr}$$

Finally, because $\tilde\nu_0^*\Cal C=1$ and $\tilde\nu$ is the subspace
measure on $\Cal C$,

$$\eqalign{\tilde\nu\{C:C\in\Cal C,\,\#(C\cap E_i)=n_i
\text{ for every }i\le r\}
&=\tilde\nu(V\cap\Cal C)
=\tilde\nu_0V\cr
&=\prod_{i=0}^r\Bover{(\gamma\mu E_i)^{n_i}}{n_i!}e^{-\gamma\mu E_i}.\cr}$$

\noindent This completes the proof of (b).

\medskip

{\bf (c)(i)} Taking $\Tau\supseteq\Tau_0$ to be the $\sigma$-algebra of
subsets of $\Cal PX$ generated by sets of the form
$\{S:\#(S\cap E)=n\}$ where $E\in\Sigma^f$ and $n\in\Bbb N$,
(b-viii) tells us that $\tilde\nu_0\restr\Tau$ satisfies the conditions
of 495D, so its completion $\nu$
is the Poisson point process as defined in 495E.   Because $\tilde\nu_0$
is complete, it extends $\nu$.  (The identity map from $\Cal PX$ to itself
is \imp\ for $\tilde\nu_0$ and $\tilde\nu_0\restr\Tau$, therefore also for
their completions $\tilde\nu_0$ and $\nu$.)   Since $\Cal C$ has full outer
measure for $\tilde\nu_0$, by (a-v), it has full outer measure for $\nu$,
and

\Centerline{$\nu_{\Cal C}(V\cap\Cal C)=\nu V=\tilde\nu_0V
=\tilde\nu(V\cap\Cal C)$}

\noindent whenever $\nu$ measures $V$, so $\tilde\nu$ extends
$\nu_{\Cal C}$.

\medskip

\quad{\bf (ii)} If $\mu$ is $\sigma$-finite, then there is a sequence
$\sequencen{H_n}$ of open sets of finite measure covering $X$.   For each
$n\in\Bbb N$, $\{S:S\subseteq X$, $S\cap H_n$ is finite$\}$ is
$\nu$-conegligible.   So $W=\{S:S\cap H_n$ is finite for every $n\}$ is
$\nu$-conegligible.   But $W\subseteq\Cal C$, so $\Cal C$ is
$\nu$-conegligible.

\medskip

{\bf (d)} If $X$ is locally compact then $\Cal C$ is Hausdorff
(4A2T(e-ii));  so $\tilde\nu$, being a quasi-Radon probability
measure on a compact
Hausdorff space, is a Radon measure (416G).

\medskip

{\bf (e)} Now suppose that $X$ is second-countable.

\medskip

\quad{\bf (i)} $\Cal C$ has a
countable network consisting of sets in $\Tau_{\Cal C}$, the subspace
$\sigma$-algebra induced by the $\sigma$-algebra $\Tau$ of (c-i).
\Prf\ Let $\Cal U$ be a countable base for $\frak T$,
closed under finite unions, consisting of sets of finite measure.
For $U_0\in\Cal U$ and finite $\Cal U_0\subseteq\Cal U$, set

\Centerline{$V(U_0,\Cal U_0)
=\{C:C\in\Cal C$, $C\cap U_0=\emptyset$,
$C\cap U\ne\emptyset$ for every $U\in\Cal U\}\in\Tau_{\Cal C}$.}

\noindent If $W\subseteq\Cal C$ is open for the Fell topology and
$C_0\in W$, there are a compact set $K\subseteq X$ and a finite family
$\Cal G\subseteq\frak T$ such that

\Centerline{$C_0
\in\{C:C\in\Cal C$, $C\cap K=\emptyset$, $C\cap G\ne\emptyset$ for every
$G\in\Cal G\}$.}

\noindent For $G\in\Cal G$ let $y_G$ be a point of $C_0\cap G$.
Now there are a $U_0\in\Cal U$ such that
$K\subseteq U_0\subseteq X\setminus C$ and a family
$\family{G}{\Cal G}{U_G}$ in $\Cal U$ such that $x_G\in U_G\subseteq G$ for
every $G\in\Cal G$.   In this case,

\Centerline{$C_0\in V(U_0,\{U_G:G\in\Cal G\})\subseteq W$.}

\noindent As $C_0$ and $W$ are arbitrary, the countable set
$\{V(U_0,\Cal U_0):U_0\in\Cal U$, $\Cal U_0\in[\Cal U]^{<\omega}\}$ is a
network for the topology of $\Cal C$.\ \Qed

\medskip

\quad{\bf (ii)} Since $\nu_{\Cal C}$ measures every set in this countable
network, it is a topological measure.
Since it is also complete, and $\tilde\nu$, being a quasi-Radon probability
measure, is the completion of its restriction to the Borel $\sigma$-algebra
of $\Cal C$, $\nu_{\Cal C}$ extends $\tilde\nu$, and the two must be equal.
}%end of proof of 495N

\leader{495O}{Proposition}\discrversionA{\footnote{Revised 2008.}}{}
Let $(X,\frak T)$ be a $\sigma$-compact
locally compact Hausdorff space and $M^{\infty+}_{\text{R}}(X)$ the set
of Radon measures on $X$.   Give $M^{\infty+}_{\text{R}}(X)$ the topology
generated by sets of the form $\{\mu:\mu G>\alpha\}$ and
$\{\mu:\mu K<\alpha\}$ where $G\subseteq X$ is open, $K\subseteq X$ is
compact and $\alpha\in\Bbb R$.   Let $\Cal C$ be the space of closed
subsets of $X$ with its Fell topology, and
$P_{\text{R}}(\Cal C)$ the set of Radon probability measures on
$\Cal C$ with its narrow topology\cmmnt{ (definition:  437Jd)}.   For
$\mu\in M^{\infty+}_{\text{R}}(X)$ and $\gamma>0$ let
$\tilde\nu_{\mu,\gamma}$ be the Radon measure on
$\Cal C$ defined from $\mu$ and $\gamma$ as in 495N.   Then the function
$(\mu,\gamma)\mapsto\tilde\nu_{\mu,\gamma}:
M^{\infty+}_{\text{R}}(X)\times\ooint{0,\infty}\to P_{\text{R}}(\Cal C)$ is
continuous.

\proof{{\bf (a)} Note that because $X$ is $\sigma$-compact, every Radon
measure on $X$ is $\sigma$-finite, therefore outer regular with respect to
the open sets (412Wb), and we can apply 495N to build the measures
$\tilde\nu_{\mu,\gamma}$.
Just as in 495E for ordinary Poisson point processes, the
uniqueness assertion in 495Na assures us that
$\tilde\nu_{\mu,\gamma}=\tilde\nu_{\gamma\mu,1}$ for all $\gamma$ and
$\mu$.   Of course the sets

\Centerline{$\{(\mu,\gamma):\gamma\mu G>\alpha\}$,
\quad$\{(\mu,\gamma):\gamma\mu K<\alpha\}$}

\noindent where $G\subseteq X$ is open,
$K\subseteq X$ is compact and $\alpha\in\Bbb R$, are all open in
$M^{\infty+}_{\text{R}}(X)\times\ooint{0,\infty}$;  so the map
$(\mu,\gamma)\mapsto\gamma\mu$ is continuous.
It will therefore be enough to show that the map
$\mu\mapsto\tilde\nu_{\mu,1}:M^{\infty+}_{\text{R}}(X)\to P_{\text{R}}(\Cal C)$ is
continuous.   Write $\tilde\nu_{\mu}$ for $\tilde\nu_{\mu,1}$.

\medskip

{\bf (b)} Fix an open set $W_0\subseteq\Cal C$, $\alpha_0>0$ and
$\mu_0\in M^{\infty+}_{\text{R}}(X)$ such that $\tilde\nu_{\mu_0}W_0>\alpha_0$.
Let $\Cal E$ be the family of relatively compact Borel subsets $E$ of
$X$ such that $\mu_0(\partial E)=0$.   Then $\Cal E$ is a subring of
$\Cal PX$ (4A2Bi).   Also
$\mu\mapsto\mu E:M^{\infty+}_{\text{R}}(X)\to\coint{0,\infty}$ is
continuous at $\mu_0$ for every $E\in\Cal E$.   \Prf\ If $E\in\Cal E$
and $\epsilon>0$, then

$$\eqalign{\{\mu:\mu_0E-\epsilon&<\mu E<\mu_0E+\epsilon\}\cr
&\supseteq\{\mu:\mu(\interior E)>\mu_0(\interior E)-\epsilon,\,
  \mu\overline{E}<\mu_0\overline{E}+\epsilon\}\cr}$$

\noindent is a neighbourhood of $\mu_0$.\ \Qed

\medskip

{\bf (c)} Next, $\Cal U=\Cal E\cap\frak T$ is a base for $\frak T$
(411Gi).   It follows that the family $\Cal V$ of sets of the form

\Centerline{$\{C:C\in\Cal C$, $C\cap U_i\ne\emptyset$ for $i<r$,
$C\cap\overline{U}=\emptyset\}$,}

\noindent where $U$, $U_0,\ldots\in\Cal U$, is a base for the
Fell topology on $\Cal C$.   \Prf\
If $W\subseteq\Cal C$ is open for the Fell topology and
$C_0\in W$, there are $r\in\Bbb N$, open sets $G_i\subseteq X$ for
$i<r$ and a compact set $K\subseteq X$ such that

\Centerline{$C_0\in\{C:C\cap G_i\ne\emptyset$ for each $i<r$,
$C\cap K=\emptyset\}\subseteq W$.}

\noindent For each $i<r$ choose $x_i\in C_0\cap G_i$ and
$U_i\in\Cal U$
such that $x_i\in U_i\subseteq G_i$.   Because $X$ is locally compact
and Hausdorff, it is regular, so every point of $K$ belongs to a member
of $\Cal U$ with closure disjoint from $C_0$;  because $\Cal U$ is
closed under finite unions, there is a $U\in\Cal U$ such that
$K\subseteq U$ and $C_0\cap\overline{U}=\emptyset$.   Now

\Centerline{$\{C:C\in\Cal C$, $C\cap U_i\ne\emptyset$ for $i<r$,
$C\cap\overline{U}=\emptyset\}$}

\noindent belongs to $\Cal V$, contains $C_0$ and is included in
$W$.   As $C_0$ and $W$ are arbitrary, $\Cal V$ is a base for the
Fell topology on $\Cal C$.\ \Qed

\medskip

{\bf (d)} If $V\in\Cal V$, then
$\mu\mapsto\tilde\nu_{\mu}V:M^{\infty+}_{\text{R}}(X)\to[0,1]$ is
continuous at $\mu_0$.   \Prf\ Express $V$ as
$\{C:C\in\Cal C$, $C\cap U_i\ne\emptyset$ for $i<r$,
$C\cap\overline{U}=\emptyset\}$, where $U_i$, $U\in\Cal U$.
Let $\Cal A$ be the set of atoms of the finite subring of $\Cal E$
generated by $\{U_i:i<r\}\cup\{\overline{U}\}$.   For
$\Cal I\subseteq\Cal A$ set

\Centerline{$V_{\Cal I}=\{C:C\in\Cal C$,
$\Cal I=\{A:A\in\Cal A$, $C\cap A\ne\emptyset\}\}$.}

\noindent Let $\frak I$ be the set of those $\Cal I\subseteq\Cal A$ such
that $A\cap\overline{U}=\emptyset$ for every $A\in\Cal I$ and for every
$i<r$ there is an $A\in\Cal I$ such that $A\subseteq U_i$.   Then
$\family{\Cal I}{\frak I}{V_{\Cal I}}$ is a partition of $V$.
Moreover, for any $\mu\in M^{\infty+}_{\text{R}}(X)$ and
$\Cal I\subseteq\Cal A$,

\Centerline{$\tilde\nu_{\mu}V_{\Cal I}
=\prod_{A\in\Cal A\setminus\Cal I}e^{-\mu A}
  \cdot\prod_{A\in\Cal I}(1-e^{-\mu A})$.}

\noindent Since each $\mu\mapsto\mu A$ is continuous at $\mu_0$, by (a),
so are the functionals $\mu\mapsto\tilde\nu_{\mu}V_{\Cal I}$, for
$\Cal I\subseteq\Cal A$, and $\mu\mapsto\tilde\nu_{\mu}V
=\sum_{\Cal I\in\frak I}\tilde\nu_{\mu}V_{\Cal I}$.\ \Qed

\medskip

{\bf (e)} Let $\Cal V^*$ be the family of Borel subsets $V$ of $\Cal C$
such that
$\mu\mapsto\tilde\nu_{\mu}V:M^{\infty+}_{\text{R}}(X)\to\coint{0,\infty}$
is continuous at $\mu_0$.   Then $\Cal V\subseteq\Cal V^*$ (by (c)),
$\Cal C\in\Cal V^*$ and $V\setminus V'\in\Cal V^*$ whenever $V$,
$V'\in\Cal V^*$ and $V'\subseteq V$.   Because $\Cal V$ is closed under
finite intersections, it follows that $\Cal V^*$ includes the algebra of
subsets of $\Cal C$ generated by $\Cal V$ (313Ga);  in particular, any
finite union of members of $\Cal V$ belongs to $\Cal V^*$.

\medskip

{\bf (f)} Let us return to the open set $W_0\subseteq\Cal C$ and the
$\alpha_0\in\Bbb R$ of part (a).   Because $\tilde\nu_{\mu_0}$ is
$\tau$-additive and $\Cal V$ is a base for the topology of $\Cal C$ ((b)
above), there is a finite family $\Cal V_0\subseteq\Cal V$ such that
$V_0=\bigcup\Cal V_0$ is included in $W_0$ and
$\tilde\nu_{\mu_0}V_0>\alpha_0$.   But this means that

\Centerline{$\{\mu:\mu\in M^{\infty+}_{\text{R}}(X)$,
$\tilde\nu_{\mu}W_0>\alpha_0\}
\supseteq\{\mu:\mu\in M^{\infty+}_{\text{R}}(X)$,
$\tilde\nu_{\mu}V_0>\alpha_0\}$}

\noindent is a neighbourhood of $\mu_0$.   As $\mu_0$ is arbitrary,
$\{\mu:\tilde\nu_{\mu}W_0>\alpha_0\}$ is open;  as $W_0$ and $\alpha_0$
are arbitrary, $\mu\mapsto\tilde\nu_{\mu}$ is continuous.
}%end of proof of 495O

\leader{495P}{}\cmmnt{ There are many constructions which, in
particular cases, can be used as an alternative to the method of
495B-495D in setting up Poisson point processes.   I give one which
applies to the half-line $\coint{0,\infty}$ with Lebesgue measure.

\medskip

\noindent}{\bf Theorem} Let $\gamma>0$, and let $\nu$ be the Poisson
point process on $\coint{0,\infty}$, with Lebesgue measure, with density
$\gamma$.   Let $\lambda_0$ be the exponential distribution with
expectation $1/\gamma$, regarded as a Radon probability measure on
$\ooint{0,\infty}$, and $\lambda$  the corresponding product measure on
$\ooint{0,\infty}^{\Bbb N}$.   Define
$\phi:\ooint{0,\infty}^{\Bbb N}\to\Cal P(\coint{0,\infty})$ by setting
$\phi(x)=\{\sum_{i=0}^nx(i):n\in\Bbb N\}$ for
$x\in\ooint{0,\infty}^{\Bbb N}$.   Then $\phi$ is a measure space
isomorphism between $\ooint{0,\infty}^{\Bbb N}$ and a $\nu$-conegligible
subset of $\Cal P(\coint{0,\infty})$.

\cmmnt{\medskip

\noindent{\bf Remark} As I seem not to have mentioned exponential
distributions earlier in this treatise, I remark now that the
{\bf exponential distribution} with parameter $\gamma$ has distribution
function

\Centerline{$F(t)=0$ if $t<0$, $1-e^{-\gamma t}$ if $t\ge 0$,}

\noindent and probability density function

\Centerline{$f(t)=0$ if $t\le 0$, $\gamma e^{-\gamma t}$ if $t>0$;}

\noindent its expectation is

\Centerline{$\biggerint_0^{\infty}\gamma te^{-\gamma t}dt
=-\int_0^{\infty}\Bover{d}{dt}
  \bigl(\Bover{\gamma t+1}{\gamma}e^{-\gamma t}\bigr)dt
=\Bover1{\gamma}$.}

\noindent Because (when regarded as a Radon probability measure on
$\Bbb R$, following my ordinary rule set out in \S271) it gives
measure zero to $\ocint{-\infty,0}$, it can be identified with the
subspace measure on $\ooint{0,\infty}$, as here.
}%end of comment

\proof{{\bf (a)} For each $n\in\Bbb N$, $\#(S\cap[0,n])$ is finite for
$\nu$-almost every $S$;  so the set

\Centerline{$Q_0
=\{S:S\subseteq\coint{0,\infty}$, $\#(S\cap[0,n])$ is finite for every
$n\}$}

\noindent is $\nu$-conegligible.   Next, the sets
$\{S:S\cap\coint{n,n+1}\ne\emptyset\}$ are $\nu$-independent and have
measure $1-e^{-\gamma}>0$, so

\Centerline{$\{S:S\cap\coint{n,n+1}\ne\emptyset$ for infinitely many
$n\}$}

\noindent is $\nu$-conegligible (273K).   Finally, $\nu\{S:0\in S\}=0$,
so $Q=\{S:S\in Q_0$, $0\notin S$, $S$ is infinite$\}$ is
$\nu$-conegligible.   For
$S\in Q$, let $\sequencen{g_n(S)}$ be the increasing enumeration of $S$.
Let $\Tau$ be the $\sigma$-algebra of subsets of
$\Cal P(\coint{0,\infty})$ generated by sets of the form
$\{S:\#(S\cap E)=n\}$ where $E\subseteq\coint{0,\infty}$ has finite
measure and $n\in\Bbb N$.   Then, for $n\in\Bbb N$ and $\alpha\ge 0$,
$\{S:g_n(S)\le\alpha\}=\{S:\#(S\cap[0,\alpha])\ge n+1\}$ belongs to
$\Tau$, so $g_n$ is $\Tau$-measurable.   Set $h_0(S)=g_0(S)$,
$h_n(S)=g_n(S)-g_{n-1}(S)$ for $n\ge 1$,
and $h(S)=\sequencen{h_n(S)}$;  then $h:Q\to\ooint{0,\infty}^{\Bbb N}$
is a bijection, and its inverse is $\phi$.

\medskip

{\bf (b)} For each $k\in\Bbb N$,
$I_S=\{i:i\in\Bbb N$, $S\cap\coint{2^{-k}i,2^{-k}(i+1)}\ne\emptyset\}$
is infinite for every $S\in Q$.   So we can define
$g_{kn}:Q\to\ooint{0,\infty}$, for each $n$, by taking
$g_{kn}(S)=2^{-k}(j+1)$ if $j\in I_S$ and $\#(I_S\cap j)=n$.   Because
all the sets $\{S:j\in I_S\}$ belong to $\Tau$, each $g_{kn}$ is
$\Tau$-measurable, and $\sequence{k}{g_{kn}}$ is a non-increasing
sequence with limit $g_n$.   Set $h_{k0}(S)=g_{k0}(S)$ and
$h_{kn}(S)=g_{kn}(S)-g_{k,n-1}(S)$ for $n\ge 1$.   Then
$h_n=\lim_{k\to\infty}h_{kn}$.

\medskip

{\bf (c)} For any $n\in\Bbb N$, $j_0,\ldots,j_n\in\Bbb N$, $k\in\Bbb N$,
set $j'_r=\sum_{i=0}^rj_i$ for $r\le n$.   Then

$$\eqalignno{\nu\{S:S\in Q,\,&h_{ki}(S)
=2^{-k}(j_i+1)\text{ for every }i\le n\}\cr
&=\nu\{S:S\in Q,\,g_{kr}(S)=2^{-k}(r+1+j'_r)
   \text{ for every }r\le n\}\cr
&=\nu\{S:S\cap\coint{2^{-k}(r+j'_r),2^{-k}(r+1+j'_r)}\ne\emptyset
     \text{ for every }r\le n,\cr
&\mskip70mu
  S\cap\coint{0,2^{-k}j_0}=\emptyset,\cr
&\mskip70mu
  S\cap\coint{2^{-k}(r+1+j'_r),2^{-k}(r+1+j'_{r+1})}=\emptyset
     \text{ for every }r<n\}\cr
&=(1-\exp(-2^{-k}\gamma))^{n+1}\exp(-2^{-k}\gamma j_0)
  \prod_{r<n}\exp(-2^{-k}\gamma(j'_{r+1}-j'_r))\cr
&=\prod_{i=0}^n(1-\exp(-2^{-k}\gamma))\exp(-2^{-k}\gamma j_i).\cr}$$

\noindent This means that the $h_{ki}$, for $i\in\Bbb N$ are
independent, with

\Centerline{$\Pr(h_{ki}=2^{-k}(j+1))
=(1-e^{-2^{-k}\gamma})e^{-2^{-k}\gamma j}$}

\noindent for each $j$.
Since $h_{ki}\to h_i\,\,\nu$-a.e.\ for each $i$, $\sequence{i}{h_i}$ is
also independent (367W).   Now, for any $\alpha>0$,

$$\eqalignno{\Pr(h_{ki}\le\alpha)
&=\sum_{2^{-k}(j+1)\le\alpha}
  (1-\exp(-2^{-k}\gamma))\exp(-2^{-k}\gamma j)\cr
&=1-\exp(-2^{-k}\gamma\lfloor 2^k\alpha\rfloor)
\to 1-e^{-\gamma\alpha}\cr}$$

\noindent as $k\to\infty$.   So

$$\eqalignno{\Pr(h_i\le\alpha)
&=\inf_{\beta>\alpha}\liminf_{k\to\infty}\Pr(h_{ki}\le\beta)\cr
\displaycause{271L}
&=\inf_{\beta>\alpha}1-e^{-\gamma\beta}
=1-e^{-\gamma\alpha}\cr}$$

\noindent for every $\alpha\ge 0$ and every $i\in\Bbb N$.

\medskip

{\bf (d)} Accordingly $\sequence{i}{h_i}$ is an independent sequence of
random variables, each exponentially distributed with expectation
$1/\gamma$.   It follows that
$h:Q\to\ooint{0,\infty}^{\Bbb N}$ is \imp\ for the subspace measure
$\nu_Q$ and $\lambda$ (254G).

Observe next that if $E\subseteq\coint{0,\infty}$ is Lebesgue measurable
and $n\in\Bbb N$, then

\Centerline{$\{x:x\in\ooint{0,\infty}^{\Bbb N}$, $\#(\phi(x)\cap E)=n\}
=\bigcup_{I\in[\Bbb N]^n}\{x:\sum_{i=0}^jx(i)\in E\iff j\in I\}
\in\Lambda$,}

\noindent writing $\Lambda$ for the domain of $\lambda$.   So $\phi$
is $(\Lambda,\Tau)$-measurable.   Now, for any $W\in\Tau$,

\Centerline{$\lambda\phi^{-1}[W]=\nu(h^{-1}[\phi^{-1}[W]])
=\nu(W\cap Q)=\nu W$.}

\noindent So $\phi$ is \imp\ for $\lambda$ and for $\nu\restr\Tau$.
Since $\lambda$ is complete and $\nu$ is defined as the completion of
its restriction to $\Tau$, $\phi$ is \imp\ for $\lambda$ and $\nu$.
Thus $\phi$ and $h$ are the two halves of an isomorphism between
$(\ooint{0,\infty}^{\Bbb N},\lambda)$ and the subspace $(Q,\nu_Q)$, as
claimed.
}%end of proof of 495P

\exercises{\leader{495X}{Basic exercises $\pmb{>}$(a)}
%\spheader 495Xa
\discrversionA{\footnote{New 2008.
Other exercises have been rearranged:
%4{}95Xa is now 495Xb
%4{}95Xb is now 495Xc
%4{}95Xc is now 495Xd
%4{}95Xd is now 495Xe
%4{}95Xe is now 495Xf
%4{}95Xf is now 495Xg
%4{}95Xg is now 495Xh
4{}95Xa-4{}95Xg are now 495Xb-495Xh,
4{}95Xh is now 495Yb,
%4{}95Xi is now 495Xi,
%4{}95Xj is now 495Xj,
%4{}95Xk is now 495Xk,
%4{}95Xl is now 495Xl,
4{}95Xm has been deleted,
%4{}95Xn is now 495Xm,
%4{}95Xo is now 495Xn
%4{}95Xp is now 495Xo
4{}95Xn-4{}95Xp are now 495Xm-495Xo,
%4{}95Ya is now 495Ya
4{}95Yb is now 495Yc,
4{}95Yc is now 495Yd.
}}{}
Let $(X,\Sigma,\mu)$ be an atomless countably separated measure
space\cmmnt{ (definition:  343D)} and $\gamma>0$.   Let $\nu$ be a complete
probability measure on $\Cal PX$ such that
$\nu\{S:S\subseteq X$, $S\cap E=\emptyset\}$ is defined and equal to
$e^{-\gamma\mu E}$ whenever $E\in\Sigma$ has finite measure.   Show that
$\nu$ extends the Poisson process with density $\gamma$ defined in 495D.
%495D

\spheader 495Xb
Let $(X,\Sigma,\mu)$ be an atomless measure space, and
$\nu$ a Poisson point process on $X$.   (i)
Show that $[X]^{\le\omega}$ has full outer measure for $\nu$.   (ii)
Show that if $\mu$ is semi-finite then $[X]^{\le\omega}$ is
conegligible iff $\mu$ is $\sigma$-finite.   (iii) Show that if $\mu$ is
semi-finite, then $[X]^{<\omega}$ is non-negligible iff $[X]^{<\omega}$
is conegligible iff $\mu$ is totally finite.
%495E

\spheader 495Xc(i) Let $(X,\Sigma,\mu)$ be an atomless measure space and
$\Cal E$ a countable partition of $X$ into measurable sets.
Let $\nu$ be the Poisson point process on $X$ with density
$1$, and for each $E\in\Cal E$ let $\nu_E$ be the Poisson point
process on $E$ with density $1$ corresponding to the subspace
measure $\mu_E$ on $E$. Let $\lambda$ be the product of the family
$\family{E}{\Cal E}{\nu_E}$.   Show that the map
$S\mapsto\family{E}{\Cal E}{S\cap E}:
\Cal PX\to\prod_{E\in\Cal E}\Cal PE$ is a measure space isomorphism for
$\nu$ and $\lambda$.
(ii) Let $(X,\Sigma,\mu)$ be a strictly localizable measure space and
$\familyiI{X_i}$ a decomposition of $X$.
Let $\nu$ be the Poisson point process on $X$ with density
$1$, and for each $i\in I$ let $\nu_i$ be the Poisson point
process on $X_i$ with density $1$ corresponding to the subspace
measure $\mu_{X_i}$ on $X_i$. Let $\lambda$ be the product of the family
$\familyiI{\nu_i}$.   Show that the map
$S\mapsto\familyiI{S\cap X_i}:\Cal PX\to\prod_{i\in I}\Cal PX_i$
is \imp\ for $\nu$ and $\lambda$.
%495E

\sqheader 495Xd Let $(X,\Sigma,\mu)$ be an atomless
measure space and $\frak T$ a topology on $X$ such that $X$ is covered
by a sequence of open sets of finite outer measure.   Let $\nu$
be a Poisson point process on $X$.   Show that
$\nu$-almost every set $S\subseteq X$ is locally finite in the sense
that $X$ is covered by the open sets meeting $S$ in finite sets;  in
particular, if $X$ is T$_1$, then
$\nu$-almost every subset of $X$ is closed.
%495E

\sqheader 495Xe(i) Let $(X,\Sigma,\mu)$ be an atomless measure space, and
for $\gamma>0$ let $\nu_{\gamma}$ be the Poisson point process on $X$
with density $\gamma$.   Show that for any $\gamma$, $\delta>0$ the map
$(S,T)\mapsto S\cup T:\Cal PX\times\Cal PX\to\Cal PX$ is \imp\ for the
product measure $\nu_{\gamma}\times\nu_{\delta}$ and
$\nu_{\gamma+\delta}$.   (ii) Let $X$ be a set, $\Sigma$ a $\sigma$-algebra
of subsets of $X$, and $\familyiI{\mu_i}$ a
countable family of measures with domain
$\Sigma$ such that $\mu=\sum_{i\in I}\mu_i$ is atomless.   Let $\nu$,
$\nu_i$ be the Poisson point processes with density $1$ corresponding to
the measures $\mu$, $\mu_i$.   Show that the map
$\familyiI{S_i}\mapsto\bigcup_{i\in I}S_i:(\Cal PX)^I\to\Cal PX$ is \imp\
for the product measure $\prod_{i\in I}\nu_i$ and $\nu$.
(iii) Compare with 495Xc(i).
%495E

\spheader 495Xf Let $(X,\Sigma,\mu)$ be an atomless semi-finite measure
space, and $\nu$ the Poisson point process on $X$ with density $1$.
Show that $\nu$ is perfect iff $\mu$ is.
%495F

\spheader 495Xg Let $(\frak A,\bar\mu)$ be a measure algebra.   Show
that there is a probability measure $\lambda$ on $\BbbR^{\frak A^f}$
such that (i) for every $a\in\frak A^f$ the corresponding marginal
measure on $\Bbb R$ is the Poisson distribution with expectation
$\bar\mu a$ (ii) whenever $a_0,\ldots,a_n\in\frak A^f$ are disjoint, the
functions $z\mapsto z(a_i):\BbbR^{\frak A^f}\to\Bbb R$ are
stochastically independent with respect to $\lambda$.   \Hint{prove the
result for finite $\frak A$ and use 454D.}   Use this to prove 495J.
%495J

\spheader 495Xh Let $U$ be a Hilbert space.   Show
that there are a probability algebra $(\frak B,\bar\lambda)$ and a
linear operator $T:U\to L^2(\frak B)$ such that (i)
for every $u\in U$, $Tu$ has a normal distribution
with expectation $0$ and variance $\|u\|_2^2$ (ii) if $\familyiI{u_i}$
is an orthogonal family in $U$ then $\familyiI{Tu_i}$
is $\bar\lambda$-independent.   \Hint{see the proof of 456K.}
%495K

\spheader 495Xi Let $\nu$ be the Poisson point process with density $1$
on $\coint{0,\infty}$ with Lebesgue measure.   Set
$Q_0=\{S:S\subseteq\coint{0,\infty}$, $S\cap[0,n]$ is finite for every
$n\}$ and for $S\in Q_0$ set $\psi(S)(t)=\#(S\cap[0,t])$ for
$t\in\coint{0,\infty}$.   Show that $\psi$ is \imp\ for the subspace
measure $\nu_{Q_0}$ and the distribution on $\BbbR^{\coint{0,\infty}}$
corresponding to the Poisson process of 455Xh.
%495K

\spheader 495Xj Let $(Y,\Tau,\nu)$ be a probability space, and
$\lambda_0$ the exponential distribution with expectation $1$, regarded
as a Radon measure on $\ooint{0,\infty}$.   Let
$\lambda$ be the product measure $\lambda_0^{\Bbb N}\times\nu^{\Bbb N}$
on $\ooint{0,\infty}^{\Bbb N}\times Y^{\Bbb N}$.   Set
$\phi(x,y)=\{(\sum_{i=0}^nx(i),y(n)):n\in\Bbb N\}$ for
$x\in\ooint{0,\infty}^{\Bbb N}$ and $y\in Y^{\Bbb N}$.   Show that
$\phi:\ooint{0,\infty}^{\Bbb N}\times Y^{\Bbb N}
\to\Cal P(\ooint{0,\infty}\times Y)$ is a measure space isomorphism
between $(\ooint{0,\infty}^{\Bbb N}\times Y^{\Bbb N},\lambda)$ and a
conegligible set for the Poisson point process on
$\ooint{0,\infty}\times Y$ with density $1$ for the c.l.d.\ product
measure $\mu_L\times\nu$, where $\mu_L$ is Lebesgue measure.
%495K

\spheader 495Xk Let $\Cal C$ be the family of closed subsets of
$\coint{0,\infty}$.   Let $\rho$ be the usual metric on
$\coint{0,\infty}$ and $\tilde\rho$ the corresponding Hausdorff metric
on $\Cal C\setminus\{\emptyset\}$ (4A2T).
Let $\nu$ be the Poisson point process
on $\coint{0,\infty}$ with density $1$ over Lebesgue measure.   Show that
every member of $\Cal C\setminus\{\emptyset\}$ has a $\nu$-negligible
$\tilde\rho$-neighbourhood.
%495N

\spheader 495Xl Show that the topology on $M_{\text{R}}^+(X)$
described in 495O is just the topology induced by the natural embedding
of $M_{\text{R}}(X)$ into $C_k(X)^{\sim}$ (436J) and the weak topology
$\frak T_s(C_k(X)^{\sim},C_k(X))$, where $C_k(X)$ is the Riesz space
of continuous real-valued functions on $X$ with compact support.

\spheader 495Xm Let $\Cal C$ be the set of closed subsets of
$\coint{0,\infty}$ with its Fell topology.   For
$\delta\in\ocint{0,1}$ let $\lambda_{\delta}$ be the measure on
$\{0,1\}^{\Bbb N}$ which is the product of copies of the measure on
$\{0,1\}$ in which $\{1\}$ is given measure $\delta$.   Define
$\phi_{\delta}:\{0,1\}^{\Bbb N}\to\Cal C$ by setting
$\phi_{\delta}(x)=\{n\delta:n\in\Bbb N$, $x(n)=1\}$, and let
$\tilde\nu_{\delta}$ be the Radon measure
$\lambda_{\delta}\phi_{\delta}^{-1}$ on $\Cal C$.   Show that the Radon
measure on $\Cal C$ representing the Poisson
point process on $\coint{0,\infty}$ with density $1$ over Lebesgue
measure is the limit
$\lim_{\delta\downarrow 0}\tilde\nu_{\delta}$ for the narrow
topology on the space of Radon probability measures on $\Cal C$.
%495O

\spheader 495Xn Show that the standard gamma distribution with
expectation $1$ is the exponential distribution with expectation $1$.
%495P

\spheader 495Xo Let $r\ge 1$ be an integer;  let $\mu$ be Lebesgue
measure on $\BbbR^r$ and $\beta_r$ the volume of the unit ball in
$\BbbR^r$.   Set $\psi(t)=(t/\beta_r)^{1/r}$ for $t\ge 0$, so that the
volume of a ball of radius $\psi(t)$ is $t$.   Let $S_{r-1}$ be the unit
sphere in $\BbbR^r$ and $\theta$ the invariant Radon probability measure
on $S_{r-1}$, so that $\theta$ is a multiple of $(r-1)$-dimensional
Hausdorff measure (see 476I).   Let $\lambda_0$ be the exponential
distribution with expectation $1$, regarded as a Radon probability
measure on $\ooint{0,\infty}$, and $\lambda$ the product measure
$\lambda_0^{\Bbb N}\times\theta^{\Bbb N}$ on
$\ooint{0,\infty}^{\Bbb N}\times S_{r-1}^{\Bbb N}$.   Set

\Centerline{$\phi(x,z)=\{\psi(\sum_{i=0}^nx(i))z(n):n\in\Bbb N\}$}

\noindent for $x\in\ooint{0,\infty}^{\Bbb N}$ and
$z\in S_{r-1}^{\Bbb N}$.   Show that
$\phi:\ooint{0,\infty}^{\Bbb N}\times S_{r-1}^{\Bbb N}
\to\Cal P(\BbbR^r)$ is a measure space isomorphism between
$\ooint{0,\infty}^{\Bbb N}\times S_{r-1}^{\Bbb N}$ and a conegligible
set for the Poisson point process on $\BbbR^r$ with density $1$.
%495Xj 495P

\leader{495Y}{Further exercises (a)}
%\spheader 495Ya
Let $U$ be an $L$-space.   Show that there are a
probability algebra $(\frak B,\bar\lambda)$ and a linear operator
$T:U\to L^0(\frak B)$ such that (i) for every
$u\in U$, $Tu$ has a Cauchy distribution with centre
$0$ and scale parameter $\|u\|$ (ii) if $\familyiI{u_i}$ is a disjoint
family in $U$ then $\familyiI{Tu_i}$ is $\bar\lambda$-independent.
%495K

\spheader 495Yb Let $U$ be an $L$-space.   Show
that there are a probability algebra $(\frak B,\bar\lambda)$ and a
linear operator $T:U\to L^1(\frak B,\bar\lambda)$ such that (i)
for every $u\in U^+$, $Tu$ has a standard gamma distribution (definition:
455Xj) with expectation $\|u\|$ (ii) if
$\familyiI{u_i}$ is a disjoint family in $U$ then
$\familyiI{Tu_i}$ is $\bar\lambda$-independent.
%495K

\spheader 495Yc Let $(\frak A,\bar\mu)$ be a measure algebra.   For
$\alpha$, $y\in\Bbb R$ set $h_y(\alpha)=e^{iy\alpha}$, and let
$\bar h_y:L^0(\frak A)\to L^0_{\Bbb C}(\frak A)$ (definition: 
366M\formerly{3{}64Yn}) be
the corresponding operator (to be defined, following the ideas of
364H\formerly{3{}64I}
or otherwise).    Show that there are a probability algebra
$(\frak B,\bar\lambda)$ and a positive linear operator
$T:L^1(\frak A,\bar\mu)\to L^1(\frak B,\bar\lambda)$ such that
(i) $\|Tu\|_1=\|u\|_1$ whenever $u\in L^1(\frak A,\bar\mu)^+$ (ii)
$\familyiI{Tu_i}$ is $\bar\lambda$-independent in $L^0(\frak B)$
whenever $\familyiI{u_i}$ is a disjoint family in
$L^1(\frak A,\bar\mu)$ (iii)
$\int\bar h_y(Tu)d\bar\lambda=\exp(\int(\bar h_y(u)-\chi 1)d\bar\mu)$
for every $u\in L^1(\frak A,\bar\mu)$ and $y\in\Bbb R$.
%495M

\spheader 495Yd
\discrversionA{\footnote{Revised 2008.}}{}
Let $(X,\rho)$ be a totally bounded metric space,
$\mu$ a Radon measure on $X$ and $\gamma>0$.   Let
$\Cal C$ be the set of closed subsets of $X$, and $\tilde\nu$ the
quasi-Radon measure of 495N;  let $\tilde\rho$ be the
Hausdorff metric on $\Cal C\setminus\{\emptyset\}$.   Show that the
subspace measure on $\Cal C\setminus\{\emptyset\}$ induced by $\tilde\nu$
is a Radon measure for the topology induced by $\tilde\rho$.
%495N
}%end of exercises

\endnotes{
\Notesheader{495} The underlying fact on which this section relies is
that the Poisson distributions form a one-parameter semigroup of
infinitely divisible distributions, with
$\nu_{\alpha}*\nu_{\beta}=\nu_{\alpha+\beta}$ for all $\alpha$,
$\beta>0$.   Other well-known families with this property are normal
distributions, Cauchy distributions and gamma distributions;  for each
of these we have results corresponding to 495B and 495K (495Xh, 495Ya,
495Yb).   The same distributions appeared, for the same reason, in the
L\'evy processes of \S455. % 455Xh 455Xi 455Xj
Observe that the version for the normal distribution is related to the
Gaussian processes of \S456.   The `compound Poisson' distributions of
495M provide further examples, which approach the general form of
infinitely divisible distributions ({\smc Lo\`eve 77}, \S23, or
{\smc Fristedt \& Gray 97}, \S16.3).

The special feature of the Poisson point process, in this context, is
the fact that (for atomless measure spaces $(X,\mu)$) it can be
represented by a measure on $\Cal PX$ rather than on some abstract
auxiliary space (495D);  so that we have a notion of `random subset',
and can discuss the expected topological properties of subsets of $X$
(495Xb, 495Xd).   In Euclidean spaces the geometric properties of these
random subsets are also of great interest;  see
{\smc Meester \& Roy 96}.   Here I look at the relations between this construction and others
which have been prominent in this book, such as \imp\ functions (495G)
and disintegrations (495H-495I).   In the latter we find ourselves in an
interesting difficulty.   If, as in 495H, we have a measure space
$X=\tilde X\times[0,1]$, where $\tilde X$ is an atomless measure space,
then it is natural to suppose that our Poisson process on $X$ can be
represented by picking a random subset $T$ of $\tilde X$ and then, for
each $t\in T$, a random $(t,\alpha)\in X$.   The obvious model for this
idea is the map $(T,z)\mapsto\{(t,z(t)):t\in T\}:
\Cal P\tilde X\times[0,1]^{\tilde X}\to\Cal PX$.   The problem with this
model is that the map is simply not measurable for the standard
$\sigma$-algebras on $\Cal P\tilde X$,
$\Cal P\tilde X\times[0,1]^{\tilde X}$ and $\Cal PX$.   When we have a
canonical ordering in order type $\omega$ of almost every subset of
$\tilde X$ (`almost every' with respect to the Poisson point process on
$\tilde X$, of course), as in 495Xo, there can be a way around this,
cutting $[0,1]^{\tilde X}$ down to a countable product and re-inventing
the representation of pairs $(T,z)$ as subsets of $X$.   But in the
general case it seems that we have to set up a disintegration of the
Poisson point process on $X$ over the Poisson point process on
$\tilde X$ which does not correspond to any measure on a product
$\Cal P\tilde X\times\Omega$.

Following my usual custom, I have expressed the theorems of this section
in terms of arbitrary (atomless) measure spaces.   The results are not
quite without interest when applied to totally finite measures, but
their natural domain is the class of non-totally-finite $\sigma$-finite
measures, as in 495N-495P.   There is an unavoidable obstacle if we wish
to extend the ideas to measure spaces which are not atomless.   The
functions $S\mapsto\#(S\cap E)$ may no longer have Poisson
distributions, since if $E$ is a singleton of positive measure then we
shall have a non-trivial two-valued random variable.   In 495N-495O I
take one of the possible resolutions of this, with measures $\tilde\nu$
on spaces of subsets for which at least the sets
$\{S:S\cap E=\emptyset\}$, for disjoint $E$, are independent.   An
alternative which is sometimes appropriate is to work with functions
$h:X\to\Bbb N$ and $\sum_{x\in E}h(x)$ in place of subsets $S$ of $X$
and $\#(S\cap E)$;  see {\smc Fristedt \& Gray 97}, \S29.

In 495J-495L we have a little cluster of results which are relevant to
rather different questions, to which I will return in Chapter 52 of
Volume 5.   % 526D 528H
The objective here is to connect the structure of a measure
algebra or Banach lattice of arbitrarily large cellularity with
something which can be realized in a probability space.   In each case,
disjointness is transformed into stochastic independence.
Once again, the special feature of the Poisson point process is that we
have a concrete representation of a linear operator which can also be
described in a more abstract way (495L).

The construction of 495B-495D seems to be the most straightforward way
to generate Poisson point processes.   It fails however to give a direct
interpretation of one of the most important approaches to these
processes, as limits of purely atomic processes in which sets are chosen
by including or excluding individual points independently
(495Xm).   In order to make sense of the limit here it seems that
we need to put some further structure onto the underlying measure space,
and `$\sigma$-finite locally compact Radon measure space' is sufficient
to give a positive result (495O).

}%end of notes

\discrpage

