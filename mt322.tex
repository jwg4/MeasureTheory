\frfilename{mt322.tex}
\versiondate{24.4.06}
\copyrightdate{1995}

\def\chaptername{Measure algebras}
\def\sectionname{Taxonomy of measure algebras}

\newsection{322}

Before going farther with the general theory of measure algebras, I run
through those parts of the classification of measure spaces in \S211
which have expressions in terms of measure algebras.   The most
important concepts at this stage are those of `semi-finite',
`localizable' and
`$\sigma$-finite' measure algebra (322Ac-322Ae);  these correspond
exactly to the same terms applied to measure spaces (322B).   I briefly
investigate the Boolean-algebra properties of
semi-finite and $\sigma$-finite measure algebras (322F, 322G), with
mentions of completions and c.l.d.\ versions (322D), subspace measures
(322I-322J), indefinite-integral measures (322K),
direct sums of measure spaces (322L, 322M) and
subalgebras of measure algebras (322N).   It turns out that
localizability
of a measure algebra is connected in striking ways to the properties of
the canonical measure on its Stone space (322O).   I end the section
with a description of the `localization' of a
semi-finite measure algebra (322P-322Q) and with some further properties
of Stone spaces (322R).

\vleader{72pt}{322A}{Definitions} Let $(\frak A,\bar\mu)$ be a measure
algebra.

\header{322Aa}{\bf (a)}\cmmnt{ I will say that} $(\frak A,\bar\mu)$ is
a {\bf probability algebra} if $\bar\mu 1=1$.

\header{322Ab}{\bf (b)} $(\frak A,\bar\mu)$ is {\bf
totally finite} if $\bar\mu 1<\infty$.

\header{322Ac}{\bf (c)} $(\frak A,\bar\mu)$ is
{\bf $\sigma$-finite} if there is a sequence $\sequencen{a_n}$ in
$\frak A$ such that $\bar\mu a_n<\infty$ for every $n\in\Bbb N$ and
$\sup_{n\in\Bbb N}a_n=1$.   Note that in this case $\sequencen{a_n}$ can be taken {\it either} to be non-decreasing\prooflet{ (consider $a'_n=\sup_{i<n}a_i$)}
{\it or} to be disjoint\prooflet{ (consider
$a''_n=a_n\Bsetminus a'_n$)}.

\header{322Ad}{\bf (d)} $(\frak A,\bar\mu)$ is
{\bf semi-finite} if whenever $a\in\frak A$ and $\bar\mu a=\infty$ there
is a non-zero $b\Bsubseteq a$ such that $\bar\mu b<\infty$.

\header{322Ae}{\bf (e)} $(\frak A,\bar\mu)$ is {\bf
localizable} if it is semi-finite and the Boolean
algebra $\frak A$ is Dedekind complete.

\leader{322B}{}\cmmnt{ The first step is to relate these concepts to
the corresponding ones for measure spaces.

\medskip

\noindent}{\bf Theorem} Let $(X,\Sigma,\mu)$ be a measure space, and
$(\frak A,\bar\mu)$ its measure algebra.   Then

(a) $(X,\Sigma,\mu)$ is a probability space iff $(\frak A,\bar\mu)$ is
a probability algebra;

(b) $(X,\Sigma,\mu)$ is totally finite iff $(\frak A,\bar\mu)$ is;

(c) $(X,\Sigma,\mu)$ is $\sigma$-finite iff $(\frak A,\bar\mu)$ is;

(d) $(X,\Sigma,\mu)$ is semi-finite iff $(\frak A,\bar\mu)$ is;

(e) $(X,\Sigma,\mu)$ is localizable iff $(\frak A,\bar\mu)$ is;

(f) if $E\in\Sigma$, then $E$ is an atom for $\mu$ iff $E^{\ssbullet}$
is an atom in $\frak A$;

(g) $(X,\Sigma,\mu)$ is atomless iff $\frak A$ is;

(h) $(X,\Sigma,\mu)$ is purely atomic iff $\frak A$ is.

\proof{{\bf (a), (b)} are trivial, since $\bar\mu 1=\mu X$.

\medskip

{\bf (c)(i)} If $\mu$ is $\sigma$-finite, let $\sequencen{E_n}$ be a
sequence of sets of finite measure covering $X$;  then $\bar\mu
E_n^{\ssbullet}<\infty$ for every $n$, and

\Centerline{$\sup_{n\in\Bbb N}E_n^{\ssbullet}=(\bigcup_{n\in\Bbb
N}E_n)^{\ssbullet}=1$,}

\noindent so $(\frak A,\bar\mu)$ is $\sigma$-finite.

\medskip

\quad{\bf (ii)} If $(\frak A,\bar\mu)$ is $\sigma$-finite, let
$\sequencen{a_n}$ be a sequence in $\frak A$ such that
$\bar\mu a_n<\infty$ for every $n$ and $\sup_{n\in\Bbb N}a_n=1$.   For each $n$, choose $E_n\in\Sigma$ such that $E_n^{\ssbullet}=a_n$.   Set
$E=\bigcup_{n\in\Bbb N}E_n$;  then
$E^{\ssbullet}=\sup_{n\in\Bbb N}a_n=1$, so $E$ is conegligible.   Now $(X\setminus E,E_0,E_1,\ldots)$
is a sequence of sets of finite measure covering $X$, so $\mu$ is
$\sigma$-finite.

\medskip

{\bf (d)(i)} Suppose that $\mu$ is semi-finite and that $a\in\frak A$,
$\bar\mu a=\infty$.   Then there is an $E\in\Sigma$ such that
$E^{\ssbullet}=a$, so that $\mu E=\bar\mu a=\infty$.   As $\mu$ is
semi-finite, there is an $F\in\Sigma$ such that $F\subseteq E$ and
$0<\mu F<\infty$.   Set $b=F^{\ssbullet}$;  then $b\Bsubseteq a$ and
$0<\bar\mu b<\infty$.

\medskip

\quad{\bf (ii)} Suppose that $(\frak A,\bar\mu)$ is semi-finite and
that $E\in\Sigma$, $\mu E=\infty$.   Then $\bar\mu
E^{\ssbullet}=\infty$, so there is a $b\Bsubseteq
E^{\ssbullet}$ such that $0<\bar\mu b<\infty$.   Let $F\in\Sigma$ be
such that $F^{\ssbullet}=b$.   Then $F\cap
E\in\Sigma$, $F\cap E\subseteq E$ and $(F\cap
E)^{\ssbullet}=E^{\ssbullet}\Bcap b=b$, so that $\mu(F\cap E)=\bar\mu
b\in\ooint{0,\infty}$.

\medskip

{\bf (e)(i)} Note first that if $\Cal E\subseteq\Sigma$ and
$F\in\Sigma$, then

$$\eqalign{E\setminus F&\text{ is negligible for every }E\in\Cal E\cr
&\iff E^{\ssbullet}\Bsetminus F^{\ssbullet}=0
    \text{ for every }E\in\Cal E\cr
&\iff F^{\ssbullet}\text{ is an upper bound for }
    \{E^{\ssbullet}:E\in\Cal E\}.\cr}$$

\noindent So if $\Cal E\subseteq\Sigma$ and $H\in\Sigma$, then $H$ is an
essential supremum of $\Cal E$ in $\Sigma$, in the sense of 211G, iff
$H^{\ssbullet}$ is the
supremum of $A=\{E^{\ssbullet}:E\in\Cal E\}$ in $\frak A$.   \Prf\
Writing $\Cal F$ for

\Centerline{$\{F:F\in\Sigma,\,E\setminus F$ is negligible
for every $E\in\Cal E\}$,}

\noindent we see that $B=\{F^{\ssbullet}:F\in\Cal F\}$ is just the set
of upper bounds of $A$, and that $H$ is an essential supremum of $\Cal
E$ iff $H\in\Cal F$ and $H^{\ssbullet}$ is a lower bound for $B$;  that
is, iff $H^{\ssbullet}=\sup A$.\ \Qed

\medskip

\quad{\bf (ii)} Thus $\frak A$ is Dedekind complete iff every family in
$\Sigma$ has an essential supremum in $\Sigma$.   Since we already know
that $(\frak A,\bar\mu)$ is semi-finite iff $\mu$ is, we see that
$(\frak A,\bar\mu)$ is localizable iff $\mu$ is.

\medskip

{\bf (f)} This is immediate from the definitions in 211I and 316K, if
we remember always that $\{b:b\Bsubseteq E^{\ssbullet}\}
=\{F^{\ssbullet}:F\in\Sigma,\,F\subseteq E\}$ (312Lb).

\medskip

{\bf (g), (h)} follow at once from (f).
}%end of proof of 322B

\leader{322C}{}\cmmnt{ I copy out the relevant parts of Theorem 211L
in the new context.

\medskip

\noindent}{\bf Theorem} (a) A probability algebra is totally finite.

(b) A totally finite measure algebra is $\sigma$-finite.

(c) A $\sigma$-finite measure algebra is localizable.

(d) A localizable measure algebra is semi-finite.

\proof{ All except (c) are trivial;  and (c) may be deduced from
211Lc-211Ld, 322Bc, 322Be and 321J, or from 316Fa and 322G below.
}%end of proof of 322C

\leader{322D}{}\cmmnt{ Of course not all the definitions in \S211 are
directly relevant to measure algebras.   The concepts of `complete',
`locally determined' and `strictly localizable' measure space do not
correspond in any direct way to properties of the measure algebras.
Indeed, completeness is just irrelevant, as the next proposition shows.

\medskip

\noindent}{\bf Proposition} Let $(X,\Sigma,\mu)$ be a measure space,
with completion $(X,\hat\Sigma,\hat\mu)$ and c.l.d.\ version
$(X,\tilde\Sigma,\tilde\mu)$\cmmnt{ (213E)}.   Write
$(\frak A,\bar\mu)$, $(\frak A_1,\bar\mu_1)$ and $(\frak A_2,\bar\mu_2)$
for the measure algebras of $\mu$, $\hat\mu$ and $\tilde\mu$
respectively.

(a) The embedding
$\Sigma\embedsinto\hat\Sigma$ corresponds to an isomorphism between
$(\frak A,\bar\mu)$ and $(\frak A_1,\bar\mu_1)$.

(b)(i) The embedding $\Sigma\embedsinto\tilde\Sigma$ defines an
order-continuous Boolean
homomorphism $\pi:\frak A\to\frak A_2$.   Setting
$\frak A^f=\{a:a\in\frak A,\,\bar\mu a<\infty\}$, $\pi\restrp\frak A^f$
is a measure-preserving bijection between $\frak A^f$ and
$\frak A_2^f=\{c:c\in\frak A_2,\,\bar\mu_2c<\infty\}$.

\quad(ii) $\pi$ is injective iff $\mu$ is semi-finite, and in this case
$\bar\mu_2(\pi a)=\bar\mu a$ for every $a\in\frak A$.

\quad(iii) If $\mu$ is localizable, $\pi$ is a bijection.

\proof{ For $E\in\Sigma$, I write $E^{\smallcirc}$ for its image in
$\frak A$;  for $F\in\hat\Sigma$, I write $F^*$ for its image in
$\frak A_1$;  and for $G\in\tilde\Sigma$, I write $G^{\ssbullet}$ for its image in $\frak A_2$.

\medskip

{\bf (a)} This is nearly trivial.   The map $E\mapsto
E^*:\Sigma\to\frak A_1$ is a
Boolean homomorphism, being the composition of the Boolean homomorphisms
$E\mapsto E:\Sigma\to\hat\Sigma$ and $F\mapsto
F^*:\hat\Sigma\to\frak A_1$.   Its kernel is
$\{E:E\in\Sigma,\,\hat\mu E=0\}=\{E:E\in\Sigma,\,\mu E=0\}$, so it
induces an injective Boolean homomorphism $\phi:\frak A\to\frak A_1$
given by the formula $\phi(E^{\smallcirc})=E^*$ for every
$E\in\Sigma$ (312F, 3A2G).   To see that $\phi$ is surjective, take any
$b\in\frak A_1$.   There is an $F\in\hat\Sigma$ such that
$F^*=b$, and there is an $E\in\Sigma$ such that $E\subseteq F$
and $\hat\mu(F\setminus E)=0$, so that

\Centerline{$\pi(E^{\smallcirc})=E^*=F^*=b$.}

\noindent Thus $\pi$ is a Boolean algebra isomorphism.   It is a measure
algebra isomorphism because for any $E\in\Sigma$

\Centerline{$\bar\mu_1\phi(E^{\smallcirc})
=\bar\mu_1E^*
=\hat\mu E
=\mu E
=\bar\mu E^{\smallcirc}$.}

\medskip

{\bf (b)(i)} The map $E\mapsto E^{\ssbullet}:\Sigma\to\frak A_2$ is a
Boolean homomorphism with kernel
$\{E:E\in\Sigma,\,\tilde\mu E=0\}\supseteq\{E:E\in\Sigma,\,\mu E=0\}$,
so induces a Boolean homomorphism
$\pi:\frak A\to\frak A_2$, defined by saying that
$\pi E^{\smallcirc}=E^{\ssbullet}$ for every $E\in\Sigma$.

If $a\in\frak A^f$, it is expressible as $E^{\smallcirc}$ where
$\mu E<\infty$.   Then $\tilde\mu E=\mu E$ (213Fa), so
$\pi a=E^{\ssbullet}$ belongs to $\frak A_2^f$, and
$\bar\mu_2(\pi a)=\bar\mu a$.   If $a$, $a'$
are distinct members of $\frak A^f$, then

\Centerline{$\bar\mu_2(\pi a\Bsymmdiff\pi a')
=\bar\mu_2\pi(a\Bsymmdiff a')
=\bar\mu(a\Bsymmdiff a')>0$,}

\noindent so $\pi a\ne\pi a'$;  thus $\pi\restrp\frak A^f$ is an
injective
map from $\frak A^f$ to $\frak A_2^f$.   If $c\in\frak A_2^f$, then
$c=G^{\ssbullet}$ where $\tilde\mu G<\infty$;   by 213Fc, there is an
$E\in\Sigma$ such that $E\subseteq G$, $\mu E=\tilde\mu G$ and
$\tilde\mu(G\setminus E)=0$, so that $E^{\smallcirc}\in\frak A^f$ and

\Centerline{$\pi E^{\smallcirc}=E^{\ssbullet}=G^{\ssbullet}=c$.}

\noindent As $c$ is arbitrary, $\phi[\frak A^f]=\frak A_2^f$.

Finally, $\pi$ is order-continuous.   \Prf\  Let $A\subseteq\frak A$ be
a non-empty downwards-directed set with infimum $0$, and $b\in\frak A_2$
a lower bound for $\pi[A]$.   \Quer\ If $b\ne 0$, then (because $(\frak
A_2,\bar\mu_2)$ is semi-finite) there is a $b_0\in\frak A_2^f$ such that
$0\ne b_0\Bsubseteq b$.   Let $a_0\in\frak A$ be such that $\pi
a_0=b_0$.   Then $a_0\ne 0$, so there is an $a\in A$ such that
$a\notBsupseteq a_0$,
that is, $a\Bcap a_0\ne a_0$.   But now, because $\pi\restrp\frak A^f$
is injective,

\Centerline{$b_0=\pi a_0\ne\pi(a\Bcap a_0)=\pi a\Bcap\pi a_0
=\pi a\Bcap b_0$,}

\noindent and $b_0\notBsubseteq\pi a$, which is
impossible.\ \BanG\ Thus $b=0$, and $0$ is the only lower bound of
$\pi[A]$.   As $A$ is arbitrary, $\pi$ is order-continuous
(313L(b-ii)).\ \Qed

\medskip

\quad{\bf (ii)} ($\alpha$) If $\mu$ is semi-finite, then
$\tilde\mu E=\mu E$ for every $E\in\Sigma$ (213Hc), so

\Centerline{$\bar\mu_2(\pi E^{\smallcirc})
=\bar\mu_2E^{\ssbullet}
=\tilde\mu E
=\mu E
=\bar\mu E^{\smallcirc}$}

\noindent for every $E\in\Sigma$.   In particular,

\Centerline{$\pi a=0\Longrightarrow 0=\bar\mu_2(\pi a)=\bar\mu a
\Longrightarrow a=0$,}

\noindent so $\pi$ is injective.   ($\beta$) If $\mu$ is not
semi-finite, there is an $E\in\Sigma$ such that $\mu E=\infty$ but $\mu
H=0$
whenever $H\in\Sigma$, $H\subseteq E$ and $\mu H<\infty$;  so that
$\tilde\mu E=0$ and

\Centerline{$E^{\smallcirc}\ne 0$,
\quad$\pi E^{\smallcirc}=E^{\ssbullet}=0$.}

\noindent So in this case $\pi$ is not injective.

\medskip

\quad{\bf (iii)} Now suppose that $\mu$ is localizable.   Then for every
$G\in\tilde\Sigma$ there is an $E\in\Sigma$ such that
$\tilde\mu(E\symmdiff G)=0$, by 213Hb;  accordingly
$\pi E^{\smallcirc}=E^{\ssbullet}=G^{\ssbullet}$.   As $G$ is arbitrary,
$\pi$ is surjective;  and we know from (ii) that $\pi$ is injective, so
it is a bijection, as claimed.
}%end of proof of 322D

\leader{322E}{Proposition} Let $(\frak A,\bar\mu)$ be a measure algebra.

(a) $(\frak A,\bar\mu)$ is semi-finite iff it has a partition of unity
consisting of elements of finite measure.

(b) If $(\frak A,\bar\mu)$ is semi-finite,
$a=\sup\{b:b\Bsubseteq a,\,\bar\mu b<\infty\}$ and
$\bar\mu a=\sup\{\bar\mu b:b\Bsubseteq a,\,\bar\mu b<\infty\}$
for every $a\in\frak A$.

\proof{ Set $\frak A^f=\{b:b\in\frak A,\,\bar\mu b<\infty\}$.

\medskip

{\bf (a)(i)} If $(\frak A,\bar\mu)$ is semi-finite, then $\frak A^f$ is
order-dense in $\frak A$, so there is a partition of unity consisting of
members of $\frak A^f$ (313K).
\medskip

\quad{\bf (ii)}  If there is a partition of unity $C\subseteq\frak A^f$,
and $\bar\mu a=\infty$, then there is a $c\in C$ such that
$a\Bcap c\ne 0$, and now $a\Bcap c\Bsubseteq a$ and
$0<\bar\mu(a\Bcap c)<\infty$;  as
$a$ is arbitrary, $(\frak A,\bar\mu)$ is semi-finite.

\medskip

{\bf (b)} Of course $\frak A^f$ is upwards-directed, by 321Bc, and we
are supposing that its supremum is $1$.   If $a\in\frak A$, then

\Centerline{$B=\{b:b\in\frak A^f,\,b\Bsubseteq a\}
=\{a\Bcap b:b\in\frak A^f\}$}

\noindent is upwards-directed and has supremum $a$ (313Ba), so
$\bar\mu a=\sup_{b\in B}\bar\mu b$, by 321D.
}%end of proof of 322E

\cmmnt{\medskip

\noindent{\bf Remark} Compare 213A.}

\leader{322F}{Proposition} If $(\frak A,\bar\mu)$ is a semi-finite
measure algebra, then $\frak A$ is a \wsid\ Boolean algebra.

\proof{ Let $\sequencen{A_n}$ be a sequence of non-empty
downwards-directed subsets of $\frak A$, all with infimum $0$.   Set

\Centerline{$B
=\{b:$ for every $n\in\Bbb N$ there is an $a\in A_n$ such that
$b\Bsupseteq a\}$.}

\noindent If $c\in\frak A\setminus\{0\}$, let $c'\Bsubseteq c$ be such
that $0<\bar\mu c'<\infty$.   For each $n\in\Bbb N$,
$\inf_{a\in A_n}\bar\mu(c'\Bcap a)=0$, by 321F;  so we may choose
$a_n\in A_n$ such
that $\bar\mu(c'\Bcap a_n)\le 2^{-n-2}\bar\mu b$.   Set
$b=\sup_{n\in\Bbb N}a_n\in B$.   Then

\Centerline{$\bar\mu(c'\Bcap b)
\le\sum_{n=0}^{\infty}\bar\mu(c'\Bcap a_n)<\bar\mu c'$,}

\noindent so $c'\notBsubseteq b$ and $c\notBsubseteq b$.   As $c$ is
arbitrary, $\inf B=0$;  as $\sequencen{A_n}$ is arbitrary, $\frak A$ is
weakly $(\sigma,\infty)$-distributive (316G).
}%end of proof of 322F

\vleader{72pt}{322G}{}\cmmnt{ Corresponding to 215B, we have the
following description of $\sigma$-finite algebras.

\medskip

\noindent}{\bf Proposition} Let $(\frak A,\bar\mu)$ be a semi-finite
measure algebra.   Then the following are equiveridical:

(i) $(\frak A,\bar\mu)$ is $\sigma$-finite;

(ii) $\frak A$ is ccc;

(iii) {\it either} $\frak A=\{0\}$ {\it or} there is a functional
$\bar\nu:\frak A\to[0,1]$ such
that $(\frak A,\bar\nu)$ is a probability algebra.

\proof{{\bf (i)$\Leftrightarrow$(ii)} By 321J,
it is enough to consider the case in
which $(\frak A,\bar\mu)$ is the measure algebra of a measure space
$(X,\Sigma,\mu)$, and $\mu$ is semi-finite, by 322Bd.   We know that
$\frak A$ is ccc iff there is no uncountable disjoint set in
$\Sigma\setminus\Cal N$, where $\Cal N$ is the null ideal of $\mu$
(316D).   But 215B(iii)
shows that this is equivalent to $\mu$ being $\sigma$-finite, which is
equivalent to $(\frak A,\bar\mu)$ being $\sigma$-finite, by 322Bc.

\medskip

{\bf (i)$\Rightarrow$(iii)} If $(\frak A,\bar\mu)$ is $\sigma$-finite,
and $\frak A\ne\{0\}$, let $\sequencen{a_n}$ be a disjoint sequence in
$\frak A$ such that $\bar\mu a_n<\infty$ for every $n$ and
$\sup_{n\in\Bbb N}a_n=1$.
Then $\bar\mu a_n>0$ for some $n$, so there are $\gamma_n>0$ such that
$\sum_{n=0}^{\infty}\gamma_n\bar\mu a_n=1$.
(Set $\gamma'_n=2^{-n}/(1+\bar\mu a_n)$,
$\gamma_n=\gamma'_n/(\sum_{i=0}^{\infty}\gamma'_i\bar\mu a_i)$.)   Set
$\bar\nu a=\sum_{n=0}^{\infty}\gamma_n\bar\mu(a\Bcap a_n)$ for every
$a\in\frak A$;
it is easy to check that $(\frak A,\bar\nu)$ is a probability algebra.

\medskip

{\bf (iii)$\Rightarrow$(i)} is a consequence of (i)$\Leftrightarrow$(ii).
}%end of proof of 322G

\leader{322H}{Principal ideals} If $(\frak A,\bar\mu)$ is a measure
algebra and $a\in\frak A$, then\cmmnt{ it is easy to see (using
314Eb) that} $(\frak A_a,\bar\mu\restrp\frak{A}_a)$
is a measure algebra, where $\frak A_a$ is the
principal ideal of $\frak A$ generated by $a$.

\leader{322I}{Subspace \dvrocolon{measures}}\cmmnt{ General
subspace measures give rise to
complications in the measure algebra (see 322Xf, 322Yd).   But subspaces
with measurable envelopes (132D, 213L) are manageable.

\medskip

\noindent}{\bf Proposition} Let $(X,\Sigma,\mu)$ be a measure space, and
$A\subseteq X$ a set with a measurable envelope $E$.   Let $\mu_A$ be
the subspace measure on $A$, and $\Sigma_A$ its domain;  let
$(\frak A,\bar\mu)$ be the measure algebra of $(X,\Sigma,\mu)$ and
$(\frak A_A,\bar\mu_A)$ the measure algebra of $(A,\Sigma_A,\mu_A)$.
Set $a=E^{\ssbullet}$ and let $\frak A_a$ be the principal ideal of
$\frak A$ generated by $a$.   Then we
have an isomorphism between $(\frak A_a,\bar\mu\restrp\frak{A}_a)$ and
$(\frak A_A,\bar\mu_A)$ given by the formula

\Centerline{$F^{\ssbullet}\mapsto (F\cap A)^{\smallcirc}$}

\noindent whenever $F\in\Sigma$ and $F\subseteq E$, writing
$F^{\ssbullet}$ for the equivalence class of $F$ in $\frak A$ and
$(F\cap A)^{\smallcirc}$
for the equivalence class of $F\cap A$ in $\frak A_A$.

\proof{ Set $\Sigma_E=\{E\cap F:F\in\Sigma\}$.   For $F$,
$G\in\Sigma_E$,

\Centerline{$F^{\ssbullet}=G^{\ssbullet}
\iff \mu(F\symmdiff G)=0
\iff \mu_A(A\cap(F\symmdiff G))=0
\iff (F\cap A)^{\smallcirc}=(G\cap A)^{\smallcirc}$,}

\noindent because $E$ is a measurable envelope of $A$.   Accordingly the
given formula defines an injective function from the image
$\{F^{\ssbullet}:F\in\Sigma_E\}$ of $\Sigma_E$ in $\frak A$ to
$\frak A_A$;  but this image is just the principal ideal $\frak A_a$.   It is
easy to check that the map is a Boolean homomorphism from $\frak A_a$ to
$\frak A_A$, and it is a Boolean isomorphism because
$\Sigma_A=\{F\cap A:F\in\Sigma_E\}$.   Finally, it is measure-preserving
because

\Centerline{$\bar\mu F^{\ssbullet}=\mu F=\mu^*(F\cap A)
=\mu_A(F\cap A)=\bar\mu_A(F\cap A)^{\smallcirc}$}

\noindent for every $F\in\Sigma_E$, again using the fact that $E$ is a
measurable envelope of $A$.
}%end of proof of 322I

\leader{322J}{Corollary} Let $(X,\Sigma,\mu)$ be a measure space, with
measure algebra $(\frak A,\bar\mu)$.

(a) If $E\in\Sigma$, then the measure algebra of the subspace measure
$\mu_E$ can be identified with the principal ideal
$\frak A_{E^{\ssbullet}}$ of $\frak A$.

(b) If $A\subseteq X$ is a set of full outer measure (in
particular, if $\mu^*A=\mu X<\infty$), then the
measure algebra of the subspace measure $\mu_A$ can be identified with
$\frak A$.

\leader{322K}{Indefinite-integral measures:  Proposition}\dvAnew{20??}
Let $(X,\Sigma,\mu)$ be a measure space and $\nu$ an
indefinite-integral measure over $\mu$\cmmnt{ (234J)}.
Then the measure
algebra of $\nu$ can be identified, as Boolean algebra, with a principal
ideal of the measure algebra of $\mu$.

\proof{ Taking $(X,\hat\Sigma,\hat\mu)$ to be the completion of
$(X,\Sigma,\mu)$, then we can identify the measure algebras of $\mu$ and
$\hat\mu$, by 322Da;  and $\nu$ is still an indefinite-integral measure
over $\hat\mu$, just because $\mu$ and $\hat\mu$ give rise to the same
theory of integration (212Fb).   Now there is a $G\in\hat\Sigma$
such that the
domain $\Tau$ of $\nu$ is $\{E:E\subseteq X$, $E\cap G\in\hat\Sigma\}$
and the null ideal $\Cal N_{\nu}$ of $\nu$ is
$\{A:A\subseteq X$, $A\cap G\in\Cal N_{\mu}\}$, where $\Cal N_{\mu}$ is the
null ideal of $\mu$ or $\hat\mu$ (234Lc\footnote{Formerly 2{}34D.}, 212Eb).
Writing $\frak A$ for the
measure algebra of $\hat\mu$, $c=G^{\ssbullet}\in\frak A$,
and $\frak A_c$ for
the principal ideal of $\frak A$ generated by $c$, we have a Boolean
homomorphism $E\mapsto(E\cap G)^{\ssbullet}:\Tau\to\frak A_c$ with kernel
$\Cal N_{\nu}$.   So, writing $E^{\smallcirc}\in\frak B$ for the
equivalence class of $E\in\Tau$, we have an injective Boolean homomorphism
$\pi:\frak B\to\frak A_c$ defined by setting
$\pi E^{\smallcirc}=(E\cap G)^{\ssbullet}$ for every $E\in\Tau$.   Of
course

\Centerline{$\pi[\frak B]\supseteq\{(E\cap G)^{\ssbullet}:E\in\hat\Sigma\}
=\{a\Bcap c:a\in\frak A\}=\frak A_c$,}

\noindent so $\pi$ is actually an isomorphism, as required.
}%end of proof of 322K

\leader{322L}{Simple products}\dvAformerly{3{}22K}
{\bf (a)} Let $\langle(\frak A_i,\bar\mu_i)\rangle_{i\in I}$ be an
indexed family of measure
algebras.   Let $\frak A$ be the simple product Boolean algebra
$\prod_{i\in I}\frak A_i$\cmmnt{ (315A)}, and for $a\in\frak A$ set
$\bar\mu a=\sum_{i\in I}\bar\mu_ia(i)$. Then\cmmnt{ it is easy to check
(using 315D(e-ii)) that} $(\frak A,\bar\mu)$ is a
measure algebra; I will call it the {\bf simple product} of the family
$\langle(\frak A_i,\bar\mu_i)\rangle_{i\in I}$.   Each of the $\frak A_i$
corresponds to a principal ideal $\frak A_{e_i}$ say in $\frak A$,
where $e_i\in\frak A$ corresponds to
$1_{\frak A_i}\in\frak A_i$\cmmnt{ (315E)},
and the Boolean isomorphism between $\frak A_i$ and $\frak A_{e_i}$ is a
measure algebra isomorphism between $(\frak A_i,\bar\mu_i)$ and
$(\frak A_{e_i},\bar\mu\restrp\frak{A}_{e_i})$.

\header{322Lb}{\bf (b)} If $\familyiI{(X_i,\Sigma_i,\mu_i)}$ is a family
of measure spaces, with direct sum
$(X,\Sigma,\mu)$\cmmnt{ (214L)}, then the measure algebra
$(\frak A,\bar\mu)$ of
$(X,\Sigma,\mu)$ can be identified with the simple product of the
measure algebras $(\frak A_i,\bar\mu_i)$ of the
$(X_i,\Sigma_i,\mu_i)$.   \prooflet{\Prf\ If, as in 214L, we set
$X=\{(x,i):i\in I,\,x\in X_i\}$, and for $E\subseteq X$, $i\in I$ we set
$E_i=\{x:(x,i)\in E\}$, then the Boolean isomorphism
$E\mapsto\familyiI{E_i}:\Sigma\to\prod_{i\in I}\Sigma_i$ induces a
Boolean isomorphism from $\frak A$ to $\prod_{i\in I}\frak A_i$, which
is also a measure algebra isomorphism, because

\Centerline{$\bar\mu E^{\ssbullet}=\mu E=\sum_{i\in I}\mu_iE_i
=\sum_{i\in I}\bar\mu_iE_i^{\ssbullet}$}

\noindent for every $E\in\Sigma$.  \Qed}

\spheader 322Lc A simple product of measure algebras is semi-finite, or
localizable, or atomless, or purely atomic, iff every factor is.
\cmmnt{(Compare 214Kb.)}

\header{322Ld}{\bf (d)} Let $(\frak A,\bar\mu)$ be a localizable measure
algebra.

\medskip

\quad{\bf (i)} If  $\langle e_i\rangle_{i\in I}$ is any partition of
unity in $\frak A$, then $(\frak A,\bar\mu)$ is isomorphic to the
product $\prod_{i\in I}(\frak A_{e_i},\bar\mu\restrp\frak{A}_{e_i})$ of
the corresponding principal ideals.   \prooflet{\Prf\ By 315F(iii),
the map $a\mapsto\langle a\Bcap e_i\rangle_{i\in I}$ is a Boolean
isomorphism between $\frak A$ and $\prod_{i\in I}\frak A_i$.   Because
$\langle e_i\rangle_{i\in I}$ is disjoint and
$a=\sup_{i\in I}a\Bcap e_i$, $\bar\mu a=\sum_{i\in I}\bar\mu(a\cap e_i)$
(321E), for every $a\in\frak A$.   So
$a\mapsto\langle a\Bcap e_i\rangle_{i\in I}$ is a measure
algebra isomorphism between $(\frak A,\bar\mu)$ and
$\prod_{i\in I}(\frak A_i,\bar\mu\restrp\frak{A}_{e_i})$.\ \Qed}

\medskip

\quad{\bf (ii)}\cmmnt{ In particular, since $\frak A$ has a partition
of unity consisting of elements of finite measure (322Ea),
$(\frak A,\bar\mu)$
is isomorphic to a simple product of totally finite measure algebras.
Each of these is isomorphic to the measure algebra of a totally finite
measure space, so} $(\frak A,\bar\mu)$ is isomorphic to the measure
algebra of a
direct sum of totally finite measure spaces, which is strictly
localizable.

\cmmnt{Thus every localizable measure algebra is isomorphic to the
measure algebra of a strictly localizable measure space.   (See also
322O below.)}

\leader{*322M}{Strictly
localizable\dvrocolon{ spaces}}\dvAformerly{3{}22L}\cmmnt{ The
following fact is occasionally useful.

\medskip

\noindent}{\bf Proposition}  Let $(X,\Sigma,\mu)$ be a strictly
localizable measure space with $\mu X>0$, and $(\frak A,\bar\mu)$ its
measure algebra.   If $\familyiI{a_i}$ is a partition of unity in
$\frak A$, there is a partition $\familyiI{X_i}$ of $X$ into members of
$\Sigma$ such that $X_i^{\ssbullet}=a_i$ for every $i\in I$ and

\Centerline{$\Sigma
=\{E:E\subseteq X,\,E\cap X_i\in\Sigma\Forall i\in I\}$,}

\Centerline{$\mu E=\sum_{i\in I}\mu(E\cap X_i)$ for every $E\in\Sigma$;}

\noindent that is, the isomorphism between $\frak A$ and the simple
product $\prod_{i\in I}\frak A_{a_i}$ of its principal
ideals\cmmnt{ (315F)} corresponds to an isomorphism between
$(X,\Sigma,\mu)$ and the direct sum of the subspace measures on $X_i$.

\proof{{\bf (a)} Suppose to begin with that $\mu X<\infty$.   In this
case $J=\{i:a_i\ne 0\}$ must be countable (322G).   For each $i\in J$,
choose $E_i\in\Sigma$ such that $E_i^{\ssbullet}=a_i$, and set
$F_i=E_i\setminus\bigcup_{j\in J,j\ne i}E_j$;  then
$F_i^{\ssbullet}=a_i$ for each $i\in J$, and $\family{i}{J}{F_i}$ is
disjoint.   Because $\mu X>0$, $J$ is non-empty;  fix some $j_0\in J$
and set

$$\eqalign{X_i
&=F_{j_0}\cup(X\setminus\bigcup_{j\in J}F_j)\text{ if }i=j_0,\cr
&=F_i\text{ for }i\in J\setminus\{j_0\},\cr
&=\emptyset\text{ for }i\in I\setminus J.\cr}$$

\noindent Then $\familyiI{X_i}$ is a disjoint family in $\Sigma$,
$\bigcup_{i\in I}X_i=X$ and $X_i^{\ssbullet}=a_i$ for every $i$.
Moreover, because only countably many of the $X_i$ are non-empty, we
certainly have

\Centerline{$\Sigma
=\{E:E\subseteq X,\,E\cap X_i\in\Sigma\Forall i\in I\}$,}

\Centerline{$\mu E=\sum_{i\in I}\mu(E\cap X_i)$ for every $E\in\Sigma$.}

\medskip

{\bf (b)} For the general case, start by taking a decomposition
$\family{j}{J}{Y_j}$ of $X$.   We can suppose that no $Y_j$ is
negligible, because there is certainly some $j_0$ such that
$\mu Y_{j_0}>0$, and we can if necessary replace $Y_{j_0}$ by
$Y_{j_0}\cup\bigcup\{Y_j:\mu Y_j=0\}$.   For each $j$, we can identify
the measure algebra of the subspace measure on $Y_j$ with the principal
ideal $\frak A_{b_j}$ generated by $b_j=Y_j^{\ssbullet}$ (322I).   Now
$\familyiI{a_i\Bcap b_j}$ is a partition of unity in $\frak A_{b_j}$, so
by (a) just above we can find a disjoint family $\familyiI{X_{ji}}$ in
$\Sigma$ such that $\bigcup_{i\in I}X_{ji}=Y_j$,
$X_{ji}^{\ssbullet}=a_i\Bcap b_j$ for every $i$ and

\Centerline{$\Sigma\cap\Cal PY_j
=\{E:E\subseteq Y_j,\,E\cap X_{ji}\in\Sigma\Forall i\in I\}$,}

\Centerline{$\mu E=\sum_{i\in I}\mu(E\cap X_{ji})$ for every
$E\in\Sigma\cap\Cal PY_j$.}

Set $X_i=\bigcup_{j\in I}X_{ji}$ for every $i\in I$.   Then
$\familyiI{X_i}$ is a partition of $X$.   Because
$X_i\cap Y_j=X_{ji}$ is measurable for every $j$, $X_i\in\Sigma$.
Because $X_i^{\ssbullet}\Bsupseteq a_i\Bcap b_j$ for every $j$, and
$\family{j}{J}{b_j}$ is a partition of unity in $\frak A$ (322Lb),
$X_i^{\ssbullet}\Bsupseteq a_i$ for each $i$;  because
$\familyiI{X_i^{\ssbullet}}$ is disjoint and $\sup_{i\in I}a_i=1$,
$X_i^{\ssbullet}=a_i$ for every $i$.   If $E\subseteq X$ is such that
$E\cap X_i\in\Sigma$ for every $i$, then $E\cap X_{ji}\in\Sigma$ for all
$i\in I$ and $j\in J$, so $E\cap Y_j\in\Sigma$ for every $j\in J$ and
$E\in\Sigma$.   If $E\in\Sigma$, then

$$\eqalign{\mu E
&=\sum_{j\in J}\mu(E\cap Y_j)
=\sum_{j\in J}\sum_{i\in I}\mu(E\cap X_{ji})\cr
&=\sum_{i\in I}\sum_{j\in J}\mu(E\cap X_i\cap Y_j)
=\sum_{i\in I}\mu(E\cap X_i).\cr}$$

\noindent Thus $\familyiI{X_i}$ is a suitable family.
}%end of proof of 322M

\leader{322N}{Subalgebras:  Proposition}\dvAformerly{322M} Let
$(\frak A,\bar\mu)$ be a
measure algebra, and $\frak B$ a $\sigma$-subalgebra of $\frak A$.  Set
$\bar\nu=\bar\mu\restrp\frak{B}$.

(a) $(\frak B,\bar\nu)$ is a measure algebra.

(b) If $(\frak A,\bar\mu)$ is totally finite, or a probability algebra,
so is $(\frak B,\bar\nu)$.

(c) If $(\frak A,\bar\mu)$ is $\sigma$-finite and $(\frak B,\bar\nu)$ is
semi-finite, then $(\frak B,\bar\nu)$ is $\sigma$-finite.

(d) If $(\frak A,\bar\mu)$ is localizable and $\frak B$ is
order-closed and $(\frak B,\bar\nu)$ is semi-finite, then
$(\frak B,\bar\nu)$ is localizable.

(e) If $(\frak B,\bar\nu)$ is a probability algebra, or totally finite,
or $\sigma$-finite, so is $(\frak A,\bar\mu)$.

\proof{{\bf (a)} By 314Eb, $\frak B$ is Dedekind $\sigma$-complete, and
the identity map $\pi:\frak B\to\frak A$ is sequentially
order-continuous;  so that $\bar\nu=\bar\mu\pi$ will be countably
additive and $(\frak B,\bar\nu)$ will be a measure algebra.

\medskip

{\bf (b)} This is trivial.

\medskip

{\bf (c)} Use 322G.   Every disjoint subset of $\frak B$ is disjoint in
$\frak A$, therefore countable, because $\frak A$ is ccc;  so $\frak B$
also is ccc and $(\frak B,\bar\nu)$ (being semi-finite) is
$\sigma$-finite.

\medskip

{\bf (d)} By 314Ea, $\frak B$ is Dedekind complete;  we are supposing
that $(\frak B,\bar\nu)$ is semi-finite, so it is localizable.

\medskip

{\bf (e)} This is elementary.
}%end of proof of 322N

\leader{322O}{The Stone space of a localizable measure
\dvrocolon{algebra}}\dvAformerly{3{}22N}\cmmnt{ I
said above that the concepts
`strictly localizable' and `locally determined' measure space have no
equivalents in the theory of measure algebras.   But when we look at the
canonical measure on the Stone space of a measure algebra, we can of
course hope that properties of the measure algebra will be reflected in
the properties of this measure, as happens in the next theorem.

\medskip

\noindent}{\bf Theorem} Let $(\frak A,\bar\mu)$ be a measure algebra,
$Z$ the Stone space of $\frak A$, and $\nu$ the standard measure on
$Z$\cmmnt{ constructed by
the method of 321J-321K}.   Then the following are equiveridical:

(i) $(\frak A,\bar\mu)$ is localizable;

(ii) $\nu$ is localizable;

(iii) $\nu$ is locally determined;

(iv) $\nu$ is strictly localizable.

\proof{ Write $\Sigma$ for the domain of $\nu$, that is,

\Centerline{$\{E\symmdiff A:E\subseteq Z$ is open-and-closed,
$A\subseteq Z$ is meager$\}$,}

\noindent and $\Cal M$ for the ideal of meager subsets of $Z$, that is,
the null ideal of $\nu$ (314M, 321K).   Then
$a\mapsto\widehat{a}^{\ssbullet}:\frak A\to\Sigma/\Cal M$ is an
isomorphism
between $(\frak A,\bar\mu)$ and the measure algebra of $(Z,\Sigma,\nu)$
(314M).   Note that because any subset of a meager set is meager, $\nu$
is surely complete.

\medskip

{\bf (a)(i)\hbox{$\Leftrightarrow$}(ii)} is a consequence of 322Be.

\medskip

{\bf (b)(ii)$\Rightarrow$(iii)} Suppose that $\nu$ is localizable.   Of
course it is semi-finite.   Let $V\subseteq Z$ be a set such that
$V\cap E\in\Sigma$ whenever $E\in\Sigma$ and $\nu E<\infty$.   Because
$\nu$ is
localizable, there is a $W\in\Sigma$ which is an essential supremum in
$\Sigma$ of $\{V\cap E:E\in\Sigma,\,\nu E<\infty\}$, that is,
$W^{\ssbullet}=\sup\{(V\cap E)^{\ssbullet}:\nu E<\infty\}$ in
$\Sigma/\Cal M$.    I claim that $W\symmdiff V$ is nowhere dense.  \Prf\
Let $G\subseteq Z$ be a
non-empty open set.   Then there is a non-zero $a\in\frak A$ such that
$\widehat{a}\subseteq G$.   Because $(\frak A,\bar\mu)$ is semi-finite,
we may suppose that $\bar\mu a<\infty$.   Now

\Centerline{$(W\cap\widehat{a})^{\ssbullet}
=W^{\ssbullet}\Bcap\widehat{a}^{\ssbullet}
=\sup_{\nu E<\infty}(V\cap E)^{\ssbullet}\Bcap\widehat{a}^{\ssbullet}
=\sup_{\nu E<\infty}(V\cap E\cap\widehat{a})^{\ssbullet}
=(V\cap\widehat{a})^{\ssbullet}$,}

\noindent so $(W\symmdiff V)\cap\widehat{a}$ is negligible, therefore
meager.   But we know that $\frak A$ is weakly
$(\sigma,\infty)$-distributive (322F),
so that meager sets in $Z$ are nowhere dense (316I), and there is a
non-empty open set $H\subseteq\widehat{a}\setminus(W\symmdiff V)$.
Now $H\subseteq G\setminus\overline{W\symmdiff V}$.   As $G$ is
arbitrary,
$\interior\overline{W\symmdiff V}=\emptyset$ and $W\symmdiff V$ is
nowhere dense.\ \Qed

But this means that $W\symmdiff V\in\Cal M\subseteq\Sigma$ and
$V=W\symmdiff(W\symmdiff V)\in\Sigma$.   As $V$ is arbitrary, $\nu$ is
locally determined.

\medskip

{\bf (c)(iii)$\Rightarrow$(iv)} Assume that $\nu$ is locally determined.
Because $(\frak A,\bar\mu)$ is semi-finite, there is a partition of
unity $C\subseteq\frak A$ consisting of elements of finite measure
(322Ea).   Set $\Cal C=\{\widehat{c}:c\in C\}$.   This is a disjoint
family of sets of finite measure for $\nu$.   Now suppose that
$F\in\Sigma$ and $\nu F>0$.   Then there is an open-and-closed set
$E\subseteq Z$ such that $F\symmdiff E$
is meager, and $E$ is of the form $\widehat{a}$ for some $a\in\frak A$.
Since

\Centerline{$\bar\mu a=\nu\widehat{a}=\nu F>0$,}

\noindent there is some $c\in C$ such that $a\Bcap c\ne 0$, and now

\Centerline{$\nu(F\cap\widehat{c})=\bar\mu(a\Bcap c)>0$.}

\noindent This means that $\nu$ satisfies the conditions of 213O and
must be strictly localizable.

\medskip

{\bf (d)(iv)$\Rightarrow$(ii)} This is just 211Ld.
}%end of proof of 322O

\leader{322P}{Theorem}\dvAformerly{3{}22O} Let
$(\frak A,\bar\mu)$ be a semi-finite
measure algebra, and let $\widehat{\frak A}$ be the Dedekind completion
of $\frak A$\cmmnt{ (314U)}.   Then there is a unique extension of
$\bar\mu$ to
a functional $\tilde\mu$ on $\widehat{\frak A}$ such that
$(\widehat{\frak A},\tilde\mu)$ is a localizable measure algebra.
The embedding $\frak A\embedsinto\widehat{\frak A}$ identifies the ideals
$\{a:a\in\frak A,\,\bar\mu a<\infty\}$ and
$\{a:a\in\widehat{\frak A},\,\tilde\mu a<\infty\}$.

\proof{ (I write the argument out as if $\frak A$ were actually a
subalgebra of $\widehat{\frak A}$.)   For $c\in\widehat{\frak A}$, set

\Centerline{$\tilde\mu c=\sup\{\bar\mu a:a\in\frak A,\,a\Bsubseteq c\}$.}

\noindent Evidently $\tilde\mu$ is a function from $\widehat{\frak A}$
to $[0,\infty]$ extending $\bar\mu$, so $\tilde\mu 0=0$.   Because
$\frak A$ is
order-dense in $\widehat{\frak A}$, $\tilde\mu c>0$ whenever $c\ne 0$,
because any such $c$ includes a non-zero member of $\frak A$.   If
$\sequencen{c_n}$ is a disjoint sequence in $\widehat{\frak A}$ with
supremum $c$, then $\tilde\mu c=\sum_{n=0}^{\infty}\tilde\mu c_n$.
\Prf\ Let $A$ be the set of all members of $\frak A$ expressible as
$a=\sup_{n\in\Bbb N}a_n$ where $a_n\in\frak A$ and $a_n\Bsubseteq c_n$ for
every $n\in\Bbb N$.   Now

$$\eqalign{\sup_{a\in A}\bar\mu a
&=\sup\{\sum_{n=0}^{\infty}\bar\mu a_n:a_n\in\frak A,\,a_n\Bsubseteq
c_n\text{ for every }n\in\Bbb N\}\cr
&=\sum_{n=0}^{\infty}\sup\{\bar\mu a_n:a_n\Bsubseteq c_n\}
=\sum_{n=0}^{\infty}\tilde\mu c_n.\cr}$$

\noindent Also, because $\frak A$ is order-dense in $\widehat{\frak A}$,
$c_n=\sup\{a:a\in\frak A,\,a\Bsubseteq c_n\}$ for each $n$, and
$\sup A$, taken in $\widehat{\frak A}$, must be $c$.
But this means that if
$a'\in\frak A$ and $a'\Bsubseteq c$ then $a'=\sup_{a\in A}a'\Bcap a$ in
$\widehat{\frak A}$ and therefore also in $\frak A$;  so that

\Centerline{$\bar\mu a'=\sup_{a\in A}\bar\mu(a'\Bcap a)
\le\sup_{a\in A}\bar\mu a$.}

\noindent   Accordingly

\Centerline{$\tilde\mu c=\sup_{a\in A}\bar\mu a
=\sum_{n=0}^{\infty}\tilde\mu c_n$.   \Qed}

This shows that $(\widehat{\frak A},\tilde\mu)$ is a measure algebra.
It is semi-finite because $(\frak A,\bar\mu)$ is and every non-zero
element of
$\widehat{\frak A}$ includes a non-zero element of $\frak A$, which in
turn includes a non-zero element of finite measure.    Since
$\widehat{\frak A}$ is Dedekind complete, $(\widehat{\frak A},\bar\mu)$
is localizable.

If $\bar\mu a$ is finite, then surely $\tilde\mu a=\bar\mu a$ is finite.
If $\tilde\mu c$ is finite, then $A=\{a:a\in\frak A,\,a\Bsubseteq c\}$ is
upwards-directed and $\sup_{a\in A}\bar\mu a=\tilde\mu c$ is finite, so
$b=\sup A$ is defined in $\frak A$ and $\bar\mu b=\tilde\mu c$.
Because $\frak A$ is order-dense in $\widehat{\frak A}$, $b=c$ (313K,
313O) and $c\in\frak A$, with $\bar\mu c=\tilde\mu c$.
}%end of proof of 322P

\leader{322Q}{Definition}\dvAformerly{3{}22P} Let
$(\frak A,\bar\mu)$ be any semi-finite
measure algebra.   I will call $(\widehat{\frak A},\tilde\mu)$, as
constructed
above, the {\bf localization} of $(\frak A,\bar\mu)$.   \cmmnt{Of
course it is unique just in so far as the Dedekind completion of
$\frak A$ is.}

\leader{322R}{Further properties of Stone
spaces:  Proposition}\dvAformerly{3{}22Q} Let
$(\frak A,\bar\mu)$ be a semi-finite measure algebra and
$(Z,\Sigma,\nu)$ its Stone space.

(a) Meager sets in $Z$ are nowhere dense;  every $E\in\Sigma$ is
uniquely expressible as $G\symmdiff M$ where $G\subseteq Z$ is
open-and-closed and $M$ is nowhere dense, and
$\nu E=\sup\{\nu H:H\subseteq E$ is open-and-closed$\}$.

(b) The c.l.d.\ version $\tilde\nu$ of $\nu$ is strictly localizable,
and has the same negligible sets as $\nu$.

(c) If $(\frak A,\bar\mu)$ is totally finite then
$\nu E=\inf\{\nu H:H\supseteq E$ is open-and-closed$\}$ for every
$E\in\Sigma$.

\proof{{\bf (a)} I have already remarked (in the proof of 322O) that
$\frak A$ is weakly $(\sigma,\infty)$-distributive, so that meager sets
in $Z$ are nowhere dense.   But we know that every member of $\Sigma$ is
expressible as $G\symmdiff M$ where $G$ is open-and-closed and $M$ is
meager, therefore nowhere dense.   Moreover, the expression is unique,
because if $G\symmdiff M=G'\symmdiff M'$ then
$G\symmdiff G'\subseteq M\cup M'$ is open and nowhere dense, therefore
empty, so $G=G'$ and $M=M'$.

Now let $a\in\frak A$ be such that $\widehat{a}=G$, and consider
$B=\{b:b\in\frak A,\,\widehat{b}\subseteq E\}$.   Then $\sup B=a$ in
$\frak A$.   \Prf\ If $b\in B$, then
$\widehat{b}\setminus\widehat{a}\subseteq M$
is nowhere dense, therefore empty;  so $a$ is an upper bound for $B$.
\Quer\ If $a$ is not the supremum of $B$, then there is a non-zero
$c\Bsubseteq a$ such that $b\Bsubseteq a\Bsetminus c$ for every
$b\in B$.   But now $\widehat{c}$ cannot be empty, so
$\widehat{c}\setminus\overline{M}$
is non-empty, and there is a non-zero $d\in\frak A$ such that
$\widehat{d}\subseteq\widehat{c}\setminus\overline{M}$.   In this case
$d\in B$ and $d\notBsubseteq a\Bsetminus c$.\ \BanG\  Thus $a=\sup B$.\
\Qed

It follows that

$$\eqalign{\nu E
&=\nu G
=\bar\mu a
=\sup_{b\in B}\bar\mu b\cr
&=\sup_{b\in B}\nu\widehat{b}
\le\sup\{\nu H:H\subseteq E\text{ is open-and-closed}\}
\le\nu E\cr}$$

\noindent and $\nu E=\sup\{\nu H:H\subseteq E$ is open-and-closed$\}$.

\medskip

{\bf (b)} This is the same as part (c) of the proof of 322O.   We have a
disjoint family $\Cal C$ of sets of finite measure for $\nu$ such that
whenever $E\in\Sigma$ and $\nu E>0$ there is a $C\in\Cal C$ such that
$\mu(C\cap E)>0$.   Now if $\tilde\nu F$ is defined and not $0$, there
is an $E\in\Sigma$ such that $E\subseteq F$ and $\nu E>0$ (213Fc), so
that there is a $C\in\Cal C$ such that $\nu(E\cap C)>0$;  since
$\nu C<\infty$, we have

\Centerline{$\tilde\nu(F\cap C)\ge\tilde\nu(E\cap C)=\nu(E\cap C)>0$.}

\noindent And of course $\tilde\nu C<\infty$ for every $C\in\Cal C$.
This means that $\Cal C$ witnesses that $\tilde\nu$ satisfies the
conditions of 213O, so that $\tilde\nu$ is strictly localizable.

Any $\nu$-negligible set is surely $\tilde\nu$-negligible.   If $M$ is
$\tilde\nu$-negligible then it is nowhere dense.   \Prf\ If
$G\subseteq Z$ is open and not empty then there is a non-empty
open-and-closed set
$H_1\subseteq G$, and now $H_1\in\Sigma$, so there is a non-empty
open-and-closed set $H\subseteq H_1$ such that $\nu H$ is finite
(because $\nu$ is semi-finite).   In this case $H\cap M$ is
$\nu$-negligible, therefore nowhere dense, and
$H\not\subseteq\overline{M}$.   But this means that
$G\not\subseteq\overline{M}$;  as $G$ is arbitrary, $M$ is nowhere
dense.\ \QeD\  Accordingly $M\in\Cal M$ and is $\nu$-negligible.

Thus $\nu$ and $\tilde\nu$ have the same negligible sets.

\medskip

{\bf (c)} Because $\nu Z<\infty$,

$$\eqalign{\nu E
&=\nu Z-\nu(Z\setminus E)
=\nu Z-\sup\{\nu H:H\subseteq Z\setminus E
  \text{ is open-and-closed}\}\cr
&=\inf\{\nu(Z\setminus H):H\subseteq Z\setminus E
  \text{ is open-and-closed}\}\cr
&=\inf\{\nu H:H\supseteq E\text{ is open-and-closed}\}.\cr}$$
}%end of proof of 322R

\exercises{
\leader{322X}{Basic exercises $\pmb{>}$(a)}
%\spheader 322Xa
Let $(\frak A,\bar\mu)$ be a measure
algebra.   Let $I_{\infty}$ be the set of those $a\in \frak A$ which are
either $0$ or
`purely infinite', that is, $\bar\mu b=\infty$ for every non-zero
$b\Bsubseteq a$.   Show that $I_{\infty}$ is a $\sigma$-ideal of
$\frak A$.   Show that there is a function
$\bar\mu_{sf}:\frak A/I_{\infty}\to[0,\infty]$ defined by setting
$\bar\mu_{sf}a^{\ssbullet}
=\sup\{\bar\mu b:b\Bsubseteq a,\,\bar\mu b<\infty\}$ for
every $a\in\frak A$.   Show that $(\frak A/I_{\infty},\bar\mu_{sf})$ is
a semi-finite measure algebra.

\header{322Xb}{\bf (b)} Let $(X,\Sigma,\mu)$ be a measure space and let
$\mu_{sf}$ be the `semi-finite version' of $\mu$, as defined in 213Xc.
Let $(\frak A,\bar\mu)$ be the measure algebra of $(X,\Sigma,\mu)$.
Show that the measure algebra of $(X,\Sigma,\mu_{sf})$ is isomorphic to
the measure algebra $(\frak A/I_{\infty},\bar\mu_{sf})$ of (a) above.

\spheader 322Xc Let $(X,\Sigma,\mu)$ be a measure space and
$(X,\tilde\Sigma,\tilde\mu)$ its c.l.d.\ version.   Let
$(\frak A,\bar\mu)$
and $(\frak A_2,\bar\mu_2)$ be the corresponding measure algebras, and
$\pi:\frak A\to\frak A_2$ the canonical homomorphism, as in 322Db.
Show that the kernel of $\pi$ is the ideal $I_{\infty}$, as described in
322Xa, so that $\frak A/I_{\infty}$ is isomorphic, as Boolean algebra,
to $\pi[\frak A]\subseteq\frak A_2$.   Show that this isomorphism
identifies $\bar\mu_{sf}$, as described in 322Xa, with
$\bar\mu_2\restr\pi[\frak A]$.
%322Xa, 322D

\spheader 322Xd Give a direct proof of 322G, not relying on 215B
and 321J.
%322G

\sqheader 322Xe Let $(\frak A,\bar\mu)$ be any measure algebra, $A$ a
non-empty subset of $\frak A$, and $c\in\frak A$ such that
$\bar\mu c<\infty$.   Show that (i) $c_0=\sup\{a\Bcap c:a\in A\}$ is
defined in
$\frak A$ (ii) there is a countable set $B\subseteq A$ such that
$c_0=\sup\{a\Bcap c:a\in B\}$.
%322G

\spheader 322Xf Let $(X,\Sigma,\mu)$ be a measure space and $A$ any
subset of $X$;  let $\mu_A$ be the subspace measure on $A$ and
$\Sigma_A$ its domain.   Write $(\frak A,\bar\mu)$ for the measure
algebra of $(X,\Sigma,\mu)$ and $(\frak A_A,\bar\mu_A)$ for the
measure algebra of $(A,\Sigma_A,\mu_A)$.   Show that the formula
$F^{\ssbullet}\mapsto (F\cap A)^{\ssbullet}$ defines a
sequentially order-continuous Boolean homomorphism
$\pi:\frak A\to\frak A_A$ which has kernel
$I=\{F^{\ssbullet}:F\in\Sigma,\,F\cap A=\emptyset\}$.
Show that for any $a\in\frak A$,
$\bar\mu_A(\pi a)=\min\{\bar\mu b:b\in\frak A,\,a\Bsetminus b\in I\}$.
%322I

\spheader 322Xg Let $(\frak A,\bar\mu)$ be a measure algebra and
$\frak B$ a regularly embedded $\sigma$-subalgebra of $\frak A$.
Suppose that
$(\frak B,\bar\mu\restrp\frak{B})$ is semi-finite.   Show that
$(\frak A,\bar\mu)$ is semi-finite.

\spheader 322Xh Let $(\frak A,\bar\mu)$ be any measure algebra and
$(Z,\Sigma,\nu)$ its Stone space.   Show that the c.l.d.\ version of
$\nu$ is strictly localizable.
%322O

\leader{322Y}{Further exercises (a)} Let $X$ be a set, $\Sigma$ a
$\sigma$-algebra of subsets of $X$, and $\Cal I$ a $\sigma$-ideal of
$\Sigma$.   Set $\Cal N=\{N:\,\exists\,F\in\Cal I,\,N\subseteq F\}$.
Show that $\Cal N$ is a $\sigma$-ideal of subsets of $X$.   Set
$\hat\Sigma=\{E\symmdiff N:E\in\Sigma,\,N\in\Cal N\}$.   Show that
$\hat\Sigma$ is a $\sigma$-algebra of subsets of $X$ and that
$\hat\Sigma/\Cal N$ is isomorphic to $\Sigma/\Cal I$.

\header{322Yb}{\bf (b)} Let $(\frak A,\bar\mu)$ be a semi-finite measure
algebra, and $(Z,\Sigma,\nu)$ its Stone space.   Let $\tilde\nu$ be the
c.l.d.\ version of $\nu$, and $\tilde\Sigma$ its domain.   Show that
$\tilde\Sigma$ is precisely the Baire-property algebra
$\{G\symmdiff A:G\subseteq Z$ is open, $A\subseteq Z$ is meager$\}$, so
that $\tilde\Sigma/\Cal M$ can be
identified with the regular open algebra of $Z$ (314Yd) and the measure
algebra of $\tilde\nu$ can be identified with the localization of
$\frak A$.

\spheader 322Yc Give an example of a localizable measure algebra
$(\frak A,\bar\mu)$ with a $\sigma$-subalgebra $\frak B$ such that
$(\frak B,\bar\mu\restrp\frak{B})$ is semi-finite and atomless, but
$\frak A$ has an atom.

\spheader 322Yd Let $(X,\Sigma,\mu)$ be a measure space and
$A\subseteq X$ a subset;  let $\mu_A$ be the subspace measure on $A$,
$\frak A$ and $\frak A_A$ the measure algebras of $\mu$ and $\mu_A$, and
$\pi:\frak A\to\frak A_A$ the canonical homomorphism, as described in
322Xf.   (i) Show that if
$\mu_A$ is semi-finite, then $\pi$ is order-continuous.   (ii) Show that
if $\mu$ is semi-finite but $\mu_A$ is not, then $\pi$ is not
order-continuous.
%322Xf, 322I

\spheader 322Ye Show that if $(\frak A,\bar\mu)$ is a semi-finite measure algebra,
with Stone space $(Z,\Sigma,\nu)$, then $\nu$ has locally determined
negligible sets in the sense of 213I.

\spheader 322Yf Let $(\frak A,\bar\mu)$ be a localizable measure algebra
and $(Z,\Sigma,\nu)$ its Stone space.   (i) Show that a function
$f:Z\to\Bbb R$
is $\Sigma$-measurable iff there is a conegligible set $G\subseteq X$
such that $f\restr G$ is continuous.   \Hint{316Yi.}
(ii) Show that $f:Z\to[0,1]$ is $\Sigma$-measurable iff there is a
continuous function $g:Z\to[0,1]$ such that $f=g\,\,\nu$-a.e.
}%end of exercises

\cmmnt{
\Notesheader{322} I have taken this leisurely tour through the concepts
of Chapter 21 partly to recall them (or persuade you to look them up)
and partly to give you practice in the elementary manipulations of
measure algebras.   The really vital result here is the correspondence
between `localizability' in measure spaces and measure algebras.   Part
of the object of this volume (particularly in Chapter 36) is to try to
make sense of the properties of localizable measure spaces, as discussed
in Chapter 24 and elsewhere, in terms of their measure algebras.   I
hope that 322Be has already persuaded you that the concept really
belongs to measure algebras, and that the formulation in terms of
`essential suprema' is a dispensable expedient.

I have given proofs of 322C and 322G depending on the realization of an
arbitrary measure algebra as the measure algebra of a measure space, and
the corresponding theorems for measure spaces, because this seems the
natural approach from where we presently stand;  but I am sympathetic to
the view that such proofs must be inappropriate, and that it is in some
sense better
style to look for arguments which speak only of measure algebras
(322Xd).

For any measure algebra $(\frak A,\bar\mu)$, the set $\frak A^f$ of
elements of finite measure is an ideal of $\frak A$;  consequently it is
order-dense iff it includes a partition of unity (322E).   In 322F we
have something deeper:  any semi-finite measure algebra must be weakly
$(\sigma,\infty)$-distributive when regarded as a Boolean algebra, and
this has significant consequences in its Stone space, which are used in
the proofs of 322O and 322R.   Of course a result of this kind must
depend on
the semi-finiteness of the measure algebra, since any Dedekind
$\sigma$-complete Boolean algebra becomes a measure algebra if we give
every non-zero element the measure $\infty$.   It is natural to look for
algebraic conditions on a Boolean algebra sufficient to make it
`measurable', in the sense that it should carry a semi-finite measure;
this is an unresolved
problem to which I will return in Chapter 39.

Subspace measures, indefinite-integral measures,
simple products, direct sums, principal ideals and
order-closed subalgebras give no real surprises;  I spell out the
details in 322H-322N %322H 322I 322J 322L 322M 322N
and 322Xf-322Xg.   It is worth noting that completing a
measure space has no effect on its measure algebra (322D, 322Ya).   We
see also that
from the point of view of measure algebras there is no distinction to be
made between `localizable' and `strictly localizable', since every
localizable measure algebra is representable as the measure algebra of a
strictly localizable measure space (322Ld).   (But strict localizability
does have implications for some processes starting in the measure
algebra;  see 322M.)   It is nevertheless remarkable that the
canonical measure on the Stone space of a semi-finite measure algebra is
localizable iff it is strictly localizable (322O).   This canonical
measure has many other interesting properties, which I skim over in
322R, 322Xh, 322Yb and 322Yf.
In Chapter 21 I discussed a number of methods of improving measure
spaces, notably `completions' (212C) and `c.l.d.\ versions' (213E).
Neither of
these is applicable in any general way to measure algebras.   But in
fact we have a more effective construction, at least for
semi-finite measure algebras, that of `localization'
(322P-322Q);  I say that it is more effective just because
localizability is more important than completeness or local
determinedness, being of vital importance in the behaviour of function
spaces (241Gb, 243Gb, 245Ec, 363M,
364M, 365M, 367M, 369A, 369C).   Note that the localization of a
semi-finite
measure algebra does in fact correspond to the c.l.d.\ version of a
certain measure (322Yb).   But of course $\frak A$ and
$\widehat{\frak A}$ do
{\it not} have the same Stone spaces, even when $\widehat{\frak A}$ can
be effectively represented as the measure algebra of a measure on the
Stone space of $\frak A$.   What is happening in 322Yb is that we are
using all
the open sets of $Z$ to represent members of $\widehat{\frak A}$, not
just the open-and-closed sets, which correspond to members of $\frak A$.
}%end of notes

\discrpage

