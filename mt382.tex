\frfilename{mt382.tex}
\versiondate{15.8.06}
\copyrightdate{2003}

\def\chaptername{Automorphisms}
\def\sectionname{Factorization of automorphisms}

\def\cycleii#1#2#3{\cycle{#1\,_{#2}\,#3}}
\def\Orbit{\mathop{\text{Orb}}}

\newsection{382}

My aim in this chapter is to investigate the automorphism groups of
measure algebras, but as usual I prefer to begin with results which can
be expressed in the language of general Boolean algebras.
The principal theorems in this section are 382M, giving a sufficient
condition
for every member of a full group of automorphism to be a product of
involutions, and 382R, describing the normal subgroups of full groups.
The former depends on Dedekind $\sigma$-completeness and the presence of
`separators' (382Aa);  the latter needs a Dedekind complete algebra and
a group with `many involutions' (382O).   Both
concepts are chosen with a view to the next section, where the results
will be applied to groups of measure-preserving automorphisms.


\leader{382A}{Definitions}
Let $\frak A$ be a Boolean algebra and $\pi\in\Aut\frak A$.

\spheader 382Aa I say that $a\in\frak A$ is a {\bf separator} for $\pi$
if $a\Bcap\pi a=0$ and $\pi b=b$
whenever $b\in\frak A$ and $b\Bcap\pi^na=0$ for every $n\in\Bbb Z$.

\spheader 382Ab I say that $a\in\frak A$ is a {\bf transversal} for
$\pi$ if $\sup_{n\in\Bbb Z}\pi^na=1$ and $\pi^nb=b$ whenever
$n\in\Bbb Z$ and $b\Bsubseteq a\Bcap\pi^na$.

%transversals
\leader{382B}{Lemma} Let $\frak A$ be a Boolean algebra and
$\pi\in\Aut\frak A$.   If every power of $\pi$ has a separator and
$\pi^n$ is the identity, where $n\ge 1$, then $\pi$ has a transversal.

\proof{{\bf (a)} For $0\le j<n$ let $a_j\in\frak A$ be a separator for
$\pi^j$.   Let $\frak B$ be the subalgebra of $\frak A$ generated by
$A=\{\pi^ia_j:0\le i$, $j<n\}$.   Because $\pi[A]=A$,
$\pi[\frak B]=\frak B$.
(The set $\{a:a\in\frak B$, $\pi a\in\frak B$, $\pi^{-1}a\in\frak B\}$
is a subalgebra of $\frak A$ including $A$, so must be $\frak B$.)
Because $A$ is finite, so is $\frak B$;  let $B$ be the set of atoms of
$\frak B$.   Then $\pi\restr B$ is a permutation of the finite set $B$.

\medskip

{\bf (b)} Let $\Cal C$ be the set of orbits of $\pi\restr B$, that is,
the family of sets of the form $\{\pi^kb:k\in\Bbb Z\}$ for $b\in B$.
If $b\in C\in\Cal C$, set $m=\#(C)$;  then $d=\pi^md$ for every
$d\Bsubseteq b$.   \Prf\ If $m=n$ this is trivial.   Otherwise, $b$ is
either disjoint from, or included in, $\pi^ia_m$ whenever $0\le i<n$,
and therefore for every $i\in\Bbb Z$.
But we have $a_m\Bcap\pi^ma_m=0$, so $\pi^ia_m\Bcap\pi^{i+m}a_m=0$ for
every $i$, and $b=\pi^mb$ must be disjoint from $\pi^ia_m$, for every $i$.
By the other clause in
the definition of `separator', $\pi^md=d$ for every $d\Bsubseteq b$.\
\Qed

\medskip

{\bf (c)} For each $C\in\Cal C$, choose $b_C\in C$.   Set
$c=\sup_{C\in\Cal C}b_C$.   Then $c$ is a transversal for $\pi$.   \Prf\
If $C\in\Cal C$, we have $\pi^nb_C=b_C$, so $k_C=\#(C)$ is a factor of
$n$.   Now

\Centerline{$\sup_{0\le k<n}\pi^kc=\sup_{C\in\Cal C,0\le k<n}\pi^kb_C
=\sup_{C\in\Cal C}\sup C=\sup(\bigcup\Cal C)=\sup B=1$.}

\noindent So certainly $\sup_{k\in\Bbb Z}\pi^kc=1$.   Now suppose that
$k\in\Bbb Z\setminus\{0\}$ and $d\Bsubseteq c\Bcap\pi^kc$.   Set
$B_0=\{b:b\in B$, $d\Bcap b\ne 0\}$.   If $b\in B_0$, then
$b\Bcap c\ne 0$, so $b=b_C$ where $C\in\Cal C$ is the orbit of
$\pi\restr B$ containing $b$.   Next, $d\Bcap b\Bcap\pi^kc\ne 0$, so
$\pi^{-k}(d\Bcap b)\Bcap c\ne 0$ and there is a $b'\in B$ such that
$\pi^{-k}(d\Bcap b)\Bcap b'\ne 0$;  in this case we must have
$b'=\pi^{-k}b\in C$.   But as
$b'\Bcap c\Bsupseteq\pi^{-k}(d\Bcap b)\Bcap c$ is non-zero, $b'=b_C=b$.
Thus $b=\pi^kb$ and $k$ is a multiple of $\#(C)$.   Since
$\pi^{\#(C)}(d\Bcap b)=d\Bcap b$, by (b), $\pi^k(d\Bcap b)=d\Bcap b$.

This is true for every $b\in B$ meeting $d$;  so

\Centerline{$\pi^kd=\pi^k(\sup_{b\in B_0}d\Bcap b)
=\sup_{b\in B_0}\pi^k(d\Bcap b)=\sup_{b\in B_0}d\Bcap b=d$.}

\noindent As $k$ and $d$ are arbitrary, $c$ is a transversal for $\pi$.\ \Qed
}%end of proof of 382B

\leader{382C}{Corollary} If $\frak A$ is a Boolean algebra and
$\pi\in\frak A$ is an involution, then $\pi$ is an exchanging involution
iff it has a separator iff it has a transversal.

\proof{ If $\pi$ exchanges $a$ and $\pi a$ then of course
$a$ is a separator for $\pi$.   If $\pi$
has a separator, then every power of $\pi$ has a separator, so 382B
tells us that $\pi$ has a transversal.   If $a$ is a transversal for $\pi$
then $a\Bcup\pi a=\sup_{n\in\Bbb Z}\pi^na=1$ and $\pi b=b$ whenever
$b\Bsubseteq a\Bcap\pi a$, so $\pi$ exchanges $a\Bsetminus\pi a$ and
$\pi a\Bsetminus a$.
}%end of proof of 382C

\leader{382D}{Lemma} Let $\frak A$ be a Dedekind $\sigma$-complete
Boolean algebra and $\pi\in\Aut\frak A$.   Then the following are
equiveridical:

(i) $\pi$ has a separator;

(ii) there is an $a\in\frak A$ such that $a\Bcap\pi a=0$ and
$a\Bcup\pi a\Bcup\pi^2a$ supports $\pi$;

(iii) there is a sequence $\sequencen{a_n}$ in $\frak A$ such that
$\sup_{n\in\Bbb N}\pi a_n\Bsetminus a_n$ supports $\pi$;

(iv) there is a
partition of unity $(a',a'',b',b'',c,e)$ in $\frak A$ such that

\Centerline{$\pi a'=b'$,
\quad$\pi a''=b''$,
\quad$\pi b''=c$,
\quad$\pi(b'\Bcup c)=a'\Bcup a''$,
\quad$\pi d=d$ for every $d\Bsubseteq e$.}

\proof{{\bf (i)$\Rightarrow$(ii)} Suppose that $a$ is a separator for
$\pi$.   Set $a^+=\sup_{n\ge 1}\pi^na$, $a^-=\sup_{n\ge 1}\pi^{-n}a$;
we are supposing that $a\Bcap\pi a=0$ and that $a\Bcup a^+\Bcup a^-$
supports $\pi$.   For $n\in\Bbb N$ set
$a_n=\pi^na\Bsetminus\sup_{0\le i<n}\pi^ia$, so that $\sequencen{a_n}$
is disjoint and has supremum $a\Bcup a^+$.   Set
$b_1=\sup_{n\in\Bbb N}a_{2n}\Bsetminus\pi^{-1}a$.
Since $a\Bcap\pi^{-1}a=\pi^{-1}(a\Bcap\pi a)=0$,
$a\Bsubseteq b_1\Bsubseteq a\Bcup a^+$.
For any $n\in\Bbb N$,

\Centerline{$\pi(a_{2n}\Bsetminus\pi^{-1}a)
=\pi^{2n+1}a\Bsetminus(a\Bcup\sup_{1\le i\le 2n}\pi^ia)=a_{2n+1}$,}

\noindent so $b_1\Bcap\pi b_1=0$.   Note that $\pi b_1\Bsubseteq a^+$,
while $a^+\Bsetminus\pi^{-1}a\Bsubseteq b_1\Bcup\pi b_1$.

Set $c=a\Bsetminus a^+$.   Then

\Centerline{$\pi^ic\Bcap\pi^jc
=\pi^j(c\Bcap\pi^{i-j}c)\Bsubseteq\pi^j(a\Bsetminus\pi^{i-j}a^+)
\Bsubseteq\pi^j(a\Bsetminus\pi^{i-j}\pi^{j-i}a)
=0$}

\noindent whenever $i<j$ in $\Bbb Z$, so $\family{k}{\Bbb Z}{\pi^kc}$ is
disjoint.   We have

$$\eqalign{\sup_{n\ge 1}\pi^{-n}c
&=\sup_{n\ge 1}(\pi^{-n}a\Bsetminus\sup_{i>-n}\pi^ia)
=\sup_{n\ge 1}(\pi^{-n}a\Bsetminus\sup_{0\le i<n}\pi^{-i}a)
  \Bsetminus(a\Bcup a^+)\cr
&=\sup_{n\ge 1}\pi^{-n}a\Bsetminus(a\Bcup a^+)
=a^-\Bsetminus(a\Bcup a^+).\cr}$$

\noindent If $k\ge 1$ and $i\ge 0$ then

\Centerline{$\pi^{-k}c\Bcap\pi^ia=\pi^{-k}(c\Bcap\pi^{i+k}a)
\Bsubseteq\pi^{-k}(c\Bcap a^+)=0$;}

\noindent as $i$ is arbitrary, $\pi^{-k}c\Bcap b_1=0$.   So if we set
$b=b_1\Bcup\sup_{k\ge 1}\pi^{-2k}c$,

$$\eqalign{b\Bcap\pi b
&=(b_1\Bcap\pi b_1)
  \Bcup(\sup_{k\ge 1}b_1\Bcap\pi^{1-2k}c)
  \Bcup(\sup_{k\ge 1}\pi^{-2k}c\Bcap\pi b_1)
  \Bcup(\sup_{j,k\ge 1}\pi^{-2j}c\Bcap\pi^{1-2k}c)\cr
&\Bsubseteq 0\Bcup 0\Bcup\pi(\sup_{k\ge 1}\pi^{-2k-1}c\Bcap b_1)
  \Bcup 0
=0.\cr}$$

\noindent Since

$$\eqalign{b\Bcup\pi b\Bcup\pi^{-1}b
&\Bsupseteq b_1\Bcup\pi b_1\Bcup\pi^{-1}a
  \Bcup\sup_{n\ge 1}\pi^{-n}c\cr
&\Bsupseteq a\Bcup a^+\Bcup(a^-\Bsetminus(a\Bcup a^+))
=a\Bcup a^+\Bcup a^-\cr}$$

\noindent supports $\pi$, $\pi^{-1}b$ witnesses that (ii) is
true.

\medskip

{\bf (ii)$\Rightarrow$(iii)} If $a\in\frak A$ is such that
$a\Bcap\pi a=0$ and $a\Bcup\pi a\Bcup\pi^2a$ supports $\pi$, then
$\pi^{n+1}a=\pi^{n+1}a\Bsetminus\pi^na$ for every $n$, so we can set
$a_n=\pi^{n-1}a$ for each $n$ to obtain a sequence witnessing (iii).

\medskip

{\bf (iii)$\Rightarrow$(i)} If $\sequencen{a_n}$ is such that
$\sup_{n\in\Bbb N}\pi a_n\Bsetminus a_n$ supports $\pi$, set
$b_n=\sup_{k\in\Bbb Z}\pi^k(\pi a_n\Bsetminus a_n)$,
$c_n=b_n\Bsetminus\sup_{0\le i<n}b_i$ for each $n\in\Bbb N$.   Then
$\pi b_n=b_n$ and $\pi c_n=c_n$ for every $n\in\Bbb N$, while
$\sequencen{c_n}$ is disjoint.   Set
$a=\sup_{n\in\Bbb N}c_n\Bcap a_n\Bsetminus\pi^{-1}a_n$.    Then

$$\eqalign{a\Bcap\pi a
&=\sup_{m,n\in\Bbb N}(c_m\Bcap a_m\Bsetminus\pi^{-1}a_m)
  \Bcap(\pi c_n\Bcap\pi a_n\Bsetminus a_n)\cr
&=\sup_{m,n\in\Bbb N}(c_m\Bcap a_m\Bsetminus\pi^{-1}a_m)
  \Bcap(c_n\Bcap\pi a_n\Bsetminus a_n)\cr
&=\sup_{n\in\Bbb N}c_n\Bcap(a_n\Bsetminus\pi^{-1}a_n)
  \Bcap(\pi a_n\Bsetminus a_n)
=0.\cr}$$

\noindent Next,

$$\eqalign{\sup_{k\in\Bbb Z}\pi^ka
&=\sup_{n\in\Bbb N,k\in\Bbb Z}c_n\Bcap\pi^ka_n\Bsetminus\pi^{k-1}a_n\cr
&=\sup_{n\in\Bbb N,k\in\Bbb Z}c_n\Bcap\pi^{k+1}a_n\Bsetminus\pi^ka_n
=\sup_{n\in\Bbb N}c_n\Bcap b_n\cr
&=\sup_{n\in\Bbb N}c_n
=\sup_{n\in\Bbb N}b_n
\Bsupseteq\sup_{n\in\Bbb N}\pi a_n\Bsetminus a_n\cr}$$

\noindent supports $\pi$.   So $a$ is a separator for $\pi$.

\medskip

{\bf (ii)$\Rightarrow$(iv)} Let $a$ be such that $a\Bcap\pi a=0$ and
$a\Bcup\pi a\Bcup\pi^2a$ supports $\pi$.   Set
$c=\pi^2a\Bsetminus(a\Bcup\pi a)$, $b''=\pi^{-1}c\Bsubseteq\pi a$,
$b'=\pi a\Bsetminus b''$, $a''=\pi^{-1}b''\Bsubseteq a$, $a'=a\Bsetminus
a''$ and $e=1\Bsetminus(a\Bcup\pi a\Bcup\pi^2a)$.   Then $(a,\pi a,c,e)$
and $(a',a'',b',b'',c,e)$ are partitions of unity in $\frak A$;
$\pi a''=b''$;   $\pi a'=\pi a\Bsetminus b''=b'$;  $\pi b''=c$;
$\pi d=d$ for every $d\Bsubseteq e$;  so

\Centerline{$\pi(b'\Bcup c)
=\pi(1\Bsetminus(a\Bcup b''\Bcup e))
=1\Bsetminus(\pi a\Bcup\pi b''\Bcup\pi e)
=1\Bsetminus(\pi a\Bcup c\Bcup e)
=a=a'\Bcup a''$.}

\medskip

{\bf (iv)$\Rightarrow$(ii)} If $a',a'',b',b'',c,e$ witness (iv), then
$a=a'\Bcup a''$ witnesses (ii).
}%end of proof of 382D

\leader{382E}{Corollary} (a) If $\frak A$ is a Dedekind
$\sigma$-complete Boolean algebra and $\pi\in\Aut\frak A$ has a
separator, then $\pi$ has a support.

(b) If $\frak A$ is a Dedekind complete Boolean algebra then every
$\pi\in\Aut\frak A$ has a separator.

\proof{{\bf (a)} Taking $a\in\frak A$ such that $a\Bcap\pi a=0$ and
$e=a\Bcup\pi a\Bcup\pi^2a$ supports $\pi$, we see that $e$ must actually
be the support of $\pi$ (381Ei, 381Ea).

\medskip

{\bf (b)} If $\frak A$ is Dedekind complete and $\pi\in\Aut\frak A$,
let $P$ be the set $\{d:d\in\frak A,\,d\Bcap\pi d=0\}$.
Then $P$ has a maximal element.   \Prf\ Of course $P\ne\emptyset$, as
$0\in P$.   If $Q\subseteq P$ is non-empty and upwards-directed, set
$a=\sup Q$, which is defined because $\frak A$ is Dedekind complete;
then $\pi a=\sup\pi[Q]$ (since $\pi$, being an automorphism, is surely
order-continuous).   If $d_1$, $d_2\in Q$, there is a $d\in Q$ such that
$d_1\Bcup d_2\Bsubseteq d$, so
$d_1\Bcap\pi d_2\Bsubseteq d\Bcap\pi d=0$.   By 313Bc, $a\Bcap\pi a=0$.
This means that $a\in P$ and is an upper bound for $Q$ in $P$.   As $Q$
is arbitrary, Zorn's Lemma tells us that $P$ has a maximal element.\
\Qed

Let $b\in P$ be maximal.   Then $b\Bcap\pi b=0$.   Set
$e=b\Bcup\pi a\Bcup\pi^{-1}b$.   \Quer\ If $e$ does not support $\pi$,
let $d\Bsubseteq 1\Bsetminus e$ be such that $d\Bcap\pi d=0$ (381Ei).
Then $d\Bcap\pi b\Bsubseteq d\Bcap e=0$, while also
$b\Bcap\pi d\Bsubseteq\pi(\pi^{-1}b\Bcap d)\Bsubseteq\pi(e\Bcap d)=0$;
so $(b\Bcup d)\Bcap\pi(b\Bcup d)=0$, and $b\Bsubset b\Bcup d\in P$,
which is impossible.\ \BanG\   So if we set $a=\pi^{-1}b$ we have a
witness of 382D(ii), and $\pi$ has a separator.
}%end of proof of 382E

\cmmnt{\medskip

\noindent{\bf Remark} 382Eb and 382D(i)$\Leftrightarrow$(ii) together amount to
`Frol\'\i k's theorem' ({\smc Frol\'\i k 68}).
}

\leader{382F}{Corollary} Let $\frak A$ be a Dedekind complete Boolean
algebra.

(a) Every involution in $\Aut\frak A$ is an exchanging involution.

(b) %3{8}5B
If $\pi\in\Aut\frak A$ is periodic with period $n\ge 2$, there is an
$a\in\frak A$ such that
$(a,\pi a,\pi^2a,\ldots,\pi^{n-1}a)$ is a partition of unity in
$\frak A$\cmmnt{;  that is (in the language of 381R) $\pi$ is of the
form $\cycle{a_1\,_{\pi}\,a_2\,_{\pi}\,\ldots\,_{\pi}\,a_n}$ where
$(a_1,\ldots,a_n)$ is a partition of unity in $\frak A$}.

\proof{{\bf (a)} By 382Eb, every involution has a separator;  now use
382C.

\medskip

{\bf (b)} Again because every automorphism has a separator, 382B tells
us that $\pi$ has a transversal $a$.   In this case, $a\Bcap\pi^ka$ must
be disjoint from the support of $\pi^k$ for every $k\in\Bbb Z$;  since
$\supp\pi^k=1$ for $0<k<n$, $a\Bcap\pi^ka=0$ for $0<k<n$;  of course it
follows that $\pi^ia\Bcap\pi^ja=\pi^i(a\Bcap\pi^{j-i}a)=0$ if
$0\le i<j<n$.   So $a,\pi a,\ldots,\pi^{n-1}a$ are disjoint;  since
$\sup_{0\le i<n}\pi^ia=\sup_{i\in\Bbb Z}\pi^ia=1$, they constitute a
partition of unity.
}%end of proof of 382F

\leader{382G}{Lemma} Let $\frak A$ be a Dedekind $\sigma$-complete
Boolean algebra and $\pi\in\Aut\frak A$.

(a) Suppose that
$\sequencen{a_n}$ is a family in $\frak A$ such that $\pi a_n=a_n$ and
$\pi\restrp\frak A_{a_n}$ has a transversal for every $n$.   Set
$a=\sup_{n\in\Bbb N}a_n$;  then $\pi a=a$ and $\pi\restrp\frak A_a$ has
a transversal.

(b) If $a$ is a transversal for $\pi$ it is a transversal for
$\pi^{-1}$.

\wheader{382G}{0}{0}{0}{36pt}

(c) Suppose that $a\in\frak A$.   Set

\Centerline{$a^*=\sup_{n\in\Bbb Z}(\pi^na\Bsetminus\sup_{i>n}\pi^ia)$,
\quad$a_*=\sup_{n\in\Bbb Z}(\pi^na\Bsetminus\sup_{i<n}\pi^ia)$.}

\noindent Then
$\pi a^*=a^*$, $\pi a_*=a_*$ and $\pi\restrp\frak A_{a^*}$,
$\pi\restrp\frak A_{a_*}$ both have transversals.

\proof{{\bf (a)} Of course $\pi a=\sup_{n\in\Bbb N}\pi a_n=a$, so we can
speak of $\pi\restrp\frak A_a$.   For each $n\in\Bbb N$, let $b_n$ be a
transversal for $\pi\restrp\frak A_{a_n}$.   Set
$b=\sup_{n\in\Bbb N}(b_n\Bsetminus\sup_{i<n}a_i)$.   Then $b$ is a
transversal for $\pi\restrp\frak A_a$.   \Prf\ Of course
$b\in\frak A_a$.   Now

$$\eqalign{\sup_{k\in\Bbb Z}\pi^kb
&=\sup_{k\in\Bbb Z}
   \sup_{n\in\Bbb N}(\pi^kb_n\Bsetminus\sup_{i<n}\pi^ka_i)
=\sup_{n\in\Bbb N}
   \sup_{k\in\Bbb Z}(\pi^kb_n\Bsetminus\sup_{i<n}a_i)\cr
&=\sup_{n\in\Bbb N}
   \bigl((\sup_{k\in\Bbb Z}\pi^kb_n)\Bsetminus\sup_{i<n}a_i\bigr)
=\sup_{n\in\Bbb N}(a_n\Bsetminus\sup_{i<n}a_i)
=\sup_{n\in\Bbb N}a_n
=a.\cr}$$

\noindent Next, suppose that $k\in\Bbb Z$ and

$$\eqalign{d
&\Bsubseteq b\Bcap\pi^kb
=\sup_{m,n\in\Bbb N}(b_m\Bsetminus\sup_{i<m}a_i)
  \Bcap(\pi^kb_n\Bsetminus\sup_{j<n}\pi^ka_j)\cr
&=\sup_{m,n\in\Bbb N}(b_m\Bcap a_m\Bsetminus\sup_{i<m}a_i)
  \Bcap(\pi^kb_n\Bcap a_n\Bsetminus\sup_{j<n}a_j)
=\sup_{n\in\Bbb N}(b_n\Bcap\pi^kb_n\Bsetminus\sup_{i<n}a_i).\cr}$$

\noindent Setting $d_n=d\Bcap b_n\Bcap\pi^kb_n$ for each $n$, we have

\Centerline{$d=\sup_{n\in\Bbb N}d_n=\sup_{n\in\Bbb N}\pi^kd_n=\pi^kd$.}

\noindent As $k$ and $d$ are arbitrary, $b$ is a transversal for
$\pi\restrp\frak A_a$.\ \Qed

\medskip

{\bf (b)} We have only to note that the definition in 382Ab is symmetric
between $\pi$ and $\pi^{-1}$.

\medskip

{\bf (c)}

$$\eqalign{\pi a^*
&=\sup_{n\in\Bbb Z}(\pi^{n+1}a\Bsetminus\sup_{i>n}\pi^{i+1}a)\cr
&=\sup_{n\in\Bbb Z}(\pi^{n+1}a\Bsetminus\sup_{i>n+1}\pi^ia)
=\sup_{n\in\Bbb Z}(\pi^na\Bsetminus\sup_{i>n}\pi^ia)
=a^*.\cr}$$

\noindent Set $b_n=\pi^na\Bsetminus\sup_{i>n}\pi^ia$ for each $n$,
$b=\sup_{n\in\Bbb Z}\pi^{-n}b_n\Bsubseteq a$.   Writing $b^*$ for
$\sup_{n\in\Bbb Z}\pi^nb$, we have
$b^*\Bsupseteq\penalty-100\sup_{n\in\Bbb Z}b_n\penalty-100=a^*$.   Note that
$\pi^{-n}b_n\Bcap\pi^ia=0$ for every $i\ge 1$.   So if $m<n$ in
$\Bbb Z$,

\Centerline{$\pi^mb\Bcap\pi^nb
\Bsubseteq\pi^m(\sup_{i\in\Bbb Z}\pi^{-i}b_i\Bcap\pi^{n-m}a)
=0$.}

\noindent Thus
$\family{i}{\Bbb Z}{\pi^ib}$ is disjoint, and $b$ is a transversal for
$\pi\restrp\frak A_{a^*}$.

Now

\Centerline{$a_*
=\sup_{n\in\Bbb Z}(\pi^na\Bsetminus\sup_{i<n}\pi^ia)
=\sup_{n\in\Bbb Z}(\pi^{-n}a\Bsetminus\sup_{i>n}\pi^{-i}a)$.}

\noindent So $\pi^{-1}a_*=a_*$ and $\pi^{-1}\restrp\frak A_{a_*}$ has a
transversal.   It follows at once that $\pi a_*=a_*$ and (using (b))
that $\pi\restrp\frak A_{a_*}$ has a transversal.
}%end of proof of 382G

\leader{382H}{Lemma} Let $\frak A$ be a Dedekind $\sigma$-complete
Boolean algebra and $\pi\in\Aut\frak A$.   If $\pi$ has a transversal,
it is expressible as the product of at most two exchanging involutions
both belonging to the countably full subgroup of $\frak A$ generated by
$\pi$.

\proof{ Let $a$ be a transversal for $\pi$.   For $n\ge 1$, set
$a_n=a\Bcap\pi^na\Bsetminus\sup_{1\le i<n}\pi^ia$;  set
$a_0=a\Bsetminus\sup_{i\ge 1}\pi^ia$.    Then $\sequencen{a_n}$ is
disjoint and $\sup_{n\in\Bbb N}a_n=a$.   We have $\pi^nb=b$ whenever
$b\Bsubseteq a_n$, while $\langle\pi^ia_0\rangle_{i\ge 1}$ is disjoint,
so $\family{i}{\Bbb Z}{\pi^ia_0}$ is disjoint.
For any $n\ge 1$, $a_n$ is disjoint from $\pi^ia_n$ for $0<i<n$, so
$\ofamily{i}{n}{\pi^ia_n}$ is disjoint.   If $0\le i<m$ and $0\le j<n$
and $i\le j$ and $\pi^ia_m\Bcap\pi^ja_n$ is non-zero, then
$1\le n-j+i\le n$ and

$$\eqalign{a_n\Bcap\pi^{n-j+i}a
&=\pi^{n-j+i}a\Bcap\pi^na_n
=\pi^{n-j}(\pi^ia\Bcap\pi^ja_n)\cr
&\Bsupseteq\pi^{n-j}(\pi^ia_m\Bcap\pi^ja_n)
\ne 0,\cr}$$

\noindent so $i=j$;  in this case $a_m\Bcap a_n\ne 0$ so $m=n$.   If
$0\le i<n$ and $j\in\Bbb Z$ and $b=\pi^ia_n\Bcap\pi^ja_0$, then
$\pi^nb=b$ and $\pi^nb$ is disjoint from $b$, so $b=0$.   This shows
that all the $\pi^ia_n$ for $0\le i<n$, and the $\pi^ja_0$ for
$j\in\Bbb Z$, are disjoint.   Also, because $\pi^na_n=a_n$ for $n\ge 1$,

\Centerline{$\sup_{0\le i<n}\pi^ia_n\Bcup\sup_{j\in\Bbb Z}\pi^ja_0
=\sup_{n\in\Bbb N,j\in\Bbb Z}\pi^ja_n\Bcup\sup_{j\in\Bbb Z}\pi^ja_0
=\sup_{j\in\Bbb Z}\pi^ja
=1$.}

For any $n\ge 1$,

\Centerline{$\langle\pi^{-2j}\pi^ja_n\rangle_{0\le j<n}
=\langle\pi^{-j}a_n\rangle_{0\le j<n}
=\langle\pi^{n-j}a_n\rangle_{0\le j<n}$,}

\Centerline{$\langle\pi^{1-2j}\pi^ja_n\rangle_{0\le j<n}
=\langle\pi^{1-j}a_n\rangle_{0\le j<n}
=\langle\pi^{n+1-j}a_n\rangle_{0\le j<n}$}

\noindent are disjoint and cover $\sup_{0\le j<n}\pi^ja_n$;  while of
course

\Centerline{$\family{j}{\Bbb Z}{\pi^{-2j}\pi^ja_0}
=\family{j}{\Bbb Z}{\pi^{-j}a_0}$,}

\Centerline{$\family{j}{\Bbb Z}{\pi^{1-2j}\pi^ja_0}
=\family{j}{\Bbb Z}{\pi^{1-j}a_0}$}

\noindent are disjoint and cover $\sup_{j\in\Bbb Z}\pi^ja_0$.
So we can define $\phi_1$, $\phi_2\in\Aut\frak A$ by setting

$$\eqalign{\phi_1d&=\pi^{-2j}d\text{ if }j\in\Bbb Z
  \text{ and }d\Bsubseteq\pi^ja_0\cr
&\mskip100mu\text{ or if }0\le j<n
  \text{ and }d\Bsubseteq\pi^ja_n\cr
\phi_2d&=\pi^{1-2j}d\text{ if }j\in\Bbb Z
  \text{ and }d\Bsubseteq\pi^ja_0\cr
&\mskip100mu\text{ or if }0\le j<n
  \text{ and }d\Bsubseteq\pi^ja_n.\cr}$$

\noindent Note that if $n\ge 1$ and $k\in\Bbb Z$ is arbitrary, then
we have $\pi^ka_n=\pi^ja_n$ where $0\le j<n$ and $j\equiv k$ mod $n$,
so if $d\Bsubseteq\pi^ka_n$ then

\Centerline{$\phi_1d=\pi^{-2j}d=\pi^{-2k}d$,
\quad$\phi_2d=\pi^{1-2j}d=\pi^{1-2k}d$}

\noindent because $\pi^nd=d$.   So if $d\Bsubseteq\pi^ja_n$ for any
$n\in\Bbb N$ and $j\in\Bbb Z$, we have
$\phi_1d=\pi^{-2j}d\Bsubseteq\pi^{-j}a_n$ and

\Centerline{$\phi_2\phi_1d=\pi^{1-2(-j)}\pi^{-2j}d=\pi d$.}

\noindent Because $\sup_{n\in\Bbb N,j\in\Bbb Z}\pi^ja_n=1$,
$\phi_2\phi_1=\pi$.   Of course both $\phi_1$ and $\phi_2$ belong to the
countably full subgroup generated by $\pi$.   Next, $\phi_1$ exchanges

$$\eqalign{&\sup_{j\ge 1}\pi^ja_0
  \Bcup\sup_{\Atop{n\ge 2}{0<j\le\lfloor(n-1)/2\rfloor}}\pi^ja_n,\cr
&\mskip80mu
\sup_{j\le-1}\pi^ja_0
\Bcup\sup_{\Atop{n\ge 2}{-\lfloor(n-1)/2\rfloor\le j<0}}\pi^ja_n,\cr}$$

\noindent so is either the identity or an exchanging involution.   In
the same way, $\phi_2$ exchanges

$$\eqalign{&\sup_{j\ge 1}\pi^ja_0
  \Bcup\sup_{\Atop{n\ge 2}{1\le j\le\lfloor n/2\rfloor}}\pi^ja_n,\cr
&\mskip80mu
\sup_{j\le 0}\pi^ja_0
  \Bcup\sup_{\Atop{n\ge 2}{-\lfloor n/2\rfloor<j\le 0}}\pi^ja_n,\cr}$$

\noindent so it too is either the identity or an exchanging involution.
Thus we have a factorization of the desired type.
}%end of proof of 382H

%countably full with sepators
\leader{382I}{Lemma} Let $\frak A$ be a Dedekind $\sigma$-complete
Boolean algebra, and $G$ a countably full subgroup of $\Aut\frak A$ such
that every member of $G$ has a separator.

(a) Every member of $G$ has a support.

(b) Suppose $\pi\in G$ and $n\ge 1$ are such that $\pi^n$ is the
identity.   Then $\pi$ has a transversal.

(c) Let $\pi\in G$, and set $e^*=\inf_{n\ge 1}\supp(\pi^n)$.   Then
$\pi\restrp\frak A_{1\Bsetminus e^*}$ has a transversal.

(d) If $e\in\frak A$ is such that $\pi e=e$ for every $\pi\in G$, then
$\{\pi\restrp\frak A_e:\pi\in G\}$ is a countably full subgroup of
$\Aut\frak A_e$, and $\pi\restrp\frak A_e$ has a separator for every
$\pi\in G$.

\proof{{\bf (a)} 382Ea.

\medskip

{\bf (b)} Induce on $n$.   If $n=1$ then $1$ is a transversal for $\pi$.
For the inductive step to $n>1$, let $a\in\frak A$ be such that
$a\Bcap\pi a=0$ and $\pi b=b$ whenever $b\Bcap\pi^ia=0$ for every
$i\in\Bbb Z$.   Let $\frak B$ be the (finite) subalgebra of $\frak A$
generated by $\{\pi^ia:0\le i<n\}$.   Then $\pi^na=a\in\frak B$, so
$\{b:\pi b\in\frak B\}$ is a subalgebra of $\frak A$ containing $\pi^ia$
whenever $i<n$, and includes $\frak B$;  thus $\pi b\in\frak B$ for
every $b\in\frak B$.   As $\pi$ is injective,
$\pi\restrp\frak B\in\Aut\frak B$.   Let $E$ be the set of atoms of
$\frak B$;  then $\pi\restr E$ is a permutation of $E$.

Let $C\subseteq E$ be an orbit of $\pi$.   Then $\pi(\sup C)=\sup C$,
and $\pi\restrp\frak A_{\sup C}$ has a transversal.   \Prf\ Take
$e\in C$, $k=\#(C)$.
Then $\pi^ie\in C\setminus\{e\}$, so $e\Bcap\pi^ie=0$, whenever
$1\le i<k$.   As $\pi^n$ is the identity, $k$ is a factor of $n$.   If
$k=1$, then $e$ itself is a transversal for
$\pi\restrp\frak A_{\sup C}=\pi\restrp\frak A_e$.   If $k>1$, define
$\phi\in\Aut\frak A$ by setting
$\phi d=\pi^k(e\Bcap d)\Bcup(d\Bsetminus e)$ for every
$d\in\frak A$.   Then $\phi\in G$, because $G$ is countably full, and
$\phi^{n/k}$ is the identity.   By the inductive hypothesis,
$\phi$ has a transversal $c\in\frak A$.   There is some $m\in\Bbb Z$
such that $e'=e\Bcap\phi^mc\ne 0$.   Now

\Centerline{$\sup_{i\in\Bbb Z}\pi^{ki}e'=\sup_{i\in\Bbb Z}\phi^ie'
=\sup_{i\in\Bbb Z}(e\Bcap\phi^{m+i}c)=e\Bcap\sup_{i\in\Bbb Z}\phi^ic
=e$,}

\noindent so

\Centerline{$\sup_{j\in\Bbb Z}\pi^je'
=\sup_{0\le j<k}\pi^j(\sup_{i\in\Bbb Z}\pi^{ki}e')
=\sup_{0\le j<k}\pi^je=\sup C$.}

\noindent Also, if $0\le j<k$ and $i\in\Bbb Z$ and

\Centerline{$0\ne d\Bsubseteq e'\Bcap\pi^{ki+j}e'
\Bsubseteq e\Bcap\pi^{ki+j}e=e\Bcap\pi^je$,}

\noindent we must have $j=0$ and $d\Bsubseteq e'\Bcap\phi^ie'$, in which
case $\pi^{ki+j}d=\phi^id=d$.   So $e'$ is a transversal for
$\pi\restrp\frak A_{\sup C}$.\ \Qed

Let $\Cal C$ be the set of orbits of $\pi\restr E$, and for $C\in\Cal C$
let $c_C$ be a transversal for $\pi\restrp\frak A_{\sup C}$.   Then
$\sup_{C\in\Cal C}c_C$ is a transversal for $\pi$ (382Ga).   Thus the
induction proceeds.

\medskip

{\bf (c)} Set $e_0=1\Bsetminus\supp\pi$,
$e_n=\inf_{1\le i\le n}\supp(\pi^i)\Bsetminus\supp(\pi^{n+1})$ for
$n\ge 1$.   Then $\sequencen{e_n}$ is a partition of unity in
$\frak A_{1\Bsetminus e^*}$, and $\pi^{n+1}a=a$ whenever
$a\Bsubseteq e_n$.   Also $\pi e_n=e_n$ for each $n$, by 381Eg.   By
(b), $\pi\restrp\frak A_{e_n}$ has a transversal for every $n$;  so
$\pi\restrp\frak A_{1\Bsetminus e^*}$ has a transversal (382Ga again).

\medskip

{\bf (d)(i)} Write $G_e$ for $\{\pi\restrp\frak A_e:\pi\in G\}$.   If
$\familyiI{a_i}$ is a countable partition of unity in $\frak A_e$,
$\familyiI{\pi_i}$ a family in $G$, and $\phi\in\Aut\frak A_e$ is such
that $\phi d=\pi_id$ whenever $i\in I$ and $d\Bsubseteq a_i$, set
$J=I\cup\{\infty\}$ for some object $\infty\notin I$,
$a_{\infty}=1\Bsetminus e$ and $\pi_{\infty}$ the identity in
$\Aut\frak A$;  then we have a $\tilde\phi\in\Aut\frak A$ defined by
setting $\tilde\phi d=\phi(d\Bcap e)\Bcup(d\Bsetminus e)$ for every
$d\in\frak A$, and $\family{i}{J}{a_i}$, $\family{i}{J}{\pi_i}$ witness
that $\tilde\phi\in G$, so $\phi=\tilde\phi\restrp\frak A_e$ belongs to
$G_e$.   As $\familyiI{a_i}$ and $\familyiI{\pi_i}$ are arbitrary, $G_e$
is countably full.

\medskip

\quad{\bf (ii)} If $\pi\in G$, let $a$ be a separator for $\pi$, and
consider $a'=a\Bcap e$.   Then $a'\Bcap\pi a'=0$ and
$\sup_{k\in\Bbb Z}\pi^ka'=\sup_{k\in\Bbb Z}\pi^ka\Bcap e=e$, so $a'$ is
a separator for $\pi\restrp\frak A_e$.
}%end of proof of 382I

\leader{382J}{Lemma} Let $\frak A$ be a Dedekind $\sigma$-complete
Boolean algebra, $G$ a countably full subgroup of $\Aut\frak A$ such that
every member of $G$ has a separator, and $\pi\in G$ an aperiodic
automorphism.   Then there is a non-increasing
sequence $\sequencen{e_n}$ in $\frak A$ such that $e_0=1$ and

\inset{(i) $\pi$ is doubly recurrent on $e_n$, and in fact
$\sup_{i\ge 1}\pi^ie_n=\sup_{i\ge 1}\pi^{-i}e_n=1$,

(ii) $e_{n+1}$, $\pi_{e_n}e_{n+1}$ and $\pi_{e_n}^2e_{n+1}$ are
disjoint}

\noindent for every $n\in\Bbb N$, where $\pi_{e_n}\in\Aut\frak A_{e_n}$
is the automorphism induced by $\pi$\cmmnt{ (381M)}.

\proof{ Construct $\sequencen{a_n}$ inductively, as follows.   Start
with $a_0=1$.   Given that
$\sup_{i\ge 1}\pi^ia_n=\sup_{i\ge 1}\pi^{-i}a_n\penalty-100=1$,
then of course $\pi$
is doubly recurrent on $a_n$ (381L).   Now there is an
$a_{n+1}\Bsubseteq a_n$ such that $a_{n+1}\Bcap\pi_{a_n}a_{n+1}=0$ and
$a_{n+1}\Bcup\pi_{a_n}a_{n+1}\Bcup\pi_{a_n}^2a_{n+1}=a_n$.   \Prf\ We
have a $\tilde\pi_{a_n}\in\Aut\frak A$ defined by setting
$\tilde\pi_{a_n} d=\pi_{a_n}d$ for $d\Bsubseteq a_n$,
$\tilde\pi_{a_n}d=d$ for
$d\Bsubseteq 1\Bsetminus a_n$.   Because $\pi$ is aperiodic, so is
$\pi_{a_n}$ (381Ng);  in particular, the support of $\pi_{a_n}$ is $a_n$
and this must also be the support of $\tilde\pi_{a_n}$.   Because $G$ is
countably full, $\tilde\pi_{a_n}\in G$ (381Ni), so $\tilde\pi_{a_n}$ has
a separator.   By 382D, there is an $a_{n+1}\in\frak A$ such that
$a_{n+1}\Bcap\tilde\pi_{a_n}a_{n+1}=0$ and
$a_{n+1}\Bcup\tilde\pi_{a_n}a_{n+1}\Bcup\tilde\pi_{a_n}^2a_{n+1}$
supports $\tilde\pi_{a_n}$, that is,

\Centerline{$a_n
=a_{n+1}\Bcup\tilde\pi_{a_n}a_{n+1}\Bcup\tilde\pi_{a_n}^2a_{n+1}
=a_{n+1}\Bcup\pi_{a_n}a_{n+1}\Bcup\pi_{a_n}^2a_{n+1}$.  \Qed}

\noindent Now

$$\eqalignno{\sup_{i\ge 1}\pi^ia_{n+1}
&=\sup_{i\ge 1}\pi^i(\sup_{j\ge 0}\pi^ja_{n+1})
\Bsupseteq\sup_{i\ge 1}\pi^i(\sup_{j\ge 0}\pi^j_{a_n}a_{n+1})\cr
\displaycause{381Nb}
&=\sup_{i\ge 1}\pi^ia_n
=1.\cr}$$

\noindent Similarly, because we can identify $\pi_{a_n}^{-1}$ with
$(\pi^{-1})_{a_n}$ (381Na), and

\Centerline{$a_{n+1}\Bcup\pi_{a_n}^{-1}a_{n+1}\Bcup\pi_{a_n}^{-2}a_{n+1}
=\pi_{a_n}^{-2}
 (a_{n+1}\Bcup\tilde\pi_{a_n}a_{n+1}\Bcup\tilde\pi_{a_n}^2a_{n+1})
=a_n$,}

\noindent we have

$$\eqalignno{\sup_{i\ge 1}\pi^{-i}ia_{n+1}
&=\sup_{i\ge 1}\pi^{-i}(\sup_{j\ge 0}\pi^{-j}a_{n+1})\cr
&\Bsupseteq\sup_{i\ge 1}\pi^{-i}(\sup_{j\ge 0}\pi^{-j}_{a_n}a_{n+1})
=\sup_{i\ge 1}\pi^{-i}a_n
=1,\cr}$$

\noindent and the induction continues.

At the end of the induction, set $e_n=a_{2n}$ for every $n$.   Then, for
each $n$, we have

\Centerline{$0=a_{2n+1}\Bcap\pi_{e_n}a_{2n+1}
=e_{n+1}\Bcap\pi_{a_{2n+1}}e_{n+1}$.}

\noindent Since we can identify $\pi_{a_{2n+1}}$ with
$(\pi_{e_n})_{a_{2n+1}}$ (381Ne), we can apply 381Nh to $\pi_{e_n}$ to
see that $e_{n+1}$, $\pi_{e_n}e_{n+1}$ and $\pi^2_{e_n}e_{n+1}$ are all
disjoint.
}%end of proof of 382J

\leader{382K}{Lemma} Let $\frak A$ be a Dedekind $\sigma$-complete
Boolean algebra.
Suppose that we have an aperiodic $\pi\in\Aut\frak A$ and a
non-increasing
sequence $\sequencen{e_n}$ in $\frak A$ such that $e_0=1$ and

\Centerline{$\sup_{i\ge 1}\pi^ie_n=\sup_{i\ge 1}\pi^{-i}e_n=1$,
\quad
$e_{n+1}$, $\pi_{e_n}(e_{n+1})$ and $\pi_{e_n}^2(e_{n+1})$ are disjoint}

\noindent for every $n\in\Bbb N$, writing
$\pi_{e_n}\in\Aut\frak A_{e_n}$ for the induced automorphism.   Let $G$
be the countably full subgroup of $\Aut\frak A$ generated by $\pi$.
Then there is a $\phi\in G$ such that $\phi$ is either the identity
or an exchanging involution and
$\inf_{n\ge 1}\supp(\pi\phi)^n=0$.

\proof{{\bf (a)} We need to check that every member of $G$ has a support.
\Prf\ If $\phi\in G$, there is a partition $\family{n}{\Bbb Z}{a_n}$ of
unity such that $\phi a=\pi^na$ whenever $n\in\Bbb Z$ and $a\Bsubseteq a_n$
(381Ib).  If $a\Bsubseteq a_0$, then $\phi a=a$, so $1\Bsetminus a_0$
supports $\phi$.   On the other hand, if $a\Bsetminus a_0\ne 0$, there is
an $n\ne 0$ such that $a\Bcap a_n\ne 0$.   As $\supp \pi^n=1$, there is a
non-zero $d\Bsubseteq a\Bcap a_n$ such that $0=d\Bcap\pi^nd=d\Bcap\phi d$.
Thus $1\Bsetminus a_0=\sup\{d:d\Bcap\phi d=0\}$ is the support of $\phi$
(381Ei).\ \Qed

\medskip

{\bf (b)} For each $n\in\Bbb N$, write $\pi_n$ for $\pi_{e_n}$
and $\tilde\pi_n\in G$ for the corresponding automorphism of $\frak A$,
as in 381Ni.   Set

\Centerline{$u'_n=\pi_n^{-1}e_{n+1}$,
\quad$u''_n=\pi_ne_{n+1}$.}

\noindent Then all the $u'_n$, $u''_n$ are disjoint.   \Prf\

\Centerline{$u'_n\Bcap u''_n
=\pi_n^{-1}(e_{n+1}\Bcap\pi_n^2e_{n+1})=0$}

\noindent for each $n$.   And if $m<n$, then
$u'_n\Bcup u''_n\Bsubseteq e_n\Bsubseteq e_{m+1}$ is disjoint from

\Centerline{$u'_m\Bcup u''_m
\Bsubseteq\pi_m^{-1}(e_{m+1})\Bcup\pi_m(e_{m+1})$.\ \Qed}

\medskip

{\bf (c)} By 381C, there is an automorphism $\phi_1\in\Aut\frak A$
defined by setting

$$\eqalign{\phi_1d
&=\pi_n\pi_{n+1}^{-1}\pi_nd=\tilde\pi_n\tilde\pi_{n+1}^{-1}\tilde\pi_nd
\text{ if }n\in\Bbb N,\,d\Bsubseteq u'_n,\cr
&=\pi_n^{-1}\pi_{n+1}\pi_n^{-1}d
=\tilde\pi_n^{-1}\tilde\pi_{n+1}\tilde\pi_n^{-1}d
\text{ if }n\in\Bbb N,\,d\Bsubseteq u''_n,\cr
&=d\text{ if }d\Bcap\sup_{n\in\Bbb N}(u'_n\Bcup u''_n)=0;\cr}$$

\noindent $\phi_1\in G$ and $\phi_1^2$ is the identity and $\phi_1$
exchanges $\sup_{n\in\Bbb N}u'_n$ with $\sup_{n\in\Bbb N}u''_n$, so is
either the identity or an exchanging involution.   Set
$c_0=\inf_{k\ge 1}\supp(\pi\phi_1)^k$ and
$c_1=\sup_{i\in\Bbb Z}\pi^ic_0$, so that $\pi c_1=c_1$
and $\phi_1c_1=c_1$ (381J).

\medskip

{\bf (d)} For $l\ge 1$, set

\Centerline{$v'_l=\pi^{-l}c_0\Bsetminus\sup_{-l<i\le l}\pi^ic_0$,
\quad$v''_l=\pi^lc_0\Bsetminus\sup_{-l\le i<l}\pi^ic_0$.}

\noindent Then $v'_k$, $v''_k$, $v'_l$ and $v''_l$ are disjoint whenever
$1\le k<l$.   For $j$, $l\ge 1$, set

\Centerline{$d'_{lj}
=v'_l\Bcap\pi^{-j}v''_l\Bsetminus\sup_{1\le i<j}\pi^{-i}v''_l$,}

\Centerline{$d''_{lj}
=v''_l\Bcap\pi^jv'_l\Bsetminus\sup_{1\le i<j}\pi^iv'_l$,}

\Centerline{$d_{lj}=d'_{lj}\Bcap\pi^{-j}d''_{lj}$;}

\noindent now define $\phi_2\in\Aut\frak A$ by setting

$$\eqalign{\phi_2d
&=\pi^jd\text{ if }d\Bsubseteq d_{lj}\text{ for some }j,\,l\ge 1,\cr
&=\pi^{-j}d\text{ if }d\Bsubseteq\pi^jd_{lj}\text{ for some }j,
  \,l\ge 1,\cr
&=d\text{ if }d\Bcap\sup_{j,l\ge 1}(d_{lj}\Bcup\pi^jd_{lj})=0,\cr}$$

\noindent so that $\phi_2\in G$, $\phi_2^2$ is the identity and

\Centerline{$\supp\phi_2=\sup_{l,j\ge 1}d_{lj}\Bcup\pi^jd_{lj}
\Bsubseteq\sup_{l\ge 1}v'_l\Bcup v''_l\Bsubseteq c_1$.}

\noindent As $\phi_2$ exchanges
$\sup_{j,l\ge 1}d_{lj}\Bsubseteq\sup_{j,l\ge 1}d'_{lj}$ with
$\sup_{j,l\ge 1}\pi^jd_{lj}\Bsubseteq\sup_{j,l\ge 1}d''_{lj}$, it too is
either trivial or an exchanging involution.

\wheader{382K}{4}{2}{2}{36pt}

{\bf (e)} Define $\phi\in\Aut\frak A$ by setting

$$\eqalign{\phi d
&=\phi_1d\text{ if }d\Bsubseteq 1\Bsetminus c_1,\cr
&=\phi_2d\text{ if }d\Bsubseteq c_1.\cr}$$

\noindent It is easy to check that $\phi$ is either the identity or an
exchanging involution.   Set $c_2=\inf_{n\ge 1}\supp(\pi\phi)^n$.

\medskip

{\bf (f)} I wish to show that $c_2=0$.   The rest of the argument does not
strictly speaking require the Stone representation (382Yb),
but I think that most readers will find it easier to follow when expressed
in terms of the Stone space $Z$ of $\frak A$.   Let
$f$, $g_1$, $g_2$ and $g$ be the autohomeomorphisms of $Z$
corresponding to $\pi$, $\phi_1$, $\phi_2$ and $\phi$;  write
$\widehat{a}\subseteq Z$ for the open-and-closed set corresponding to
$a\in\frak A$.   For each $n\in\Bbb N$, let
$f_n:\widehat{e_n}\to\widehat{e_n}$ be the autohomeomorphism
corresponding to $\pi_{e_n}$.   Since

\inset{$\supp\pi^k=1$ for every $k\ge 1$,

$\sup_{i\ge k}\pi^ie_n=\sup_{i\ge k}\pi^{-i}e_n=1$ for every
$n\in\Bbb N$, $k\in\Bbb Z$ (381L),

$c_0=\inf_{k\ge 1}\supp(\pi\phi_1)^k$,

$c_1=\sup_{i\in\Bbb Z}\pi^ic_0$,

$\supp\phi_2\Bsetminus\sup_{l\ge 1}(v'_l\Bcup v''_l)=0$,

$c_2=\inf_{k\ge 1}\supp(\pi\phi)^k$,}

\noindent the sets

\inset{$\{z:f^k(z)=z\}$, for $k\ge 1$,

$Z\setminus\bigcup_{i\ge k}f^{-i}[\widehat{e_n}]$, for
$n\in\Bbb N$ and $k\in\Bbb Z$,

$Z\setminus\bigcup_{i\le k}f^{-i}[\widehat{e_n}]$, for $n\in\Bbb N$ and
$k\in\Bbb Z$,

$\widehat{c_0}\symmdiff\bigcap_{k\ge 1}\{z:(g_1f)^k(z)\ne z\}$,

$\widehat{c_1}\symmdiff\bigcup_{i\in\Bbb Z}f^{-i}[\widehat{c_0}]$,

$\widehat{\supp\phi_2}
  \setminus\bigcup_{l\ge 1}(\widehat{v'_l}\cup\widehat{v''_l})$,

$\widehat{c_2}\symmdiff\{z:(gf)^k(x)\ne x$ for every $k\ge 1\}$,}

\noindent as well as the sets

\inset{$\{z:g_1(z)\notin\{f^i(z):i\in\Bbb Z\}\}$,

$\{z:g_2(z)\notin\{f^i(z):i\in\Bbb Z\}\}$}

\noindent are all meager (using 381Qb), and their union $Y$ is meager.
Set $Y'=\bigcup_{i\in\Bbb Z}f^{-i}[Y]$;  then $Y'$ also is meager, and
$X=Z\setminus Y'$ is comeager, therefore dense, by Baire's theorem
(3A3G).   Of course $f^i(x)\in X$ whenever $x\in X$ and $i\in\Bbb Z$.

\medskip

{\bf (g)} Fix $x\in X\cap\widehat{c_1}$ for the time being.
Because $f^k(x)\ne x$ for any $k\ge 1$, the map
$i\mapsto f^i(x):\Bbb Z\to X$ is injective.   Because
$g_k(f^i(z))\in\{f^{i+j}(z):j\in\Bbb Z\}$ for every $i\in\Bbb Z$ and
both $k\in\{1,2\}$, we can define $g_1^x$, $g_2^x:\Bbb Z\to\Bbb Z$ by
saying that $g_k^x(i)=j$ if $g_k(f^i(x))=f^j(x)$.   Similarly, $f$ is
represented on $\{f^i(x):i\in\Bbb Z\}$ by $s$, where $s(i)=i+1$ for
every $i\in\Bbb Z$.

\medskip

\quad{\bf (i)} For $n\in\Bbb N$, set

\Centerline{$E_n=\{i:i\in\Bbb Z$, $f^i(x)\in\widehat{e_n}\}$,}

\Centerline{$U'_n=\{i:f^i(x)\in\widehat{u'_n}\}$,
\quad$U''_n=\{i:f^i(x)\in\widehat{u''_n}\}$.}

\noindent Because $x\in\bigcup_{i\ge k}f^{-i}[\widehat{e_n}]
\cap\bigcup_{i\le k}f^{-i}[\widehat{e_n}]$ for every $k$, $E_n$ is
unbounded above and
below.   If $i\in E_n$, then $f_n(f^i(x))=f^{k+i}(x)$ where
$k\ge 1$ is the first such that $f^{k+i}(x)\in\widehat{e_n}$ (381Qe),
that is, such that $k+i\in E_n$.   Turning this round,
$f_n^{-1}(f^i(x))=f^j(x)$ where $j$ is the greatest member of $E_n$ less
than $i$.   In particular, $i\in U'_n$ iff $i$ is the next point of
$E_n$ above a point of $E_{n+1}$, and $i\in U''_n$ iff $i$ is the next
point of $E_n$ below a point of $E_{n+1}$.   If $i\in U'_n$, then
$f_n^{-1}f^i(x)=f^j(x)$ where $j\in E_{n+1}$ is the next point of $E_n$
below $i$, and
$f_{n+1}f_n^{-1}f^i(x)=f^k(x)$ where $k$ is the next point of $E_{n+1}$
above $j$.   Since $g_1$ must agree with $f_n^{-1}f_{n+1}f_n^{-1}$ on
$\widehat{u'_n}$ (381Qa),
$g_1f^i(x)=f_n^{-1}f_{n+1}f_n^{-1}f_i(x)=f^l(x)$ where $l$ is the next
point of $E_n$ below $f^k(x)$.    This means that $g_1^x$ exchanges
pairs $i<l$ exactly when $i$, $l\in E_n$ are the first and last
points in $E_n\cap\ooint{j,k}$ where $j$, $k$ are successive points of
$E_{n+1}$.   In this case, there is no point of $E_{n+1}$ in the
interval $[i,l]$.   Accordingly, if $i'<l'$ and $g_1^x$ exchanges
$i'$ and $l'$ and either $i'$ or $l'$ is in $\ooint{i,l}$,
we must have $i'$, $l'\in E_m$ for some $m<n$;  and as the interval
$[i',l']$ cannot meet $E_{m+1}\supseteq E_n$, it is included in
$\ooint{i,l}$.   Thus $g_1^x$ fixes $\ooint{i,l}$ in the sense that if
$i<i'<l$ then $g_1^x(i')=l'$ for some $l'\in\ooint{i,l}$.
It follows that $g_1^xs$ fixes $\coint{i,l}$.   In this case, of course,
every point of $\coint{i,l}$ must be fixed by some power of
$g_1^xs$.

The following diagram attempts to show how $g_1^x$ links pairs of
integers.   The points of $E_n$, as $n$ increases, are shown as
progressively multiplied circles.

\leaveitout{
\ifdim\pagewidth=370pt
\def\Caption{\hskip45pt Pairs of points exchanged by $g_1^x$}
\hskip-45pt\vbox{\picture{mt382h1}{37pt}}
\else\ifdim\pagewidth>467pt
\def\Caption{\hskip0pt Pairs of points exchanged by $g_1^x$}
\hskip-13pt\vbox{\picture{mt382h1}{45pt}}
\else
\def\Caption{\hskip45pt Pairs of points exchanged by $g_1^x$}
\hskip-45pt\vbox{\picture{mt382h1}{37pt}}
\fi\fi
}

\def\Caption{Pairs of points exchanged by $g_1^x$}
\picture{mt382h1}{37pt}

\noindent Note that because $e_{n+1}$, $\phi_{e_n}e_{n+1}$ and
$\pi_{e_n}^2e_{n+1}$ are always disjoint, there are always at least two
points of $E_n$ between any two successive points of $E_{n+1}$.

\medskip

\quad{\bf (ii)} Set $C_0=\{i:f^i(x)\in\widehat{c_0}\}$.   Then

\Centerline{$C_0=\Bbb Z\setminus\bigcup\{\coint{i,l}:i<l=g_1^x(i)\}$.}

\noindent\Prf\ Because $X$ does not meet
$\widehat{c_0}\Bsymmdiff\bigcap_{k\ge 1}\{z:(g_1f)^k(z)\ne z\}$,

\Centerline{$C_0=\{i:(g_1f)^kf^i(x)\ne f^i(x)$ for every $k\ge 1\}
=\{i:(g_1^xs)^k(i)\ne i$ for every $k\ge 1\}$.}

\noindent If $i<l=g_1^x(i)$ then (i) tells us that every point of
$\coint{i,l}$ is fixed by some power of $g_1^xs$ and cannot belong to
$C_0$.   Conversely, if $j\in\Bbb Z$ does not belong to any such interval
$\coint{i,l}$, then $g_1^x(i)>j$ for every $i>j$, so $g_1^xs(i)>j$ for
every $i\ge j$ and $j\notin C_0$.\ \Qed

Because $X$ does not meet
$\widehat{c_1}\setminus\bigcup_{i\in\Bbb Z}f^{-i}[\widehat{c_0}]$,
$C_0$ is not empty.   Now $C_0$ has no greatest member.   \Prf\ Let
$j_0\in C_0$.   Then $j_0\notin\coint{i,l}$ for any pair $i$, $l$
exchanged by $g_1^x$.   If $j_0+1\in C_0$ we can stop.   Otherwise,
there are $i_0$, $l_0$ exchanged by $g_1^x$ such that
$i_0\le j_0+1<l_0$.   \Quer\ If
$l_0\notin C_0$ there are $i_1$, $l_1$ exchanged by $g_1^x$ such that
$i_1\le l_0<l_1$.   But in this case $i_1\le j_0<l_1$.\ \BanG\  Thus
$j_0<l_0\in C_0$ and $j_0$ cannot be the greatest member of $C_0$.\ \Qed

Similarly, $C_0$ has no least member.   \Prf\ If $j_0\in C_0$ but
$j_0-1\notin C_0$, take $i_0$, $l_0$ exchanged by $g_1^x$ such that
$i_0\le j_0-1<l_0$.   \Quer\ If $i_0-1\notin C_0$, take $i_1$, $l_1$
exchanged by $g_1^x$ such that $i_1\le i_0-1<l_1$;  then
$i_1\le j_0= l_0<l_1$.\ \BanG\  So $i_0-1$ is a member of $C_0$ less
than $j_0$.\ \Qed

Thus $C_0$ is unbounded above and below.

\medskip

\quad{\bf (iii)} For $l\ge 1$,

\Centerline{$\widehat{v'_l}
=f^l[\widehat{c_0}]\setminus\bigcup_{-l\le j<l}f^j[\widehat{c_0}]$,
\quad$\widehat{v''_l}
=f^{-l}[\widehat{c_0}]
  \setminus\bigcup_{-l<j\le l}f^j[\widehat{c_0}]$;}

\noindent so setting

\Centerline{$V'_l=\{i:f^i(x)\in\widehat{v'_l}\}$,
\quad$V''_l=\{i:f^i(x)\in\widehat{v''_l}\}$,}

\noindent we see that

\Centerline{$V'_l=\{i:i-l\in C_0$, $i+j\notin C_0$ if $-l<j\le l\}
=\{i+l:i\in C_0$, $C_0\cap\ocint{i,i+2l}=\emptyset\}$,}

\Centerline{$V''_l=\{i:i+l\in C_0$, $i+j\notin C_0$ if $-l\le j<l\}
=\{i-l:i\in C_0$, $C_0\cap\coint{i-2l,i}=\emptyset\}$;}

\noindent that is to say, if $j$, $k$ are successive members of $C_0$,
and $j+l<k-l$, then $j+l\in V'_l$ and $k-l\in V''_l$.   Looking at this
from the other direction, if $j$ and $k$ are successive members of
$C_0$, and $l_0=\lfloor\Bover{k-j-1}2\rfloor$, then if $1\le l\le l_0$
we have exactly one $i'\in V'_l\cap[j,k]$ and exactly one
$i''\in V''_l\cap[j,k]$ and $i'<i''$, while if $l>l_0$ then neither
$V'_l$ nor $V''_l$ meets $[j,k]$.

\medskip

\quad{\bf (iv)} Now the point is that every
$V'_l$ is unbounded above.   \Prf\ Because there are at least two
points of $E_n$ between any two points of $E_{n+1}$, successive points
of $E_n$ always differ by at least $3^n$, for every $n$.
Take $n$ such that $3^n\ge 2l+1$.   For any $i_0\in\Bbb Z$, there are an
$i_1\in C_0$ such that $i_1\ge i_0$, and a $j\in E_{n+1}$ such that
$j\ge i_1$;  let $k$ be the next point of $E_{n+1}$ above $j$.   Then we
have points $j'$, $k'$ of $E_n\cap\ooint{j,k}$ such that $C_0$ is
disjoint from $\coint{j',k'}$.  So if we take
$i=\max(C_0\Bcap\ooint{-\infty,j'})$ and
$i'=\min(C_0\Bcap\coint{j',\infty})$, $i'-i\ge k'-j'\ge 2l+1$ and
$i+l\in V'_l$, while $i+l\ge i\ge i_1\ge i_0$.   As $i_0$ is arbitrary,
$V'_l$ is unbounded above.\ \QeD\  Similarly, turning the argument
upside down, $V''_l$ is unbounded below.

\medskip

\quad{\bf (v)} Next consider

$$\eqalign{D'_{lj}=\{i:f^i(x)\in\widehat{d'_{lj}}\}
&=V'_l\Bcap(V''_l+j)\setminus\bigcup_{1\le i<j}V''_l+i\cr
&=\{i:i\in V'_l,\,i-j=\max(V''_l\cap\ooint{-\infty,i}\},\cr
D''_{lj}=\{i:f^i(x)\in\widehat{d'_{lj}}\}
&=V''_l\Bcap(V'_l+j)\setminus\bigcup_{1\le i<j}V'_l+i\cr
&=\{i:i\in V''_l,\,i+j=\min(V'_l\cap\ooint{i,\infty}\},\cr
D_{lj}=\{i:f^i(x)\in\widehat{d_{lj}}\}
&=D'_{lj}\cap(D''_{lj}+j).\cr}$$

\noindent Since $\phi_2$ agrees with $\pi^j$ on $\frak A_{d_{lj}}$,
$g_2$ agrees with $f^j$ on $\widehat{\pi^jd_{lj}}$, and
$g_2^x(i)=i+j$ whenever $f^i(x)\in f^{-j}[\widehat{d_{lj}}]$,
that is, whenever $i+j\in D_{lj}$.   This means that $g_2^x$ exchanges
pairs $i''<i'$ exactly when, for some $l$, $i''$ is the greatest member
of $V''_l$ less than $i'$ and $i'$ is the least member of $V'_l$ greater
than $i''$.   Since $X$ does not meet $\widehat{\supp\phi_2}
\setminus\bigcup_{l\ge 1}(\widehat{v'_l}\cup\widehat{v''_l})$,
$g_2^x$ does not move any other $i$.

But, starting from any $l\ge 1$ and $i'\in V'_l$,
let $i''$ be the greatest element of $V''_l$ less than $i'$.   Then
$i'-l$ and $i''+l$ belong to $C_0$, and if $k$, $k'$ are any successive
members of $C_0$ such that $i''<k<k'<i'$ then there is no member of
$V''_l$ in $[k,k']$ and therefore no member of $V'_l$ in $[k,k']$.   So
$i'$ is the least member of $V'_l$ greater than $i''$, and
$g_2^x(i')=i''$.   Similarly, every member of every $V''_l$ is moved by
$g_2^x$.

At the same time we see that if $i''\in V''_l$ and
$i'\in V'_l$ are exchanged by $g_2^x$, and $m>l$, then there can be no
interval of $C_0$ of length $2m+1$ or greater between $i''$ and $i'$, so
there is no point of $V''_m\cup V'_m$ in $[i'',i']$.   For the same
reason, if $m<l$ then no pair of points in $V''_m\cup V'_m$ exchanged by
$g_2^x$ can bracket either $i''$ or $i'$.
So $g_2^x$ leaves the interval $[i'',i']$ invariant.   Accordingly
$g_2s$ leaves $\coint{i'',i'}$ invariant.

The next diagram attempts to illustrate $g_2^x$.   Members of $C_0$ are
shown as multiple circles\footnote{I have made no attempt to arrange
these in a configuration compatible with the process by which $C_0$ was
constructed;  the diagram aims only to show how the links would be
formed from a particular set.}.

\leaveitout{
\ifdim\pagewidth=370pt
\vskip15pt
\def\Caption{\hskip60pt Pairs of points exchanged by $g_2^x$}
\hskip-43pt\vbox{\picture{mt382h2}{33pt}}
\else\ifdim\pagewidth>467pt
\vskip20pt
\def\Caption{\hskip60pt Pairs of points exchanged by $g_2^x$}
\hskip-12pt\vbox{\picture{mt382h2}{40pt}}
\else
\vskip15pt
\def\Caption{\hskip60pt Pairs of points exchanged by $g_2^x$}
\hskip-43pt\vbox{\picture{mt382h2}{33pt}}
\fi\fi
}

\vskip20pt

\def\Caption{Pairs of points exchanged by $g_2^x$}
\picture{mt382h2}{33pt}

At this point observe that $0$ belongs to some $g_2^xs$-invariant
interval.   \Prf\ Let $k$, $k'$ be successive members of $C_0$
such that $k\le 0<k'$.   Take $l$ such that $k'-k\le 2l$.   Let $i'$ be
the least member of $V'_l$ greater than $0$, and $i''$ the greatest
member of $V''_l$ less than $0$;  since neither $V'_l$ nor $V''_l$ meets
$[k,k']$, $i''$ and $i'$ are exchanged by $g_2^x$, while
$0\in\coint{i'',i'}$.\ \QeD\
This means that there is a $k\ge 1$ such that $(g_2^xs)^k(0)=0$, that
is, $(g_2f)^k(x)=x$.

\medskip

\quad{\bf (vi)} We know that $g$ agrees with $g_2$ on
$\widehat{\phi_2c_1}=\widehat{c_1}$.   Since $x\in\widehat{c_1}$ and
$f^{-1}[\widehat{c_1}]=\widehat{c_1}$, $(gf)^k(x)=x$.   Because $X$ does
not meet
$\widehat{c_2}\symmdiff\{z:(gf)^k(x)\ne x$ for every $k\ge 1\}$,
$x\notin\widehat{c_2}$.

This is true for every $x\in X\cap\widehat{c_1}$.   Since $X$ is dense
in $Z$, $\widehat{c_1}\cap\widehat{c_2}$ is empty, that is,
$c_1\Bcap c_2=0$.

\medskip

{\bf (h)} Since $\pi\phi$ agrees with $\pi\phi_1$ on
$\frak A_{1\Bsetminus c_1}$, and $c_1=\pi\phi c_1$,
$\supp(\pi\phi)^k\Bsetminus c_1=\supp(\pi\phi_1)^k\Bsetminus c_1$ for
every $k$, and

\Centerline{$c_2\Bsetminus c_1
=\inf_{k\ge 1}\supp(\pi\phi_1)^k\Bsetminus c_1
\Bsubseteq\inf_{k\ge 1}\supp(\pi\phi_1)^k\Bsetminus c_0=0$.}

\noindent Putting this together with (g), we see that $c_2=0$, as
required.
}%end of proof of 382K

\leader{382L}{Lemma} Let $\frak A$ be a Dedekind $\sigma$-complete
Boolean algebra, and $G$ a countably full subgroup of $\Aut\frak A$ such
that every member of $G$ has a separator.   If $\pi\in G$, there is a
$\phi\in G$ such that $\phi$ is either the identity or an exchanging
involution and $\pi\phi$ has a transversal.

\proof{{\bf (a)} We may suppose that $G$ is the countably full subgroup
of $\Aut\frak A$ generated by $\pi$.   $\pi^n$ has a support for every
$n\ge 1$ (382Ia);  set $e=\inf_{n\ge 1}\supp\pi^n$, so that $\pi e=e$ and
$\pi\restrp\frak A_{1\Bsetminus e}$ has a transversal (382Ic), while
$\pi\restrp\frak A_e$ is aperiodic (381H).   By 381J, $\psi e=e$ for
every $\psi\in G$;  by 382Id,
$G_e=\{\psi\restr\frak A_e:\psi\in G\}$ is a countably full subgroup of
$\Aut\frak A_e$ and every member of $G_e$ has a separator.

\medskip

{\bf (b)} Applying 382J to $\pi\restr\frak A_e$, we can find
$\langle e_n\rangle_{n\ge 1}$ such that $e_0=e$, $\sequencen{e_n}$ is
non-increasing, $\sup_{i\ge 1}\pi^ie_n=\sup_{i\ge 1}\pi^{-i}e_n=e$ for
every $n$, and $e_{n+1}$, $\pi_{e_n}e_{n+1}$ and $\pi_{e_n}^2e_{n+1}$
are disjoint for every $n$.   (By 381Ne or otherwise, we can compute
$\pi_{e_n}$ either in $\Aut\frak A$ or in $\Aut\frak A_e$.   Note that
$\pi_e=\pi\restrp\frak A_e$, by 381Nf or otherwise.)
Now 382K tells us that there is a $\phi\in G_e$ such that $\phi$ is
either the identity or an exchanging involution,
and $\inf_{n\ge 1}\supp(\pi_e\phi)^n=0$.

\medskip

{\bf (c)} Take $\tilde\phi\in\Aut\frak A$ to agree with $\phi$ on
$\frak A_e$ and with the identity on
$\frak A_{1\Bsetminus e}$, so that $\tilde\phi$ is
either the identity or an exchanging involution.
Now $\pi\tilde\phi\restrp\frak A_{1\Bsetminus e}
=\pi\restrp\frak A_{1\Bsetminus e}$ and
$\pi\tilde\phi\restrp\frak A_e=\pi\phi\restrp\frak A_e$
both have transversals (using 382I again).   So $\pi\tilde\phi$ has a
transversal (382Ga).
}%end of proof of 382L

\leader{382M}{Theorem} Let $\frak A$ be a Dedekind $\sigma$-complete
Boolean algebra, and $G$ a countably full subgroup of $\Aut\frak A$ such
that every member of $G$ has a separator.   If $\pi\in G$, it can be
expressed as the product of at most three exchanging involutions
belonging to $G$.

\proof{ By 382L, there is a $\phi\in G$, either the identity or an
exchanging involution, such that $\pi\phi$ has a transversal.   By 382H,
$\pi\phi$ is the product of at most two exchanging involutions in $G$,
so $\pi=\pi\phi\phi^{-1}$ is the product of at most three exchanging
involutions.
}%end of proof of 382M

\leader{382N}{Corollary} If $\frak A$ is a Dedekind complete Boolean
algebra and $G$ is a full subgroup of $\Aut\frak A$, every $\pi\in G$
is expressible as the product
of at most three involutions all belonging to $G$ and all supported by
$\supp\pi$.

\proof{ We may suppose that $G$ is the full subgroup of $\Aut\frak A$
generated by $\pi$.   By 382Eb, every member of $G$ has a separator.
By 382M, $\pi$ is the product of at most three involutions all belonging
to $G$;  by 381Jb, they are all supported by $\supp\pi$.
}%end of proof of 382N

%simple Aut\frak A
\leader{382O}{Definition} Let $\frak A$ be a Boolean algebra, and $G$ a
subgroup of the automorphism group $\Aut\frak A$.   I will say that $G$
{\bf has many involutions} if for every non-zero $a\in\frak A$ there is
an involution $\pi\in G$ which is supported by $a$.

\leader{382P}{Lemma} Let $\frak A$ be an atomless homogeneous Boolean
algebra.   Then $\Aut\frak A$ has many involutions, and in fact every
non-zero element of $\frak A$ is the support of an exchanging
involution.

\proof{ If $a\in\frak A\setminus\{0\}$, then there is a $b$ such that
$0\ne b\Bsubset a$.   Let $\psi:\frak A_b\to\frak A_{a\Bsetminus b}$ be
an isomorphism;  define $\pi\in\Aut\frak A$ to agree with $\psi$ on
$\frak A_b$, with $\psi^{-1}$ on $\frak A_{a\Bsetminus b}$, and with the
identity on $\frak A_{1\Bsetminus a}$.   Then $\pi$ is an exchanging
involution with support $a$.
}%end of proof of 382P

\leader{382Q}{Lemma} Let $\frak A$ be a Dedekind complete Boolean
algebra, and $G$ a full subgroup of $\Aut\frak A$ with many involutions.
Then every non-zero element of $\frak A$ is the support of an
exchanging involution belonging to $G$.

\proof{ By the definition 382O,

\Centerline{$C=\{\supp\pi:\pi\in G$ is an involution$\}$}

\noindent is order-dense in $\frak A$.   So if
$a\in\frak A\setminus\{0\}$ there is a
disjoint $B\Bsubseteq C$ such that $\sup B=a$ (313K).   For each
$b\in B$ let $\pi_b\in G$ be an involution with support $b$.   Define
$\pi\in G$ by setting $\pi d=\pi_bd$ for $d\Bsubseteq b\in B$, $\pi d=d$
if $d\Bcap a=0$;  then $\pi\in G$ is an involution with support $a$.
By 382Fa it is an exchanging involution.
}%end of proof of 382Q
%AC

\leader{382R}{Theorem} Let $\frak A$ be a Dedekind complete Boolean
algebra, and $G$ a full subgroup of $\Aut\frak A$ with many involutions.
Then a subset $H$ of $G$ is a normal subgroup of $G$ iff it is of the
form

\Centerline{$\{\pi:\pi\in G$, $\supp\pi\in I\}$}

\noindent for some ideal $I\normalsubgroup\frak A$ which is
{\bf $G$-invariant}, that is, such that $\pi a\in I$ for every $a\in I$
and $\pi\in G$.

\proof{{\bf (a)} I deal with the easy implication first.   Let
$I\normalsubgroup\frak A$ be a $G$-invariant ideal and set
$H=\{\pi:\pi\in G$, $\supp\pi\in I\}$.   Because the support of the
identity automorphism $\iota$ is $0\in I$, $\iota\in H$.   If $\phi$,
$\psi\in H$ and $\pi\in G$, then

\Centerline{$\supp(\phi\psi)\Bsubseteq\supp\phi\Bcup\supp\psi\in I$,}

\Centerline{$\supp(\psi^{-1})=\supp\psi\in I$,}

\Centerline{$\supp(\pi\psi\pi^{-1})=\pi(\supp\psi)\in I$}

\noindent (381E),
and $\phi\psi$, $\psi^{-1}$, $\pi\psi\pi^{-1}$ all belong to
$H$;  so $H\normalsubgroup G$.

\medskip

{\bf (b)} For the rest of the proof, therefore, I suppose that $H$ is a
normal subgroup of $G$ and seek to express it in the given form.   We
can in fact describe the ideal $I$ immediately, as follows.   Set

\Centerline{$J=\{a:a\in\frak A,\,\pi\in H$ whenever $\pi\in G$ is an
involution and $\supp\pi\Bsubseteq a\}$;}

\noindent then $0\in J$ and $a\in J$ whenever $a\Bsubseteq b\in J$.
Also $\pi a\in J$ whenever $a\in J$ and $\pi\in G$.   \Prf\ If $\phi\in G$
is an involution and $\supp\phi\Bsubseteq\pi a$ then
$\phi_1=\pi^{-1}\phi\pi$ is an involution in $G$ and

\Centerline{$\supp\phi_1=\pi^{-1}(\supp\phi)\Bsubseteq a$,}

\noindent so $\phi_1\in H$ and $\phi=\pi\phi_1\pi^{-1}\in H$.   As
$\phi$ is arbitrary, $\pi a\in J$.\ \Qed

I do not know how to prove directly that $J$ is an ideal, so let us set

\Centerline{$I=\{a_0\Bcup a_1\Bcup\ldots\Bcup a_n:
a_0,\ldots,a_n\in J\}$;}

\noindent then $I\normalsubgroup\frak A$, and $\pi a\in I$ for every
$a\in I$ and $\pi\in G$.

\medskip

{\bf (c)} If $a\in\frak A$, $\psi\in H$ and $a\Bcap\psi a=0$ then
$a\in J$.   \Prf\ If $a=0$, this is trivial.   Otherwise, let $\pi\in G$
be an involution with $\supp\pi\Bsubseteq a$;
say $\pi=\cycleii{b}{\pi}{c}$ where $b\Bcup c\Bsubseteq a$.   By 382Q
there is an involution $\pi_1\in G$ such that $\supp\pi_1=b$;  say
$\pi_1=\cycleii{b'}{\pi_1}{b''}$ where $b'\Bcup b''=b$.   Set

\Centerline{$c'=\pi b'$,
\quad$c''=\pi b''=c\Bsetminus c'$,}

\Centerline{$\pi_2=\pi_1\pi\pi_1\pi^{-1}
=\cycleii{b'}{\pi_1}{b''}\cycleii{c'}{\pi\pi_1\pi^{-1}}{c''}$,
\quad$\pi_3=\cycleii{b'}{\pi}{c'}$,}

\Centerline{$\phi=\pi_2^{-1}\psi\pi_2\psi^{-1}\in H$,}

\Centerline{$\bar\pi=\pi_3^{-1}\phi\pi_3\phi^{-1}
=\pi_3^{-1}\pi_2^{-1}\psi\pi_2\psi^{-1}\pi_3\psi\pi_2^{-1}\psi^{-1}\pi_2
\in H$.}

\noindent Now

\Centerline{$\supp(\psi\pi_2\psi^{-1})=\psi(\supp\pi_2)
=\psi(b\Bcup c)\Bsubseteq \psi a$}

\noindent is disjoint from

\Centerline{$\supp\pi_3=b'\Bcup c'\Bsubseteq a$,}

\noindent so $\pi_3$ commutes with $\psi\pi_2\psi^{-1}$, and

$$\eqalign{\bar\pi
&=\pi_3^{-1}\pi_2^{-1}\pi_3
\psi\pi_2\psi^{-1}\psi\pi_2^{-1}\psi^{-1}\pi_2\cr
&=\pi_3^{-1}\pi_2^{-1}\pi_3\pi_2\cr
&=\cycleii{b'}{\pi}{c'}\cycleii{b'}{\pi_1}{b''}
  \cycleii{c'}{\pi\pi_1\pi^{-1}}{c''}\cycleii{b'}{\pi}{c'}
  \cycleii{b'}{\pi_1}{b''}\cycleii{c'}{\pi\pi_1\pi^{-1}}{c''}\cr
&=\cycleii{b'}{\pi}{c'}\cycleii{b''}{\pi}{c''}\cr
&=\pi.\cr}$$

\noindent So $\pi\in H$.   As $\pi$ is arbitrary, $a\in J$.\
\Qed

\medskip

{\bf (d)} If $\pi=\cycleii{a}{\pi}{b}$ is an involution in $G$ and $a\in
J$, then $\pi\in H$.   \Prf\ By 382Q again, there is an involution
$\psi\in G$ such that $\supp\psi = a$;  because $a\in J$, $\psi\in H$.
Express $\psi$ as $\cycleii{a'}{\psi}{a''}$ where $a'\Bcup a''=a$.   Set
$b'=\pi a'$ and $b''=\pi a''$,
so that $\pi=\cycleii{a'}{\pi}{b'}\cycleii{a''}{\pi}{b''}$, and

\Centerline{$\psi_1=\psi\pi\psi\pi^{-1}
=\cycleii{a'}{\psi}{a''}\cycleii{b'}{\pi\psi\pi^{-1}}{b''}\in H$.}

As $\psi_1(a'\Bcup b')=a''\Bcup b''$ is disjoint from
$a'\Bcup b'$, $a'\Bcup b'\in J$, by (c), and
$\pi_1=\cycleii{a'}{\pi}{b'}\in H$;  similarly, $a''\Bcup b''\in J$, so
$\pi_2=\cycleii{a''}{\pi}{b''}\in H$ and
$\pi=\pi_1\pi_2$ belongs to $H$.\ \Qed

\medskip

{\bf (e)} If $\pi\in G$ is an involution and $\supp\pi\in I$, then
$\pi\in H$.   \Prf\ Express $\pi$ as $\cycleii{a}{\pi}{b}$.   Let
$a_0,\ldots,a_n\in J$ be such that
$a\Bcup b\Bsubseteq a_0\Bcup\ldots\Bcup a_n$.   Set

\Centerline{$c_j=a\Bcap a_j\Bsetminus\sup_{i<j}a_i$,
\quad$b_j=\pi c_j$,
\quad$\pi_j=\cycleii{c_j}{\pi}{b_j}$}

\noindent for $j\le n$;  then every $c_j$ belongs to $J$, so every
$\pi_j$ belongs to $H$ (by (d)) and $\pi=\pi_0\ldots\pi_n\in H$.\ \Qed

\medskip

{\bf (f)} If $\pi\in G$ and $\supp\pi\in I$ then $\pi\in H$.   \Prf\ By
382N,  $\pi$ is a product of involutions in $G$ all with supports
included in $\supp\pi$;  by (e), they all belong to $H$, so $\pi$ also
does.\ \Qed

\medskip

{\bf (g)} We are nearly home.   So far we know that $I$ is a
$G$-invariant ideal and that $\pi\in H$ whenever $\pi\in G$ and
$\supp\pi\in I$.   On the other hand, $\supp\pi\in I$ for every
$\pi\in H$.   \Prf\ By 382Eb, $\pi$ has a separator;  take $a'$, $a''$,
$b'$, $b''$, $c$ from 382D(iv).   Then

\Centerline{$a'\Bcap\pi a'=b'\Bcap\pi b'=\ldots=c\Bcap\pi c=0$,}

\noindent so $a',\ldots,c$ all belong to $J$, by (c), and
$\supp\pi=a'\Bcup\ldots\Bcup c$ belongs to $I$.\ \Qed

So $H$ is precisely the set of members of $G$ with supports in $I$, as
required.
}%end of proof of 382R

\leader{382S}{Corollary} Let $\frak A$ be a homogeneous Dedekind
complete Boolean algebra.   Then $\Aut\frak A$ is simple.

\proof{ If $\frak A$ is $\{0\}$ or $\{0,1\}$ this is trivial.
Otherwise, let $H$ be a normal subgroup of $\Aut\frak A$.   Then by 382R
and 382P there is an invariant ideal $I$ of $\frak A$ such that
$H=\{\pi:\supp\pi\in I\}$.   But if $H$ is non-trivial so is $I$;  say
$a\in I\setminus \{0\}$.   If $a=1$ then certainly $1\in I$ and
$H=\Aut\frak A$.   Otherwise, there is a $\pi\in\Aut\frak A$ such that
$\pi a=1\Bsetminus a$ (as in 381D), so $1\Bsetminus a\in I$, and again
$1\in I$ and $H=\Aut\frak A$.
}%end of proof of 382S

\cmmnt{\medskip

\noindent{\bf Remark} I ought to remark that in fact $\Aut\frak A$ is
simple for any homogeneous Dedekind $\sigma$-complete Boolean
algebra;  see {\smc \v St\v ep\'anek \& Rubin 89}, Theorem 5.9b.
}

\exercises{\leader{382X}{Basic exercises (a)}
%\spheader 382Xa
Let $\frak A$ be a Boolean algebra and $Z$ its Stone
space.   Suppose that $\pi\in\Aut\frak A$ is represented by
$f_{\pi}:Z\to Z$.   For $z\in Z$, write
$\Orbit_{\pi}(z)=\{f_{\pi}^n(z):n\in\Bbb Z\}$.   (i) Show that $a\in\frak A$
is a separator for $\pi$ iff $f_{\pi}^{-1}[\widehat{a}]\cap\widehat{a}$
is empty and $\{z:\Orbit_{\pi}(z)\cap\widehat{a}\}\ne\emptyset\}$ is
dense in $\{z:f_{\pi}(z)\ne z\}$.   (ii) Show that $a\in\frak A$ is a
transversal
for $\pi$ iff $\{z:\Orbit_{\pi}(z)\cap\widehat{a}\ne\emptyset\}$ is
dense in $Z$ and $\#(\Orbit_{\pi}(z)\cap\widehat{a})\le 1$ for every
$z$.
%382A

\sqheader 382Xb %3{8}1Xe
Let $X$ be any set.   Show that any automorphism of the Boolean algebra
$\Cal PX$ is expressible as a product of at most two involutions.
%382M

\sqheader 382Xc ({\smc Miller 04}) Let $X$ be a set and $\Sigma$ a
$\sigma$-algebra of subsets of $X$.   Suppose that
$(X,\Sigma)$ is countably separated in the sense that there is a
countable subset of $\Sigma$ separating the points of $X$ (cf.\ 343D).
Let $G$ be the group of permutations $f:X\to X$ such that
$\Sigma=\{f^{-1}[E]:E\in\Sigma\}$.
Show that every automorphism of the Boolean algebra $\Sigma$ has a
separator, so that every member of $G$ is expressible as the product of
at most three involutions belonging to $G$.
%382M 382Xl

\spheader 382Xd %3{8}1Xf
Recall that in any group $G$, a
{\bf commutator} in $G$ is an element of the form $ghg^{-1}h^{-1}$ where
$g$, $h\in G$.   Show that if $\frak A$ is a Dedekind complete
Boolean algebra and $G$ is a full subgroup of $\Aut\frak A$ with many
involutions then every involution in $G$ is a commutator in $G$, so that
every element of $G$ is expressible as a product of three commutators.
%382M

\spheader 382Xe %3{8}1Xg
Give an example of a Dedekind complete Boolean
algebra $\frak A$ such that not every member of $\Aut\frak A$ is a
product of commutators in $\Aut\frak A$.
%382Xd 382M  \#(\frak A)=4

\spheader 382Xf %3{8}1Xh
Let $\frak A$ be a Dedekind complete Boolean algebra,
and suppose that $\Aut\frak A$ has many involutions.   Show that if
$H\normalsubgroup\Aut\frak A$ then every member of $H$ is expressible as
the product of at most three involutions belonging to $H$.
%382R

\spheader 382Xg %3{8}1Xi
Let $\frak A$ be a Dedekind complete Boolean algebra and $G$ a full
subgroup of $\Aut\frak A$ with many involutions.   Show that the
partially ordered set $\Cal H$ of normal subgroups of $G$ is a
distributive lattice, that is, $H\cap K_1K_2=(H\cap K_1)(H\cap K_2)$,
$H(K_1\cap K_2)=HK_1\cap HK_2$ for all $H$, $K_1$, $K_2\in\Cal H$.
%382R

\spheader 382Xh %3{8}1Xj
Let $\frak A$ be a Dedekind complete Boolean algebra and
$G$ a full subgroup of $\Aut\frak A$ with many involutions.   Show that
if $H$ is the normal subgroup of $G$ generated by a finite subset of
$G$, then it is the normal subgroup generated by a single involution.
%382R

\spheader 382Xi %3{8}1Xk
Let $\frak A$ be a Dedekind complete Boolean algebra and
$G$ a full subgroup of $\Aut\frak A$ with many involutions.   Show (i)
that there is an involution $\pi\in G$ such that every member of $G$ is
expressible as a product of conjugates of $\pi$ in $G$ (ii) any proper
normal subgroup of $G$ is included in a maximal proper normal subgroup
of $G$.
%382R

\spheader 382Xj Let $(\frak A,\bar\mu)$ be an atomless probability algebra.
Show that if $\pi:\frak A\to\frak A$ is an ergodic
measure-preserving automorphism it has no transversal.  
%382 notes

\spheader 382Xk Show that if $\frak A$ is a Dedekind $\sigma$-complete
Boolean algebra with countable Maharam type (definition:  331F),
then every automorphism of $\frak A$ has a separator.   \Hint{show that if
$b\in\frak A$ then $\{a:a\Bsymmdiff\pi a\Bsubseteq b\}$ is an
order-closed subalgebra.}
%382D 382Xl out of order query

\spheader 382Xl\dvAnew{2009} 
Let $\frak A$ be a Dedekind $\sigma$-complete Boolean
algebra and $\pi\in\Aut\frak A$.   Show that $\pi$ has a separator iff
there is a sequence $\sequencen{a_n}$ in $\frak A$ such that $\pi$ is
supported by $\sup_{n\in\Bbb N}a_n\Bsymmdiff\pi a_n$.
%382D  out of order query
%a\Bsymmdiff\pi a 
%  = (\pi a\Bsetminus a)\Bcup((1\Bsetminus\pi a)\Bsetminus(1\Bsetminus a))

\spheader 382Xm\dvAnew{2010}
Let $\frak A$ be a Dedekind complete Boolean algebra and $G$ a subgroup of
$\Aut\frak A$ with many involutions.   Show that for every $n\ge 2$ and
every $a\in\frak A\setminus\{0\}$ there is a $\pi\in G$ with period $n$ and
support $a$.
%382Q out of order query
    
\leader{382Y}{Further exercises (a)}
%\spheader 382Ya
Find a Dedekind $\sigma$-complete Boolean algebra with an involution
which is not an exchanging involution.
%382E

\spheader 382Yb\dvAnew{2008} 
Devise an expression of the ideas of parts (f)-(h) of the
proof of 382K which does not involve the Stone representation.
\Hint{show that there is a non-increasing sequence in $\frak A^+$ which
makes enough decisions to play the role of the Boolean homomorphism
$x:\frak A\to\Bbb Z_2$.}
%382K

\spheader 382Yc\dvAformerly{3{}82Yb}
Let $\Cal B$ be the algebra of Borel subsets of
$\Bbb R$.   Show that $\Aut\Cal B$ has exactly three proper normal
subgroups.   \Hint{re-work the proof of 382R, paying particular attention
to calls on Lemma 382Q.   You will need to know
that if $E\in\Cal B$ is uncountable then the subspace $\sigma$-algebra
on $E$ is isomorphic to $\Cal B$;  see \S424 in Volume 4.}
%382R

\spheader 382Yd\dvAformerly{3{}81Yc}
Find a Dedekind $\sigma$-complete Boolean
algebra $\frak A$ with an automorphism which cannot be expressed either
as a product of finitely many involutions in $\Aut\frak A$, or
as a product of finitely many commutators in $\Aut\frak A$.   (This
seems to require a certain amount of ingenuity.)
%382R  382Xd handwritten note maybe, from Dijon trip?

\leaveitout{3{8}1Y? Show that if $\frak A$ is any homogeneous Boolean
algebra, then the action of $\Aut\frak A$ on $\frak A$ is {\bf
oligomorphic}, that is, the induced action of $\frak A^n$ has only
finitely many orbits, for any $n\in\Bbb N$.
}%end of leaveitout
}%end of exercises

%Question:  in any group, the product of two involutions is conjugate to
%its inverse.   In \Aut\frak A, if \pi is conjugate to its inverse, is
%it the product of two involutions?

\endnotes{
\Notesheader{382} The ideas of 382A and 382G-382N are adapted from
{\smc Miller 04}, and (most conspicuously in part (g) of the proof
of 382K) betray their origin in a study of Borel automorphisms of
$\Bbb R$ (see 382Xc).   The magic number of three involutions appears in
{\smc Ryzhikov 93} and {\smc Truss 89}.   The idea of the method
presented here is to shift from a `separator' to a `transversal'.
Since there are many automorphisms without transversals (382Xj),
something quite surprising has to happen.   The diagrams in the proof of
382K are supposed to show the two steps involved in the argument.   We
are trying to draw non-overlapping links to build a function $g^x$ such
that every point of $\Bbb Z$ will belong to a finite orbit of $g^xs$.
This must be done by some uniform, translation-invariant, process based
on configurations already present;  in particular, we are {\it not}
permitted to single out any point of $\Bbb Z$ as a centre for the
construction.   The first attempt is based on the sequence
$\sequencen{E_n}$ of sets corresponding to the decreasing sequence
$\sequencen{e_n}$.   The construction of such a sequence
(382J) requires that there be many separators, which is why these
results cannot be applied to all Boolean algebras, or even to all
homogeneous ones.   If this first attempt fails, however, the points not
recurrent under $g_1^xs$ provide a set $C_0$ with arbitrarily large gaps
both to left and to right, from which the second method can build an
adequate family of links.

Of course the search for these factorizations was inspired by the
well-known corresponding fact for algebras $\Cal PX$ (382Xb).   In those
algebras we can use the axiom of choice unscrupulously to pick out a
point of each orbit, thereby forming a transversal in one step without
considering separators, and then apply 382H in its original simple form.
Perhaps the principal psychological barrier we need to overcome in 382K
is raised in the phrase `fix $x\in X\cap\widehat{c_1}$'.   What I could
have said is `fix an orbit of $f$ meeting $\widehat{c_1}$, and order it
by the transitive closure of the relation $f$';  because the whole point
of the subsequent argument is that we do not have a marker to work from.

This volume is concerned with measure algebras, and all the most
important measure algebras are Dedekind complete.   I take the trouble
to express the ideas down to Theorem 382M in terms of $\sigma$-complete
algebras partly because this is the natural boundary of the arguments
given and partly because in Volume 4 I will look at Borel automorphisms,
as in 382Xc, and 382M as stated may then be illuminating.   But note
that in 382N $\sigma$-completeness is insufficient
(382Yd).   In 382S I allow myself for once to present a result with a
stronger hypothesis than is required for the conclusion;  the point
being that homogeneous
semi-finite measure algebras are necessarily Dedekind complete (383E),
and the arguments for the more general case do not seem to tell us
anything which we can use elsewhere in this treatise.

It is natural to ask whether the number `three' in 382M is best
possible (cf.\ 382Xb).   It seems to be quite difficult to exhibit an automorphism
requiring three involutions;  examples may be found in {\smc Anzai 51} and {\smc Ornstein \& Sheilds 73}\footnote{I am indebted to P.Biryukov and G.Hjorth for the references.}.

Just as well-known facts about symmetry groups lead us to the
factorization theorem 382M, they suggest that automorphism groups of
Boolean algebras may often have few normal subgroups;  and once again we
find that the form of the theorem changes significantly.   However the
root of the phenomenon remains the fact that our groups are multiply
transitive.
382O-382S are derived from {\smc \v St\v ep\'anek \& Rubin 89} and
{\smc Fathi 78}.   An obvious question arising from 382S is:  does {\it
every} homogeneous
Boolean algebra have a simple automorphism group?    This leads into
deep water.   As remarked after 382S, every homogeneous Dedekind
$\sigma$-complete algebra has a simple automorphism group.   Using the
continuum hypothesis, it is possible to construct a homogeneous Boolean
algebra which does not have a simple automorphism group;  but as far as
I am aware no such construction is known which does not rely on some
special axiom outside ordinary set theory.   See
{\smc \v St\v ep\'anek \& Rubin 89}, \S5.
}%end of notes

\discrpage

