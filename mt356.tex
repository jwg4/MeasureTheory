\frfilename{mt356.tex}
\versiondate{5.10.04}
\copyrightdate{1996}

\def\chaptername{Riesz spaces}
\def\sectionname{Dual spaces}

\newsection{356}

As always in functional analysis, large parts of the theory of Riesz
spaces are based on the study of linear functionals.   Following the
scheme of the last section, I define spaces $U^{\sim}$, $U^{\sim}_c$ and
$U^{\times}$, the `order-bounded', `sequentially order-continuous'
and `order-continuous' duals of a Riesz space $U$ (356A).   These are
Dedekind complete Riesz spaces (356B).   If $U$ carries a Riesz norm
they are closely connected with the normed space dual $U^*$, which is
itself a Banach lattice (356D).   For each of them, we have a canonical
Riesz homomorphism from $U$ to the corresponding bidual.   The map from
$U$ to $U^{\times\times}$ is particularly
important (356I);   when this map is an isomorphism we call $U$
`perfect' (356J).   The last third of the section deals with $L$- and
$M$-spaces and the duality between them (356N, 356P), with two important
theorems on uniform integrability (356O, 356Q).

\leader{356A}{Definition} Let $U$ be a Riesz space.

\spheader 356Aa I write $U^{\sim}$ for the space
$\eusm L^{\sim}(U;\Bbb R)$ of order-bounded real-valued linear functionals on $U$, the {\bf order-bounded dual} of $U$.

\spheader 356Ab $U^{\sim}_c$ will be the space $\eusm L^{\sim}_c(U;\Bbb
R)$ of differences of sequentially order-continuous positive real-valued
linear functionals on $U$, the {\bf sequentially order-continuous dual}
of $U$.

\spheader 356Ac $U^{\times}$ will be the space $\eusm L^{\times}(U;\Bbb
R)$ of differences of order-continuous positive real-valued linear
functionals on $U$, the {\bf order-continuous dual} of $U$.

\cmmnt{\medskip

\noindent{\bf Remark} It is easy to check that the three spaces
$U^{\sim}$, $U^{\sim}_c$ and $U^{\times}$ are in general different
(356Xa-356Xc).   But the examples there leave open the question:  can we
find a Riesz space $U$, for which $U^{\sim}_c\ne U^{\times}$, and which
is actually Dedekind complete, rather than just Dedekind
$\sigma$-complete, as in 356Xc?   This leads to unexpectedly deep water;
it is yet another form of the Banach-Ulam problem.   Really this is a
question for Volume 5, but in 363S below I collect the relevant ideas
which are within the scope of the present volume.
}%end of comment

\leader{356B}{Theorem} For any Riesz space $U$,\cmmnt{ its
order-bounded dual} $U^{\sim}$ is a Dedekind
complete Riesz space in which $U^{\sim}_c$ and $U^{\times}$ are bands,
therefore Dedekind complete Riesz spaces in their own right.   For
$f\in U^{\sim}$, $f^+$ and $|f|\in U^{\sim}$ are defined by the formulae

\Centerline{$f^+(w)=\sup\{f(u):0\le u\le w\}$,
\quad$|f|(w)=\sup\{f(u):|u|\le w\}$}

\noindent for every $w\in U^+$.   A non-empty upwards-directed set
$A\subseteq U^{\sim}$ is bounded above iff $\sup_{f\in A}f(u)$ is finite
for every $u\in U$, and in this case $(\sup A)(u)=\sup_{f\in A}f(u)$ for
every $u\in U^+$.

\proof{ 355E, 355H, 355I.
}%end of proof of 356B

\leader{356C}{Proposition} Let $U$ be any Riesz space and $P$ a band
projection on $U$.   Then its adjoint $P':U^{\sim}\to U^{\sim}$, defined
by setting $P'(f)=fP$ for every $f\in U^{\sim}$, is a band projection on
$U^{\sim}$.

\proof{ Because $P:U\to U$ is a positive linear operator,
$P'f\in U^{\sim}$ for every $f\in U^{\sim}$ (355Bd), and $P'$ is a
positive linear operator from $U^{\sim}$ to itself.   Set $Q=I-P$, the
complementary band projection on $U$;  then $Q'$ is another positive
linear operator on $U^{\sim}$, and $P'f+Q'f=f$ for every $f$.   Now
$P'f\wedge Q'f=0$ for every $f\ge 0$.   \Prf\ For any $w\in U^+$,

$$\eqalignno{(P'f-Q'f)^+(w)
&=\sup_{0\le u\le w}(P'f-Q'f)(u)\cr
&=\sup_{0\le u\le w}f(Pu-Qu)
=f(Pw)\cr
\displaycause{because $Pu-Qu\le Pu\le Pw=P(Pw)-Q(Pw)$ whenever
$0\le u\le w$}
&=(P'f)(w),\cr}$$

\noindent so $(P'f-Q'f)^+=P'f$, that is, $P'f\wedge Q'f=0$.\ \QeD\  By
352Rd, $P'$ is a band projection.
}%end of proof of 356C

\leader{356D}{Proposition} Let $U$ be a Riesz space with a Riesz norm.

(a) The normed space dual $U^*$ of $U$ is a solid linear subspace of
$U^{\sim}$, and in itself is a Banach lattice with a Fatou norm and has
the Levi property.

(b) The norm of $U$ is order-continuous iff $U^*\subseteq U^{\times}$.

(c) If $U$ is a Banach lattice, then $U^*=U^{\sim}$, so that $U^{\sim}$,
$U^{\times}$ and $U^{\sim}_c$ are all Banach lattices.

(d) If $U$ is a Banach lattice with order-continuous norm then
$U^*=U^{\times}=U^{\sim}$.

\proof{{\bf (a)(i)} If $f\in U^*$ then

\Centerline{$\sup_{|u|\le w}f(u)\le\sup_{|u|\le w}\|f\|\|u\|
=\|f\|\|w\|<\infty$}

\noindent for every $w\in U^+$, so $f\in U^{\sim}$ (355Ba).   Thus
$U^*\subseteq U^{\sim}$.

\medskip

\quad{\bf (ii)} If $f\in U^{\sim}$, $g\in U^*$ and $|f|\le|g|$, then
for any $w\in U$

\Centerline{$|f(w)|\le|f|(|w|)\le|g|(|w|)=\sup_{|u|\le|w|}g(u)
\le\sup_{|u|\le|w|}\|g\|\|u\|\le\|g\|\|w\|$.}

\noindent As $w$ is arbitrary, $f\in U^*$ and $\|f\|\le\|g\|$;  as $f$
and $g$ are arbitrary, $U^*$ is a solid linear subspace of $U^{\sim}$
and the norm of $U^*$ is a Riesz norm.   Because $U^*$ is a Banach space
it is also a Banach lattice.

\medskip

\quad{\bf (iii)} If $A\subseteq(U^*)^+$ is non-empty and upwards-directed
and $M=\sup_{f\in A}\|f\|$ is finite, then $\sup_{f\in A}f(u)\le M\|u\|$
is finite for every $u\in U^+$, so $g=\sup A$ is defined in $U^{\sim}$
(355Ed).   Now $g(u)=\sup_{f\in A}f(u)$ for every $u\in U^+$, as also
noted in 355Ed, so

\Centerline{$|g(u)|\le g(|u|)\le M\||u|\|=M\|u\|$}

\noindent for every $u\in U$, and $\|g\|\le M$.   But as $A$ is
arbitrary, this proves simultaneously that the norm of $U^{\sim}$ is
Fatou and has the Levi property.

\medskip

{\bf (b)(i)} Suppose that the norm of $U$ is order-continuous.
If $f\in U^*$
and $A\subseteq U$ is a non-empty downwards-directed set with infimum
$0$, then

\Centerline{$\inf_{u\in A}|f|(u)\le\inf_{u\in A}\|f\|\|u\|=0$,}

\noindent so $|f|\in U^{\times}$ and $f\in U^{\times}$.   Thus
$U^*\subseteq U^{\times}$.

\medskip

\quad{\bf (ii)} Now suppose that the norm is not order-continuous.
Then there is a non-empty downwards-directed set $A\subseteq U$, with
infimum $0$, such that $\inf_{u\in A}\|u\|=\delta>0$.   Set

\Centerline{$B=\{v:v\ge u$ for some $u\in A\}$.}

\noindent Then $B$ is convex.   \Prf\ If $v_1$, $v_2\in B$ and
$\alpha\in[0,1]$, there are $u_1$, $u_2\in A$ such that $v_i\ge u_i$ for
both $i$;  now there is a $u\in A$ such that $u\le u_1\wedge u_2$, so
that

\Centerline{$u=\alpha u+(1-\alpha)u\le\alpha v_1+(1-\alpha)v_2$,}

\noindent and $\alpha v_1+(1-\alpha)v_2\in B$.\ \QeD\  Also
$\inf_{v\in B}\|v\|=\delta>0$.   By the Hahn-Banach theorem (3A5Cb), there is an
$f\in U^*$ such that $\inf_{v\in B}f(v)>0$.   But now

\Centerline{$\inf_{u\in A}|f|(u)\ge\inf_{u\in A}f(u)>0$}

\noindent and $|f|$ is not order-continuous;  so $U^*\not\subseteq
U^{\times}$.

\medskip

{\bf (c)}  By 355C, $U^{\sim}\subseteq U^*$, so $U^{\sim}=U^*$.   Now
$U^{\times}$ and $U^{\sim}_c$, being bands, are closed linear subspaces
(354Bd), so are Banach lattices in their own right.

\medskip

{\bf (d)} Put (b) and (c) together.
}%end of proof of 356D


\leader{356E}{\dvrocolon{Biduals}}\cmmnt{ If you have studied any
functional analysis at all, it will come as no surprise that
duals-of-duals are important in the theory of Riesz spaces.   I start
with a simple lemma.

\medskip

\noindent}{\bf Lemma} Let $U$ be a Riesz space and $f:U\to\Bbb R$ a
positive
linear functional.   Then for any $u\in U^+$ there is a positive linear
functional $g:U\to\Bbb R$ such that $0\le g\le f$, $g(u)=f(u)$ and
$g(v)=0$ whenever $u\wedge v=0$.

\proof{ Set $g(v)=\sup_{\alpha\ge 0}f(v\wedge\alpha u)$ for every $v\in
U^+$.   Then it is easy to see that $g(\beta v)=\beta g(v)$ for every
$v\in U^+$, $\beta\in\coint{0,\infty}$.   If $v$, $w\in U^+$ then

\Centerline{$(v\wedge\alpha u)+(w\wedge\alpha u)
\le(v+w)\wedge 2\alpha u
\le (v\wedge 2\alpha u)+(w\wedge 2\alpha u)$}

\noindent for every $\alpha\ge 0$ (352Fa), so $g(v+w)=g(v)+g(w)$.
Accordingly $g$ has an extension to a linear functional from $U$ to
$\Bbb R$ (355D).   Of course $0\le g(v)\le f(v)$ for $v\ge 0$, so
$0\le g\le f$ in $U^{\sim}$.   We have $g(u)=f(u)$, while if $u\wedge v=0$
then $\alpha u\wedge v=0$ for every $\alpha\ge 0$, so $g(v)=0$.
}%end of proof of 356E

\leader{356F}{Theorem} Let $U$ be a Riesz space and $V$ a solid linear
subspace of $U^{\sim}$.   For $u\in U$ define $\hat u:V\to\Bbb R$ by
setting $\hat u(f)=f(u)$ for every $f\in V$.   Then $u\mapsto\hat u$ is
a Riesz homomorphism from $U$ to $V^{\times}$.

\proof{{\bf (a)} By the definition of addition and scalar multiplication
in $V$, $\hat u$ is linear for every $u$;  also
$\widehat{\alpha u}=\alpha\hat u$ and $(u_1+u_2)\sphat=\hat u_1+\hat u_2$
for all $u$, $u_1$,
$u_2\in U$ and $\alpha\in\Bbb R$.   If $u\ge 0$ then
$\hat u(f)=f(u)\ge  0$ for every $f\in V^+$, so $\hat u\ge 0$;
accordingly every $\hat u$
is the difference of two positive functionals, and $u\mapsto\hat u$ is a
linear operator from $U$ to $V^{\sim}$.

\medskip

{\bf (b)} If $B\subseteq V$ is a non-empty downwards-directed set with
infimum $0$, then $\inf_{f\in B}f(u)=0$ for every $u\in U^+$, by
355Ee.   But this means that $\hat u$ is order-continuous for every
$u\in U^+$, so that $\hat u\in V^{\times}$ for every $u\in U$.

\medskip

{\bf (c)} If $u\wedge v=0$ in $U$, then for any $f\in V^+$ there is a
$g\in[0,f]$ such that $g(u)=f(u)$ and $g(v)=0$ (356E).   So

\Centerline{$(\hat u\wedge\hat v)(f)\le\hat u(f-g)+\hat v(g)
=f(u)-g(u)+g(v)=0$.}

\noindent As $f$ is arbitrary, $\hat u\wedge\hat v=0$.   As $u$ and $v$
are arbitrary, $u\mapsto\hat u$ is a Riesz homomorphism (352G).
}%end of proof of 356F

\leader{356G}{Lemma} Suppose that $U$ is a Riesz space such that
$U^{\sim}$ separates the points of $U$.   Then $U$ is Archimedean.

\proof{ \Quer\ Otherwise, there are $u$, $v\in U$ such that $v>0$ and
$nv\le u$ for every $n\in\Bbb N$.   Now there is an $f\in U^{\sim}$ such
that $f(v)\ne 0$;  but $|f(v)|\le|f|(v)\le\bover1n|f|(u)$ for every $n$,
so this is impossible.\ \Bang
}%end of proof of 356G

\leader{356H}{Lemma} Let $U$ be an Archimedean Riesz space and $f>0$ in
$U^{\times}$.   Then there is a $u\in U$ such that (i) $u>0$ (ii)
$f(v)>0$ whenever $0<v\le u$ (iii) $g(u)=0$ whenever $g\wedge f=0$ in
$U^{\times}$.   Moreover, if $u_0\in U^+$ is such that $f(u_0)>0$, we
can arrange that $u\le u_0$.

\proof{{\bf (a)} Because $f>0$ there is certainly some $u_0\in U$ such that
$f(u_0)>0$.   Set $A=\{v:0\le v\le u_0,\,f(v)=0\}$.   Then
$(v_1+v_2)\wedge u_0\in A$ for all $v_1$, $v_2\in A$, so $A$ is
upwards-directed.   Because $f(u_0)>0=\sup f[A]$ and $f$ is
order-continuous,
$u_0$ cannot be the least upper bound of $A$, and there is another upper
bound $u_1$ of $A$ strictly less than $u_0$.

Set $u=u_0-u_1>0$.   If $0\le v\le u$ and $f(v)=0$, then

\Centerline{$w\in A\Longrightarrow w\le u_1\Longrightarrow w+v\le u_0
\Longrightarrow w+v\in A$;}

\noindent consequently $nv\in A$ and $nv\le u_0$ for every $n\in\Bbb N$,
so $v=0$.   Thus $u$ has properties (i) and (ii).

\medskip

{\bf (b)} Now suppose that $g\wedge f=0$ in $U^{\times}$.   Let
$\epsilon>0$.   Then for each $n\in\Bbb N$ there is a $v_n\in[0,u]$ such
that $f(v_n)+g(u-v_n)\le 2^{-n}\epsilon$ (355Ec).   If $v\le v_n$ for
every $n\in\Bbb N$ then $f(v)=0$ so $v=0$;  thus $\inf_{n\in\Bbb
N}v_n=0$.
Set $w_n=\inf_{i\le n}v_i$ for each $n\in\Bbb N$;  then
$\sequencen{w_n}$ is non-increasing and has infimum $0$ so (because $g$
is order-continuous) $\inf_{n\in\Bbb N}g(w_n)=0$.   But

\Centerline{$u-w_n=\sup_{i\le n}u-v_i\le\sum_{i=0}^nu-v_i$,}

\noindent so

\Centerline{$g(u-w_n)\le\sum_{i=0}^ng(u-v_i)\le 2\epsilon$}

\noindent for every $n$, and

\Centerline{$g(u)\le 2\epsilon+\inf_{n\in\Bbb N}g(w_n)=2\epsilon$.}

\noindent As $\epsilon$ is arbitrary, $g(u)=0$;  as $g$ is arbitrary,
$u$ has the third required property.
}%end of proof of 356H

\leader{356I}{Theorem} Let $U$ be any Archimedean Riesz space.   Then
the canonical map from $U$ to $U^{\times\times}$\cmmnt{ (356F)} is an
order-continuous Riesz homomorphism from $U$ onto an order-dense Riesz
subspace of $U^{\times\times}$.   If $U$ is Dedekind complete, its image
in $U^{\times\times}$ is solid.

\proof{{\bf (a)} By 356F, $u\mapsto\hat u:U\to U^{\times\times}$ is a
Riesz homomorphism.

To see that it is order-continuous, take any non-empty
downwards-directed set $A\subseteq U$ with infimum $0$.   Then
$C=\{\hat u:u\in A\}$ is downwards-directed, and for any
$f\in(U^{\times})^+$

\Centerline{$\inf_{\phi\in C}\phi(f)=\inf_{u\in A}f(u)=0$}

\noindent because $f$ is order-continuous.   As $f$ is arbitrary,
$\inf C=0$ (355Ee);  as $A$ is arbitrary, $u\mapsto\hat u$ is
order-continuous (351Ga).

\medskip

{\bf (b)} Now suppose that $\phi>0$ in $U^{\times\times}$.   By 356H,
there is an $f>0$ in $U^{\times}$ such that $\phi(f)>0$ and $\phi(g)=0$
whenever $g\wedge f=0$.   Next, there is a $u>0$ in $U$ such that
$f(u)>0$.   Since $u\ge 0$, $\hat u\ge 0$;  since $\hat u(f)>0$,
$\hat u\wedge\phi>0$.

Because $U^{\times\times}$ (being Dedekind complete) is Archimedean,
$\inf_{\alpha>0}\alpha\hat u=0$, and there is an $\alpha>0$ such that

\Centerline{$\psi=(\hat u\wedge\phi-\alpha\hat u)^+>0$.}

\noindent Let $g\in (U^{\times})^+$ be such that $\psi(g)>0$
%and $\psi(h)>0$ whenever $0<h\le g$
and $\theta(g)=0$ whenever $\theta\wedge\psi =0$ in $U^{\times\times}$.
Let $v\in U^+$ be such that $g(v)>0$
%and $g(w)>0$ whenever $0<w\le v$
and $h(v)=0$ whenever $h\wedge g=0$ in $U^{\times}$.

Because $\hat v(g)=g(v)>0$, $\hat v\wedge\psi>0$.   As $\psi\le\hat u$,
$\hat v\wedge\hat u>0$ and $\hat v\wedge\alpha\hat u>0$.   Set
$w=v\wedge\alpha u$;  then $\hat w=\hat v\wedge\alpha\hat u$, by
356F, so $\hat w>0$.

\Quer\ Suppose, if possible, that $\hat w\not\le\phi$.   Then
$\theta=(\hat w-\phi)^+>0$, so there is an $h\in(U^{\times})^+$ such
that $\theta(h)>0$ and $\theta(h')>0$ whenever $0<h'\le h$ (356H, for
the fourth and last time).   Now examine

$$\eqalignno{\theta(h\wedge g)
&\le(\alpha\hat u-\phi\wedge\hat u)^+(g)\cr
\noalign{\noindent (because $\hat w\le\alpha\hat u$, $\phi\wedge\hat
u\le\phi$, $h\wedge g\le g$)}
&=0\cr}$$

\noindent because
$(\alpha\hat u-\phi\wedge\hat u)^+\wedge\psi=0$.
So $h\wedge g=0$ and $h(v)=0$.   But this means that

\Centerline{$\theta(h)\le\hat w(h)\le\hat v(h)=0$,}

\noindent which is impossible.\ \Bang

Thus $0<\hat w\le\phi$.  As $\phi$ is arbitrary, the image $\hat U$ of
$U$ is quasi-order-dense in $U^{\times\times}$, therefore order-dense
(353A).

\medskip

{\bf (c)} Now suppose that $U$ is Dedekind complete and that
$0\le\phi\le\psi\in\hat U$.   Express $\psi$ as $\hat u$ where $u\in U$,
and set $A=\{v:v\in U,\,v\le u^+,\,\hat v\le\phi\}$.   If $v\in U$ and
$0\le\hat v\le\phi$, then $w=v^+\wedge u^+\in A$ and $\hat w=\hat v$;
thus $\phi=\sup\{\hat v:v\in A\}=\hat v_0$, where $v_0=\sup A$.   So
$\phi\in\hat U$.   As $\phi$ and $\psi$ are arbitrary, $\hat U$ is solid
in $U^{\times\times}$.
}%end of proof of 356I

\leader{356J}{Definition} A Riesz space $U$ is {\bf perfect} if the
canonical map from $U$ to $U^{\times\times}$ is an isomorphism.

\leader{356K}{Proposition} A Riesz space $U$ is perfect iff (i) it is
Dedekind complete (ii) $U^{\times}$ separates the points of $U$ (iii)
whenever $A\subseteq U$ is non-empty and upwards-directed and
$\{f(u):u\in A\}$ is bounded for every $f\in U^{\times}$, then $A$ is
bounded above in $U$.

\proof{{\bf (a)} Suppose that $U$ is perfect.   Because it is isomorphic
to $U^{\times\times}$, which is surely Dedekind complete, $U$ also is
Dedekind complete.   Because the map $u\mapsto\hat u:U\to
U^{\times\times}$ is injective, $U^{\times}$ separates the points of
$U$.   If $A\subseteq U$ is non-empty and upwards-directed ad
$\{f(u):u\in A\}$ is bounded above for every $f\in U^{\times}$, then
$B=\{\hat u:u\in A\}$ is non-empty and upwards-directed and
$\sup_{\phi\in B}\phi(f)<\infty$ for every $f\in U^{\times}$, so
$\sup B$ is defined in $U^{\times\sim}$ (355Ed);  but $U^{\times\times}$ is
a band in $U^{\times\sim}$, so $\sup B$ belongs to
$U^{\times\times}$ and is of
the form $\hat w$ for some $w\in U$.   Because $u\mapsto\hat u$ is a
Riesz space isomorphism, $w=\sup A$ in $U$.   Thus $U$ satisfies the
three conditions.

\medskip

{\bf (b)} Suppose that $U$ satisfies the three conditions.   We know
that $u\mapsto\hat u$ is an order-continuous Riesz homomorphism
onto an order-dense Riesz
subspace of $U^{\times\times}$ (356I).   It is injective because
$U^{\times}$ separates the points of $U$.   If $\phi\ge 0$ in
$U^{\times\times}$, set $A=\{u:u\in U^+,\,\hat u\le\phi\}$.   Then $A$
is non-empty and upwards-directed and for any $f\in U^{\times}$

\Centerline{$\sup_{u\in A}f(u)\le\sup_{u\in A}|f|(u)
\le\sup_{u\in A}\hat u(|f|)\le\phi(|f|)<\infty$,}

\noindent so by condition (iii) $A$ has an upper bound in $U$.   Since
$U$ is Dedekind complete, $w=\sup A$ is defined in $U$.   Now

\Centerline{$\hat w=\sup_{u\in A}\hat u=\phi$.}

\noindent As $\phi$ is arbitrary, the image of $U$ includes
$(U^{\times\times})^+$, therefore is the whole of $U^{\times\times}$,
and $u\mapsto\hat u$ is a bijective Riesz homomorphism, that is, a Riesz
space isomorphism.
}%end of proof of 356K

\leader{356L}{Proposition} (a) Any band in a perfect Riesz space is a
perfect Riesz space in its own right.

(b) For any Riesz space $U$, $U^{\sim}$ is perfect;  consequently $U^{\sim}_c$ and $U^{\times}$ are perfect.

\proof{{\bf (a)} I use the criterion of 356K.   Let $U$ be a perfect
Riesz space and $V$ a band in $U$.   Then $V$ is Dedekind complete
because $U$ is (353Jb).   If $v\in V\setminus\{0\}$ there is an
$f\in U^{\times}$ such that $f(v)\ne 0$;  but the embedding
$V\embedsinto U$ is
order-continuous (352N), so $g=f\restrp V$ belongs to $V^{\times}$,
and $g(v)\ne 0$.   Thus $V^{\times}$ separates the points of $V$.   If
$A\subseteq V$ is
non-empty and upwards-directed and $\sup_{v\in A}g(v)$ is finite for
every $g\in V^{\times}$, then $\sup_{v\in A}f(v)<\infty$ for every $f\in
U^{\times}$ (again because $f\restrp V\in V^{\times}$), so $A$ has an
upper bound in $U$;  because $U$ is Dedekind complete, $\sup A$ is
defined in $U$;  because $V$ is a band, $\sup A\in V$ and is an upper
bound for $A$ in $V$.   Thus $V$ satisfies the conditions of 356K and is
perfect.

\medskip

{\bf (b)} $U^{\sim}$ is Dedekind complete, by 355Ea.   If
$f\in U^{\sim}\setminus\{0\}$, there is a $u\in U$ such that $f(u)\ne 0$;  now
$\hat u(f)\ne 0$, where $\hat u\in U^{\sim\times}$ (356F).   Thus
$U^{\sim\times}$ separates the points of $U^{\sim}$.   If $A\subseteq
U^{\sim}$ is non-empty and upwards-directed and $\sup_{f\in A}\phi(f)$
is finite for every $\phi\in U^{\sim\times}$, then, in particular,

\Centerline{$\sup_{f\in A}f(u)=\sup_{f\in A}\hat u(f)<\infty$}

\noindent for every $u\in U$, so $A$ is bounded above in $U^{\sim}$, by
355Ed.   Thus $U^{\sim}$ satisfies the conditions of 356K and is
perfect.

By (a), it follows at once that $U^{\times}$ and $U^{\sim}_c$ are
perfect.
}%end of proof of 356L

\leader{356M}{Proposition} If $U$ is a Banach lattice in which the norm
is order-continuous and has the Levi property, then $U$ is perfect.

\proof{ By 356Db, $U^*=U^{\times}$;  since $U^*$ surely separates the
points of $U$, so does $U^{\times}$.   By 354Ee, $U$ is Dedekind
complete.   If $A\subseteq U$ is non-empty and upwards-directed and
$f[A]$ is bounded for every $f\in U^{\times}$, then $A$ is norm-bounded,
by the Uniform Boundedness Theorem (3A5Hb).   Because the norm is
supposed to have the Levi property, $A$ is bounded above in $U$.
Thus $U$ satisfies all the conditions of 356K and is perfect.
}%end of proof of 356M

\leader{356N}{$L$- and \dvrocolon{$M$-spaces}}\cmmnt{ I come now to
the duality between $L$-spaces and $M$-spaces which I hinted at in
\S354.

\medskip

\noindent}{\bf Proposition} Let $U$ be an Archimedean Riesz space with
an order-unit norm.

(a) $U^*=U^{\sim}$ is an $L$-space.

(b) If $e$ is the
standard order unit of $U$, then $\|f\|=|f|(e)$ for every $f\in U^*$.

(c) A linear functional $f:U\to\Bbb R$ is positive iff it belongs to
$U^*$ and $\|f\|=f(e)$.

(d) If $e\ne 0$ there is a positive linear functional $f$ on $U$ such
that $f(e)=1$.

\proof{{\bf (a)-(b)} We know already that $U^*\subseteq U^{\sim}$ is a
Banach lattice (356Da).   If $f\in U^{\sim}$ then

\Centerline{$\sup\{|f(u)|:\|u\|\le 1\}=\sup\{|f(u)|:|u|\le e\}=|f|(e)$,}

\noindent so $f\in U^*$ and $\|f\|=|f|(e)$;  thus $U^{\sim}=U^*$.   If
$f$, $g\ge 0$ in $U^*$, then

\Centerline{$\|f+g\|=(f+g)(e)=f(e)+g(e)=\|f\|+\|g\|$;}

\noindent thus $U^*$ is an $L$-space.

\medskip

{\bf (c)} As already remarked, if $f$ is positive then $f\in U^*$ and
$\|f\|=f(e)$.   On the other hand, if $f\in U^*$ and $\|f\|=f(e)$, take
any $u\ge 0$.   Set $v=(1+\|u\|)^{-1}u$.   Then $0\le v\le e$ and
$\|e-v\|\le 1$ and

\Centerline{$f(e-v)\le|f(e-v)|\le \|f\|=f(e)$.}

\noindent But this means that $f(v)\ge 0$ so $f(u)\ge 0$.   As $u$ is
arbitrary, $f\ge 0$.

\medskip

{\bf (d)} By the Hahn-Banach theorem (3A5Ac), there is an $f\in U^*$
such that $f(e)=\|f\|=1$;  by (c), $f$ is positive.
}%end of proof of 356N

\leader{356O}{Theorem} Let $U$ be an Archimedean Riesz space with
order-unit norm.   Then a set $A\subseteq U^*=U^{\sim}$ is uniformly
integrable iff it is norm-bounded and
$\lim_{n\to\infty}\sup_{f\in A}|f(u_n)|=0$ for every order-bounded disjoint sequence
$\sequencen{u_n}$ in $U^+$.

\proof{{\bf (a)} Suppose that $A$ is uniformly integrable.   Then it is
surely norm-bounded (354Ra).   If $\sequencen{u_n}$ is a disjoint
sequence in $U^+$ bounded above by $w$, then for any $\epsilon>0$ we can
find an $h\ge 0$ in $U^*$ such that $\|(|f|-h)^+\|\le\epsilon$ for every
$f\in A$.   Now $\sum_{i=0}^nh(u_i)\le h(w)$ for every $n$, and
$\lim_{n\to 0}h(u_n)=0$;  since at the same time

\Centerline{$|f(u_n)|\le|f|(u_n)\le h(u_n)+(|f|-h)^+(u_n)
\le h(u_n)+\epsilon\|u_n\|\le h(u_n)+\epsilon\|w\|$}

\noindent for every $f\in A$ and $n\in\Bbb N$,
$\limsup_{n\to\infty}\sup_{f\in A}|f|(u_n)\le\epsilon\|w\|$.   As
$\epsilon$ is arbitrary,

\Centerline{$\lim_{n\to\infty}\sup_{f\in A}|f|(u_n)=0$,}

\noindent and the conditions are satisfied.

\medskip

{\bf (b)(i)} Now suppose that $A$ is norm-bounded but not uniformly
integrable.   Write $B$ for the solid hull of $A$, $M$ for
$\sup_{f\in A}\|f\|=\sup_{f\in B}\|f\|$;  then there is a disjoint sequence
$\sequencen{g_n}$ in $B\cap(U^*)^+$ which is not
norm-convergent to $0$ (354R(b-iv)), that is,

\Centerline{$\delta=\bover12\limsup_{n\to\infty}g_n(e)
=\bover12\limsup_{n\to\infty}\|g_n\|>0$,}

\noindent where $e$ is the standard order unit of $U$.

\medskip

\quad{\bf (ii)} Set

\Centerline{$C=\{v:0\le v\le e,\,\sup_{g\in B}g(v)\ge\delta\}$,}

\Centerline{$D=\{w:0\le w\le e,\,\limsup_{n\to\infty}g_n(w)>\delta\}$.}

\noindent Then for any $u\in D$ we can find $v\in C$ and $w\in D$ such that
$v\wedge w=0$.   \Prf\ Set $\delta'=\limsup_{n\to\infty}g_n(u)$,
$\eta=(\delta'-\delta)/(3+M)>0$;  take $k\in\Bbb N$ so large that
$k\eta\ge M$.

Because $g_n(u)\ge\delta'-\eta$ for infinitely many $n$, we can find a
set $K\subseteq\Bbb N$, of size $k$, such that $g_i(u)\ge\delta'-\eta$
for every $i\in K$.   Now we know that, for each $i\in K$,
$g_i\wedge k\sum_{j\in K,j\ne i}g_j=0$, so there is a $v_i\le u$ such that
$g_i(u-v_i)+k\sum_{j\in K,j\ne i}g_j(v_i)\le\eta$ (355Ec).   Now

\Centerline{$g_i(v_i)\ge g_i(u)-\eta\ge\delta'-2\eta$,
\quad$g_i(v_j)\le\Bover{\eta}k$ for $i$, $j\in K$, $i\ne j$.}

Set $v_i'=(v_i-\sum_{j\in K,j\ne i}v_j)^+$ for each $i\in K$;  then

\Centerline{$g_i(v'_i)\ge g_i(v_i)-\sum_{j\in K,j\ne
i}g_i(v_j)\ge\delta'-3\eta$}

\noindent for every $i\in K$, while $v'_j\wedge v'_i=0$ for distinct
$i$, $j\in K$.

For each $n\in\Bbb N$,

\Centerline{$\sum_{i\in K}g_n(u\wedge\Bover1{\eta}v'_i)
\le g_n(u)\le\|g_n\|\le\eta k$,}

\noindent so there is some $i(n)\in K$ such that

\Centerline{$g_n(u\wedge\Bover1{\eta}v'_{i(n)})\le\eta$,
\quad $g_n(u-\Bover1{\eta}v'_{i(n)})^+\ge g_n(u)-\eta$.}

\noindent Since $\{n:g_n(u)\ge\delta+2\eta\}$ is infinite, there is some
$m\in K$ such that $J=\{n:g_n(u)\ge\delta+2\eta,\,i(n)=m\}$ is infinite.
Try

\Centerline{$v=(v'_m-\eta u)^+$,
\quad$w=(u-\Bover1{\eta}v'_m)^+$.}

\noindent Then $v$, $w\in[0,u]$ and $v\wedge w=0$.   Next,

\Centerline{$g_m(v)\ge g_m(v'_m)-\eta M\ge\delta'-3\eta-\eta M=\delta$,}

\noindent so $v\in C$, while for any $n\in J$

\Centerline{$g_n(w)=g_n(u-\Bover1{\eta}v'_{i(n)})^+
\ge g_n(u)-\eta\ge\delta+\eta$;}

\noindent since $J$ is infinite,

\Centerline{$\limsup_{n\to\infty}g_n(w)\ge\delta+\eta>\delta$}

\noindent and $w\in D$.\ \Qed

\medskip

\quad{\bf (iii)} Since $e\in D$, we can choose inductively sequences
$\sequencen{w_n}$ in $D$, $\langle v_n\rangle_{n\in\Bbb N}$ in $C$ such
that $w_0=e$, $v_n\wedge w_{n+1}=0$, $v_n\vee w_{n+1}\le w_n$ for every
$n\in\Bbb N$.   But in this case $\sequencen{v_n}$ is a disjoint
order-bounded sequence in $[0,u]$, while
for each $n\in\Bbb N$, we can find $f_n\in A$ such that
$|f_n|(v_n)>\bover23\delta$.   Now there is a $u_n\in[0,v_n]$ such that
$|f_n(u_n)|\ge\bover13\delta$.   \Prf\ Set $\gamma=\sup_{0\le v\le
v_n}|f_n(v)|$.   Then $f_n^+(v_n)$, $f^-_n(v_n)$ are both less than or
equal to $\gamma$, so $|f_n|(v_n)\le 2\gamma$ and
$\gamma>\bover13\delta$;  so there is a $u_n\in[0,v_n]$ such that
$|f_n(u_n)|\ge\bover13\delta$.\ \Qed

Accordingly we have a disjoint sequence $\sequencen{u_n}$ in $[0,e]$
such that $\sup_{f\in A}|f(u_n)|\ge\bover13\delta$ for every
$n\in\Bbb N$.

\medskip

\quad{\bf (iv)} All this is on the assumption that $A$ is norm-bounded and
not uniformly integrable.   So, turning it round, we see that if $A$ is
norm-bounded and $\lim_{n\to\infty}\sup_{f\in A}|f(u_n)|=0$ for every
order-bounded disjoint sequence $\sequencen{u_n}$, $A$ must be uniformly
integrable.

This completes the proof.
}%end of proof of 356O

\leader{356P}{Proposition} Let $U$ be an $L$-space.

(a) $U$ is perfect.

(b) $U^*=U^{\sim}=U^{\times}$ is an $M$-space;  its standard
order unit is the functional $\int$ defined by setting
$\int u=\|u^+\|-\|u^-\|$ for every $u\in U$.

(c) If $A\subseteq U$ is non-empty and upwards-directed and
$\sup_{u\in A}\int u$ is finite, then $\sup A$ is defined in $U$ and $\int\sup A=\sup_{u\in A}\int u$.

\proof{{\bf (a)} By 354N we know that the norm on $U$ is
order-continuous and has
the Levi property, so 356M tells us that $U$ is perfect.

\medskip

{\bf (b)} 356Dd tells us that $U^*=U^{\sim}=U^{\times}$.

The $L$-space property tells us that the functional
$u\mapsto\|u\|:U^+\to\Bbb R$ is additive;  of course it is also
homogeneous, so by 355D it has an extension to a linear functional
$\int:U\to\Bbb R$ satisfying the given formula.   Because
$\int u=\|u\|\ge 0$ for $u\ge 0$, $\int\in(U^{\sim})^+$.   For
$f\in U^{\sim}$,

$$\eqalign{|f|\le\int
&\iff|f|(u)\le\int u\text{ for every }u\in U^+\cr
&\iff|f(v)|\le\|u\|\text{ whenever }|v|\le u\in U\cr
&\iff|f(v)|\le\|v\|\text{ for every }v\in U\cr
&\iff\|f\|\le 1,\cr}$$

\noindent so the norm on $U^*=U^{\sim}$ is the order-unit norm defined
from $\int$, and $U^{\sim}$ is an $M$-space, as claimed.

\wheader{356P}{4}{2}{2}{60pt}

{\bf (c)} Fix $u_0\in A$, and set $B=\{u^+:u\in A,\,u\ge u_0\}$.   Then
$B\subseteq U^+$ is upwards-directed, and

$$\eqalign{\sup_{v\in B}\|v\|
&=\sup_{u\in A,u\ge u_0}\int u^+
=\sup_{u\in A,u\ge u_0}\int u+\int u^-\cr
&\le\sup_{u\in A,u\ge u_0}\int u+\int u_0^-  %
<\infty.\cr}$$

\noindent Because $\|\,\|$ has the Levi property, $B$ is bounded above.
But (because $A$ is upwards-directed) every member of $A$ is dominated
by some member of $B$, so $A$ also is bounded above.   Because $U$ is
Dedekind complete, $\sup A$ is defined in $U$.
Finally, $\int\sup A=\sup_{u\in A}\int u$ because $\int$, being a
positive member of $U^{\times}$, is order-continuous.
}%end of proof of 356P

\leader{356Q}{Theorem} Let $U$ be any $L$-space.   Then a subset of $U$
is uniformly integrable iff it is relatively weakly compact.

\proof{{\bf (a)} Let $A\subseteq U$ be a uniformly integrable set.

\medskip

\quad{\bf (i)} Suppose that $\Cal F$ is an ultrafilter on $X$ containing
$A$.   Then $A\ne\emptyset$.      Because $A$ is norm-bounded,
$\sup_{u\in A}|f(u)|<\infty$ and $\phi(f)=\lim_{u\to\Cal F}f(u)$ is
defined in $\Bbb R$ for every $f\in U^*$ (2A3Se).

If $f$, $g\in U^*$ then

\Centerline{$\phi(f+g)=\lim_{u\to\Cal F}f(u)+g(u)
=\lim_{u\to\Cal F}f(u)+\lim_{u\to\Cal F}g(u)
=\phi(f)+\phi(g)$}

\noindent (2A3Sf).   Similarly,

\Centerline{$\phi(\alpha f)=\lim_{u\to\Cal F}\alpha f(u)
=\alpha\phi(f)$}

\noindent whenever $f\in U^*$ and $\alpha\in\Bbb R$.   Thus
$\phi:U^*\to\Bbb R$ is linear.   Also

\Centerline{$|\phi(f)|\le\sup_{u\in A}|f(u)|\le\|f\|\sup_{u\in
A}\|u\|$,}

\noindent so $\phi\in U^{**}=U^{*\sim}$.

\medskip

\quad{\bf (ii)}
Now the point of this argument is that $\phi\in U^{*\times}$.   \Prf\
Suppose that $B\subseteq U^*$ is non-empty and downwards-directed and
has infimum $0$.   Fix $f_0\in B$.   Let $\epsilon>0$.   Then there is a
$w\in U^+$ such that $\|(|u|-w)^+\|\le\epsilon$ for every $u\in A$,
which means that

\Centerline{$|f(u)|\le |f|(|u|)\le |f|(w)+|f|(|u|-w)^+\le
|f|(w)+\epsilon\|f\|$}

\noindent for every $f\in U^*$ and every $u\in A$.   Accordingly
$|\phi(f)|\le |f|(w)+\epsilon\|f\|$ for every $f\in U^*$.   Now
$\inf_{f\in B}f(w)=0$ (using 355Ee, as usual), so there is an
$f_1\in B$ such that $f_1\le f_0$ and $f_1(w)\le\epsilon$.   In this case

\Centerline{$|\phi|(f_1)
=\sup_{|f|\le f_1}|\phi(f)|
\le\sup_{|f|\le f_1}|f|(w)+\epsilon\|f\|
\le f_1(w)+\epsilon\|f_1\|
\le \epsilon(1+\|f_0\|)$.}

\noindent As $\epsilon$ is arbitrary, $\inf_{f\in B}|\phi|(f)=0$;  as
$B$ is arbitrary, $|\phi|$ is order-continuous and $\phi\in
U^{*\times}$.\ \Qed

\medskip

\quad{\bf (iii)} At this point, we recall that $U^*=U^{\times}$ and that
the canonical map from $U$ to $U^{\times\times}$ is surjective (356P).
So there is a $u_0\in U$ such that $\hat u_0=\phi$.   But now we see
that

\Centerline{$f(u_0)=\phi(f)=\lim_{u\to\Cal F}f(u)$}

\noindent for every $f\in U^*$;  which is just what is meant by saying
that $\Cal F\to u_0$ for the weak topology on $U$ (2A3Sd).

Accordingly every ultrafilter on $U$ containing $A$ has a limit in $U$.
But because the weak topology on $U$ is regular (3A3Be), it follows
that the closure of $A$ for the weak topology is compact (3A3De),
so that $A$ is relatively weakly compact.

\medskip

{\bf (b)} For the converse I use the criterion of 354R(b-iv).  Suppose
that $A\subseteq U$ is relatively weakly compact.   Then $A$ is
norm-bounded, by the Uniform Boundedness Theorem.   Now let
$\sequencen{u_n}$ be any disjoint sequence in the solid hull of $A$.
For each $n$, let $U_n$ be the band in $U$ generated by $u_n$.   Let
$P_n$ be the band projection from $U$ onto $U_n$ (353Hb).   Let
$v_n\in A$ be such that $|u_n|\le|v_n|$;  then

\Centerline{$|u_n|=P_n|u_n|\le P_n|v_n|=|P_nv_n|$,}

\noindent so $\|u_n\|\le\|P_nv_n\|$ for each $n$.   Let $g_n\in U^*$ be
such that $\|g_n\|=1$ and $g_n(P_nv_n)=\|P_nv_n\|$.

Define $T:U\to\BbbR^{\Bbb N}$ by setting $Tu=\sequencen{g_n(P_nu)}$ for
each $u\in U$.   Then $T$ is a continuous linear operator from $U$ to
$\ell^1$.   \Prf\ For $m\ne n$, $U_m\cap U_n=\{0\}$, because
$|u_m|\wedge|u_n|=0$.   So, for any $u\in U$, $\sequencen{P_nu}$ is a
disjoint sequence in $U$, and

\Centerline{$\sum_{i=0}^n\|P_iu\|=\|\sum_{i=0}^n|P_iu|\|
=\|\sup_{i\le n}|P_iu|\|\le\|u\|$}

\noindent for every $n$;  accordingly

\Centerline{$\|Tu\|_1=\sum_{i=0}^{\infty}|g_iP_iu|
\le\sum_{i=0}^{\infty}\|P_iu\|\le\|u\|$.}

\noindent Since $T$ is certainly a linear operator (because every
coordinate functional $g_iP_i$ is linear), we have the result.\ \Qed

Consequently $T[A]$ is relatively weakly compact in $\ell^1$, because
$T$ is continuous for the weak topologies (2A5If).   But $\ell^1$ can be
identified with $L^1(\mu)$, where $\mu$ is counting measure on $\Bbb N$.
So $T[A]$ is uniformly integrable in $\ell^1$, by 247C, and in
particular $\lim_{n\to\infty}\sup_{w\in T[A]}|w(n)|=0$.   But this means
that

\Centerline{$\lim_{n\to\infty}\|u_n\|
\le\lim_{n\to\infty}|g_n(P_nv_n)|
=\lim_{n\to\infty}|(Tv_n)(n)|
=0$.}

\noindent As $\sequencen{u_n}$ is arbitrary, $A$ satisfies the
conditions of 354R(b-iv) and is uniformly integrable.
}%end of proof of 356Q

\exercises{\leader{356X}{Basic exercises (a)}
%\spheader 356Xa
Show that if $U=\ell^{\infty}$ then
$U^{\times}=U^{\sim}_c$ can be identified with $\ell^1$, and is properly
included in $U^{\sim}$.   \Hint{show that if $f\in U^{\sim}_c$ then
$f(u)=\sum_{n=0}^{\infty}u(n)f(e_n)$, where $e_n(n)=1$, $e_n(i)=0$ for
$i\ne n$.}
%356A

\spheader 356Xb Show that if $U=C([0,1])$ then
$U^{\times}=U^{\sim}_c=\{0\}$.   \Hint{show that if $f\in
(U^{\sim}_c)^+$ and $\sequencen{q_n}$ enumerates $\Bbb Q\cap[0,1]$, then
for each $n\in\Bbb N$ there is a $u_n\in U^+$ such that $u_n(q_n)=1$ and
$f(u_n)\le 2^{-n}$.}
%356A

\spheader 356Xc Let $X$ be an uncountable set, $\mu$ the
countable-cocountable measure on $X$ and $\Sigma$ its domain (211R).   Let $U$ be the space of bounded $\Sigma$-measurable real-valued functions on
$X$.   Show that $U$ is a Dedekind $\sigma$-complete Banach lattice if
given the supremum norm $\|\,\|_{\infty}$.   Show that $U^{\times}$ can
be identified with $\ell^1(X)$ (cf.\ 356Xa), and that
$u\mapsto\int u\,d\mu$ belongs to $U^{\sim}_c\setminus U^{\times}$.
%356A

\spheader 356Xd Let $U$ be a Dedekind $\sigma$-complete Riesz space and
$f\in U^{\sim}_c$.   Let $\sequencen{u_n}$ be an order-bounded sequence
in $U$ which is order-convergent to $u\in U$ in the sense that
$u=\inf_{n\in\Bbb N}\sup_{m\ge n}u_m=\sup_{n\in\Bbb N}\inf_{m\ge n}u_m$.
Show that $\lim_{n\to\infty}f(u_n)$ exists and is equal to $f(u)$.
%356A

\spheader 356Xe Let $U$ be any Riesz space.   Show that the band
projection $P:U^{\sim}\to U^{\times}$ is defined by the formula

$$\eqalign{(Pf)(u)=\inf\{\sup_{v\in A}f(v):A\subseteq U
&\text{ is non-empty, upwards-directed}\cr
&\qquad\qquad\qquad\qquad\text{and has supremum }u\}\cr}$$

\noindent for every $f\in(U^{\sim})^+$, $u\in U^+$.   \Hint{show that
the formula for $Pf$ always defines an order-continuous linear
functional.   Compare 355Yh, 356Yb and 362Bd.}
%356B

\spheader 356Xf Let $U$ be any Riesz space.   Show that the band
projection $P:U^{\sim}\to U^{\sim}_c$ is defined by the formula

\Centerline{$(Pf)(u)=\inf\{\sup_{n\in\Bbb N}f(v_n):\sequencen{v_n}$ is a
non-decreasing sequence with supremum $u\}$}

\noindent for every $f\in(U^{\sim})^+$, $u\in U^+$.
%356B

\spheader 356Xg Let $U$ be a Riesz space with a Riesz norm.   Show that
$U^*$ is perfect.
%356D

\spheader 356Xh  Let $U$ be a Riesz space with a Riesz norm.   Show that
the canonical map from $U$ to $U^{**}$ is a Riesz homomorphism.
%356F

\spheader 356Xi Let $V$ be a perfect Riesz space and $U$ any Riesz
space.   Show that $\eusm L^{\sim}(U;V)$ is perfect.   \Hint{show that
if $u\in U$ and $g\in V^{\times}$ then $T\mapsto g(Tu)$ belongs to
$\eusm L^{\sim}(U;V)^{\times}$.}
%356K

\spheader 356Xj Let $U$ be an $M$-space.   Show that it is perfect iff
it is Dedekind complete and $U^{\times}$ separates the points of $U$.
%356K

\spheader 356Xk Let $U$ be a Banach lattice which, as a Riesz space, is
perfect.   Show that its norm has the Levi property.
%356M

\spheader 356Xl Write out a proof from first principles that if
$\sequencen{u_n}$ is a sequence in $\ell^1$ such that
$|u_n(n)|\ge\delta>0$ for every $n\in\Bbb N$, then $\{u_n:n\in\Bbb N\}$
is not relatively weakly compact.
%356Q

\spheader 356Xm Let $U$ be an $L$-space and $A\subseteq U$ a non-empty
set.   Show that the following are equiveridical:  (i) $A$ is uniformly
integrable (ii) $\inf_{f\in B}\sup_{u\in A}|f(u)|$ for every non-empty
downwards-directed set $B\subseteq U^{\times}$ with infimum $0$ (iii)
$\inf_{n\in\Bbb N}\sup_{u\in A}|f_n(u)|=0$ for every non-increasing
sequence $\sequencen{f_n}$ in $U^{\times}$ with infimum $0$ (iv) $A$ is
norm-bounded and $\lim_{n\to\infty}\sup_{u\in A}|f_n(u)|=0$ for every
disjoint order-bounded sequence $\sequencen{f_n}$ in $U^{\times}$.
%356Q

\leader{356Y}{Further exercises (a)}
%\spheader 356Ya
Let $U$ be a Riesz space with the countable sup
property.   Show that $U^{\times}=U^{\sim}_c$.
%356A

\spheader 356Yb Let $U$ be a Riesz space, and $\Cal A$ a family of
non-empty downwards-directed subsets of $U^+$ all with infimum $0$.
(i) Show that $U^{\sim}_{\Cal A}=\{f:f\in U^{\sim},\,\inf_{u\in
A}|f|(u)=0$ for every $A\in\Cal A\}$ is a band in $U^{\sim}$.   (ii) Set
$\Cal A^*=\{A_0+\ldots+A_n:A_0,\ldots,A_n\in\Cal A\}$.   Show that
$U^{\sim}_{\Cal A}=U^{\sim}_{\Cal A^*}$.    (iii) Take any
$f\in(U^{\sim})^+$, and let $g$, $h$ be the components of $f$ in
$U^{\sim}_{\Cal A}$, $(U^{\sim}_{\Cal A})^{\perp}$ respectively.   Show
that

\Centerline{$g(u)=\inf_{A\in\Cal A^*}\sup_{v\in A}f(u-v)^+$,
\quad$h(u)=\sup_{A\in\Cal A^*}\inf_{v\in A}f(u\wedge v)$}

\noindent for every $u\in U^+$.   (Cf.\ 362Xi.)
%356Xe, 356B

\spheader 356Yc Let $U$ be a Riesz space.   For any band $V\subseteq U$
write $V^{\smallcirc}$ for
$\{f:f\in U^{\times}$, $f(v)\le 1$ for every $v\in V\}
=\{f:f\in U^{\times},\,f(v)=0$ for every $v\in V\}$.
Show that $V\mapsto (V^{\perp})^{\smallcirc}$ is a surjective
order-continuous
Boolean homomorphism from the algebra of complemented bands of $U$ onto
the band algebra of $U^{\times}$, and that it is injective iff
$U^{\times}$ separates the points of $U$.
%356C

\spheader 356Yd Let $U$ be a Dedekind complete Riesz space such that
$U^{\times}$ separates the points of $U$ and $U$ is the solid linear
subspace of itself generated by a countable set.   Show that $U$ is
perfect.
%356K, 356Xj

\spheader 356Ye  Let $U$ be an $L$-space and
$\sequencen{u_n}$ a sequence in $U$ such that $\sequencen{f(u_n)}$ is
Cauchy for every $f\in U^*$.   Show that $\sequencen{u_n}$ is convergent
for the weak topology of $U$.   \Hint{use 356Xm(iv) to show that
$\{u_n:n\in\Bbb N\}$ is relatively weakly compact.}
%356Xm, 356Q

\spheader 356Yf  Let $U$ be a perfect Banach lattice with
order-continuous norm and
$\sequencen{u_n}$ a sequence in $U$ such that $\sequencen{f(u_n)}$ is
Cauchy for every $f\in U^*$.   Show that $\sequencen{u_n}$ is convergent
for the weak topology of $U$.   \Hint{set
$\phi(f)=\lim_{n\to\infty}f_n(u)$.   For any $g\in(U^*)^+$ let $V_g$ be
the solid linear subspace of $U^*$ generated by $g$,
$W_g=\{u:g(|u|)=0\}^{\perp}$, $\|u\|_g=g(|u|)$ for $u\in W_g$.   Show
that the completion of $W_g$ under $\|\,\|_g$ is an $L$-space with dual
isomorphic to $V_g$, and hence (using 356Ye) that $\phi\restrp V_g$
belongs to $V_g^{\times}$;  as $g$ is arbitrary, $\phi\in V^{\times}$
and may be identified with an element of $U$.}
%356Ye, 356Xm, 356Q

\spheader 356Yg Let $U$ be a uniformly complete Archimedean Riesz space
with complexification $V$ (354Yl).   (i) Show that the complexification
of $U^{\sim}$ can be identified with the space of linear functionals
$f:V\to\Bbb C$ such that $\sup_{|v|\le u}|f(v)|$ is finite for every
$u\in U^+$.   (ii) Show that if $U$ is a Banach lattice, then the
complexification of $U^{\sim}=U^*$ can be identified (as normed space)
with $V^*$.   (See 355Yk.)
%+

\spheader 356Yh Let $U$ be a perfect Banach lattice.
Show that the family of closed balls in $U$ is a compact class.
\Hint{342Ya.}
}%end of exercises

\endnotes{
\Notesheader{356} The section starts easily enough, with special cases
of results in \S355 (356B).   When $U$ has a Riesz norm,  the
identification of $U^*$ as a subspace of $U^{\sim}$, and the
characterization of order-continuous norms (356D) are pleasingly
comprehensive and straightforward.   Coming to biduals, we need to think
a little (356F), but there is still no real difficulty at first.   In
356H-356I, however, something more substantial is happening.   I have
written these arguments out in what seems to be the shortest route to
the main theorem, at the cost perhaps of neglecting any intuitive
foundation.   What I think we are really doing is matching bands in $U$,
$U^{\times}$ and $U^{\times\times}$, as in 356Yc.

From now on, almost the first thing we shall ask of any new Riesz space
will be whether it is perfect, and if not, which of the three conditions
of 356K it fails to satisfy.   For reasons which will I hope appear in
the next chapter, perfect Riesz spaces are especially important in
measure theory;  in particular, all $L^p$ spaces for
$p\in\coint{1,\infty}$  are perfect (366Dd), as are the $L^{\infty}$
spaces of localizable measure spaces (365N).   Further examples
will be discussed in \S369 and \S374.   Of course we have to remember
that there are also important Riesz spaces which are not perfect, of
which $C([0,1])$ and $\pmb{c}_0$ are two of the simplest examples.

The duality between $L$- and $M$-spaces (356N, 356P) is natural and
satisfying.   We are now in a position to make a determined attempt to
tidy up the notion of `uniform integrability'.   I give two major
theorems.   The first is yet another `disjoint-sequence'
characterization of uniformly integrable sets, to go with 246G and 354R.
The essential difference here is that we are looking at disjoint
sequences in a predual;  in a sense, this means that the result is a
sharper one, because the $M$-space $U$ need not be Dedekind complete
(for instance, it could be $C([0,1])$ -- this indeed is the archetype
for applications of the theorem) and therefore need not have as many
disjoint sequences as its dual.   (For instance, in the dual of
$C([0,1])$ we have all the point masses $\delta_t$, where
$\delta_t(u)=u(t)$;  these form a disjoint family in $C([0,1])^{\sim}$
not corresponding to any disjoint family in $C([0,1])$.)   The essence
of the proof is a device to extract a disjoint sequence in $U$ to match
approximately a subsequence of a given disjoint sequence in $U^{\sim}$.
In the example just suggested, this would correspond, given a sequence
$\sequencen{t_n}$ of distinct points in $[0,1]$, to finding a
subsequence $\sequence{i}{t_{n(i)}}$ which is discrete, so that we can
find disjoint $u_i\in C([0,1])$ with $u_i(t_{n(i)})=1$ for each $i$.

The second theorem, 356Q, is a new version of a result already given in
\S247:  in any $L$-space, uniform integrability is the same as relative
weak compactness.   I hope you are not exasperated by having been asked,
in Volume 2, to master a complex argument (one of the more difficult
sections of that volume) which was going to be superseded.   Actually it
is worse than that.   A theorem of Kakutani (369E) tells us that
every $L$-space is isomorphic to an $L^1$ space.   So 356Q is itself a
consequence of 247C.   I do at least owe you an explanation for writing
out two proofs.   The first point is that the result is sufficiently
important for it to be well worth while spending time in its
neighbourhood, and the contrasts and similarities between the two
arguments are instructive.   The second is that the proof I have just
given was not really accessible at the level of Volume 2.   It does not
rely on every single page of this chapter, but the key idea (that $U$ is
isomorphic to $U^{\times\times}$, so it will be enough if we can show
that $A$ is relatively compact in $U^{\times\times}$) depends
essentially on 356I, which lies pretty deep in the abstract theory of
Riesz spaces.   The third is an aesthetic one:  a theorem about
$L$-spaces ought to be proved in the category of normed Riesz spaces,
without calling on a large body of theory outside.   Of course this is a
book on measure theory, so I did the measure theory first, but if you
look at everything that went into it, the proof in \S247 is I believe
longer, in the formal sense, than the one here, even setting aside the
labour of proving Kakutani's theorem.

Let us examine the ideas in the two proofs.   First, concerning the
proof that uniformly integrable sets are relatively compact, the method
here is very smooth and natural;  the definition I chose of `uniform
integrability' is exactly adapted to showing that uniformly integrable
sets are relatively compact in the order-continuous bidual;  all the
effort goes into the proof that $L$-spaces are perfect.   The previous
argument depended on identifying the dual of $L^1$ as $L^{\infty}$ --
and was disagreeably complicated by the fact that the identification is
not always valid, so that I needed to reduce the problem to the
$\sigma$-finite case (part (b-ii) of the proof of 247C).   After that,
the Radon-Nikod\'ym theorem did the trick.   Actually Kakutani's theorem
shows that the side-step to $\sigma$-finite spaces is irrelevant.   It
directly represents an abstract $L$-space as $L^1(\mu)$ for a
localizable measure $\mu$, in which case $(L^1)^*\cong L^{\infty}$
exactly.

In the other direction, both arguments depend on a disjoint-sequence
criterion for uniform integrability (246G(iii) or 354R(b-iv)).   These
criteria belong to the `easy' side of the topic;  straightforward
Riesz space arguments do the job, whether written out in that language
or not.   (Of course the new one in this section, 356O, lies a little
deeper.)   I go a bit faster this time because I feel that you ought by
now to be happy with the Hahn-Banach theorem and the Uniform Boundedness
Theorem, which I was avoiding in Volume 2.   And then of course I quote
the result for $\ell^1$.   This looks like cheating.   But $\ell^1$
really is easier, as you will find if you just write out part (a) of the
proof of 247C for this case.   It is not exactly that you can dispense
with any particular element of the argument;  rather it is that the
formulae become much more direct when you can write $u(i)$ in place of
$\int_{F_i}u$, and `cluster points for the weak topology' become
pointwise limits of subsequences, so that the key step (the `sliding
hump', in which $u_{k(j)}(n(k(j)))$ is the only significant coordinate
of $u_{k(j)}$), is easier to find.

We now have a wide enough variety of conditions equivalent to uniform
integrability for it to be easy to find others;  I give a couple in
356Xm, corresponding in a way to those in 246G.   You may have noticed,
in the proof of 247C, that in fact the full strength of the hypothesis
`relatively weakly compact' is never used;  all that is demanded is
that a couple of sequences should have cluster points for the weak
topology.   So we see that a set $A$ is uniformly integrable iff every
sequence in $A$ has a weak cluster point.   But this extra refinement is
nothing to do with $L$-spaces;  it is generally true, in any normed
space $U$, that a set $A\subseteq U$ is relatively weakly compact iff
every sequence in $A$ has a cluster point in $U$ for the weak topology
(`Eberlein's theorem';  see 462D in Volume 4, {\smc K\"othe 69}, 24.2.1,
or {\smc Dunford \& Schwartz 57}, V.6.1).

There is a very rich theory concerning weak compactness in perfect Riesz
spaces, based on the ideas here;  some of it is explored in {\smc
Fremlin 74a}.   As a sample, I give one of the basic properties of
perfect Banach lattices with order-continuous norms:  they are `weakly
sequentially complete' (356Yf).
}%end of comment

\frnewpage

