\frfilename{mt2a3.tex}
\versiondate{25.7.07}
\copyrightdate{1995}

\def\varBbb#1{\mathchoice{\hbox{$\Bbb#1$\hskip0.02em}}
  {\hbox{$\Bbb#1$\hskip0.02em}}
  {\hbox{$\scriptstyle\Bbb#1$\hskip0.04em}}
  {\hbox{$\scriptscriptstyle\Bbb#1$\hskip0.04em}}}

\def\chaptername{Appendix}
\def\sectionname{General topology}

\newsection{2A3}

At various points -- principally \S\S245-247, %\S245 \S246 \S247
but also for certain
ideas in Chapter 27 -- we need to know something about non-metrizable
topologies.   I must say that you should probably
take the time to look at some book on elementary functional analysis
which has the phrases `weak compactness' or `weakly compact' in the
index.   But I can
list here the concepts actually used in this volume, in a good deal less
space than any orthodox, complete treatment would employ.

\leader{2A3A}{Topologies} \cmmnt{First we need to know what a {\lq
topology'}
is.}   If $X$ is
any set, a {\bf topology} on $X$ is a family $\frak T$ of subsets of $X$
such that (i) $\emptyset$, $X\in\frak T$ (ii) if $G$, $H\in\frak T$ then
$G\cap H\in\frak T$ (iii) if $\Cal G\subseteq\frak T$ then
$\bigcup\Cal G\in\frak T$\cmmnt{ (cf.\ 1A2B)}.   \cmmnt{The pair} $(X,\frak T)$ is now a
{\bf topological space}.   In this context, members of $\frak T$ are
called {\bf open} and their complements\cmmnt{ (in $X$)} are called
{\bf closed}\cmmnt{ (cf.\ 1A2E-1A2F)}.

\leader{2A3B}{Continuous functions (a)} If $(X,\frak T)$ and $(Y,\frak
S)$ are topological spaces, a
function $\phi:X\to Y$ is {\bf continuous} if $\phi^{-1}[G]\in\frak T$
for every $G\in\frak S$.   \cmmnt{(By 2A2Ca above, this is consistent
with the
$\epsilon$-$\delta$ definition of continuity for functions from one
Euclidean space to another.   See also 2A3H below.)}

\header{2A3Bb}{\bf (b)} If $(X,\frak T)$, $(Y,\frak S)$ and
$(Z,\frak U)$ are topological spaces
and $\phi:X\to Y$ and $\psi:Y\to Z$ are continuous, then
$\psi\phi:X\to Z$
is continuous.   \prooflet{\Prf\ If $G\in\frak U$ then
$\psi^{-1}[G]\in\frak S$ so
$(\psi\phi)^{-1}[G]=\phi^{-1}[\psi^{-1}[G]]\in\frak T$.\ \Qed}

\header{2A3Bc}{\bf (c)} If $(X,\frak T)$ is a topological space, a
function $f:X\to\Bbb R$ is continuous iff $\{x:a<f(x)<b\}$ is open
whenever $a<b$ in $\Bbb R$.  \prooflet{\Prf\ {\bf (i)} Every interval
$\ooint{a,b}$ is open in $\Bbb R$, so if $f$ is continuous its inverse
image $\{x:a<f(x)<b\}$ must be open.   {\bf (ii)} Suppose that
$f^{-1}[\,\ooint{a,b}\,]$ is open whenever $a<b$, and let
$H\subseteq\Bbb R$ be any open set.   By the definition of `open' set
in $\Bbb R$\cmmnt{ (1A2A)},

$$\eqalign{H&
=\bigcup\{\ooint{y-\delta,y+\delta}:y\in\Bbb R,\,\delta>0,\,
\ooint{y-\delta,y+\delta}\subseteq H\},\cr}$$

\noindent so

\Centerline{$f^{-1}[H]=
\bigcup\{f^{-1}[\,\ooint{y-\delta,y+\delta}\,]:y\in\Bbb R,\,\delta>0,\,
\ooint{y-\delta,y+\delta}\subseteq H\}$}

\noindent is a union of open sets in $X$, therefore open.\ \Qed}

\header{2A3Bd}{\bf (d)} If $r\ge 1$, $(X,\frak T)$ is a topological
space, and $\phi:X\to\BbbR^r$ is a function, then $\phi$ is continuous
iff $\phi_i:X\to\Bbb R$ is continuous for each $i\le r$, where
$\phi(x)=(\phi_1(x),\ldots,\phi_r(x))$ for each $x\in X$.
\prooflet{\Prf\ {\bf (i)} Suppose that $\phi$ is continuous.   For $i\le
r$, $y=(\eta_1,\ldots,\eta_r)\in\BbbR^r$, set $\pi_i(y)=\eta_i$.   Then
$|\pi_i(y)-\pi_i(z)|\le\|y-z\|$ for all $y$, $z\in\BbbR^r$ so
$\pi_i:\BbbR^r\to\Bbb R$
is continuous.   Consequently $\phi_i=\pi_i\phi$ is continuous, by
(b) above.   {\bf (ii)} Suppose that every $\phi_i$ is continuous, and that
$H\subseteq\BbbR^r$ is open.   Set

\Centerline{$\Cal G
=\{G:G\subseteq X$ is open, $G\subseteq\phi^{-1}[H]\}$.}

\noindent Then $G_0=\bigcup\Cal G$ is open, and
$G_0\subseteq \phi^{-1}[H]$.   But suppose that $x_0$ is any point of
$\phi^{-1}[H]$.   Then there is a $\delta>0$ such that
$U(\phi(x_0),\delta)\subseteq H$, because $H$ is open and contains
$\phi(x_0)$.   For $1\le i\le r$ set
$V_i=\{x:\phi_i(x_0)-\bover{\delta}{\sqrt{r}}
<\phi_i(x)<\phi_i(x_0)+\bover{\delta}{\sqrt{r}}\}$;
then $V_i$ is the inverse image of an open set under the continuous map
$\phi_i$, so is open.   Set $G=\bigcap_{i\le r}V_i$.    Then $G$ is open
(using (ii) of the definition 2A3A), $x_0\in G$, and
$\|\phi(x)-\phi(x_0\|<\delta$ for every $x\in G$, so
$G\subseteq\phi^{-1}[H]$, $G\in\Cal G$ and $x_0\in G_0$.   This shows
that $\phi^{-1}[H]=G_0$ is open.   As $H$ is arbitrary, $\phi$ is
continuous.\ \Qed}

\spheader 2A3Be If $(X,\frak T)$ is a topological space,
$f_1,\ldots,f_r$ are continuous functions from $X$ to $\Bbb R$, and
$h:\BbbR^r\to\Bbb R$ is continuous, then $h(f_1,\ldots,f_r):X\to\Bbb R$
is continuous.   \prooflet{\Prf\ Set
$\phi(x)=(f_1(x),\ldots,f_r(x))\in\BbbR^r$ for $x\in X$.   By (d),
$\phi$ is continuous, so by 2A3Bb $h(f_1,\ldots,f_r)=h\phi$ is
continuous.\ \QeD}   In particular, $f+g$, $f\times g$ and $f-g$ are
continuous for all continuous functions $f$, $g:X\to\Bbb R$.

\spheader 2A3Bf If $(X,\frak T)$ and $(Y,\frak S)$ are topological
spaces and $\phi:X\to Y$ is a continuous function, then $\phi^{-1}[F]$
is closed in $X$ for every closed set $F\subseteq Y$.   \prooflet{(For
$X\setminus\phi^{-1}[F]=\phi^{-1}[Y\setminus F]$ is open.)}

\leader{2A3C}{Subspace topologies}
If $(X,\frak T)$ is a topological space and $D\subseteq X$,
then $\frak T_D=\{G\cap D:G\in\frak T\}$ is a topology on $D$.
\prooflet{\Prf\
(i) $\emptyset=\emptyset\cap D$ and $D=X\cap D$ belong to $\frak T_D$.
(ii) If $G$, $H\in\frak T_D$ there are $G'$, $H'\in\frak T$ such that
$G=G'\cap D$, $H=H'\cap D$;  now $G\cap H=G'\cap H'\cap D\in\frak T_D$.
(iii) If $\Cal G\subseteq\frak T_D$ set $\Cal H=\{H:H\in\frak T,\,H\cap
D\in\Cal G\}$;  then
$\bigcup\Cal G=(\bigcup\Cal H)\cap D\in\frak T_D$.\ \Qed}

$\frak T_D$ is called the {\bf subspace topology} on $D$, or the
topology on $D$ {\bf induced} by $\frak T$.   If $(Y,\frak S)$ is
another topological space, and $\phi:X\to Y$ is
$(\frak T,\frak S)$-continuous, then $\phi\restr D:D\to Y$ is
$(\frak T_D,\frak S)$-continuous.   \prooflet{(For if $H\in\frak S$ then

\Centerline{$(\phi\restr D)^{-1}[H]=D\cap\phi^{-1}[H]\in\frak T_D$.)}}

\leader{2A3D}{Closures and interiors (a)} \cmmnt{In the proof of 2A3Bd
I have already used the following idea.}   Let $(X,\frak T)$ be any
topological space and $A$ any subset of $X$.   Write

\Centerline{$\interior A=\bigcup\{G:G\in\frak T,\,G\subseteq A\}$.}

\noindent Then $\interior A$ is\cmmnt{ an open set, being a union of
open sets,
and is of course included in $A$;  it must be} the largest open set
included in $A$, and is called the {\bf interior} of $A$.


\header{2A3Db}{\bf (b)} Because a set is closed iff its complement is
open, we have a complementary notion:

\dvro{$$\eqalign{\overline{A}
&=\bigcap\{F:F\text{ is closed},\,A\subseteq F\}
=X\setminus\interior(X\setminus A).\cr}$$}
{$$\eqalign{\overline{A}
&=\bigcap\{F:F\text{ is closed},\,A\subseteq F\}\cr
&=X\setminus\bigcup\{X\setminus F:F\text{ is closed},\,A\subseteq F\}\cr
&=X\setminus\bigcup\{G:G\text{ is open},\,A\cap G=\emptyset\}\cr
&=X\setminus\bigcup\{G:G\text{ is open},\,G\subseteq X\setminus A\}
=X\setminus\interior(X\setminus A).\cr}$$}

\noindent$\overline{A}$ is\cmmnt{ closed (being the complement of an
open set) and is} the smallest closed set including $A$;  it is called
the {\bf closure} of $A$.   \cmmnt{(Compare 2A2A.)}   \cmmnt{Because
the union of two closed sets is closed (cf.\ 1A2Fc),}
$\overline{A\cup B}=\overline{A}\cup\overline{B}$ for all $A$,
$B\subseteq X$.
%4{}A2G

\header{2A3Dc}{\bf (c)}\cmmnt{ There are innumerable ways of looking
at these concepts;  a useful description of the closure of a set is}

\dvro{$$\eqalign{x\in\overline A
&\iff\,\text{every open set containing }x\text{ meets }A.\cr}$$}
{$$\eqalign{x\in\overline A
&\iff\,x\notin\interior(X\setminus A)\cr
&\iff\,\text{there is no open set containing }x
\text{ and included in }X\setminus A\cr
&\iff\,\text{every open set containing }x\text{ meets }A.\cr}$$}

\leader{2A3E}{Hausdorff topologies (a)}\dvro{ A}{ The concept
of `topological space' is so widely drawn, and
so widely applicable, that a vast number of different types of
topological space have been studied.   For this volume we shall not need
much of the (very extensive) vocabulary which has been developed to
describe this variety.   But one useful word (and one of the most
important concepts) is that of `Hausdorff space';  a} topological
space $X$ is {\bf Hausdorff} if for all distinct $x$, $y\in X$ there are
disjoint open sets $G$, $H\subseteq X$ such that $x\in G$ and $y\in H$.

\spheader 2A3Eb In a Hausdorff space $X$, finite sets are closed.
\prooflet{\Prf\ If $z\in X$, then for any $x\in X\setminus\{z\}$ there
is an open set containing $x$ but not $z$, so $X\setminus\{z\}$ is open
and
$\{z\}$ is closed.   So a finite set is a finite union of closed sets
and is therefore closed.\ \Qed}

\leader{2A3F}{Pseudometrics}\cmmnt{ Many important topologies (not
all!) can be
defined by families of pseudometrics;  it will be useful to have a
certain amount of technical skill with these.

\medskip

}{\bf (a)} Let $X$ be a set.   A {\bf pseudometric} on $X$
is a function $\rho:X\times X\to\coint{0,\infty}$ such that

\inset{$\rho(x,z)\le\rho(x,y)+\rho(y,z)$ for all $x$, $y$, $z\in X$}

\cmmnt{\noindent (the `triangle inequality';)}

\inset{$\rho(x,y)=\rho(y,x)$ for all $x$, $y\in X$;

$\rho(x,x)=0$ for all $x\in X$.}

\noindent A {\bf metric} is a pseudometric $\rho$ satisfying the further
condition

\inset{if $\rho(x,y)=0$ then $x=y$.}

\header{2A3Fb}{\bf (b) Examples (i)} For $x$, $y\in\Bbb R$, set
$\rho(x,y)=|x-y|$\cmmnt{;  then $\rho$ is a metric on $\Bbb R$} (the
`usual metric' on $\Bbb R$).

\medskip

\quad{\bf (ii)} For $x$, $y\in\BbbR^r$, where $r\ge 1$, set
$\rho(x,y)=\|x-y\|$, defining $\|z\|=\sqrt{\sum_{i=1}^r\zeta_i^2}$,
as usual.   Then $\rho$ is\cmmnt{ a metric,} the {\bf Euclidean
metric} on $\Bbb
R^r$.   \cmmnt{(The triangle inequality for $\rho$ comes from Cauchy's
inequality in 1A2C:  if $x$, $y$, $z\in\BbbR^r$, then

\Centerline{$\rho(x,z)=\|x-z\|=\|(x-y)+(y-z)\|
\le\|x-y\|+\|y-z\|=\rho(x,y)+\rho(y,z)$.}

\noindent The other required properties of $\rho$ are elementary.
Compare 2A4Bb below.)}

\cmmnt{\medskip

\quad{\bf (iii)} For an example of a pseudometric which is not a metric,
take $r\ge 2$ and define $\rho:\BbbR^r\times\Bbb
R^r\to\coint{0,\infty}$ by setting $\rho(x,y)=|\xi_1-\eta_1|$ whenever
$x=(\xi_1,\ldots,\xi_r)$, $y=(\eta_1,\ldots,\eta_r)\in\BbbR^r$.
}%end of comment

\header{2A3Fc}{\bf (c)} Now let $X$ be a set and $\Rho$ a non-empty
family of
pseudometrics on $X$.   Let $\frak T$ be the family of those subsets $G$
of $X$ such that for every $x\in G$ there are
$\rho_0,\ldots,\rho_n\in\Rho$ and $\delta>0$ such that

\Centerline{$U(x;\rho_0,\ldots,\rho_n;\delta)
=\{y:y\in X,\,\max_{i\le n}\rho_i(y,x)<\delta\}\subseteq G$.}

\noindent Then $\frak T$ is a topology on $X$.

\prooflet{\Prf\ (Compare 1A2B.)  (i)
$\emptyset\in\frak T$ because the condition is vacuously satisfied.
$X\in\frak T$ because $U(x;\rho;1)\subseteq X$ for any $x\in X$,
$\rho\in\Rho$.   (ii) If $G$, $H\in\frak T$ and $x\in G\cap H$, take
$\rho_0,\ldots,\rho_m,\rho'_0,\ldots,\rho'_n\in\Rho$, $\delta$,
$\delta'>0$ such that $U(x;\rho_0,\ldots,\rho_m;\delta)\subseteq G$,
$U(x;\rho'_0,\ldots,\rho'_n;\delta')
\ifdim\pagewidth=390pt\penalty-1000\fi
\subseteq G$;   then


\Centerline{
$U(x;\rho_0,\ldots,\rho_m,\rho'_0,\ldots,\rho'_n;\min(\delta,\delta'))
\subseteq G\cap H$.}

\noindent   As $x$ is arbitrary, $G\cap H\in\frak T$.   (iii)
If $\Cal G\subseteq\frak T$ and $x\in\bigcup\Cal G$, there is a
$G\in\Cal G$ such that $x\in G$;  now there are
$\rho_0,\ldots,\rho_n\in\Rho$ and $\delta>0$ such that


\Centerline{$U(x;\rho_0,\ldots,\rho_n;\delta)\subseteq
G\subseteq\bigcup\Cal G$.}

\noindent As $x$ is arbitrary, $\bigcup\Cal G\in\frak T$.
\Qed}

$\frak T$ is the {\bf topology defined by}
$\Rho$.

\header{2A3Fd}{\bf (d)} You may wish to have a convention to deal with
the case in which $\Rho$
is the empty set;   in this case the topology on $X$ defined by $\Rho$
is $\{\emptyset,X\}$.

\cmmnt{\spheader 2A3Fe In many important cases,
$\Rho$ is upwards-directed in the sense that for any $\rho_1$,
$\rho_2\in\Rho$ there is a
$\rho\in\Rho$ such that $\rho_i(x,y)\le\rho(x,y)$ for all $x$, $y\in X$
and both $i$.   In this case, of course, any set
$U(x;\rho_0,\ldots,\rho_n;\delta)$, where $\rho_0,\ldots,\rho_n\in\Rho$,
includes some set of the form $U(x;\rho;\delta)$, where $\rho\in\Rho$.
Consequently, for instance, a set $G\subseteq X$ is open iff for every
$x\in G$ there are $\rho\in\Rho$, $\delta>0$ such that
$U(x;\rho;\delta)\subseteq G$.
}

\spheader 2A3Ff A topology $\frak T$ is {\bf metrizable} if it
is the topology
defined
by a family $\Rho$ consisting of a single metric.   Thus the {\bf
Euclidean topology} on $\BbbR^r$ is the metrizable topology defined by
$\{\rho\}$, where $\rho$ is the metric of (b-ii) above.

\leader{2A3G}{Proposition} Let $X$ be a set with a topology defined by  a non-empty set $\Rho$
of pseudometrics on $X$.   Then $U(x;\rho_0,\ldots,\rho_n;\epsilon)$ is
open for all $x\in X$, $\rho_0,\ldots,\rho_n\in\Rho$ and $\epsilon>0$.

\proof{ (Compare 1A2D.)  Take $y\in U(x;\rho_0,\ldots,\rho_n;\epsilon)$.   Set

\Centerline{$\eta=\max_{i\le n}\rho_i(y,x)$,
\quad $\delta=\epsilon-\eta>0$.}

\noindent    If $z\in U(y;\rho_0,\ldots,\rho_n;\delta)$ then

\Centerline{$\rho_i(z,x)\le\rho_i(z,y)+\rho_i(y,x)<\delta+\eta
=\epsilon$}

\noindent for each $i\le n$, so $U(y;\rho_0,\ldots,\rho_n;\delta)
\subseteq U(x;\rho_0,\ldots,\rho_n;\epsilon)$.   As $y$ is arbitrary,
$U(x;\rho_0,\ldots,\rho_n;\epsilon)$ is open.
}%end of proof of 2A3G

\leader{2A3H}{}\cmmnt{ Now we have a result corresponding to 2A2Ca,
describing continuous functions between topological spaces defined by
families of pseudometrics.

\medskip

\noindent}{\bf Proposition} Let $X$ and $Y$ be sets;  let $\Rho$ be a
non-empty family of
pseudometrics on $X$, and $\Theta$ a non-empty family of pseudometrics
on $Y$;  let $\frak T$ and $\frak S$ be the corresponding topologies.
Then a function $\phi:X\to Y$ is continuous iff whenever $x\in X$,
$\theta\in\Theta$ and $\epsilon>0$, there are
$\rho_0,\ldots,\rho_n\in\Rho$ and $\delta>0$ such that
$\theta(\phi(y),\phi(x))\le\epsilon$ whenever $y\in X$ and $\max_{i\le
n}\rho_i(y,x)\le\delta$.

\proof{{\bf (a)} Suppose that $\phi$ is continuous;  take $x\in X$,
$\theta\in\Theta$ and $\epsilon>0$.   By 2A3G,
$U(\phi(x);\theta;\epsilon)\in\frak S$.
So $G=\phi^{-1}[U(\phi(x);\theta;\epsilon)]\in\frak T$.   Now $x\in G$,
so there are $\rho_0,\ldots,\rho_n\in\Rho$ and $\delta>0$ such that
$U(x;\rho_0,\ldots,\rho_n;\delta)\subseteq G$.   In this case
$\theta(\phi(y),\phi(x))\le\epsilon$ whenever $y\in X$ and $\max_{i\le
n}\rho_i(y,x)\le\bover12\delta$.   As $x$, $\theta$ and $\epsilon$ are
arbitrary, $\phi$ satisfies the condition.

\medskip

{\bf (b)} Suppose $\phi$ satisfies the condition.   Take
$H\in\frak S$ and consider $G=\phi^{-1}[H]$.   If $x\in G$, then
$\phi(x)\in H$, so there are $\theta_0,\ldots,\theta_n\in\Theta$ and
$\epsilon>0$ such that
$U(\phi(x);\theta_0,\ldots,\theta_n;\epsilon)\subseteq H$.   For each
$i\le n$ there are $\rho_{i0},\ldots,\rho_{i,m_i}\in\Rho$ and
$\delta_i>0$ such that $\theta(\phi(y),\phi(x))\le\bover12\epsilon$
whenever $y\in X$ and $\max_{j\le m_i}\rho_{ij}(y,x)\le\delta_i$.   Set
$\delta=\min_{i\le n}\delta_i>0$;  then

\Centerline{$U(x;\rho_{00},\ldots,\rho_{0,m_0},\ldots,\rho_{n0},\ldots,
\rho_{n,m_n};\delta)\subseteq G$.}

\noindent As $x$ is arbitrary, $G\in\frak T$.   As $H$ is arbitrary,
$\phi$ is continuous.
}%end of proof of 2A3H

\cmmnt{
\leader{2A3I}{Remarks (a)} If $\Rho$ is upwards-directed, the condition
simplifies to:  for every $x\in X$,
$\theta\in\Theta$ and $\epsilon>0$, there are
$\rho\in\Rho$ and $\delta>0$ such that
$\theta(\phi(y),\phi(x))\le\epsilon$ whenever $y\in X$ and
$\rho(y,x)\le\delta$.




\spheader 2A3Ib Suppose we have a set $X$ and two non-empty
families $\Rho$,
$\Theta$ of pseudometrics on $X$, generating topologies $\frak T$ and
$\frak S$ on $X$.   Then $\frak S\subseteq\frak T$ iff the identity map
$\phi$ from $X$ to itself is a continuous function when regarded as a
map from $(X,\frak T)$ to $(X,\frak S)$, because this will mean that
$G=\phi^{-1}[G]$ belongs to $\frak T$ whenever $G\in\frak S$.   Applying
the proposition above to $\phi$, we see that this happens iff for every
$\theta\in\Theta$,
$x\in X$ and $\epsilon>0$ there are $\rho_0,\ldots,\rho_n\in\Rho$ and
$\delta>0$ such that $\theta(y,x)\le\epsilon$ whenever $y\in X$ and
$\max_{i\le n}\rho_i(y,x)\le\delta$.   Similarly, reversing the roles
of $\Rho$ and $\Theta$, we get a criterion for when $\frak
T\subseteq\frak S$, and putting the two together we obtain a criterion
to determine when $\frak T=\frak S$.
}%end of comment

\leader{2A3J}{Subspaces:  Proposition} If $X$ is a set, $\Rho$ a
non-empty family of pseudometrics on
$X$ defining a topology $\frak T$ on $X$, and $D\subseteq X$, then

(a) for every $\rho\in\Rho$, the restriction $\rho^{(D)}$ of $\rho$
to $D\times D$ is a pseudometric on $D$;

(b) the topology defined by $\Rho_D=\{\rho^{(D)}:\rho\in\Rho\}$ on
$D$ is precisely the subspace topology $\frak T_D$ described in 2A3C.

\proof{(a) is just a matter of reading through the
definition in 2A3Fa.   For (b), we have to think for a moment.

\medskip

{\bf (i)} Suppose that $G$ belongs to the topology defined by $\Rho_D$.
Set

\Centerline{$\Cal H=\{H:H\in\frak T,\,H\cap D\subseteq G\}$,}

\Centerline{$H^*=\bigcup\Cal
H\in\frak T$,\quad $G^*=H^*\cap D\in\frak T_D$;}

\noindent  then $G^*\subseteq G$.   On
the other hand, if $x\in G$, then there are
$\rho_0,\ldots,\rho_n\in\Rho$ and $\delta>0$ such that

\Centerline{$U(x;\rho_0^{(D)},\ldots,\rho_n^{(D)};\delta)
=\{y:y\in D,\,\max_{i\le n}\rho_i^{(D)}(y,x)<\delta\}\subseteq G$.}

\noindent Consider

\Centerline{$H=U(x;\rho_0,\ldots,\rho_n;\delta)
=\{y:y\in X,\,\max_{i\le n}\rho_i(y,x)<\delta\}\subseteq X$.}

\noindent Evidently

\Centerline{$H\cap D
=U(x;\rho_0^{(D)},\ldots,\rho_n^{(D)};\delta)\subseteq G$.}

\noindent   Also $H\in\frak T$.   So $H\in\Cal H$ and

\Centerline{$x\in H\cap D\subseteq H^*\cap D=G^*$.}

\noindent Thus $G=G^*\in\frak T_D$.

\medskip

{\bf (ii)} Now suppose that $G\in\frak T_D$.   Then there is an
$H\in\frak T$ such that $G=H\cap D$.
Consider the identity map
$\phi:D\to
X$, defined by saying that $\phi(x)=x$ for every $x\in D$.   $\phi$
obviously satisfies the criterion of 2A3H, if we endow $D$ with $\Rho_D$
and $X$ with $\Rho$, because $\rho(\phi(x),\phi(y))=\rho^{(D)}(x,y)$
whenever $x$, $y\in D$ and $\rho\in\Rho$;  so $\phi$ must be continuous
for the associated topologies, and $\phi^{-1}[H]$ must belong to the
topology defined by $\Rho_D$.   But $\phi^{-1}[H]=G$.   Thus every set
in $\frak T_D$ belongs to the topology defined by $\Rho_D$, and the two
topologies are the same, as claimed.
}%end of proof of 2A3J

\leader{2A3K}{Closures and interiors} Let $X$ be a set, $\Rho$ a
non-empty family of pseudometrics on $X$ and $\frak T$ the topology
defined by $\Rho$.

\header{2A3Ka}{\bf (a)} For any $A\subseteq X$ and $x\in X$,

\dvro{$$\eqalign{x\in\interior A
&\iff\,\text{there are }\rho_0,\ldots,\rho_n\in\Rho,\,\delta>0
\text{ such that }U(x;\rho_0,\ldots,\rho_n;\delta)\subseteq A.\cr}$$}
{$$\eqalign{x\in\interior A
&\iff\,\text{there is an open set included in }A
\text{ containing }x\cr
&\iff\,\text{there are }\rho_0,\ldots,\rho_n\in\Rho,\,\delta>0
\text{ such that }U(x;\rho_0,\ldots,\rho_n;\delta)\subseteq A.\cr}$$}

\header{2A3Kb}{\bf (b)} For any $A\subseteq X$ and $x\in X$,
$x\in\overline A$ iff
$U(x;\rho_0,\ldots,\rho_n;\delta)\cap A\ne\emptyset$ for every
$\rho_0,\ldots,\rho_n\in\Rho$ and $\delta>0$.
\cmmnt{(Compare 2A2B(ii), 2A3Dc.)}

\leader{2A3L}{Hausdorff topologies}\cmmnt{ Recall that a topology
$\frak T$ is Hausdorff if any two
points can be separated by open sets (2A3E).}   Now a topology defined
on a set $X$ by a non-empty family $\Rho$ of pseudometrics is Hausdorff
iff for any
two different points $x$, $y$ of $X$ there is a $\rho\in\Rho$ such that
$\rho(x,y)>0$.   \prooflet{\Prf\ {\bf (i)} Suppose that the topology is
Hausdorff and that $x$, $y$ are distinct points in $X$.   Then there is
an open set $G$ containing $x$ but not containing $y$.   Now there are
$\rho_0,\ldots,\rho_n\in\Rho$ and $\delta>0$ such that
$U(x;\rho_0),\ldots,\rho_n;\delta)\subseteq G$, in which case
$\rho_i(y,x)\ge\delta>0$ for some $i\le n$.   {\bf (ii)} If $\Rho$
satisfies the condition, and $x$, $y$ are distinct points of $X$, take
$\rho\in\Rho$ such that $\rho(x,y)>0$, and set
$\delta=\bover12\rho(x,y)$.   Then $U(x;\rho;\delta)$ and
$U(y;\rho;\delta)$ are disjoint (because if $z\in X$, then

\Centerline{$\rho(z,x)+\rho(z,y)\ge\rho(x,y)=2\delta$,}

\noindent so at least one of $\rho(z,x)$, $\rho(z,y)$ is greater than or
equal to $\delta$), and they are open sets containing $x$, $y$
respectively.   As $x$ and $y$ are arbitrary, the topology is Hausdorff.
\Qed}

In particular, metrizable topologies are Hausdorff.

\leader{2A3M}{Convergence of sequences (a)} If $(X,\frak T)$ is any
topological space, and $\sequencen{x_n}$ is a sequence in $X$, we say
that $\sequencen{x_n}$ {\bf converges} to $x\in X$, or that $x$ is a
{\bf limit} of $\sequencen{x_n}$, or $\sequencen{x_n}\to x$, if for
every open set $G$ containing $x$ there is an $n_0\in\Bbb N$ such that
$x_n\in G$ for every $n\ge n_0$.

\header{2A3Mb}{\bf (b) Warning} In general topological spaces, it is
possible for a sequence to have more than one limit\cmmnt{, and we cannot
safely write $x=\lim_{n\to\infty}x_n$}.   But in Hausdorff spaces, this
does not occur.   \prooflet{\Prf\ If $\frak T$ is Hausdorff, and $x$,
$y$ are distinct points of $X$, there are disjoint open sets $G$, $H$
such that $x\in G$ and $y\in H$.   If now $\sequencen{x_n}$ converges to
$x$, there is an $n_0$ such that $x_n\in G$ for every $n\ge n_0$, so
$x_n\notin H$ for every $n\ge n_0$, and $\sequencen{x_n}$ cannot
converge to $y$.\ \Qed}   In particular, a sequence in a metrizable space
can have at most one limit.

\header{2A3Mc}{\bf (c)} Let $X$ be a set, and $\Rho$ a non-empty family
of pseudometrics on $X$, generating a topology $\frak T$;  let
$\sequencen{x_n}$ be a sequence in $X$ and $x\in X$.   Then
$\sequencen{x_n}$ converges to $x$ iff $\lim_{n\to\infty}\rho(x_n,x)=0$
for every $\rho\in\Rho$.   \prooflet{\Prf\ {\bf (i)} Suppose that
$\sequencen{x_n}\to x$ and that $\rho\in\Rho$.   Then for any
$\epsilon>0$ the set $G=U(x;\rho;\epsilon)$ is an open set containing
$x$, so there is an $n_0$ such that $x_n\in G$ for every $n\ge n_0$,
that is, $\rho(x_n,x)<\epsilon$ for every $n\ge n_0$.   As $\epsilon$ is
arbitrary, $\lim_{n\to\infty}\rho(x_n,x)=0$.    {\bf (ii)} If the
condition is satisfied, take any open set $G$ containing $X$.   Then
there are $\rho_0,\ldots,\rho_k\in\Rho$ and $\delta>0$ such that
$U(x;\rho_0,\ldots,\rho_k;\delta)\subseteq G$.   For each $i\le k$ there
is an $n_i\in\Bbb N$ such that $\rho_i(x_n,x)<\delta$ for every $n\ge
n_i$.   Set $n^*=\max(n_0,\ldots,n_k)$;  then $x_n\in
U(x;\rho_0,\ldots,\rho_k;\delta)\subseteq G$ for every $n\ge n^*$.   As
$G$ is arbitrary, $\sequencen{x_n}\to x$.\ \Qed}

\spheader 2A3Md Let $(X,\rho)$ be a metric space, $A$ a subset of $X$
and $x\in X$.   Then $x\in\overline{A}$ iff there is a sequence in $A$
converging to $x$.   \prooflet{\Prf {\bf (i)} If $x\in\overline{A}$,
then for every $n\in\Bbb N$ we can choose a point
$x_n\in A\cap U(x;\rho;2^{-n})$
(2A3Kb);  now $\sequencen{x_n}\to x$.   {\bf (ii)} If $\sequencen{x_n}$
is a sequence in $A$ converging to $x$, then for every open set $G$
containing $x$ there is an $n$ such that $x_n\in G$, so that $A\cap
G\ne\emptyset$;  by 2A3Dc, $x\in\overline{A}$.\ \Qed}

\leader{2A3N}{Compactness} \cmmnt{ The next concept we need is the
idea of `compactness' in general topological spaces.

\header{2A3Na}}{\bf (a)} If $(X,\frak T)$ is any
topological space, a subset $K$ of $X$ is {\bf compact} if whenever
$\Cal G$ is a family in $\frak T$ covering $K$, then there is a finite
$\Cal G_0\subseteq\Cal G$ covering $K$.   \cmmnt{(Cf.\ 2A2D.  A {\bf
warning}:  many authors reserve the term `compact' for Hausdorff
spaces.)}   A set $A\subseteq X$ is {\bf relatively compact} in $X$ if
there is a compact subset of $X$ including $A$.   \cmmnt{({\bf Warning!}
in non-Hausdorff spaces, this is not the same thing as saying that
$\overline{A}$ is compact.)}

\header{2A3Nb}{\bf (b)}\cmmnt{ Just as in 2A2E-2A2G %2A2E 2A2F 2A2G
(and the proofs
are the same in the general case), we have the following results.

\medskip

\quad}{\bf (i)} If $K$ is compact and $E$ is closed, then $K\cap E$ is
compact.

\medskip

\quad {\bf (ii)} If $K\subseteq X$ is compact and $\phi:K\to Y$ is
continuous,
where $(Y,\frak S)$ is another topological space, then $\phi[K]$ is a
compact subset of $Y$.

\medskip

\quad{\bf (iii)} If $K\subseteq X$ is compact and $\phi:K\to\Bbb R$ is
continuous, then $\phi$ is bounded and attains its bounds.

\leader{2A3O}{Cluster points (a)}  If
$(X,\frak T)$ is a topological space, and $\sequencen{x_n}$ is a
sequence
in $X$, then a {\bf cluster point} of $\sequencen{x_n}$ is an $x\in X$
such that whenever $G$ is an open set containing $x$ and $n\in\Bbb N$
then there is a $k\ge n$ such that $x_k\in G$.

\header{2A3Ob}{\bf (b)} Now if $(X,\frak T)$ is
a topological space and $A\subseteq X$ is relatively compact, every
sequence $\sequencen{x_n}$ in $A$ has a cluster point in $X$.
\prooflet{\Prf\
Let $K$ be a compact subset of $X$ including $A$.    Set

\Centerline{$\Cal G=\{G:G\in\frak T,\,\{n:x_n\in G\}\text{ is
finite}\}$.}

\noindent\Quer\ If $\Cal G$ covers $K$, then there is a finite
$\Cal G_0\subseteq\Cal G$ covering $K$.   Now

\Centerline{$\Bbb N=\{n:x_n\in A\}=\{n:x_n\in\bigcup\Cal G_0\}
=\bigcup_{G\in\Cal G_0}\{n:x_n\in G\}$}

\noindent is a finite union of finite sets, which is absurd.   \Bang\
Thus $\Cal G$ does not cover $K$.
Take any $x\in K\setminus\bigcup\Cal G$.   If $G\in\frak T$ and $x\in G$
and $n\in\Bbb N$, then $G\notin\Cal G$ so $\{k:x_k\in G\}$ is infinite
and there is a $k\ge n$ such that $x_k\in G$.   Thus $x$ is a cluster
point of $\sequencen{x_n}$, as required.\ \Qed}

\cmmnt{
\leader{2A3P}{Filters} In $\BbbR^r$, and more generally in all
metrizable spaces, topological ideas can be effectively
discussed in terms of convergent sequences.   (To be sure, this
occasionally necessitates the use of a weak form of the axiom of choice,
in order to
choose a sequence;  but as measure theory without such choices is
changed utterly -- see Chapter 56 in Volume 5 -- there
is no point in fussing about them here.)   For
topological spaces in general, however, sequences are quite inadequate,
for very interesting reasons which I shall not enlarge upon.   Instead
we need to use `nets' or `filters'.   The latter take a moment's
more effort at the beginning, but are then (in my view) much easier to
work with, so I describe this method now.
}%end of comment

\leader{2A3Q}{Convergent filters (a)} Let $(X,\frak T)$ be a topological
space, $\Cal F$ a filter on $X$\cmmnt{ (see 2A1I)} and $x$ a point of
$X$.   We say that $\Cal F$ is
{\bf convergent} to $x$, or that $x$ is a {\bf limit} of $\Cal F$, and
write $\Cal F\to x$, if
every open set containing $x$ belongs to $\Cal F$.

\header{2A3Qb}{\bf (b)} Let $(X,\frak T)$ and $(Y,\frak S)$ be
topological spaces,
$\phi:X\to Y$ a continuous function, $x\in X$ and $\Cal F$ a filter on
$X$ converging to $x$.   Then $\phi[[\Cal F]]$\cmmnt{ (as defined in
2A1Ib)} converges to $\phi(x)$\prooflet{ (because $\phi^{-1}[G]$ is an
open set containing $x$ whenever $G$ is an
open set containing $\phi(x)$)}.

\leader{2A3R}{}\cmmnt{ Now we have the following characterization of
compactness.

\medskip

\noindent}{\bf Theorem} Let $X$ be a topological space, and $K$ a subset
of $X$.   Then $K$ is compact iff every ultrafilter on $X$  containing
$K$ has a limit in $K$.

\proof{{\bf (a)} Suppose that $K$ is compact and that $\Cal F$
is an ultrafilter on $X$ containing $K$.   Set

\Centerline{$\Cal G=\{G:G\subseteq X$ is open,
$X\setminus G\in\Cal F\}$.}

\noindent Then the union of any two members of $\Cal G$ belongs to $\Cal
G$, so the union of any finite number of members of $\Cal G$ belongs to
$\Cal G$;  also no member of $\Cal G$ can include $K$, because
$X\setminus K\notin\Cal F$.   Because $K$ is compact, it follows that
$\Cal G$ cannot cover $K$.   Let $x$ be any point of
$K\setminus\bigcup\Cal G$.   If $G$ is any open set containing $x$, then
$G\notin\Cal G$ so $X\setminus G\notin\Cal F$;  but this means that $G$
must belong to $\Cal F$, because $\Cal F$ is an ultrafilter.   As $G$ is
arbitrary, $\Cal F\to x$.
Thus every ultrafilter on $X$ containing $K$ has a limit in $K$.

\medskip

{\bf (b)} Now suppose that every ultrafilter on $X$ containing $K$ has a
limit in $K$.   Let $\Cal G$ be a cover of $K$ by open sets in $X$.
\Quer\ Suppose, if possible, that $\Cal G$ has no finite subcover.   Set

\Centerline{$\Cal F=\{F:$ there is a finite $\Cal G_0\subseteq
\Cal G,\,F\cup\bigcup\Cal G_0\supseteq K\}$.}

\noindent Then $\Cal F$ is a filter on $X$.   \Prf\ (i)
$X\cup\bigcup\emptyset\supseteq K$ so $X\in\Cal F$.


\Centerline{$\emptyset\cup\bigcup\Cal G_0=\bigcup\Cal G_0\not\supseteq
K$}

\noindent for any
finite $\Cal G_0\subseteq\Cal G$, by hypothesis, so
$\emptyset\notin\Cal F$.
(ii) If $E$, $F\in\Cal F$ there are finite sets $\Cal G_1$,
$\Cal G_2\subseteq\Cal G$ such that $E\cup\bigcup\Cal G_1$ and
$F\cup\bigcup\Cal G_2$ both include $K$;  now
$(E\cap F)\cup\bigcup(\Cal G_1\cup\Cal G_2)\supseteq K$ so
$E\cap F\in\Cal F$.   (iii) If
$X\supseteq E\supseteq F\in\Cal F$ then there is a finite
$\Cal G_0\subseteq\Cal G$ such that $F\cup\Cal G_0\supseteq K$;  now
$E\cup\bigcup\Cal G_0\supseteq K$ and $E\in\Cal F$.\ \Qed

By the Ultrafilter Theorem (2A1O), there is an ultrafilter $\Cal F^*$ on
$X$ including $\Cal F$.   Of course $K$ itself belongs to $\Cal F$, so
$K\in\Cal F^*$.   By hypothesis, $\Cal F^*$ has a limit $x\in K$.   But
now there is a set $G\in\Cal G$ containing $x$, and
$(X\setminus G)\cup G\supseteq K$, so
$X\setminus G\in\Cal F\subseteq\Cal F^*$;  which means
that $G$ cannot belong to $\Cal F^*$, and $x$ cannot be a limit of
$\Cal F^*$.   \Bang

So $\Cal G$ has a finite subcover.   As $\Cal G$ is arbitrary, $K$ must
be compact.
}%end of proof of 2A3R

\cmmnt{
\medskip

\noindent{\bf Remark} Note that part (b) of the proof of
this theorem depends vitally on the
Ultrafilter Theorem and therefore on the axiom of choice.
}

\leader{2A3S}{Further calculations with filters (a)}
I\cmmnt{n general, i}t is
possible for a filter to have more than one limit;  but in Hausdorff
spaces this does not occur.  \prooflet{\Prf\ (Compare 2A3Mb.)  If
$(X,\frak T)$ is Hausdorff, and $x$,
$y$ are distinct points of $X$, there are disjoint open sets $G$, $H$
such that $x\in G$ and $y\in H$.   If now a filter $\Cal F$ on $X$
converges to
$x$, $G\in\Cal F$ so $H\notin\Cal F$ and $\Cal F$ does not converge to
$y$.\ \Qed}

Accordingly we can safely write $x=\lim\Cal F$ when $\Cal F\to x$ in a
Hausdorff space.

\spheader 2A3Sb Now suppose that $X$ is a set, $\Cal F$ is a filter on
$X$, $(Y,\frak S)$ is a Hausdorff space, $D\in\Cal F$ and $\phi:D\to Y$
is a function.   Then we write $\lim_{x\to\Cal F}\phi(x)$ for
$\lim\phi[[\Cal F]]$ if this is defined in $Y$;  that is,
$\lim_{x\to\Cal F}\phi(x)=y$ iff $\phi^{-1}[H]\in\Cal F$ for every open
set $H$ containing $y$.

\dvAnew{2014}If $Z$ is another set, $\Cal G$ is a filter on $Z$, and
$\psi:Z\to X$ is such that $\Cal F=\psi[[\Cal G]]$, then 
the composition $\phi\psi$ is defined on $\psi^{-1}[D]\in\Cal G$, and
if one of the limits $\lim_{x\to\Cal F}\phi(x)$, 
$\lim_{z\to\Cal G}\phi\psi(z)$ is defined in $Y$ so is the other, and they
are then equal.   \prooflet{\Prf\ Suppose that $y\in Y$ and let 
$\Cal U$ be the family of open subsets of $Y$ containing $y$.   Then

$$\eqalign{\lim_{x\to\Cal F}\phi(x)=y
&\iff\phi^{-1}[G]\in\Cal F\text{ for every }G\in\Cal U
\cr&\iff\psi^{-1}[\phi^{-1}[G]]\in\Cal G\text{ for every }G\in\Cal U
\cr&\iff(\phi\psi)^{-1}[G]\in\Cal G\text{ for every }G\in\Cal U
\iff\lim_{z\to\Cal G}\phi\psi(z)=y. \text{ \Qed}\cr}$$}

In the special case $Y=\Bbb R$, $\lim_{x\to\Cal F}\phi(x)=a$ iff
$\{x:|\phi(x)-a|\le\epsilon\}\in\Cal F$ for every
$\epsilon>0$\prooflet{ (because every open set containing $a$ includes a
set of the form $[a-\epsilon,a+\epsilon]$, which in turn includes the
open set $\ooint{a-\epsilon,a+\epsilon}$)}.

\spheader 2A3Sc Suppose that $X$ and $Y$ are sets, $\Cal F$ is a filter
on $X$, $\Theta$ is a non-empty family
of pseudometrics on $Y$ defining a topology $\frak S$ on
$Y$, and
$\phi:X\to Y$ is a function.   Then the image filter $\phi[[\Cal F]]$
converges to $y\in Y$ iff $\lim_{x\to\Cal F}\theta(\phi(x),y)=0$ in
$\Bbb R$
for every $\theta\in\Theta$.   \prooflet{\Prf\ {\bf (i)} Suppose that
$\phi[[\Cal F]]\to
y$.   For every $\theta\in\Theta$ and $\epsilon>0$,
$U(y;\theta;\epsilon)=\{z:\theta(z,y)<\epsilon\}$ is an open set
containing $y$ (2A3G), so belongs to $\phi[[\Cal F]]$, and its inverse
image $\{x:0\le\theta(\phi(x),y)<\epsilon\}$ belongs to $\Cal F$.
As $\epsilon$ is arbitrary, $\lim_{x\to\Cal F}\theta(\phi(x),y)=0$.
As $\theta$ is arbitrary, $\phi$ satisfies the condition.   {\bf (ii)}
Now
suppose that $\lim_{x\to\Cal F}\theta(\phi(x),y)=0$ for every
$\theta\in\Theta$.   Let $G$ be any open set in $Y$ containing $y$.
Then there are $\theta_0,\ldots,\theta_n\in\Theta$ and $\epsilon>0$ such
that

\Centerline{$U(y;\theta_0,\ldots,\theta_n;\epsilon)
=\bigcap_{i\le n}U(y;\theta_i;\epsilon)\subseteq G$.}

\noindent For each $i\le n$,

\Centerline{$\phi^{-1}[U(y;\theta_i;\epsilon)]
=\{x:\theta(\phi(x),y)<\epsilon\}$}

\noindent belongs to $\Cal F$;  because $\Cal F$ is closed under finite
intersections, so do
$\phi^{-1}[U(y;\theta_0,\ldots,\theta_n;\epsilon)]$ and its superset
$\phi^{-1}[G]$.   Thus $G\in\phi[[\Cal F]]$.   As $G$ is arbitrary,
$\phi[[\Cal F]]\to y$.\ \Qed}

%2A3S

\spheader 2A3Sd In particular,\cmmnt{ taking $X=Y$ and $\phi$ the
identity map,} if
$X$ has a topology $\frak T$ defined by a non-empty family $\Rho$ of
pseudometrics, then a filter $\Cal F$ on $X$ converges to $x\in X$ iff
$\lim_{y\to\Cal F}\rho(y,x)=0$ for every $\rho\in\Rho$.

\spheader 2A3Se{\bf (i)} If $X$ is any set, $\Cal F$ is an ultrafilter
on $X$, $(Y,\frak S)$ is a Hausdorff space, and
$h:X\to Y$ is a function such that $h[F]$ is relatively compact in $Y$
for some $F\in\Cal F$, then $\lim_{x\to\Cal F}h(x)$ is defined in $Y$.
\prooflet{\Prf\ Let $K\subseteq Y$ be a compact set including $h[F]$.
Then $K\in h[[\Cal F]]$, which is an ultrafilter (2A1N), so
$h[[\Cal F]]$ has a limit in $Y$ (2A3R), which is
$\lim_{x\to\Cal F}h(x)$.\ \Qed}

\medskip

\quad{\bf (ii)} If $X$ is any set, $\Cal F$ is an ultrafilter on $X$,
and $h:X\to\Bbb R$ is a function such that $h[F]$ is bounded in $\Bbb R$
for some set $F\in\Cal F$, then $\lim_{x\to\Cal F}h(x)$ exists in
$\Bbb R$.
\prooflet{\Prf\ $\overline{h[F]}$ is closed and bounded, therefore
compact (2A2F), so $h[F]$ is relatively compact and we can use
(i).\ \Qed}

\spheader 2A3Sf \cmmnt{ The concepts of $\limsup$, $\liminf$ can be
applied to filters.}  Suppose that $\Cal F$ is a filter on a set $X$,
and that $f:X\to[-\infty,\infty]$ is any function.   Then

\dvro{$$\eqalign{\limsup_{x\to\Cal F}f(x)
&=\inf_{F\in\Cal F}\sup_{x\in F}f(x)
\in[-\infty,\infty],\cr}$$}
{$$\eqalign{\limsup_{x\to\Cal F}f(x)
&=\inf\{u:u\in[-\infty,\infty],\,\{x:f(x)\le u\}\in\Cal F\}\cr
&=\inf_{F\in\Cal F}\sup_{x\in F}f(x)
\in[-\infty,\infty],\cr}$$}

\dvro{$$\eqalign{\liminf_{x\to\Cal F}f(x)
&=\sup_{F\in\Cal F}\inf_{x\in F}f(x).\cr}$$}
{$$\eqalign{\liminf_{x\to\Cal F}f(x)
&=\sup\{u:u\in[-\infty,\infty],\,\{x:f(x)\ge u\}\in\Cal F\}\cr
&=\sup_{F\in\Cal F}\inf_{x\in F}f(x).\cr}$$}

\noindent\dvro{For}{It is easy to see that, for} any two functions $f$,
$g:X\to\Bbb R$,

\Centerline{$\lim_{x\to\Cal F}f(x)=a$\quad iff\quad
$a=\limsup_{x\to\Cal F}f(x)=\liminf_{x\to\Cal F}f(x)$,}

\noindent and

\Centerline{$\limsup_{x\to\Cal F}f(x)+g(x)
\le\limsup_{x\to\Cal F}f(x)+\limsup_{x\to\Cal F}g(x)$,}

\Centerline{$\liminf_{x\to\Cal F}f(x)+g(x)
\ge\liminf_{x\to\Cal F}f(x)+\liminf_{x\to\Cal F}g(x)$,}

\Centerline{$\liminf_{x\to\Cal F}(-f(x))=-\limsup_{x\to\Cal F}f(x)$,
\quad$\limsup_{x\to\Cal F}(-f(x))=-\liminf_{x\to\Cal F}f(x)$,}

\Centerline{$\liminf_{x\to\Cal F}cf(x)=c\liminf_{x\to\Cal F}f(x)$,
\quad$\limsup_{x\to\Cal F}cf(x)=c\limsup_{x\to\Cal F}f(x)$}

\noindent whenever the right-hand-sides are defined in
$[-\infty,\infty]$ and $c\ge 0$.   So if $a=\lim_{x\to\Cal F}f(x)$ and
$b=\lim_{x\to\Cal F}(x)$ exist in $\Bbb R$, $\lim_{x\to\Cal F}f(x)+g(x)$
exists and is equal to $a+b$ and $\lim_{x\to\Cal F}cf(x)$ exists and is
equal to $c\lim_{x\to\Cal F}f(x)$ for every $c\in\Bbb R$.

\dvro{If}{We also see that if} $f:X\to\Bbb R$ is such that

\Centerline{for every $\epsilon>0$ there is an $F\in\Cal F$ such that
$\sup_{x\in F}f(x)\le\epsilon+\inf_{x\in F}f(x)$,}

\noindent then
\cmmnt{$\limsup_{x\to\Cal F}f(x)\le\epsilon+\liminf_{x\to\Cal F}f(x)$
for every $\epsilon>0$, so that} $\lim_{x\to\Cal F}f(x)$ is defined in
$[-\infty,\infty]$.

\spheader 2A3Sg \cmmnt{Note that the standard limits of real analysis
can be represented in the form described here.   For instance,}
$\lim_{n\to\infty}$, $\limsup_{n\to\infty}$, $\liminf_{n\to\infty}$
correspond to $\lim_{n\to\CalFr}$, $\limsup_{n\to\CalFr}$,
$\liminf_{n\to\CalFr}$ where $\CalFr$ is the {\bf Fr\'echet filter}
on $\Bbb N$, the filter $\{\Bbb N\setminus A:A\subseteq\Bbb N$ is
finite$\}$ of cofinite subsets of $\Bbb N$.   Similarly,
$\lim_{\delta\downarrow a}$, $\limsup_{\delta\downarrow a}$,
$\liminf_{\delta\downarrow a}$ correspond to $\lim_{\delta\to\Cal F}$,
$\limsup_{\delta\to\Cal F}$, $\liminf_{\delta\to\Cal F}$ where

\Centerline{$\Cal F=\{A:A\subseteq\Bbb R,\,\exists\,h>0$ such that
$\ocint{a,a+h}\subseteq A\}$.}
%2A3S

\leader{2A3T}{Product topologies}\cmmnt{ We need some brief remarks
concerning topologies on product spaces.

\medskip

} {\bf (a)} Let $(X,\frak T)$ and $(Y,\frak S)$ be
topological spaces.
Let $\frak U$ be the set of subsets $U$ of $X\times Y$ such that for
every $(x,y)\in U$ there are $G\in\frak T$, $H\in\frak S$ such that
$(x,y)\in G\times H\subseteq U$.   Then $\frak U$ is a topology on
$X\times Y$.   \prooflet{\Prf\ (i) $\emptyset\in\frak U$ because the
condition for
membership of $\frak U$ is vacuously satisfied.   $X\times Y\in\frak U$
because $X\in\frak T$, $Y\in\frak S$ and $(x,y)\in X\times Y\subseteq
X\times Y$ for every $(x,y)\in X\times Y$.   (ii) If $U$, $V\in\frak U$
and $(x,y)\in U\cap V$, then there are $G$, $G'\in\frak T$, $H$,
$H'\in\frak S$ such that

\Centerline{$(x,y)\in G\times H\subseteq U$,
\quad $(x,y)\in G'\times H'\subseteq V$;}

\noindent now $G\cap G'\in \frak T$, $H\cap H'\in\frak S$ and

\Centerline{$(x,y)\in(G\cap G')\times(H\cap H')\subseteq U\cap V$.}

\noindent As $(x,y)$ is arbitrary, $U\cap V\in\frak U$.   (iii) If
$\Cal U\subseteq\frak U$ and $(x,y)\in\bigcup\Cal U$, then there is a
$U\in\Cal U$ such that $(x,y)\in U$;  now there are $G\in\frak T$,
$H\in\frak S$ such that $(x,y)\in G\times H\subseteq
U\subseteq\bigcup\Cal U$.   As $(x,y)$ is arbitrary, $\bigcup\Cal
U\in\frak U$.\ \Qed}

$\frak U$ is called the {\bf product topology} on $X\times Y$.


\spheader 2A3Tb Suppose, in (a), that $\frak T$ and $\frak S$ are
defined by
non-empty families $\Rho$, $\Theta$ of pseudometrics\cmmnt{ in the manner of 2A3F}.   Then $\frak U$ is defined by the family
$\Upsilon=\{\tilde\rho:\rho\in\Rho\}\cup\{\bar\theta:\theta\in\Theta\}$
of pseudometrics on $X\times Y$, where

\Centerline{$\tilde\rho((x,y),(x',y'))=\rho(x,x')$,
\quad$\bar\theta((x,y),(x',y'))=\theta(y,y')$}

\noindent whenever $x$, $x'\in X$, $y$, $y'\in Y$, $\rho\in\Rho$ and
$\theta\in\Theta$.

\prooflet{\Prf\ {\bf (i)} Of course you should check that every
$\tilde\rho$, $\bar\theta$ is a pseudometric on $X\times Y$.

\quad{\bf (ii)} If $U\in\frak U$ and $(x,y)\in U$, then there are
$G\in\frak T$, $H\in\frak S$ such that $(x,y)\in G\times H\subseteq U$.
There are $\rho_0,\ldots,\rho_m\in\Rho$,
$\theta_0,\ldots,\theta_n\in\Theta$, $\delta$, $\delta'>0$ such that (in
the language of 2A3Fc) $U(x;\rho_0,\ldots,\rho_m;\delta)\subseteq G$,
$U(x;\theta_0,\ldots,\theta_n;\delta)\subseteq H$.   Now

\Centerline{$U((x,y);\tilde\rho_0,\ldots,\tilde\rho_m,
\bar\theta_0,\ldots,\bar\theta_n;\min(\delta,\delta'))\subseteq U$.}

\noindent As $(x,y)$ is arbitrary, $U$ is open for the topology
generated by $\Upsilon$.

\quad{\bf (iii)} If $U\subseteq X\times Y$ is open for the topology
defined by $\Upsilon$, take any $(x,y)\in U$.   Then there are
$\upsilon_0,\ldots,\upsilon_k\in\Upsilon$ and $\delta>0$ such that
$U((x,y);\upsilon_0,\ldots,\upsilon_k;\delta)\subseteq U$.   Take
$\rho_0,\ldots,\rho_m\in\Rho$ and $\theta_0,\ldots,\theta_n\in\Theta$
such that $\{\upsilon_0,\ldots,\upsilon_k\}
\subseteq\{\tilde\rho_0,\ldots,\tilde\rho_m,
\bar\theta_0,\ldots,\bar\theta_n\}$;    then
$G=U(x;\rho_0,\ldots,\rho_m;\delta)\in\frak T$ (2A3G),
$H=U(y;\theta_0,\ldots,\theta_n;\delta)\in\frak S$, and

\Centerline{$(x,y)\in G\times H
=U((x,y);\tilde\rho_0,\ldots,\rho_m,
\bar\theta_0,\ldots,\bar\theta_n;\delta)
\subseteq U((x,y);\upsilon_0,\ldots,\upsilon_k;\delta)\subseteq U$.}

\noindent As $(x,y)$ is arbitrary, $U\in\frak U$.   This completes the
proof that $\frak U$ is the topology defined by $\Upsilon$.\ \Qed}

\spheader 2A3Tc In particular, the product topology on
$\BbbR^r\times\BbbR^s$ is the Euclidean topology if we identify
$\BbbR^r\times\BbbR^s$ with $\BbbR^{r+s}$.  \prooflet{\Prf\ The product
topology is defined by the two pseudometrics $\upsilon_1$, $\upsilon_2$,
where for $x$, $x'\in \BbbR^r$ and $y$, $y'\in\BbbR^s$ I write

\Centerline{$\upsilon_1((x,y),(x',y'))=\|x-x'\|$,
\quad$\upsilon_2((x,y),(x',y'))=\|y-y'\|$}

\noindent (2A3F(b-ii)).   Similarly, the Euclidean topology on
$\BbbR^r\times\BbbR^s\cong\BbbR^{r+s}$ is defined by the metric $\rho$,
where

\Centerline{$\rho((x,y),(x',y'))=\|(x-y)-(x',y')\|
=\sqrt{\|x-x'\|^2+\|y-y'\|^2}$.}

\noindent Now if $(x,y)\in\BbbR^r\times\BbbR^s$ and $\epsilon>0$, then

\Centerline{$U((x,y);\rho;\epsilon)
\subseteq U((x,y);\upsilon_j;\epsilon)$}

\noindent for both $j$, while

\Centerline{$U((x,y);\upsilon_1,\upsilon_2;\Bover{\epsilon}{\sqrt2})
\subseteq U((x,y);\rho;\epsilon)$.}

\noindent Thus, as remarked in 2A3Ib, each topology is included in the
other, and they are the same.\ \Qed}

\leader{2A3U}{Dense sets (a)} If $X$ is a topological space, a set
$D\subseteq X$ is {\bf dense} in $X$ if $\overline{D}=X$\cmmnt{, that
is, if every non-empty open set meets $D$}.   More
generally, if $D\subseteq A\subseteq X$, then $D$ is dense in $A$ if it
is dense for the subspace topology of $A$\cmmnt{ (2A3C), that is, if
$A\subseteq\overline{D}$}.   %for 253

\spheader 2A3Ub If $\frak T$ is defined by a non-empty family $\Rho$ of
pseudometrics on $X$, then $D\subseteq X$ is dense iff
\discrcenter{468pt}
{$U(x;\rho_0,\ldots,\rho_n;\delta)\cap D\ne\emptyset$ }whenever
$x\in X$, $\rho_0,\ldots,\rho_n\in\Rho$ and $\delta>0$.

\spheader 2A3Uc If $(X,\frak T)$, $(Y,\frak S)$ are topological spaces,
of which $Y$ is Hausdorff (in particular, if $(X,\rho)$
and $(Y,\theta)$ are metric spaces),
and $f$, $g:X\to Y$ are continuous functions
which agree on some dense subset $D$ of $X$, then $f=g$.
\prooflet{\Prf\Quer\ Suppose, if possible, that there is an $x\in X$
such that $f(x)\ne g(x)$.   Then there are open sets $G$, $H\subseteq Y$
such that $f(x)\in G$, $g(x)\in H$ and $G\cap H=\emptyset$.   Now
$f^{-1}[G]\cap g^{-1}[H]$ is an open set, containing $x$ and therefore
not empty, but it cannot meet $D$, so $x\notin\overline{D}$ and $D$ is
not dense.\ \Bang\QeD}

\spheader 2A3Ud A topological space is called {\bf separable} if it has
a countable dense subset.   \cmmnt{For instance, $\BbbR^r$ is
separable for every $r\ge 1$, since $\varBbb{Q}^r$ is dense.}

\discrpage

