\frfilename{mt214.tex}
\versiondate{22.5.09}
\copyrightdate{2009}

\def\chaptername{Taxonomy of measure spaces}
\def\sectionname{Subspaces}

\newsection{214}

In \S131 I described a construction for subspace measures on measurable
subsets.   It is now time to give the generalization to subspace
measures
on arbitrary subsets of a measure space.   The relationship between this
construction and the properties listed in \S211 is not quite as
straightforward as one might imagine, and in this section I try to give
a full account of what can be expected of subspaces in general.   I
think that for the present volume only (i) general subspaces of
$\sigma$-finite spaces and
(ii) measurable subspaces of general measure spaces will be
needed in any essential way, and these do not give any difficulty;  but
in later volumes we shall need the full theory.

I begin with a general construction for `subspace measures'
(214A-214C), %214A 214B 214C
with an account of integration with respect to a subspace
measure (214E-214G); %214E 214F 214G
these (with 131E-131H) %131E 131F 131G 131H
give a solid foundation for
the concept of `integration over a subset' (214D).   I present
this work in its full natural generality, which will eventually be
essential, but even for Lebesgue measure alone it is important to be
aware of the ideas here.    I continue with answers to some obvious
questions concerning subspace measures and the properties of measure
spaces so far considered, both for general subspaces (214I) and for
measurable subspaces (214K), and I mention a basic construction for
assembling measure spaces side-by-side, the `direct sums' of 214L-214M.
At the end of the section I discuss a measure extension problem
(214O-214P).

\leader{214A}{Proposition} Let $(X,\Sigma,\mu)$ be a measure space, and
$Y$ any subset of $X$.   Let $\mu^*$ be the outer measure defined from
$\mu$\cmmnt{ (132A-132B)}, and set $\Sigma_Y=\{E\cap Y:E\in\Sigma\}$;  let
$\mu_Y$ be the restriction of $\mu^*$ to $\Sigma_Y$.   Then
$(Y,\Sigma_Y,\mu_Y)$ is a measure space.

\proof{{\bf (a)} I have noted in 121A that $\Sigma_Y$ is a
$\sigma$-algebra of subsets of $Y$.

\medskip

{\bf (b)} Of course $\mu_YF\in[0,\infty]$ for
every $F\in\Sigma_Y$.

\medskip

{\bf (c)}  $\mu_Y\emptyset=\mu^*\emptyset=0$.


\medskip

{\bf (d)} If $\sequencen{F_n}$ is a disjoint sequence in $\Sigma_Y$ with
union $F$,  then choose $E_n$, $E'_n$, $E\in\Sigma$ such that
$F_n=Y\cap E_n$, $F_n\subseteq E'_n$ and $\mu_YF_n=\mu E'_n$ for each
$n$, $F\subseteq E$ and $\mu_YF=\mu E$ (using 132Aa repeatedly).   Set
$G_n=E_n\cap E'_n\cap E\setminus\bigcup_{m<n}E_m$ for each $n\in\Bbb N$;
then $\sequencen{G_n}$ is disjoint and $F_n\subseteq G_n\subseteq E'_n$
for each $n$, so $\mu_YF_n=\mu G_n$.   Also
$F\subseteq\bigcup_{n\in\Bbb N}G_n\subseteq E$, so

\Centerline{$\mu_YF=\mu(\bigcup_{n\in\Bbb N}G_n)
=\sum_{n=0}^{\infty}\mu G_n=\sum_{n=0}^{\infty}\mu_YF_n$.}

\noindent As $\sequencen{F_n}$ is
arbitrary, $\mu_Y$ is a measure.
}%end of proof of 214A

\leader{214B}{Definition} If $(X,\Sigma,\mu)$ is any measure space and
$Y$ is any subset of $X$, then $\mu_Y$, defined as in 214A, is the {\bf
subspace measure} on $Y$.

\cmmnt{It is worth noting the following.}

\leader{214C}{Lemma} Let $(X,\Sigma,\mu)$ be a measure space, $Y$ a
subset of $X$, $\mu_Y$ the subspace measure on $Y$ and $\Sigma_Y$ its
domain.  Then

(a) for any $F\in\Sigma_Y$, there is an $E\in\Sigma$ such that
$F=E\cap Y$ and $\mu E=\mu_YF$;

(b) for any $A\subseteq Y$, $A$ is $\mu_Y$-negligible iff it is
$\mu$-negligible;

(c)(i) if $A\subseteq X$ is $\mu$-conegligible, then $A\cap Y$
is $\mu_Y$-conegligible;

\quad (ii) if $A\subseteq Y$ is
$\mu_Y$-conegligible, then $A\cup(X\setminus Y)$ is $\mu$-conegligible;

(d) $(\mu_Y)^*$\cmmnt{, the outer measure on $Y$ defined from
$\mu_Y$,} agrees with $\mu^*$ on $\Cal PY$;

(e) if $Z\subseteq Y\subseteq X$, then
$\Sigma_Z=(\Sigma_Y)_Z$, the subspace $\sigma$-algebra of
subsets of $Z$ regarded as a subspace of $(Y,\Sigma_Y)$, and
$\mu_Z=(\mu_Y)_Z$ is the subspace measure on $Z$
regarded as a subspace of $(Y,\mu_Y)$;

(f) if $Y\in\Sigma$, then $\mu_Y$, as defined here, is exactly the
subspace measure on $Y$ defined in 131A-131B\cmmnt{;  that is,
$\Sigma_Y=\Sigma\cap\Cal PY$ and $\mu_Y=\mu\restr\Sigma_Y$}.

\proof{{\bf (a)} By the definition of $\Sigma_Y$, there is an
$E_0\in\Sigma$ such that $F=E_0\cap Y$.   By 132Aa, there is an
$E_1\in\Sigma$ such that $F\subseteq E_1$ and $\mu^*F=\mu E_1$.   Set
$E=E_0\cap E_1$;  this serves.

\medskip

{\bf (b)}(i) If $A$ is $\mu_Y$-negligible, there
is a set $F\in\Sigma_Y$ such that $A\subseteq F$ and $\mu_YF=0$;   now
$\mu^*A\le\mu^*F=0$ so $A$ is $\mu$-negligible, by 132Ad.   (ii)
If $A$ is
$\mu$-negligible, there is an $E\in\Sigma$ such that $A\subseteq E$ and
$\mu E=0$;  now $A\subseteq E\cap Y\in\Sigma_Y$ and $\mu_Y(E\cap Y)=0$,
so $A$ is $\mu_Y$-negligible.

\medskip

{\bf (c)} If $A\subseteq X$ is $\mu$-conegligible, then $A\cap Y$
is $\mu_Y$-conegligible, because $Y\setminus A=Y\cap(X\setminus A)$ is
$\mu$-negligible, therefore $\mu_Y$-negligible.   If $A\subseteq Y$ is
$\mu_Y$-conegligible, then $A\cup(X\setminus Y)$ is $\mu$-conegligible
because $X\setminus(A\cup(X\setminus Y))=Y\setminus A$ is
$\mu_Y$-negligible, therefore $\mu$-negligible.

\medskip

{\bf (d)} Let $A\subseteq Y$.   (i) If
$A\subseteq E\in\Sigma$, then $A\subseteq E\cap Y\in\Sigma_Y$, so
$\mu_Y^*A\le\mu_Y(E\cap Y)\le\mu E$;  as $E$ is arbitrary,
$\mu_Y^*A\le\mu^*A$.    (ii) If $A\subseteq F\in\Sigma_Y$, there is
an $E\in\Sigma$ such that $F\subseteq E$ and $\mu_YF=\mu^*F=\mu E$;
now $A\subseteq E$ so $\mu^*A\le\mu E=\mu_YF$.   As $F$ is arbitrary,
$\mu^*A\le\mu_Y^*A$.

\medskip

{\bf (e)} That $\Sigma_Z=(\Sigma_Y)_Z$ is immediate from the definition
of $\Sigma_Y$, etc.;   now

\Centerline{$(\mu_Y)_Z=
\mu_Y^*\restr\Sigma_Z=\mu^*\restr\Sigma_Z=\mu_Z$}

\noindent by (d).

\medskip

{\bf (f)}  This is elementary, because $E\cap Y\in\Sigma$ and
$\mu^*(E\cap Y)=\mu(E\cap Y)$ for every $E\in\Sigma$.
}%end of proof of 214C

\leader{214D}{Integration over subsets:  Definition} Let
$(X,\Sigma,\mu)$ be a measure
space, $Y$ a subset of $X$ and $f$ a $[-\infty,\infty]$-valued function
defined on a
subset of $X$.   By $\int_Y f$\cmmnt{ (or $\int_Y f(x)\mu(dx)$, etc.)}
I mean $\int(f\restrp Y)d\mu_Y$, if this exists in
$[-\infty,\infty]$\cmmnt{, following the definitions of
214A-214B, 133A and 135F, and taking $\dom(f\restrp Y)=Y\cap\dom f$,
$(f\restrp Y)(x)=f(x)$ for $x\in Y\cap\dom f$}.   \cmmnt{(Compare
131D.)}

\leader{214E}{Proposition} Let $(X,\Sigma,\mu)$ be a measure space,
$Y\subseteq X$, and $f$ a
$[-\infty,\infty]$-valued function defined on a subset $\dom f$ of $X$.

(a) If $f$ is $\mu$-integrable then $f\restrp Y$ is $\mu_Y$-integrable, and
$\int_Y f\le\int f$ if $f$ is non-negative.

(b) If $\dom f\subseteq Y$ and $f$ is $\mu_Y$-integrable, then there is
a $\mu$-integrable function $\tilde f$ on $X$, extending $f$, such that
$\int_F\tilde f=\int_{F\cap Y}f$ for every $F\in\Sigma$.

\proof{{\bf (a)(i)} If $f$ is $\mu$-simple, it is expressible as
$\sum_{i=0}^na_i\chi E_i$, where $E_0,\ldots,E_n\in\Sigma$,
$a_0,\ldots,a_n\in\Bbb R$ and $\mu E_i<\infty$ for each $i$.   Now
$f\restrp Y=\sum_{i=0}^na_i\chi_Y(E_i\cap Y)$, where $\chi_Y(E_i\cap
Y)=(\chi E_i)\restrp Y$ is the indicator function of $E_i\cap Y$
regarded as a subset of $Y$;  and each $E_i\cap Y$ belongs to
$\Sigma_Y$, with $\mu_Y(E_i\cap Y)\le\mu E_i<\infty$, so $f\restr
Y:Y\to\Bbb R$ is $\mu_Y$-simple.

If $f:X\to \Bbb R$ is a non-negative simple function, it is expressible
as $\sum_{i=0}^na_i\chi E_i$ where $E_0,\ldots,E_n$ are disjoint sets of
finite measure (122Cb).   Now
$f\restrp Y=\sum_{i=0}^na_i\chi_Y(E_i\cap Y)$ and

\Centerline{$\int (f\restrp Y)d\mu_Y
=\sum_{i=0}^na_i\mu_Y(E_i\cap Y)\le\sum_{i=0}^na_i\mu E_i=\int fd\mu$}

\noindent because $a_i\ge 0$ whenever $E_i\ne\emptyset$, so that
$a_i\mu_Y(E_i\cap Y)\le a_i\mu E_i$ for every $i$.

\medskip

\quad{\bf (ii)} If $f$ is a non-negative $\mu$-integrable function,
there is a non-decreasing sequence $\sequencen{f_n}$ of non-negative
$\mu$-simple functions converging to $f\,\,\mu$-almost everywhere;  now
$\sequencen{f_n\restrp Y}$ is a non-decreasing sequence of $\mu_Y$-simple
functions increasing to $f\restrp Y\,\,\mu_Y$-a.e.\ (by 214Cb), and

\Centerline{$\sup_{n\in\Bbb N}\int(f_n\restrp Y)d\mu_Y
\le\sup_{n\in\Bbb N}\int f_nd\mu
=\int fd\mu<\infty$,}

\noindent so $\int(f\restrp Y)d\mu_Y$ exists and is at most $\int fd\mu$.

\medskip

\quad{\bf (iii)} Finally, if $f$ is any $\mu$-integrable real-valued
function, it is expressible as $f_1-f_2$ where $f_1$ and $f_2$ are
non-negative $\mu$-integrable functions, so that
$f\restrp Y=(f_1\restrp Y)-(f_2\restrp Y)$ is $\mu_Y$-integrable.

\medskip

{\bf (b)} Let us say that if $f$ is a $\mu_Y$-integrable function, then
an  `enveloping extension' of $f$ is a $\mu$-integrable function
$\tilde f$, extending $f$, real-valued on $X\setminus Y$, such that
$\int_F\tilde f=\int_{F\cap Y}f$ for every $F\in\Sigma$.

\medskip

\quad{\bf (i)} If $f$ is of the form $\chi H$, where $H\in\Sigma_Y$ and
$\mu_YH<\infty$, let $E_0\in\Sigma$ be such that $H=Y\cap E_0$ and
$E_1\in\Sigma$ a measurable envelope for $H$ (132Ee);   then
$E=E_0\cap E_1$ is a measurable envelope for $H$ and $H=E\cap Y$.   Set
$\tilde f=\chi E$, regarded as a function from $X$ to $\{0,1\}$.   Then
$\tilde f\restrp Y=f$, and for any $F\in\Sigma$ we have

\Centerline{$\int_F\tilde f=\mu_F(E\cap F)=\mu(E\cap F)
=\mu^*(H\cap F)=\mu_{Y\cap F}(H\cap F)=\int_{Y\cap F}f$.}

\noindent So $\tilde f$ is an enveloping extension of $f$.

\medskip

\quad{\bf (ii)} If $f$, $g$ are $\mu_Y$-integrable functions with
enveloping extensions $\tilde f$, $\tilde g$, and $a$, $b\in\Bbb R$,
then $a\tilde f+b\tilde g$ extends $af+bg$ and

$$\eqalign{\int_Fa\tilde f+b\tilde g
&=a\int_F\tilde f+b\int_F\tilde g\cr
&=a\int_{F\cap Y}f+b\int_{F\cap Y}g
=\int_{F\cap Y}af+bg\cr}$$

\noindent for every $F\in\Sigma$, so $a\tilde f+b\tilde g$ is an
enveloping extension of $af+bg$.

\medskip

\quad{\bf (iii)} Putting (i) and (ii) together, we see that every
$\mu_Y$-simple function $f$ has an enveloping extension.

\medskip

\quad{\bf (iv)} Now suppose that $\sequencen{f_n}$ is a non-decreasing
sequence of non-negative $\mu_Y$-simple functions converging
$\mu_Y$-almost everywhere to a $\mu_Y$-integrable function $f$.   For
each $n\in\Bbb N$ let $\tilde f_n$ be an enveloping extension of $f_n$.
Then $\tilde f_n\leae\tilde f_{n+1}$.   \Prf\ If $F\in\Sigma$ then

\Centerline{$\int_F\tilde f_n=\int_{F\cap Y}f_n
\le\int_{F\cap Y}f_{n+1}=\int_F\tilde f_{n+1}$.}

\noindent So $\tilde f_n\leae\tilde f_{n+1}$, by 131Ha.\ \QeD\   Also

\Centerline{$\lim_{n\to\infty}\int_F\tilde f_n
=\lim_{n\to\infty}\int_{F\cap Y}f_n
=\int_{F\cap Y}f$}

\noindent for every $F\in\Sigma$.   Taking $F=X$ to begin with, B.Levi's
theorem tells us that $h=\lim_{n\to\infty}\tilde f_n$ is defined (as a
real-valued function) $\mu$-almost everywhere;  now letting $F$ vary, we
have $\int_Fh=\int_{F\cap Y}f$ for every $F\in\Sigma$, because
$h\restr F=\lim_{n\to\infty}\tilde f_n\restr F\,\,\mu_F$-a.e.   (I seem
to be using 214Cb here.)   Now $h\restrp Y=f\,\,\mu_Y$-a.e., by 214Cb
again.   If we define $\tilde f$ by setting

\Centerline{$\tilde f(x)=f(x)$ for $x\in\dom f$, $h(x)$ for
$x\in\dom h\setminus\dom f$, $0$ for other $x\in X$,}

\noindent then $\tilde f$ is defined everywhere in $X$ and is equal to
$h\,\,\mu$-almost everywhere;  so that if $F\in\Sigma$,
$\tilde f\restr F$ will be equal to $h\restr F\,\,\mu_F$-almost
everywhere, and

\Centerline{$\int_F\tilde f=\int_Fh=\int_{F\cap Y}f$.}

\noindent As $F$ is arbitrary, $\tilde f$ is an enveloping
extension of $f$.

\medskip

\quad{\bf (v)} Thus every non-negative $\mu_Y$-integrable function has
an enveloping extension.   Using (ii) again, every $\mu_Y$-integrable
function has an enveloping extension, as claimed.
}%end of proof of 214E

\leader{214F}{Proposition} Let $(X,\Sigma,\mu)$ be a measure space, $Y$
a subset of $X$, and $f$ a $[-\infty,\infty]$-valued function such that
$\int_Xf$ is defined in $[-\infty,\infty]$.   If {\it either} $Y$ has
full outer measure in $X$ {\it or} $f$ is zero almost everywhere in
$X\setminus Y$, then $\int_Yf$ is defined and equal to $\int_Xf$.

\proof{{\bf (a)} Suppose first that $f$ is non-negative,
$\Sigma$-measurable and defined everywhere in $X$.   In this case
$f\restrp Y$ is $\Sigma_Y$-measurable.   Set
$F_{nk}=\{x:x\in X,\,f(x)\ge 2^{-n}k\}$ for $k$, $n\in\Bbb N$,
$f_n=\sum_{k=1}^{4^n}2^{-n}\chi F_{nk}$ for $n\in\Bbb N$, so that
$\sequencen{f_n}$ is a non-decreasing sequence of real-valued measurable
functions converging everywhere to $f$, and
$\int_Xf=\lim_{n\to\infty}\int_Xf_n$.   For each $n\in\Bbb N$ and $k\ge 1$,

\Centerline{$\mu_Y(F_{nk}\cap Y)=\mu^*(F_{nk}\cap Y)=\mu F_{nk}$}

\noindent either because $F_{nk}\setminus Y$ is negligible or because
$X$ is a measurable envelope of $Y$.   So

$$\eqalignno{\int_Yf
&=\lim_{n\to\infty}\int_Yf_n
=\lim_{n\to\infty}\sum_{k=1}^{4^n}2^{-n}\mu_Y(F_{nk}\cap Y)\cr
&=\lim_{n\to\infty}\sum_{k=1}^{4^n}2^{-n}\mu F_{nk}
=\lim_{n\to\infty}\int_Xf_n
=\int_Xf.\cr}$$

\medskip

{\bf (b)} Now suppose that $f$ is non-negative, defined almost
everywhere in $X$ and $\mu$-virtually
measurable.   In this case there is a conegligible measurable set
$E\subseteq\dom f$ such that $f\restr E$ is measurable.   Set
$\tilde f(x)=f(x)$ for $x\in E$, $0$ for $x\in X\setminus E$;  then
$\tilde f$
satisfies the conditions of (a) and $f=\tilde f\,\,\mu$-a.e.
Accordingly $f\restrp Y=\tilde f\restrp Y\,\,\mu_Y$-a.e.\ (214Cc), and

\Centerline{$\int_Yf=\int_Y\tilde f=\int_X\tilde f=\int_Xf$.}

\medskip

{\bf (c)} Finally, for the general case, we can apply (b) to the
positive and negative parts $f^+$, $f^-$ of $f$ to get

\Centerline{$\int_Yf=\int_Yf^+-\int_Yf^-=\int_Xf^+-\int_Xf^-=\int_Xf$.}
}%end of proof of 214F

\leader{214G}{Corollary} Let $(X,\Sigma,\mu)$ be a measure space, $Y$ a
subset of $X$, and $E\in\Sigma$ a measurable envelope of $Y$.   If $f$
is a $[-\infty,\infty]$-valued function such that $\int_Ef$ is defined
in $[-\infty,\infty]$, then $\int_Yf$ is defined and equal to $\int_Ef$.

\proof{ By 214Ce, we can identify the subspace measure $\mu_Y$ with the
subspace measure $(\mu_E)_Y$ induced by the subspace measure on $E$.
Now, regarded as a subspace of $E$, $Y$ has full outer measure, so
214F gives the result.
}%end of proof of 214G

\leader{214H}{Subspaces and \Caratheodory's
\dvrocolon{method}}\cmmnt{ The following easy technical results will
occasionally be useful.

\medskip

\noindent}{\bf Lemma} Let $X$ be a set, $Y\subseteq X$ a subset, and
$\theta$ an outer measure on $X$.

(a) $\theta_Y=\theta\restrp\Cal PY$ is an outer measure on $Y$.

(b) Let $\mu$, $\nu$ be the measures on $X$, $Y$ defined by
\Caratheodory's method from the outer measures $\theta$, $\theta_Y$, and
$\Sigma$, $\Tau$ their domains;  let $\mu_Y$ be the subspace measure on
$Y$ induced by $\mu$, and $\Sigma_Y$ its domain.   Then

\quad (i) $\Sigma_Y\subseteq\Tau$ and $\nu F\le\mu_YF$ for every
$F\in\Sigma_Y$;

\quad(ii) if $Y\in\Sigma$ then $\nu=\mu_Y$;

\quad(iii) if $\theta=\mu^*$\cmmnt{ (that is, $\theta$ is
`regular')} then $\nu$ extends $\mu_Y$;

\quad(iv) if $\theta=\mu^*$ and $\theta Y<\infty$ then $\nu=\mu_Y$.

\proof{{\bf (a)} You have only to read the definition of `outer
measure' (113A).

\medskip

{\bf (b)(i)} Suppose that $F\in\Sigma_Y$.   Then it is of the form
$E\cap Y$ where $E\in\Sigma$.   If $A\subseteq Y$, then

\Centerline{$\theta_Y(A\cap F)+\theta_Y(A\setminus F)
=\theta(A\cap F)+\theta(A\setminus F)
=\theta(A\cap E)+\theta(A\setminus E)
=\theta A=\theta_YA$,}

\noindent so $F\in\Tau$.   Now

\Centerline{$\nu F=\theta_YF=\theta F\le\mu^*F=\mu_YF$.}

\medskip

\quad{\bf (ii)} Suppose that $F\in\Tau$.   If $A\subseteq X$, then

$$\eqalign{\theta A
&=\theta(A\cap Y)+\theta(A\setminus Y)
=\theta_Y(A\cap Y)+\theta(A\setminus Y)\cr
&=\theta_Y(A\cap Y\cap F)+\theta_Y(A\cap Y\setminus F)
  +\theta(A\setminus Y)\cr
&=\theta(A\cap F)+\theta(A\cap Y\setminus F)+\theta(A\setminus Y)\cr
&=\theta(A\cap F)+\theta((A\setminus F)\cap Y)
  +\theta((A\setminus F)\setminus Y)
=\theta(A\cap F)+\theta(A\setminus F);\cr}$$

\noindent as $A$ is arbitrary, $F\in\Sigma$ and therefore
$F\in\Sigma_Y$.
Also

\Centerline{$\mu_YF=\mu F=\theta F=\theta_YF=\nu F$.}
\noindent Putting this together with (i), we see that $\mu_Y$ and $\nu$
are identical.

\medskip

\quad{\bf (iii)} Let $F\in\Sigma_Y$.   Then $F\in\Tau$, by (i).
Now $\nu F=\theta F=\mu^*F=\mu_YF$.   As $F$ is arbitrary,
$\nu$ extends $\mu_Y$.

\medskip

\quad{\bf (iv)} Now suppose that $F\in\Tau$.   Because
$\mu^*Y=\theta Y<\infty$, we have measurable envelopes $E_1$, $E_2$ of
$F$ and $Y\setminus F$ for $\mu$ (132Ee).   Then

$$\eqalign{\theta Y
&=\theta_YY
=\theta_YF+\theta_Y(Y\setminus F)
=\theta F+\theta(Y\setminus F)\cr
&=\mu^*F+\mu^*(Y\setminus F)
=\mu E_1+\mu E_2\ge\mu(E_1\cup E_2)
=\theta(E_1\cup E_2)\ge\theta Y,\cr}$$

\noindent so $\mu E_1+\mu E_2=\mu(E_1\cup E_2)$ and

\Centerline{$\mu(E_1\cap E_2)=\mu E_1+\mu E_2-\mu(E_1\cup E_2)=0$.}

\noindent   As $\mu$ is complete (212A) and
$E_1\cap Y\setminus F\subseteq E_1\cap E_2$ is $\mu$-negligible,
therefore belongs to
$\Sigma$, $F=Y\cap(E_1\setminus(E_1\cap Y\setminus F))$ belongs to
$\Sigma_Y$.   Thus $\Tau\subseteq\Sigma_Y$;  putting this together
with (iii), we see that $\nu=\mu_Y$.
}%end of proof of 214H

\leader{214I}{}\cmmnt{ I now turn to the relationships between
subspace measures and the classification of measure spaces developed in
this chapter.

\medskip

\noindent}{\bf Theorem} Let $(X,\Sigma,\mu)$ be a measure
space and $Y$ a subset of $X$.   Let $\mu_Y$ be the subspace measure on
$Y$ and $\Sigma_Y$ its domain.

(a) If $(X,\Sigma,\mu)$ is complete, or totally finite, or
$\sigma$-finite, or strictly localizable, so is $(Y,\Sigma_Y,\mu_Y)$.
If $\langle X_i\rangle_{i\in I}$ is a decomposition of $X$ for $\mu$,
then $\langle X_i\cap Y\rangle_{i\in I}$ is a decomposition of $Y$ for
$\mu_Y$.

(b)
Writing $\hat\mu$ for the completion of $\mu$, the subspace measure
$\hat\mu_Y=(\hat\mu)_Y$ is the completion of $\mu_Y$.

(c) If $(X,\Sigma,\mu)$ has locally determined negligible sets,
then $\mu_Y$ is semi-finite.

(d) If $(X,\Sigma,\mu)$ is complete and locally determined, then
$(Y,\Sigma_Y,\mu_Y)$ is complete and semi-finite.

(e) If $(X,\Sigma,\mu)$ is complete, locally determined and localizable
then so is $(Y,\Sigma_Y,\mu_Y)$.

\proof{{\bf (a)(i)} Suppose that $(X,\Sigma,\mu)$ is complete.
If $A\subseteq U\in\Sigma_Y$ and $\mu_YU=0$, there is an $E\in\Sigma$
such that $U=E\cap Y$ and $\mu E=\mu_YU=0$;  now $A\subseteq E$ so
$A\in\Sigma$ and $A=A\cap Y\in\Sigma_Y$.

\medskip

\quad{\bf (ii)}
$\mu_YY=\mu^*Y\le\mu X$, so $\mu_Y$ is totally finite if $\mu$ is.

\medskip

\quad{\bf (iii)} If $\sequencen{X_n}$ is a sequence of sets of finite
measure for $\mu$ which covers $X$, then $\sequencen{X_n\cap Y}$ is a
sequence of sets of finite measure for $\mu_Y$ which covers $Y$.   So
$(Y,\Sigma_Y,\mu_Y)$ is $\sigma$-finite if $(X,\Sigma,\mu)$ is.

\medskip

\quad{\bf (iv)} Suppose that $\langle X_i\rangle_{i\in I}$ is a
decomposition of $X$ for $\mu$.   Then $\langle X_i\cap
Y\rangle_{i\in I}$ is a decomposition of $Y$ for $\mu_Y$.   \Prf\
Because $\mu_Y(X_i\cap Y)\le\mu X_i<\infty$ for each $i$,
$\langle X_i\cap Y\rangle_{i\in I}$ is a partition of $Y$ into sets of finite measure.   Suppose that $U\subseteq Y$ is such that
$U_i=U\cap X_i\cap Y\in\Sigma_Y$ for every $i$.   For each $i\in I$, choose $E_i\in\Sigma$
such that $U_i=E_i\cap Y$ and $\mu E_i=\mu_YU_i$;  we may of course
suppose that $E_i\subseteq X_i$.   Set $E=\bigcup_{i\in I}E_i$.   Then
$E\cap X_i=E_i\in\Sigma$ for every $i$, so $E\in\Sigma$ and
$\mu E=\sum_{i\in I}\mu E_i$.   Now $U=E\cap Y$ so $U\in\Sigma_Y$ and

\Centerline{$\mu_YU\le\mu E=\sum_{i\in I}\mu E_i
=\sum_{i\in I}\mu_YU_i$.}

\noindent On the other hand, $\mu_YU$ is surely greater than or equal to
$\sum_{i\in I}\mu_YU_i
=\sup_{J\subseteq I\text{ is finite}}\sum_{i\in J}\mu_YU_i$, so they are
equal.   As $U$ is arbitrary,
$\langle X_i\cap Y\rangle_{i\in I}$ is a decomposition of $Y$ for
$\mu_Y$.\ \Qed

Consequently $(Y,\Sigma_Y,\mu_Y)$ is strictly localizable if
$(X,\Sigma,\mu)$ is.

\medskip

{\bf (b)} The domain of the completion $(\mu_Y)\sphat\mskip5mu$ is

$$\eqalignno{\hat\Sigma_Y
&=\{F\symmdiff A:F\in\Sigma_Y,\,A\subseteq Y
   \text{ is }\mu_Y\text{-negligible}\}\cr
&=\{(E\cap Y)\symmdiff(A\cap Y):E\in\Sigma,\,
   A\subseteq X\text{ is }\mu\text{-negligible}\}\cr
\displaycause{214Cb}
&=\{(E\symmdiff A)\cap Y:E\in\Sigma,\,
   A\text{ is }\mu\text{-negligible}\}
=\dom\hat\mu_Y.\cr}$$

\noindent If $H\in\hat\Sigma_Y$ then

\Centerline{$(\mu_Y)\sphat\mskip3mu(H)
=\mu_Y^*H=\mu^*H=(\hat\mu)^*H=\hat\mu_YH$,}

\noindent using 214Cd for the second step, and 212Ea for the third.

\medskip

{\bf (c)} Take $U\in\Sigma_Y$ such
that $\mu_YU>0$.   Then there is an $E\in\Sigma$ such that
$\mu E<\infty$
and $\mu^*(E\cap U)>0$.   \Prf\Quer\ Otherwise, $E\cap U$ is
$\mu$-negligible whenever $\mu E<\infty$;  because $\mu$ has locally determined negligible sets, $U$ is $\mu$-negligible and $\mu_YU=\mu^*U=0$.\ \Bang\QeD   Now
$E\cap U\in\Sigma_Y$ and

\Centerline{$0<\mu^*(E\cap U)=\mu_Y(E\cap U)\le\mu E<\infty$.}

\medskip

{\bf (d)} By (a), $\mu_Y$ is complete;  by 213J and (c) here, it is
semi-finite.

\medskip

{\bf (e)} By (d), $\mu_Y$ is complete and semi-finite.
To see that it is locally
determined, take any $U\subseteq Y$ such that $U\cap V\in\Sigma_Y$
whenever $V\in\Sigma_Y$ and $\mu_YV<\infty$.   By 213J and 213L,
there is a
measurable envelope $E$ of $U$ for $\mu$;  of course
$E\cap Y\in\Sigma_Y$.

I claim that $\mu(E\cap Y\setminus U)=0$.   \Prf\ Take any $F\in\Sigma$
with $\mu F<\infty$.   Then $F\cap U\in\Sigma_Y$, so
\Centerline{$\mu_Y(F\cap E\cap Y)\le\mu(F\cap E)=\mu^*(F\cap
U)=\mu_Y(F\cap U)\le\mu_Y(F\cap E\cap Y)$;}

\noindent thus $\mu_Y(F\cap E\cap Y)=\mu_Y(F\cap U)$ and

\Centerline{$\mu^*(F\cap E\cap Y\setminus U)
=\mu_Y(F\cap E\cap Y\setminus U)=0$.}

\noindent Because $\mu$ is complete, $\mu(F\cap E\cap Y\setminus U)=0$;
because $\mu$ is locally determined and $F$ is arbitrary,
$\mu(E\cap Y\setminus U)=0$.\ \QeD\  But this means that
$E\cap Y\setminus U\in\Sigma_Y$ and $U\in\Sigma_Y$.   As $U$ is
arbitrary, $\mu_Y$ is locally determined.

To see that $\mu_Y$ is localizable,
let $\Cal U$ be any family in $\Sigma_Y$.   Set

\Centerline{$\Cal E=\{E:E\in\Sigma,\,\mu E<\infty,\,
  \mu E=\mu^*(E\cap U)$ for some $U\in\Cal U\}$,}

\noindent and let $G\in\Sigma$ be an essential supremum for $\Cal E$ in
$\Sigma$.   I claim that $G\cap Y$ is an essential supremum for $\Cal U$
in $\Sigma_Y$.   \Prf\ (i) \Quer\ If $U\in\Cal U$ and
$U\setminus(G\cap Y)$ is not negligible, then (because $\mu_Y$ is
semi-finite) there is a
$V\in\Sigma_Y$ such that $V\subseteq U\setminus G$ and
$0<\mu_YV<\infty$.   Now there is an $E\in\Sigma$ such that
$V\subseteq E$ and $\mu E=\mu^*V$.
We have $\mu^*(E\cap U)\ge\mu^*V=\mu E$, so $E\in\Cal E$ and
$E\setminus G$
must be negligible;  but $V\subseteq E\setminus G$ is not negligible.\
\BanG\   Thus $U\setminus(G\cap Y)$ is negligible for every
$U\in\Cal U$.
(ii)  If $W\in\Sigma_Y$ is such that $U\setminus W$ is negligible for
every $U\in\Cal U$, express $W$ as $H\cap Y$ where $H\in\Sigma$.   If
$E\in\Cal E$, there is a $U\in\Cal U$ such that $\mu E=\mu^*(E\cap U)$;
now $\mu^*(E\cap U\setminus W)=0$, so
$\mu E=\mu^*(E\cap U\cap W)\le\mu(E\cap H)$ and $E\setminus H$ is
negligible.   As $E$ is
arbitrary, $H$ is an essential upper bound for $\Cal E$ and
$G\setminus H$ is negligible;  but this means that $G\cap Y\setminus W$
is negligible.   As $W$ is arbitrary, $G\cap Y$ is an essential supremum
for $\Cal U$.\ \Qed

As $\Cal U$ is arbitrary, $\mu_Y$ is localizable.
}%end of proof of 214I

\leader{214J}{Upper and
lower \dvrocolon{integrals}}\cmmnt{ The
following elementary facts are sometimes useful.

\medskip

\noindent}{\bf Proposition} Let $(X,\Sigma,\mu)$ be a measure space,
$A$ a subset of $X$ and $f$ a real-valued function defined almost
everywhere in $X$.   Then

(a)  if {\it either} $f$ is non-negative {\it or}
$A$ has full outer measure in $X$,
$\overline{\intop}(f\restr A)d\mu_A\le\overline{\intop}fd\mu$;

(b) if $A$ has full outer measure in $X$,
$\underline{\int}fd\mu\le\underline{\int}(f\restr A)d\mu_A$.

\proof{{\bf (a)(i)} Suppose that $f$ is non-negative.
If $\overline{\intop}fd\mu=\infty$, the result is trivial.
Otherwise, there is a $\mu$-integrable function $g$ such that
$f\le g\,\,\mu$-a.e.\ and $\overline{\intop}fd\mu=\int g\,d\mu$, by
133J(a-i).   Now $f\restr A\le g\restr A\,\,\mu_A$-a.e., by 214Cb, and
$\int(g\restr A)\,d\mu_A$
is defined and less than or equal to $\int g\,d\mu$, by 214Ea;  so

\Centerline{$\overlineint(f\restr A)d\mu_A\le\int(g\restr A)d\mu_A
\le\int g\,d\mu=\overlineint fd\mu$.}

\medskip

\quad{\bf (ii)} Now suppose that $A$ has full outer measure in $X$.
If $g$ is such that $f\le g\,\,\mu$-a.e.\ and $\int g\,d\mu$ is defined in
$[-\infty,\infty]$, then $f\restr A\le g\restr A\,\,\mu_A$-a.e.\ and
$\int(g\restr A)d\mu_A=\int g\,d\mu$, by 214F.   So
$\overline{\int}(f\restr A)d\mu_A\le\int g\,d\mu$.   As $g$ is arbitrary,
$\overline{\int}(f\restr A)d\mu_A\le\overline{\int}f\,d\mu$.

\medskip

{\bf (b)} Apply (a) to $-f$, and use 133J(b-iv).
}%end of proof of 214J

\leader{214K}{Measurable subspaces:  Proposition}
Let $(X,\Sigma,\mu)$ be a measure space.

(a) Let $E\in\Sigma$ and let $\mu_E$ be the subspace measure, with
$\Sigma_E$ its domain.   If $(X,\Sigma,\mu)$ is complete, or totally
finite, or $\sigma$-finite, or strictly localizable, or semi-finite, or
localizable, or locally determined, or atomless, or purely atomic, so is
$(E,\Sigma_E,\mu_E)$.

(b) Suppose that $\langle X_i\rangle_{i\in I}$ is a partition of
$X$ into measurable sets\cmmnt{ (not necessarily of finite measure)} such that

\Centerline{$\Sigma=\{E:E\subseteq X,\,
E\cap  X_i\in\Sigma$ for every $i\in I\}$,}

\Centerline{$\mu E=\sum_{i\in I}\mu(E\cap X_i)$ for every $E\in\Sigma$.}
\noindent Then $(X,\Sigma,\mu)$ is complete, or strictly localizable, or
semi-finite, or localizable, or locally determined, or atomless, or
purely atomic, iff $(X_i,\Sigma_{X_i},\mu_{X_i})$ has that property for
every
$i\in I$.

\proof{ I really think that if you have read attentively up
to this point, you ought to find this easy.   If you are in any doubt,
this makes a very suitable set of sixteen exercises to do.
}%end of proof of 214K

\leader{214L}{Direct sums} Let
$\langle(X_i,\Sigma_i,\mu_i)\rangle_{i\in I}$ be any indexed family of
measure spaces.   Set
$X=\bigcup_{i\in I}(X_i\times\{i\})$;  for $E\subseteq X$, $i\in I$ set
$E_i=\{x:(x,i)\in E\}$.   Write

\Centerline{$\Sigma
=\{E:E\subseteq X,\,E_i\in\Sigma_i$ for every $i \in I\}$,}

\Centerline{$\mu E=\sum_{i\in I}\mu_iE_i$ for every $E\in\Sigma$.}

\noindent Then it is easy to check that $(X,\Sigma,\mu)$ is a measure
space;  I will call it the {\bf direct sum} of the family
$\familyiI{(X_i,\Sigma_i,\mu_i)}$.   Note that if $(X,\Sigma,\mu)$ is
any strictly localizable measure space, with decomposition
$\familyiI{X_i}$,
then we have a natural isomorphism between $(X,\Sigma,\mu)$ and the
direct sum
$(X',\Sigma',\mu')=\bigoplus_{i\in I}(X_i,\Sigma_{X_i},\mu_{X_i})$ of
the subspace measures, if we match $(x,i)\in X'$ with $x\in X$ for every
$i\in I$ and $x\in X_i$.

\cmmnt{For some of the elementary properties (to put
it plainly, I know of no properties which are not elementary) of direct
sums, see 214M and 214Xh-214Xk.}
% 214Xh, 214Xi, 214Xj, 214Xk

\leader{214M}{Proposition}
Let $\familyiI{(X_i,\Sigma_i,\mu_i)}$ be a
family of measure spaces, with direct sum $(X,\Sigma,\mu)$.   Let $f$ be
a real-valued function defined on a subset of $X$.   For each $i\in I$,
set $f_i(x)=f(x,i)$ whenever $(x,i)\in\dom f$.

(a) $f$ is measurable iff $f_i$ is measurable for every $i\in I$.

(b) If $f$ is non-negative, then
$\int fd\mu=\sum_{i\in I}\int f_id\mu_i$ if either is defined in
$[0,\infty]$.

\proof{{\bf (a)} For $a\in\Bbb R$, set
$F_a=\{(x,i):(x,i)\in\dom f,\,f(x,i)\ge a\}$.   (i) If $f$ is
measurable, $i\in I$ and $a\in\Bbb R$, then there is an $E\in\Sigma$
such that $F_a=E\cap\dom f$;  now

\Centerline{$\{x:f_i(x)\ge a\}
=\dom f_i\cap\{x:(x,i)\in E\}$}

\noindent belongs to the subspace $\sigma$-algebra on $\dom f_i$ induced
by $\Sigma_i$.   As $a$ is arbitrary, $f_i$ is measurable.   (ii) If
every $f_i$ is measurable and $a\in\Bbb R$, then for each $i\in I$ there
is an $E_i\in\Sigma_i$ such that $\{x:(x,i)\in F_a\}=E_i\cap\dom f$;
setting $E=\{(x,i):i\in I,\,x\in E_i\}$, $F_a=\dom f\cap E$ belongs to
the subspace $\sigma$-algebra on $\dom f$.   As $a$ is arbitrary, $f$ is
measurable.

\medskip

{\bf (b)(i)} Suppose first that $f$ is measurable and defined
everywhere.   Set $F_{nk}=\{(x,i):(x,i)\in X,\,f(x,i)\ge 2^{-n}k\}$ for
$k$, $n\in\Bbb N$, $g_n=\sum_{k=1}^{4^n}2^{-n}\chi F_{nk}$ for
$n\in\Bbb N$, $F_{nki}=\{x:(x,i)\in F_{nk}\}$ for $k$, $n\in\Bbb N$ and
$i\in I$, $g_{ni}(x)=g_n(x,i)$ for $i\in I$, $x\in X_i$.   Then

$$\eqalign{\int fd\mu
&=\lim_{n\to\infty}\int g_nd\mu
=\sup_{n\in\Bbb N}\sum_{k=1}^{4^n}2^{-n}\mu F_{nk}\cr
&=\sup_{n\in\Bbb N}\sum_{k=1}^{4^n}\sum_{i\in I}2^{-n}\mu F_{nki}
=\sum_{i\in I}\sup_{n\in\Bbb N}\sum_{k=1}^{4^n}2^{-n}\mu F_{nki}\cr
&=\sum_{i\in I}\sup_{n\in\Bbb N}\int g_{ni}d\mu_i
=\sum_{i\in I}\int f_id\mu_i.\cr}$$

\medskip

\quad{\bf (ii)} Generally, if $\int fd\mu$ is defined, there are a
measurable $g:X\to\coint{0,\infty}$ and a conegligible measurable set
$E\subseteq\dom f$ such that $g=f$ on $E$.   Now
$E_i=\{x:(x,i)\in X_i\}$ belongs to $\Sigma_i$ for each $i$, and
$\sum_{i\in I}\mu_i(X_i\setminus E_i)=\mu(X\setminus E)=0$, so $E_i$ is
$\mu_i$-conegligible for every $i$.   Setting $g_i(x)=g(x,i)$ for
$x\in X_i$, (i) tells us that

\Centerline{$\sum_{i\in I}\int f_id\mu_i
=\sum_{i\in I}\int g_id\mu_i=\int g\,d\mu=\int fd\mu$.}

\medskip

\quad{\bf (iii)} On the other hand, if $\int f_id\mu_i$ is defined for
every $i\in I$, then for each $i\in I$ we can find a measurable function
$g_i:X_i\to\coint{0,\infty}$ and a $\mu_i$-conegligible measurable set
$E_i\subseteq\dom f_i$ such that $g_i=f_i$ on $E_i$.   Setting
$g(x,i)=g_i(x)$ for $i\in I$, $x\in X_i$, (a) tells us that $g$ is
measurable, while $g=f$ on $\{(x,i):i\in I,\,x\in E_i\}$, which is
conegligible (by the calculation in (ii) just above);  so

\Centerline{$\int fd\mu=\int g\,d\mu
=\sum_{i\in I}\int g_id\mu_i=\sum_{i\in I}\int f_id\mu_i$,}

\noindent again using (i) for the middle step.
}%end of proof of 214M

\leader{214N}{Corollary}
Let $(X,\Sigma,\mu)$ be a measure space with a
decomposition $\familyiI{X_i}$.   If $f$ is a real-valued function
defined on a subset of $X$, then

(a) $f$ is measurable iff $f\restr X_i$ is measurable for every
$i\in I$,

(b) if $f\ge 0$, then $\int f=\sum_{i\in I}\int_{X_i}f$ if either is
defined in $[0,\infty]$.

\proof{ Apply 214M to the direct sum of
$\familyiI{(X_i,\Sigma_{X_i},\mu_{X_i})}$, identified with
$(X,\Sigma,\mu)$ as in 214L.
}%end of proof of 214N

\leader{*214O}{}\cmmnt{ I make space here for a general theorem
which puts rather heavy demands on the reader.   So I ought to say that I
advise skipping it on first reading.   It
will not be quoted in this volume, in
the full form here I do not expect to use it anywhere in this treatise,
only the special case of 214Xm is at all often applied, and the proof
depends on a concept (`ideal of sets') and a technique
(`transfinite induction', part (d) of the proof of 214P) which are
used nowhere else in this volume.   However,
`extension of measures' is one of the central themes of Volume 4, and this
result may help to make sense of some of the patterns which will
appear there.

\medskip

\noindent}{\bf Lemma} Let $(X,\Sigma,\mu)$ be a measure space, and $\Cal I$
an ideal of subsets of $X$, that is, a family of subsets of $X$ such that
$\emptyset\in\Cal I$, $I\cup J\in\Cal I$ for all $I$, $J\in\Cal I$, and
$I\in\Cal I$ whenever $I\subseteq J\in\Cal I$.   Then there is a measure
$\lambda$ on $X$ such that $\Sigma\cup\Cal I\subseteq\dom\lambda$,
$\mu E=\lambda E+\sup_{I\in\Cal I}\mu^*(E\cap I)$ for every $E\in\Sigma$,
and $\lambda I=0$ for every $I\in\Cal I$.

\proof{{\bf (a)} Let $\Lambda$ be the set of those $F\subseteq X$ such that
there are $E\in\Sigma$ and a countable $\Cal J\subseteq\Cal I$ such that
$E\symmdiff F\subseteq\bigcup\Cal J$.   Then $\Lambda$ is a $\sigma$-algebra
of subsets of $X$ including $\Sigma\cup\Cal I$.   \Prf\
$\Sigma\subseteq\Lambda$ because
$E\symmdiff E\subseteq\bigcup\emptyset$ for every $E\in\Sigma$.   $\Cal
I\subseteq\Lambda$ because $\emptyset\symmdiff I\subseteq\bigcup\{I\}$ for
every $I\in\Cal I$.   In particular, $\emptyset\in\Lambda$.   If $F\in\Lambda$,
let $E\in\Sigma$ and $\Cal J\subseteq\Cal I$ be such that $\Cal J$ is
countable and $F\symmdiff E\subseteq\bigcup\Cal J$;  then
$(X\setminus F)\symmdiff(X\setminus E)\subseteq\bigcup\Cal J$ so
$X\setminus F\in\Lambda$.   If $\sequencen{F_n}$ is a sequence in $\Lambda$ with
union $F$, then for each $n\in\Bbb N$ choose $E_n\in\Sigma$,
$\Cal J_n\subseteq\Cal I$ such that $\Cal J_n$ is countable and
$E_n\symmdiff F_n\subseteq\bigcup\Cal J_n$;  then
$E=\bigcup_{n\in\Bbb N}E_n$ belongs to $\Sigma$,
$\Cal J=\bigcup_{n\in\Bbb N}\Cal J_n$ is a countable subset of $\Cal I$ and
$E\symmdiff F\subseteq\bigcup\Cal J$, so $F\in\Sigma$.   Thus $\Lambda$ is a
$\sigma$-algebra.\ \Qed

\medskip

{\bf (b)} For $F\in\Lambda$ set

\Centerline{$\lambda F=\sup\{\mu E:E\in\Sigma$,
$E\subseteq F$, $\mu^*(E\cap I)=0$ for every $I\in\Cal I\}$.}

\noindent Then $\lambda$ is a measure.   \Prf\ The only subset of
$\emptyset$ is $\emptyset$, so $\lambda\emptyset=0$.   Let
$\sequencen{F_n}$ be a disjoint sequence in $\Lambda$ with union $F$;
set $u=\sum_{n=0}^{\infty}\lambda F_n$.
(i) If $E\in\Sigma$, $E\subseteq F$ and $\mu^*(E\cap I)=0$ for every
$I\in\Cal I$, then for each $n$ set $E_n=E\cap F_n$.   As
$\mu^*(E_n\cap I)=0$ for every $I\in\Cal I$, $\mu E_n\le\lambda F_n$
for each $n$.   Now
$\sequencen{E_n}$ is disjoint and has union $E$, so

\Centerline{$\mu E
=\sum_{n=0}^{\infty}\mu E_n\le\sum_{n=0}^{\infty}\lambda F_n=u$.}

\noindent As $E$ is arbitrary,
$\lambda F\le u$.   (ii) Take any $\gamma<u$.   For $n\in\Bbb N$, set
$\gamma_n=\lambda F_n-2^{-n-1}\min(1,u-\gamma)$ if $\lambda F_n$ is finite,
$\gamma$ otherwise.   For each $n$, we
can find an $E_n\in\Sigma$ such that $E_n\subseteq F_n$,
$\mu^*(E_n\cap I)=0$
for every $I\in\Cal I$, and $\mu E_n\ge\gamma_n$.   Set
$E=\bigcup_{n\in\Bbb N}E_n$;  then $E\subseteq F$ and
$E\cap I=\bigcup_{n\in\Bbb N}E_n\cap I$ is $\mu$-negligible for every
$I\in\Cal I$, so $\lambda F\ge\mu E=\sum_{n=0}^{\infty}\mu E_n\ge\gamma$.
As $\gamma$ is arbitrary, $\lambda F\ge u$.   (iii) As
$\sequencen{F_n}$ is arbitrary, $\lambda$ is a measure.\ \Qed

\medskip

{\bf (c)} Now take any $E\in\Sigma$ and set
$u=\sup_{I\in\Cal I}\mu^*(E\cap I)$.   If $u=\infty$ then
we certainly have
$\mu E=\infty=\lambda E+u$.
Otherwise, let $\sequencen{I_n}$ be a sequence in $\Cal I$ such that
$\lim_{n\to\infty}\mu^*(E\cap I_n)=u$;  replacing $I_n$ by
$\bigcup_{m\le n}I_m$ for each $n$ if necessary, we may suppose that
$\sequencen{I_n}$ is non-decreasing.   Set
$A=E\cap\bigcup_{n\in\Bbb N}I_n$;   because $E\cap I_n$ has finite outer
measure for each $n$, $A$ can be covered by a sequence of sets of finite
measure, and has a measurable envelope $H$ for $\mu$ included in
$E$ (132Ee).   Observe that

\Centerline{$\mu H=\mu^*A=\sup_{n\in\Bbb N}\mu^*(E\cap I_n)=u$}

\noindent by 132Ae.

Set $G=E\setminus H$.   Then $\mu^*(G\cap I)=0$ for every $I\in\Cal I$.
\Prf\ For any $n\in\Bbb N$ there is an $F\in\Sigma$
such that $F\supseteq E\cap(I_n\cup I)$ and $\mu F\le u$;  in which case

\Centerline{$\mu^*(G\cap I)+\mu^*(E\cap I_n)
\le\mu(F\setminus H)+\mu(F\cap H)\le u$.}

\noindent As $n$ is arbitrary, $\mu^*(G\cap I)=0$.\ \QeD\   Accordingly

\Centerline{$u+\lambda E\ge\mu H+\mu G=\mu E$.}

\noindent On the other hand, if $F\in\Sigma$ is such that $F\subseteq E$
and $\mu^*(F\cap I)=0$ for every $I\in\Cal I$, then

\Centerline{$\mu^*(E\cap I_n)
\le\mu(E\setminus F)+\mu^*(F\cap I_n)=\mu(E\setminus F)$}

\noindent for every $n$, so

\Centerline{$u+\mu F\le\mu(E\setminus F)+\mu F=\mu E$;}

\noindent as $F$ is arbitrary, $u+\lambda E\le\mu E$.

\medskip

{\bf (d)} If $J\in\Cal I$, $F\in\Sigma$, $F\subseteq J$ and
$\mu^*(F\cap I)=0$ for every $I\in\Cal I$, then $F\cap J=F$ is
$\mu$-negligible;  as $F$ is arbitrary, $\lambda J=0$.
Thus $\lambda$ has all the required properties.
}%end of proof of 214O

\leader{*214P}{Theorem}
Let $(X,\Sigma,\mu)$ be a measure space, and
$\Cal A$ a family of subsets of $X$ which is well-ordered by the relation
$\subseteq$.   Then there is an extension of $\mu$ to a measure $\lambda$
on $X$ such that $\lambda(E\cap A)$ is defined and equal to
$\mu^*(E\cap A)$ whenever $E\in\Sigma$ and $A\in\Cal A$.

\proof{{\bf (a)} Adding $\emptyset$ and $X$ to $\Cal A$ if necessary, we
may suppose that
$\Cal A$ has $\emptyset$ as its least member and $X$ as its greatest
member.   By 2A1Dg, $\Cal A$ is isomorphic, as
ordered set, to some ordinal;  since $\Cal A$ has a greatest member, this
ordinal is a successor, expressible as $\zeta+1$;  let
$\xi\mapsto A_{\xi}:\zeta+1\to\Cal A$ be the order-isomorphism, so that
$\langle A_{\xi}\rangle_{\xi\le\zeta}$ is a non-decreasing family of
subsets of $X$, $A_0=\emptyset$ and $A_{\zeta}=X$.

\medskip

{\bf (b)} For each ordinal $\xi\le\zeta$, write $\mu_{\xi}$ for
the subspace measure on $A_{\xi}$, $\Sigma_{\xi}$ for its domain
and $\Cal I_{\xi}$ for $\bigcup_{\eta<\xi}\Cal PA_{\eta}$.
Because $A_{\eta}\cup A_{\eta'}=A_{\max(\eta,\eta')}$ for $\eta$,
$\eta'<\xi$, $\Cal I_{\xi}$ is an ideal of subsets of $A_{\xi}$.
By 214O, we have a measure $\lambda_{\xi}$ on $A_{\xi}$, with domain
$\Lambda_{\xi}\supseteq\Sigma_{\xi}\cup\Cal I_{\xi}$, such that
$\mu_{\xi}E=\lambda_{\xi}E+\sup_{I\in\Cal I_{\xi}}\mu_{\xi}^*(E\cap I)$
for every $E\in\Sigma_{\xi}$ and $\lambda_{\xi}I=0$ for every
$I\in\Cal I_{\xi}$.   Because every member of $\Cal I_{\xi}$ is
included in $A_{\eta}$ for some $\eta<\xi$, we have

\Centerline{$\mu^*(E\cap A_{\xi})
=\lambda_{\xi}(E\cap A_{\xi})+\sup_{\eta<\xi}\mu_{\xi}^*(E\cap A_{\eta})
=\lambda_{\xi}(E\cap A_{\xi})+\sup_{\eta<\xi}\mu^*(E\cap A_{\eta})$}

\noindent (214Cd) for every $E\in\Sigma$.   Also, of course,
$\lambda_{\xi}A_{\eta}=0$ for every $\eta<\xi$.

\medskip

{\bf (c)} Now set

\Centerline{$\Lambda=\{F:F\subseteq X$, $F\cap A_{\xi}\in\Lambda_{\xi}$ for
every $\xi\le\zeta\}$,}

\Centerline{$\lambda F=\sum_{\xi\le\zeta}\lambda_{\xi}(F\cap A_{\xi})$}

\noindent for every $F\in\Lambda$.   Because $\Lambda_{\xi}$ is a
$\sigma$-algebra of subsets of $A_{\xi}$ for each $\xi$, $\Lambda$ is a
$\sigma$-algebra of subsets of $X$;  because every $\lambda_{\xi}$ is a
measure, so is $\lambda$.   If $E\in\Sigma$, then

\Centerline{$E\cap A_{\xi}\in\Sigma_{\xi}\subseteq\Lambda_{\xi}$}

\noindent for each $\xi$, so $E\in\Lambda$.   If $\eta\le\zeta$, then for
each $\xi\le\zeta$ either $\eta<\xi$ and

\Centerline{$A_{\eta}\cap A_{\xi}=A_{\eta}
\in\Cal I_{\xi}\subseteq\Lambda_{\xi}$}

\noindent or $\eta\ge\xi$ and $A_{\eta}\cap A_{\xi}=A_{\xi}$ belongs to
$\Lambda_{\xi}$.   So $A_{\eta}\in\Lambda$ for every $\eta\le\zeta$.

\medskip

{\bf (d)} Finally, $\lambda(E\cap A_{\xi})=\mu^*(E\cap A_{\xi})$ whenever
$E\in\Sigma$ and $\xi\le\zeta$.   \Prf\Quer\ Otherwise, because the ordinal
$\zeta+1$ is well-ordered, there is a least
$\xi$ such that $\lambda(E\cap A_{\xi})\ne\mu^*(E\cap A_{\xi})$.
As $A_0=\emptyset$ we surely have $\lambda(E\cap A_0)=\mu^*(E\cap A_0)$
and $\xi>0$.
Note that if $\eta>\xi$, then $\lambda_{\eta}(E\cap A_{\xi})=0$;  so

\Centerline{$\lambda(E\cap A_{\xi})
=\sum_{\eta\le\xi}\lambda_{\eta}(E\cap A_{\xi}\cap A_{\eta})
=\sum_{\eta\le\xi}\lambda_{\eta}(E\cap A_{\eta})$.}

Now

$$\eqalignno{\mu^*(E\cap A_{\xi})
&=\lambda_{\xi}(E\cap A_{\xi})+\sup_{\xi'<\xi}\mu^*(E\cap A_{\xi'})\cr
\displaycause{(b) above}
&=\lambda_{\xi}(E\cap A_{\xi})
  +\sup_{\xi'<\xi}\sum_{\eta\le\xi'}\lambda_{\eta}(E\cap A_{\eta})\cr
\displaycause{because $\xi$ was the first problematic ordinal}
&=\lambda_{\xi}(E\cap A_{\xi})
  +\sup_{\xi'<\xi}\sup_{K\subseteq\xi'+1\text{ is finite}}
  \sum_{\eta\in K}\lambda_{\eta}(E\cap A_{\eta})\cr
\displaycause{see the definition of `sum' in 112Bd, or 226A below}
&=\lambda_{\xi}(E\cap A_{\xi})
  +\sup_{K\subseteq\xi\text{ is finite}}
  \sum_{\eta\in K}\lambda_{\eta}(E\cap A_{\eta})\cr
&=\sup_{K\subseteq\xi+1\text{ is finite}}
  \sum_{\eta\in K}\lambda_{\eta}(E\cap A_{\eta})
=\sum_{\eta\le\xi}\lambda_{\eta}(E\cap A_{\eta})
\ne\mu^*(E\cap A_{\xi})\cr}$$

\noindent by the choice of $\xi$;  but this is absurd.\ \Bang\Qed

In particular,

\Centerline{$\lambda E=\lambda(E\cap A_{\zeta})=\mu^*(E\cap A_{\zeta})
=\mu E$}

\noindent for every $E\in\Sigma$.
This completes the proof of the theorem.
}%end of proof of 214P

\leader{*214Q}{Proposition}\dvAformerly{2{}14Xd}
Suppose that $(X,\Sigma,\mu)$ is an atomless
measure space and $Y$ a subset of $X$ such that
the subspace measure $\mu_Y$ is semi-finite.   Then $\mu_Y$ is atomless.
%for 548B 548C
%\query what about general $Y$?

\proof{ Let $F\subseteq Y$ be such that $\mu_YF$ is defined and not $0$.
Because $\mu_Y$ is semi-finite, there is an $F'\subseteq F$ such that
$\mu_YF'$ is defined, finite and not zero.   In this case,
$\mu^*F'=\mu_YF'$
is finite, so $F'$ has a measurable envelope $E$ say with respect to $\mu$.
Because $\mu$ is atomless, there is an $E_1\in\Sigma$ such that
$E_1\subseteq E$ and neither $E_1$ nor
$E\setminus E_1$ is
$\mu$-negligible.   Now $F\cap E_1$ is measured by $\mu_Y$ and

\Centerline{$\mu_Y(F\cap E_1)\ge\mu^*(F'\cap E_1)=\mu(E\cap E_1)>0$,}

\Centerline{$\mu_Y(F\setminus E_1)\ge\mu^*(F'\setminus E_1)
=\mu(E\setminus E_1)>0$.}

\noindent As $F$ is arbitrary, $\mu_Y$ is atomless.
}%end of proof of 214Q

\exercises{
\leader{214X}{Basic exercises (a)}
%\spheader 214Xa
Let $(X,\Sigma,\mu)$ be a localizable measure space.   Show
that there is an $E\in\Sigma$ such that the subspace measure $\mu_E$
is purely atomic and $\mu_{X\setminus E}$ is atomless.
%214/

\spheader 214Xb Let $X$ be a set, $\theta$ a regular outer measure on
$X$, and $Y$ a subset of $X$.   Let $\mu$ be the measure on $X$ defined
by \Caratheodory's method from $\theta$, $\mu_Y$ the subspace measure on
$Y$, and $\nu$ the measure on $Y$ defined by \Caratheodory's method from
$\theta\restrp\Cal PY$.   Show that if $\mu_Y$ is locally determined
(in particular, if $\mu$ is locally
determined and localizable) then $\nu=\mu_Y$.
%214I

\spheader 214Xc Let $(X,\Sigma,\mu)$ be a localizable measure space, and
$Y$ a subset of $X$ such that the subspace measure $\mu_Y$ is
semi-finite.   Show that $\mu_Y$ is localizable.
%214I

\sqheader 214Xd Let $(X,\Sigma,\mu)$ be a measure space, and $Y$ a
subset of $X$ such that the subspace measure $\mu_Y$ is semi-finite.
(i) Show that a set $F\subseteq Y$ is an atom for $\mu_Y$ iff it is of the
form $E\cap Y$ where $E$ an atom for $\mu$.
(ii) Show that if $\mu$ is purely atomic, so is $\mu_Y$.
%214I  used in 5x47A

\spheader 214Xe Let $(X,\Sigma,\mu)$ be a localizable measure
space, and $Y$ any subset of $X$.   Show that the c.l.d.\ version of the
subspace measure on $Y$ is localizable.
%214I

\spheader 214Xf Let $(X,\Sigma,\mu)$ be a measure space with locally
determined negligible sets, and $Y$ a
subset of $X$, with its subspace measure $\mu_Y$.
Show that $\mu_Y$ has locally determined negligible sets.
%214I

\sqheader 214Xg Let $(X,\Sigma,\mu)$ be a measure space.   Show that
$(X,\Sigma,\mu)$ has locally determined
negligible sets iff the subspace measure $\mu_Y$ is semi-finite for
every $Y\subseteq X$.
%214I, 214Xf

\sqheader 214Xh Let $\langle (X_i,\Sigma_i,\mu_i)\rangle_{i\in I}$ be a
family
of measure spaces, with direct sum $(X,\Sigma,\mu)$ (214L).   Set
$X'_i=X_i\times\{i\}\subseteq X$ for each $i\in I$.   Show that $X'_i$,
with the subspace measure, is isomorphic to $(X_i,\Sigma_i,\mu_i)$.
Under what circumstances is $\langle X'_i\rangle_{i\in I}$ a
decomposition of $X$?   Show that $\mu$ is complete, or strictly
localizable, or localizable, or locally determined, or semi-finite, or
atomless, %used in 548C
or purely atomic iff every $\mu_i$ is.
Show that a measure space is strictly localizable iff it is isomorphic
to a direct sum of totally finite spaces.
%214L

\sqheader 214Xi Let $\langle(X_i,\Sigma_i,\mu_i)\rangle_{i\in I}$ be a
family of measure spaces, and $(X,\Sigma,\mu)$ their direct sum.   Show
that the completion of $(X,\Sigma,\mu)$ can be identified with the
direct sum of the completions of the $(X_i,\Sigma_i,\mu_i)$, and that
the c.l.d.\ version of $(X,\Sigma,\mu)$ can be identified with the
direct sum of the c.l.d.\ versions of the $(X_i,\Sigma_i,\mu_i)$.
%214L

\spheader 214Xj Let
$\langle(X_i,\Sigma_i,\mu_i)\rangle_{i\in I}$ be a family of measure
spaces.   Show that their direct sum has
locally determined negligible sets iff every $\mu_i$ has.
%214L, 214Xf

\spheader 214Xk
Let $\langle(X_i,\Sigma_i,\mu_i)\rangle_{i\in I}$ be a family of measure
spaces, and $(X,\Sigma,\mu)$ their direct sum.   Show that
$(X,\Sigma,\mu)$ has the measurable envelope property (213Xl) iff every
$(X_i,\Sigma_i,\mu_i)$ has.
%214L

\spheader 214Xl
Let $(X,\Sigma,\mu)$ be a measure space, $Y$ a subset of
$X$, and $f:X\to[0,\infty]$
a function such that $\int_Yf$ is defined in $[0,\infty]$.
Show that $\int_Yf=\overline{\int}f\times\chi Y\,d\mu$.
%214E

\sqheader 214Xm\dvAnew{2009}
Write out a direct proof of 214P in the special case in which
$\Cal A=\{A\}$.   ({\it Hint\/}:  for $E$, $F\in\Sigma$,

\Centerline{$\lambda((E\cap A)\cup(F\setminus A))
=\mu^*(E\cap A)+\sup\{\mu G:G\in\Sigma$, $G\subseteq F\setminus A\}$.)}
%214P

\sqheader 214Xn\dvAnew{2009}
Let $(X,\Sigma,\mu)$ be a measure space and $\Cal A$ a finite family of
subsets of $X$.   Show that there is a measure on $X$,
extending $\mu$, which measures every member of $\Cal A$.
%214Xm 214P

\leader{214Y}{Further exercises (a)}\dvAnew{2009}
%\spheader 214Ya
Let $(X,\Sigma,\mu)$ be a measure space and $A$ a subset of $X$
such that the subspace measure on $A$ is semi-finite.   Set
$\alpha=\sup\{\mu E:E\in\Sigma$, $E\subseteq A\}$.
Show that if $\alpha\le\gamma\le\mu^*A$ then there is a measure
$\lambda$ on $X$, extending $\mu$, such that $\lambda A=\gamma$.
%214P 214Xm

\spheader 214Yb\dvAnew{2009}
Let $(X,\Sigma,\mu)$ be a measure space and
$\family{n}{\Bbb Z}{A_n}$ a double-ended sequence of subsets of $X$ such
that $A_m\subseteq A_n$ whenever $m\le n$ in $\Bbb Z$.   Show that there
is a measure on $X$, extending $\mu$, which measures every $A_n$.
\Hint{use 214P twice.}
%214P

\spheader 214Yc\dvAnew{2009}
Let $X$ be a set and $\Cal A$ a family of subsets of $X$.
Show that the following are equiveridical:  (i) for every measure $\mu$ on
$X$ there is a measure on $X$ extending $\mu$ and measuring every member of
$\Cal A$;  (ii) for every totally finite measure $\mu$ on
$X$ there is a measure on $X$ extending $\mu$ and measuring every member of
$\Cal A$.   \Hint{213Xa.}

\spheader 214Yd\dvAnew{2015}
For this exercise only, I will say that a measure $\mu$ on a set $X$ is
{\bf nowhere all-measuring} if whenever $A\subseteq X$ is not
$\mu$-negligible there is a subset of $A$ which is not measured by the
subspace measure on $A$.   Show that if $X$ is a set and
$\mu_0,\ldots,\mu_n$ are nowhere all-measuring complete totally finite
measures on $X$, then there are disjoint
$A_0,\ldots,A_n\subseteq X$ such that $\mu_i^*A_i=\mu_iX$ for every
$i\le n$.   \Hint{start with the case $n=1$, $\mu_0=\mu_1$.}
%mt21bits
}%end of exercises

\endnotes{
\Notesheader{214} I take the first part of the section, down to 214H,
slowly and carefully, because while none of the arguments are deep
(214Eb is the longest) the patterns formed by the results are not always
easy to predict.   There is a counter-example to a tempting extension of
214H/214Xb in 216Xb.

The message of the second part of the section (214I-214L, 214Q) is that
subspaces inherit
many, but not all, of the properties of a measure space;  and in
particular there can be a difficulty with semi-finiteness, unless we have
locally determined negligible sets (214Xg).   (I give an example in
216Xa.)   Of course
213Hb shows that if we start with a localizable space, we can convert
it into a complete locally determined localizable space without doing
great violence to the structure of the space, so the difficulty is
ordinarily superable.

By far the most important case of 214P is when $\Cal A=\{A\}$ is a
singleton, so that the argument simplifies dramatically (214Xm).   In
\S439 of Volume 4 I will return to the problem of extending a measure to
a given larger $\sigma$-algebra in the absence of any helpful
auxiliary structure.   That section will mostly offer counter-examples,
in particular showing that there is no general theorem extending
214Xn from finite families to countable families, and that the special
conditions in 214P and 214Yb
are there for good reasons.   But in Chapter 54 and \S552 of Volume 5 I
will discuss mathematical systems in which much stronger extension
theorems are true, at least if we start from Lebesgue measure.
}%end of notes

\discrpage


