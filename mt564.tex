\frfilename{mt564.tex}
\versiondate{9.2.14}
\copyrightdate{2006}

\def\chaptername{Choice and determinacy}
\def\sectionname{Integration without choice}

\newsection{564}

I come now to the problem
of defining an integral with respect to a Borel- or Baire-coded measure.
Since a Borel-coded measure can be regarded as a Baire-coded measure on a
second-countable space\cmmnt{ (562U)},
I will give the basic results in terms of the wider class.
I seek to follow the general plan of Chapter 12, starting from simple
functions and taking integrable functions to be almost-everywhere
limits of sequences of simple functions (564A);  the concept of
`virtually measurable' function has to be re-negotiated (564Ab).   The
basic convergence theorems from \S123 are restricted but recognisable
(564F).   We also have versions of two of the
representation theorems from \S436 (564H, 564I).

There is a significant change when we come to the completeness of
$L^p$ spaces (564K) and the Radon-Nikod\'ym theorem (564L),
where it becomes necessary to
choose sequences, and we need a well-orderable dense set of functions to
pick from.   Subject to this, we have workable notions of conditional
expectation operator (564Mc) and product measures (564N, 564O).

\leader{564A}{Definitions (a)} Given a topological space $X$ and a
Baire-coded measure $\mu$ on $X$\cmmnt{ (563J)}, I will write
$\CalBa_c(X)^f$ for
the ring of codable Baire sets of finite measure;  $S=S(\CalBa_c(X)^f)$
will be the linear subspace of $\BbbR^X$ generated by
$\{\chi E:E\in\CalBa_c(X)^f\}$\cmmnt{ (see 122Ab,
361D\footnote{\S\S361-362
are written on a general assumption of AC.   The only essential use of it
to begin with, however, is in asserting that an arbitrary Boolean ring
can be faithfully represented as a ring of sets;  and even that can be
dispensed with for a while if we work a little harder, as in 361Ya.})}.
\cmmnt{Then }$S$ is a Riesz subspace of $\BbbR^X$, and also an
$f$-algebra\cmmnt{ in the sense of 352W}.

\spheader 564Ab I will write $\eusm L^0$
for the space of real-valued functions $f$ defined almost
everywhere in $X$ such that there is a codable Baire function
$g:X\to\Bbb R$ such that $f\eae g$.

\spheader 564Ac Let $\int:S\to\Bbb R$ be the
positive linear functional defined by
saying that $\int\chi E=\mu E$ for every $E\in\CalBa_c(X)^f$.
\cmmnt{(The arguments of 361E-361G still apply, so there is such a
functional.)}

\spheader 564Ad $\eusm L^1$ will be the set of those real-valued
functions $f$ defined almost everywhere in $X$ for
which there is a codable sequence $\sequencen{h_n}$ in
$S$ converging to $f$ almost everywhere and such that
$\sum_{n=0}^{\infty}\int|h_{n+1}-h_n|<\infty$;  I will call such functions
{\bf integrable}.

\leader{564B}{Lemma} Let $X$ be a topological space and $\mu$ a
Baire-coded measure on $X$.

(a) $\eusm L^1\subseteq\eusm L^0$.

(b) If $\sequencen{h_n}$ is a non-increasing codable sequence in
$S=S(\CalBa_c(X)^f)$ and
$\lim_{n\to\infty}h_n(x)=0$ for almost every $x$, then
$\lim_{n\to\infty}\int h_n=0$.

(c) If $\sequencen{h_n}$ and $\sequencen{h'_n}$ are two codable
sequences in
$S$ such that $\lim_{n\to\infty}h_n$ and $\lim_{n\to\infty}h'_n$
are defined and equal almost everywhere, and
$\sum_{n=0}^{\infty}\int|h_{n+1}-h_n|$
and $\sum_{n=0}^{\infty}\int|h'_{n+1}-h'_n|$ are both finite, then
$\lim_{n\to\infty}\int h_n$ and $\lim_{n\to\infty}\int h'_n$ are defined
and equal.

(d) If $\sequencen{h_n}$ is a codable sequence in $S$ and
$\sum_{n=0}^{\infty}\int|h_{n+1}-h_n|$ is finite, then
$\sequencen{h_n}$ converges almost everywhere.   In particular, if
$\sequencen{h_n}$ is a non-decreasing codable sequence in $S$ and
$\sup_{n\in\Bbb N}\int h_n$ is finite, $\sequencen{h_n}$ converges a.e.

(e) If $\sequencen{h_n}$ is a codable sequence in $S^+$ and
$\liminf_{n\to\infty}\int h_n=0$, then $\liminf_{n\to\infty}h_n=0$ a.e.

\proof{{\bf (a)} If $f\in\eusm L^1$, there is a codable sequence
$\sequencen{h_n}$ in $S$ converging almost everywhere to $f$.
Now 562T(c-iii) tells us that there is a codable Baire function $g$ equal
to
$\lim_{n\to\infty}h_n$ wherever this is defined as a real number, so that
$f\eae g$ and $f\in\eusm L^0$.

\medskip

{\bf (b)} Set $E=\{x:h_0(x)>0\}$.   Take any $\epsilon>0$.
For each
$n\in\Bbb N$ set $E_n=\{x:h_n(x)>\epsilon\}$.   Then $\sequencen{E_n}$
is a non-increasing codable sequence in $\CalBa_c(X)^f$ (562Td), and
$\bigcap_{n\in\Bbb N}E_n\subseteq\{x:\lim_{n\to\infty}h_n(x)\ne 0\}$ is
negligible;  also $E_0$ has finite measure.
Accordingly $\lim_{n\to\infty}\mu E_n=0$ (563K(b-iii)).   But

\Centerline{$h_n\le\|h_0\|_{\infty}\chi E_n+\epsilon\chi E$,
\quad$\int h_n\le\|h_0\|_{\infty}\mu E_n+\epsilon\mu E$}

\noindent for every $n$, so
$\limsup_{n\to\infty}\int h_n\le\epsilon\mu E$.
As $\epsilon$ is arbitrary, $\lim_{n\to\infty}\int h_n=0$.

\medskip

{\bf (c)} Since $\int$ is a positive linear functional on the Riesz space
$S$,

\Centerline{$\sum_{n=0}^{\infty}|\int h_{n+1}-\int h_n|
\le\sum_{n=0}^{\infty}\int|h_{n+1}-h_n|$}

\noindent is finite, and
the limit $\lim_{n\to\infty}\int h_n$ is defined in $\Bbb R$.   Similarly,
$\lim_{n\to\infty}\int h'_n$ is defined.   To see that the limits are
equal, set $g_n=h_n-h'_n$ for each $n$, so that $\lim_{n\to\infty}g_n=0$
a.e.\ and $\sum_{n=0}^{\infty}\int|g_{n+1}-g_n|<\infty$.   Then
$\int|g_n|\le\sum_{m=n}^{\infty}\int|g_{m+1}-g_m|$ for every $n$.   \Prf\
For $k\ge n$, set $f_k=(|g_n|-\sum_{m=n}^k|g_{m+1}-g_m|)^+$.   Then
$0\le f_k\le|g_{k+1}|$ for each $k$, so
$\langle f_k\rangle_{k\ge n}$ is a non-increasing codable
sequence in $S$ converging to
$0$ almost everywhere.   (To check that $\langle f_k\rangle_{k\ge n}$
is codable, use 562T(c-ii) and the idea of 562Se.)
By (b), $\lim_{n\to\infty}\int f_k=0$;  but
$\int f_k\ge\int|g_n|-\sum_{m=n}^k\int|g_{m+1}-g_m|$ for every $k$.\ \Qed

Consequently

$$\eqalign{|\lim_{n\to\infty}\int h_n-\lim_{n\to\infty}\int h'_n|
&=\lim_{n\to\infty}|\int h_n-\int h'_n|
\le\lim_{n\to\infty}\int|g_n|
\cr&\le\lim_{n\to\infty}\sum_{m=n}^{\infty}\int|g_{m+1}-g_m|
=0\cr}$$

\noindent as required.

\medskip

{\bf (d)} For $k\in\Bbb N$ let $n_k\in\Bbb N$ be the least integer
such that
$\sum_{i=n_k}^{\infty}\int|h_{i+1}-h_i|\le 2^{-k}$, and for $m\ge n_k$ set

\Centerline{$G_{km}=\{x:\sum_{i=n_k}^m|h_{i+1}(x)-h_i(x)|\ge 1\}$.}

\noindent Then $\mu G_{km}\le 2^{-k}$, because
$\chi G_{km}\le\sum_{i=n_k}^m|h_{i+1}-h_i|$, so
$\mu G_k\le 2^{-k}$, where $G_k=\bigcup_{m\ge n_k}G_{km}$, by 563K(b-i).
(Of course we have to check that all the sequences of sets and functions
involved here are codable.)   Accordingly,
setting $E=\bigcap_{k\in\Bbb N}G_k$, $\mu E=0$.   But observe
that if $x\in X\setminus E$ there is a $k\in\Bbb N$ such that
$x\notin G_k$ and $\sum_{i=n_k}^{\infty}|h_{i+1}(x)-h_i(x)|\le 1$;
in which case $\lim_{n\to\infty}h_n(x)$ is defined.


\medskip

{\bf (e)} For $k\in\Bbb N$ let $n_k$ be the least integer such that
$n_k>n_i$ for $i<k$ and $\int h_{n_k}\le 4^{-n_k}$.   Set
$G_k=\{x:h_{n_k}(x)\ge 2^{-k}\}$;  then $\mu G_k\le 2^{-k}$ and
$\sequence{k}{G_k}$ is codable.   So
$\mu(\bigcup_{k\ge n}G_k)\le 2^{-n+1}$ for every $n$ and
$E=\bigcap_{n\in\Bbb N}\bigcup_{k\ge n}G_k$ is negligible.   But
$E\supseteq\{x:\liminf_{n\to\infty}h_n(x)>0\}$.
}%end of proof of 564B

\leader{564C}{Definition} Let $X$ be a topological space and $\mu$ a
Baire-coded measure on $X$.   For $f\in\eusm L^1$, define its integral
$\int f$ by saying that $\int f=\lim_{n\to\infty}\int h_n$ whenever
$\sequencen{h_n}$ is a codable sequence in $S=S(\CalBa_c(X)^f)$ converging to $f$ almost
everywhere and $\sum_{n=0}^{\infty}\int|h_{n+1}-h_n|$ is finite.
\cmmnt{By
564Bc, this definition is sound;  and clearly it is consistent with the
previous definition of the integral on $S$.}

\leader{564D}{Lemma} Let $X$ be a topological space and $\sequencen{f_n}$ a
codable sequence of codable Baire functions on $X$.
Let $\sequence{i}{q_i}$ be an enumeration of $\Bbb Q\cap\coint{0,\infty}$,
starting with $q_0=0$.   Set

\Centerline{$f'_n(x)=\max\{q_i:i\le n$, $q_i\le\max(0,f_n(x))\}$}

\noindent for $n\in\Bbb N$ and $x\in X$.   Then $\sequencen{f'_n}$ is a
codable sequence of codable Baire functions.

\proof{ Take a sequence running over a base for the topology of
$\BbbR^{\Bbb N}$ and the corresponding interpretation
$\phi:\Cal T\to\Cal B_c(\BbbR^{\Bbb N})$ of Borel codes, as in 562B, and
let $\tilde\phi:\tilde T\to\BbbR^X$ be the corresponding interpretation of codes for
real-valued codable Borel functions, as in 562N.   By
562T(c-ii), there are a continuous function $g:X\to\BbbR^{\Bbb N}$ and
a sequence $\sequencen{\tau_n}$ of codes such that
$f_n=\tilde\phi(\tau_n)\frsmallcirc g$ for every $n$.   We need
a sequence $\sequencen{\tau'_n}$ of codes such that

$$\eqalign{\phi(\tau'_n(\alpha))
&=\bigcup_{\Atop{i\le n}{q_i>\alpha}}
  \bigcap_{\Atop{j\in\Bbb N}{q_j<q_i}}\phi(\tau_n(q_j))
\text{ if }\alpha\ge 0,\cr
&=X\text{ if }\alpha<0;\cr}$$

\noindent and this is easy to build using complementation and
general union operators as in
562C.   Now take $f'_n=\tilde\phi(\tau'_n)\frsmallcirc g$ for each $n$.
}%end of proof of 564D

\leader{564E}{Theorem} Let $X$ be a
topological space and $\mu$ a Baire-coded measure on $X$.

(a)(i) If $f$, $g\in\eusm L^0$ and $\alpha\in\Bbb R$,
then $f+g$, $\alpha f$, $|f|$ and $f\times g$ belong to $\eusm L^0$.

\quad(ii) If $h:\Bbb R\to\Bbb R$ is a codable Borel function,
$hf\in\eusm L^0$ for every $f\in\eusm L^0$.

(b) If $f$, $g\in\eusm L^1$ and $\alpha\in\Bbb R$,
then

\quad(i) $f+g$, $\alpha f$ and $|f|$ belong to $\eusm L^1$;

\quad(ii) $\int f+g=\int f+\int g$, $\int\alpha f=\alpha\int f$;

\quad(iii) if $f\leae g$ then $\int f\le\int g$.

(c)(i) If $f\in\eusm L^0$, $g\in\eusm L^1$ and $|f|\leae g$, then
$f\in\eusm L^1$.

\quad(ii) If $E\in\CalBa_c(X)$ and $\chi E\in\eusm L^1$ then $\mu E$ is
finite.

\proof{{\bf (a)(i)} Use 562T(c-iv).

\medskip

\quad{\bf (ii)} We know that $f$ is equal almost everywhere to a product
$f'g$ where
$g:X\to\BbbR^{\Bbb N}$ is continuous and $f':\BbbR^{\Bbb N}\to\Bbb R$ is a
codable Borel function.   Now $hf'$ is a codable Borel function, by
562Mb, so $hf'g$ is a codable Baire function and $hf\eae hf'g$ belongs to
$\eusm L^0$.

\medskip

{\bf (b)(i)-(ii)} These proceed by the same arguments as in (a-i).
To deal with $|f|$, we need to note that if $\sequencen{h_n}$ is any
codable sequence in $S=S(\CalBa_c(X)^f)$ then $\sum_{n=0}^{\infty}\int||h_{n+1}|-|h_n||
\le\sum_{n=0}^{\infty}\int|h_{n+1}-h_n|$.

\medskip

\quad{\bf (iii)} If $f\leae g$, let $\sequencen{f_n}$, $\sequencen{g_n}$ be
codable sequences in $S$ converging almost everywhere to $f$, $g$
respectively, and such that $\sum_{n=0}^{\infty}\int|f_{n+1}-f_n|$ and
$\sum_{n=0}^{\infty}\int|g_{n+1}-g_n|$ are finite.   Set
$h_n=f_n\wedge g_n$ for each $n$.   Then $\sequencen{h_n}$ is codable,
$f\eae\lim_{n\to\infty}h_n$,
$\sum_{n=0}^{\infty}\int|h_{n+1}-h_n|<\infty$ and

\Centerline{$\int f=\lim_{n\to\infty}\int h_n\le\lim_{n\to\infty}\int g_n
=\int g$.}

\medskip

{\bf (c)(i)} Let $\sequencen{h_n}$ be a codable sequence in $S$ such that
$g\eae\lim_{n\to\infty}h_n$ and $\sum_{n=0}^{\infty}\int|h_{n+1}-h_n|$ is
finite.   Set $h'_n=\sup_{i\le n}h_i^+$ for each $n$;  then
$\sequencen{h'_n}$ is a non-decreasing codable sequence in $S$ and
$\sup_{n\in\Bbb N}\int h'_n$ is finite, while
$|f|\leae g\leae\sup_{n\in\Bbb N}h'_n$.   There is a codable
Baire function $\tilde f$ such that
$f\eae\tilde f$.   Now $\tilde f^+$ is a codable Baire function, so
$\sequencen{\tilde f^+\wedge h'_n}$ is a codable sequence of
non-negative codable Baire functions.

For each $n\in\Bbb N$ consider $h''_n$ where

\Centerline{$h''_n(x)
  =\max\{q_i:i\le n$, $q_i\le\max(0,(\tilde f^+\wedge h'_n)(x)\}$}

\noindent for $x\in X$.
By 564D, $\sequencen{h''_n}$ is a codable
sequence;  it is non-decreasing and converges a.e.\ to
$\tilde f^+\eae f^+$.   Because $0\le h''_n\le h'_n$,
$h''_n\in S$ for each $n$, and
$\sup_{n\in\Bbb N}\int h''_n\le\sup_{n\in\Bbb N}\int h'_n$ is finite;  so
564Bd tells us that $f^+$ is integrable.

Similarly, $f^-$ is integrable, so $f$ is integrable.

\medskip

\quad{\bf (ii)} Let $\sequencen{h_n}$ be a codable sequence in $S$ such
that $\chi E\eae\lim_{n\to\infty}h_n$ and
$\sum_{n=0}^{\infty}\int|h_{n+1}-h_n|$ is finite.   Set
$h'_n=\sup_{i\le n}h_i$ for each $n$;  then $\sequencen{h'_n}$ is a codable
sequence in $S$ and
$\sup_{n\in\Bbb N}\int h'_n$ is finite.   Set
$E_n=\{x:h'_n(x)>\bover12\}$;  then $\sequencen{E_n}$ is a non-decreasing
codable sequence in $\CalBa_c(X)^f$.   Also
$E\setminus\bigcup_{n\in\Bbb N}E_n$ is negligible, so

\Centerline{$\mu E\le\mu(\bigcup_{n\in\Bbb N}E_n)
=\sup_{n\in\Bbb N}\mu E_n=\sup_{n\in\Bbb N}\int\chi E_n
\le 2\sup_{n\in\Bbb N}\int h'_n$}

\noindent is finite.
}%end of proof of 564E

\leader{564F}{}\cmmnt{ I come now to versions of the fundamental
convergence theorems.

\medskip

\noindent}{\bf Theorem} Let $X$ be a topological space and
$\mu$ a Baire-coded measure on $X$.   Suppose that $\sequencen{f_n}$ is a
codable sequence of integrable codable Baire functions on $X$.

(a) If $\sequencen{f_n}$ is non-decreasing and
$\gamma=\sup_{n\in\Bbb N}\int f_n$ is finite, then
$f=\lim_{n\to\infty}f_n$ is defined a.e.\ and is integrable, and
$\int f=\gamma$.

(b) If every $f_n$ is non-negative and $\liminf_{n\to\infty}\int f_n$
is finite, then $f=\liminf_{n\to\infty}f_n$ is defined a.e.\ and is
integrable, and $\int f\le\liminf_{n\to\infty}\int f_n$.

(c) Suppose that
there is a $g\in\eusm L^1$ such that $|f_n|\leae g$ for every $n$,
and $f=\lim_{n\to\infty}f_n$ is defined a.e.   Then $\int f$ and
$\lim_{n\to\infty}\int f_n$ are defined and equal.

(d) If $\sum_{n=0}^{\infty}\int|f_{n+1}-f_n|$
is finite, then $f=\lim_{n\to\infty}f_n$ is defined a.e., and $\int f$ and
$\lim_{n\to\infty}\int f_n$ are defined and equal.

(e) If $\sum_{n=0}^{\infty}\int|f_n|$ is
finite, then $f=\sum_{n=0}^{\infty}f_n$ is defined a.e., and $\int f$ and
$\sum_{n=0}^{\infty}\int f_n$ are defined and equal.

\proof{{\bf (a)} Replacing $f_n$ by $f_n-f_0$ for each $n$, we may suppose
that $f_n\ge 0$ for each $n$.
Let $\sequence{i}{q_i}$ be an enumeration of $\Bbb Q\cap\coint{0,\infty}$
and set

\Centerline{$h_n(x)
  =\max\{q_i:i\le n$, $q_i\le\max(0,f_n(x))\}$}

\noindent for each $x\in X$.   Then $\sequencen{h_n}$ is a codable sequence
of codable Baire functions (use 564D again).   Moreover, $h_n$ takes only
finitely many values, all non-negative, and for $\alpha>0$ the set
$E_{n\alpha}=\{x:h_n(x)>\alpha\}$ is a codable Baire set such that
$\chi E_{n\alpha}\leae\bover1{\alpha}f_n$;  by 564Ec, $E_{n\alpha}$ has
finite measure;  as $\alpha$ is arbitrary, $h_n\in S$.

Now $\sequencen{h_n}$ is non-decreasing, and $\int h_n\le\int f_n\le\gamma$
for every $n$;   so by 564Bd $\sequencen{h_n}$ converges almost
everywhere to an integrable function $f_1$, with $\int f_1\le\gamma$.
Of course $f_1\eae\lim_{n\to\infty}f_n=f$;  as $f\geae f_n$ for every
$n$, $\int f=\int f_1=\gamma$ exactly.

\medskip

{\bf (b)} By 562T(c-ii) and 562Oc,
$\sequencen{f'_n}$ is codable, where $f'_n=\inf_{m\ge n}f_m$ for
every $n$, and of course

\Centerline{$\int f'_n\le\inf_{m\ge n}\int f_m
\le\liminf_{m\to\infty}\int f_m$}

\noindent for every $n$.   Now $\sequencen{f'_n}$ is non-decreasing,
so (a) tells us that
$\int\liminf_{n\to\infty}f_n=\int\lim_{n\to\infty}f'_n$ is defined and
equal to $\lim_{n\to\infty}\int f'_n\le\liminf_{n\to\infty}\int f_n$.

\medskip

{\bf (c)} Let $g'$ be a codable Baire function such that $g'\eae g$, and
set $f'_n=\med(-g',f_n,g')$ for each $n$;  once again, 562T(c-ii) and the
ideas of 562Oc show that $\sequencen{g'+f'_n}$ is codable.
So we can use (b) to see that
$\int\liminf_{n\to\infty}g'+f'_n$ is defined and is at most
$\liminf_{n\to\infty}\int g'+f'_n=\int g'+\liminf_{n\to\infty}\int f_n$.
Subtracting $g'$, we get
$\int\liminf_{n\to\infty}f'_n\le\liminf_{n\to\infty}\int f_n$.
Similarly, $\int\limsup_{n\to\infty}f'_n\ge\limsup_{n\to\infty}\int f_n$.

Once again, the sequences
$\sequencen{f_n}$, $\sequencen{f'_n}$, $\sequencen{|f'_n-f_n|}$ and
$\sequencen{\{x:f'_n(x)\ne f_n(x)\}}$ are all codable.
Since all the sets
$\{x:f'_n(x)\ne f_n(x)\}$ are negligible, so is their union;  but this
means that $\lim_{n\to\infty}f'_n\eae\lim_{n\to\infty}f_n$ is defined
almost everywhere.   So (just as in 123C) the integrals are sandwiched, and
$\int\lim_{n\to\infty}f_n=\lim_{n\to\infty}\int f_n$.

\medskip

{\bf (d)} Of course $\sum_{n=0}^{\infty}|\int f_{n+1}-\int f_n|$ is finite,
so $\gamma=\lim_{n\to\infty}\int f_n$ is defined.
Next, (a) tells us that $g=|f_0|+\sum_{n=0}^{\infty}|f_{n+1}-f_n|$ is
defined a.e.\ and is integrable (of course this depends on our having a
procedure -- induction is allowed -- for building a sequence of Baire codes
representing $\sequencen{|f_0|+\sum_{i=0}^n|f_{i+1}-f_i|}$ out of a
sequence of codes representing $\sequencen{f_n}$).
Since $\lim_{n\to\infty}f_n(x)$ is
defined whenever $g(x)$ is defined and finite, which is almost everywhere,
and $|f_n|\leae g$ for every $n$, (c) gives the result we're looking for.

\medskip

{\bf (e)} Similarly, $\sequencen{\sum_{i=0}^nf_i}$ is
codable and we can apply (d).
}%end of proof of 564F

\vleader{72pt}{564G}{Integration over subsets:  Proposition} Let $X$ be a
topological space and $\mu$ a Baire-coded measure on $X$.

(a) If $f\in\eusm L^1$, the functional
$E\mapsto\int f\times\chi E:\CalBa_c(X)\to\Bbb R$ is additive and truly
continuous with respect to $\mu$.\cmmnt{\footnote{The
definition of `truly continuous' in 232Ab assumed that $\mu$ was defined on
a $\sigma$-algebra.   I hope it is obvious that the same formulation makes
sense when the domain of $\mu$ is any Boolean algebra.}}

(c) If $f$, $g\in\eusm L^1$, then $f\leae g$ iff
$\int f\times\chi E\le\int g\times\chi E$ for every $E\in\CalBa_c(X)$.
\cmmnt{So }$f\eae g$ iff
$\int f\times\chi E=\int g\times\chi E$ for every $E\in\CalBa_c(X)$.

\proof{{\bf (a)} If $E\in\CalBa_c(X)$ then $\chi E$ is a codable Baire
function (use 562Nf), so that $f\times\chi E$ is
integrable (564E(a-i), 564E(c-i)).
Because $\chi:\CalBa_c(X)\to\eusm L^0$ is additive,
$E\mapsto\int f\times\chi E$ is additive.   To see that it is truly
continuous, take $\epsilon>0$.   There is a codable
sequence $\sequencen{h_n}$ in $S=S(\CalBa_c(X)^f)$ such that
$\sum_{n=0}^{\infty}\int|h_{n+1}-h_n|<\infty$ and
$f\eae\lim_{n\to\infty}h_n$.   For each $n$,

\Centerline{$\int|f-h_n|=\lim_{m\to\infty}\int|h_m-h_n|
\le\sum_{m=n}^{\infty}\int|h_{m+1}-h_m|$,}

\noindent so there is an $n$ such that $\int|f-h_n|\le\bover12\epsilon$.
Set $E=\{x:h_n(x)\ne 0\}$ and $\delta=\epsilon/(1+2\|h_n\|_{\infty})$.
Then $E$ has finite measure.   If $F\in\Cal B_c(X)$ and
$\mu(E\cap F)\le\delta$, then

\Centerline{$|\int f\times\chi F|
\le\int|f-h_n|+\int|h_n|\times\chi F
\le\Bover12\epsilon+\|h_n\|_{\infty}\mu(E\cap F)
\le\epsilon$.}

\noindent As $\epsilon$ is arbitrary, the functional is truly continuous.

\medskip

{\bf (b)(i)} If $f\leae g$ and $E\in\Cal B_c(X)$, then
$f\times\chi E\leae g\times\chi E$ so
$\int f\times\chi E\le\int g\times\chi E$.

\medskip

\quad{\bf (ii)} If $\int f\times\chi E\le\int g\times\chi E$ for every
$E\in\Cal B_c(X)$, let $\sequencen{h_n}$ be a codable sequence in $S$ such
that $f-g\eae\lim_{n\to\infty}h_n$ and
$\sum_{n=0}^{\infty}\int|h_{n+1}-h_n|<\infty$.   For $k\in\Bbb N$ let
$n_k$ be the least integer such that
$\sum_{m=n_k}^{\infty}\int|h_{m+1}-h_m|\le 2^{-k}$.   For $m$, $k\in\Bbb N$
set $E_{mk}=\{x:h_{n_k}(x)\ge 2^{-m}\}$.   Then

$$\eqalign{\mu E_{mk}
&\le 2^m\int h_{n_k}\times\chi E_{mk}
\le 2^m\int(h_{n_k}-f+g)\times\chi E_{mk}\cr
&=2^m\lim_{i\to\infty}\int(h_{n_k}-h_i)\times\chi E_{mk}
\le 2^m\lim_{i\to\infty}\int|h_{n_k}-h_i|
\le 2^{m-k}.\cr}$$

\noindent Also
$\langle E_{mk}\rangle_{m,k\in\Bbb N}$ is a codable family in
$\CalBa_c(X)$, so $\mu(\bigcup_{k\ge 2m}E_{mk})\le 2^{-m+1}$ for every
$m$ and $\mu E=0$, where

\Centerline{$E
=\bigcup_{l\in\Bbb N}\bigcap_{m\ge l}\bigcup_{k\ge 2m}E_{mk}$.}

\noindent But for $x\in X\setminus E$,
$\limsup_{k\to\infty}h_{n_k}(x)\le 0$.   Since
$f-g\eae\lim_{k\to\infty}h_{n_k}$, $f\leae g$.
}%end of proof of 564G

\leader{564H}{Theorem} Let $X$ be a topological space, and
$f:C_b(X)\to\Bbb R$ a sequentially smooth
positive linear functional\cmmnt{, where $C_b(X)$ is the space of
bounded continuous real-valued functions on $X$}.
Then there is a totally finite Baire-coded measure $\mu$ on
$X$ such that $f(u)=\int u\,d\mu$ for every $u\in C_b(X)$.

\proof{{\bf (a)} For cozero sets $G\subseteq X$ set
$\mu_0 G=\sup\{f(u):u\in C_b(X)$, $0\le u\le\chi G\}$.   Then
$\mu_0 G=\lim_{n\to\infty}f(u_n)$ whenever $G\subseteq X$ is a cozero set
and $\sequencen{u_n}$ is a
non-decreasing sequence in $C_b(X)^+$ with supremum $\chi G$ in $\BbbR^X$.
\Prf\ Setting
$\gamma=\sup_{n\in\Bbb N}f(u_n)$, then of course

\Centerline{$\mu_0 G\ge\gamma=\lim_{n\to\infty}f(u_n)$.}

\noindent On the other hand, if $v\in C_b(X)$ and $0\le v\le\chi G$,
$\sequencen{(v-u_n)^+}$ is a non-increasing sequence
converging to $\tbf{0}$ pointwise, so

\Centerline{$f(v)\le f(u_n)+f(v-u_n)^+\le\gamma+f(v-u_n)^+\to\gamma$}

\noindent as $n\to\infty$.   As $v$ is arbitrary, $\mu_0 G\le\gamma$.\ \Qed

\medskip

{\bf (b)} It follows that $\mu_0$ satisfies the conditions of 563L.
\Prf\ Of course $\mu_0\emptyset=0$ and $\mu_0$ is monotonic.
If $G$, $H\subseteq X$
are cozero sets, express them as $\{x:u(x)>0\}$ and $\{x:v(x)>0\}$ where
$u$, $v\in C_b(X)^+$.   Set $u_n=nu\wedge\chi X$, $v_n=nv\wedge\chi X$ for
each $n$;  then $\sequencen{u_n}$, $\sequencen{v_n}$ are non-decreasing
sequences in $C_b(X)^+$ converging pointwise to $\chi G$, $\chi H$
respectively.   Now $\sequencen{u_n\wedge v_n}$ and
$\sequencen{u_n\vee v_n}$ are also non-decreasing sequences in $C_b(X)^+$
converging to $\chi(G\cap H)$, $\chi(G\cup H)$;  so (a) tells us that

$$\eqalign{\mu_0(G\cup H)+\mu_0(G\cap H)
&=\lim_{n\to\infty}f(u_n\wedge v_n)+\lim_{n\to\infty}f(u_n\vee v_n)\cr
&=\lim_{n\to\infty}f(u_n\wedge v_n+u_n\vee v_n)\cr
&=\lim_{n\to\infty}f(u_n+v_n)
=\mu_0 G+\mu_0 H.\cr}$$

\noindent As for the penultimate condition in 563L,
let $\sequencen{G_n}$ be a
non-decreasing sequence of cozero sets such that there is a sequence
$\sequencen{v_n}$ in $C(X)$ such that $G_n=\{x:v_n(x)\ne 0\}$ for each $n$.
Set $u_n=\chi X\wedge n\sup_{i\le n}|v_i|$ for each $n$, and
$G=\bigcup_{n\in\Bbb N}G_n$;  then
$\sequencen{u_n}\uparrow\chi G$, so

\Centerline{$\mu_0 G
=\lim_{n\to\infty}f(u_n)\le\lim_{n\to\infty}\mu_0 G_n\le\mu_0G$,}

\noindent as required.\ \Qed

\medskip

{\bf (c)} We therefore have a Baire-coded measure $\mu$ on $X$ extending
$\mu_0$.   Now take any $u\in C_b(X)$ such that $0\le u\le\chi X$, and
$n\ge 1$.   For each $i<n$ set $G_i=\{x:u(x)>\bover{i}{n}\}$;  then

\Centerline{$\Bover1n\sum_{i=0}^{n-1}\chi G_i\le u+\Bover1n\chi X$,}

\noindent so

\Centerline{$\Bover1n\sum_{i=0}^{n-1}\mu G_i
\le\int u\,d\mu+\Bover1n\mu X$.}

\noindent Next, setting

\Centerline{$v_i=u\wedge\Bover{i+1}n\chi X-u\wedge\Bover{i}n\chi X$}

\noindent for $i<n$, $u=\sum_{i=0}^{n-1}v_i$ and $nv_i\le\chi G_i$ for
each $i$, so

\Centerline{$f(u)=\sum_{i=0}^{n-1}f(v_i)\le\Bover1n\sum_{i=0}^{n-1}\mu G_i
\le\int u\,d\mu+\Bover1n\mu X$.}

\noindent As $n$ is arbitrary, $f(u)\le\int u\,d\mu$.   On the other hand,
$f(\chi X)=\mu X=\int\chi X\,d\mu$ and $f(\chi X-u)\le\int(\chi X-u)d\mu$;
so in fact $f(u)=\int u\,d\mu$.

\medskip

{\bf (d)} It follows at once that $f(u)=\int u\,d\mu$ for every
$u\in C_b(X)^+$ and therefore for every $u\in C_b(X)$, as required.
}%end of proof of 564H

\leader{564I}{Riesz Representation Theorem} Let $X$ be a completely
regular locally compact space, and $f:C_k(X)\to\Bbb R$ a positive
linear functional\cmmnt{, where $C_k(X)$ is the space of continuous
real-valued functions with compact support}.
Then there is a Baire-coded measure $\mu$ on $X$ such
that $\int u\,d\mu$ is defined and equal to $f(u)$ for every $u\in C_k(X)$.

\proof{ We can follow the plan of 564H, with minor modifications.

\medskip

{\bf (a)} For open sets $G\subseteq X$ write
$D_G=\{u:u\in C_k(X)$, $0\le u\le\chi X$, $\supp u\subseteq G\}$, where
$\supp u=\overline{\{x:u(x)\ne 0\}}$.   We need to know that if $G$,
$H\subseteq X$ are open and $K\subseteq G\cup H$, $K'\subseteq G\cap H$
are compact, there are
$u\in D_G$, $v\in D_H$ such that $\chi K\le u\vee v$ and
$\chi K'\le u\wedge v$.   \Prf\ Because
$X$ is completely regular, the family
$\{\interior\{x:u(x)=1\}:u\in D_G\cup D_H\}$ is an open cover of $G\cup H$
and has a finite subfamily covering $K$;  because $D_G$ and $D_H$ are
upwards-directed, we can reduce this finite subfamily to two terms, one
corresponding to $u_1\in D_G$ and the other to $v_1\in D_H$, so that
$\chi K\le u_1\vee v_1$.
Next, $\{\interior\{x:u(x)=1\}:u\in D_G\}$ is an open cover of
$G\supseteq K'$, so we can find a $u_2\in D_G$ such that $\chi K'\le u_2$;
similarly, there is a $v_2\in D_H$ such that $\chi K'\le v_2$;  set
$u=u_1\vee u_2$ and $v=v_1\wedge v_2$.\ \Qed

\medskip

{\bf (b)} For cozero $G\subseteq X$, set
$\mu_0G=\sup\{f(u):u\in D_G\}$.   If $G$, $H\subseteq X$ are cozero sets,
$u\in D_G$ and $v\in D_H$, then $u\vee v\in D_{G\cup H}$ and
$u\wedge v\in D_{G\cap H}$;  this is enough to show that
$\mu_0G+\mu_0H\le\mu_0(G\cup H)+\mu_0(G\cap H)$.   If $w\in D_{G\cup H}$
and $w'\in D_{G\cap H}$, (a) tells us that
there are $u\in D_G$ and $v\in D_H$ such that
$u\vee v\ge\chi(\supp w)\ge w$
and $u\wedge v\ge\chi(\supp w')\ge w'$;  this is what we need to show that so that
$\mu_0G+\mu_0H\ge\mu_0(G\cup H)+\mu_0(G\cap H)$.

If $\sequencen{G_n}$ is a non-decreasing sequence of cozero sets, defined
from a sequence of continuous functions so that
$G=\bigcup_{n\in\Bbb N}G_n$ is a cozero set, then
$D_G=\bigcup_{n\in\Bbb N}D_{G_n}$ so that
$\mu_0G=\sup_{n\in\Bbb N}\mu_0G_n$.

If $G$ is a relatively compact cozero set then $\mu_0G<\infty$.
\Prf\ There is a $w\in C_k(X)$ such that $\chi\overline{G}\le w$,
so that $\mu G\le f(w)$.\ \QeD\   If $G$ is a cozero set and
$\gamma<\mu_0G$, there is a $u\in D_G$ such that $f(u)\ge\gamma$.
Now there is a $v\in D_G$ such that $\chi(\supp(u))\le v$, so that
$\mu_0H\ge\gamma$, where $H=\{x:v(x)>0\}$;  as $H$ is relatively compact,
$\mu_0H$ is finite.   Thus $\mu_0G=\sup\{\mu_0H:H\subseteq G$ is a cozero
set, $\mu_0H<\infty\}$.

The other hypotheses of 563L are elementary, so we have a Baire-coded
measure on $X$ extending $\mu_0$.

\medskip

{\bf (c)} If $u\in C_k(X)$ and $0\le u\le\chi X$ and $\epsilon>0$,
let $G$ be a relatively
compact cozero set including $\supp u$, and $v\in D_G$ such that
$\chi(\supp u)\le v$ and $f(v)\ge\mu G-\epsilon$.   The argument of
part (c) of the proof of 564H, with $v$ in place of $\chi X$, shows that
$f(u)\le\int u\,d\mu+\bover1n\int v\,d\mu$ for every $n$, so that
$f(u)\le\int u\,d\mu$.   On the other hand,

$$\eqalign{f(u)
&=f(v)-f(v-u)
\ge\mu G-\epsilon-\int v-u\,d\mu\cr
&=\mu G-\int v\,d\mu+\int u\,d\mu-\epsilon
\ge\int u\,d\mu-\epsilon.\cr}$$

\noindent As $\epsilon$ is arbitrary, $f(u)=\int u\,d\mu$.   Of course it
follows at once that $f$ agrees with $\int\,d\mu$ on the whole of $C_k(X)$.
}%end of proof of 564I

\leader{564J}{The space $L^1$} Let $X$ be a topological space and
$\mu$ a Baire-coded measure on $X$.

\spheader 564Ja If $f$, $g\in\eusm L^1$ then $f\eae g$ iff $\int|f-g|=0$.
\prooflet{\Prf\
If $f\eae g$ then $|f-g|=0$ a.e.\ and $\int|f-g|=0$ by the definition
in 564Ad.   If $\int|f-g|=0$, let $f_1$, $g_1$ be codable Baire functions
such that $f\eae f_1$ and $g\eae g_1$ (564Ba);  then $|f_1-g_1|$ is
codable.
For each $n\in\Bbb N$, set $E_n=\{x:|f_1(x)-g_1(x)|\ge 2^{-n}\}$.
Then $E_n\in\CalBa_c(X)$ and
$|f_1-g_1|\ge 2^{-n}\chi E_n$ so $\mu E_n=\int\chi E_n=0$.
But $\sequencen{E_n}$ is a
codable sequence so $\bigcup_{n\in\Bbb N}E_n=\{x:f_1(x)\ne g_1(x)\}$ is
negligible and

\Centerline{$f\eae f_1\eae g_1\eae g$.  \Qed}}

\spheader 564Jb As in \S242, we have an equivalence
relation $\sim$ on $\eusm L^1$ defined by saying that $f\sim g$ if
$f\eae g$.   The set $L^1$ of equivalence classes has a Riesz space
structure and a Riesz norm inherited from the addition, scalar
multiplication, ordering and integral on $\eusm L^1$.

\spheader 564Jc\cmmnt{ As in \S242,} I\cmmnt{ will} define 
$\int:L^1\to\Bbb R$ by saying
that $\int f^{\ssbullet}=\int f$ for every $f\in\eusm L^1$.   Similarly, we
can define $\int_Eu$, for $u\in L^1$ and $E\in\CalBa_c(X)$, by saying that
$\int_Ef^{\ssbullet}=\int f\times\chi E$ for $f\in\eusm L^1$.

\leader{564K}{}\cmmnt{ In order to prove that an $L^1$-space is
norm-complete, it seems that we need extra conditions.

\medskip

\noindent}{\bf Theorem} Let $X$ be a second-countable space and $\mu$ a
codably $\sigma$-finite Borel-coded measure on $X$.   Then $L^1(\mu)$ is a
separable $L$-space.

\proof{ (Compare 563N.)

\medskip

{\bf (a)} There is a codable sequence of sets of finite measure
covering $X$.   By 562Pb, we can find a codably Borel equivalent
zero-dimensional second-countable topology on $X$ for which all the
these sets
are open, so that $\mu$ becomes locally finite.   Since this procedure does
not change $\eusm L^1$ and $L^1$, we may suppose from the beginning that
$X$ is regular and $\mu$ is locally finite.   Let $\sequencen{U_n}$ run
over a countable base for the topology of $X$ containing $\emptyset$
and closed under finite unions.

\medskip

{\bf (b)} If $E\in\Cal B_c(X)^f$ and $\epsilon>0$, there is an open
$G\subseteq X$ such that $E\subseteq G$ and $\mu(G\setminus E)\le\epsilon$,
by 563Fd.
Next, $G=\bigcup\{U_n:n\in\Bbb N$, $U_n\subseteq G\}$, so there is a finite
set $I\subseteq\Bbb N$ such that $G'=\bigcup_{n\in I}U_n\subseteq G$ and
$\mu(G\setminus G')\le\epsilon$;  now $\mu(G'\symmdiff E)\le 2\epsilon$ and
$G'=U_m$ for some $m$.

\medskip

{\bf (c)} If $f\in\eusm L^1$ and $\epsilon>0$, there is an
$h\in S(\Cal B_c(X)^f)$ such that $\int|f-h|\le\epsilon$;  now there must
be an $n\in\Bbb N$ and a family $\langle q_i\rangle_{i\le n}$
in $\Bbb Q$ such that
$\int|h-\sum_{i=0}^nq_i\chi U_i|\le\epsilon$.   The set $D$ of such
rational linear combinations of the $\chi U_i$ is countable;  enumerate it
as $\sequencen{h_n}$.   All the $h_n$ are differences of semi-continuous
functions, therefore resolvable, so $\sequencen{h_n}$ is a codable
sequence;  and for any $u\in L^1$ and $\epsilon>0$ there is an $n$ such
that $\|u-h_n^{\ssbullet}\|_1\le 2\epsilon$.

\medskip

{\bf (d)} This shows that $L^1$ is separable.   To see that it is complete,
take a Cauchy filter $\Cal F$ on $L^1$.   For each
$k\in\Bbb N$ we can take the first $n_k\in\Bbb N$ such that
$\{u:\|u-h_{n_k}^{\ssbullet}\|_1\le 2^{-k}\}$ belongs to $\Cal F$.   Now
$\int|h_{n_k}-h_{n_{k+1}}|\le 2^{-k}+2^{-k-1}$ for every $k$, so the
codable sequence $\sequence{k}{h_{n_k}}$ converges a.e.\ to some
$f\in\eusm L^1$ (564Fd),
and $\int|f-h_{n_k}|\le 3\cdot 2^{-k}$ for every $k$.   So

\Centerline{$f^{\ssbullet}
=\lim_{k\to\infty}h_{n_k}^{\ssbullet}=\lim\Cal F$.}

\medskip

{\bf (e)} Thus $L^1$ is norm-complete.   We know it is a
Riesz space with a Riesz norm, so it is a Banach lattice.   As for the
additivity of the norm on the positive cone, we have only to observe that
if $f$, $g\in\eusm L^1$ and $f^{\ssbullet}$, $g^{\ssbullet}$ are
non-negative, then

$$\eqalign{\|f^{\ssbullet}+g^{\ssbullet}\|_1
&=\||f|^{\ssbullet}+|g|^{\ssbullet}\|_1
=\|(|f|+|g|)^{\ssbullet}\|_1\cr
&=\int|f|+|g|
=\int|f|+\int|g|
=\|f^{\ssbullet}\|_1+\|g^{\ssbullet}\|_1.\cr}$$
}%end of proof of 564K

\leader{564L}{Radon-Nikod\'ym theorem} Let $X$ be a second-countable space
with a codably $\sigma$-finite
Borel-coded measure $\mu$.   Let $\nu:\Cal B_c(X)\to\Bbb R$ be a
truly continuous additive functional.   Then there is an
$f\in\eusm L^1(\mu)$
such that $\nu E=\int f\times\chi E$ for every $E\in\Cal B_c(X)$.

\proof{{\bf (a)} Let $M$ be the space of bounded additive functionals on
$\Cal B_c(X)$;  as in 362B, $M$ is an $L$-space.
I will write $\Cal L^1$ for the Riesz space of integrable real-valued
codable Borel functions on $X$.   For $f\in\Cal L^1$ and $E\in\Cal B_c(X)$,
set $\nu_fE=\int f\times\chi E$;  this is defined by 564Ea and
564E(c-i).   The map $f\mapsto\nu_f:\Cal L^1\to M$ is a Riesz
homomorphism, and
norm-preserving in the sense that $\|\nu_f\|=\int|f|$ for every
$f\in\Cal L^1$.   Accordingly $M_1=\{\nu_f:f\in\Cal L^1\}$
is a Riesz subspace
of $M$ isomorphic, as normed Riesz space, to $L^1$;  in particular, it is
norm-complete, by 564K, therefore norm-closed.

\medskip

{\bf (b)} If $\nu\in M^+$ is truly continuous and $\epsilon>0$, there are
an $E\in\Cal B_c(X)^f$ and a $\gamma>0$ such that
$\|(\nu-\gamma\nu_{\chi E})^+\|\le\epsilon$.   \Prf\ There
are $E\in\Cal B_c(X)$ and $\delta>0$ such that $\mu E<\infty$ and
$\nu F\le\epsilon$ whenever $\mu(E\cap F)\le\delta$.   Set
$\gamma=\Bover{\|\nu\|}{\delta}$.   Then

$$\eqalign{(\nu-\gamma\nu_{\chi E})(F)
&=\nu F-\Bover{\|\nu\|}{\delta}\mu(F\cap E)\cr
&\le 0\text{ if }\mu(F\cap E)\ge\delta,\cr
&\le\epsilon\text{ otherwise}.\cr}$$

\noindent So $\|(\nu-\gamma\nu_{\chi E})^+\|\le\epsilon$.\ \Qed

\medskip

{\bf (c)} Suppose that $\nu\in M$, $E\in\Cal B_c(X)^f$ and $\gamma>0$
are such that $0\le\nu\le\gamma\nu_{\chi E}$.   Let $\epsilon>0$.   Then
there are an $f\in\Cal L^1$ and a $\nu'\in M^+$ such that
$\|\nu-\nu_f-\nu'\|\le\epsilon$ and $\nu'\le\bover12\gamma\nu_{\chi E}$.
\Prf\
Set $\alpha=\sup_{F\in\Cal B_c(X)}\nu F-\bover12\gamma\mu F$;
let $H\in\Cal B_c(X)$ be such that
$\nu H-\bover12\gamma\mu H\ge\alpha-\bover13\epsilon$;  set
$f=\bover12\gamma\chi(H\cap E)$ and
$\nu'=(\nu-\nu_f)^+\wedge\bover12\gamma\nu_{\chi E}$.

If $F\in\Cal B_c(X)$ then

$$\eqalignno{(\nu_f-\nu)(F)
&=\Bover12\gamma\mu(F\cap H\cap E)-\nu F
\le\Bover12\gamma\mu(F\cap H)-\nu(F\cap H)\cr
&=\Bover12\gamma\mu H-\nu H-\Bover12\gamma\mu(H\setminus F)
  +\nu(H\setminus F)\cr
\displaycause{of course $\mu H$ must be finite, as
$\nu H-\bover12\gamma\mu H$ is finite}
&\le-\alpha+\Bover13\epsilon+\alpha
=\Bover13\epsilon,\cr
(\nu-\nu_f-\Bover12\gamma\nu_{\chi E})(F)
&=\nu F-\Bover12\gamma\mu(F\cap E\cap H)
  -\Bover12\gamma\mu(F\cap E)\cr
&=\nu(F\cap E\setminus H)-\Bover12\gamma\mu(F\cap E\setminus H)\cr
&\mskip100mu
  +\nu(F\setminus E)+\nu(F\cap E\cap H)-\gamma\mu(F\cap E\cap H)\cr
&\le\nu(F\cap E\setminus H)-\Bover12\gamma\mu(F\cap E\setminus H)\cr
\displaycause{because $\nu\le\gamma\nu_{\chi E}$}
&=\nu((F\cap E)\cup H)-\Bover12\gamma\mu((F\cap E)\cup H)
  -\nu H+\Bover12\gamma\mu H\cr
&\le\alpha-(\alpha-\Bover13\epsilon)
=\Bover13\epsilon.\cr}$$

\noindent So $\|(\nu_f-\nu)^+\|\le\bover13\epsilon$ and
$\|(\nu-\nu_f-\bover12\gamma\nu_{\chi E})^+\|\le\bover13\epsilon$.
But this means that

$$\eqalign{\|\nu-\nu_f-\nu'\|
&=\|\nu-\nu_f-(\nu-\nu_f)^++((\nu-\nu_f)^+-\Bover12\gamma\nu_{\chi E})^+\|
  \cr
&\le 2\|\nu-\nu_f-(\nu-\nu_f)^+\|
   +\|(\nu-\nu_f-\Bover12\gamma\nu_{\chi E})^+\|\cr
&\le 2\|(\nu_f-\nu)^+\|+\Bover13\epsilon
\le\epsilon,\cr}$$

\noindent as required.\ \Qed

\medskip

{\bf (d)} Again suppose that $\nu\in M$, $E\in\Cal B_c(X)^f$, $\gamma>0$
and $\epsilon>0$ are such that $0\le\nu\le\gamma\nu_{\chi E}$.
Then for any $n\in\Bbb N$
there are an $f\in\Cal L^1$ and a $\nu'\in M^+$ such that
$\|\nu-\nu_f-\nu'\|\le\epsilon$ and $\nu'\le 2^{-n}\gamma\nu_{\chi E}$.
\Prf\ Induce on $n$.\ \Qed

\medskip

{\bf (e)} If $\nu\in M^+$ is truly continuous and $\epsilon>0$, there is
an $f\in\Cal L^1$ such that $\|\nu-\nu_f\|\le\epsilon$.   \Prf\ By
(b), there are an $E\in\Cal B_c(X)^f$ and a $\gamma>0$ such that
$\|(\nu-\gamma\nu_{\chi E})^+\|\le\bover13\epsilon$.   Let $n\in\Bbb N$ be
such that $2^{-n}\gamma\mu E\le\bover13\epsilon$.   By (d), we have an
$f\in\Cal L^1$ and a $\nu'\in M$ such that
$\|(\nu\wedge\gamma\nu_{\chi E})-\nu_f-\nu'\|\le\bover13\epsilon$ and
$0\le\nu'\le 2^{-n}\gamma\nu_{\chi E}$.   But this means that

$$\eqalign{\|\nu-\nu_f\|
&\le\|(\nu-\gamma\nu_{\chi E})^+\|+\|(\nu\wedge\gamma\nu_{\chi E})-\nu_f\|
   \cr
&\le\Bover13\epsilon+\Bover13\epsilon+\|\nu'\|
\le\Bover23\epsilon+2^{-n}\gamma\mu E
\le\epsilon. \text{ \Qed}\cr}$$

\medskip

{\bf (f)} Since any truly continuous $\nu\in M$ has truly continuous
positive and negative parts, the space
$M_{tc}$ of truly continuous functionals is included in the closure of
$M_1=\{\nu_f:f\in\Cal L^1\}$.   But I noted in (a)
that $M_1$ is norm-isomorphic to $L^1$, so is complete, therefore closed,
and must include $M_{tc}$.
}%end of proof of 564L

\leader{564M}{Inverse-measure-preserving functions (a)} Let $X$ and $Y$ be
second-countable spaces, with Borel-coded measures $\mu$ and $\nu$.
Suppose that $\varphi:X\to Y$ is a codable Borel function such that
$\mu\varphi^{-1}[F]=\nu F$ for every $F\in\Cal B_c(Y)$.   Then
$h\varphi\in S_X$ and $\int h\varphi\,d\mu=\int h\,d\nu$ for every
$h\in S_Y$, writing $S_X=S(\Cal B_c(X)^f)$, $S_Y$ for the spaces of
simple functions.
\cmmnt{By 562Mb,} $f\varphi\in\eusm L^0(\mu)$ for every
$f\in\eusm L^0(\nu)$.   \cmmnt{By 562Sd,}
$\sequencen{h_n\varphi}$ is a codable sequence in $S_X$ whenever
$\sequencen{h_n}$ is a codable sequence in $S_Y$;  \cmmnt{consequently}
$f\varphi\in\eusm L^1(\mu)$ whenever $f\in\eusm L^1(\nu)$, and
we have a norm-preserving Riesz homomorphism $T:L^1(\nu)\to L^1(\mu)$
defined by setting
$Tf^{\ssbullet}=(f\varphi)^{\ssbullet}$ for $f\in\eusm L^1(\mu)$.

\spheader 564Mb If $\nu$ is codably $\sigma$-finite, we have a conditional
expectation operator in the reverse direction\cmmnt{, as follows}.   For 
any $f\in\eusm L^1(\mu)$, consider the functional
$\lambda_f$ defined by setting
$\lambda_fF=\int f\times\chi(\varphi^{-1}[F])$ for $F\in\Cal B_c(Y)$.
This is additive and truly continuous.   \prooflet{\Prf\ Let $\epsilon>0$.
By 564Ga, there are an $E_0\in\Cal B_c(X)$ and a $\delta>0$ such that
$\mu E_0<\infty$ and $\int|f|\times\chi E\le\epsilon$ whenever
$E\in\Cal B_c(X)$ and
$\mu(E\cap E_0)\le 2\delta$.   Next, there is a non-decreasing
codable sequence $\sequencen{F_n}$ in $\Cal B_c(Y)$ such that
$\nu F_n<\infty$ for every $n$ and $Y=\bigcup_{n\in\Bbb N}F_n$.   In this
case, $\sequencen{\varphi^{-1}[F_n]}$ is a non-decreasing codable sequence
in $\Cal B_c(X)$ with union $X$, so there is an $n$ such that
$\mu(E_0\setminus\varphi^{-1}[F_n])\le\delta$.   Now suppose that
$F\in\Cal B_c(Y)$ and $\nu(F\cap F_n)\le\delta$.   In this case,

\Centerline{$\mu(E_0\cap\varphi^{-1}[F])
\le\mu(E_0\setminus\varphi^{-1}[F_n])+\mu(\varphi^{-1}[F_n\cap F])
\le 2\delta$,}

\noindent so

\Centerline{$|\lambda_fF|\le\int|f|\times\chi(\varphi^{-1}[F])
\le\epsilon$.}

\noindent As $\epsilon$ is arbitrary, $\lambda_f$ is truly continuous.\
\Qed}

There is\cmmnt{ therefore} a unique $v_f\in L^1(\nu)$ such that
$\int_Fv_f=\lambda_fF$ for every $F\in\Cal B_c(Y)$.  \prooflet{\Prf\
By 564L, there is a $g\in\eusm L^1(\nu)$ such that
$\lambda_fF=\int g\times\chi F$ for every $F\in\Cal B_c(Y)$.   By 564Gb,
any two such functions are equal almost everywhere, so have the same
equivalence class in $L^1$, which we may call $v_f$.\ \Qed}

We may call $v_f$ the {\bf conditional expectation} of $f$ with respect to
the \imp\ function $\varphi$.

\spheader 564Mc Still supposing that $\nu$ is codably $\sigma$-finite,
\cmmnt{we see that}
$\lambda_f=\lambda_{f'}$ whenever $f$, $f'\in\eusm L^1(\mu)$ are
equal almost everywhere, so that we have an operator
$P:L^1(\mu)\to L^1(\nu)$ defined by saying that $Pf^{\ssbullet}=v_f$ for
every $f\in\eusm L^1(\mu)$;  \cmmnt{that is, that}
$\int_FPu=\int_{\varphi^{-1}[F]}u$ for every $u\in L^1(\mu)$ and
$F\in\Cal B_c(Y)$.   \cmmnt{Because this defines each $Pu$ uniquely,}
$P$ is
linear.   It is positive\dvro{. }{ because if $f^{\ssbullet}\ge 0$ then
$\lambda_f\ge 0$;  if now $g\in\eusm L^1(\nu)$ is such that
$\int g\times\chi F=\lambda_fF\ge 0$ for every $F\in\Cal B_c(X)$,
$g\ge 0$ a.e., by 564Gb, and

\Centerline{$Pf^{\ssbullet}=u_f=g^{\ssbullet}\ge 0$.}

\noindent}%end of dvro
It is elementary to check that if $T$ is the operator of (a)
above then $PT$ is the identity operator on $L^1(\nu)$.

\spheader 564Md Now consider the special case in which $Y=X$,
the topology of $Y$ is the topology generated by a codable sequence
$\sequencen{V_n}$ in $\Cal B_c(X)^f$, $\nu=\mu\restr\Cal B_c(Y)$ and
$\varphi$ is the identity function.   \cmmnt{(Of course this can be
done only when
$\mu$ is codably $\sigma$-finite.)}   In this case, we can identify
$L^1(\nu)$ with its image in $L^1(\mu)$ under $T$, and $P$ becomes a
conditional expectation operator of the kind examined in 242J.

\leader{564N}{Product measures:  Theorem} Let $X$ and $Y$ be
second-countable spaces, and $\mu$, $\nu$ semi-finite
Borel-coded measures on $X$, $Y$ respectively.

(a) There is a Borel-coded measure $\lambda$
on $X\times Y$ such that
$\lambda(E\times F)=\mu E\cdot\nu F$ for all
$E\in\Cal B_c(X)$ and $F\in\Cal B_c(Y)$.

(b) If $\nu$ is codably $\sigma$-finite then we can arrange that
$\iint f(x,y)\nu(dy)\mu(dx)$ is defined and equal to $\int fd\lambda$
for every $\lambda$-integrable real-valued function $f$.

(c) If $\mu$ and $\nu$ are both codably $\sigma$-finite then
$\lambda$ is uniquely defined by the formula in (a).

\proof{{\bf (a)(i)} Start by fixing sequences $\sequencen{U_n}$,
$\sequencen{V_n}$ running over bases for the topologies of $X$, $Y$
respectively containing $\emptyset$, and a bijection
$n\mapsto(i_n,j_n):\Bbb N\to\Bbb N$;  then
$\sequencen{U_{i_n}\times V_{j_n}}$ runs over a base for the topology of
$X\times Y$ containing $\emptyset$.   Let

\Centerline{$\phi_X:\Cal T\to\Cal B_c(X)$,
\quad$\tilde{\Cal T}_X\subseteq\Cal T^{\Bbb R}$,
\quad$\tilde\phi_X:\tilde{\Cal T}_X\to\BbbR^X$,}

\Centerline{$\phi_Y:\Cal T\to\Cal B_c(Y)$,}

\Centerline{$\phi:\Cal T\to\Cal B_c(X\times Y)$,
\quad$\tilde{\Cal T}\subseteq\Cal T^{\Bbb R}$,
\quad$\tilde\phi:\tilde{\Cal T}\to\BbbR^{X\times Y}$}

\noindent be the interpretations of codes
associated with the sequences $\sequencen{U_n}$,
$\sequencen{V_n}$ and $\sequencen{U_{i_n}\times V_{j_n}}$,
as described in 562B and 562N.
Let $\Cal R_X$ be the space of resolvable real-valued functions on $X$,
and $\tilde\psi_X:\Cal R_X\to\tilde{\Cal T}_X$ a function such that
$\tilde\phi_X(\tilde\psi_X(f))=f$ for ever $f\in\Cal R_X$,
as in 562R.
The argument will depend on the existence of a number of
further functions;  it may help if I lay them out explicitly.
Fix a member $\tau_0$ of $\tilde{\Cal T}_X$.

\medskip

\qquad\grheada\
Let $\Theta_2':\Cal T\times\Cal T\to\Cal T$ be such that

\Centerline{$\phi(\Theta_2'(T,T'))=\phi(T)\cap\phi(T')$,
\quad$r(\Theta_2'(T,T'))=\max(r(T),r(T'))$}

\noindent for all $T$, $T'\in\Cal T$ (562Cc);  now define
$\Theta_2^*:\bigcup_{I\in[\Bbb N]^{<\omega}}\Cal T^I\to\Cal T$ by setting

$$\eqalign{\Theta_2^*(\familyiI{T_i})
&=\{\emptyset\}\cup\{\fraction{n}:n\in\Bbb N\}\text{ if }I=\emptyset,\cr
&=\Theta_2'(\Theta_2^*(\family{i}{I\cap n}{T_i}),T_n)
  \text{ if }n=\max I.\cr}$$

\noindent Then

\Centerline{$\phi(\Theta_2^*(\familyiI{T_i}))
=(X\times Y)\cap\bigcap_{i\in I}\phi(T_i)$,
\quad$r(\Theta_2^*(\familyiI{T_i}))=\max(1,\sup_{i\in I}r(T_i))$}

\noindent whenever $I\subseteq\Bbb N$ is finite and $T_i\in\Cal T$ for
$i\in I$.

\medskip

\qquad\grheadb\ Let $\tilde\Theta_1:\Cal T^{\Bbb N}\to\Cal T$ be such that

\Centerline{$\phi_X(\tilde\Theta_1(\sequencen{T_n}))
=\bigcup_{n\in\Bbb N}\phi_X(T_n)$}

\noindent for every sequence $\sequencen{T_n}$ in $\Cal T$ (526Cb).

\medskip

\qquad\grheadc\ There is a function
$\Theta_2:\tilde{\Cal T}_X\times\Cal T\to\tilde{\Cal T}_X$ such that

\Centerline{$\tilde\phi_X(\Theta_2(\tau,T'))
=(\nu\phi_Y(T'))\chi X-\tilde\phi_X(\tau)$}

\noindent whenever $\tau\in\tilde{\Cal T}_X$ and $T'\in\Cal T$ is such that
$\nu(\phi_Y(T'))$ is finite.   \Prf\ Taking $\Theta_0:\Cal T\to\Cal T$
such that $\phi_X(\Theta_0(T))=X\setminus\phi_X(T)$ for every $T\in\Cal T$
(562Ca), set

\Centerline{$\hat\Theta(\tau,\beta)
=\tilde\Theta_1(\sequencen{\Theta_0(\tau(\beta-2^{-n}))})$}

\noindent for $\tau\in\tilde{\Cal T}_X$ and $\beta\in\Bbb R$, so that
$\hat\Theta$ is a function from $\tilde{\Cal T}_X\times\Bbb R$ to
$\Cal T$ and

\Centerline{$\phi_X(\hat\Theta(\tau,\beta))
=\bigcup_{n\in\Bbb N}X\setminus\phi_X(\tau(\beta-2^{-n}))
=\{x:\tilde\phi_X(\tau)(x)<\beta\}$}

\noindent for $\tau\in\tilde{\Cal T}_X$ and $\beta\in\Bbb R$.   If
$\nu(\phi_Y(T'))=\infty$, take $\Theta_2(\tau,T')=\tau_0$
for every $\tau\in\tilde{\Cal T}_X$;  otherwise set

\Centerline{$\Theta_2(\tau,T')(\alpha)
=\hat\Theta(\tau,\nu\phi_Y(T')-\alpha)$}

\noindent for $\tau\in\tilde{\Cal T}_X$, $T'\in\Cal T$ and
$\alpha\in\Bbb R$, so that

$$\eqalign{\phi_X(\Theta_2(\tau,T')(\alpha))
&=\{x:\tilde\phi_X(\tau)(x)<\nu\phi_Y(T')-\alpha\}
\cr&=\{x:\nu\phi_Y(T')-\tilde\phi_X(\tau)(x)>\alpha\}\cr}$$

\noindent for every $\alpha$,
$\Theta_2(\tau,T')\in\tilde{\Cal T}_X$ and
$\tilde\phi_X(\Theta_2(\tau,T'))(x)=\nu\phi_Y(T')-\tilde\phi_X(\tau)(x)$
for every $x\in X$.\ \Qed

\medskip

\qquad\grheadd\ Define
$\Theta_1^*:\tilde{\Cal T}_X^{\Bbb N}\to\Cal T^{\Bbb R}$ by saying that

\Centerline{$\Theta_1^*(\sequencen{\tau_n})(\alpha)
=\tilde\Theta_1(\sequencen{\tau_n(\alpha)})$}

\noindent for every sequence $\sequencen{\tau_n}$ in $\tilde{\Cal T}_X$, so
that $\Theta_1^*(\sequencen{\tau_n})\in\tilde{\Cal T}_X$ and

\Centerline{$\tilde\phi_X(\tilde\Theta_1(\sequencen{\tau_n}))
=\sup_{n\in\Bbb N}\tilde\phi_X(\tau_n)$}

\noindent whenever $\sequencen{\tau_n}$ is a sequence in
$\tilde{\Cal T}_X$ such that $\sup_{n\in\Bbb N}\tilde\phi_X(\tau_n)$ is
defined in $\BbbR^X$.

\medskip

\qquad\grheade\ As in 562Ob,
we can find a function $\Theta^*:\Cal T\times X\to\Cal T$ such that

\Centerline{$\phi_Y(\Theta^*(T,x))=\{y:(x,y)\in\phi(T)\}$}

\noindent for $T\in\Cal T$ and $x\in X$.

\medskip

\quad{\bf (ii)} If $W\subseteq X\times Y$ is open and $F\in\Cal B_c(Y)$,
$x\mapsto\nu(F\cap W[\{x\}]):X\to[0,\infty]$ is lower semi-continuous.
\Prf\ Take $\gamma\in\Bbb R$ and consider
$G=\{x:\nu(F\cap W[\{x\}])>\gamma\}$.
Given $x\in G$ let $K$ be $\{(m,n):x\in U_m$, $U_m\times V_n\subseteq W\}$;
then $W[\{x\}]=\bigcup_{(m,n)\in K}V_n$.   Now $\family{(m,n)}{K}{V_n}$ and
$\family{(m,n)}{K}{F\cap V_n}$ are codable families (562J),
so there is a finite set
$L\subseteq K$ such that $\nu(\bigcup_{(m,n)\in L}F\cap V_n)>\gamma$
(563B(a-ii)).   In this case,
$H=X\cap\bigcap_{(m,n)\in L}U_m$ is an open neighbourhood of $x$ included
in $G$, and $\nu(F\cap W[\{x'\}])>\gamma$ for every $x'\in H$.
As $x$ is arbitrary, $G$ is open;  as $\gamma$ is arbitrary, the
function is lower semi-continuous.\ \Qed

\medskip

\quad{\bf (iii)} For $T$, $T'\in\Cal T$ and $x\in X$, set

\Centerline{$h_{TT'}(x)=\nu\{y:y\in\phi_Y(T')$, $(x,y)\in\phi(T)\}
=\nu(\phi_Y(T')\cap\phi_Y(\Theta^*(T,x)))$.}

\noindent Then there is a function
$\Theta:\Cal T\times\Cal T\to\tilde{\Cal T}_X$ such that

\Centerline{$\tilde\phi_X(\Theta(T,T'))=h_{TT'}$}

\noindent whenever $T$, $T'\in\Cal T$ are such that
$\nu\phi_Y(T')$ is finite.   \Prf\
If $\nu\phi_Y(T')=\infty$ set $\Theta(T,T')=\tau_0$.
For other $T'$, build $\Theta$ by
induction on the rank of $T$, as usual.
If $r(T)\le 1$, then $\phi(T)$ is open;  by (ii), $h_{TT'}$ is
lower semi-continuous, therefore resolvable (562Qa).   So we can set
$\Theta(T,T')=\tilde\psi_X(h_{TT'})$.

For the inductive step to $r(T)\ge 2$, set
$A_T=\{n:\fraction{n}\in T\}$, so that

$$\eqalign{\phi(T)
&=\bigcup_{n\in A_T}(X\times Y)\setminus\phi(T_{\fraction{n}})\cr
&=\bigcup_{m\in\Bbb N}(X\times Y)
  \setminus((X\times Y)\cap\bigcap_{n\in A_T\cap m}
     \phi(T_{\fraction{n}}))\cr
&=\bigcup_{m\in\Bbb N}(X\times Y)
    \setminus\phi(\Theta_2^*(\family{n}{A_T\cap m}{T_{\fraction{n}}}))
\cr}$$

\noindent and

\Centerline{$h_{TT'}(x)
=\lim_{m\to\infty}\nu\phi_Y(T')-h_{T^{(m)}T'}(x)
=\sup_{m\in\Bbb N}\nu\phi_Y(T')-h_{T^{(m)}T'}(x)$}

\noindent for every $x$, where

\Centerline{$T^{(m)}=\Theta_2^*(\family{n}{A_T\cap m}{T_{\fraction{n}}})$,
\quad$\phi(T^{(m)})
=(X\times Y)\cap\bigcap_{n\in A_T\cap m}\phi(T_{\fraction{n}})$}

\noindent for $m\in\Bbb N$.   Now $r(T^{(m)})<r(T)$ for every $m$,
so each $\Theta(T^{(m)},T')$ has been defined, and we can speak of
$\Theta_2(\Theta(T^{(m)},T'),\penalty-100T')$ for each $m$;  we shall have

$$\eqalign{\tilde\phi_X(\Theta_2(\Theta(T^{(m)},T'),T'))(x)
&=\nu\phi_Y(T')-\tilde\phi_X(\Theta(T^{(m)},T'))(x)
=\nu\phi_Y(T')-h_{T^{(m)}T'}(x)\cr
&=\nu\{y:y\in\phi_Y(T'),\,(x,y)\in\bigcup_{n\in A_T\cap m}
(X\times Y)\setminus\phi(T_{\fraction{n}})\}\cr}$$

\noindent for $m\in\Bbb N$ and $x\in X$.   So if we set

\Centerline{$\Theta(T,T')
=\Theta_1^*(\sequence{m}{\Theta_2(\Theta(T^{(m)},T'),T')})$,}

\noindent we shall have

$$\eqalign{\tilde\phi_X(\Theta(T,T'))
&=\sup_{m\in\Bbb N}\tilde\phi_X(\Theta_2(\Theta(T^{(m)},T'),T'))\cr
&=\sup_{m\in\Bbb N}(\nu\phi_Y(T'))\chi X
   -\tilde\phi_X(\Theta(T^{(m)},T'))\cr
&=\sup_{m\in\Bbb N}(\nu\phi_Y(T'))\chi X-h_{T^{(m)}T'}
=h_{TT'},\cr}$$

\noindent as required for the induction to proceed.\ \Qed

\medskip

\quad{\bf (iv)} Thus we see that $h_{TT'}\in\eusm L^0(\mu)$ whenever
$T$, $T'\in\Cal T$ and $\nu\phi_Y(T')$ is finite.

\medskip

\quad{\bf (v)}
Let $\Cal B_c(Y)^f$ be the ring of subsets of $Y$ of finite measure.
For $F\in\Cal B_c(Y)^f$ and $W\in\Cal B_c(X\times Y)$ we have $T$,
$T'\in\Cal T$ such that $\phi(T)=W$ and $\phi_Y(T')=F$, and now
$\nu(F\cap W[\{x\}])=h_{TT'}(x)$ for every $x\in X$.   So
we have a functional
$\lambda_F:\Cal B_c(X\times Y)\to[0,\infty]$ defined by saying that

$$\eqalign{\lambda_FW&=\int\nu(F\cap W[\{x\}])\mu(dx)
  \text{ if the integral is defined in }\Bbb R,\cr
&=\infty\text{ otherwise}.\cr}$$

\noindent Of course $\lambda_F$ is additive.   If $E\in\Cal B_c(X)$ and
$F'\in\Cal B_c(Y)$, then

$$\eqalign{\lambda_F(E\times F')
&=0=\mu E\cdot\nu(F\cap F')\text{ if }\nu(F\cap F')=0,\cr
&=\int\nu(F\cap F')\chi E\,d\mu=\mu E\cdot\nu(F\cap F')
   \text{ if }\mu E<\infty,\cr
&=\infty=\mu E\cdot\nu(F\cap F')\text{ if }\mu E=\infty
   \text{ and }\nu(F\cap F')>0.\cr}$$

\noindent (To see that
$E\times F'\in\Cal B_c(X\times Y)$, use 562Mc.)

\medskip

\quad{\bf (vi)} Now suppose that $\sequencen{W_n}$ is a codable disjoint
sequence in $\Cal B_c(X\times Y)$ with union $W$, and that
$F\in\Cal B_c(Y)^f$.   We surely have
$\lambda_FW\ge\sum_{n=0}^{\infty}\lambda_FW_n$.   If
$\sum_{n=0}^{\infty}\lambda_FW_n$ is finite, let $\sequencen{T_n}$ be a
sequence in $\Cal T$ such that $\phi(T_n)=W_n$ for each $n$, and take
$T'\in\Cal T$ such that $\phi_Y(T')=F$.   Then
$\sequencen{h_{T_nT'}}=\sequencen{\tilde\phi_X(\Theta(T_n,T'))}$
is a codable sequence of integrable Borel functions,
so 564Fe tells us that the sum of the
integrals is the integral of the sum;  but

$$\eqalign{\sum_{n=0}^{\infty}h_{T_nT'}(x)
&=\sum_{n=0}^{\infty}\nu(\phi_Y(T')\cap\phi_Y(\Theta^*(T_n,x)))
=\nu(\bigcup_{n\in\Bbb N}\phi_Y(T')\cap\phi_Y(\Theta^*(T_n,x)))\cr
&=\nu(\bigcup_{n\in\Bbb N}F\cap W_n[\{x\}])
=\nu(F\cap W[\{x\}])\cr}$$

\noindent for each $x$, so we have

$$\eqalign{\lambda_FW
&=\int\nu(F\cap W[\{x\}])\mu(dx)
=\int\sum_{n=0}^{\infty}h_{T_nT'}d\mu\cr
&=\sum_{n=0}^{\infty}\int h_{T_nT'}d\mu
=\sum_{n=0}^{\infty}\lambda_FW_n.\cr}$$

\noindent As $\sequencen{W_n}$ is arbitrary, $\lambda_F$ is a Borel-coded
measure.

\medskip

\quad{\bf (vii)} If $W\in\Cal B_c(X\times Y)$
and $F\subseteq F'$ in $\Cal B_c(Y)^f$, then

$$\eqalignno{\lambda_FW
&=\int\nu(F\cap W[\{x\}])\mu(dx)\cr
\displaycause{counting $\int h\,d\mu$ as $\infty$ for a non-negative
function $h\in\eusm L^0(\mu)\setminus\eusm L^1(\mu)$}
&=\int\nu(F'\cap(W\cap(X\times F))[\{x\}])\mu(dx)
=\lambda_{F'}(W\cap(X\times F))
\le\lambda_{F'}W.\cr}$$

\noindent Thus $\langle\lambda_F(W)\rangle_{F\in\Cal B_c(Y)^f}$ is an
upwards-directed family for each $W\in\Cal B_c(X\times Y)$;  let
$\lambda W$ be its supremum.   Then $\lambda$ is a Borel-coded measure on
$X\times Y$ (563E).   Also

\Centerline{$\lambda(E\times F)
=\lambda_F(E\times F)=\mu E\cdot\nu_FF=\mu E\cdot\nu F$}

\noindent whenever $E\in\Cal B_c(X)$ and $F\in\Cal B_c(Y)^f$
have finite measure.
For other measurable $E$ and $F$, if either is negligible then
$\lambda(E\times F)=0$, while if one has infinite measure and the other has
non-zero measure then $\lambda(E\times F)=\infty$ because $\mu$ and $\nu$
are both semi-finite.

Observe that the construction ensures that if $\lambda W<\infty$ and
$W\subseteq X\times F$ for some $F\in\Cal B_c(Y)^f$, then
$\lambda W=\int\nu W[\{x\}]\mu(dx)$.

\medskip

{\bf (b)} Now suppose that $\nu$ is codably $\sigma$-finite.

\medskip

\quad{\bf (i)} Let
$\sequencen{F_n}$ be a codable sequence in $\Cal B_c(Y)^f$ covering $Y$;
since $\sequencen{\bigcup_{i\le n}F_i}$ also is codable, we can suppose
that $\sequencen{F_n}$ is non-decreasing.   Let $\sequencen{T'_n}$ be
a sequence in $\Cal T$ such that $\phi_Y(T'_n)=F_n$ for each $n$.   By
562Mc, as usual, $\sequencen{X\times F_n}$ is a codable sequence in
$\Cal B_c(X\times Y)$, so
$\lambda W=\sup_{n\in\Bbb N}\lambda(W\cap(X\times F_n))$ whenever
$\lambda$ measures $W$.

\medskip

\quad{\bf (ii)} Let
$f:X\times Y\to\coint{0,\infty}$ be an integrable codable Borel function.
Then $\iint f(x,y)\nu(dy)\mu(dx)$ is defined and equal to
$\int fd\lambda$.

\quad\Prf\grheada\ For $n$, $k\in\Bbb N$ set

\Centerline{$W_{nk}=\{(x,y):y\in F_n$, $f(x,y)\ge 2^{-n}k\}$;}

\noindent then $\langle W_{nk}\rangle_{n,k\in\Bbb N}$ is a codable family
in $\Cal B_c(X\times Y)$.   Let $\langle T_{nk}\rangle_{n,k\in\Bbb N}$ be a
family in $\Cal T$ such that $W_{nk}=\phi(T_{nk})$ for $n$, $k\in\Bbb N$.
For $n\in\Bbb N$, define $v_n:X\times Y\to\Bbb R$ by setting

\Centerline{$v_n=2^{-n}\sum_{k=1}^{4^n}\chi W_{nk}$;}

\noindent then $\sequencen{v_n}$ is a codable sequence of codable Borel
functions on $X\times Y$.   Moreover, setting $v_{nx}(y)=v_n(x,y)$,
$\sequencen{v_{nx}}$ is a codable sequence of codable Borel functions on
$Y$, for each $x\in X$.   Now set

\Centerline{$u_{nk}(x)=\nu W_{nk}[\{x\}]$,
\quad$u_n(x)=\int v_n(x,y)\nu(dy)$}

\noindent for $x\in X$ and $n$, $k\in\Bbb N$.   Then, in the language of
part (a) of this proof,

\Centerline{$u_{nk}=\tilde\phi_X(\Theta(T_{nk},T'_n))$}

\noindent for all $n$ and $k$, so $\langle u_{nk}\rangle_{n,k\in\Bbb N}$ is
a codable family of codable Borel functions on $X$.   Since

\Centerline{$u_n=2^{-n}\sum_{k=1}^{4^n}u_{nk}$}

\noindent for each $n$, $\sequencen{u_n}$ is a codable sequence of codable
Borel functions on $X$.

\medskip

\qquad\grheadb\ Next, for each $n\in\Bbb N$,

$$\eqalignno{\int u_nd\mu
&=2^{-n}\sum_{k=1}^{4^n}\int\nu W_{nk}[\{x\}]\mu(dx)
=2^{-n}\sum_{k=1}^{4^n}\lambda W_{nk}\cr
\displaycause{by the final remark in part (a) of the proof}
&=\int v_nd\lambda.\cr}$$

\noindent At this point, observe that $\sequencen{v_n}$ is a non-decreasing
codable sequence with limit $f$.   So

\Centerline{$\lim_{n\to\infty}\int u_nd\mu
=\lim_{n\to\infty}\int v_nd\lambda=\int fd\lambda$}

\noindent is finite;  since $\sequencen{u_n}$ also is non-decreasing,
$u(x)=\lim_{n\to\infty}u_n(x)$ is finite for $\mu$-almost all $x$, and

\Centerline{$\int u\,d\mu
=\lim_{n\to\infty}\int u_nd\mu=\int fd\lambda$}

\noindent (564Fa).   On the other hand, for each $x\in X$,
$\sequencen{v_{nx}}$ is a non-decreasing codable sequence with limit
$f_x$, where $f_x(y)=f(x,y)$ for $y\in Y$;  so

\Centerline{$u(x)=\lim_{n\to\infty}\int v_{nx}d\nu=\int f_xd\nu$}

\noindent for almost all $x$, and

\Centerline{$\iint f(x,y)\nu(dy)\mu(dx)
=\iint f_xd\nu\mu(dx)=\int u\,d\mu=\int fd\lambda$.  \Qed}

\medskip

\quad{\bf (iii)} It follows at once, taking the difference of positive and
negative parts, that

\Centerline{$\iint f(x,y)\nu(dy)\mu(dx)=\int fd\lambda$}

\noindent for every $\lambda$-integrable codable Borel function $f$.

\medskip

\quad{\bf (iv)} In particular (or more directly), if
$W\in\Cal B_c(X\times Y)$ is $\lambda$-negligible,
then $\mu$-almost every vertical section of $W$ is $\nu$-negligible.   So
starting from a general $\lambda$-integrable function $f$, we move to a
codable Borel function $g$ such that $f\eae g$;  now
$\int f(x,y)\nu(dy)$ must be defined and equal to $\int g(x,y)\nu(dy)$ for
almost every $x$, and

\Centerline{$\iint f(x,y)\nu(dy)\mu(dx)
=\iint g(x,y)\nu(dy)\mu(dx)
=\int gd\lambda=\int fd\lambda$.}

\noindent This completes the proof of (b).

\medskip

{\bf (c)} Let $\sequencen{E_n}$, $\sequencen{F_n}$ be codable sequences of
sets of finite measure covering $X$, $Y$ respectively;  we may suppose that
both sequences are non-decreasing.   Then
$\sequencen{E_n\times F_n}=\sequencen{(E_n\times Y)\cap(X\times F_n)}$ is a
codable sequence (562Mc).   Suppose that
$\lambda$, $\lambda'$ are two Borel-coded measures on $X\times Y$ agreeing
on measurable rectangles.
For each $n\in\Bbb N$ let $\lambda_n$, $\lambda'_n$ be
the totally finite measures defined by setting

\Centerline{$\lambda_nW=\lambda(W\cap(E_n\times F_n))$,
\quad$\lambda'_nW=\lambda'(W\cap(E_n\times F_n))$}

\noindent for $W\in\Cal B_c(X\times Y)$.   Now, given $n$, set
$\Cal W_n=\{W:W\in\Cal B_c(X\times Y)$, $\lambda_nW=\lambda'_nW\}$.
Then $W\cup W'\in\Cal W_n$ whenever $W$, $W'\in\Cal W_n$ are disjoint,
and $E\times F\in\Cal W_n$ whenever $E\in\Cal B_c(X)$ and
$F\in\Cal B_c(Y)$.   So $\Cal W_n$ includes the algebra of subsets of
$X\times Y$ generated by
$\{E\times F:E\in\Cal B_c(X)$, $F\in\Cal B_c(Y)\}$.
In particular, $\Cal W_n$ includes any
set of the form $\bigcup_{(i,j)\in K}U_i\times V_j$ where
$K\subseteq\Bbb N\times\Bbb N$ is finite.   But any open subset of
$X\times Y$ is expressible as the union of a non-decreasing
codable sequence of such sets,
so also belongs to $\Cal W_n$.   By 563Fg, $\lambda_n=\lambda'_n$.

This is true for every $n\in\Bbb N$.   Since

\Centerline{$\lambda W=\sup_{n\in\Bbb N}\lambda_nW$,
\quad$\lambda'W=\sup_{n\in\Bbb N}\lambda'_nW$}

\noindent for every $W\in\Cal B_c(X\times Y)$, $\lambda=\lambda'$, as
claimed.
}%end of proof of 564N

\leader{564O}{Theorem} Let $\sequencen{(X_k,\rho_k)}$ be a sequence of
complete metric spaces, and suppose that we have a double sequence
$\langle U_{ki}\rangle_{k,i\in\Bbb N}$ such that $\{U_{ki}:i\in\Bbb N\}$ is
a base for the topology of $X_k$
for each $k$.   Let $\sequencen{\mu_k}$ be a sequence such
that $\mu_k$ is a Borel-coded probability measure on $X_k$ for each $k$.
Set $X=\prod_{k\in\Bbb N}X_k$.   Then $X$ is a Polish space and
there is a Borel-coded probability measure $\lambda$ on $X$
such that $\lambda(\prod_{k\in\Bbb N}E_k)=\prod_{k\in\Bbb N}\mu_kE_k$
whenever $\sequence{k}{E_k}\in\prod_{k\in\Bbb N}\Cal B_c(X_k)$ and
$\{k:E_k\ne X_k\}$ is finite.

\proof{{\bf (a)(i)}
Of course $X$ is Polish;  we have a complete metric $\rho$
on $X$ defined by saying that
$\rho(x,y)=\sup_{k\in\Bbb N}\min(2^{-k},\penalty-100\rho_k(x(k),y(k)))$
for $x$, $y\in X$, and a countable base generated by sets of the form
$\{x:x(k)\in U_{ki}\}$.

\medskip

\quad{\bf (ii)} Writing $\Cal F_k$ for the family of closed subsets of
$X_k$ for $k\in\Bbb N$, we have a choice function $\zeta$ on
$\bigcup_{k\in\Bbb N}\Cal F_k\setminus\{\emptyset\}$.   \Prf\ Given
a non-empty
$F\in\bigcup_{k\in\Bbb N}\Cal F_k$, take the first $k$ such that
$F\in\Cal F_k$, and define $\sequence{m}{F_m}$, $\sequence{m}{i_m}$ by
saying that

\inset{$F_0=F$,

$i_m=\min\{i:i\in\Bbb N$, $U_{ki}\cap F_m\ne\emptyset$,
$\diam U_{ki}\le 2^{-m}\}$}

\noindent (taking the diameter as measured by $\rho_k$, of course),

\inset{$F_{m+1}=\overline{F_m\cap U_{ki_m}}$}

\noindent for each $m$.   Now $\sequence{m}{F_m}$ generates a Cauchy
filter in $X_k$ which must have a unique limit;  take this limit for
$\zeta(F)$.\ \Qed

\medskip

{\bf (b)(i)} Let
$\Tau=\bigotimes_{k\in\Bbb N}\Cal B_c(X_k)$ be the algebra of subsets of
$X$ generated by $\{\{x:x(k)\in E\}:k\in\Bbb N$, $E\in\Cal B_c(X_k)\}$.
Note that all these sets belong to $\Cal B_c(X)$, by 562Md, so
$\Tau\subseteq\Cal B_c(X)$.   Set

\Centerline{$\Cal C
=\{\prod_{k\in\Bbb N}E_k:E_k\in\Cal B_c(X_k)$ for every $k\in\Bbb N$,
$\{k:E_k\ne X_k\}$ is finite$\}$,}

\Centerline{$\Cal C_o
=\{\prod_{k\in\Bbb N}G_k:G_k\subseteq X_k$ is open for every
$k\in\Bbb N$, $\{k:G_k\ne X_k\}$ is finite$\}$,}

\Centerline{$\Cal C_c
=\{\prod_{k\in\Bbb N}F_k:F_k\subseteq X_k$ is closed
for every $k\in\Bbb N$, $\{k:F_k\ne X_k\}$ is finite$\}$.}

\noindent Then every member of $\Tau$ can be
expressed as the union of a finite disjoint family in $\Cal C$.
$\overline{C}\in\Cal C_c$ for every $C\in\Cal C$, so the closure of any
member of $\Tau$ can be expressed as the union
of finitely many members of $\Cal C_c$ and belongs to $\Tau$.
The complement of a member of $\Cal C_c$
can be expressed as the union of finitely many members of $\Cal C_o$, so
any open set belonging to $\Tau$ can be expressed as the union of
finitely many members of $\Cal C_o$.

\medskip

\quad{\bf (ii)} For $m\in\Bbb N$ write
$\Tau_m=\bigotimes_{k\ge m}\Cal B_c(X_k)$ for the algebra of
subsets of $\prod_{k\ge m}X_k$ generated by sets of the form
$\{x:x(k)\in E_k\}$ for $k\ge m$ and $E_k\in\Cal B_c(X_k)$.   Then we have an
additive functional $\nu_m:\Tau_m\to[0,1]$ defined by saying that
$\nu_m(\prod_{k\ge m}E_k)=\prod_{k=m}^{\infty}\mu_kE_k$ whenever
$E_k\in\Cal B_c(X_k)$ for every $k\ge m$ and $\{k:E_k\ne X_k\}$ is finite
(326E).   Now if $m\in\Bbb N$ and $W\in\Tau_m$ then

\Centerline{$\nu_mW=\int\nu_{m+1}\{v:\fraction{t}^{\smallfrown}v\in W\}\mu_m(dt)$,}

\noindent where I write $\fraction{t}^{\smallfrown}v\in\prod_{k\ge m}X_k$
for
$(t,v(m+1),v(m+2),\ldots)$.   \Prf\ This is elementary for cylinder sets
$W=\prod_{k\ge m}E_k$;  now any other member of $\Tau_m$ is
expressible as a finite disjoint union of such sets.\ \Qed

\medskip

{\bf (c)(i)} For open sets $W\subseteq X$ define

\Centerline{$\lambda_0 W
=\sup\{\nu_0V:V\in\Tau$, $\overline{V}\subseteq W\}$.}

\noindent Then if $\sequencen{W_n}$ is a non-decreasing sequence of open
sets with union $X$, $\lim_{n\to\infty}\lambda_0 W_n=1$.   \Prf\ Starting
from the double sequence $\langle U_{ki}\rangle_{k,i\in\Bbb N}$, it is easy
to build a sequence $\sequencen{U_n}$ in $\Cal C_o$ which runs over a base
for the topology of $X$.   Set
$W'_n=\bigcup\{U_i:i\le n$, $\overline{U}_i\subseteq W_n\}$ for each $n$;
then $\sequencen{W'_n}$ is a non-decreasing sequence of open sets belonging
to $\Tau=\Tau_0$, and $\bigcup_{n\in\Bbb N}W'_n=X$.   \Quer\
Suppose, if possible, that $\lim_{n\to\infty}\nu_0W'_n\le 1-2^{-l}$ for
some $l\in\Bbb N$.   Then we can
define $\sequence{k}{t_k}$ inductively, as follows.   The inductive
hypothesis will be that $\nu_mV_{mn}\le 1-2^{-l-m}$ for every $n$, where

\Centerline{$V_{mn}=\{v:v\in\prod_{k\ge m}X_k$,
$\ofamily{k}{m}{t_k}\cup v\in W'_n\}$.}

\noindent In this case, define $f_{mn}:X_m\to[0,1]$ by setting

\Centerline{$f_{mn}(t)
=\nu_{m+1}\{w:w\in\prod_{k\ge m+1}X_k$,
$\fraction{t}^{\smallfrown}w\in V_{mn}\}$.}

\noindent By (b-ii),
$\nu_mV_{mn}=\int f_{mn}d\mu_m$ for every $m$, while $\sequencen{f_{mn}}$
is non-decreasing.

Because every $W'_n$ is a finite union of open cylinder sets, so
is $V_{mn}$, and $f_{mn}$ is lower semi-continuous, therefore resolvable;
so

\Centerline{$\int\sup_{n\in\Bbb N}f_{mn}d\mu_m
=\sup_{n\in\Bbb N}\int f_{mn}d\mu_m
=\lim_{n\to\infty}\nu_mV_{mn}\le 1-2^{-l-m}$.}

\noindent The set
$F=\{t:\sup_{n\in\Bbb N}f_{mn}(t)\le 1-2^{-l-m-1}\}$ must be closed and
non-empty, and we can
set $t_m=\zeta(F)$, where $\zeta$ is the choice function of (a-ii).
In this case,

\Centerline{$V_{m+1,n}=\{w:\fraction{t_m}^{\smallfrown}w\in V_{mn}\}$,
\quad$\nu_{m+1}V_{m+1,n}=f_{mn}(t_m)\le 1-2^{-l-m-1}$}

\noindent for every $n$, and the induction continues.

At the end of the induction, however, $x=\sequence{k}{t_k}$ belongs to $X$,
so belongs to $W'_n$ for some $n$.   There must be an $m$ such that $W'_n$
is determined by coordinates less than $m$, and now
$V_{mn}=\prod_{k\ge m}X_k$, so $\nu_mV_{mn}=1$;  which is supposed to be
impossible.\ \Bang

We conclude that

\Centerline{$1=\lim_{n\to\infty}\nu_0W'_n
=\lim_{n\to\infty}\nu_0\overline{W'_n}
=\lim_{n\to\infty}\lambda_0 W_n$}

\noindent because $\overline{W'_n}$ is a closed member of $\Tau$ included
in $W_n$ for each $n$.\ \Qed

\medskip

\quad{\bf (ii)} $\lambda_0$ satisfies the conditions of 563H.   \Prf\

\medskip

\qquad\grheada\ Of course $\lambda_0\emptyset=\emptyset$ and
$\lambda_0W\le\lambda_0W'$ whenever $W\subseteq W'$;  also $\lambda_0X=1$
is finite.

\medskip

\qquad\grheadb\ Suppose that $\sequencen{W_n}$ is a non-decreasing sequence
of open sets in $X$ with union $W$, and $\epsilon>0$.   Then
there is a closed
$V\in\Tau$ such that $V\subseteq W$ and $\nu_0V\ge\lambda_0W-\epsilon$.
Set $W'_n=(X\setminus V)\cup W_n$ for each $n$;  then $\sequencen{W'_n}$
is a non-decreasing sequence of open sets with union $X$, so by (i)
there are an
$n\in\Bbb N$ such that $\lambda_0W'_n\ge 1-\epsilon$, and a closed
$V'\in\Tau$ such that $V'\subseteq W'_n$ and $\nu_0V'\ge 1-2\epsilon$.
Now $V\cap V'$ is a closed member of $\Tau$ included in $W_n$ and

\Centerline{$\lambda_0W_n
\ge\nu_0(V\cap V')\ge\nu_0V-2\epsilon\ge\lambda_0W-3\epsilon$.}

\noindent As $\epsilon$ is arbitrary,
$\lambda_0W\le\lim_{n\to\infty}\lambda_0W_n$;  the reverse inequality is
trivial, so we have equality.

\medskip

\qquad\grheadc\ Let
$W$, $W'\subseteq X$ be open sets.   As in (i), we have
non-decreasing sequences
$\sequencen{W_n}$, $\sequencen{W'_n}$ of open members of $\Tau$ such that

\Centerline{$W=\bigcup_{n\in\Bbb N}W_n=\bigcup_{n\in\Bbb N}\overline{W}_n$,
\quad$W'=\bigcup_{n\in\Bbb N}W'_n=\bigcup_{n\in\Bbb N}\overline{W'_n}$.}

\noindent In this case

\Centerline{$W\cap W'
=\bigcup_{n\in\Bbb N}W_n\cap W'_n
=\bigcup_{n\in\Bbb N}\overline{W_n\cap W'_n}$,}

\Centerline{$W\cup W'=\bigcup_{n\in\Bbb N}W_n\cup W'_n
=\bigcup_{n\in\Bbb N}\overline{W_n\cup W'_n}$.}

\noindent Also

\Centerline{$\lambda_0W=\lim_{n\to\infty}\lambda_0W_n
\le\lim_{n\to\infty}\nu_0W_n
\le\lim_{n\to\infty}\nu_0\overline{W}_n\le\lambda_0W$,}

\noindent so these are all equal;  the same applies to the sequences
converging to $W'$, $W\cap W'$ and $W\cup W'$, so

$$\eqalign{\lambda_0W+\lambda_0W'
&=\lim_{n\to\infty}\nu_0W_n+\nu_0W'_n\cr
&=\lim_{n\to\infty}\nu_0(W_n\cap W'_n)+\nu_0(W_n\cup W'_n)\cr
&=\lambda_0(W\cap W')+\lambda_0(W\cup W'). \text{ \Qed}\cr}$$

\medskip

{\bf (d)} By 563H, we have a Borel-coded measure $\lambda$ on $X$
extending $\lambda_0$.   Now $\lambda$ extends $\nu_0$.
\Prf\ If $C\in\Cal C$ and $\epsilon>0$, express $C$ as
$\prod_{k\in\Bbb N}E_k$ where $E_k\in\Cal B_c(X_k)$ for every $k$ and there is
an $m$ such that $E_k=X_k$ for $k>m$.   For each $k\le m$, there is an open
set $G_k\supseteq E_k$ such that
$\mu_kG_k\le\mu_kE_k+\bover{\epsilon}{m+1}$ (563Fd again);
setting $G_k=X_k$ for $k>m$
and $W=\prod_{k\in\Bbb N}G_k$, $C\subseteq W$ and

\Centerline{$\lambda C\le\lambda W
=\lambda_0W\le\nu_0W=\prod_{k=0}^m\mu_kG_k\le\nu_0C+\epsilon$.}

\noindent As $\epsilon$ is arbitrary, $\lambda C\le\nu_0C$.   This is true
for every $C\in\Cal C$.   As both $\lambda$ and $\nu_0$ are additive,
$\lambda W\le\nu_0W$ for every $W\in\Tau$;  as $\lambda X=\nu_0X=1$,
$\lambda$ agrees with $\nu_0$ on $\Tau$.\ \Qed

In particular, $\lambda$ agrees with $\nu_0$ on $\Cal C$, as required.
}%end of proof of 564O

\exercises{\leader{564X}{Basic exercises (a)}
%\spheader 564Xa
Let $X$ be a
second-countable space and $\mu$ a Borel-coded measure on $X$.
Let $E\in\Cal B_c(X)$ and let $\mu_E$ be the
Borel-coded measure on $X$ defined as in 563Fa.   Show that
$\int fd\mu_E$ is defined and equal to $\int f\times\chi E\,d\mu$ for
every $f\in\eusm L^1(\mu)$.
%564G

\spheader 564Xb\dvAnew{2014} Let $X$ be a topological space, $\mu$ a
Baire-coded measure on $X$, and $f$ a non-negative
integrable real-valued function
defined almost everywhere in $X$.   Set $\nu E=\int f\times\chi E$ for
$E\in\CalBa_c(X)$.   Show that $\nu$ is a Baire-coded measure, and that
$\int g\,d\nu=\int g\times f\,d\mu$ for every $\nu$-integrable $g$, if we
interpret $(g\times f)(x)$ as $0$ when $f(x)=0$ and $g(x)$ is undefined.
(Compare 235K.)
%564G

\spheader 564Xc Let $X$ be a countably compact topological space.   (i)
Show that $C(X)=C_b(X)$.   (ii) Show that every positive linear functional
$f:C(X)\to\Bbb R$ is sequentially smooth.   (iii) Show that a
norm-bounded sequence
$\sequencen{u_n}$ in the normed space $C(X)$ is weakly convergent to $0$
iff it is pointwise convergent to $0$.
(iv) Prove this
without using measure theory.   \Hint{{\smc Fremlin 74}, A2F.   Also see
564Ya.}
%564H

\spheader 564Xd Let $X$ be a topological space and $\mu$ a Baire-coded
measure on $X$.
(i) Describe constructions for normed Riesz spaces $L^p(\mu)$
for $1<p\le\infty$.
(ii) Show that if $X$ is second-countable,
$\mu$ is codably $\sigma$-finite and $1<p<\infty$ then
$L^p(\mu)$ is a Dedekind complete Banach lattice with an order-continuous
norm, while $L^2(\mu)$ is a Hilbert space.
%564K

\spheader 564Xe In 564O, show that $\lambda$ is uniquely defined.   Hence
show that we have commutative and associative laws for the product measure
construction.
%564O

\leader{564Y}{Further exercises (a)}\dvAnew{2014}
%\spheader 564Ya
Let $X$ be a topological space and $\sequencen{f_n}$ a
codable sequence of bounded codable Baire real-valued functions on $X$
such that
$\{\int f_nd\mu:n\in\Bbb N\}$ is bounded for every totally finite
Baire-coded measure $\mu$ on $X$.   (i) Show that if $\sequencen{E_n}$ is a
disjoint codable sequence in $\CalBa_c(X)$ and $\mu$ is a Baire-coded
measure on $X$, then $\lim_{n\to\infty}\int f_n\times\chi E_nd\mu=0$.
(ii) Now suppose in addition that
$\lim_{n\to\infty}f_n(x)=0$ for every $x\in X$.
Show that if $\mu$ is a Baire-coded
measure on $X$, then $\lim_{n\to\infty}\int f_nd\mu=0$.
(iii) Use this result to strengthen (iii) of 564Xc to `a sequence
$\sequencen{u_n}$ in $C(X)$ is weakly convergent to $0$ iff it is bounded
for the weak topology and pointwise convergent to $0$'.
%564Xc 564H 56bits 564Xb

\spheader 564Yb
Let $X$ be a locally compact completely regular topological group.
Show that there is a non-zero \lti\ Baire-coded measure on $X$.
%564I

\spheader 564Yc
Let $I$ be a set and $X=\{0,1\}^I$.   Write $Z$ for
$\{0,1\}^{\Bbb N}$.   For
$\theta:\Bbb N\to I$ define $g_{\theta}:X\to Z$ by setting
$g_{\theta}(x)=x\theta$ for $x\in X$.   Let
$\phi:\Cal T\to\Cal B_c(Z)$ be an interpretation of Borel codes for subsets
of $Z$ defined from a sequence running over a base for the topology of $Z$.
Let $\Sigma$ be the family of
subsets of $X$ of the form $\phi'(\theta,T)=g_{\theta}^{-1}[\phi(T)]$ where
$\theta\in I^{\Bbb N}$ and $T\in\Cal T$;  say that a codable family in
$\Sigma$ is one of the form $\familyiI{\phi'(\theta_i,T_i)}$.
Show that there is a functional $\mu:\Sigma\to[0,1]$ such that
$\mu\emptyset=0$, $\mu(\bigcup_{n\in\Bbb N}E_n)=\sum_{n=0}^{\infty}\mu E_n$
whenever $\sequencen{E_n}$ is a disjoint codable sequence in $\Sigma$, and
$\mu\{x:x\restr J=w\}=2^{-\#(J)}$ whenever $J\subseteq I$ is finite and
$w\in\{0,1\}^J$.
%564N

\spheader 564Yd
Suppose there is a disjoint sequence $\sequencen{I_n}$ of doubleton sets
such that for every function $f$ with domain $\Bbb N$ the set
$\{n:f(n)\in I_n\}$ is finite ({\smc Jech 73}, 4.4).   Set
$I=\bigcup_{n\in\Bbb N}I_n$ and let $\Sigma$ be the algebra of subsets of
$\{0,1\}^I$ determined by coordinates in finite sets.   Let
$\lambda:\Sigma\to[0,1]$ be the additive functional such that
$\lambda\{x:z\subseteq x\}=2^{-k}$ whenever $J\in[I]^k$ and
$z\in\{0,1\}^J$.   Show that there is a sequence $\sequencen{E_n}$ in
$\Sigma$, covering $\{0,1\}^I$, such that
$\sum_{n=0}^{\infty}\lambda E_n<1$.
% E_n=\{x:x\restr I_m$ constant for $n\le m\le 4^{n+1}\}$.
%564N
}%end of exercises

\endnotes{
\Notesheader{564}
In the definitions of 564A, I follow the principles of earlier volumes in
allowing virtually measurable functions with conegligible domains to be
counted as integrable.   But you will see that in
564F and elsewhere I work with real-valued Baire measurable
functions defined everywhere.   The point is that while, if you wish
to work through the basic theorems of Fourier analysis under the new rules,
you will certainly need to deal with functions which are not defined
everywhere, all the main theorems will depend on establishing that you have
sequences of sets and functions which are codable in appropriate senses.
There is no way of coding members of $\eusm L^0$ or $\eusm L^1$ as I have
defined them in 564A.   What you will need to do is to build parallel
structures, so that associated with each almost-everywhere-summable Fourier
series $f(x)=\bover12a_0+\sum_{k=1}^{\infty}a_k\cos kx+b_k\sin kx$ you
have in hand a code $\tau$ for a codable Borel function $\tilde f$ equal
almost everywhere to $f$, together with a code $T$ for a conegligible
codable Borel set $E$ included in $\{x:x\in\dom f$, $f(x)=\tilde f(x)\}$.
Provided that associated with every relevant sequence $\sequencen{f_n}$ you
can define appropriate sequences $\sequencen{\tau_n}$ and
$\sequencen{T_n}$, you can hope to deduce the reqired properties of
$\sequencen{f_n}$ by applying 564F to the sequence coded by
$\sequencen{\tau_n}$.

Of course there are further significant technical differences between the
treatment here and the more orthodox one I have employed elsewhere.
In the ordinary theory, using the axiom of choice whenever convenient,
a measure $\mu$, thought of as a function defined on a $\sigma$-algebra of
sets, carries in itself all the information needed to describe the space
$\eusm L^0(\mu)$.   In the present context, we are dealing with functions
$\mu$ defined on algebras which do not directly code the topologies on
which the definition relies.   So it would be safer to write
$\eusm L^0(\frak T,\mu)$.   But of course what really matters is the
collection of codable families of codable sets,
and perhaps we should be thinking of a
different level of abstraction.   In the proof of 564N I have tried to cast
the proof in a language which might be adaptable to other ways of coding
sets and functions.

From 564K on, most of
the results seem to depend on second-countability;  it may be
that something can be done with spaces which have well-orderable bases.

In the shift from 564Xc(iii) to 564Ya(iii) I find myself asking for a
reason why a weakly bounded sequence in $C(X)$ should be norm-bounded.
As far as I know, there is no useful general result in ZF in this
direction.   But in 564Ya I have suggested a method which will serve in
this special context.

I offer 564Yc and 564Yd as positive and negative examples.   The point is
that in 564Yc there may be few sequences of
functions from $\Bbb N$ to $I$, so that we get few codable sequences of
sets.   Of course, if $I$ is well-orderable then $\{0,1\}^I$ is compact
(561D) and we can use 564H.   For well-orderable $I$,
any continuous
real-valued function on $\{0,1\}^I$ is determined by coordinates in some
countable set, so that the methods of 564H and 564Yc will give the same
measure.
}%end of notes

\discrpage

