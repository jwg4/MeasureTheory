\frfilename{mt314.tex}
\versiondate{26.7.07}

\def\chaptername{Boolean algebras}
\def\sectionname{Order-completeness}

\newsection{314}

The results of \S313 are valid in all Boolean algebras, but
of course are of most value when many suprema and infima exist.
I now set out the most useful definitions which guarantee the existence
of suprema and infima (314A) and work through their elementary
relationships with the concepts introduced so far (314C-314J).   I then
embark on the principal theorems concerning order-complete Boolean
algebras:  the extension theorem for homomorphisms to a Dedekind
complete algebra (314K), the Loomis-Sikorski representation of a
Dedekind $\sigma$-complete algebra as a quotient of a $\sigma$-algebra
of sets (314M), the characterization of Dedekind complete algebras in
terms of their Stone spaces (314S), and the idea of `Dedekind
completion' of a Boolean algebra (314T-314U).
On the way I describe `regular open algebras' (314O-314Q).

\vleader{48pt}{314A}{Definitions}  Let $P$ be a partially ordered set.

(a) $P$ is {\bf Dedekind complete}\cmmnt{, or {\bf order-complete}, or
{\bf conditionally complete}} if every
non-empty subset of $P$ with an upper bound has a least upper bound.

(b) $P$ is {\bf Dedekind $\sigma$-complete}\cmmnt{, or
{\bf $\sigma$-order-complete},} if (i) every countable non-empty subset
of $P$ with an upper bound has a least upper bound (ii) every countable
non-empty subset of $P$ with a lower bound has a greatest lower bound.

\cmmnt{
\leader{314B}{Remarks (a)} I give these definitions in the
widest possible generality because they are in fact of great interest
for general partially ordered sets, even though for the moment we shall
be concerned only with Boolean algebras.   Indeed I have already
presented the same idea in the context of Riesz spaces (241F).

\header{314Bb}{\bf (b)} You will observe that the definition in (a) of
314A is asymmetric, unlike that in (b).   This is because the inverted
form of the definition is equivalent to that given;  that is, $P$ is
Dedekind complete (on the definition 314Aa) iff every non-empty subset
of $P$ with a lower bound has a greatest lower bound.   \prooflet{\Prf\
(i) Suppose that $P$ is Dedekind complete, and that $B\subseteq P$ is
non-empty and bounded below.   Let $A$ be the set of lower bounds for
$B$.   Then $A$ has at least one upper bound (since any member of $B$ is
an upper bound for $A$) and is not empty;  so $a_0=\sup A$ is defined.
Now if $b\in B$, $b$ is an upper bound for $A$, so $a_0\le b$;  thus
$a_0\in A$ and must be the greatest member of $A$, that is, the greatest
lower bound of $B$.   (ii) Similarly, if every non-empty subset of $P$
with a lower bound has a greatest lower bound, $P$ is Dedekind
complete.\ \Qed}

\header{314Bc}{\bf (c)} In the special case of Boolean algebras, we do
not need both halves of the definition 314Ab;  in fact we have,
for any Boolean algebra $\frak A$,

$$\eqalign{A&\text{ is Dedekind }\sigma\text{-complete}\cr
&\iff\text{ every non-empty countable subset of }\frak A
\text{ has a least upper bound}\cr
&\iff\text{ every non-empty countable subset of }\frak A
\text{ has a greatest lower bound}.\cr}$$

\noindent\prooflet{\Prf\ Because $\frak A$ has a least element $0$ and a
greatest element $1$, every subset of $\frak A$ has upper and lower
bounds;  so the two one-sided conditions together are equivalent to
saying that $\frak A$ is Dedekind $\sigma$-complete.   I therefore have
to show that they are equiveridical.   Now if $A\subseteq\frak A$ is a
non-empty countable set, so is $B=\{1\Bsetminus a:a\in A\}$, and

\Centerline{$\inf A=1\Bsetminus\sup B$,
\quad$\sup A=1\Bsetminus\inf B$}

\noindent whenever the right-hand-sides are defined (313A).   So if the
existence of a supremum (resp.\ infimum) of $B$ is guaranteed, so is the
existence of an infimum (resp.\ supremum) of $A$.\ \Qed}

The real point here is of course that $(\frak A,\Bsubseteq)$ is
isomorphic to $(\frak A,\Bsupseteq)$.

\header{314Bd}{\bf (d)} Most specialists in Boolean algebra speak of
`complete', or `$\sigma$-complete', Boolean algebras.   I prefer the
longer phrases `Dedekind complete' and
`Dedekind $\sigma$-complete' because we shall be studying metrics on
Boolean algebras and shall need the notion of metric completeness as
well as that of order-completeness.

\spheader 314Be I have had to make some rather arbitrary choices in the
definition here.   The principal examples of partially ordered set to
which we shall apply these definitions are Boolean algebras and Riesz
spaces, which are all lattices.   Consequently it is not possible to
distinguish in these contexts between the property of Dedekind
completeness, as defined above, and the weaker property, which we might
call `monotone order-completeness',

\inset{(i) whenever $A\subseteq P$ is non-empty, upwards-directed and
bounded above then $A$ has a least upper bound in $P$ (ii) whenever
$A\subseteq P$ is non-empty, downwards-directed and bounded below then
$A$ has a greatest lower bound in $P$.}

\noindent (See 314Xa below.   `Monotone order-completeness' is the
property involved in 314Ya, for instance.)   Nevertheless I am prepared
to say, on the
basis of my own experience of working with other partially ordered sets,
that `Dedekind completeness', as I have defined it, is at least of
sufficient importance to deserve a name.   Note that it does not imply
that $P$ is a lattice, since it allows two elements of $P$ to have no
common upper bound.

\spheader 314Bf The phrase {\bf complete lattice} is sometimes used to
mean a Dedekind complete lattice with greatest and least elements;
equivalently, a Dedekind complete partially ordered set with greatest
and least elements.   Thus a Dedekind complete Boolean algebra is a
complete lattice in this sense, but $\Bbb R$ is not.

\spheader 314Bg The most important Dedekind complete Boolean
algebras (at least from the point of view of measure theory) are the
`measure algebras' of the next chapter.   I shall not pause here to give
other examples, but will proceed directly with the general theory.
}%end of comment

\leader{314C}{Proposition} Let $\frak A$ be a Dedekind $\sigma$-complete
Boolean algebra and $I$ a $\sigma$-ideal of $\frak A$.   Then the
quotient Boolean algebra $\frak A/I$ is Dedekind $\sigma$-complete.

\proof{ I use the description in 314Bc.   Let $B\subseteq\frak A/I$ be a
non-empty countable set.   For each $u\in B$, choose an $a_u\in \frak A$
such that $u=a_u^{\ssbullet}$.   Then $c=\sup_{u\in B}a_u$ is
defined in $\frak A$;  consider $v=c^{\ssbullet}$ in $\frak A/I$.
Because the map $a\mapsto a^{\ssbullet}$ is sequentially
order-continuous (313Qb), $v=\sup B$.
As $B$ is arbitrary, $\frak A/I$ is Dedekind $\sigma$-complete.
}%end of proof of 314C

\leader{314D}{Corollary} Let $X$ be a set, $\Sigma$ a $\sigma$-algebra
of subsets of $X$, and $\Cal I$ a $\sigma$-ideal of subsets of $X$.
Then $\Sigma\cap\Cal I$ is a $\sigma$-ideal of the Boolean algebra
$\Sigma$, and $\Sigma/\Sigma\cap \Cal I$ is Dedekind $\sigma$-complete.

\proof{ Of course $\Sigma$ is Dedekind $\sigma$-complete, because if
$\sequencen{E_n}$ is any sequence in $\Sigma$ then $\bigcup_{n\in\Bbb
N}E_n$ is the least upper bound of $\{E_n:n\in\Bbb N\}$ in $\Sigma$.
It is also easy to see that $\Sigma\cap\Cal I$ is a $\sigma$-ideal of
$\Sigma$, since $F\cap\bigcup_{n\in\Bbb N}E_n\in \Cal I$ whenever
$F\in\Sigma$ and $\sequencen{E_n}$ is a sequence in $\Sigma\cap\Cal I$.
So 314C gives the result.
}%end of proof of 314D

\leader{314E}{Proposition} Let $\frak A$ be a Boolean algebra.

(a) If $\frak A$ is Dedekind complete, then all its order-closed
subalgebras and principal ideals are Dedekind complete.

(b) If $\frak A$ is Dedekind $\sigma$-complete, then all its
$\sigma$-subalgebras and principal ideals are Dedekind
$\sigma$-complete.

\proof{ All we need to note is that if $\frak C$ is either an
order-closed subalgebra or a principal ideal of $\frak A$, and
$B\subseteq\frak C$ is such that $b=\sup B$ is defined in $\frak A$,
then $b\in\frak C$ (see 313E(a-i-$\beta$)), so $b$ is still the supremum
of $B$ in $\frak C$;
while the same is true if $\frak C$ is a $\sigma$-subalgebra and
$B\subseteq\frak C$ is countable, using 313E(a-ii-$\beta$).
}%end of proof of 314E

\leader{314F}{}\cmmnt{ I spell out some further connexions between the
concepts `order-closed set', `order-continuous function' and
`Dedekind complete Boolean algebra' which are elementary without
being quite transparent.

\medskip

\noindent}{\bf Proposition} Let $\frak A$ and $\frak B$ be Boolean
algebras and $\pi:\frak A\to\frak B$ a Boolean homomorphism.

(a)(i) If $\frak A$ is Dedekind complete and $\pi$ is order-continuous,
then $\pi[\frak A]$ is order-closed in $\frak B$.

\quad(ii) If $\frak B$ is Dedekind complete and $\pi$ is injective and
$\pi[\frak A]$ is order-closed then $\pi$ is order-continuous.

(b)(i) If $\frak A$ is Dedekind $\sigma$-complete and $\pi$ is
sequentially order-continuous, then $\pi[\frak A]$ is a
$\sigma$-subalgebra of $\frak B$.

\quad(ii) If $\frak B$ is Dedekind $\sigma$-complete and $\pi$ is
injective and $\pi[\frak A]$ is a $\sigma$-subalgebra of $\frak B$ then
$\pi$ is sequentially order-continuous.

\proof{{\bf (a)(i)} If $B\subseteq\pi[\frak A]$, then
$a_0=\sup(\pi^{-1}[B])$ is defined in $\frak A$;  now

\Centerline{$\pi a_0=\sup(\pi[\pi^{-1}[B]])=\sup B$}

\noindent in $\frak B$ (313L(b-iv)), and of course
$\pi a_0\in\pi[\frak A]$.   By 313E(a-i-$\beta$) again, this is enough
to show that $\pi[\frak A]$ is order-closed in $\frak B$.

\medskip

\quad{\bf (ii)} Suppose that $A\subseteq\frak A$ and $\inf A=0$ in
$\frak A$.   Then $\pi[A]$ has an infimum $b_0$ in $\frak B$, which
belongs to $\pi[\frak A]$ because $\pi[\frak A]$ is an order-closed
subalgebra of $\frak B$ (313E(a-i-$\beta'$)).   Now if $a_0\in\frak A$
is such that $\pi a_0=b_0$, we have

\Centerline{$\pi(a\Bcap a_0)=\pi a\Bcap\pi a_0=\pi a_0$}

\noindent for every $a\in A$, so (because $\pi$ is injective) $a\Bcap
a_0=a_0$ and $a_0\Bsubseteq a$ for every $a\in A$.   But this means that
$a_0=0$ and $b_0=\pi 0=0$.   As $A$ is arbitrary, $\pi$ is
order-continuous (313L(b-ii)).

\medskip

{\bf (b)} Use the same arguments, but with sequences in place of the
sets $B$, $A$ above.
}%end of proof of 314F

\leader{314G}{Corollary} Let $\frak A$ be a Boolean algebra and $\frak B$
a subalgebra of $\frak A$.

(a) If $\frak A$ is Dedekind complete, then $\frak B$ is order-closed
iff it is Dedekind complete in itself and is regularly embedded in
$\frak A$.

(b) If $\frak A$ is Dedekind $\sigma$-complete, then $\frak B$ is a
$\sigma$-subalgebra iff it is Dedekind $\sigma$-complete in itself and
the identity map from $\frak B$ to $\frak A$ is sequentially
order-continuous.

\proof{{\bf (a)} Let $\iota:\frak B\to\frak A$ be the identity map;  then
it is an injective Boolean homomorphism.

\medskip

\quad{\bf (i)} If $\frak B$ is order-closed, then it is Dedekind
complete in itself by 314Ea.   By 314F(a-ii), $\iota:\frak B\to\frak A$
is order-continuous, that is, $\frak B$ is regularly embedded in
$\frak A$.

\medskip

\quad{\bf (ii)} If $\frak B$ is Dedekind complete in itself and
$\iota$ is order-continuous, then $\frak B=\iota[\frak B]$ is order-closed
in $\frak A$ by 314F(a-i).

\medskip

{\bf (b)} Use the same arguments, but with 314Eb and 314Fb in place of
314Ea and 314Fa.
}%end of proof of 314G

\leader{314H}{Corollary} Let $\frak A$ be a Dedekind complete Boolean
algebra, $\frak B$ a Boolean algebra and $\pi:\frak A\to\frak B$ an
order-continuous Boolean homomorphism.   If $C\subseteq\frak A$ and
$\frak C$ is the order-closed subalgebra of $\frak A$ generated by $C$,
then $\pi[\frak C]$ is the order-closed subalgebra of $\frak B$
generated by $\pi[C]$.

\proof{ Let $\frak D$ be the order-closed subalgebra of $\frak B$
generated by $\pi[C]$.   By 313Mb, $\pi[\frak C]\subseteq\frak D$.   On
the other hand, the identity homomorphism $\iota:\frak C\to\frak A$ is
order-continuous, by 314Ga, so $\pi\iota:\frak C\to\frak B$ is
order-continuous, and $\pi[\frak C]=\pi\iota[\frak C]$
is order-closed in $\frak B$, by 314F(a-i).   But since $\pi[C]$ is
surely included in $\pi[\frak C]$, $\frak D$ also is included in
$\pi[\frak C]$.   Accordingly $\pi[\frak C]=\frak D$, as claimed.
}%end of proof of 314H

\leader{314I}{Corollary} (a)
If $\frak A$ is a Dedekind complete Boolean algebra, $\frak B$ is a
Boolean algebra, $\pi:\frak A\to\frak B$ is an injective Boolean
homomorphism and $\pi[\frak A]$ is order-dense in $\frak B$, then $\pi$
is an isomorphism.

(b) If $\frak A$ is a Boolean algebra and
$\frak B$ is an order-dense subalgebra of $\frak A$ which is Dedekind
complete in itself, then $\frak B=\frak A$.

\proof{{\bf (a)} Because $\pi[\frak A]$ is order-dense, it is regularly
embedded in $\frak B$ (313O);
also, the kernel of $\pi$ is $\{0\}$, which is
surely order-closed in $\frak A$, so 313P(a-ii) tells us that $\pi$ is
order-continuous.   By 314F(a-i), $\pi[\frak A]$ is order-closed in
$\frak B$;  being order-dense, it must be the whole of $\frak B$ (313K).
Thus $\pi$ is surjective;  being injective, it is an isomorphism.

\medskip

{\bf (b)} Apply (a) to the identity map from $\frak B$ to $\frak A$.
}%end of proof of 314I

\leader{314J}{}\cmmnt{ When we come to applications of the extension
procedure in 312O, the following will sometimes be needed.

\woddheader{314J}{6}{2}{2}{72pt}

\noindent}{\bf Lemma} Let $\frak A$ be a Boolean algebra and $\frak A_0$
a subalgebra of $\frak A$.   Take any $c\in\frak A$, and set

\Centerline{$\frak A_1
=\{(a\Bcap c)\Bcup(b\Bsetminus c):a,\,b\in\frak A_0\}$,}

\noindent the subalgebra of $\frak A$ generated by
$\frak A_0\cup\{c\}$\cmmnt{ (312N)}.

(a) Suppose that $\frak A$ is Dedekind complete.   If $\frak A_0$ is
order-closed in $\frak A$, so is $\frak A_1$.

(b) Suppose that $\frak A$ is Dedekind $\sigma$-complete.   If
$\frak A_0$ is a $\sigma$-subalgebra of $\frak A$, so is $\frak A_1$.

\proof{{\bf (a)} Let $D$ be any subset of $\frak A_1$.   Set
\Centerline{$E=\{e:e\in\frak A$, there is some $d\in D$ such that
$e\Bsubseteq d\}$,}

\Centerline{$A=\{a:a\in\frak A_0,\,a\Bcap c\in E\}$,
\quad$B=\{b:b\in\frak A_0,\,b\Bsetminus c\in E\}$.}

\noindent Because $\frak A$ is Dedekind complete, $a^*=\sup A$ and
$b^*=\sup B$ are defined in $\frak A$;  because $\frak A_0$ is
order-closed, both belong to $\frak A_0$, so $d^*=(a^*\Bcap
c)\Bcup(b^*\Bsetminus c)$ belongs to $\frak A_1$.

Now if $d\in D$, it is expressible as $(a\Bcap c)\Bcup(b\Bsetminus c)$
for some $a$, $b\in\frak A_0$;  since $a\in A$ and $b\in B$, we have
$a\Bsubseteq a^*$ and $b\Bsubseteq b^*$, so $d\Bsubseteq d^*$.   Thus
$d^*$ is an upper bound for $D$.   On the other hand, if $d'$ is any
other upper bound for $D$ in $\frak A$, it is also an upper bound for
$E$, so we must have

\Centerline{$a^*\Bcap c=\sup_{a\in A}a\Bcap c\Bsubseteq d'$,
\quad $b^*\Bsetminus c=\sup_{b\in B}b\Bsetminus c\Bsubseteq d'$,}

\noindent and $d^*\Bsubseteq d'$.   Thus $d^*=\sup D$.   This shows that
the supremum of any subset of $\frak A_1$ belongs to $\frak A_1$, so
that $\frak A_1$ is order-closed.

\medskip

{\bf (b)} The argument is the same, except that we replace $D$ by a
sequence $\sequencen{d_n}$, and $A$, $B$ by sequences $\sequencen{a_n}$,
$\sequencen{b_n}$ in $\frak A_0$ such that $d_n=(a_n\Bcap
c)\Bcup(b_n\Bsetminus c)$ for every $n$.
}%end of proof of 314J

\leader{314K}{Extension of \dvrocolon{homomorphisms}}\cmmnt{ The
following is one
of the most striking properties of Dedekind complete Boolean algebras.

\medskip

\noindent}{\bf Theorem} Let $\frak A$ be a
Boolean algebra and $\frak B$ a Dedekind complete Boolean algebra.   Let
$\frak A_0$ be a Boolean subalgebra of $\frak A$ and
$\pi_0:\frak A_0\to\frak B$ a Boolean homomorphism.
Then there is a Boolean homomorphism $\pi_1:\frak A\to\frak B$ extending
$\pi_0$.

\proof{{\bf (a)} Let $P$ be the set of all Boolean homomorphisms $\pi$
such that $\dom\pi$ is a Boolean subalgebra of $\frak A$ including
$\frak A_0$ and $\pi$ extends $\pi_0$.  Identify each member of $P$ with
its graph, which is a subset of $\frak A\times\frak B$, and order $P$ by
inclusion, so that $\pi\subseteq\theta$ means just that $\theta$ extends
$\pi$. Then any non-empty totally ordered subset $Q$ of $P$ has an upper
bound in $P$.   \Prf\   Let $\pi^*$ be the simple union of these graphs.
(i) If $(a,b)$ and $(a,b')$ both belong to $\pi^*$, then there are
$\pi$, $\pi'\in Q$ such that $\pi a=b$, $\pi'a=b'$;  now either
$\pi\subseteq\pi'$ or $\pi'\subseteq\pi$;  in either case,
$\theta=\pi\cup\pi'\in Q$, so that

\Centerline{$b=\pi a=\theta a=\pi'a=b'$.}

\noindent This shows that $\pi^*$ is a function.   (ii) Because
$Q\ne\emptyset$,

\Centerline{$\dom\pi_0\subseteq\dom\pi\subseteq\dom\pi^*$}

\noindent for some $\pi\in Q$;  thus $\pi^*$ extends $\pi_0$ (and, in
particular, $0\in\dom\pi^*$).   (iii) Now suppose that $a$,
$a'\in\dom(\pi^*)$.  Then there are $\pi$, $\pi'\in Q$ such that
$a\in\dom\pi$, $a'\in\dom\pi'$;  once again, $\theta=\pi\cup\pi'\in Q$,
so that $a$, $a'\in\dom\theta$, and

\Centerline{$a\Bcap a'\in\dom\theta\subseteq\dom\pi^*$,\quad
$1\Bsetminus a\in\dom\theta\subseteq\dom\pi^*$,}

\Centerline{$\pi^*(a\Bcap a')=\theta(a\Bcap a')
=\theta a\Bcap\theta a'=\pi^*a\Bcap\pi^*a'$,}

\Centerline{$\pi^*(1\Bsetminus a)
=\theta(1\Bsetminus a)
=1\Bsetminus\theta a
=1\Bsetminus\pi^*a$.}

\noindent (iv) This shows that $\dom\pi^*$ is a subalgebra of $\frak A$
and that $\pi^*$ is a Boolean homomorphism, that is, that $\pi^*\in P$;
and of course $\pi^*$ is an upper bound for $Q$ in $P$.\ \Qed

\medskip

{\bf (b)} By Zorn's Lemma, $P$ has a maximal element $\pi_1$ say.

\Quer\ Suppose, if possible, that $\frak A_1=\dom\pi_1$ is not the whole
of $\frak A$;  take $c\in\frak A\setminus\frak A_1$.   Set
$A=\{a:a\in \frak A_1,\,a\Bsubseteq c\}$.
Because $\frak B$ is Dedekind complete,
$d=\sup\pi_1[A]$ is defined in $\frak B$.   If $a'\in\frak A_1$ and
$c\Bsubseteq a'$, then of course $a\Bsubseteq a'$ and
$\pi_1a\Bsubseteq\pi_1a'$ whenever $a\in A$,  so that $\pi_1a'$ is an
upper bound for $\pi_1[A]$, and $d\Bsubseteq\pi_1a'$.

But this means that there is an extension of $\pi_1$ to a Boolean
homomorphism $\pi$ on the Boolean subalgebra of $\frak A$ generated by
$\frak A_1\cup\{c\}$ (312O).   And this $\pi$ must be a member of
$P$ properly extending $\pi_1$, which is supposed to be
maximal.\ \Bang

Thus $\dom\pi_1=\frak A$ and $\pi_1$ is an extension of $\pi_0$ to
$\frak A$, as required.
}%end of proof of 314K

\leader{314L}{}\cmmnt{{\bf The Loomis-Sikorski representation of a
Dedekind
$\sigma$-complete Boolean algebra} The construction in 314D is not only
the commonest way in which new Dedekind $\sigma$-complete Boolean
algebras appear, but is adequate to describe them all.   I start with an
elementary general fact.

\medskip

\noindent}{\bf Lemma} Let $X$ be any topological space, and write
$\Cal M$ for the family of meager subsets of $X$.   Then $\Cal M$ is a
$\sigma$-ideal of subsets of $X$.

\proof{ The point is that if $A\subseteq X$ is nowhere dense, so is
every subset of $A$;  this is obvious, since if $B\subseteq A$ then
$\overline{B}\subseteq\overline{A}$ so
$\interior\overline{B}\subseteq\interior\overline{A}=\emptyset$.
So if $B\subseteq A\in\Cal M$, let $\sequencen{A_n}$ be a sequence of
nowhere dense sets with union $A$;  then $\sequencen{B\cap A_n}$ is a
sequence of nowhere dense sets with union $B$, so $B\in\Cal M$.   If
$\sequencen{A_n}$ is a sequence in $\Cal M$ with union $A$, then for
each $n$ we may choose a sequence $\sequence{m}{A_{nm}}$ of nowhere
dense sets with union $A_n$;  then the countable family $\langle
A_{nm}\rangle_{n,m\in\Bbb N}$ may be re-indexed as a sequence of nowhere
dense sets with union $A$, so $A\in\Cal M$.   Finally, $\emptyset$ is
nowhere dense, so belongs to $\Cal M$.
}%end of proof of 314L

\leader{314M}{Theorem} Let $\frak A$ be a Dedekind $\sigma$-complete
Boolean algebra, and $Z$ its Stone space.   Let $\Cal E$ be the algebra
of open-and-closed subsets of $Z$, and $\Cal M$ the $\sigma$-ideal of
meager subsets of $Z$.   Then
$\Sigma=\{E\symmdiff A:E\in\Cal E,\,A\in\Cal M\}$ is a $\sigma$-algebra
of subsets of $Z$, $\Cal M$ is a
$\sigma$-ideal of $\Sigma$, and $\frak A$
is isomorphic, as Boolean algebra, to $\Sigma/\Cal M$.

\proof{{\bf (a)} I start by showing that $\Sigma$ is a
$\sigma$-algebra.   \Prf\  Of course
$\emptyset=\emptyset\symmdiff\emptyset\in\Sigma$.   If $F\in\Sigma$,
express it as $E\symmdiff A$ where $E\in\Cal E$, $A\in\Cal M$;  then
$Z\setminus F=(Z\setminus E)\symmdiff A\in\Sigma$.

If $\sequencen{F_n}$ is a sequence in $\Sigma$, express each $F_n$ as
$E_n\symmdiff A_n$, where $E_n\in\Cal E$ and $A_n\in\Cal M$.   Now each
$E_n$ is expressible as $\widehat a_n$, where $a_n\in\frak A$.
Because $\frak A$ is Dedekind $\sigma$-complete,
$a=\sup_{n\in\Bbb N}a_n$ is
defined in $\frak A$.   Set $E=\widehat a\in\Cal E$.   By 313Ca,
$E=\overline{\bigcup_{n\in\Bbb N}E_n}$,  so the closed set
$E\setminus\bigcup_{n\in\Bbb N}E_n$ has empty interior and is nowhere
dense.   Accordingly, setting $A=E\symmdiff\bigcup_{n\in\Bbb N}F_n$, we
have

\Centerline{$A\subseteq(E\setminus\bigcup_{n\in\Bbb N}E_n)
  \cup\bigcup_{n\in\Bbb N}A_n\in\Cal M$,}

\noindent so that $\bigcup_{n\in\Bbb N}F_n=E\symmdiff A\in\Sigma$.
Thus $\Sigma$ is closed under countable unions and is a
$\sigma$-algebra.\ \Qed

Evidently $\Cal M\subseteq\Sigma$, because $\emptyset\in\Cal E$.

\medskip

{\bf (b)} For each $F\in\Sigma$, there is exactly one $E\in\Cal E$ such
that $F\symmdiff E\in\Cal M$.   \Prf\ There is surely some $E\in\Cal E$
such that $F$ is expressible as $E\symmdiff A$ where $A\in\Cal M$, so
that $F\symmdiff E=A\in\Cal M$.   If $E'$ is any other member of
$\Cal E$, then $E'\symmdiff E$ is a non-empty open set in $X$, while
$E'\symmdiff E\subseteq A\cup(F\symmdiff E')$;
by Baire's theorem for compact Hausdorff spaces (3A3G),
$A\cup(F\symmdiff E')\notin\Cal M$ and $F\symmdiff E'\notin\Cal M$.
Thus $E$ is unique.\ \Qed

\medskip

{\bf (c)} Consequently the maps
$E\mapsto E^{\ssbullet}:\Cal E\to\Sigma/\Cal M$ is a bijection.
But since it is also a Boolean homomorphism, it is an
isomorphism, and $\frak A\cong\Cal E\cong \Sigma/\Cal M$, as claimed.
}%end of proof of 314M

\leader{314N}{Corollary} A Boolean algebra $\frak A$ is Dedekind
$\sigma$-complete iff it is isomorphic to a quotient $\Sigma/\Cal I$
where $\Sigma$ is a $\sigma$-algebra of sets and $\Cal I$
is a $\sigma$-ideal of $\Sigma$.

\proof{ Put 314D and 314M together.
}%end of proof of 314N

\leader{314O}{Regular open \dvrocolon{algebras}}\cmmnt{ For Boolean
algebras which are Dedekind complete in the full sense, there is another
general method of representing them, which
leads to further very interesting ideas.

\medskip

\noindent}{\bf Definition} Let $X$ be a topological space.   A
{\bf regular open set} in $X$ is an open set $G\subseteq X$ such that
$G=\interior\overline{G}$.

Note that if $F\subseteq X$ is any closed set, then $G=\interior F$ is a
regular open set\prooflet{, because $G\subseteq\overline{G}\subseteq F$
so

\Centerline{$G\subseteq\interior{\overline{G}}\subseteq\interior F=G$}

\noindent and $G=\interior\overline{G}$}.

\leader{314P}{Theorem} Let $X$ be any topological space, and write
$\RO(X)$ for the set of regular open sets in $X$.   Then $\RO(X)$ is a
Dedekind complete Boolean algebra, with $1_{\RO(X)}=X$ and
$0_{\RO(X)}=\emptyset$, and with Boolean operations given by

\Centerline{$G\hskip.2em\Bcapshort_{\RO}\hskip.2em H=G\cap H$,
\quad$G\hskip.2em\Bsymmdiffshort_{\RO}\hskip.2em H
=\interior\overline{G\symmdiff H}$,}

\Centerline{$G\hskip.2em\Bcupshort_{\RO}\hskip.2em H
=\interior\overline{G\cup H}$,
\quad$G\hskip.2em\Bsetminusshort_{\RO}\hskip.2em H=G\setminus\overline{H}$,}

\noindent with Boolean ordering given by

\Centerline{$G\hskip.2em\Bsubseteqshort_{\RO}\hskip.2em H
\iff G\subseteq H$,}

\noindent and with suprema and infima given by

\Centerline{$\sup\Cal H=\interior\overline{\bigcupop\Cal H}$,
\quad$\inf\Cal H=\interior\bigcap\Cal H
=\interior\overline{\bigcapop\Cal H}$}

\noindent for all non-empty $\Cal H\subseteq\RO(X)$.

\cmmnt{\medskip

\noindent{\bf Remark} I use the expressions

\Centerline{$\Bcapshort_{\RO}\hskip.2em
\quad\Bcupshort_{\RO}\hskip.2em
\quad\Bsymmdiff_{\RO}
\quad\Bsetminusshort_{\RO}\hskip.2em
\quad\Bsubseteqshort_{\RO}\hskip.2em$}

\noindent in case the distinction between

\Centerline{$\cap\quad\cup\quad\symmdiff\quad\setminus\quad\subseteq$}

\noindent and

\Centerline{$\Bcap\quad\Bcup\quad\Bsymmdiff
\quad\Bsetminus\quad\Bsubseteq$}

\noindent is insufficiently marked.
}%end of comment

\proof{ I base the proof on the study of an auxiliary algebra of sets
which involves some of the ideas already used in 314M.

\medskip

{\bf (a)} Let $\Cal I$ be the family of nowhere dense subsets of $X$.
Then $\Cal I$ is an ideal of subsets of $X$.   \Prf\ Of course
$\emptyset\in\Cal I$.   If $A\subseteq B\in\Cal I$ then
$\interior\overline{A}\subseteq\interior\overline{B}=\emptyset$.   If
$A$, $B\in\Cal I$ and $G$ is a non-empty open set, then
$G\setminus\overline{A}$ is a non-empty open set and
$(G\setminus\overline{A})\setminus\overline{B}$ is non-empty;
accordingly $G$ cannot be a subset of
$\overline{A}\cup\overline{B}=\overline{A\cup B}$.   This shows that
$\interior\overline{A\cup B}=\emptyset$, so that
$A\cup B\in\Cal I$.\ \Qed

\medskip

{\bf (b)} For any set $A\subseteq X$, write $\partial A$ for the
boundary of $A$, that is, $\overline{A}\setminus\interior A$.
Set

\Centerline{$\Sigma=\{E:E\subseteq X,\,\partial E\in\Cal I\}$.}

\noindent The $\Sigma$ is an algebra of subsets of $X$.   \Prf\ (i)
$\partial\emptyset=\emptyset\in\Cal I$ so $\emptyset\in\Sigma$.   (ii)
If $A$, $B\subseteq X$, then
$\overline{A\cup B}=\overline{A}\cup\overline{B}$, while $\interior(A\cup B)\supseteq\interior A\cup\interior B$;  so $\partial(A\cup B)\subseteq\partial A\cup\partial B$.   So if $E$, $F\in\Sigma$,
$\partial(E\cup F)\subseteq\partial E\cup\partial F\in\Cal I$ and
$E\cup F\in\Sigma$.   (iii) If $A\subseteq X$, then

\Centerline{$\partial(X\setminus A)
=\overline{X\setminus A}\setminus\interior(X\setminus A)
=(X\setminus\interior A)\setminus(X\setminus\overline{A})
=\overline A\setminus\interior A=\partial A$.}

\noindent So if $E\in\Sigma$, $\partial(X\setminus E)
=\partial E\in\Cal I$ and $X\setminus E\in\Sigma$.\ \Qed

If $A\in\Cal I$, then of course $\partial A=\overline{A}\in\Cal I$, so
$A\in\Sigma$;  accordingly $\Cal I$ is an ideal in the Boolean algebra
$\Sigma$, and we can form the quotient $\Sigma/\Cal I$.

It will be helpful to note that every open set belongs to $\Sigma$,
since if $G$ is open then $\partial G=\overline{G}\setminus G$ cannot
include any non-empty open set (since any open set meeting
$\overline{G}$ must meet $G$).

\medskip

{\bf (c)} For each $E\in\Sigma$, set $V_E=\interior\overline{E}$;  then
$V_E$ is the unique member of $\RO(X)$ such that
$E\symmdiff V_E\in\Cal I$.
\Prf\ (i) Being the interior of a closed set, $V_E\in\RO(X)$.
Since $\interior E\subseteq V_E\subseteq\overline{E}$,
$E\symmdiff V_E\subseteq\partial E\in\Cal I$.   (ii) If $G\in\RO(X)$ is such that $E\symmdiff G\in\Cal I$, then

\Centerline{$G\setminus\overline{V_E}\subseteq G\setminus V_E
\subseteq (G\symmdiff E)\cup(V_E\symmdiff E)\in\Cal I$,}

\noindent so $G\setminus\overline{V_E}$, being open, must be actually
empty, and $G\subseteq\overline{V_E}$;  but this means that
$G\subseteq\interior\overline{V_E}=V_E$.   Similarly, $V_E\subseteq G$
and $V_E=G$.   This shows that $V_E$ is unique.\ \Qed

\medskip

{\bf (d)} It follows that the map
$G\mapsto G^{\ssbullet}:\RO(X)\to\Sigma/\Cal I$ is a bijection, and we have a Boolean algebra
structure on $\RO(X)$ defined by the Boolean algebra structure of
$\Sigma/\Cal I$.  What this means is that for each of the binary Boolean
operations $\Bcapshort_{\RO}\hskip.2em$,
$\Bsymmdiffshort_{\RO}\hskip.2em$, $\Bcupshort_{\RO}\hskip.2em$,
$\Bsetminusshort_{\RO}\hskip.2em$ and for $G$,
$H\in\RO(X)$ we must have
$G{*}_{\RO}H=\interior\overline{G*H}$,
writing ${*}_{\RO}$ for the operation on the algebra
$\RO(X)$ and $*$ for the corresponding operation on $\Sigma$ or
$\Cal PX$.

\medskip

{\bf (e)} Before working through the identifications, it will be helpful
to observe that if $\Cal H$ is any non-empty subset of $\RO(X)$, then
$\interior\bigcap\Cal H=\interior\overline{\bigcap\Cal H}$.   \Prf\ Set
$G=\interior\overline{\bigcap\Cal H}$.   For every $H\in\Cal H$,
$G\subseteq\overline{H}$ so $G\subseteq\interior\overline{H}=H$;  thus

\Centerline{$G\subseteq\interior\bigcap\Cal H
\subseteq\interior\overline{\bigcapop\Cal H}=G$,}

\noindent so $G=\interior\bigcap\Cal H$.\ \QeD\
Consequently $\interior\bigcap\Cal H$, being the interior of a closed
set, belongs to $\RO(X)$.

\medskip

{\bf (f)(i)} If $G$, $H\in\RO(X)$ then their intersection in the
algebra $\RO(X)$ is

\Centerline{$G\hskip.2em\Bcapshort_{\RO}\hskip.2em H
=\interior\overline{G\cap H}=\interior(G\cap H)=G\cap H$,}

\noindent using (d) for the first equality and (e) for the second.

\medskip

\quad{\bf (ii)} Of course $X\in\RO(X)$ and
$X^{\ssbullet}=1_{\Sigma/\Cal I}$, so $X=1_{\RO(X)}$.

\medskip

\quad{\bf (iii)} If $G\in\RO(X)$ then its complement
$1_{\RO(X)}\hskip.2em\Bsetminusshort_{\RO}\hskip.2em G$ in $\RO(X)$ is

\Centerline{$\interior\overline{X\setminus G}
=\interior(X\setminus G)=X\setminus\overline{G}$.}

\medskip

\quad{\bf (iv)} If $G$, $H\in\RO(X)$, then the relative complement in
$\RO(X)$ is

\Centerline{$G\hskip.2em\Bsetminusshort_{\RO}\hskip.2em H
=G\hskip.2em\Bcapshort_{\RO}\hskip.2em
  (1_{\RO(X)}\hskip.2em\Bsetminusshort_{\RO}\hskip.2em H)
=G\cap(X\setminus\overline{H})=G\setminus\overline{H}
=\interior(G\setminus H)$.}

\medskip

\quad{\bf (v)} If $G$, $H\in\RO(X)$, then
$G\hskip.2em\Bcupshort_{\RO}\hskip.2em H
=\interior\overline{G\cup H}$ and $G\hskip.2em\Bsymmdiffshort_{\RO}\hskip.2em H
=\interior\overline{G\symmdiff H}$, by the remarks in (d).

\medskip

{\bf (g)} We must note that for $G$, $H\in\RO(X)$,

\Centerline{$G\hskip.2em\Bsubseteqshort_{\RO}\hskip.2em H
\iff G\hskip.2em\Bcapshort_{\RO}\hskip.2em H=G
\iff G\cap H=G\iff G\subseteq H$;}

\noindent that is, the ordering of the Boolean algebra $\RO(X)$ is just
the partial ordering induced on $\RO(X)$ by the Boolean ordering
$\subseteq$ of $\Cal PX$ or $\Sigma$.

\medskip

{\bf (h)} If $\Cal H$ is any non-empty subset of $\RO(X)$, consider
$G_0=\interior\bigcap\Cal H$ and
$G_1=\interior\overline{\bigcup\Cal H}$.

$G_0=\inf\Cal H$ in $\RO(X)$.   \Prf\ By (e), $G_0\in\RO(X)$.   Of
course $G_0\subseteq H$ for every $H\in\Cal H$, so $G_0$ is a lower
bound for $\Cal H$.   If $G$ is any lower bound for $\Cal H$ in
$\RO(X)$, then $G\subseteq H$ for every $H\in\Cal H$, so
$G\subseteq\bigcap\Cal H$;  but also $G$ is open, so
$G\subseteq\interior\bigcap\Cal H=G_0$.   Thus $G_0$ is the greatest
lower bound for $\Cal H$.\ \Qed

$G_1=\sup\Cal H$ in $\RO(X)$.   \Prf\ Being the interior of a closed
set, $G_1\in\RO(X)$, and of course

\Centerline{$H
=\interior\overline{H}\subseteq\interior\overline{\bigcupop\Cal H}=G_1$}

\noindent for every $H\in\Cal H$, so $G_1$ is an upper bound for
$\Cal H$ in $\RO(X)$.   If $G$ is any upper bound for $\Cal H$ in $\RO(X)$, then

\Centerline{$G=\interior\overline{G}
\supseteq\interior\overline{\bigcupop\Cal H}
=G_1$;}

\noindent thus $G_1$ is the least upper bound for $\Cal H$ in
$\RO(X)$.\ \Qed

This shows that every non-empty $\Cal H\subseteq\RO(X)$ has a supremum
and an infimum in $\RO(X)$;  consequently $\RO(X)$ is Dedekind
complete, and the proof is finished.
}%end of proof of 314P

\leader{314Q}{Remark\cmmnt{s (a)}} $\RO(X)$ is called the
{\bf regular open algebra} of the topological space $X$.

\cmmnt{
\spheader 314Qb Note that the map $E\mapsto V_E:\Sigma\to\RO(X)$
of part (c) of the proof above is a Boolean homomorphism, if $\RO(X)$
is given its Boolean algebra structure.   Its kernel is of course
$\Cal I$;  the induced map
$E^{\ssbullet}\mapsto V_E:\Sigma/\Cal I\to\RO(X)$
is just the inverse of the isomorphism
$G\mapsto G^{\ssbullet}:\RO(X)\to\Sigma/\Cal I$.
}

\leader{*314R}{}\cmmnt{ I interpolate a lemma corresponding to 313R, with a
couple of other occasionally useful facts.

\medskip

\noindent}{\bf Lemma} (a) Let $X$ and $Y$ be topological spaces, and
$f:X\to Y$ a continuous function such that $f^{-1}[M]$ is nowhere dense
in $X$ for every nowhere dense $M\subseteq Y$.   Then we have an
order-continuous Boolean homomorphism $\pi$ from the regular open
algebra $\RO(Y)$ of $Y$ to the regular open algebra $\RO(X)$ of $X$
defined by setting $\pi H=\interior\overline{f^{-1}[H]}$ for every
$H\in\RO(Y)$.

(b)\dvAnew{2011} Let $X$ be a topological space.

\quad(i) If $U\subseteq X$ is open, then $G\mapsto G\cap U$ is a surjective
order-continuous Boolean homomorphism from $\RO(X)$ onto $\RO(U)$.

\quad(ii) If $U\in\RO(X)$ then $\RO(U)$ is the principal ideal of $\RO(X)$
generated by $U$.

\proof{{\bf (a)(i)} By the remark in 314O, the formula for $\pi H$ always
defines a member of $\RO(X)$;  and of course $\pi$ is order-preserving.

Observe that if $H\in\RO(Y)$, then $f^{-1}[H]$ is open, so
$f^{-1}[H]\subseteq\pi H$.   It will be convenient to note straight away
that if $V\subseteq Y$ is a dense open set then $f^{-1}[V]$ is dense in
$X$.   \Prf\
$M=Y\setminus V$ is nowhere dense, so $f^{-1}[M]$ is nowhere dense and
its complement $f^{-1}[V]$ is dense.\ \Qed

\medskip

\quad{\bf (ii)} If $H_1$, $H_2\in\RO(Y)$ then
$\pi(H_1\cap H_2)=\pi H_1\cap\pi H_2$.   \Prf\ Because $\pi$ is
order-preserving, $\pi(H_1\cap H_2)\subseteq\pi H_1\cap\pi H_2$.
\Quer\ Suppose, if possible, that they are not equal.   Then (because
$\pi(H_1\cap H_2)$ is a regular open set)
$G=\pi H_1\cap\pi H_2\setminus\overline{\pi(H_1\cap H_2)}$ is non-empty.
Set $M=\overline{f[G]}$.   Then $f^{-1}[M]\supseteq G$ is not nowhere
dense, so $H=\interior M$ must be non-empty.   Now
$G\subseteq\pi H_1\subseteq\overline{f^{-1}[H_1]}$, so

\Centerline{$f[G]\subseteq f[\overline{f^{-1}[H_1]}]
\subseteq\overline{f[f^{-1}[H_1]]}\subseteq\overline{H}_1$,}

\noindent so $M\subseteq\overline{H}_1$ and
$H\subseteq\interior\overline{H}_1=H_1$.   Similarly, $H\subseteq H_2$,
and $f^{-1}[H]\subseteq f^{-1}[H_1\cap H_2]\subseteq\pi(H_1\cap H_2)$.
But also $H\cap f[G]$ is not empty, so

\Centerline{$\emptyset\ne G\cap f^{-1}[H]
\subseteq G\cap\pi(H_1\cap H_2)$,}

\noindent which is impossible.\ \Bang\Qed

\medskip

\quad{\bf (iii)} If $H\in\RO(Y)$ and $H'=Y\setminus\overline{H}$ is its
complement in $\RO(Y)$ then $\pi H'=X\setminus\overline{\pi H}$ is the
complement of $\pi H$ in $\RO(X)$.   \Prf\ By (b), $\pi H$ and $\pi H'$
are disjoint.   Now $H\cup H'$ is a dense open subset of $Y$, so

\Centerline{$\pi H\cup\pi H'\supseteq f^{-1}[H]\cup f^{-1}[H']
=f^{-1}[H\cup H']$}

\noindent is dense in $X$, and the regular open set $\pi H'$ must
include the complement of $\pi H$ in $\RO(X)$.\ \Qed

Putting this together with (b), we see that the conditions of 312H(ii) are
satisfied, so that $\pi$ is a Boolean homomorphism.

\medskip

\quad{\bf (iv)} To see that it is order-continuous, let
$\Cal H\subseteq\RO(Y)$ be a non-empty set with supremum $Y$.   Then
$H_0=\bigcup\Cal H$ is a dense open subset of $Y$ (see the formula in
314P).   So

\Centerline{$\bigcup_{H\in\Cal H}\pi H
\supseteq\bigcup_{H\in\Cal H}f^{-1}[H]=f^{-1}[H_0]$}

\noindent is dense in $X$, and $\sup_{H\in\Cal H}\pi H=X$ in $\RO(X)$.
By 313L(b-iii), $\pi$ is order-continuous.

\medskip

{\bf (b)(i)} The idea is to apply (a) to the identity function
$f:U\to X$.   If $M\subseteq X$ is nowhere dense, then any non-empty open
subset of $U$ has a non-empty open subset disjoint from $M$, so
$f^{-1}[M]=M\cap U$ is nowhere dense in $U$;  thus the condition is
satisfied, and we have an order-continuous Boolean homomorphism
$\pi:\RO(X)\to\RO(U)$ defined by setting
$\pi H=\interior_U\overline{H\cap U}^{(U)}$ for every $H\in\RO(X)$.
(I write $\interior_U$, $\overline{\phantom{H}}^{(U)}$ to indicate interior
and closure in the subspace topology.)   Now for any open set
$G\subseteq X$,

\Centerline{$U\cap\overline{G}
=U\cap(\overline{G\cap U}\cup\overline{G\setminus U})
=U\cap\overline{G\cap U}
=\overline{G\cap U}^{(U)}$.}

\noindent So if $H\in\RO(X)$, then

\Centerline{$\pi H
=\interior_U\overline{H\cap U}^{(U)}
=\interior_U(U\cap\overline{H})
=U\cap\interior\overline{H}
=U\cap G$.}

\noindent So $\pi$ takes the required form.   To see that it is surjective,
take any $V\in\RO(U)$.   Then $\interior\overline{V}\in\RO(X)$, and

\Centerline{$V=\interior_U\overline{V}^{(U)}
=\interior_U(U\cap\overline{V})=U\cap\interior\overline{V}
=\pi(\interior\overline{V})$}

\noindent is a value of $\pi$.

\medskip

\quad{\bf (ii)}
If $G\in\RO(X)$ and $G\subseteq U$, then $G=G\cap U\in\RO(U)$.
Conversely, if $V\in\RO(U)$, there is a $G\in\RO(X)$ such that
$V=G\cap U$;  but $G\cap U\in\RO(X)$, by 314P, so $V\in\RO(X)$.
}%end of proof of 314R

\leader{314S}{}\cmmnt{ It is now easy to characterize the Stone spaces
of Dedekind complete Boolean algebras.

\medskip

\noindent}{\bf Theorem} Let $\frak A$ be a Boolean algebra, and $Z$ its
Stone space;  write $\Cal E$ for the algebra of open-and-closed subsets
of $Z$, and $\RO(Z)$ for the regular open algebra of $Z$.   Then the
following are equiveridical:

(i) $\frak A$ is Dedekind complete;

(ii) $Z$ is extremally disconnected\cmmnt{ (definition:  3A3Af)};

(iii) $\Cal E=\RO(Z)$.

\proof{ For $a\in\frak A$, let $\widehat{a}$ be
the corresponding member of $\Cal E$.

\medskip

{\bf (i)$\Rightarrow$(ii)} If $\frak A$ is Dedekind complete,
let $G$ be any open set in $Z$.   Set
$A=\{a:a\in\frak A,\,\widehat a\subseteq G\}$, $a_0=\sup A$.   Then
$G=\bigcup\{\widehat a:a\in A\}$,
because $\Cal E$ is a base for the topology of $Z$, so
$\widehat a_0=\overline{G}$, by 313Ca.   Consequently $\overline{G}$ is
open.   As $G$ is arbitrary, $Z$ is extremally disconnected.

\medskip

{\bf (ii)$\Rightarrow$(iii)}
If $E\in\Cal E$, then of course $E=\overline{E}=\interior\overline{E}$,
so $E$ is a regular open set.    Thus $\Cal E\subseteq\RO(Z)$.   On the
other hand, suppose that $G\subseteq Z$ is a regular open set.
Because $Z$ is extremally disconnected, $\overline{G}$ is open;  so
$G=\interior\overline{G}=\overline{G}$ is open-and-closed, and belongs
to $\Cal E$.   Thus $\Cal E=\RO(Z)$.

\medskip

{\bf (iii)$\Rightarrow$(i)}
Since $\RO(Z)$ is Dedekind complete (314P), $\Cal E$ and $\frak A$ are
also Dedekind complete Boolean algebras.
}%end of proof of 314S

\cmmnt{\medskip

\noindent{\bf Remark} Note that if the conditions above are satisfied,
either 312M or the formulae in 314P show that the Boolean structures of
$\Cal E$ and $\RO(Z)$ are identical.
}%end of comment

\leader{314T}{}\cmmnt{ I come now to a construction of great
importance, both as a foundation for further constructions and as a
source of insight into the nature of Dedekind completeness.

\medskip

\noindent}{\bf Theorem} Let $\frak A$ be a Boolean algebra, with Stone
space $Z$;  for $a\in\frak A$ let $\widehat a$ be the corresponding
open-and-closed subset of $Z$.   Let $\widehat{\frak A}$ be the
regular open algebra of $Z$\cmmnt{ (314P)}.

(a) The map $a\mapsto\widehat a$ is an injective order-continuous
Boolean homomorphism from $\frak A$ onto an order-dense subalgebra of
$\widehat{\frak A}$.

(b) If $\frak B$ is any Dedekind complete Boolean algebra and
$\pi:\frak A\to\frak B$ is an order-continuous Boolean homomorphism,
there is a unique order-continuous Boolean homomorphism
$\pi_1:\widehat{\frak A}\to\frak B$ such that $\pi_1\widehat a=\pi a$
for every $a\in\frak A$.

\proof{{\bf (a)(i)}  Setting $\Cal E=\{\widehat a:a\in\frak A\}$, every
member of $\Cal E$ is open-and-closed, so is surely equal to the
interior of its closure, and is a regular open set;  thus
$\widehat a\in\widehat{\frak A}$ for every $a\in\frak A$.   The formulae
in 314P tell us that if $a$, $b\in\frak A$, then
$\widehat a\Bcap\widehat b$,
taken in $\widehat{\frak A}$, is just the set-theoretic intersection
$\widehat a\cap\widehat b=(a\Bcap b)\widehat{\phantom{m}}$;  while
$1\Bsetminus\widehat a$, taken in $\widehat{\frak A}$, is

\Centerline{$Z\setminus\overline{\widehat a}
=Z\setminus\widehat a=(1\Bsetminus a)\widehat{\phantom{m}}$.}

\noindent And of course $\widehat 0=\emptyset$ is the zero of
$\widehat{\frak A}$.   Thus the map
$a\mapsto\widehat a:\frak A\to\widehat{\frak A}$ preserves $\Bcap$ and
complementation, so is a
Boolean homomorphism (312H).   Of course it is injective.

\medskip

\quad{\bf (ii)} If $A\subseteq\frak A$ is non-empty and $\inf A=0$, then
$\bigcap_{a\in A}\widehat a$ is nowhere dense in $Z$ (313Cc), so

\Centerline{$\inf\{\widehat a:a\in A\}
=\interior(\bigcap_{a\in A}\widehat a)=\emptyset$}

\noindent (314P again).   As $A$ is arbitrary, the map
$a\mapsto\widehat a:\frak A\to\widehat{\frak A}$ is order-continuous.

\medskip

\quad{\bf (iii)} If $G\in\widehat{\frak A}$ is not empty, then there is
a non-empty member of $\Cal E$ included in it, by the definition of the
topology of $Z$ (311I).   So $\Cal E$ is an order-dense subalgebra of
$\widehat{\frak A}$.

\medskip

{\bf (b)} Now suppose that $\frak B$ is a Dedekind complete Boolean
algebra and $\pi:\frak A\to\frak B$ is an order-continuous Boolean
homomorphism.   Write $\iota a=\widehat{a}$ for $a\in\frak A$, so that
$\iota:\frak A\to\widehat{\frak A}$ is an isomorphism between $\frak A$
and the order-dense subalgebra $\Cal E$ of $\widehat{\frak A}$.
Accordingly $\pi\iota^{-1}:\Cal E\to\frak B$ is an order-continuous
Boolean homomorphism, being the composition of the order-continuous
Boolean homomorphisms $\pi$ and $\iota^{-1}$.
By 314K, it has an extension to a Boolean homomorphism
$\pi_1:\widehat{\frak A}\to\frak B$, and $\pi_1\iota=\pi$, that is,
$\pi_1\widehat{a}=\pi a$ for every $a\in\frak A$.   Now $\pi_1$ is
order-continuous.   \Prf\ Suppose that
$\Cal H\subseteq\widehat{\frak A}$
has supremum $1$ in $\widehat{\frak A}$.   Set

\Centerline{$\Cal H'=\{E:E\in\Cal E$,
$E\subseteq H$ for some $H\in\Cal H\}$.}

\noindent Because $\Cal E$ is order-dense in $\widehat{\frak A}$,

\Centerline{$H=\sup_{E\in\Cal E,E\subseteq H}E
=\sup_{E\in\Cal H',E\subseteq H}E$}

\noindent for every $H\in\Cal H$ (313K), and $\sup\Cal H'=1$ in
$\widehat{\frak A}$.   It follows at once that $\sup\Cal H'=1$ in
$\Cal E$, so $\sup\pi_1[\Cal H']=\sup(\pi\iota^{-1})[\Cal H']=1$.
Since any upper bound for $\pi_1[\Cal H]$
must also be an upper bound for $\pi_1[\Cal H']$, $\sup\pi_1[\Cal H]=1$
in $\frak B$.   As $\Cal H$ is arbitrary, $\pi_1$ is order-continuous
(313L(b-iii)).\ \Qed

If $\pi'_1:\widehat{\frak A}\to\frak B$ is any other Boolean
homomorphism such
that $\pi'_1\widehat a=\pi a$ for every $a\in\frak A$, then $\pi_1$ and
$\pi'_1$ agree on $\Cal E$, and the argument just above shows that
$\pi'_1$ is also order-continuous.   But if $G\in\widehat{\frak A}$, $G$
is the supremum (in $\widehat{\frak A}$) of
$\Cal F=\{E:E\in\Cal E,\,E\subseteq G\}$, so

\Centerline{$\pi_1'G=\sup_{E\in\Cal F}\pi'_1E
=\sup_{E\in\Cal F}\pi_1E=\pi_1G$.}

\noindent As $G$ is arbitrary, $\pi'_1=\pi_1$.   Thus $\pi_1$ is unique.
}%end of proof of 314T

\leader{314U}{The Dedekind completion of a Boolean algebra (a)} For any
Boolean algebra $\frak A$, I will say that the Boolean algebra
$\widehat{\frak A}$ constructed in 314T is the {\bf Dedekind completion}
of $\frak A$.

\cmmnt{When using this concept I shall frequently suppress the
distinction between $a\in\frak A$ and $\widehat a\in\widehat{\frak A}$,
and treat $\frak A$ as itself an order-dense subalgebra of
$\widehat{\frak A}$.}

\spheader 314Ub\cmmnt{ The universal mapping theorem in 314Tb assures us that the
Dedekind completion is essentially unique.
The commonest way in which this fact appears is
the following.}
If $\frak C$ is a Dedekind complete Boolean algebra and $\frak A$ is
an order-dense subalgebra of $\frak C$,
then the embedding $\frak A\embedsinto\frak C$
induces an isomorphism from $\widehat{\frak A}$ to $\frak C$.
\prooflet{\Prf\ Write
$\pi a=a$ for $a\in\frak A$.   Because $\frak A$ is order-dense, $\pi$ is
order-continuous (313O), so extends to an order-continuous Boolean
homomorphism $\pi_1:\widehat{\frak A}\to\frak C$.
If $b\in\widehat{\frak A}$ is non-zero, there
is a non-zero $a\in\frak A$ such that $a\Bsubseteq b$;  now

\Centerline{$0\ne a=\pi a=\pi_1a\Bsubseteq\pi_1b$.}

\noindent As $b$ is arbitrary, $\pi_1$ is injective.   Next,
$\pi_1[\widehat{\frak A}]$ must be order-closed in $\frak C$, by 314F(a-i);
since it includes $\frak A$ and $\frak A$ is order-dense in $\frak C$,
$\pi_1[\widehat{\frak A}]=\frak C$ and $\pi_1$ is an isomorphism.\ \Qed}

\exercises{
\leader{314X}{Basic exercises $\pmb{>}$(a)}
%\spheader 314Xa
Let $\frak A$ be a Boolean algebra.
(i) Show that the following are equiveridical:  ($\alpha$) $\frak A$ is
Dedekind complete ($\beta$) every upwards-directed subset of
$\frak A$ has a least upper bound ($\gamma$) every
downwards-directed subset of $\frak A$ has a greatest
lower bound ($\delta$) every disjoint subset of $\frak A$ has a least
upper bound.
(ii) Show that the following are equiveridical:  ($\alpha$) $\frak A$ is
Dedekind $\sigma$-complete ($\beta$) every non-decreasing sequence in
$\frak A$  has a least upper bound ($\gamma$) every non-increasing sequence
in $\frak A$ has a greatest lower bound ($\delta$) every disjoint sequence
in $\frak A$ has a least upper bound.
%314B

\spheader 314Xb Let $\frak A$ be a Boolean algebra.   Show that any
principal ideal of $\frak A$ is order-closed.   Show that $\frak A$ is
Dedekind complete iff every order-closed ideal is principal.
%314B

\spheader 314Xc Let $\frak A$ be a Dedekind complete Boolean algebra,
$\frak B$ an order-closed subalgebra of $\frak A$, and $a\in\frak A$;
let $\frak A_a$ be the principal ideal of $\frak A$ generated by $a$.
Show that $\{a\Bcap b:b\in\frak B\}$ is an order-closed subalgebra of
$\frak A_a$.
%314F

\sqheader 314Xd Let $\frak A$ be a Dedekind complete Boolean algebra,
$\frak B$ a
Boolean algebra and $\pi:\frak A\to\frak B$ a surjective order-continuous
Boolean homomorphism.   (i) Show
that the kernel of $\pi$ is a principal ideal in $\frak A$.
(ii) Show that $\frak B$
is isomorphic to the complementary principal ideal in $\frak A$,
and in particular is Dedekind complete.
%314Xb 314F new 2005 used in 514Ic

\spheader 314Xe Let $\frak A$ be a Dedekind complete Boolean algebra and
$\frak C$ an order-closed subalgebra of $\frak A$.   Show that an
element $a$ of $\frak A$ belongs to $\frak C$ iff
$\upr(1\Bsetminus a,\frak C)=1\Bsetminus\upr(a,\frak C)$
iff $\upr(1\Bsetminus a,\frak C)\Bcap\upr(a,\frak C)=0$, writing
$\upr(a,\frak C)$ for the upper envelope of $a$ in $\frak C$, as in
313S.
%3{}13S 314G

\sqheader 314Xf Let $\frak A$ be a Dedekind complete Boolean algebra,
$\frak C$ an order-closed subalgebra of $\frak A$, $a_0\in\frak A$ and
$c_0\in\frak C$.   Show that the following are equiveridical:  (i) there
is a Boolean homomorphism
$\pi:\frak A\to\frak C$ such that $\pi c=c$ for every $c\in\frak C$ and
$\pi a_0=c_0$ (ii) $1\Bsetminus\upr(1\Bsetminus a_0,\frak C)
\Bsubseteq c_0\Bsubseteq\upr(a_0,\frak C)$.
%3{}13S 314Xe 314G

\sqheader 314Xg Let $\frak A$ be a Dedekind $\sigma$-complete Boolean
algebra, $\frak B$ a Boolean algebra and $\pi:\frak A\to\frak B$ a
sequentially order-continuous Boolean homomorphism.   If
$C\subseteq\frak A$ and
$\frak C$ is the $\sigma$-subalgebra of $\frak A$ generated by $C$, show
that $\pi[\frak C]$ is the $\sigma$-subalgebra of $\frak B$ generated by
$\pi[C]$.
%314F, 314H

\spheader 314Xh Let $X$ and $Y$ be extremally disconnected compact Hausdorff spaces,
$\RO(X)$ and $\RO(Y)$ their regular open algebras,
and $\phi:X\to Y$ a continuous surjection.   Show that the following are
equiveridical:  (i) the Boolean homomorphism $V\mapsto\phi^{-1}[V]$ from
$\RO(Y)$ to $\RO(X)$ (312Q, 314S) is order-continuous;
(ii) $\phi[U]$ is open-and-closed in $Y$ for every open-and-closed set
$U\subseteq X$;  (iii) $\phi[G]$ is open in $Y$ for every open set $G\subseteq X$.
%314S  new 2002

\spheader 314Xi Find a proof of 314Tb which does not appeal to 314K.
%314T

\spheader 314Xj Let $\frak B$ be a Dedekind complete Boolean algebra, and
$\frak A$ a Boolean algebra which can be regularly embedded in $\frak B$.
Show that the Dedekind completion of $\frak A$ can be regularly embedded in
$\frak B$.
%314T

\spheader 314Xk\dvAnew{2011}
Let $X$ be a topological space and $Y$ a dense subset of
$X$.   Show that $G\mapsto G\cap Y$ is a Boolean isomorphism from
$\RO(X)$ to $\RO(Y)$.
%314R

\leader{314Y}{Further exercises (a)}
%\spheader 314Ya
Let $P$ be a Dedekind complete partially ordered set.   Show that a set
$Q\subseteq P$ is order-closed iff $\sup R$, $\inf R$ belong to $Q$
whenever $R\subseteq Q$ is a totally ordered subset of $Q$ with upper
and lower bounds in $P$.   ({\it Hint\/}:  show by induction on $\kappa$
that if $A\subseteq Q$ is upwards-directed and bounded above and
$\#(A)\le\kappa$ then $\sup A\in Q$.)
%314A

\spheader 314Yb Let $P$ be a lattice.   Show that $P$ is Dedekind
complete iff every non-empty totally ordered subset of $P$ with an upper
bound in $P$ has a least upper bound in $P$.   \Hint{if $A\subseteq P$
is non-empty and bounded below in $P$, let $B$ be the set of lower
bounds of $A$ and use Zorn's Lemma to find a maximal element of $B$.}
%314A

\spheader 314Yc Give an example of a Boolean
algebra $\frak A$ with an order-closed subalgebra $\frak A_0$ and an
element $c$ such that the subalgebra generated by $\frak A_0\cup\{c\}$
is not order-closed.
%314J

\spheader 314Yd Let $X$ be any topological space.
Let $\Cal M$ be the $\sigma$-ideal of meager subsets of $X$, and set

\Centerline{$\widehat{\Cal B}=\{G\symmdiff A:G\subseteq X$ is open,
$A\in\Cal M\}$.}

\noindent (i) Show that $\widehat{\Cal B}$ is a $\sigma$-algebra of
subsets of $X$,
and that $\widehat{\Cal B}/\Cal M$ is Dedekind complete.   (Members of
$\widehat{\Cal B}$
are said to be the subsets of $X$ {\bf with the Baire property};
$\widehat{\Cal B}$ is the {\bf Baire-property algebra} of $X$.)
(ii) Show that if $A\subseteq X$ and $\bigcup\{G:G\subseteq X$ is open,
$A\cap G\in\widehat{\Cal B}\}$ is dense, then $A\in\widehat{\Cal B}$.
(iii) Show that there is a largest open set $V\in\Cal M$.   (iv) Let
$\RO(X)$ be the regular open algebra of $X$.   Show that the map
$G\mapsto G^{\ssbullet}$ is an order-continuous Boolean homomorphism
from $\RO(X)$ onto $\widehat{\Cal B}/\Cal M$, so induces a Boolean
isomorphism between
the principal ideal of $\RO(X)$ generated by $X\setminus\overline{V}$
and $\widehat{\Cal B}/\Cal M$.   ($\widehat{\Cal B}/\Cal M$ is the
{\bf category algebra} of $X$;  it is a Dedekind complete Boolean
algebra.   $X$ is called a {\bf Baire space} if
$V=\emptyset$;  in this case $\RO(X)\cong\widehat{\Cal B}/\Cal M$.
See 4A3R in Volume 4.)
%314P

\spheader 314Ye Let $\frak A$ be a Dedekind $\sigma$-complete Boolean
algebra, and $\sequencen{a_n}$ any sequence in $\frak A$.   For
$n\in\Bbb N$ set $E_n=\{x:x\in\{0,1\}^{\Bbb N},\,x(n)=1\}$, and let
$\Cal B$ be the $\sigma$-algebra of subsets of $\{0,1\}^{\Bbb N}$
generated by $\{E_n:n\in\Bbb N\}$.   ($\Cal B$ is the `Borel
$\sigma$-algebra' of $\{0,1\}^{\Bbb N}$;  see 4A3E in Volume 4.)
Show that there is a unique sequentially order-continuous Boolean
homomorphism
$\theta:\Cal B\to\frak A$ such that $\theta(E_n)=a_n$ for every
$n\in\Bbb N$.   \Hint{define a suitable function $\phi$ from the Stone
space $Z$ of $\frak A$ to $\{0,1\}^{\Bbb N}$, and consider
$\{E:E\subseteq\{0,1\}^{\Bbb N},\,\phi^{-1}[E]$ has the Baire property
in $Z\}$.}   Show that $\theta[\Cal B]$ is the $\sigma$-subalgebra of
$\frak A$ generated by $\{a_n:n\in\Bbb N\}$.
%314Yd 314P

\spheader 314Yf Let $\frak A$ be a Boolean algebra, and $Z$ its
Stone space.   Show that $\frak A$ is Dedekind $\sigma$-complete iff
$\overline{G}$ is open whenever $G$ is a cozero set in $Z$.   (Such
spaces are called {\bf basically disconnected} or {\bf quasi-Stonian}.)
%not in Engelking
%314S

\spheader 314Yg Let $\frak A$, $\frak B$ be Dedekind complete Boolean
algebras and $D\subseteq\frak A$ an order-dense set.
Suppose that $\phi:D\to\frak B$ is such that (i) $\phi[D]$ is
order-dense in $\frak B$ (ii) for all $d$, $d'\in D$, $d\Bcap d'=0$ iff
$\phi d\Bcap\phi d'=0$.   Show that $\phi$ has a unique extension to a
Boolean isomorphism from $\frak A$ to $\frak B$.
%314T

\spheader 314Yh Let $\frak A$ be any Boolean algebra.   Let $\Cal J$ be
the family of order-closed ideals in $\frak A$.   Show that (i) $\Cal J$
is a Dedekind complete Boolean algebra with operations defined by the
formulae $I\Bcap J=I\cap J$,
$1\Bsetminus J=\{a:a\Bcap b=0$ for every $b\in J\}$ (ii) the map
$a\mapsto \frak A_a$, the principal ideal generated by $a$, is an
injective order-continuous Boolean homomorphism from $\frak A$ onto an
order-dense subalgebra of $\Cal J$ (iii) $\Cal J$ is isomorphic to the
Dedekind completion of $\frak A$.
%314U
}%end of exercises

\cmmnt{
\Notesheader{314} At the risk of being tiresomely long-winded, I have
taken the trouble to spell out a large proportion of the results in this
section and the last in their `sequential' as well as their
`unrestricted' forms.   The point is that while (in my view) the
underlying ideas are most clearly and dramatically expressed in terms of
order-closed sets, order-continuous functions and Dedekind complete
algebras, a large proportion of the applications in measure theory deal
with sequentially order-closed sets, sequentially order-continuous
functions and Dedekind $\sigma$-complete algebras.   As a matter of
simple technique, therefore, it is necessary to master both, and for the
sake of later reference I generally give the statements of both versions
in full.   Perhaps the points to look at most keenly are just those
where there is a difference in the ideas involved, as in 314Bb, or in
which there is only one version given, as in 314M and 314T.

If you have seen the Hahn-Banach theorem (3A5A), it may
have been recalled to your mind by Theorem 314K;  in both cases we use
an order relation and a bit of algebra to make a single step towards an
extension of a function, and Zorn's lemma to turn this into the
extension we seek.
A good part of this section has turned out to be on
the borderland between the theory of Boolean algebra and general
topology;  naturally enough, since (as always with the general theory of
Boolean algebra) one of our first concerns is to establish connexions
between algebras and their Stone spaces.

I think 314T is the first substantial `universal mapping theorem' in
this volume;  it is by no means the last.   The idea of the construction
$\widehat{\frak A}$ is not just that we obtain a Dedekind complete
Boolean algebra in which $\frak A$ is embedded as an order-dense
subalgebra, but that we simultaneously obtain a theorem on the canonical
extension to $\widehat{\frak A}$ of order-continuous Boolean
homomorphisms defined on $\frak A$.   This characterization is enough to
define the pair $(\widehat{\frak A},a\mapsto\widehat a)$ up to
isomorphism, so the exact method of construction of $\widehat{\frak A}$
becomes of secondary importance.   The one used in 314T is very
natural (at least, if we believe in Stone spaces), but there are
others (see 314Yh), with different virtues.

314K and 314T both describe circumstances in which we can find
extensions of Boolean homomorphisms.   Clearly such results are
fundamental in the theory of Boolean algebras, but I shall not attempt
any systematic presentation here.   314Ye can also be regarded as
belonging to this family of ideas.
}%end of notes

\discrpage

