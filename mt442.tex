\frfilename{mt442.tex}
\versiondate{21.3.07}
\copyrightdate{1998}

\def\chaptername{Topological groups}
\def\sectionname{Uniqueness of Haar measures}

\newsection{442}

Haar measure has an extraordinary wealth of special properties, and it
will be impossible for me to cover them all properly in this chapter.
But surely the second thing to take on board, after the existence of
Haar measures on locally compact Hausdorff groups\cmmnt{ (441E)}, is
the fact that they are, up to scalar multiples, unique.   This is the
content of 442B.   We find also that while left and right Haar measures
can be different\cmmnt{ (442Xf)}, they are not only direct mirror
images of each other (442C)
-- as is, I suppose, to be expected --  but even more closely related
(442F, 442H, 442L).   Investigating this relation, we are led naturally
to the `modular function' of a group (442I).

\leader{442A}{Lemma} Let $X$ be a topological group and $\mu$ a left
Haar measure on $X$.

(a) $\mu$ is strictly positive and locally finite.

(b) If $G\subseteq X$ is open and $\gamma<\mu G$, there are an open set
$H$ and an open neighbourhood $U$ of the identity such that
$HU\subseteq G$\cmmnt{ (writing $HU$ for $\{xy:x\in H,\,y\in U\}$)} and $\mu H\ge\gamma$.

(c) If $X$ is locally compact and Hausdorff, $\mu$ is a Radon measure.

\proof{{\bf (a)(i)} \Quer\ If $G\subseteq X$ were a non-empty open set
such that $\mu G=0$, then we should have $\mu(xG)=0$ for every $x\in X$,
so that $X$ would be covered by negligible open sets;  but as $\mu$ is
supposed to be $\tau$-additive, $\mu X=0$.\ \Bang

\medskip

\quad{\bf (ii)} Because $\mu$ is effectively locally finite, there is
some non-empty open set $G$ such that $\mu G<\infty$;  but now
$\{xG:x\in X\}$ is a cover of $X$ by open sets of finite measure.

\medskip

{\bf (b)} Let $\Cal U$ be the family
of open sets containing the identity $e$, and $\Cal H$ the family of
open sets $H$ such that $HU\subseteq G$ for some $U\in\Cal U$.   Because
$\Cal U$ is downwards-directed,
$\Cal H$ is upwards-directed;  because $e\in U$ for every $U\in\Cal U$,
$\bigcup\Cal H\subseteq G$.   If $x\in G$, then $x^{-1}G\in\Cal U$, and
there is a $U\in\Cal U$ such that $UU\subseteq x^{-1}G$;  but now
$xUU\subseteq G$, so $xU\in\Cal H$.   Thus $\bigcup\Cal H=G$.   Because
$\mu$ is $\tau$-additive, there is an $H\in\Cal H$ such that $\mu
H\ge\gamma$.

\medskip

{\bf (c)} Use (a) and 416G.
}%end of proof of 442A

\leader{442B}{Theorem} Let $X$ be a topological group.   If $\mu$ and
$\nu$ are left Haar measures on $X$, they are multiples of each other.

\proof{{\bf (a)} Let $\Cal G$ be the family of non-empty open sets $G$
such that $\mu G$ and $\nu G$ are both finite;  because $\mu$ and $\nu$
are locally finite (442Aa), $\Cal G$ is a base for
the topology of $X$.   Note that $G\cup H\in\Cal G$ for all $G$,
$H\in\Cal G$.   Set $\Cal U=\{U:U\in\Cal G,\,U=U^{-1},\,e\in U\}$,
where $e$ is the identity of $G$;
then $\Cal U$ is a base of neighbourhoods of $e$ (4A5Ec).   Let $\Cal F$
be the filter on $\Cal U$ generated by the sets
$\{U:U\in\Cal U,\,U\subseteq V\}$ as $V$ runs over $\Cal U$.

\medskip

{\bf (b)} (The key.)  If $G\in\Cal G$ and $0<\epsilon<1$, there is a
$V_1\in\Cal U$ such that

\Centerline{$(1-\epsilon)\Bover{\mu G}{\nu G}\le\Bover{\mu U}{\nu U}$}

\noindent whenever $U\in\Cal U$ and $U\subseteq V_1$.   \Prf\ By 442Aa,
$\mu G$ and $\nu G$ are both non-zero.   By 442Ab, there are an open set
$H$ and a neighbourhood $V_1$ of $e$ such that $HV_1\subseteq G$ and
$\mu H\ge(1-\epsilon)\mu G$;  shrinking $V_1$ if need be, we may suppose
that $V_1\in\Cal U$.   Take any $U\in\Cal U$ such that $U\subseteq V_1$,
so that $HU\subseteq G$.   Consider the product quasi-Radon measure
$\lambda$ of $\mu$ and $\nu$ on $X\times X$ (417R), and the set
$W=\{(x,y):x,\,y\in G,\,x^{-1}y\in U\}$.   Because the function
$(x,y)\mapsto x^{-1}y$ is continuous (4A5Eb), $W$ is open.
Consequently

\Centerline{$\int\mu\{x:(x,y)\in W\}\nu(dy)
=\lambda W=\int\nu\{y:(x,y)\in W\}\mu(dx)$}

\noindent (417C(iv)).   But we see that if $x\in H$ then
$xU\subseteq G$, so $(x,y)\in W$ whenever $y\in xU$, and

\Centerline{$\int\nu\{y:(x,y)\in W\}\mu(dx)
\ge\int_H\nu(xU)\mu(dx)=\mu H\cdot\nu U$.}

\noindent On the other hand,

$$\eqalign{\int\mu\{x:(x,y)\in W\}\nu(dy)
&\le\int_G\mu\{x:x^{-1}y\in U\}\nu(dy)
=\int_G\mu\{x:y^{-1}x\in U\}\nu(dy)\cr
&=\int_G\mu(yU)\nu(dy)
=\mu U\cdot\nu G,\cr}$$

\noindent so that

\Centerline{$(1-\epsilon)\mu G\cdot\nu U
\le\mu H\cdot\nu U
\le\lambda W
\le\mu U\cdot\nu G$.}

\noindent  Dividing both sides by $\nu U\cdot\nu G$, we have the
result.\ \Qed

\medskip

{\bf (c)} In the same way, there is a $V_2\in\Cal V$ such that

\Centerline{$(1-\epsilon)\Bover{\nu G}{\mu G}\le\Bover{\nu U}{\mu U}$}

\noindent whenever $U\in\Cal U$ and $U\subseteq V_2$.   So if
$U\in\Cal U$ and $U\subseteq V_1\cap V_2$, we have

\Centerline{$(1-\epsilon)\Bover{\mu G}{\nu G}\le\Bover{\mu U}{\nu U}
\le\Bover{1}{1-\epsilon}\Bover{\mu G}{\nu G}$.}

\noindent As $\epsilon$ is arbitrary,

\Centerline{$\lim_{U\to\Cal F}\Bover{\mu U}{\nu U}
=\Bover{\mu G}{\nu G}$.}

\noindent And this is true for every $G\in\Cal G$.

\medskip

{\bf (d)} So if we set $\alpha=\lim_{U\to\Cal F}\bover{\mu U}{\nu U}$,
we shall have $\mu G=\alpha\nu G$ for every $G\in\Cal G$.   Now $\mu$
and $\alpha\nu$ are quasi-Radon measures agreeing on the
base $\Cal G$, which is closed under finite unions, so are identical, by
415H(iv).


}%end of proof of 442B

\leader{442C}{Proposition} Let $X$ be a topological group and $\mu$ a
left Haar measure on $X$.
Setting $\nu E=\mu(E^{-1})$ whenever $E\subseteq X$ is such that
$E^{-1}=\{x^{-1}:x\in E\}$ is measured by $\mu$, $\nu$ is a right Haar
measure on $X$.

\proof{ Set $\phi(x)=x^{-1}$ for $x\in X$.   Then $\phi$ is a
homeomorphism, so the image measure $\nu=\mu\phi^{-1}$ is a quasi-Radon
measure.   It is
non-zero because $\nu X=\mu X$.   If $E\in\dom\nu$ and $x\in X$, then

\Centerline{$\nu(Ex)=\mu(x^{-1}E^{-1})=\mu E^{-1}=\nu E$.}

\noindent So $\nu$ is a right Haar measure.
}%end of proof of 442C

\leader{442D}{Remark} Clearly all the arguments of 442A-442C must be
applicable to right Haar
measures\cmmnt{;  that is, any right Haar measure must be locally
finite and strictly
positive;  two right Haar measures on the same group must be multiples
of each other;  and if $X$ carries a right Haar measure $\nu$ then
$E\mapsto\nu E^{-1}$ will be a left Haar measure on $X$}.
\cmmnt{(If you are unhappy with such a bold appeal to the symmetry
between `left' and `right' in topological groups, write the reflected
version of 442C out in full, and use it to reflect 442A-442B.)

}Thus we may say that a topological group {\bf carries Haar measures} if
it has either a left or a right Haar measure.   \cmmnt{These can, of
course, be the same;  in fact it takes a certain amount of exploration
to find a group in which they are different (e.g., 442Xf).}

\leader{442E}{Lemma} Let $X$ be a topological group, $\mu$ a left Haar
measure on $X$ and $\nu$ a right Haar measure on $X$.
If $G$, $H\subseteq X$ are open, then

\Centerline{$\mu G\cdot\nu H
=\int_H\nu(xG^{-1})\mu(dx)$.}

\proof{ Let $\lambda$ be the quasi-Radon product measure of
$\mu$ and $\nu$ on
$X\times X$.   The sets $W_1=\{(x,y):y^{-1}x\in G,\,x\in H\}$ and
$W_2=\{(x,y):x\in G,\,yx\in H\}$ are both open, so 417C(iv) tells
us that

$$\eqalign{\int_H\nu(xG^{-1})\mu(dx)
&=\int\nu\{y:x\in H,\,y^{-1}x\in G\}\mu(dx)
=\lambda W_1\cr
&=\int\mu\{x:x\in H,\,y^{-1}x\in G\}\nu(dy)
=\int\mu(H\cap yG)\nu(dy)\cr
&=\int\mu(y^{-1}H\cap G)\nu(dy)
=\int\mu\{x:x\in G,\,yx\in H\}\nu(dy)\cr
&=\lambda W_2
=\int_G\nu\{y:yx\in H\}\mu(dx)\cr
&=\int_G\nu(Hx^{-1})\mu(dx)
=\mu G\cdot\nu H.\cr}$$
}%end of proof of 442E

\leader{442F}{Domains of Haar \dvrocolon{measures}}\cmmnt{ 442B tells
us, in part, that any two left Haar measures on a topological
group must have the same domain and the same negligible sets;
similarly, any two right Haar measures have the same domain and the same
negligible sets.   In fact left and right Haar measures agree on both.

\medskip

\noindent}{\bf Proposition} Let $X$ be a topological group which carries
Haar measures.   If $\mu$ is a left Haar measure and $\nu$ is a right
Haar measure on $X$, then they have the same domains and the same null ideals.

\proof{{\bf (a)} Suppose that $F\subseteq X$ is a closed set such that
$\mu F=0$.   Then $\nu F=0$.   \Prf\Quer\ Otherwise, there is an open
set $H$ such that $\nu H<\infty$ and $\nu(F\cap H)>0$.   Let $G$ be any
open set such that $0<\mu G<\infty$.   By 442E,

\Centerline{$\mu G\cdot\nu H
=\int_H\nu(xG^{-1})\mu(dx)$.}

\noindent But also

\Centerline{$\mu G\cdot\nu(H\setminus F)
=\int_{H\setminus F}\nu(xG^{-1})\mu(dx)
=\int_H\nu(xG^{-1})\mu(dx)=\mu G\cdot\nu H$,}

\noindent so $\mu G\cdot\nu(H\cap F)=0$.\ \Bang\Qed

\medskip

{\bf (b)} It follows that

\Centerline{$\nu E=\sup_{F\subseteq E\text{ is closed}}\nu F=0$}

\noindent whenever $E$ is a Borel set such that $\mu E=0$.   Now take
any $E\in\dom\mu$.   Set

\Centerline{$\Cal G=\{G:G\subseteq X$ is open, $\mu G<\infty$,
$\nu G<\infty\}$.}

\noindent Because both $\mu$ and $\nu$ are locally finite, $\Cal G$
covers $X$.   If $G\in\Cal G$, there are Borel sets $E'$, $E''$ such
that $E'\subseteq E\cap G\subseteq E''$ and $\mu(E''\setminus E')=0$.
In this case $\nu(E''\setminus E')=0$ so $E\cap G\in\dom\nu$.   Because
$\nu$ is complete, locally determined and
$\tau$-additive, $E\in\dom\nu$ (414I).   If $\mu E=0$, it follows that

\Centerline{$\nu E=\sup_{F\subseteq E\text{ is closed}}\nu F=0$}

\noindent just as above.

\medskip

{\bf (c)} Thus $\nu$ measures $E$ whenever $\mu$ measures $E$, and $E$
is $\nu$-negligible whenever it is $\mu$-negligible.   I am sure you
will have no difficulty in believing that all the arguments above, in
particular that of 442E, can be re-cast to show that
$\dom\nu\subseteq\dom\mu$;  alternatively, apply the result in the form
just demonstrated to the left Haar measure $\nuprime$ and the right Haar
measure $\mu'$, where

\Centerline{$\nuprime E=\nu E^{-1}$,
\quad$\mu'E=\mu E^{-1}$}

\noindent as in 442C.
}%end of proof of 442F

\vleader{60pt}{442G}{Corollary} 
Let $X$ be a topological group and $\mu$ a left
Haar measure on $X$ with domain $\Sigma$.   Then, for $E\subseteq X$ and
$a\in X$,

\Centerline{$E\in\Sigma\iff E^{-1}\in\Sigma\iff Ea\in\Sigma$,}

\Centerline{$\mu E=0\iff\mu E^{-1}=0\iff\mu(Ea)=0$.}

\proof{ Apply 442F with $\nu E=\mu E^{-1}$.}

\leader{442H}{Remark} \dvro{If}{From 442F-442G we see that if} $X$ is
any topological group which
carries Haar measures, there is a distinguished $\sigma$-algebra
$\Sigma$ of subsets of $X$, which we may call the algebra of {\bf Haar
measurable sets}, which is the domain of any Haar measure on $X$.
Similarly, there is a $\sigma$-ideal $\Cal N$ of $\Cal PX$, the ideal of
{\bf Haar negligible sets}\footnote{{\bf Warning!} do not confuse with the
`Haar null' sets described in 444Ye below.}, which is the null ideal for
any Haar measure on $X$.   Both $\Sigma$ and $\Cal N$ are
translation-invariant and also invariant under the inversion operation
$x\mapsto x^{-1}$.

If we form the quotient $\frak A=\Sigma/\Cal N$,
then we have a fixed Dedekind complete Boolean algebra which is the
{\bf Haar measure algebra} of the group $X$ in the sense that any Haar
measure on $X$, whether left or right, has measure algebra based on
$\frak A$.   If $a\in X$, the maps $x\mapsto ax$, $x\mapsto xa$,
$x\mapsto x^{-1}$ give rise to Boolean automorphisms of $\frak A$.

\cmmnt{For a member of $\Sigma$, we have a notion of `$\sigma$-finite'
which is symmetric between left and right (442Xd).   We do not in
general have a corresponding two-sided notion of `finite measure'
(442Xg(i));  but of course we can if we wish speak of a set as having
`finite left Haar measure' or `finite right Haar measure' without
declaring which Haar measure we are thinking of.
It is the case, however, that if the group $X$ itself has finite left
Haar measure, it also has finite right Haar measure;  see
442Ic-d below.}

\leader{442I}{The modular function} Let $X$ be a topological group which
carries Haar measures.

\spheader 442Ia There is a group homomorphism
$\Delta:X\to\ooint{0,\infty}$ defined by the formula

\Centerline{$\mu(Ex)=\Delta(x)\mu E$ whenever $\mu$ is a left Haar
measure on $X$ and $E\in\dom\mu$.}

\noindent\prooflet{\Prf\ Fix on a left Haar measure $\tilde\mu$ on $X$.
For $x\in X$, let $\mu_x$ be the function defined by saying

\Centerline{$\mu_xE=\tilde\mu(Ex)$ whenever $E\subseteq X$,
$Ex\in\dom\mu$,}

\noindent that is, for every Haar measurable set $E\subseteq X$.
Because the function $\phi_x:X\to X$ defined by setting
$\phi_x(y)=yx^{-1}$ is a homeomorphism, $\mu_x=\tilde\mu\phi_x^{-1}$ is
a quasi-Radon measure on $X$;  and

\Centerline{$\mu_x(yE)=\tilde\mu(yEx)=\tilde\mu(Ex)=\mu_xE$}

\noindent whenever $\mu_x$ measures $E$, so $\mu_x$ is a left Haar
measure on $X$.   By 442B, there is a $\Delta(x)\in\ooint{0,\infty}$
such that $\mu_x=\Delta(x)\tilde\mu$;  because $\tilde\mu$ surely takes
at least one value in $\ooint{0,\infty}$, $\Delta(x)$ is uniquely
defined.

If $\mu$ is any other left Haar measure on $X$, then
$\mu=\alpha\tilde\mu$ for some $\alpha>0$, so that

\Centerline{$\mu(Ex)=\alpha\tilde\mu(Ex)
=\alpha\Delta(x)\tilde\mu E=\Delta(x)\mu E$.}

\noindent Thus $\Delta:X\to\ooint{0,\infty}$ has the property asserted
in the formula offered.

To see that $\Delta$ is a group homomorphism, take any $x$, $y\in X$ and
a Haar measurable set $E$ such that $0<\tilde\mu E<\infty$, and observe
that

\Centerline{$\Delta(xy)\tilde\mu E
=\tilde\mu(Exy)
=\Delta(y)\tilde\mu(Ex)
=\Delta(y)\Delta(x)\tilde\mu E$,}

\noindent so that
$\Delta(xy)=\Delta(y)\Delta(x)=\Delta(x)\Delta(y)$.\ \Qed
}%end of prooflet

$\Delta$ is called the {\bf left modular function} of $X$.

\spheader 442Ib We find now that $\nu(xE)=\Delta(x^{-1})\nu E$ whenever
$\nu$ is a right Haar measure on $X$, $x\in X$ and $E\subseteq X$ is
Haar measurable.
\prooflet{\Prf\ Let $\mu$ be the left Haar measure derived from $\nu$,
so that $\mu E=\nu E^{-1}$ whenever $E$ is Haar measurable.   If $x\in
X$ and $E\in\dom\nu$, then

\Centerline{$\nu(xE)=\mu(E^{-1}x^{-1})
=\Delta(x^{-1})\mu E^{-1}=\Delta(x^{-1})\nu E$.  \Qed}
}%end of prooflet

Thus we may call $x\mapsto\Delta(x^{-1})=\Bover1{\Delta(x)}$ the {\bf
right modular function} of $X$.

\spheader 442Ic If $X$ is abelian, then\cmmnt{ obviously}
$\Delta(x)=1$ for
every $x\in X$\cmmnt{, because $\mu(Ex)=\mu(xE)=\mu E$ whenever
$x\in X$, $\mu$ is a left Haar measure on $X$ and $E$ is Haar measurable}.
Equally, if any\cmmnt{ (therefore every)} left (or right) Haar measure
$\mu$ on $X$ is totally
finite, then\cmmnt{ $\mu(Xx)=\mu(xX)=\mu X$, so again} $\Delta(x)=1$
for every $x\in X$.   This will be the case\cmmnt{, in particular,}
for any compact
Hausdorff topological group\cmmnt{ (recall that by 441E any such group
carries Haar measures), because its Haar measures are locally finite,
therefore totally finite}.

Groups in which $\Delta(x)=1$ for every $x$ are called {\bf unimodular}.

\spheader 442Id \dvro{A}{From the definition of $\Delta$, we see
that a} topological group carrying Haar measures is
unimodular iff every left Haar measure is a right Haar measure.

\spheader 442Ie In particular, if a group has any totally finite (left
or right) Haar measure, its left and right Haar measures are the same,
and it has a unique Haar probability measure, which we may call its {\bf
normalized Haar measure}.

In the other direction, any group with its discrete topology is unimodular, since counting measure is a two-sided Haar measure.

\leader{442J}{Proposition} For any topological group carrying Haar
measures, its left modular function is continuous.

\proof{ Let $X$ be a topological group carrying Haar measures, with left
modular function $\Delta$.

{\bf (a)} If $\epsilon>0$, there is an open set $U_{\epsilon}$
containing the identity $e$ of $X$ such that $\Delta(x)\le 1+\epsilon$
for every $x\in U$.   \Prf\ Take any left Haar measure $\mu$ on $X$, and
an open set $G$ such that $0<\mu G<\infty$.   By 442Ab, there are an
open set $H$ and a neighbourhood $U_{\epsilon}$ of the identity such
that $HU_{\epsilon}\subseteq G$ and $\mu G\le(1+\epsilon)\mu H$.   If
$x\in U_{\epsilon}$, then $Hx\subseteq G$, so
$\Delta(x)=\Bover{\mu(Hx)}{\mu H}\le 1+\epsilon$.\ \Qed

\medskip

{\bf (b)} Now, given $x_0\in X$ and $\epsilon>0$, $V=\{x:x^{-1}x_0\in
U_{\epsilon},\,x_0^{-1}x\in U_{\epsilon}\}$ is an open set containing
$x_0$.   If $x\in V$, then

\Centerline{$\Delta(x)
=\Delta(x_0)\Delta(x_0^{-1}x)\le(1+\epsilon)\Delta(x_0)$,}

\Centerline{$\Delta(x_0)
=\Delta(x)\Delta(x^{-1}x_0)\le(1+\epsilon)\Delta(x)$,}

\noindent so

\Centerline{$\Bover1{1+\epsilon}\Delta(x_0)\le\Delta(x)
\le(1+\epsilon)\Delta(x)$.}

\noindent As $\epsilon$ is arbitrary, $\Delta$ is continuous at $x_0$;
as $x_0$ is arbitrary, $\Delta$ is continuous.
}%end of proof of 442J

\leader{442K}{Theorem} Let $X$ be a topological group and $\mu$ a
left Haar measure on $X$.   Let $\Delta$ be the left modular function of
$X$.

(a) $\mu(E^{-1})=\int_E\Delta(x^{-1})\mu(dx)$ for every $E\in\dom\mu$.

(b)(i) $\int f(x^{-1})\mu(dx)=\int\Delta(x^{-1})f(x)\mu(dx)$
whenever $f$ is a real-valued function such that either integral is
defined in $[-\infty,\infty]$;

\quad(ii) $\int f(x)\mu(dx)=\int\Delta(x^{-1})f(x^{-1})\mu(dx)$
whenever $f$ is a real-valued function such that either integral is
defined in $[-\infty,\infty]$.

(c) $\int f(xy)\mu(dx)=\Delta(y^{-1})\int f(x)\mu(dx)$ whenever $y\in X$
and $f$ is a real-valued function such that either integral is defined
in $[-\infty,\infty]$.

\proof{{\bf (a)(i)} Setting $\nu_1 E=\mu E^{-1}$ for Haar measurable
sets $E\subseteq X$, we
know that $\nu_1$ is a right Haar measure, so 442E tells us that

\Centerline{$\mu G\cdot\nu_1 H
=\int_H\nu_1(xG^{-1})\mu(dx)
=\int_H\mu(Gx^{-1})\mu(dx)
=\mu G\int_H\Delta(x^{-1})\mu(dx)$}

\noindent for all open sets $G$, $H\subseteq X$.   Since there is an
open set $G$ such that $0<\mu G<\infty$,
$\mu H^{-1}=\int_H\Delta(x^{-1})\mu(dx)$ for every open set
$H\subseteq X$.

\medskip

\quad{\bf (ii)} Now let $\nu_2$ be the indefinite-integral measure
defined by setting
\discrcenter{390pt}{$\nu_2 E=\int\Delta(x^{-1})\chi E(x)\mu(dx)$ }whenever
this is defined in $[0,\infty]$ (234J\formerly{2{}34B}).
Then $\nu_2$ is effectively
locally finite.   \Prf\ If $\nu_2 E>0$, then $\mu E>0$, so there is an
$n\in\Bbb N$ such that $\mu(E\cap H)>0$, where
$H$ is the open set $\{x:\Delta(x^{-1})<n\}$.   Now there is an open set
$G\subseteq H$
such that $\mu G<\infty$ and $\mu(E\cap G)>0$, in which case
$\nu_2 G\le n\mu G<\infty$ and $\nu_2(E\cap G)>0$.\ \Qed

Accordingly $\nu_2$ is a quasi-Radon measure (415Ob).   Since it agrees
with the quasi-Radon measure $\nu_1$ on open sets, by (i), the two are
equal;  that is, $\mu E^{-1}=\int_E\Delta(x^{-1})\mu(dx)$ for every
$E\in\dom\mu$.

\medskip

{\bf (b)(i)} Apply 235E with $X=Y$, $\Sigma=\Tau=\dom\mu$, $\mu=\nu$ and
$\phi(x)=x^{-1}$, $J(x)=\Delta(x^{-1})$, $g(x)=\Delta(x^{-1})f(x)$ for
$x\in X$.   From (a) we have

\Centerline{$\int J\times\chi(\phi^{-1}[F])d\mu
=\int_{F^{-1}}\Delta(x^{-1})\mu(dx)=\mu F=\nu F$}

\noindent for every $F\in\Tau$ (using 442G to see that
$F^{-1}\in\Sigma$).   So we get

$$\eqalign{\int f(x^{-1})\mu(dx)
&=\int\Delta(x^{-1})g(x^{-1})\mu(dx)
=\int J\times g\phi\,d\mu\cr
&=\int g\,d\nu
=\int\Delta(x^{-1})f(x)\mu(dx)\cr}$$

\noindent if any of the integrals is defined in $[-\infty,\infty]$.

\medskip

\quad{\bf (ii)} Set $\Reverse{f}(x)=f(x^{-1})$ whenever this is defined
(4A5C(c-ii)), and apply (i) to $\Reverse{f}$.

\wheader{442K}{4}{2}{2}{30pt}

{\bf (c)} Similarly, apply 235E with $\mu=\nu$, $\phi(x)=xy$,
$J(x)=\Delta(y)$ for every $x\in X$;  then

\Centerline{$\int J\times\chi(\phi^{-1}[F])d\mu
=\Delta(y)\mu(Fy^{-1})=\mu F$}

\noindent for every $F\in\dom\mu$, so

\Centerline{$\int f(xy)\mu(dx)
=\Delta(y^{-1})\int J\times f\phi\,d\mu
=\Delta(y^{-1})\int f(x)\mu(dx)$.}
}%end of proof of 442K

\leader{442L}{Corollary} Let $X$ be a group carrying Haar measures.   If
$\mu$ is a left Haar measure on $X$ and $\nu$ is a right Haar measure,
then each is an indefinite-integral measure over the other.

\proof{ Let $\Reverse\mu$ be the right Haar measure defined by setting
$\Reverse{\mu} E=\mu E^{-1}$ for every Haar measurable $E\subseteq X$.
Then $\Reverse{\mu} E=\int_E\Delta(x^{-1})\mu(dx)$ for every
$E\in\dom\mu=\dom\Reverse{\mu}$, so $\Reverse{\mu}$ is an 
indefinite-integral
measure over $\mu$;  because $\nu$ is a multiple of $\Reverse{\mu}$, 
it also
is an indefinite-integral measure over $\mu$.   Similarly, or because
$\Delta$ is strictly positive, $\mu$ is an indefinite-integral measure
over $\nu$.
}%end of proof of 442L

\exercises{\leader{442X}{Basic exercises $\pmb{>}$(a)}
%\spheader 442Xa
Let $X$ and $Y$ be topological groups with (left) Haar probability
measures $\mu$ and $\nu$, and $\phi:X\to Y$ a continuous surjective
group homomorphism.   Show that $\phi$ is \imp\ for $\mu$ and $\nu$.
%442B

\spheader 442Xb
(i) Let $X$ and $Y$ be two topological groups carrying Haar measures.
Show that the product topological group $X\times Y$ (4A5G) carries Haar
measures.
(ii) Let $\familyiI{X_i}$ be any family of topological groups carrying
totally finite Haar measures.   Show that the product group
$\prod_{i\in I}X_i$ carries a totally finite Haar measure.
\Hint{417O.}
%442D

\spheader 442Xc Let $X$ be a subgroup of the group $(\Bbb R,+)$.   Show
that $X$ carries Haar measures iff it is either discrete (so that
counting measure is a Haar measure on $X$) or of full outer Lebesgue
measure (so that the subspace measure on $X$ is a Haar measure).
\Hint{if $G$ has a Haar measure $\nu$ and is not discrete, then
$\nu(G\cap[\alpha,\beta])=(\beta-\alpha)\nu(G\cap[0,1])$ whenever
$\alpha\le\beta$.}   In particular, $\Bbb Q$ does not carry Haar
measures.
%442D

\spheader 442Xd Let $X$ be a topological group carrying Haar measures;
let $\Sigma$ be the algebra of Haar measurable subsets of $X$.   Let
$\mu$ and $\nu$ be any Haar measures on $X$ (either left or right).
Show that a set $E\in\Sigma$ can be covered by a sequence of sets of
finite measure for $\mu$ iff it can be covered by a sequence of sets of
finite measure for $\nu$.
%442H

\spheader 442Xe Let $X$ be a topological group carrying Haar measures
and $\frak A$ its Haar measure algebra (in the sense of 442H).   Show
that we have left, right and conjugacy actions of $X$ on $\frak A$ given
by the formulae $z\action E^{\ssbullet}=(zE)^{\ssbullet}$, 
$z\action E^{\ssbullet}=(Ez^{-1})^{\ssbullet}$ and
$z\action E^{\ssbullet}=(zEz^{-1})^{\ssbullet}$ for every Haar
measurable $E\subseteq X$ and every $z\in X$.
%442H

\sqheader 442Xf On $\BbbR^2$ define a binary operation $*$ by setting
$(\xi_1,\xi_2)*(\eta_1,\eta_2)=(\xi_1+\eta_1,\xi_2+e^{\xi_1}\eta_2)$.
(i) Show that $*$ is a group operation under which $\BbbR^2$ is a
locally compact topological group.   (ii) Show that Lebesgue measure
$\mu$ is a right Haar measure for $*$.   (iii) Let $\nu$
be the indefinite-integral measure on $\BbbR^2$ defined by setting $\nu
E=\int_Ee^{-\xi_1}d\xi_1d\xi_2$ for Lebesgue measurable sets
$E\subseteq\BbbR^2$.   Show that $\nu$ is a left Haar measure for $*$.
\Hint{263D.}
(iv) Thus $(\BbbR^2,*)$ is not unimodular.   (v) Show that the left
modular function of $(\BbbR^2,*)$ is $(\xi_1,\xi_2)\mapsto e^{-\xi_1}$.

\sqheader 442Xg Let $X$ be any topological group which is not
unimodular.   (i) Show that there is an open subset of $X$ which is of
finite measure for all left Haar measures on $X$ and of infinite measure
for all right Haar measures.   \Hint{the modular function is unbounded.}
(ii) Let $\mu$ be a left Haar measure on $X$ and $\nu$ a right Haar
measure.   Show that $L^0(\mu)=L^0(\nu)$ and
$L^{\infty}(\mu)=L^{\infty}(\mu)$, but that $L^p(\mu)\ne L^p(\nu)$ for
any $p\in\coint{1,\infty}$.
%442I

\spheader 442Xh Let $X$ and $Y$ be topological groups carrying Haar
measures, with left modular functions $\Delta_X$ and $\Delta_Y$
respectively.   Show that the left modular function of $X\times Y$ is
$(x,y)\mapsto\Delta_X(x)\Delta_Y(y)$.
%442I

\spheader 442Xi Let $X$ be any topological group and
$\Delta:X\to\ooint{0,\infty}$ a group homomorphism such that
$\{x:\Delta(x)\le 1+\epsilon\}$ is a neighbourhood of the identity in
$X$ for every $\epsilon>0$.    Show that $\Delta$ is continuous.
%442J

\spheader 442Xj Let $X$ be a topological group with a right Haar measure
$\nu$ and left modular function $\Delta$.   Show that
$\nu E^{-1}=\int_E\Delta(x)\nu(dx)$ for every Haar measurable set
$E\subseteq X$.
%442K

\leader{442Y}{Further exercises (a)}
%\spheader 442Ya
In 441Yb, show that the only $G$-invariant Radon measures on $\Cal C_s$
are multiples of Hausdorff $s(r-s)$-dimensional measure on $\Cal C_s$.
\Hint{$G$ itself is $\bover{r(r-1)}2$-dimensional (cf.\ 441Yd), and for
any $C\in\Cal C_s$ the stabilizer of $C$ is
$\bover{s(s-1)}2+\bover{(r-s)(r-s-1)}2$-dimensional.
See {\smc Federer 69}, 3.2.28.}
%442B

\spheader 442Yb Let $r\ge 1$, and let $X$ be the group of
non-singular $r\times r$ real matrices.   Regarding $X$ as an open
subset of
$\BbbR^{r^2}$, show that a two-sided Haar measure $\mu$ can be defined
on $X$ by setting $\mu E=\int_E\Bover1{|\det A|^r}\mu_L(dA)$, where
$\mu_L$ is Lebesgue measure on $\BbbR^{r^2}$;  so that $X$ is
unimodular.
%442I

\spheader 442Yc Show that there is a set $A\subseteq[0,1]$, of Lebesgue
outer measure $1$, such that no countable set of translates of $A$
covers any set of Lebesgue measure greater than $0$.   \Hint{let
$\langle F_{\xi}\rangle_{\xi<\frak c}$ run over the uncountable closed
subsets of $[0,1]$ with cofinal repetitions (4A3Fa), and enumerate the
countable subsets of $\Bbb R$ as $\langle I_{\xi}\rangle_{\xi<\frak c}$.
Choose inductively $x_{\xi}$, $x'_{\xi}\in F_{\xi}$ such that
$x_{\xi}\notin\bigcup_{\eta,\zeta<\xi}x'_{\eta}-I_{\zeta}$,
$x'_{\xi}\notin\bigcup_{\eta,\zeta\le\xi}x_{\eta}+I_{\zeta}$;  set
$A=\{x_{\xi}:\xi<\frak c\}$.}   Show that we can extend Lebesgue measure
on $\Bbb R$ to a translation-invariant measure for which $A$ is
negligible.   \Hint{417A.}
%+
}%end of exercises

\leader{442Z}{Problem} Let $X$ be a compact Hausdorff space, and $G$ the
group of autohomeomorphisms of $X$.   Suppose that $G$ acts transitively
on $X$.   Does it follow that there is at most one $G$-invariant Radon
probability measure on $X$?

\endnotes{
\Notesheader{442} Haar measure dominates the theory of locally compact
topological groups for two reasons:  it is ubiquitous (the existence
theorem, 441E) and essentially defined by the group structure (the
uniqueness theorem, 442B).   I have tried to show that these are rather
different results by setting the theorems out with different hypotheses.
I presented the existence theorem as a special case of 441C, which
demands a locally compact space and a group, but allows them to be
different.
In the uniqueness theorem (roughly following {\smc Halmos 50}) I demand
a group with an invariant quasi-Radon measure, but do not (at this
point) ask for any hypothesis of compactness.   In fact it will become
apparent in the next section that this is a somewhat spurious
generality;  442B and 442I here can be deduced from the traditional
forms in which the group is assumed to be locally compact and Hausdorff.
From the point of view of the topological measure theory to which this
volume is devoted, however, I think the small extra labour involved in
tracing through the arguments without relying on the Riesz
Representation Theorem is instructive.   For instance, it emphasizes
interesting features of the domains and null ideals of Haar
measures (442H).

There is however a more serious question concerning the uniqueness
theorem.   I do not know whether it really belongs to the theory of
topological groups, as described here, or to the theory of group actions
along with 441C.   The trouble is that I know of no example of a
Hausdorff space $X$ and a transitive group $G$ of homeomorphisms of $X$
such that $X$ carries $G$-invariant Radon measures which are not
multiples of each other (see 442Z).   443U and 443Xy below eliminate the
simplest possibilities.
We do need to put some restriction on the measures;  for instance,
counting measure on $\Bbb R$ is translation-invariant, but has nothing
to do with Lebesgue measure.   There are also proper
translation-invariant extensions of Lebesgue measure (442Yc);  for
far-reaching elaborations of this idea see {\smc Hewitt \& Ross 63},
\S16.
}%end of notes

\discrpage

