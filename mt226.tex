\frfilename{mt226.tex}
\versiondate{16.11.13}
\copyrightdate{2000}

\def\chaptername{The Fundamental Theorem of Calculus}
\def\sectionname{The Lebesgue decomposition of a function of bounded
variation}

\newsection{226}

I end this chapter with some notes on a method of analysing a general
function of bounded variation which may help to give a picture of what
such functions can be, though (apart from 226A)
it is hardly needed in this volume.

\leader{226A}{Sums over arbitrary index sets} \cmmnt{To get a full
picture of this
fragment of real analysis, a bit of preparation will be
helpful.   This concerns the notion of a sum over an arbitrary
index set, which I have rather been skirting around so far.

\header{226Aa}}{\bf (a)} If $I$ is any set and
$\langle a_i\rangle_{i\in I}$ any family in $[0,\infty]$, we set

\Centerline{$\sum_{i\in I}a_i
=\sup\{\sum_{i\in K}a_i:K\text{ is a finite subset of }I\}$,}

\noindent with the convention that $\sum_{i\in\emptyset}a_i=0$.
\cmmnt{(See 112Bd, 222Ba.)}   For general $a_i\in[-\infty,\infty]$, we
can set

\Centerline{$\sum_{i\in I}a_i=\sum_{i\in I}a_i^+-\sum_{i\in I}a_i^-$}

\noindent if this is defined in $[-\infty,\infty]$\cmmnt{, that is, at
least one of $\sum_{i\in I}a_i^+$, $\sum_{i\in I}a_i^-$ is finite},
where $a^+=\max(a,0)$ and $a^-=\max(-a,0)$ for each $a$.   If
$\sum_{i\in I}a_i$ is defined and finite, we say
that $\familyiI{a_i}$ is {\bf summable}.

\header{226Ab}{\bf (b)}\cmmnt{ Since this is a book on measure theory,
I will immediately describe the relationship between this kind of
summability
and an appropriate notion of integration.}   For any set $I$, we have
the corresponding `counting measure' $\mu$ on $I$\cmmnt{ (112Bd)}.
Every\cmmnt{ subset
of $I$ is measurable, so every} family $\langle a_i\rangle_{i\in I}$ of
real numbers is a measurable real-valued function on $I$.
A\cmmnt{ subset of $I$ has finite measure iff it is finite;
so a} real-valued
function $f$ on $I$ is `simple' if $K=\{i:f(i)\ne 0\}$ is finite.
\cmmnt{In this case,

\Centerline{$\int fd\mu=\sum_{i\in K}f(i)=\sum_{i\in I}f(i)$}

\noindent as defined in part (a).  The measure $\mu$ is semi-finite
(211Nc) so a non-negative function $f$ is integrable iff
$\int f=\sup_{\mu K<\infty}\int_Kf$ is finite (213B);   but of course
this supremum is precisely

\Centerline{$\sup\{\sum_{i\in K}f(i):K\subseteq I\text{ is finite}\}
=\sum_{i\in I}f(i)$.}


\noindent}%end of comment
Now a general function $f:I\to\Bbb R$ is integrable\cmmnt{ iff it is
measurable and $\int|f|d\mu<\infty$, that is,} iff
$\sum_{i\in I}|f(i)|<\infty$, and in this case

\dvro{\Centerline{$\int fd\mu=\sum_{i\in I}f(i)$,}}
{\Centerline{$\int fd\mu
=\int f^+d\mu-\int f^-d\mu
=\sum_{i\in I}f(i)^+-\sum_{i\in I}f(i)^-
=\sum_{i\in I}f(i)$,}}

\noindent\cmmnt{writing $f^{\pm}(i)=f(i)^{\pm}$ for each $i$.  }Thus
we have

\Centerline{$\sum_{i\in I}a_i=\int_Ia_i\mu(di)$,}

\noindent and the standard rules under which we allow $\infty$ as the
value of an integral\cmmnt{ (133A, 135F)} match\cmmnt{ well with} the
interpretations in (a) above.

\header{226Ac}{\bf (c)}\cmmnt{ Accordingly, and unsurprisingly, the
operation
of summation is a linear operation on the linear space of summable
families of real numbers.

}  I observe here that this notion of summability is `absolute';  a
family $\langle a_i\rangle_{i\in I}$ is summable iff it is absolutely
summable.   \cmmnt{This is necessary because it must also be
`unconditional';  we have no structure on an arbitrary set $I$ to guide
us to take the sum in any particular order.   See 226Xa.   In
particular, I distinguish between `$\sum_{n\in\Bbb N}a_n$', which in
this book will always be interpreted as in 226A above, and
`$\sum_{n=0}^{\infty}a_n$' which (if it makes a difference) should be
interpreted as $\lim_{m\to\infty}\sum_{n=0}^ma_n$.   So, for instance,
$\sum_{n=0}^{\infty}\Bover{(-1)^n}{n+1}=\ln 2$, while
$\sum_{n\in\Bbb N}\Bover{(-1)^n}{n+1}$ is undefined.   Of course
$\sum_{n=0}^{\infty}a_n=\sum_{n\in\Bbb N}a_n$ whenever the latter is
defined in $[-\infty,\infty]$.}

\header{226Ad}{\bf (d)}\cmmnt{ There is another, and very important,
approach
to the sum described here.}   If $\langle a_i\rangle_{i\in I}$ is an
(absolutely) summable family of real numbers, then for every
$\epsilon>0$ there is a finite $K\subseteq I$ such that
$\sum_{i\in I\setminus K}|a_i|\le\epsilon$.   \prooflet{\Prf\ This is
nothing but a
special case of 225A;  there is a set $K$ with $\mu K<\infty$ such that
$\int_{I\setminus K}|a_i|\mu(di)\le\epsilon$, but

\Centerline{$\int_{I\setminus K}|a_i|\mu(di)
=\sum_{i\in I\setminus K}|a_i|$.   \Qed}

\noindent}\cmmnt{(Of course there are `direct' proofs of this
result from
the definition in (a), not mentioning measures or integrals.   But
I think you will see that they rely on the same idea as that in the
proof of 225A.)  }Consequently, for any family
$\langle a_i\rangle_{i\in I}$ of real numbers and any $s\in\Bbb R$, the
following are equiveridical:

\inset{(i) $\sum_{i\in I}a_i=s$;}

\inset{(ii) for every $\epsilon>0$ there is a finite $K\subseteq I$ such
that $|s-\sum_{i\in J}a_i|\le\epsilon$ whenever $J$ is finite and
$K\subseteq J\subseteq I$.}

\noindent\prooflet{\Prf\ {\bf (i)$\Rightarrow$(ii)} Take $K$ such that
$\sum_{i\in
I\setminus K}|a_i|\le\epsilon$.   If $K\subseteq J\subseteq I$, then

\Centerline{$|s-\sum_{i\in J}a_i|=|\sum_{i\in I\setminus J}a_i|
\le\sum_{i\in I\setminus K}|a_i|\le\epsilon$.}

{\bf (ii)$\Rightarrow$(i)} Let $\epsilon>0$, and let $K\subseteq I$ be
as described in (ii).   If $J\subseteq I\setminus K$ is any finite set,
then set $J_1=\{i:i\in J,\,a_i\ge 0\}$, $J_2=J\setminus J_1$.   We have

$$\eqalign{\sum_{i\in J}|a_i|
&=|\sum_{i\in J_1\cup K}a_i-\sum_{i\in J_2\cup K}a_i|\cr
&\le|s-\sum_{i\in J_1\cup K}a_i|+|s-\sum_{i\in J_2\cup K}a_i|
\le 2\epsilon.\cr}$$

\noindent As $J$ is arbitrary, $\sum_{i\in I\setminus K}|a_i|\le
2\epsilon$ and

\Centerline{$\sum_{i\in I}|a_i|\le\sum_{i\in K}|a_i|+2\epsilon<\infty$.}

\noindent Accordingly $\sum_{i\in I}a_i$ is well-defined in $\Bbb R$.
Also

\Centerline{$|s-\sum_{i\in I}a_i|
\le|s-\sum_{i\in K}a_i|+|\sum_{i\in I\setminus K}a_i|
\le\epsilon+\sum_{i\in I\setminus K}|a_i|
\le 3\epsilon$.}

\noindent As $\epsilon$ is arbitrary, $\sum_{i\in I}a_i=s$, as required.
\Qed}

\cmmnt{In this way, we express $\sum_{i\in I}a_i$ directly as a limit;
we could write it as

\Centerline{$\sum_{i\in I}a_i=\lim_{K\uparrow I}\sum_{i\in K}a_i$,}

\noindent on the understanding that we look at finite sets $K$ in the
right-hand formula.
}%end of comment

\header{226Ae}{\bf (e)}\cmmnt{ Yet another approach is through the
following
fact.}   If $\sum_{i\in I}|a_i|<\infty$, then\cmmnt{ for any
$\delta>0$ the
set $\{i:|a_i|\ge\delta\}$ is finite, indeed can have at most
${1\over{\delta}}\sum_{i\in I}|a_i|$ members;  consequently}

\Centerline{$J=\{i:a_i\ne 0\}
=\bigcup_{n\in\Bbb N}\{i:|a_i|\ge 2^{-n}\}$}

\noindent is countable\cmmnt{ (1A1F)}.   If $J$ is finite,
then\cmmnt{ of course} $\sum_{i\in I}a_i=\sum_{i\in J}a_i$ reduces to a finite sum.   Otherwise, we can
enumerate $J$ as $\sequencen{j_n}$, and we shall have

\Centerline{$\sum_{i\in I}a_i=\sum_{i\in
J}a_i=\lim_{n\to\infty}\sum_{k=0}^na_{j_k}=\sum_{n=0}^{\infty}a_{j_n}$
\dvro{.}{}}

\noindent\cmmnt{(using (d) to reduce the sum $\sum_{i\in J}a_i$ to a
limit of
finite sums).  }Conversely, if $\langle a_i\rangle_{i\in I}$ is such
that there is a countably infinite $J\subseteq\{i:a_i\ne 0\}$ enumerated
as $\sequencen{j_n}$, and if $\sum_{n=0}^{\infty}|a_{j_n}|<\infty$, then
$\sum_{i\in I}a_i$ will be $\sum_{n=0}^{\infty}a_{j_n}$.

\spheader 226Af
%new 2008
\cmmnt{It will be useful later to have a fragment of general theory.}
Let $I$ and
$J$ be sets and $\langle a_{ij}\rangle_{i\in I,j\in J}$ a family in
$[0,\infty]$.   Then

\Centerline{$\sum_{(i,j)\in I\times J}a_{ij}
=\sum_{i\in I}(\sum_{j\in J}a_{ij})
=\sum_{j\in J}(\sum_{i\in I}a_{ij})$.}

\prooflet{\noindent\Prf\ (i) If
$\sum_{(i,j)\in I\times J}a_{ij}>u$, then there is a finite set
$M\subseteq I\times J$ such that
$\sum_{(i,j)\in M}a_{ij}>u$.   Now $K=\{i:(i,j)\in M\}$ and
$L=\{j:(i,j)\in M\}$ are finite, so

$$\eqalignno{\sum_{i\in I}\sum_{j\in J}a_{ij}
&\ge\sum_{i\in K}\sum_{j\in J}a_{ij}
\ge\sum_{i\in K}\sum_{j\in L}a_{ij}\cr
\displaycause{because $\sum_{j\in J}a_{ij}\ge\sum_{j\in L}a_{ij}$ for every
$i$}
&=\sum_{(i,j)\in K\times L}a_{ij}
\ge\sum_{(i,j)\in M}a_{ij}
>u.\cr}$$

\noindent As $u$ is arbitrary,
$\sum_{i\in I}\sum_{j\in J}a_{ij}\ge\sum_{(i,j)\in I\times J}a_{ij}$.
(ii) If $\sum_{i\in I}\sum_{j\in J}a_{ij}>u\ge 0$, there is a
non-empty finite
set $K\subseteq I$ such that $\sum_{i\in K}\sum_{j\in J}a_{ij}>u$.
Let $\epsilon\in\ooint{0,1}$ be such that
$\sum_{i\in K}\sum_{j\in J}a_{ij}>u+\epsilon$, and set
$\delta=\Bover{\epsilon}{\#(K)}$.   For each $i\in K$ set
$\gamma_i=\min(u+1,\sum_{j\in J}a_{ij})-\delta$;
then

\Centerline{$\epsilon+\sum_{i\in K}\gamma_i
=\sum_{i\in K}\min(u+1,\sum_{j\in J}a_{ij})
\ge\min(u+1,\sum_{i\in K}\sum_{j\in J}a_{ij})
>u+\epsilon$,}

\noindent so $\sum_{i\in K}\gamma_i>u$.   For each $i\in K$,
$\gamma_i<\sum_{j\in J}a_{ij}$, so there is a finite
$L_i\subseteq J$ such that $\sum_{j\in L_i}a_{ij}\ge\gamma_i$.
Set $M=\{(i,j):i\in K$, $j\in L_i\}$, so that $M$ is a finite subset of
$I\times J$;  then

\Centerline{$\sum_{(i,j)\in I\times J}a_{ij}
\ge\sum_{(i,j)\in M}a_{ij}
=\sum_{i\in K}\sum_{j\in L_i}a_{ij}
\ge\sum_{i\in K}\gamma_i
>u$.}

\noindent As $u$ is arbitrary,
$\sum_{(i,j)\in I\times J}a_{ij}\ge\sum_{i\in I}\sum_{j\in J}a_{ij}$ and
these two sums are equal.   (iii) Similarly,
$\sum_{(i,j)\in I\times J}a_{ij}=\sum_{j\in J}\sum_{i\in I}a_{ij}$.\ \Qed}

\leader{226B}{Saltus functions}\cmmnt{ Now we are ready for a special
type of
function of bounded variation on $\Bbb R$.}   Suppose that $a<b$ in
$\Bbb R$.

\header{226Ba}{\bf (a)} A (real) {\bf saltus function} on $[a,b]$ is
a function $F:[a,b]\to\Bbb R$ expressible in the form

\Centerline{$F(x)=\sum_{t\in\coint{a,x}}u_t+\sum_{t\in[a,x]}v_t$}

\noindent for $x\in [a,b]$, where $\langle
u_t\rangle_{t\in\coint{a,b}}$,
$\langle v_t\rangle_{t\in[a,b]}$ are real-valued families such that
$\sum_{t\in\coint{a,b}}|u_t|$ and
$\sum_{t\in[a,b]}|v_t|$ are finite.

\header{226Bb}{\bf (b)} For any function $F:[a,b]\to\Bbb R$ we can write

\Centerline{$F(x^+)=\lim_{y\downarrow x}F(y)$ if $x\in\coint{a,b}$ and
the limit exists,}

\Centerline{$F(x^-)=\lim_{y\uparrow x}F(y)$ if $x\in\ocint{a,b}$ and the
limit exists.}

\noindent\cmmnt{(I hope that this will not lead to confusion with the
alternative interpretation of $x^+$ as $\max(x,0)$.)  }Observe that if
$F$ is a saltus function, as defined in (b),
with associated families $\langle u_t\rangle_{t\in\coint{a,b}}$ and
$\langle v_t\rangle_{t\in[a,b]}$, then $v_a=F(a)$, $v_x=F(x)-F(x^-)$ for
$x\in\ocint{a,b}$ and $u_x=F(x^+)-F(x)$ for $x\in\coint{a,b}$.
\prooflet{\Prf\ Let
$\epsilon>0$.   As remarked in 226Ad, there is a finite
$K\subseteq[a,b]$ such that

\Centerline{$\sum_{t\in\coint{a,b}\setminus
K}|u_t|+\sum_{t\in[a,b]\setminus K}|v_t|\le\epsilon$.}

\noindent Given $x\in[a,b]$, let $\delta>0$ be such that
$[x-\delta,x+\delta]$ contains no point of $K$ except perhaps $x$.
In this case, if $\max(a,x-\delta)\le y<x$, we must have

$$\eqalign{|F(y)-(F(x)-v_x)|
&=|\sum_{t\in\coint{y,x}}u_t+\sum_{t\in\ooint{y,x}}v_t|\cr
&\le\sum_{t\in\coint{a,b}\setminus K}|u_t|
  +\sum_{t\in[a,b]\setminus K}|v_t|
\le\epsilon,\cr}$$

\noindent while if $x<y\le\min(b,x+\delta)$ we shall have

$$\eqalign{|F(y)-(F(x)+u_x)|
&=|\sum_{t\in\ooint{x,y}}u_t+\sum_{t\in\ocint{x,y}}v_t|\cr
&\le\sum_{t\in\coint{a,b}\setminus K}|u_t|
  +\sum_{t\in[a,b]\setminus K}|v_t|
\le\epsilon.\cr}$$

\noindent As $\epsilon$ is arbitrary, we get $F(x^-)=F(x)-v_x$ (if
$x>a$) and $F(x^+)=F(x)+u_x$ (if $x<b$).   \Qed}

\cmmnt{It follows that }$F$ is continuous at $x\in\ooint{a,b}$ 
iff $u_x=v_x=0$,
while $F$ is continuous at $a$ iff $u_a=0$ and $F$ is continuous at $b$
iff $v_b=0$.   In particular, $\{x:x\in[a,b],\,F\text{ is not continuous
at }x\}$ is countable\cmmnt{ (see 226Ae)}.

\header{226Bc}{\bf (c)} If $F$ is a saltus function defined on $[a,b]$,
with associated families $\langle u_t\rangle_{t\in\coint{a,b}}$ and
$\langle v_t\rangle_{t\in[a,b]}$, then $F$ is of bounded variation on
$[a,b]$, and

\Centerline{$\Var_{[a,b]}(F)
\le\sum_{t\in\coint{a,b}}|u_t|+\sum_{t\in\ocint{a,b}}|v_t|$.}

\noindent\prooflet{\Prf\ If $a\le x<y\le b$, then

\Centerline{$F(y)-F(x)=u_x+\sum_{t\in\ooint{x,y}}(u_t+v_t)+v_y$,}

\noindent so

\Centerline{$|F(y)-F(x)|
\le\sum_{t\in\coint{x,y}}|u_t|+\sum_{t\in\ocint{x,y}}|v_t|$.}

\noindent If $a\le a_0\le a_1\le\ldots\le a_n\le b$, then

$$\eqalign{\sum_{i=1}^n|F(a_i)-F(a_{i-1})|
&\le\sum_{i=1}^n\bigl(\sum_{t\in\coint{a_{i-1},a_i}}|u_t|
   +\sum_{t\in\ocint{a_{i-1},a_i}}|v_t|\bigr)\cr
&\le\sum_{t\in\coint{a,b}}|u_t|+\sum_{t\in\ocint{a,b}}|v_t|.\cr}$$

\noindent Consequently

\Centerline{$\Var_{[a,b]}(F)
\le\sum_{t\in\coint{a,b}}|u_t|+\sum_{t\in\ocint{a,b}}|v_t|<\infty$. \Qed}}

\header{226Bd}{\bf (d)} The inequality in (c) is actually an equality.
\prooflet{To see this, note first that if $a\le x<y\le b$, then
$\Var_{[x,y]}(F)\ge|u_x|+|v_y|$.   \Prf\ I noted in (b) that
$u_x=\lim_{t\downarrow x}F(t)-F(x)$ and
$v_y=F(y)-\lim_{t\uparrow y}F(t)$.
So, given $\epsilon>0$, we can find $t_1$, $t_2$ such that
$x<t_1\le t_2<y$ and

\Centerline{$|F(t_1)-F(x)|\ge|u_x|-\epsilon$,
\quad$|F(y)-F(t_2)|\ge|v_y|-\epsilon$.}

\noindent Now

\Centerline{$\Var_{[x,y]}(F)
\ge|F(t_1)-F(x)|+|F(t_2)-F(t_1)|+|F(y)-F(t_2)|
\ge|u_x|+|v_y|-2\epsilon$.}

\noindent As $\epsilon$ is arbitrary, we have the result.   \Qed

Now, given $a\le t_0<t_1<\ldots<t_n\le b$, we must have

$$\eqalignno{\Var_{[a,b]}(F)
&\ge\sum_{i=1}^n\Var_{[t_{i-1},t_i]}(F)\cr
\noalign{\noindent(using 224Cc)}
&\ge\sum_{i=1}^n|u_{t_{i-1}}|+|v_{t_i}|.\cr}$$

\noindent As $t_0,\ldots,t_n$ are arbitrary,

\Centerline{$\Var_{[a,b]}(F)
\ge\sum_{t\in\coint{a,b}}|u_t|+\sum_{t\in\ocint{a,b}}|v_t|$,}

\noindent as required.}

\header{226Be}{\bf (e)} Because a saltus function is of bounded
variation\cmmnt{ ((c) above)}, it is differentiable almost
everywhere\cmmnt{ (224I)}.
In fact its derivative is zero almost everywhere.   \prooflet{\Prf\ Let
$F:[a,b]\to\Bbb R$ be a saltus function, with associated families
$\langle u_t\rangle_{t\in\coint{a,b}}$, $\langle
v_t\rangle_{t\in[a,b]}$.
Let $\epsilon>0$.   Let $K\subseteq[a,b]$ be a finite set such that

\Centerline{$\sum_{t\in\coint{a,b}\setminus
K}|u_t|+\sum_{t\in[a,b]\setminus K}|v_t|\le\epsilon$.}

\noindent Set

$$\eqalign{u'_t
&=u_t\text{ if }t\in\coint{a,b}\cap K,\cr
&=0\text{ if }t\in\coint{a,b}\setminus K,\cr
v'_t&=v_t\text{ if }t\in K,\cr
&=0\text{ if }t\in[a,b]\setminus K,\cr
u''_t&=u_t-u'_t\text{ for }t\in\coint{a,b},\cr
v''_t&=v_t-v'_t\text{ for }t\in[a,b].\cr}$$

\noindent Let $G$, $H$ be the saltus functions corresponding to
$\langle u'_t\rangle_{t\in\coint{a,b}}$,
$\langle v'_t\rangle_{t\in[a,b]}$ and
$\langle u''_t\rangle_{t\in\coint{a,b}}$
$\langle v''_t\rangle_{t\in[a,b]}$, so that $F=G+H$.   Then $G'(t)=0$
for every $t\in\ooint{a,b}\setminus K$, since $\ooint{a,b}\setminus K$
comprises a finite number of open intervals on each of which $G$ is
constant.   So $G'=0$ a.e.\ and $F'\eae H'$.   On the other hand,

\Centerline{$\int_a^b|H'|\le\Var_{[a,b]}(H)
=\sum_{t\in\coint{a,b}\setminus K}|u_t|
  +\sum_{t\in\ocint{a,b}\setminus K}|v_t|
\le\epsilon$,}

\noindent using 224I and (d) above.   So

\Centerline{$\int_a^b|F'|=\int_a^b|H'|\le\epsilon$.}

\noindent As $\epsilon$ is arbitrary, $\int_a^b|F'|=0$ and $F'=0$ a.e.,
as claimed.   \Qed}

\leader{226C}{The Lebesgue decomposition of a function of bounded
variation}  Take $a$, $b\in\Bbb R$ with $a<b$.

\header{226Ca}{\bf (a)} If $F:[a,b]\to\Bbb R$ is non-decreasing, set
$v_a=0$, $v_t=F(t)-F(t^-)$ for $t\in\ocint{a,b}$, $u_t=F(t^+)-F(t)$ for
$t\in\coint{a,b}$\cmmnt{, defining $F(t^+)$, $F(t^-)$ as in 226Bb}.
Then all the $v_t$, $u_t$ are
non-negative, and\cmmnt{ if $a<t_0<t_1<\ldots<t_n< b$, then

\Centerline{$\sum_{i=0}^n(u_{t_i}+v_{t_i})
=\sum_{i=0}^n(F(t_i^+)-F(t_i^-))\le F(b)-F(a)$.}

\noindent   Accordingly}
$\sum_{t\in\coint{a,b}}u_t$ and $\sum_{t\in[a,b]}v_t$ are both finite.
Let $F_p$ be the corresponding saltus function\cmmnt{, as defined in
226Ba, so that

\Centerline{$F_p(x)
=F(a^+)-F(a)+\sum_{t\in\ooint{a,x}}(F(t^+)-F(t^-))+F(x)
-F(x^-)$}

\noindent if $a<x\le b$}.   \cmmnt{If $a\le x<y\le b$ then

$$\eqalign{F_p(y)-F_p(x)
&=F(x^+)-F(x)+\sum_{t\in\ooint{x,y}}(F(t^+)-F(t^-))+F(y)-F(y^-)\cr
&\le F(y)-F(x)\cr}$$

\noindent because if $x=t_0<t_1<\ldots<t_n<t_{n+1}=y$ then

$$\eqalign{F(x^+)-F(x)
&+\sum_{i=1}^{n}(F(t_i^+)-F(t_i^-))+F(y)-F(y^-)\cr
&=F(y)-F(x)-\sum_{i=1}^{n+1}(F(t_i^-)-F(t_{i-1}^+))
\le F(y)-F(x).\cr}$$

\noindent Accordingly both }$F_p$ and $F_c=F-F_p$ are
non-decreasing.   \cmmnt{Also, because

\Centerline{$F_p(a)=0=v_a$,}

\Centerline{$F_p(t)-F_p(t^-)=v_t=F(t)-F(t^-)$ for
$t\in\ocint{a,b}$,}

\Centerline{$F_p(t^+)-F_p(t)=u_t=F(t^+)-F(t)$ for
$t\in\coint{a,b}$,}

\noindent we shall have

\Centerline{$F_c(a)=F(a)$,}

\Centerline{$F_c(t)=F_c(t^-)$ for
$t\in\ocint{a,b}$,}

\Centerline{$F_c(t)=F_c(t^+)$ for $t\in\coint{a,b}$,}

\noindent and }$F_c$ is continuous.

Clearly this expression of $F=F_p+F_c$ as the sum of a saltus function
and a continuous function is unique, except that we can freely add a
constant to one if we subtract it from the other.

\header{226Cb}{\bf (b)}\cmmnt{ Still taking $F:[a,b]\to\Bbb R$ to be
non-decreasing, we know that $F'$ is integrable (222C);
moreover,
$F'\eae F'_c$, by 226Be.}   Set $F_{ac}(x)=F(a)+\int_a^xF'$
for each $x\in[a,b]$.   \cmmnt{We have

\Centerline{$F_{ac}(y)-F_{ac}(x)=\int_x^yF_c'\le F_c(y)-F_c(x)$}

\noindent for $a\le x\le y\le b$ (222C again), so} $F_{cs}=F_c-F_{ac}$
is still non-decreasing;  \cmmnt{$F_{ac}$ is continuous (225A), 
so }$F_{cs}$ is
continuous;  \cmmnt{$F_{ac}'\eae F'\eae F_c'$ (222E), 
so }$F_{cs}'=0$ a.e.

Again, the expression of $F_c=F_{ac}+F_{cs}$ as the sum of an absolutely
continuous function and a function with zero derivative almost
everywhere is unique, except for the possibility of moving a constant
from one to the other\cmmnt{, because two absolutely continuous
functions whose
derivatives are equal almost everywhere must differ by a constant
(225D)}.

\header{226Cc}{\bf (c)} Putting\cmmnt{ all} these together:  if
$F:[a,b]\to\Bbb R$ is any
non-decreasing function, it is expressible as $F_p+F_{ac}+F_{cs}$, where
$F_p$ is a saltus function, $F_{ac}$ is absolutely continuous, and
$F_{cs}$ is continuous and differentiable, with zero derivative, almost
everywhere;  all three components are non-decreasing;  and the
expression is unique if we say that $F_{ac}(a)=F(a)$ and
$F_p(a)=F_{cs}(a)=0$.

The Cantor function $f:[0,1]\to[0,1]$\cmmnt{ (134H)} is continuous and
$f'=0$ a.e.\cmmnt{ (134Hb)}, so $f_p=f_{ac}=0$ and $f=f_{cs}$.
Setting $g(x)=\bover12(x+f(x))$ for $x\in[0,1]$\cmmnt{, as in 134I},
we get $g_p(x)=0$, $g_{ac}(x)={x\over 2}$ and
$g_{cs}(x)={1\over 2}f(x)$.

\header{226Cd}{\bf (d)} Now suppose that $F:[a,b]\to\Bbb R$ is of
bounded variation.
Then it is expressible as a difference $G-H$ of
non-decreasing functions\cmmnt{ (224D)}.   So writing $F_p=G_p-H_p$,
etc., we can express $F$ as a sum $F_p+F_{cs}+F_{ac}$, where $F_p$ is a
saltus function, $F_{ac}$ is absolutely continuous, $F_{cs}$ is
continuous, $F_{cs}'=0$ a.e., $F_{ac}(a)=F(a)$ and $F_{cs}(a)=F_{p}(a)=0$.
Under these conditions the expression is unique\dvro{.}{, because (for
instance) $F_p(t^+)-F_p(t)=F(t^+)-F(t)$ for $t\in\coint{a,b}$, while
$F_{ac}'\eae(F-F_p)'\eae F'$.}

This is a {\bf Lebesgue decomposition} of the function $F$.
\cmmnt{(I have to say `a' Lebesgue decomposition because of course
the assignments $F_{ac}(a)=F(a)$, $F_{p}(a)=F_{cs}(a)=0$ are arbitrary.)}   I will call $F_p$ the {\bf saltus part} of $F$.

\leader{226D}{Complex functions} \cmmnt{The modifications needed to
deal with complex functions are elementary.

\medskip

}{\bf (a)} If $I$ is any set and
$\langle a_j\rangle_{j\in I}$ is a family of complex numbers, then the
following are equiveridical:

\inset{(i) $\sum_{j\in I}|a_j|<\infty$;}

\inset{(ii) there is an $s\in\Bbb C$ such that for every $\epsilon>0$
there is a finite $K\subseteq I$ such that
$|s-\sum_{j\in J}a_j|\le\epsilon$ whenever $J$ is finite and
$K\subseteq J\subseteq I$.}

\noindent In this case

\Centerline{$s=\sum_{j\in I}\Real(a_j)+i\sum_{j\in
I}\Imag(a_j)=\int_Ia_j\mu(dj)$,}

\noindent where $\mu$ is counting measure on $I$, and we write
$s=\sum_{j\in I}a_j$.

\header{226Db}{\bf (b)} If $a<b$ in $\Bbb R$, a complex {\bf saltus
function} on $[a,b]$ is a function $F:[a,b]\to\Bbb C$ expressible in the
form

\Centerline{$F(x)=\sum_{t\in\coint{a,x}}u_t+\sum_{t\in[a,x]}v_t$}

\noindent for $x\in [a,b]$, where $\langle
u_t\rangle_{t\in\coint{a,b}}$,
$\langle v_t\rangle_{t\in[a,b]}$ are complex-valued families such that
$\sum_{t\in\coint{a,b}}|u_t|$ and
$\sum_{t\in[a,b]}|v_t|$ are finite\cmmnt{;  that is, if the real and
imaginary parts of $F$ are saltus functions}.   In this case $F$ is
continuous except at countably many points and differentiable, with zero
derivative, almost everywhere in $[a,b]$, and

\Centerline{$u_x=\lim_{t\downarrow x}F(t)-F(x)$ for every
$x\in\coint{a,b}$,}

\Centerline{$v_x=\lim_{t\uparrow x}F(x)-F(t)$ for every
$x\in\ocint{a,b}$\dvro{.}{}}

\noindent\cmmnt{(apply the results of 226B to the real and imaginary
parts of $F$).   }$F$ is of bounded variation, and its variation is

\Centerline{$\Var_{[a,b]}(F)
=\sum_{t\in\coint{a,b}}|u_t|+\sum_{t\in\ocint{a,b}}|v_t|$\dvro{.}{}}

\cmmnt{\noindent (repeat the arguments of 226Bc-d).}

\header{226Dc}{\bf (c)} If $F:[a,b]\to\Bbb C$ is a function of bounded
variation, where $a<b$ in $\Bbb R$, it is uniquely expressible as
$F=F_p+F_{cs}+F_{ac}$, where $F_p$ is a saltus function, $F_{ac}$ is
absolutely continuous, $F_{cs}$ is continuous and has zero derivative
almost everywhere, and $F_{ac}(a)=F(a)$, $F_p(a)=F_{cs}(a)=0$.
\cmmnt{(Apply
226C to the real and imaginary parts of $F$.)}

\leader{226E}{}\cmmnt{ As an elementary exercise in the language of
226A, I interpolate a version of a theorem of B.Levi
%Levi 1906
which is sometimes useful.

\medskip

\noindent}{\bf Proposition} Let $(X,\Sigma,\mu)$ be a measure space, $I$
a {\it countable} set, and $\familyiI{f_i}$ a family of $\mu$-integrable
real- or complex-valued functions such that $\sum_{i\in I}\int|f_i|d\mu$
is finite.   Then
$f(x)=\sum_{i\in I}f_i(x)$ is defined almost everywhere and
$\int fd\mu=\sum_{i\in I}\int f_id\mu$.

\proof{ If $I$ is finite this is elementary.   Otherwise, since there
must be a bijection between $I$ and $\Bbb N$, we may take it that
$I=\Bbb N$.   Setting $g_n=\sum_{i=0}^n|f_i|$ for each $n$, we have a
non-decreasing sequence $\sequencen{g_n}$ of integrable functions such
that $\int g_n\le\sum_{i\in\Bbb N}\int|f_i|$ for every $n$, so that
$g=\sup_{n\in\Bbb N}g_n$ is integrable, by B.Levi's theorem as stated in
123A.   In particular, $g$ is finite almost everywhere.   Now if
$x\in X$ is such that $g(x)$ is defined and finite,
$\sum_{i\in J}|f_i(x)|\le g(x)$ for every finite $J\subseteq\Bbb N$, so
$\sum_{i\in\Bbb N}|f_i(x)|$ and $\sum_{i\in\Bbb N}f_i(x)$ are defined.
In this case, of course,
$\sum_{i\in\Bbb N}f_i(x)=\lim_{n\to\infty}\sum_{i=0}^nf_i(x)$.   But
$|\sum_{i=0}^nf_i|\leae g$ for each $n$, so Lebesgue's Dominated
Convergence Theorem tells us that

\Centerline{$\int\sum_{i\in\Bbb N}f_i
=\lim_{n\to\infty}\int\sum_{i=0}^nf_i
=\lim_{n\to\infty}\sum_{i=0}^n\int f_i
=\sum_{i\in\Bbb N}\int f_i$.}
}%end of proof of 226E

\exercises{
\leader{226X}{Basic exercises $\pmb{>}$(a)}
%\sqheader 226Xa
Suppose that $I$ and $J$ are sets and that
$\familyiI{a_i}$ is a summable family of real numbers.   (i) Show that
if $f:J\to I$ is injective then $\family{j}{J}{a_{f(j)}}$ is summable.
(ii) Show that if $g:I\to J$ is any function, then
$\sum_{j\in J}\sum_{i\in g^{-1}[\{j\}]}a_i$ is defined and equal to
$\sum_{i\in I}a_i$.
%226A

\sqheader 226Xb A {\bf step-function} on an
interval $[a,b]$ is a function $F$
such that, for suitable $t_0,\ldots,t_n$ with $a=t_0\le\ldots\le t_n=b$,
$F$ is constant on each interval $\ooint{t_{i-1},t_i}$.   Show that
$F:[a,b]\to\Bbb R$ is a saltus function iff for every $\epsilon>0$ there
is a step-function $G:[a,b]\to\Bbb R$ such that
$\Var_{[a,b]}(F-G)\le\epsilon$.
%226B

\spheader 226Xc Let $F$, $G$ be real-valued functions of bounded
variation defined on an interval $[a,b]\subseteq\Bbb R$.   Show that, in
the language of 226C,

\Centerline{$(F+G)_p=F_p+G_p$,\quad $(F+G)_c=F_c+G_c$,}

\Centerline{$(F+G)_{cs}=F_{cs}+G_{cs}$,
\quad$(F+G)_{ac}=F_{ac}+G_{ac}$.}
%226C

\sqheader 226Xd Let $F$ be a real-valued function of bounded
variation on an interval $[a,b]\subseteq\Bbb R$.   Show that, in the
language of 226C,

\Centerline{$\Var_{[a,b]}(F)
=\Var_{[a,b]}(F_p)+\Var_{[a,b]}(F_c)
=\Var_{[a,b]}(F_p)+\Var_{[a,b]}(F_{cs})+\Var_{[a,b]}(F_{ac})$.}
%226C

\spheader 226Xe Let $F$ be a real-valued function of bounded
variation on an interval $[a,b]\subseteq\Bbb R$.   Show that $F$ is
absolutely continuous iff $\Var_{[a,b]}(F)=\int_a^b|F'|$.
%226Xd

\spheader 226Xf Consider the function $g$ of 134I/226Cc.   Show
that $g^{-1}:[0,1]\to[0,1]$ is differentiable almost everywhere in
$[0,1]$, and find $\mu\{x:(g^{-1})'(x)\le a\}$ for each $a\in\Bbb R$.
%226C

\leader{226Y}{Further exercises (a)}
%\spheader 226Ya
Show that a set $I$ is countable iff there is a summable
family $\familyiI{a_i}$ of non-zero real numbers.
%226A

\spheader 226Yb
Explain modifications which might be appropriate in the description
of the Lebesgue decomposition of a function of bounded variation
if we wish to consider functions on open or
half-open intervals, including unbounded intervals.
%226C

\spheader 226Yc Suppose that $F:[a,b]\to\Bbb R$ is a function of bounded
variation, and set $h(y)=\#(F^{-1}[\{y\}])$ for $y\in\Bbb R$.   Show
that $\int h=\Var_{[a,b]}(F_c)$, where $F_c$ is the `continuous part'
of $F$ as defined in 226Ca/226Cd.
%226C

\spheader 226Yd Suppose that $a<b$ in $\Bbb R$, and that
$F:[a,b]\to\Bbb R$ is a function of bounded variation;  let $F_p$ be its saltus part.   Show that
$|F(b)-F(a)|\le\mu F[\,[a,b]\,]+\Var_{[a,b]}F_p$, where $\mu$ is Lebesgue measure on $\Bbb R$.
%226C
}%end of exercises


\endnotes{
\Notesheader{226} In 232I and 232Yb below I will revisit these ideas,
linking them to a decomposition of the Lebesgue-Stieltjes measure
corresponding to a non-decreasing real function, and thence to more
general measures.   All this work is peripheral to the main concerns of
this volume, but I think it is illuminating, and certainly it is part of
the basic knowledge assumed of anyone working in real analysis.
}%end of notes

\frnewpage

