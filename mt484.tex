\frfilename{mt484.tex}
\versiondate{21.1.10}
\copyrightdate{2001}

\def\varinnerprod#1#2{#1\dotproduct#2}

\def\chaptername{Gauge integrals}
\def\sectionname{The Pfeffer integral}

\newsection{484}

I give brief notes on what seems at present to be the most
interesting of the multi-dimensional versions of the Henstock integral,
leading to Pfeffer's Divergence Theorem (484N).

\leader{484A}{Notation}\cmmnt{ This section will depend heavily on
Chapter 47, and will use much of the same notation.}   $r\ge 2$ will be
a fixed integer, and $\mu$ will be Lebesgue measure on
$\BbbR^r$\cmmnt{, while
$\mu_{r-1}$ is Lebesgue measure on $\BbbR^{r-1}$}.   As in
\S\S473-475, let
$\nu$ be `normalized' $(r-1)$-dimensional Hausdorff measure on
$\BbbR^r$\cmmnt{, as
described in \S265;  that is, $\nu=2^{-r+1}\beta_{r-1}\mu_{H,r-1}$,
where $\mu_{H,r-1}$ is
$(r-1)$-dimensional Hausdorff measure on $\BbbR^r$ as described in
\S264}, and

$$\eqalign{\beta_{r-1}
&=\Bover{2^{2k}k!\pi^{k-1}}{(2k)!}\text{ if }r=2k\text{ is even},\cr
&=\Bover{\pi^k}{k!}\text{ if }r=2k+1\text{ is odd}\cr}$$

\noindent\cmmnt{is} the Lebesgue measure of a ball of radius $1$ in
$\BbbR^{r-1}$\cmmnt{ (264I)}.
For this section only, let us say that a subset of $\BbbR^r$ is {\bf
thin} if it is of the form $\bigcup_{n\in\Bbb N}A_n$ where $\nu^*A_n$ is
finite for every $n$.
\cmmnt{Note that every thin set is $\mu$-negligible (471L).
For $A\subseteq\BbbR^r$, write $\partial A$ for
its ordinary topological boundary.   If $x\in\BbbR^r$ and $\epsilon>0$,
$B(x,\epsilon)$ will be the closed ball $\{y:\|y-x\|\le\epsilon\}$.
}%end of comment

I will use the term {\bf dyadic cube} for sets of the
form $\prod_{i<r}\coint{2^{-m}k_i,2^{-m}(k_i+1)}$ where $m$,
$k_0,\ldots,k_{r-1}\in\Bbb Z$;  write $\Cal D$ for the set of dyadic
cubes in $\BbbR^r$.   Note that if\cmmnt{ $D$, $D'\in\Cal D$, either
$D\subseteq D'$ or $D'\subseteq D$ or $D\cap D'=\emptyset$;
so if} $\Cal D_0\subseteq\Cal D$, the
maximal members of $\Cal D_0$ are disjoint.

It will be helpful to have an abbreviation for the following expression:
set

\Centerline{$\alpha^*
=\min(\Bover1{r^{r/2}},\Bover{2^{r-2}}{r\beta_r^{(r-1)/r}})$.}

\noindent\cmmnt{(As will become apparent, the
actual value of this
constant is of no importance;  but the strict logic of the arguments
below depends on $\alpha^*$ being small enough.)}

\cmmnt{As in \S475, I write $\intstar A$, $\clstar A$ and
$\partstar A$ for the
essential interior, essential closure and essential boundary of a set
$A\subseteq\BbbR^r$ (475B).   Recall that a set $A\subseteq\BbbR^r$ has
finite perimeter
in the sense of 474D iff $\nu(\partstar A)$ is finite, and then

\Centerline{$\nu(\partstar A)=\lambda^{\partial}_A(\BbbR^r)=\per A$}

\noindent is the perimeter of $A$ (475M);  we shall also need to
remember that $A$ is necessarily Lebesgue measurable.}%end of comment

$\Cal C$ will be the family of subsets of $\BbbR^r$ with locally finite
perimeter, and
$\Cal V$ the family of bounded sets in $\Cal C$\cmmnt{, that is, the
family of bounded sets with finite perimeter}.
%`Cacciopoli sets' in Pfeffer, Tamanini & Giacomelli
\cmmnt{Note that $\Cal C$ is an algebra of subsets of $\BbbR^r$
(475Ma), and that $\Cal V$ is an ideal in $\Cal C$.
}

\leader{484B}{Theorem}\cmmnt{ ({\smc Tamanini \& Giacomelli 89})}
%Theorem 2.3
Let $E\subseteq\BbbR^r$ be a Lebesgue measurable set of
finite measure and perimeter, and $\epsilon>0$.   Then there is a
Lebesgue measurable set $G\subseteq E$ such that $\per G\le\per E$,
$\mu(E\setminus G)\le\epsilon$ and $\clstar G=\overline{G}$.

\proof{({\smc Pfeffer 91b}) {\bf (a)} Set
$\alpha=\bover1{\epsilon}\per E$.   For measurable sets
$G\subseteq E$ set $q(G)=\per G-\alpha\mu G$.   Then there is a
self-supporting measurable set $G\subseteq E$ such that $q(G)\le q(G')$
whenever $G'\subseteq E$ is measurable.

\Prf\ Write $\Sigma$ for the family of Lebesgue measurable subsets of
$\BbbR^r$;  give $\Sigma$ the topology of convergence in measure defined
by the pseudometrics $\rho_H(G,G')=\mu((G\symmdiff G')\cap H)$ for
measurable sets $H$ of finite measure\cmmnt{ (cf.\ 474T)}.
Extend $q$ to $\Sigma$ by setting
$q(G)=\per(E\cap G)-\alpha\mu(E\cap G)$ for every $G\in\Sigma$.
Because $\per:\Sigma\to[0,\infty]$ is lower semi-continuous for the
topology of convergence in measure (474Ta), and $G\mapsto E\cap G$,
$G\mapsto\mu(E\cap G)$ are continuous, $q:\Sigma\to\coint{0,\infty}$ is
lower semi-continuous (4A2Bd).   Next,
$\Cal K=\{G:G\in\Sigma,\,\per G\le\per E\}$ is compact (474Tb), while
$\Cal L=\{G:\mu(G\setminus E)=0\}$ is closed, so there is a
$G_0\in\Cal L\cap\Cal K$ such that
$q(G_0)=\inf_{G\in\Cal L\cap\Cal K}q(G)$ (4A2Gl).   Since
$G_0\in\Cal L$,
$\per(G_0\cap E)=\per(G_0)$ and $\mu(G_0\cap E)=\mu G_0$, so we may
suppose that $G_0\subseteq E$.   Moreover, there is a self-supporting
set $G\subseteq G_0$ such that $G_0\setminus G$ is negligible (414F),
and we still have $q(G)=q(G_0)$.   Of course $q(G)\le q(E)$, just
because $E\in\Cal L\cap\Cal K$.

\Quer\ If there is a measurable set $G'\subseteq E$ such that
$q(G')<q(G)$, then

\Centerline{$\per G'=q(G')+\alpha\mu G'\le q(E)+\alpha\mu E=\per E$,}

\noindent so $G'\in\Cal K$;  but this means that $G'\in\Cal L\cap\Cal K$
and $q(G)=q(G_0)\le q(G')$.\ \BanG\  So $G$ has the required
properties.\ \Qed

\medskip

{\bf (b)} Since $q(G)\le q(E)$, we must have

\Centerline{$\per G+\alpha\mu(E\setminus G)
=q(G)+\alpha\mu E\le q(E)+\alpha\mu E=\per E=\alpha\epsilon$.}

\noindent So $\mu(E\setminus G)\le\epsilon$.

\medskip

{\bf (c)} Next, $\overline{G}\subseteq\clstar G$.   \Prf\ Let
$x\in\overline{G}$.   For every $t>0$, set
$U_t=\{y:\|y-x\|<t\}$;  then

$$\eqalignno{\per(G\cap U_t)+\per(G\setminus U_t)
&=\nu(\partstar(G\cap U_t))+\nu(\partstar(G\setminus U_t))\cr
&\le\nu(\partstar G\cap U_t)+\nu(\clstar G\cap\partial U_t)\cr
&\qquad\qquad\qquad
      +\nu(\partstar G\setminus U_t)+\nu(\clstar G\cap\partial U_t)\cr
\displaycause{475Cf, because
$\partial(\BbbR^r\setminus U_t)=\partial U_t)$}
&=\nu(\partstar G)+2\nu(\clstar G\cap\partial U_t)
=\per G+2\nu(G\cap\partial U_t)\cr}$$

\noindent for almost every $t>0$, because

\Centerline{$\int_0^{\infty}
  \nu((G\symmdiff\clstar G)\cap\partial U_t)dt
=\mu(G\symmdiff\clstar G)=0$}

\noindent (265G).   So, for almost every $t$,

$$\eqalignno{\mu(G\cap U_t)^{(r-1)/r}
&\le\per(G\cap U_t)\cr
\displaycause{474La}
&\le\per G+2\nu(G\cap\partial U_t)-\per(G\setminus U_t)\cr
&=q(G)+\alpha\mu(G\cap U_t)+2\nu(G\cap\partial U_t)-q(G\setminus U_t)\cr
&\le\alpha\mu(G\cap U_t)+2\nu(G\cap\partial U_t)\cr}$$

\noindent because $q(G)$ is minimal.

For $t>0$, set

\Centerline{$g(t)=\mu(G\cap U_t)=\int_0^t\nu(G\cap\partial U_s)ds$,}

\noindent so that

\Centerline{$g'(t)=\nu(G\cap\partial U_t)
\ge\Bover12(g(t)^{(r-1)/r}-\alpha g(t))$}

\noindent for almost every $t$.   Because $G$ is self-supporting and
$U_t$ is open and $G\cap U_t\ne\emptyset$, $g(t)>0$ for every $t>0$;
and $\lim_{t\downarrow 0}g(t)=0$.

Set

\Centerline{$h(t)=\Bover{d}{dt}g(t)^{1/r}
=\Bover{g'(t)}{rg(t)^{(r-1)/r}}
\ge\Bover1{2r}(1-\alpha g(t)^{1/r})$}

\noindent for almost every $t$.   Then

\Centerline{$\limsup_{t\downarrow 0}\Bover{g(t)^{1/r}}t
=\limsup_{t\downarrow 0}\Bover1t\int_0^th\ge\Bover1{2r}$,}

\noindent and

\Centerline{$\limsup_{t\downarrow 0}
\Bover{\mu(G\cap B(x,t)))}{\mu B(x,t))}
=\limsup_{t\downarrow 0}\Bover{g(t)}{\beta_rt^r}>0$.}

Thus $x\in\clstar G$.   As $x$ is arbitrary,
$\overline{G}\subseteq\clstar G$.\ \Qed

Since we certainly have
$\clstar G\subseteq\overline{G}$, this $G$ serves.
}%end of proof of 484B

\leader{484C}{Lemma} Let $E\in\Cal V$ and $l\in\Bbb N$ be such that
$\max(\per E,\diam E)\le l$.   Then $E$ is expressible as
$\bigcup_{i<n}E_i$ where $\ofamily{i}{n}{E_i}$ is disjoint,
$\per E_i\le 1$ for each $i<n$ and $n$ is at most
$2^r(l+1)^r(4r(2l^2+1))^{r/(r-1)}+2^{r+1}l^2$.

\proof{ For $D\in\Cal D$, write $\Cal D_D$ for
$\{D':D'\in\Cal D,\,D'\subseteq D,\,\diam D'=\bover12\diam D\}$, the
family of the $2^r$ dyadic subcubes of $D$ at the next level down.

\medskip

{\bf (a)} If $l\le 1$ the result is trivial, so let us suppose that
$l\ge 2$.   Let $m\in\Bbb N$ be minimal subject to
$4r(2l^2+1)\le 2^{m(r-1)}$, so that
$2^m\le 2(4r(2l^2+1))^{1/(r-1)}$.   Then we can cover $E$ by a family
$\Cal L_0$ of dyadic cubes of side $2^{-m}$ with

\Centerline{$\#(\Cal L_0)\le (2^ml+1)^r\le 2^{mr}(l+1)^r
\le 2^r(l+1)^r(4r(2l^2+1))^{r/(r-1)}$.}

\medskip

{\bf (b)} Let $\Cal L_1$ be the set of those $D\in\Cal D$ such that
$\lfloor 2\nu(D'\cap\partstar E)\rfloor
<\lfloor 2\nu(D\cap\partstar E)\rfloor$ for every $D'\in\Cal D_D$.
Then $\#(\Cal L_1)\le 2l^2$.   \Prf\
For $k\ge 1$, set

\Centerline{$\Cal L_1^{(k)}=\{F:D\in\Cal L_1$,
$\lfloor 2\nu(D\cap\partstar E)\rfloor=k\}$.}

\noindent If $D$, $D'\in\Cal L_1^{(k)}$ are distinct, neither can be
included in the other, so they are disjoint.   Accordingly
$k\#(\Cal L_1^{(k)})\le 2\nu(\partstar E)\le 2l$ and
$\#(\Cal L_1^{(k)})\le 2l$.   Since
$\Cal L_1=\bigcup_{1\le k\le l}\Cal L_1^{(k)}$,
$\#(\Cal L_1)\le 2l^2$.\ \Qed

\medskip

{\bf (c)} For $D\in\Cal D$, set
$\tilde D=D\setminus\bigcup\{D':D'\in\Cal L_1$, $D'\subseteq D\}$.
Then $\nu(\tilde D\cap\partstar E)\le\bover12$.   \Prf\Quer\ Otherwise,
set $j=\lfloor 2\nu(D\cap\partstar E)\rfloor\ge 1$, and
choose $\sequence{i}{D_i}$ inductively, as follows.   $D_0=D$.   Given that
$D_i\in\Cal D$, $D_i\subseteq D$ and
$\lfloor 2\nu(D_i\cap\partstar E)\rfloor=j$,
$\nu((D\setminus D_i)\cap\partstar E)<\bover12$ and
$D_i\cap\tilde D$ is non-empty, so $D_i\notin\Cal L_1$ and there must be a
$D_{i+1}\in\Cal D_{D_i}$ such that
$\lfloor 2\nu(D_{i+1}\cap\partstar E)\rfloor=j$.   Continue.   This gives
us a strictly decreasing sequence $\sequence{i}{D_i}$ in $\Cal D$ such that
$\nu(D_i\cap\partstar E)\ge\bover{j}2$ for every $i$.   But (because
$\per E$ is finite) this means that, writing $x$ for the
unique member of $\bigcap_{i\in\Bbb N}\overline{D}_i$,
$\nu\{x\}\ge\bover{j}2$, which is absurd.\ \Bang\Qed

\medskip

{\bf (d)} Set


\Centerline{$\Cal L'_1
=\{D:D\in\Cal L_1$ is included in some member of $\Cal L_0\}$,}

\Centerline{$\Cal L_2=\Cal L_0\cup\bigcup\{\Cal D_D:D\in\Cal L'_1\}$,
\quad$\Cal K=\{\tilde D:D\in\Cal L_2\}$.}

\medskip

\quad{\bf (i)}

$$\eqalign{\#(\Cal K)
&\le\#(\Cal L_2)
\le\#(\Cal L_0)+2^r\#(\Cal L_1)\cr
&\le 2^r(l+1)^r(4r(2l^2+1))^{r/(r-1)}+2^{r+1}l^2.\cr}$$

\medskip

\quad{\bf (ii)} $\bigcup\Cal K\supseteq E$.   \Prf\ If $x\in E$, there
is a smallest member $D$ of $\Cal L_2$ containing it, because
certainly $x\in\bigcup\Cal L_0$.   But now $x$ cannot belong to any
member of $\Cal L_1$ included in $D$, so $x\in\tilde D$.\ \Qed

\medskip

\quad{\bf (iii)} $\Cal K$ is disjoint.   \Prf\ If $D_1$, $D_2\in\Cal L_2$
are disjoint, then of course $\tilde D_1\cap\tilde D_2=\emptyset$.   If
$D_1\subset D_2$, then $D_1\notin\Cal L_0$, so there is a $D\in\Cal L'_1$
such that $D_1\in\Cal D_D$;  in this case $D\subseteq D_2$ so
$\tilde D_2\subseteq D_2\setminus D$ is disjoint from $D_1$.\ \Qed

\medskip

\quad{\bf (iv)} $\per(D\cap E)\le 1$ for every $D\in\Cal K$.
\Prf\ Take $D_0\in\Cal L_2$ such that $D=\tilde D_0$;  then

\Centerline{$\nu(\partial D)
\le\nu(\partial D_0)+\sum_{D'\in\Cal L'_1}\nu(\partial D')
\le 2r(2l^2+1)2^{-m(r-1)}
\le\Bover12$}

\noindent by the choice of $m$.   So

\Centerline{$\per(D\cap E)\le\nu(\partial D)+\nu(D\cap\partstar E)
\le\Bover12+\Bover12=1$}

\noindent by 475Cf.\ \Qed

\medskip

{\bf (e)} So if we take $\ofamily{i}{n}{E_i}$ to be an enumeration of
$\{E\cap D:D\in\Cal K\}$, we shall have the required result.
}%end of proof of 484C

\leader{484D}{Definitions}\cmmnt{ The gauge integrals of this section
will be based on the following residual families.}   Let $\Eta$ be the
family of strictly positive
sequences $\eta=\sequence{i}{\eta(i)}$ in $\Bbb R$.   For $\eta\in\Eta$,
write $\Cal M_{\eta}$ for the set of disjoint sequences
$\sequence{i}{E_i}$ of measurable subsets of $\BbbR^r$ such that
$\mu E_i\le\eta(i)$ and $\per E_i\le 1$ for every
$i\in\Bbb N$, and $E_i$ is empty for all but finitely many $i$.
For $\eta\in\Eta$ and $V\in\Cal V$ set

\Centerline{$\Cal R_{\eta}
=\{\bigcup_{i\in\Bbb N}E_i:\sequence{i}{E_i}\in\Cal M_{\eta}\}
\subseteq\Cal C$,
\quad$\Cal R^{(V)}_{\eta}
=\{R:R\subseteq\BbbR^r,\,R\cap V\in\Cal R_{\eta}\}$;}

\noindent
finally, set $\frak R=\{\Cal R^{(V)}_{\eta}:V\in\Cal V,\,\eta\in\Eta\}$.

\leader{484E}{Lemma} (a)(i) For every $\Cal R\in\frak R$, there is an
$\eta\in\Eta$ such that $\Cal R_{\eta}\subseteq\Cal R$.

\quad(ii) If $\Cal R\in\frak R$ and $C\in\Cal C$, there is an
$\Cal R'\in\frak R$ such that $C\cap R\in\Cal R$ whenever $R\in\Cal R'$.

(b)(i) If $\eta\in\Eta$ and $\gamma\ge 0$, there is an $\epsilon>0$ such
that $R\in\Cal R_{\eta}$ whenever $\mu R\le\epsilon$, $\diam R\le\gamma$
and $\per R\le\gamma$.

\quad(ii) If $\Cal R\in\frak R$ and $\gamma\ge 0$, there is an
$\epsilon>0$ such that $R\in\Cal R$ whenever $\mu R\le\epsilon$
and $\per R\le\gamma$.

(c) If $\Cal R\in\frak R$ there is an $\Cal R'\in\frak R$ such that
$R\cup R'\in\Cal R$ whenever $R$, $R'\in\Cal R'$ and
$R\cap R'=\emptyset$.

(d)(i) If $\eta\in\Eta$ and $A\subseteq\BbbR^r$ is a thin set, then
there is a set $\Cal D_0\subseteq\Cal D$ such that every point of $A$
belongs to the interior of $\bigcup\Cal D_1$ for some finite
$\Cal D_1\subseteq\Cal D_0$, and $\bigcup\Cal D_1\in\Cal R_{\eta}$ for
every finite set $\Cal D_1\subseteq\Cal D_0$.

\quad(ii) If $\Cal R\in\frak R$ and $A\subseteq\BbbR^r$ is a thin
set, then there is a set $\Cal D_0\subseteq\Cal D$ such that every point
of $A$ belongs to the interior of $\bigcup\Cal D_1$ for some finite set
$\Cal D_1\subseteq\Cal D_0$, and $\bigcup\Cal D_1\in\Cal R$ for every
finite set $\Cal D_1\subseteq\Cal D_0$.

\proof{{\bf (a)(i)} Express $\Cal R$ as $\Cal R^{(V)}_{\eta'}$ where
$\eta'\in\Eta$ and $V\in\Cal V$.   Let $l\in\Bbb N$ be such that
$\max(\diam V,1+\per V)\le l$, and take
$n\ge 2^r(l+1)^r(4r(2l^2+1))^{r/(r-1)}+2^{r+1}l^2$.   Set
$\eta(i)=\min\{\eta'(j):ni\le j<n(i+1)\}$ for every $i\in\Bbb N$, so
that $\eta\in\Eta$.   If $R\in\Cal R_{\eta}$, express it as
$\bigcup_{i\in\Bbb N}E_i$ where $\sequence{i}{E_i}\in\Cal M_{\eta}$.
For each $i$, $\max(\diam(E_i\cap V),\per(E_i\cap V))\le l$, so
by 484C we can express $E_i\cap V$ as $\bigcup_{ni\le j<n(i+1)}E'_j$,
where $\langle E'_j\rangle_{ni\le j<n(i+1)}$ is disjoint and
$\per E'_j\le 1$ for each $j$.   Now

\Centerline{$\mu E'_j\le\mu E_i\le\eta(i)\le\eta'(j)$}

\noindent for $ni\le j<n(i+1)$.   Also $\{j:E'_j\ne\emptyset\}$ is
finite because $\{j:E_j=\emptyset\}$ is finite.   So
$\sequence{j}{E'_j}\in\Cal M_{\eta'}$ and
$R\cap V=\bigcup_{j\in\Bbb N}E'_j$ belongs to $\Cal R_{\eta'}$, that is,
$R\in\Cal R$.   As $R$ is arbitrary, $\Cal R_{\eta}\subseteq\Cal R$.

\medskip

\quad{\bf (ii)} Express $\Cal R$ as $\Cal R^{(V)}_{\eta}$, where
$V\in\Cal V$ and $\eta\in\Eta$.   By (i), there is an $\eta'\in\Eta$
such that $\Cal R_{\eta'}\subseteq\Cal R^{(C\cap V)}_{\eta}$.   Set
$\Cal R'=\Cal R^{(V)}_{\eta'}\in\frak R$.   If $R\in\Cal R'$, then
$R\cap V\in\Cal R_{\eta'}$, so
$C\cap R\cap V\in\Cal R_{\eta}$ and $C\cap R\in\Cal R$.

\medskip

{\bf (b)(i)} Take $l\ge\gamma$ and
$n\ge 2^r(l+1)^r(4r(2l^2+1))^{r/(r-1)}+2^{r+1}l^2$, and set
$\epsilon=\min_{i<n}\eta(i)$.   If $\mu R\le\epsilon$, $\diam R\le l$
and $\per R\le l$, then $R$ is expressible as
$\bigcup_{i<n}E_i$ where $\ofamily{i}{n}{E_i}$ is disjoint and
$\per E_i\le 1$ for each $i<n$.   Since $\mu E_i\le\mu R\le\eta(i)$ for
each $i$, $R\in\Cal R_{\eta}$.

\quad{\bf (ii)} Express $\Cal R$ as $\Cal R^{(V)}_{\eta}$.   By
(i), there is an $\epsilon>0$ such that $R\in\Cal R_{\eta}$
whenever $\diam R\le\diam V$, $\per R\le\gamma+\per V$ and
$\mu R\le\epsilon$;  and this $\epsilon$ serves.

\medskip

{\bf (c)} Express $\Cal R$ as $\Cal R^{(V)}_{\eta}$.
Set $\eta'(i)=\min(\eta(2i),\eta(2i+1))$
for every $i$;  then $\eta'\in\Eta$ and if $\sequence{i}{E_i}$,
$\sequence{i}{E'_i}$ belong to $\Cal M_{\eta'}$ and have
disjoint unions, then
$(E_0,E'_0,E_1,E'_1,\ldots)\in\Cal M_{\eta}$;  this is
enough to show that $R\cup R'\in\Cal R_{\eta}$ whenever $R$,
$R'\in\Cal R_{\eta'}$ are disjoint, so that $R\cup R'\in\Cal R$ whenever
$R$, $R'\in\Cal R^{(V)}_{\eta'}$ are disjoint.

\medskip

{\bf (d)(i)}\grheada\ We can express $A$ as $\bigcup_{i\in\Bbb N}A_i$
where $\mu^*_{H,r-1}A_i<2^{-r}/2r$ for every $i$.   \Prf\ Because $A$
is thin, it is the union of a sequence of sets of finite outer measure
for $\nu$, and therefore for $\mu_{H,r-1}$.   On each of these the
subspace measure is atomless (471E, 471Dg), so that the set can be
dissected into
finitely many sets of measure less than $2^{-r}/2r$ (215D).\ \Qed

\medskip

\qquad\grheadb\ For each $i\in\Bbb N$,
we can cover $A_i$ by a sequence $\sequence{j}{A_{ij}}$ of sets such
that $\diam A_{ij}<\eta(i)$ for every $j$ and
$\sum_{j=0}^{\infty}(\diam A_{ij})^{r-1}<2^{-r}/2r$;  enlarging the
$A_{ij}$ slightly if need be, we can suppose that they are all open.
Now we can
cover each $A_{ij}$ by $2^r$ cubes $D_{ijk}\in\Cal D$ in such a way
that the side length of each $D_{ijk}$ is at most the diameter of
$A_{ij}$.

Setting $\Cal D_0=\{D_{ijk}:i$, $j\in\Bbb N$, $k<2^r\}$, we see that
every point of $A$ belongs to an open set $A_{ij}$ which is covered by a
finite subset of $\Cal D_0$.   If $\Cal D_1\subseteq\Cal D_0$ is finite,
let $\Cal D'_1$ be the family of maximal elements of $\Cal D_1$, so that
$\Cal D'_1$ is disjoint and $\bigcup\Cal D'_1=\bigcup\Cal D_1$.
Express $\Cal D'_1$ as $\{D_{ijk}:(i,j,k)\in I\}$ where
$I\subseteq\Bbb N\times\Bbb N\times 2^r$ is finite and
$\family{(i,j,k)}{I}{D_{ijk}}$ is disjoint.   Set
$I_i=\{(j,k):(i,j,k)\in I\}$ and $E_i=\bigcup_{(j,k)\in I_i}D_{ijk}$ for
$i\in\Bbb N$.   Then $\sequence{i}{E_i}$ is disjoint, $E_i=\emptyset$
for all but finitely many $i$, and for each $i\in\Bbb N$

$$\eqalign{\mu E_i
&\le\sum_{j=0}^{\infty}\sum_{k=0}^{2^r-1}\mu D_{ijk}
\le\sum_{j=0}^{\infty}2^r(\diam A_{ij})^r\cr
&\le 2^r\sum_{j=0}^{\infty}\eta(i)(\diam A_{ij})^{r-1}
\le\eta(i),\cr
\nu(\partial E_i)
&\le\sum_{(j,k)\in I_i}\nu(\partial D_{ijk})
\le\sum_{j=0}^{\infty}\sum_{k=0}^{2^r-1}\nu(\partial D_{ijk})\cr
&\le\sum_{j=0}^{\infty}2^r\cdot 2r(\diam A_{ij})^{r-1}
\le 1.\cr}$$

\noindent So $\bigcup\Cal D_1=\bigcup_{i\in\Bbb N}E_i$ belongs to
$\Cal R_{\eta}$.

\medskip

\quad{\bf (ii)}  By (a-i), there is an $\eta\in\Eta$ such that
$\Cal R_{\eta}\subseteq\Cal R$.   By (i) here, there is a
$\Cal D_0\subseteq\Cal D$ such that every point of $A$ belongs to the
interior of $\bigcup\Cal D_1$ for some finite
$\Cal D_1\subseteq\Cal D_0$, and
$\bigcup\Cal D_1\in\Cal R_{\eta}\subseteq\Cal R$ for
every finite set $\Cal D_1\subseteq\Cal D_0$.
}%end of proof of 484E

\leader{484F}{A family of tagged-partition structures} For $\alpha>0$,
let $\Cal C_{\alpha}$ be the family of those $C\in\Cal V$ such that
$\mu C\ge\alpha(\diam C)^r$ and $\alpha\per C\le(\diam C)^{r-1}$, and
let $T_{\alpha}$ be the straightforward set of tagged partitions
generated by the set

\Centerline{$\{(x,C):C\in\Cal C_{\alpha},\,x\in\clstar C\}$.}

\noindent Let $\Theta$ be the set of functions
$\theta:\BbbR^r\to\coint{0,\infty}$ such that $\{x:\theta(x)=0\}$ is
thin\cmmnt{ (definition:  484A)},
and set $\Delta=\{\delta_{\theta}:\theta\in\Theta\}$, where
$\delta_{\theta}
=\{(x,A):x\in\BbbR^r,\,\theta(x)>0,\,\|y-x\|<\theta(x)$ for every
$y\in A\}$.

Then whenever $0<\alpha<\alpha^*$, $(\BbbR^r,T_{\alpha},\Delta,\frak R)$
is a tagged-partition structure allowing subdivisions, witnessed by
$\Cal C$.

\proof{{\bf (a)} We had better look again at all the conditions in 481G.

(i) and (vi) really are trivial.
(iv) and (v) are true because $\Cal C$ is actually an algebra of sets
and $\emptyset\in\Cal R$ for every $\Cal R\in\frak R$.

For (ii), we have to observe that the
union of two thin sets is thin, so that $\theta\wedge\theta'\in\Theta$
for all $\theta$, $\theta'\in\Theta$;  since
$\delta_{\theta}\cap\delta_{\theta'}
=\delta_{\theta\wedge\theta'}$, this is all we need.

(iii)($\alpha$) follows from 484Ea:  given $V$, $V'\in\Cal V$ and
$\eta$, $\eta'\in\Eta$, take $\tilde\eta$, $\tilde\eta'\in\Eta$ such
that $\Cal R_{\tilde\eta}\subseteq\Cal R^{(V)}_{\eta}$ and
$\Cal R_{\tilde\eta'}\subseteq\Cal R^{(V')}_{\eta'}$.
Then $V\cup V'\in\Cal V$ and
$\Cal R^{(V)}_{\eta}\cap\Cal R^{(V')}_{\eta'}
\supseteq\Cal R^{(V\cup V')}_{\tilde\eta\wedge\tilde\eta'}$.
($\beta$) is just 484Ec.

\medskip

{\bf (b)} Now let us turn to 481G(vii).

\medskip

\quad{\bf (i)} Fix $C\in\Cal C$, $\delta\in\Delta$ and
$\Cal R\in\frak R$.   Express $\delta$ as $\delta_{\theta}$ where
$\theta\in\Theta$.   Let $\Cal R'\in\frak R$ be such that
$A\cup A'\in\Cal R$ whenever $A$, $A'\in\Cal R'$ are disjoint.   Express
$\Cal R'$ as $\Cal R^{(V)}_{\eta}$ where $V\in\Cal V$ and $\eta\in\Eta$.
By 484E(b-ii), there is an $\epsilon>0$ such that $R\in\Cal R'$ whenever
$\per R\le 2\per(C\cap V)$ and $\mu R\le\epsilon$.

By 484B, there is an $E\subseteq C\cap V$ such that
$\per E\le\per(C\cap V)$,
$\mu((C\cap V)\setminus E)\le\epsilon$ and $\clstar E=\overline{E}$.
In this case,

\Centerline{$\per(C\cap V\setminus E)\le\per(C\cap V)+\per E
\le 2\per(C\cap V)$,}

\noindent so $C\cap V\setminus E\in\Cal R'$, by the choice of
$\epsilon$.   By
484E(a-ii), there is an $\Cal R''\in\frak R$ such that
$E\cap R\in\Cal R'$ whenever $R\in\Cal R''$.

Now consider

\Centerline{$A=\{x:\theta(x)=0\}\cup\partstar E
   \cup\{x:\limsup_{\zeta\downarrow 0}
     \sup_{x\in G,0<\diam G\le\zeta}
     \Bover{\nu^*(G\cap\partstar E)}{(\diam G)^{r-1}}>0\}$.}

Because $\per E<\infty$  and

\Centerline{$\{x:x\in\BbbR^r\setminus\partstar E,\,
  \limsup_{\zeta\downarrow 0}
     \sup_{x\in G,0<\diam G\le\zeta}
     \Bover{\nu^*(G\cap\partstar E)}{(\diam G)^{r-1}}>0\}$}

\noindent is $\nu$-negligible (471Pc), $A$ is thin.   By 484E(d-ii),
there is a set $\Cal D_0\subseteq\Cal D$ such that

\Centerline{$A\subseteq\bigcup\{\interior(\bigcup\Cal D_1):
  \Cal D_1\in[\Cal D_0]^{<\omega}\}$,
\quad $\bigcup\Cal D_1\in\Cal R''$ for every
  $\Cal D_1\in[\Cal D_0]^{<\omega}$.}

\medskip

\quad{\bf (ii)} Write $T'$ for the set of those $\delta$-fine
$\pmb{t}\in T_{\alpha}$ such that every
member of $\pmb{t}$ is of the form $(x,D\cap E)$ for some $D\in\Cal D$.
If $x\in\intstar E\setminus A$, there is an $h(x)>0$
such that $h(x)<\theta(x)$ and $\{(x,D\cap E)\}\in T'$ whenever
$D\in\Cal D$, $x\in\overline{D}$ and $\diam D\le h(x)$.
\Prf\ Let $\epsilon_1>0$ be such that $r^{r/2}\epsilon_1\le\bover12$ and

\Centerline{$\alpha\le\Bover{1-r^{r/2}\epsilon_1}{r^{r/2}}$,
\quad$\Bover1{\alpha}\ge(2r+r^{r/2}\epsilon_1)\cdot\Bover1{2^{r-1}}
  \bigl(\Bover{\beta_r}{1-r^{r/2}\epsilon_1}\bigr)^{(r-1)/r}$;}

\noindent this is where we need to know that $\alpha<\alpha^*$.
Because $x\notin A$, $\theta(x)>0$;  let $h(x)\in\ooint{0,\theta(x)}$ be
such that ($\alpha$)
$\nu^*(D\cap\partstar E)\le\epsilon_1(\diam D)^{r-1}$ whenever
$x\in\overline{D}$ and $\diam D\le h(x)$ ($\beta$)
$\mu(B(x,t)\setminus E)\le\epsilon_1t^r$ whenever
$0\le t\le h(x)$.   Now suppose that $D\in\Cal D$ and
$x\in\overline{D}$ and $\diam D\le h(x)$.   Then, writing $\gamma$ for
the side length of $D$,

$$\eqalign{\mu(D\cap E)
&\ge\mu D-\mu(B(x,\diam D)\setminus E)
\ge\gamma^r-\epsilon_1(\diam D)^r\cr
&=\gamma^r(1-r^{r/2}\epsilon_1)
\ge\alpha\gamma^rr^{r/2}
=\alpha(\diam D)^r
\ge\alpha\diam(D\cap E)^r.\cr}$$

\noindent Using 264H, we see also that

\Centerline{$\diam(D\cap E)
\ge\bigl(\Bover{2^r}{\beta_r}\mu(D\cap E)\bigr)^{1/r}
\ge 2\gamma\bigl(\Bover{1-r^{r/2}\epsilon_1}{\beta_r}\bigr)^{1/r}$.}

\noindent Next,

$$\eqalignno{\per(D\cap E)
&\le\nu(\partial D)+\nu(D\cap\partstar E)\cr
\displaycause{475Cf}
&\le 2r\gamma^{r-1}+\epsilon_1(\diam D)^{r-1}
\le\gamma^{r-1}(2r+r^{r/2}\epsilon_1)\cr
&\le(2r+r^{r/2}\epsilon_1)\cdot\Bover1{2^{r-1}}
  \bigl(\Bover{\beta_r}{1-\epsilon_1r^{r/2}}\bigr)^{(r-1)/r}
  \diam(D\cap E)^{r-1}\cr
&\le\Bover1{\alpha}\diam(D\cap E)^{r-1}.\cr}$$

\noindent So $D\cap E\in\Cal C_{\alpha}$.   Also, for every $s>0$, there
is a $D'\in\Cal D$ such that
$D'\subseteq D$ and $x\in\overline{D'}$ and $\diam D'\le s$.
In this case

\Centerline{$\mu(B(x,\diam D')\setminus E)
\le\epsilon_1(\diam D')^r=\epsilon_1r^{r/2}\mu D'\le\Bover12\mu D'$,}

\noindent so

$$\eqalign{\mu(D\cap E\cap B(x,\diam D'))
&\ge\mu D'-\mu(B(x,\diam D')\setminus E)\cr
&\ge\Bover12\mu D'
=\Bover1{2\beta_rr^{r/2}}\mu B(x,\diam D').\cr}$$

\noindent As $\diam D'$ is arbitrarily small, $x\in\clstar(D\cap E)$ and
$\pmb{t}=\{(x,D\cap E)\}\in T_{\alpha}$.   Finally, since
$\diam D<\theta(x)$, $(x,D\cap E)\in\delta$ and $\pmb{t}\in T'$.\ \Qed

\medskip

\quad{\bf (iii)} Let $\Cal H$ be the set of those $H\subseteq\BbbR^r$
such that
$W_{\pmb{t}}\subseteq E\cap H\subseteq W_{\pmb{t}}\cup\bigcup\Cal D_1$
for some $\pmb{t}\in T'$ and finite
$\Cal D_1\subseteq\Cal D_0$.
Then $H\cup H'\in\Cal H$ whenever $H$, $H'\in\Cal H$ are disjoint.
If $\sequencen{D_n}$ is any strictly decreasing sequence in $\Cal D$,
then some $D_n$ belongs to $\Cal H$.   \Prf\
Let $x$ be the unique point of $\bigcap_{n\in\Bbb N}\overline{D_n}$.

\qquad{\bf case 1} If $x\in A$, then there is a finite subset $\Cal D_1$
of $\Cal D_0$ whose union is a neighbourhood of $x$, and therefore
includes $D_n$ for some $n$;  so $\pmb{t}=\emptyset$ and
$\Cal D_1$ witness that $D_n\in\Cal H$.

\qquad{\bf case 2} If $x\in E\setminus A$, then $x\in\intstar E$, so
$h(x)>0$ and there is some $n\in\Bbb N$ such that
$\diam D_n\le h(x)$.   In this case $\pmb{t}=\{(x,D_n\cap E)\}$ belongs
to $T'$, by the choice of $h(x)$, so that $\pmb{t}$ and $\emptyset$
witness that $D_n\in\Cal H$.

\qquad{\bf case 3} Finally, if $x\notin E\cup A$, then
$x\notin\clstar E$ so $x\notin\overline{E}$ and there is some $n$ such
that $D_n\cap E=\emptyset$, in which case $\pmb{t}=\Cal D_1=\emptyset$
witness that $D_n\in\Cal H$.\ \Qed

\medskip

\quad{\bf (iv)} In fact $\BbbR^r\in\Cal H$.   \Prf\Quer\ Otherwise,
because $E\subseteq V$ is bounded, it can be covered by a finite
disjoint family in $\Cal D$, and there must be some
$D_0\in\Cal D\setminus\Cal H$.   Now we can find
$\langle D_n\rangle_{n\ge 1}$ in $\Cal D\setminus\Cal H$ such that
$D_n\subseteq D_{n-1}$ and $\diam D_n=\bover12\diam D_{n-1}$ for every
$n$.   But this contradicts (iii).\Bang\Qed

\medskip

\quad{\bf (v)} We therefore have a $\pmb{t}\in T'$ and a
finite set $\Cal D_1\subseteq\Cal D_0$ such that
$W_{\pmb{t}}\subseteq E\subseteq W_{\pmb{t}}\cup\bigcup\Cal D_1$.
Now we can find $\pmb{t}'\subseteq\pmb{t}$ and
$\Cal D'_1\subseteq\Cal D_1$ such that
$W_{\pmb{t}'}\cap\bigcup\Cal D'_1=\emptyset$ and
$E\subseteq W_{\pmb{t}'}\cup\bigcup\Cal D'_1$.
\Prf\ Express $\pmb{t}$ as $\familyiI{(x_i,D_i\cap E)}$ where
$D_i\in\Cal D$ for each $i$.   Then
$E\subseteq\bigcup_{i\in I}D_i\cup\bigcup\Cal D_1$.   Set
$\Cal D_1'=\{D:D\in\Cal D_1$, $D\not\subseteq D_i$ for every $i\in I\}$,
$J=\{i:i\in I$, $D_i\not\subseteq D$ for every $D\in\Cal D'_1\}$,
$\pmb{t}'=\{(x_i,D_i\cap E):i\in J\}$.\ \Qed

By the choice of $\Cal D_0$, $\bigcup\Cal D'_1\in\Cal R''$;  by the
choice of $\Cal R''$,
$E\setminus W_{\pmb{t}'}=E\cap\bigcup\Cal D'_1$ belongs to $\Cal R'$.
But we know also that $C\cap V\setminus E\in\Cal R'$, that is,
$C\setminus E\in\Cal R'$, because $\Cal R'=\Cal R^{(V)}_{\eta}$.   By
the choice of $\Cal R'$, $C\setminus W_{\pmb{t}'}\in\Cal R$.   And
$\pmb{t}'\in T'$ is a $\delta$-fine member of $T_{\alpha}$.   As $C$,
$\delta$ and $\Cal R$ are
arbitrary, 481G(vii) is satisfied, and the proof is complete.
}%end of proof of 484F

\leader{484G}{The Pfeffer integral (a)} For
$\alpha\in\ooint{0,\alpha^*}$, write $I_{\alpha}$ for the linear
functional defined by setting

\Centerline{$I_{\alpha}(f)
=\lim_{\pmb{t}\to\Cal F(T_{\alpha},\Delta,\frak R)}
  S_{\pmb{t}}(f,\mu)$}

\noindent whenever $f:\BbbR^r\to\Bbb R$ is such that the limit is
defined.   (See 481F for the notation $\Cal F(T_{\alpha},\Delta,\frak R)$.)
Then if $0<\beta\le\alpha<\alpha^*$ and $I_{\beta}(f)$ is defined, so
is $I_{\alpha}(f)$, and the
two are equal.   \prooflet{\Prf\ All we have to observe is that
$\Cal C_{\alpha}\subseteq\Cal C_{\beta}$ so that
$T_{\alpha}\subseteq T_{\beta}$, while
$\Cal F(T_{\alpha},\Delta,\frak R)$ is just
$\{A\cap T_{\alpha}:A\in\Cal F(T_{\beta},\Delta,\frak R)\}$.\ \Qed}

\spheader 484Gb Let $f:\BbbR^r\to\Bbb R$ be a function.   I will say
that it
is {\bf Pfeffer integrable}, with {\bf Pfeffer integral} $\Pfint f$, if

\Centerline{$\biggerPfint f
=\lim_{\alpha\downarrow 0}I_{\alpha}(f)$}

\noindent is defined;  that is to say, if $I_{\alpha}(f)$ is
defined whenever $0<\alpha<\alpha^*$.

\leader{484H}{}\cmmnt{ The first step is to work through the results
of \S482 to see which ideas apply directly to the limit integral
$\Pfint\,$.

\medskip

\noindent}{\bf Proposition} (a) The domain of $\Pfint\,$ is a linear
space of functions, and $\Pfint\,$ is a positive linear functional.

(b) If $f$, $g:\BbbR^r\to\Bbb R$ are such that $|f|\le g$ and
$\Pfint g=0$, then $\Pfint f$ is defined and equal to $0$.

(c) If $f:\BbbR^r\to\Bbb R$ is Pfeffer
integrable, then there is a unique additive functional
$F:\Cal C\to\Bbb R$ such that whenever $\epsilon>0$ and
$0<\alpha<\alpha^*$ there are
$\delta\in\Delta$ and $\Cal R\in\frak R$ such that

\Centerline{$\sum_{(x,C)\in\pmb{t}}|F(C)-f(x)\mu C|\le\epsilon$ for
every $\delta$-fine $\pmb{t}\in T_{\alpha}$,}

\Centerline{$|F(E)|\le\epsilon$ whenever $E\in\Cal C\cap\Cal R$.}

\noindent Moreover, $F(\BbbR^r)=\Pfint f$.

(d) Every Pfeffer integrable function is Lebesgue measurable.

(e) Every Lebesgue integrable function is Pfeffer integrable, with the
same integral.

(f) A non-negative function is Pfeffer integrable iff it is Lebesgue
integrable.

\proof{{\bf (a)-(b)} Immediate from 481C.

\medskip

{\bf (c)} For each $\alpha\in\ooint{0,\alpha^*}$ let $F_{\alpha}$ be the
Saks-Henstock indefinite integral corresponding to the the structure
$(\BbbR^r,T_{\alpha},\Delta,\penalty-100\frak R,\mu)$.
Then all the $F_{\alpha}$
coincide.   \Prf\ Suppose that $0<\beta\le\alpha<\alpha^*$.
Then, for any $\epsilon>0$, there are $\delta\in\Delta$,
$\Cal R\in\frak R$ such that

\Centerline{$\sum_{(x,C)\in\pmb{t}}|F_{\beta}(C)-f(x)\mu C|
\le\epsilon$ for every $\delta$-fine $\pmb{t}\in T_{\beta}$,}

\Centerline{$|F_{\beta}(E)|\le\epsilon$
whenever $E\in\Cal R$.}

\noindent Since $T_{\alpha}\subseteq T_{\beta}$, this
means that

\Centerline{$\sum_{(x,C)\in\pmb{t}}|F_{\beta}(C)-f(x)\mu C|
\le\epsilon$
for every $\delta$-fine $\pmb{t}\in T_{\alpha}$.}

\noindent And this works for any $\epsilon>0$.   By the uniqueness
assertion in 482B, $F_{\beta}$ must be exactly the same as
$F_{\alpha}$.\ \Qed

So we have a single functional $F$;  and 482B also tells us that

\Centerline{$F(\BbbR^r)=I_{\alpha}(f)=\biggerPfint f$}

\noindent for every $\alpha$.

\medskip

{\bf (d)} In fact if there is any $\alpha$ such that
$I_{\alpha}(f)$ is defined, $f$ must be Lebesgue measurable.
\Prf\ We have only to check that the conditions of 482E are satisfied by
$\mu$, $\Cal C_{\alpha}$, $\{(x,C):C\in\Cal C_{\alpha}$, $x\in\clstar C\}$,
$T_{\alpha}$, $\Delta$ and $\frak R$.    (i), (iii) and (v) are built
into the definitions above, and (iv) and (vii) are covered by 484F.
482E(ii) is true because $C\setminus\clstar C$ is negligible
for every $C\in\Cal C$ (475Cg).

As for 482E(vi), this is true because if
$\mu E<\infty$ and
$\epsilon>0$, there are $n\in\Bbb N$ and $\eta\in\Eta$ such that
$\mu(E\setminus B(\tbf{0},n))\le\bover12\epsilon$ and
$\sum_{i=0}^{\infty}\eta(i)\le\bover12\epsilon$, so that

\Centerline{$\mu(E\cap R)
\le\mu(E\setminus B(\tbf{0},n))+\mu(R\cap B(\tbf{0},n))
\le\epsilon$}

\noindent for every $R\in\Cal R^{(B(\tbf{0},n))}_{\eta}$.\ \Qed

\medskip

{\bf (e)} This time, we have to check that the conditions of 482F are
satisfied by $T_{\alpha}$, $\Delta$ and $\frak R$ whenever
$0<\alpha<\alpha^*$.   \Prf\ Of course $\mu$
is inner regular with respect to the closed sets and outer regular with
respect to the open sets (134F).   Condition 482F(v) just repeats
482E(v), verified in (d) above.\ \Qed

\medskip

{\bf (f)} If $f\ge 0$ is integrable in the ordinary sense, then it is
Pfeffer integrable, by (e).   If it is Pfeffer integrable, then it is
measurable;  but also $\int g\,d\mu=\Pfint g\le\Pfint f$ for every
simple function $g\le f$, so $f$ is integrable (213B).
}%end of proof of 484H

\leader{484I}{Definition} If $f:\BbbR^r\to\Bbb R$ is Pfeffer
integrable, I will call the function $F:\Cal C\to\Bbb R$ defined in
484Hc the {\bf Saks-Henstock indefinite integral} of $f$.

\leader{484J}{}\cmmnt{ In fact 484Hc characterizes the Pfeffer
integral, just as the Saks-Henstock lemma can be used to define general
gauge integrals based on tagged-partition structures allowing
subdivisions.

\medskip

\noindent}{\bf Proposition} Suppose that $f:\BbbR^r\to\Bbb R$ and
$F:\Cal C\to\Bbb R$ are such that

\inset{(i) $F$ is additive,

(ii) whenever $0<\alpha<\alpha^*$ and $\epsilon>0$ there is a
$\delta\in\Delta$ such that
$\sum_{(x,C)\in\pmb{t}}|F(C)-f(x)\mu C|\le\epsilon$ for every
$\delta$-fine $\pmb{t}\in T_{\alpha}$,

(iii) for every $\epsilon>0$ there is an $\Cal R\in\frak R$ such that
$|F(E)|\le\epsilon$ for every $E\in\Cal C\cap\Cal R$.}

\noindent Then $f$ is Pfeffer integrable and $F$ is the Saks-Henstock
indefinite integral of $f$.

\proof{ By 482D, the gauge integral $I_{\alpha}(f)$ is defined and
equal to $F(\BbbR^r)$ for every $\alpha\in\ooint{0,\alpha^*}$.   So $f$
is Pfeffer integrable.   Now 484Hc tells us that $F$ must be its
Saks-Henstock indefinite integral.
}%end of proof of 484J

\leader{484K}{Lemma} Suppose that $\alpha>0$ and $0<\alpha'
<\alpha\min(\bover12,2^{r-1}(\Bover{\alpha}{2\beta_r})^{(r-1)/r})$.
If $E\in\Cal C$ is such that $E\subseteq\clstar E$, then there is a
$\delta\in\Delta$ such that $\{(x,C\cap E)\}\in T_{\alpha'}$ whenever
$(x,C)\in\delta$, $x\in E$ and $\{(x,C)\}\in T_{\alpha}$.

\proof{ Take $\epsilon>0$ such that
$\beta_r\epsilon\le\bover12\alpha$ and

\Centerline{$\Bover1{\alpha}+2^{r-1}\epsilon
\le\Bover{2^{r-1}}{\alpha'}
  \bigl(\Bover{\alpha}{2\beta_r}\bigr)^{(r-1)/r}$.}

\noindent Set

\Centerline{$A=\partstar E
   \cup\{x:\lim_{\zeta\downarrow 0}
     \sup_{x\in G,0<\diam G\le\zeta}
     \Bover{\nu^*(G\cap\partstar E)}{(\diam G)^{r-1}}>0\}$,}

\noindent so that $A$ is a thin set, as in (b-i) of the proof of 484F.
(Of course $\partstar E$ is thin because
$\nu(\partstar E\cap B(\tbf{0},n))$ is finite for every $n\in\Bbb N$.)

For $x\in E\setminus A$, we have $x\in\intstar E$ (because
$E\subseteq\clstar E$), so there is a $\theta(x)>0$ such that

\Centerline{$\mu(B(x,\zeta)\setminus E)\le\epsilon\mu B(x,\zeta)$,
\quad$\nu(\partstar E\cap B(x,\zeta))\le\epsilon(2\zeta)^{r-1}$}

\noindent whenever $0<\zeta\le 2\theta(x)$.   If we set $\theta(x)=0$
for $x\in E\cap A$ and $\theta(x)=1$ for $x\in\BbbR^r\setminus E$, then
$\theta\in\Theta$ and $\delta_{\theta}\in\Delta$.

Now suppose that $x\in E$, $(x,C)\in\delta_{\theta}$ and
$\{(x,C)\}\in T_{\alpha}$, that is, that $C\in\Cal C_{\alpha}$ and
$x\in(E\cap\clstar C)\setminus A$ and $\|x-y\|<\theta(x)$ for every
$y\in C$.   Set $\gamma=\diam C\le 2\theta(x)$.   Then

\Centerline{$\mu(C\setminus E)\le\mu(B(x,\gamma)\setminus E)
\le\epsilon\mu B(x,\gamma)=\beta_r\epsilon\gamma^r$,}

\noindent so

$$\eqalign{\mu(C\cap E)
&\ge\mu C-\beta_r\epsilon\gamma^r
\ge(\alpha-\beta_r\epsilon)\gamma^r\cr
&\ge\Bover12\alpha\gamma^r
\ge\alpha'\gamma^r
\ge\alpha'\diam(C\cap E)^r.\cr}$$

\noindent Next,

$$\eqalign{\per(C\cap E)
&=\nu(\partstar(C\cap E))
\le\nu(B(x,\gamma)\cap(\partstar C\cup\partstar E))\cr
&\le\nu(\partstar C)+\nu(B(x,\gamma)\cap\partstar E)
\le\Bover1{\alpha}\gamma^{r-1}+\epsilon(2\gamma)^{r-1}
=(\Bover1{\alpha}+2^{r-1}\epsilon)\gamma^{r-1}.\cr}$$

\noindent Moreover,

\Centerline{$2^{-r}\beta_r\diam(C\cap E)^r
\ge\mu(C\cap E)
\ge\Bover12\alpha\gamma^r$}

\noindent (264H), so
$\diam(C\cap E)\ge 2(\Bover{\alpha}{2\beta_r})^{1/r}\gamma$
and

$$\eqalign{\per(C\cap E)
&\le(\Bover1{\alpha}+2^{r-1}\epsilon)\cdot\Bover1{2^{r-1}}
  \bigl(\Bover{2\beta_r}{\alpha}\bigr)^{(r-1)/r}\diam(C\cap E)^{r-1}
\le\Bover1{\alpha'}\diam(C\cap E)^{r-1}.\cr}$$

\noindent Putting these together, we see that $C\in\Cal C_{\alpha'}$.

Finally, because $x\in\clstar C\cap\intstar E\subseteq\clstar(C\cap E)$
(475Ce), $\{(x,C\cap E)\}\in T_{\alpha'}$.
}%end of proof of 484K

\leader{484L}{Proposition} Suppose that $f:\BbbR^r\to\BbbR$ is Pfeffer
integrable, and that $F:\Cal C\to\Bbb R$ is its Saks-Henstock indefinite
integral.   Then
$\Pfint f\times\chi E$ is defined and equal to $F(E)$ for every
$E\in\Cal C$.

\proof{{\bf (a)} To begin with (down to the end of (d) below),
suppose that $E\in\Cal C$ is such that
$\intstar E\subseteq E\subseteq\clstar E$.   For $C\in\Cal C$ set
$F_1(C)=F(C\cap E)$.   I seek to show that $F_1$ satisfies the
conditions of 484J.

Of course $F_1:\Cal C\to\Bbb R$ is additive.   If $\epsilon>0$, there is
an $\Cal R\in\frak R$ such that $|F(G)|\le\epsilon$ whenever
$G\in\Cal R$, by 484H;  now there is an $\Cal R'\in\frak R$ such that
$G\cap E\in\Cal R$ for every $G\in\Cal R'$ (484E(a-ii)), so that
$|F_1(G)|\le\epsilon$ for every $G\in\Cal C\cap\Cal R'$.   Thus $F_1$
satisfies (iii) of 484J.

\medskip

{\bf (b)} Take $\alpha\in\ooint{0,\alpha^*}$ and $\epsilon>0$.   Take
$\alpha'$ such that $0<\alpha'
<\alpha\min(\bover12,2^{r-1}(\Bover{\alpha}{2\beta_r})^{(r-1)/r})$.
Applying 484K to $E$ and its complement, and appealing to the definition
of $F$, we see that there is a $\delta\in\Delta$ such that

\inset{($\alpha$) $\{(x,C\cap E)\}\in T_{\alpha'}$ whenever
$(x,C)\in\delta$, $x\in E$ and $\{(x,C)\}\in T_{\alpha}$,

($\beta$) $\{(x,C\setminus E)\}\in T_{\alpha'}$ whenever
$(x,C)\in\delta$,
$x\in\BbbR^r\setminus E$ and $\{(x,C)\}\in T_{\alpha}$,

($\gamma$) $\sum_{(x,C)\in\pmb{t}}|F(C)-f(x)\mu C|\le\epsilon$ for every
$\delta$-fine $\pmb{t}\in T_{\alpha'}$.}

\noindent (For ($\alpha$), we need to know that $E\subseteq\clstar E$
and for ($\beta$) we need $\intstar E\subseteq E$.)   Next, choose for
each $n\in\Bbb N$ closed sets
$H_n\subseteq E$, $H'_n\subseteq\BbbR^r\setminus E$ such that
$\mu(E\setminus H_n)\le 2^{-n}\epsilon$ and
$\mu((\BbbR^r\setminus E)\setminus H'_n)\le 2^{-n}\epsilon$.   Define
$\theta:\BbbR^r\to\ooint{0,\infty}$ by setting

$$\eqalign{\theta(x)
&=\min(1,\Bover12\rho(x,H'_n))\text{ if }x\in E
  \text{ and }n\le|f(x)|<n+1,\cr
&=\min(1,\Bover12\rho(x,H_n))\text{ if }x\in\BbbR^r\setminus E
  \text{ and }n\le|f(x)|<n+1,\cr}$$

\noindent writing $\rho(x,A)=\inf_{y\in A}\|x-y\|$ if
$A\subseteq\BbbR^r$ is non-empty, $\infty$ if $A=\emptyset$.   Then
$\delta_{\theta}\in\Delta$.   Let $\Cal R'\in\frak R$ be such that
$A\cap E\in\Cal R$ whenever $A\in\Cal R'$.

\medskip

{\bf (c)} Write $f_E$ for $f\times\chi E$.   Then
$\sum_{(x,C)\in\pmb{t}}|F_1(C)-f_E(x)\mu(C)|\le 11\epsilon$ whenever
$\pmb{t}\in T_{\alpha}$ is $(\delta\cap\delta_{\theta})$-fine.   \Prf\ Set

\Centerline{$\pmb{t}'=\{(x,C\cap E):(x,C)\in\pmb{t},\,x\in E\}$.}

\noindent By clause ($\alpha$) of the choice of $\delta$,
$\pmb{t}'\in T_{\alpha'}$, and of course it is $\delta$-fine.   So

\Centerline{$\sum_{(x,C)\in\pmb{t},x\in E}|F(C\cap E)-f(x)\mu(C\cap E)|
\le\epsilon$}

\noindent by clause ($\gamma$) of the choice of $\delta$.   Next,

$$\eqalignno{\sum_{(x,C)\in\pmb{t},x\in E}|f(x)\mu(C\setminus E)|
&=\sum_{n=0}^{\infty}\sum_{\Atop{(x,C)\in\pmb{t},x\in E}
  {n\le|f(x)|<n+1}}
   |f(x)|\mu(C\setminus E)\cr
&\le\sum_{n=0}^{\infty}(n+1)\mu((\BbbR^r\setminus E)\setminus H'_n)\cr
\displaycause{because $\diam C\le\theta(x)$, so $C\cap H'_n=\emptyset$
whenever $(x,C)\in\pmb{t}$, $x\in E$ and $n\le|f(x)|<n+1$}
&\le\sum_{n=0}^{\infty}2^{-n}(n+1)\epsilon
=4\epsilon,\cr}$$

\noindent and

\Centerline{$\sum_{(x,C)\in\pmb{t},x\in E}|F(C\cap E)-f(x)\mu C|
\le 5\epsilon$.}

\noindent Similarly,

\Centerline{$\sum_{(x,C)\in\pmb{t},x\notin E}
  |F(C\setminus E)-f(x)\mu C)|
\le 5\epsilon$.}

\noindent But as

\Centerline{$\sum_{(x,C)\in\pmb{t},x\notin E}|F(C)-f(x)\mu C|
\le\epsilon$}

\noindent (because surely $\pmb{t}$ itself belongs to $T_{\alpha'}$),
we have

\Centerline{$\sum_{(x,C)\in\pmb{t},x\notin E}|F(C\cap E)|
\le 6\epsilon$.}

Putting these together,

$$\eqalign{\sum_{(x,C)\in\pmb{t}}|F_1(C)-f_E(x)\mu C|
&=\sum_{\Atop{(x,C)\in\pmb{t}}{x\in E}}
      |F(C\cap E)-f(x)\mu C|
   +\sum_{\Atop{(x,C)\in\pmb{t}}{x\notin E}}
      |F(C\cap E)|\cr
&\le 5\epsilon+6\epsilon
=11\epsilon.  \text{ \Qed}\cr}$$

\medskip

{\bf (d)} As $\alpha$ and $\epsilon$ are arbitrary, condition (ii) of
484J is satisfied by $F_1$ and $f_E$, so
$\Pfint f_E=F_1(\BbbR^r)=F(E)$.

\medskip

{\bf (e)} This completes the proof when
$\intstar E\subseteq E\subseteq\clstar E$.   For a general set
$E\in\Cal C$, set $E_1=(E\cup\intstar E)\cap\clstar E$.   Then
$E\symmdiff E_1$ is negligible, so

\Centerline{$\intstar E_1=\intstar E\subseteq E_1
\subseteq\clstar E=\clstar E_1$.}

\noindent Also $\Pfint f\times\chi(E\setminus E_1)
=\int f\times\chi(E\setminus E_1)d\mu$,
$\Pfint f\times\chi(E_1\setminus E)
=\int f\times\chi(E\setminus E_1)d\mu$ are both zero, and

\Centerline{$\biggerPfint f\times\chi E=\Pfint f\times\chi E_1
=F(E_1)=F(E)$.}

\noindent (To see that $F(E_1)=F(E)$, note that $E\setminus E_1$ and
$E_1\setminus E$, being negligible sets, have empty essential boundary
and zero perimeter, so belong to every member of $\frak R$, by
484E(b-ii), or otherwise.)
}%end of proof of 484L

\leader{484M}{Lemma} Let $G$, $H\in\Cal C$ be disjoint and
$\phi:\BbbR^r\to\BbbR^r$ a continuous function.   If {\it either}
$G\cup H$ is bounded {\it or} $\phi$ has compact support,

\Centerline{$\int_{\partstar(G\cup H)}\varinnerprod{\phi}
  {\psi_{G\cup H}}\,d\nu
=\int_{\partstar G}\varinnerprod{\phi}
  {\psi_G}\,d\nu
+\int_{\partstar H}\varinnerprod{\phi}
  {\psi_G}\,d\nu$,}

\noindent where $\psi_G$, $\psi_H$ and $\psi_{G\cup H}$ are the
canonical outward-normal functions\cmmnt{ (474G)}.

\proof{{\bf (a)} Suppose first that $\phi$ is a Lipschitz function with
compact support.   Then 475N tells us that

$$\eqalign{\int_{\partstar(G\cup H)}
  \varinnerprod{\phi}{\psi_{G\cup H}}\,d\nu
&=\int_{G\cup H}\diverg\phi\,d\mu
=\int_G\diverg\phi\,d\mu+\int_H\diverg\phi\,d\mu\cr
&=\int_{\partstar G}\varinnerprod{\phi}
  {\psi_G}\,d\nu
+\int_{\partstar H}\varinnerprod{\phi}
  {\psi_G}\,d\nu.\cr}$$

\noindent (Recall from 474R that we can identify canonical outward-normal
functions with Federer exterior normals, as in the statement of 475N.)

\medskip

{\bf (b)} Now suppose that $\phi$ is a continuous function with compact
support.   Let $\sequencen{\tilde h_n}$ be the smoothing sequence of
473E.   Then all the functions $\phi*\tilde h_n$ are Lipschitz and
$\sequencen{\phi*\tilde h_n}$ converges uniformly to $\phi$ (473Df,
473Ed).   So

$$\eqalign{\int_{\partstar(G\cup H)}\varinnerprod{\phi}
  {\psi_{G\cup H}}\,d\nu
&=\lim_{n\to\infty}\int_{\partstar(G\cup H)}
  \varinnerprod{(\phi*\tilde h_n)}{\psi_{G\cup H}}\,d\nu\cr
&=\lim_{n\to\infty}\int_{\partstar G}
  \varinnerprod{(\phi*\tilde h_n)}{\psi_G}\,d\nu
+\lim_{n\to\infty}\int_{\partstar H}
  \varinnerprod{(\phi*\tilde h_n)}{\psi_H}\,d\nu\cr
&=\int_{\partstar G}\varinnerprod{\phi}
  {\psi_G}\,d\nu
+\int_{\partstar H}\varinnerprod{\phi}
  {\psi_G}\,d\nu.\cr}$$

\medskip

{\bf (c)} If, on the other hand, $\phi$ is continuous and $G$ and $H$
are bounded, then we can find a continuous function $\tilde\phi$ with
compact support agreeing with $\phi$ on $\overline{G\cup H}$ (4A2G(e-i),
or otherwise);  applying (b) to $\tilde\phi$, we get the required result
for $\phi$.
}%end of proof of 484M

\leader{484N}{Pfeffer's Divergence Theorem}  Let
$E\subseteq\BbbR^r$ be a set with locally finite perimeter, and
$\phi:\BbbR^r\to\BbbR^r$ a continuous function with compact support such
that $\{x:x\in\BbbR^r,\,\phi$ is not differentiable at $x\}$ is thin.
Let $v_x$ be the Federer exterior normal to $E$ at any point $x$ where
the normal exists.   Then $\Pfint\diverg\phi\times\chi E$ is defined and
equal to $\int_{\partstar E}\varinnerprod{\phi(x)}{v_x}\nu(dx)$.

\proof{{\bf (a)} Let $n$ be such that $\phi(x)=\tbf{0}$ for
$\|x\|\ge n$.   For $C\in\Cal C$, set
$F(C)=\int_{\partstar C}\varinnerprod{\phi}{\psi_C}\nu(dx)$, where
$\psi_C$ is the canonical outward-normal function;  recall that
$\psi_E(x)=v_x$ for $\nu$-almost every $x\in\partstar E$ (474R, 475D).
By 484M, $F$ is additive.

\medskip

{\bf (b)} If $0<\alpha<\alpha^*$ and $\epsilon>0$ and $x\in\BbbR^r$ is
such that $\phi$ is differentiable at $x$, there is a $\gamma>0$ such
that $|F(C)-\diverg\phi(x)\mu C|\le\epsilon\mu C$ whenever
$C\in\Cal C_{\alpha}$, $x\in\overline{C}$ and $\diam C\le\gamma$.
\Prf\ Let $T$ be the derivative of $\phi$ at $x$.   Let $\gamma>0$ be
such that $\|\phi(y)-\phi(x)-T(y-x)\|\le\alpha^2\epsilon\|y-x\|$
whenever $\|y-x\|\le\gamma$.   Let $\tilde\phi:\BbbR^r\to\BbbR^r$ be a
Lipschitz function with compact support such that
$\tilde\phi(y)=\phi(x)+(y-x)$ whenever $\|y-x\|\le\gamma$ (473Cf).
If $C\in\Cal C_{\alpha}$ has diameter at most $\gamma$ and
$x\in\overline{C}$, then

$$\eqalignno{|F(C)-\diverg\phi(x)\mu C|
&=|\int_{\partstar C}\varinnerprod{\phi}{\psi_C}\nu(dx)
  -\int_C\diverg\tilde\phi\,d\mu|\cr
\displaycause{because $T$ is the derivative of $\tilde\phi$ everywhere
on $B(x,\gamma)$, so $\diverg\tilde\phi(y)=\diverg\phi(x)$ for every
$y\in C$}
&=|\int_{\partstar C}\varinnerprod{(\phi-\tilde\phi)}{\psi_C}\nu(dx)|\cr
\displaycause{applying the Divergence Theorem 475N to $\tilde\phi$}
&\le\nu(\partstar C)\sup_{y\in C}\|\phi(y)-\tilde\phi(y)\|\cr
&\le\alpha^2\epsilon\diam C\per C
\le\alpha\epsilon(\diam C)^r
\le\epsilon\mu C\cr}$$

\noindent because $C\in\Cal C_{\alpha}$.\ \Qed

\medskip

{\bf (c)} If $\epsilon>0$ and $\alpha\in\ooint{0,\alpha^*}$, there is a
$\delta\in\Delta$ such that
$\sum_{(x,C)\in\pmb{t}}|F(C)-\diverg\phi(x)\mu C|\le\epsilon$ whenever
$\pmb{t}\in T_{\alpha}$ is $\delta$-fine.
\Prf\ Let $\zeta>0$ be such that $\zeta\mu B(\tbf{0},n+2)\le\epsilon$.
Set $A=\{x:x\in\BbbR^r,\,\phi$ is not differentiable at $x\}$, and for
$x\in\BbbR^r\setminus A$ let $\theta(x)\in\ocint{0,\bover12}$ be such
that $|F(C)-\diverg\phi(x)\mu C|\le\zeta\mu C$ whenever
$C\in\Cal C_{\alpha}$, $x\in\overline{C}$ and $\diam C\le\theta(x)$;
for $x\in A$ set $\theta(x)=0$.
Now suppose that $\pmb{t}\in T_{\alpha}$ is $\delta_{\theta}$-fine.
Then

$$\eqalignno{\sum_{(x,C)\in\pmb{t}}|F(C)-\diverg\phi(x)\mu C|
&=\sum_{\Atop{(x,C)\in\pmb{t}}{x\in B(\tbf{0},n+1)}}
|F(C)-\diverg\phi(x)\mu C|\cr
\displaycause{because $\diam C\le 1$ whenever $(x,C)\in\pmb{t}$, so
if $\|x\|>n+1$ then $F(C)=\diverg\phi(x)=0$}
&\le\sum_{\Atop{(x,C)\in\pmb{t}}{x\in B(\tbf{0},n+1)}}\zeta\mu C
\le\zeta\mu B(\tbf{0},n+2)
\le\epsilon. \text{ \Qed}\cr}$$

\medskip

{\bf (d)} Because $\phi$ is a continuous function with compact support,
it is uniformly continuous (apply 4A2Jf to each of the coordinates of
$\phi$).   For $\zeta>0$, let $\gamma(\zeta)>0$ be such that
$\|\phi(x)-\phi(y)\|\le\zeta$ whenever $\|x-y\|\le\gamma(\zeta)$.

If $C\in\Cal C$ and $\per C\le 1$ and $\mu C\le\zeta\gamma(\zeta)$,
where $\zeta>0$, then
$|F(C)|\le r\zeta(2\|\phi\|_{\infty}+\bover12)$, writing
$\|\phi\|_{\infty}$ for $\sup_{x\in\BbbR^r}\|\phi(x)\|$.
\Prf\ For $1\le i\le r$, let $\phi_i:\BbbR^r\to\Bbb R$ be the $i$th
component of $\phi$, and $v_i$ the $i$th unit vector
$(0,\ldots,1,\ldots,0)$;  write

\Centerline{$\alpha_i
=\int_{\partstar C}\phi_i(x)\innerprod{v_i}{\psi_E(x)}\nu(dx)$,}

\noindent so that $F(C)=\sum_{i=1}^r\alpha_i$.   I start by examining
$\alpha_r$.   By 475O, we have sequences $\sequencen{H_n}$,
$\sequencen{g_n}$ and $\sequencen{g'_n}$ such that

\inset{(i) for each $n\in\Bbb N$, $H_n$ is a Lebesgue measurable subset
of $\BbbR^{r-1}$, and $g_n$, $g'_n:H_n\to[-\infty,\infty]$ are Lebesgue
measurable functions such that $g_n(u)<g'_n(u)$ for every $u\in H_n$;

(ii) if $m$, $n\in\Bbb N$ then $g_m(u)\ne g'_n(u)$ for every
$u\in H_m\cap H_n$;

(iii) $\sum_{n=0}^{\infty}\int_{H_n}g'_n-g_nd\mu_{r-1}=\mu C$;

(iv)

\Centerline{$\alpha_r
=\sum_{n=0}^{\infty}\int_{H_n}\phi_r(u,g'_n(u))-\phi_r(u,g_n(u))
  \mu_{r-1}(du)$,}

\noindent where we interpret $\phi_r(u,\infty)$ and $\phi_r(u,-\infty)$
as $0$ if necessary;

(v) for $\mu_{r-1}$-almost every $u\in\BbbR^{r-1}$,

$$\eqalign{\{t:(u,t)\in\partstar C\}
&=\{g_n(u):n\in\Bbb N,\,u\in H_n,\,g_n(u)\ne-\infty\}\cr
&\hskip8em
  \cup\{g'_n(u):n\in\Bbb N,\,u\in H_n,\,g'_n(u)\ne\infty\}.\cr}$$
}%end of inset

\noindent From (iii) we see that $g'_n$ and $g_n$ are both finite almost
everywhere on $H_n$, for every $n$.   Consequently, by (v) and 475H,

\Centerline{$2\sum_{n=0}^{\infty}\mu_{r-1}H_n
=\int\#(\{t:(u,t)\in\partstar C\})\mu_{r-1}(du)
\le\nu(\partstar C)\le 1$.}

\noindent For each $n$, set

\Centerline{$H'_n=\{u:u\in H_n,\,g'_n(u)-g_n(u)>\gamma(\zeta)\}$.}

\noindent Then $\gamma(\zeta)\sum_{n=0}^{\infty}\mu H'_n\le\mu C$ so
$\sum_{n=0}^{\infty}\mu H'_n\le\zeta$ and

$$|\sum_{n=0}^{\infty}\int_{H'_n}
  \phi_r(u,g'_n(u))-\phi_r(u,g_n(u))\mu_{r-1}(du)|
\le 2\|\phi\|_{\infty}\sum_{n=0}^{\infty}\mu H'_n
\le 2\zeta\|\phi\|_{\infty}.$$

\noindent On the other hand, for $n\in\Bbb N$ and
$u\in H_n\setminus H'_n$,
$|\phi_r(u,g'_n(u))-\phi_r(u,g_n(u))|\le\zeta$, so

$$|\sum_{n=0}^{\infty}\int_{H_n\setminus H'_n}
  \phi_r(u,g'_n(u))-\phi_r(u,g_n(u))\mu_{r-1}(du)|
\le\sum_{n=0}^{\infty}\zeta\mu_{r-1}(H_n\setminus H'_n)
\le\Bover12\zeta.$$

\noindent Putting these together,

\Centerline{$\alpha_r\le 2\zeta\|\phi\|_{\infty}+\Bover12\zeta$.}

\noindent But of course the same arguments apply to all the $\alpha_i$,
so

\Centerline{$|F(C)|\le\sum_{i=1}^r|\alpha_i|
\le r\zeta(2\|\phi\|_{\infty}+\Bover12)$,}

\noindent as claimed.\ \Qed

\medskip

{\bf (e)} If $\epsilon>0$ there is an $\Cal R\in\frak R$ such that
$|F(E)|\le\epsilon$ for every $E\in\Cal C\cap\Cal R$.   \Prf\ Let
$\sequence{i}{\epsilon_i}$ be a sequence of strictly positive real
numbers such that
$r(2\|\phi\|_{\infty}+\bover12)\sum_{i=0}^{\infty}\epsilon_i
\le\epsilon$.
For each $i\in\Bbb N$, set $\eta(i)=\epsilon_i\gamma(\epsilon_i)>0$.
Set $V=B(\tbf{0},n+1)$, and take any
$E\in\Cal C\cap\Cal R^{(V)}_{\eta}$.   Then $F(E\setminus V)=0$, so
$F(E)=F(E\cap V)$, while $E\cap V\in\Cal R_{\eta}$.   Express $E\cap V$
as $\bigcup_{i\le n}E_i$, where $\langle E_i\rangle_{i\le n}$ is
disjoint, $\per E_i\le 1$ and $\mu E_i\le\eta(i)$ for each $i$.
Then $|F(E_i)|\le r\epsilon_i(2\|\phi\|_{\infty}+\bover12)$ for each
$i$, by (d), so

\Centerline{$|F(E)|=|F(E\cap V)|\le\sum_{i=0}^n|F(E_i)|\le\epsilon$,}

\noindent as required.\ \Qed

\medskip

{\bf (f)} By 484J, $\diverg\phi$ is Pfeffer integrable.   Moreover, by
the uniqueness assertion in 484Hc, its Saks-Henstock indefinite integral
is just the function $F$ here.   By 484L,
$F(E)=\Pfint\diverg\phi\times\chi E$ for every $E\in\Cal C$, as required.
}%end of proof of 484N

\leader{484O}{Differentiating the indefinite integral:  Theorem} Let
$f:\BbbR^r\to\Bbb R$ be a Pfeffer integrable function, and $F$ its
Saks-Henstock indefinite integral.   Then whenever $0<\alpha<\alpha^*$,

$$\eqalign{f(x)
&=\lim_{\zeta\downarrow 0}\sup\{\Bover{F(C)}{\mu C}:
  C\in\Cal C_{\alpha},\,x\in\overline{C},\,0<\diam C\le\zeta\}\cr
&=\lim_{\zeta\downarrow 0}\inf\{\Bover{F(C)}{\mu C}:
  C\in\Cal C_{\alpha},\,x\in\overline{C},\,0<\diam C\le\zeta\}\cr}$$

\noindent for $\mu$-almost every $x\in\BbbR^r$.

\proof{{\bf (a)} It will be useful to know the following:  if
$C\in\Cal C_{\alpha}$, $\diam C>0$, $x\in\overline{C}$ and $\epsilon>0$,
then for any sufficiently small $\zeta>0$,
$C\cup B(x,\zeta)\in\Cal C_{\alpha/2}$ and
$|F(C\cup B(x,\zeta))-F(C)|\le\epsilon$.   \Prf\ Let $\Cal R\in\frak R$
be such that $|F(R)|\le\epsilon$ whenever $R\in\Cal C\cap\Cal R$, let
$\Cal R'\in\frak R$ be such that $(\BbbR^r\setminus C)\cap R\in\Cal R$
whenever $R\in\Cal R'$ (484E(a-ii)), and let $\eta\in\Eta$ be such that
$\Cal R_{\eta}\subseteq\Cal R'$ (484E(a-i)).
Then for all sufficiently small $\zeta>0$, we shall have
$\per B(x,\zeta)\le 1$ and $\mu B(x,\zeta)\le\eta(0)$, so that
$B(x,\zeta)\in\Cal R_{\eta}$, $B(x,\zeta)\setminus C\in\Cal R$ and

\Centerline{$|F(C\cup B(x,\zeta))-F(C)|
=|F(B(x,\zeta)\setminus C)|\le\epsilon$.}

\noindent Next, for all sufficiently small $\zeta>0$,

$$\eqalignno{\mu(C\cup B(x,\zeta))
&\ge\mu C
\ge\alpha(\diam C)^r\cr
&\ge\Bover{\alpha}2(\zeta+\diam C)^r
\ge\Bover{\alpha}2\diam(C\cup B(x,\zeta))^r\cr
\noalign{\noindent (because $x\in\overline{C}$) and}
\per(C\cup B(x,\zeta))
&\le\per C+\per B(x,\zeta)
\le\Bover1{\alpha}(\diam C)^{r-1}+\per B(x,\zeta)\cr
&\le\Bover2{\alpha}(\diam C)^{r-1}
\le\Bover2{\alpha}\diam(C\cup B(x,\zeta))^{r-1},\cr}$$

\noindent so that $C\cup B(x,\zeta)\in\Cal C_{\alpha/2}$.\ \Qed

\medskip

{\bf (b)} For $x\in\Bbb R$, set

\Centerline{$g(x)=\lim_{\zeta\downarrow 0}
  \sup\{\Bover{F(C)}{\mu C}:C\in\Cal C_{\alpha},\,
     x\in\overline{C},\,\diam C\le\zeta\}$.}

\noindent \Quer\ Suppose, if possible, that there are rational numbers
$q<q'$ and $n\in\Bbb N$ such that
$A=\{x:\|x\|\le n,\,f(x)\le q<q'<g(x)\}$ is not $\mu$-negligible.
Set

\Centerline{$\epsilon=\Bover{(q'-q)\alpha}{4\beta_r}\mu^*A
>0$.}

\noindent Let $\theta\in\Theta$ be such that

\Centerline{$\sum_{(x,C)\in\pmb{t}}|F(C)-f(x)\mu C|
\le\epsilon$}

\noindent for every $\delta_{\theta}$-fine $\pmb{t}\in T_{\alpha/2}$.
Let $\Cal I$ be the family of all balls $B(x,\zeta)$ where $x\in A$,
$0<\zeta\le\theta(x)$ and there is a $C\in\Cal C_{\alpha}$ such that
$x\in\overline{C}$,
$\diam C=\zeta$ and $\Bover{F(C)}{\mu C}>q'$.   Then every member of
$A_1=A\setminus\theta^{-1}[\{0\}]$ is the centre of arbitrarily small
members of $\Cal I$, so by 472C there is a countable disjoint family
$\Cal J_0\subseteq\Cal I$ such that

\Centerline{$\mu(\bigcup\Cal J_0)>\Bover12\mu^*A_1
=\Bover12\mu^*A$.}

\noindent There is therefore a finite family $\Cal J_1\subseteq\Cal J_0$
such that $\mu(\bigcup\Cal J_1)>\Bover12\mu^*A$;  enumerate
$\Cal J_1$ as $\langle B(x_i,\zeta_i)\rangle_{i\le n}$ where, for each
$i\le n$, $x_i\in A$, $0<\zeta_i\le\theta_i(x)$ and there is a
$C_i\in\Cal C_{\alpha}$ such that $x_i\in\overline{C_i}$,
$\diam C_i=\zeta_i$ and $F(C_i)>q'\mu C_i$.   By (a), we can enlarge
$C_i$ by adding a sufficiently small ball around $x_i$ to form a
$C'_i\in\Cal C_{\alpha/2}$ such that $x_i\in\interior C'_i$,
$C'_i\subseteq B(x_i,\zeta_i)$ and $F(C'_i)\ge q'\mu C'_i$.

Consider $\pmb{t}=\{(x_i,C'_i):i\le n\}$.   Then, because the balls
$B(x_i,\zeta_i)$ are disjoint, and
$x_i\in\interior C'_i\subseteq\clstar C'_i$ for every $i$, $\pmb{t}$ is
a $\delta_{\theta}$-fine member of $T_{\alpha/2}$.   So
$\sum_{i=0}^nF(C'_i)\le\epsilon+\sum_{i=0}^nf(x_i)\mu C'_i$.   But as
$F(C'_i)\ge q'\mu C'_i$ and $f(x_i)\le q$ for every $i$, this means that
$(q'-q)\sum_{i=0}^n\mu C'_i\le\epsilon$.

But now remember that $\diam C'_i\ge\diam C_i=\zeta_i$ and that
$C'_i\in\Cal C_{\alpha/2}$ for each $i$.   This means that

\Centerline{$\mu C'_i
\ge\Bover{\alpha}2\zeta_i^r
\ge\Bover{\alpha}{2\beta_r}\mu B(x_i,\zeta_i)$}

\noindent for each $i$, and

$$\eqalign{\epsilon
&\ge(q'-q)\sum_{i=0}^n\mu C'_i
\ge\Bover{(q'-q)\alpha}{2\beta_r}\sum_{i=0}^n\mu B(x_i,\zeta_i)\cr
&>\Bover{(q'-q)\alpha}{4\beta_r}\mu^*A
=\epsilon\cr}$$

\noindent which is absurd.\ \Bang

\medskip

{\bf (c)} Since $q$, $q'$ and $n$ are arbitrary, this means that
$g\leae f$.   Similarly (or applying (b) to $-f$ and $-F$)

\Centerline{$f(x)
\le\lim_{\zeta\downarrow 0}\inf\{\Bover{F(C)}{\mu C}:
  C\in\Cal C_{\alpha},\,x\in\overline{C},\,0<\diam C\le\zeta\}$}

\noindent for almost all $x$, as required.
}%end of proof of 484O

\vleader{72pt}{484P}{Lemma} Let $\phi:\BbbR^r\to\BbbR^r$ be an injective
Lipschitz function, and $H$ the set of points at which it is
differentiable;  for $x\in H$, write $T(x)$ for the derivative of $\phi$
at $x$ and $J(x)$ for $|\det T(x)|$.
Then, for $\mu$-almost every $x\in\BbbR^r$,

$$\eqalign{J(x)
&=\lim_{\zeta\downarrow 0}\sup\{\Bover{\mu\phi[C]}{\mu C}:
  C\in\Cal C_{\alpha},\,x\in\overline{C},\,0<\diam C\le\zeta\}\cr
&=\lim_{\zeta\downarrow 0}\inf\{\Bover{\mu\phi[C]}{\mu C}:
  C\in\Cal C_{\alpha},\,x\in\overline{C},\,0<\diam C\le\zeta\}\cr}$$

\noindent for every $\alpha>0$.

\proof{ By Rademacher's theorem (262Q), $H$ is conegligible.   Let $H'$
be the Lebesgue set of $J$, so that $H'$ also is conegligible (261E).
Take any $x\in H'$ and $\epsilon>0$.   Then there is a $\zeta_0>0$ such
that $\int_{B(x,\zeta)}|J(y)-J(x)|\mu(dy)\le\epsilon\mu B(x,\zeta)$ for
every $\zeta\in[0,\zeta_0]$.   Now suppose that $C\in\Cal C_{\alpha}$,
$x\in\overline{C}$ and $0<\diam C\le\zeta_0$.   Then
$\mu(C\setminus H)=0$ so $\mu\phi[C\setminus H]=0$ (262D), and

\Centerline{$\mu\phi[C]=\mu\phi[C\cap H]=\int_{C\cap H}J\,d\mu$}

\noindent (263D(iv)).   So

$$\eqalign{|\mu\phi[C]-J(x)\mu C|
&=|\int_{C\cap H}J\,d\mu-J(x)\mu(C\cap H)|
\le\int_{C\cap H}|J(y)-J(x)|d\mu\cr
&\le\int_{B(x,\diam C)}|J(y)-J(x)|d\mu
\le\epsilon\mu B(x,\diam C)\cr
&=\beta_r\epsilon(\diam C)^r
\le\Bover{\beta_r\epsilon}{\alpha}\mu C.\cr}$$

\noindent Thus
$|\Bover{\mu\phi[C]}{\mu C}-J(x)|\le\Bover{\beta_r}{\alpha}\epsilon$
whenever $C\in\Cal C_{\alpha}$, $x\in\overline{C}$ and
$0<\diam C\le\zeta_0$;  as $\epsilon$ is arbitrary,

$$\eqalign{J(x)
&=\lim_{\zeta\downarrow 0}\sup\{\Bover{\mu\phi[C]}{\mu C}:
  C\in\Cal C_{\alpha},\,x\in\overline{C},\,0<\diam C\le\zeta\}\cr
&=\lim_{\zeta\downarrow 0}\inf\{\Bover{\mu\phi[C]}{\mu C}:
  C\in\Cal C_{\alpha},\,x\in\overline{C},\,0<\diam C\le\zeta\}.\cr}$$

\noindent And this is true for $\mu$-almost every $x$.
}%end of proof of 484P

\leader{484Q}{Definition} If $(X,\rho)$ and $(Y,\sigma)$ are metric
spaces, a function $\phi:X\to Y$ is a {\bf lipeomorphism} if it is
bijective and both $\phi$ and $\phi^{-1}$ are Lipschitz.   \cmmnt{Of
course a lipeomorphism is a homeomorphism.}

\leader{484R}{Lemma} Let $\phi:\BbbR^r\to\BbbR^r$ be a lipeomorphism.

(a) For any set $A\subseteq\BbbR^r$,

\Centerline{$\clstar(\phi[A])=\phi[\clstar A]$,
\quad$\intstar(\phi[A])=\phi[\intstar A]$,
\quad$\partstar(\phi[A])=\phi[\partstar A]$.}

(b) $\phi[C]\in\Cal C$ for every $C\in\Cal C$, and $\phi[V]\in\Cal V$
for every $V\in\Cal V$.

(c) For any $\alpha>0$ there is an $\alpha'\in\ocint{0,\alpha}$ such
that $\phi[C]\in\Cal C_{\alpha'}$ for every $C\in\Cal C_{\alpha}$ and
$\{(\phi(x),\phi[C]):(x,C)\in\pmb{t}\}$ belongs to $T_{\alpha'}$ for
every $\pmb{t}\in T_{\alpha}$.

(d) For any $\Cal R\in\frak R$ there is an $\Cal R'\in\frak R$ such that
$\phi[R]\in\Cal R$ for every $R\in\Cal R'$.

(e) $\theta\phi:\BbbR^r\to\coint{0,\infty}$ belongs to $\Theta$ for
every $\theta\in\Theta$.

\proof{ Let $\gamma$ be so large that it is a Lipschitz constant for
both $\phi$ and $\phi^{-1}$.   Observe that in this case

\Centerline{$\phi^{-1}[B(\phi(x),\Bover{\zeta}{\gamma})]
\subseteq B(x,\zeta)$,
\quad$\phi[B(x,\zeta)]\supseteq B(\phi(x),\Bover{\zeta}{\gamma})$}

\noindent for every $x\in\BbbR^r$ and $\zeta\ge 0$, while

\Centerline{$\mu^*A=\mu^*\phi^{-1}[\phi[A]]\le\gamma^r\mu^*\phi[A]$,
\quad$\nu^*A\le\gamma^{r-1}\nu^*\phi[A]$}

\noindent for every $A\subseteq\BbbR^r$ (264G/471J).

\medskip

{\bf (a)} If $A\subseteq\BbbR^r$ and $x\in\clstar A$, set

\Centerline{$\epsilon
=\Bover12\limsup_{\zeta\downarrow 0}
   \Bover{\mu^*(B(x,\zeta)\cap A)}{\mu B(x,\zeta)}>0$.}

\noindent Take any $\zeta_0>0$.   Then there is a $\zeta$ such that
$0<\zeta\le\zeta_0$ and
$\mu^*(B(x,\bover{\zeta}{\gamma})\cap A)
\ge\epsilon\mu B(x,\bover{\zeta}{\gamma})$, so that

$$\eqalign{\mu^*(B(\phi(x),\zeta)\cap\phi[A])
&\ge\mu^*\phi[B(x,\Bover{\zeta}{\gamma})\cap A]
\ge\Bover1{\gamma^r}\mu^*(B(x,\Bover{\zeta}{\gamma})\cap A)\cr
&\ge\Bover{\epsilon}{\gamma^r}\mu B(x,\Bover{\zeta}{\gamma})
=\Bover{\epsilon}{\gamma^{2r}}\mu B(x,\zeta).\cr}$$

\noindent As $\zeta_0$ is arbitrary,

\Centerline{$\limsup_{\zeta\downarrow 0}
   \Bover{\mu^*(B(\phi(x),\zeta)\cap\phi[A]}{\mu B(\phi(x),\zeta)}
\ge\Bover{\epsilon}{\gamma^{2r}}>0$,}

\noindent and $\phi(x)\in\clstar(\phi[A])$.

This shows that $\phi[\clstar A]\subseteq\clstar(\phi[A])$.   The same
argument applies to $\phi^{-1}$ and $\phi[A]$, so that $\phi[\clstar A]$
must be equal to $\clstar(\phi[A])$.   Taking complements,
$\phi[\intstar A]=\intstar(\phi[A])$, so that
$\phi[\partstar A]=\partstar(\phi[A])$.

\medskip

{\bf (b)} Take $C\in\Cal C$.   Then, for any $n\in\Bbb N$,
$\phi^{-1}[B(\tbf{0},n)]$ is bounded, so is included in $B(\tbf{0},m)$
for some $m$.   Now

$$\eqalign{\nu(\partstar\phi[C]\cap B(\tbf{0},n))
&=\nu(\phi[\partstar C]\cap B(\tbf{0},n))\cr
&\le\nu(\phi[\partstar C\cap B(\tbf{0},m)])
\le\gamma^{r-1}\nu(\partstar C\cap B(\tbf{0},m))\cr}$$

\noindent is finite.   This shows that $\phi[C]$ has
locally finite perimeter and belongs to $\Cal C$.   Since $\phi[V]$ is
bounded whenever $V$ is bounded, $\phi[V]\in\Cal V$ whenever
$V\in\Cal V$.

\medskip

{\bf (c)} Set $\alpha'=\gamma^{-2r}\alpha$.   Note that as $\gamma^2$ is
a Lipschitz constant for the identity map, $\gamma\ge 1$, and
$\alpha'\le\min(\alpha,\gamma^{2-2r}\alpha)$.   If
$C\in\Cal C_{\alpha}$, then

$$\eqalign{\mu\phi[C]
&\ge\Bover1{\gamma^r}\mu C
\ge\Bover{\alpha}{\gamma^r}(\diam C)^r
\ge\alpha\diam\phi[C])^r
\ge\alpha'(\diam\phi[C])^r,\cr
\cr
\per\phi[C]
&=\nu(\partstar(\phi[C]))
=\nu(\phi[\partstar C])
\le\gamma^{r-1}\nu(\partstar C)\cr
&\le\Bover{\gamma^{r-1}}{\alpha}(\diam C)^{r-1}
\le\Bover{\gamma^{r-1}}{\alpha}(\gamma\diam\phi[C])^{r-1}
\le\Bover1{\alpha'}(\diam\phi[C])^{r-1}.\cr}$$

\noindent So $C\in\Cal C_{\alpha'}$.

If now $\pmb{t}\in T_{\alpha}$, then, for any $(x,C)\in\pmb{t}$,
$\phi[C]\in\Cal C_{\alpha'}$ and $\phi(x)\in\clstar\phi[C]$;  also,
because $\phi$ is injective, $\langle\phi[C]\rangle_{(x,C)\in\pmb{t}}$
is disjoint, so $\{(\phi(x),\phi[C]):(x,C)\in\pmb{t}\}\in T_{\alpha'}$.

\medskip

{\bf (d)} Express $\Cal R$ as $\Cal R^{(V)}_{\eta}$ where $V\in\Cal V$
and $\eta\in\Eta$, so that $R\in\Cal R$ whenever $R\cap V\in\Cal R$.
By 484Ec and 481He, there is a sequence
$\sequence{i}{\Cal Q_i}$ in $\frak R$ such that
$\bigcup_{i\le n}A_i\in\Cal R$ whenever $n\in\Bbb N$,
$\langle A_i\rangle_{i\le n}$ is disjoint and $A_i\in\Cal Q_i$ for every
$i$.   By 484E(b-ii), there is an $\eta'\in\Eta$ such that
$R\in\Cal Q_i$ whenever $i\in\Bbb N$ and $R$ is such that
$\mu R\le\gamma^r\eta'(i)$ and  $\per R\le\gamma^{r-1}$.   Try
$\Cal R'=\Cal R^{(\phi^{-1}[V])}_{\eta'}\in\frak R$.   If $R\in\Cal R'$,
we can express $R\cap\phi^{-1}[V]$ as $\bigcup_{i\le n}E_i$ where
$\per E_i\le 1$ and $\mu E_i\le\eta'_i$ for each $i\le n$, and
$\langle E_i\rangle_{i\le n}$ is disjoint.   So
$\phi[R]\cap V=\bigcup_{i\le n}\phi[E_i]$ and
$\langle\phi[E_i]\rangle_{i\le n}$ is disjoint.   Now, for each $i$,

\Centerline{$\mu\phi[E_i]\le\gamma^r\mu E_i\le\gamma^r\eta'(i)$,
\quad$\per\phi[E_i]\le\gamma^{r-1}\per E_i\le\gamma^{r-1}$,}

\noindent so $\phi[E_i]\in\Cal Q_i$.   By the choice of
$\sequence{i}{\Cal Q_i}$, $\phi[R]\cap V\in\Cal R$ and
$\phi[R]\in\Cal R$.   So $\Cal R'$ has the property we need.

\medskip

{\bf (e)} We have only to observe that if $A$ is the thin set
$\theta^{-1}[\{0\}]$, then $(\theta\phi)^{-1}[\{0\}]=\phi^{-1}[A]$ is
also thin.   \Prf\ If $A=\bigcup_{n\in\Bbb N}A_n$ where $\nu^*A_n$ is
finite for every $n$, then
$\phi^{-1}[A]=\bigcup_{n\in\Bbb N}\phi^{-1}[A_n]$, while
$\nu^*\phi^{-1}[A_n]\le\gamma^{r-1}\nu^*A_n$ is finite for
every $n\in\Bbb N$.\ \Qed
}%end of proof of 484R

\leader{484S}{Theorem} Let $\phi:\BbbR^r\to\BbbR^r$ be a lipeomorphism.
Let $H$ be the set of points at which $\phi$ is differentiable.   For
$x\in H$, write $T(x)$ for the derivative of $\phi$ at $x$;  set
$J(x)=|\det T(x)|$ for $x\in H$, $0$ for $x\in\BbbR^r\setminus H$.
Then, for any function $f:\BbbR^r\to\BbbR^r$,

\Centerline{$\biggerPfint f=\Pfint J\times f\phi$}

\noindent if either is defined in $\Bbb R$.

\proof{{\bf (a)} Let $H'\subseteq H$ be a conegligible set such that

$$\eqalign{J(x)
&=\lim_{\zeta\downarrow 0}\sup\{\Bover{\mu\phi[C]}{\mu C}:
  C\in\Cal C_{\alpha},\,x\in\overline{C},\,0<\diam C\le\zeta\}\cr
&=\lim_{\zeta\downarrow 0}\inf\{\Bover{\mu\phi[C]}{\mu C}:
  C\in\Cal C_{\alpha},\,x\in\overline{C},\,0<\diam C\le\zeta\}\cr}$$

\noindent for every $\alpha>0$ and every $x\in H'$ (484P).
To begin with (down to the end of (c)), suppose that $f$ is Pfeffer
integrable
and that $f\phi(x)=0$ for every $x\in\BbbR^r\setminus H'$.   Let $F$ be
the Saks-Henstock indefinite integral of $f$, and define
$G:\Cal C\to\Bbb R$ by setting $G(C)=F(\phi[C])$ for every $C\in\Cal C$
(using 484Rb to see that this is well-defined).

\medskip

{\bf (b)} $G$ and $J\times f\phi$ satisfy the conditions of 484J.

\medskip

\Prf{\bf (i)} Of course $G$ is additive, because $F$ is.

\medskip

\quad{\bf (ii)} Suppose that $0<\alpha<\alpha^*$ and $\epsilon>0$.   Let
$\alpha'\in\ooint{0,\alpha^*}$ be such that
$\{(\phi(x),\phi[C]):(x,C)\in\pmb{t}\}\in T_{\alpha'}$ whenever
$\pmb{t}\in T_{\alpha}$ (484Rc).   Let $\theta_1\in\Theta$ be such that
$\sum_{(x,C)\in\pmb{t}}|F(C)-f(x)\mu C|\le\bover12\epsilon$ for every
$\delta_{\theta_1}$-fine $\pmb{t}\in T_{\alpha'}$.   Let
$\theta_2:\BbbR^r\to\ocint{0,1}$ be such that whenever $x\in H'$,
$n\le\|x\|+|f\phi(x)|<n+1$, $C\in\Cal C_{\alpha'}$, $x\in\overline{C}$
and $\diam C\le 2\theta_2(x)$ then

\Centerline{$|\mu\phi[C]-J(x)\mu C|
\le\Bover{\epsilon\mu C}{2^{n+2}\beta_r(n+2)^r(n+1)}$.}

\noindent Set
$\theta(x)=\min(\Bover1{\gamma}\theta_1\phi(x),\theta_2(x))$ for
$x\in\BbbR^r$, where $\gamma>0$ is a Lipschitz constant for $\phi$, so
that $\theta\in\Theta$ (484Re).

If $\pmb{t}\in T_{\alpha}$ is $\delta_{\theta}$-fine, set
$\pmb{t}'=\{(\phi(x),\phi[C]):(x,C)\in\pmb{t}\}$.   Then
$\pmb{t}'\in T_{\alpha'}$, by the choice of $\alpha'$.   If
$(x,C)\in\pmb{t}$, then $\theta(x)>0$ so $\theta_1\phi(x)>0$;  also, for
any $y\in\phi[C]$,

\Centerline{$\|\phi(x)-y\|\le\gamma\|x-\phi^{-1}(y)\|
<\gamma\theta(x)\le\theta_1\phi(x)$.}

\noindent This shows that $\pmb{t}'$ is $\delta_{\theta_1}$-fine.   We
therefore have

$$\eqalignno{\sum_{(x,C)\in\pmb{t}}|G(C)-J(x)f(\phi(x))\mu C|
&\le\sum_{(x,C)\in\pmb{t}}|F(\phi[C])-f(\phi(x))\mu\phi[C]|\cr
&\hskip 5em +\sum_{(x,C)\in\pmb{t}}|f(\phi(x))||\mu\phi[C]-J(x)\mu C|\cr
&\le\sum_{(x,C)\in\pmb{t}'}|F(C)-f(x)\mu C|\cr
&\hskip 5em
  +\sum_{\Atop{(x,C)\in\pmb{t}}{x\in H'}}
      |f(\phi(x))||\mu\phi[C]-J(x)\mu C|\cr
\displaycause{because if $x\notin H'$ then $f(\phi(x))=0$}
&\le\Bover12\epsilon+\sum_{n=0}^{\infty}
 \sum_{\Atop{(x,C)\in\pmb{t},x\in H'}{n\le\|x\|+|f(\phi(x))|<n+1}}
   \Bover{(n+1)\epsilon\mu C}{2^{n+2}\beta_r(n+2)^r(n+1)}\cr
&\le\Bover12\epsilon+\sum_{n=0}^{\infty}
   \Bover{\epsilon\mu B(\tbf{0},n+2)}
     {2^{n+2}\beta_r(n+2)^r}\cr
\displaycause{remembering that $\theta_2(x)\le 1$, so
$C\subseteq B(\tbf{0},n+2)$ whenever $(x,C)\in\pmb{t}$ and $\|x\|<n+1$}
&=\epsilon.\cr}$$

\noindent As $\pmb{t}$ is arbitrary, this shows that $G$ and
$J\times f\phi$ satisfy (ii) of 484J.

\medskip

\quad{\bf (iii)} Given $\epsilon>0$, there is an $\Cal R\in\frak R$ such
that $|F(C)|\le\epsilon$ for every $C\in\Cal C\cap\Cal R$.   Now by
484Rd there is an $\Cal R'\in\frak R$ such that $\phi[R]\in\Cal R$ for
every $R\in\Cal R'$, so that $|G(C)|\le\epsilon$ for every
$C\in\Cal C\cap\Cal R'$.    Thus $G$ satisfies (iii) of 484J.\ \Qed

\medskip

{\bf (c)} This shows that $J\times f\phi$ is Pfeffer integrable, with
Saks-Henstock indefinite integral $G$;  so, in particular,

\Centerline{$\biggerPfint f\times\phi=G(\BbbR^r)=F(\BbbR^r)=\Pfint f$.}

\medskip

{\bf (d)} Now suppose that $f$ is an arbitrary Pfeffer integrable
function.   In this case set $f_1=f\times\chi\phi[H']$.   Because
$\BbbR^r\setminus H'$ is $\mu$-negligible, so is
$\phi[\BbbR^r\setminus H']$, and $f_1=f\,\,\mu$-a.e.   Also, of course,
$f\phi=f_1\phi\,\,\mu$-a.e.   Because the Pfeffer integral extends the
Lebesgue integral (484He),

\Centerline{$\biggerPfint J\times f\phi=\Pfint J\times f_1\phi
=\Pfint f_1=\Pfint f$.}

\medskip

{\bf (e)} All this has been on the assumption that $f$ is Pfeffer
integrable.   If $g=J\times f\phi$ is Pfeffer integrable, consider
$\tilde J\times g\phi^{-1}$, where $\tilde J(x)=|\det\tilde T(x)|$
whenever the derivative $\tilde T(x)$ of $\phi^{-1}$ at $x$ is defined,
and otherwise is zero.   Now

\Centerline{$\tilde J(x)g(\phi^{-1}(x))
=\tilde J(x)\cdot J(\phi^{-1}(x))f(x)$}

\noindent for every $x$.   But, for $\mu$-almost every $x$,

\Centerline{$\tilde J(x)J(\phi^{-1}(x))
=|\det\tilde T(x)||\det T(\phi^{-1}(x))|
=|\det\tilde T(x)T(\phi^{-1}(x))|
=1$}

\noindent because $\tilde T(x)T(\phi^{-1}(x))$ is (whenever it is
defined) the derivative at $\phi^{-1}(x)$ of the identity function
$\phi^{-1}\phi$, by 473Bc.   (I see that we need to know that
$\{x:\phi$ is differentiable at $\phi^{-1}(x)\}=\phi[H]$ is
conegligible.)   So $\tilde J\times g\phi^{-1}=f\,\mu$-a.e., and
$f$ is Pfeffer integrable.   This completes the proof.
}%end of proof of 484S

\exercises{\leader{484X}{Basic exercises $\pmb{>}$(a)}
%\spheader 484Xa
Show that for every $\Cal R\in\frak R$ there are
$\eta\in\Eta$ and $n\in\Bbb N$ such that
$\Cal R^{(B(\tbf{0},n))}_{\eta}\subseteq\Cal R$.
%484D

\sqheader 484Xb ({\smc Pfeffer 91a}) For $\alpha>0$ let
$\Cal C'_{\alpha}$ be the family of bounded Lebesgue measurable sets $C$
such that
$\mu C\ge\alpha\diam C\per C$.   Show that
$\Cal C_{\sqrt{\alpha}}\subseteq\Cal C'_{\alpha}
\subseteq\Cal C_{\min(\alpha,\alpha^r)}$.
\Hint{474La.}   %if E bdd (\mu E)^{(r-1)/r}\le\per E
%484F

\sqheader 484Xc For $\alpha>0$, let $\Cal C''_{\alpha}$ be the family of
bounded convex sets $C\subseteq\BbbR^r$ such that
$\mu C\ge\alpha(\diam C)^r$.   Show that if $0<\alpha<1/2r$
then $\Cal C''_{\alpha}\subseteq\Cal C_{\alpha}$ ({\it hint\/}: 475T)
and $T_{\alpha}\cap[\BbbR^r\times\Cal C''_{\alpha}]^{<\omega}$
is compatible with $\Delta$ and $\frak R$.
\Hint{use the argument of 484F, but in part (b) take $C=\BbbR^r$, $E=V$
a union of members of $\Cal D$.}
%484F

\spheader 484Xd Describe a suitable filter $\Cal F$ to express the
Pfeffer integral directly in the form considered in 481C.
%484G

\spheader 484Xe Let $f:\BbbR^r\to\Bbb R$ be a Pfeffer integrable
function.   Show that there is some $n\in\Bbb N$ such that
$\int_{\Bbb R^r\setminus B(\tbf{0},n)}|f|d\mu$ is finite.
%484G

\spheader 484Xf\dvAnew{2010} (Here take $r=2$.)
Let $\sequencen{\delta_n}$ be a strictly
decreasing summable sequence in $\ocint{0,1}$.   Define
$f:\BbbR^2\to\Bbb R$ by saying that
$f(x)=\Bover{(-1)^n}{(n+1)(\delta_n^2-\delta_{n+1}^2)}$ if $n\in\Bbb N$ and
$\delta_{n+1}\le\|x\|<\delta_n$, $0$ otherwise.   Show that
$\lim_{\delta\downarrow 0}\int_{\BbbR^2\setminus B(\tbf{0},\delta)}fd\mu$
is defined, but that $f$ is not Pfeffer integrable.
\Hint{484J.}
%484J

\spheader 484Xg\dvAnew{2010} (Again take $r=2$.) Show that there are a
Lebesgue integrable $f_1:\Bbb R\to\Bbb R$ and a Henstock integrable
$f_2:\Bbb R\to\Bbb R$, both with bounded support, such that
$(\xi_1,\xi_2)\mapsto f_1(\xi_1)f_2(\xi_2):\BbbR^2\to\Bbb R$ is not Pfeffer
integrable.
%484Xf 484J mt48bits

\spheader 484Xh Let $E\subseteq\BbbR^r$ be a bounded set with finite
perimeter, and $\phi:\BbbR^r\to\BbbR^r$ a differentiable function.   Let
$v_x$ be the Federer exterior normal to $E$ at any point $x$ where the
normal exists.   Show that $\Pfint\diverg\phi\times\chi E$ is defined
and equal to $\int_{\partstar E}\varinnerprod{\phi(x)}{v_x}\nu(dx)$.
%484N

\spheader 484Xi\dvAnew{2010} Show that there is a Lipschitz function
$f:\BbbR^r\to[0,1]$ such that $\BbbR^r\setminus\dom f'$ is not thin.
\Hint{there is a Lipschitz function $f:\Bbb R\to[0,1]$ not differentiable
at any point of the Cantor set.}
%484N

\leader{484Y}{Further exercises (a)}
%\spheader 484Ya
Let $E\subseteq\BbbR^r$ be any Lebesgue measurable set, and
$\epsilon>0$.   Show that there is a Lebesgue measurable set
$G\subseteq E$ such that $\per G\le\per E$,
$\mu(E\setminus G)\le\epsilon$ and $\clstar G=\overline{G}$.
%484B

\spheader 484Yb
Give an example of a compact set $K\subseteq\BbbR^2$
with zero one-dimensional Hausdorff measure such that whenever
$\theta:K\to\ooint{0,\infty}$ is a strictly positive function, and
$\gamma\in\Bbb R$, there is a disjoint family
$\langle B(x_i,\zeta_i)\rangle_{i\le n}$ of balls such that $x_i\in K$
and $\zeta_i\le\theta(x_i)$ for every $i$, while
$\per(\bigcup_{i\le n}B(x_i,\zeta_i))\ge\gamma$.
%notes %mt48bits
}%end of exercises

\endnotes{
\Notesheader{484}
Listing the properties of the Pfeffer integral as developed above, we have

\inset{expected relations with Lebesgue measure and integration
(484Hd-484Hf);

Saks-Henstock indefinite integrals (484H-484J);  %484Hc 484I 484J

integration over suitable subsets (484L);

a divergence theorem (484N);

a density theorem (484O);

a change-of-variable theorem for lipeomorphisms (484S).}

\noindent The results on indefinite integrals and integration over
subsets are restricted in comparison with
what we have for the Lebesgue integral, since we can deal only with sets
with locally finite perimeter;  and 484S is similarly narrower in scope
than 263D(v).   Pfeffer's Divergence Theorem, on the other hand,
certainly applies to many functions $\phi$ for which $\diverg\phi$ is not
Lebesgue integrable, though it does not entirely cover 475N (see 484Xi).
In comparison with the one-dimensional case, the
Pfeffer integral does not share the most basic property of the special
Denjoy integral (483Bd, 484Xf), but 484N is a step
towards the Perron integral (483J).   484O is a satisfactory rendering of
the idea in 483I, and even for Lebesgue integrable functions adds something
to 261C.
Throughout, I have written on the assumption that $r\ge 2$.
It would be possible to work through the same arguments
with $r=1$, but in this case we should find that `thin' sets became
countable, therefore easily controllable by neighbourhood gauges, making
the methods here inappropriate.

The whole point of `gauge integrals' is that we have an enormous amount
of freedom within the framework of \S\S481-482.   There is a
corresponding difficulty in making definitive choices.   The essential
ideology of the Pfeffer integral is that we take an intersection
of a family of gauge integrals, each determined by a family
$\Cal C_{\alpha}$ of sets which are `Saks regular' in the sense that
their measures, perimeters and diameters are linked (compare 484Xb).
Shrinking
$\Cal C_{\alpha}$ and $T_{\alpha}$, while leaving $\Delta$ and $\frak R$
unchanged, of course leads to a more `powerful' integral (supposing, at
least, that we do not go so far that $T_{\alpha}$ is no longer
compatible with $\Delta$ and $\frak R$), so that Pfeffer's Divergence
Theorem will remain true.   One possibility is to turn to convex sets
(484Xc), though we could not then expect invariance under lipeomorphisms.

The family $\Delta=\{\delta_{\theta}:\theta\in\Theta\}$ of gauges is
designed to permit the exclusion of tags from thin sets;  apart from this
refinement, we are looking at neighbourhood gauges, just as with the
Henstock integral.   This feature, or something like it,
seems to be essential when we come to
the identification $F(E)=\Pfint f\times\chi E$ in 484L, which is demanded
by the formula in the target theorem 484N.   In order to make our
families $T_{\alpha}$ compatible in the sense of 481F, we are
then forced to allow non-trivial residual families;
with some effort (484C, 484Ed), we can
get tagged-partition structures allowing subdivisions (484F).
Note that this is one of the
cases in which our residual families $\Cal R^{(V)}_{\eta}$ are defined by
`shape' as well as by `size'.   In the indefinite-integral characterization
of the Pfeffer integral (484J), we certainly cannot demand `for every
$\epsilon>0$ there is an $\Cal R\in\frak R$ such that $|F(E)|\le\epsilon$
whenever $E\in\Cal C$ is included in a member of $\Cal R$', since all
small balls belong to $\Cal R$, and we should immediately be driven to the
Lebesgue integral.   However I use the construction
$\Cal R^{(V)}_{\eta}=\{R:R\cap V\in\Cal R_{\eta}\}$ (484D) as a quick
method of eliminating any difficulties at infinity (484Xe).   We do not
of course need to look at arbitrary sets $V\in\Cal V$ here (484Xa).

Observe that 484B can be thought of as a refinement of 475I.
As usual, the elaborate formula in the statement of 484C is there only
to emphasize that we have a bound depending only on $l$ and $r$.
Note that 484S depends much more on the fact that the Pfeffer integral can
be characterized in the language of 484J, than on the exact choices made
in forming $\frak R$ and the $\Cal C_{\alpha}$.   For a
discussion of integrals {\it defined} by Saks-Henstock lemmas, see
{\smc Pfeffer 01}.

It would be agreeable to be able to think of the Pfeffer integral as a
product in some sense, so we naturally look for Fubini-type theorems.
I give 484Xg to indicate one of the obstacles.
}%end of notes

\discrversionA{Any
hope of a form of Fubini theorem?  or other relationship between
$\Pfint_{r+s}$ ,  $\Pfint_r$  and  $\Pfint_s$ ?\query
see 484Xg
}{}

\discrpage

