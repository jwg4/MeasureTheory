\frfilename{mt5a5.tex} 
\versiondate{3.10.13} 
\copyrightdate{2007} 
 
\def\chaptername{Appendix} 
\def\sectionname{Real analysis} 
 
\newsection{5A5} 
 
For the sake of an argument in \S534 I sketch a fragment of theory. 
 
\leader{5A5A}{Entire functions} A real function $f$ is {\bf real-analytic}  
if its domain is an open subset $G$ of $\Bbb R$ and for every $a\in G$  
there are a $\delta>0$ and a real sequence $\sequencen{c_n}$ such that  
$f(x)=\sum_{n=0}^{\infty}c_n(x-a)^n$ whenever $|x-a|<\delta$.   It is  
{\bf real-entire} if in addition its domain is the whole of $\Bbb R$. 
 
We need the following facts:  (i) if $f$ and $g$ are real-entire  
functions so is $f-g$;  (ii) if $\sequencen{c_n}$ is a real sequence such  
that $f(x)=\sum_{n=0}^{\infty}c_nx^n$ is defined in  
$\Bbb R$ for every $x\in\Bbb R$, then $f$ is real-entire;   
(iii) if in this expression not every $c_n$ is zero, then every point of  
$F=\{x:x\in\Bbb R$, $f(x)=0\}$ is isolated in $F$, so that $F$ is  
countable.    
If you have done a basic course in complex functions you should recognise  
this.    
If either you missed this out, or you are not sure you understood the  
proof of Cauchy's theorem,  
the following is a sketch of a real-variable argument. 
 
(i) is elementary.    
For (ii), observe first that if $\sequencen{c_nx^n}$ is summable then  
$\lim_{n\to\infty}c_nx^n=0$ so $\sum_{n=0}^{\infty}|c_n|t^n$ is finite  
whenever $0\le t<|x|$.    
In the present case, $\sum_{n=0}^{\infty}|c_n|t^n<\infty$ for every  
$t\ge 0$.   So if $a$, $x\in\Bbb R$,  
 
$$\eqalignno{\sum_{n=0}^{\infty}\sum_{k=0}^n 
  |\Bover{n!}{k!(n-k)!}c_nx^ka^{n-k}| 
&\le\sum_{n=0}^{\infty}\sum_{k=0}^n\Bover{n!}{k!(n-k)!}|c_n|R^n\cr 
\displaycause{where $R=\max(|x|,|a|)$} 
&=\sum_{n=0}^{\infty}|c_n|(2R)^n<\infty.\cr}$$ 
 
\noindent We therefore have 
 
\Centerline{$f(x+a) 
=\sum_{n=0}^{\infty}c_n(x+a)^n
=\sum_{n=0}^{\infty}\sum_{k=0}^n\Bover{n!}{k!(n-k)!}c_nx^ka^{n-k} 
=\sum_{k=0}^{\infty}c_{ak}x^k$}
 
\noindent where $c_{ak}=\sum_{n=k}^{\infty}\Bover{n!}{k!(n-k)!}c_na^{n-k}$  
for each $k$.   Turning this round, 
$f(x)=\sum_{k=0}^{\infty}c_{ak}(x-a)^k$ for every $x$.   This shows that  
$f$ is real-entire.   As for (iii), 
if not every $c_n$ is zero, there must be  
some neighbourhood of $0$ in which the first non-zero term $c_nx^n$  
dominates, so $f$ is not identically zero.   (The point is that 
$\sum_{k=0}^{\infty}|c_k|<\infty$, so there is some $\delta>0$ such that 
$\sum_{k=n+1}^{\infty}|c_k\delta^{k+1}|<|c_n\delta^n|$.)  
In this case, if $a\in\Bbb R$, not every  
$c_{ak}$ can be zero, and there must be some neighbourhood of $a$ in which  
the first non-zero term $c_{ak}(x-a)^k$ dominates, so that there can be no  
zeroes of $f$ in that neighbourhood except perhaps $a$ itself. 
 
\discrpage 
 
 
 
