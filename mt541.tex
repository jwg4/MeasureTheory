\frfilename{mt541.tex}
\versiondate{10.12.12}
\copyrightdate{2004}

\def\chaptername{Real-valued-measurable cardinals}
\def\sectionname{Saturated ideals}

\def\covSh{\cov_{\text{Sh}}}
\def\Esic{$\exists$sic}
\def\Ewic{$\exists$wic}
\def\Tr{\mathop{\text{Tr}}}

\newsection{541}

If $\nu$ is a totally finite measure with domain $\Cal PX$ and null
ideal $\Cal N(\nu)$, then its measure algebra $\Cal PX/\Cal N(\nu)$ is
ccc, that is to say, $\sat(\Cal PX/\Cal N(\nu))\le\omega_1$;  while the
additivity of $\Cal N(\nu)$ is at least $\omega_1$.   It turns out that
an ideal $\Cal I$ of $\Cal PX$ such that
$\sat(\Cal PX/\Cal I)\le\add\Cal I$ is either trivial or extraordinary.
In this section I present a little of the theory of such ideals.   To
begin with, the quotient algebra has to be Dedekind complete (541B).
Further elementary ideas are in 541C (based on a method already used in
\S525) and 541D-541E.   In a less expected direction, we have a useful
fact concerning transversal numbers $\Tr_{\Cal I}(X;Y)$ (541F).

The most remarkable properties of saturated ideals arise because of
their connexions with `normal' ideals (541G).   These ideals share the
properties of non-stationary ideals (541H-541I).   If $\Cal I$ is an
$(\add\Cal I)$-saturated ideal of $\Cal PX$, we have corresponding
normal ideals on $\add\Cal I$ (541J).   Now there can be a
$\kappa$-saturated normal ideal on $\kappa$ only if there is a great
complexity of cardinals below $\kappa$ (541L).

The original expression of these ideas ({\smc Keisler \& Tarski 64})
concerned `\2vm' cardinals, on which we have normal ultrafilters (541M).
The dichotomy of {\smc Ulam 30}\cmmnt{ (438Ce-438Cf)} reappears in the
context of $\kappa$-saturated normal ideals (541P).   For
$\kappa$-saturated ideals, `normality' implies some far-reaching
extensions (541Q).   Finally, I include a technical lemma concerning the
covering numbers $\covSh(\alpha,\beta,\gamma,\delta)$ (541S).

\leader{541A}{Definition} If $\frak A$ is a Boolean algebra, $I$ is an
ideal of $\frak A$ and $\kappa$ is a cardinal, I will say that $I$ is
{\bf $\kappa$-saturated} in $\frak A$ if
$\kappa\ge\sat(\frak A/I)$\cmmnt{;  that is, if for every family
$\ofamily{\xi}{\kappa}{a_{\xi}}$ in $\frak A\setminus I$ there are
distinct $\xi$, $\eta<\kappa$ such that
$a_{\xi}\Bcap a_{\eta}\notin I$.}

\leader{541B}{Proposition} Let $\frak A$ be a Dedekind complete Boolean
algebra and $I$ an ideal of $\frak A$ which is $(\add I)^+$-saturated in
$\frak A$.   Then the quotient algebra $\frak A/I$ is Dedekind complete.

\proof{ Take any $B\subseteq\frak A/I$.   Let $C$ be the set of those
$c\in\frak A/I$ such that {\it either} $c\Bcap b=0$ for every $b\in B$
{\it or} there is a $b\in B$ such that $c\Bsubseteq b$.   Then $C$ is
order-dense in $\frak A/I$, so there is a partition of unity
$D\subseteq C$ (313K).   Enumerate $D$ as
$\ofamily{\xi}{\kappa}{d_{\xi}}$, where

\Centerline{$\kappa=\#(D)<\sat(\frak A/I)\le(\add I)^+$.}

\noindent For each $\xi<\kappa$, choose $a_{\xi}\in\frak A$ such that
$a_{\xi}^{\ssbullet}=d_{\xi}$.   Now $\#(\xi)<\kappa\le\add I$, so
$\sup_{\eta<\xi}a_{\xi}\Bcap a_{\eta}\in I$.   Set
$\tilde a_{\xi}=a_{\xi}\Bsetminus\sup_{\eta<\xi}a_{\eta}$;  then
$\tilde a_{\xi}^{\ssbullet}=a_{\xi}^{\ssbullet}=d_{\xi}$.   Set
$L=\{\xi:\xi<\kappa$, $d_{\xi}\Bsubseteq b$ for some $b\in B\}$ and
$a=\sup_{\xi\in L}\tilde a_{\xi}$ in $\frak A$.

\Quer\ If $b\in B$ and $b\notBsubseteq a^{\ssbullet}$, there must be a
$\xi<\kappa$ such that $d_{\xi}\Bcap(b\setminus a^{\ssbullet})\ne 0$.
But $d_{\xi}\in C$, so there must be a $b'\in B$ such that
$d_{\xi}\Bsubseteq b'$;  accordingly $\xi\in L$,
$\tilde a_{\xi}\Bsubseteq a$
and $d_{\xi}\Bsubseteq a^{\ssbullet}$.\ \BanG\  Thus $a^{\ssbullet}$ is
an upper bound of $B$ in $\frak A/I$.

\Quer\ If there is an upper bound $b^*$ of $B$ such that
$a^{\ssbullet}\notBsubseteq b^*$, there must be a $\xi<\kappa$ such that
$d_{\xi}\Bcap a^{\ssbullet}\Bsetminus b^*\ne 0$.   As
$d_{\xi}\notBsubseteq b^*$, $d_{\xi}\notBsubseteq b$ for every $b\in B$,
and $\xi\notin L$.   But this means that
$\tilde a_{\xi}\Bcap\tilde a_{\eta}=0$ for every $\eta\in L$, so
$\tilde a_{\xi}\Bcap a=0$ (313Ba) and $d_{\xi}\Bcap a^{\ssbullet}=0$.\
\Bang

Thus $a^{\ssbullet}=\sup B$ in $\frak A/I$;  as $B$ is arbitrary,
$\frak A/I$ is Dedekind complete.
}%end of proof of 541B

\leader{541C}{Proposition} Let $X$ be a set, $\kappa$ a regular infinite
cardinal, $\Sigma$ an algebra of subsets of $X$ such that
$\bigcup\Cal E\in\Sigma$ whenever $\Cal E\subseteq\Sigma$ and
$\#(\Cal E)<\kappa$, and $\Cal I$ a $\kappa$-saturated $\kappa$-additive
ideal of $\Sigma$.

(a) If $\Cal E\subseteq\Sigma$ there is an
$\Cal E'\in[\Cal E]^{<\kappa}$ such that
$E\setminus\bigcup\Cal E'\in\Cal I$ for every $E\in\Cal E$.

(b) If $\ofamily{\xi}{\kappa}{E_{\xi}}$ is any family
in $\Sigma\setminus\Cal I$, and $\theta<\kappa$ is a cardinal, then
$\{x:x\in X$, $\#(\{\xi:x\in E_{\xi}\})\ge\theta\}$ includes a member of
$\Sigma\setminus\Cal I$\cmmnt{ (and, in particular, is not empty)}.

(c) Suppose that no element of $\Sigma\setminus\Cal I$ can be covered by
$\kappa$ members of $\Cal I$.   Then $\kappa$ is a precaliber of
$\Sigma/\Cal I$.

\proof{ Write $\frak A$ for $\Sigma/\Cal I$.

\medskip

{\bf (a)} Consider
$A=\{E^{\ssbullet}:E\in\Cal E\}\subseteq\frak A$.   By 514Db, there is
an $\Cal E'\in[\Cal E]^{<\sat(\frak A)}$ such that
$\{E^{\ssbullet}:E\in\Cal E'\}$ has the same upper bounds as $A$.
Now $\#(\Cal E')<\sat(\frak A)\le\kappa$, so
$F=\bigcup\Cal E'$ belongs to $\Sigma$, and $F^{\ssbullet}$ must be an
upper bound for $A$, that is, $E\setminus F\in\Cal I$ for every
$E\in\Cal E$.

\medskip

{\bf (b)} For $\alpha\le\beta<\kappa$ set
$F_{\alpha\beta}=\bigcup_{\alpha\le\xi<\beta}E_{\xi}\in\Sigma$.
Then for every $\alpha<\kappa$ we have a $g(\alpha)<\kappa$ such that
$g(\alpha)\ge\alpha$ and
$E_{\xi}\setminus F_{\alpha,g(\alpha)}\in\Cal I$ whenever
$g(\alpha)\le\xi<\kappa$, by (a) and the regularity of $\kappa$.

Define $\ofamily{\alpha}{\kappa}{h(\alpha)}$ by setting $h(0)=0$,
$h(\alpha+1)=g(h(\alpha))$ for each $\alpha<\kappa$, and
$h(\alpha)=\sup_{\beta<\alpha}h(\beta)$ for non-zero limit ordinals
$\alpha<\kappa$.   Set $G_{\alpha}=F_{h(\alpha),h(\alpha+1)}$ for each
$\alpha$.   If $\beta<\alpha$, then

\Centerline{$G_{\alpha}\setminus G_{\beta}
=\bigcup_{h(\alpha)\le\xi<h(\alpha+1)}E_{\xi}
  \setminus F_{h(\beta),g(h(\beta))}\in\Cal I$}

\noindent because $\Cal I$ is $\kappa$-additive.   Consequently
$G_{\theta}\setminus\bigcap_{\beta<\theta}G_{\beta}$ belongs to
$\Cal I$;  and $G_{\theta}\supseteq E_{h(\theta)}\notin\Cal I$, so
$G=\bigcap_{\beta<\theta}G_{\beta}\in\Sigma\setminus\Cal I$.   But if
$x\in G$ then $\{\xi:x\in E_{\xi}\}$ meets $\coint{h(\beta),h(\beta+1)}$
for every $\beta<\theta$ and has cardinal at least $\theta$.

\medskip

{\bf (c)} Let $\ofamily{\xi}{\kappa}{a_{\xi}}$ be a family of non-zero
elements in $\frak A$.   For each $\xi<\kappa$, choose
$\tilde E_{\xi}\in\Sigma$ such that
$\tilde E_{\xi}^{\ssbullet}=a_{\xi}$.   Let
$\Cal K$ be the family of all those finite subsets $K$ of $\kappa$ such
that $H_K=\bigcap_{\xi\in K}\tilde E_{\xi}$ belongs to $\Cal I$.
Now set $E_{\xi}
=\tilde E_{\xi}\setminus\bigcup\{H_K:K\in\Cal K$, $K\subseteq\xi\}$;
then $E_{\xi}\in\Sigma\setminus\Cal I$ and $E_{\xi}^{\ssbullet}=a_{\xi}$
for each $\xi<\kappa$.

Repeat the argument of (b).   Once again we get a family
$\ofamily{\alpha}{\kappa}{G_{\alpha}}$ in $\Sigma\setminus\Cal I$ such
that $G_{\alpha}\setminus G_{\beta}\in\Cal I$ whenever
$\beta\le\alpha<\kappa$.   Now, applying (a) to
$\ofamily{\alpha}{\kappa}{X\setminus G_{\alpha}}$, we have a
$\gamma<\kappa$ such that
$\bigcap_{\alpha<\gamma}G_{\alpha}\setminus G_{\beta}\in\Cal I$ for
every $\beta<\kappa$.   On the other hand,
$G_{\gamma}\setminus G_{\alpha}\in\Cal I$ for every $\alpha<\gamma$, so
in fact $G_{\gamma}\setminus G_{\beta}\in\Cal I$ for every
$\beta<\kappa$, while $G_{\gamma}\notin\Cal I$.   At this point, recall
that we are now assuming that $G_{\gamma}$ cannot be covered by
$\bigcup_{\beta<\kappa}G_{\gamma}\setminus G_{\beta}$, and there is an
$x\in\bigcap_{\beta<\kappa}G_{\beta}$.

As in (b), it follows that $\Gamma=\{\xi:\xi<\kappa$, $x\in E_{\xi}\}$
has cardinal $\kappa$.   If $K\subseteq\Gamma$ is finite and not empty,
take $\zeta\in\Gamma$ such that $K\subseteq\zeta$.   Then

\Centerline{$x\in E_{\zeta}\cap\bigcap_{\xi\in K}\tilde E_{\xi}
\subseteq E_{\zeta}\cap H_K$;}

\noindent it follows that $K\notin\Cal K$ and $H_K\notin\Cal I$,
that is, that $\inf_{\xi\in K}a_{\xi}\ne 0$ in
$\frak A$.   So $\family{\xi}{\Gamma}{a_{\xi}}$ is centered;  as
$\ofamily{\xi}{\kappa}{a_{\xi}}$ is arbitrary, $\kappa$ is a precaliber
of $\frak A$.
}%end of proof of 541C

\leader{541D}{Lemma} %RVMC 5A
Let $X$ be a set, $\Cal I$ an ideal of $\Cal PX$, $Y$ a set of
cardinal less than $\add\Cal I$ and $\kappa$ a cardinal such that
$\Cal I$ is $(\cf\kappa)$-saturated in $\Cal PX$.   Then for any
function $f:X\to[Y]^{<\kappa}$ there is an
$M\in[Y]^{<\kappa}$ such that $\{x:f(x)\not\subseteq M\}\in\Cal I$.
%do we really need quite such an elaborate version? maybe enough to have
%regular $\kappa$

\proof{ If $\kappa>\#(Y)$ this is trivial;
suppose that $\kappa\le\#(Y)<\add\Cal I$.   For cardinals
$\lambda<\kappa$ set $X_{\lambda}=\{x:\#(f(x))=\lambda\}$.   If
$A=\{\lambda:X_{\lambda}\notin\Cal I\}$ then $\Cal I$ is not
$\#(A)$-saturated in $\Cal PX$, so $\#(A)<\cf\kappa$ and
$\theta=\sup A$ is less than $\kappa$.   Set
$X'=\{x:\#(f(x))\le\theta\}$;  then
$X\setminus X'$ is the union of at most $\lambda<\add\Cal I$ members of
$\Cal I$, so belongs to $\Cal I$.

For each $x\in X'$ let
$\langle h_{\xi}(x)\rangle_{\xi<\theta}$
run over a set including $f(x)$.   For each $\xi<\theta$,

\Centerline{$Y_{\xi}=\{y:h_{\xi}^{-1}[\{y\}]\notin\Cal I\}$}

\noindent has cardinal less than $\cf\kappa$, and because
$\#(Y)<\add\Cal I$,
$h_{\xi}^{-1}[Y\setminus Y_{\xi}]\in\Cal I$.   Set
$M=\bigcup_{\xi<\theta}Y_{\xi}\in[Y]^{<\kappa}$.   (If $\kappa$ is
regular, $M$ is the union of fewer than $\kappa$ sets of size less than
$\kappa$, so $\#(M)<\kappa$;  if $\kappa$ is not regular, then
$M$ is the union of fewer than $\kappa$ sets of size at most
$\cf\kappa$, so again $\#(M)<\kappa$.)
Because $\theta<\add\Cal I$,

\Centerline{$\{x:f(x)\not\subseteq M\}
\subseteq(X\setminus X')
  \cup\bigcup_{\xi<\theta}h_{\xi}^{-1}[Y\setminus Y_{\xi}]
\in\Cal I$,}

\noindent as required.
}%end of proof of 541D

\leader{541E}{Corollary} Let $X$ be a set, $\Cal I$ an ideal of
$\Cal PX$, $Y$ a set of
cardinal less than $\add\Cal I$ and $\kappa$ a cardinal such that
$\Cal I$ is $(\cf\kappa)$-saturated in $\Cal PX$.   Then for any
function $g:X\to Y$ there is an
$M\in[Y]^{<\kappa}$ such that $g^{-1}[Y\setminus M]\in\Cal I$.

\proof{ Apply 541D to the function $x\mapsto\{g(x)\}$.   (In the trivial
case $\kappa=1$, $\Cal I=\Cal PX$.)
}%end of proof of 541E

\leader{541F}{Lemma} Let $X$ and $Y$ be sets, $\kappa$ a regular 
uncountable cardinal, and $\Cal I$ a proper
$\kappa$-saturated $\kappa$-additive ideal of subsets of $X$.
Then $\Tr_{\Cal I}(X;Y)$\cmmnt{ (definition:  5A1La)} is
attained, in the sense that there is a set $G\subseteq Y^X$ such
that $\#(G)=\Tr_{\Cal I}(X;Y)$ and
$\{x:x\in X$, $g(x)=g'(x)\}\in\Cal I$ for all distinct $g$,
$g'\in G$.

\proof{ It is enough to consider the case in which
$Y=\lambda$ is a cardinal.   Set $\theta=\Tr_{\Cal I}(X;\lambda)$.

\medskip

{\bf (a)} If $\lambda^+<\kappa$ then
$\theta\le\lambda$.   \Prf\Quer\ Suppose, if possible,
that we have a family $\ofamily{\xi}{\lambda^+}{f_{\xi}}$ in $\lambda^X$
such that $\{x:f_{\xi}(x)=f_{\eta}(x)\}\in\Cal I$ whenever
$\eta<\xi<\lambda^+$.
Then $\lambda$ is surely infinite, so $\lambda^+$ is uncountable and
regular.   For each $x\in X$ there is an $\alpha_x<\lambda^+$ such that
$\{f_{\xi}(x):\xi<\alpha_x\}=\{f_{\xi}(x):\xi<\lambda^+\}$.
Setting

\Centerline{$F_{\alpha}=\{x:x\in X$, $\alpha_x=\alpha\}
\subseteq\bigcup_{\eta<\alpha}\{x:f_{\eta}(x)=f_{\alpha}(x)\}$,}

\noindent we see that $F_{\alpha}\in\Cal I$ for each $\alpha<\lambda^+$.
But $X=\bigcup_{\alpha<\lambda^+}F_{\alpha}$ and $\lambda^+<\kappa$, so
this is impossible.\ \Bang\Qed

Since we can surely find a family $\ofamily{\xi}{\lambda}{f_{\xi}}$ in
$\lambda^X$ such that $f_{\xi}(x)\ne f_{\eta}(x)$ whenever $x\in X$ and
$\eta<\xi<\lambda$, we have the result when $\lambda^+<\kappa$.

\medskip

{\bf (b)} We may therefore suppose from now on that
$\lambda^+\ge\kappa$.

If $H\subseteq\lambda^X$ is such that

\Centerline{$F=\{f:f\in\lambda^X$, $\{x:f(x)\le h(x)\}\in\Cal I$ for
every $h\in H\}\ne\emptyset$,}

\noindent then there is an $f_0\in F$ such that

\Centerline{$\{x:f(x)<f_0(x)\}\in\Cal I$ for every $f\in F$.}

\noindent\Prf\Quer\ If not, choose a family
$\langle f_{\xi} \rangle_{\xi<\kappa}$ in $F$ inductively, as follows.
$f_0$ is to be any member of $F$.   Given $f_{\xi}$, there is an
$f\in F$ such that $\{x:f(x)<f_{\xi}(x)\}\notin\Cal I$;  set
$f_{\xi+1}(x)=\min(f(x),f_{\xi}(x))$ for every $x$;  then
$f_{\xi+1}\in F$.   Given that $f_{\eta}\in F$ for every
$\eta<\xi$, where $\xi<\kappa$ is a non-zero limit
ordinal, set $f_{\xi}(x)=\min_{\eta<\xi}f_{\eta}(x)$ for each $x$;
then for any $h\in H$ we shall have

\Centerline{$\{x:f_{\xi}(x)\le h(x)\}
=\bigcup_{\eta<\xi}\{x:f_{\eta}(x)\le h(x)\}\in\Cal I$,}

\noindent so $f_{\xi}\in F$ and the induction continues.

Now consider

\Centerline{$E_{\xi}=\{x:f_{\xi+1}(x)<f_{\xi}(x)\}
\in\Cal PX\setminus\Cal I$}

\noindent for $\xi<\kappa$.   By 541Cb there is an $x\in X$ such that
$A=\{\xi:x\in E_{\xi}\}$ is infinite.   But if
$\langle\xi(n)\rangle_{n\in \Bbb N}$ is any strictly increasing sequence
in $A$, $\langle f_{\xi(n)}(x)\rangle_{n\in\Bbb N}$ is a strictly
decreasing sequence of ordinals, which is impossible.\ \Bang\Qed

\medskip

{\bf (c)} Choose a family
$\langle g_{\xi}\rangle_{\xi<\delta}$ in $\lambda^X$ as follows.   Given
$\langle g_{\eta}\rangle_{\eta<\xi}$, set

\Centerline{$F_{\xi}=\{f:f\in\lambda^X$,
$\{x:f(x)\le g_{\eta}(x)\}\in\Cal I$ for every $\eta<\xi\}$.}

\noindent If $F_{\xi}=\emptyset$, set $\delta=\xi$ and stop.
If $F_{\xi}\ne\emptyset$ use (b) to find $g_{\xi}\in F_{\xi}$ such that
$\{x:f(x)<g_{\xi}(x)\}\in\Cal I$ for every $f\in F_{\xi}$,
and continue.   Note that for $\xi<\min(\lambda,\kappa)$,
$\{x:g_{\xi}(x)\ne \xi\}\in\Cal I$.   \Prf\ Induce on $\xi$.   If
$\xi<\min(\lambda,\kappa)$ and $\{x:g_{\eta}(x)\ne\eta\}\in\Cal I$ for
every $\eta<\xi$, then the constant function with value $\xi$ belongs to
$F_{\xi}$, so $g_{\xi}$ is defined and
$\{x:g_{\xi}(x)>\xi\}\in\Cal I$.   On the other hand,
$\{x:g_{\xi}(x)=\eta\}\in\Cal I$ for $\eta<\xi$;
as $\xi<\add\Cal I$, $\{x:g_{\xi}(x)<\xi\}\in\Cal I$.\ \QeD\  Accordingly
$\delta\ge\min(\lambda,\kappa)$.

\medskip

{\bf (d)} Because $g_{\xi}\in F_{\xi}$,
$\{x:g_{\xi}(x)=g_{\eta}(x)\}\in\Cal I$ whenever $\eta<\xi<\delta$, so
$\#(\delta)\le\theta$.   On the other hand, suppose that
$F\subseteq\lambda^X$ is such that $\{x:f(x)=f'(x)\}\in\Cal I$ for all
distinct $f$, $f'\in F$.   For each $f\in F$, set

\Centerline{$\zeta'_f=\min\{\xi:\xi\le\delta$, $f\notin F_{\xi}\};$}

\noindent this must be defined because $F_{\delta}=\emptyset$.
Also $F_0=\lambda^X$ and $F_{\xi}=\bigcap_{\eta<\xi}F_{\eta}$
if $\xi\le\delta$ is a non-zero limit ordinal, so $\zeta'_f$ must
be a successor ordinal;  let $\zeta_f$ be its predecessor.
We have $f\in F_{\zeta_f}$ and

\Centerline{$\{x:f(x)<g_{\zeta_f}(x)\}\in\Cal I$,
\quad$\{x:f(x)\le g_{\zeta_f}(x)\}\notin\Cal I$,}

\noindent so that

\Centerline{$E_f=\{x:f(x)=g_{\zeta_f}(x)\}\notin\Cal I$.}

If $f$, $f'$ are distinct members of $F$ and $\zeta_f=\zeta_{f'}$,
then $E_f\cap E_{f'}\in\Cal I$.   So

\Centerline{$\{f:f\in F$, $\zeta_f=\zeta\}$}

\noindent must have cardinal less than $\kappa$ for every
$\zeta<\delta$.

If $\kappa=\lambda^+$, $\#(\{f:f\in F$, $\zeta_f=\zeta\})\le\lambda$ for
every $\zeta<\delta$, so $\#(F)\le\max(\delta,\lambda)=\delta$.   On the
other hand, if $\kappa\le\lambda$, then
$\#(F)\le\max(\delta,\kappa)=\delta$.   As $F$ is arbitrary,
$\theta=\delta$ and we may take $G=\{g_{\xi}:\xi<\delta\}$ as our
witness that $\Tr_{\Cal I}(X;\lambda)$ is attained.
}%end of proof of 541F

\leader{541G}{Definition} Let $\kappa$ be a regular uncountable
cardinal.   A {\bf normal ideal} on $\kappa$ is a proper ideal $\Cal I$
of $\Cal P\kappa$, including $[\kappa]^{<\kappa}$, such that

\Centerline{$\{\xi:\xi<\kappa$, $\xi\in\bigcup_{\eta<\xi}I_{\eta}\}$}

\noindent belongs to $\Cal I$ for every family
$\ofamily{\xi}{\kappa}{I_{\xi}}$ in $\Cal I$.
\cmmnt{It is easy to check that a proper ideal $\Cal I$ of
$\Cal P\kappa$ is
normal iff the dual filter $\{\kappa\setminus I:I\in\Cal I\}$ is normal
in the sense of 4A1Ic.}

\leader{541H}{Proposition} Let $\kappa$ be a regular uncountable
cardinal and $\Cal I$ a proper ideal of $\Cal P\kappa$ including
$[\kappa]^{<\kappa}$.   Then the following are equiveridical:

(i) $\Cal I$ is normal;

(ii) $\Cal I$ is $\kappa$-additive and whenever
$S\in\Cal P\kappa\setminus\Cal I$ and $f:S\to\kappa$ is regressive, then
there is an $\alpha<\kappa$ such that
$\{\xi:\xi\in S$, $f(\xi)\le\alpha\}$ is not in $\Cal I$;

(iii) whenever
$S\in\Cal P\kappa\setminus\Cal I$ and $f:S\to\kappa$ is regressive, then
there is a $\beta<\kappa$ such that
$\{\xi:\xi\in S$, $f(\xi)=\beta\}$ is not in $\Cal I$.

\proof{{\bf (i)$\Rightarrow$(ii)} Suppose that $\Cal I$ is normal.

\medskip

\quad\grheada\ (Cf.\ 4A1J.)  Suppose that
$\ofamily{\eta}{\alpha}{I_{\eta}}$ is a
family in $\Cal I$, where $0<\alpha<\kappa$, and
$I=\bigcup_{\eta<\alpha}I_{\eta}$.
Then $I\setminus\alpha\subseteq\{\xi:\xi\in\bigcup_{\eta<\xi}I_{\eta}\}$
belongs to $\Cal I$;  as $\alpha\in\Cal I$, $I\in\Cal I$;  as
$\ofamily{\eta}{\alpha}{I_{\eta}}$ is arbitrary, $\Cal I$ is
$\kappa$-additive.

\medskip

\quad\grheadb\ Take $S$ and $f$ as in (ii).   \Quer\ If
$I_{\alpha}=\{\xi:\xi\in S$, $f(\xi)\le\alpha\}$ belongs to $\Cal I$ for
every $\alpha$, then
$I=\{\xi:\xi<\kappa$, $\xi\in\bigcup_{\alpha<\xi}I_{\alpha}\}$ belongs
to $\Cal I$.   But if $\xi\in S$ then $f(\xi)<\xi$ and
$\xi\in I_{f(\xi)}$, so $S\subseteq I$.\ \BanG\  As $S$ and $f$ are
arbitrary, (ii) is true.

\medskip

{\bf (ii)$\Rightarrow$(iii)} Suppose (ii) is true and that $S$, $f$ are
as in (iii).   By (ii), there is an $\alpha<\kappa$ such that
$\{\xi:\xi\in S$, $f(\xi)\le\alpha\}\notin\Cal I$.   As $\Cal I$ is
$\kappa$-additive, there is a $\beta\le\alpha$ such that
$\{\xi:\xi\in S$, $f(\xi)=\beta\}\notin\Cal I$.   As $S$ and $f$ are
arbitrary, (iii) is true.

\medskip

{\bf (iii)$\Rightarrow$(i)}
Now suppose that (iii) is true, and that
$\ofamily{\xi}{\kappa}{I_{\xi}}$ is any family in $\Cal I$;  set
$S=\{\xi:\xi<\kappa$, $\xi\in\bigcup_{\eta<\xi}I_{\eta}\}$.   Then we
have a regressive function $f:S\to\kappa$ such that $\xi\in I_{f(\xi)}$
for every $\xi\in S$.   Since
$\{\xi:\xi\in S$, $f(\xi)=\beta\}\subseteq I_{\beta}\in\Cal I$ for every
$\beta<\kappa$, (iii) tells us that $S\in\Cal I$.
Since we are assuming that $\Cal I$ is a proper ideal including
$[\kappa]^{<\kappa}$, it is normal.
}%end of proof of 541H

\leader{541I}{Lemma} Let $\kappa$ be a regular uncountable cardinal.

(a) The family of non-stationary subsets of $\kappa$ is a normal ideal
on $\kappa$, and is included in every normal ideal on $\kappa$.

(b) If $\Cal I$ is a normal ideal on $\kappa$, and
$\family{K}{[\kappa]^{<\omega}}{I_K}$ is any family in $\Cal I$, then
$\{\xi:\xi<\kappa$, $\xi\in\bigcup_{K\in[\xi]^{<\omega}}I_K\}$ belongs
to $\Cal I$.

\proof{{\bf (a)} Let $\Cal I$ be the family of non-stationary subsets of
$\kappa$.

\medskip

\quad{\bf (i)} Since a subset of $\kappa$ is non-stationary iff it is
disjoint from some closed cofinal set (4A1Ca), any subset of a
non-stationary set is non-stationary.   Because the intersection of two
closed cofinal sets is again a closed cofinal set (4A1Bd), $\Cal I$ is
an ideal.   Because $\kappa\setminus\xi$ is a closed cofinal set for any
$\xi<\kappa$, and $\kappa$ is regular,
$[\kappa]^{<\kappa}\subseteq\Cal I$.

Now suppose that $\ofamily{\xi}{\kappa}{I_{\xi}}$ is any family in
$\Cal I$, and that
$I=\{\xi:\xi<\kappa$, $\xi\in\bigcup_{\eta<\xi}I_{\eta}\}$.   For each
$\xi<\kappa$ let $F_{\xi}$ be a closed cofinal subset of $\kappa$
disjoint from $I_{\xi}$, and let $F$ be the diagonal intersection of
$\ofamily{\xi}{\kappa}{F_{\xi}}$;  then $F$ is a closed cofinal set
(4A1B(c-ii)), and it is easy to check that $F$ is disjoint from $I$, so
$I\in\Cal I$.   Thus $\Cal I$ is normal.

\medskip

\quad{\bf (ii)} Let $\Cal J$ be any normal ideal on $\kappa$.   If
$F\subseteq\kappa$ is a closed cofinal set containing $0$, we have a
regressive function $f:\kappa\setminus F\to F$ defined by setting
$f(\xi)=\sup(F\cap\xi)$ for every $\xi\in\kappa\setminus F$.   If
$\alpha<\kappa$, $\{\xi:f(\xi)\le\alpha\}$ is bounded above by
$\min(F\setminus\alpha)$ so belongs to
$[\kappa]^{<\kappa}\subseteq\Cal J$;  by 541H(ii), $\kappa\setminus F$
must belong to $\Cal J$.   This works for any closed cofinal set
containing $0$;  but as $\{0\}$ surely belongs to $J$,
$\kappa\setminus F\in\Cal J$ for every closed cofinal set $F$, that is,
$\Cal I\subseteq\Cal J$.

\medskip

{\bf (b)} Set $J_{\xi}=\bigcup_{K\in[\xi+1]^{<\omega}}I_K$;  because
$\Cal I$ is $\kappa$-additive, $J_{\xi}\in\Cal I$ for each $\xi$.   Now

\Centerline{$\{\xi:\xi<\kappa$,
  $\xi\in\bigcup_{K\in[\xi]^{<\omega}}I_K\}
=\{\xi:\xi<\kappa$, $\xi\in\bigcup_{\eta<\xi}J_{\eta}\}\in\Cal I$}

\noindent because $\Cal I$ is normal.
}%end of proof of 541I

\leader{541J}{Theorem}\cmmnt{ ({\smc Solovay 71})} Let $X$ be a set
and $\Cal J$ an ideal of
subsets of $X$.   Suppose that $\add\Cal J=\kappa>\omega$ and that
$\Cal J$ is $\lambda$-saturated in $\Cal PX$, where $\lambda\le\kappa$.
Then there are $Y\subseteq X$ and $g:Y\to\kappa$ such that
$\{B:B\subseteq\kappa$, $g^{-1}[B]\in\Cal J\}$ is a $\lambda$-saturated
normal ideal on $\kappa$.

\proof{ (Cf.\ 4A1K.) Let $\ofamily{\xi}{\kappa}{J_{\xi}}$ be a family in
$\Cal J$
such that $Y=\bigcup_{\xi<\kappa}J_{\xi}\notin\Cal J$.   Let $F$ be the
set of functions $f:Y\to\kappa$ such that $f^{-1}[\alpha]\in\Cal J$ for
every $\alpha<\kappa$.   Set $f_0(y)=\min\{\xi:y\in J_{\xi}\}$ for
$y\in Y$;  then $f_0\in F$.   \Prf\ If $\alpha<\kappa$, then
$f^{-1}[\alpha]=\bigcup_{\xi<\alpha}J_{\xi}$ belongs to $\Cal J$ because
$\Cal J$ is $\kappa$-additive.\ \Qed

The point is that there is a $g\in F$ such that
$\{y:y\in Y$, $f(y)<g(y)\}\in\Cal J$ for every $f\in F$.   \Prf\Quer\
Otherwise, choose $f_{\xi}$, for $0<\xi<\kappa$, as follows.   Given
$f_{\xi}\in F$, where $\xi<\kappa$, there is an $f\in F$ such that
$A_{\xi}=\{y:f(y)<f_{\xi}(y)\}\notin\Cal J$;  set
$f_{\xi+1}(y)=\min(f(y),f_{\xi}(y))$ for every $y$.   Then

\Centerline{$f_{\xi+1}^{-1}[\alpha]
=f^{-1}[\alpha]\cup f_{\xi}^{-1}[\alpha]\in\Cal J$}

\noindent for every $\alpha<\kappa$, so $f_{\xi+1}\in F$.   Given that
$f_{\eta}\in F$ for every $\eta<\xi$, where $\xi<\kappa$ is a
non-zero limit ordinal, set $f_{\xi}(y)=\min\{f_{\eta}(y):\eta<\xi\}$
for each $y\in Y$;  then

\Centerline{$f_{\xi}^{-1}[\alpha]
=\bigcup_{\eta<\xi}f_{\eta}^{-1}[\alpha]\in\Cal J$}

\noindent for every $\alpha<\kappa$, because
$\#(\xi)<\kappa=\add\Cal J$.

This construction ensures that $\ofamily{\xi}{\kappa}{f_{\xi}(y)}$ is
non-increasing for every $y$, and that
$\{y:f_{\xi+1}(y)<f_{\xi}(y)\}=A_{\xi}\notin\Cal J$ for every
$\xi<\kappa$.   But as $\Cal J$ is $\kappa$-saturated in $\Cal PX$,
there must be a point $y$ belonging to infinitely many $A_{\xi}$ (541Cb),
so that there is a strictly decreasing sequence in
$\{f_{\xi}(y):\xi<\kappa\}$, which is impossible.\ \Bang\Qed

Now consider $\Cal I=\{B:B\subseteq\kappa$, $g^{-1}[B]\in\Cal J\}$.
Because $\Cal J$ is $\lambda$-saturated in $\Cal PX$, $\Cal I$ is
$\lambda$-saturated in $\Cal P\kappa$.   \Prf\ If
$\ofamily{\xi}{\lambda}{B_{\xi}}$ is a family in
$\Cal P\kappa\setminus\Cal I$, then
$\ofamily{\xi}{\lambda}{g^{-1}[B_{\xi}]}$  is a family in
$\Cal PX\setminus\Cal J$, so there are distinct $\xi$, $\eta<\lambda$
such that
$g^{-1}[B_{\xi}\cap B_{\eta}]=g^{-1}[B_{\xi}]\cap g^{-1}[B_{\eta}]$ does
not belong to $\Cal J$, and $B_{\xi}\cap B_{\eta}$ does not belong to
$\Cal I$.\ \QeD\   Next, $\Cal I$ is normal.   \Prf\ Of course
$\kappa=\add\Cal J$ is regular (513C(a-i)), and we are supposing that it
is uncountable.   If $S\in\Cal P\kappa\setminus\Cal I$ and
$h:S\to\kappa$
is regressive, set $f(y)=hg(y)$ if $y\in g^{-1}[S]$, $g(y)$ otherwise.
Then $\{y:f(y)<g(y)\}=g^{-1}[S]\notin\Cal J$, so $f\notin F$ and there
is an $\alpha<\kappa$ such that $f^{-1}[\alpha]\notin\Cal J$.   But

\Centerline{$f^{-1}[\alpha]
\subseteq g^{-1}[\alpha]
  \cup\bigcup_{\beta<\alpha}g^{-1}[h^{-1}[\{\beta\}]]$;}

\noindent as $\alpha<\add\Cal J$, there is a $\beta<\alpha$ such that
$g^{-1}[h^{-1}[\{\beta\}]]\notin\Cal J$ and
$h^{-1}[\{\beta\}]\notin\Cal I$.   As $h$ is arbitrary, $\Cal I$ is
normal (541H).\ \Qed
}%end of proof of 541J

\leader{541K}{Lemma} %RVMC 4Lc
Let $\kappa$ be a regular uncountable cardinal and $\Cal I$ a
normal ideal on $\kappa$ which is $\kappa'$-saturated in $\Cal P\kappa$,
where $\kappa'\le\kappa$.

(a) If $S\in\Cal P\kappa\setminus\Cal I$ and $f:S\to\kappa$ is
regressive, then there is a set $A\in[\kappa]^{<\kappa'}$ such that
$S\setminus f^{-1}[A]\in\Cal I$;  \cmmnt{consequently} there is an
$\alpha<\kappa$ such that
$\{\xi:\xi\in S$, $f(\xi)\ge\alpha\}\in\Cal I$.

(b) If $\lambda<\kappa$, then
$\{\xi:\xi<\kappa$, $\cf\xi\le\lambda\}\in\Cal I$.

(c) If for each $\xi<\kappa$ we are given a relatively closed set
$C_{\xi}\subseteq\xi$ which is cofinal with $\xi$, then

\Centerline{$C=\{\alpha:\alpha<\kappa$,
$\{\xi:\alpha\notin C_{\xi}\}\in\Cal I\}$}

\noindent is a cofinal closed set in $\kappa$.

\proof{{\bf (a)} Choose $\langle S_{\eta}\rangle_{\eta\le\gamma}$ and
$\ofamily{\eta}{\gamma}{\alpha_{\eta}}$ inductively, as follows.
$S_0=S$.   If $S_{\eta}\in\Cal I$, set $\gamma=\eta$ and stop.
Otherwise, $f\restr S_{\eta}$ is regressive, so (because $\Cal I$ is
normal) there is an $\alpha_{\eta}<\kappa$ such that
$\{\xi:\xi\in S_{\eta}$, $f(\xi)=\alpha_{\eta}\}\notin\Cal I$ (541H(iii)).
Set $S_{\eta+1}=\{\xi:\xi\in S_{\eta}$, $f(\xi)\ne\alpha_{\eta}\}$.
Given $\ofamily{\zeta}{\eta}{S_{\zeta}}$ for a non-zero limit ordinal
$\eta$, set $S_{\eta}=\bigcap_{\zeta<\eta}S_{\zeta}$.   Now
$\ofamily{\eta}{\gamma}{S_{\eta}\setminus S_{\eta+1}}$ is a disjoint
family in $\Cal P\kappa\setminus\Cal I$, so $\#(\gamma)<\kappa'$ and
$A=\{\alpha_{\eta}:\eta<\gamma\}\in[\kappa]^{<\kappa'}$, while
$S\setminus f^{-1}[A]=S_{\gamma}$ belongs to $\Cal I$.
Setting $\alpha=\sup A + 1$, $\alpha<\kappa$
(because $\kappa$ is regular) and
$\{\xi:\xi\in S$, $f(\xi)\ge\alpha\}$ belongs to $\Cal I$.

\medskip

{\bf (b)} \Quer\ Otherwise, set
$S=\{\xi:0<\xi<\kappa$, $\cf\xi\le\lambda\}$ and for $\xi\in S$ choose a
cofinal set $A_{\xi}\subseteq\xi$ with $\#(A_{\xi})\le\lambda$.   Let
$\ofamily{\eta}{\lambda}{f_{\eta}}$ be a family of functions defined on
$S$ such that $A_{\xi}=\{f_{\eta}(\xi):\eta<\lambda\}$ for each
$\xi\in S$.   By (a), we have for each $\eta<\lambda$ an
$\alpha_{\eta}<\kappa$ such that
$B_{\eta}=\{\xi:\xi\in S$, $f_{\eta}(\xi)\ge\alpha_{\eta}\}\in\Cal I$.
Set $\alpha=\sup_{\eta<\lambda}\alpha_{\eta}<\kappa$;  as
$\lambda<\kappa=\add\Cal I$ (541H), there is a
$\xi\in S\setminus\bigcup_{\eta<\lambda}B_{\eta}$ such that $\xi>\alpha$.
But now $A_{\xi}\subseteq\alpha$ is not cofinal with $\xi$.\ \Bang

\medskip

{\bf (c)} For $\alpha<\kappa$, $0<\xi<\kappa$ set

$$\eqalign{f_{\alpha}(\xi)&=\min(C_{\xi}\setminus\alpha)
\text{ if }\xi>\alpha,\cr
&=0\text{ otherwise}.\cr}$$

\noindent Then $f_{\alpha}$ is regressive, so by (a) there is a
$\zeta_{\alpha}<\kappa$ such that
$\kappa\setminus f_{\alpha}^{-1}[\zeta_{\alpha}]\in\Cal I$, that is,
$\{\xi:C_{\xi}\cap\zeta_{\alpha}\setminus\alpha=\emptyset\}\in\Cal I$.
Set $\tilde C=\{\alpha:\alpha<\kappa$, $\zeta_{\beta}<\alpha$ for every
$\beta<\alpha\}$;  then $\tilde C$ is cofinal with $\kappa$.   If
$\alpha\in\tilde C$, then

$$\eqalign{\{\xi:\xi>\alpha,\,\alpha\notin C_{\xi}\}
&\subseteq\{\xi:C_{\xi}\cap\alpha\text{ is not cofinal with }\alpha\}\cr
&\subseteq\{\xi:C_{\xi}\cap\zeta_{\beta}\setminus\beta=\emptyset
  \text{ for some }\beta<\alpha\}\cr}$$

\noindent is the union of fewer than $\kappa$ members of $\Cal I$, so
belongs to $\Cal I$, and $\alpha\in C$.   Thus $C$ is cofinal with
$\kappa$.   If $\alpha<\kappa$ and $\alpha=\sup(C\cap\alpha)$, then

\Centerline{$\{\xi:\xi>\alpha,\,\alpha\notin C_{\xi}\}
\subseteq\{\xi:\beta\notin C_{\xi}\text{ for some }\beta\in
C\cap\alpha\}$}

\noindent is again the union of fewer than $\kappa$ members of $\Cal I$,
so $\alpha\in C$.   Thus $C$ is closed.
}%end of proof of 541K

\vleader{72pt}{541L}{Theorem} Let $\kappa$ be an uncountable cardinal
such that there is a proper $\kappa$-saturated $\kappa$-additive ideal of
$\Cal P\kappa$ containing singletons.

(a) There is a $\kappa$-saturated normal ideal on $\kappa$.

(b) $\kappa$ is weakly inaccessible.

(c) The set of weakly inaccessible cardinals less than $\kappa$ is
stationary in $\kappa$.

\proof{{\bf (a)} Let $\Cal J$ be a proper
$\kappa$-saturated $\kappa$-additive ideal of $\Cal P\kappa$.
The additivity of $\Cal J$ must be exactly $\kappa$, so 541J
tells us that there is a $\kappa$-saturated normal ideal $\Cal I$
on $\kappa$.

\medskip

{\bf (b)} Of course $\kappa=\add\Cal J=\add\Cal I$ is regular.
\Quer\ Suppose, if possible, that $\kappa=\lambda^+$ is
a successor cardinal.   For each $\alpha<\kappa$ let
$\phi_{\alpha}:\alpha\to\lambda$ be an injection.   For $\beta<\kappa$ and
$\xi<\lambda$ set $A_{\beta\xi}
=\{\alpha:\beta<\alpha<\kappa$, $\phi_{\alpha}(\beta)=\xi\}$.
Then $\bigcup_{\xi<\lambda}A_{\beta\xi}=\kappa\setminus(\beta+1)
\notin\Cal I$, so there is a $\xi_{\beta}<\lambda$ such that
$A_{\beta,\xi_{\beta}}\notin\Cal I$.   Now there must be an
$\eta<\lambda$ such that $B=\{\beta:\beta<\kappa$, $\xi_{\beta}=\eta\}$
has cardinal $\kappa$.   But in this case
$\family{\beta}{B}{A_{\beta\eta}}$ is a
disjoint family in $\Cal P\kappa\setminus\Cal I$, and $\Cal I$ is not
$\kappa$-saturated in $\Cal P\kappa$.\ \Bang

Thus $\kappa$ is a regular uncountable limit cardinal, i.e., is weakly
inaccessible.

\medskip

{\bf (c)} Write $R$ for the set
of regular infinite cardinals less than $\kappa$ and $L$ for the set of
limit cardinals less than $\kappa$.

\medskip

\quad{\bf (i)} $\kappa\setminus R\in\Cal I$.  \Prf\Quer\ Otherwise,
$A=(\kappa\setminus R)\setminus\{0,1\}\notin\Cal I$.   For $\xi\in A$,
set $f(\xi)=\cf\xi$;  then $f:A\to\kappa$ is regressive.   Because
$\Cal I$ is normal, there must be a $\delta<\kappa$ such that
$B=\{\xi:\xi<\kappa$, $\cf\xi=\delta\}\notin\Cal I$.
For each $\xi\in B$, let $\ofamily{\eta}{\delta}{g_{\eta}(\xi)}$ enumerate
a cofinal subset of $\xi$.   If $\eta<\delta$, then
$g_{\eta}:B\to\kappa$ is regressive, so by 541Ka there is a
$\gamma_{\eta}<\kappa$ such that
$J_{\eta}=\{\xi:\xi\in B$, $g_{\eta}(\xi)\ge\gamma_{\eta}\}\in\Cal I$.
Set $\gamma=\sup_{\eta<\delta}\gamma_{\eta}$;  as $\kappa$ is regular,
$\gamma<\kappa$;  while
$B\setminus(\gamma+1)\subseteq\bigcup_{\eta<\delta}J_{\eta}$ belongs to
$\Cal I$, which is impossible.\ \Bang\Qed

\medskip

\quad{\bf (ii)} $R\setminus L\in\Cal I$.   \Prf\ We have a regressive
function $f:R\setminus L\to\kappa$ defined by setting
$f(\lambda^+)=\lambda$ for every infinite cardinal $\lambda<\kappa$.
Now $f^{-1}[\{\xi\}]$ is empty or a singleton for every $\xi$, so always
belongs to $\Cal I$;  because
$\Cal I$ is normal, $R\setminus L\in\Cal I$.\ \Qed

\medskip

\quad{\bf (iii)} Accordingly the set $R\cap L$ of weakly inaccessible
cardinals less than $\kappa$ cannot belong to $\Cal I$ and must be
stationary, by 541Ia.
}%end of proof of 541L

\leader{541M}{Definition (a)} A regular uncountable cardinal $\kappa$ is
{\bf \2vm}\cmmnt{ (often just {\bf measurable})} if there is a proper
$\kappa$-additive
$2$-saturated ideal of $\Cal P\kappa$ containing singletons.

\cmmnt{Of course a proper ideal $\Cal I$ of $\Cal P\kappa$ is
$2$-saturated iff it is maximal, that is, the dual filter
$\{\kappa\setminus I:I\in\Cal I\}$ is an ultrafilter;  thus $\kappa$ is
\2vm\ iff there is a non-principal $\kappa$-complete ultrafilter on
$\kappa$.   From 541J we see also that if $\kappa$ is \2vm\ then there
is a normal maximal ideal of $\Cal P\kappa$, that is, there is a normal
ultrafilter on $\kappa$, as considered in \S4A1.}

\spheader 541Mb An uncountable cardinal $\kappa$ is {\bf weakly compact}
if for every $S\subseteq[\kappa]^2$ there is a $D\in[\kappa]^{\kappa}$
such that $[D]^2$ is either included in $S$ or disjoint from $S$.

\leader{541N}{Theorem} (a) A \2vm\ cardinal is weakly compact.

(b) A weakly compact cardinal is strongly inaccessible.

\proof{{\bf (a)} If $\kappa$ is a \2vm\ cardinal, there is a
non-principal normal ultrafilter on $\kappa$, so 4A1L tells us that
$\kappa$ is weakly compact.

\medskip

{\bf (b)} Let $\kappa$ be a weakly compact cardinal.

\medskip

\quad{\bf (i)} Set $\lambda=\cf\kappa$;  let $A\in[\kappa]^{\lambda}$ be
a cofinal subset of $\kappa$, and
$\ofamily{\zeta}{\lambda}{\alpha_{\zeta}}$ the increasing enumeration of
$A$.   For $\xi<\kappa$ set $f(\xi)=\min\{\zeta:\xi<\alpha_{\zeta}\}$;
now set $S=\{I:I\in[\kappa]^2$, $f$ is constant on $I\}$.   If
$D\in[\kappa]^{\kappa}$, take any $\xi\in D$;  then there is an
$\eta\in D\setminus\alpha_{f(\xi)}$, so $f$ is not constant on
$\{\xi,\eta\}$ and $[D]^2\not\subseteq S$.   There must therefore be a
$D\in[\kappa]^2$ such that $[D]^2\cap S=\emptyset$.   But in this case
$f$ is injective on $D$, so $\lambda\ge\#(f[D])=\kappa$ and
$\cf\kappa=\kappa$.

Thus $\kappa$ is regular.

\medskip

\quad{\bf (ii)} \Quer\ Suppose, if possible, that $\kappa$ is not
strongly inaccessible.   Then there is a least cardinal $\lambda<\kappa$
such that $2^{\lambda}\ge\kappa$;  let $\phi:\kappa\to\Cal P\lambda$ be
an injective function.   Set

\Centerline{$S=\{\{\xi,\eta\}:\xi<\eta<\kappa$,
$\min(\phi(\xi)\symmdiff\phi(\eta))\in\phi(\eta)\}$.}

\noindent Because $\kappa$ is weakly compact, there is a
$D\in[\kappa]^{\kappa}$ such that either $[D]^2\subseteq S$ or
$[D]^2\cap S=\emptyset$.   Set
$B=\{\phi(\xi)\cap\gamma:\xi\in D$, $\gamma<\lambda\}$.   Then

\Centerline{$\#(B)\le\#(\bigcup_{\gamma<\lambda}\Cal P\gamma)
\le\max(\lambda,\sup_{\gamma<\lambda}2^{\gamma})<\kappa$}

\noindent because $\kappa$ is regular, $\lambda<\kappa$ and
$2^{\gamma}<\kappa$ for every $\gamma<\lambda$.   So there must be an
$\eta\in D$ such that
$B=\{\phi(\xi)\cap\gamma:\xi\in D\cap\eta$, $\gamma<\lambda\}$.   Take
$\zeta\in D$ such that $\zeta>\eta$, set
$\gamma=\min(\phi(\eta)\symmdiff\phi(\zeta))$ and take
$\xi\in D\cap\eta$ such that
$\phi(\xi)\cap(\gamma+1)=\phi(\zeta)\cap(\gamma+1)$.   Now
$\gamma=\min(\phi(\xi)\symmdiff\phi(\eta))$, so

\Centerline{$\{\xi,\eta\}\in S
\iff\gamma\in\phi(\eta)
\iff\gamma\notin\phi(\zeta)
\iff\{\eta,\zeta\}\notin S$.}

\noindent But this means that $[D]^2$ can be neither included in $S$ nor
disjoint from $S$;  contrary to the choice of $D$.\ \Bang

Thus $\kappa$ is strongly inaccessible.
}%end of proof of 541N

\leader{541O}{Lemma} Let $X$ be a set and $\Cal I$ a proper ideal of
subsets of $X$ such that $\Cal PX/\Cal I$ is atomless.   If $\Cal I$ is
$\lambda$-saturated and $\kappa$-additive, with $\lambda\le\kappa$, then
$\kappa\le\cov\Cal I\le\sup_{\theta<\lambda}2^{\theta}$.

\proof{ We may take it that $\lambda=\sat(\Cal PX/\Cal I)$.
If $\lambda>\kappa$ the result is trivial because $\Cal I$ contains
singletons.   So suppose that $\lambda\le\kappa$.
For each $A\in\Cal PX\setminus\Cal I$ choose $A'\subseteq A$ such that
neither $A'$ nor $A\setminus A'$ belongs to $\Cal I$;  this is possible because
$\Cal PX/\Cal I$ is atomless.   Define
$\ofamily{\xi}{\lambda}{\Cal A_{\xi}}$ inductively, as follows.
$\Cal A_0=\{X\}$.   Given that
$\Cal A_{\xi}\subseteq\Cal PX\setminus\Cal I$, then set
$\Cal A_{\xi+1}
=\{A':A\in\Cal A_{\xi}\}\cup\{A\setminus A':A\in\Cal A_{\xi}\}$.   For a
non-zero limit ordinal $\xi<\lambda$, set
$E_{\xi}=\bigcap_{\eta<\xi}\bigcup\Cal A_{\eta}$;  for
$x\in E_{\xi}$ set $C_{\xi x}
=\bigcap\{A:x\in A\in\bigcup_{\eta<\xi}\Cal A_{\eta}\}$;  set
$\Cal A_{\xi}=\{C_{\xi x}:x\in E_{\xi}\}\setminus\Cal I$, and
continue.   Observe that this construction ensures that each
$\Cal A_{\xi}$ is disjoint, and that if $\eta\le\xi$ and
$A\in\Cal A_{\xi}$ then there is a $B\in\Cal A_{\eta}$ such that
$A\subseteq B$.

If $x\in X$, then
$\alpha_x=\{\xi:\xi<\lambda$, $x\in\bigcup\Cal A_{\xi}\}$ is an
initial segment of $\lambda$, so is an ordinal less than or equal to
$\lambda$.   In fact $\alpha_x<\lambda$.   \Prf\ For each
$\xi<\alpha_x$ take $A_{\xi}\in\Cal A_{\xi}$ such that
$x\in A_{\xi}$, and let $B_{\xi}$ be either $A'_{\xi}$ or
$A_{\xi}\setminus A'_{\xi}$ and such that $x\notin B_{\xi}$.   Then
$\ofamily{\xi}{\alpha_x}{B_{\xi}}$ is a disjoint family in
$\Cal P\kappa\setminus\Cal I$ so has cardinal less than $\lambda$.\ \Qed

Of course each $\alpha_x$ is a non-zero limit ordinal, because
$\bigcup\Cal A_{\xi}=\bigcup\Cal A_{\xi+1}$ for each $\xi$.   Now set
$\Cal A=\bigcup_{\xi<\lambda}\Cal A_{\xi}$;  then
$\#(\Cal A)\le\lambda$.   Next, for any $x\in X$,
$\Cal B_x=\{A:A\in\Cal A$, $x\in A\}$ has cardinal less than
$\lambda$ and $C_x=\bigcap\Cal B_x$ belongs to $\Cal I$ and
contains $x$.   So $\Cal C=\{C_x:x\in X\}$ has cardinal
at most $\#([\lambda]^{<\lambda})=\sup_{\theta<\lambda}2^{\theta}$
(because $\lambda=\sat(\Cal PX/\Cal I)$ is regular, by 514Da),
and $\Cal C\subseteq\Cal I$ covers $X$, so

\Centerline{$\kappa\le\add\Cal I\le\cov\Cal I
\le\#(\Cal C)\le\sup_{\theta<\lambda}2^{\theta}$.}
}%end of proof of 541O

\vleader{48pt}{541P}{Theorem}\cmmnt{ ({\smc Tarski 45}, {\smc Solovay 71})}
Suppose that $\kappa$ is a regular uncountable cardinal with a proper
$\lambda$-saturated $\kappa$-additive ideal
$\Cal I$ of $\Cal P\kappa$ containing singletons, where
$\lambda\le\kappa$.   Set $\frak A=\Cal P\kappa/\Cal I$.   Then

{\it either} $\kappa\le\sup_{\theta<\lambda}2^{\theta}$ and $\frak A$ is
atomless

{\it or} $\kappa$ is \2vm\ and $\frak A$ is purely atomic.

\proof{{\bf (a)} Let us begin by noting that $\Cal I$ is
$\lambda$-saturated iff $\lambda\ge\sat(\frak A)$;  so it will be enough
to prove the result when $\lambda=\sat(\frak A)$, in which case
$\lambda$ is either finite or regular and uncountable (514Da again).

\medskip

{\bf (b)} Suppose that $\frak A$ is atomless.   By 541O,
$\kappa\le\sup_{\theta<\lambda}2^{\theta}$.
So in this case we have the first alternative of the dichotomy.

\medskip

{\bf (c)} Before continuing with an analysis of atoms in $\frak A$, I
draw out some further features of the structure discussed in
the proof of 541O.   We
find that if $\frak A$ is atomless then $\kappa$ is not weakly compact.
\Prf\ Construct
$\ofamily{\xi}{\lambda}{\Cal A_{\xi}}$, $\Cal A$ and $\Cal C$
as in 541O.   Consider
$\alpha^*=\sup\{\xi:\xi<\lambda$, $\Cal A_{\xi}\ne\emptyset\}$.

\medskip

\quad{\bf case 1} If $\alpha^*<\kappa$, then $\#(\Cal A)<\kappa$,
because $\kappa$ is
regular and $\#(\Cal A_{\xi})<\lambda\le\kappa$ for every $\xi$.   In
this case $\kappa\le\#(\Cal C)\le 2^{\#(\Cal A)}$ and $\kappa$ is not
strongly inaccessible, therefore not weakly compact, by 541Nb.

\medskip

\quad{\bf case 2} If $\alpha^*=\kappa$, then $\#(\Cal A')=\kappa$, where
$\Cal A'=\bigcup_{\xi<\kappa}\Cal A_{\xi+1}$.   Note that each
$D\in\Cal A'$ has a companion $D^*\in\Cal A'$ defined by saying that if
$D\in\Cal A_{\xi+1}$ then $D^*=D_0\setminus D$ where $D_0$ is the unique
member of $\Cal A_{\xi}$ including $D$.   Consider the relation
$S=\{(D,D'):D$, $D'\in\Cal A'$, $D\cap D'=\emptyset\}$.   Take any
$\Cal D\in[\Cal A']^{\kappa}$.   Then $[\Cal D]^2\not\subseteq S$, because
$\Cal I$ is $\kappa$-saturated.   \Quer\ If
$[\Cal D]^2\cap S=\emptyset$, any two members of $\Cal D$ meet.   If $D_1$ and
$D_2$ are distinct members of $\Cal D$, then they cannot both belong to
$\Cal A_{\xi+1}$ for any $\xi$, so one must belong to $\Cal A_{\eta+1}$
and the other to $\Cal A_{\xi+1}$ where $\eta<\xi$;  say
$D_1\in\Cal A_{\eta+1}$ and $D_2\in\Cal A_{\xi+1}$.   Now
$D_2\cup D_2^*\in\Cal A_{\xi}$ meets $D_1$ and is therefore included in
$D_1$;  so $D_1^*\cap D_2^*=\emptyset$.   Thus $\{D^*:D\in\Cal D\}$ is a
disjoint family in $\Cal A$ of size $\kappa$, contrary to the hypothesis
that $\Cal I$ is $\kappa$-saturated.\ \Bang

Thus if $\Cal D\in[\Cal A']^{\kappa}$, $[\Cal D]^2$ is neither included
in nor disjoint from $S$.   Since $\#(\Cal A')=\kappa$, this shows that
$\kappa$ cannot be weakly compact.\ \Qed

\medskip

{\bf (d)} Now suppose that $\frak A$ has an atom $a$.
Let $A\in\Cal P\kappa\setminus\Cal I$ be such that $A^{\ssbullet}=a$.
Set $\Cal I_A=\{I:I\subseteq\kappa$, $I\cap A\in\Cal I\}$;  then
$\Cal I_A$ is a $\kappa$-additive maximal ideal of $\kappa$ containing
singletons, so $\kappa$ is \2vm.   It follows that $\kappa$ is weakly
compact (541Na).

\Quer\ Suppose, if possible, that $\frak A$ is not purely atomic.   Then
there is a $C\in\Cal P\kappa\setminus\Cal I$ such that
$\Cal P\kappa/\Cal I_C$ is atomless, where
$\Cal I_C=\{I:I\subseteq\kappa$, $I\cap C\in\Cal I\}$.   Also $\Cal I_C$ is
$\kappa$-additive and $\lambda$-saturated.   But this is
impossible, by (c).\ \BanG\  Thus $\frak A$ is purely atomic, and we
have the second alternative of the dichotomy.
}%end of proof of 541P

\leader{541Q}{Theorem} %RVMC 5B
Let $\kappa$ be a regular uncountable cardinal and $\Cal I$ a normal
ideal on $\kappa$.   Let $\theta<\kappa$ be a cardinal of uncountable
cofinality such that
$\Cal I$ is $(\cf\theta)$-saturated in $\Cal P\kappa$, and
$f:[\kappa]^{<\omega}\to[\kappa]^{<\theta}$ any
function.   Then there are $C\in\Cal I$ and
$f^*:[\kappa\setminus C]^{<\omega}\to[\kappa]^{<\theta}$ such that
$f(I)\cap\eta\subseteq f^*(I\cap\eta)$ whenever
$I\in[\kappa\setminus C]^{<\omega}$ and $\eta<\kappa$.
%again:  do we need all this?  probably not, but it's fun

\proof{{\bf (a)} I show by induction on $n\in\Bbb N$ that if
$g:[\kappa]^{\le n}\to[\kappa]^{<\theta}$ is a
function then there are $A\in\Cal I$ and
$g^*:[\kappa\setminus A]^{\le n}\to[\kappa]^{<\theta}$ such that
$g(I)\cap\eta\subseteq g^*(I\cap\eta)$ for every
$I\in[\kappa\setminus A]^{\le n}$ and $\eta<\kappa$.

\Prf\ If $n=0$ this is trivial;  take $A=\emptyset$,
$g^*(\emptyset)=g(\emptyset)$.   For the inductive step to $n+1$, given
$g:[\kappa]^{\le n+1}\to[\kappa]^{<\theta}$, then for each $\xi<\kappa$
define $g_{\xi}:[\kappa]^{\le n}\to[\kappa]^{<\theta}$ by
setting $g_{\xi}(J)=g(J\cup\{\xi\})$ for every
$J\in[\kappa]^{\le n}$.   Set

\Centerline{$D=\{\xi:\xi<\kappa$, $\cf(\xi)\ge\theta\}$;}

\noindent then $\kappa\setminus D\in\Cal I$ (541Kb).  For $\xi\in D$ and
$J\in[\kappa]^{\le n}$ set
$\zeta_{J\xi}=\sup(\xi\cap g_{\xi}(J))<\xi$.   Then for each
$J\in[\kappa]^{\le n}$
the function $\xi\mapsto\zeta_{J\xi}:D\to\kappa$ is regressive,
so there is a $\zeta^*_J<\kappa$ such that
$\{\xi:\zeta_{J\xi}\ge\zeta^*_J\}\in\Cal I$ (541Ka).   Now
$\add\Cal I=\kappa$, by 541H, so 541D tells us
that there is an $h(J)\in[\zeta^*_J]^{<\theta}$ such that
$\{\xi:\xi\cap g_{\xi}(J)\not\subseteq h(J)\}\in\Cal I$.   By the
inductive hypothesis, there are $B\in\Cal I$ and
$h^*:[\kappa\setminus B]^{\le n}\to[\kappa]^{<\theta}$ such that
$h(J)\cap\eta\subseteq h^*(J\cap\eta)$ for every
$J\in[\kappa\setminus B]^{\le n}$ and $\eta<\kappa$.

Try setting

\Centerline{$A_J=\{\xi:\xi\cap g_{\xi}(J)\not\subseteq h(J)\}$ for
$J\in[\kappa]^{\le n}$,}

\Centerline{$A=B\cup\{\xi:\xi\in\bigcup_{J\in[\xi]^{\le n}}A_J\}$,}

$$\eqalign{g^*(I)
&=g(I)\text{ if }I\in[\kappa\setminus A]^{n+1},\cr
&=g(I)\cup h^*(I)\text{ if }I\in[\kappa\setminus A]^{\le n}.\cr}$$

\noindent Then $A_J$ always belongs to $\Cal I$, by the choice of
$h(J)$, so $A\in\Cal I$, by 541Ib,
while $g^*(I)\in[\kappa]^{<\theta}$ for every
$I\in[\kappa\setminus A]^{\le n+1}$.   Take $\eta<\kappa$ and
$I\in[\kappa\setminus A]^{\le n+1}$.   If $I\subseteq\eta$ then
$g(I)\cap\eta\subseteq g^*(I)=g^*(I\cap\eta)$.   Otherwise, set
$\xi=\max I$ and $J=I\setminus\{\xi\}$.   Then
$\eta\le\xi\in\kappa\setminus A_J$, so

\Centerline{$g(I)\cap\eta
=g_{\xi}(J)\cap\xi\cap\eta
\subseteq h(J)\cap\eta
\subseteq h^*(J\cap\eta)
=h^*(I\cap\eta)
\subseteq g^*(I\cap\eta)$.}

\noindent Thus the induction continues.\ \Qed

\medskip

{\bf (b)} Now applying (a) to $f\restr[\kappa]^{\le n}$
we obtain sets $C_n\in\Cal I$ and functions
$f^*_n:[\kappa\setminus C_n]^{\le n}\to[\kappa]^{<\theta}$ such that
$f(I)\cap\eta\subseteq f_n^*(I\cap\eta)$
whenever $I\in[\kappa\setminus C_n]^{\le n}$ and $\eta<\kappa$.   Set
$C=\bigcup_{n\in\Bbb N}C_n\in\Cal I$ and
$f^*(I)=\bigcup_{n\ge\#(I)}f^*_n(I)$ for each
$I\in[\kappa\setminus C]^{<\omega}$.   Because $\cf\theta>\omega$,
$f(I)\in[\kappa]^{<\theta}$ for every $I$.
If $I\in[\kappa\setminus C]^{<\omega}$ and
$\eta<\kappa$, set $n=\#(I)$;  then $I\in[\kappa\setminus C_n]^n$ so
$f(I)\cap\eta\subseteq f_n^*(I\cap\eta)\subseteq f^*(I\cap\eta)$, as
required.
}%end of proof of 541Q

\vleader{72pt}{541R}{Corollary} %RVMC 5C
Let $\kappa$ be a regular uncountable cardinal, $\Cal I$ a normal
ideal on $\kappa$, and $\theta<\kappa$ a cardinal of uncountable
cofinality such that
$\Cal I$ is $(\cf\theta)$-saturated in $\Cal P\kappa$.

(a) If $Y$ is a set of
cardinal less than $\kappa$ and $f:[\kappa]^{<\omega}\to[Y]^{<\theta}$ a
function, then there are $C\in\Cal I$ and $M\in[Y]^{<\theta}$ such
that $f(I)\subseteq M$ for every $I\in[\kappa\setminus C]^{<\omega}$.

(b) If $Y$ is any set and $g:\kappa\to[Y]^{<\theta}$ a function, then there
are $C\in\Cal I$ and $M\in[Y]^{<\theta}$ such that
$g(\xi)\cap g(\eta)\subseteq M$ for all distinct $\xi$,
$\eta\in\kappa\setminus C$.

\proof{{\bf (a)} We may suppose that $Y\subseteq\kappa$.   In this case, by 541Q,
we have a $C_0\in\Cal I$ and an
$f^*:[\kappa\setminus C_0]^{<\omega}\to[\kappa]^{<\theta}$ such that
$f(I)\cap\eta\subseteq f^*(I\cap\eta)$ whenever
$I\subseteq\kappa\setminus C_0$ is finite and $\eta<\kappa$.   Let
$\gamma<\kappa$ be such that $Y\subseteq\gamma$ and set
$M=Y\cap f^*(\emptyset)$, $C=C_0\cup\gamma$.   Then
$M\in[Y]^{<\theta}$, $C\in\Cal I$ and if
$I\in[\kappa\setminus C]^{<\omega}$ then

\Centerline{$f(I)=f(I)\cap\gamma
\subseteq Y\cap f^*(I\cap\gamma)\cap\gamma=M$.}

\medskip

{\bf (b)} Since $\bigcup_{\xi<\kappa}g(\xi)$ has cardinal at most $\kappa$,
we may again suppose that $Y\subseteq\kappa$.   Apply 541Q with
$f(\{\xi\})=g(\xi)$ for $\xi<\kappa$.
Taking $C$ and $f^*$ from 541Q, set $M=Y\cap f^*(\emptyset)$.
Set $F=\{\xi:\xi<\kappa$, $g(\eta)\subseteq\xi$ for every $\eta<\xi\}$;
then $F$ is a closed cofinal subset of $\kappa$ (because $\theta\le\kappa$
and $\kappa$ is regular), so $C'=C\cup(\kappa\setminus F)\in\Cal I$
(541Ia).   If $\xi$, $\eta$ belong to $\kappa\setminus C'=F\setminus C$
and $\eta<\xi$, then

\Centerline{$g(\xi)\cap g(\eta)\subseteq\xi\cap g(\xi)
=\xi\cap f(\{\xi\})\subseteq f^*(\xi\cap\{\xi\})
=f^*(\emptyset)$,}

\noindent so $g(\xi)\cap g(\eta)\subseteq M$.   Thus $C'$ serves.
}%end of proof of 541R

\vleader{36pt}{541S}{Lemma} %RVMC 7O
Let $\kappa$ be a regular uncountable cardinal and $\Cal I$ a normal
ideal on $\kappa$.   Suppose that $\gamma$ and $\delta$ are cardinals
such that
$\omega\le\gamma<\delta<\kappa\le 2^{\delta}$,
$\Cal I$ is $\delta$-saturated in
$\Cal P\kappa$, $2^{\beta}=2^{\gamma}$
for $\gamma\le\beta<\delta$, but $2^{\delta}>2^{\gamma}$.   Then
$\delta$ is regular and

\Centerline{$2^{\delta}=\covSh(2^{\gamma},\kappa,\delta^+,\delta)
=\covSh(2^{\gamma},\kappa,\delta^+,\omega_1)
=\covSh(2^{\gamma},\kappa,\delta^+,2)$.}

\proof{ By 5A1Eh, $\delta$ is regular.    Of course

\Centerline{$\covSh(2^{\gamma},\kappa,\delta^+,\delta)
\le\covSh(2^{\gamma},\kappa,\delta^+,\omega_1)
\le\covSh(2^{\gamma},\kappa,\delta^+,2)
\le\#([2^{\gamma}]^{\le\delta})\le 2^{\delta}$}

\noindent (5A2D, 5A2Ea).   For the reverse inequality, let
$\Cal E\subseteq[2^{\gamma}]^{<\kappa}$ be a set with cardinal
$\covSh(2^{\gamma},\kappa,\delta^+,\delta)$
such that every member of $[2^{\gamma}]^{\le\delta}$ is covered by
fewer than $\delta$ members of
$\Cal E$.   For each ordinal $\xi<\delta$ let
$\phi_{\xi}:\Cal P\xi\to 2^{\gamma}$ be an injective function.   For
$A\subseteq\delta$ define $f_A:\delta\to 2^{\gamma}$ by

\Centerline{$f_A(\xi)=\phi_{\xi}(A\cap\xi)$ for every $\xi<\delta$.}

\noindent Choose $E_A\in\Cal E$ such that $f^{-1}_A[E_A]$ is cofinal
with $\delta$;  such must exist because $\delta$ is regular and
$f_A[\delta]$ can be covered by fewer than $\delta$ members of $\Cal E$.

\Quer\ If $2^{\delta}>\#(\Cal E)$ then there must be an $E\in\Cal E$
and an $\Cal A\subseteq\Cal P\delta$ such that $\#(\Cal A)=\kappa$ and
$E_A=E$ for every $A\in\Cal A$.   For each pair $A$, $B$ of distinct
members of $\Cal A$ set $\xi_{AB}=\min(A\symmdiff B)<\delta$.   By 541Ra,
there is a set
$\Cal B\subseteq\Cal A$, with cardinal $\kappa$, such that
$M=\{\xi_{AB}:A$, $B\in\Cal B$, $A\ne B\}$ has cardinal less than
$\delta$.   Set $\zeta=\sup M<\delta$.   Next, for each $A\in\Cal B$,
take $\eta_A>\zeta$ such that $f_A(\eta_A)\in E$.
Let $\eta<\delta$ be such that $\Cal C=\{A:A\in\Cal B$, $\eta_A=\eta\}$
has cardinal $\kappa$.   Then we have a map

\Centerline{$A\mapsto f_A(\eta)=\phi_{\eta}(A\cap\eta):\Cal C\to E$}

\noindent which is injective, because if $A$, $B$ are distinct members
of $\Cal C$ then $\xi_{AB}\le\zeta<\eta$, so $A\cap\eta\ne B\cap\eta$.
So $\#(E)\ge\kappa$;  but $E\in\Cal E\subseteq[2^{\gamma}]^{<\kappa}$.\
\Bang

As $\Cal E$ is arbitrary,
$\covSh(2^{\gamma},\kappa,\delta^+,\delta)\ge 2^{\delta}$.
}%end of proof of 541S

\exercises{\leader{541X}{Basic exercises (a)}
%\spheader 541Xa
Let $\kappa$ be a regular infinite cardinal.   Show that
$\Cal P\kappa/[\kappa]^{<\kappa}$ is not Dedekind complete, so
$[\kappa]^{<\kappa}$ is not $\kappa^+$-saturated in $\Cal P\kappa$.
\Hint{construct a disjoint family $\ofamily{\xi}{\kappa}{A_{\xi}}$ in
$[\kappa]^{\kappa}$;  show that if $\#(A_{\xi}\setminus A)<\kappa$ for
every $\xi$ there is a $B\in[A]^{\kappa}$ such that
$\#(B\cap A_{\xi})<\kappa$ for every $\xi$.}
%541B

\spheader 541Xb
Let $\frak A$ be a Boolean algebra and $I$ an ideal of
$\frak A$.   Suppose there is a cardinal $\kappa$ such that $I$ is
$\kappa$-additive and $\kappa^+$-saturated and $\frak A$ is
Dedekind $\kappa^+$-complete in the sense that $\sup A$ is defined in
$\frak A$ whenever $A\in[\frak A]^{\le\kappa}$.   Show that $\frak A/I$
is Dedekind complete.
%541B

\spheader 541Xc Suppose that $X$ and $Y$ are sets and $\Cal I$, $\Cal J$
ideals of subsets of $X$, $Y$ respectively.   Suppose that $\kappa$ is
an infinite cardinal such that both $\Cal I$ and $\Cal J$ are
$\kappa$-saturated and $\kappa^+$-additive.   Show that
$\Cal I\ltimes\Cal J$ (definition:  527Ba) is $\kappa$-saturated and
$\kappa^+$-additive.
%541C

\spheader 541Xd Simplify the argument of 541D to give a direct proof of
541E in the case in which $\kappa$ is regular.
%541E

\sqheader 541Xe Show that there is a \2vm\ cardinal iff there are a set $I$
and a non-principal $\omega_1$-complete ultrafilter on $I$.
%541M

\sqheader 541Xf Let $\kappa$ be a \2vm\ cardinal and $\Cal I$ a normal
maximal proper ideal of $\Cal P\kappa$.   (i) Show that if
$S\subseteq[\kappa]^{<\omega}$, there is a $C\in\Cal I$ such that, for
each $n\in\Bbb N$, $[\kappa\setminus C]^n$ is either disjoint from $S$
or included in it.
(ii) Show that if $\#(Y)<\kappa$, and $f:[\kappa]^{<\omega}\to Y$ is
any function, then there is a $C\in\Cal I$ such that $f$ is constant on
$[\kappa\setminus C]^n$ for each $n\in\Bbb N$.
\Hint{4A1L.}
%541N

\spheader 541Xg Let $\kappa$ be a regular uncountable cardinal, $\Cal I$
a $\kappa$-saturated normal ideal on $\kappa$ and
$f:[\kappa]^2\to\kappa$ a function.   Show that there are a $C\in\Cal I$
and an $f^*:\kappa\to\kappa$ such that whenever
$\eta\in\kappa\setminus C$ and $\xi\in\eta\setminus C$ then either
$f(\{\xi,\eta\})\ge\eta$ or $f(\{\xi,\eta\})<f^*(\xi)$.
%541Q

\leader{541Y}{Further exercises (a)}
%\spheader 541Ya
Show that if $\kappa$ is a regular uncountable cardinal and
$S\subseteq\kappa$ is stationary, then $S$ can be partitioned into
$\kappa$ stationary sets.   ({\it Hint\/}:  reduce to the case in which there is a
$\kappa$-saturated normal ideal $\Cal I$ of $\kappa$ containing
$\kappa\setminus S$.   Define $f:S\to\kappa$ inductively by saying that

\Centerline{$f(\xi)=\min(\bigcup\{\kappa\setminus f[C]:C\subseteq\xi$ is
relatively closed and cofinal$\}$.}

\noindent Set $S_{\gamma}=f^{-1}[\{\gamma\}]$.   Apply 541Kc with
$C_{\xi}\cap S_{\gamma}=\emptyset$ for $\xi\in S_{\gamma}$ to show that
$S_{\gamma}\in\Cal I$ for every $\gamma$.   Hence show that $S_{\gamma}$
is stationary for every $\gamma$.   See
{\smc Solovay 71}.)
%541K mt54bits

\spheader 541Yb Show that if $\kappa$ is
\2vm\ and $\Cal F$ is a normal ultrafilter on $\kappa$, then the set of
weakly compact cardinals less than $\kappa$ belongs to $\Cal F$.
%541N
}%end of exercises

\endnotes{
\Notesheader{541}
The ordinary principle of exhaustion (215A) can be regarded as an
expression of $\omega_1$-saturation (compare 316E and 514Db).   In
541B-541E %541B 541C 541D 541E
we have versions of results already given in special cases;  but note
that 541B, for instance, goes a step farther than the arguments offered
in 314C and 316Fa can reach.   541Ca corresponds to 215B(iv);
541Cb-541Cc are associated with 516Q and 525C.   In all this work you
will probably find it helpful to use the words `negligible' and
`conegligible' and `almost everywhere', so that the conclusion of 541E
becomes `$g(x)\in M$ a.e.($x$)'.   I don't use this language in the
formal exposition because of the obvious danger of confusing a reader
who is skimming through without reading introductions to sections very
carefully;  but in the principal applications I have in mind, $\Cal I$
will indeed be a null ideal.   The cardinals
$\Tr_{\Cal I}(X;Y)$ will appear only occasionally in this book, but
are of great importance in infinitary combinatorics generally.   Note
that the key step in the proof of 541F (part (b) of the proof) develops
an idea from the proof of 541J.

For the special purposes of \S438 I mentioned `normal filters' in \S4A1;
I have now attempted an introduction to the general theory of normal
filters and ideals.   The central observation of {\smc Keisler \& Tarski
64} was that if $\kappa$ is uncountable then any $\kappa$-complete
non-principal ultrafilter gives rise to a normal ultrafilter on
$\kappa$.   It was noticed very quickly that something similar happens
if we have a $\kappa$-complete filter of conegligible sets in a totally
finite measure space;  the extension of the idea to general
$\kappa$-additive $\kappa$-saturated ideals is in {\smc Solovay 71}.
In this chapter I speak oftener of ideals than of filters but the ideas
are necessarily identical.   Observe that the Pressing-Down Lemma
(4A1Cc) is the special case of 541H(iii) when $\Cal I$ is the ideal of
non-stationary sets (541Ia).

541Lb here is a re-working of Ulam's theorem
(438C, {\smc Ulam 30}).   The dramatic further step in 541Lc derives
from {\smc Keisler \& Tarski 64}.   The proof of 541Lc already makes it
plain that much more can be said;  for
extensions of these ideas, see {\smc Fremlin 93}, {\smc Levy 71} and
{\smc Baumgartner Taylor \& Wagon 77}.   In 541P we have an extension of
Ulam's dichotomy (438C, 543B).   `Weak compactness' of a cardinal
corresponds to Ramsey's theorem (4A1G);  the idea was the basis of the
proof of 451Q.   Here I treat it as a purely combinatorial concept,
but its real importance is in model theory ({\smc Kanamori 03}, \S4).

541Q is a fairly strong version of one of the typical properties of
saturated normal ideals.   The simplest not-quite-trivial case is when
we have a function $f:[\kappa]^2\to\kappa$.   In this case we find that
if we discard an appropriate
negligible set $C$ then, for the remaining doubletons
$I\in[\kappa\setminus C]^2$, $f(I)$
is either greater than or equal to $\max I$ or in a `small' set determined
by $\min I$.   In this form, with the appropriate definition of `small', it
is enough for the ideal to be
$\kappa$-saturated (541Xg).   In the intended applications of 541Q,
however, we shall be looking at functions
$f:[\kappa]^{<\omega}\to[\kappa]^{\le\omega}$ and shall need to start
from an $\omega_1$-saturated ideal to obtain the full strength of the
result.

Shelah's four-cardinal covering numbers $\covSh$ are not immediately
digestible;  in \S5A2 I give the basic pcf theory linking them to
cofinalities of products.   541S is here because it relies on a normal
ideal being saturated.

Perhaps I have not yet sufficiently emphasized that there is a good
reason why I have given no examples of normal ideals other than the
non-stationary ideals, and no discussion of the saturation of those
beyond Solovay's theorem 541Ya.   We have in fact come to an area of
mathematics which demands further acts of faith.   I will continue,
whenever possible, to express ideas as arguments in ordinary ZFC;  but
in most of the principal theorems the hypotheses will include assertions
which can be satisfied only in rather special models of set theory.
Moreover, these are special in a new sense.   By and large, the
assumptions used in the first three chapters of this volume (Martin's
axiom, the continuum hypothesis, and so on) have been proved to be
relatively consistent with ZFC (indeed, with ZF);  that is, we know how
to convert any proof in ZFC that `$\frak m=\omega_1$' into a proof in ZF
that `$0=1$'.   The formal demonstration that this can be done is of
course normally expressed in a framework reducible to Zermelo-Fraenkel
set theory;  but it is sufficiently compelling to be itself part of the
material which must be encompassed by any formal system claiming to
represent twenty-first century mathematics.   In the present chapter,
however, we are coming to results like 541P which have no content unless
there can be non-trivial $\kappa$-saturated $\kappa$-additive ideals.
And such objects are known to be strange in a different way from Souslin
lines.

To describe this difference I turn to the simplest of the new
propositions.   Write \Esic, \Ewic\ for the sentences `there is a
strongly inaccessible cardinal', `there is a weakly inaccessible
cardinal'.   Of course \Esic\ implies \Ewic, while `there is a cardinal
which is not measure-free' also implies \Ewic, by Ulam's theorem.   We
have no such implications in the other direction, but it is easy to
adapt G\"odel's argument for the relative consistency of GCH to show
that if ZFC + \Ewic\ is consistent so is ZFC + GCH + \Ewic, while
ZFC + GCH + \Ewic\ implies \Esic.   But we find also that there is a
proof in ZFC + \Esic\ that `ZFC is consistent'.   So if there were a
proof in ZFC that
`if ZFC is consistent, then ZFC + \Esic\ is consistent', there would be
a proof in ZFC + \Esic\ that `ZFC + \Esic\ is consistent';  by G\"odel's
incompleteness theorem, this would give us a proof that ZFC + \Esic\ was
in fact {\it inconsistent} (and therefore that ZFC and ZF are
inconsistent).

The last paragraph is expressed crudely, in a language which blurs some
essential distinctions;  for a more careful account see {\smc Kunen 80},
{\S}IV.10.   But what I am trying to say is that any theory involving
inaccessible cardinals -- and the theory of this chapter involves
unthinkably many such cardinals -- necessarily leads us to propositions
which are not merely unprovable, but have high `consistency strength';
we have long strings $\phi_0,\ldots,\phi_n$ of statements such that (i)
if ZFC + $\phi_{i+1}$ is consistent, so is ZFC + $\phi_i$ (ii) there can
be no proof of the reverse unless ZF is inconsistent.

We do not (and in my view cannot) know for sure that ZF is consistent.
It has now survived for a hundred years or so, which is empirical
evidence of a sort.   I do not suppose that the century of my own birth
was also the century in which the structure of formal mathematics was
determined for eternity;  I hope and trust that mathematicians will come
to look on ZFC as we now think of Euclidean geometry, as a glorious
achievement and an enduring source of inspiration but inadequate for the
expression of many of our deepest ideas.   But (arguing from the weakest
of historical analogies) I suggest that if and when a new paradox shakes
the foundations of mathematics, it will be because some new Cantor has
sought to extend apparently trustworthy methods to a totally new
context.   And I think that the mathematicians of that generation will
stretch their ingenuity to the utmost to find a resolution of the
paradox which is conservative, in that it retains as much as possible of
their predecessors' ideas, subject perhaps to re-writing a good many
proofs and tut-tutting over the naivety of essays such as this.

I think indeed (I am going a bit farther here) that they will be as
reluctant to discard measurable cardinals as our forebears were to
discard Cantor's cardinals.   There is a flourishing theory of large
cardinals in which very much stronger statements than `there is a \2vm\
cardinal' have been explored in depth without catastrophe.   (I
mention a couple of these in \S545;  another is the Axiom of Determinacy
in \S567.)   Occasionally a proof that
there are no measurable cardinals is announced;  but the last real
fright was in 1976, and most of these arguments have easily been shown
not to reach the claimed conclusion.   My best guess is that measurable
cardinals are safe.   But even if I am wrong, and they are
irreconcilable with ZFC as now formulated, it does not follow that ZFC
will be kept and measurable cardinals discarded.   It could equally
happen that one of the axioms of ZF will be modified;  or, at least,
that a modified form will become a recognized option.   This is a
partisan view from somebody who has a substantial investment to protect.
But if you wish to prove me wrong, I do not see how you can do so
without giving part of your own life to the topic.
}%end of notes

\discrpage

