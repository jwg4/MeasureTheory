\frfilename{mt437.tex}
\versiondate{5.11.12}
\copyrightdate{2011}

\def\frakTKR{\frak T_{\text{KR}}}
\def\MqR{M^+_{\text{qR}}}
\def\MR{M^+_{\text{R}}}
\def\PqR{P_{\text{qR}}}
\def\PR{P_{\text{R}}}
\def\sigmaKR{\sigma_{\text{KR}}}

\def\chaptername{Topologies and measures II}
\def\sectionname{Spaces of measures}

\newsection{437}

Once we have started to take the correspondence between measures and
integrals as something which operates in both directions, we can go a
very long way.   While `measures', as dealt with in this treatise, are
essentially positive, an `integral' can be thought of as a member of a
linear space, dual in some sense to a space of functions.   Since the
principal spaces of functions are Riesz spaces, we find ourselves
looking at dual Riesz spaces as discussed in \S356;  while the
corresponding spaces of measures are close to those of \S362.   Here I
try to draw these ideas together with an
examination of spaces $U^{\sim}_{\sigma}$ and $U^{\sim}_{\tau}$ of
sequentially smooth and smooth functionals, and the matching spaces
$M_{\sigma}$ and $M_{\tau}$ of countably additive and $\tau$-additive
measures (437A-437I).   Because a (sequentially) smooth functional
corresponds to a countably additive measure, which can be expected to
integrate many more functions than those in the original Riesz space
(typically, a space of continuous functions), we find that relatively
large spaces of bounded measurable functions can be canonically embedded
into the biduals $(U^{\sim}_{\sigma})^*$ and $(U^{\sim}_{\tau})^*$
(437C, 437H, 437I).

The guiding principles of functional analysis encourage us not only to
form linear spaces, but also to examine linear space topologies,
starting with norm and weak topologies.   The theory of Banach lattices
described in \S354, particularly the theory of $M$- and $L$-spaces, is
an important part of the structure here.   In addition, our spaces
$U^{\sim}_{\sigma}$ have natural weak* topologies which can be
regarded as topologies on spaces of measures;  these are the `vague'
topologies of 437J, which have already been considered, in a
special case, in \S285.

It turns out that on the positive cone of $M_{\tau}$, at least, the
vague topology may have an alternative description directly in terms of
the behaviour of the measures on
open sets (437L).   This leads us to a parallel idea, the `narrow'
topology on non-negative additive functionals (437Jd).   The second half
of the section
is devoted to the elementary properties of narrow topologies
(437K-437N), %437K 437L 437M 437P 437R 437N
with especial reference to compact sets in these topologies (437P, 437Rf,
437T).
Seeking to identify narrowly compact sets, we come to the concept of
`uniform tightness' (437O).
Bounded uniformly tight sets are narrowly relatively compact (437P);  in
`Prokhorov spaces' (437U) the converse is true.  I end the section with
a list of the best-known Prokhorov spaces (437V).

\leader{437A}{Smooth and sequentially smooth duals} Let $X$ be a
set, and $U$ a Riesz subspace of $\Bbb R^X$.   \cmmnt{Recall that
$U^{\sim}$ is the Dedekind complete Riesz space of order-bounded linear
functionals on $U$, that $U^{\sim}_c$ is the band of differences of
sequentially order-continuous positive linear functionals, and that
$U^{\times}$ is the band of differences of
order-continuous positive linear functionals (356A).   A functional
$f\in(U^{\sim})^+$ is `sequentially smooth' if
$\inf_{n\in\Bbb N}f(u_n)=0$ whenever $\sequencen{u_n}$
is a non-increasing sequence in $U$ and $\lim_{n\to\infty}u_n(x)=0$ for
every $x\in X$, and `smooth' if $\inf_{u\in A}f(u)=0$ whenever
$A\subseteq U$ is a non-empty
downwards-directed set and $\inf_{u\in A}u(x)=0$ for every $x\in X$
(436A, 436G).}

\spheader 437Aa Set $U^{\sim}_{\sigma}=\{f:f\in U^{\sim}$, $|f|$ is
sequentially smooth$\}$, the {\bf sequentially smooth dual} of $U$.
Then $U^{\sim}_{\sigma}$ is a band in $U^{\sim}$.   \prooflet{\Prf\ (i)
If $f\in U^{\sim}_{\sigma}$, $g\in U^{\sim}$ and $|g|\le|f|$, then

\Centerline{$|g|(u_n)\le|f|(u_n)\to 0$}

\noindent whenever $\sequencen{u_n}$ is a non-increasing sequence in $U$
and $\lim_{n\to\infty}u_n(x)=0$ for every $x$, so $|g|$ is sequentially
smooth and $g\in U^{\sim}_{\sigma}$.   Thus $U^{\sim}_{\sigma}$ is a
solid subset of $U^{\sim}$.   (ii) If $f$, $g\in U^{\sim}_{\sigma}$ and
$\alpha\in\Bbb R$, then

\Centerline{$|f+g|(u_n)\le|f|(u_n)+|g|(u_n)\to 0$,
\quad$|\alpha f|(u_n)=|\alpha||f|(u_n)\to 0$}

\noindent whenever $\sequencen{u_n}$ is a non-increasing sequence in $U$
and $\lim_{n\to\infty}u_n(x)=0$ for every $x$, so $|f+g|$ and
$|\alpha f|$ are sequentially smooth.   Thus $U^{\sim}_{\sigma}$ is a
Riesz subspace of $U^{\sim}$.   (iii) Now suppose that
$B\subseteq(U^{\sim}_{\sigma})^+$ is an upwards-directed set with
supremum $g\in U^{\sim}$, and that $\sequencen{u_n}$ is a non-increasing
sequence in
$U$ such that $\lim_{n\to\infty}u_n(x)=0$ for every $x\in X$.   Then,
given $\epsilon>0$, there is an $f\in B$ such that
$(g-f)(u_0)\le\epsilon$ (355Ed), so that $g(u_n)\le f(u_n)+\epsilon$
for every $n$, and

\Centerline{$\limsup_{n\to\infty}g(u_n)
\le\epsilon+\lim_{n\to\infty}f(u_n)\le\epsilon$.}

\noindent As $\epsilon$ and $\sequencen{u_n}$ are arbitrary,
$g\in U^{\sim}_{\sigma}$;  as $B$ is arbitrary, $U^{\sim}_{\sigma}$ is a
band (352Ob).\ \Qed
}%end of prooflet

\cmmnt{As remarked in 436A, sequentially order-continuous positive
linear functionals are sequentially smooth,
so} $U^{\sim}_c\subseteq U^{\sim}_{\sigma}$.

\spheader 437Ab Set $U^{\sim}_{\tau}=\{f:f\in U^{\sim}$, $|f|$ is
smooth$\}$, the {\bf smooth dual} of $U$.   Then $U^{\sim}_{\tau}$ is a
band in $U^{\sim}$.   \prooflet{\Prf\ (i) Suppose that $f$,
$g\in U^{\sim}_{\tau}$, $\alpha\in\Bbb R$, $h\in U^{\sim}$ and
$|h|\le|f|$.   If $A\subseteq U$ is a non-empty downwards-directed set
and $\inf_{u\in A}u(x)=0$ for every $x\in X$, and $\epsilon>0$, then
there are $u_0$,
$u_1\in A$ such that $|f|(u_0)\le\epsilon$ and $|g|(u_1)\le\epsilon$,
and a $u\in A$ such that $u\le u_0\wedge u_1$.   In this case

\Centerline{$|h|(u)\le|f|(u)\le\epsilon$,}

\Centerline{$|f+g|(u)\le|f|(u)+|g|(u)\le 2\epsilon$,}

\Centerline{$|\alpha f|(u)=|\alpha||f|(u)\le|\alpha|\epsilon$.}

\noindent As $A$ and $\epsilon$ are arbitrary, $h$, $f+g$ and $\alpha f$
all belong to $U^{\sim}_{\tau}$;  so that $U^{\sim}_{\tau}$ is a solid
Riesz subspace of $U^{\sim}$.   (ii) Now suppose that $B\subseteq
(U^{\sim}_{\tau})^+$ is an upwards-directed set with supremum $g\in
U^{\sim}$, and that $A\subseteq U$ is a non-empty downwards-directed set
such that $\inf_{u\in A}u(x)=0$ for every $x\in X$.   Fix any $u_0\in
A$.   Then, given $\epsilon>0$, there is an $f\in B$ such that
$(g-f)(u_0)\le\epsilon$,
so that $g(u)\le f(u)+\epsilon$ whenever $u\in A$
and $u\le u_0$.   But $A_0=\{u:u\in A,\,u\le u_0\}$ is also a
downwards-directed set,
and $\inf_{u\in A_0}u(x)=0$ for every $x\in X$, so

\Centerline{$\inf_{u\in A}g(u)
\le\epsilon+\inf_{u\in A_0}f(u)\le\epsilon$.}

\noindent As $\epsilon$ and $A$ are arbitrary, $g\in U^{\sim}_{\tau}$;
as $B$ is arbitrary, $U^{\sim}_{\tau}$ is a band.\ \Qed
}%end of prooflet

\cmmnt{Just as $U^{\sim}_c\subseteq U^{\sim}_{\sigma}$,}
$U^{\times}\subseteq U^{\sim}_{\tau}$.

\leader{437B}{Signed measures}\cmmnt{ Collecting these ideas together
with those of \S\S362-363, we are ready to approach `signed measures'.}
Recall that if $X$ is a set and $\Sigma$ is a $\sigma$-algebra of
subsets of
$X$, we can identify $L^{\infty}=L^{\infty}(\Sigma)$\cmmnt{, as
defined in \S363,} with the space
$\eusm L^{\infty}=\eusm L^{\infty}(\Sigma)$ of bounded
$\Sigma$-measurable real-valued functions\cmmnt{ (363H)}.
\cmmnt{Now, because $\eusm L^{\infty}$ is sequentially order-closed in
$\Bbb R^X$, sequentially smooth functionals on $\eusm L^{\infty}$ are
actually sequentially order-continuous,
so }$(\eusm L^{\infty})^{\sim}_{\sigma}=(\eusm L^{\infty})^{\sim}_c$.
Next, we can
identify $(L^{\infty})^{\sim}_c$ with the space $M_{\sigma}$ of
countably additive functionals\cmmnt{, or `signed measures',} on
$\Sigma$\cmmnt{ (363K)};  if $\nu\in M_{\sigma}$, the corresponding
member of $(L^{\infty})^{\sim}_c$ is the unique
order-bounded\cmmnt{ (or norm-continuous)} linear functional $f$ on
$L^{\infty}$ such that
$f(\chi E)=\nu E$ for every $E\in\Sigma$.   \cmmnt{If $\nu\ge 0$, so
that $\nu$
is a totally finite measure with domain $\Sigma$, then of course
$f$, when interpreted as a functional on $\eusm L^{\infty}$, must be
just integration with respect to $\nu$.}

The identification between $(L^{\infty})^{\sim}_c$ and
$M_{\sigma}$\cmmnt{ described in 363K} is an $L$-space isomorphism.
So\cmmnt{ it tells us, for instance, that if we are willing to use the
symbol $\dashint$ for the duality between $L^{\infty}$ and the space of
bounded finitely additive functionals on $\Sigma$, as in 363L, then we
can write}

\Centerline{$\dashint u\,d(\mu+\nu)=\dashint u\,d\mu+\dashint u\,d\nu$}

\noindent for every $u\in\eusm L^{\infty}$ and all $\mu$,
$\nu\in M_{\sigma}$.

\leader{437C}{Theorem} Let $X$ be a set and $U$ a Riesz subspace of
$\ell^{\infty}(X)$\cmmnt{, the $M$-space of bounded real-valued
functions on $X$,} containing the constant functions.

(a) Let $\Sigma$ be the smallest $\sigma$-algebra of subsets of $X$ with
respect to which every member of $U$ is measurable.   Let
$M_{\sigma}=M_{\sigma}(\Sigma)$ be the $L$-space of countably additive
functionals on $\Sigma$\cmmnt{ (326I\formerly{3{}26E}, 362B)}.
Then there is a Banach lattice
isomorphism $T:M_{\sigma}\to U^{\sim}_{\sigma}$ defined by saying that
$(T\mu)(u)=\int u\,d\mu$ whenever $\mu\in M_{\sigma}^+$ and $u\in U$.

(b) We\cmmnt{ now} have
a sequentially order-continuous norm-preserving Riesz
homomorphism $S$, embedding the
$M$-space $\eusm L^{\infty}=\eusm L^{\infty}(\Sigma)$ of bounded
real-valued
$\Sigma$-measurable functions on $X$\cmmnt{ (363Hb)} into the
$M$-space $(U^{\sim}_{\sigma})^{\sim}=(U^{\sim}_{\sigma})^*
=(U^{\sim}_{\sigma})^{\times}$, defined by saying that
$(Sv)(T\mu)=\int v\,d\mu$ whenever $v\in\eusm L^{\infty}$ and
$\mu\in M_{\sigma}^+$.   If
$u\in U$, then $(Su)(f)=f(u)$ for every $f\in U^{\sim}_{\sigma}$.

\proof{{\bf (a)(i)} The norm $\|\,\|_{\infty}$ is an order-unit norm on
$U$ (354Ga), so $U^*=U^{\sim}$ is an $L$-space (356N), and the band
$U^{\sim}_{\sigma}$ (437Aa) is an $L$-space in its own right (354O).

\medskip

\quad{\bf (ii)} As noted in 437B, we have a Banach lattice isomorphism
$T_0:M_{\sigma}\to(\eusm L^{\infty})^{\sim}_c$ defined by saying that
$(T_0\mu)(u)=\int u\,d\mu$ whenever $u\in\eusm L^{\infty}$ and
$\mu\in M_{\sigma}^+$.   If we set $T\mu=T_0\mu\restr U$, then $T$
is a positive linear operator from $M_{\sigma}$ to $U^{\sim}$, just
because $U$ is a linear subspace of $\eusm L^{\infty}$;  and since
$T\mu\in U^{\sim}_{\sigma}$ for every $\mu\in M_{\sigma}^+$, $T$ is an
operator from $M_{\sigma}$ to $U^{\sim}_{\sigma}$.   Now every
$f\in(U^{\sim}_{\sigma})^+$ is of the form $T\mu$ for some
$\mu\in M_{\sigma}^+$.   \Prf\ By 436D, there is some measure $\lambda$
such that $\int u\,d\lambda=f(u)$ for every $u\in U$.   Completing
$\lambda$ if necessary, we see that we may suppose that every member of
$U$ is $(\dom\lambda)$-measurable, that is, that
$\Sigma\subseteq\dom\lambda$;  take $\mu=\lambda\restr\Sigma$.\ \QeD\
So $T$ is surjective.

\medskip

\quad{\bf (iii)} Write $\Cal K$ for the family of sets $K\subseteq X$
such that $\chi K=\inf_{n\in\Bbb N}u_n$ for some sequence
$\sequencen{u_n}$ in $U$.   (See the proof of 436D.)   We need to know
the following.
($\alpha$) $\Cal K\subseteq\Sigma$.   ($\beta$) If $K\in\Cal K$, then
there is a non-increasing sequence $\sequencen{u_n}$ in $U$ such that
$\chi K=\inf_{n\in\Bbb N}u_n$.   (For if $\sequencen{u'_n}$ is any
sequence in $U$ such that $\chi K=\inf_{n\in\Bbb N}u'_n$, we can set
$u_n=\inf_{i\le n}u'_i$ for each $i$.)   ($\gamma$) The $\sigma$-algebra
of subsets of $X$ generated by $\Cal K$ is $\Sigma$.   \Prf\ Let $\Tau$
be the $\sigma$-algebra of subsets of $X$ generated by $\Cal K$.
$\Tau\subseteq\Sigma$ because $\Cal K\subseteq\Sigma$.   If $u\in U$ and
$\alpha>0$ then $\{x:u(x)\ge\alpha\}\in\Cal K$ (see part (b) of the
proof of 436D).   So every member of $U^+$, therefore every member of
$U$, is $\Tau$-measurable, and $\Sigma\subseteq\Tau$.\ \Qed

\medskip

\quad{\bf (iv)} $T$ is injective.   \Prf\ If $\mu_1$,
$\mu_2\in M_{\sigma}$ and $T\mu_1=T\mu_2$, set
$\nu_i=\mu_i+\mu_1^-+\mu_2^-$ for
each $i$, so that $\nu_i$ is non-negative and $T\nu_1=T\nu_2$.   If
$K\in\Cal K$ then there is a non-increasing sequence $\sequencen{u_n}$
in $U$ such that $\chi K=\inf_{n\in\Bbb N}u_n$ in $\Bbb R^X$, so

\Centerline{$\nu_1K=\inf_{n\in\Bbb N}\biggerint u_nd\nu_1
=\inf_{n\in\Bbb N}\int u_nd\nu_2
=\nu_2K$.}

\noindent Now $\Cal K$ contains $X$ and is closed under finite
intersections and $\nu_1$ and $\nu_2$ agree on $\Cal K$.   By the
Monotone Class Theorem (136C), $\nu_1$ and $\nu_2$ agree on the
$\sigma$-algebra generated by $\Cal K$, which is $\Sigma$;  so
$\nu_1=\nu_2$ and $\mu_1=\mu_2$.\ \Qed

Thus $T$ is a linear space isomorphism between $M_{\sigma}$ and
$U^{\sim}_{\sigma}$.

\medskip

\quad{\bf (v)} As noted in (ii),
$T[M_{\sigma}^+]=(U^{\sim}_{\sigma})^+$;  so $T$ is a Riesz space
isomorphism.

\medskip

\quad{\bf (vi)} Now if $\mu\in M_{\sigma}$,

$$\eqalignno{\|T\mu\|
&=|T\mu|(\chi X)\cr
\noalign{\noindent (356Nb)}
&=(T|\mu|)(\chi X)\cr
\noalign{\noindent (because $T$ is a Riesz homomorphism)}
&=|\mu|(X)
=\|\mu\|\cr}$$

\noindent (362Ba).   So $T$ is norm-preserving and is an $L$-space
isomorphism, as claimed.

\medskip

{\bf (b)(i)} By 356Pb, $(U^{\sim}_{\sigma})^*=(U^{\sim}_{\sigma})^{\sim}
=(U^{\sim}_{\sigma})^{\times}$ is an $M$-space.

\medskip

\quad{\bf (ii)} We have a canonical map
$S_0:\eusm L^{\infty}\to((\eusm L^{\infty})^{\sim}_c)^{\times}$ defined
by saying that $(S_0v)(h)=h(v)$
for every $v\in\eusm L^{\infty}$ and $h\in(\eusm L^{\infty})^{\sim}_c$;
and by 356F, $S_0$ is a Riesz homomorphism.   If $\sequencen{v_n}$ is a
non-increasing sequence in $\eusm L^{\infty}$ with infimum $0$, then
$\inf_{n\in\Bbb N}(S_0v_n)(h)=\inf_{n\in\Bbb N}h(v_n)=0$ for every
$h\in((\eusm L^{\infty})^{\sim}_c)^+$, so $\inf_{n\in\Bbb N}S_0v_n=0$
(355Ee);  as $\sequencen{v_n}$ is arbitrary, $S_0$ is sequentially
order-continuous (351Gb).

Also $S_0$ is norm-preserving.   \Prf\ ($\alpha$) If
$h\in(\eusm L^{\infty})^{\sim}_c$ and $v\in\eusm L^{\infty}$, then

\Centerline{$|(S_0v)(h)|=|h(v)|\le\|h\|\|v\|_{\infty}$,}

\noindent so $\|S_0v\|\le\|v\|_{\infty}$.   ($\beta$)
If $v\in\eusm L^{\infty}$ and $0\le\gamma<\|v\|_{\infty}$, take $x\in X$
such that $|v(x)|>\gamma$, and define $h_x\in(\eusm
L^{\infty})^{\sim}_c$ by setting $h_x(w)=w(x)$ for every
$w\in\eusm L^{\infty}$;  then $\|h_x\|=1$ and

\Centerline{$|(S_0v)(h_x)|=|h_x(v)|=|v(x)|\ge\gamma$,}

\noindent so $\|S_0v\|\ge\gamma$.   As $\gamma$ is arbitrary,
$\|S_0v\|\ge\|v\|_{\infty}$ and $\|S_0v\|=\|v\|_{\infty}$.\ \Qed

\medskip

\quad{\bf (iii)} Now $T_0:M_{\sigma}\to(\eusm L^{\infty})^{\sim}_c$ and
$T:M_{\sigma}\to U^{\sim}_{\sigma}$ are both norm-preserving Riesz space
isomorphisms, so $T_0T^{-1}:U^{\sim}_{\sigma}\to(\eusm
L^{\infty})^{\sim}_c$ is another, and its adjoint $S_1:((\eusm
L^{\infty})^{\sim}_c)^*\to(U^{\sim}_{\sigma})^*$ must also be a
norm-preserving Riesz space isomorphism.   So if we set $S=S_1S_0$, $S$
will be a norm-preserving sequentially order-continuous Riesz
homomorphism from $\eusm L^{\infty}$ to
$(U^{\sim}_{\sigma})^{\times}=(U^{\sim}_{\sigma})^*$.

\medskip

\quad{\bf (iv)} Setting the construction out in this way tells us a lot
about the properties of the operator $S$, but undeniably leaves it
somewhat obscure.   So let us start again from $v\in\eusm L^{\infty}$
and $\mu\in M_{\sigma}^+$, and seek to calculate
$(Sv)(T\mu)$.   We have

$$\eqalignno{(Sv)(T\mu)
&=(S_1S_0v)(T\mu)
=(S_0v)(T_0T^{-1}T\mu)\cr
\noalign{\noindent (because $S_1$ is the adjoint of $T_0T^{-1}$)}
&=(S_0v)(T_0\mu)
=(T_0\mu)(v)
=\int v\,d\mu,\cr}$$

\noindent as claimed.

If $u\in U$, then $(T\mu)(u)=(T_0\mu)(u)$ for every $\mu\in M_{\sigma}$,
so if $f\in U^{\sim}_{\sigma}$ then

\Centerline{$(Su)(f)
=(S_1S_0u)(f)
=(S_0u)(T_0T^{-1}f)
=(T_0T^{-1}f)(u)
=(TT^{-1}f)(u)
=f(u)$.}

\noindent This completes the proof.
}%end of proof of 437C

\cmmnt{
\leader{437D}{Remarks} What is happening here is that the
canonical Riesz homomorphism $u\mapsto\hat u$ from $U$ to
$(U^{\sim}_{\sigma})^*$ (356F) has a natural extension to
$\eusm L^{\infty}(\Sigma)$.   The original homomorphism $u\mapsto\hat u$
is not, as a rule, sequentially order-continuous, just because
$U^{\sim}_{\sigma}$ is generally larger than $U^{\sim}_c$;  but the
extension to $\eusm L^{\infty}$ is sequentially order-continuous.   If
you like, it is sequential smoothness which is carried over to the
extension, and because the embedding of $\eusm L^{\infty}$ in $\Bbb R^X$
is sequentially order-continuous, a sequentially smooth operator on
$\eusm L^{\infty}$ is sequentially order-continuous.

\cmmnt{In the statement of 437C, I have used the formulae
$(T\mu)(u)=\int u\,d\mu$ and $(Sv)(T\mu)=\int v\,d\mu$ on the assumption
that $\mu\in M_{\sigma}^+$, so that $\mu$ is actually a measure on the
definition used in this treatise, and $\int\,d\mu$ is the ordinary
integral as constructed in \S122.   Since the functions $u$ and $v$ are
bounded, measurable and defined everywhere, we can choose to extend the
notion of integration to signed measures, as in 363L, in which case the
formulae $(T\mu)(u)=\dashint u\,d\mu$ and $(Sv)(T\mu)=\dashint v\,d\mu$
become meaningful, and true, for all $\mu\in M_{\sigma}$, $u\in U$ and
$v\in\eusm L^{\infty}$.}

In fact the ideas here can be pushed farther, as in 437Ib, 437Xf and
437Yd.
}%end of comment

\leader{437E}{Corollary} Let $X$ be a completely regular Hausdorff
space, and $\CalBa\cmmnt{\mskip5mu=\CalBa(X)}$ its Baire $\sigma$-algebra.
Then we can identify
$C_b(X)^{\sim}_{\sigma}$ with the $L$-space $M_{\sigma}(\CalBa)$ of
countably additive functionals on $\CalBa$, and we have a
norm-preserving sequentially order-continuous Riesz homomorphism $S$ from
$\eusm L^{\infty}(\CalBa)$ to $(C_b(X)^{\sim}_{\sigma})^*$ defined by
setting $(Sv)(f)=\int v\,d\mu_f$ for every $v\in\eusm L^{\infty}$ and
$f\in(C_b(X)^{\sim}_{\sigma})^+$, where $\mu_f$ is the Baire measure
associated with $f$.

\proof{ Apply 437C with $U=C_b(X)$ (cf.\ 436E).}

\leader{437F}{Proposition} Let $X$ be a topological space and
$\Cal B\cmmnt{\mskip5mu=\Cal B(X)}$ its
Borel $\sigma$-algebra.   Let $M_{\sigma}$ be the $L$-space of countably
additive functionals on $\Cal B$.

(a) Write $M_{\tau}\subseteq M_{\sigma}$ for the set of differences of
$\tau$-additive totally finite Borel measures.   Then $M_{\tau}$ is a
band in $M_{\sigma}$, so is an $L$-space in its own right.

(b) Write $M_t\subseteq M_{\tau}$ for the set of
differences of totally finite Borel measures which are
tight\cmmnt{ (that is, inner regular with respect to the
closed compact sets)}.   Then $M_t$ is a band in $M_{\sigma}$, so is an
$L$-space in its own right.

\proof{{\bf (a)(i)} Let $\mu_1$, $\mu_2$ be totally finite
$\tau$-additive
Borel measures on $X$, $\alpha\ge 0$, and $\mu\in M_{\sigma}$ such that
$0\le\mu\le\mu_1$.   Then $\mu_1+\mu_2$, $\alpha\mu_1$ and $\mu$ are
totally finite $\tau$-additive Borel measures.   \Prf\ They all belong
to $M_{\sigma}$, that is, are totally finite Borel measures.
Now let $\Cal G$ be a
non-empty upwards-directed family of open sets in $X$ with union $H$,
and $\epsilon>0$.   Then there are $G_1$, $G_2\in\Cal G$ such that
$\mu_1G_1\ge\mu_1H-\epsilon$ and $\mu_2G_2\ge\mu_2H-\epsilon$, and a
$G\in\Cal G$ such that $G\supseteq G_1\cup G_2$.   In this case,

\Centerline{$(\mu_1+\mu_2)(G)\ge(\mu_1+\mu_2)(H)-2\epsilon$,}

\Centerline{$(\alpha\mu_1)(G)\ge(\alpha\mu_1)(H)-\alpha\epsilon$}

\noindent and

\Centerline{$\mu G=\mu H-\mu(H\setminus G)\ge\mu H-\mu_1(H\setminus G)
\ge\mu H-\epsilon$.}

\noindent As $\Cal G$ and $\epsilon$ are arbitrary, $\mu_1+\mu_2$,
$\alpha\mu_1$ and $\mu$ are all $\tau$-additive.\ \Qed

It follows that $M_{\tau}$ is a solid linear subspace of $M_{\sigma}$.

\medskip

\quad{\bf (ii)} Now suppose that $B\subseteq M_{\tau}^+$ is non-empty
and upwards-directed and has a supremum $\nu$ in $M_{\sigma}$.   Then
$\nu\in M_{\tau}$.   \Prf\  If $\Cal G$ is a non-empty
upwards-directed family of open sets with union $H$, then

$$\eqalignno{\nu H
&=\sup_{\mu\in B}\mu H\cr
\noalign{\noindent (362Be)}
&=\sup_{\mu\in B,G\in\Cal G}\mu G
=\sup_{G\in\Cal G}\nu G;\cr}$$

\noindent as $\Cal G$ is arbitrary, $\nu$ is $\tau$-additive and belongs
to $M_{\tau}$.\ \Qed

As $B$ is arbitrary, $M_{\tau}$ is a band in $M_{\sigma}$.   By 354O, it
is itself an $L$-space.

\medskip

{\bf (b)} We can use the same arguments.   Suppose that $\mu_1$,
$\mu_2\in M_{\sigma}^+$ are tight,
$\alpha\ge 0$, and $\mu\in M_{\sigma}$ is such that
$0\le\mu\le\mu_1$.   If $E\in\Cal B$ and $\epsilon>0$, there are
closed compact sets $K_1$, $K_2\subseteq E$ such that
$\mu_1K_1\ge\mu_1E-\epsilon$ and $\mu_2K_2\ge\mu_2E-\epsilon$.   Set
$K=K_1\cup K_2$, so that $K$ also is a closed compact subset of $E$.
Then

\Centerline{$(\mu_1+\mu_2)(K)\ge(\mu_1+\mu_2)(E)-2\epsilon$,}

\Centerline{$(\alpha\mu_1)(K)\ge(\alpha\mu_1)(E)-\alpha\epsilon$}

\noindent and

\Centerline{$\mu K=\mu E-\mu(E\setminus K)\ge\mu E-\mu_1(E\setminus K)
\ge\mu E-\epsilon$.}

\noindent As $\Cal G$ and $\epsilon$ are arbitrary, $\mu_1+\mu_2$,
$\alpha\mu_1$ and $\mu$ are all tight;  as $\mu_1$, $\mu_2$, $\mu$ and
$\alpha$ are arbitrary, $M_t$ is a solid linear subspace of $M_{\sigma}$.

Now suppose that $B\subseteq M_t^+$ is non-empty and
upwards-directed and has a supremum $\nu$ in $M_{\sigma}$.   Take any
$E\in\Cal B$ and $\epsilon>0$.   Then there is a $\mu\in B$ such that
$\mu E\ge\nu E-\epsilon$;  there is a closed compact set $K\subseteq E$
such that $\mu K\ge\mu E-\epsilon$;  and now $\nu K\ge\nu E-2\epsilon$.
As $E$ and $\epsilon$ are arbitrary, $\nu$ is tight;  as $B$ is
arbitrary, $M_t$ is a band in $M_{\sigma}$, and is in itself an
$L$-space.
}%end of proof of 437F

\leader{437G}{Definitions} Let $X$ be a topological space.   A {\bf
signed Baire measure} on $X$ will be a countably additive functional on
the Baire
$\sigma$-algebra $\CalBa(X)$\cmmnt{, which by the Jordan decomposition
theorem (231F) is expressible as the difference of two totally finite
Baire measures};
a {\bf signed Borel measure} will be a countably additive functional on
the Borel $\sigma$-algebra $\Cal B(X)$\cmmnt{, that is, the difference
of two totally finite
Borel measures};  a {\bf signed $\tau$-additive Borel measure} will
be\cmmnt{ a member of the $L$-space $M_{\tau}$ as described in 437F,
that is,} the difference of two $\tau$-additive totally finite
Borel measures;  and a {\bf signed
tight Borel measure} will be\cmmnt{ a member of the $L$-space $M_t$ as
described in 437F, that is,}
the difference of two tight totally finite Borel measures.

\leader{437H}{Theorem} Let $X$ be a set and $U$ a Riesz subspace of
$\ell^{\infty}(X)$ containing the constant functions.   Let $\frak T$ be
the coarsest topology on $X$ rendering every member of $U$ continuous,
and $\Cal B$ the corresponding Borel $\sigma$-algebra.

(a) Let $M_{\tau}$ be the $L$-space of signed
$\tau$-additive Borel measures on $X$.   Then we have a Banach
lattice isomorphism $T:M_{\tau}\to U^{\sim}_{\tau}$ defined by saying
that $(T\mu)(u)=\int u\,d\mu$ whenever $\mu\in M_{\tau}^+$ and $u\in U$.

(b) We\cmmnt{ now} have
a sequentially order-continuous norm-preserving Riesz
homomorphism $S$, embedding\cmmnt{ the
$M$-space} $\cmmnt{\eusm L^{\infty}=\mskip5mu}\eusm L^{\infty}(\Cal
B)$\cmmnt{ of bounded Borel
measurable functions on $X$} into
$(U^{\sim}_{\tau})^{\sim}=(U^{\sim}_{\tau})^*
=(U^{\sim}_{\tau})^{\times}$, defined by saying that
$(Sv)(T\mu)=\int v\,d\mu$ whenever $v\in\eusm L^{\infty}$ and
$\mu\in M_{\tau}^+$.   If
$u\in U$, then $(Su)(f)=f(u)$ for every $f\in U^{\sim}_{\tau}$.

\proof{ The proof follows the same lines as that of 437C.

\medskip

{\bf (a)(i)} As before, the norm $\|\,\|_{\infty}$ is an order-unit norm
on $U$, so $U^*=U^{\sim}$ is an $L$-space, and the band
$U^{\sim}_{\tau}$ (437Ab) is an $L$-space in its own right, like
$M_{\tau}$ (437F).

\medskip

\quad{\bf (ii)} Let $M_{\sigma}$ be the $L$-space of all countably
additive functionals on $\Cal B$, so that $M_{\tau}$ is a band in
$M_{\sigma}$.   Let $T_0:M_{\sigma}\to(\eusm L^{\infty})^{\sim}_c$ be
the canonical Banach lattice isomorphism defined by saying that
$(T_0\mu)(u)=\int u\,d\mu$ whenever $u\in\eusm L^{\infty}$ and
$\mu\in M_{\sigma}^+$.   If we set
$T\mu=T_0\mu\restr U$ for $\mu\in M_{\tau}$,
then $T$ is a positive linear operator from $M_{\tau}$ to $U^{\sim}$,
just because $U$ is a Riesz subspace of $\eusm L^{\infty}$;  and since
$T\mu\in U^{\sim}_{\tau}$ for every $\mu\in M_{\tau}^+$ (436H), $T$ is
an operator from $M_{\tau}$ to $U^{\sim}_{\tau}$.   Now every
$f\in(U^{\sim}_{\tau})^+$ is of the form $T\mu$ for some
$\mu\in M_{\tau}^+$.
\Prf\ By 436H, there is a quasi-Radon measure $\lambda$
such that $\int u\,d\lambda=f(u)$ for every $u\in U$;  set
$\mu=\lambda\restr\Cal B$.\ \QeD\  So $T$ is surjective.

\medskip

\quad{\bf (iii)} Let $\Cal K$ be the family of subsets $K$ of $X$ such
that $\chi K=\inf A$ in $\Bbb R^X$ for some non-empty subset $A$ of $U$.
Then $\Cal K$ is just the family of closed sets for $\frak T$.   \Prf\
As noted in part (b) of the proof of 436H, every member of $\Cal K$ is
closed, and $K\setminus G\in\Cal K$ whenever $K\in\Cal K$ and
$G\in\frak T$;  but as, in the present case, $X\in\Cal K$, every closed
set belongs to $\Cal K$.\ \Qed

\medskip

\quad{\bf (iv)} $T$ is injective.   \Prf\ If $\mu_1$, $\mu_2\in
M_{\tau}$ and $T\mu_1=T\mu_2$, set $\nu_i=\mu_i+\mu_1^-+\mu_2^-$ for
each $i$, so that $\nu_i$ is non-negative and $T\nu_1=T\nu_2$.   If
$K\in\Cal K$, set $A=\{u:u\in U,\,u\ge\chi K\}$, so that $\chi
K=\inf A$ in $\Bbb R^X$, and $A$ is downwards-directed.   By 414Bb,

\Centerline{$\nu_1K=\inf_{u\in A}\int u\,d\nu_1
=\inf_{u\in A}\int u\,d\nu_2=\nu_2K$.}

\noindent Now $\Cal K$ contains $X$ and is closed under finite
intersections and $\nu_1$ and $\nu_2$ agree on $\Cal K$.   By the
Monotone Class Theorem, $\nu_1$ and $\nu_2$ agree on the
$\sigma$-algebra generated by $\Cal K$, which is $\Cal B$;  so
$\nu_1=\nu_2$ and $\mu_1=\mu_2$.\ \Qed

Thus $T$ is a linear space isomorphism between $M_{\tau}$ and
$U^{\sim}_{\tau}$.

\medskip

\quad{\bf (v)} As noted in (ii), $T[M_{\tau}^+]=(U^{\sim}_{\tau})^+$;
so $T$ is a Riesz space isomorphism.

\medskip

\quad{\bf (vi)} Now if $\mu\in M_{\tau}$,

\Centerline{$\|T\mu\|
=|T\mu|(\chi X)
=(T|\mu|)(\chi X)
=|\mu|(X)
=\|\mu\|$.}

\noindent So $T$ is norm-preserving and is an $L$-space isomorphism, as
claimed.

\medskip

{\bf (b)(i)} By 356Pb, $(U^{\sim}_{\tau})^*=(U^{\sim}_{\tau})^{\sim}
=(U^{\sim}_{\tau})^{\times}$ is an $M$-space.

\medskip

\quad{\bf (ii)} Because $T_0:M_{\sigma}\to(\eusm L^{\infty})^{\sim}_c$
is a Banach lattice isomorphism, and $M_{\tau}$ is a band in
$M_{\sigma}$, $W=T_0[M_{\tau}]$ is a band in
$(\eusm L^{\infty})^{\sim}_c$.   We therefore have a Riesz homomorphism
$S_0:\eusm L^{\infty}\to W^{\times}$ defined by writing $(S_0v)(h)=h(v)$
for $v\in\eusm L^{\infty}$, $h\in W$ (356F).   Just as in (b-ii) of the
proof of 437C, $S_0$ is sequentially order-continuous and
norm-preserving.   (We need to observe that $h_x$ in the second half of
the argument there always belongs to $W$;  this is because
$h_x=T_0(\delta_x)$, where $\delta_x\in M_{\tau}$ is defined by setting
$\delta_x(E)=\chi E(x)$ for every Borel set $E$.)

\medskip

\quad{\bf (iii)} Now $T_0:M_{\sigma}\to(\eusm L^{\infty})^{\sim}_c$ and
$T:M_{\tau}\to U^{\sim}_{\tau}$ are both norm-preserving Riesz space
isomorphisms, so $T_0T^{-1}:U^{\sim}_{\tau}\to W$ is another, and its
adjoint $S_1:W^*\to(U^{\sim}_{\tau})^*$ must also be a
norm-preserving Riesz space isomorphism.   So if we set $S=S_1S_0$, $S$
will be a norm-preserving sequentially order-continuous Riesz
homomorphism from $\eusm L^{\infty}$ to
$(U^{\sim}_{\tau})^{\times}=(U^{\sim}_{\tau})^*$.

\medskip

\quad{\bf (iv)} If $v\in\eusm L^{\infty}$ and $\mu\in M_{\tau}^+$,

$$\eqalignno{(Sv)(T\mu)
&=(S_1S_0v)(T\mu)
=(S_0v)(T_0T^{-1}T\mu)\cr
&=(S_0v)(T_0\mu)
=(T_0\mu)(v)
=\int v\,d\mu;\cr}$$

\noindent if $u\in U$ and $f\in U^{\sim}_{\tau}$, then
$(T\mu)(u)=(T_0\mu)(u)$ for every $\mu\in M_{\tau}$, so

\Centerline{$(Su)(f)=(S_1S_0u)(f)=(S_0u)(T_0T^{-1}f)
=(T_0T^{-1}f)(u)=(TT^{-1}f)(u)=f(u)$.}
}%end of proof of 437H

\leader{437I}{Proposition} Let $X$ be a locally compact Hausdorff space,
$\Cal B\cmmnt{\mskip5mu=\Cal B(X)}$
its Borel $\sigma$-algebra, and $\eusm L^{\infty}(\Cal B)$
the $M$-space of  bounded Borel measurable real-valued functions on $X$.

(a) Let $M_t$ be the $L$-space of signed tight Borel measures on $X$.
Then we have
a Banach lattice isomorphism $T:M_t\to C_0(X)^*$ defined by saying that
$(T\mu)(u)=\int u\,d\mu$ whenever $\mu\in M_t^+$ and $u\in C_0(X)$.

(b) Let $\Sigma_{\text{uRm}}$ be the algebra of universally
Radon-measurable subsets of $X$\cmmnt{ (definition: 434E)}, and
$\eusm L^{\infty}(\Sigma_{\text{uRm}})$ the $M$-space of bounded
$\Sigma_{\text{uRm}}$-measurable real-valued functions on $X$.   Then we
have a norm-preserving sequentially order-continuous Riesz
homomorphism $S:\eusm L^{\infty}(\Sigma_{\text{uRm}})\to C_0(X)^{**}$
defined by saying that $(Sv)(T\mu)=\int v\,d\mu$ whenever
$v\in\eusm L^{\infty}(\Sigma_{\text{uRm}})$ and
$\mu\in M_t^+$;  and $(Su)(f)=f(u)$ for every $u\in C_0(X)$,
$f\in C_0(X)^*$.

\proof{{\bf (a)} The point is just that in this context $M_t$ is equal
to $M_{\tau}$, as defined in 437F-437H (416H), while
$C_0(X)^*=C_0(X)^{\sim}_{\tau}$ (see part (a) of the proof of 436J), and
the topology of $X$ is completely regular, so we just have a special
case of 437Ha.

\medskip

{\bf (b)(i)} As in 437Hb, we have a
sequentially order-continuous Riesz
homomorphism $S_0:\eusm L^{\infty}(\Cal B)\to C_0(X)^{**}$ defined by
saying that $(S_0v)(T\nu)=\int v\,d\nu$ whenever
$v\in\eusm L^{\infty}(\Cal B)$ and $\nu\in M_t^+$.

\medskip

\quad{\bf (ii)} If $v\in\eusm L^{\infty}(\Sigma_{\text{uRm}})$, then

\Centerline{$\sup\{S_0w:w\in\eusm L^{\infty}(\Cal B),\,w\le v\}
=\inf\{S_0w:w\in\eusm L^{\infty}(\Cal B),\,w\ge v\}$}

\noindent in $C_0(X)^{**}$.   \Prf\ Set

\Centerline{$A=\{w:w\in\eusm L^{\infty}(\Cal B),\,w\le v\}$,
\quad$B=\{w:w\in\eusm L^{\infty}(\Cal B),\,w\ge v\}$.}

\noindent Because the constant functions belong to
$\eusm L^{\infty}(\Cal B)$, $A$ and $B$ are both non-empty;   of course
$w\le w'$ and $S_0w\le S_0w'$ for every $w\in A$ and $w'\in B$;  because
$C_0(X)^{**}$ is Dedekind complete, $\phi=\sup S_0[A]$ and
$\psi=\inf S_0[B]$ are both defined in $C_0(X)^{**}$, and $\phi\le\psi$.
If $f\ge 0$ in $C_0(X)^*$, then there is a $\nu\in M_t^+$ such that
$T\nu=f$.
Since $v$ is $\nu$-virtually measurable (see 434Ec), there are (bounded)
Borel measurable functions $w$, $w'$ such that $w\le v\le w'$ and
$w=w'\,\,\nu$-a.e., that is, $w\in A$, $w'\in B$ and

\Centerline{$(S_0w)(f)=\biggerint w\,d\nu=\int w'd\nu=(S_0w')(f)$.}

\noindent But as

\Centerline{$(S_0w)(f)\le\phi(f)\le\psi(f)\le(S_0w')(f)$,}

\noindent $\phi(f)=\psi(f)$;  as $f$ is arbitrary, $\phi=\psi$.\ \Qed

\medskip

\quad{\bf (iii)} We can therefore define
$S:\eusm L^{\infty}(\Sigma_{\text{uRm}})\to C_0(X)^{**}$ by setting

\Centerline{$Sv=\sup\{S_0w:w\in\eusm L^{\infty}(\Cal B),\,w\le v\}
=\inf\{S_0w:w\in\eusm L^{\infty}(\Cal B),\,w\ge v\}$}

\noindent for every $v\in\eusm L^{\infty}$.   The argument in (ii) tells
us also that $(Sv)(T\nu)=\int v\,d\nu$ for every $\nu\in M_t^+$;  that
is, that $(Sv)(T\mu)=\int v\,d\mu$ for every Radon measure $\mu$ on
$X$.

\medskip

\quad{\bf (iv)} Now $S$ is a norm-preserving
sequentially order-continuous Riesz homomorphism.   \Prf\ (Compare
355F.)   ($\alpha$) The
non-trivial part of this is actually the check that $S$ is additive.
But the formula

\Centerline{$Sv=\sup\{S_0w:w\in\eusm L^{\infty}(\Cal B),\,w\le v\}$}

\noindent ensures that $Sv_1+Sv_2\le S(v_1+v_2)$ for all $v_1$,
$v_2\in\eusm L^{\infty}$, while the formula

\Centerline{$Sv=\inf\{S_0w:w\in\eusm L^{\infty}(\Cal B),\,w\ge v\}$}

\noindent ensures that $Sv_1+Sv_2\ge S(v_1+v_2)$ for all $v_1$, $v_2$.
($\beta$) It is easy to check that $S(\alpha v)=\alpha Sv$ whenever
$v\in\eusm L^{\infty}(\Sigma_{\text{uRm}})$ and $\alpha>0$, so that $S$
is linear.   ($\gamma$) If $v_1\wedge v_2=0$ in
$\eusm L^{\infty}(\Sigma_{\text{uRm}})$,

$$\eqalignno{Sv_1\wedge Sv_2
&=\sup\{S_0w:w\in\eusm L^{\infty}(\Cal B),\,w\le v_1\}
    \wedge\sup\{S_0w:w\in\eusm L^{\infty}(\Cal B),\,w\le v_2\}\cr
&=\sup\{S_0w_1\wedge S_0w_2:w_1,\,w_2\in\eusm L^{\infty}(\Cal B),\,
   w_1\le v_1,\, w_2\le v_2\}\cr
\displaycause{352Ea}
&=\sup\{S_0(w_1\wedge w_2):w_1,\,w_2\in\eusm L^{\infty}(\Cal B),\,
   w_1\le v_1,\, w_2\le v_2\}\cr
\displaycause{because $S_0$ is a Riesz homomorphism}
&=0.\cr}$$

\noindent So $S$ is a Riesz homomorphism (352G(iv)).   ($\delta$) Now
suppose that $\sequencen{v_n}$ is a non-increasing sequence in
$\eusm L^{\infty}(\Sigma_{\text{uRm}})$ with infimum $0$ in
$\eusm L^{\infty}(\Sigma_{\text{uRm}})$.   Then
$\inf_{n\in\Bbb N}v_n(x)=0$ for every $x\in X$, so
$\inf_{n\in\Bbb N}\int v_nd\nu=0$ for every $\nu\in M_t^+$ and
$\inf_{n\in\Bbb N}(Sv_n)(f)=0$ for every $f\in(C_0(X)^*)^+$.   So
$\inf_{n\in\Bbb N}Sv_n=0$;  as $\sequencen{v_n}$ is arbitrary, $S$ is
sequentially
order-continuous.   ($\epsilon$)
If $v\in\eusm L^{\infty}(\Sigma_{\text{uRm}})$,
then $|v|\le\|v\|_{\infty}\chi X$, so

\Centerline{$\|Sv\|\le\|v\|_{\infty}\|S(\chi X)\|=\|v\|_{\infty}$.}

\noindent On the other hand, for any $x\in X$ we have the Dirac measure
$\delta_x$ on $X$ concentrated at $x$, with the matching
functional $h_x\in C_0(X)^*$, and

\Centerline{$\|Sv\|=\||Sv|\|=\|S|v|\|\ge(S|v|)(h_x)
=\biggerint|v|d\delta_x=|v(x)|$,}

\noindent so $\|Sv\|\ge\|v\|_{\infty}$;  thus $S$ is
norm-preserving.\ \Qed
}%end of proof of 437I

\cmmnt{\medskip

\noindent{\bf Remark} As in 437D, we can write
$(Sv)(T\mu)=\dashint v\,d\mu$ whenever $v\in\eusm L^{\infty}(\Cal B)$
and $u\in U$ and $\mu\in M_{\tau}$ (in 437H) or
$v\in\eusm L^{\infty}(\Sigma_{\text{uRm}})$ and $u\in C_0(X)$ and
$\mu\in M_t$ (in 437I).
}%end of comment

\leader{437J}{Vague and narrow topologies}\cmmnt{ We are ready for
another look
at `vague' topologies on spaces of measures.}   Let $X$ be a topological
space.

\spheader 437Ja Let $\Sigma$ be an algebra of subsets of $X$.   I will
say that $\Sigma$ {\bf separates zero sets} if whenever $F$,
$F'\subseteq X$ are disjoint zero sets then there is an $E\in\Sigma$
such that $F\subseteq E$ and $E\cap F'=\emptyset$.

\spheader 437Jb If $\Sigma$ is any algebra of subsets of $X$,
we can identify the Banach algebra and Banach lattice
$L^{\infty}(\Sigma)$\cmmnt{, as defined in \S363,} with the
$\|\,\|_{\infty}$-closed linear subspace of
$\ell^{\infty}(X)$ generated by $\{\chi E:E\in\Sigma\}$\cmmnt{ (363C,
363Ha)}.   If we do this, then $C_b(X)\subseteq L^{\infty}(\Sigma)$ iff
$\Sigma$ separates zero sets.   \prooflet{\Prf\ (i) Suppose that
$C_b(X)\subseteq L^{\infty}(\Sigma)$ and that $F_1$, $F_2\subseteq X$
are disjoint zero sets.   Let
$u_1$, $u_2:X\to\Bbb R$ be continuous functions such that
$F_i=u_i^{-1}[\{0\}]$ for both $i$;  then $|u_1(x)|+|u_2(x)|>0$ for
every $x$;  set
$v=\Bover{|u_1|}{|u_1|+|u_2|}$, so that $v:X\to[0,1]$ is continuous,
$v(x)=0$ for $x\in F_1$ and $v(x)=1$ for $x\in F_2$.   Now
$v\in C_b(X)\subseteq L^{\infty}(\Sigma)$,
so there is a $w\in S(\Sigma)$, the linear subspace of
$L^{\infty}(\Sigma)$ generated by $\{\chi E:E\in\Sigma\}$, such that
$\|v-w\|_{\infty}<\bover12$ (363C).
Set $E=\{x:w(x)\le\bover12\}$;  then $E\in\Sigma$ and
$F_1\subseteq E\subseteq X\setminus F_2$.   As $F_1$ and $F_2$ are
arbitrary, $\Sigma$ separates zero sets.

(ii) Now suppose that $\Sigma$ separates zero sets, that $u:X\to[0,1]$
is continuous, and that $n\ge 1$ is an integer.   For $i\le n$, set
$F_i=\{x:x\in X$, $u(x)\le\bover{i}{n}\}$, $F'_i=\{x:x\in X$,
$u(x)\ge\bover{i+1}{n}\}$.   Then $F_i$ and $F'_i$ are disjoint zero
sets so there is an
$E_i\in\Sigma$ such that $F'_i\subseteq E_i\subseteq X\setminus F_i$.
Set $w=\bover1n\sum_{i=1}^n\chi E_i\in S(\Sigma)$.   If $x\in X$, let
$j\le n$ be such that
$\bover{j}{n}\le u(x)<\bover{j+1}{n}$;  then for $i\le n$

\Centerline{$i<j\Longrightarrow x\in F'_i\Longrightarrow x\in E_i\Longrightarrow
x\notin F_i\Longrightarrow i\le j$,}

\noindent and $w(x)=\bover1n\#(\{i:i\le n$, $x\in E_i\})$ is either
$\bover{j}n$ or $\bover{j+1}n$.   Thus $|w(x)-u(x)|\le\bover1n$.  As $x$
is arbitrary,
$\|u-w\|_{\infty}\le\bover1n$;  as $n$ is arbitrary, $u\in
L^{\infty}(\Sigma)$.   As $L^{\infty}(\Sigma)$ is a linear subspace of
$\ell^{\infty}(X)$,
this is enough to show that $C_b(X)\subseteq L^{\infty}(\Sigma)$.\
\Qed}%end of prooflet

\spheader 437Jc It follows that if $\Sigma$ is an algebra of subsets of
$X$ separating the zero sets, and $\nu:\Sigma\to\Bbb R$ is a bounded
additive functional, we can speak of $\dashint u\,d\nu$
for any $u\in C_b(X)$\cmmnt{;  $\dashint d\nu$ is the
unique norm-continuous linear functional on $L^{\infty}(\Sigma)$ such
that $\dashint\chi E\,d\nu=\nu E$ for every $E\in\Sigma$ (363L)}.   The map
$\nu\mapsto\dashint\,d\nu$
is a Banach lattice isomorphism from the $L$-space $M(\Sigma)$ of
bounded additive functionals on $\Sigma$ to
$L^{\infty}(\Sigma)^*=L^{\infty}(\Sigma)^{\sim}$\cmmnt{ (363K)}.
We therefore have a positive linear operator $T:M(\Sigma)\to C_b(X)^*$
defined by setting $(T\nu)(u)=\dashint u\,d\nu$ for every
$\nu\in M(\Sigma)$ and $u\in C_b(X)$.
Except in the trivial case $X=\emptyset$, $\|T\|=1$\prooflet{ (if
$x\in X$, we have $\delta_x\in M(\Sigma)$ defined by setting
$\delta_x(E)=\chi E(x)$ for $E\in\Sigma$, and $\|T(\delta_x)\|=1$)}.

The {\bf vague topology} on $M(\Sigma)$ is\cmmnt{ now} the 
topology generated by
the functionals $\nu\mapsto\dashint u\,d\nu$ as $u$ runs over $C_b(X)$;
that is, the coarsest topology on $M(\Sigma)$ such that the canonical
map $T:M(\Sigma)\to C_b(X)^*$ is
continuous for the weak* topology of $C_b(X)^*$.   Because the
functionals $\nu\mapsto|\dashint u\,d\nu|$ are seminorms on $M(\Sigma)$,
the vague topology is a locally convex linear space topology.

\spheader 437Jd \cmmnt{There is a variant of the vague topology which
can be applied directly to spaces of (non-negative) totally finite
measures.}
Let $\tilde M^+$ be the set of all non-negative real-valued additive
functionals defined on algebras of subsets of $X$ which contain every
open set.
The {\bf narrow topology} on $\tilde M^+$ is that generated by sets of
the form

\Centerline{$\{\nu:\nu\in\tilde M^+$, $\nu G>\alpha\}$,
\quad$\{\nu:\nu\in\tilde M^+$, $\nu X<\alpha\}$}

\noindent for open sets $G\subseteq X$ and real numbers $\alpha$.
\cmmnt{(See {\smc Tops{\o}e 70b}, 8.1.)}

Observe that $\nu\mapsto\nu X:\tilde M^+\to\coint{0,\infty}$ is
continuous for the narrow topology, and if
$G\subseteq X$ is open then $\nu\mapsto\nu G$ is lower semi-continuous
for the narrow topology.
Writing $P_{\text{top}}$ for the set of topological probability
measures on $X$, then the narrow topology on $P_{\text{top}}$ is
generated by sets of the form
$\{\mu:\mu\in P_{\text{top}},\,\mu G>\alpha\}$ for open sets
$G\subseteq X$.

Writing $\tilde M^+_{\sigma}$ for the set of
totally finite topological measures on $X$, then
$\nu\mapsto\nu E:\tilde M^+_{\sigma}\to\coint{0,\infty}$ is Borel
measurable, for the narrow topology on $\tilde M^+_{\sigma}$, for every
Borel set $E\subseteq X$\prooflet{ (because the family of
sets $E$ for which
$\nu\mapsto\nu E$ is Borel measurable is a Dynkin class containing the
open sets)}.   Similarly,
$\nu\mapsto\int u\,d\nu:\tilde M^+_{\sigma}\to\Bbb R$ is Borel measurable for
every bounded measurable function $u:X\to\Bbb R$\cmmnt{,
being the limit of a sequence
of linear combinations of Borel measurable functions}.

\spheader 437Je Vague topologies, being linear space
topologies, are necessarily\cmmnt{ associated with uniformities (3A4Ad),
therefore} completely regular\cmmnt{ (4A2Ja)}.
In the very general context of (c) here\cmmnt{, in which
we have a space $M(\Sigma)$ of all finitely additive functionals on an
algebra $\Sigma$}, we do not expect the vague topology to be Hausdorff.
\cmmnt{But if we look
at particular subspaces, such as the space $M_{\sigma}(\CalBa(X))$ of
signed Baire measures, or the space
$M_{\tau}$ of signed $\tau$-additive Borel measures on a
completely regular space $X$, we may well have a Hausdorff vague
topology (437Xg).}

Similarly, the narrow topology on $\tilde M^+$ is rarely Hausdorff.
But on important subspaces we can get Hausdorff topologies.
In particular, if $X$ is Hausdorff, then the
narrow topology on the space $\MR$ of totally finite Radon
measures on $X$ is Hausdorff\cmmnt{ (437R(a-ii))}.

\spheader 437Jf\dvAnew{2007}
It will be useful to know that if $u:X\to\Bbb R$ is bounded
and lower semi-continuous, then
$\nu\mapsto\dashint u\,d\nu:\tilde M^+\to\Bbb R$ is lower semi-continuous
for the narrow topology.
\prooflet{\Prf\ (i) Perhaps I should start by explaining why
$\dashint u\,d\nu$ is always defined;  this is because the algebra $\Tau$
generated by the open sets is always a subalgebra of $\dom\nu$, and
$\{x:u(x)>\alpha\}\in\Tau$ for every $\alpha$, so
$u\in L^{\infty}(\Tau)$ (363Ha).
(ii) Now suppose for a moment that $u\ge 0$.   If
$\nu_0\in\tilde M^+$ and $\gamma<\dashint u\,d\nu_0$,
let $\epsilon>0$ be such
that $\gamma+\epsilon(1+\nu_0X)<\dashint u\,d\nu_0$,
let $n\ge 1$ be such that
$\|u\|_{\infty}\le n\epsilon$, and for $i\le n$ set
$G_i=\{x:u(x)>i\epsilon\}$.   Then

\Centerline{$\epsilon\sum_{i=1}^n\chi G_i
\le u\le\epsilon(\chi X+\sum_{i=1}^n\chi G_i)$,}

\Centerline{$\dashint u\,d\nu_0
\le\epsilon(\nu_0X+\sum_{i=1}^n\nu_0G_i)$,}

\Centerline{$V=\{\nu:\nu\in\tilde M^+$,
$\sum_{i=0}^n\nu G_i>\sum_{i=0}^n\nu_0G_i-1\}$}

\noindent is a neighbourhood of $\nu_0$ in $\tilde M^+$, and

\Centerline{$\dashint u\,d\nu
\ge\epsilon\sum_{i=1}^n\nu G_i
>\gamma$}

\noindent for every $\nu\in V$.   As $\nu_0$ and $\gamma$ are
arbitrary, $\nu\mapsto\dashint u\,d\nu$ is lower semi-continuous.    (iii)
In general, $u$ is expressible as the sum of a constant function and a
non-negative lower semi-continuous function;  as $\nu\mapsto\nu X$ is
continuous, $\nu\mapsto\dashint u\,d\nu$ is the sum of two lower
semi-continuous functions and is lower semi-continuous.\ \Qed}

Of course it follows at once that if $u:X\to\Bbb R$ is bounded and
continuous, then $\nu\mapsto\dashint u\,d\nu$ is continuous for the narrow
topology;  that is, the vague topology is coarser than the narrow topology
in contexts in which both make sense.

\spheader 437Jg\cmmnt{ With the more liberal definitions I use when 
considering
integrals with respect to $\sigma$-additive measures, we have another
version of the same idea.}   If $u:X\to[0,\infty]$ is a
lower semi-continuous function, then
$\nu\mapsto\int u\,d\nu:\tilde M^+_{\sigma}\to[0,\infty]$ is lower
semi-continuous for the narrow topology.
\prooflet{\Prf\ It is the supremum of the lower
semi-continuous functions $\nu\mapsto\int(u\wedge n\chi X)d\nu$.\ \Qed}

\spheader 437Jh\dvAnew{2010}
Let $X$ and $Y$ be topological spaces, $\phi:X\to Y$ a
continuous function, and $\tilde M^+(X)$, $\tilde M^+(Y)$ the spaces of
functionals described in (d).   For a functional $\nu$ defined on a subset
of $\Cal PX$, define
$\nu\phi^{-1}$ by saying that $(\nu\phi^{-1})(F)=\nu(\phi^{-1}[F])$
whenever $F\subseteq Y$ and $\phi^{-1}[F]\in\dom\nu$.   Then
$\nu\phi^{-1}\in\tilde M^+(Y)$ whenever $\nu\in\tilde M^+(X)$, and
the map $\nu\mapsto\nu\phi^{-1}:\tilde M^+(X)\to\tilde M^+(Y)$ is
continuous for the narrow topologies\prooflet{ (use 4A2B(a-ii))}.

\spheader 437Ji I am trying to
maintain the formal distinctions between `quasi-Radon measure' and
`$\tau$-additive effectively locally finite Borel measure inner regular
with respect to the closed sets', and between `Radon measure' and
`tight locally finite Borel measure'.   \cmmnt{There are obvious problems 
in interpreting the sum and
difference of measures with different domains, which are readily soluble
(see 234G and 416De) but in the context of this section
are unilluminating.}   If\cmmnt{, however,} we 
take $\MqR$ to be the set of totally
finite quasi-Radon measures on $X$, and $X$ is completely regular,
we have a canonical embedding of $\MqR$ into a cone in the
$L$-space $C_b(X)^*$;
\cmmnt{more generally,} even if\cmmnt{ our space} $X$ is 
not completely regular, the map
$\mu\mapsto\mu\restr\Cal B(X):\MqR\to M_{\sigma}(\Cal B(X))$
is still injective,
and we can identify $\MqR$ with a cone in the $L$-space
$M_{\tau}$ of
signed $\tau$-additive Borel measures\cmmnt{ (often the whole positive 
cone of $M_{\tau}$, as in 415M)}.
Similarly, when $X$ is Hausdorff, we can identify totally finite Radon
measures with tight totally finite Borel measures\cmmnt{ (416F)}.   The 
definition in 437Jd makes it plain that these
identifications are homeomorphisms for the narrow topology,

\cmmnt{It is even possible to extend these ideas to measures which are not
totally finite (437Yi), though there may be new difficulties (415Ya).}

\cmmnt{\spheader 437Jj For a different kind of narrow topology, adapted
to the space of all Radon measures on a Hausdorff space, see 495O below.
}%end of comment

\leader{437K}{Proposition} Let $X$ be a topological space, and
$\tilde M^+$ the set of all non-negative real-valued additive
functionals defined on algebras of subsets of $X$ containing every open
set.

(a) We have a function $T:\tilde M^+\to C_b(X)^*$ defined by the formula
$(T\nu)(u)=\dashint u\,d\nu$ whenever $\nu\in\tilde M^+$ and
$u\in C_b(X)$.

(b) $T$ is continuous for the narrow topology
$\frak S$ on $\tilde M^+$ and the weak* topology on $C_b(X)^*$.

(c) Suppose now that $X$ is completely regular, and that
$W\subseteq\tilde M^+$ is a family of $\tau$-additive totally finite
topological measures such that
two members of $W$ which agree on the Borel $\sigma$-algebra are equal.
Then $T\restr W$ is a homeomorphism between $W$, with the narrow
topology, and $T[W]$, with the weak* topology.

\proof{{\bf (a)} We have only to assemble the operators of 437Jc, noting
that if an algebra of subsets of $X$ contains every open set then it
certainly separates the
zero sets (indeed, it actually contains every zero set).

\medskip

{\bf (b)} As already noted in 437Jf,
$\nu\mapsto (T\nu)(u)=\dashint u\,d\nu$ is
$\frak S$-continuous for every $u\in C_b(X)$.
Since the weak* topology on $C_b(X)^*$ is the coarsest
topology on $C_b(X)^*$
for which all the functionals $f\mapsto f(u)$ are continuous,
$T$ is continuous.

\medskip

{\bf (c)(i)}
Write $\frak T$ for the topology on $W$ induced by $T$, that is, the
family of sets of the form $W\cap T^{-1}[V]$ where $V\subseteq C_b(X)^*$
is weak*-open.   If $G\subseteq X$ is open, then
$A=\{u:u\in C_b(X),\,0\le u\le\chi G\}$ is upwards-directed and has
supremum $\chi G$, so $\mu G=\sup_{u\in A}\int u\,d\mu$ for every
$\mu\in W$ (414Ba).   Accordingly $\{\mu:\mu\in W$,
$\mu G>\alpha\}=\bigcup_{u\in A}\{\mu:(T\mu)(u)>\alpha\}$ belongs
to $\frak T$ for every $\alpha\in\Bbb R$.   Also, of course,
$\{\mu:\mu X<\alpha\}=\{\mu:(T\mu)(\chi X)<\alpha\}\in\frak T$ for
every $\alpha$.   So if $\frak S'$ is the narrow topology on $W$,
$\frak S'\subseteq\frak T$.
Putting this together with (b), we see that $\frak S'=\frak T$.

\medskip

\quad{\bf (ii)} Now the same formulae show that $T\restr W$ is
injective.   \Prf\ Suppose that $\mu_1$, $\mu_2\in W$ and that
$T\mu_1=T\mu_2$.   Then $\mu_1G=\mu_2G$ for every open set
$G\subseteq X$.   By the Monotone Class Theorem, $\mu_1$ and
$\mu_2$ agree on all Borel sets;
but our hypothesis is that this is enough to ensure that $\mu_1=\mu_2$.\
\Qed

Since $T:W\to T[W]$ is continuous and open, it is a homeomorphism.
}%end of proof of 437K

\leader{437L}{Corollary} Let $X$ be a completely regular topological
space, and $M_{\tau}$ the space of signed $\tau$-additive Borel
measures on $X$.   Then the narrow and vague topologies on $M_{\tau}^+$
coincide.

\leader{437M}{Theorem}\discrversionA{\footnote{Revised
2009.}}{}\cmmnt{ ({\smc Ressel 77})} For a topological space
$X$, write $\MqR(X)$ for the space of totally finite quasi-Radon
measures on $X$, $\PqR(X)$ for the space of quasi-Radon probability
measures on $X$, and $M_{\tau}(X)$ for the
$L$-space of signed $\tau$-additive Borel measures on $X$.

(a) Let $X$ and $Y$ be topological spaces.   If
$\mu\in\MqR(X)$ and $\nu\in\MqR(Y)$, write
$\mu\times\nu$ for their $\tau$-additive
product measure on $X\times Y$\cmmnt{ (417G)}.
Then $(\mu,\nu)\mapsto\mu\times\nu$ is continuous for the narrow
topologies on $\MqR(X)$, $\MqR(Y)$ and $\MqR(X\times Y)$.

(b) Let $\familyiI{X_i}$ be a family of topological
spaces, with product $X$.   If $\familyiI{\mu_i}$ is a family of
probability measures such that $\mu_i\in\PqR(X_i)$ for each
$i$, write $\prod_{i\in I}\mu_i$ for its
$\tau$-additive product on $X$.   Then
$\familyiI{\mu_i}\mapsto\prod_{i\in I}\mu_i$ is continuous for the
narrow topology on $\PqR(X)$ and the product of the narrow
topologies on $\prod_{i\in I}\PqR(X_i)$.

(c) Let $X$ and $Y$ be topological spaces.

\quad(i) We have a unique bilinear operator
$\psi:M_{\tau}(X)\times M_{\tau}(Y)\to M_{\tau}(X\times Y)$
such that $\psi(\mu,\nu)$ is the restriction of the $\tau$-additive product
of $\mu$ and $\nu$ to the
Borel $\sigma$-algebra of $X\times Y$ whenever $\mu$, $\nu$ are totally
finite Borel measures on $X$, $Y$ respectively.

\quad(ii) $\|\psi\|\le 1$\cmmnt{ (definition:  253Ab)}.

\quad(iii) $\psi$ is separately continuous for the vague topologies on
$M_{\tau}(X)$, $M_{\tau}(Y)$ and $M_{\tau}(X\times Y)$.

(d) In (c), suppose that $X$ and $Y$ are compact and Hausdorff.
If $B\subseteq M_{\tau}(X)$ and $B'\subseteq M_{\tau}(Y)$ are
norm-bounded, then $\psi\restr B\times B'$ is continuous for the vague
topologies.

\proof{{\bf (a)(i)} I ought to note that we need 417N to assure us that
$\mu\times\nu\in\MqR(X\times Y)$ whenever
$\mu\in\MqR(X)$ and $\nu\in\MqR(Y)$.

\medskip

\quad{\bf (ii)} If $W\subseteq X\times Y$ is open and
$\alpha\in\Bbb R$, then $Q=\{(\mu,\nu):(\mu\times\nu)(W)>\alpha\}$ is
open in $\MqR(X)\times\MqR(Y)$.   \Prf\
Suppose that $(\mu_0,\nu_0)\in Q$.   Because $\mu_0\times\nu_0$ is
$\tau$-additive, there is a subset $W'\subseteq W$, expressible in the
form $\bigcup_{i\le n}G_i\times H_i$ where $G_i\subseteq X$ and
$H_i\subseteq Y$ are open for every $i$, such that

\Centerline{$\alpha<(\mu_0\times\nu_0)(W')
=\int\nu_0 W'[\{x\}]\mu_0(dx)$}

\noindent (417C(iv)).   Set $u(x)=\nu_0W'[\{x\}]$ for $x\in X$, so that
$u$ is lower semi-continuous (417Ba).   Let $\eta>0$ be such that
$\int u\,d\mu_0>\alpha+(1+2\mu_0X)\eta$, and set
$E_i=\{x:u(x)>\eta i\}$ for $i\in\Bbb N$, so that
$\eta\sum_{i=1}^{\infty}\mu_0E_i>\int u\,d\mu_0-\eta\mu_0X$.   Because
every $E_i$ is open, there is a neighbourhood $U$ of $\mu_0$ in
$\MqR(X)$ such that

\Centerline{$\int u\,d\mu_0-\eta\mu_0X\le\eta\sum_{i=1}^{\infty}\mu E_i
\le\int u\,d\mu=(\mu\times\nu_0)(W')$}

\noindent for every $\mu\in U$;  shrinking $U$ if necessary, we can
arrange at the same time that $\mu X<\mu_0X+1$ for every $\mu\in U$.
Next, observe that $\Cal H=\{W'[\{x\}]:x\in X\}
\subseteq\{\bigcup_{i\in I}H_i:I\subseteq\{0,\ldots,n\}\}$ is finite, so
there is a neighbourhood $V$ of $\nu_0$ in $\MqR(Y)$ such
that $\nu H\ge\nu_0 H-\eta$ for every $H\in\Cal H$ and $\nu\in V$.   If
$\mu\in U$ and $\nu\in V$, we have

$$\eqalign{(\mu\times\nu)(W)
&\ge(\mu\times\nu)(W')
=\int\nu W'[\{x\}]\mu(dx)
\ge\int u(x)-\eta\,\mu(dx)\cr
&=\int u\,d\mu-\eta\mu X
\ge\int u\,d\mu_0-\eta\mu_0X-\eta(1+\mu_0X)
>\alpha.\cr}$$

\noindent As $\mu_0$ and $\nu_0$ are arbitrary, $Q$ is open.\ \Qed

\medskip

\quad{\bf (iii)} Since $(\mu\times\nu)(X\times Y)=\mu X\cdot\nu Y$, the
sets $\{(\mu,\nu):(\mu\times\nu)(X\times Y)<\alpha\}$ are also open for
every $\alpha\in\Bbb R$.   So $(\mu,\nu)\mapsto\mu\times\nu$ is
continuous (4A2B(a-ii) again).

\medskip

{\bf (b)} Similarly, we can refer to 417O to check that
$\prod_{i\in I}\mu_i\in\PqR(X)$ whenever $\mu_i\in\PqR(X_i)$ for each
$i$.   For finite sets $I$, the result is a simple induction on $\#(I)$,
using 417Db and part (a) just above.
For infinite $I$, let $W\subseteq X$ be an open set and
$\alpha\in\Bbb R$, and consider

\Centerline{$Q
=\{\familyiI{\mu_i}:\mu_i\in\PqR(X_i)$ for each $i$,
$(\prod_{i\in I}\mu_i)(W)>\alpha\}$.}

\noindent If $\familyiI{\mu_i}\in Q$, then
there is an open set $W'\subseteq W$, determined by coordinates in a
finite set $J\subseteq I$, such that $(\prod_{i\in I}\mu_i)(W')>\alpha$.
Setting $V=\{x\restr J:x\in W'\}$, we have
$(\prod_{i\in J}\mu_i)(V)>\alpha$.   Now we can find open sets $U_i$
in $\PqR(X_i)$, for $i\in J$, such that
$(\prod_{i\in J}\nu_i)(V)>\alpha$ whenever $\nu_i\in U_i$ for $i\in J$.
If now
$\familyiI{\nu_i}\in\prod_{i\in I}\PqR(X_i)$ is such that
$\nu_i\in U_i$ for every $i\in J$,

\Centerline{$(\prod_{i\in I}\nu_i)(W)\ge(\prod_{i\in I}\nu_i)(W')
=(\prod_{i\in J}\nu_i)(V)>\alpha$,}

\noindent so $\prod_{i\in I}\nu_i\in Q$.   As $\familyiI{\mu_i}$ is
arbitrary, $Q$ is open.

As $W$ and $\alpha$ are arbitrary,
$\familyiI{\mu_i}\mapsto\prod_{i\in I}\mu_i$ is continuous.

\medskip

{\bf (c)(i)} Start by writing
$\psi(\mu,\nu)=(\mu\times\nu)\restr\Cal B(X\times Y)$ for
$\mu\in M^+_{\tau}(X)$ and $\nu\in M^+_{\tau}(Y)$,
where $\Cal B(X\times Y)$ is the Borel $\sigma$-algebra of $X\times Y$.
If $\mu$, $\mu_1$, $\mu_2\in M^+_{\tau}(X)$ and
$\nu$, $\nu_1$, $\nu_2\in M^+_{\tau}(Y)$ and $\alpha\ge 0$, then

\Centerline{$\psi(\mu_1+\mu_2,\nu)
=\psi(\mu_1,\nu)+\psi(\mu_2,\nu)$.}

\noindent\Prf\ On each side of the equation we have a $\tau$-additive
Borel measure, and the two measures agree on the standard base $\Cal W$
for the topology of $X\times Y$ consisting of products of open sets;
since $\Cal W$ is closed under finite intersections, they agree on the
algebra generated by $\Cal W$ and therefore on all open sets and
therefore (using the Monotone Class Theorem yet again) on all Borel
sets.\ \QeD\   Similarly,

\Centerline{$\psi(\mu,\nu_1+\nu_2)
=\psi(\mu,\nu_1)+\psi(\mu,\nu_2)$,
\quad$\psi(\alpha\mu,\nu)=\psi(\mu,\alpha\nu)
=\alpha\psi(\mu,\nu)$}

\noindent whenever $\mu\in M^+_{\tau}(X)$, $\nu$, $\nu_1$,
$\nu_2\in M^+_{\tau}(Y)$ and $\alpha\in\Bbb R$.
Now if $\mu'_1$, $\mu'_2\in M^+_{\tau}(X)$ and $\nu'_1$,
$\nu'_2\in M^+_{\tau}(Y)$ are such that $\mu_1-\mu_2=\mu'_1-\mu'_2$ and
$\nu_1-\nu_2=\nu'_1-\nu'_2$, we shall have

$$\eqalign{\psi(\mu_1,\nu_1)-\psi(\mu_1,&\nu_2)
  -\psi(\mu_2,\nu_1)+\psi(\mu_2,\nu_2)\cr
&=\psi(\mu_1,\nu_1+\nu'_2)-\psi(\mu_1+\mu'_2,\nu'_2)
        +\psi(\mu'_2,\nu'_2)\cr
&\mskip75mu
   -\psi(\mu_1,\nu'_1+\nu_2)+\psi(\mu_1+\mu'_2,\nu'_1)
   -\psi(\mu'_2,\nu'_1)\cr
&\mskip75mu
   -\psi(\mu_2,\nu_1+\nu'_2)+\psi(\mu_2+\mu'_1,\nu'_2)
     -\psi(\mu'_1,\nu'_2)\cr
&\mskip75mu
   +\psi(\mu_2,\nu'_1+\nu_2)-\psi(\mu_2+\mu'_1,\nu'_1)
     +\psi(\mu'_1,\nu'_1)\cr
&=\psi(\mu'_2,\nu'_2)
 -\psi(\mu'_2,\nu'_1)
 -\psi(\mu'_1,\nu'_2)
 +\psi(\mu'_1,\nu'_1).\cr}$$

\noindent We can therefore extend $\psi$ to an operator on
$M_{\tau}(X)\times M_{\tau}(Y)$ by setting

\Centerline{$\psi(\mu_1-\mu_2,\nu_1-\nu_2)
  =\psi(\mu_1,\nu_1)
  -\psi(\mu_1,\nu_2)
  -\psi(\mu_2,\nu_1)
  +\psi(\mu_2,\nu_2)$}

\noindent whenever $\mu_1$, $\mu_2\in M^+_{\tau}(X)$ and $\nu_1$,
$\nu_2\in M^+_{\tau}(Y)$, and it is straightforward to check that $\psi$
is bilinear.

\medskip

\quad{\bf (ii)} If $\mu\in M_{\tau}(X)$, then
$\|\mu\|=\mu^+(X)+\mu^-(X)$, where $\mu^+$ and $\mu^-$ are evaluated in
the Riesz space $M_{\tau}(X)$.   Now if $\nu\in M_{\tau}(Y)$,

$$\eqalign{|\psi(\mu,\nu)|
&=|\psi(\mu^+,\nu^+)-\psi(\mu^+,\nu^-)
  -\psi(\mu^-,\nu^+)+\psi(\mu^-,\nu^-)|\cr
&\le\psi(\mu^+,\nu^+)+\psi(\mu^+,\nu^-)
  +\psi(\mu^-,\nu^+)+\psi(\mu^-,\nu^-),\cr}$$

\noindent so

$$\eqalign{\|\psi(\mu,\nu)\|
&=|\psi(\mu,\nu)|(X\times Y)\cr
&\le\psi(\mu^+,\nu^+)(X\times Y)+\psi(\mu^+,\nu^-)(X\times Y)\cr
&\mskip100mu
  +\psi(\mu^-,\nu^+)(X\times Y)+\psi(\mu^-,\nu^-)(X\times Y)\cr
&=\mu^+(X)\cdot\nu^+(Y)+\mu^+(X)\cdot\nu^-(Y)
  +\mu^-(X)\cdot\nu^+(Y)+\mu^-(X)\cdot\nu^-(Y)\cr
&=\|\mu\|\|\nu\|.\cr}$$

\noindent As $\mu$ and $\nu$ are arbitrary, $\|\psi\|\le 1$.

\medskip

\quad{\bf (iii)} Fix $\nu\in M^+_{\tau}(Y)$ and $w\in C_b(X\times Y)^+$,
and consider the map
$\mu\mapsto\dashint w\,d\psi(\mu,\nu):M_{\tau}(X)\to\Bbb R$.   Note first
that if $\mu\in M^+_{\tau}(X)$,

\Centerline{$\dashint w\,d\psi(\mu,\nu)=\int w\,d(\mu\times\nu)
=\iint w(x,y)\nu(dy)\mu(dx)=\dashiint w(x,y)\nu(dy)\mu(dx)$}

\noindent (417H).   Since both sides of this equation are linear in
$\mu$, we have

\Centerline{$\dashint w\,d\psi(\mu,\nu)=\dashiint w(x,y)\nu(dy)\mu(dx)$}

\noindent for every $\mu\in M_{\tau}(X)$.   Now
$x\mapsto\int w(x,y)\nu(dy)$ is
continuous.   \Prf\ By 417Bc, it is lower semi-continuous;  but if
$\alpha\ge\|w\|_{\infty}$ and $w'=\alpha\chi(X\times Y)-w$, then
$x\mapsto\int w'(x,y)\nu(dy)$ is lower semi-continuous, so

\Centerline{$x\mapsto\alpha\nu Y-\biggerint w'(x,y)\nu(dy)
=\int w(x,y)\nu(dy)$}

\noindent is also upper semi-continuous, therefore continuous.\ \QeD\
It follows at once that $\mu\mapsto\dashiint w(x,y)\nu(dy)\mu(dx)$ is
continuous for the vague topology on $M_{\tau}(X)$.   The argument has
supposed that $w$ and $\nu$ are positive;  but taking positive and
negative parts as usual, we see that $\mu\mapsto\dashint w\,d\psi(\mu,\nu)$
is vaguely continuous for every $w\in C_b(X\times Y)$ and
$\nu\in M_{\tau}(Y)$.   As $w$ is arbitrary, $\mu\mapsto\psi(\mu,\nu)$
is vaguely continuous, for every $\nu$.   Similarly,
$\nu\mapsto\psi(\mu,\nu)$ is vaguely continuous for every $\mu$, and
$\psi$ is separately continuous.

\medskip

{\bf (d)} Now suppose that $X$ and $Y$ are compact.   Let $W$ be the
linear subspace of $C(X\times Y)$ generated by
$\{u\otimes v:u\in C(X)$, $v\in C(Y)\}$, writing
$(u\otimes v)(x,y)=u(x)v(y)$ as in 253B.   Then $W$ is a subalgebra of
$C(X\times Y)$ separating the points of
$X\times Y$ and containing the constant functions, so is
$\|\,\|_{\infty}$-dense in $C(X\times Y)$ (281E).   Now

\Centerline{$(\mu,\nu)\mapsto\dashint u\otimes v\,d\psi(\mu,\nu)
=\dashint u\,d\mu\cdot\dashint v\,d\nu$}

\noindent is continuous whenever $u\in C(X)$ and $v\in C(Y)$, so

\Centerline{$(\mu,\nu)\mapsto\dashint w\,d\psi(\mu,\nu)$}

\noindent is continuous whenever $w\in W$.

Next suppose that $B\subseteq M_{\tau}(X)$ and $B'\subseteq M_{\tau}(Y)$
are bounded.   Let $\gamma\ge 0$ be such that $\|\mu\|\le\gamma$ for
every $\mu\in B$ and $\|\nu\|\le\gamma$ for every $\nu\in B$.   If
$w\in C(X\times Y)$ and $\epsilon>0$, there is a $w'\in W$ such that
$\|w-w'\|_{\infty}\le\epsilon$.   In this case

$$\eqalign{|\dashint w\,d\psi(\mu,\nu)-\dashint w'\,d\psi(\mu,\nu)|
&\le\|w-w'\|_{\infty}\|\psi(\mu,\nu)\|\cr
&\le\epsilon\|\mu\|\|\nu\|
\le\gamma^2\epsilon\cr}$$

\noindent whenever $\mu\in B$ and $\nu\in B'$.   As $\epsilon$ is
arbitrary, the function $(\mu,\nu)\mapsto\dashint w\,d\psi(\mu,\nu)$ is
uniformly approximated on $B\times B'$ by vaguely continuous functions,
and is therefore itself vaguely continuous on $B\times B'$.
}%end of proof of 437M

\leader{437N}{}\cmmnt{ One of the standard constructions of Radon measures
is as image measures.   It leads naturally to maps between spaces of
Radon measures, and of course we wish to know whether they are continuous.

\medskip

\noindent}{\bf Proposition}\dvAformerly{4{}37Q} (a) Let $X$ and $Y$ be
Hausdorff spaces, and $\phi:X\to Y$ a continuous function.   Let $\MR(X)$,
$\MR(Y)$ be the spaces
of totally finite Radon measures on $X$ and $Y$ respectively.   Write
$\tilde\phi(\mu)$ for the image measure $\mu\phi^{-1}$ for
$\mu\in\MR(X)$.

\quad(i) $\tilde\phi:\MR(X)\to\MR(Y)$ is continuous for
the narrow topologies on $\MR(X)$ and $\MR(Y)$.

\quad(ii) $\tilde\phi(\mu+\nu)=\tilde\phi(\mu)+\tilde\phi(\nu)$ and
$\tilde\phi(\alpha\mu)=\alpha\tilde\phi(\mu)$ for all $\mu$, $\nu\in\MR(X)$
and $\alpha\ge 0$.

(b) If $Y$ is a Hausdorff space, $X$ a subset of $Y$, and $\phi:X\to Y$
the identity map, then $\tilde\phi$ is a homeomorphism between
$\MR(X)$ and
$\{\nu:\nu\in\MR(Y)$, $\nu(Y\setminus X)=0\}$.

\proof{{\bf (a)(i)} All we have to do is to recall from 418I that
$\mu\phi^{-1}\in\MR(Y)$ for every
$\mu\in\MR(X)$, and observe that

\Centerline{$\{\mu:(\mu\phi^{-1})(H)>\alpha\}
=\{\mu:\mu\phi^{-1}[H]>\alpha\}$,
\quad$\{\mu:(\mu\phi^{-1})(Y)<\alpha\}=\{\mu:\mu X<\alpha\}$}

\noindent are narrowly open in $\MR(X)$ for every open set
$H\subseteq Y$ and $\alpha\in\Bbb R$.

\medskip

\quad{\bf (ii)} As usual, since all the measures here are Radon measures,
it is enough to check that
$\tilde\phi(\mu+\nu)(E)=\tilde\phi(\mu)(E)+\tilde\phi(\nu)(E)$ and
$\tilde\phi(\alpha\mu)(E)=\alpha\tilde\phi(\mu)(E)$ for every Borel set
$E\subseteq X$, and this is easy.

\medskip

{\bf (b)} First note that if $\mu\in\MR(X)$, then certainly
$\tilde\phi(\mu)(Y\setminus X)=0$;  while if $\nu\in\MR(Y)$
and $\nu(Y\setminus X)=0$,
then $\mu=\nu\restr\Cal PX$ is a Radon measure on $X$ (416Rb) and
$\nu=\tilde\phi(\mu)$.   Thus $\tilde\phi$ is a continuous bijection
from $\MR(X)$ to
$\{\nu:\nu\in\MR(Y)$, $\nu(Y\setminus X)=0\}$.   Now if
$G\subseteq X$ is relatively open and $\alpha\in\Bbb R$, there is an
open set $H\subseteq Y$ such that $G=H\cap X$, so that

\Centerline{$\{\mu:\mu\in\MR(X)$,
$\mu G>\alpha\}=\{\mu:\tilde\phi(\mu)(H)>\alpha\}$}

\noindent is the inverse image of a narrowly open set in
$\MR(Y)$;  and of course

\Centerline{$\{\mu:\mu\in\MR(X)$,
$\mu X<\alpha\}=\{\mu:\tilde\phi(\mu)(Y)<\alpha\}$}

\noindent is also the inverse image of an open set.   So $\tilde\phi$ is
a homeomorphism between $\MR(X)$ and
$\{\nu:\nu\in\MR(Y)$, $\nu(Y\setminus X)=0\}$.
}%end of proof of 437N

\leader{437O}{Uniform tightness}\dvAformerly{4{}37T}
Let $X$ be a topological space.   If $\nu$ is a bounded additive functional
on an algebra of subsets of $X$,\cmmnt{ I say that} it is {\bf tight} if

\Centerline{$\nu E
\in\overline{\{\nu K:K\subseteq E,\,K\in\dom\nu,\,
  K\text{ is closed and compact}\}}$}

\noindent for every $E\in\dom\nu$, and that a set $A$ of tight functionals
is {\bf uniformly tight} if every member of $A$ is tight
and for every $\epsilon>0$ there is a
closed compact set $K\subseteq X$ such that $\nu K$ is defined and
$|\nu E|\le\epsilon$ whenever $\nu\in A$ and $E\in\dom\nu$ is disjoint from
$K$.

\leader{437P}{Proposition}\discrversionA{\footnote{Formerly 4{}37U;
revised 2009.}}{} Let $X$ be a topological space.

(a) Let $\MqR$ be the
set of totally finite quasi-Radon measures on $X$.   Suppose that
$A\subseteq\MqR$ is uniformly totally finite\cmmnt{ (that is,
$\{\mu X:\mu\in A\}$ has a finite upper bound)} and for every
$\epsilon>0$ there is a
closed compact $K\subseteq X$ such that $\mu(X\setminus K)\le\epsilon$ for
every $\mu\in A$.   Then $A$ is relatively compact in $\MqR$ for the narrow
topology.

(b) Suppose now that $X$ is Hausdorff, and that $\MR$ is the set of Radon
measures on $X$.   If $A\subseteq\MR$ is uniformly totally finite and
uniformly tight, then
it is relatively compact in $\MR$ for the narrow topology.

\proof{{\bf (a)(i)} I show first that the closure $\overline{A}$ of $A$ in
$\MqR$ has the same two properties.   \Prf\ Because $\mu\mapsto\mu X$ is
continuous for the narrow topology,
$\{\mu X:\mu\in\overline{A}\}\subseteq\overline{\{\mu X:\mu\in A\}}$ is
bounded.   If $\epsilon>0$, there is a closed compact set $K\subseteq X$
such that $\mu(X\setminus K)\le\epsilon$ for every $\mu\in A$.
In this case $\{\mu:\mu\in\MqR$, $\mu(X\setminus K)\le\epsilon\}
=\MqR\setminus\{\mu:\mu(X\setminus K)>\epsilon\}$
is closed in $\MqR$, so includes $\overline{A}$.   As $\epsilon$ is
arbitrary, we have the result.\ \Qed

\medskip

\quad{\bf (ii)} Now let $\Cal F$ be an ultrafilter on $\MqR$ containing
$\overline{A}$.

\medskip

\qquad\grheada\ For Borel sets
$E\subseteq X$, set $\theta E=\lim_{\nu\to\Cal F}\nu E$;  this is defined
in $\Bbb R$ because $\sup_{\nu\in\overline{A}}\nu X$ is finite.
$\theta$ is
a non-negative additive functional on the Borel $\sigma$-algebra of $X$.
The family $\Cal K$ of closed compact subsets of $X$ is a compact
class closed under finite unions and countable intersections,
so 413S tells us that there is a complete measure $\mu$ on $X$
such that $\mu X\le\theta X$, $\Cal K\subseteq\dom\mu$, $\mu K\ge\theta K$
for every $K\in\Cal K$, and $\mu$ is inner regular with respect to
$\Cal K$.

If $F\subseteq X$ is closed, then $F\cap K\in\Cal K$ for every
$K\in\Cal K$, so $\mu$ measures $F$ (412Ja).   Thus $\mu$ is a topological
measure.   Because $\mu$ is tight, it is $\tau$-additive (411E).
So $\mu$ is a complete totally finite $\tau$-additive measure which is
inner regular with respect to the closed sets,
and is a quasi-Radon measure.

\medskip

\qquad\grheadb\ Given $\epsilon>0$, there is a $K\in\Cal K$ such that
$\nu(X\setminus K)\le\epsilon$ for every $\nu\in\overline{A}$, so
$\theta(X\setminus K)\le\epsilon$ and

\Centerline{$\theta X-\epsilon\le\theta K\le\mu K\le\mu X\le\theta X$.}

\noindent As $\epsilon$ is arbitrary,
$\mu X=\theta X=\lim_{\nu\to\Cal F}\nu X$.

\medskip

\qquad\grheadc\
If $G\subseteq X$ is open and $\epsilon>0$, there is a $K\in\Cal K$ such
that $\nu(X\setminus K)\le\epsilon$ for every $\nu\in\overline{A}$, so
$\theta(X\setminus K)\le\epsilon$ and

$$\eqalignno{\mu G
&=\mu X-\mu(X\setminus G)
\le\theta X-\mu(K\setminus G)
\le\theta X-\theta(K\setminus G)\cr
\displaycause{because $K\setminus G\in\Cal K$}
&\le\theta X-\theta(X\setminus G)+\theta(X\setminus K)
\le\theta G+\epsilon.\cr}$$

\noindent As $\epsilon$ is arbitrary,

\Centerline{$\mu G\le\theta G=\lim_{\nu\to\Cal F}\nu G$.}

\medskip

\qquad\grheadd\ Putting ($\beta$) and ($\gamma$) together, we see that
$\Cal F\to\mu$ for the narrow topology on
$\MqR$;  it follows that $\mu\in\overline{A}$.   Thus every ultrafilter on
$\MqR$ containing $\overline{A}$ has a limit in $\overline{A}$, and
$\overline{A}$ is compact.   Accordingly $A$ is relatively compact in
$\MqR$, as claimed.

\medskip

{\bf (b)} We know from the proof of (a) that the closure $\overline{A}$ of
$A$ in $\MqR$ is compact, so it will be enough to show that
$\overline{A}\subseteq\MR$.   If $\mu\in\overline{A}$ and $E\in\dom\mu$,
then for every
$\epsilon>0$ there is a compact set $K\subseteq X$ such that
$\mu(X\setminus K)\le\epsilon$, by (a-i) above;  also there is a closed set
$F\subseteq E$ such that $\mu(E\setminus F)\le\epsilon$.   But now
$F\cap K$ is compact and $\mu(E\setminus(F\cap K))\le 2\epsilon$.   As $E$
and $\epsilon$ are arbitrary, $\mu$ is inner regular with respect to the
compact sets, so is a Radon measure.
}%end of proof of 437P

\leader{437Q}{Two metrics}\dvAnew{2012}
\cmmnt{So far, as elsewhere in this volume, I have been writing about
topologies with as few restrictions on their nature as possible.   Of
course the repeated invocation of $L$-spaces in the first part of the
section indicates that there are norms and their associated metrics about,
and when the underlying set $X$ is metrizable we rather hope that the
constructions of 437J will lead to metrizable topologies on the spaces of
measures considered there.   I offer two definitions which
seem to give us interesting paths to explore.

\medskip

}{\bf (a)(i)} If $X$ is a set and $\mu$, $\nu$
are bounded additive functionals defined on algebras of subsets of $X$,
then $\mu-\nu:\dom\mu\cap\dom\nu\to\Bbb R$ is bounded and
additive, and we can set

\Centerline{$\rhotv(\mu,\nu)=|\mu-\nu|(X)
=\sup_{E,F\in\dom\mu\cap\dom\nu}(\mu-\nu)(E)-(\mu-\nu)(F)$.}

\noindent In this generality, $\rhotv$
is not even a pseudometric, but if we
have a class $M$ of totally finite measures on $X$ all of which are inner
regular with respect to a subset $\Cal K$ of $\bigcap_{\mu\in M}\dom\mu$,
then we have

\Centerline{$\rhotv(\mu,\nu)
=\sup_{K,L\in\Cal K}(\mu K-\mu L)-(\nu K-\nu L)$}

\noindent for all $\mu$, $\nu\in M$, and $\rhotv\restr M\times M$ is a
pseudometric on $M$.   If moreover $M$ is such that distinct members of $M$
differ on $\Cal K$ (as when $\Cal K$ is the family of
closed sets in a topological space $X$ and $M=\MqR(X)$, or when
$\Cal K$ the family of compact sets in a Hausdorff space $X$ and
$M=\MR(X)$), then $\rhotv$
gives us a metric on $M$.   In such a case I will call
$\rhotv\restr M\times M$ the {\bf total variation metric} on $M$.
\cmmnt{(Compare the `total variation norms' of 362B.)}

\medskip

\quad{\bf (ii)} Note that if $\Sigma\subseteq\dom\mu\cap\dom\nu$
is a $\sigma$-algebra then

\Centerline{$|\int u\,d\mu-\int u\,d\nu|\le\|u\|_{\infty}\rhotv(\mu,\nu)$}

\noindent whenever $u\in\eusm L^{\infty}(\Sigma)$.   So if, for instance,
$X$ is a topological space and $M\subseteq M^+_{\text{qR}}(X)$, then
$u\mapsto\int u\,d\mu$ will be continuous for the total variation metric on
$M$ whenever $u:X\to\Bbb R$ is a bounded universally measurable function.

\medskip

\quad{\bf (iii)}\cmmnt{ It is of course worth knowing when to expect a
complete
metric.}   When our set $M$ can be identified with the positive cone
of a band in some $L$-space
$M_{\sigma}$ of countably additive functions\cmmnt{, as in 437F}, then
we\cmmnt{ naturally}
have a complete metric\cmmnt{, because bands in $L$-spaces are closed
subspaces
(354Bd)}.   In particular, for any Hausdorff space $X$,
$M^+_{\text{R}}(X)$ can be identified with the
positive cone of the $L$-space of tight Borel measures on $X$, so is
complete.    \cmmnt{See also 437Xo.}

\cmmnt{\medskip

\quad{\bf (iv)} There is an obvious variation on $\rhotv$ as defined here:
the function

\Centerline{$(\mu,\nu)\mapsto\sup_{E\in\dom\mu\cap\dom\nu}|\mu E-\nu E|$,}

\noindent which will be a metric on nearly all occasions when $\rhotv$ is a
metric, and will then be uniformly equivalent to $\rhotv$.   But the more
complex formulation gives a better match to the Riesz norm metric of the
leading examples.
}%end of comment

\spheader 437Qb Suppose that $(X,\rho)$ is a metric space.   Write $\MqR$
for the set of totally finite quasi-Radon measures on $X$.
For $\mu$, $\nu\in\MqR$ set

\Centerline{$\rhoKR(\mu,\nu)
=\sup\{|\int u\,d\mu-\int u\,d\nu|:
   u:X\to[-1,1]$ is $1$-Lipschitz$\}$.}

\noindent Then $\rhoKR$ is a metric on $\MqR$.  \prooflet{\Prf\ It is
immediate from the form of the definition that $\rhoKR$ is a pseudometric.
If $\mu$, $\nu\in\MqR$ are different, there is an open set
$G$ such that $\mu G\ne\nu G$ (415G(iii));  suppose that $\mu G<\nu G$.
Set $u(x)=\rho(x,X\setminus G)$ for $x\in X$.
There must be an $n\in\Bbb N$ such that $\mu G<\nu F_n$
where $F_n=\{x:u(x)\ge 2^{-n}\}$.   In this case, setting
$u'=u\wedge 2^{-n}\chi X$,

\Centerline{$\int u'd\mu\le 2^{-n}\mu G<2^{-n}\nu F_n\le\int u'd\nu$,}

\noindent so $\rhoKR(\mu,\nu)\ge 2^{-n}(\nu F_n-\mu G)>0$.
As $\mu$ and $\nu$ are arbitrary, $\rhoKR$ is a metric.\ \Qed}

\cmmnt{\medskip

\noindent{\bf Remark} $\rhoKR$ here is taken from {\smc Bogachev 07},
\S8.3, where it is called the `Kantorovich-Rubinshtein metric'.
For its principal properties, see 437R(g)-(h) below.   A variation of
this construction will be used in 457L;  see also 437Xs.}
%Hutchinson J.E.\ [81] `Fractals and self similarity',
%Indiana Univ.\ Math.\ J.\ 30 (1981) 713-747.
%thanks to J.Pachl for reference.

\leader{437R}{Theorem}\dvAformerly{4{}37Je, 4{}37O and 4{}37R;
revised and expanded 2009-2012} Let $X$ be a topological space;  write
$\MqR=\MqR(X)$ for the set of totally finite
quasi-Radon measures on $X$, and if $X$ is Hausdorff write $\MR=\MR(X)$
for the set of totally finite Radon measures on $X$, both endowed with
their narrow topologies.

(a)(i) If $X$ is regular then $\MqR$ is Hausdorff.

\quad(ii) If $X$ is Hausdorff then $\MR$ is Hausdorff.

(b)\dvAnew{2011} If $X$ has a countable network then $\MqR$ has a countable
network.

(c) Suppose that $X$ is separable.

\quad(i) If $X$ is a T$_1$ space, then $\MqR$ is separable.

\quad(ii) If $X$ is Hausdorff, $\MR$ is separable.

(d)\dvAnew{2011} If $X$ is a K-analytic Hausdorff space, so is $\MqR=\MR$.

(e)\dvAnew{2011} If $X$ is an analytic Hausdorff space, so is $\MqR=\MR$.

(f)(i) If $X$ is compact, then for any real $\gamma\ge 0$ the sets
$\{\mu:\mu\in\MqR$, $\mu X\le\gamma\}$ and $\{\mu:\mu\in\MqR$,
$\mu X=\gamma\}$ are compact.

\quad(ii) If $X$ is compact and Hausdorff, then for any real
$\gamma\ge 0$ the sets
$\{\mu:\mu\in\MR$, $\mu X\le\gamma\}$ and $\{\mu:\mu\in\MR$,
$\mu X=\gamma\}$ are compact.  In particular, the set $\PR$ of
Radon probability measures on $X$ is compact.

(g) Suppose that $X$ is metrizable and
$\rho$ is a metric on $X$ inducing its topology.

\quad(i) The metric $\rhoKR$ on $\MqR$\cmmnt{ (437Qb)}
induces the narrow topology on $\MqR$.

\quad(ii) If $(X,\rho)$ is complete then $\MqR=\MR$ is complete
under $\rhoKR$.

(h) If $X$ is Polish, so is $\MqR=\MR$.

\proof{{\bf (a)(i)} (Cf.\ 437Qb.)   Take distinct
$\mu_0$, $\mu_1\in\MqR$.   If $\mu_0X\ne\mu_1X$ then they
can be separated by open sets of the form $\{\mu:\mu X<\alpha\}$,
$\{\mu:\mu X>\alpha\}$.   Otherwise, set $\gamma=\mu_0X=\mu_1X$.   There
is certainly an open set $G$ such that $\mu_0G\ne\mu_1G$ (415H);
suppose that $\mu_0G<\mu_1G$.   Because $\mu_1$ is inner regular with
respect to the closed sets, there is a closed set $F\subseteq G$ such
that $\mu_0G<\mu_1F$.   Now consider

\Centerline{$\Cal H
=\{H:H$ is open, $\overline{H}\subseteq G\}$.}

\noindent Then $\Cal H$ is upwards-directed;  because $X$ is regular,
$\bigcup\Cal H=G$;  because $\mu_1$ is quasi-Radon,

\Centerline{$\sup_{H\in\Cal H}\mu_1H
\ge\sup_{H\in\Cal H}\mu_1(H\cap F)=\mu_1F>\mu_0G$}

\noindent and there is an $H\in\Cal H$ such that $\mu_1H>\mu_0G$.   Now

\Centerline{$\mu_1H+\mu_0(X\setminus\overline{H})
\ge\mu_1H+\gamma-\mu_0G>\gamma$.}

\noindent Let $\alpha$, $\beta$ be such that $\mu_1H>\alpha$,
$\mu_0(X\setminus\overline{H})>\beta$ and $\alpha+\beta>\gamma$.   Then

\Centerline{$\{\mu:\mu\in\MqR$, $\mu(X\setminus\overline{H})>\beta\}$,
\quad$\{\mu:\mu\in\MqR$, $\mu H>\alpha$, $\mu X<\alpha+\beta\}$}

\noindent are disjoint open sets containing $\mu_0$, $\mu_1$ respectively,
so again we have separation.

\medskip

\quad{\bf (ii)} We can use the same ideas.   Take distinct
$\mu_0$, $\mu_1\in M^+_{\text{R}}$.   If $\mu_0X\ne\mu_1X$ then $\mu_0$ and
$\mu_1$ can be separated by open sets of the form $\{\mu:\mu X<\alpha\}$,
$\{\mu:\mu X>\alpha\}$.   Otherwise, set $\gamma=\mu_0X=\mu_1X$, and take
an open set $G$ such that $\mu_0G\ne\mu_1G$;
suppose that $\mu_0G<\mu_1G$.   Then $\mu_0(X\setminus G)+\mu_1G>\gamma$.
Because $\mu_0$ and $\mu_1$ are inner regular with respect to the compact
sets, there are compact sets
$K_0\subseteq X\setminus G$, $K_1\subseteq G$ such
that $\mu_0K_0+\mu_1K_1>\gamma$.   Now there are disjoint open sets
$H_0$, $H_1$ such that $K_i\subseteq H_i$ for both $i$ (4A2F(h-i)), 
in which case $\mu_0H_0+\mu_1H_1>\gamma$.   Take $\alpha_0<\mu_0H_0$ and
$\alpha_1<\mu_1H_1$
such that $\alpha_0+\alpha_1>\gamma$.   In this case,
$\{\mu:\mu H_0>\alpha_0\}$ and
$\{\mu:\mu H_1>\alpha_1$, $\mu X<\alpha_0+\alpha_1\}$
are disjoint open sets containing $\mu_0$, $\mu_1$ respectively.

\medskip

{\bf (b)} Let $\Cal A$ be a countable network for the topology of
$X$;  replacing $\Cal A$ by
$\{\bigcup\Cal A_0:\Cal A_0\in[\Cal A]^{<\omega}\}$ if necessary,
we can suppose that $\Cal A$ is closed under finite unions.   Let $\Cal D$
be the family of sets of the form

\Centerline{$\{\mu:\mu\in\MqR$, $\mu X<\gamma$, $\mu^*A_i>\gamma_i$ for
$i\le n\}$}

\noindent where $n\in\Bbb N$, $A_0,\ldots,A_n\in\Cal A$ and $\gamma$,
$\gamma_0,\ldots,\gamma_n\in\Bbb Q$.   Then $\Cal D$ is countable.   If
$V\subseteq\MqR$ is an open set and $\mu_0\in V$, there must be
open sets $G_0,\ldots,G_n\subseteq X$ and $\gamma$,
$\gamma_0,\ldots,\gamma_n\in\Bbb Q$ such that

\Centerline{$\mu_0\in\{\mu:\mu X<\gamma$,
$\mu G_i>\gamma_i$ for every $i\le n\}
\subseteq V$.}

\noindent For each $i\le n$, $\{A:A\in\Cal A$, $A\subseteq G_i\}$ is a
countable upwards-directed set with union $G_i$, so there is a
non-decreasing sequence $\sequence{j}{A_{ij}}$ in $\Cal A$
with union $G_i$, and there
must be a $j_i\in\Bbb N$ such that $\mu_0^*A_{ij_i}>\gamma_i$ (132Ae).
Now

\Centerline{$\{\mu:\mu X<\gamma$,
$\mu^*A_{ij_i}>\gamma_i$ for every $i\le n\}$}

\noindent belongs to $\Cal D$, contains $\mu$ and is included in $V$.   As
$\mu$ and $V$ are arbitrary, $\Cal D$ is a countable network for the
topology of $\MqR$.

\medskip

{\bf (c)(i)} If $X$ is empty, then $\MqR$ is a
singleton, and
we can stop.   Otherwise, let $D$ be a countable dense subset of $X$.   Set
$D'=\{\sum_{i=0}^n\alpha_i\delta_{x_i}:x_0,\ldots,x_n\in D$,
$\alpha_0,\ldots,\alpha_n\in\Bbb Q\cap\coint{0,\infty}\}$, writing
$\delta_x$ for the Dirac measure concentrated at $x$ for each $x\in X$.
Because $X$ is T$_1$, $D'\subseteq\MqR$.   In fact $D'$ is dense in $\MqR$.
\Prf\ Take any $\mu\in\MqR$, a finite family $\Cal G$
of open subsets of $X$, and $\epsilon>0$.   Let $\Cal E$ be the
algebra of subsets of $X$ generated by $\Cal G$, and $\Cal A$ the set of
atoms of $\Cal E$.   For each $E\in\Cal A$ choose
$x_E\in D\cap\bigcap\{G:E\subseteq G\in\Cal G\}$ and
$\alpha_E\in\Bbb Q\cap\coint{0,\infty}$ such that
$|\alpha_E-\mu E|\le\Bover{\epsilon}{\#(\Cal A)}$.   Try
$\nu=\sum_{E\in\Cal A}\alpha_E\delta_{x_E}\in D'$.   If $G\in\Cal G$, then

$$\eqalign{\mu G
&=\sum_{E\in\Cal A,E\subseteq G}\mu E
\le\sum_{E\in\Cal A,x_E\in G}\mu E\cr
&\le\epsilon+\sum_{E\in\Cal A,x_E\in G}\alpha_E
=\epsilon+\nu G;\cr}$$

\noindent while

\Centerline{$\nu X
=\sum_{E\in\Cal A}\alpha_E
\le\epsilon+\sum_{E\in\Cal A}\mu E
=\epsilon+\mu X$.}

\noindent As $\mu$, $\Cal G$ and $\epsilon$ are arbitrary, $D'$ is dense in
$\MqR$.\ \QeD\   So $\MqR$ is separable.

\medskip

\quad{\bf (ii)} If $X$ is Hausdorff, use the same construction;  in this
case $D'\subseteq\MR$, so $\MR$ also is separable.

\medskip

{\bf (d)} Most of the argument will be devoted to proving
that the set $\PR$ of Radon probability measures on $X$ is K-analytic in
its narrow topology.

\medskip

\quad{\bf (i)} We are supposing that there is an usco-compact relation
$R\subseteq\NN\times X$ such that $R[\NN]=X$ (422F).   Set
$R_1=\{(\alpha,x):$ there is a $\beta\le\alpha$
such that $(\beta,\alpha)\in R\}$;  then $R_1$ is also usco-compact
(422Dh).

Set

$$\eqalign{\tilde R
&=\{(\pmb{\alpha},\mu):\pmb{\alpha}\in(\NN)^{\Bbb N},\,\mu\in\PR,\,
\mu R_1[\{\pmb{\alpha}(n)\}]\ge 1-2^{-n}\text{ for every }n\in\Bbb N\}\cr
&\subseteq(\NN)^{\Bbb N}\times\PR.\cr}$$

\noindent(Of course $R_1[\{\pmb{\alpha}(n)\}]$ is compact, therefore
universally measurable, whenever $\pmb{\alpha}\in(\NN)^{\Bbb N}$ and
$n\in\Bbb N$.)

\medskip

\quad{\bf (ii)} $\tilde R[(\NN)^{\Bbb N}]=\PR$.   \Prf\ $\NN\times X$ is
K-analytic (422Ge), while $R$ is a closed subset of $\NN\times X$ (422Da),
so is itself K-analytic (422Gf).   Let $\pi_1:R\to\NN$ and
$\pi_2:R\to X$ be the coordinate maps.   If $\mu\in\PR$, there is a Radon
probability measure $\lambda$ on $R$ such that $\mu=\lambda\pi_2^{-1}$
(432G).   For each $n\in\Bbb N$ let $L_n\subseteq R$ be a compact set such
that $\lambda L_n>1-2^{-n}$;  then $\pi_1[L_n]$ is a non-empty
compact subset of $\NN$.   Define $\pmb{\alpha}$ by setting

\Centerline{$\pmb{\alpha}(n)(m)=\sup\{\beta(m):\beta\in\pi_1[L_n]\}$}

\noindent for $m$, $n\in\Bbb N$.   Then

\Centerline{$\mu R_1[\{\pmb{\alpha}(n)\}]
\ge\mu R[\pi_1[L_n]]
\ge\mu\pi_2[L_n]
\ge\lambda L_n
\ge 1-2^{-n}$}

\noindent for every $n\in\Bbb N$, so $(\pmb{\alpha},\mu)\in\tilde R$ and
$\mu\in\tilde R[(\NN)^{\Bbb N}]$.\ \Qed

\medskip

\quad{\bf (iii)} $\tilde R[\{\pmb{\alpha}\}]$ is a compact subset of $\PR$
for every $\pmb{\alpha}\in(\NN)^{\Bbb N}$.   \Prf\ Since
$R_1[\{\pmb{\alpha}(n)\}]$ is compact for every $n$,
$\tilde R[\{\pmb{\alpha}\}]$ is uniformly tight, therefore
relatively compact in $\MR$, by
437Pb.   On the other hand, $\{\mu:\mu\in\MR$, $\mu X=1\}$
and $\{\mu:\mu\in\MR$,
$\mu(X\setminus R_1[\{\pmb{\alpha}(n)\}])\le 1-2^{-n}\}$ are closed for
every $n$, so $\tilde R[\{\pmb{\alpha}\}]$ is closed, therefore compact.\
\Qed

\medskip

\quad{\bf (iv)} If $F\subseteq\PR$ is closed, then $\tilde R^{-1}[F]$ is
closed in $(\NN)^{\Bbb N}$.   \Prf\ Let $\sequence{k}{\pmb{\alpha}_k}$ be a
sequence in $\tilde R^{-1}[F]$ converging to $\pmb{\alpha}$ in
$(\NN)^{\Bbb N}$.   For each $k\in\Bbb N$ choose $\mu_k\in F$ such that
$(\pmb{\alpha}_k,\mu_k)\in\tilde R$.   For $n$, $k\in\Bbb N$ set
$L_{nk}=\{\pmb{\alpha}(n)\}\cup\{\pmb{\alpha}_l(n):l\ge k\}$.   Then
$L_{nk}$ is a compact subset of $\NN$, so $R_1[L_{nk}]$ is a compact subset
of $X$.   \Quer\ If
$x\in\bigcap_{k\in\Bbb N}R_1[L_{nk}]\setminus R_1[\{\pmb{\alpha}(n)\}]$, then
for every $k\in\Bbb N$ there is an $l_k\ge k$ such that
$(\pmb{\alpha}_{l_k}(n),x)\in R_1$;  but $R_1^{-1}[\{x\}]$ is closed in
$\NN$, so contains
$\lim_{k\to\infty}\pmb{\alpha}_{l_k}(n)=\pmb{\alpha}(n)$, and
$x\in R_1[\{\pmb{\alpha}(n)\}]$.\ \BanG\   Thus
$\bigcap_{k\in\Bbb N}R_1[L_{nk}]=R_1[\{\pmb{\alpha}(n)\}]$.

For any $n$ and $k$,
$\mu_lR_1[L_{nk}]\ge\mu_lR_1[\{\pmb{\alpha}_l(n)\}]\ge 1-2^{-n}$ for every
$l\ge k$.   In the first place, taking $k=0$,
$\{\mu_l:l\in\Bbb N\}$ is uniformly tight, therefore relatively compact and
$\sequence{l}{\mu_l}$ has
a cluster point $\mu$ say, which must belong to $F$.   Now, for any $n$,

$$\eqalignno{\mu R_1[\{\pmb{\alpha}(n)\}]
&=\inf_{k\in\Bbb N}\mu R_1[L_{nk}]
\ge\inf_{k\in\Bbb N,l\ge k}\mu_lR_1[L_{nk}]\cr
\displaycause{because $R_1[L_{nk}]$ is compact, therefore closed, and
$\mu\in\overline{\{\mu_l:l\ge k\}}$, for each $k$}
&\ge\inf_{k\in\Bbb N,l\ge k}\mu_lR_1[\{\pmb{\alpha}_l(n)\}]
\ge 1-2^{-n}.\cr}$$

\noindent So $(\pmb{\alpha},\mu)\in\tilde R$ and
$\pmb{\alpha}\in\tilde R^{-1}[F]$.   As $\sequence{k}{\pmb{\alpha}_k}$ is
arbitrary, $\tilde R^{-1}[F]$ is closed.\ \Qed

\medskip

\quad{\bf (v)} Thus $\tilde R\subseteq(\NN)^{\Bbb N}\times\PR$ is
usco-compact.   Since $(\NN)^{\Bbb N}$, like $\NN$, is Polish (4A2Ub,
4A2Qc), $\tilde R[(\NN)^{\Bbb N}]$ is K-analytic (422Gd).   But we saw in
(ii) that $\tilde R[(\NN)^{\Bbb N}]=\PR$.   So $\PR$ is K-analytic.

\medskip

\quad{\bf (vi)} Now observe that
$(\alpha,\mu)\mapsto\alpha\mu:\coint{0,\infty}\times\PR\to\MR$ is
continuous.   \Prf\ We have only to note that

\Centerline{$(\alpha,\mu)\mapsto(\alpha\mu)(X)=\alpha$}

\noindent is continuous, and that

\Centerline{$\{(\alpha,\mu):(\alpha\mu)(G)>\gamma\}
=\bigcup_{\beta>0}\{(\alpha,\mu):\alpha>\beta$,
$\mu G>\Bover{\gamma}{\beta}\}$}

\noindent is open for every open $G\subseteq X$ and $\gamma\ge 0$.\ \QeD\
Since $\coint{0,\infty}$ and $\PR$ are K-analytic, and (except in the
trivial case $X=\emptyset$) every member of
$\MR$ is expressible as a non-negative multiple of a probability measure,
$\MR$ is K-analytic (using 422Ge and 422Gd again).

\medskip

\quad{\bf (vii)} Finally, $\MqR=\MR$ by 432E.

\medskip

{\bf (e)} Put (d), (b) and 423C together.

\medskip

{\bf (f)(i)} Because $X$ is compact, every quasi-Radon measure on $X$ is
tight, and $\MqR$ itself is uniformly tight;
by 437Pa, $\{\mu:\mu\in\MqR$, $\mu X\le\gamma\}$
is relatively compact in $\MqR$.   But as
it is also closed in $\MqR$, it is actually compact.   The same argument
applies to $\{\mu:\mu\in\MqR$, $\mu X=\gamma\}$.

\medskip

\quad{\bf (ii)} Use the same idea, but with 437Pb in place of 437Pa.

\medskip

{\bf (g)(i)} Write $\frakTKR$ for the topology generated by
$\rhoKR$.

\medskip

\qquad\grheada\ If $\mu\in\MqR$ and $\mu X>\alpha$, then $\nu X>\alpha$
whenever $\nu\in\MqR$ and $\rhoKR(\mu,\nu)<\mu X-\alpha$, just because
$\chi X$ is a $1$-Lipschitz function;  so
$\{\mu:\mu\in\MqR$, $\mu X>\alpha\}\in\frakTKR$ for every
$\alpha\in\Bbb R$.

If $G\subseteq X$ is open, $\alpha\ge 0$ and
$\mu\in\MqR$ is such that $\mu G>\alpha$, there is a
$\delta\in\ocint{0,1}$ such that $\mu F>\alpha+\delta$, where
$F=\{x:\rho(x,X\setminus G)\ge\delta\}$.   Let $u$ be a $1$-Lipschitz
function such that $\delta\chi F\le u\le\delta\chi G$.
If $\nu\in\MqR$ and $\rhoKR(\mu,\nu)\le\delta^2$, then

\Centerline{$\delta\nu G
\ge\int u\,d\nu
\ge\int u\,d\mu-\delta^2
\ge\delta\mu F-\delta^2
>\delta\alpha$}

\noindent and $\nu G>\alpha$.   This shows that
$\{\mu:\mu\in\MqR$, $\mu G>\alpha\}\in\frakTKR$.   As $G$ and $\alpha$ are
arbitrary, $\frakTKR$ is finer than the narrow topology.

\medskip

\qquad\grheadb\ Suppose that $\mu\in\MqR$ and
$\epsilon>0$;  let $\delta>0$ be such that
$\delta(3\delta+6\mu X+7)\le\epsilon$.   Then there is a
totally bounded closed set $F\subseteq X$ such that
$\mu(X\setminus F)\le\delta$ (434L).   Set $G=\{x:\rho(x,F)<\delta\}$.
Let $x_0,\ldots,x_n\in X$ be such
that $F\subseteq\bigcup_{i\le n}B(x_i,\delta)$;
then $G\subseteq\bigcup_{i\le n}B(x_i,2\delta)$ and
there are $v_0,\ldots,v_n\in C_b(X)^+$
such that $\chi G\le\sum_{i=0}^nv_i(x)\le\chi X$ and
$\{x:v_i(x)>0\}\subseteq B(x_i,3\delta)$ for every $i\le n$.   Let
$w\in C_b(X)$ be such that $\chi(X\setminus G)\le w\le\chi(X\setminus F)$.
By the choice of $F$, $\int w\,d\mu\le\delta$.

If $u:X\to[-1,1]$ is $1$-Lipschitz then

\Centerline{$|u-\sum_{i=0}^nu(x_i)v_i|\le 3\delta\chi X+2w$.}

\noindent\Prf\ If $x\in G$, then

$$\eqalignno{|u(x)-\sum_{i=0}^nu(x_i)v_i(x)|
&=|\sum_{i=0}^n(u(x)-u(x_i))v_i(x)|\cr
\displaycause{because $\sum_{i=0}^nv_i(x)=1$}
&\le\sum_{i=0}^n|u(x)-u(x_i)|v_i(x)
\le\sum_{i=0}^n3\delta v_i(x)\cr
\displaycause{because whenever $v_i(x)>0$,
$|u(x)-u(x_i)|\le\rho(x,x_i)\le 3\delta$}
&\le 3\delta.\cr}$$

\noindent If $x\in X\setminus G$, then

\Centerline{$|u(x)-\sum_{i=0}^nu(x_i)v_i(x)|
\le|u(x)|+\sum_{i=0}^n|u(x_i)|v_i(x)
\le 2=2w(x)$.  \Qed}

\noindent So if $\nu\in\MqR$,

\Centerline{$|\int u\,d\nu-\sum_{i=0}^nu(x_i)\int v_id\nu|
\le 3\delta\nu X+2\int w\,d\nu$.}

By 437Jf or 437L,
there is a neighbourhood $V$ of $\mu$ for the narrow topology
in $\MqR$ such that if $\nu\in V$ then $\nu X\le\mu X+\delta$,
$\int w\,d\nu\le 2\delta$ and
$|\int v_id\mu-\int v_id\nu|\le\Bover{\delta}{n+1}$ for every
$i\le n$.   So if $\nu\in V$ and
$u:X\to[-1,1]$ is $1$-Lipschitz, we shall have

$$\eqalign{\bigl|\int u\,d\nu-\int u\,d\mu\bigr|
&\le 3\delta(\nu X+\mu X)+2(\int w\,d\nu+\int w\,d\mu)\cr
&\mskip180mu
   +\sum_{i=0}^n\bigl|\int v_id\nu-\int v_id\mu\bigr|\cr
&\le 3\delta(2\mu X+\delta)+6\delta
   +\sum_{i=0}^n\Bover{\delta}{n+1}
\le\delta(6\mu X+3\delta+7)\le\epsilon.\cr}$$

\noindent Thus $\{\nu:\rhoKR(\nu,\mu)\le\epsilon\}\supseteq V$
is a neighbourhood
of $\mu$ for the narrow topology;  as $\mu$ and $\epsilon$ are arbitrary,
the narrow topology is finer than $\frakTKR$, and the two topologies are
equal.

\medskip

\quad{\bf (ii)}
If $X$ is $\rho$-complete then $\MqR=\MR$ by 434Jg and 434Jb.
Now suppose that $\sequencen{\mu_n}$ is a $\rhoKR$-Cauchy sequence
in $\MR$.

\medskip

\qquad\grheada\ For every $\epsilon\in\ocint{0,1}$ there is a compact
$K\subseteq X$
such that $\mu_n(X\setminus U(K,\epsilon))\le\epsilon$ for every
$n\in\Bbb N$, where
$U(K,\epsilon)=\{x:\rho(x,y)<\epsilon$ for some $y\in K\}$.   \Prf\
Take $m\in\Bbb N$ such that $\rhoKR(\mu_m,\mu_n)\le\bover12\epsilon^2$
for every $n\ge m$.   Let $K\subseteq X$ be a compact set such that
$\mu_n(X\setminus K)\le\bover12\epsilon$ for every $n\le m$.
Set $G=U(K,\epsilon)$.   There is a
$1$-Lipschitz function $u:X\to[0,\epsilon]$ such that
$\epsilon\chi(X\setminus G)\le u\le\chi(X\setminus K)$.   If $n\le m$, then
of course $\mu_n(X\setminus G)\le\epsilon$.   If $n\ge m$, then

$$\eqalign{\mu_n(X\setminus G)
&\le\Bover1{\epsilon}\int u\,d\mu_n
\le\Bover1{\epsilon}\bigl(\rhoKR(\mu_n,\mu_m)+\int u\,d\mu_m\bigr)\cr
&\le\Bover1{\epsilon}
  \bigl(\Bover{\epsilon^2}2+\epsilon\mu_m(X\setminus K)\bigr)
\le\Bover{\epsilon}2+\mu_m(X\setminus K)
\le\epsilon.\cr}$$

\noindent So we have an appropriate $K$.\ \Qed

\medskip

\qquad\grheadb\ $\{\mu_n:n\in\Bbb N\}$ is uniformly totally finite and
uniformly tight.   \Prf\ Since $|\mu_mX-\mu_nX|\le\rhoKR(\mu_m,\mu_n)$
for all $m$, $n\in\Bbb N$, $\{\mu_nX:n\in\Bbb N\}$ is bounded.
Of course all the $\mu_n$ are tight.   Now take any
$\epsilon\in\ocint{0,1}$.   For each $m\in\Bbb N$, (i) tells us that
there is a compact set $K_m\subseteq X$ such that
$\mu_n(X\setminus U(K_m,2^{-m}\epsilon))\le 2^{-m}\epsilon$ for every
$n\in\Bbb N$.   Set $E=\bigcap_{m\in\Bbb N}U(K_m,2^{-m}\epsilon)$,
$K=\overline{E}$.   Then $E$ and $K$ are totally bounded;  because
$(X,\rho)$ is complete, $K$ is compact.   And

\Centerline{$\mu_n(X\setminus K)
\le\sum_{m=0}^{\infty}\mu_n(X\setminus U(K_m,2^{-m}\epsilon))
\le 2\epsilon$}

\noindent for every $n\in\Bbb N$.   As $\epsilon$ is arbitrary,
$\{\mu_n:n\in\Bbb N\}$ is uniformly tight.\ \Qed

\medskip

\qquad\grheadc\ By 437Pb, $\sequencen{\mu_n}$ has a cluster point $\mu$
in $\MR$ for the narrow topology.   Now, for $m\in\Bbb N$,

$$\eqalignno{\rhoKR(\mu,\mu_m)
&=\sup\{|\int u\,d\mu-\int u\,d\mu_m|:
  u:X\to[-1,1]\text{ is }1\text{-Lipschitz}\}\cr
&\le\sup\{|\int u\,d\mu_n-\int u\,d\mu_m|:
  u:X\to[-1,1]\text{ is }1\text{-Lipschitz},\,n\ge m\}\cr
\displaycause{437Jf again}
&\le\sup_{n\ge m}\rhoKR(\mu_n,\mu_m),\cr}$$

\noindent and $\lim_{n\to\infty}\rhoKR(\mu,\mu_m)=0$.

\medskip

\qquad\grheadd\ Thus every $\rhoKR$-Cauchy sequence in $\MR$ has a
limit in $\MR$, and $\MR$ is complete.

\medskip

{\bf (h)} Put (g-ii) and (c-ii) together.
}%end of proof of 437R

\leader{437S}{}\cmmnt{ The sets of measures we have been considering have
generally been convex, if addition and multiplication by non-negative
scalars are defined as in 234G and 234Xf.   We can therefore look for
extreme points, in the hope that they will have straightforward
characterizations, as in the following.

\medskip

\noindent}{\bf Proposition}\dvAformerly{4{}37P} Let $X$ be a Hausdorff
space, and $\PR$ the
set of Radon probability measures on $X$.   Then the extreme points
of $\PR$ are just the Dirac measures on $X$.

\proof{{\bf (a)} Suppose that $x\in X$, and that $\delta_x$ is the Dirac
measure on $X$ concentrated at $x$.   If $\mu_1$,
$\mu_2\in\PR$ are
such that $\delta_x=\bover12(\mu_1+\mu_2)$, then we must have
$\mu_1E\le 2\mu E$ for every Borel set $E$;
in particular, $\mu_1(X\setminus\{x\})=0$ and $\mu_1\{x\}=1$, that is,
$\mu_1=\delta_x$.   Similarly, $\mu_2=\delta_x$;  as $\mu_1$ and $\mu_2$
are arbitrary, $\delta_x$ is an extreme point of $\PR$.

\medskip

{\bf (b)} Suppose that $\mu$ is an extreme point of $\PR$.   Let $K$ be
the support of $\mu$.   \Quer\ If $K$ has more than one point, take
distinct $x$, $y\in K$.
As $X$ is Hausdorff, there are disjoint open sets $G$, $H$ such that
$x\in G$ and $y\in H$.   Set $E=G\cap K$, $\alpha=\mu E$.   Because $K$
is the support of $\mu$,
$\alpha>0$.   But similarly $\mu(H\cap K)>0$ and $\alpha<1$.   Let
$\mu_1$, $\mu_2$ be the indefinite-integral measures defined over $\mu$
by $\Bover1{\alpha}\chi E$ and $\Bover1{\beta}\chi(X\setminus E)$
respectively.   Then both are Radon probability measures on $X$ (416S),
so belong to $\PR$.
Now $\mu F=\alpha\mu_1F+(1-\alpha)\mu_2F$ for every Borel set $F$;  as
$\mu$ and $\alpha\mu_1+(1-\alpha)\mu_2$ are both Radon measures, they
coincide;  as
neither $\mu_1$ nor $\mu_2$ is equal to $\mu$, $\mu$ is not extreme in
$\PR$.\ \Bang

Thus $K=\{x\}$ for some $x\in X$.   But this means that $\mu\{x\}=1$ and
$\mu(X\setminus\{x\})=0$, so $\mu=\delta_x$ is of the declared form.
}%end of proof of 437S

%437xT
\leader{437T}{}\cmmnt{ We now have a language in which to express a
fundamental result in the theory of dynamical systems.

\medskip

\noindent}{\bf Theorem}\dvAformerly{4{}37S} Let $X$ be a non-empty
compact Hausdorff space, and $\phi:X\to X$ a continuous function.
Write $Q_{\phi}$ for the set of Radon probability measures on $X$ for
which $\phi$ is \imp.   Then $Q_{\phi}$ is convex and not empty, and is
compact for the narrow topology.

\proof{{\bf (a)} Write $\MR$ for the set of totally finite Radon
measures on $X$, and let $\tilde\phi:\MR\to\MR$ be the function
corresponding to $\phi:X\to X$ as described in  437N.
Now, for $\mu\in\MR$, $\mu\in Q_{\phi}$ iff $\mu X=1$ and
$\mu(\phi^{-1}[E])=\mu E$ whenever $\mu$ measures $E$, that is, iff the
image measure $\mu\phi^{-1}=\tilde\phi(\mu)$ extends $\mu$.   But as
$\tilde\phi(\mu)$ and $\mu$ are Radon measures, $\mu\in Q_{\phi}$ iff
$\mu X=1$ and $\tilde\phi(\mu)=\mu$.

Since $\tilde\phi$ is continuous (and $\MR$ is Hausdorff, see 437Ra),
$Q_{\phi}$ is closed for the narrow topology.   By 437Pb/437R(f-ii),
it is compact.   Because $\tilde\phi$ respects addition and scalar
multiplication, $Q_{\phi}$ is convex.

\medskip

{\bf (b)} To see that $Q_{\phi}$ is not empty,
take any $x_0\in X$ and a non-principal ultrafilter $\Cal F$ on
$\Bbb N$.   Define $f:C(X)\to\Bbb R$ by setting
$f(u)=\lim_{n\to\Cal F}\bover1{n+1}\sum_{i=0}^nu(\phi^i(x_0))$ for every
$u\in C(X)$.   Then $f$ is a positive linear functional and
$f(\chi X)=1$.
So there is a $\mu\in\MR$ such that $f(u)=\int u\,d\mu$ for every
$u\in C(X)$.

If $u\in C(X)$, then $f(u)=f(u\phi)$.   \Prf\

$$\eqalign{|f(u\phi)-f(u)|
&=|\lim_{n\to\Cal F}\Bover1{n+1}
  \sum_{i=0}^nu(\phi^{i+1}(x_0))-u(\phi^i(x_0))|\cr
&=|\lim_{n\to\Cal F}\Bover1{n+1}u(\phi^{n+1}(x_0))-u(x_0)|\cr
&\le\lim_{n\to\Cal F}\Bover1{n+1}|u(\phi^{n+1}(x_0))-u(x_0)|
\le\lim_{n\to\Cal F}\Bover{2\|u\|_{\infty}}{n+1}
=0.  \text{ \Qed}\cr}$$

On the other hand,

$$\eqalignno{\int u\,d(\tilde\phi(\mu))
&=\int u\phi\,d\mu\cr
\displaycause{235G}
&=f(u\phi)
=f(u)
=\int u\,d\mu\cr}$$

\noindent for every $u\in C(X)$.   By the uniqueness of the representation
of $f$ as an integral, $\mu=\tilde\phi(\mu)$.   Of course
$\mu X=f(\chi X)=1$ so $\mu\in Q_{\phi}$, as required.
}%end of proof of 437T

\leader{437U}{}\cmmnt{ In important cases, the narrowly compact
subsets of $M^+_{\text{R}}(X)$ are exactly the bounded uniformly tight
sets.   Once again, it is worth introducing a word to describe when this
happens.

\medskip

\noindent}{\bf Definition}\dvAformerly{4{}37V} Let $X$ be a Hausdorff
space and $P_{\text{R}}(X)$ the set
of Radon probability measures on $X$.   $X$ is a {\bf Prokhorov space}
if every subset of $P_{\text{R}}(X)$ which is compact for the narrow
topology is uniformly tight.

\vleader{60pt}{437V}{Theorem}\dvAformerly{4{}37W} 
(a) Compact Hausdorff spaces are Prokhorov spaces.

(b) A closed subspace of a Prokhorov Hausdorff space is a Prokhorov
space.

(c) An open subspace of a Prokhorov Hausdorff space is a Prokhorov
space.

(d) The product of a countable family of Prokhorov Hausdorff spaces is a
Prokhorov space.

(e) Any G$_{\delta}$ subset of a Prokhorov Hausdorff space is a
Prokhorov space.

(f) \v{C}ech-complete spaces are Prokhorov spaces.

(g) Polish spaces are Prokhorov spaces.

\proof{{\bf (a)} This is trivial;  on a compact Hausdorff space the set
of all Radon probability measures is uniformly tight.

\medskip

{\bf (b)} Let $X$ be a Prokhorov Hausdorff space, $Y$ a closed subset of
$X$, and $A\subseteq P_{\text{R}}(Y)$ a narrowly compact set.   Taking
$\phi$ to be the identity map from $Y$ to $X$, and defining
$\tilde\phi:M^+_{\text{R}}(Y)\to M^+_{\text{R}}(X)$ as in 437N,
$\tilde\phi[A]$ is narrowly compact in
$P_{\text{R}}(X)$, so is uniformly tight.   For any $\epsilon>0$, there
is a compact set $K\subseteq X$ such that
$\tilde\phi(\mu)(X\setminus K)\le\epsilon$
for every $\mu\in A$.   Now $K\cap Y$ is a compact subset of $Y$ and
$\mu(Y\setminus(K\cap Y))\le\epsilon$ for every $\mu\in A$.   As
$\epsilon$ is arbitrary,
$A$ is uniformly tight in $P_{\text{R}}(Y)$.

\medskip

{\bf (c)} Let $X$ be a Prokhorov Hausdorff space, $Y$ an open subset of
$X$, and $A\subseteq P_{\text{R}}(Y)$ a narrowly compact set.   Once
again, take $\phi$ to be the
identity map from $Y$ to $X$, so that
$\tilde\phi[A]\subseteq P_{\text{R}}(X)$ is narrowly compact and
uniformly tight in $P_{\text{R}}(X)$.

\Quer\ Suppose, if possible, that $A$ is not uniformly tight in
$P_{\text{R}}(Y)$.   Then there is an $\epsilon>0$ such that
$A_K=\{\mu:\mu\in A$, $\mu(Y\setminus K)\ge 5\epsilon\}$ is non-empty
for every compact set $K\subseteq Y$.   Note that $A_K\subseteq A_{K'}$
whenever
$K\supseteq K'$, so $\{A_K:K\subseteq Y$ is compact$\}$ has the finite
intersection property, and there is an ultrafilter $\Cal F$ on
$P_{\text{R}}(Y)$
containing every $A_K$.   Because $A$ is narrowly compact, there is a
$\lambda\in P_{\text{R}}(Y)$ such that $\Cal F\to\lambda$.   Let
$K^*\subseteq Y$ be
a compact set such that $\lambda(Y\setminus K^*)\le\epsilon$.

As $\tilde\phi[A]$ is uniformly tight, there is a
compact set $L\subseteq X$ such that
$\mu(Y\setminus L)=\tilde\phi(\mu)(X\setminus L)\le\epsilon$
for every $\mu\in A$.   Now $K^*$ and $L\setminus Y$ are disjoint
compact sets in the Hausdorff space $X$, so there are disjoint open sets
$G$, $H\subseteq X$
such that $K^*\subseteq G$ and $L\setminus Y\subseteq H$ (4A2F(h-i) again).   Set
$K=L\setminus H\supseteq L\cap G$;  then
$K$ is a compact subset of $Y$.   As $A_K\in\Cal F$, there must be a
$\mu\in A_K$ such that $\mu Y\le\lambda Y+\epsilon$ and
$\mu(G\cap Y)\ge\lambda(G\cap Y)-\epsilon$.   Accordingly

$$\eqalign{\mu(Y\setminus L)
&\le\epsilon,\cr
\mu(Y\setminus G)
&=\mu Y-\mu(G\cap Y)
\le\lambda Y+\epsilon-\lambda(G\cap Y)+\epsilon\cr
&=\lambda(Y\setminus G)+2\epsilon
\le\lambda(Y\setminus K^*)+2\epsilon\le 3\epsilon,\cr
\mu((Y\setminus L)\cup(Y\setminus G))
&=\mu(Y\setminus(L\cap G))\ge\mu(Y\setminus K)\ge 5\epsilon,\cr}$$

\noindent which is impossible.\ \Bang

Thus $A$ is uniformly tight.   As $A$ is arbitrary, $Y$ is a Prokhorov
space.

\medskip

{\bf (d)} Let $\sequencen{X_n}$ be a sequence of Prokhorov Hausdorff
spaces with product $X$.   Let $A\subseteq P_{\text{R}}(X)$ be a
narrowly compact set.
Let $\epsilon>0$.   For each $n\in\Bbb N$ let $\pi_n:X\to X_n$ be the
canonical map and
$\tilde\pi_n:M^+_{\text{R}}(X)\to M^+_{\text{R}}(X_n)$ the
associated function.
Then $\tilde\pi_n[A]$ is narrowly compact in $P_{\text{R}}(X_n)$,
therefore uniformly tight, and there is a compact set $K_n\subseteq X_n$
such that
$(\tilde\pi_n\mu)(X_n\setminus K_n)\le 2^{-n-1}\epsilon$ for every
$\mu\in A$.   Set
$K=\prod_{n\in\Bbb N}K_n$, so that $K$ is a compact subset of $X$ and
$X\setminus K=\bigcup_{n\in\Bbb N}\pi_n^{-1}[X_n\setminus K_n]$.   If
$\mu\in A$, then

\Centerline{$\mu(X\setminus K)
\le\sum_{n=0}^{\infty}\mu\pi_n^{-1}[X_n\setminus K_n]
\le\sum_{n=0}^{\infty}2^{-n-1}\epsilon=\epsilon$.}

\noindent As $\epsilon$ is arbitrary, $A$ is uniformly tight;  as $A$ is
arbitrary, $X$ is a Prokhorov space.

\medskip

{\bf (e)} Let $X$ be a Prokhorov Hausdorff space and $Y$ a G$_{\delta}$
subset of $X$.   Express $Y$ as $\bigcap_{n\in\Bbb N}Y_n$ where every
$Y_n\subseteq X$ is open.
Set $Z=\{z:z\in\prod_{n\in\Bbb N}Y_n$, $z(m)=z(n)$ for all $m$,
$n\in\Bbb N\}$.
Because $X$ is Hausdorff, $Z$ is a closed subspace of
$\prod_{n\in\Bbb N}Y_n$ homeomorphic to $Y$.
Putting (c), (d) and (b) together, $Z$ and $Y$ are Prokhorov spaces.

\medskip

{\bf (f)} Put (a), (e) and the definition of `\v{C}ech-complete'
together.

\medskip

{\bf (g)} This is a special case of (f) (4A2Md).
}%end of proof of 437V

\exercises{
\leader{437X}{Basic exercises (a)}
%\spheader 437Xa
Let $X$ be a set, $U$ a Riesz subspace of $\Bbb R^X$ and
$f\in U^{\sim}$.   (i) Show that $f\in U^{\sim}_{\sigma}$ iff
$\lim_{n\to\infty}f(u_n)=0$ whenever $\sequencen{u_n}$ is a
non-increasing sequence in $U$ such that $\lim_{n\to\infty}u_n(x)=0$ for
every $x\in X$.   \Hint{show that in this case, if $0\le v_n\le u_n$, we
can find $k(n)$ such that $|f(v_n\vee u_{k(n)})-f(v_n)|\le 2^{-n}$.}
(ii) Show that $f\in U^{\sim}_{\tau}$ iff $\inf_{u\in A}|f(u)|=0$
whenever $A\subseteq U$ is a non-empty downwards-directed set and
$\inf_{u\in A}u(x)=0$ for every $x\in X$.   \Hint{given $\epsilon>0$,
set $B=\{v:f(v)\ge\inf_{u\in A}f^+(u)-\epsilon,\,\exists\,w\in A,\,v\ge
w\}$ and show that $B$ is a downwards-directed set with infimum $0$ in
$\Bbb R^X$.}
%437A

\spheader 437Xb Let $X$ be a set, $U$ a Riesz subspace of
$\ell^{\infty}(X)$ containing the constant functions, and $\Sigma$ the
smallest $\sigma$-algebra of subsets of $X$ with respect to which every
member of $U$ is measurable.   Let $\mu$ and $\nu$ be two totally finite
measures on $X$ with domain $\Sigma$, and $f$, $g$ the corresponding
linear functionals on $U$.   Show that $f\wedge g=0$ in $U^{\sim}$ iff
there is an $E\in\Sigma$ such that $\mu E=\nu(X\setminus E)=0$.
\Hint{326M\formerly{3{}26I}.}
%437C

\sqheader 437Xc\dvAnew{2012} Let $I\subseteq\Bbb R$ be a
interval with at least two points.   (i) Show that if $g:I\to\Bbb R$ is of
bounded variation on every compact subinterval of $I$,
there is a unique signed
tight Borel measure $\mu_g$ on $I$ such that
$\mu_g[a,b]=\lim_{x\downarrow b}g(x)-\lim_{x\uparrow a}g(x)$ whenever
$a\le b$ in $I$, counting $\lim_{x\uparrow a}g(x)$ as $g(a)$ if
$a=\min I$, and $\lim_{x\downarrow b}g(x)$ as $g(b)$ if $b=\max I$.
(ii) Show that if $h:I\to\Bbb R$ is another
function of bounded variation on every compact subinterval, then
$\mu_h=\mu_g$ iff $\{x:h(x)\ne g(x)\}$ is countable iff
$\{x:h(x)=g(x)\}$ is dense in $I$.
(iii) Show that if $\nu$ is any signed Baire
measure on $I$ there is a $g$ of bounded variation on every compact
subinterval such that $\nu=\mu_g$.
%437G

\spheader 437Xd(i) Show that $S$, in 437C, is the unique sequentially
order-continuous positive linear operator from $\eusm L^{\infty}$ to
$(U^{\sim}_{\sigma})^*$ which extends the canonical embedding of $U$ in
$(U^{\sim}_{\sigma})^*$.   (ii) Show that $S$, in 437H, is the unique
sequentially order-continuous positive linear operator from $\eusm
L^{\infty}$ to $(U^{\sim}_{\tau})^*$ which extends the canonical
embedding of $U$ in $(U^{\sim}_{\tau})^*$ and is `$\tau$-additive' in
the sense that whenever $\Cal G$ is a non-empty upwards-directed family
of open sets with union $H$ then $S(\chi H)=\sup_{G\in\Cal G}S(\chi G)$
in $(U^{\sim}_{\tau})^*$.
%437H

\spheader 437Xe Let $X$ and $Y$ be completely regular
topological spaces and $\phi:X\to Y$
a continuous function.   Define $T:C_b(Y)\to C_b(X)$ by setting
$T(v)=v\phi$ for every $v\in C_b(Y)$, and let $T':C_b(X)^*\to C_b(Y)^*$
be its adjoint.   (i) Show that $T'$ is a norm-preserving Riesz
homomorphism.   (ii) Show that
$T'[C_b(X)^{\sim}_{\sigma}]\subseteq C_b(Y)^{\sim}_{\sigma}$, and that
if $f\in C_b(X)^{\sim}_{\sigma}$ corresponds to a Baire measure $\mu$ on
$X$, then $T'f$ corresponds to the Baire measure
$\mu\phi^{-1}\restr\CalBa(Y)$.   (iii) Show that
$T'[C_b(X)^{\sim}_{\tau}]\subseteq C_b(Y)^{\sim}_{\tau}$, and that if
$f\in C_b(X)^{\sim}_{\tau}$ corresponds to a Borel measure $\mu$ on $X$,
then $T'f$ corresponds to the Borel measure
$\mu\phi^{-1}\restr\Cal B(Y)$.   (iv) Write $\Cal L^{\infty}_X$ and
$\Cal L^{\infty}_Y$ for the
$M$-spaces of bounded real-valued Borel measurable functions on $X$, $Y$
respectively, and $S_X:\eusm L^{\infty}_X\to(C_b(X)^{\sim}_{\tau})^*$,
$S_Y:\eusm L^{\infty}_Y\to(C_b(Y)^{\sim}_{\tau})^*$ for the canonical
Riesz homomorphisms as constructed in 437Hb.   Show that if
$T'':(C_b(Y)^{\sim}_{\tau})^*\to(C_b(X)^{\sim}_{\tau})^*$ is the adjoint
of $T'\restr C_b(X)^{\sim}_{\tau}$, then $T''S_Y(v)=S_X(v\phi)$ for
every $v\in\eusm L^{\infty}_Y$.
%437H

\spheader 437Xf Let $X$ be a topological space,
$\eusm L^{\infty}(\Sigma_{\text{um}})$ the
space of bounded universally measurable real-valued functions on $X$,
and $M_{\sigma}$ the space of countably additive functionals on the
Borel $\sigma$-algebra of $X$.
Show that we have a sequentially order-continuous
Riesz homomorphism
$S:\eusm L^{\infty}(\Sigma_{\text{um}})\to M_{\sigma}^*$ defined by the
formula $(Sv)(\mu)=\int v\,d\mu$ whenever
$v\in\eusm L^{\infty}(\Sigma_{\text{um}})$ and $\mu\in M_{\sigma}^+$.
%437I

\spheader 437Xg Let $X$ be a completely regular topological space.
Show that the vague topology on the space $M_{\tau}$ of differences of
$\tau$-additive totally finite Borel measures on $X$ is Hausdorff.
%437J

\sqheader 437Xh Let $X$ and $Y$ be topological spaces, and $\phi:X\to Y$
a continuous function.   Write $M_{\#}(X)$ for any of
$M(\CalBa(X))$, $M_{\sigma}(\CalBa(X))$, $M(\Cal B(X))$,
$M_{\sigma}(\Cal B(X))$, $M_{\tau}(X)$ or $M_t(X)$, where
$M_{\tau}(X)\subseteq M_{\sigma}(\Cal B(X))$
is the space of signed $\tau$-additive Borel measures and
$M_t(X)\subseteq M_{\tau}(X)$ is the space of signed tight
Borel measures;  and $M_{\#}(Y)$ for the corresponding space based on
$Y$.   Show that there is a positive linear operator
$\tilde\phi:M_{\#}(X)\to M_{\#}(Y)$ defined by saying that
$\tilde\phi(\mu)(E)=\mu\phi^{-1}[E]$ whenever $\mu\in M_{\#}(X)$
and $E$ belongs to $\CalBa(Y)$ or $\Cal B(Y)$, as appropriate, and that
$\tilde\phi$ is continuous for the vague topologies on $M_{\#}(X)$ and
$M_{\#}(Y)$.
%437J

\sqheader 437Xi Let $X$ be a zero-dimensional compact Hausdorff
space and $\Cal E$ the algebra of open-and-closed subsets of $X$.   (i)
Show that $\Cal E$ separates
zero sets.   (ii) Show that the vague topology on $M(\Cal E)$ is just
the topology of pointwise convergence induced by the usual topology of
$\BbbR^{\Cal E}$.
(iii) Writing $M_t$ for the space of signed tight Borel measures on $X$,
show that $\mu\mapsto\mu\restr\Cal E:M_t\to M(\Cal E)$ is a Banach
lattice isomorphism
between the $L$-spaces $M_t$ and $M(\Cal E)$, and is also a
homeomorphism when $M_t$ and $M(\Cal E)$ are given their vague
topologies.
%437J

\spheader 437Xj\dvAnew{2012}
(i) Let $X$ be a topological space, and $\Sigma$ an algebra of
subsets of $X$ containing every open set;  let $M(\Sigma)^+$ be the set of
non-negative real-valued additive functionals on $\Sigma$, endowed with its
narrow topology, $E$ a member
of $\Sigma$, and $\partial E$ its boundary.
Show that $\nu\mapsto\nu E:M(\Sigma)^+\to\coint{0,\infty}$
is continuous at $\nu_0\in M(\Sigma)^+$ iff $\nu_0(\partial E)=0$.
%Mushtari 96, p 2
(ii) Let $X$ be a completely regular topological space, and $\Sigma$ a
$\sigma$-algebra of subsets of $X$ including the Baire $\sigma$-algebra.
Write $M_{\sigma}$ for the $L$-space of countably additive functionals on
$\Sigma$.   Let $\Cal F$ be a filter on the positive cone $M_{\sigma}^+$
and $\mu$ a member of $M_{\sigma}^+$.
Show that $\Cal F\to\mu$ for the vague topology on $M_{\sigma}$ iff
$\mu E=\lim_{\nu\to\Cal F}\nu E$ whenever $E\in\Sigma$ and
$\mu(\partial E)=0$.
%437J

\spheader 437Xk\dvAnew{2009}
Let $X$ be a compact Hausdorff space, $M^+_{\text{R}}$
the set of Radon measures on $X$ and $P_{\text{R}}$ the set of Radon
probability measures on $X$.
(i) Show that $M^+_{\text{R}}$, with its narrow
topology and its natural convex structure,
can be identified with the positive cone of $C(X)^*$ with its
weak* topology.
(ii) Show that $P_{\text{R}}$, with its narrow
topology and its natural convex structure, can be identified with
$\{f:f\in C(X)^*$, $f\ge 0$, $f(\chi X)=1\}$ with its weak* topology.
%437L

\spheader 437Xl In 437Mc, show that $|\psi(\mu,\nu)|=\psi(|\mu|,|\nu|)$
for every $\mu\in M_{\tau}(X)$ and $\nu\in M_{\tau}(Y)$.
%437M

\spheader 437Xm\dvAnew{2009} Let $X$ be a topological space, $Y$ a regular
topological space and $\MqR(X)$, $\MqR(Y)$ the spaces of totally finite
quasi-Radon measures on $X$, $Y$ respectively.   For a continuous
$\phi:X\to Y$ define $\tilde\phi:\MqR(X)\to\MqR(Y)$ by saying that $\phi$
is \imp\ for $\mu$ and $\tilde\phi(\mu)$ for every $\mu\in\MqR(X)$ (418Hb).
Show that $\tilde\phi$ is continuous for the narrow topologies.
%437N

\sqheader 437Xn\dvAnew{2010} Let
$\familyiI{(X_i,\frak T_i,\Sigma_i,\mu_i)}$ be a countable family of Radon
probability spaces, and $Q$ the set of Radon probability measures $\mu$ on
$X=\prod_{i\in I}X_i$ such that the image of $\mu$ under the map
$x\mapsto x(i)$ is
$\mu_i$ for every $i\in I$.   Show that $Q$ is uniformly tight and is
compact for the narrow topology on the set of
totally finite topological measures on $X$.
%what if I uncountable? fails on  [0,1[^{\non\Cal N}  I think
%437N 4{}57L

\spheader 437Xo\dvAnew{2009}
Let $X$ be any topological space, and $M^+_{\text{qR}}$ the
space of totally finite quasi-Radon measures on $X$.   Show that
$M^+_{\text{qR}}$ is complete in the total variation metric.
%437Qa

\spheader 437Xp\dvAnew{2009}
Let $X$ and $Y$ be topological spaces, and $\rhotv^{(X)}$,
$\rhotv^{(Y)}$, $\rhotv^{(X\times Y)}$
the total variation metrics on the spaces
$M^+_{\text{qR}}(X)$, $M^+_{\text{qR}}(Y)$ and $M^+_{\text{qR}}(X\times Y)$
of quasi-Radon measures.   Let $\mu_1$, $\mu_2$ be totally finite
quasi-Radon measures
on $X$, $\nu_1$, $\nu_2$ totally finite quasi-Radon measures on $Y$, and
$\mu_1\times\nu_1$, $\mu_2\times\nu_2$ the quasi-Radon product measures.
Show that

\Centerline{$\rhotv^{(X\times Y)}(\mu_1\times\nu_1,\mu_2\times\nu_2)
\le\rhotv^{(X)}(\mu_1,\mu_2)\cdot\nu_2Y
  +\mu_1X\cdot\rhotv^{(Y)}(\nu_1,\nu_2)$.}
%437Qa

\spheader 437Xq(i)\dvAformerly{4{}37Xi} Show that the set
$M^+_{\sigma}(\Cal B(X))$ of totally finite
Borel probability measures on $X$ is T$_0$ in its narrow topology for any
topological space $X$.
(ii)\dvAformerly{4{}37Xk} Give $X=\omega_1+1$ its order topology.
Show that the narrow topology on $M^+_{\sigma}(\Cal B(X))$ is not T$_1$.
\Hint{consider interpretations of
Dieudonn\'e's measure on $\omega_1$ and the Dirac measure concentrated at
$\omega_1$ as Borel measures on $X$.}
%437R

\spheader 437Xr Let $X$ be any topological space and $\tilde M^+$ the
set of non-negative additive functionals defined on subalgebras of
$\Cal PX$ containing every open set.   For $\mu$,
$\nu\in\tilde M^+$ define $\mu+\nu\in\tilde M^+$ by setting
$(\mu+\nu)(E)=\mu E+\nu E$ for $E\in\dom\mu\cap\dom\nu$.   (i) Show that
addition on $\tilde M^+$ is continuous for the narrow topology.   (ii)
Show that $(\alpha,\mu)\mapsto\alpha\mu:
\coint{0,\infty}\times\tilde M^+\to \tilde M^+$ is continuous for the
narrow topology on $\tilde M^+$.   (iii) Writing $\tilde P$ for
$\{\mu:\mu\in\tilde M^+$, $\mu X=1\}$, and $\delta_x$ for the Dirac measure
concentrated at $x$ for each $x\in X$, show that the convex hull of
$\{\delta_x:x\in X\}$ is dense in $\tilde P$ for the narrow topology.
(iv) Suppose that $A$ and $B$
are uniformly tight subsets of $\tilde M^+$ and $\gamma\ge 0$.
Show that $A\cup B$, $A+B=\{\mu+\nu:\mu\in A$, $\nu\in B\}$ and
$\{\alpha\mu:\mu\in A$, $0\le\alpha\le\gamma\}$ are uniformly tight.
%437R

\spheader 437Xs\dvAnew{2012} Let $(X,\rho)$ be a metric space, $\Sigma$ a
subalgebra of $\Cal PX$ containing all the open sets, and $M=M(\Sigma)$ the
set of bounded finitely additive functionals on
$\Sigma$.   For $\mu$, $\nu\in M$ set

\def\rhoLP{\rho_{\text{LP}}}

\Centerline{$\rhoKR(\mu,\nu)=\sup\{|\dashint u\,d\mu-\dashint u\,d\nu|:
   u:X\to[-1,1]$ is $1$-Lipschitz$\}$,}

$$\eqalign{\rhoLP(\mu,\nu)
&=\inf\{\epsilon:\epsilon>0,\,
   \nu F-\mu U(F;\epsilon)\le\epsilon
   \text{ and }\mu F-\nu U(F;\epsilon)\le\epsilon\cr
&\mskip200mu\text{ for every non-empty closed }F\subseteq X\},\cr}$$

\noindent where $U(F;\epsilon)=\{x:\rho(x,F)<\epsilon\}$ for non-empty
subsets $F$ of $X$.   (i) Show that $\rhoKR$ and $\rhoLP$ are pseudometrics
on $M$.
(ii) Show that if $\mu$, $\nu\in M$ and $\rhoLP(\mu,\nu)=\delta$ then
$\min(1,\delta^2)\le\rhoKR(\mu,\nu)\le 2\delta(1+\delta+|\nu|X)$.
(iii) Show that $\rhoKR$ and $\rhoLP$ induce the
same topology on $M$ and the same uniformity on
$\{\mu:\mu\in M$, $|\mu|X\le\gamma\}$, for any $\gamma\ge 0$.
\Hint{{\smc Bogachev 07}, 8.10.43.
$\rhoLP$ is the {\bf L\'evy-Prokhorov} pseudometric.}
%437J  437R  but does either  \rhoKR  or  \rhoLP  induce vague topy?

%    "Fortet-Mourier metric of order p "  ask Kawabe

\spheader 437Xt\discrversionA{\footnote{Revised 2009.}}{} Let $X$ be a
Hausdorff space, and $\PR$ the set of Radon
probability measures on $X$ with its narrow topology.   For $x\in X$ let
$\delta_x$ be the Dirac measure on $X$ concentrated at $x$.   Show that
$x\mapsto\delta_x$ is a homeomorphism between $X$ and its image in $\PR$.
%437S

\spheader 437Xu Let $X$ be a Hausdorff space and $P_{\text{R}}'$
the set of Radon measures $\mu$ on $X$ such that $\mu X\le 1$.
Show that the extreme points of $P_{\text{R}}'$ are the
Dirac measures on $X$, as in 437S, together with the zero measure.
%437S

\spheader 437Xv\dvAnew{2009} Let $X$ be a T$_0$ topological space, and
$\PqR$ the
set of quasi-Radon probability measures on $X$.   Show that
the extreme points of $\PqR$ are just the Dirac measures on $X$
concentrated on points $x$ such that $\{x\}$ is closed.
%437S

\sqheader 437Xw Let $X$ be a Prokhorov Hausdorff space, and $A$ a set of
totally finite
Radon measures on $X$ which is compact for the narrow topology.   Show
that $A$ is uniformly tight.   \Hint{(i) $\gamma=\sup_{\mu\in A}\mu X$
is finite;  (ii) for any $\epsilon>0$ the set
$\{\bover1{\mu X}\mu:\mu\in A$, $\mu X\ge\epsilon\}$ is narrowly
compact, therefore uniformly tight;  for any $\epsilon>0$ the set
$\{\mu:\mu\in A$, $\mu X\ge\epsilon\}$ is uniformly tight.}
%437U 437Xr(iv)

\spheader 437Xx Give $\omega_1$ its order topology, and let $M_t$ be the
$L$-space of signed tight Borel measures on $\omega_1$.
(i) Show that $\omega_1$ is a Prokhorov space.   (ii) For
$\xi<\omega_1$, define $\mu_{\xi}\in M_t$
by setting $\mu_{\xi}(E)=\chi E(\xi)-\chi E(\xi+1)$ for every Borel set
$E\subseteq\omega_1$.   Show that $\{\mu_{\xi}:\xi<\omega_1\}$ is
relatively compact in
$M_t$ for the vague topology, but is not uniformly tight.   (Compare 437Yy
below.)
%437V

\leader{437Y}{Further exercises (a)}
%\spheader 437Ya
Let $X$ be a set and $U$ a Riesz subspace of $\Bbb R^X$.
Give formulae for the components of a given element of $U^{\sim}$ in the
bands $U^{\sim}_{\sigma}$, $(U^{\sim}_{\sigma})^{\perp}$,
$U^{\sim}_{\tau}$ and $(U^{\sim}_{\tau})^{\perp}$.   \Hint{356Yb.}
%437A

\spheader 437Yb Let $X$ be a compact Hausdorff space.   Show that the
dual $C(X;\Bbb C)^*$ of the
complex linear space of continuous functions from $X$ to $\Bbb C$ can be
identified with the space of `complex tight Borel measures' on $X$, that
is, the space of functionals $\mu:\Cal B(X)\to\Bbb C$ expressible as a
complex linear
combination of tight totally finite Borel measures;  explain how this
may be identified,
as Banach space, with the complexification of the $L$-space $M_t$
of signed tight Borel measures
as described in 354Yl.   Show that the complex Banach space
$\eusm L^{\infty}_{\Bbb C}(\Cal B(X))$ is canonically embedded in
$C(X;\Bbb C)^{**}$.
%437I

\spheader 437Yc Write $\mu_c$ for counting measure on $[0,1]$, and
$\mu_L$ for Lebesgue measure;  write $\mu_c\times\mu_L$ for the product
measure on $[0,1]^2$, and $\mu$ for the direct sum of $\mu_c$ and
$\mu_c\times\mu_L$.   Show that the $L$-space $C([0,1])^{\sim}$ is
isomorphic, as $L$-space, to $L^1(\mu)$.   \Hint{every Radon measure on
$[0,1]$ has countable Maharam type.}
%437I

\spheader 437Yd Let $X$ be a set, $U$ a Riesz subspace of
$\ell^{\infty}(X)$ containing the constant functions, and $\Sigma$ the
smallest $\sigma$-algebra of subsets of $X$ with respect to which every
member of $U$ is measurable.   Write $\tilde\Sigma$ for the intersection
of the domains of the completions of the totally finite measures with
domain $\Sigma$.   Show that there is a unique sequentially
order-continuous
norm-preserving Riesz homomorphism from $\eusm L^{\infty}(\tilde\Sigma)$
to $(U^{\sim}_{\sigma})^*\cong M_{\sigma}^*$ such that $(Su)(f)=f(u)$
whenever $u\in U$ and $f\in U^{\sim}_{\sigma}$.
%437I 437Xf

\spheader 437Ye\dvAnew{2010}
Show that in 437Ib the operator
$S:\eusm L^{\infty}(\Sigma_{\text{uRm}})\to C_0(X)^{**}$ is
multiplicative if $C_0(X)^{**}$ is given the Arens multiplication
described in 4A6O based on the ordinary multiplication
$(u,v)\mapsto u\times v$ on $C_0(X)$.
%437I out of order query

\spheader 437Yf Explain how to express the proof of
285L(iii)$\Rightarrow$(ii) as ($\alpha$) a proof that if the
characteristic functions of a sequence $\sequencen{\nu_n}$ of Radon
probability measures on $\BbbR^r$ converge pointwise to a characteristic
function, then $\{\nu_n:n\in\Bbb N\}$ is uniformly tight ($\beta$) the
observation that any subalgebra of $C_b(\BbbR^r)$ which separates the
points of $\BbbR^r$ and contains the constant functions will define the
vague topology on any vaguely compact set of measures.
%437J, 437Yf

\spheader 437Yg Let $X$ be a topological space.   (i)
Let $\Cal Ba$ be the Baire $\sigma$-algebra of $X$,
$M_{\sigma}(\Cal Ba)$ the space of signed Baire measures on
$X$, and $u:X\to\Bbb R$ a bounded Baire measurable function.   Show that
we have a linear functional
$\mu\mapsto\dashint u\,d\mu:M_{\sigma}(\CalBa)\to\Bbb R$ agreeing with
ordinary integration with respect to non-negative
measures.  Show that this functional is Baire
measurable with respect to the vague topology on $M_{\sigma}(\CalBa)$.
(ii) Let $M_{\tau}$ be the space of signed $\tau$-additive Borel
measures on $X$, and $u:X\to\Bbb R$ a bounded Borel measurable function.
Show that
we have a linear functional $\mu\mapsto\dashint u\,d\mu:M_{\tau}\to\Bbb R$
agreeing with ordinary integration with respect to non-negative
measures.  Show that this functional is Borel
measurable with respect to the vague topology on $M_{\tau}$.
%437J

\spheader 437Yh\dvAnew{2009}
Let $X$ be a topological space, $\mu_0$ a totally finite
$\tau$-additive topological measure on $X$, and $u:X\to\Bbb R$ a bounded
function which is continuous $\mu_0$-a.e.   Let $\tilde M_{\sigma}^+$
be the set of totally finite topological measures on $X$, with its narrow
topology.   Show that
$\nu\mapsto\overline{\int}u\,d\nu:\tilde M^+_{\sigma}\to\Bbb R$ is
continuous at $\mu_0$.
%437J

\spheader 437Yi Let $X$ be a Hausdorff space, and
$M^{\infty+}_{\text{R}}$ the set of all Radon measures on $X$.   Define
addition and scalar multiplication (by positive scalars) on
$M^{\infty+}_{\text{R}}$ as in 234G and
234Xf, and $\le$ by the formulae of 234P.
(i) Show that there is a Dedekind complete Riesz space $V$ such
that the positive cone of $V$ is isomorphic to
$M^{\infty+}_{\text{R}}$.   (ii) Show that every principal band in $V$
is an $L$-space.   (iii) Show that if $X$ is first-countable then $V$ is
perfect.
%437Ji

\spheader 437Yj For a topological space $X$ let $M_{\tau}(X)$ be the
$L$-space of signed $\tau$-additive Borel measures on $X$, and
$\psi:M_{\tau}(X)\times M_{\tau}(X)\to M_{\tau}(X\times X)$ the
canonical bilinear operator (437M);  give $M_{\tau}(X)$ and
$M_{\tau}(X\times X)$ their vague topologies.   (i) Show that if
$X=[0,1]$ then $\psi$ is not continuous.   (ii) Show that $X=\Bbb N$ and
$B$ is the unit ball of $M_{\tau}(X)$ then $\psi\restr B\times B$ is not
continuous.
%437M

\spheader 437Yk Let $X$ and $Y$ be topological spaces, and
$\psi:M_{\tau}(X)\times M_{\tau}(Y)\to M_{\tau}(X\times Y)$ the bilinear
map of 437Mc.   Write $M_t(X)$, etc., for the spaces of signed tight
Borel measures.   (i) Show that $\psi(\mu,\nu)\in M_t(X\times Y)$ for
every $\mu\in M_t(X)$, $\nu\in M_t(Y)$.   (ii) Show that if
$B\subseteq M_t(X)$, $B'\subseteq M_t(Y)$ are norm-bounded and uniformly
tight, then $\psi\restr B\times B'$ is continuous for the vague
topologies.
%437O 437M

\spheader 437Yl\dvAnew{2009} Let $X$ be a topological space, and
$\tilde M$ the
space of bounded additive functionals defined on subalgebras of $\Cal PX$
containing every open set.   For $\nu\in\tilde M$, say that
$|\nu|(E)=\sup\{\nu F-\nu(E\setminus F):F\in\dom\nu$, $F\subseteq E\}$ for
$E\in\dom\nu$.   Show that a set $A\subseteq\tilde M$ is uniformly tight
in the sense of 437O iff every member of $A$ is tight and
$\{|\nu|:\nu\in A\}$ is uniformly tight.
%437O

\spheader 437Ym Let $X$ be a completely regular space and
$\PqR$ the space of quasi-Radon probability measures on $X$.   Let
$B\subseteq\PqR$ be a non-empty set.   Show that the following are
equiveridical:  (i) $B$ is relatively compact in $\PqR$ for the
narrow topology;  (ii) whenever $A\subseteq C_b(X)$ is non-empty and
downwards-directed and $\inf_{u\in A}u(x)=0$ for every $x\in A$, then
$\inf_{u\in A}\sup_{\mu\in B}\int u\,d\mu=0$;  (iii) whenever $\Cal G$
is an upwards-directed family of open sets with union $X$, then
$\sup_{G\in\Cal G}\inf_{\mu\in B}\mu G=1$.
%437P

\spheader 437Yn\dvAnew{2010} (i) Let $X$ be a regular topological space,
and
$\MqR$ the space of totally finite quasi-Radon measures on $X$, with its
narrow topology.    Show that $\MqR$ is regular.   (ii)
Find a second-countable Hausdorff space $X$ such that the space
$P_{\text{qR}}$ of quasi-Radon probability measures on $X$ is not
Hausdorff in its narrow topology.
%437J 437Ra

\spheader 437Yo\dvAnew{2009}
Let $(X,\rho)$ and $(Y,\sigma)$ be metric
spaces, and give $\MqR(X)$ and $\MqR(Y)$ the corresponding metrics
$\rhoKR$, $\sigmaKR$ as in 437Rg.   For a continuous function
$\phi:X\to Y$, let $\tilde\phi:\MqR(X)\to\MqR(Y)$ be the map
described in 437Xm.   (i) Show that if $\phi$ is $\gamma$-Lipschitz,
where $\gamma\ge 1$, then $\tilde\phi$ is $\gamma$-Lipschitz.   (ii)
(J.Pachl) Show
that if $\phi$ is uniformly continuous, then $\tilde\phi$ is uniformly
continuous on any uniformly totally finite subset of $\MqR(X)$.
(iii) Show that if $(X,\rho)$ is $\Bbb R$ with its usual
metric, then $\rhoKR$ is not uniformly equivalent to L\'evy's metric as
described in 274Yc\formerly{2{}74Ya}.
(For a discussion of various metrics related to
$\rhoKR$, see {\smc Bogachev 07}, 8.10.43-8.10.48.)
%437Rg 437Xm 43bits

\spheader 437Yp\dvAnew{2009}
Let $(X,\rho)$ be a metric space.   For $f\in C_b(X)^{\sim}_{\sigma}$, set
$\|f\|_{\text{KR}}
=\sup\{|f(u)|:u\in C_b(X),\,\|u\|_{\infty}\le 1,\,u$ is $1$-Lipschitz$\}$.
(i) Show that $\|\,\|_{\text{KR}}$ is a
norm on $C_b(X)^{\sim}_{\sigma}$.
%Bogachev \S8.3 calls it `Kantorovich-Rubinstein norm'
(ii) Let $(X',\rho')$ and $(X'',\rho'')$ be metric spaces, and $\rho$ the
$\ell^1$-product metric on $X=X'\times X''$ defined by saying that
$\rho((x',x''),(y',y''))=\rho'(x',y')+\rho''(x'',y'')$.   Identifying the
spaces
$M_{\tau}(X')$, $M_{\tau}(X'')$ and $M_{\tau}(X)$ of signed $\tau$-additive
Borel measures with subspaces of
$C_b(X')^{\sim}_{\sigma}$, $C_b(X'')^{\sim}_{\sigma}$ and
$C_b(X')^{\sim}_{\sigma}$, as in 437E-437H, show that the bilinear map
$\psi:M_{\tau}(X')\times M_{\tau}(X'')\to M_{\tau}(X)$ described in
437Mc has norm $1$ when $M_{\tau}(X')$, $M_{\tau}(X'')$ and $M_{\tau}(X)$
are given the appropriate norms $\|\,\|_{\text{KR}}$.
%437Rg 437Yo 437E 437H 437M

\spheader 437Yq\dvAnew{2009}
Let $X$ be a topological space and $\tilde M^+$
the set of non-negative real-valued additive functionals defined on
algebras of subsets of $X$ containing every open set, endowed with its
narrow topology.   Show that the weight $w(\tilde M^+)$ of
$\tilde M^+$ is at most $\max(\omega,w(X))$.
%437Rb

\spheader 437Yr Let $X$ be a \v{C}ech-complete completely regular
Hausdorff space
and $P_{\text{R}}$ the set of Radon probability measures on $X$, with
its narrow topology.
Show that $P_{\text{R}}$ is \v{C}ech-complete.
%437Rh

\spheader 437Ys\dvAnew{2008}
Let $\familyiI{(\frak A_i,\bar\mu_i)}$ be a non-empty family of probability
algebras, and $\Cal F$ an ultrafilter on $I$.   Let
$(\frak A,\bar\mu)=\prod_{i\in I}(\frak A_i,\bar\mu_i)|\Cal F$
be the reduced product as defined in 328C.   For each $i\in I$, let
$(Z_i,\nu_i)$ be the Stone space of $(\frak A_i,\bar\mu_i)$;
give $W=\{(z,i):i\in I$, $z\in Z_i\}$ its disjoint union topology, and let
$\beta W$ be the Stone-\v{C}ech compactification of $W$.   For each
$i\in I$, define $\phi_i:Z_i\to W\subseteq\beta W$ by setting
$\phi_i(z)=(z,i)$ for $z\in Z_i$, and let $\nu_i\phi_i^{-1}$
be the image measure on $\beta W$.
Let $\nu$ be the limit $\lim_{i\to\Cal F}\nu_i\phi_i^{-1}$ for the narrow
topology on the space of Radon probability measures on $\beta W$,
and $Z$ its support.   Show that
$(Z,\nu)$ can be identified with the Stone space of $(\frak A,\bar\mu)$.
%437R

\spheader 437Yt\dvAnew{2009} (i) Show that there are a continuous
$\phi:\{0,1\}^{\Bbb N}\to[0,1]$ and a positive linear operator
$T:C(\{0,1\}^{\Bbb N})\to C([0,1])$ such that $T(u\phi)=u$ for every
$u\in C([0,1])$.   \Hint{if
$I_{\sigma}=\{x:\sigma\subseteq x\in\{0,1\}^{\Bbb N}$ for
$\sigma\in\bigcup_{n\in\Bbb N}\{0,1\}^n$, arrange
that $\{t:T(\chi I_{\sigma})(t)>0\}$ is always an interval of length
$(\bover23)^{\#(\sigma)}$.}   (ii) Show that there are a continuous
$\tilde\phi:(\{0,1\}^{\Bbb N})^{\Bbb N}\to[0,1]^{\Bbb N}$ and a positive
linear operator
$\tilde T:C((\{0,1\}^{\Bbb N})^{\Bbb N})\to C([0,1]^{\Bbb N})$ such that
$\tilde T(h\tilde\phi)=h$ for every
$h\in C([0,1]^{\Bbb N})$.   \Hint{if, in (i), $(Tg)(t)=\int g\,d\nu_t$ for
$t\in[0,1]$ and $g\in C(\{0,1\}^{\Bbb N})$, take $\nu_{\pmb{t}}$ to be the
product measure $\prod_{n\in\Bbb N}\nu_{t_n}$ for
$\pmb{t}=\sequencen{t_n}\in[0,1]^{\Bbb N}$.}
%for 437Yu (437R)

\spheader 437Yu\dvAnew{2009} Let $X$ be a separable metrizable space
and $P=\PR(X)$ the set of Radon probability measures on $X$, with its
narrow topology.
Show that there is a family $\family{\mu}{P}{f_{\mu}}$ of
functions from
$[0,1]$ to $X$ such that (i) $(\mu,t)\mapsto f_{\mu}(t)$ is Borel
measurable (ii) writing $\mu_L$ for Lebesgue measure on
$[0,1]$, $\mu=\mu_Lf_{\mu}^{-1}$ for every $\mu\in P$ (iii) whenever
$\sequencen{\mu_n}$ is a sequence in $P$ converging
to $\mu\in P$, there is a countable set $A\subseteq[0,1]$ such that
$f_{\mu}(t)=\lim_{n\to\infty}f_{\mu_n}(t)$ for every
$t\in[0,1]\setminus A$.   \Hint{first consider
the cases $X=[0,1]$ and $X=\{0,1\}^{\Bbb N}$,
then use 437Yt to deal with $[0,1]^{\Bbb N}$ and its subspaces.
See {\smc Bogachev 07}, \S8.5.}
%437Yt 437R+

\spheader 437Yv\dvAnew{2012} Let $(\frak A,\bar\mu)$ be a measure algebra, and
$\frak A^f$ the ideal of elements of $\frak A$ of finite measure.   For
$a\in\frak A^f$ and $u\in L^0=L^0(\frak A)$, let $\nu_{au}$ be the totally
finite Radon measure on $\Bbb R$ defined by saying that
$\nu_{au}(E)=\bar\mu(a\Bcap\Bvalue{u\in E})$ (definition:  364G, 434T) for Borel
sets $E\subseteq\Bbb R$.   For $a\in\frak A^f$ and $u$, $v\in L^0$ set
$\bar\rho_a(u,v)=\rho_{KR}(\nu_{au},\nu_{av})$, where $\rho_{KR}$ is the metric
on $\MR=\MR(\Bbb R)$ defined from the usual metric on $\Bbb R$.   (i) Show that
the family $\Rho=\{\bar\rho_a:a\in\frak A^f\}$ of pseudometrics
defines the topology of convergence in measure on $L^0$ (definition:  367L).
(ii) Show that if $(\frak A,\bar\mu)$ is
semi-finite then the uniformity $\Cal U$
defined from $\Rho$ is metrizable iff $(\frak A,\bar\mu)$ is $\sigma$-finite and
$\frak A$ has countable Maharam type.   (iii) Show that if $(\frak A,\bar\mu)$
is semi-finite then $L^0$ is complete under $\Cal U$ (definition:  3A4F)
iff $\frak A$ is purely
atomic.   \Hint{if $(\frak A,\bar\mu)$ is an atomless probability algebra and
$\sequencen{c_n}$ is an independent sequence of elements of $\frak A$ of measure
$\bover12$, show that $\sequencen{\nu_{a,\chi c_n}}$ is convergent in
$\MR$ for every $a\in\frak A$, so $\sequencen{\chi c_n}$ is $\Cal U$-Cauchy.}
%437R
%\query check Dudley

\spheader 437Yw\dvAnew{2009} Let $X$ be a non-empty compact metrizable
space, and $\phi:X\to X$ a continuous function.   Show that there is an
$x\in X$ such that $\lim_{n\to\infty}\Bover1{n+1}\sum_{i=0}^nu(\phi^i(x))$
is defined for every $u\in C(X)$.   \Hint{372H\formerly{3{}72I}, 4A2Pe.}
%437T;  372H Ergodic Theorem, 4A2Pe C(X) separable
%fails if X the Stone space of Lebesgue prob algebra

\spheader 437Yx(i) Let $X$ be a metrizable space, and $A$ a narrowly
compact subset of the set of Radon probability measures on $X$.   Show
that there is a separable subset
$Y$ of $X$ which is conegligible for every measure in $A$.   (ii) Show
that a metrizable space is Prokhorov iff all its closed separable
subspaces are Prokhorov.
%437V

\spheader 437Yy I say that a completely regular Hausdorff space
$X$ is {\bf strongly Prokhorov}
if every vaguely compact subset of the space $M_t(X)$ of signed tight
Borel measures
on $X$ is uniformly tight.   (i) Check that a strongly Prokhorov
completely regular Hausdorff space is Prokhorov.  (ii) Show that a
closed subspace of a strongly Prokhorov
completely regular Hausdorff space is strongly Prokhorov.   (iii) Show
that the product of a countable family of strongly Prokhorov completely
regular Hausdorff spaces is
strongly Prokhorov.   (iv) Show that a G$_{\delta}$ subset of a strongly
Prokhorov metrizable space is strongly Prokhorov.   (v) Show that if
$(X,\rho)$ is a complete metric space then $X$ is strongly Prokhorov.
%437V 437Xx mt43bits

%query: is every metrizable Prokhorov space strongly Prokhorov?

\spheader 437Yz\dvAnew{2010}
Let $X$ be a regular Hausdorff topological space and $C$ a non-empty
narrowly compact
set of totally finite topological measures on $X$, all inner regular with
respect to the closed sets.   Set $c(A)=\sup_{\mu\in C}\mu^*A$ for
$A\subseteq X$.   Show that $c:\Cal PX\to\coint{0,\infty}$
is a Choquet capacity.
%437V
}%end of exercises

\endnotes{\Notesheader{437}
The ramifications of the results here are enormous.   For completely
regular topological spaces $X$, the theorems of \S436 give effective
descriptions of the totally finite Baire, quasi-Radon and Radon measures
on $X$ as linear functionals on $C_b(X)$ (436E, 436Xl, 436Xn).   This
makes it possible, and natural, to integrate the topological measure
theory of $X$ into functional analysis, through the theory of
$C_b(X)^*$.   (See {\smc Wheeler 83} for an extensive discussion of this
approach.)   For the rest of this volume we shall never be far away from
such considerations.   In 437C-437I I give only a sample of the results,
heavily slanted towards the abstract theory of Riesz spaces in Chapter
35 and the first part of Chapter 36.

Note that while the constructions of the dual spaces $U^{\sim}$,
$U^{\sim}_c$ and $U^{\times}$ are `intrinsic' to a Riesz space $U$, in
that we can identify these functions as soon as we know the linear and
order structure of $U$, the spaces
$U^{\sim}_{\sigma}$ and $U^{\sim}_{\tau}$ are
definable only when $U$ is presented as a Riesz subspace of $\Bbb R^X$.
In the same way, while the space $M_{\sigma}(\Sigma)$ of countably
additive
functionals on a $\sigma$-algebra $\Sigma$ depends only on the Boolean
algebra structure, the spaces $M_{\tau}$ here (not to be confused with
the space of completely additive functionals considered in 362B) depend
on the topology as well as the Borel algebra.   (For an example in which
radically different topologies give rise to the same Borel algebra, see
{\smc Juh\'asz Kunen \& Rudin 76}.)

You may have been puzzled by the shift from `quasi-Radon' measures in
436H to `$\tau$-additive' measures in 437H;  somewhere the requirement
of inner regularity has got lost.   The point is that the topologies
being considered here, being defined by declaring certain families of
functions continuous, are (completely) regular;  so that $\tau$-additive
measures are necessarily inner regular with respect to the closed sets
(414Mb).

The theory of `vague' and `narrow' topologies in 437J-437V here hardly
impinges on
the questions considered in \S\S274 and 285, where vague topologies
first appeared.   This is because the earlier investigation was
dominated by the very special position of the functions
$x\mapsto e^{iy\dotproduct x}$ (what we shall in \S445 come to call the
`characters' of the additive groups of $\Bbb R$ or $\BbbR^r$).   One
idea which does appear essentially in the proof of 285L, and has a
natural interpretation in the general theory, is that of a `uniformly
tight' family of Radon measures (437O).   In 445Yh below I
set out a generalization of 285L to abelian locally compact groups.

In \S461 I will return to the general theory of extreme points in
compact convex sets.   Here I remark only that it is never surprising
that extreme points should be
special in some way, as in 437S and 461Q-461R;
but the precise ways in which they are special are often unexpected.
A good deal of work has been done on relationships between the
topological properties of a topological space $X$ and the space
$P_{\text{R}}$ of Radon probability measures on $X$ with the
narrow topology.   Here I give only a sample of basic facts in 437R
and 437Yq-437Yr.   Having observed that $\MqR(X)$ is metrizable
whenever $X$ is (437Rg), it is natural to seek ways of defining a metric on
$\MqR(X)$ from a metric on $X$.   The Kantorovich-Rubinstein metric
$\rhoKR$ I have
chosen here is only one of many possibilities;  compare the L\'evy metric
of 274Yc and the metric $\rho_{\text{W}}$ of 457K below.   Note that it can
make a difference whether we look at quasi-Radon (or $\tau$-additive)
probability measures, or at general Borel measures (438Yl).

The terms `vague' and `narrow' both appear in the literature on this
topic, and I take the opportunity to use them both, meaning slightly
different things.   Vague topologies,
in my usage, are linear space topologies on linear spaces of
functionals;  narrow topologies are topologies on spaces of (finitely
additive) measures, which are not linear spaces, though we can
in some cases define addition and multiplication by non-negative
scalars (437N, 437Xr, 437Yi).
I must warn you that this distinction is not standard.   I see that the
word `narrow' appears above
a good deal oftener than the word `vague', which
is in part a reflection of
a simple prejudice against signed measures;  but from the point of view
of this treatise as a whole, it is more natural to work with a concept
well adapted to measures with
variable domains, even if we are considering questions (like compactness
of sets of measures) which originate in linear analysis.
I should mention also that the definition in 437Jc includes a choice.
The duality considered there uses the space $C_b(X)$;  for locally
compact $X$, we have the rival spaces $C_0(X)$ and $C_k(X)$ (see 436J
and 436K), and there are occasions when one of these gives a more
suitable topology on a space of measures (as in 495Xl below).

The elementary theory of uniform tightness and Prokhorov spaces
(437O-437V) %437O 437P 437U 437V
is both pretty and useful.   The emphasis I give it here, however, is
partly because it provides the background to a remarkable construction
by D.Preiss (439S below), showing that $\Bbb Q$ is not a Prokhorov
space.
}%end of notes

\discrpage

