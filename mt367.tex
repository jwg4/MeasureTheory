\frfilename{mt367.tex}
\versiondate{18.9.03}
\copyrightdate{1998}
\long\def\varinset#1{{\narrower{#1}}}
\def\starto{\to^*}
     
\def\chaptername{Function spaces}
\def\sectionname{Convergence in measure}
     
\newsection{367}
     
Continuing through the ideas of Chapter 24, I come to `convergence in
measure'.   The basic results of \S245
all translate easily into the new language (367L-367M, 367P).   The
associated concept of (sequential) order-convergence can also be
expressed in abstract terms (367A), and I take the trouble to do this in
the context of general lattices (367A-367B), since the
concept can be applied in many ways (367C-367E,
367K\cmmnt{, 367Xa-367Xn}).   In
the particular case of $L^0$ spaces, which are the first aim of this
section, the idea is most naturally expressed by 367F.   It enables us
to express some of the fundamental theorems from Volumes 1 and 2 in the
language of this chapter (367I-367J).
     
In 367N and 367O I give two of the most characteristic properties of the
topology of convergence in measure on $L^0$;  it is one of the
fundamental types of topological Riesz space.   Another striking fact is
the way it is determined by the Riesz space structure (367T).   In 367U
I set out a theorem which is the basis of many remarkable applications
of the concept;  for the sake of a result in \S369 I give one such
application (367V).
     
\leader{367A}{\dvrocolon{Order*-convergence}}\cmmnt{ As I have
remarked before, the function spaces of measure theory have three
interdependent structures:  they are linear spaces, they have a variety
of interesting topologies, and they are ordered spaces.   Ordinary
elementary functional analysis studies interactions between topologies
and linear structures, in the theory of normed spaces and, more
generally, of linear topological spaces.   Chapter 35 in this volume
looked at interactions between linear and order structures.   It is
natural to seek to complete the triangle with a theory of topological
ordered spaces.   The relative obscurity of any such theory is in part
due to the difficulty of finding convincing definitions;  that is,
isolating concepts which lead to elegant and useful general
theorems.   Among the many rival ideas, however, I believe it is
possible to identify one which is particularly important in the context
of measure theory.
     
In its natural home in the theory of $L^0$ spaces, this notion of
`order*-convergence' has a very straightforward expression
(367F).   But, suitably interpreted, the same idea can be applied
in other contexts, some of which will be very useful to us, and I
therefore begin with a definition which is applicable in any lattice.
     
\medskip
     
\noindent}{\bf Definition} Let $P$ be a lattice, $p$ an element of $P$ and $\sequencen{p_n}$ a sequence in $P$.   I will say that
$\sequencen{p_n}$ {\bf order*-converges} to $p$, or that $p$ is the
{\bf order*-limit} of $\sequencen{p_n}$, if
     
$$\eqalign{p
&=\inf\{q:\,\exists\,n\in\Bbb N,\,q\ge(p'\vee p_i)\wedge p''
\Forall i\ge n\}\cr
&=\sup\{q:\,\exists\,n\in\Bbb N,\,q\le p'\vee(p_i\wedge p'')
\Forall i\ge n\}\cr}$$
     
\noindent whenever $p'\le p\le p''$ in $P$.
     
\vleader{108pt}{367B}{Lemma} Let $P$ be a lattice.
     
(a) A sequence in $P$ can order*-converge to at most one point.
     
(b) A constant sequence order*-converges to its constant value.
     
(c) Any subsequence of an order*-convergent sequence is
order*-convergent, with the same limit.
     
(d) If $\sequencen{p_n}$ and $\sequencen{p'_n}$ both order*-converge
to $p$, and $p_n\le q_n\le p'_n$ for every $n$, then $\sequencen{q_n}$
order*-converges to $p$.
     
(e) If $\sequencen{p_n}$ is an order-bounded sequence in $P$, then it
order*-converges to $p\in P$ iff
     
$$\eqalign{p
&=\inf\{q:\,\exists\,n\in\Bbb N,\,q\ge p_i\Forall i\ge n\}\cr
&=\sup\{q:\,\exists\,n\in\Bbb N,\,q\le p_i\Forall i\ge n\}.\cr}$$
     
(f) If $P$ is a Dedekind $\sigma$-complete lattice\cmmnt{ (314Ab)} and
$\sequencen{p_n}$ is an order-bounded sequence in $P$, then it
order*-converges to $p\in P$ iff
     
\Centerline{$p=\sup_{n\in\Bbb N}\inf_{i\ge n}p_i
=\inf_{n\in\Bbb N}\sup_{i\ge n}p_i$.}
     
\proof{{\bf (a)} Suppose that $\sequencen{p_n}$ is order*-convergent
to both $p$ and $\tilde p$.   Set $p'=p\wedge\tilde p$, $p''=p\vee\tilde
p$;  then
     
\Centerline{$p
=\inf\{q:\,\exists\,n\in\Bbb N,\,q\ge(p'\vee p_i)\wedge
p''\Forall i\ge n\}
=\tilde p$.}
     
\medskip
     
{\bf (b)} is trivial.
     
\medskip
     
{\bf (c)} Suppose that $\sequencen{p_n}$ is order*-convergent to $p$,
and that $\sequencen{p'_n}$ is a subsequence of $\sequencen{p_n}$.
Take $p'$, $p''$ such that $p'\le p\le p''$, and set
     
\Centerline{$B=\{q:\,\exists\,n\in\Bbb N,\,q\le p'\vee(p_i\wedge
p'')\Forall i\ge n\}$,}
     
\Centerline{$B'=\{q:\,\exists\,n\in\Bbb N,\,q\le p'\vee(p'_i\wedge
p'')\Forall i\ge n\}$,}
     
\Centerline{$C=\{q:\,\exists\,n\in\Bbb N,\,q\ge(p'\vee p_i)\wedge
p''\Forall i\ge n\}$,}
     
\Centerline{$C'=\{q:\,\exists\,n\in\Bbb N,\,q\ge(p'\vee p'_i)\wedge
p''\Forall i\ge n\}$.}
     
\noindent If $q\in B'$ and $q'\in C$, then for all sufficiently large
$i$
     
\Centerline{$q\le p'\vee(p'_i\wedge p'')\le(p'\vee p'_i)\wedge p''\le
q'$.}
     
\noindent As $p=\inf C$, we must have $q\le p$;  thus $p$ is an upper
bound for $B'$.   On the other hand,
$\{p'_i:i\ge n\}\subseteq\{p_i:i\ge n\}$ for every $n$, so $B\subseteq
B'$ and $p$ must be the least upper bound of $B'$, since $p=\sup B$.
     
Similarly, $p=\inf C'$.   As $p'$ and $p''$ are arbitrary,
$\sequencen{p'_n}$ order*-converges to $p$.
     
\medskip
     
{\bf (d)} Take $p'$, $p''$ such that $p'\le p\le p''$, and set
     
\Centerline{$B=\{q:\,\exists\,n\in\Bbb N,\,q\le p'\vee(p_i\wedge
p'')\Forall i\ge n\}$,}
     
\Centerline{$B'=\{q:\,\exists\,n\in\Bbb N,\,q\le p'\vee(q_i\wedge
p'')\Forall i\ge n\}$,}
     
\Centerline{$C=\{q:\,\exists\,n\in\Bbb N,\,q\ge(p'\vee p'_i)\wedge
p''\Forall i\ge n\}$,}
     
\Centerline{$C'=\{q:\,\exists\,n\in\Bbb N,\,q\ge(p'\vee q_i)\wedge
p''\Forall i\ge n\}$.}
     
\noindent If $q\in B'$ and $q'\in C$, then for all sufficiently large
$i$
     
\Centerline{$q\le p'\vee(q_i\wedge p'')\le(p'\vee p'_i)\wedge p''\le
q'$.}
     
\noindent As $p=\inf C$, we must have $q\le p$;  thus $p$ is an upper
bound for $B'$.   On the other hand,
$p'\vee(p_i\wedge p'')\le p'\vee(q_i\wedge p'')$ for every $i$, so
$B\subseteq B'$ and $p=\sup B'$.
Similarly, $p=\inf C'$.   As $p'$ and $p''$ are arbitrary,
$\sequencen{q_n}$ order*-converges to $p$.
     
\medskip
     
{\bf (e)} Set
     
\Centerline{$B=\{q:\,\exists\,n\in\Bbb N,\,q\le p_i\Forall i\ge
n\}$,}
     
\Centerline{$C=\{q:\,\exists\,n\in\Bbb N,\,q\ge p_i\Forall i\ge
n\}$.}
     
\medskip
     
\quad{\bf (i)} Suppose that $\sequencen{p_n}$ order*-converges to
$p$.   Let $p'$, $p''$ be such that $p'\le p_n\le p''$ for every
$n\in\Bbb N$ and $p'\le p\le p''$.   Then
     
\Centerline{$B
=\{q:\,\exists\,n\in\Bbb N,\,q\le p'\vee(p_i\wedge p'')\Forall i\ge
n\}$,}
     
\noindent so $\sup B=p$.   Similarly, $\inf C=p$, so the condition is
satisfied.
     
\medskip
     
\quad{\bf (ii)} Suppose that $\sup B=\inf C=p$.   Take any $p'$, $p''$
such that $p'\le p\le p''$ and set
     
\Centerline{$B'=\{q:\,\exists\,n\in\Bbb N,\,q\le p'\vee(p_i\wedge
p'')\Forall i\ge n\}$,}
     
\Centerline{$C'=\{q:\,\exists\,n\in\Bbb N,\,q\ge(p'\vee p_i)\wedge
p''\Forall i\ge n\}$.}
     
\noindent If $q\in B'$ and $q'\in C$, then for all large enough $i$
     
\Centerline{$q\le p'\vee(p_i\wedge p'')\le p'\vee q'=q'$}
     
\noindent because $p\le q'$.   As $\inf C=p$, $p$ is an upper bound for
$B'$.   On the other hand,
if $q\in B$, then $q\le p$, so $q\le p'\vee(p_i\wedge p'')$ whenever
$q\le p_i$, which is so for all sufficiently large $i$, and $q\in B'$.
Thus $B'\supseteq B$ and $p$ must be the supremum of $B'$.   Similarly,
$p=\inf C'$;  as $p'$ and $p''$ are arbitrary, $\sequencen{p_n}$
order*-converges to $p$.
     
\medskip
     
{\bf (f)} This follows at once from (e).   Setting
     
\Centerline{$B
=\{q:\,\exists\,n\in\Bbb N,\,q\le p_i\Forall i\ge n\}$,
\quad$B'=\{\inf_{i\ge n}p_i:i\in\Bbb N\}$,}
     
\noindent then $B'\subseteq B$ and for every $q\in B$ there is a $q'\in
B'$ such that $q\le q'$;  so $\sup B=\sup B'$ if either is defined.
Similarly,
     
\Centerline{$\inf\{q:\,\exists\,n\in\Bbb N,\,q\ge p_i\Forall i\ge n\}
=\inf_{n\in\Bbb N}\sup_{i\ge n}p_i$}
     
\noindent if either is defined.
}%end of proof of 367B
     
\leader{367C}{Proposition} Let $U$ be a Riesz space and
$\sequencen{u_n}$, $\sequencen{v_n}$ two sequences in $U$
order*-converging to $u$, $v$ respectively.
     
(a) If $w\in U$, $\sequencen{u_n+w}$
order*-converges to $u+w$, and $\alpha u_n$ order*-converges to
$\alpha u$ for every $\alpha\in\Bbb R$.
     
(b) $\sequencen{u_n\vee v_n}$ order*-converges to $u\vee v$
and $\sequencen{u_n\wedge v_n}$ order*-converges to $u\wedge v$.
     
(c) If $\sequencen{w_n}$ is any sequence in $U$, then it
order*-converges to $w\in U$ iff $\sequencen{|w_n-w|}$
order*-converges to $0$.
     
(d) $\sequencen{u_n+v_n}$ order*-converges to $u+v$.
     
(e) If $U$ is Archimedean, and $\sequencen{\alpha_n}$ is a sequence in
$\Bbb R$ converging to
$\alpha\in\Bbb R$, then $\sequencen{\alpha_nu_n}$
order*-converges to $\alpha u$.
     
(f) Again suppose that $U$ is Archimedean.   Then a sequence
$\sequencen{w_n}$ in $U^+$ is {\it not} order*-convergent to $0$ iff
there is a $\tilde w>0$ such that 
$\tilde w=\sup_{i\ge n}\tilde w\wedge w_i$ for every $n\in\Bbb N$.
     
\proof{{\bf (a)(i)} $\sequencen{u_n+w}$ order*-converges to $u+w$
because the ordering of $U$ is translation-invariant;  the map
$w'\mapsto w'+w$ is an order-isomorphism.
     
\medskip
     
\quad{\bf (ii)}\grheada\ If $\alpha>0$, then the map 
$w'\mapsto\alpha w'$ is an order-isomorphism, so $\sequencen{\alpha u_n}$
order*-converges to $\alpha u$.
     
\medskip
     
\qquad\grheadb\ If $\alpha=0$ then $\sequencen{\alpha u_n}$
order*-converges to $\alpha u=0$ by 367Bb.
     
\medskip
     
\qquad\grheadc\ If $w'\le -u\le w''$ then $-w''\le u\le -w'$ so
     
$$\eqalign{u
&=\inf\{w:\,\exists\,n\in\Bbb N,\,
w\ge((-w'')\vee u_i)\wedge (-w')\Forall i\ge n\}\cr
&=\sup\{w:\,\exists\,n\in\Bbb N,\,
w\le (-w'')\vee(u_i\wedge (-w'))\Forall i\ge n\}.\cr}$$
     
\noindent Turning these formulae upside down,
     
$$\eqalign{-u
&=\sup\{w:\,\exists\,n\in\Bbb N,\,
w\le(w''\wedge(-u_i))\vee w'\Forall i\ge n\}\cr
&=\inf\{w:\,\exists\,n\in\Bbb N,\,
w\ge w''\wedge((-u_i)\vee w')\Forall i\ge n\}.\cr}$$
     
\noindent As $w'$ and $w''$ are arbitrary, $\sequencen{-u_n}$
order*-converges to $-u$.
     
\medskip
     
\qquad\grheadd\ Putting ($\alpha$) and ($\gamma$) together,
$\sequencen{\alpha u_n}$ order*-converges to $\alpha u$ for every
$\alpha<0$.
     
\medskip
     
{\bf (b)} Suppose that $w'\le u\vee v\le w''$.   Set
     
\Centerline{$B=\{w:\,\exists\,n\in\Bbb N,\,w\le w'\vee((u_i\vee
v_i)\wedge w'')\Forall i\ge n\}$,}
     
\Centerline{$C=\{w:\,\exists\,n\in\Bbb N,\,w\ge(w'\vee(u_i\vee
v_i))\wedge w''\Forall i\ge n\}$,}
     
\Centerline{$B_1=\{w:\,\exists\,n\in\Bbb N,\,w\le (w'\wedge
u)\vee(u_i\wedge w'')\Forall i\ge n\}$,}
     
\Centerline{$B_2=\{w:\,\exists\,n\in\Bbb N,\,w\le (w'\wedge
v)\vee(v_i\wedge w'')\Forall i\ge n\}$,}
     
\Centerline{$C_1=\{w:\,\exists\,n\in\Bbb N,\,w\ge((w'\wedge u)\vee
u_i)\wedge w''\Forall i\ge n\}$,}
     
\Centerline{$C_2=\{w:\,\exists\,n\in\Bbb N,\,w\ge((w'\wedge v)\vee
v_i)\wedge w''\Forall i\ge n\}$,}
     
If $w_1\in B_1$ and $w_2\in B_2$ then $w_1\vee w_2\in B$.   \Prf\ There
is an $n\in\Bbb N$ such that $w_1\le (w'\wedge u)\vee(u_i\wedge w'')$
for every $i\ge n$, while $w_2\le (w'\wedge v)\vee(v_i\wedge w'')$ for
every $i\ge n$.   So
     
$$\eqalignno{w_1\vee w_2
&\le(w'\wedge u)\vee(w'\wedge v)\vee(u_i\wedge w'')
  \vee(v_i\wedge w'')\cr
&=(w'\wedge(u\vee v))\vee((u_i\vee v_i)\wedge w'')\cr
\noalign{\noindent (352Ec)}
&=w'\vee((u_i\vee v_i)\wedge w'')\cr}$$
     
\noindent for every $i\ge n$, and $w_1\vee w_2\in B$.\ \Qed
     
Similarly, if $w_1\in C_1$ and $w_2\in C_2$ then $w_1\vee w_2\in C$.
\Prf\ There is an $n\in\Bbb N$ such that
$w_1\ge((w'\wedge u)\vee u_i)\wedge w''$ and
$w_2\ge((w'\wedge v)\vee v_i)\wedge w''$ for every $i\ge n$.   So
     
$$\eqalign{w_1\vee w_2
&\ge(((w'\wedge u)\vee u_i)\wedge w'')
  \vee(((w'\wedge v)\vee v_i)\wedge w'')\cr
&=((w'\wedge u)\vee u_i\vee(w'\wedge v)\vee v_i)\wedge w''\cr
&=((w'\wedge(u\vee v))\vee(u_i\vee v_i))\wedge w''\cr
&=(w'\vee(u_i\vee v_i))\wedge w''\cr}$$
     
\noindent for every $i\ge n$, so $w_1\vee w_2\in C$.\ \Qed
     
At the same time, of course, $w\le\tilde w$ whenever $w\in B$ and
$\tilde w\in C$, since there is some $i\in\Bbb N$ such that
     
\Centerline{$w\le w'\vee((u_i\vee v_i)\wedge w'')
\le(w'\vee(u_i\vee v_i))\wedge w''
\le\tilde w$.}
     
\noindent Since
     
\Centerline{$\sup\{w_1\vee w_2:w_1\in B_1,\,w_2\in B_2\}
=(\sup B_1)\vee(\sup B_2)
=u\vee v$,}
     
\Centerline{$\inf\{w_1\vee w_2:w_1\in C_1,\,w_2\in C_2\}
=(\inf C_1)\vee(\inf C_2)
=u\vee v$}
     
\noindent (using the generalized distributive laws in 352E), we must
have $\sup B=\inf C=u\vee v$.   As $w'$ and $w''$ are arbitrary,
$\sequencen{u_n\vee v_n}$ is order*-convergent to $u\vee v$.
     
Putting this together with (a), we see that
$\sequencen{u_n\wedge v_n}=\sequencen{-((-u_n)\vee(-v_n))}$
order*-converges to $-((-u)\vee(-v))=u\wedge v$.
     
\medskip
     
{\bf (c)} The hard parts are over.   (i) If $\sequencen{w_n}$
order*-converges to $w$, then $\sequencen{w_n-w}$,
$\sequencen{w-w_n}$ and
$\sequencen{|w_n-w|}
=\sequencen{(w_n-w)\vee(w-w_n)}$ all order*-converge to $0$, putting
(a) and (b) together.   (ii) If $\sequencen{|w_n-w|}$
order*-converges to $0$, then so do $\sequencen{-|w_n-w|}$ and
$\sequencen{w_n-w}$, by (a) and 367Bd;  so $\sequencen{w_n}$
order*-converges to $0$, by (a) again.
     
\medskip
     
{\bf (d)} $\sequencen{|u_n-u|}$ and $\sequencen{|v_n-v|}$
order*-converge to $0$, by (c), so
$\sequencen{2(|u_n-u|\vee|v_n-v|)}$ also order*-converges to $0$, by
(b) and (a).   But
     
\Centerline{$0\le|(u_n+v_n)-(u+v)|\le|u_n-u|+|v_n-v|
\le 2(|u_n-u|\vee|v_n-v|)$}
     
\noindent for every $n$, so $\sequencen{|(u_n+v_n)-(u+v)|}$
order*-converges to $0$, by 367Bb and 367Bd, and
$\sequencen{u_n+v_n}$ order*-converges to $u+v$.
     
\medskip
     
{\bf (e)} Set $\beta_n=\sup_{i\ge n}|\alpha_i-\alpha|$ for each $n$.
Then $\sequencen{\beta_n}\to 0$, so $\inf_{n\in\Bbb N}\beta_n|u|=0$,
because $U$ is Archimedean.   Consequently $\sequencen{\beta_n|u|}$
order*-converges to $0$, by 367Be.   But we also have
$\beta_0|u_n-u|$ order*-converging to $0$, by (c) and (a), so
$\sequencen{\beta_0|u_n-u|+\beta_n|u|}$ order*-converges to $0$, by
(d).   As $|\alpha_nu_n-\alpha u|\le\beta_0|u_n-u|+\beta_n|u|$ for every
$n$, $\sequencen{\alpha_nu_n}$ order*-converges to $\alpha u$, as
required.
     
\medskip
     
{\bf (f)(i)} Suppose that $\sequencen{w_n}$ is not order*-convergent
to $0$.   Then there are $w'$, $w''$ such that $w'\le 0\le w''$ and
either
     
\Centerline{$B=\{w:\,\exists\,n\in\Bbb N,\,w\le w'\vee(w_i\wedge
w'')\Forall i\ge n\}$}
     
\noindent does not have supremum $0$, or
     
\Centerline{$C=\{w:\,\exists\,n\in\Bbb N,\,w\ge(w'\vee w_i)\wedge
w''\Forall i\ge n\}$}
     
\noindent does not have infimum $0$.   Now $0\in B$, because every
$w_i\ge 0$, and every member of $B$ is a lower bound for $C$;  so $0$
cannot be the greatest lower bound of $C$.   Let $\tilde w>0$ be a lower
bound for $C$.
     
Let $n\in\Bbb N$, and set
     
\Centerline{$C_n=\{w:w\ge(w'\vee w_i)\wedge w''\Forall i\ge n\}
=\{w:w\ge w_i\wedge w''\Forall i\ge n\}$.}
     
\noindent Because $U$ is Archimedean, we know that $\inf(C_n-A_n)=0$,
where $A_n=\{w_i\wedge w'':i\ge n\}$ (353F).   Now $\tilde w$ is a lower
bound for $C_n$, so
     
     
$$\eqalign{\inf_{i\ge n}(\tilde w-w_i)^+
&\le\inf\{(w-w_i)^+:w\in C,\,i\ge n\}\cr
&\le\inf\{(w-(w_i\wedge w''))^+:w\in C,\,i\ge n\}\cr
&=\inf\{w-(w_i\wedge w''):w\in C,\,i\ge n\}
=\inf(C_n-A_n)
=0.\cr}$$
     
\noindent As this is true for every $n\in\Bbb N$, $\tilde w$ has the
property declared.
     
\medskip
     
\quad{\bf (ii)} If $\tilde w>0$ is such that $\tilde w=\sup_{i\ge
n}\tilde w\wedge w_i$ for every $n\in\Bbb N$, then
     
\Centerline{$\{w:\,\exists\,n\in\Bbb N,\,w\ge(0\vee w_i)\wedge\tilde
w\Forall i\ge n\}$}
     
\noindent cannot have infimum $0$, and $\sequencen{w_n}$ is not
order*-convergent to $0$.
}%end of proof of 367C
     
\leader{367D}{}\cmmnt{ As examples of the use of this concept in a
relatively abstract setting, I offer the following.
     
\medskip
     
\noindent}{\bf Proposition} (a) Let $U$ be a Banach lattice and
$\sequencen{u_n}$ a sequence in $U$ which is norm-convergent to
$u\in U$.   Then $\sequencen{u_n}$ has a subsequence which is
order-bounded
and order*-convergent to $u$.   So if $\sequencen{u_n}$ itself is
order*-convergent, its order*-limit is $u$.

(b)\dvAformerly{3{}67E} Let $U$ be a Riesz space with an
order-continuous norm.   Then any order-bounded order*-convergent
sequence is norm-convergent.
     
\proof{{\bf (a)} Let $\sequencen{u'_n}$ be a subsequence of 
$\sequencen{u_n}$
such that $\|u'_n-u\|\le 2^{-n}$ for each $n\in\Bbb N$.   Then
$v_n=\sup_{i\ge n}|u'_i-u|$ is defined in $U$, and $\|v_n\|\le 2^{-n+1}$,
for each $n$ (354C).   Because $\inf_{n\in\Bbb N}\|v_n\|=0$,
$\inf_{n\in\Bbb N}v_n$ must be $0$, while $u-v_n\le u'_i\le u+v_n$
whenever $i\ge n$;  so $\sequencen{u'_n}$ order*-converges to $u$, by
367Be.
     
Now if $\sequencen{u_n}$ has an order*-limit, this must be $u$, by
367Ba and 367Bc.

\medskip

{\bf (b)} Suppose that $\sequencen{u_n}$ is order*-convergent to $u$.
Then $\sequencen{|u_n-u|}$ is order*-convergent to $0$ (367Cc), so
     
\Centerline{$C
=\{v:\,\exists\,n\in\Bbb N,\,v\ge|u_i-u|\Forall i\ge n\}$}
     
\noindent has infimum $0$ (367Be).   Because $U$ is a lattice, $C$ is
downwards-directed, so $\inf_{v\in C}\|v\|=0$.   But
     
\Centerline{$\inf_{v\in C}\|v\|
\ge\inf_{n\in\Bbb N}\sup_{i\ge n}\|u_i-u\|$,}
     
\noindent so $\lim_{n\to\infty}\|u_n-u\|=0$, that is, $\sequencen{u_n}$
is norm-convergent to $u$.
}%end of proof of 367D
     
\vleader{48pt}{367E}{}\cmmnt{ One of the fundamental obstacles to the
development of any satisfying general theory of ordered topological
spaces is the erratic nature of the relations between subspace
topologies of order topologies and order topologies on subspaces.   The
particular virtue of
order*-convergence in the context of function spaces is that it is
relatively robust when transferred to the subspaces we are interested
in.
     
\medskip
     
\noindent}{\bf Proposition}\dvAformerly{3{}67F}
Let $U$ be an Archimedean Riesz space and
$V$ a regularly embedded Riesz subspace.   \cmmnt{(For instance, $V$
might be either solid or order-dense.)}   If $\sequencen{v_n}$ is a
sequence in $V$ and $v\in V$, then $\sequencen{v_n}$ order*-converges
to $v$ when regarded as a sequence in $V$, iff it order*-converges to
$v$ when regarded as a sequence in $U$.
     
\proof{{\bf (a)} Since, in either $V$ or $U$, $\sequencen{v_n}$
order*-converges to $v$ iff $\sequencen{|v_n-v|}$ order*-converges
to $0$ (367Cc), it is enough to consider the case $v_n\ge 0$, $v=0$.
     
\medskip
     
{\bf (b)} If $\sequencen{v_n}$ is {\it not} order*-convergent to $0$
in $U$, then, by 367Cf, there is a $u>0$ in $U$ such that $u=\sup_{i\ge
n}u\wedge v_i$ for every $n\in\Bbb N$ (the supremum being taken in $U$,
of course).   In particular, there is a $k\in\Bbb N$ such that $u\wedge
v_k>0$.   Now consider the set
     
\Centerline{$C=\{w:w\in V,\,\exists\,n\in\Bbb N,\,
w\ge v_i\wedge v_k\Forall i\ge n\}$.}
     
\noindent Then for any $w\in C$,
     
\Centerline{$u\wedge v_k=\sup_{i\ge n}u\wedge v_i\wedge v_k\le w$,}
     
\noindent using the generalized distributive law in $U$, so $0$ is not
the greatest lower bound of $C$ in $U$.   But as the embedding of $V$ in
$U$ is order-continuous, $0$ is not the greatest lower bound of $C$ in
$V$, and $\sequencen{v_n}$ cannot be
order*-convergent to $0$ in $V$.
     
\medskip
     
{\bf (c)} Now suppose that $\sequencen{v_n}$ is not order*-convergent
to $0$ in $V$.   Because $V$ also is Archimedean (351Rc), there is a
$w>0$ in $V$ such that $w=\sup_{i\ge n}w\wedge v_i$ for every
$n\in\Bbb N$, the suprema being taken in $V$.   Again because $V$ is regularly
embedded in $U$, we have the same suprema in $U$, so, by 367Cf in the
other direction, $\sequencen{v_n}$ is not order*-convergent to $0$ in
$U$.
}%end of proof of 367E
     
\leader{367F}{}\cmmnt{ I now spell out the connexion between the
definition above and the concepts introduced in 245C.
     
\medskip
     
\noindent}{\bf Proposition}\dvAformerly{3{}67F}
Let $X$ be a set, $\Sigma$ a
$\sigma$-algebra of subsets of $X$, $\frak A$ a Boolean algebra and 
$\pi:\Sigma\to\frak A$ a sequentially order-continuous surjective Boolean 
homomorphism;  let
$\Cal I$ be its kernel.   Write $\eusm L^0$ for the space of
$\Sigma$-measurable real-valued functions on $X$, and let
$T:\eusm L^0\to L^0=L^0(\frak A)$ be
the canonical Riesz homomorphism\cmmnt{ (364C, 364P)}.   Then for any
$\sequencen{f_n}$ and $f$ in $\eusm L^0$, $\sequencen{Tf_n}$
order*-converges to $Tf$ in $L^0$ iff
$X\setminus\{x:f(x)=\lim_{n\to\infty}f_n(x)\}\in\Cal I$.
     
\proof{ Set $H=\{x:\lim_{n\to\infty}f_n(x)$ exists $=f(x)\}$;  of course
$H\in\Sigma$.
Set $g_n(x)=|f_n(x)-f(x)|$ for $n\in\Bbb N$ and $x\in X$.
     
\medskip
     
{\bf (a)} If $X\setminus H\in\Cal I$, set
$h_n(x)=\sup_{i\ge n}g_i(x)$ for $x\in H$ and $h_n(x)=0$ for $x\in
X\setminus H$.   Then $\sequencen{h_n}$ is a non-increasing sequence
with infimum $0$ in $\eusm L^0$, so $\inf_{n\in\Bbb N}Th_n=0$ in $L^0$,
because $T$ is sequentially order-continuous (364Pa).   But as
$X\setminus H\in\Cal I$, $Th_n\ge Tg_i=|Tf_i-Tf|$ whenever $i\ge n$, so
$\sequencen{|Tf_n-Tf|}$ order*-converges to $0$, by 367Be or 367Bf,
and $\sequencen{Tf_n}$ order*-converges to $Tf$, by 367Cc.
     
\medskip
     
{\bf (b)} Now suppose that $\sequencen{Tf_n}$ order*-converges to
$Tf$.   Set $g'_n(x)=\min(1,g_n(x))$ for $n\in\Bbb N$,
$x\in X$;  then $\sequencen{Tg'_n}=\sequencen{e\wedge|Tf_n-Tf|}$
order*-converges to $0$, where $e=T(\chi X)$.
By 367Bf, $\inf_{n\in\Bbb N}\sup_{i\ge n}Tg'_i=0$ in $L^0$.   But $T$ is
a sequentially order-continuous Riesz homomorphism, so
$T(\inf_{n\in\Bbb N}\sup_{i\ge n}g'_i)=0$, that is,
     
\Centerline{$X\setminus H=\{x:\inf_{n\in\Bbb N}\sup_{i\ge n}g'_i>0\}$}
     
\noindent belongs to $\Cal I$.
}%end of proof of 367F
     
\leader{367G}{Corollary}\dvAformerly{3{}67H}
Let $\frak A$ be a Dedekind $\sigma$-complete Boolean algebra.
     
(a) Any order*-convergent sequence in $L^0=L^0(\frak A)$ is
order-bounded.
     
(b) If $\sequencen{u_n}$ is a sequence in $L^0$, then it is
order*-convergent to $u\in L^0$ iff
     
\Centerline{$u=\inf_{n\in\Bbb N}\sup_{i\ge n}u_i
=\sup_{n\in\Bbb N}\inf_{i\ge n}u_i$.}
     
\proof{{\bf (a)} We can express $\frak A$ as a quotient $\Sigma/\Cal I$
of a $\sigma$-algebra of sets, in which case $L^0$ can be identified
with the
canonical image of $\eusm L^0=\eusm L^0(\Sigma)$ (364C).   If
$\sequencen{u_n}$ is an order*-convergent sequence in $L^0$, then it
is expressible as $\sequencen{Tf_n}$, where $T:\eusm L^0\to L^0$ is the
canonical map, and 367F tells us that $\sequencen{f_n(x)}$ converges for
every $x\in H$, where $X\setminus H\in\Cal I$.    If we set
$h(x)=\sup_{n\in\Bbb N}|f_n(x)|$ for $x\in H$, $0$ for
$x\in X\setminus H$, then we see that $|u_n|\le Th$ for every
$n\in\Bbb N$, so that $\sequencen{u_n}$ is order-bounded in $L^0$.
     
\medskip
     
{\bf (b)} This now follows from 367Bf, because $L^0$ is Dedekind
$\sigma$-complete.
}%end of proof of 367G
     
\leader{367H}{Proposition}\dvAformerly{3{}67I}
Suppose that $E\subseteq\Bbb R$ is a Borel
set and $h:E\to\Bbb R$ is a continuous function.   Let $\frak A$ be a
Dedekind $\sigma$-complete Boolean algebra and set
$Q_E=\{u:u\in L^0,\,\Bvalue{u\in E}=1\}$, where $L^0=L^0(\frak A)$.
Let $\bar h:Q_E\to L^0$ be the function defined by $h$\cmmnt{ (364H)}.
Then $\sequencen{\bar h(u_n)}$ order*-converges to $\bar h(u)$
whenever $\sequencen{u_n}$ is a sequence in $Q_E$ order*-converging
to $u\in Q_E$.
     
\proof{ This is an easy consequence of 367F.   We can represent
$\frak A$ as $\Sigma/\Cal I$ where $\Sigma$ is a $\sigma$-algebra of
subsets of
some set $X$ and $\Cal I$ is a $\sigma$-ideal of $\Sigma$ (314M);  let
$T:\eusm L^0\to L^0(\frak A)$ be the corresponding homomorphism
(364C, 367F).   Now we can find $\Sigma$-measurable functions
$\sequencen{f_n}$, $f$ such that $Tf_n=u_n$, $Tf=u$, as in 367F;   and
the hypothesis $\Bvalue{u_n\in E}=1$, $\Bvalue{u\in E}=1$ means just
that, adjusting $f_n$ and $f$ on a member of $\Cal I$ if necessary, we
can suppose that $f_n(x)$, $f(x)\in E$ for every $x\in X$.   (I am
passing over the trivial case $E=\emptyset$, $X\in\Cal I$,
$\frak A=\{0\}$.)   Accordingly
$\bar h(u_n)=T(hf_n)$ and $\bar h(u)=T(hf)$, and (because $h$ is
continuous)
     
\Centerline{$\{x:h(f(x))\ne\lim_{n\to\infty}h(f_n(x))\}
\subseteq\{x:f(x)\ne\lim_{n\to\infty}f_n(x)\}\in\Cal I$,}
     
\noindent so $\sequencen{\bar h(u_n)}$ order*-converges to
$\bar h(u)$.
}%end of proof of 367H
     
\leader{367I}{Dominated \dvrocolon{convergence}}\cmmnt{ We now have a
suitable language in which to express an abstract version of Lebesgue's
Dominated Convergence Theorem.
     
\medskip
     
\noindent}{\bf Theorem}\dvAformerly{3{}67J}
Let $(\frak A,\bar\mu)$ be a measure algebra.
If $\sequencen{u_n}$ is a sequence in $L^1=L^1(\frak A,\bar\mu)$ which
is order-bounded and order*-convergent in $L^1$, then
$\sequencen{u_n}$ is norm-convergent to $u$ in $L^1$;  in particular,
$\int u=\lim_{n\to\infty}\int u_n$.
     
\proof{ The norm of $L^1$ is order-continuous (365C), so
$\sequencen{u_n}$ is norm-convergent to $u$, by 367Db.   As $\int$ is
norm-continuous, $\int u=\lim_{n\to\infty}\int u_n$.
}%end of proof of 367I
     
\leader{367J}{The Martingale Theorem}\cmmnt{ In the same way, we can
re-write theorems from \S275 in this language.
     
\medskip
     
\noindent{\bf Theorem}}\dvAformerly{3{}67K}
Let $(\frak A,\bar\mu)$ be a probability
algebra, and $\sequencen{\frak B_n}$ a non-decreasing sequence of closed
subalgebras of $\frak A$.   For each $n\in\Bbb N$ let
$P_n:L^1=L^1(\frak A,\bar\mu)\to L^1\cap L^0(\frak B_n)$ be the conditional expectation
operator\cmmnt{ (365R)};  let $\frak B$ be the closed subalgebra of
$\frak A$ generated by $\bigcup_{n\in\Bbb N}\frak B_n$, and $P$ the
conditional expectation operator onto $L^1\cap L^0(\frak B)$.
     
(a) If $\sequencen{u_n}$ is a norm-bounded sequence in $L^1$ such that
$P_n(u_{n+1})=u_n$ for every $n\in\Bbb N$, then $\sequencen{u_n}$ is
order*-convergent in $L^1$.
     
(b) If $u\in L^1$ then $\sequencen{P_nu}$ is order*-convergent and
$\|\,\|_1$-convergent to $Pu$.
     
\proof{ If we represent $(\frak A,\bar\mu)$ as the measure algebra of a
probability space, these become mere translations of 275G and 275I.
(Note that this argument relies on the description of
order*-convergence in $L^0$ in terms of a.e.\ convergence of
functions, as in 367F;  so that we need to know that
order*-convergence in $L^1$ matches order*-convergence in $L^0$,
which is what 367E is for.)
}%end of proof of 367J
     
\leader{367K}{}\cmmnt{ Some of the most important applications of
these ideas concern spaces of continuous functions.   I do not think
that this is
the time to go very far along this road, but one particular fact will be
useful in \S376.
     
\medskip
     
\noindent}{\bf Proposition}\dvAformerly{3{}67L}
Let $X$ be a locally compact Hausdorff
space, and $\sequencen{u_n}$ a sequence in $C(X)$\cmmnt{, the space of
continuous real-valued functions on $X$}.   Then $\sequencen{u_n}$
order*-converges to $0$ in $C(X)$ iff
$\{x:x\in X,\,\limsup_{n\to\infty}|u_n(x)|>0\}$ is
meager.   In particular, $\sequencen{u_n}$ order*-converges to $0$ if
$\lim_{n\to\infty}u_n(x)=0$ for every $x$.
     
\proof{{\bf (a)} The following elementary fact is worth noting:  if
$A\subseteq C(X)^+$ is non-empty and $\inf A=0$ in $C(X)$, then
$G=\bigcup_{u\in A}\{x:u(x)<\epsilon\}$ is dense for every $\epsilon>0$.
\Prf\Quer\ If not, take $x_0\in X\setminus\overline{G}$.   Because $X$
is completely regular (3A3Bb), there is a continuous function
$w:X\to[0,1]$
such that $w(x_0)=1$ and $w(x)=0$ for every $x\in\overline{G}$.   But in
this case $0<\epsilon w\le u$ for every $u\in A$, which is
impossible.\ \Bang\Qed\
     
\medskip
     
{\bf (b)} Suppose that $\sequencen{u_n}$ order*-converges to $0$.
Set $v_n=|u_n|\wedge\chi X$, so that $\sequencen{v_n}$
order*-converges to $0$ (using 367C, as usual).   Set
     
\Centerline{$B=\{v:v\in C(X),\,\exists\,n\in\Bbb N,\,v_i\le
v\Forall i\ge n\}$,}
     
\noindent so that $\inf B=0$ in
$C(X)$ (367Be).   For each $k\in\Bbb N$, set
$G_k=\bigcup_{v\in B}\{x:v(x)<2^{-k}\}$;  then $G_k$ is dense, by (a),
and of course is open.   So $H=\bigcup_{k\in\Bbb N}X\setminus G_k$ is a
countable union of
nowhere dense sets and is meager.   But this means that
     
$$\eqalign{\{x:\limsup_{n\to\infty}|u_n(x)|>0\}
&=\{x:\limsup_{n\to\infty}v_n(x)>0\}\cr
&\subseteq\{x:\inf_{v\in B}v(x)>0\}
\subseteq H\cr}$$
     
\noindent is meager.
     
\medskip
     
{\bf (c)} Now suppose that $\sequencen{u_n}$ does not order*-converge
to $0$.   By 367Cf, there is a $w>0$ in $C(X)$ such that $w=\sup_{i\ge
n}w\wedge|u_i|$ for every $n\in\Bbb N$;  that is,
$\inf_{i\ge n}(w-|u_i|)^+=0$ for every $n$.   Set
     
\Centerline{$G_n
=\{x:\inf_{i\ge n}(w-|u_i|)^+(x)<2^{-n}\}
=\{x:\sup_{i\ge n}|u_i(x)|>w(x)-2^{-n}\}$}
     
\noindent for each $n$.   Then
     
\Centerline{$H=\bigcap_{n\in\Bbb N}G_n
=\{x:\limsup_{n\to\infty}u_n(x)\ge w(x)\}$}
     
\noindent is the intersection of a sequence of dense open sets, and its
complement is meager.
     
Let $G$ be the non-empty open set $\{x:w(x)>0\}$.   Then $G$ is not
meager, by Baire's theorem (3A3Ha);  so $G\cap H$ cannot be meager.
But $\{x:\limsup_{n\to\infty}|u_n(x)|>0\}$ includes $G\cap H$, so is
also not meager.
}%end of proof of 367K
     
\cmmnt{\medskip
     
\noindent{\bf Remark} Unless the topology of $X$ is discrete, $C(X)$ is
not regularly embedded in $\Bbb R^X$, and we expect to find sequences in
$C(X)$ which order*-converge to $0$ in $C(X)$ but not in $\Bbb R^X$.
But the proposition tells us that if we have a sequence in $C(X)$ which
order*-converges in $\Bbb R^X$ to a member of $C(X)$, then it
order*-converges in $C(X)$.
}%end of comment
     
\leader{367L}{}\dvAformerly{3{}67M}\cmmnt{ Everything above concerns a 
particular notion
of sequential convergence.   There is inevitably a suggestion that there
ought to be a topological interpretation of this convergence (see
367Yb, 367Yk, 3A3P), but I have taken care to avoid spelling one out at 
this stage;  I will return to the point in \S393.   
(For a general discussion in the context of Boolean algebras, 
see {\smc Vladimirov 02}, chap.\ 4.)   I come
now to something which really is a topology, and is as closely involved
with order-convergence as any.
     
\medskip
     
\noindent}{\bf Convergence in measure} Let $(\frak A,\bar\mu)$ be a
measure algebra.   For $a\in\frak A^f=\{a:\bar\mu a<\infty\}$ and
$u\in L^0=L^0(\frak A)$ set $\tau_a(u)=\int|u|\wedge\chi a$,
$\tau_{a\epsilon}(u)=\bar\mu(a\Bcap\Bvalue{|u|>\epsilon})$.   Then the
{\bf topology of convergence in measure} on $L^0$ is defined {\it
either} as the topology generated by the pseudometrics
$(u,v)\mapsto\tau_a(u-v)$ {\it or} by saying that $G\subseteq L^0$ is
open iff for every $u\in G$ there are $a\in\frak A^f$ and $\epsilon>0$ such that $v\in G$ whenever $\tau_{a\epsilon}(u-v)\le\epsilon$.
     
\cmmnt{\medskip
     
\noindent{\bf Remark} The sentences above include a number of assertions
which need proving.   But at this point, rather than write out any of
the relevant arguments, I refer you to \S245.   Since we
know that $L^0(\frak A)$ can be identified with $L^0(\mu)$ for a
suitable measure space $(X,\Sigma,\mu)$ (321J, 364Ic), everything we
know about general spaces $L^0(\mu)$ can be applied directly to
$L^0(\frak A)$ for measure algebras $(\frak A,\bar\mu)$;  and that is
what I will
do for the next few paragraphs.   So far, all I have done is to write
$\tau_a$ in place of the $\bar\tau_F$ of 245Ac, and call on the
remarks in 245Bb and 245F.
}%end of comment
     
\leader{367M}{Theorem}\dvAformerly{3{}67N}
(a) For any measure algebra $(\frak A,\bar\mu)$,
the topology $\frak T$ of convergence in measure on $L^0=L^0(\frak A)$
is a linear space topology, and any order*-convergent sequence in
$L^0$ is $\frak T$-convergent to the same limit.
%maybe add 367Xp here
     
(b) $(\frak A,\bar\mu)$ is semi-finite iff $\frak T$ is Hausdorff.
     
(c) $(\frak A,\bar\mu)$ is localizable iff $\frak T$ is Hausdorff and
$L^0$ is complete under the uniformity corresponding to $\frak T$.
     
(d) $(\frak A,\bar\mu)$ is $\sigma$-finite iff $\frak T$ is metrizable.
     
\proof{ 245D, 245Cb, 245E.   Of course we need 322B to assure us that
the phrases `semi-finite', `localizable', `$\sigma$-finite'
here correspond to the same phrases used in \S245, and 367F to identify
order*-convergence in $L^0$ with the order-convergence studied in
\S245.
}%end of proof of 367M
     
\leader{367N}{Proposition}\dvAformerly{3{}67O} 
Let $(\frak A,\bar\mu)$ be a measure algebra
and give $L^0=L^0(\frak A)$ its topology of convergence in measure.
     
(a) If $A\subseteq L^0$ is a non-empty, downwards-directed set with
infimum
$0$, then for every neighbourhood $G$ of $0$ in $L^0$ there is a
$u\in A$ such that $v\in G$ whenever $|v|\le u$.
     
(b) If $U\subseteq L^0$ is an order-dense Riesz subspace, it is
topologically dense.
     
(c) In particular, $S(\frak A)$ and $L^{\infty}(\frak A)$ are
topologically dense.
     
\proof{{\bf (a)} Let $a\in\frak A^f$, $\epsilon>0$ be such that $u\in G$
whenever $\int|u|\wedge\chi a\le\epsilon$ (see 245Bb).   Since
$\{u\wedge\chi a:u\in A\}$ is a downwards-directed set in
$L^1=L^1_{\bar\mu}$ with infimum $0$
in $L^1$, there must be a $u\in A$ such that
$\int u\wedge\chi a\le\epsilon$ (365Da).   But now $[-u,u]\subseteq G$,
as required.
     
\medskip
     
{\bf (b)} Write $\overline{U}$ for the closure of $U$.   Then
$(L^0)^+\subseteq\overline{U}$.   \Prf\ If $v\in(L^0)^+$, then
$\{u:u\in U$, $u\le v\}$ is an upwards-directed set with supremum $u$,
so $A=\{v-u:u\in U$, $u\le v\}$ is a downwards-directed set with infimum
$0$ (351Db).   By (a), every neighbourhood of $0$ meets $A$, and
(because subtraction is continuous) every neighbourhood of $v$ meets
$U$, that is, $v\in\overline{U}$.\ \Qed
     
Since $\overline{U}$ is a linear subspace of $L^0$ (2A5Ec), it includes
$(L^0)^+-(L^0)^+=L^0$ (352D).
     
\medskip
     
{\bf (c)} By 364Ja, $S(\frak A)$ and $L^{\infty}(\frak A)$ are
order-dense Riesz subspaces of $L^0$.
}%end of proof of 367N
     
\leader{367O}{Theorem}\dvAformerly{3{}67P}
Let $U$ be a Banach lattice and
$(\frak A,\bar\mu)$ a measure algebra.   Give $L^0=L^0(\frak A)$ its
topology of convergence in measure.   If $T:U\to L^0$ is a
positive linear operator, then it is continuous.
     
\proof{ Take any open set $G\subseteq L^0$.   \Quer\ Suppose, if
possible, that $T^{-1}[G]$ is not open.   Then we can find $u$,
$\sequencen{u_n}\in U$ such that $Tu\in G$ and $\|u_n-u\|\le 2^{-n}$,
$Tu_n\notin G$ for every $n$.   Set $H=G-Tu$;  then $H$ is an open set
containing $0$ but not $T(u_n-u)$, for any $n\in\Bbb N$.   Since
$\sum_{n=0}^{\infty}n\|u_n-u\|<\infty$, $v=\sum_{n=0}^{\infty}n|u_n-u|$
is defined in $U$, and $|T(u_n-u)|\le\bover1nTv$ for every $n\ge 1$.
But by 367Na (or otherwise) we know that there is some $n$ such that
$w\in H$
whenever $|w|\le\bover1nTv$, so that $T(u_n-u)\in H$ for some $n$, which
is impossible.\ \Bang
}%end of proof of 367O
     
\vleader{60pt}{367P}{Proposition}\dvAformerly{3{}67Q} 
Let $(\frak A,\bar\mu)$ be a $\sigma$-finite measure algebra.
     
(a) A sequence $\sequencen{u_n}$ in $L^0=L^0(\frak A)$ converges in
measure to $u\in L^0$ iff every subsequence of $\sequencen{u_n}$ has a
sub-subsequence which order*-converges to $u$.
     
(b) A set $F\subseteq L^0$ is closed for the topology of convergence in
measure iff $u\in F$ whenever there is a sequence $\sequencen{u_n}$ in
$F$ order*-converging to $u\in L^0$.
     
\proof{ 245K, 245L.
}%end of proof of 367P

\leader{367Q}{}\cmmnt{ As an example of the power of the language we now
have available, I give abstract versions of some martingale
convergence theorems.

\medskip

\noindent}{\bf Theorem}\dvAnew{2009, 2012}
Let $(\frak A,\bar\mu)$ be a probability algebra; 
for each closed subalgebra $\frak B$ of $\frak A$, let 
$P_{\frak B}$ be the corresponding conditional expectation operator.

(a) If $\Bbb B$ is a non-empty downwards-directed family of closed
subalgebras of $\frak A$ with intersection $\frak C$, then for every 
$u\in L^1$, $P_{\frak C}u$ is the $\|\,\|_1$-limit of $P_{\frak B}u$ as
$\frak B$ decreases through $\Bbb B$, in the sense that

\doubleinset{for every $\epsilon>0$ there is a $\frak B_0\in\Bbb B$ such 
that $\|P_{\frak B}u-P_{\frak C}u\|_1\le\epsilon$ whenever 
$\frak B\in\Bbb B$ and $\frak B\subseteq\frak B_0$.}

(b) If $\Bbb B$ is a non-empty upwards-directed family of closed
subalgebras of $\frak A$ and $\frak C$ is the closed subalgebra generated
by $\bigcup\Bbb B$, then for every 
$u\in L^1$, $P_{\frak C}u$ is the $\|\,\|_1$-limit of $P_{\frak B}u$ as
$\frak B$ increases through $\Bbb B$, in the sense that

\doubleinset{for every $\epsilon>0$ there is a $\frak B_0\in\Bbb B$ such 
that $\|P_{\frak B}u-P_{\frak C}u\|_1\le\epsilon$ whenever 
$\frak B\in\Bbb B$ and $\frak B\supseteq\frak B_0$.}

(c) Suppose that $\Bbb B$ is a non-empty upwards-directed family of closed
subalgebras of $\frak A$, and $\family{\frak B}{\Bbb B}{u_{\frak B}}$ is a
$\|\,\|_1$-bounded family in $L^1$ such that 
$u_{\frak B}=P_{\frak B}u_{\frak C}$ whenever $\frak B$, 
$\frak C\in\Bbb B$ and $\frak B\subseteq\frak C$.   Then
there is a $u\in L^1$ which is the limit 
$\lim_{\frak B\to\Cal F(\Bbb B\closeuparrow)}u_{\frak B}$ 
for the topology of convergence in measure, where 
$\Cal F(\Bbb B\closeuparrow)$ is the filter on $\Bbb B$ generated by 
$\{\{\frak C:\frak B\subseteq\frak C\in\Bbb B\}:\frak B\in\Bbb B\}$.

%do we have full pointwise limits?\query

\proof{{\bf (a)} Take $u\in L^1$.

\medskip

\quad{\bf (i)} Note first that $\{P_{\frak B}u:\frak B\in\Bbb B\}$ is
uniformly integrable (246D, or directly), therefore relatively weakly
compact in $L^1$ (247C).   Consequently there must be a
$v\in L^1$ which is a weak cluster point of $P_{\frak B}u$ as $\frak B$
decreases through $\Bbb B$, in the sense that 
$v$ belongs to the weak closure 
$\overline{\{P_{\frak B}u:\frak B\in\Bbb B,\,\frak B\subseteq\frak B_0\}}$
for every $\frak B_0\in\Bbb B$.

It follows that $v=P_{\frak C}u$.   \Prf\ For every 
$\frak B_0\in\Bbb B$, $L^1\cap L^0(\frak B_0)$, identified with
$L^1(\frak B_0,\bar\mu\restrp\frak B_0)$, is a norm-closed linear
subspace of $L^1$ containing $P_{\frak B}u$ whenever 
$\frak B\subseteq\frak B_0$.   It is therefore weakly closed (3A5Ee)
and contains
$v$.   Consequently $\Bvalue{v>\alpha}\in\frak B_0$ for every
$\alpha\in\Bbb R$.   As $\frak B_0$ is arbitrary,
$\Bvalue{v>\alpha}\in\frak C$ for every $\alpha\in\Bbb R$, and 
$v\in L^1(\frak C,\bar\mu\restrp\frak C)$.   Next, if $c\in\frak C$,
then 

\Centerline{$\int_cv\in\overline{\{\intop_cP_{\frak B}u:\frak B\in\Bbb B\}}
=\{\int_cu\}$;}

\noindent so $v=P_{\frak C}u$.\ \Qed

\medskip

\quad{\bf (ii)} Now take $\epsilon>0$.   Then
there is a $\frak B_0\in\Bbb B$ such 
that $\|P_{\frak B}u-P_{\frak B_0}u\|_1\le\bover12\epsilon$ whenever 
$\frak B\in\Bbb B$ and $\frak B\subseteq\frak B_0$.
\Prf\Quer\ Otherwise, we can find a non-increasing sequence
$\sequencen{\frak B_n}$ in $\Bbb B$ such that  
$\|P_{\frak B_{n+1}}u-P_{\frak B_n}u\|_1>\bover12\epsilon$ for every $n\in\Bbb N$.
By the reverse martingale theorem (275K), $\sequencen{P_{\frak B_n}u}$ is
order*-convergent to $w$ say.   But as $\{P_{\frak B_n}u:n\in\Bbb N\}$ is
uniformly integrable, $\sequencen{P_{\frak B_n}u}$ is $\|\,\|_1$-convergent
to $w$ (246Ja), and 
$\lim_{n\to\infty}\|P_{\frak B_{n+1}}u-P_{\frak B_n}u\|_1=0$.\ \Bang\Qed

At this point, however, observe that 
$C=\{w:\|w-P_{\frak B_0}u\|_1\le\bover12\epsilon\}$ is convex and
$\|\,\|_1$-closed, therefore weakly closed, in $L^1$.   Since it
contains $P_{\frak B}u$ whenever $\frak B\in\Bbb B$ and 
$\frak B\subseteq\frak B_0$, it contains $v=P_{\frak C}u$.   Consequently

\Centerline{$\|P_{\frak B}u-P_{\frak C}u\|_1
\le\|P_{\frak B}u-P_{\frak B_0}u\|_1+\|P_{\frak B_0}u-v\|_1
\le\epsilon$}

\noindent whenever $\frak B\in\Bbb B$ and $\frak B\subseteq\frak B_0$.
As $\epsilon$ and $u$ are arbitrary, (a) is true.

\medskip

{\bf (b)} We can use the same method.   Again take any $u\in L^1$.

\medskip

\quad{\bf (i)} This time, observe that $P_{\frak B}u$ must have a weak
cluster point $v$ as $\frak B$ increases through $\Bbb B$.   Since
$P_{\frak B}u$ belongs to $L^1\cap L^0(\frak C)$ for every 
$\frak B\in\Bbb B$, so does $v$.   Next, if $b\in\frak B_0\in\Bbb B$, then
$\int_bP_{\frak B}u=\int_bu$ whenever $\frak B\supseteq\frak B_0$, so
$\int_bv=\int_bu$.   Thus $\frak D=\{b:b\in\frak A$, $\int_bv=\int_bu\}$ 
includes $\bigcup\Bbb B$.   But $\frak D$ is closed for the measure algebra
topology of $\frak A$, so $\frak D\supseteq\frak C$ and $\int_cv=\int_cu$
for every $c\in\frak C$.   Thus once again we have $v=P_{\frak C}u$.

\medskip

\quad{\bf (ii)} Now repeat the argument of (a-ii) almost word for word, but
taking `$\frak B\supseteq\frak B'$' in place of every 
`$\frak B\subseteq\frak B'$', and quoting the ordinary martingale theorem
instead of the reverse martingale theorem.

\medskip

{\bf (c)(i)} If $\sequencen{\frak B_n}$ is a non-decreasing sequence in
$\Bbb B$, then $\sequencen{u_{\frak B_n}}$ is order*-convergent, by
Doob's martingale theorem (367Ja).   

\medskip

\quad{\bf (ii)}
It follows that the image $\Cal G$ of $\Cal F(\Bbb B\closeuparrow)$
under the map $\frak B\mapsto u_{\frak B}:\Bbb B\to L^0$ is
Cauchy for the linear space topology $\frak T$ of convergence in measure.
\Prf\Quer\ Otherwise, set $\tau(v)=\int|v|\wedge\chi 1$ for $v\in L^0$,
there is an $\epsilon>0$ such that $\sup_{v,v'\in C}\tau(v-v')>2\epsilon$
for every $C\in\Cal G$;  in which case, for any $\frak B\in\Bbb B$, there
must be a $\frak C\in\Bbb B$ such that 
$\tau(u_{\frak C}-u_{\frak B})\ge\epsilon$.   But now there will be a
non-decreasing sequence $\sequencen{\frak B_n}$ in $\Bbb B$ such that
$\tau(u_{\frak B_{n+1}}-u_{\frak B_n})\ge\epsilon$ for every $n\in\Bbb N$
and $\sequencen{u_{\frak B_n}}$ cannot be order*-convergent.\ \Bang\Qed

\medskip

\quad{\bf (iii)} By 367Mc,
$u=\lim\Cal G=\lim_{\frak B\to\Cal F(\Bbb B\closeuparrow)}u_{\frak B}$
is defined in $L^0$ for $\frak T$.   But as $u$ belongs to the
$\frak T$-closure of the $\|\,\|_1$-bounded set 
$\{u_{\frak B}:\frak B\in\Bbb B\}$, $u\in L^1$, by 245J(b-i).
}%end of proof of 367Q

\leader{367R}{}\cmmnt{ It will be useful later to be able to quote
the following straightforward facts.
     
\medskip
     
\noindent}{\bf Proposition} Let $(\frak A,\bar\mu)$ be a measure
algebra.   Give $\frak A$ its measure-algebra topology\cmmnt{ (323A)}
and $L^0=L^0(\frak A)$ the topology of convergence in measure.   

(a) The map $\chi:\frak A\to L^0$ is a homeomorphism between $\frak A$ and
its image in $L^0$.

(b)\dvAnew{2011} If $\frak A$ has countable Maharam type, then $L^0$ is
separable.

(c)\dvAnew{2011} Suppose that $\frak B$ is a subalgebra of $\frak A$ which
is closed for the measure-algebra topology.   Then
$L^0(\frak B)$ is closed in $L^0(\frak A)$.

\proof{{\bf (a)} Of course $\chi$ is injective (364Jc).   
The measure-algebra
topology of $\frak A$ is defined by the pseudometrics
$\rho_a(b,c)=\bar\mu(a\Bcap(b\symmdiff c))$, while the topology of $L^0$
is defined by the pseudometrics $\sigma_a(u,v)=\int|u-v|\wedge\chi a$, in
both cases taking $a$ to run over elements of $\frak A$ of finite
measure;  as $\sigma_a(\chi b,\chi c)$ is always equal to $\rho_a(b,c)$,
we have the result.

\medskip

{\bf (b)} By 331O, $\frak A$ is separable in its measure-algebra topology;
let $B\subseteq\frak A$ be a countable dense set.   Set

\Centerline{$B^*=\{\sum_{i=0}^n\alpha_i\chi b_i:n\in\Bbb N$,
$\alpha_0,\ldots,\alpha_n\in\Bbb Q$, $b_0,\ldots,b_n\in B\}$.}
 
\noindent $B^*$ is a countable subset of $L^0$;   let $V$ be its
closure.   Then $V$ includes $S(\frak A)$.   \Prf\ For any $n\in\Bbb N$,
the function 
$(\alpha_0,\ldots,\alpha_n,a_0,\ldots,a_n)\mapsto
\sum_{i=0}^n\alpha_i\chi a_i:\BbbR^{n+1}\times\frak A^{n+1}\to L^0$ is
continuous, just because $\chi:\frak A\to L^0$ and
addition and scalar multiplication in $L^0$ are continuous ((a) above,
367M).   So

\Centerline{$D_n=\{(\alpha_0,\ldots,\alpha_n,a_0,\ldots,a_n):
\sum_{i=0}^n\alpha_i\chi a_i\in V\}$}

\noindent is a closed subset of $\BbbR^{n+1}\times\frak A^{n+1}$ including
$\BbbQ^{n+1}\times B^{n+1}$.  But $\BbbQ^{n+1}\times B^{n+1}$ is dense in
$\BbbR^{n+1}\times\frak A^{n+1}$ (3A3Ie), so
$D_n=\BbbR^{n+1}\times\frak A^{n+1}$, that is,
$\sum_{i=0}^n\alpha_i\chi a_i\in V$ whenever 
$\alpha_0,\ldots,\alpha_n\in\Bbb R$ and $a_0,\ldots,a_n\in V$.   As $n$ is
arbitrary, $S(\frak A)\subseteq V$.\ \Qed

Since $S(\frak A)$ is dense in $L^0$ (367Nc), $V=L^0$, $B^*$ is dense in
$L^0$ and $L^0$ is separable.

\medskip

{\bf (c)(i)} Note first that $\frak B$ is order-closed in $\frak A$
(323D(c-i)), so that $L^0(\frak B)$, defined as in 364A, is a subset of
$L^0(\frak A)$ (cf.\ 364Xt).   Applying 364P to the identity map
$\frak B\embedsinto\frak A$, we see that the map
$L^0(\frak B)\embedsinto L^0(\frak A)$ identifies the operations
of addition, scalar multiplication and supremum in $L^0(\frak B)$ with the
restrictions of the corresponding operations on $L^0(\frak A)$.

Suppose that $u\in L^0(\frak A)$ is in the closure of $L^0(\frak B)$,
and $\alpha\in\Bbb R$;  let $n\in\Bbb N$ be such that $|\alpha|<n$, and
fix $a\in\frak A^f$ for the moment.
For each $k\in\Bbb N$, choose
$v_k\in L^0(\frak B)$ such that
$\int|u-v_k|\times\chi a\le 2^{-k}$ (367L).
Consider $v'_k=\med(-n\chi 1,v_k,n\chi 1)$ for $k\in\Bbb N$, and
$v=\inf_{k\in\Bbb N}\sup_{j\ge k}v'_k$.   We do not
need to ask whether the operations here are being performed in 
$L^0(\frak A)$ or in $L^0(\frak B)$, and $v$ will belong to $L^0(\frak B)$.
Accordingly, now necessarily working in $L^0(\frak A)$, 
we shall have

\Centerline{$v\times\chi a
=\inf_{k\in\Bbb N}\sup_{j\ge k}v'_k\times\chi a$.}

\noindent Now observe that, for each $k$, 
$w_k=\sup_{j\ge k}|u-v_j|\times\chi a$ is defined in 
$L^1(\frak A,\bar\mu)$ and $\int w_k\le 2^{-k+1}$.   Set
$u'=\med(-n\chi 1,u,n\chi 1)$.
For $j\ge k$, 

$$\eqalign{|u'\times\chi a-v'_j\times\chi a|
&=|\med(-n\chi 1,u\times\chi a,n\chi 1)
   -\med(-n\chi 1,v_j\times\chi a,n\chi 1)|\cr
&\le|u-v_j|\times\chi a
\le w_k.\cr}$$

\noindent So, for any $m\in\Bbb N$,

$$\eqalign{u'\times\chi a-v\times\chi a
&=\sup_{k\in\Bbb N}\inf_{j\ge k}u'\times\chi a-v'_k\times\chi a\cr
&=\sup_{k\ge m}\inf_{j\ge k}u'\times\chi a-v'_k\times\chi a
\le\sup_{k\ge m}w_k,\cr
v\times\chi a-u'\times\chi a
&=\inf_{k\in\Bbb N}\sup_{j\ge k}v'_k\times\chi a-u'\times\chi a\cr
&\le\sup_{j\ge m}v'_k\times\chi a-u'\times\chi a
\le\sup_{j\ge m}w_k.\cr}$$

\noindent Putting these together,

\Centerline{$|u'\times\chi a-v\times\chi a|\le\sup_{j\ge m}w_k$}

\noindent for every $m\in\Bbb N$, and $u'\times\chi a=v\times\chi a$.
But this means that
$a\Bcap\Bvalue{u'>\alpha}=a\Bcap\Bvalue{v>\alpha}$;  at the same time,
because $-n<\alpha<n$, $\Bvalue{u'>\alpha}=\Bvalue{u>\alpha}$.

Thus we see that for every $a\in\frak A^f$ there is a $b\in\frak B$ such
that $a\Bcap(b\Bsymmdiff\Bvalue{u>\alpha})=0$.   It follows at once that
$\Bvalue{u>\alpha}$ belongs to the closure of $\frak B$, which is 
$\frak B$ itself.   As $\alpha$ is arbitrary, $u\in L^0(\frak B)$;  as
$u$ is arbitrary, $L^0(\frak B)$ is closed.
}%end of proof of 367R
     
\leader{367S}{Proposition} Let $E\subseteq\Bbb R$ be a Borel set, and
$h:E\to\Bbb R$ a continuous function.   Let $(\frak A,\bar\mu)$ be a
measure algebra, and $\bar h:Q_E\to L^0=L^0(\frak A)$ the associated
function, where
$Q_E=\{u:u\in L^0$, $\Bvalue{u\in E}=1\}$\cmmnt{ (364H)}.   Then
$\bar h$ is continuous for the topology of convergence in measure.
     
\proof{ (Compare 245Dd.)   Express $(\frak A,\bar\mu)$ as the measure
algebra of a measure space $(X,\Sigma,\mu)$.   Take
any $u\in Q_E$, any $a\in\frak A$ such that $\bar\mu a<\infty$, and any
$\epsilon>0$.
Express $u$ as $f^{\ssbullet}$ where $f:X\to\Bbb R$ is a measurable
function, and $a$ as $F^{\ssbullet}$ where $F\in\Sigma$.
Then $f(x)\in E$ a.e.($x$).   Set $\eta=\epsilon(2+\mu F)$.   For each $n\in\Bbb N$, write $E_n$ for
     
\Centerline{$\{t:t\in E$, $|h(s)-h(t)|\le\eta$ whenever
$s\in E$ and $|s-t|\le 2^{-n}\}$.}
     
\noindent Then $\sequencen{E_n}$ is a non-decreasing
sequence of Borel sets with union $E$, so there is an $n$ such that
$\mu\{x:x\in F$, $f(x)\notin E_n\}\le\eta$.
     
Now suppose that $v\in Q_E$ is such that
$\int|v-u|\wedge\chi a\le 2^{-n}\eta$.   Express $v$ as
$g^{\ssbullet}$ where
$g:X\to\Bbb R$ is a measurable function.   Then $g(x)\in E$ for almost
every $x$, and
     
\Centerline{$\int_F\min(1,|g(x)-f(x)|)\mu(dx)\le 2^{-n}\eta$,}
     
\noindent so $\mu\{x:x\in F,\,|f(x)-g(x)|>2^{-n}\}\le\eta$, and
     
$$\eqalign{\{x:x\in F,\,|&h(g(x))-h(f(x))|>\eta\}\cr
&\subseteq\{x:x\in F,\,f(x)\notin E_n\}
\cup\{x:g(x)\notin E\}\cr
&\qquad\qquad\qquad\qquad\cup\{x:x\in F,\,|f(x)-g(x)|>2^{-n}\}\cr}$$
     
\noindent has measure at most $2\eta$.   But this means that
     
\Centerline{$\int|\bar h(v)-\bar h(u)|\wedge\chi a
=\int_F\min(1,|hg(x)-hf(x)|)\mu(dx)\le 2\eta+\eta\mu F=\epsilon$.}
     
\noindent As $u$, $a$ and $\epsilon$ are arbitrary, $\bar h$ is
continuous.
}%end of proof of 367S
     
\leader{367T}{Intrinsic description of convergence in
\dvrocolon{measure}}\cmmnt{ It is a remarkable fact that the
topology of convergence in measure, not only on $L^0$ but on its
order-dense Riesz subspaces, can be described in terms of the Riesz
space structure alone, without referring at all to the underlying
measure algebra or to integration.   (Compare 324H.)   There is more
than one way of doing this.   As far as
I know, none is outstandingly convincing;  I present a formulation which
seems to me to exhibit some, at least, of the essence of the phenomenon.
     
\wheader{367T}{4}{2}{2}{72pt}
     
\noindent}{\bf Proposition} Let $(\frak A,\bar\mu)$ be a semi-finite
measure algebra, and $U$ an order-dense Riesz subspace of
$L^0=L^0(\frak A)$.   Suppose that $A\subseteq U$ and $u^*\in U$.   Then
$u^*$ belongs
to the closure of $A$ for the topology of convergence in measure iff
     
\inset{\noindent there is an order-dense Riesz subspace $V$ of $U$ such
that
     
\varinset{\noindent for every $v\in V^+$ there is a non-empty
downwards-directed $B\subseteq U$, with infimum $0$, such that
     
\varinset{\noindent for every $w\in B$ there is a $u\in A$ such that
     
\Centerline{$|u-u^*|\wedge v\le w$.}}}}
     
\proof{{\bf (a)} Suppose first that $u^*\in\overline{A}$.   Take $V$ to
be $U\cap L^1_{\bar\mu}$;  then $V$ is an order-dense Riesz
subspace of $L^0$, by 352Nc, and is therefore order-dense in $U$.
(This is where I use the hypothesis that $(\frak A,\bar\mu)$ is
semi-finite, so that $L^1_{\bar\mu}$ is order-dense in $L^0$, by 365Ga.)
     
Take any $v\in V^+$.   For each $n\in\Bbb N$, set
$a_n=\Bvalue{v>2^{-n}}\in\frak A^f$.   Because $u^*\in\overline{A}$,
there is a $u_n\in A$ such that $\bar\mu b_n\le 2^{-n}$, where
     
\Centerline{$b_n=a_n\Bcap\Bvalue{|u_n-u^*|>2^{-n}}
=\Bvalue{|u_n-u^*|\wedge v>2^{-n}}$.}
     
\noindent Set $c_n=\sup_{i\ge n}b_i$;  then $\bar\mu c_n\le 2^{-n+1}$
for each $n$, so $\inf_{n\in\Bbb N}c_n=0$ and $\inf_{n\in\Bbb N}w_n=0$
in $L^0$, where $w_n=v\times\chi c_n+2^{-n}\chi 1$.   Also
$|u_n-u^*|\wedge v\le w_n$ for each $n$.
     
The $w_n$ need not belong to $U$, so we cannot set $B=\{w_n:n\in\Bbb
N\}$.   But if instead we write
\Centerline{$B=\{w:w\in U,\,w\ge v\wedge w_n$ for some $n\in\Bbb N\}$,}
     
\noindent then $B$ is non-empty and downwards-directed (because
$\sequencen{w_n}$ is non-increasing);  and
$$\eqalignno{\inf B
&=v-\sup\{v-w:w\in B\}\cr
&=v-\sup\{w:w\in U,\,w\le (v-w_n)^+\text{ for some }n\in\Bbb N\}\cr
&=v-\sup_{n\in\Bbb N}(v-w_n)^+\cr
\noalign{\noindent (because $U$ is order-dense in $L^0$)}
&=0.\cr}$$
     
\noindent Since for every $w\in B$ there is an $n$ such that
$v\wedge|u_n-u^*|\le v\wedge w_n\le w$, $B$ witnesses that the condition
is satisfied.
     
\medskip
     
{\bf (b)} Now suppose that the condition is satisfied.   Fix
$a\in\frak A^f$ and $\epsilon>0$.   Because $V$ is order-dense in $U$ and therefore in
$L^0$, there is a $v\in V$ such that $0\le v\le\chi a$ and
$\int v\ge\bar\mu a-\epsilon$.   Let $B$ be a downwards-directed set,
with infimum $0$, such that for every $w\in B$ there is a $u\in A$ with
$v\wedge|u-u^*|\le w$.   Then there is a $w\in B$ such that
$\int w\wedge v\le\epsilon$.   Now there is a $u\in A$ such that
$|u-u^*|\wedge v\le w$, so that
     
\Centerline{$\int|u-u^*|\wedge\chi a\le\epsilon+\int|u-u^*|\wedge
v
\le\epsilon+\int w\wedge v\le 2\epsilon$.}
     
\noindent As $a$ and $\epsilon$ are arbitrary, $u^*\in\overline{A}$.
}%end of proof of 367T
     
\leader{*367U}{Theorem} Let $(\frak A,\bar\mu)$ be a semi-finite
measure algebra;  write $L^1$ for $L^1(\frak A,\bar\mu)$.   Let
$P:(L^1)^{**}\to L^1$ be the linear operator corresponding to the band
projection from
$(L^1)^{**}=(L^1)^{\times\sim}$ onto $(L^1)^{\times\times}$ and the
canonical isomorphism between $L^1$ and $(L^1)^{\times\times}$.   For
$A\subseteq L^1$ write $A^*$ for the weak* closure of the image of
$A$ in $(L^1)^{**}$.   Then for every $A\subseteq L^1$
     
\Centerline{$P[A^*]\subseteq\overline{\Gamma(A)}$,}
     
\noindent where $\Gamma(A)$ is the convex hull of $A$ and
$\overline{\Gamma(A)}$ is the closure of $\Gamma(A)$ in
$L^0=L^0(\frak A)$ for the topology of convergence in measure.
     
\proof{{\bf (a)} The statement of the theorem includes a number of
assertions:  that $(L^1)^*=(L^1)^{\times}$;  that
$(L^1)^{**}=((L^1)^*)^{\sim}$;  that the natural embedding of $L^1$ into
$(L^1)^{**}=(L^1)^{\times\sim}$ identifies $L^1$ with
$(L^1)^{\times\times}$;  and that $(L^1)^{\times\times}$ is a band in
$(L^1)^{\times\sim}$.   For proofs of these see 365C, 356D and 356B.
     
Now for the new argument.   First, observe that the statement of the
theorem involves the measure algebra $(\frak A,\bar\mu)$ and the space
$L^0$ only in the definition of `convergence in measure';  everything
else depends only on the Banach lattice structure of $L^1$.   And since
we are concerned only with the question of whether members of $P[A^*]$,
which is surely a subset of $L^1$, belong to $\overline{\Gamma(A)}$,
367T shows that this also can be answered in terms of the Riesz space
structure of $L^1$.   What this means is that we can suppose that
$(\frak A,\bar\mu)$ is localizable.   \Prf\ Let
$(\widehat{\frak A},\tilde\mu)$ be the localization of
$(\frak A,\bar\mu)$ (322Q).   The
natural expression of $\frak A$ as an order-dense subalgebra of
$\widehat{\frak A}$ identifies
$\frak A^f=\{a:a\in\frak A,\,\bar\mu a<\infty\}$ with
$\widehat{\frak A}^f$ (322P), so that $L^1_{\bar\mu}$ 
becomes identified with $L^1_{\tilde\mu}$, by 365Od.   
Thus we can think of $L^1$ as $L^1_{\tilde\mu}$,
and $(\widehat{\frak A},\tilde\mu)$ is localizable.\ \Qed
     
\medskip
     
{\bf (b)} Take $\phi\in A^*$ and set $u_0=P\phi$;  I have to show that
$u_0\in\overline{\Gamma(A)}$.   Write $R$ for the canonical map from
$L^1$ to $(L^1)^{**}$, so that $\phi$ belongs to the weak* closure of
$R[A]$.
     
Consider first the case $u_0=0$.   Take any $c\in\frak A^f$ and
$\epsilon>0$.   We know that $(L^1)^*=(L^1)^{\sim}=(L^1)^{\times}$ can
be identified with $L^{\infty}=L^{\infty}(\frak A)$ (365Mc), so that
$\phi\in (L^{\infty})^*=(L^{\infty})^{\sim}$ must be in the band
orthogonal to $(L^{\infty})^{\times}$.   Now we can identify
$(L^{\infty})^{\sim}$ with the Riesz space $M$ of bounded additive
functionals on $\frak A$, and if we do so then $(L^{\infty})^{\times}$
corresponds to the space $M_{\tau}$ of completely additive functionals
(363K).   Writing $P_{\tau}:M\to M_{\tau}$ for the band projection, we
must have $P_{\tau}(\nu)=0$, where $\nu\in M$ is defined by setting 
$\nu a=\phi(\chi a)$ for each $a\in\frak A$;  consequently
$P_{\tau}(|\nu|)=0$ and
there is an upwards-directed family $C\subseteq\frak A$, with supremum
$1$, such that $|\nu|(a)=0$ for every $a\in C$ (362D).   Since
$\bar\mu c=\sup_{a\in C}\bar\mu(a\Bcap c)$, there is an $a\in C$ such
that $\bar\mu(c\Bsetminus a)\le\epsilon$.
     
Consider the map $Q:L^1\to L^1$ defined by setting $Qw=w\times\chi a$
for every $w\in L^1$.   Then its adjoint $Q':L^{\infty}\to L^{\infty}$
(3A5Ed) can be defined by the same formula:  $Q'v=v\times\chi a$ for
every $v\in L^{\infty}$.   Since $|\phi|\in (L^{\infty})^{\sim}$
corresponds to $|\nu|\in M$, we have
     
\Centerline{$|\phi(Q'v)|\le\|v\|_{\infty}|\phi|(\chi
a)=\|v\|_{\infty}|\nu|(a)=0$}
     
\noindent for every $v\in L^{\infty}$, and $Q''\phi=0$, where
$Q'':(L^{\infty})^*\to(L^{\infty})^*$ is the adjoint of $Q'$.   Since
$Q''$ is continuous for the weak* topology on $(L^{\infty})^*$, $0\in
\overline{Q''[R[A]]}$, where $\overline{Q''R[A]}$ is the closure for the
weak* topology of $(L^{\infty})^*$.   But of course $Q''R=RQ$, while
the weak* topology of $(L^{\infty})^*$ corresponds, on the image
$R[L^1]$ of $L^1$, to the weak topology of $L^1$;  so that $0$ belongs
to the closure of $Q[A]$ for the weak topology of $L^1$.
     
Because $Q$ is linear, $Q[\Gamma(A)]$ is convex.   Since $0$ belongs to
the closure of $Q[\Gamma(A)]$ for the weak topology of $L^1$, it belongs
to the closure of $Q[\Gamma(A)]$ for the norm topology (3A5Ee).   So
there is a $w\in\Gamma(A)$ such that $\|w\times\chi a\|_1\le\epsilon^2$.
But this means that $\bar\mu(a\Bcap\Bvalue{|w|\ge\epsilon})\le\epsilon$
and $\bar\mu(c\Bcap\Bvalue{|w|\ge\epsilon})\le2\epsilon$.   Since $c$
and
$\epsilon$ are arbitrary, $0\in\overline{\Gamma(A)}$.
     
\medskip
     
{\bf (c)} This deals with the case $u_0=0$.   Now the general case
follows at once if we set $B=A-u_0$ and observe that $\phi-Ru_0\in B^*$,
so
     
\Centerline{$0=P(\phi-Ru_0)\in\overline{\Gamma(B)}
=\overline{\Gamma(A)-u_0}=\overline{\Gamma(A)}-u_0$.}
}%end of proof of 367U
     
\cmmnt{\medskip
     
\noindent{\bf Remark} This is a version of a theorem from {\smc
Bukhvalov 95}.
}
     
\leader{*367V}{Corollary} Let $(\frak A,\bar\mu)$ be a localizable
measure algebra.  Let $\Cal C$ be a family of convex subsets of
$L^0=L^0(\frak A)$, all closed for the topology of convergence in
measure, with the finite intersection property, and suppose that for
every non-zero $a\in\frak A$ there are a
non-zero $b\subseteq a$ and a $C\in\Cal C$ such that
$\sup_{u\in C}\int_b|u|<\infty$.   Then $\bigcap\Cal C\ne\emptyset$.
     
\proof{ Because $\Cal C$ has the finite intersection property, there is
an ultrafilter $\Cal F$ on $L^0$ including $\Cal C$.   Set
     
\Centerline{$I
=\{a:a\in\frak A,\,\inf_{F\in\Cal F}\sup_{u\in
F}\int_a|u|<\infty\}$;}
     
\noindent because $\Cal F$ is a filter, $I$ is an ideal in $\frak A$,
and the condition on $\Cal C$ tells us that $I$ is order-dense.   For
each $a\in I$, define $Q_a:L^0\to L^0$ by setting $Q_au=u\times\chi a$.
Then there is an $F\in\Cal F$ such that $Q_a[F]$ is a norm-bounded set
in $L^1$, so $\phi_a=\lim_{u\to\Cal F}RQ_au$ is defined in
$(L^{\infty})^*$ for the weak* topology on $(L^{\infty})^*$, writing
$R$ for the canonical map from $L^1$ to $(L^{\infty})^*\cong(L^1)^{**}$.
If $P:(L^{\infty})^*\to L^1$ is the map corresponding to the
band projection $\tilde P$ from $(L^{\infty})^{\sim}$ onto
$(L^{\infty})^{\times}$, as in 367U, and $C\in\Cal C$, then 367U tells
us that $P(\phi_a)$
must belong to the closure of the convex set $Q_a[C]$ for the topology
of convergence in measure.   Moreover, if $a\Bsubseteq b\in I$, so that
$Q_a=Q_aQ_b$, then $P(\phi_a)=Q_aP(\phi_b)$.   \Prf\ $Q_a\restr L^1$ is
a band projection on $L^1$, so its adjoint $Q'_a$ is a band projection
on $L^{\infty}\cong(L^1)^{\sim}$ (356C) and $Q''_a$ is a band
projection on $(L^{\infty})^*\cong(L^{\infty})^{\sim}$.   This means
that $Q''_a$ will commute with $\tilde P$ (352Sb).   But also
$Q''_a$ is continuous for the weak* topology of $(L^{\infty})^*$, so
     
\Centerline{$Q''_a(\phi_b)=\lim_{u\to\Cal F}Q''_aRQ_bu
=\lim_{u\to\Cal F}RQ_aQ_bu=\phi_a$,}
     
\noindent and
     
\Centerline{$P(\phi_a)=R^{-1}\tilde P(\phi_a)
=R^{-1}\tilde PQ''_a(\phi_b)=R^{-1}Q''_a\tilde P(\phi_b)
=Q_aR^{-1}\tilde P(\phi_b)=Q_aP(\phi_b)$.   \Qed}
     
Generally, if $a$, $b\in I$, then
     
\Centerline{$Q_aP(\phi_b)=Q_aQ_bP(\phi_b)=Q_{a\Bcap b}P(\phi_b)
=P(\phi_{a\Bcap b})=Q_bP(\phi_a)$.}
     
\noindent 
What this means is that if we take a partition $D$ of unity included in
$I$ (313K), so that $L^0\cong\prod_{d\in D}L^0(\frak A_d)$ (315F(iii),
364R), and define $w\in L^0$ by saying that $w\times\chi d=P(\phi_d)$
for every $d\in D$, then we shall have
$w\times\chi a\times\chi d=P(\phi_a)\times\chi d$ whenever $a\in I$ and 
$d\in D$, so $w\times\chi a=P(\phi_a)$ for
every $a\in I$.   But now, given $a\in\frak A^f$ and $\epsilon>0$ and
$C\in\Cal C$, there is a $b\in I$ such that
$\bar\mu(a\Bsetminus b)\le\epsilon$;
$w\times\chi b\in\overline{Q_b[C]}$, so there is a
$u\in C$ such that
$\bar\mu(b\Bcap\Bvalue{|w-u|\ge\epsilon})\le\epsilon$;  and
$\bar\mu(a\Bcap\Bvalue{|w-u|\ge\epsilon})\le2\epsilon$.   As $a$ and
$\epsilon$ are arbitrary and $C$ is closed, $w\in C$;  as $C$ is
arbitrary, $w\in\bigcap\Cal C$ and $\bigcap\Cal C\ne\emptyset$.
}%end of proof of 367V
     
\leader{*367W}{\dvrocolon{Independence}}\cmmnt{ I have given myself
very little room in this chapter to discuss stochastic independence.
There are direct translations of results from \S272 in 364Xe-364Xf.
However the language here is adapted to a significant result not
presented in \S272.   I had better begin by repeating a definition from
364Xe.}   Let $(\frak A,\bar\mu)$ be a probability algebra.   Then a
family $\familyiI{u_i}$ in $L^0(\frak A)$ is {\bf stochastically
independent}
if $\bar\mu(\inf_{i\in J}\Bvalue{u_i>\alpha_i})
=\prod_{i\in J}\bar\mu\Bvalue{u_i>\alpha_i}$ whenever $J\subseteq I$ is
a non-empty finite set and $\alpha_i\in\Bbb R$ for every $i\in I$.
\cmmnt{(The direct translation of the definition in 272Ac would rather
be `$\bar\mu(\inf_{i\in J}\Bvalue{u_i\le\alpha_i})
=\prod_{i\in J}\bar\mu\Bvalue{u_i\le\alpha_i}$ whenever $J\subseteq I$
is a non-empty finite set and $\alpha_i\in\Bbb R$ for every $i\in I$',
intepreting $\Bvalue{u_i\le\alpha_i}$ as in 364Xa.   Of course 272F
tells us that this comes to the same thing.)}   \cmmnt{Now the new
fact is the following.}
     
\medskip
     
\noindent{\bf Proposition} Let $(\frak A,\bar\mu)$ be a probability
algebra, and $I$ any set.   Give $L^0=L^0(\frak A)$ its topology of
convergence in measure.   Then the collection of independent families
$\familyiI{u_i}$ is a closed set in $(L^0)^I$.
     
\proof{ Suppose that $\familyiI{u_i}\in(L^0)^I$ is not independent.
Then there are a finite set $J\subseteq I$ and a family
$\family{i}{J}{\alpha_i}$ of real numbers such that
$\bar\mu(\inf_{i\in J}\Bvalue{u_i>\alpha_i})
\ne\prod_{i\in J}\bar\mu\Bvalue{u_i>\alpha_i}$.   Set
$a_i=\Bvalue{u_i>\alpha_i}$ for each $i$.   Let $\delta>0$ be such that
$\gamma\ne\prod_{i\in J}\gamma_i$ whenever
$|\gamma-\bar\mu(\inf_{i\in J}a_i)|\le 2\delta\#(J)$ and
$|\gamma_i-\bar\mu a_i|\le 2\delta$ for every
$i\in J$.   Let $\eta\in\ocint{0,1}$ be such that
$\bar\mu\Bvalue{u_i>\alpha_i+2\eta}
\ge\bar\mu a_i-\delta$ for every $i\in J$.
     
Now if $\familyiI{v_i}\in(L^0)^I$ and
$\bar\mu\Bvalue{|v_i-u_i|>\eta}\le\delta$ for each $i\in J$,
$\familyiI{v_i}$ is not independent.   \Prf\ For each $i\in J$, consider
$b_i=\Bvalue{v_i>\alpha_i+\eta}$, $a'_i=\Bvalue{u_i>\alpha_i+2\eta}$.
We have
     
\Centerline{$a'_i=\Bvalue{u_i>\alpha_i+2\eta}
\Bsubseteq\Bvalue{v_i>\alpha_i+\eta}\Bcup\Bvalue{u_i-v_i>\eta}
\Bsubseteq b_i\Bcup\Bvalue{|u_i-v_i|>\eta}$}
     
\noindent (364Ea), and
     
\Centerline{$b_i=\Bvalue{v_i>\alpha_i+\eta}
\Bsubseteq\Bvalue{u_i>\alpha_i}\Bcup\Bvalue{v_i-u_i>\eta}
\Bsubseteq a_i\Bcup\Bvalue{|v_i-u_i|>\eta}$,}
     
\noindent so
     
\Centerline{$b_i\Bsymmdiff a_i=(b_i\Bsetminus a_i)\Bcup(a_i\Bsetminus
b_i)
\Bsubseteq\Bvalue{|v_i-u_i|>\eta}\Bcup(a_i\Bsetminus a'_i)$}
     
\noindent has measure at most $2\delta$.   It follows that
$(\inf_{i\in J}b_i)\Bsymmdiff(\inf_{i\in J}a_i)$ has measure at most
$2\delta\#(J)$, and
$|\bar\mu(\inf_{i\in J}b_i)\penalty-100-\bar\mu(\inf_{i\in J}a_i)|\le 2\delta\#(J)$.
At the same time, for each $i\in J$,
$|\bar\mu b_i-\bar\mu a_i|\le 2\delta$.   By the choice of $\delta$,
$\bar\mu(\inf_{i\in J}b_i)\ne\prod_{i\in J}\bar\mu b_i$, and
$\familyiI{v_i}$ is not independent.\ \Qed
     
This shows that the set of non-independent families is open in
$(L^0)^I$, so that the set of independent families is closed, as
claimed.
}%end of proof of 367W
     
\exercises{\leader{367X}{Basic exercises $\pmb{>}$(a)}
%\spheader 367Xa
Let $P$ be a lattice.   (i) Show that if $p\in P$ and $\sequencen{p_n}$
is a non-decreasing sequence in $P$, then $\sequencen{p_n}$ is
order*-convergent to $p$ iff $p=\sup_{n\in\Bbb N}p_n$.
(ii) Suppose that $\sequencen{p_n}$ is a sequence in
$P$ order*-converging to $p\in P$.   Show that
$p=\sup_{n\in\Bbb N}p\wedge p_n=\inf_{n\in\Bbb N}p\vee p_n$.
(iii) Let $\sequencen{p_n}$, $\sequencen{q_n}$ be two sequences in $P$
which are order*-convergent to $p$, $q$ respectively.   Show that if
$p_n\le q_n$ for every $n$ then $p\le q$.
(iv) Let $\sequencen{p_n}$ be a sequence in
$P$.   Show that $\sequencen{p_n}$ order*-converges to
$p\in P$ iff $\sequencen{p_n\vee p}$ and $\sequencen{p_n\wedge p}$
order*-converge to $p$.
%367B
     
\spheader 367Xb Let $P$ and $Q$ be lattices, and $f:P\to Q$ an
order-preserving function.   Suppose that $\sequencen{p_n}$ is an
order-bounded sequence which
order*-converges to $p$ in $P$.   Show that $\sequencen{f(p_n)}$
order*-converges to $f(p)$ in $Q$ if {\it either} $f$ is
order-continuous {\it or} $P$ is Dedekind $\sigma$-complete and $f$ is
sequentially order-continuous.
%367B
     
\spheader 367Xc Let $P$ be {\it either} a Boolean algebra {\it or} a
Riesz space.   Suppose that $\sequencen{p_n}$ is a sequence in $P$ such
that $\sequencen{p_{2n}}$ and $\sequencen{p_{2n+1}}$ are both
order*-convergent to $p\in P$.   Show that $\sequencen{p_n}$ is
order*-convergent to $p$.   \Hint{313B, 352E.}
%367B
     
\sqheader 367Xd Let $\frak A$ be a Boolean algebra and
$\sequencen{a_n}$, $\sequencen{b_n}$ two sequences in $\frak A$
order*-converging to $a$, $b$ respectively.   Show that
$\sequencen{a_n\Bcup b_n}$, $\sequencen{a_n\Bcap b_n}$,
$\sequencen{a_n\Bsetminus b_n}$, $\sequencen{a_n\Bsymmdiff b_n}$
order*-converge to $a\Bcup b$, $a\Bcap b$, $a\Bsetminus b$ and
$a\Bsymmdiff b$ respectively.
%367C
     
\spheader 367Xe Let $\frak A$ be a Boolean algebra and $\sequencen{a_n}$
a sequence in $\frak A$.   Show that $\sequencen{a_n}$ does not
order*-converge to $0$ iff there is a non-zero $a\in\frak A$ such
that $a=\sup_{i\ge n}a\wedge a_i$ for every $n\in\Bbb N$.
%367C
     
\sqheader 367Xf(i) Let $U$ be a Riesz space and $\sequencen{u_n}$ an
order*-convergent sequence in $U^+$ with limit $u$.
Show that $h(u)\le\liminf_{n\to\infty}h(u_n)$ for every $h\in
(U^{\times})^+$.
(ii) Let $U$ be a Riesz space and $\sequencen{u_n}$ an
order-bounded order*-convergent sequence in $U$ with limit $u$.
Show that $h(u)=\lim_{n\to\infty}h(u_n)$ for every $h\in U^{\times}$.
(Compare 356Xd.)
%367D
     
\sqheader 367Xg Let $U$ be a Riesz space with a Fatou norm $\|\,\,\|$.
(i) Show that if $\sequencen{u_n}$ is an order*-convergent sequence
in $U$ with limit $u$, then $\|u\|\le\liminf_{n\to\infty}\|u_n\|$.
\Hint{$\sequencen{|u_n|\wedge|u|}$ is order*-convergent to $|u|$.}
(ii) Show that if $\sequencen{u_n}$ is a norm-convergent sequence in $U$
it has an order*-convergent subsequence.   \Hint{if
$\sum_{n=0}^{\infty}\|u_n\|<\infty$ then $\sequencen{u_n}$
order*-converges to $0$.}
%367D
     
\spheader 367Xh Let $U$ and $V$ be Archimedean Riesz spaces and
$T:U\to V$ an order-continuous Riesz homomorphism.   Show that if
$\sequencen{u_n}$ is a sequence in $U$ which order*-converges to
$u\in U$, then $\sequencen{Tu_n}$ order*-converges to $Tu$ in $V$.
%367E
     
\spheader 367Xi Let $\frak A$ be a Boolean algebra and $\frak B$ 
a regularly embedded subalgebra.   
Show that if $\sequencen{b_n}$ is a sequence
in $\frak B$ and $b\in\frak B$, then $\sequencen{b_n}$
order*-converges to $b$ in $\frak B$ iff it order*-converges to
$b$ in $\frak A$.
%367E
     
\spheader 367Xj Let $\frak A$ be a Dedekind $\sigma$-complete Boolean
algebra and $\sequencen{u_n}$, $\sequencen{v_n}$ two sequences in
$L^0(\frak A)$ which are order*-convergent to $u$, $v$
respectively.   Show that $\sequencen{u_n\times v_n}$
order*-converges
to $u\times v$.   Show that if $u$, $u_n$ all have
multiplicative inverses $u^{-1}$, $u_n^{-1}$ then
$\sequencen{u_n^{-1}}$ order*-converges to $u^{-1}$.
%367F
     
\spheader 367Xk Let $\frak A$ be a Dedekind $\sigma$-complete Boolean
algebra and $\Cal I$ a $\sigma$-ideal of $\frak A$.   Show that for any
$\sequencen{a_n}$, $a\in\frak A$, $\sequencen{a_n^{\ssbullet}}$
order*-converges to $a^{\ssbullet}$ in $\frak A/\Cal I$ iff
$\inf_{n\in\Bbb N}\sup_{m\ge n}a_m\symmdiff a\in\Cal I$.
%367F
     
\sqheader 367Xl Let $\frak A$ be a Dedekind $\sigma$-complete Boolean
algebra, and $\sequencen{h_n}$ a sequence of Borel measurable functions
from $\Bbb R$ to itself such that $h(t)=\lim_{n\to\infty}h_n(t)$ is
defined for every $t\in\Bbb R$.   Show that $\sequencen{\bar h_n(u)}$
order*-converges to $\bar h(u)$ for every $u\in L^0=L^0(\frak A)$,
where $\bar h_n$, $\bar h:L^0\to L^0$ are defined as in 364H.
%367F
     
\spheader 367Xm Let $(\frak A,\bar\mu)$ be a measure algebra, and
$\sequencen{u_n}$ a sequence in $L^1=L^1_{\bar\mu}$ which is
order*-convergent to $u\in L^1$.   Show that
$\sequencen{u_n}$ is norm-convergent to $u$ iff $\{u_n:n\in\Bbb N\}$ is
uniformly integrable iff $\|u\|_1=\lim_{n\to\infty}\|u_n\|_1$.
\Hint{245H, 246J.}
%367H
     
\spheader 367Xn Let $U$ be an $L$-space and $\sequencen{u_n}$ a
norm-bounded sequence in $U$.   Show that there are a $v\in U$ and a
subsequence $\sequencen{v_n}$ of $\sequencen{u_n}$ such that
$\sequencen{\bover1{n+1}\sum_{i=0}^nw_i}$ order*-converges to $v$ for
every subsequence $\sequencen{w_n}$ of $\sequencen{v_n}$.   \Hint{276H.}
%367J
     
\spheader 367Xo Let $(\frak A,\bar\mu)$ be a measure algebra and
$p\in\coint{1,\infty}$.   For $v\in (L^p)^+=(L^p_{\bar\mu})^+$
define $\rho_v:L^0\times L^0\to\coint{0,\infty}$ by setting
$\rho_v(u_1,u_2)=\||u_1-u_2|\wedge v\|_p$ for all $u_1$, $u_2\in U$.
Show that each $\rho_v$ is a pseudometric and that the topology on
$L^0(\frak A)$ defined by $\{\rho_v:v\in(L^p)^+\}$ is the topology
of convergence in measure.
%367L
     
\sqheader 367Xp Let $(\frak A,\bar\mu)$ be a measure algebra and give
$L^0(\frak A)$ its topology of convergence in measure.   Show that
$u\mapsto |u|$, $(u,v)\mapsto u\vee v$ and $(u,v)\mapsto u\times v$ are
continuous.
%367M  useful for 444F
     
\spheader 367Xq Let $(\frak A,\bar\mu)$ be a $\sigma$-finite measure
algebra.   Suppose we have a double sequence
$\family{(i,j)}{\Bbb N\times\Bbb N}{u_{ij}}$ in $L^0=L^0(\frak A)$ such that $\sequence{j}{u_{ij}}$ order*-converges to $u_i$ in $L^0$ for each
$i$, while $\sequence{i}{u_i}$ order*-converges to $u$.   Show that
there is a strictly
increasing sequence $\sequence{i}{n(i)}$ such that
$\sequence{i}{u_{i,n(i)}}$ order*-converges to $u$.
%367M
     
\spheader 367Xr Let $(X,\Sigma,\mu)$ be a semi-finite measure space.
Show that $L^0(\mu)$ is separable for the topology of convergence in
measure iff $\mu$ is $\sigma$-finite and has countable Maharam type.
(Cf.\ 365Xp.)
%367M
     
\spheader 367Xs Let $(\frak A,\bar\mu)$ be a measure algebra.   (i) Show
that if $\sequencen{a_n}$ is order*-convergent to $a\in\frak A$, then
$\sequencen{a_n}\to a$ for the measure-algebra topology.
(ii) Show that if $(\frak A,\bar\mu)$ is $\sigma$-finite, then
($\alpha$) a sequence converges to $a$ for the topology of $\frak A$ iff
every subsequence has a sub-subsequence which is order*-convergent to
to $a$ ($\beta$) a set $F\subseteq\frak A$ is closed for the topology of
$\frak A$ iff $a\in F$ whenever there is a sequence $\sequencen{a_n}$ in
$F$ which is order*-convergent to $a\in\frak A$.
%367P
     
\spheader 367Xt Let $(\frak A,\bar\mu)$ be a measure algebra which is
not $\sigma$-finite.   Show that there is a set
$A\subseteq L^0(\frak A)$ such that the limit of any order*-convergent sequence in $A$
belongs to $A$, but $A$ is not closed for the topology of convergence in
measure.
%367P
     
\spheader 367Xu Let $U$ be a Banach lattice with an order-continuous
norm.   (i) Show that a sequence $\sequencen{u_n}$ is norm-convergent to
$u\in U$ iff every subsequence has a sub-subsequence which is
order-bounded and order*-convergent to $u$.   (ii) Show that a set
$F\subseteq U$ is closed for
the norm topology iff $u\in F$ whenever there is an order-bounded
sequence $\sequencen{u_n}$ in $F$ order*-converging to $u\in U$.
%367P
     
\spheader 367Xv Let $(\frak A,\bar\mu)$ be a probability algebra.   For
$u\in L^0=L^0(\frak A)$ let $\nu_u$ be the distribution of $u$ (364Xd).
Show that $u\mapsto\nu_u$ is continuous when $L^0$ is given the topology
of convergence in measure and the space of probability distributions on
$\Bbb R$ is given the vague topology (274Ld).
%367P sort of
     
\spheader 367Xw Let $(\frak A,\bar\mu)$ be a probability algebra and
$\sequencen{u_n}$ a stochastically independent sequence in
$L^0(\frak A)$, all
with the Cauchy distribution $\nu_{C,1}$ with centre $0$ and
scale parameter
$1$ (285Xm).   For each $n$ let $C_n$ be the convex hull of
$\{u_i:i\ge n\}$, and $\overline{C_n}$ its closure for the topology of
convergence in measure.   Show that every $u\in\overline{C_0}$ has
distribution $\nu_{C,1}$.   \Hint{consider first $u\in C_0$.}   Show
that $\overline{C_0}$ is bounded for the topology of convergence in
measure.   Show that $\bigcap_{n\in\Bbb N}\overline{C_n}=\emptyset$.
%367V
     
\spheader 367Xx If $U$ is a linear space and $C\subseteq U$ is a convex
set, a function $f:C\to\Bbb R$ is {\bf convex} if
$f({\alpha}x+(1-{\alpha})y)\le {\alpha}f(x)+(1-{\alpha})f(y)$
whenever $x$, $y\in C$ and $\alpha\in[0,1]$.
Let $(\frak A,\bar\mu)$ be a localizable measure algebra and
$C\subseteq L^1_{\bar\mu}$ a non-empty convex
norm-bounded set which is closed in $L^0(\frak A)$ for the topology of
convergence in measure.   Show that any convex function $f:C\to\Bbb R$
which is lower semi-continuous for the topology of convergence in
measure is bounded below and attains its infimum.
%367V
     
\spheader 367Xy Let $(\frak A,\bar\mu)$ be the measure algebra of
Lebesgue measure on $[0,1]$.   Show that there are a sequence
$\sequencen{u_n}$ in $L^1=L^1_{\bar\mu}$ and $u$, $v\in L^1$ such
that $u_n$ and $v$ are independent for every $n$, $\sequencen{u_n}$
converges weakly to $u$, but $u$ and $v$ are not independent.
%367W

\spheader 367Xz\dvAnew{2009}
Let $(\frak A,\bar\mu)$ be a measure algebra, and
give $L^0=L^0(\frak A)$ its topology of convergence in measure.   
(i) Show that
a set $A\subseteq L^0$ is bounded in the sense of 3A5N iff for every
$a\in\frak A^f$ and $\epsilon>0$ there is an $n\in\Bbb N$ such that
$\bar\mu(a\Bcap\Bvalue{|u|>n})\le\epsilon$ for every $u\in A$.
(ii) Show that if $(\frak A,\bar\mu)$ is semi-finite, then a set
$A\subseteq L^0$ is bounded in this sense iff $\{\alpha_nx_n:n\in\Bbb N\}$
is order-bounded for every sequence $\sequencen{x_n}$ in $A$ and every
sequence $\sequencen{\alpha_n}$ in $\Bbb R$ converging to $0$.
%??
     
\leader{367Y}{Further exercises (a)}
%\spheader 367Ya
Give an example of an Archimedean Riesz space $U$ and an
order-bounded sequence $\sequencen{u_n}$ in $U$ which is
order*-convergent to $0$, but such that there is no non-increasing
sequence $\sequencen{v_n}$, with infimum $0$, such that $u_n\le v_n$ for
every $n\in\Bbb N$.
%367A
     
\spheader 367Yb Let $P$ be any lattice.  (i) Show that there is a
topology on $P$ for which a set $A\subseteq P$ is closed iff $p\in A$
whenever there is a sequence in $A$ which is order*-convergent to
$p$.   Show that any closed set for this topology is sequentially
order-closed.   (ii) Now let $Q$ be another lattice, with the
topology defined in the same way, and $f:P\to Q$ an order-preserving
function.  Show that if $f$ is topologically
continuous it is sequentially order-continuous.
%367A
     
\spheader 367Yc Give an example of a distributive lattice $P$ with $p$, $q\in P$ and a sequence $\sequencen{p_n}$, order*-convergent to $p$, such that $\sequencen{p_n\wedge q}$ is not order*-convergent to
$p\wedge q$.
%367B
     
\spheader 367Yd Let us say that a lattice $P$ is {\bf
$(2,\infty)$-distributive} if ($\alpha$) whenever $A$, $B\subseteq P$
are non-empty sets with infima $p$, $q$ respectively, then $\inf\{a\vee
b:a\in A,\,b\in B\}=p\vee q$
($\beta$) whenever $A$, $B\subseteq P$ are non-empty sets with suprema
$p$, $q$ respectively, then $\sup\{a\wedge b:a\in A,\,b\in B\}=p\wedge
q$.   Show that, in this case, if $\sequencen{p_n}$ order*-converges
to $p$ and $\sequencen{q_n}$ order*-converges to $q$,
$\sequencen{p_n\vee q_n}$ order*-converges to $p\vee q$.
%367C
     
\spheader 367Ye(i) Give an example of a Riesz space $U$ with an
order-dense Riesz subspace $V$ of $U$ and a sequence $\sequencen{v_n}$
in $V$ such that
$\sequencen{v_n}$ order*-converges to $0$ in $V$ but does not
order*-converge in $U$.
(ii) Give an example of a Riesz space $U$ with an order-dense Riesz
subspace $V$ of $U$ and a sequence $\sequencen{v_n}$ in $V$,
order-bounded in $V$, such that
$\sequencen{v_n}$ order*-converges to $0$ in $U$ but does not
order*-converge in $V$.
%367E mt36bits
     
\spheader 367Yf Let $U$ be an Archimedean $f$-algebra.   Show that if
$\sequencen{u_n}$, $\sequencen{v_n}$ are sequences in $U$
order*-converging to $u$, $v$ respectively, then
$\sequencen{u_n\times v_n}$ order*-converges to $u\times v$.
%367F, 367Xj
     
\spheader 367Yg Let $\frak A$ be a Dedekind $\sigma$-complete Boolean
algebra and $r\ge 1$.   Let $E\subseteq\BbbR^r$ be a Borel set and
write $Q_E=\{(u_1,\ldots,u_r):\Bvalue{(u_1,\ldots,u_r)\in E}=1\}
\subseteq L^0(\frak A)^r$ (364Yb).   Let $h:E\to\Bbb R$ be a
continuous function and $\bar h:Q_E\to L^0=L^0(\frak A)$ the 
corresponding map
(364Yc).   Show that if $\sequencen{\tbf{u}_n}$ is a sequence in $Q_E$
which is order*-convergent to $\tbf{u}\in Q_E$ (in the lattice
$(L^0)^r$), then $\sequencen{\bar h(\tbf{u}_n)}$ is
order*-convergent to $\bar h(\tbf{u})$.
%367G
     
\spheader 367Yh Let $X$ be a completely regular Baire space (definition:
314Yd), and $\sequencen{u_n}$ a sequence in $C(X)$.   Show
that $\sequencen{u_n}$ order*-converges to $0$ in $C(X)$ iff
$\{x:\limsup_{n\to\infty}|u_n(x)|>0\}$ is meager in $X$.
%367J
     
\spheader 367Yi(i) Give an example of a sequence $\sequencen{u_n}$ in
$C([0,1])$ such that $\lim_{n\to\infty}u_n(x)=0$ for every $x\in[0,1]$,
but $\{u_n:n\in\Bbb N\}$ is not order-bounded in $C([0,1])$.   (ii) Give
an example of an order-bounded sequence $\sequencen{u_n}$ in $C(\Bbb Q)$
such that $\lim_{n\to\infty}u_n(q)=0$ for every $q\in\Bbb Q$, but
$\sup_{i\ge
n}u_i=\chi\Bbb Q$ in $C(\Bbb Q)$ for every $n\in\Bbb N$.   (iii) Give an
example of a sequence $\sequencen{u_n}$ in $C([0,1])$ such that
$\sequencen{u_n}$ order*-converges to $0$ in $C([0,1])$, but
$\lim_{n\to\infty}u_n(q)>0$ for every $q\in\Bbb Q\cap[0,1]$.
%367Yh, 367J
     
\spheader 367Yj Write out an alternative proof of 367J/367Yh based on
the fact that, for a Baire space $X$, $C(X)$ can be identified with an
order-dense Riesz subspace of a quotient of the space of
$\widehat{\Cal B}$-measurable functions, where $\widehat{\Cal B}$ is the 
Baire-property algebra of $X$.
%367J, 367Yh
     
\spheader 367Yk Let $\frak A$ be a ccc \wsid\
Boolean algebra.   Show that there is a topology on $\frak A$ such that
the closure of any $A\subseteq\frak A$ is precisely the set of limits of
order*-convergent sequences in $A$.
%367K see \S393
     
\spheader 367Yl Give an example of a set $X$ and a double sequence
$\langle u_{mn}\rangle_{m,n\in\Bbb N}$ in $\Bbb R^X$ such that
$\lim_{n\to\infty}u_{mn}(x)=u_m(x)$ exists for every $m\in\Bbb N$ and
$x\in X$, $\lim_{m\to\infty}u_m(x)=0$ for every $x\in X$, but there is
no sequence $\sequence{k}{v_k}$ in $\{u_{mn}:m,\,n\in\Bbb N\}$ such that
$\lim_{k\to\infty}v_k(x)=0$ for every $x$.
%367Yk, 367K
     
\spheader 367Ym Let $U$ be a Banach lattice with an order-continuous
norm.   For $v\in U^+$ define $\rho_v:U\times U\to\coint{0,\infty}$ by
setting $\rho_v(u_1,u_2)=\||u_1-u_2|\wedge v\|$ for all $u_1$, 
$u_2\in U$.   Show that every $\rho_v$ is a pseudometric on $U$, and that $\{\rho_v:v\in U^+\}$ defines a Hausdorff linear space topology on $U$.
%367K
     
\spheader 367Yn Let $U$ be any Riesz space.   For $h\in(U^{\sim}_c)^+$
(356Ab), $v\in U^+$ define $\rho_{vh}:U\times U\to\coint{0,\infty}$ by
setting $\rho_{vh}(u_1,u_2)=h(|u_1-u_2|\wedge v)$ for all $u_1$, $u_2\in
U$.   Show that each $\rho_{vh}$ is a pseudometric on $U$, and that
$\{\rho_{vh}:h\in(U^{\sim}_c)^+,\,v\in U^+\}$ defines a linear space
topology on $U$.
%367K
     
\spheader 367Yo Let $(\frak A,\bar\mu)$ be a $\sigma$-finite measure
algebra.   Show that the function
$(\alpha,u)\mapsto\Bvalue{u>\alpha}:\Bbb R\times L^0\to\frak A$ is Borel
measurable when $L^0=L^0(\frak A)$ is given the topology of convergence
in measure and $\frak A$ is given its measure-algebra topology.
\Hint{if $a\in\frak A$, $\gamma\ge 0$ then
$\{(\alpha,u):\bar\mu(a\Bcap\Bvalue{u>\alpha})>\gamma\}$ is open.}
%367M
     
\spheader 367Yp Let $\frak G$ be the regular open algebra of $\Bbb R$.
Show that there is no Hausdorff topology $\frak T$ on $L^0(\frak G)$
such that $\sequencen{u_n}$ is $\frak T$-convergent to $u$ whenever
$\sequencen{u_n}$ is order*-convergent to $u$.   \Hint{Let $H$ be any
$\frak T$-open set containing $0$.   Enumerate $\Bbb Q$ as
$\sequencen{q_n}$.   Find inductively a non-decreasing sequence
$\sequencen{G_n}$ in $\frak G$ such that $\chi G_n\in H$, $q_n\in G_n$
for every $n$.   Conclude that $\chi\Bbb R\in\overline{H}$.}
%367P
     
\spheader 367Yq Give an example of a Banach lattice with a norm which is
not order-continuous, but in which every order-bounded
order*-convergent sequence
is norm-convergent.
%367P
     
\spheader 367Yr Let $\frak A$ be a Dedekind $\sigma$-complete Boolean
algebra and $r\ge 1$.   Let $E\subseteq\BbbR^r$ be a Borel set and
write $Q_E=\{(u_1,\ldots,u_r):\Bvalue{(u_1,\ldots,u_r)\in E}=1\}
\subseteq L^0(\frak A)^r$ (364Yb).   Let $h:E\to\Bbb R$ be a
continuous function and $\bar h:Q_E\to L^0=L^0(\frak A)$ the 
corresponding map
(364Yc).   Show that if $\bar h$ is continuous if $L^0$ is given its
topology of convergence in measure and $(L^0)^r$ the product topology.
%367S
     
\spheader 367Ys Show that 367U is true for all measure algebras,
whether semi-finite or not.
%367U

\spheader 367Yt In 367Qc, show that 
$u=\lim_{\frak B\to\Cal F(\Bbb B\closeuparrow)}u_{\frak B}$ for the norm
topology of $L^1$ iff $\{u_{\frak B}:\frak B\in\Bbb B\}$
is uniformly integrable, and that in this case
$u_{\frak B}=P_{\frak B}u$ for every $\frak B\in\Bbb B$.
%367Q out of order query
}%end of exercises
     
\cmmnt{
\Notesheader{367} I have given a very general definition of
`order*-convergence'.   The general theory of convergence structures
on ordered spaces is complex and full of traps for the unwary.   I have
tried to lay out a safe path to
the results which are important in the context of this book.   But the
propositions here are necessarily full of little conditions (e.g.,
the requirement that $U$ should be Archimedean in 367E) whose
significance may not be immediately obvious.   In particular, the
definition is very much better adapted to distributive lattices than to
others (367Yc, 367Yd).   It is useful in the study of Riesz
spaces and Boolean algebras largely because these satisfy strong
distributive laws (313B, 352E).   The special feature which
distinguishes the definition here from other definitions of
order-convergence is the fact that it can be applied to sequences which
are not order-bounded.   For order-bounded sequences there are useful
simplifications (367Be-f), but the Martingale Theorem (367J), for
instance, if we want to express it in terms of its natural home in the
Riesz space
$L^1$, refers to sequences which are hardly ever order-bounded.
     
The * in the phrase `order*-convergent' is supposed to be a
warning that it may not represent exactly the concept you expect.   I
think nearly any author using the phrase `order-convergent' would accept
sequences fulfilling the conditions of 367Bf;  but beyond this no
standard definitions have taken root.
     
The fact that order*-convergent sequences in an $L^0$ space are
order-bounded (367G) is actually one of the characteristic properties
of $L^0$.   Related ideas will be important in the next section (368A,
368M).
     
It is one of the outstanding characteristics of measure algebras in this
context that they provide non-trivial linear space topologies on their
$L^0$ spaces, related in striking ways to the order structure.   Not all
$L^0$ spaces have such topologies (367Yp).   A
topology corresponding to `convergence in measure' can be defined
on $L^0(\frak A)$ for any Maharam algebra $\frak A$;  see 393K below.
     
367T shows that the topology of convergence in measure on $L^0(\frak A)$
is (at least for semi-finite measure algebras) determined by the Riesz
space structure of $L^0$;  and that indeed the same is true of its
order-dense Riesz subspaces.   This fact is important for a full
understanding of the representation theorems in \S369 below.   If a
Riesz space $U$ can be embedded as an order-dense subspace of any such
$L^0$, then there is already a `topology of convergence in measure'
on $U$, independent of the embedding.   It is therefore not surprising
that there should be alternative descriptions of the topology of
convergence in measure on the important subspaces of $L^0$
(367Xo, 367Ym).
     
For $\sigma$-finite measure algebras, the topology of convergence in
measure is easily described in terms of order-convergence (367P).
For other measure algebras, the formula fails (367Xt).
367Yp shows that trying to apply the same ideas to Riesz spaces in
general gives rise to some very curious phenomena.
     
367V enables us to prove results which would ordinarily be associated
with some form of compactness.   Of course compactness is indeed
involved, as the proof through 367U makes clear;  but it is weak*
compactness in $(L^1)^{**}$, rather than in the space immediately to
hand.
     
I hardly mention `uniform integrability' in this section, not because
it is uninteresting, but because I have nothing to add at this point to
246J and the exercises in \S246.   But I do include translations of
Lebesgue's Dominated Convergence Theorem (367I) and the Martingale
Theorem (367J) to show how these can be expressed in the language of
this chapter.
}%end of comment
     
\discrpage
     
