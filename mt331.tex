\frfilename{mt331.tex}
\versiondate{1.2.05}
\copyrightdate{1995}

\def\chaptername{Maharam's theorem}
\def\sectionname{Maharam types and homogeneous measure algebras}

\newsection{331}

I embark directly on the principal theorem of this chapter (331I), split
between 331B, 331D and 331I;  331B and 331D will be the basis of many of
the results in later sections of this chapter.   In 331E-331H I
introduce the concepts of `Maharam type' and `Maharam-type-homogeneity'.
I discuss the measure algebras of products $\{0,1\}^{\kappa}$, showing
that these provide a complete set of examples of \Mth\
probability algebras (331J-331L).

\leader{331A}{Definition}\cmmnt{ The following idea is almost the key
to the whole chapter.}   Let $\frak A$ be a Boolean algebra and
$\frak B$ an order-closed subalgebra of $\frak A$.
A non-zero element $a$ of $\frak A$ is
a {\bf relative atom} over $\frak B$ if every $c\Bsubseteq a$ is of the
form $a\Bcap b$ for some $b\in\frak B$\cmmnt{;  that is,
$\{a\Bcap b:b\in\frak B\}$ is the principal ideal generated by $a$}.
\cmmnt{We say that} $\frak A$ is {\bf relatively atomless} over $\frak B$
if there are no relative atoms in $\frak A$ over $\frak B$.

\cmmnt{(I'm afraid the phrases `relative atom', `relatively atomless'
are bound to seem opaque at this stage.   I hope that after the
structure theory of \S333 they will seem more natural.   For the moment,
note only that $a$ is an atom in $\frak A$ iff it is a relative atom
over the smallest subalgebra $\{0,1\}$, and every element of $\frak A$
is a relative atom over the largest subalgebra $\frak A$.   In a way,
$a$ is a relative atom over $\frak B$ if its image is an atom in a kind
of quotient $\frak A/\frak B$.   But we are two volumes away from any
prospect of making sense of this kind of quotient.)}

\leader{331B}{}\dvArevised{2010}\cmmnt{ The first lemma is the heart of
Maharam's theorem.

\medskip

\noindent}{\bf Lemma} Let $(\frak A,\bar\mu)$ be a totally finite
measure algebra, $\frak B$ a closed subalgebra of $\frak A$ such that
$\frak A$ is relatively atomless over $\frak B$, and $a_0\in\frak A$.
Let $\nu:\frak B\to\Bbb R$ be an additive functional such
that $0\le\nu b\le\bar\mu(b\Bcap a_0)$ for every $b\in\frak B$.
Then there is a $c\in\frak A$ such that
$c\Bsubseteq a_0$ and $\nu b=\bar\mu(b\Bcap c)$ for every $b\in\frak B$.

\cmmnt{\medskip

\noindent{\bf Remark} Recall that by 323H we need not distinguish
between `order-closed' and `topologically closed' subalgebras.
}

\proof{{\bf (a)} It is worth noting straight away that $\nu$ is
necessarily countably additive.   This is easy to check from first
principles, but if you want to trace the underlying ideas they are in
313O (the identity map from $\frak B$ to $\frak A$ is
order-continuous), 326Jf (so $\mu\restrp\frak B:\frak B\to\Bbb R$ is
countably additive) and 326Kb (therefore $\nu$ is countably additive).

\medskip

{\bf (b)} For each $a\in\frak A$ set
$\nu_ab=\bar\mu(b\Bcap a)$ for every $b\in\frak B$;  
then $\nu_a:\frak B\to\Bbb R$ is
countably additive (326Jd).   Note that $\nu_{c\Bcup d}=\nu_c+\nu_d$ whenever
$c$. $d\in\frak A$ are disjoint.   The key idea is the
following fact:  for every non-zero $a\in\frak A$ there is a non-zero
$d\Bsubseteq a$ such that $\nu_d\le\bover12\nu_a$.   \Prf\ Because
$\frak A$ is relatively atomless over $\frak B$, there is an
$e\Bsubseteq a$ such that $e\ne a\Bcap b$ for any $b\in\frak B$.
Consider the countably additive
functional $\lambda=\nu_a-2\nu_e:\frak B\to\Bbb R$.   By 326M, there is
a $b_0\in\frak B$ such that $\lambda b\ge 0$ whenever $b\in\frak B$ and
$b\Bsubseteq b_0$, while $\lambda b\le 0$ whenever $b\in\frak B$ and
$b\Bcap b_0=0$.

If $e\Bcap b_0\ne 0$, try $d=e\Bcap b_0$.   Then $0\ne d\Bsubseteq a$,
and for every $b\in\frak B$

\Centerline{$\nu_db=\nu_e(b\Bcap b_0)
=\Bover12(\nu_a(b\Bcap b_0)-\lambda(b\Bcap b_0))\le\Bover12\nu_ab$}


\noindent (because $\lambda(b\Bcap b_0)\ge 0$) so
$\nu_d\le\bover12\nu_a$.

If $e\Bcap b_0=0$, then (by the choice of $e$) $e\ne a\Bcap(1\Bsetminus
b_0)$, so $d=a\Bsetminus(e\Bcup b_0)\ne 0$, and of course $d\Bsubseteq
a$.   In this case, for every $b\in\frak B$,

\Centerline{$\nu_db=\nu_a(b\Bsetminus b_0)-\nu_e(b\Bsetminus b_0)
=\Bover12(\lambda(b\Bsetminus b_0)+\nu_a(b\Bsetminus b_0))
\le\Bover12\nu_ab$}


\noindent (because $\lambda(b\Bsetminus b_0)\le 0$), so once again
$\nu_d\le\bover12\nu_a$.


Thus in either case we have a suitable $d$.\   \Qed

\medskip

{\bf (c)} It follows at once, by induction on $n$, that if $a$ is any
non-zero element of $\frak A$ and $n\in\Bbb N$ then there is a non-zero
$d\Bsubseteq a$ such that $\nu_d\le 2^{-n}\nu_a$.

\medskip

{\bf (d)}
Now suppose that $a\in\frak A$ and that $\lambda:\frak B\to\coint{0,\infty}$ is a
non-zero countably additive functional such that $\lambda\le\nu_a$.   Then there is a
non-zero $d\Bsubseteq a$ such that $\nu_d\le\lambda$.   \Prf\ Let $b^*\in\frak B$ be
such that $\lambda b^*>0$;  then

\Centerline{$\lambda(b^*\setminus a)\le\nu_a(b^*\setminus a)=0$,}

\noindent so $\lambda b^*=\lambda(b^*\Bcap a)$.   Take $n\in\Bbb N$ such that
$2^{-n}\nu_ab^*<\lambda b^*$;  set $\lambda_1=\lambda-2^{-n}\nu_a$.   
By 326M (for the
second time), there is a $b_1\in\frak B$ such that $\lambda_1b\ge 0$ if
$b\Bsubseteq b_1$ and $\lambda_1b\le 0$ if $b\Bcap b_1=0$.   Set $c=a\Bcap b_1$.
Now

\Centerline{$2^{-n}\nu_a(a\Bcap b^*)
=2^{-n}\nu_ab^*<\lambda b^*=\lambda(a\Bcap b^*)$,}

\noindent so $\lambda_1(a\Bcap b^*)>0$ and $a\Bcap b^*\Bcap b_1\ne 0$ and $c\ne 0$.
By (c), we have a non-zero $d\Bsubseteq c$ such that $\nu_d\le 2^{-n}\nu_c$.
If $b\in\frak B$ then

\Centerline{$\nu_db\le 2^{-n}\nu_cb=2^{-n}\nu_a(b\Bcap b_1)
=\lambda(b\Bcap b_1)-\lambda_1(b\Bcap b_1)\le\lambda(b\Bcap b_1)\le\lambda b$,}

\noindent so $\nu_d\le\lambda$, as required.\ \Qed

\medskip

{\bf (e)} Let $C$ be the set

\Centerline{$\{a:a\in\frak A$, $a\Bsubseteq a_0$, $\nu_a\le\nu\}$.}

\noindent Then $0\in C$, so $C\ne\emptyset$.    If $D\subseteq C$ is
upwards-directed and not empty, then $a=\sup D$ is defined in $\frak A$ and
included in $a_0$, and

\Centerline{$\nu_{\sup D}b=\bar\mu(b\Bcap\sup D)
=\bar\mu(\sup_{d\in D}b\Bcap d)=\sup_{d\in D}\bar\mu(b\Bcap d)
=\sup_{d\in D}\nu_db\le\nu b$}

\noindent  using 313Ba and 321C.   So $a\in C$ and is an upper bound for
$D$ in $C$.   In particular, any non-empty totally ordered subset of $C$
has an upper bound in $C$.   By Zorn's Lemma, $C$ has a maximal element
$c$ say.   Of course $c\Bsubseteq a_0$.

\medskip

{\bf (e)} \Quer\ Suppose, if possible, that $\nu_c\ne\nu$.
Set $\lambda=\nu-\nu_c$;  note that

\Centerline{$\lambda b
=\nu b-\nu_cb\le\bar\mu(b\Bcap a_0)-\bar\mu(b\Bcap c)
=\nu_{a_0\Bsetminus c}b$}

\noindent for every $b\in\frak B$, so $\lambda\le\nu_{a_0\Bsetminus c}$.
Because $c\in C$, $\lambda\ge 0$, and we are supposing that $\lambda\ne 0$.
By (d), there is a non-zero $d\Bsubseteq a_0\Bsetminus c$ such that
$\nu_d\le\lambda$.   But now $c\Bcup d\Bsubseteq a_0$,

\Centerline{$\nu_{c\Bcup d}=\nu_c+\nu_d\le\nu_c+\lambda=\nu$}

\noindent and $c\Bcup d\in C$, so $c$ is not maximal in $C$.\ \Bang

Thus $c$ is an element of $\frak A$, included in $a_0$,
giving a representation of $\nu$.
}%end of proof of 331B

\leader{331C}{Corollary} Let $(\frak A,\bar\mu)$ be an atomless
semi-finite measure algebra, and $a\in\frak A$.   Suppose that
$0\le\gamma\le\bar\mu a$.   Then there is a $c\in\frak A$ such that
$c\Bsubseteq a$ and $\bar\mu c=\gamma$.

\proof{ If $\gamma=\bar\mu a$, take $c=a$.    If $\gamma<\bar\mu a$,
there is a
$d\in\frak A$ such that $d\Bsubseteq a$ and $\gamma\le\bar\mu d<\infty$
(322Eb).   Apply 331B to
the principal ideal $\frak A_d$ generated by $d$, with $a_0=d$,
$\frak B=\{0,d\}$
and $\nu d=\gamma$.   (The point is that because $\frak A$ is atomless,
no non-trivial principal ideal of $\frak A_d$ can be of the form
$\{c\Bcap b:b\in\frak B\}=\{0,c\}$, so $\frak A_d$ is relatively atomless
over $\{0,d\}$.)
}%end of proof of 331C

\cmmnt{\medskip

\noindent{\bf Remark} Of course this is also an easy consequence of
either 215D or the one-dimensional case of 326H.
}%end of comment

\leader{331D}{Lemma} Let $(\frak A,\bar\mu)$, $(\frak B,\bar\nu)$ be
totally finite measure algebras and $\frak C\subseteq\frak A$ a closed
subalgebra.   Suppose that $\pi:\frak C\to\frak B$ is a
measure-preserving Boolean homomorphism such that $\frak B$ is
relatively atomless over $\pi[\frak C]$.
Take any $a\in\frak A$, and let $\frak C_1$ be the subalgebra
of $\frak A$ generated by $\frak C\cup\{a\}$.   Then  there is a
measure-preserving homomorphism from $\frak C_1$ to $\frak B$ extending
$\pi$.

\proof{ We know that $\pi[\frak C]$ is a closed subalgebra of $\frak B$
(324Kb), and that $\pi$ is a Boolean isomorphism between $\frak C$ and
$\pi[\frak C]$.   Consequently the countably additive functional
$c\mapsto\bar\mu(c\Bcap a):\frak C\to\Bbb R$ is transferred to a
countably additive functional $\lambda:\pi[\frak C]\to\Bbb R$, writing
$\lambda(\pi c)=\bar\mu(c\Bcap a)$ for every $c\in\frak C$.   Of course
$\lambda(\pi c)\le\bar\mu c=\bar\nu(\pi c)$ for every $c\in\frak C$.
So by 331B there is a $b\in\frak B$ such that
$\lambda(\pi c)=\bar\nu(b\Bcap\pi c)$ for every $c\in\frak C$.

If $c\in\frak C$ and $c\Bsubseteq a$ then

\Centerline{$\bar\nu(b\Bcap\pi c)=\lambda(\pi c)
=\bar\mu(a\Bcap c)=\bar\mu c=\bar\nu(\pi c)$,}

\noindent so $\pi c\Bsubseteq b$.   Similarly, if $a\Bsubseteq c\in\frak C$, then

\Centerline{$\bar\nu(b\Bcap\pi c)=\bar\mu(a\Bcap c)=\bar\mu(a\Bcap 1)
=\bar\nu(b\Bcap\pi 1)=\bar\nu b$,}

\noindent so $b\Bsubseteq\pi c$.   It follows from 312O that there is
a Boolean homomorphism $\pi_1:\frak C_1\to\frak B$, extending $\pi$,
such that $\pi_1a=b$.

To see that $\pi_1$ is measure-preserving, take any member of $\frak C_1$.
By 312N, this is expressible as $e=(c_1\Bcap a)\Bcup(c_2\Bsetminus a)$,
where $c_1$, $c_2\in\frak C$.   Now

$$\eqalign{\bar\nu(\pi_1e)
&=\bar\nu((\pi c_1\Bcap b)\Bcup(\pi c_2\Bsetminus b))
=\bar\nu(\pi c_1\Bcap b)+\bar\nu(\pi c_2)-\bar\nu(\pi c_2\Bcap b)\cr
&=\bar\mu(c_1\Bcap a)+\bar\mu c_2-\bar\mu(c_2\Bcap a)
=\bar\mu e.\cr}$$

\noindent As $e$ is arbitrary, $\pi_1$ is measure-preserving.
}%end of proof of 331D

\leader{331E}{Generating sets}\ifresultsonly{ If}
\else{ For the sake of the next definition, we
need a language a little more precise than I have felt the need to use
so far.   The point is that if }\fi
$\frak A$ is a Boolean algebra and $B$ is
a subset of $\frak A$\cmmnt{, there is more than one subalgebra of $\frak A$
which can be said to be `generated' by $B$, because} we can look at any
of the three algebras

\quad-- $\frak B$, the smallest subalgebra of $\frak A$ including $B$;

\quad-- $\frak B_{\sigma}$, the smallest $\sigma$-subalgebra of
$\frak A$ including $B$;

\quad-- $\frak B_{\tau}$, the smallest order-closed subalgebra of
$\frak A$ including $B$.

\noindent\cmmnt{(See 313Fb.)  Now} I will say henceforth\cmmnt{, in this context,} that

\quad-- $\frak B$ is the subalgebra of $\frak A$ generated by $B$, and
$B$ {\bf generates} $\frak A$ if $\frak A=\frak B$;

\quad-- $\frak B_{\sigma}$ is the $\sigma$-subalgebra
of $\frak A$ generated by $B$, and $B$ {\bf $\sigma$-generates}
$\frak A$ if $\frak A=\frak B_{\sigma}$;

\quad-- $\frak B_{\tau}$ is the order-closed subalgebra of $\frak A$
generated by $B$, and $B$ {\bf $\tau$-generates} or {\bf completely generates} %Monk
$\frak A$ if $\frak A=\frak B_{\tau}$.

\cmmnt{There is a danger inherent in these phrases, because if we have
$B\subseteq\frak A'$, where $\frak A'$ is a subalgebra of $\frak A$, it
is possible that the smallest order-closed subalgebra of $\frak A'$
including $B$ might not be recoverable from the smallest order-closed
subalgebra of $\frak A$ including $B$.   (See
331Yb-331Yc.)   This problem will not seriously
interfere with the ideas below;  but for definiteness let me say that
the phrases `$B\,\,\sigma$-generates $\frak A$',
`$B\,\,\tau$-generates $\frak A$' will always refer to suprema and
infima taken in $\frak A$ itself, not in any larger algebra in which it
may be embedded.
}

\leader{331F}{Maharam types (a)}
\ifresultsonly{ If $\frak A$ is a Boolean algebra, its {\bf Maharam
type} $\tau(\frak A)$}
\else{ With the language of 331E established,
I can now define the {\bf Maharam type} or
{\bf complete generation number}
$\tau(\frak A)$ of any Boolean algebra $\frak A$;  it }\fi
is the smallest cardinal of any subset of $\frak A$ which $\tau$-generates
$\frak A$.

\cmmnt{(I think that this is the first `cardinal function' which I have
mentioned in this treatise.   All you need to know, to confirm that the
definition is well-conceived, is that there is {\it some} set which
$\tau$-generates $\frak A$;  and obviously $\frak A\,\,\tau$-generates
itself.   For this means that the set
$A=\{\#(B):B\subseteq\frak A\,\,\tau$-generates $\frak A\}$ is a non-empty
class of cardinals, and
therefore, assuming the axiom of choice, has a least member (2A1Lf).
In 331Ye-331Yf I mention a further function, the `density' of a
topological space, which is closely related to
Maharam type.)
}

\header{331Fb}{\bf (b)} A Boolean algebra $\frak A$ is {\bf\Mth} if
$\tau(\frak A_a)=\tau(\frak A)$ for every non-zero
$a\in\frak A$, writing $\frak A_a$ for the principal ideal of $\frak A$
generated by $a$.

\header{331Fc}{\bf (c)} Let $(X,\Sigma,\mu)$ be a measure space, with
measure algebra $(\frak A,\bar\mu)$.   Then the {\bf Maharam type} of
$(X,\Sigma,\mu)$, or of $\mu$, is the Maharam type of $\frak A$;  and
$(X,\Sigma,\mu)$, or $\mu$, is {\bf\Mth} if $\frak A$ is.

\cmmnt{
\medskip

\noindent{\bf Remark} I should perhaps remark that the phrases `Maharam
type' and `\Mth', while well established in the context
of probability algebras, are not in common use for general Boolean
algebras.   But the cardinal $\tau(\frak A)$ is important in the general
context, and is such an obvious extension of Maharam's idea ({\smc
Maharam 42}) that I am happy to propose this extension of terminology.
}

\leader{331G}{}\cmmnt{ For the sake of those who have not mixed set
theory and
algebra before, I had better spell out some basic facts.

\medskip

\noindent}{\bf Proposition} Let $\frak A$ be a Boolean algebra, $B$ a
subset of $\frak A$.   Let $\frak B$ be the subalgebra of $\frak A$
generated by $B$, $\frak B_{\sigma}$ the
$\sigma$-subalgebra of $\frak A$ generated by $B$, and $\frak B_{\tau}$
the order-closed subalgebra of $\frak A$ generated by $B$.

(a) $\frak B\subseteq\frak B_{\sigma}\subseteq\frak B_{\tau}$.

(b) If $B$ is finite, so is $\frak B$, and in this case
$\frak B=\frak B_{\sigma}=\frak B_{\tau}$.

(c) For every $a\in\frak B$, there is a finite $B'\subseteq B$ such that
$a$ belongs to the subalgebra of $\frak A$ generated by $B'$.
Consequently $\#(\frak B)\le\max(\omega,\#(B))$.

(d) For every $a\in\frak B_{\sigma}$, there is a countable
$B'\subseteq B$ such that $a$ belongs to the $\sigma$-subalgebra of
$\frak A$ generated by $B'$.

(e) If $\frak A$ is ccc, then $\frak B_{\sigma}=\frak B_{\tau}$.

\proof{{\bf (a)} All we need to know is that $\frak B_{\sigma}$ is a
subalgebra of $\frak A$ including $B$, and that $\frak B_{\tau}$ is a
$\sigma$-subalgebra of $\frak A$ including $B$.

\medskip

{\bf (b)} Induce on $\#(B)$, using 312N for the inductive step, to see
that $\frak B$ is finite.   In this case it must be order-closed, so is
equal to $\frak B_{\tau}$.

\medskip

{\bf (c)(i)} For $I\subseteq B$, let $\frak C_I$ be the subalgebra of
$\frak A$ generated by $I$.   If $I$, $J\subseteq B$ then $\frak
C_I\cup\frak C_J\subseteq\frak C_{I\cup J}$.   So $\bigcup\{\frak
C_I:I\subseteq B$ is finite$\}$ is a subalgebra of $\frak A$, and must
be equal to $\frak B$, as claimed.

\medskip

\quad{\bf (ii)} To estimate the size of $\frak B$, recall that the set
$[B]^{<\omega}$ of all finite subsets of $B$ has cardinal at most
$\max(\omega,\#(B))$ (3A1Cd).    For each $I\in[B]^{<\omega}$,
$\frak C_I$ is finite, so

\Centerline{$\#(\frak B)=\#(\bigcup_{I\in[B]^{<\omega}}\frak C_I)
\le\max(\omega,\#(I),\sup_{I\in [B]^{<\omega}}\#(\frak C_I))
\le\max(\omega,\#(B))$}

\noindent by 3A1Cc.

\medskip

{\bf (d)} For $I\subseteq B$, let $\frak D_I\subseteq \frak B_{\sigma}$
be the $\sigma$-subalgebra of $\frak A$ generated by $I$.   If $I$,
$J\subseteq B$ then $\frak D_I\cup\frak D_J\subseteq\frak D_{I\cup J}$,
so $\frak B'_{\sigma}=\bigcup\{\frak D_I:I\subseteq B$ is countable$\}$
is a subalgebra of $\frak A$.   But also it is sequentially
order-closed in $\frak A$.   \Prf\ Let $\sequencen{a_n}$ be a
non-decreasing sequence in $\frak B'_{\sigma}$ with supremum $a$ in
$\frak
A$.   For each $n\in\Bbb N$ there is a countable $I(n)\subseteq B$ such
that $a_n\in\frak C_{I(n)}$.   Set $K=\bigcup_{n\in\Bbb N}I(n)$;  then
$K$ is a countable subset of $B$ and every $a_n$ belongs to $\frak D_K$,
so $a\in\frak D_K\subseteq\frak B'_{\sigma}$.\  \Qed\   So $\frak
B'_{\sigma}$ is a $\sigma$-subalgebra of $\frak A$
including $B$ and must be the whole of $\frak B_{\sigma}$.

\medskip

{\bf (e)} By 316Fb, $\frak B_{\sigma}$ is order-closed in $\frak A$, so
must be equal to $\frak B_{\tau}$.
}%end of proof of 331G

\leader{331H}{Proposition} Let $\frak A$ be a Boolean algebra.

(a)(i) $\tau(\frak A)=0$ iff $\frak A$ is either $\{0\}$ or $\{0,1\}$.

\quad(ii) $\tau(\frak A)$ is finite iff $\frak A$ is finite.

(b) If $\frak B$ is another Boolean algebra and $\pi:\frak A\to\frak B$
is a surjective order-continuous Boolean homomorphism, then $\tau(\frak
B)\le\tau(\frak A)$.

(c) If $a\in\frak A$ then $\tau(\frak A_a)\le\tau(\frak A)$, where
$\frak A_a$ is the principal ideal of $\frak A$ generated by $a$.

(d) If $\frak A$ has an atom and is \Mth, then $\frak A=\{0,1\}$.

\proof{{\bf (a)(i)} $\tau(\frak A)=0$ iff $\frak A$ has no proper
subalgebras.

\medskip

\quad{\bf (ii)} If $\frak A$ is finite, then
$\tau(\frak A)\le\#(\frak A)$ is
finite.   If $\tau(\frak A)$ is finite, then there
is a finite set $B\subseteq\frak A$ which $\tau$-generates $\frak A$;
by 331Gb, $\frak A$ is finite.

\medskip

{\bf (b)} We know that there is a set $A\subseteq\frak A$,
$\tau$-generating $\frak A$, with $\#(A)=\tau(\frak A)$.   Now $\pi[A]$
$\tau$-generates $\pi[\frak A]=\frak B$ (313Mb), so

\Centerline{$\tau(\frak B)\le\#(\pi[A])\le\#(A)=\tau(\frak A)$.}

\medskip

{\bf (c)} Apply (b) to the map $b\mapsto a\Bcap b:\frak A\to\frak A_a$.

\medskip

{\bf (d)} If $a\in\frak A$ is an atom, then $\tau(\frak A_a)=0$, so if
$\frak A$ is \Mth\ then $\tau(\frak A)=0$ and $\frak A=\{0,a\}=\{0,1\}$.
}% end of proof of 331H

\leader{331I}{}\cmmnt{ We are now ready for the theorem.

\medskip

\noindent}{\bf Theorem} Let $(\frak A,\bar\mu)$ and $(\frak B,\bar\nu)$
be \Mth\ measure algebras of the same Maharam type, with
$\bar\mu 1=\bar\nu 1<\infty$.
Then they are isomorphic as measure algebras.

\proof{{\bf (a)} If $\tau(\frak A)=\tau(\frak B)=0$, this is trivial.
So let us take $\kappa=\tau(\frak A)=\tau(\frak B)>0$.   In this case,
because $\frak A$ and $\frak B$ are \Mth, they can have
no atoms and must be infinite, so $\kappa$ is
infinite (331H).   Let $\langle a_{\xi}\rangle_{\xi<\kappa}$ and
$\langle b_{\xi}\rangle_{\xi<\kappa}$ enumerate $\tau$-generating
subsets of $\frak A$, $\frak B$ respectively.

The strategy of the proof is to define a measure-preserving isomorphism
$\pi:\frak A\to\frak B$ as the last of an increasing family
$\langle\pi_{\xi}\rangle_{\xi\le\kappa}$ of isomorphisms between closed
subalgebras $\frak C_{\xi}$, $\frak D_{\xi}$ of $\frak A$ and $\frak B$.
The inductive hypothesis will be that, for some families $\langle
a'_{\xi}\rangle_{\xi<\kappa}$, $\langle b'_{\xi}\rangle_{\xi<\kappa}$ to
be determined,

\inset{$\frak C_{\xi}$ is the closed subalgebra of $\frak A$
generated by $\{a_{\eta}:\eta<\xi\}\cup\{a'_{\eta}:\eta<\xi\}$,

$\frak D_{\xi}$ is the closed subalgebra of $\frak B$
generated by $\{b_{\eta}:\eta<\xi\}\cup\{b'_{\eta}:\eta<\xi\}$,

$\pi_{\xi}:\frak C_{\xi}\to\frak D_{\xi}$ is a measure-preserving
isomorphism,

$\pi_{\xi}$ extends $\pi_{\eta}$ whenever $\eta<\xi$.}

\noindent (Formally speaking, this will be a transfinite recursion,
defining a function
$\xi\mapsto f(\xi)=(\frak C_{\xi},\frak D_{\xi},\pi_{\xi},a'_{\xi},b'_{\xi})$
on the ordinal $\kappa+1$ by a
rule which chooses $f(\xi)$ in terms of $f\restr\xi$, as described in
2A1B.   The construction of an actual function $F$ for which
$f(\xi)=F(f\restr\xi)$ will necessitate the axiom of choice.)

\medskip

{\bf (b)} The induction starts with $\frak C_0=\{0,1\}$,
$\frak D_0=\{0,1\}$, $\pi_0(0)=0$, $\pi_0(1)=1$.   (The hypothesis
$\bar\mu 1=\bar\nu 1$ is what we need to ensure that $\pi_0$ is
measure-preserving.)

\medskip

{\bf (c)} For the inductive step to a successor ordinal $\xi+1$, where
$\xi<\kappa$, suppose that $\frak C_{\xi}$, $\frak D_{\xi}$ and
$\pi_{\xi}$ have been defined.

\medskip

\quad{\bf (i)} For any non-zero $b\in\frak B$, the principal ideal
$\frak B_b$ of $\frak B$ generated by $b$ has Maharam type $\kappa$,
because $\frak B$ is \Mth.   On the other hand, the
Maharam type of $\frak D_{\xi}$ is at most

\Centerline{$\#(\{b_{\eta}:\eta<\xi\}\cup\{b'_{\eta}:\eta<\xi\})
\le\#(\xi\times\{0,1\})<\kappa$,}

\noindent because if $\xi$ is finite so is $\xi\times\{0,1\}$, while if
$\xi$ is infinite then $\#(\xi\times\{0,1\})=\#(\xi)\le\xi<\kappa$.
Consequently $\frak B_b$ cannot be an order-continuous image of
$\frak D_{\xi}$ (331Hb).   Now the map
$c\mapsto c\Bcap b:\frak D_{\xi}\to\frak B_b$ is order-continuous, because
$\frak D_{\xi}$ is closed, so that the
embedding $\frak D_{\xi}\embedsinto\frak B$ is order-continuous.
It therefore cannot be surjective, and

\Centerline{$\{b\Bcap\pi_{\xi}a:a\in\frak C_{\xi}\}
=\{b\Bcap d:d\in\frak D_{\xi}\}\ne\frak B_b$.}

This means that $\pi_{\xi}:\frak C_{\xi}\to\frak D_{\xi}$ satisfies the
conditions of 331D, and must have an extension $\phi_{\xi}$ to a
measure-preserving homomorphism from the subalgebra $\frak C'_{\xi}$ of
$\frak A$ generated by $\frak C_{\xi}\cup\{a_{\xi}\}$ to $\frak B$.
We know that $\frak C'_{\xi}$ is a closed subalgebra of $\frak A$
(314Ja), so it must be the closed subalgebra of $\frak A$ generated by
$\{a_{\eta}:\eta\le\xi\}\cup\{a'_{\eta}:\eta<\xi\}$.   Also $\frak
D'_{\xi}=\phi_{\xi}[\frak C'_{\xi}]$ will be the subalgebra of $\frak B$
generated by $\frak D_{\xi}\cup\{b'_{\xi}\}$, where
$b'_{\xi}=\phi_{\xi}(a_{\xi})$, so is closed in $\frak B$, and is
the closed subalgebra of $\frak B$ generated by
$\{b_{\eta}:\eta<\xi\}\cup\{b'_{\eta}:\eta\le\xi\}$.

\medskip

\quad{\bf (ii)} The next step is to repeat the whole of the argument
above, but applying it to 
$\phi_{\xi}^{-1}:\frak D'_{\xi}\to\frak C_{\xi}$, $b_{\xi}$ in place of 
$\pi_{\xi}:\frak C_{\xi}\to\frak D_{\xi}$ and $a_{\xi}$.   Once again, we 
have $\tau(\frak D'_{\xi})<\kappa=\tau(\frak A_a)$ for every 
$a\in\frak A$, so we can use
Lemma 331D to find a measure-preserving isomorphism 
$\psi_{\xi}:\frak D_{\xi+1}\to\frak C_{\xi+1}$ extending 
$\phi_{\xi}^{-1}$, where $\frak D_{\xi+1}$ is the subalgebra of $\frak B$ 
generated by 
$\frak D'_{\xi}\cup\{b_{\xi}\}$, and $\frak C_{\xi+1}$ is the subalgebra of
$\frak A$ generated by $\frak C'_{\xi}\cup\{a'_{\xi}\}$, setting
$a'_{\xi}=\psi_{\xi}(b_{\xi})$.   As in (i), we find that 
$\frak C_{\xi+1}$ is the closed subalgebra of $\frak A$ generated by
$\{a_{\eta}:\eta\le\xi\}\cup\{a'_{\eta}:\eta\le\xi\}$, while 
$\frak D_{\xi+1}$ is the closed subalgebra of $\frak B$ generated by
$\{b_{\eta}:\eta\le\xi\}\cup\{b'_{\eta}:\eta\le\xi\}$.

\medskip

\quad{\bf (iii)} We can therefore take $\pi_{\xi+1}
=\psi_{\xi}^{-1}:\frak C_{\xi+1}\to\frak D_{\xi+1}$, and see that
$\pi_{\xi+1}$ is a measure-preserving isomorphism, extending
$\pi_{\xi}$, such that $\pi_{\xi+1}(a_{\xi})=b'_{\xi}$ and
$\pi_{\xi+1}(a'_{\xi})=b_{\xi}$.   Evidently $\pi_{\xi+1}$ extends
$\pi_{\eta}$ for every $\eta\le\xi$ because it extends $\pi_{\xi}$ and
(by the inductive hypothesis) $\pi_{\xi}$ extends $\pi_{\eta}$ for every
$\eta<\xi$.

\medskip

{\bf (d)} For the inductive step to a limit ordinal $\xi$, where
$0<\xi\le\kappa$, suppose that $\frak C_{\eta}$, $\frak D_{\eta}$,
$a'_{\eta}$, $b'_{\eta}$, $\pi_{\eta}$ have been defined for $\eta<\xi$.
Set $\frak C^*_{\xi}=\bigcup_{\eta<\xi}\frak C_{\xi}$.   Then $\frak
C^*_{\xi}$ is a subalgebra of $\frak A$, because it is the union of an
upwards-directed family of subalgebras;  similarly, $\frak
D^*_{\xi}=\bigcup_{\eta<\xi}\frak D_{\xi}$ is a subalgebra of $\frak B$.
Next, we have a function $\pi^*_{\xi}:\frak C^*_{\xi}\to\frak D^*_{\xi}$
defined by setting $\pi^*_{\xi}a=\pi_{\eta}a$ whenever $\eta<\xi$ and
$a\in\frak C_{\eta}$;  for if $\eta$, $\zeta<\xi$ and $a\in\frak
C_{\eta}\cap\frak C_{\zeta}$, then
$\pi_{\eta}a=\pi_{\max(\eta,\zeta)}a=\pi_{\zeta}a$.   Clearly

\Centerline{$\pi^*_{\xi}[\frak C^*_{\xi}]
=\bigcup_{\eta<\xi}\pi_{\eta}[\frak C_{\eta}]
=\frak D^*_{\xi}$.}

\noindent Moreover, $\bar\nu(\pi^*_{\xi}a)=\bar\mu a$ for every $a\in\frak C^*_{\xi}$,
since $\bar\nu(\pi_{\eta}a)=\bar\mu a$ whenever $\eta<\xi$ and
$a\in\frak C_{\eta}$.

Now let $\frak C_{\xi}$ be the smallest closed subalgebra of
$\frak A$ including $\frak C^*_{\xi}$, that is, the topological closure of
$\frak C^*_{\xi}$ in $\frak A$ (323J).   Since $\frak C_{\xi}$ is the
smallest closed subalgebra of $\frak A$ including $\frak C_{\eta}$ for every
$\eta<\xi$, it must be the closed subalgebra of $\frak A$ generated by
$\{a_{\eta}:\eta<\xi\}\cup\{a'_{\eta}:\eta<\xi\}$.
By 324O, $\pi^*_{\xi}$ has an extension to a measure-preserving
homomorphism $\pi_{\xi}:\frak C_{\xi}\to\frak B$.   Set
$\frak D_{\xi}=\pi_{\xi}[\frak C_{\xi}]$;  by 324Kb, $\frak D_{\xi}$ is a
closed subalgebra of $\frak B$.   Because
$\pi_{\xi}:\frak C_{\xi}\to\frak B$ is continuous (also noted in 324Kb),

\Centerline{$\frak D^*_{\xi}=\pi^*_{\xi}[\frak C^*_{\xi}]
=\pi_{\xi}[\frak C^*_{\xi}]$}

\noindent is topologically dense in $\frak D_{\xi}$ (3A3Eb), and
$\frak D_{\xi}=\overline{\frak D^*_{\xi}}$ is the closed subalgebra of
$\frak B$-generated by
$\{b_{\eta}:\eta<\xi\}\cup\{b'_{\eta}:\eta<\xi\}$.   Finally, if
$\eta<\xi$, $\pi_{\xi}$ extends $\pi_{\eta}$ because $\pi^*_{\xi}$
extends $\pi_{\eta}$.   Thus the induction continues.

\medskip

{\bf (e)} The induction ends with $\xi=\kappa$, $\frak C_{\kappa}=\frak A$,
$\frak D_{\kappa}=\frak B$ and $\pi=\pi_{\kappa}:\frak A\to\frak B$
the required measure algebra isomorphism.
}%end of proof of 331I

\leader{331J}{Lemma} Let $(\frak A,\bar\mu)$ be a totally finite measure
algebra, and $\kappa$ an infinite cardinal.

(a)\dvAnew{2008}
If there is a family $\ofamily{\xi}{\kappa}{e_{\xi}}$ in $\frak A$ such
that $\inf_{\xi\in I}a_{\xi}=0$ and $\sup_{\xi\in I}a_{\xi}=1$ for every
infinite $I\subseteq\kappa$, then $\tau(\frak A_d)\ge\kappa$ for every
non-zero $d\in\frak A$.

(b)\discrversionA{\footnote{Elaborated 2008.}}{}
Let $\nu_{\kappa}$ be the usual measure on
$\{0,1\}^{\kappa}$\cmmnt{ (254J)} and
$(\frak B_{\kappa},\bar\nu_{\kappa})$ its measure algebra.   If there is an
order-continuous Boolean homomorphism
from $\frak B_{\kappa}$ to $\frak A$, $\tau(\frak A_d)\ge\kappa$ for every
non-zero $d\in\frak A$.

\proof{{\bf (a)(i)} To begin with (down to the end of (iii)), let us take
it that $d=1$.   For $a\in\frak A$, $\delta>0$ set
$U(a,\delta)=\{a':\bar\mu(a'\Bsymmdiff a)<\delta\}$, the ordinary open
$\delta$-neighbourhood of $a$.   If $a\in\frak A$, then there is a
$\delta>0$ such that $\{\xi:\xi<\kappa,\,a_{\xi}\in U(a,\delta)\}$ is
finite.    \Prf\Quer\ Suppose, if possible, otherwise.   Then there is a
sequence $\sequencen{\xi_n}$ of distinct elements of $\kappa$ such that
$\bar\mu(a\Bsymmdiff a_{\xi_n})\le 2^{-n-2}\bar\mu 1$ for every
$n\in\Bbb N$.    Now $\inf_{n\in\Bbb N}a_{\xi_n}=0$, so

\Centerline{$\bar\mu a
=\bar\mu(a\Bsetminus\inf_{n\in\Bbb N}a_{\xi_n})
\le\sum_{n=0}^{\infty}\bar\mu(a\Bsetminus a_{\xi_n})$.}

\noindent Similarly

\Centerline{$\bar\mu(1\Bsetminus a)
=\bar\mu(\sup_{n\in\Bbb N}a_{\xi_n}\Bsetminus a)
\le\sum_{n=0}^{\infty}\bar\mu( a_{\xi_n}\Bsetminus a)$.}

\noindent Putting these together,

$$\eqalign{\bar\mu 1
&=\bar\mu a+\bar\mu(1\Bsetminus a)
\le\sum_{n=0}^{\infty}\bar\mu(a\Bsetminus a_{\xi_n})
  +\sum_{n=0}^{\infty}\bar\mu( a_{\xi_n}\Bsetminus a)\cr
&=\sum_{n=0}^{\infty}\bar\mu(a\Bsymmdiff a_{\xi_n})
\le\sum_{n=0}^{\infty}2^{-n-2}\bar\mu 1
<\bar\mu 1,\cr}$$

\noindent which is impossible.\ \Bang\Qed

\medskip

\quad{\bf (ii)} Note that $\frak A$ is infinite;  for if $a\in\frak A$
the set
$\{\xi: a_{\xi}=a\}$ must be finite, and $\kappa$ is supposed to be
infinite.   So $\tau(\frak A)$ must be infinite.

\medskip

\quad{\bf (iii)} Now take a set $C\subseteq\frak A$, of cardinal
$\tau(\frak A)$, which $\tau$-generates $\frak A$.   By (ii), $C$ is
infinite.   Let $\frak C$ be the subalgebra of $\frak A$ generated by
$C$;  then $\#(\frak C)=\#(C)=\tau(\frak A)$, by 331Gc, and $\frak C$ is
topologically dense in $\frak A$ (323J again).
If $a\in\frak A$, there are
$c\in\frak C$ and $k\in\Bbb N$ such that $a\in U(c,2^{-k})$ and
$\{\xi: a_{\xi}\in U(c,2^{-k})\}$ is finite.   \Prf\ By (b), there is a
$\delta>0$ such that $\{\xi: a_{\xi}\in U(a,\delta)\}$ is finite.
Take $k\in\Bbb N$ such that $2\cdot 2^{-k}\le\delta$, and
$c\in\frak C\cap U(a,2^{-k})$;  then
$U(c,2^{-k})\subseteq U(a,\delta)$ can contain only finitely many
$ a_{\xi}$, so these $c$, $k$ serve.\   \Qed

Consider

\Centerline{$\Cal U
=\{U(c,2^{-k}):c\in\frak C,\,k\in\Bbb N,\,\{\xi: a_{\xi}\in U(c,2^{-k})\}$
is finite$\}$.}

\noindent Then $\#(\Cal U)\le\max(\#(\frak C),\omega)=\tau(\frak A)$.
Also $\Cal U$ is a cover of $\frak A$.   In particular,
$\kappa=\bigcup_{U\in\Cal U}J_U$, where $J_U=\{\xi: a_{\xi}\in U\}$.
But this means that

\Centerline{$\kappa
=\#(\kappa)\le\max(\omega,\#(\Cal U),\sup_{U\in\Cal U}\#(J_U))
=\tau(\frak A)$.}

\noindent This proves the result when $d=1$.

\medskip

\quad{\bf (iv)} For the general case, given $d\in\frak A\setminus\{0\}$,
set $a'_{\xi}=a_{\xi}\Bcap d$ for each $\xi$.   Since
$\inf_{\xi\in I}a_{\xi}\Bcap d=0$ and $\sup_{\xi\in I}a_{\xi}\Bcap d=d$ for
every infinite $I\subseteq\kappa$, we can apply (i)-(iii) to
$(\frak A_d,\bar\mu\restrp\frak A_d,\ofamily{\xi}{\kappa}{a'_{\xi}})$ to
see that $\tau(\frak A_d)\ge\kappa$, as required.

\medskip

{\bf (b)} Let $\pi:\frak B_{\kappa}\to\frak A$ be an order-continuous
Boolean homomorphism.
Set $E_{\xi}=\{x:x\in\{0,1\}^{\kappa},\,x(\xi)=1\}$,
$e_{\xi}=E_{\xi}^{\ssbullet}\in\frak B_{\kappa}$ and
$a_{\xi}=\pi e_{\xi}\in\frak A$ for each $\xi<\kappa$.
If $\sequencen{\xi_n}$ is any sequence of distinct elements of $\kappa$,

\Centerline{$\nu_{\kappa}(\bigcap_{n\in\Bbb N}E_{\xi_n})
=\lim_{n\to\infty}\nu_{\kappa}(\bigcap_{i\le n}E_{\xi_n})
=\lim_{n\to\infty}2^{-n-1}=0$,}

\noindent so that $\bar\nu_{\kappa}(\inf_{n\in\Bbb N}e_{\xi_n})=0$ and
$\inf_{n\in\Bbb N}e_{\xi_n}=0$.
Because $\pi$ is order-continuous,
$\inf_{n\in\Bbb N}a_{\xi_n}=0$ in $\frak A$.    Similarly,
$\nu_{\kappa}(\bigcup_{n\in\Bbb N}E_{\xi_n})=1$ and
$\sup_{n\in\Bbb N}a_{\xi_n}=1$.   As $\sequencen{\xi_n}$ is arbitrary,
$\inf_{\xi\in I}a_{\xi}=0$ and $\sup_{\xi\in I}a_{\xi}=1$ for every
infinite $I\subseteq\kappa$.   So we can apply (a) to get the result.
}%end of proof of 331J

\leader{331K}{Theorem}  Let $\kappa$ be any infinite cardinal.   Let
$\nu_{\kappa}$ be the usual measure on $\{0,1\}^{\kappa}$ and
$(\frak B_{\kappa},\bar\nu_{\kappa})$
its measure algebra.   Then $\frak B_{\kappa}$ is \Mth, with Maharam type
$\kappa$.

\proof{ Set $X=\{0,1\}^{\kappa}$ and write $\Sigma$ for the domain of
$\nu_{\kappa}$.

\medskip

{\bf (a)} To see that $\tau(\frak B_{\kappa})\le\kappa$, set
$E_{\xi}=\{x:x\in X,\,x(\xi)=1\}$ and
$e_{\xi}=E_{\xi}^{\ssbullet}$ for each
$\xi<\kappa$.    Writing $\Cal E$ for the algebra of subsets of $X$
generated by
$\{E_{\xi}:\xi<\kappa\}$, we see that every measurable cylinder in $X$,
as defined in 254Aa, belongs to $\Cal E$, so that every member of
$\Sigma$ is approximated, in measure, by members of $\Cal E$ (254Fe),
that is, $\{E^{\ssbullet}:E\in\Cal E\}$ is topologically dense in
$\frak A$.   But this means just that the subalgebra
$\frak E$ of $\frak B_{\kappa}$ generated
by $\{e_{\xi}:\xi<\kappa\}$ is topologically dense in $\frak B_{\kappa}$,
so that
$\{e_{\xi}:\xi<\kappa\}\,\,\tau$-generates $\frak B_{\kappa}$, and
$\tau(\frak B_{\kappa})\le\kappa$.

\medskip

{\bf (b)} Next, if $c\in\frak B_{\kappa}\setminus\{0\}$ and
$(\frak B_{\kappa})_c$
is the principal ideal of $\frak B_{\kappa}$ generated by $c$, the map
$b\mapsto b\Bcap c$ is an order-continuous Boolean homomorphism from
$\frak B_{\kappa}$ to $(\frak B_{\kappa})_c$, so by 331Jb we must have
$\tau((\frak B_{\kappa})_c)\ge\kappa$.   Thus

\Centerline{$\kappa\le\tau((\frak B_{\kappa})_c)
\le\tau(\frak B_{\kappa})\le\kappa$.}

\noindent As $c$ is arbitrary, $\frak B_{\kappa}$ is \Mth\ of
Maharam type $\kappa$.
}%end of proof of 331K

\leader{331L}{Theorem} Let $(\frak A,\bar\mu)$ be a \Mth\
probability algebra.   Then there is exactly one $\kappa$, either $0$ or
an infinite cardinal, such that $(\frak A,\bar\mu)$ is isomorphic, as
measure algebra, to the measure algebra $(\frak B_{\kappa},\bar\nu_{\kappa})$
of the usual measure on $\{0,1\}^{\kappa}$.

\proof{ If $\tau(\frak A)$ is finite, it is zero, and $\frak A=\{0,1\}$
(331Ha, 331Hd) so that (interpreting $\{0,1\}^0$ as $\{\emptyset\}$) we
have the case $\kappa=0$.   If $\kappa=\tau(\frak A)$ is infinite, then
by 331K we know that $(\frak B_{\kappa},\bar\nu_{\kappa})$ also is
\Mth\ with Maharam type $\kappa$, so 331I gives the required
isomorphism.   Of course $\kappa$ is uniquely defined by $\frak A$.
}%end of proof of 331L

\leader{331M}{Homogeneous Boolean algebras} Of course a homogeneous Boolean
algebra\cmmnt{ (definition:  316N)} must be
\Mth, since $\tau(\frak A)=\tau(\frak A_c)$ whenever
$\frak A$ is isomorphic to $\frak A_c$.
In general, a Boolean algebra can be \Mth\ without being
homogeneous
(331Xj, 331Yg).   But for $\sigma$-finite measure algebras this doesn't
happen.

\leader{331N}{Proposition} Let $(\frak A,\bar\mu)$ be a \Mth\
$\sigma$-finite measure algebra.   Then it is homogeneous as
a Boolean algebra.

\proof{ If $\frak A=\{0\}$ this is trivial;  so suppose that
$\frak A\ne\{0\}$.
By 322G, there is a measure $\bar\nu$ on $\frak A$ such
that $(\frak A,\bar\nu)$ is a probability algebra.   Now let $c$ be any
non-zero member of $\frak A$, and set $\gamma=\bar\nu c$,
$\bar\nuprime_c=\gamma^{-1}\bar\nu_c$, where $\bar\nu_c$ is the restriction
of $\bar\nu$ to the principal ideal $\frak A_c$ of $\frak A$ generated
by $c$.   Then $(\frak A,\bar\nu)$ and $(\frak A_c,\bar\nuprime_c)$ are
\Mth\ probability algebras of the same Maharam type, so
are isomorphic as measure algebras, and {\it a fortiori} as Boolean
algebras.
}%end of proof of 331N

\leader{331O}{}\dvAnew{2011}
I will wait until Chapter 52 of Volume 5 for a systematic
discussion of properties of measure algebras
which depend on their Maharam
types.   There is however a result which is easy, useful and expressible in
terms already introduced.

\medskip

\noindent{\bf Proposition} Let $(\frak A,\bar\mu)$ be a measure algebra
with countable Maharam type.   Then $\frak A$ is separable in its
measure-algebra topology.

\proof Let $B\subseteq\frak A$ be a countable set which $\tau$-generates
$\frak A$.   Then the subalgebra $\frak B$ of $\frak A$ generated by $B$ is
countable (331Gc).   Now $\frak B$ is dense for the measure-algebra
topology.   \Prf\ Let $G$ be a non-empty open subset of $\frak A$, and $c$
any element of $G$.   Let $\Rho=\{\rho_a:a\in\frak A^f\}$ be the
upwards-directed family of pseudometrics defining the topology of
$\frak A$, as described in 323A.   Then there must be an $a\in\frak A^f$
and an $\epsilon>0$ such that $\{b:\rho_a(b,c)\le\epsilon\}\subseteq G$.
Let $\frak C$ be the order-closed subalgebra of the principal ideal
$\frak A_a$ generated by $\frak B_a=\{b\Bcap a:b\in\frak B\}$.
Because $b\mapsto b\Bcap a:\frak A\to\frak A_a$ is an order-continuous
Boolean homomorphism, $\{b:b\in\frak A$, $b\Bcap a\in\frak C\}$ is
an order-closed subalgebra of $\frak A$, and must be the whole of
$\frak A$, because it includes $B$.   So $\frak C=\frak A_a$.   By 323J,
$\frak C$ is the topological closure of $\frak B_a$ in $\frak A_a$, and
there must be a $b\in\frak B_a$ such that
$\bar\mu(b\Bsymmdiff(c\Bcap a))\le\epsilon$;  that is, there is a
$b\in\frak B$ such that $\bar\mu(a\Bcap(b\Bsymmdiff c))\le\epsilon$ and
$b\in G$.   Thus $\frak B$ meets $G$;  as $G$ is arbitrary, $\frak B$ is
dense.\ \Qed

So $\frak B$ is a countable dense subset of $\frak A$ and $\frak A$ is
separable.

\exercises{
\leader{331X}{Basic exercises (a)}
%\spheader 331Xa
Let $(X,\Sigma,\mu)$ be a probability space and $\Tau$ a
$\sigma$-subalgebra of $\Sigma$ such that for any
non-negligible $E\in\Sigma$
there is an $F\in\Sigma$ such that $F\subseteq E$ and
$\mu(F\symmdiff(E\cap H))>0$ for every $H\in\Tau$.   Suppose that
$f:X\to[0,1]$ is a measurable function.   Show that there is an
$F\in\Sigma$ such that $\int_Hf=\mu(H\cap F)$ for every $H\in\Tau$.
%331B

\sqheader 331Xb Write out a direct proof of 331C not relying on
331B or 321J.
%331C

\spheader 331Xc Let $\frak A$ be a finite Boolean algebra with $n$
atoms.   Show that $\tau(\frak A)$ is the least $k$ such that $n\le 2^k$.
%331F

\sqheader 331Xd Show that the measure algebra of Lebesgue measure on
$\Bbb R$ is \Mth\ with Maharam type $\omega$.
\Hint{show that it is $\tau$-generated by
$\{\ocint{-\infty,q}^{\ssbullet}:q\in\Bbb Q\}$.}
%331H

\spheader 331Xe Show that the measure algebra of Lebesgue measure on
$\BbbR^r$ is \Mth\ with Maharam type $\omega$, for any
$r\ge 1$. \Hint{show that it is $\tau$-generated by
$\{\ocint{-\infty,q}^{\ssbullet}:q\in\Bbb Q^r\}$.}
%331H, 331Xd

\spheader 331Xf Show that the measure algebra of any Radon measure on
$\BbbR^r$ (256Ad) has countable Maharam type.    \Hint{show that
it is $\tau$-generated by
$\{\ocint{-\infty,q}^{\ssbullet}:q\in\Bbb Q^r\}$.}
%331H, 331Xd, 331Xe

\sqheader 331Xg Show that $\Cal P\Bbb R$ has Maharam type $\omega$.
\Hint{show that it is $\tau$-generated by
$\{\ocint{-\infty,q}:q\in\Bbb Q\}$.}
%331H

\sqheader 331Xh Show that the regular open algebra of $\Bbb R$ is
\Mth\ with Maharam type $\omega$.   \Hint{show that it is
$\tau$-generated by $\{\ocint{-\infty,q}^{\ssbullet}:q\in\Bbb Q\}$.}
%331H

\spheader 331Xi Let $(\frak A,\bar\mu)$ be a totally finite measure
algebra, and $\kappa$ an infinite cardinal.   Suppose that there is a
family $\langle a_{\xi}\rangle_{\xi<\kappa}$ in $\frak A$ such that
$\inf_{\xi\in I}a_{\xi}=0$, $\sup_{\xi\in I}a_{\xi}=1$ for every
infinite $I\subseteq\kappa$.   Show that $\tau(\frak A_a)\ge\kappa$ for
every non-zero principal ideal $\frak A_a$ of $\frak A$.
%331K

\spheader 331Xj Let $\frak A$ be the measure algebra of Lebesgue measure
on $\Bbb R$, and $\frak G$ the regular open algebra of $\Bbb R$.   Show
that the simple product $\frak A\times\frak G$ is \Mth\ of
Maharam type $\omega$, but is not homogeneous.   \Hint{$\frak A$ is
weakly
$(\sigma,\infty)$-distributive, but $\frak G$ is not, so they are not
isomorphic.}
%316N

\spheader 331Xk Show that a homogeneous semi-finite measure algebra is
$\sigma$-finite.
%316N

\spheader 331Xl Let $(X,\Sigma,\mu)$ be a measure space, and $A$ a subset
of $X$ which has a measurable envelope.
Show that the Maharam type of the subspace measure on $A$ is less
than or equal to the Maharam type of $\mu$.
%331H  useful in 552M (i) => (iii)

\spheader 331Xm Let $\frak A$ be a Boolean algebra, and $\frak B$ an
order-dense subalgebra of $\frak A$.   Show that
$\tau(\frak A)\le\tau(\frak B)$.
%331H

\spheader 331Xn Let $\mu$ be a semi-finite measure, and
$\tilde\mu$ the c.l.d.\ version of $\mu$.   Show that the Maharam type
of $\tilde\mu$ is at most the Maharam type of $\mu$.   \Hint{322Db.}
%331Xm 331H

\spheader 331Xo Let $\familyiI{\mu_i}$ be a non-empty
countable family of $\sigma$-finite
measures all with the same domain;  let $\mu$ be the sum measure
$\sum_{i\in I}\mu_i$.   Writing $\tau(\mu)$, $\tau(\mu_i)$ for the Maharam
types of the measures, show that
$\sup_{i\in I}\tau(\mu_i)
\le\tau(\mu)\le\max(\omega,\sup_{i\in I}\tau(\mu_i))$.
%331H

\leader{331Y}{Further exercises (a)}
%\spheader 331Ya
Suppose that $\frak A$ is a Dedekind complete Boolean algebra,
$\frak B$ is an order-closed subalgebra of $\frak A$ and $\frak C$ is an
order-closed
subalgebra of $\frak B$.   Show that if $a\in\frak A$ is a relative atom
in $\frak A$ over $\frak C$, then $\upr(a,\frak B)$ is a relative atom
in $\frak B$ over $\frak C$.   So if $\frak B$ is relatively atomless
over $\frak C$, then $\frak A$ is relatively atomless over $\frak C$.
%331B

\spheader 331Yb Give an example of a Boolean algebra $\frak A$
with a subalgebra $\frak A'$ and a proper subalgebra $\frak B$ of $\frak
A'$ which is order-closed in $\frak A'$, but $\tau$-generates $\frak A$.
\Hint{take $\frak A$ to be the measure algebra $\frak A_L$ of
Lebesgue measure on $\Bbb R$ and $\frak B$ the subalgebra $\frak B_{\Bbb
Q}$ of $\frak A$ generated by $\{[a,b]^{\ssbullet}:a,\,b\in\Bbb Q\}$.
Take $E\subseteq\Bbb R$ such that $I\cap E$, $I\setminus E$ have
non-zero measure for every
non-trivial interval $I\subseteq\Bbb R$ (134Jb), and let $\frak A'$ be
the subalgebra of $\frak A$ generated by
$\frak B\cup\{E^{\ssbullet}\}$.}
%331E

\spheader 331Yc Give an example of a Boolean algebra $\frak A$
with a subalgebra $\frak A'$ and a proper subalgebra $\frak B$ of
$\frak A'$ which is order-closed in $\frak A$, but $\tau$-generates $\frak A'$.
\Hint{in the notation of 331Yb, take $Z$ to be the Stone space of
$\frak A_L$, and set $\frak A'=\{\widehat{a}:a\in\frak A_L\}$,
$\frak B=\{\widehat{a}:a\in\frak B_{\Bbb Q}\}$;  let $\frak A$ be the
subalgebra of $\Cal PZ$ generated by $\frak A'\cup\{\{z\}:z\in Z\}$.}
%331E

\spheader 331Yd Let $\frak A$ be a Dedekind complete purely atomic
Boolean algebra, and $A$ the set of its atoms.   Show that
$\tau(\frak A)$ is the least
cardinal $\kappa$ such that $\#(A)\le 2^{\kappa}$.
%331F, 331Xc

\spheader 331Ye Let $(\frak A,\bar\mu)$ be a measure algebra.   Write
$d(\frak A)$ for the smallest cardinal of any subset of $\frak A$ which
is dense for the measure-algebra topology.   Show that
$d(\frak A)\le\max(\omega,\tau(\frak A))$.   Show that if
$(\frak A,\bar\mu)$ is semi-finite, then $\tau(\frak A)\le d(\frak A)$.
%331F mt33bits

\spheader 331Yf Let $(X,\rho)$ be a metric space.   Write $d(X)$ for the
{\bf density} of $X$, the smallest cardinal of any dense subset of $X$.
(i) Show that if $\Cal G$ is any family of open subsets of $X$, there is
a family $\Cal H\subseteq\Cal G$ such that $\bigcup\Cal H=\bigcup\Cal G$
and $\#(\Cal H)\le\max(\omega,d(X))$.   (ii) Show that if
$\kappa>\max(\omega,d(X))$ and $\langle x_{\xi}\rangle_{\xi<\kappa}$ is
any family in $X$, then there is an $x\in X$ such that
$\#(\{\xi:x_{\xi}\in G\})>\max(\omega,d(X))$ for every open set $G$
containing $x$, and that there is a strictly increasing
sequence $\sequencen{\xi_n}$ in $\kappa$ such that $x=\lim_{n\to\infty}x_{\xi_n}$.
%331J

\spheader 331Yg Let $(\frak A,\bar\mu)$ be the simple product (322L) of
$\omega_1$ copies of the measure algebra of the usual measure on
$\{0,1\}^{\omega_1}$.   Show that $\frak A$ is \Mth\ but not
homogeneous.
%316N

\spheader 331Yh Let $\kappa$ be an infinite cardinal,
$\nu_{\kappa}$ the usual measure on $\{0,1\}^{\kappa}$ and
$(\frak B_{\kappa},\bar\nu_{\kappa})$ its measure algebra.   Suppose that
$(\frak A,\bar\mu)$ is a totally finite measure algebra and such that
$\tau(\frak A)<\kappa$, and $\pi:\frak B_{\kappa}\to\frak A$
a Boolean homomorphism.   Show that (i) for every $\epsilon>0$ there is a
$b\in\frak B_{\kappa}$ such that $\bar\nu_{\kappa}b\ge 1-\epsilon$ and
$\bar\mu(\pi b)\le\epsilon$ (ii) $\pi$ is not injective.
%331J out of order query

\spheader 331Yi Give an example of a semi-finite measure space
$(X,\Sigma,\mu)$ such that the Maharam type of $\mu$ is greater than the
Maharam type of its c.l.d.\ version.
%331Xn 331H

\spheader 331Yj\dvAnew{2011}
Let $(\frak A,\bar\mu)$ be a semi-finite measure algebra
which is separable when given
its measure-algebra topology.   Show that it has countable Maharam type.
%331O
}%end of exercises

\endnotes{
\Notesheader{331}
Maharam's theorem belongs with the Radon-Nikod\'ym theorem, Fubini's
theorem and the strong law of large numbers as one of the theorems which
make measure theory what it is.   Once you have this theorem and its
consequences in the next section properly absorbed, you will never again
look at a measure space without trying to classify its measure algebra in
terms of the Maharam types of its homogeneous principal ideals.
As one might expect,
a very large proportion of the important measure algebras of analysis are
homogeneous, and indeed a great many are homogeneous with Maharam type
$\omega$.

In this section I have contented myself with the basic statement of
Theorem 331I on the isomorphism of \Mth\ measure algebras
and the identification of representative
homogeneous probability algebras (331K).   The same techniques lead to
an enormous number of further facts, some of which I will describe in
the rest of the chapter.
For the moment, it gives us a complete description of \Mth\ probability algebras (331L).   There is the atomic algebra
$\{0,1\}$, with Maharam type $0$, and for each infinite cardinal
$\kappa$ there is the measure algebra of $\{0,1\}^{\kappa}$, with
Maharam type $\kappa$;  these are all non-isomorphic, and
every \Mth\ probability algebra is isomorphic to exactly
one of them.   The isomorphisms here are not unique;  indeed, it is
characteristic of measure algebras that they have very large
automorphism groups (see Chapter 38 below), and there are
correspondingly large numbers of isomorphisms between any isomorphic
pair.   The proof of 331I already suggests this, since we have such a
vast amount of choice concerning the lists
$\langle a_{\xi}\rangle_{\xi<\kappa}$ and
$\langle b_{\xi}\rangle_{\xi<\kappa}$, and even with these fixed there
remains a good deal of scope in the choice of
$\langle a'_{\xi}\rangle_{\xi<\kappa}$ and
$\langle b'_{\xi}\rangle_{\xi<\kappa}$.

The isomorphisms described in Theorem 331I are measure algebra
isomorphisms, that is, measure-preserving Boolean isomorphisms.
Obvious questions arise concerning Boolean isomorphisms which are not
necessarily measure-preserving;  the theorem also helps us to settle
many of these (see 331N).   But we can observe straight
away the remarkable fact that two homogeneous probability
algebras which are isomorphic as Boolean algebras are also isomorphic as
probability algebras, since they must have the same Maharam type.

I have already mentioned certain measure space isomorphisms (254K,
255A).   Of course any isomorphism between measure spaces must induce an
isomorphism between their measure algebras (see 324M), and any
isomorphism between measure algebras corresponds to an isomorphism
between their Stone spaces (see 324N).   But there are many important
examples of isomorphisms between measure algebras which do not
correspond to isomorphisms between the measure spaces most naturally
involved.   (I describe one in 343J.)   Maharam's theorem really is a
theorem about measure {\it algebras} rather than measure {\it spaces}.

The particular method I use to show that the measure algebra of the
usual measure on $\{0,1\}^{\kappa}$ is homogeneous for infinite $\kappa$
(331J-331K) is chosen with a view to a question in the next section
(332O).   There are other ways of doing it.   But I recommend study of
this particular one because of the way in which it involves the
topological, algebraic and order properties of the algebra $\frak B$.
I have extracted some of the elements of the argument in 331Xi and
331Ye-331Yf.   These use the concept of `density' of a topological
space.   This does not seem the moment to go farther along this road,
but I hope you can see that there are likely to be many further
`cardinal functions' to provide useful measures of complexity in both
algebraic and topological structures.
}%end of comment

\discrpage





