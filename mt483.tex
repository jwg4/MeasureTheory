\frfilename{mt483.tex}
\versiondate{6.9.10}
\copyrightdate{2000}

\def\chaptername{Gauge integrals}
\def\sectionname{The Henstock integral}
\def\FSH{F^{_{\text{SH}}}}

\newsection{483}

I come now to the original gauge integral, the `Henstock integral' for
real functions.   The first step is to check that the results of \S482
can be applied to show that this is an extension of both the Lebesgue
integral and the improper Riemann integral (483B), coinciding with
the Lebesgue integral for non-negative functions (483C).   It turns out
that any Henstock integrable function can be approximated in a strong
sense by a sequence of Lebesgue integrable functions (483G).   The
Henstock integral can be identified with the Perron and special Denjoy
integrals (483J, 483N\cmmnt{, 483Yh}).
Much of the rest of the section is concerned
with indefinite Henstock integrals.   Some of the results of \S482 on
tagged-partition structures allowing subdivisions condense into a
particularly strong Saks-Henstock lemma (483F).   If $f$ is Henstock
integrable, it is equal almost everywhere to the derivative of its
indefinite Henstock integral (483I).   
Finally, indefinite Henstock integrals can be characterized as 
continuous ACG$_*$ functions (483R).

\leader{483A}{Definition} The following notation will apply throughout
the section.   Let $\Cal C$ be the family of non-empty bounded intervals
in $\Bbb R$, and let
$T\subseteq[\Bbb R\times\Cal C]^{<\omega}$ be the straightforward set of
tagged partitions generated by
$\{(x,C):C\in\Cal C$, $x\in\overline{C}\}$.   Let $\Delta$ be the set of
all neighbourhood gauges on $\Bbb R$.    Set
$\frak R=\{\Cal R_{ab}:a\le b\in\Bbb R\}$, where
$\Cal R_{ab}=\{\Bbb R\setminus[c,d]:c\le a,\,d\ge b\}\cup\{\emptyset\}$.
Then $(\Bbb R,T,\Delta,\frak R)$ is a tagged-partition structure
allowing subdivisions\cmmnt{ (481K)}, so $T$ is compatible with
$\Delta$ and $\frak R$\cmmnt{ (481Hf)}.   The {\bf Henstock integral}
is the gauge integral defined\cmmnt{ by the process of 481E-481F} from
$(\Bbb R,T,\Delta,\frak R)$ and one-dimensional
Lebesgue measure $\mu$.   For a function
$f:\Bbb R\to\Bbb R$ I will say that $f$ is {\bf Henstock integrable},
and that $\Heint f=\gamma$, if
$\lim_{\pmb{t}\to\Cal F(T,\Delta,\frak R)}S_{\pmb{t}}(f,\mu)$ is defined
and equal to $\gamma\in\Bbb R$.   For $\alpha$,
$\beta\in[-\infty,\infty]$ I will write $\Heint_{\alpha}^{\beta}f$ for
$\Heint f\times\chi\ooint{\alpha,\beta}$ if this is defined in $\Bbb R$.
I will use the symbol $\int$ for the ordinary integral, so that
$\int fd\mu$ is the Lebesgue integral of $f$.

\leader{483B}{}\cmmnt{ Tracing through the theorems of \S482, we have
the following.

\medskip

\noindent}{\bf Theorem} (a) Every Henstock integrable function on
$\Bbb R$ is Lebesgue measurable.

(b) Every Lebesgue integrable function $f:\Bbb R\to\Bbb R$ is Henstock
integrable, with the same integral.

(c) If $f$ is Henstock integrable so is $f\times\chi C$ for every
interval $C\subseteq\Bbb R$.

(d) Suppose that $f:\Bbb R\to\Bbb R$ is any function, and
$-\infty\le\alpha<\beta\le\infty$.   Then

\Centerline{$\biggerHeint_{\alpha}^{\beta}f
=\lim_{a\downarrow\alpha}\biggerHeint_a^{\beta}f
=\lim_{b\uparrow\beta}\biggerHeint_{\alpha}^bf
=\lim_{a\downarrow\alpha,b\uparrow\beta}\biggerHeint_a^bf$}

\noindent if any of the four terms is defined in $\Bbb R$.

\proof{{\bf (a)} Apply 482E.

\medskip

{\bf (b)} Apply 482F, referring to 134F to confirm that condition
482F(ii) is satisfied.

It will be useful to note at once that this shows that
$\Heint f\times\chi\{a\}=0$ for every $f:\Bbb R\to\Bbb R$ and every
$a\in\Bbb R$.   Consequently $\Heint_a^bf=\Heint_a^cf+\Heint_c^bf$ whenever
$a\le c\le b$ and the right-hand side is defined.

\medskip

{\bf (c)-(d)} In the following argument, $f$ will always be a function from
$\Bbb R$ to itself;  when $f$ is Henstock integrable, $\FSH$ will be its
Saks-Henstock indefinite integral (482B-482C).

\medskip

\quad{\bf (i)} The first thing to check is that the conditions of 482G
are satisfied by $\Bbb R$, $T$, $\Delta$, $\frak R$, $\Cal C$ and
$\mu\restr\Cal C$.   \Prf\  482G(i) is just 481K, and 482G(ii) is trivial.
482G(iii-$\alpha$) and 482G(v) are elementary, and so is
482G(iii-$\beta$) --- if you like, this is a special case of
412W(b-iii).   As for 482G(iv), if
$E\in\Cal C$ is a singleton $\{x\}$, then whenever
$(x,C)\in\Cal C$ we can express $C$ as a union of one or more of the
sets $C\cap\ooint{-\infty,x}$, $C\cap\{x\}$ and $C\cap\ooint{x,\infty}$,
and for any non-empty $C'$ of these we have $\{(x,C')\}\in T$ and either
$C'\subseteq E$ or
$C'\cap E=\emptyset$.   Otherwise, let $\eta>0$ be half the length of
$E$, and let $\delta$ be the uniform metric gauge
$\{(x,A):A\subseteq\ooint{x-\eta,x+\eta}\}$.    Then if $x\in\partial E$
and $(x,C)\in T\cap\delta$, we can again express $C$ as a union
of one or more of the sets
$C\cap\ooint{-\infty,x}$, $C\cap\{x\}$ and $C\cap\ooint{x,\infty}$, and
these will witness that 482G(iv) is satisfied.\ \Qed

\medskip

\quad{\bf (ii)} Now suppose that $f$ is Henstock integrable.   Then
$\Heint_a^bf$ is defined whenever $a\le b$ in $\Bbb R$.
\Prf\ 482G tells us that $\Heint f\times\chi C$ is
defined and equal to $\FSH(C)$ for every $C\in\Cal C$;  in particular,
$\Heint_a^bf=\Heint f\times\chi\ooint{a,b}$ is defined whenever $a<b$
in $\Bbb R$.\ \QeD\   Note that because
$\Heint f\times\chi\{c\}=0$ for every $c$,
$\Heint f\times\chi C=\Heint_{\inf C}^{\sup C}f$ whenever $C\in\Cal C$ is
non-empty.

This proves (c) for bounded intervals;  we shall come to unbounded
intervals in (vii) below.

\medskip

\quad{\bf (iii)} If $f$ is Henstock integrable, then
$\lim_{a\to-\infty,b\to\infty}\Heint_a^bf$ is defined and equal to
$\Heint f$.   \Prf\ Given
$\epsilon>0$, there is an $\Cal R\in\frak R$ such that
$|\FSH(\Bbb R\setminus C)|\le\epsilon$ whenever $C\in\Cal C$ and
$\Bbb R\setminus C\in\Cal R$;  that is, there are $a_0\le b_0$ such that

\Centerline{$|\biggerHeint f-\biggerHeint_a^bf|
=|\FSH(\Bbb R\setminus\ooint{a,b})|=|\FSH(\Bbb R\setminus[a,b])|\le\epsilon$}

\noindent whenever $a\le a_0\le b_0\le b$.
As $\epsilon$ is arbitrary, $\lim_{a\to-\infty,b\to\infty}\Heint_a^bf$
is defined and equal to $\Heint f$.\ \Qed

\medskip

\quad{\bf (iv)} It follows that if $f$ is Henstock integrable and
$c\in\Bbb R$, $\lim_{b\to\infty}\Heint_c^bf$ is defined.   \Prf\ Let
$\epsilon>0$.   Then there are $a_0\le c$, $b_0\ge c$ such that
$|\Heint_a^bf-\Heint f|\le\epsilon$ whenever $a\le a_0$ and $b\ge b_0$.
But this means that

\Centerline{$|\biggerHeint_c^bf-\biggerHeint_c^{b'}f|
=|\biggerHeint_{a_0}^bf-\biggerHeint_{a_0}^{b'}f|\le 2\epsilon$}

\noindent whenever $b$, $b'\ge b_0$.   As $\epsilon$ is arbitrary,
$\lim_{b\to\infty}\int_c^bf$ is defined.\ \Qed

Similarly, $\lim_{a\to-\infty}\Heint_a^cf$ is defined.

\medskip

\quad{\bf (v)} Moreover, if $f$ is Henstock integrable and
$a<b$ in $\Bbb R$, then $\lim_{c\uparrow b}\Heint_a^cf$ is defined and
equal to $\int_a^bf$.   \Prf\ Let $\epsilon>0$.   Then
there is a $\delta\in\Delta$ such that
$\sum_{(x,C)\in\pmb{t}}|\FSH(C)-f(x)\mu C|\le\epsilon$ whenever
$\pmb{t}\in T$ is $\delta$-fine.   Let $\eta>0$ be such that
$\eta|f(b)|\le\epsilon$ and $(b,\coint{c,b})\in\delta$ whenever
$b-\eta\le c<b$.   Then whenever $\max(a,b-\eta)\le c<b$,

\Centerline{$\biggerHeint_a^bf-\biggerHeint_a^cf
=\FSH(\coint{c,b})
\le|\FSH(\coint{c,b})-f(b)\mu\coint{c,b}|+|f(b)|\mu\coint{c,b}
\le 2\epsilon$.}

\noindent As $\epsilon$ is arbitrary,
$\lim_{c\uparrow b}\Heint_a^cf=\Heint_a^bf$.\ \Qed

Similarly, $\lim_{c\downarrow a}\Heint_c^bf$ is defined and equal to
$\Heint_a^bf$.

\medskip

\quad{\bf (vi)} Now for a much larger step.
If $-\infty\le\alpha<\beta\le\infty$ and $f:\Bbb R\to\Bbb R$ is such that
$\lim_{a\downarrow\alpha,b\uparrow\beta}\Heint_a^bf$ is defined and equal
to $\gamma$, then $\Heint_{\alpha}^{\beta}f$ is defined and equal to
$\gamma$.
\Prf\ I seek to apply 482H, with $H=\ooint{\alpha,\beta}$ and
$H_n=\ooint{a_n,b_n}$, where $\sequencen{a_n}$ is a strictly decreasing
sequence with limit $\alpha$, $\sequencen{b_n}$ is a strictly increasing
sequence with limit $\beta$, and $a_0<b_0$.
We have already seen that the conditions of
482G are satisfied;  482H(vi) is elementary, and 482H(vii) is covered by
(ii) above.   So we are left with 482H(viii).   Given
$\epsilon>0$, let $m\in\Bbb N$ be such that
$|\gamma-\Heint_a^bf|\le\epsilon$ whenever $\alpha<a\le a_m$ and
$b_m\le b<\beta$.   For $x\in\Bbb R$ let $G_x$ be an open set,
containing $x$, such that

$$\eqalign{\overline{G}_x
&\subseteq H\text{ if }x\in H,\cr
&\subseteq\ooint{-\infty,a_m}\text{ if }x<a_m,\cr
&\subseteq\ooint{b_m,\infty}\text{ if }x>b_m,\cr}$$

\noindent and let $\delta\in\Delta$ be the neighbourhood gauge
corresponding to $\family{x}{\Bbb R}{G_x}$.
Now suppose that $\pmb{t}\in T$ is $\delta$-fine and
$\Cal R_{a_mb_m}$-filling.
Then $W_{\pmb{t}}$ is a closed bounded interval
including $[a_m,b_m]$.   Since $\family{(x,C)}{\pmb{t}}{C}$ is a disjoint
family of intervals, $\pmb{t}$ must have an
enumeration $\langle(x_i,C_i)\rangle_{i\le r}$ where $x\le y$ whenever
$i\le j\le r$, $x\in C_i$ and $y\in C_j$.   As $H\subseteq\Bbb R$ is an
interval, there are $i_0\le i_1$ such that

\Centerline{$\pmb{t}\restr H=\{(x_i,C_i):i\le r$, $x_i\in H\}
=\{(x_i,C_i):i_0\le i\le i_1\}$.}

\noindent Because $W_{\pmb{t}}$ is an interval, so is
$W_{\pmb{t}\restr H}=\bigcup_{i_0\le i\le i_1}C_i$;
set $a=\inf W_{\pmb{t}\restr H}$ and
$b=\sup W_{\pmb{t}\restr H}$;  then
$\Heint f\times\chi W_{\pmb{t}}=\Heint_a^bf$ (see (ii)).
Next, $\overline{W}_{\pmb{t}\restr H}\subseteq H$ (because
$\overline{G}_x\subseteq H$ if $x\in H$) and
$[a_m,b_m]\subseteq W_{\pmb{t}\restr H}$
(because $[a_m,b_m]\subseteq W_{\pmb{t}}$ and
$\overline{G}_x\cap[a_m,b_m]=\emptyset$ for $x\notin[a_m,b_m]$).
So $\alpha<a\le a_m$, $b_m\le b<\beta$, and

\Centerline{$|\gamma-\biggerHeint f\times\chi W_{\pmb{t}\restr H}|
=|\gamma-\biggerHeint_a^bf|\le\epsilon$.}

As $\epsilon$ is arbitrary, this shows that

\Centerline{$\lim_{\pmb{t}\to\Cal F(T,\Delta,\frak R)}
\biggerHeint f\times\chi W_{\pmb{t}\restr H}=\gamma$,}

\noindent as required by 482H(viii).   So
$\Heint_{\alpha}^{\beta}f=\Heint f\times\chi H$ is defined and equal to
$\gamma$, as claimed.\ \Qed

\medskip

\quad{\bf (vii)} We are now in a position to confirm that if $f$ is Henstock
integrable then $\Heint_c^{\infty}f=\lim_{b\to\infty}\Heint_c^bf$ is
defined for every $c\in\Bbb R$.
\Prf\ By (iii) and (iv),
$\lim_{a\downarrow c}\Heint_a^{c+1}f=\Heint_c^{c+1}f$ and
$\lim_{b\to\infty}\Heint_{c+1}^bf$ are both defined.   So

$$\eqalign{\lim_{a\downarrow c,b\to\infty}\Heint_a^bf
&=\lim_{a\downarrow c}\Heint_a^{c+1}f
+\lim_{b\to\infty}\Heint_{c+1}^bf\cr
&=\Heint_c^{c+1}f+\lim_{b\to\infty}\Heint_{c+1}^b
=\lim_{b\to\infty}\Heint_c^bf\cr}$$

\noindent is defined, and is equal to $\Heint_c^{\infty}f$, by (vi).\
\Qed

Similarly, $\Heint_{-\infty}^cf=\lim_{a\to-\infty}\Heint_a^cf$ is
defined.   So $\Heint f\times\chi C$ is defined for sets $C$ of the form
$\ooint{c,\infty}$ or $\ooint{-\infty,c}$, and therefore for any unbounded
interval, since the case $C=\Bbb R$ is immediate.   So the proof of (c) is
complete.

\medskip

\quad{\bf (viii)} As for (d), (vi) has already given us part of it:  if
$\lim_{a\downarrow\alpha,b\uparrow\beta}\Heint_a^bf$ is defined, this is
$\Heint_{\alpha}^{\beta}f$.   In the other direction, if
$\Heint_{\alpha}^{\beta}f$ is defined, set
$g=f\times\chi\ooint{\alpha,\beta}$, so that $g$ is Henstock integrable,
and take any
$c\in\ooint{\alpha,\beta}$.   Then $\lim_{b\uparrow\beta}\Heint_c^bg$ is
defined, and equal to $\Heint_c^{\beta}g$, by (v) if $\beta$ is finite
and by (vii) if $\beta=\infty$.   Similarly,
$\lim_{a\downarrow\alpha}\Heint_a^cg$ is defined and equal to
$\Heint_{\alpha}^cg$.   Consequently

\Centerline{$\lim_{a\downarrow\alpha,b\uparrow\beta}\biggerHeint_a^bf
=\lim_{a\downarrow\alpha,b\uparrow\beta}\biggerHeint_a^bg
=\lim_{a\downarrow\alpha}\biggerHeint_a^cg
+\lim_{b\uparrow\beta}\biggerHeint_c^bg$}

\noindent is defined, and must be equal to
$\Heint_{\alpha}^{\beta}g=\Heint_{\alpha}^{\beta}f$, while also

\Centerline{$\lim_{a\downarrow\alpha}\biggerHeint_a^{\beta}f
=\lim_{a\downarrow\alpha}\biggerHeint_a^{\beta}g
=\lim_{a\downarrow\alpha}\biggerHeint_a^cg
   +\biggerHeint_c^{\beta}g
=\biggerHeint_{\alpha}^cg+\biggerHeint_c^{\beta}g
=\biggerHeint_{\alpha}^{\beta}g
=\biggerHeint_{\alpha}^{\beta}f$,}

\noindent and similarly

\Centerline{$\lim_{b\uparrow\beta}\biggerHeint_{\alpha}^bf
=\biggerHeint_{\alpha}^{\beta}f$.}

\medskip

\quad{\bf (ix)} Finally, we need to consider the case in which we are told
that $\lim_{a\downarrow\alpha}\Heint_a^{\beta}f$ is defined.
Taking any $c\in\ooint{\alpha,\beta}$, we know that $\Heint_c^{\beta}f$
is defined, by (ii) or (vii) applied to $f\times\chi\ooint{a,\beta}$ for
some $a\le c$, and equal to
$\lim_{b\uparrow\beta}\Heint_c^bf$, by (v) or (vii).   But this means that

\Centerline{$\lim_{a\downarrow\alpha,b\uparrow\beta}\biggerHeint_a^bf
=\lim_{a\downarrow\alpha}\biggerHeint_a^cf
+\lim_{b\uparrow\beta}\biggerHeint_c^bf$}

\noindent is defined, so (vi) and (viii) tell us that
$\Heint_{\alpha}^{\beta}f$ is defined and equal to
$\lim_{a\downarrow\alpha}\Heint_a^{\beta}f$.   The same argument, suitably
inverted, deals with the case in which
$\lim_{b\uparrow \beta}\Heint_{\alpha}^bf$ is defined.
}%end of proof of 483B

\leader{483C}{Corollary} The Henstock and Lebesgue integrals agree on
non-negative functions, in the sense that if
$f:\Bbb R\to\coint{0,\infty}$ then $\Heint f=\int fd\mu$ if either is
defined in $\Bbb R$.

\proof{ If $f$ is Lebesgue integrable, it is Henstock integrable, with
the same integral, by 483Bb.   If it is Henstock integrable, then it is
measurable, by 483Ba, so that $\int fd\mu$ is defined in $[0,\infty]$;
but

$$\eqalignno{\int fd\mu
&=\sup\{\int g\,d\mu:
  g\le f\text{ is a non-negative simple function}\}\cr
\displaycause{213B}
&=\sup\{\Heint g:g\le f\text{ is a non-negative simple function}\}
\le\Heint f\cr}$$

\noindent (481Cb) is finite, so $f$ is Lebesgue integrable.
}%end of proof of 483C

\leader{483D}{Corollary} If $f:\Bbb R\to\Bbb R$ is Henstock integrable,
then
$\alpha\mapsto\Heint_{-\infty}^{\alpha}f:[\infty,\infty]\to\Bbb R$ and
$(\alpha,\beta)\mapsto\Heint_{\alpha}^{\beta}f:
[-\infty,\infty]^2\to\Bbb R$ are continuous and bounded.

\proof{ Let $F$ be the indefinite Henstock integral of $f$.   Take any
$x_0\in\Bbb R$ and $\epsilon>0$.   By 483Bd, there is an
$\eta_1>0$ such that
$|\Heint_{-\infty}^{x}f-\Heint_{-\infty}^{x_0}f|\le\epsilon$
whenever $x_0-\eta_1\le x\le x_0$.   By 483Bd again, there
is an $\eta_2>0$ such that
$|\Heint_{x}^{\infty}f-\Heint_{x_0}^{\infty}f|\le\epsilon$
whenever $x_0\le x\le x_0+\eta_2$.   But this means that
$|F(x)-F(x_0)|\le\epsilon$ whenever
$x_0-\eta_1\le x\le x_0+\eta_2$.   As $\epsilon$ is
arbitrary, $F$ is continuous at $x_0$.

We know also that $\lim_{x\to\infty}F(x)=\Heint f$ is defined
in $\Bbb R$;  while

\Centerline{$\lim_{x\to-\infty}F(x)
=\biggerHeint f-\lim_{x\to-\infty}\biggerHeint_x^{\infty}f=0$}

\noindent is also defined, by 483Bd once more.   So $F$ is
continuous at $\pm\infty$.

Now writing $G(\alpha,\beta)=\Heint_{\alpha}^{\beta}f$, we have
$G(\alpha,\beta)=F(\beta)-F(\alpha)$ if $\alpha\le\beta$ and zero if
$\beta\le\alpha$.   So $G$ also is continuous.   $F$ and $G$ are bounded
because $[-\infty,\infty]$ is compact.
}%end of proof of 483D

\leader{483E}{Definition} If $f:\Bbb R\to\Bbb R$ is Henstock integrable,
then its {\bf indefinite Henstock integral} is the function
$F:\Bbb R\to\Bbb R$ defined by saying that $F(x)=\Heint_{-\infty}^xf$
for every $x\in\Bbb R$.

\leader{483F}{}\cmmnt{ In the present context, the Saks-Henstock lemma
can be sharpened, as follows.

\medskip

\noindent}{\bf Theorem} Let $f:\Bbb R\to\Bbb R$ and
$F:\Bbb R\to\Bbb R$ be functions.   Then the following are
equiveridical:

(i) $f$ is Henstock integrable and $F$ is its indefinite Henstock
integral;

(ii)($\alpha$) $F$ is continuous,

\quad($\beta$) $\lim_{x\to-\infty}F(x)=0$ and $\lim_{x\to\infty}F(x)$ is
defined in $\Bbb R$,

\quad($\gamma$) for every $\epsilon>0$ there are a gauge
$\delta\in\Delta$ and a non-decreasing function
$\phi:\Bbb R\to[0,\epsilon]$ such that
$|f(x)(b-a)-F(b)+F(a)|\le\phi(b)-\phi(a)$ whenever $a\le x\le b$ in
$\Bbb R$ and $(x,[a,b])\in\delta$.

\proof{{\bf (i)$\Rightarrow$(ii)} ($\alpha$)-($\beta$) are covered by
483D.   As for ($\gamma$), 482G tells us that we can identify the
Saks-Henstock indefinite integral of $f$ with
$E\mapsto\Heint f\times\chi E:\Cal E\to\Bbb R$, where $\Cal E$ is the
algebra generated by $\Cal C$.   Let
$\epsilon>0$.   Then there is a $\delta\in\Delta$ such
that $\sum_{(x,C)\in\pmb{t}}|f(x)\mu C-\Heint f\times\chi C|\le\epsilon$
for every $\delta$-fine $\pmb{t}\in T$.   Set

\Centerline{$\phi(a)
=\sup_{\pmb{t}\in T\cap\delta}
\sum_{(x,C)\in\pmb{t},C\subseteq\ocint{-\infty,a}}
$$|f(x)\mu C-\biggerHeint f\times\chi C|$,}

\noindent so that $\phi:\Bbb R\to[0,\epsilon]$ is a non-decreasing
function.   Now suppose that $a\le y\le b$ and that
$(y,[a,b])\in\delta$.   In this case, whenever $\pmb{t}\in T\cap\delta$,
$\pmb{s}=\{(x,C):(x,C)\in\pmb{t},\,
C\subseteq\ocint{-\infty,a}\}\cup\{(y,\ooint{a,b})\}$ also belongs to
$T\cap\delta$.   Now $F(b)-F(a)=\Heint_a^bf$, so

$$\eqalign{\phi(b)
&\ge\sum_{(x,C)\in\pmb{s}}|f(x)\mu C-\Heint f\times\chi C|\cr
&=\sum_{(x,C)\in\pmb{t},C\subseteq\ocint{-\infty,a}}
  |f(x)\mu C-\Heint f\times\chi C|
  +|f(y)(b-a)-F(b)+F(a)|.\cr}$$

\noindent As $\pmb{t}$ is arbitrary,

\Centerline{$\phi(b)\ge\phi(a)+|f(y)(b-a)-F(b)+F(a)|$,}

\noindent that is, $|f(y)(b-a)-F(b)+F(a)|\le\phi(b)-\phi(a)$, as called
for by ($\gamma$).

\medskip

{\bf (ii)$\Rightarrow$(i)} Assume (ii).   Set
$\gamma=\lim_{x\to\infty}F(x)$.   Let $\epsilon>0$.   Let $a\le b$ be
such that $|F(x)|\le\epsilon$ for every $x\le a$ and
$|F(x)-\gamma|\le\epsilon$ for every $x\ge b$.   Let $\delta\in\Delta$,
$\phi:\Bbb R\to[0,\epsilon]$ be such that $\phi$ is non-decreasing and
$|f(x)(\beta-\alpha)-F(\beta)+F(\alpha)|\le\phi(\beta)-\phi(\alpha)$
whenever $\alpha\le x\le\beta$ and $(x,[\alpha,\beta])\in\delta$.   Let
$\delta'\in\Delta$ be such that $(x,\overline{A})\in\delta$ whenever
$(x,A)\in\delta'$.   For $C\in\Cal C$ set
$\lambda C=F(\sup C)-F(\inf C)$, $\nu C=\phi(\sup C)-\phi(\inf C)$;
then if $(x,C)\in\delta'$, $(x,[\inf C,\sup C])\in\delta$, so
$|f(x)\mu C-\lambda C|\le\nu C$.   Note that $\lambda$ and $\nu$ are
both additive in the sense that
$\lambda(C\cup C')=\lambda C+\lambda C'$, $\nu(C\cup C')=\nu C+\nu C'$
whenever $C$, $C'$ are disjoint members of $\Cal C$ such that
$C\cup C'\in\Cal C$ (cf.\ 482G(iii-$\alpha$)).

Let $\pmb{t}\in T$ be $\delta'$-fine and $\Cal R_{ab}$-filling.   Then
$W_{\pmb{t}}$ is of the form $[c,d]$ where $c\le a$ and $b\le d$.   So

$$\eqalign{|S_{\pmb{t}}(f,\mu)-\gamma|
&\le 2\epsilon+|S_{\pmb{t}}(f,\mu)-F(d)+F(c)|
=2\epsilon+|S_{\pmb{t}}(f,\mu)-\lambda[c,d]|\cr
&=2\epsilon+|\sum_{(x,C)\in\pmb{t}}f(x)\mu C-\lambda C|
\le 2\epsilon+\sum_{(x,C)\in\pmb{t}}\nu C\cr
&=2\epsilon+\nu[c,d]
\le 3\epsilon.\cr}$$

\noindent As $\epsilon$ is arbitrary, $f$ is Henstock integrable, with
integral $\gamma$.

I still have to check that $F$ is the indefinite integral of $f$.   Set
$F_1(x)=\Heint_{-\infty}^xf$ for $x\in\Bbb R$, and $G=F-F_1$.   Then
(ii) applies equally to the pair $(f,F_1)$, because
(i)$\Rightarrow$(ii).   So, given $\epsilon>0$, we have
$\delta$, $\delta_1\in\Delta$ and non-decreasing functions $\phi$,
$\phi_1:\Bbb R\to[0,\epsilon]$ such that

\inset{$|f(x)(b-a)-F(b)+F(a)|\le\phi(b)-\phi(a)$ whenever $a\le x\le b$
in $\Bbb R$ and $(x,[a,b])\in\delta$,

$|f(x)(b-a)-F_1(b)+F_1(a)|\le\phi_1(b)-\phi_1(a)$ whenever $a\le x\le b$
in $\Bbb R$ and $(x,[a,b])\in\delta_1$.}

\noindent Putting these together,

\inset{$|G(b)-G(a)|\le\psi(b)-\psi(a)$ whenever $a\le x\le b$ in
$\Bbb R$ and $(x,[a,b])\in\delta\cap\delta_1$,}

\noindent where $\psi=\phi+\phi_1$.   But if $a\le b$ in $\Bbb R$, there
are $a_0\le x_0\le a_1\le x_1\le\ldots\le x_{n-1}\le a_n$ such that
$a=a_0$, $a_n=b$ and $(x_i,[a_i,a_{i+1}])\in\delta$ for $i<n$ (481J), so
that

$$\eqalign{|G(b)-G(a)|
&\le\sum_{i=0}^{n-1}|G(a_{i+1})-G(a_i)|\cr
&\le\sum_{i=0}^{n-1}\psi(a_{i+1})-\psi(a_i)
=\psi(b)-\psi(a)\le 2\epsilon.\cr}$$

\noindent As $\epsilon$ is arbitrary, $G$ is constant.   As
$\lim_{x\to-\infty}F(x)=\lim_{x\to-\infty}F_1(x)=0$, $F=F_1$, as
required.
}%end of proof of 483F

\leader{483G}{Theorem} %\cmmnt{ (see {\smc Gordon 94}, 9.18)}
Let $f:\Bbb R\to\Bbb R$ be a Henstock integrable function.
Then there is a
countable cover $\Cal K$ of $\Bbb R$ by compact sets such that
$f\times\chi K$ is Lebesgue integrable for every $K\in\Cal K$.

\proof{{\bf (a)} For $n\in\Bbb N$ set
$E_n=\{x:|x|\le n,\,|f(x)|\le n\}$.   By 483Ba, $f$ is Lebesgue
measurable, so for each $n\in\Bbb N$ we can find a compact set
$K_n\subseteq E_n$ such that $\mu(E_n\setminus K_n)\le 2^{-n}$;  $f$ is
Lebesgue integrable over $K_n$, and
$Y=\Bbb R\setminus\bigcup_{n\in\Bbb N}K_n$ is Lebesgue negligible.

Let $F$ be the indefinite Henstock integral of $f$,
and take a gauge $\delta_0\in\Delta$ and a non-decreasing function
$\phi:\Bbb R\to[0,1]$ such that
$|f(x)(b-a)-F(b)+F(a)|\le\phi(b)-\phi(a)$ whenever
$a\le x\le b$ and $(x,[a,b])\in\delta_0$ (483F).   Because
$\Heint|f|\times\chi Y=\int_Y|f|d\mu=0$ (483Bb), there is a
$\delta_1\in\Delta$ such that $S_{\pmb{t}}(|f|\times\chi Y,\mu)\le 1$
whenever $\pmb{t}\in T$ is $\delta_1$-fine (482Ad).   For $C\in\Cal C$,
set $\lambda C=F(\sup C)-F(\inf C)$, $\nu C=\phi(\sup C)-\phi(\inf C)$.
Set

\Centerline{$D_n
=\{x:x\in E_n\cap Y,\,(x,[a,b])\in\delta_0\cap\delta_1$
whenever $x-2^{-n}\le a\le x\le b\le x+2^{-n}\}$,}

\noindent so that $\bigcup_{n\in\Bbb N}D_n=Y$;  set
$K'_n=\overline{D}_n$, so that $K'_n$ is compact and
$\bigcup_{n\in\Bbb N}K'_n\supseteq Y$.

\medskip

{\bf (b)} The point is that $f\times\chi K'_n$ is Lebesgue integrable
for each $n$.   \Prf\ For $k\in\Bbb N$, let $A_k$ be the set of points
$x\in K'_n$ such that $\ocint{x,x+2^{-k}}\cap K'_n=\emptyset$;  then
$A_k$ is finite, because $K'_n\subseteq[-n,n]$ is bounded.   Similarly,
if $A'_k=\{x:x\in K'_n,\,\coint{x-2^{-k},x}\cap K'_n=\emptyset\}$,
$A'_k$ is finite.   Set
$B=K'_n\setminus\bigcup_{k\in\Bbb N}(A_k\cup A'_k)$, so
that $K'_n\setminus B$ is countable.

Set

\Centerline{$\delta
=\delta_0\cap\delta_1\cap
\{(x,A):x\in\Bbb R,\,A\subseteq\ooint{x-2^{-n-1},x+2^{-n-1}}\}$,}

\noindent so that $\delta\in\Delta$.   Note that if $C\in\Cal C$,
$x\in B\cap\overline{C}$, $(x,C)\in\delta$ and $\mu C>0$, then
$\interior C$ meets $K'_n$ (because there are points of $K'_n$
arbitrarily close to $x$ on both sides) so $\interior C$ meets
$D_n$;  and if $y\in D_n\cap\interior C$ then
$(y,C)\in\delta_0\cap\delta_1$, because $\diam C\le 2^{-n}$.   This
means that if $\pmb{t}\in T$ is $\delta$-fine and
$\pmb{t}\subseteq B\times\Cal C$, then there is a
$\delta_0\cap\delta_1$-fine $\pmb{s}\in T$ such that
$\pmb{s}\subseteq D_n\times\Cal C$, $W_{\pmb{s}}\subseteq W_{\pmb{t}}$
and whenever $(x,C)\in\pmb{t}$ and $C$ is not a singleton, there is a
$y$ such that $(y,C)\in\pmb{s}$.   Accordingly

$$\eqalign{\sum_{(x,C)\in\pmb{t}}|\lambda C|
&\le\sum_{(y,C)\in\pmb{s}}|\lambda C|
\le\sum_{(y,C)\in\pmb{s}}|f(y)\mu C-\lambda C|
  +\sum_{(y,C)\in\pmb{s}}|f(y)|\mu C\cr
&\le\sum_{(y,C)\in\pmb{s}}\nu C+S_{\pmb{s}}(|f|\times\chi Y,\mu)
\le 2.\cr}$$

But this means that if $\pmb{t}\in T$ is $\delta$-fine,

$$\eqalignno{S_{\pmb{t}}(|f\times\chi B|,\mu)
&=\sum_{(x,C)\in\pmb{t}\restr B}|f(x)\mu C|\cr
\displaycause{where $\pmb{t}\restr B=\pmb{t}\cap(B\times\Cal C)$}
&\le\sum_{(x,C)\in\pmb{t}\restr B}|f(x)\mu C-\lambda C|
  +\sum_{(x,C)\in\pmb{t}\restr B}|\lambda C|\cr
&\le\sum_{(x,C)\in\pmb{t}\restr B}\nu C+2
\le 3.\cr}$$

\noindent It follows that if $g$ is a $\mu$-simple function and
$0\le g\le|f\times\chi B|$,

$$\eqalignno{\int g\,d\mu
&=\Heint g
\le\sup_{\pmb{t}\in T\text{ is }\delta\text{-fine}}
  S_{\pmb{t}}(g,\mu)\cr
&\le\sup_{\pmb{t}\in T\text{ is }\delta\text{-fine}}
  S_{\pmb{t}}(|f\times\chi B|,\mu)
\le 3,\cr}$$

\noindent and $|f\times\chi B|$ is $\mu$-integrable, by 213B, so
$f\times\chi B$ is $\mu$-integrable, by 122P.   As $K'_n\setminus B$ is
countable, therefore negligible, $f\times\chi K'_n$ is
$\mu$-integrable.\ \Qed

\medskip

{\bf (c)} So if we set
$\Cal K=\{K_n:n\in\Bbb N\}\cup\{K'_n:n\in\Bbb N\}$, we have a suitable
family.
}%end of proof of 483G

\leader{483H}{Upper and lower derivates:  Definition} Let
$F:\Bbb R\to\Bbb R$ be any function.   For $x\in\Bbb R$, set

\Centerline{$\DiniD F(x)=\limsup_{y\to x}\Bover{F(y)-F(x)}{y-x}$,
\quad$\Dinid F(x)=\liminf_{y\to x}\Bover{F(y)-F(x)}{y-x}$}

\noindent in $[-\infty,\infty]$\cmmnt{, that is,
$\DiniD F(x)=\max(\DiniD^+F(x),\DiniD^-F(x))$ and
$\Dinid F(x)=\min(\Dinid^+F(x),\Dinid^-F(x))$ as defined in 222J}.

\leader{483I}{Theorem} Suppose that $f:\Bbb R\to\Bbb R$ is Henstock
integrable, and $F$ is its indefinite Henstock integral.   Then $F'(x)$
is defined and equal to $f(x)$ for almost every $x\in\Bbb R$.

\proof{ For $n\in\Bbb N$, set
$A_n=\{x:|x|\le n,\,\DiniD F(x)>f(x)+2^{-n}\}$.   Then
$\mu^*A_n\le 2^{-n+1}$.   \Prf\ Let $\delta\in\Delta$ and
$\phi:\Bbb R\to[0,4^{-n}]$ be such that $\phi$ is non-decreasing and
$|f(x)(b-a)-F(b)+F(a)|\le\phi(b)-\phi(a)$ whenever $a\le x\le b$ and
$(x,[a,b])\in\delta$ (483F).   Let $\Cal I$ be the set of non-trivial
closed intervals $[a,b]\subseteq\Bbb R$ such that, for some
$x\in[a,b]\cap A_n$, $(x,[a,b])\in\delta$ and
$\Bover{F(b)-F(a)}{b-a}\ge f(x)+2^{-n}$.   By Vitali's theorem (221A) we
can find a countable disjoint family
$\Cal I_0\subseteq\Cal I$ such that $A_n\setminus\bigcup\Cal I_0$ is
negligible;  so we have a finite family $\Cal I_1\subseteq\Cal I_0$ such
that $\mu^*(A_n\setminus\bigcup\Cal I_1)\le 2^{-n}$.   Enumerate
$\Cal I_1$ as $\ofamily{i}{m}{[a_i,b_i]}$, and for each $i<m$ take
$x_i\in[a_i,b_i]\cap A_n$ such that $(x_i,[a_i,b_i])\in\delta$ and
$F(b_i)-F(a_i)\ge(b_i-a_i)(f(x_i)+2^{-n})$.   Then

\Centerline{$\phi(b_i)-\phi(a_i)\ge|f(x_i)(b_i-a_i)-F(b_i)+F(a_i)|
\ge 2^{-n}(b_i-a_i)$}

\noindent for each $i<m$, so

\Centerline{$\mu(\bigcup\Cal I_1)=\sum_{i<m}b_i-a_i
\le 2^n\sum_{i<m}\phi(b_i)-\phi(a_i)\le 2^{-n}$,}

\noindent and $\mu^*A_n\le 2^{-n+1}$.\ \Qed

Accordingly
$\{x:\DiniD F(x)>f(x)\}=\bigcup_{m\in\Bbb N}\bigcap_{n\ge m}A_n$
is negligible.   Similarly, or applying the argument to $-f$,
$\{x:\Dinid F(x)<f(x)\}$ is negligible.   So
$\DiniD F\leae f\leae\Dinid F$.
Since $\Dinid F\le\DiniD F$ everywhere, $\DiniD F\eae\Dinid F\eae f$.
But $F'(x)=f(x)$
whenever $\DiniD F(x)=\Dinid F(x)=f(x)$, so we have the result.
}%end of proof of 483I

\vleader{48pt}{483J}{Theorem} Let $f:\Bbb R\to\Bbb R$ be a function.   Then
the following are equiveridical:

(i) $f$ is Henstock integrable;

(ii) for every $\epsilon>0$ there are functions $F_1$,
$F_2:\Bbb R\to\Bbb R$, with finite limits at both $-\infty$ and
$\infty$, such that $\DiniD F_1(x)\le f(x)\le\Dinid F_2(x)$ and
$0\le F_2(x)-F_1(x)\le\epsilon$ for every $x\in\Bbb R$.

\proof{{\bf (i)$\Rightarrow$(ii)} Suppose that $f$ is Henstock
integrable and that $\epsilon>0$.   Let $F$ be the indefinite Henstock
integral of $f$.   Let $\delta\in\Delta$,
$\phi:\Bbb R\to[0,\bover12\epsilon]$ be such that $\phi$ is
non-decreasing and $|f(x)(b-a)-F(b)+F(a)|\le\phi(b)-\phi(a)$ whenever
$a\le x\le b$ and $(x,[a,b])\in\delta$ (483F).   Set $F_1=F-\phi$,
$F_2=F+\phi$;  then $F_1(x)\le F_2(x)\le F_1(x)+\epsilon$ for every
$x\in\Bbb R$, and the limits at $\pm\infty$ are defined because $F$ and
$\phi$ both have limits at both ends.   If $x\in\Bbb R$, there is an
$\eta>0$ such that $(x,A)\in\delta$ whenever
$A\subseteq[x-\eta,x+\eta]$.   So if
$x-\eta\le a\le x\le b\le x+\eta$ and $a<b$,

\Centerline{$|\Bover{F(b)-F(a)}{b-a}-f(x)|
\le\Bover{\phi(b)-\phi(a)}{b-a}$,}

\noindent and

\Centerline{$\Bover{F_1(b)-F_1(a)}{b-a}\le f(x)
\le\Bover{F_2(b)-F_2(a)}{b-a}$.}

\noindent In particular, this is true whenever $x-\eta\le a<x=b$ or
$x=a<b\le x+\eta$.   So $\DiniD F_1(x)\le f(x)\le\Dinid F_2(x)$.   As $x$ is
arbitrary, we have a suitable pair $F_1$, $F_2$.

\medskip

{\bf (ii)$\Rightarrow$(i)} Suppose that (ii) is true.   Take
any $\epsilon>0$.   Let $F_1$, $F_2:\Bbb R\to\Bbb R$ be as in the
statement of (ii).

\medskip

\quad\grheada\ We need to know that $F_2-F_1$ is non-decreasing.   \Prf\
Set $G=F_2-F_1$.   Then

$$\eqalignno{\liminf_{y\to x}\Bover{G(y)-G(x)}{y-x}
&=\liminf_{y\to x}\Bover{F_2(y)-F_2(x)}{y-x}
  -\Bover{F_1(y)-F_1(x)}{y-x}\cr
&\ge\liminf_{y\to x}\Bover{F_2(y)-F_2(x)}{y-x}
  -\limsup_{y\to x}\Bover{F_1(y)-F_1(x)}{y-x}\cr
\displaycause{2A3Sf}
&=\Dinid F_2(x)-\DiniD F_1(x)\ge 0\cr}$$

\noindent for any $x\in\Bbb R$.   \Quer\ If $a<b$ and $G(a)>G(b)$, set
$\gamma=\Bover{G(a)-G(b)}{2(b-a)}$, and choose $\sequencen{a_n}$,
$\sequencen{b_n}$ inductively as follows.   $a_0=a$ and $b_0=b$.   Given
that $a_n<b_n$ and $G(a_n)-G(b_n)>\gamma(b_n-a_n)$, set
$c=\bover12(a_n+b_n)$;  then either $G(a_n)-G(c)>\gamma(c-a_n)$ or
$G(c)-G(b_n)\ge\gamma(b_n-c)$;  in the former case, take $a_{n+1}=a_n$
and $b_{n+1}=c$;  in the latter, take $a_{n+1}=c$ and $b_{n+1}=b_n$.
Set $x=\lim_{n\to\infty}a_n=\lim_{n\to\infty}b_n$.   Then for each $n$,
either $G(a_n)-G(x)>\gamma(x-a_n)$ or $G(x)-G(b_n)>\gamma(b_n-x)$.   In
either case, we have a $y$ such that $0<|y-x|\le 2^{-n}(b-a)$ and
$\Bover{G(y)-G(x)}{y-x}<-\gamma$.   So $\Dinid G(x)\le-\gamma<0$, which is
impossible.\ \Bang

Thus $G$ is non-decreasing, as required.\ \Qed

\medskip

\quad\grheadb\ Let $a\le b$ be such that $|F_1(x)-F_1(a)|\le\epsilon$
whenever $x\le a$ and
$|F_1(x)-F_1(b)|\le\epsilon$ whenever $x\ge b$.   Let
$h:\Bbb R\to\ooint{0,\infty}$ be a strictly positive integrable function
such that $\int h\,d\mu\le\epsilon$.   Then $\Heint h\le\epsilon$, by
483Bb, so there is a $\delta_0\in\Delta$ such that
$S_{\pmb{t}}(h,\mu)\le 2\epsilon$ for every $\delta_0$-fine
$\pmb{t}\in T$ (482Ad).   For $x\in\Bbb R$ let $\eta_x>0$ be such that

\Centerline{$\Bover{F_1(y)-F_1(x)}{y-x}\le f(x)+h(x)$,
\quad$\Bover{F_2(y)-F_2(x)}{y-x}\ge f(x)-h(x)$}

\noindent whenever $0<|y-x|\le 2\eta_x$;  set $\delta
=\{(x,A):(x,A)\in\delta_0,\,A\subseteq\ooint{x-\eta_x,x+\eta_x}\}$, so
that $\delta\in\Delta$.   Note that if $x\in\Bbb R$ and
$x-\eta_x\le\alpha\le x\le\beta\le x+\eta_x$, then

\Centerline{$F_1(\beta)-F_1(x)\le(\beta-x)(f(x)+h(x))$,
\quad$F_1(x)-F_1(\alpha)\le(x-\alpha)(f(x)+h(x))$,}

\noindent so that $F_1(\beta)-F_1(\alpha)\le(\beta-\alpha)(f(x)+h(x))$;
and similarly
$F_2(\beta)-F_2(\alpha)\ge(\beta-\alpha)(f(x)-h(x))$.

For $C\in\Cal C$, set

\Centerline{$\lambda_1C=F_1(\sup C)-F_1(\inf C)$,
\quad$\lambda_2C=F_2(\sup C)-F_2(\inf C)$.}

\noindent Then if $C\in\Cal C$, $x\in\overline{C}$ and $(x,C)\in\delta$,

\Centerline{$\lambda_1C\le(f(x)+h(x))\mu C$,
\quad$\lambda_2C\ge(f(x)-h(x))\mu C$.}

Suppose that $\pmb{t}\in T$ is $\delta$-fine and $\Cal R_{ab}$-filling.
Then $W_{\pmb{t}}=[\alpha,\beta]$ for some $\alpha\le a$ and $\beta\ge b$,
so that

$$\eqalign{S_{\pmb{t}}(f,\mu)
&=\sum_{(x,C)\in\pmb{t}}f(x)\mu C
\le\sum_{(x,C)\in\pmb{t}}\lambda_2C+h(x)\mu C
=\lambda_2[\alpha,\beta]+S_{\pmb{t}}(h,\mu)\cr
&\le F_2(\beta)-F_2(\alpha)+2\epsilon
\le F_1(\beta)-F_1(\alpha)+3\epsilon
\le F_1(b)-F_1(a)+5\epsilon.\cr}$$

\noindent Similarly,

$$\eqalign{S_{\pmb{t}}(f,\mu)
&=\sum_{(x,C)\in\pmb{t}}f(x)\mu C
\ge\sum_{(x,C)\in\pmb{t}}\lambda_1C-h(x)\mu C\cr
&=\lambda_1[\alpha,\beta]-S_{\pmb{t}}(h,\mu)
\ge F_1(\beta)-F_1(\alpha)-2\epsilon
\ge F_1(b)-F_1(a)-4\epsilon.\cr}$$

\noindent But this means that if $\pmb{t}$, $\pmb{t}'$ are two
$\delta$-fine $\Cal R_{ab}$-filling members of $T$,
$|S_{\pmb{t}}(f,\mu)-S_{\pmb{t}'}(f,\mu)|\le 9\epsilon$.   As $\epsilon$
is arbitrary,

\Centerline{$\lim_{\pmb{t}\to\Cal F(T,\Delta,\frak R)}S_{\pmb{t}}(f,\mu)
=\biggerHeint f$}

\noindent is defined.
}%end of proof of 483J

\cmmnt{\medskip

\noindent{\bf Remark} The formulation (ii) above is a version of the
method of integration described by {\smc Perron 1914}.
}%end of comment

\leader{483K}{Proposition} Let $f:\Bbb R\to\Bbb R$ be a Henstock
integrable function, and $F$ its indefinite Henstock integral.
Then $F[E]$ is Lebesgue negligible for every Lebesgue negligible set
$E\subseteq\Bbb R$.

\proof{ Let $\epsilon>0$.   By 483C and 482Ad, as usual, together with
483F, there are a $\delta\in\Delta$ and a non-decreasing
$\phi:\Bbb R\to[0,\epsilon]$ such that

\Centerline{$S_{\pmb{t}}(|f|\times\chi E,\mu)\le\epsilon$,
\quad$|f(x)(b-a)-F(b)+F(a)|\le\phi(b)-\phi(a)$}

\noindent whenever $\pmb{t}\in T$ is $\delta$-fine, $a\le x\le b$ and
$(x,[a,b])\in\delta$.   For $n\in\Bbb N$ and $i\in\Bbb Z$, set

\Centerline{$E_{ni}=\{x:x\in E\cap\coint{2^{-n}i,2^{-n}(i+1)},
  \,(x,A)\in\delta$ whenever $A\subseteq[x-2^{-n},x+2^{-n}]\}$.}

\noindent Set
$J_n=\{i:i\in\Bbb Z,\,-4^n<i\le 4^n,\,E_{ni}\ne\emptyset\}$.
Observe that

\Centerline{$E
=\bigcup_{n\in\Bbb N}\bigcap_{m\ge n}\bigcup_{i\in J_m}E_{mi}$.}

\noindent For $i\in J_n$, take $x_{ni}$, $y_{ni}\in E_{ni}$ such that
$x_{ni}\le y_{ni}$ and

\Centerline{$\min(F(x_{ni}),F(y_{ni}))
\le\inf F[E_{ni}]+4^{-n}\epsilon$,}

\Centerline{$\max(F(x_{ni}),F(y_{ni}))
\ge\sup F[E_{ni}]-4^{-n}\epsilon$,}

\noindent so that
$\mu^*F[E_{ni}]\le|F(y_{ni})-F(x_{ni})|+2^{-2n+1}\epsilon$.   Now, for
each $i\in J_n$, $(x_{ni},[x_{ni},y_{ni}])\in\delta$, while
$[x_{ni},y_{ni}]\subseteq\coint{2^{-n}i,2^{-n}(i+1)}$, so
$\pmb{t}=\{(x_{ni},[x_{ni},y_{ni}]):i\in J_n\}$ is a $\delta$-fine
member of $T$, and

$$\eqalign{\mu^*F[\bigcup_{i\in J_n}E_{ni}]
&\le\sum_{i\in J_n}\mu^*F[E_{ni}]
\le\sum_{i\in J_n}2^{-2n+1}\epsilon+|F(y_{ni})-F(x_{ni})|\cr
&\le 4\epsilon+\sum_{i\in J_n}|f(x_{ni})(y_{ni}-x_{ni})|
  +\sum_{i\in J_n}\phi(y_{ni})-\phi(x_{ni})\cr
&\le 4\epsilon+S_{\pmb{t}}(|f|\times\chi E,\mu)+\epsilon
\le 6\epsilon.\cr}$$

Since this is true for every $n\in\Bbb N$,

$$\eqalignno{\mu^*F[E]
&=\mu^*F[\bigcup_{n\in\Bbb N}\bigcap_{m\ge n}\bigcup_{i\in J_m}E_i]\cr
&=\mu^*(\bigcup_{n\in\Bbb N}F[\bigcap_{m\ge n}\bigcup_{i\in J_m}E_i])
=\sup_{n\in\Bbb N}\mu^*F[\bigcap_{m\ge n}\bigcup_{i\in J_m}E_i]\cr
\displaycause{132Ae}
&\le 6\epsilon.\cr}$$

\noindent As $\epsilon$ is arbitrary, $F[E]$ is negligible, as claimed.
}%end of proof of 483K

\cmmnt{\medskip

\noindent{\bf Remark} Compare 225M.}

\leader{483L}{Definition} If $f:\Bbb R\to\Bbb R$ is Henstock integrable,
I write $\|f\|_H$ for $\sup_{a\le b}|\Heint_a^bf|$.   It is elementary
to check that this is a seminorm on the linear space of all Henstock
integrable functions.   \cmmnt{(It is finite-valued by 483D.)}

\leader{483M}{Proposition} (a) If $f:\Bbb R\to\Bbb R$ is Henstock
integrable, then $|\Heint f|\le\|f\|_H$, and $\|f\|_H=0$ iff $f=0$ a.e.

(b) Write $\eusm{HL}^1$ for the linear space of all Henstock integrable
real-valued functions on $\Bbb R$, and $HL^1$ for
$\{f^{\ssbullet}: f\in\eusm{HL}^1\}\subseteq L^0(\mu)$\cmmnt{ (\S241)}.
If we write $\|f^{\ssbullet}\|_H=\|f\|_H$ for every
$f\in\eusm{HL}^1$, then $HL^1$ is a normed space.   The ordinary space
$L^1(\mu)$ of equivalence classes of Lebesgue integrable functions is a
linear subspace of $HL^1$, and $\|u\|_H\le\|u\|_1$ for every
$u\in L^1(\mu)$.

(c) We have a linear operator $T:HL^1\to C_b(\Bbb R)$ defined by saying
that $T(f^{\ssbullet})$ is the indefinite Henstock integral of $f$ for
every $f\in\eusm{HL}^1$, and $\|T\|=1$.

\proof{{\bf (a)} Of course

\Centerline{$|\biggerHeint f|
=\lim_{a\to-\infty,b\to\infty}|\biggerHeint_a^bf|\le\|f\|_H$}

\noindent (using 483Bd).   Let $F$ be the indefinite Henstock integral of
$f$, so that
$F(b)-F(a)=\Heint_a^bf$ whenever $a\le b$.   If $f=0$ a.e., then
$F(x)=\int_{-\infty}^xfd\mu=0$ for every $x$, by 483Bb, so $\|f\|_H=0$.
If $\|f\|_H=0$, then $F$ is constant, so $f=F'=0$ a.e., by 483I.

\medskip

{\bf (b)} That $HL^1$ is a normed space follows immediately from (a).
(Compare the definitions
of the norms $\|\,\|_p$ on $L^p$, for $1\le p\le\infty$, in
\S\S242-244.)   By 483Bb, $L^1(\mu)\subseteq HL^1$, and

\Centerline{$\|u\|_H\le\|u^+\|_H+\|u^-\|_H=\|u^+\|_1+\|u^-\|_1=\|u\|_1$}

\noindent for every $u\in L^1(\mu)$, writing $u^+$ and $u^-$ for the
positive and negative parts of $u$, as in Chapter 24.

\medskip

{\bf (c)} If $f$, $g\in\eusm{HL}^1$ and $f^{\ssbullet}=g^{\ssbullet}$,
then $f$ and $g$ have the same indefinite Henstock integral, by 483Bb or
otherwise;  so $T$ is defined as a function from $HL^1$ to
$\BbbR^{\Bbb R}$.   By 483F, $Tu$ is continuous and bounded for every
$u\in HL^1$, and by 481Ca $T$ is linear.   If $f\in\eusm{HL}^1$ and
$Tf^{\ssbullet}=F$, then $\|f\|_H=\sup_{x,y\in\Bbb R}|F(y)-F(x)|$;
since $\lim_{x\to-\infty}F(x)=0$, $\|f\|_H\ge\|F\|_{\infty}$;  as $f$ is
arbitrary, $\|T\|\le 1$.   On the other hand, for any non-negative
Lebesgue integrable function $f$,
$\|Tf^{\ssbullet}\|_{\infty}=\|f\|_1=\|f\|_H$, so $\|T\|=1$.
}%end of proof 483M

\leader{483N}{Proposition} Suppose that $\family{m}{M}{I_m}$ is a
disjoint family of open intervals in $\Bbb R$ with union $G$, and that
$f:\Bbb R\to\Bbb R$ is a function such that $f_m=f\times\chi I_m$ is
Henstock integrable for every $m\in M$.   If
$\sum_{m\in M}\|f_m\|_H<\infty$, then $f\times\chi G$ is Henstock
integrable, and $\Heint f\times\chi G=\sum_{m\in M}\Heint f_m$.

\proof{ I seek to apply 482H again.   We have already seen, in the proof of
483Bc, that the conditions of 482G are satisfied by
$\Bbb R$, $T$, $\Delta$, $\frak R$, $\Cal C$, $\frak T$ and $\mu$.   Of
course $G=\bigcup_{m\in M}I_m$ is the union of a sequence of open sets
over which $f$ is Henstock integrable.   So we have only to check
482H(viii).

Set

\Centerline{$\delta_0
=\bigcup_{m\in M}\{(x,A):x\in I_m,\,A\subseteq I_m\}
\cup\{(x,A):x\in\Bbb R\setminus G,\,A\subseteq\Bbb R\}$,}

\noindent so that $\delta_0\in\Delta$.   For each $m\in M$ let $\FSH_m$ be
the Saks-Henstock indefinite integral of $f_m$.
Let $\epsilon>0$.   Then there is a finite set $M_0\subseteq M$ such
that $\sum_{m\in M\setminus M_0}\|f_m\|_H\le\epsilon$.   Next, there
must be $\delta_1\in\Delta$ and $\Cal R\in\frak R$ such that

\Centerline{$\sum_{m\in M_0}|$$\biggerHeint f_m-S_{\pmb{t}}(f_m,\mu)|
\le\epsilon$}

\noindent for every $\delta_1$-fine $\Cal R$-filling $\pmb{t}\in T$, and
$\delta_2\in\Delta$ such that
$\sum_{m\in M_0}|S_{\pmb{t}}(f_m,\mu)-\FSH_m(W_{\pmb{t}})|\le\epsilon$ for
every $\delta_2$-fine $\pmb{t}\in T$.

Now let $\pmb{t}\in T$ be $(\delta_0\cap\delta_1\cap\delta_2)$-fine and
$\Cal R$-filling.   For each $m\in M$ set
$\pmb{t}_m=\pmb{t}\restr I_m$, so that
$\pmb{t}\restr G=\bigcup_{m\in M}\pmb{t}_m$.   Because $W_{\pmb{t}}$ is
an interval, each $W_{\pmb{t}_m}$ must be an interval, as in part
(c)-(d)(vi) of the proof of 483B, and $W_{\pmb{t}_m}$ is a
subinterval of $I_m$ because $\pmb{t}$ is $\delta_0$-fine.   So (using
482G)

\Centerline{$|\FSH_m(W_{\pmb{t}_m})|
=|\biggerHeint f_m\times\chi W_{\pmb{t}_m}|\le\|f_m\|_H$.}

\noindent Also

$$\eqalign{\sum_{m\in M_0}
  |\Heint f_m-\Heint f\times\chi W_{\pmb{t}_m}|
&\le\sum_{m\in M_0}|\Heint f_m-S_{\pmb{t}}(f_m,\mu)|
  +\sum_{m\in M_0}|S_{\pmb{t}}(f_m,\mu)-\FSH_m(W_{\pmb{t}_m}|\cr
&\le 2\epsilon.\cr}$$

\noindent On the other hand,

\Centerline{$\sum_{m\in M\setminus M_0}
   $$|\biggerHeint f_m-\biggerHeint f\times\chi W_{\pmb{t}_m}|
\le 2$$\sum_{m\in M\setminus M_0}\|f_m\|_H
\le 2\epsilon$.}

Putting these together,

$$\eqalignno{|\Heint f\times\chi W_{\pmb{t}\restr G}
  -\sum_{m\in M}\Heint f_m|
&=|\sum_{m\in M}\Heint f\times\chi W_{\pmb{t}_m}
  -\sum_{m\in M}\Heint f_m|\cr
\displaycause{because $\pmb{t}$ is finite, so all but finitely many
terms in the sum $\sum_{m\in M}f\times\chi W_{\pmb{t}_m}$ are zero}
&\le\sum_{m\in M}|\Heint f\times\chi W_{\pmb{t}_m}-\Heint f_m|
\le 4\epsilon.\cr}$$

\noindent As $\epsilon$ is arbitrary, condition 482H(viii) is satisfied,
with

\Centerline{$\lim_{\pmb{t}\rightarrow\Cal F(T,\Delta,\frak R)}
  \biggerHeint f\times\chi W_{\pmb{t}\restr G}
=$$\sum_{m\in M}$$\biggerHeint f_m$,}

\noindent and 482H gives the result we seek.
}%end of proof of 483N

\leader{483O}{Definitions (a)} For any real-valued function $F$, write
$\omega(F)$ for
$\sup_{x,y\in\dom F}|F(x)-F(y)|$, the {\bf oscillation} of $F$.
(Interpret $\sup\emptyset$ as $0$, so that $\omega(\emptyset)=0$.)

\leaveitout{\spheader 483O\query For $A\subseteq\Bbb R$, set

\Centerline{$\Var^*_A(F)
=\sup\{\sum_{I\in\Cal I}\omega(F;\overline{I}):\Cal I$ is a disjoint
family of open intervals with endpoints in $A\}$.}

\noindent (Again, $\Var^*_{\emptyset}(F)=0$.)   \cmmnt{Note that
$\Var^*_A(F)$ is {\it not} determined by $F\restr A$.}
We say that $F$ is {\bf BV$_*$ on $A$} if $\Var^*_A(F)$ is finite.
}%end of leaveitout

\spheader 483Ob Let $F:\Bbb R\to\Bbb R$ be a function.   For
$A\subseteq\Bbb R$, we say that $F$ is {\bf AC$_*$
on $A$} if for every $\epsilon>0$ there is an $\eta>0$ such that
$\sum_{I\in\Cal I}\omega(F\restr\overline{I})\le\epsilon$ whenever $\Cal
I$
is a disjoint family of open intervals with endpoints in $A$ and
$\sum_{I\in\Cal I}\mu I\le\eta$.   \cmmnt{Note that whether $F$ is
AC$_*$ on $A$ is {\it not} determined by $F\restr A$, since it depends
on the behaviour of $F$ on intervals with endpoints in $A$.}

\spheader 483Oc\cmmnt{ Finally,} $F$ is
{\bf ACG$_*$} if it is continuous and there is a countable family
$\Cal A$ of sets, covering
$\Bbb R$, such that $F$ is AC$_*$ on every member of $\Cal A$.

\leader{483P}{Elementary results (a)(i)}
If $F$, $G:\Bbb R\to\Bbb R$ are functions and
$A\subseteq B\subseteq\Bbb R$, then
$\omega(F+G\restr A)\le\omega(F\restr A)+\omega(G\restr A)$ and
$\omega(F\restr A)\le\omega(F\restr B)$.

\medskip

\quad{\bf (ii)} If $F$ is the indefinite Henstock integral of
$f:\Bbb R\to\Bbb R$ and $C\subseteq\Bbb R$ is an interval, then
$\|f\times\chi C\|_H=\omega(F\restr C)$.

\medskip

\quad{\bf (iii)} If $F:\Bbb R\to\Bbb R$ is continuous, then
$(a,b)\mapsto\omega(F\restrp[a,b]):\Bbb R^2\to\Bbb R$ is continuous, and
$\omega(F\restr\overline{A})=\omega(F\restr A)$ for every set
$A\subseteq\Bbb R$.

\leaveitout{\spheader 483P\query
Then $\Var^*_A(F+G)\le\Var^*_A(F)+\Var^*_A(G)$ and
$\Var^*_A(F)\le\Var^*_B(F)$ if $A\subseteq B\subseteq\Bbb R$,
$\Var^*_A(F)=\omega(F\restr A)$ if $F$ is monotonic.}

\spheader 483Pb{\bf (i)}
If $F:\Bbb R\to\Bbb R$ is AC$_*$ on $A\subseteq\Bbb R$, it is AC$_*$ on
every subset of $A$.

\medskip

\quad{\bf (ii)} If $F:\Bbb R\to\Bbb R$ is continuous and is AC$_*$ on
$A\subseteq\Bbb R$, it is AC$_*$ on $\overline{A}$.   \prooflet{\Prf\
Let $\epsilon>0$.   Let $\eta>0$ be such that
$\sum_{I\in\Cal I}\omega(F\restr\overline{I})\le\epsilon$ whenever
$\Cal I$ is a disjoint family of open intervals with endpoints in $A$
and $\sum_{I\in\Cal I}\mu I\le\eta$.   Let $\Cal I$ be a disjoint family
of open intervals with endpoints in $\overline{A}$ and
$\sum_{I\in\Cal I}\mu I\le\bover12\eta$.   Let $\Cal I_0\subseteq\Cal I$
be a non-empty finite set;  then we can enumerate $\Cal I_0$ as
$\langle\ooint{a_i,b_i}\rangle_{i\le n}$ where
$a_0,b_0,\ldots,a_n,b_n\in\overline{A}$ and
$a_0\le b_0\le a_1\le b_1\le\ldots\le a_n\le b_n$.   Because
$(a,b)\mapsto\omega(F\restrp[a,b])$ is continuous, as noted in (a-ii)
above, we can find $a'_0,\ldots,b'_n\in A$ such that
$a'_0\le b'_0\le a'_1\le\ldots\le a'_n\le b'_n$,
$\sum_{i=0}^nb'_i-a'_i\le\eta$, and
$\sum_{i=0}^n|\omega(F\restrp[a'_i,b'_i])-\omega(F\restrp[a_i,b_i])|
\le\epsilon$;  so that

\Centerline{$\sum_{I\in\Cal I_0}\omega(F\restr\overline{I})
\le\sum_{i=0}^n\omega(F\restrp[a'_i,b'_i])\le 2\epsilon$.}

\noindent As $\Cal I_0$ is arbitrary,
$\sum_{I\in\Cal I}\omega(F\restr\overline{I})\le 2\epsilon$;  as
$\epsilon$
is arbitrary, $F$ is AC$_*$ on $\overline{A}$.\ \Qed}

\leader{483Q}{Lemma} Let $F:\Bbb R\to\Bbb R$ be a continuous function,
and $K\subseteq\Bbb R$ a non-empty compact set such that $F$ is AC$_*$
on $K$.   Write $\Cal I$ for the family of non-empty bounded open
intervals, disjoint from $K$, with endpoints in $K$.

(a) $\sum_{I\in\Cal I}\omega(F\restr I)$ is finite.

(b) Write $a^*$ for $\inf K=\min K$.   Then there is a Lebesgue
integrable function $g:\Bbb R\to\Bbb R$, zero off $K$, such that

\Centerline{$F(x)-F(a^*)
=\int_{a^*}^xg
  +\sum_{J\in\Cal I,J\subseteq[a^*,x]}F(\sup J)-F(\inf J)$}

\noindent for every $x\in K$.

\proof{{\bf (a)} Let $\eta>0$ be such that
$\sum_{I\in\Cal J}\omega(F\restr\overline{I})\le 1$ whenever $\Cal J$ is
a disjoint family of open intervals with endpoints in $K$ and
$\sum_{I\in\Cal J}\mu I\le\eta$.   Let
$m_0$, $m_1\in\Bbb Z$ be such that $K\subseteq[m_0\eta,m_1\eta]$.   For
integers $m$ between $m_0$ and $m_1$, let $\Cal I_m$ be the set of
intervals in $\Cal I$ included in $\ooint{m\eta,(m+1)\eta}$.   Then
$\sum_{I\in\Cal I_m}\mu I\le\eta$, so
$\sum_{I\in\Cal I_m}\omega(F\restr I)\le 1$ for each $m$.   Also every
member of $\Cal J=\Cal I\setminus\bigcup_{m_0\le m<m_1}\Cal I_m$
contains $m\eta$ for some $m$ between $m_0$ and $m_1$, so
$\#(\Cal J)\le m_1-m_0$.   Accordingly

$$\eqalign{\sum_{I\in\Cal I}\omega(F\restr\overline{I})
&\le\sum_{I\in\Cal J}\omega(F\restr\overline{I})
    +\sum_{m=m_0}^{m_1-1}
      \sum_{I\in\Cal I_m}\omega(F\restr\overline{I})\cr
&\le\sum_{I\in\Cal J}\omega(F\restr\overline{I})+m_1-m_0
<\infty\cr}$$

\noindent because $F$ is continuous, therefore bounded on every bounded
interval.

\medskip

{\bf (b)(i)} Set $b^*=\sup K=\max K$.   Define $G:[a^*,b^*]\to\Bbb R$ by
setting

$$\eqalign{G(x)&=F(x)\text{ if }x\in K,\cr
&=\Bover{F(b)(x-a)+F(a)(b-x)}{b-a}
  \text{ if }x\in\ooint{a,b}\in\Cal I.\cr}$$

\noindent Then $G$ is absolutely continuous.   \Prf\ $G$ is continuous
because $F$ is.   Let $\epsilon>0$.   Let $\eta_1>0$ be such that
$\sum_{I\in\Cal J}\omega(F\restr\overline{I})\le\epsilon$ whenever
$\Cal J$ is a disjoint family of open intervals with endpoints in $K$
and $\sum_{I\in\Cal J}\mu I\le\eta_1$.   Let
$\Cal I_0\subseteq\Cal I$ be a finite set such that
$\sum_{I\in\Cal I\setminus\Cal I_0}\omega(F\restr\overline{I})
\le\epsilon$,
and take $M>0$ such that $|F(b)-F(a)|\le M(b-a)$ whenever
$\ooint{a,b}\in\Cal I_0$;  set $\eta=\min(\eta_1,\Bover{\epsilon}M)>0$.

Let $\Cal J^*$ be the set of non-empty open subintervals $J$ of
$[a^*,b^*]$ such that either $J\cap K=\emptyset$ or both endpoints of
$J$ belong to $K$.   Let $\Cal J\subseteq\Cal J^*$ be a disjoint family
such that $\sum_{I\in\Cal J}\mu I\le\eta$.   Set
$\Cal J'=\{J:J\in\Cal J,\,J\cap K=\emptyset\}$.   Then

$$\eqalign{\sum_{J\in\Cal J\setminus\Cal J'}|G(\sup J)-G(\inf J)|
&=\sum_{J\in\Cal J\setminus\Cal J'}|F(\sup J)-F(\inf J)|\cr
&\le\sum_{J\in\Cal J\setminus\Cal J'}\omega(F\restr\overline{J})
\le\epsilon.\cr}$$

\noindent On the other hand,

$$\eqalignno{\sum_{J\in\Cal J'}|G(\sup J)-G(\inf J)|
&=\sum_{I\in\Cal I_0}\sum_{\Atop{J\in\Cal J}{J\subseteq I}}
  |G(\sup J)-G(\inf J)|\cr
&\qquad\qquad+\sum_{I\in\Cal I\setminus\Cal I_0}
        \sum_{\Atop{J\in\Cal J}{J\subseteq I}}|G(\sup J)-G(\inf J)|\cr
&\le M\sum_{I\in\Cal I_0}\sum_{\Atop{J\in\Cal J}{J\subseteq I}}\mu J
   +\sum_{I\in\Cal I\setminus\Cal I_0}|F(\sup I)-F(\inf I)|\cr
\displaycause{because $G$ is monotonic on $\overline{I}$ for each
$I\in\Cal I$}
&\le M\eta+\epsilon
\le 2\epsilon,\cr}$$

\noindent so $\sum_{J\in\Cal J}|G(\sup J)-G(\inf J)|\le 3\epsilon$.

Generally, if $J$ is any non-empty open subinterval of $[a^*,b^*]$, we
can split it into at most three intervals belonging to $\Cal J^*$.
So if $\Cal J$ is any disjoint family of non-empty open
subintervals of $[a^*,b^*]$ with $\sum_{J\in\Cal J}\mu J\le\eta$, we can
find a family $\tilde\Cal J\subseteq\Cal J^*$ with
$\sum_{J\in\tilde\Cal J}\mu J=\sum_{J\in\Cal J}\mu J$ and
$\sum_{J\in\tilde\Cal J}|G(\sup J)-G(\inf J)|
\ge\sum_{J\in\Cal J}|G(\sup J)-G(\inf J)|$.   But this means that
$\sum_{J\in\Cal J}|G(\sup J)-G(\inf J)|\le 3\epsilon$.   As $\epsilon$
is arbitrary, $G$ is absolutely continuous.\ \Qed

\medskip

\quad{\bf (ii)} By 225E, $G'$ is Lebesgue integrable and
$G(x)=G(a^*)+\int_{a^*}^xG'$ for every $x\in[a^*,b^*]$.   Set
$g(x)=G'(x)$ when $x\in K$ and $G'(x)$ is defined, $0$ for other
$x\in\Bbb R$, so that $g:\Bbb R\to\Bbb R$ is Lebesgue integrable.   Now
take any $x\in K$.   Then

$$\eqalignno{F(x)
&=G(x)
=G(a^*)+\int_{a^*}^{x}G'
=F(a^*)+\int_{a^*}^{x}g+\int_{[a^*,x]\setminus K}G'\cr
&=F(a^*)+\int_{a^*}^{x}g
  +\sum_{\Atop{I\in\Cal I}{I\subseteq[a^*,x]}}\int_IG'\cr
\displaycause{because $\Cal I$ is a disjoint countable family of
measurable sets, and
$\bigcup_{I\in\Cal I,I\subseteq[a^*,x]}I=[a^*,x]\setminus K$}
&=F(a^*)+\int_{a^*}^xg
  +\sum_{\Atop{I\in\Cal I}{I\subseteq[a^*,x]}}G(\sup I)-G(\inf I)\cr
\displaycause{note that this sum is absolutely summable}
&=F(a^*)+\int_{a^*}^{x}g
  +\sum_{\Atop{I\in\Cal I}{I\subseteq[a^*,x]}}F(\sup I)-F(\inf I)\cr
}$$

\noindent as required.

}%end of proof of 483Q

\leader{483R}{Theorem} %\cmmnt{ (see {\smc Gordon 94}, 11.14)}
Let $F:\Bbb R\to\Bbb R$ be a function.   Then $F$ is an indefinite Henstock
integral iff it is ACG$_*$,
$\lim_{x\to-\infty}F(x)=0$ and $\lim_{x\to\infty}F(x)$ is defined in
$\Bbb R$.

\proof{{\bf (a)} Suppose that $F$ is the indefinite Henstock integral of
$f:\Bbb R\to\Bbb R$.

\medskip

\quad{\bf (i)} By 483F, $F$ is continuous, with limit zero at $-\infty$
and finite at $\infty$.   So I have just to show that there is a
sequence of sets, covering $\Bbb R$, on each of which $F$ is AC$_*$.
Recall that there is a sequence $\sequence{m}{K_m}$ of compact sets,
covering $\Bbb R$, such that $f\times K_m$ is Lebesgue integrable for
every $m\in\Bbb N$ (483G).
By the arguments of (i)$\Rightarrow$(ii) in the proof of 483J, there is
a function $F_2\ge F$ such that $\Dinid F_2\ge f$ and $F_2-F$ is
non-decreasing and takes values between $0$ and $1$.   For $n\in\Bbb N$,
$j\in\Bbb Z$ set

\Centerline{$I_{nj}=[2^{-n}j,2^{-n}(j+1)]$,}

\Centerline{$B_{nj}=\{x:x\in I_{nj},\,\Bover{F_2(y)-F_2(x)}{y-x}\ge-n
  \text{ whenever }y\in I_{nj}\setminus\{x\}\}$.}

\noindent Observe that $\bigcup_{n\in\Bbb N,j\in\Bbb Z}B_{nj}=\Bbb R$,
so that $\{B_{nj}\cap K_m:m,\,n\in\Bbb N,\,j\in\Bbb Z\}$ is a countable
cover of $\Bbb R$.   It will therefore be enough to show that $F$ is
AC$_*$ on every $B_{nj}\cap K_m$.

\medskip

\quad{\bf (ii)} Fix $m$, $n\in\Bbb N$ and $j\in\Bbb Z$, and set
$A=B_{nj}\cap K_m$.   If $A=\emptyset$, then of course $F$ is AC$_*$ on
$A$;  suppose that $A$ is not empty.   Set $G(x)=F_2(x)+nx$ for
$x\in I_{nj}$, so that $F=G-(F_2-F)-H$ on $I_{nj}$, where $H(x)=nx$.
Whenever $a$, $b\in B_{nj}$ and $a\le x\le b$, then
$G(a)\le G(x)\le G(b)$, because $x\in I_{nj}$.   So if
$a_0,b_0,a_1,b_1,\ldots,a_k,b_k\in A$ and
$a_0\le b_0\le a_1\le b_1\le\ldots\le a_k\le b_k$,

\Centerline{$\sum_{i=0}^k\omega(G\restrp[a_i,b_i])
=\sum_{i=0}^kG(b_i)-G(a_i)
\le G(b_k)-G(a_0)\le\omega(G\restr I_{nj})$,}

\noindent and

$$\eqalignno{\sum_{i=0}^k\omega(F\restrp[a_i,b_i])
&\le\sum_{i=0}^k\omega(G\restrp[a_i,b_i])
  +\sum_{i=0}^k\omega(F_2-F\restrp[a_i,b_i])
  +\sum_{i=0}^k\omega(H\restrp[a_i,b_i])\cr
&\le\omega(G\restr I_{nj})+\omega(F_2-F\restr I_{nj})
  +\omega(H\restr I_{nj})\cr
\displaycause{because $F_2-F$ and $H$ are monotonic}
&\le\omega(F\restr I_{nj})+2\omega(F_2-F\restr I_{nj})
  +2\omega(H\restr I_{nj})\cr
&\le\omega(F\restr I_{nj})+2(1+n\mu I_{nj})
=M\cr}$$

\noindent say, which is finite, because $F$ is bounded.

\medskip

\quad{\bf (iii)} By 2A2I (or 4A2Rj), the open set
$\Bbb R\setminus\overline{A}$ is expressible as a countable union of
disjoint non-empty open intervals.   Two of these are unbounded;  let
$\Cal I$ be the set consisting of the rest, so that
$\overline{A}\cup\bigcup\Cal I=[a^*,b^*]$ is a closed interval included
in $I_{nj}$.   If $I$, $I'$ are distinct members of $\Cal I$ and
$\inf I\le\inf I'$, then $\sup I\le\inf I'$, because
$I\cap I'=\emptyset$, and there must be a point of $\overline{A}$ in the
interval $[\sup I,\inf I']$;  so in fact there must be a point of $A$ in
this interval, since $A$ does not meet either $I$ or $I'$.
It follows that
$\sum_{I\in\Cal I}\omega(F\restr I)\le M$.   \Prf\ If
$\Cal I_0\subseteq\Cal I$ is finite and non-empty, we can enumerate it
as $\langle I_i\rangle_{i\le k}$ where $\sup I_i\le\inf I_{i'}$ whenever
$i<i'\le k$.   We can find $a_0,\ldots,a_{k+1}\in A$ such that
$a_0\le\inf I_0$, $\sup I_i\le a_{i+1}\le\inf I_{i+1}$ for every $i<k$,
and $\sup I_k\le a_{k+1}$;  so that

\Centerline{$\sum_{I\in\Cal I_0}\omega(F\restr I)
=\sum_{i=0}^k\omega(F\restr I_i)
\le\sum_{i=0}^k\omega(F\restrp[a_i,a_{i+1}])
\le M$.}

\noindent As $\Cal I_0$ is arbitrary, this gives the result.\ \Qed

\medskip

\quad{\bf (iv)} Whenever $a^*\le x\le y\le b^*$,

\Centerline{$|F(y)-F(x)|
\le\int_{\overline{A}\cap[x,y]}|f|d\mu
  +\sum_{I\in\Cal I}\omega(F\restr I\cap\ooint{x,y})$.}

\noindent \Prf\ For each $I\in\Cal I$,

\Centerline{$\|f\times\chi(I\cap\ooint{x,y})\|_H
=\omega(F\restr I\cap\ooint{x,y})\le\omega(F\restr I)$}

\noindent (483Pa).   So

\Centerline{$\sum_{I\in\Cal I}\|f\times\chi(I\cap\ooint{x,y})\|_H
\le\sum_{I\in\Cal I}\omega(F\restr I)$}

\noindent is finite.   Writing $H=\ooint{x,y}\cap\bigcup\Cal I$,
$\Heint f\times\chi H$ is defined and equal to
$\sum_{I\in\Cal I}\Heint f\times\chi(I\cap\ooint{x,y})$, by 483N.
On the other hand,

\Centerline{$\ooint{x,y}\setminus H\subseteq\overline{A}\subseteq K_m$,}

\noindent so $f\times\chi(\ooint{x,y}\setminus H)$ is Lebesgue
integrable.   Accordingly

$$\eqalign{|F(y)-F(x)|
&=|\Heint f\times\chi\ooint{x,y}|\cr
&\le|\Heint f\times\chi H|
  +|\int f\times\chi(\ooint{x,y}\setminus H)d\mu|\cr
&\le\sum_{I\in\Cal I}|\Heint f\times\chi(I\cap\ooint{x,y})|
     +\int_{\overline{A}\cap[x,y]}|f|d\mu\cr
&\le\sum_{I\in\Cal I}\omega(F\restr I\cap\ooint{x,y})
     +\int_{\overline{A}\cap[x,y]}|f|d\mu,\cr}$$

\noindent as claimed.\ \Qed

\medskip

\quad{\bf (v)} So if $a^*\le a\le b\le b^*$,

\Centerline{$\omega(F\restrp[a,b])
\le\sum_{I\in\Cal I}\omega(F\restr I\cap[a,b])
  +\int_{\overline{A}\cap[a,b]}|f|d\mu$.}

\noindent\Prf\ We have only to observe that if $a\le x\le y\le b$, then
$\omega(F\restr I\cap\ooint{x,y})\le\omega(F\restr I\cap[a,b])$ for
every $I\in\Cal I$, and
$\int_{\overline{A}\cap\ooint{x,y}}|f|d\mu
\le\int_{\overline{A}\cap[a,b]}|f|d\mu$.\ \Qed

\medskip

\quad{\bf (vi)} Now let $\epsilon>0$.   Setting
$\tilde F(x)=\int_{a^*}^x|f|\times\chi\overline{A}\,d\mu$ for
$x\in[a^*,b^*]$, $\tilde F$ is absolutely continuous (225E), and there
is an $\eta_0>0$ such that
$\sum_{i=0}^k\tilde F(b_i)-\tilde F(a_i)\le\epsilon$ whenever
$a^*\le a_0\le b_0\le a_1\le b_1\le\ldots\le a_k\le b_k\le b^*$ and
$\sum_{i=0}^kb_i-a_i\le\eta_0$.   Take $\Cal I_0\subseteq\Cal I$ to be a
finite set such that
$\sum_{I\in\Cal I\setminus\Cal I_0}\omega(F\restr I)\le\epsilon$, and
let
$\eta>0$ be such that $\eta\le\eta_0$ and $\eta<\diam I$ for every
$I\in\Cal I_0$.

Suppose that $a_0,b_0,\ldots,a_k,b_k\in A$ are such that
$a_0\le b_0\le\ldots\le a_k\le b_k$ and $\sum_{i=0}^kb_i-a_i\le\eta$.
Then no member of $\Cal I_0$ can be included in any interval
$[a_i,b_i]$, and therefore, because no $a_i$ or $b_i$ can belong to
$\bigcup\Cal I$, no member of $\Cal I_0$ meets any $[a_i,b_i]$.   Also,
of course, $a^*\le a_0$ and $b_k\le b^*$.   We therefore have

$$\eqalignno{\sum_{i=0}^k\omega(F\restrp[a_i,b_i])
&\le\sum_{i=0}^k\bigl(\sum_{I\in\Cal I}\omega(F\restr I\cap[a_i,b_i])
  +\int_{\overline{A}\cap[a_i,b_i]}|f|d\mu\bigr)\cr
&=\sum_{i=0}^k\sum_{I\in\Cal I\setminus\Cal I_0}
   \omega(F\restr I\cap[a_i,b_i])
  +\sum_{i=0}^k\tilde F(b_i)-\tilde F(a_i)\cr
&\le\sum_{i=0}^k
  \sum_{\Atop{I\in\Cal I\setminus\Cal I_0}{I\subseteq[a_i,b_i]}}
      \omega(F\restr I)
  +\epsilon\cr
\displaycause{because if $I\in\Cal I$ meets $[a_i,b_i]$, it is included
in it}
&\le\sum_{I\in\Cal I\setminus\Cal I_0}\omega(F\restr I)+\epsilon
\le 2\epsilon.\cr}$$

\noindent As $\epsilon$ is arbitrary, $F$ is AC$_*$ on $A$.   This
completes the proof that $F$ is ACG$_*$ and therefore satisfies the
conditions given.

\medskip

{\bf (b)} Now suppose that $F$ satisfies the conditions.   Set
$F(-\infty)=0$ and $F(\infty)=\lim_{x\to\infty}F(x)$, so that
$F:[-\infty,\infty]\to\Bbb R$ is continuous.   For $x\in\Bbb R$, set
$f(x)=F'(x)$ if this is defined, $0$ otherwise.   Let $\Cal J$ be the
family of all
non-empty intervals $C\subseteq\Bbb R$ such that $\Heint f\times\chi C$
is defined and equal to $F(\sup C)-F(\inf C)$, and let $\Cal I$ be the
set of non-empty open intervals $I$ such that every non-empty
subinterval of $I$ belongs to $\Cal J$.
I seek to show that $\Bbb R$ belongs to $\Cal I$.

\medskip

\quad{\bf (i)} Of course singleton intervals belong to $\Cal J$.   If
$C_1$, $C_2\in\Cal J$ are disjoint and $C=C_1\cup C_2$ is an interval,
then

$$\eqalign{\Heint f\times\chi C
&=\Heint f\times\chi C_1+\Heint f\times\chi C_2\cr
&=F(\sup C_1)-F(\inf C_1)+F(\sup C_2)-F(\inf C_2)\cr
&=F(\sup C)-F(\inf C)\cr}$$

\noindent and $C\in\Cal J$.   If $I_1$, $I_2\in\Cal I$ and $I_1\cap I_2$
is non-empty, then $I_1\cup I_2$ is an interval;  also any subinterval
$C$ of $I_1\cup I_2$
is either included in one of the $I_j$ or is expressible as a disjoint
union $C_1\cup C_2$ where $C_j$ is a subinterval of $I_j$ for each $j$;
so $C\in\Cal J$ and $I\in\Cal I$.   If $I_1$, $I_2\in\Cal I$ and
$\sup I_1=\inf I_2$, then $I=I_1\cup I_2\cup\{\sup I_1\}\in\Cal I$,
because any subinterval of $I$ is expressible as the disjoint union of
at most three intervals in $\Cal J$.

\medskip

\quad{\bf (ii)} If $\Cal I_0\subseteq\Cal I$ is non-empty and
upwards-directed, then $\bigcup\Cal I_0\in\Cal I$.   \Prf\ This is a
consequence of 483Bd.   If we take a non-empty open subinterval $J$ of
$\bigcup\Cal I_0$ and express it as $\ooint{\alpha,\beta}$, where
$-\infty\le\alpha<\beta\le\infty$, then whenever $\alpha<a<b<\beta$ there
are members of $\Cal I_0$ containing $a$ and $b$, and therefore a member of
$\Cal I_0$ containing both, so that $[a,b]\in\Cal J$.   Accordingly
$\Heint_a^bf$ is defined and equal to $F(b)-F(a)$.   Since $F$ is
continuous, $\lim_{a\downarrow\alpha,b\uparrow\beta}\Heint_a^bf$ is defined
and equal to $F(\beta)-F(\alpha)$;  by 483Bd, $\Heint_{\alpha}^{\beta}f$ is
defined and equal to $F(\beta)-F(\alpha)$, so that $J\in\Cal J$.   I
wrote this out for open intervals, for convenience;  but any non-empty
subinterval of $\bigcup\Cal I_0$ is either a singleton or expressible as
an open interval with at most two points added, so belongs to $\Cal J$.
Accordingly $\bigcup\Cal I_0\in\Cal I$.\ \Qed

\medskip

\quad{\bf (iii)} It follows that every member of $\Cal I$ is included in
a maximal member of $\Cal I$.   Let $\Cal I^*$ be the set of maximal
members of $\Cal I$.   By (i), these are all disjoint, so no endpoint of
any member of $\Cal I^*$ can belong to $\bigcup\Cal I^*$.

\Quer\ Suppose, if possible, that $\Bbb R\notin\Cal I$.   Then
$\bigcup\Cal I=\bigcup\Cal I^*$ cannot be $\Bbb R$, and
$V=\Bbb R\setminus\bigcup\Cal I$ is a non-empty closed set.   By (i), no
two distinct members of $\Cal I^*$ can share a boundary point, so $V$
has no isolated points.

We are supposing that $F$ is ACG$_*$, so there is a countable family
$\Cal A$ of sets, covering $\Bbb R$, such that $F$ is AC$_*$ on $A$ for
every $A\in\Cal A$.   By Baire's theorem (3A3G or 4A2Ma), applied to the
locally compact Polish space $V$, $V\setminus\overline{A}$
cannot be dense in $V$ for every $A\in\Cal A$, so there are an
$A\in\Cal A$ and a bounded open interval $\tilde J$ such that
$\emptyset\ne V\cap\tilde J\subseteq\overline{A}$.   Set
$K=\overline{\tilde J\cap A}$;  by 483Pb, $F$ is AC$_*$ on $K$, and
$V\cap\tilde J\subseteq K$.   Because $V$ has no isolated points,
$V\cap\tilde J$ is infinite, so, setting $a^*=\min K$ and $b^*=\max K$,
$V\cap\ooint{a^*,b^*}$ is non-empty.

Let $\Cal I_0$ be the family of non-empty bounded open intervals, disjoint from $K$, with endpoints in $K$.   By 483Q,
$\sum_{I\in\Cal I_0}\omega(F\restr I)$ is
finite, and there is a Lebesgue integrable function $g:\Bbb R\to\Bbb R$
such that $g(x)=0$ for $x\in\Bbb R\setminus K$ and

\Centerline{$F(x)
=F(a^*)+\int_{a^*}^xg
  +\sum_{I\in\Cal I_0,I\subseteq[a^*,x]}F(\sup I)-F(\inf I)$}

\noindent for $x\in K$.   Since every member of $\Cal I_0$ is disjoint from $V$, it is
included in some member of $\Cal I^*$ and belongs to $\Cal J$, so
$F(\sup I)-F(\inf I)=\Heint f\times\chi I$ for every $I\in\Cal I_0$.
If $I\in\Cal I_0$, then

$$\eqalign{\|f\times\chi I\|_H
&=\sup_{C\in\Cal C}|\Heint f\times\chi I\times\chi C|
=\sup_{C\in\Cal C,C\subseteq I}|\Heint f\times\chi C|\cr
&=\sup_{C\in\Cal C,C\subseteq I}|F(\sup C)-F(\inf C)|
=\omega(F\restr I)\cr}$$

\noindent because every non-empty subinterval of $I$ belongs to
$\Cal J$.   So $\sum_{I\in\Cal I_0}\|f\times\chi I\|_H$ is finite, and
$f\times\chi H$ is Henstock integrable, where $H=\bigcup\Cal I_0$, by
483N.   Moreover, if $x\in K$, then

$$\eqalign{\Heint_{a^*}^xf\times\chi H
&=\Heint f\times\chi(\bigcup\{I:I\in\Cal I_0,\,I\subseteq[a^*,x]\})\cr
&=\sum_{I\in\Cal I_0,I\subseteq[a^*,x]}\Heint f\times\chi I
=\sum_{I\in\Cal I_0,I\subseteq[a^*,x]}F(\sup I)-F(\inf I).\cr}$$

\noindent But this means that if $y\in[a^*,b^*]$, and
$x=\max(K\cap[a^*,y])$, so that $\ooint{x,y}\subseteq H$, then

$$\eqalign{F(y)
&=F(x)+F(y)-F(x)\cr
&=F(a^*)
   +\int_{a^*}^xg
   +\sum_{I\in\Cal I_0,I\subseteq[a^*,x]}(F(\sup I)-F(\inf I))
   +\Heint f\times\chi\ooint{x,y}\cr
&=F(a^*)
   +\int g\times\chi(K\cap\coint{a^*,y})
   +\Heint_{a^*}^xf\times\chi H
   +\Heint f\times\chi\ooint{x,y}\cr
&=F(a^*)+\Heint_{-\infty}^yh\cr}$$

\noindent where $h=f\times\chi H+g\times\chi K$ is Henstock integrable
because $f\times\chi H$ is Henstock integrable and $g\times\chi K$ is
Lebesgue integrable.

Accordingly, if $C$ is any non-empty subinterval of $[a^*,b^*]$,
$F(\sup C)-F(\inf C)=\Heint h\times\chi C$.   But we also know that
$\bover{d}{dy}\Heint_{-\infty}^yh=h(y)$ for almost every $y$, by 483I.
So $F'(y)$ is defined and equal to $h(y)$ for almost every
$y\in[a^*,b^*]$, and $h=f$ a.e.\ on $[a^*,b^*]$.   This means that
$F(\sup C)-F(\inf C)=\Heint f\times\chi C$ for any non-empty subinterval
$C$ of $[a^*,b^*]$, and $\ooint{a^*,b^*}\in\Cal I$.   But
$\ooint{a^*,b^*}$ meets $V$, so this is impossible.\ \Bang

\medskip

\quad{\bf (iv)} This contradiction shows that $\Bbb R\in\Cal I$ and that
$F$ is an indefinite Henstock integral, as required.
}%end of proof of 483R

\exercises{\leader{483X}{Basic exercises $\pmb{>}$(a)}
%\spheader 483Xa
Let $[a,b]\subseteq\Bbb R$ be a non-empty closed
interval, and let $I_{\mu}$ be the gauge integral on $[a,b]$ defined
from Lebesgue measure and the tagged-partition structure defined in
481J.   Show that, for $f:\Bbb R\to\Bbb R$,
$I_{\mu}(f\restrp[a,b])=\Heint f\times\chi[a,b]$ if either is defined.
%483A

\sqheader 483Xb Extract ideas from the proofs of 482G and 482H to give
a direct proof of 483B(c)-(d).
%483B

\spheader 483Xc Set $f(0)=0$ and $f(x)=\bover{\sin x}{x}$ for other real
$x$.   Show that $f$ is Henstock integrable, and that
$\Heint_{\mskip4mu 0}^{\infty}f=\bover{\pi}2$.   \Hint{283Da.}
%483B

\spheader 483Xd Set $f(x)=\Bover1x\cos(\Bover1{x^2})$ for $0<x\le 1$,
$0$ for other real $x$.   Show that $f$ is Henstock integrable but not
Lebesgue integrable.   \Hint{by considering
$\Bover{d}{dx}(x^2\sin\Bover1{x^2})$, show that
$\lim_{a\downarrow 0}\int_a^1f$ is defined.}
%483B

\spheader 483Xe  Let $f:\Bbb R\to\Bbb R$ be a Henstock integrable
function.   Show that there is a finitely additive functional
$\lambda:\Cal P\Bbb R\to\Bbb R$ such that for every $\epsilon>0$ there
are a gauge $\delta\in\Delta$ and a Radon measure $\nu$ on $\Bbb R$ such
that $\nu\Bbb R\le\epsilon$ and
$|S_{\pmb{t}}(f,\mu)-\lambda W_{\pmb{t}}|\le\nu W_{\pmb{t}}$ for every
$\delta$-fine $\pmb{t}\in T$.
%483F

\sqheader 483Xf Let $f:\Bbb R\to\Bbb R$ be a Henstock integrable
function.   Show that $\bigcup\{G:G\subseteq\Bbb R$ is open,
$f$ is Lebesgue integrable over $G\}$ is dense.   \Hint{483G.}
%483G

\sqheader 483Xg Let $f:\Bbb R\to\Bbb R$ be a Henstock integrable
function and $F$ its indefinite Henstock integral.   Show that $f$ is
Lebesgue integrable iff $F$ is of bounded variation on $\Bbb R$.
\Hint{224I.}
%483I

\sqheader 483Xh Let $F:\Bbb R\to\Bbb R$ be a continuous function such
that $\lim_{x\to-\infty}F(x)=0$, $\lim_{x\to\infty}F(x)$ is defined in
$\Bbb R$, and $F'(x)$ is defined for all but countably many
$x\in\Bbb R$.   Show that $F$ is the indefinite Henstock integral of any
function $f:\Bbb R\to\Bbb R$ extending $F'$.   \Hint{in 483J, take $F_1$
and $F_2$ differing from $F$ by saltus functions.}
%483J

\spheader 483Xi Let $f:\Bbb R\to\Bbb R$ be a Henstock integrable
function, and $\sequencen{I_n}$ a disjoint sequence of intervals in
$\Bbb R$.   Show that $\lim_{n\to\infty}\|f\times\chi I_n\|_H=0$.
%483D

\spheader 483Xj Show that $HL^1$ is not a Banach space.   \Hint{there is
a continuous function which is nowhere differentiable (477K).}
%483M

\sqheader 483Xk Use 483N to replace part of the proof of 483Bd.
%483N

\spheader 483Xl Let $F:\Bbb R\to\Bbb R$ be such that $\DiniD F$ and $\Dinid F$
are both finite everywhere.   Show that
$F(b)-F(a)=\Heint\DiniD F\times\chi[a,b]$ whenever $a\le b$ in $\Bbb R$.
\Hint{$F$ is AC$_*$ on
$\{x:|F(y)-F(x)|\le n|y-x|$ whenever $|y-x|\le 2^{-n}\}$.}
%483R

\spheader 483Xm\dvAnew{2011}
For integers $r\ge 1$, write $\Cal C_r$ for the family of
subsets of $\BbbR^r$ of the form $\prod_{i<r}C_i$ where 
$C_i\subseteq\Bbb R$ is a bounded interval for each $i<r$.   Set
$Q_r=\{(x,C):C\in\Cal C_r$, $x\in\overline{C}\}$;  let $T_r$ be the
straightforward set of tagged partitions generated by $Q_r$, 
$\Delta_r$ the set
of neighbourhood gauges on $\BbbR^r$, and 
$\frak R_r=\{\Cal R_C:C\in\Cal C_r\}$
where $\Cal R_C
=\{\BbbR^r\setminus C':C\subseteq C'\in\Cal C_r\}\cup\{\emptyset\}$ for 
$C\in\Cal C_r$.   Let $\nu_r$ be the restriction of $r$-dimensional
Lebesgue measure to $\Cal C_r$.   
(i) Show that $(\BbbR^r,T_r,\Delta_r,\frak R_r)$ is
a tagged-partition structure allowing subdivisions, witnessed by 
$\Cal C_r$.   (ii) For a function $f:\BbbR^r\to\Bbb R$ write 
$\Heint f(x)dx$ for the gauge integral $I_{\nu_r}(f)$ associated with this
structure when it is defined.   Show that if $r$, $s\ge 1$ are integers,
$f:\BbbR^{r+s}\to\Bbb R$ has compact support and $\Heint f(z)dz$ is
defined, then, identifying $\BbbR^{r+s}$ with $\BbbR^r\times\BbbR^s$,
$\Heint g(x)dx$ is defined and equal to $\Heint f(x,y)d(x,y)$ whenever
$g:\BbbR^r\to\Bbb R$ is such that $g(x)=\Heint f(x,y)dy$ for every
$x\in\BbbR^r$ for which this is defined.   
%482M

\leader{483Y}{Further exercises (a)}
%\spheader 483Ya
Let us say that a {\bf Lebesgue measurable neighbourhood
gauge} on $\Bbb R$ is a neighbourhood gauge of the form
$\{(x,A):x\in\Bbb R,\,A\subseteq\ooint{x-\eta_x,x+\eta_x}\}$ where
$x\mapsto\eta_x$ is a Lebesgue measurable function from $\Bbb R$ to
$\ooint{0,\infty}$.   Let $\tilde\Delta$ be the set of Lebesgue
measurable neighbourhood gauges.   Show that the gauge integral defined
by the tagged-partition structure $(\Bbb R,T,\tilde\Delta,\frak R)$ and
$\mu$ is the Henstock integral.
%483F

\spheader 483Yb Show that if $\Delta_0\subseteq\Delta$ is any set of
cardinal at most $\frak c$, then the gauge integral defined by
$(\Bbb R,T,\Delta_0,\frak R)$ and $\mu$ does not extend the Lebesgue
integral, so is not the Henstock integral.
%483Ya, 483B

\spheader 483Yc Let $f:\Bbb R\to\Bbb R$ be a Henstock integrable
function with indefinite Henstock integral $F$, and $\nu$ a totally
finite Radon measure on $\Bbb R$.   Set $G(x)=\nu\ocint{-\infty,x}$ for
$x\in\Bbb R$.   Show that $f\times G$ is Henstock integrable, with
indefinite Henstock integral $H$, where
$H(x)=F(x)G(x)-\int_{\ocint{-\infty,x}}F\,d\nu$ for $x\in\Bbb R$.
%483F

\spheader 483Yd Let $\nu$ be any Radon measure on $\Bbb R$,
$I_{\nu}$ the gauge integral defined from $\nu$ and the
tagged-partition structure of 481K and this section, and
$f:\Bbb R\to\Bbb R$ a function.

\quad(i) Show that if $I_{\nu}(f)$ is defined, then $f$ is
$\dom\nu$-measurable.

\quad(ii) Show that if $\int fd\nu$ is
defined in $\Bbb R$, then $I_{\nu}(f)$ is defined and equal to
$\int fd\nu$.

\quad(iii) Show that if $\alpha\in\ocint{-\infty,\infty}$ then
$I_{\nu}(f\times\chi\ooint{-\infty,\alpha})
=\lim_{\beta\uparrow\alpha}
  I_{\nu}(f\times\chi\ooint{-\infty,\beta})$ if either is defined in
$\Bbb R$.

\quad(iv) Suppose that $I_{\nu}(f)$ is defined.   ($\alpha$) Let $\FSH$ be
the Saks-Henstock indefinite integral of $f$ with respect to $\nu$.   Show
that for any $\epsilon>0$ there are a Radon measure $\zeta$ on $\Bbb R$
and a $\delta\in\Delta$ such that $\zeta\Bbb R\le\delta$ and
$|\FSH(W_{\pmb{t}})-S_{\pmb{t}}(f,\nu)|\le\zeta W_{\pmb{t}}$ whenever
$\pmb{t}\in T$ is $\delta$-fine.
($\beta$) Show that there is a
countable cover of $\Bbb R$ by compact sets $K$ such that
$\int_K|f|d\nu<\infty$.
%+ 483B 483Xe 483G 483F

\spheader 483Ye Let $f:\Bbb R\to\Bbb R$ be a Henstock integrable
function, and $G:\Bbb R\to\Bbb R$ a function of bounded variation.
Show that $f\times G$ is Henstock integrable, and

\Centerline{$\biggerHeint f\times G
\le(\lim_{x\to\infty}|G(x)|+\Var_{\Bbb R}G)
  \sup_{x\in\Bbb R}|\biggerHeint_{-\infty}^xf|$.}

\noindent(Compare 224J.)
%483Yc, 483M

\spheader 483Yf Let $f:\Bbb R\to\Bbb R$ be a Henstock integrable
function and $g:\Bbb R\to\Bbb R$ a Lebesgue integrable function;  let
$F$ and $G$ be their indefinite (Henstock) integrals.   Show that
$\Heint f\times G+\int g\times F\,d\mu$ is defined and equal to
$\lim_{x\to\infty}F(x)G(x)$.
%483Ye, 483M

\spheader 483Yg Let $f:\Bbb R\to\Bbb R$ be a function.   Show that $f$
is Lebesgue integrable iff $f\times g$ is Henstock integrable for every
bounded continuous function $g:\Bbb R\to\Bbb R$.
%483Ye, 483M

\spheader 483Yh Let $U$ be a linear subspace of $\BbbR^{\Bbb R}$ and
$\phi:U\to\Bbb R$ a linear functional such that (i) $f\in U$ and
$\phi f=\int fd\mu$ for every Lebesgue integrable function
$f:\Bbb R\to\Bbb R$ (ii) $f\times\chi C\in U$ whenever $f\in U$ and
$C\in\Cal C$ (iii) whenever $f\in\BbbR^{\Bbb R}$
and $\Cal I$ is a disjoint family
of non-empty open intervals such that $f\times\chi I\in U$ for every
$I\in\Cal I$ and
$\sum_{I\in\Cal I}\sup_{C\subseteq I,C\in\Cal C}
  |\phi(f\times\chi C)|<\infty$, then $f\times\chi(\bigcup\Cal I)\in U$
and
$\phi(f\times\chi(\bigcup\Cal I))=\sum_{I\in\Cal I}\phi(f\times\chi I)$.
Show that if $f:\Bbb R\to\Bbb R$ is any Henstock integrable function,
then $f\in U$ and $\phi(f)=\Heint f$.   \Hint{use the argument of part
(b) of the proof of 483R.} 
%483R 483Xf

\spheader 483Yi ({\smc Bongiorno Piazza \& Preiss 00}) Let $\Cal C$ be the set of non-empty subintervals of a closed interval
$[a,b]\subseteq\Bbb R$, and $T$ the straightforward set of tagged
partitions generated by $[a,b]\times\Cal C$.   Let $\Delta_{[a,b]}$
be the set of neighbourhood gauges on $[a,b]$.   For $\alpha\ge 0$ set

\Centerline{$T_{\alpha}
=\{\pmb{t}:\pmb{t}\in T,\,\sum_{(x,C)\in\pmb{t}}\rho(x,C)\le\alpha\}$,}

\noindent writing $\rho(x,C)=\inf_{y\in C}|x-y|$ as usual.   Show that
$T_{\alpha}$ is compatible with $\Delta_{[a,b]}$ and
$\frak R=\{\{\emptyset\}\}$
in the sense of 481F.   Show that if $I_{\alpha}$ is the gauge integral
defined from $[a,b]$, $T_{\alpha}$, $\Delta_{[a,b]}$, $\frak R$ and
Lebesgue
measure, then $I_{\alpha}$ extends the ordinary Lebesgue integral and
$I_{\alpha}(F')=F(b)-F(a)$ whenever $F:[a,b]\to\Bbb R$ is
differentiable relative to its domain.
%+

\spheader 483Yj Let $V$ be a Banach space and $f:\Bbb R\to V$ a
function.   For $\pmb{t}\in T$, set
$S_{\pmb{t}}(f,\mu)=\sum_{(x,C)\in\pmb{t}}\mu C\cdot f(x)$.
We say that $f$ is {\bf Henstock integrable}, with {\bf Henstock integral}
$v=\Heint f\in V$, if
$v=\lim_{\pmb{t}\to\Cal F(T,\Delta,\frak R)}S_{\pmb{t}}(f,\mu)$.

\quad(i) Show that the set $\eusm{HL}^1_V$ of Henstock integrable
functions from $\Bbb R$ to $V$ is a linear subspace of $V^{\Bbb R}$
including the space $\eusm L^1_V$ of Bochner integrable functions
(253Yf), and that $\Heint:\eusm{HL}^1_V\to V$ is a linear operator
extending the Bochner integral.

\quad(ii) Show that if $f:\Bbb R\to V$ is Henstock integrable, so is
$f\times\chi C$ for every interval $C\subseteq\Bbb R$, and that
$(a,b)\mapsto\Heint f\times\chi\ooint{a,b}$ is continuous.   Set
$\|f\|_H=\sup_{C\in\Cal C}\|\Heint f\times\chi C\|$.

\quad(iii) Show that if $\Cal I$ is a disjoint family of open intervals
in $\Bbb R$, and $f:\Bbb R\to V$ is such that
$f\times\chi I\in\eusm{HL}^1_V$ for every $I\in\Cal I$ and
$\sum_{I\in\Cal I}\|f\times\chi I\|_H$ is finite, then
$\Heint f\times\chi(\bigcup\Cal I)$ is defined and equal to
$\sum_{I\in\Cal I}\Heint f\times\chi I$.

\quad(iv) Define $f:\Bbb R\to\ell^{\infty}([0,1])$ by setting
$f(x)=\chi([0,1]\cap\ocint{-\infty,x})$ for $x\in\Bbb R$.   Show that
$f$ is Henstock integrable, but that if
$F(x)=\Heint f\times\chi\ooint{-\infty,x}$ for $x\in\Bbb R$, then
$\lim_{y\to x}\Bover1{y-x}(F(y)-F(x))$ is not defined in
$\ell^{\infty}([0,1])$ for any $x\in[0,1]$.

\spheader 483Yk\dvAnew{2011}
Find a function $f:\BbbR^2\to\Bbb R$, with compact support, such that
$\Heint f$ is defined in the sense of 483Xm, but $\Heint fT$ is not
defined, where $T(x,y)=\Bover1{\sqrt2}(x+y,x-y)$ for $x$, $y\in\Bbb R$.
%483Xm

\spheader 483Yl\dvAnew{2011}
Show that for a function $g:\Bbb R\to\Bbb R$ the following are 
equiveridical:  (i) there is a function $h:\Bbb R\to\Bbb R$, of bounded 
variation, such that $g\eae h$ (ii) $g$ is a {\bf multiplier} for the 
Henstock integral, that is, $f\times g$ is Henstock integrable for every 
Henstock integrable $f:\Bbb R\to\Bbb R$.
%483Ye mt48bits
}%end of exercises

\endnotes{
\Notesheader{483} I hope that the brief account here (largely taken from
{\smc Gordon 94}) will give an idea of the extraordinary power of gauge
integrals.   While what I am calling the `Henstock integral', regarded
as a linear functional on a space of real functions, was constructed
long ago by Perron and Denjoy, the gauge integral approach makes it far
more accessible, and gives clear pathways to corresponding Stieltjes and
vector integrals (483Yd, 483Yj).

Starting from our position in the fourth volume of a book on measure
theory, it is natural to try to describe the Henstock integral in terms
of the Lebesgue integral, as in 483C (they agree on non-negative
functions) and 483Yh (offering an extension process to generate the
Henstock integral from the Lebesgue integral);  on the way, we see that
Henstock integrable functions are necessarily Lebesgue integrable over
many intervals (483Xf).   Alternatively, we can set out to understand
indefinite Henstock integrals and their derivatives, just as Lebesgue
integrable functions can be characterized as almost everywhere equal to
derivatives of absolutely continuous functions (222E, 225E), because if
$f$ is Henstock integrable then it is equal almost everywhere to the
derivative of its indefinite integral (483I).   Any differentiable
function (indeed, any continuous function differentiable except on a
countable set) is an indefinite Henstock integral (483Xh).   Recall that
the Cantor function (134H) is continuous and differentiable almost
everywhere but is not an indefinite integral, so we have to look for a
characterization which can exclude such cases.   For this we have to
work quite hard, but we find that `ACG$_*$ functions' are the
appropriate class (483R).

Gauge integrals are good at integrating derivatives (see 483Xh), but bad
at integrating over subspaces.   Even to show that
$f\times\chi\coint{0,\infty}$ is Henstock integrable whenever $f$ is
(483Bc) involves us in some unexpected manoeuvres.   I give an argument
which is designed to show off the general theory of \S482, and I
recommend you to look for short cuts (483Xb), but any method must depend
on careful examination of the exact classes $\Cal C$ and $\frak R$
chosen for the definition of the integral.   We do however have a new
kind of convergence theorem in 482H and 483N.

One of the incidental strengths of the Henstock integral is that it
includes the improper Riemann integral (483Bd, 483Xc);  so that, for
instance, Carleson's theorem (286U) can be written in the form

\Centerline{$\varhatf(y)
=\Bover1{\sqrt{2\pi}}\biggerHeint e^{-ixy}f(x)dx$
for almost every $y$ if $f:\Bbb R\to\Bbb C$ is square-integrable.}

\noindent But to represent the many expressions of the type
$\lim_{a\to\infty}\int_{-a}^af$ in \S283 (e.g., 283F, 283I and 283L)
directly in the form $I_{\mu}(f)$ we need to change $\frak R$, as in
481L or 481Xc.
}%end of notes

\discrpage

