\frfilename{mt345.tex}
\versiondate{27.6.06}
\copyrightdate{1995}

\def\upr{\mathop{\text{upr}}}

\def\chaptername{Liftings}
\def\sectionname{Translation-invariant liftings}

\newsection{345}

In this section and the next I complement the work of \S341 by
describing some
important special properties which can, in appropriate circumstances, be
engineered into our liftings.   I begin with some remarks on
translation-invariance.   I restrict
my attention to measure spaces which we have already seen, delaying a
general discussion of translation-invariant measures on groups until
Volume 4.

\vleader{60pt}{345A}{Translation-invariant liftings} I shall
consider two forms of translation-invariance, as follows.

\spheader 345Aa Let $\mu$ be Lebesgue measure on $\BbbR^r$, and
$\Sigma$ its domain.   A lifting $\phi:\Sigma\to\Sigma$ is {\bf
translation-invariant} if $\phi(E+x)=\phi E+x$ for every $E\in\Sigma$,
$x\in\BbbR^r$.   \cmmnt{(Recall from 134A that $E+x=\{y+x:y\in E\}$
belongs to $\Sigma$ for every $E\in\Sigma$, $x\in\BbbR^r$.)}

Similarly, writing $\frak A$ for the measure algebra of $\mu$, a
lifting $\theta:\frak A\to\Sigma$ is {\bf translation-invariant} if
$\theta(E+x)^{\ssbullet}=\theta E^{\ssbullet}+x$ for every $E\in\Sigma$,
$x\in\BbbR^r$.

\cmmnt{It is easy to see that if $\theta$ and $\phi$ correspond to
each other in the manner of 341B, then one is translation-invariant if
and only if the other is.}

\spheader 345Ab Now let $I$ be any set, and let $\nu_I$ be the usual
measure on $X=\{0,1\}^I$, with $\Tau_I$ its domain and $\frak B_I$ its
measure algebra.   For $x$, $y\in X$, define $x+y\in X$ by setting
$(x+y)(i)=x(i)+_2y(i)$ for every $i\in I$\cmmnt{;  that is, give $X$
the group structure of the product group $\Bbb Z_2^I$}.
\cmmnt{This makes $X$ an abelian
group (isomorphic to the additive group $(\Cal PI,\symmdiff)$ of the
Boolean algebra $\Cal PI$, if we match $x\in X$ with
$\{i:x(i)=1\}\subseteq I$).

\cmmnt{Recall that the measure $\nu_I$ is a product measure (254J),
being the product of copies of the fair-coin probability measure on the
two-element set
$\{0,1\}$.}   If $x\in X$,\cmmnt{ then for each $i\in I$ the map
$\epsilon\mapsto\epsilon+_2x(i):\{0,1\}\to\{0,1\}$ is a measure space
automorphism of $\{0,1\}$, since the two singleton sets $\{0\}$ and
$\{1\}$ have the same measure $\bover12$.   It follows at once that} the
map $y\mapsto y+x:X\to X$ is a measure space automorphism.
}

\ifresultsonly{We }\else{Accordingly we can again }\fi
say that a lifting $\theta:\frak B_I\to\Tau_I$,
or $\phi:\Tau_I\to\Tau_I$, is {\bf translation-invariant} if

\Centerline{$\theta(E+x)^{\ssbullet}=\theta E^{\ssbullet}+x$,
\quad$\phi(E+x)=\phi E+x$}

\noindent whenever $E\in\Sigma$ and $x\in X$.

\leader{345B}{Theorem} For any $r\ge 1$, there is a
translation-invariant lifting for Lebesgue measure on $\BbbR^r$.

\proof{{\bf (a)} Write $\mu$ for Lebesgue measure on $\BbbR^r$,
$\Sigma$ for its domain.   Let $\undphi:\Sigma\to\Sigma$ be lower
Lebesgue density (341E).   Then $\undphi$ is translation-invariant in
the sense that $\undphi(E+x)=\undphi E+x$ for every $E\in\Sigma$,
$x\in\BbbR^r$.   \Prf\

$$\eqalignno{\undphi(E+x)
&=\{y:y\in\BbbR^r,\,\lim_{\delta\downarrow 0}
\Bover{\mu(E+x)\cap B(y,\delta)}{\mu(B(y,\delta))}=1\}\cr
&=\{y:y\in\BbbR^r,\,\lim_{\delta\downarrow 0}
\Bover{\mu(E\cap B(y-x,\delta))}{\mu(B(y-x,\delta))}=1\}\cr
\noalign{\noindent (because $\mu$ is translation-invariant)}
&=\{y+x:y\in\BbbR^r,\,\lim_{\delta\downarrow 0}
\Bover{\mu(E\cap B(y,\delta))}{\mu(B(y,\delta))}=1\}\cr
&=\undphi E+x.   \text{  \Qed}\cr}$$

\medskip

{\bf (b)} Let $\phi_0$ be any lifting for $\mu$ such that
$\phi_0E\supseteq\undphi E$ for every $E\in\Sigma$ (341Jb).    Consider

\Centerline{$\phi E=\{y:\tbf{0}\in\phi_0(E-y)\}$}

\noindent for $E\in\Sigma$.
It is easy to check that $\phi:\Sigma\to\Sigma$ is a Boolean
homomorphism because $\phi_0$ is, so that, for instance,

$$\eqalign{y\in\phi E\symmdiff\phi F
&\iff \tbf{0}\in\phi_0(E-y)\symmdiff\phi_0(F-y)\cr
&\iff \tbf{0}\in\phi_0((E-y)\symmdiff(F-y))=\phi_0((E\symmdiff F)-y)\cr
&\iff y\in\phi(E\symmdiff F).\cr}$$

\medskip

{\bf (c)} If $\mu E=0$, then $E-y$ is negligible for every $y\in\Bbb
R^r$, so $\phi_0(E-y)$ is always empty and $\phi E=\emptyset$.

\medskip

{\bf (d)} Next, $\undphi E\subseteq\phi E$ for every $E\in\Sigma$.
\Prf\ If $y\in\undphi E$, then

\Centerline{$\tbf{0}=y-y\in\undphi E-y
=\undphi(E-y)\subseteq\phi_0(E-y)$,}

\noindent so $y\in\phi E$.\ \QeD\  By 341Ib, $\phi$ is a lifting for
$\mu$.

\medskip

{\bf (e)} Finally, $\phi$ is translation-invariant, because if
$E\in\Sigma$ and $x$, $y\in\BbbR^r$ then

$$\eqalign{y\in\phi(E+x)
&\iff\tbf{0}\in\phi_0(E+x-y)=\phi_0(E-(y-x))\cr
&\iff y-x\in\phi E\cr
&\iff y\in\phi E+x.\cr}$$
}%end of proof of 345B

\leader{345C}{Theorem} For any set $I$, there is a translation-invariant
lifting for the usual measure on $\{0,1\}^I$.

\proof{ I base the argument on the same programme as in 345B.   This
time we have to work rather harder, as we have no simple formula for a
translation-invariant lower density.   However, the ideas already used
in 341F-341H are in fact adequate, if we take care, to produce one.

\medskip

{\bf (a)} Since there is certainly a bijection between
$I$ and its cardinal $\kappa=\#(I)$, it is enough to consider the case
$I=\kappa$.   Write $\nu_{\kappa}$ for the usual measure on
$X=\{0,1\}^I=\{0,1\}^{\kappa}$ and $\Tau_{\kappa}$ for its domain.   For each
$\xi<\kappa$ set $E_{\xi}=\{x:x\in X,\,x(\xi)=1\}$, and let
$\Sigma_{\xi}$ be the $\sigma$-algebra generated by
$\{E_{\eta}:\eta<\xi\}$.   Because $x+E_{\eta}$ is either $E_{\eta}$ or
$X\setminus E_{\eta}$, and in either case belongs to $\Sigma_{\xi}$, for
every $\eta<\xi$ and $x\in X$, $\Sigma_{\xi}$ is translation-invariant.
(Consider
the algebra

\Centerline{$\Sigma_{\xi}'=\{E:E+x\in\Sigma_{\xi}$ for every $x\in
X\}$;}

\noindent this must be $\Sigma_{\xi}$.)   Let $\Phi_{\xi}$ be the set of
partial lower densities $\undphi:\Sigma_{\xi}\to\Tau_{\kappa}$ which are
translation-invariant in the sense that $\undphi(E+x)=\undphi E+x$ for
any $E\in\Sigma_{\xi}$, $x\in X$.


\medskip

{\bf (b)(i)} For $\xi<\kappa$,
$\Sigma_{\xi+1}$ is just the algebra of subsets of $X$ generated by
$\Sigma_{\xi}\cup\{E_{\xi}\}$, that is, sets of the form
$(F\cap E_{\xi})\cup(G\setminus E_{\xi})$ where $F$, $G\in\Sigma_{\xi}$
(312N).
Moreover, the expression is unique.   \Prf\ Define $x_{\xi}\in X$ by
setting $x_{\xi}(\xi)=1$, $x_{\xi}(\eta)=0$ if $\eta\ne\xi$.   Then
$x_{\xi}+E_{\eta}=E_{\eta}$ for every $\eta<\xi$, so $x_{\xi}+F=F$ for
every $F\in\Sigma_{\xi}$.   If
$H=(F\cap E_{\xi})\cup(G\setminus E_{\xi})$ where $F$,
$G\in\Sigma_{\xi}$, then

\Centerline{$x_{\xi}+H
=((x_{\xi}+F)\cap(x_{\xi}+E_{\xi}))\cup((x_{\xi}+G)\setminus
(x_{\xi}+E_{\xi}))
=(F\setminus E_{\xi})\cup(G\cap E_{\xi})$,}

\noindent so

\Centerline{$F=(H\cap E_{\xi})\cup((x_{\xi}+H)\setminus E_{\xi})=F_H$,}

\Centerline{$G=(H\setminus E_{\xi})\cup((x_{\xi}+H)\cap E_{\xi})=G_H$}

\noindent are determined by $H$.\ \Qed

\medskip

\quad{\bf (ii)} The functions $H\mapsto F_H$,
$H\mapsto G_H:\Sigma_{\xi+1}\to\Sigma_{\xi}$ defined above are clearly
Boolean homomorphisms;  moreover, if $H$, $H'\in\Sigma_{\xi+1}$ and
$H\symmdiff H'$ is negligible, then

\Centerline{$(F_H\symmdiff F_{H'})\cup(G_H\symmdiff G_{H'})
\subseteq(H\symmdiff H')\cup(x_{\xi}+(H\symmdiff H'))$}

\noindent is negligible.   It follows at once that if
$\xi<\kappa$ and $\undphi\in\Phi_{\xi}$, we can define
$\undphi_1:\Sigma_{\xi+1}\to\Tau_{\kappa}$ by setting

\Centerline{$\undphi_1H
=(\undphi F_H\cap E_{\xi})\cup(\undphi G_H\setminus E_{\xi})$,}

\noindent and $\undphi_1$ will be a lower density.   If
$H\in\Sigma_{\xi}$ then $F_H=G_H=H$, so $\undphi_1H=\undphi H$.
Generally, if $H$, $H'\in\Sigma_{\xi}$ then

\Centerline{$\undphi_1((H\cap E_{\xi})\cup(H'\setminus E_{\xi}))
=(\undphi F_H\cap E_{\xi})\cup(\undphi G_{H'}\setminus E_{\xi})
=(\undphi H\cap E_{\xi})\cup(\undphi H'\setminus E_{\xi})$.}

\medskip

\quad{\bf (iii)} To see that $\undphi_1$ is translation-invariant,
observe that if $x\in X$ and $x(\xi)=0$ then $x+E_{\xi}=E_{\xi}$, so,
for any $F$, $G\in\Sigma_{\xi}$,

$$\eqalign{\undphi_1(x+((F\cap E_{\xi})\cup(G\setminus E_{\xi})))
&=\undphi_1(((F+x)\cap E_{\xi})\cup((G+x)\setminus E_{\xi}))\cr
&=(\undphi(F+x)\cap E_{\xi})\cup(\undphi(G+x)\setminus E_{\xi})\cr
&=((\undphi F+x)\cap E_{\xi})\cup((\undphi G+x)\setminus E_{\xi})\cr
&=x+(\undphi F\cap E_{\xi})\cup(\undphi G\setminus E_{\xi})\cr
&=x+\undphi_1((F\cap E_{\xi})\cup(G\setminus E_{\xi})).\cr}$$

\noindent While if $x(\xi)=1$ then $x+E_{\xi}=X\setminus E_{\xi}$, so

$$\eqalign{\undphi_1(x+((F\cap E_{\xi})\cup(G\setminus E_{\xi})))
&=\undphi_1(((F+x)\setminus E_{\xi})\cup((G+x)\cap E_{\xi}))\cr
&=(\undphi(F+x)\setminus E_{\xi})\cup(\undphi(G+x)\cap E_{\xi})\cr
&=((\undphi F+x)\setminus E_{\xi})\cup((\undphi G+x)\cap E_{\xi})\cr
&=x+(\undphi F\cap E_{\xi})\cup(\undphi G\setminus E_{\xi})\cr
&=x+\undphi_1((F\cap E_{\xi})\cup(G\setminus E_{\xi})).\cr}$$

\noindent So $\undphi_1\in\Phi_{\xi+1}$.

\medskip

\quad{\bf (iv)} Thus every member of $\Phi_{\xi}$ has an extension to a
member of $\Phi_{\xi+1}$.

\medskip

{\bf (c)} Now suppose that $\sequencen{\zeta(n)}$ is a non-decreasing
sequence in $\kappa$ with supremum $\xi<\kappa$.   Then $\Sigma_{\xi}$
is just the $\sigma$-algebra generated by
$\bigcup_{n\in\Bbb N}\Sigma_{\zeta(n)}$.   If we have a sequence
$\sequencen{\undphi_n}$
such that $\undphi_n\in\Phi_{\zeta(n)}$ and $\undphi_{n+1}$ extends
$\undphi_n$ for every $n$, then there is a $\undphi\in\Phi_{\xi}$
extending every $\undphi_n$.   \Prf\ I repeat the ideas of 341G.

\medskip

\quad{\bf (i)} For $E\in\Sigma_{\xi}$, $n\in\Bbb N$ choose
$g_{En}$ such that $g_{En}$ is a conditional
expectation of $\chi E$ on $\Sigma_{\zeta(n)}$;  that is,

\Centerline{$\int_Fg_{En}=\int_F\chi E=\nu_{\kappa}(F\cap E)$}

\noindent for every $E\in\Sigma_{\zeta(n)}$.   Moreover, make these
choices in such a way that ($\alpha$) every $g_{En}$ is
$\Sigma_{\zeta(n)}$-measurable and defined everywhere on $X$ ($\beta$)
$g_{En}=g_{E'n}$ for every $n$ if $E\symmdiff E'$ is negligible.
Now $\lim_{n\to\infty}g_{En}$ exists and is equal
to $\chi E$ almost everywhere, by L\'evy's martingale
theorem (275I).

\medskip

\quad{\bf (ii)} For $E\in\Sigma_{\xi}$, $k\ge 1$, $n\in\Bbb N$ set

\Centerline{$H_{kn}(E)
=\{x:x\in X,\,g_{En}(x)\ge 1-2^{-k}\}\in\Sigma_{\zeta(n)}$,
\quad $\tilde H_{kn}(E)=\undphi_n(H_{kn}(E))$,}

\Centerline{$\undphi E
=\bigcap_{k\ge 1}\bigcup_{n\in\Bbb N}\bigcap_{m\ge n}\tilde H_{km}(E)$.}

\medskip

\quad{\bf (iii)} Every $g_{\emptyset n}$ is zero almost
everywhere, every $H_{kn}(\emptyset)$ is negligible and every
$\tilde H_{kn}(\emptyset)$
is empty;  so $\undphi\emptyset=\emptyset$.   If $E$,
$E'\in\Sigma_{\xi}$ and $E\symmdiff E'$ is negligible, $g_{En}=g_{E'n}$
for every $n$, $H_{nk}(E)=H_{nk}(E')$ and
$\tilde H_{nk}(E)=\tilde H_{nk}(E')$ for all $n$, $k$,
and $\undphi E=\undphi E'$.

\medskip

\quad{\bf (iv)} If $E\subseteq F$ in $\Sigma_{\xi}$, then $g_{En}\le
g_{Fn}$ almost everywhere for every $n$, every $H_{kn}(E)\setminus
H_{kn}(F)$ is negligible, $\tilde H_{kn}(E)\subseteq\tilde H_{kn}(F)$
for every $n$, $k$, and $\undphi E\subseteq \undphi F$.

\medskip

\quad{\bf (v)} If $E$, $F\in\Sigma_{\xi}$ then
$\chi(E\cap F)\geae\chi E+\chi F-1$ so
$g_{E\cap F,n}\geae g_{En}+g_{Fn}-1$ for every $n$.   Accordingly

\Centerline{$H_{k+1,n}(E)\cap H_{k+1,n}(F)\setminus H_{kn}(E\cap F)$}

\noindent is negligible, and (because $\undphi_n$ is a lower density)

\Centerline{$\tilde H_{kn}(E\Bcap F)\supseteq
\undphi_n(H_{k+1,n}(E)\cap H_{k+1,n}(F))
=\tilde H_{k+1,n}(E)\cap\tilde H_{k+1,n}(F)$}

\noindent for all $k\ge 1$, $n\in\Bbb N$.   Now, if $x\in\undphi
E\cap\undphi F$, then, for any $k\ge 1$, there are $n_1$, $n_2\in\Bbb
N$ such that

\Centerline{$x\in\bigcap_{m\ge n_1}\tilde H_{k+1,m}(E)$,
\quad$x\in\bigcap_{m\ge n_2}\tilde H_{k+1,m}(F)$.}

\noindent But this means that

\Centerline{$x\in\bigcap_{m\ge\max(n_1,n_2)}\tilde H_{km}(E\cap F)$.}

\noindent As $k$ is arbitrary, $x\in\undphi (E\cap F)$;  as $x$ is
arbitrary, $\undphi E\cap\undphi F\subseteq\undphi(E\cap F)$.
We know already from (iv) that $\undphi(E\cap F)\subseteq\undphi
E\cap\undphi F$, so $\undphi(E\cap F)=\undphi E\cap\undphi F$.

\medskip

\quad{\bf (vi)} If $E\in\Sigma_{\xi}$, then $g_{En}\to\chi E$ a.e., so
setting

\Centerline{$V=\bigcap_{k\ge 1}\bigcup_{n\in\Bbb N}\bigcap_{m\ge
n}H_{km}(E)=\{x:\limsup_{n\to\infty}g_{En}(x)\ge 1\}$,}

\noindent $V\symmdiff E$ is negligible;  but

\Centerline{$\undphi E\symmdiff V
\subseteq\bigcup_{k\ge 1,n\in\Bbb N}H_{kn}(E)\symmdiff\tilde H_{kn}(E)$}

\noindent is also negligible, so $\undphi E\symmdiff E$ is negligible.
Thus $\undphi$ is a partial lower density with domain $\Sigma_{\xi}$.

\medskip

\quad{\bf (vii)}  If $E\in\Sigma_{\zeta(n)}$, then
$E\in\Sigma_{\zeta(m)}$ for every $m\ge
n$, so $g_{Em}\eae\chi E$ for every $m\ge n$;  $H_{km}(E)\symmdiff E$ is negligible for $k\ge 1$, $m\ge n$;

\Centerline{$\tilde H_{km}(E)=\undphi_mE=\undphi_nE$}

\noindent for $k\ge 1$, $m\ge n$; and
$\undphi E=\undphi_nE$.    Thus $\undphi$ extends every $\undphi_n$.

\medskip

\quad{\bf (viii)} I have still to check the translation-invariance of
$\undphi$.   If $E\in\Sigma_{\xi}$ and $x\in X$, consider $g'_n$,
defined by setting

\Centerline{$g'_n(y)=g_{En}(y-x)$}

\noindent for every $y\in X$, $n\in\Bbb N$;  that is, $g'_n$ is the
composition $g_{En}\psi$, where $\psi(y)=y-x$ for $y\in X$.   (I am not
sure whether it is more, or less, confusing to distinguish between the
operations of addition and subtraction in $X$.   Of course
$y-x=y+(-x)=y+x$ for every $y$.)   Because $\psi$ is a measure space
automorphism, and in particular is \imp, we have

\Centerline{$\int_{F+x}g'_n
=\int_{\psi^{-1}[F]}g'_n=\int_Fg_{En}=\nu_{\kappa}(E\cap F)$}

\noindent whenever $F\in\Sigma_{\zeta(n)}$ (235Gc\formerly{2{}35I}).
But because
$\Sigma_{\zeta(n)}$ is itself translation-invariant, we can apply this
to $F-x$ to get

\Centerline{$\int_Fg'_n=\nu_{\kappa}(E\cap(F-x))=\nu_{\kappa}((E+x)\cap F)$}

\noindent for every $F\in\Sigma_{\zeta(n)}$.   Moreover, for any
$\alpha\in\Bbb R$,

\Centerline{$\{y:g'_n(y)\ge\alpha\}=\{y:g_{En}(y)\ge\alpha\}+x
\in\Sigma_{\zeta(n)}$}

\noindent for every $\alpha$, and $g'_n$ is
$\Sigma_{\zeta(n)}$-measurable.   So $g'_n$ is a conditional expectation
of $\chi(E+x)$ on $\Sigma_{\zeta(n)}$, and must be equal almost
everywhere to $g_{E+x,n}$.

This means that if we set

\Centerline{$H'_{kn}=\{y:g'_n(y)\ge 1-2^{-k}\}=H_{kn}(E)+x$}

\noindent for $k$, $n\in\Bbb N$, we shall have
$H'_{kn}\in\Sigma_{\zeta(n)}$ and $H'_{kn}\symmdiff H_{kn}(E+x)$ will be
negligible, so

$$\eqalign{\tilde H_{kn}(E+x)
&=\undphi_n(H_{kn}(E+x))
=\undphi_n(H'_{kn})\cr
&=\undphi_n(H_{kn}(E)+x)
=\undphi_n(H_{kn}(E)) + x
=\tilde H_{kn}(E)+x.\cr}$$

\noindent Consequently

$$\eqalign{\undphi(E+x)
&=\bigcap_{k\ge 1}\bigcup_{n\in\Bbb N}\bigcap_{m\ge n}
\tilde H_{kn}(E+x)\cr
&=\bigcap_{k\ge 1}\bigcup_{n\in\Bbb N}\bigcap_{m\ge n}
\tilde H_{kn}(E)+x
=\undphi E+x.\cr}$$

\noindent As $E$ and $x$ are arbitrary, $\undphi$ is
translation-invariant and belongs to $\Phi_{\xi}$.\ \Qed

\medskip

{\bf (d)} We are now ready for the proof that there is a
translation-invariant lower density for $\nu_{\kappa}$.   \Prf\ Build inductively a
family $\langle\undphi_{\xi}\rangle_{\xi\le\kappa}$ such that ($\alpha$)
$\undphi_{\xi}\in\Phi_{\xi}$ for each $\xi$ ($\beta$) $\undphi_{\xi}$
extends
$\undphi_{\eta}$ whenever $\eta\le\xi\le\kappa$.   The induction starts
with $\Sigma_0=\{\emptyset,X\}$, $\undphi_0\emptyset=\emptyset$,
$\undphi_0X=X$.   The inductive step to a successor ordinal is dealt
with in (b), and the inductive step to a non-zero ordinal of countable
cofinality is dealt with in (c).   If $\xi\le\kappa$ has uncountable
cofinality, then $\Sigma_{\xi}=\bigcup_{\eta<\xi}\Sigma_{\eta}$, so we
can (and must) take $\undphi_{\xi}$ to be the unique common extension of
all the previous $\undphi_{\eta}$.

The induction ends with
$\undphi_{\kappa}:\Sigma_{\kappa}\to\Tau_{\kappa}$.   Note that
$\Sigma_{\kappa}$ is not in general the whole of $\Tau_{\kappa}$.   But for
every $E\in\Tau_{\kappa}$ there is an $F\in\Sigma_{\kappa}$ such that
$E\symmdiff F$ is negligible (254Ff).   So we can extend
$\undphi_{\kappa}$ to a function $\undphi$ defined on the whole of
$\Tau_{\kappa}$ by setting

\Centerline{$\undphi E=\undphi_{\kappa}F$ whenever $E\in\Tau_{\kappa}$,
$F\in\Sigma_{\kappa}$ and $\nu_{\kappa}(E\symmdiff F)=0$}

\noindent (the point being that $\undphi_{\kappa}F=\undphi_{\kappa}F'$
if $F$, $F'\in\Sigma_{\kappa}$ and $\nu_{\kappa}(E\symmdiff
F)=\nu_{\kappa}(E\symmdiff F')=0$).   It is easy to check that $\undphi$ is a
lower density, and it is translation-invariant because if $E\in\Tau_{\kappa}$,
$x\in X$, $F\in\Sigma_{\kappa}$ and $E\symmdiff F$ is negligible, then
$(E+x)\symmdiff(F+x)=(E\symmdiff F)+x$ is negligible, so

\Centerline{$\undphi(E+x)=\undphi_{\kappa}(F+x)
=\undphi_{\kappa}F+x=\undphi E+x$.  \Qed}

\medskip

{\bf (e)} The rest of the argument is exactly that of parts (b)-(e) of
the proof of 345B;  you have to change $\BbbR^r$ into $X$ wherever it
appears, but otherwise you can use it word for word, interpreting
`$\tbf{0}$' as the identity of the group $X$, that is, the constant
function with value $0$.
}%end of proof of 345C

\leader{345D}{}\cmmnt{ Translation-invariant liftings are of great
importance, and I will return to them in \S447 with a theorem
dramatically generalizing the results above.   Here I shall content
myself with giving one of their basic properties, set out for the two
kinds of translation-invariant lifting we have seen.

\medskip

\noindent}{\bf Proposition} Let $(X,\Sigma,\mu)$ be {\it either}
Lebesgue measure on $\BbbR^r$ {\it or} the usual measure on $\{0,1\}^I$
for some set $I$, and let $\phi:\Sigma\to\Sigma$ be a
translation-invariant lifting.   Then for any open set $G\subseteq X$ we
must have $G\subseteq\phi G\subseteq\overline{G}$, and for any closed
set $F$ we must have $\interior F\subseteq\phi F\subseteq F$.

\proof{{\bf (a)} Suppose that $G\subseteq X$ is open and that $x\in G$.
Then there is an open set $U$ such that $\tbf{0}\in U$ and
$x+U-U=\{x+y-z:y,\,z\in U\}\subseteq G$.   \Prf\ ($\alpha$) If
$X=\Bbb R^r$, take $\delta>0$ such that
$\{y:\|y-x\|\le\delta\}\subseteq G$, and
set $U=\{y:\|y-x\|<\bover12\delta\}$.   ($\beta$) If $X=\{0,1\}^I$, then
there is a finite set $K\subseteq I$ such that
$\{y:y\restr K=x\restr K\}\subseteq G$ (3A3K);
set $U=\{y:y(i)=0$ for every $i\in K\}$.\ \Qed

It follows that $x\in\phi G$.   \Prf\ Consider $H=x+U$.   Then
$\mu H=\mu U>0$ so $H\cap\phi H\ne\emptyset$.   Let $y\in U$ be such that
$x+y\in\phi H$.   Then

\Centerline{$x=(x+y)-y\in\phi(H-y)\subseteq\phi G$}

\noindent because

\Centerline{$H-y\subseteq x+U-U\subseteq G$.  \Qed}

\medskip

{\bf (b)} Thus $G\subseteq\phi G$ for every open set $G\subseteq X$.
But it follows at once that if $G$ is open and $F$ is closed,

\Centerline{$\interior F\subseteq\phi(\interior F)\subseteq\phi F$,}

\Centerline{$\overline{G}=X\setminus\interior(X\setminus G)
\supseteq X\setminus\phi(X\setminus G)
=\phi G$,}

\Centerline{$F=X\setminus(X\setminus F)\supseteq
X\setminus\phi(X\setminus F)=\phi F$.}
}%end of proof of 345D

\leader{345E}{}\cmmnt{ I remarked in 341Lg that it is 
undecidable in
ordinary set theory whether there is a lifting for Borel measure on
$\Bbb R$.   It is however known that there can be no
translation-invariant Borel lifting.   The argument depends on the
following fact about measurable sets in $\{0,1\}^{\Bbb N}$.

\medskip

\noindent}{\bf Lemma}\dvArevised{2011}
Give $X=\{0,1\}^{\Bbb N}$ its usual measure
$\nu_{\Bbb N}$, and let $E\subseteq X$ be any non-negligible measurable
set.   Then there is an $n\in\Bbb N$ such that for every $k\ge n$ there are
$x$, $x'\in E$ which differ at $k$ and nowhere else.

\proof{ By 254Fe, there is a set $F$, determined by coordinates in a
finite set, such that
$\nu_{\Bbb N}(E\symmdiff F)\le\bover14\nu_{\Bbb N}E$;  we have
$\nu_{\Bbb N}F\ge\bover34\nu_{\Bbb N}E$, so
$\nu_{\Bbb N}(E\symmdiff F)\le\bover13\nu_{\Bbb N}F$.
Let $n\in\Bbb N$ be such that 
$F$ is determined by coordinates in $\{0,\ldots,n-1\}$.   Take any
$k\ge n$.   Then the map
$\psi:X\to X$, defined by setting $(\psi x)(k)=1-x(k)$,
$(\psi x)(i)=x(i)$ for $i\ne k$, is a measure space automorphism, and

\Centerline{$\nu_{\Bbb N}(\psi^{-1}[E\symmdiff F]\cup(E\symmdiff F))
\le 2\nu_{\Bbb N}(E\symmdiff F)<\nu_{\Bbb N}F$.}

\noindent Take any
$x\in F\setminus((E\symmdiff F)\cup\psi^{-1}[E\symmdiff F])$.   Then
$x'=\psi x$ differs from $x$ at $k$, and only there;  but also
$x'\in F$, by the choice of $n$, so both $x$ and $x'$ belong to $E$.
}%end of proof of 345E

\leader{345F}{\bf Proposition} Let $\mu$ be the restriction of
Lebesgue measure to the algebra $\Cal B$ of Borel subsets
of $\Bbb R$.   Then $\mu$ is translation-invariant, but has no
translation-invariant lifting.

\proof{{\bf (a)} To see that $\mu$ is translation-invariant all we
have to know is that $\Cal B$ is translation-invariant and that Lebesgue
measure is translation-invariant.   I have already cited 134A for the
proof that Lebesgue measure is invariant, and $\Cal B$ is invariant
because $G+x$ is open for every open set $G$ and every $x\in\Bbb R$.

\medskip

{\bf (b)} The argument below is most easily expressed in terms of the
geometry of the Cantor set $C$.
Recall that $C$ is defined as the intersection $\bigcap_{n\in\Bbb N}C_n$
of a sequence of closed subsets of $[0,1]$;  each $C_n$ consists of
$2^n$ closed intervals of length $3^{-n}$;  $C_{n+1}$ is obtained from
$C_n$ by deleting the middle third of each interval of $C_n$.   Any
point of $C$ is uniquely expressible as
$f(e)=\bover23\sum_{n=0}^{\infty}3^{-n}e(n)$ for some
$e\in\{0,1\}^{\Bbb N}$.   (See 134Gb.)
Let $\nu_{\Bbb N}$ be the usual measure of $\{0,1\}^{\Bbb N}$.
Because the map
$e\mapsto e(n):\{0,1\}^{\Bbb N}\to\{0,1\}$ is
measurable for each $n$, $f:\{0,1\}^{\Bbb N}\to\Bbb R$ is
measurable.

We can label the closed intervals constituting $C_n$ as
$\langle J_z\rangle_{z\in\{0,1\}^n}$, taking $J_{\emptyset}$ to be the
unit interval $[0,1]$ and, for $z\in\{0,1\}^n$, taking
$J_{z^{\smallfrown}\fraction{0}}$
to be the left-hand third of $J_z$ and
$J_{z^{\smallfrown}\fraction{1}}$ to be the
right-hand third of $J_z$.   (If the notation here seems odd to you,
there is an explanation in 3A1H.)

For $n\in\Bbb N$ and $z\in\{0,1\}^n$, let $J'_z$ be the open interval with
the same centre as $J_z$ and twice the length.   Then $J'_z\setminus
J_z$ consists of two open intervals of length $3^{-n}/2$ on either side
of $J_z$;  call the left-hand one $V_z$ and the right-hand one $W_z$.
Thus $V_{z^{\smallfrown}\fraction{1}}$ is the right-hand half of the middle third
of
$J_z$, and $W_{z^{\smallfrown}\fraction{0}}$ is the left-hand half of the middle
third of $J_z$.

Construct sets $G$, $H\subseteq\Bbb R$ as follows.

\inset{$G$ is to be the union of
the intervals $V_z$ where $z$ takes the value $1$ an even number of
times, together with the intervals $W_z$ where $z$ takes the value $0$
an odd number of times;}

\inset{$H$ is to be the union of the intervals $V_z$ where $z$ takes
the value $1$ an odd number of times, together with the intervals $W_z$
where $z$ takes the value $0$ an even number of times.}

\noindent $G$ and $H$ are open sets.   The intervals $V_z$, $W_z$
between them cover the whole of the interval
$\ooint{-\bover12,\bover32}$ with the exception of the set $C$ and the
countable set of midpoints of the intervals $J_z$;  so that
$\ooint{-\bover12,\bover32}\setminus(G\cup H)$ is negligible.   We have
to observe that $G\cap H=\emptyset$.   \Prf\ For each $z$,
$J'_{z^{\smallfrown}\fraction{0}}$ and $J'_{z^{\smallfrown}\fraction{1}}$ are disjoint subsets
of $J'_z$.   Consequently $J'_z\cap J'_{w}$ is non-empty just when one
of $z$, $w$ extends the other, and we need consider only the
intersections of the four sets $V_z$, $W_z$, $V_{w}$, $W_{w}$ when
$w$ is a proper extension of $z$;  say $w\in\{0,1\}^n$ and $z= w\restr
m$, where $m<n$.  ($\alpha$)
If in the extension $(w(m),\ldots,w(n-1))$ both values $0$ and $1$
appear, $J'_{w}$ will be a subset of $J_z$, and certainly the four sets
will all be disjoint.   ($\beta$) If $w(i)=0$ for $m\le i<n$, then
$W_{w}\subseteq J_z$ is disjoint from the rest, while $V_{w}\subseteq
V_z$;  but $z$ and $w$ take the value $1$ the same number of times, so
$V_{w}$ is assigned to $G$ iff $V_z$ is, and otherwise both are
assigned to $H$.   ($\gamma$) Similarly, if $w(i)=1$ for $m\le i<n$,
$V_{w}\subseteq J_z$, $W_{w}\subseteq W_z$ and $z$, $w$ take the
value $0$ the same number of times, so $W_z$ and $W_{w}$ are assigned
to the same set.\ \Qed

The following diagram may help you to see what is supposed to be
happening:

\def\Caption{}
\ifdim\pagewidth>467pt\picture{mt345f}{125pt}
\else\picture{mt345f}{104pt}\fi
%122pt made diagram a little indented on the right in \fullsize

\noindent The assignment rule can be restated as follows:

\inset{$V=V_{\emptyset}$ is assigned to $G$, $W=W_{\emptyset}$ is assigned to $H$;

$V_{z^{\smallfrown}\fraction{0}}$ is assigned to the same set as $V_z$, and
$V_{z^{\smallfrown}\fraction{1}}$ to the other;

$W_{z^{\smallfrown}\fraction{1}}$ is assigned to the same set as $W_z$, and
$W_{z^{\smallfrown}\fraction{0}}$ to the other.}

\medskip

{\bf (c)} Now take any $n\in\Bbb N$ and $z\in\{0,1\}^n$.   Consider the
two open intervals $I_0=J'_{z^{\smallfrown}\fraction{0}}$,
$I_1=J'_{z^{\smallfrown}\fraction{1}}$.   These are both of length
$\gamma=2\cdot 3^{-n-1}$ and abut at the centre of $J_z$, so $I_1$ is
just
the translate
$I_0+\gamma$.   I claim that $I_1\cap H=(I_0\cap G)+\gamma$.   \Prf\ Let
$A$ be the set

\Centerline{$\bigcup_{m>n}\{w:w\in\{0,1\}^m,\,w$ extends
$z^{\smallfrown}\fraction{0}\}$,}

\noindent and for $w\in A$ let $w'$ be the finite sequence obtained from
$w$ by changing $w(n)=0$ into $w'(n)=1$ but leaving the other values of
$w$ unaltered.   Then
$V_{w'}=V_w+\gamma$ and $W_{w'}=W_w+\gamma$ for every $w\in A$.   Now

$$\eqalign{I_0\cap G
&=\bigcup\{V_w:w\in A,\,
w\text{ takes the value }1\text{ an even number of times}\}\cr
&\qquad\cup\bigcup\{W_w:w\in A,\,
w\text{ takes the value }0\text{ an odd number of times}\},\cr}$$

\noindent so

$$\eqalign{(I_0\cap G)+\gamma
&=\bigcup\{V_{w'}:w\in A,\,
w\text{ takes the value }1\text{ an even number of times}\}\cr
&\qquad\cup\bigcup\{W_{w'}:w\in A,\,
w\text{ takes the value }0\text{ an odd number of times}\}\cr
&=\bigcup\{V_{w'}:w\in A,\,
w'\text{ takes the value }1\text{ an odd number of times}\}\cr
&\qquad\cup\bigcup\{W_{w'}:w\in A,\,
w'\text{ takes the value }0\text{ an even number of times}\}\cr
&=I_1\cap H.  \text{  \Qed}\cr}$$

\medskip

{\bf (d)} \Quer\ Now suppose, if possible, that $\phi:\Cal B\to\Cal B$
is a translation-invariant lifting.   Note first that $U\subseteq\phi U$
for every open $U\subseteq\Bbb R$.   \Prf\  The argument is exactly that
of 345D as applied to $\Bbb R=\BbbR^1$.\ \QeD\ Consequently

\Centerline{$J'_{\emptyset}=\ooint{-\bover12,\bover32}\subseteq\phi
J'_{\emptyset}$.}

\noindent But as $J'_{\emptyset}\setminus(G\cup H)$ is negligible,

\Centerline{$C\subseteq\ooint{-\bover12,\bover32}\subseteq\phi G\cup\phi
H$.}

\noindent Consider the sets $E=f^{-1}[\phi G]$, $F=\{0,1\}^{\Bbb
N}\setminus E=f^{-1}[\phi H]$.   Because $f$ is measurable and
$\phi G$, $\phi H$ are Borel sets, $E$ and $F$ are measurable subsets of
$\{0,1\}^{\Bbb N}$,
and at least one of them has positive measure for $\nu_{\Bbb N}$.
There must therefore be
$e$, $e'\in\{0,1\}^{\Bbb N}$, differing at exactly one coordinate, such
that either both belong to $E$ or both belong to $F$ (345E).   Let us
suppose that $n$ is such that $e(n)=0$, $e'(n)=1$ and $e(i)=e'(i)$ for
$i\ne n$.   Set $z=e\restr n=e'\restr n$.
Then $f(e)$ belongs to the open interval
$I_0=J'_{z^{\smallfrown}\fraction{0}}$, so
$f(e)\in\phi I_0$ and $f(e)\in\phi G$ iff $f(e)\in\phi(I_0\cap G)$.
But now

\Centerline{$f(e')=f(e)+2\cdot 3^{-n-1}
\in I_1=J'_{z^{\smallfrown}\fraction{1}}$,}

\noindent so

$$\eqalignno{e\in E
&\iff f(e)\in\phi G
\iff f(e)\in\phi(I_0\cap G)\cr
&\iff f(e')\in\phi((I_0\cap G)+2\cdot 3^{-n-1})\cr
\noalign{\noindent (because $\phi$ is translation-invariant)}
&\iff f(e')\in\phi(I_1\cap H)\cr
\noalign{\noindent (by (c) above)}
&\iff f(e')\in\phi H\cr
\noalign{\noindent (because $f(e')\in I_1\subseteq\phi I_1$)}
&\iff e'\in F.\cr}$$

\noindent But this contradicts the choice of $e$.   \Bang

Thus there is no translation-invariant lifting for $\mu$.
}%end of proof of 345F

\cmmnt{\medskip

\noindent{\bf Remark} This result is due to {\smc Johnson 80};  the
proof here follows {\smc Talagrand 82b}.   For references to various
generalizations see {\smc Burke 93}, $\S$3.
}

\exercises{
\leader{345X}{Basic exercises (a)} In 345Ab I wrote `It follows at once
that the map $y\mapsto y+x:X\to X$ is a measure space automorphism'.
Write the details out in full, using 254G or otherwise.
%345A

\spheader 345Xb Let $S^1$ be the unit circle in $\BbbR^2$, and let
$\mu$ be one-dimensional Hausdorff measure on $S^1$ (\S\S264-265).
Show that $\mu$ is translation-invariant, if $S^1$ is given its
usual group operation corresponding to complex multiplication (255M),
and that it has a translation-invariant lifting $\phi$.
\Hint{Identifying $S^1$ with $\ocint{-\pi,\pi}$ with the group operation
$+_{2\pi}$, show that we can set
$\phi E=\ocint{-\pi,\pi}\cap\phi'(\bigcup_{n\in\Bbb Z}E+2\pi n)$, where
$\phi'$ is any translation-invariant lifting for Lebesgue measure.}
%345B

\sqheader 345Xc   Show that there is no lifting $\phi$ of Lebesgue
measure on $\Bbb R$ which is `symmetric' in the sense that
$\phi(-E)=-\phi E$ for every measurable set $E$, writing
$-E=\{-x:x\in E\}$.   \Hint{can $0$ belong to $\phi(\coint{0,\infty})$?}
%345B

\sqheader 345Xd Let $\mu$ be Lebesgue measure on
$X=\Bbb R\setminus\{0\}$.   Show that there is a lifting $\phi$ of $\mu$
such that $\phi(xE)=x\phi E$ for every $x\in X$ and every measurable
$E\subseteq X$, writing $xE=\{xy:y\in E\}$.
%345B

\spheader 345Xe Let $\nu_I$ be the usual measure on $X=\{0,1\}^I$, for
some set $I$, $\Tau_I$ its domain, and $(\frak B_I,\bar\nu_I)$ its measure
algebra.   (i) Show that we can define $\pi_x(a)=a+x$, for $a\in\frak B_I$
and $x\in X$, by the formula $E^{\ssbullet}+x=(E+x)^{\ssbullet}$;  and
that $x\mapsto\pi_x$ is a group homomorphism from $X$ to the group of
measure-preserving automorphisms of $\frak A$.   (ii) Define
$\Sigma_{\xi}$ as in the proof of 345C, and set
$\frak A_{\xi}=\{E^{\ssbullet}:E\in\Sigma_{\xi}\}$.   Say that a partial
lifting $\undtheta:\frak A_{\xi}\to\Tau_I$ is
translation-invariant if $\undtheta(a+x)=\undtheta a+x$ for every
$a\in\frak A_{\xi}$ and $x\in X$.   Show that any such partial lifting
can be extended to a translation-invariant partial lifting on
$\frak A_{\xi+1}$.   (iii) Write out a proof of 345C in the language of
341F-341H.  %341F 341G 341H
%345C

\sqheader 345Xf Let $\undphi$ be a lower density for Lebesgue
measure on $\BbbR^r$ which is translation-invariant in the sense that
$\undphi(E+x)=\undphi E+x$ for every $x\in\BbbR^r$ and every measurable
set $E$.   Show that $\undphi G\supseteq G$ for every open set
$G\subseteq\BbbR^r$.
%345D

\spheader 345Xg Let $\mu$ be 1-dimensional Hausdorff measure on $S^1$, as in 345Xb.   Show that there is no translation-invariant lifting $\phi$ of $\mu$ such that $\phi E$ is a Borel set for every $E\in\dom\mu$.
%345F

\leader{345Y}{Further exercises (a)}  Let $(X,\Sigma,\mu)$ be a
complete measure space, and suppose that $X$ has a group operation
$(x,y)\mapsto xy$ (not necessarily abelian!) such that $\mu$ is
left-translation-invariant, in the sense that
$xE=\{xy:y\in E\}\in\Sigma$ and $\mu(xE)=\mu E$ whenever $E\in\Sigma$
and $x\in X$.   Suppose that $\undphi:\Sigma\to\Sigma$ is a lower
density which is
left-translation-invariant in the sense that $\undphi(xE)=x(\undphi E)$
for every $E\in\Sigma$ and $x\in X$.   Show that there is a
left-translation-invariant lifting $\phi:\Sigma\to\Sigma$ such that
$\undphi E\subseteq\phi E$ for every $E\in\Sigma$.
%345C

\spheader 345Yb Write $\Sigma$ for the $\sigma$-algebra of Lebesgue
measurable subsets of $\Bbb R$, and $\eusm L^0(\Sigma)$ for the linear
space of $\Sigma$-measurable functions from $\Bbb R$ to itself.   Show
that there is a linear operator
$T:L^0(\mu)\to\eusm L^0(\Sigma)$ such that ($\alpha$)
$(Tu)^{\ssbullet}=u$ for every $u\in L^0(\mu)$ ($\beta$)
$\sup_{x\in\Bbb R}|(Tu)(x)|=\|u\|_{\infty}$ for every
$u\in L^{\infty}(\mu)$ ($\gamma$) $Tu\ge 0$ whenever
$u\in L^{\infty}(\mu)$ and $u\ge 0$ ($\delta$) $T$ is
translation-invariant in the sense that
$T(S_xf)^{\ssbullet}=S_xTf^{\ssbullet}$ for every $x\in\Bbb R$ and
$f\in\eusm L^0(\Sigma)$, where $(S_xf)(y)=f(x+y)$ for
$f\in\eusm L^0(\Sigma)$ and $x$, $y\in\Bbb R$ ($\epsilon$) $T$ is
reflection-invariant in the sense that
$T(Rf)^{\ssbullet}=RTf^{\ssbullet}$ for every $f\in\eusm L^0(\Sigma)$,
where $(Rf)(x)=f(-x)$ for $f\in\eusm L^0(\Sigma)$ and $x\in\Bbb R$.
({\it Hint\/}:  for $f\in\eusm L^0(\Sigma)$, set

\Centerline{$p(f^{\ssbullet})
=\inf\{\alpha:\alpha\in[0,\infty],\,\lim_{\delta\downarrow 0}
  \Bover1{2\delta}\mu\{x:|x|\le\delta,\,|f(x)|>\alpha\}=0\}$.}

\noindent Set $V=\{u:u\in L^0(\mu),\,p(u)<\infty\}$ and show that $V$ is
a linear subspace of $L^0(\mu)$ and that $p\restr V$ is a seminorm.
Let $h_0:V\to\Bbb R$ be a linear functional such that
$h_0(\chi\Bbb R)^{\ssbullet}=1$ and $h_0(u)\le p(u)$ for every $u\in V$.
Extend $h_0$ arbitrarily to a linear functional $h_1:L^0(\mu)\to\Bbb R$;
set $h(f^{\ssbullet})=\bover12(h_1(f^{\ssbullet})+h_1(Rf)^{\ssbullet})$.
Set $(Tf^{\ssbullet})(x)=h(S_{-x}f)^{\ssbullet}$.   You will need 223C.)
Show that
there must be a $u\in L^1(\mu)$ such that $u\ge 0$ but $Tu\not\ge 0$.
%341Xg

\spheader 345Yc Show that there is no translation-invariant lifting
$\phi$ of the usual measure on $\{0,1\}^{\Bbb N}$ such that $\phi E$ is a
Borel set for every measurable set $E$.
%345F  mt34bits
}%end of exercises

\cmmnt{
\Notesheader{345} I have taken a great deal of care over the concept of
`translation-invariance'.   I hope that you are already a little
impatient with some of the details as I have written them out;  but
while it is very easy to guess at the structure of such arguments as
part (e) of
the proof of 345B, or (b-iii) and (c-viii) in the proof of 345C, I am
not sure that one can always be certain of guessing correctly.   A fair
test of your intuition will be how quickly you can generate the formulae
appropriate to a non-abelian group operation, as in 345Ya.

Part (b) of the proof of 345C is based on the same idea as the proof
of 341F.   There is a useful simplification because the set $E_{\xi}$ in
345C, corresponding to the set $E$ of the proof of 341F, is independent
of the algebra $\Sigma_{\xi}$ in a very strong sense, so that the
expression of an element of $\Sigma_{\xi+1}$ in the form
$(F\cap E_{\xi})\cup(G\setminus E_{\xi})$ is unique.   Interpreted in
the terms of 341F, we have $w=v=1$, so that the formula

\Centerline{$\undtheta_1((a\Bcap e)\Bcup(b\Bsetminus e))
=\bigl(\undtheta((a\Bcap v)\Bcup(b\Bsetminus v))\cap E\bigr)
   \cup\bigl(\undtheta((a\Bsetminus w)\Bcup(b\Bcap w))
     \setminus E\bigr)$}

\noindent used there becomes

\Centerline{$\undtheta_1((a\Bcap e)\Bcup(b\Bsetminus e))
=(\undtheta a\cap E)\cup(\undtheta b\setminus E)$,}

\noindent matching the formula for $\undphi_1$ in the proof of 345C.

The results of this section are satisfying and natural;  they have
obvious generalizations, many of which are true.   The most important
measure spaces come equipped with a variety of automorphisms, and we can
always ask which of these can be preserved by a lifting.   The answers
are not always obvious;  I offer 345Xc and 346Xc as warnings, and 345Xd
as an encouragement.   345Yb is striking (I have made it as striking as
I can), but slightly off the most natural target;  the sting is in the
last sentence (see 341Xg).
}%end of comment

\discrpage

