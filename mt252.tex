\frfilename{mt252.tex}
\versiondate{6.12.07}
\copyrightdate{2000}

\def\displaycause#1{\noalign{\noindent (#1)}}

\def\chaptername{Product measures}
\def\sectionname{Fubini's theorem}

\newsection{252}

Perhaps the most important feature of the concept of `product
measure' is the fact that we can use it to discuss repeated integrals.
In this section I give versions of Fubini's theorem and Tonelli's
theorem (252B, 252G) with a variety of corollaries, the most useful ones
being versions for $\sigma$-finite spaces (252C, 252H).   As
applications I describe the relationship between integration and
measuring ordinate sets (252N) and calculate
the $r$-dimensional volume of a ball in $\BbbR^r$
(252Q\cmmnt{, 252Xi}).
I mention counter-examples showing the difficulties which can arise with
non-$\sigma$-finite measures and non-integrable functions
(252K-252L\cmmnt{, 252Xf-252Xg}).

\leader{252A}{Repeated integrals}  Let $(X,\Sigma,\mu)$ and
$(Y,\Tau,\nu)$ be
measure spaces, and $f$ a real-valued function defined on a set
$\dom f\subseteq X\times Y$.   We can seek to form the {\bf repeated
integral}

\Centerline{$\iint f(x,y)\nu(dy)\mu(dx)
=\int\bigl(\int f(x,y)\nu(dy)\bigr)\mu(dx)$,}

\noindent which should be interpreted as follows:  set

\Centerline{$D=\{x:x\in X,\,\int f(x,y)\nu(dy)$ is defined in
$[-\infty,\infty]\}$,}

\Centerline{$g(x)=\int f(x,y)\nu(dy)$ for $x\in D$,}

\noindent and then write $\iint f(x,y)\nu(dy)\mu(dx)=\int g(x)\mu(dx)$
if this is defined.   \cmmnt{Of course the subset of $Y$ on which
$y\mapsto f(x,y)$ is defined may vary with $x$, but it must always be
conegligible, as must $D$.

Similarly, exchanging the roles of $X$ and $Y$, we can seek
a repeated integral

\Centerline{$\iint f(x,y)\mu(dx)\nu(dy)
=\int\bigl(\int f(x,y)\mu(dx)\bigr)\nu(dy)$.}

\noindent The point is that, under appropriate conditions on $\mu$ and
$\nu$, we can relate these repeated integrals to each other
by connecting them both with the integral of $f$ itself with respect to
the product measure on $X\times Y$.

As will become apparent shortly, it is essential here to allow oneself
to discuss the integral of a function which is not everywhere defined.   It
is of less importance whether one allows integrands and integrals to
take infinite values, but for definiteness let me say that I shall be
following the rules of 135F;  that is, $\int f=\int f^+-\int f^-$
provided that $f$ is defined almost everywhere, takes values in
$[-\infty,\infty]$ and is virtually measurable, and at most one of
$\int f^+$, $\int f^-$ is infinite.
}%end of comment

\leader{252B}{Theorem} Let $(X,\Sigma,\mu)$ and $(Y,\Tau,\nu)$ be
measure spaces, with c.l.d.\ product
$(X\times Y,\Lambda,\lambda)$\cmmnt{ (251F)}.   Suppose that $\nu$ is
$\sigma$-finite and that $\mu$ is {\it either} strictly localizable
{\it or} complete and locally determined.   Let $f$ be a
$[-\infty,\infty]$-valued function such that $\int fd\lambda$ is defined
in $[-\infty,\infty]$.   Then $\iint f(x,y)\nu(dy)\mu(dx)$ is defined
and is equal to $\int fd\lambda$.

\proof{ The proof of this result involves substantial technical
difficulties.   If you have not seen these ideas before, you should
almost certainly not go straight to the full generality of the version
announced above.   I will therefore start by writing out a proof in the
case in which both $\mu$ and $\nu$ are totally finite;  this is already
lengthy enough.   I will present it in such a way that only the central
section (part (b) below) needs to be amended in the general case, and
then, after completing the proof of the special case, I will give the
alternative version of (b) which is required for the full result.

\medskip

{\bf (a)} Write $\eusm L$ for the family of $[0,\infty]$-valued
functions $f$ such that $\int fd\lambda$ and
\discrversionA{\break}{}$\iint f(x,y)\nu(dy)\mu(dx)$ are defined and
equal.   My
aim is to show first that $f\in\eusm L$ whenever $f$ is non-negative and
$\int fd\lambda$ is defined, and then to look at differences of
functions in $\eusm L$.   To prove that enough functions belong to
$\eusm L$, my strategy will be to start with `elementary' functions and
work outwards through progressively larger classes.   It is most
efficient to begin by describing ways of building new members of $\eusm
L$ from old, as follows.

\medskip

\quad{\bf (i)} $f_1+f_2\in\eusm L$ for all $f_1$, $f_2\in\eusm L$, and
$cf\in\eusm L$ for all $f\in\eusm L$, $c\in\coint{0,\infty}$;  this is
because

\Centerline{$\int(f_1+f_2)(x,y)\nu(dy)
=\int f_1(x,y)\nu(dy)+\int f_2(x,y)\nu(dy)$,}

\Centerline{$\int (cf)(x,y)\nu(dy)=c\int f(x,y)\nu(dy)$}

\noindent whenever the right-hand sides are defined, which we are
supposing to be the case for almost every $x$, so that

$$\eqalign{\iint(f_1+f_2)(x,y)\nu(dy)\mu(dx)
&=\iint f_1(x,y)\nu(dy)\mu(dx)+\iint f_2(x,y)\nu(dy)\mu(dx)\cr
&=\int f_1d\lambda+\int f_2d\lambda
=\int(f_1+f_2)d\lambda,\cr}$$

$$\iint(cf)(x,y)\nu(dy)\mu(dx)
=c\int f(x,y)\nu(dy)\mu(dx)
=c\int fd\lambda
=\int(cf)d\lambda.$$

\medskip

\quad{\bf (ii)} If $\sequencen{f_n}$ is a sequence in $\eusm L$ such
that $f_n(x,y)\le f_{n+1}(x,y)$ whenever $n\in\Bbb N$ and
$(x,y)\in\dom f_n\cap\dom f_{n+1}$, then $\sup_{n\in\Bbb N}f_n\in\eusm
L$.   \Prf\
Set $f=\sup_{n\in\Bbb N}f_n$;  for $x\in X$, $n\in\Bbb N$ set
$g_n(x)=\int f_n(x,y)\nu(dy)$ when the integral is defined in
$[0,\infty]$.   Since here I am allowing $\infty$ as a value of a
function, it is natural to regard $f$ as defined on $\bigcap_{n\in\Bbb
N}\dom f_n$.   By
B.Levi's theorem, $\int fd\lambda=\sup_{n\in\Bbb N}\int f_nd\lambda$;
write $u$ for this common value in $[0,\infty]$.   Next,
because $f_n\le f_{n+1}$ wherever both are defined, $g_n\le g_{n+1}$
wherever both are defined, for each $n$;   we are supposing that
$f_n\in\eusm L$, so $g_n$ is defined $\mu$-almost everywhere for each
$n$, and

\Centerline{$\sup_{n\in\Bbb N}\int g_nd\mu
=\sup_{n\in\Bbb N}\int f_nd\lambda=u$.}

\noindent By B.Levi's theorem again, $\int g\,d\mu=u$, where
$g=\sup_{n\in\Bbb N}g_n$.   Now take any $x\in\bigcap_{n\in\Bbb N}\dom
g_n$, and consider the functions $f_{xn}$ on $Y$, setting
$f_{xn}(y)=f_n(x,y)$ whenever this is defined.   Each $f_{xn}$ has an
integral in $[0,\infty]$, and $f_{xn}(y)\le f_{x,n+1}(y)$ whenever both
are defined, and

\Centerline{$\sup_{n\in\Bbb N}\int f_{xn}d\nu=g(x)$;}

\noindent so, using B.Levi's theorem for a third time,
$\int(\sup_{n\in\Bbb
N}f_{xn})d\nu$ is defined and equal to $g(x)$, that is,

\Centerline{$\int f(x,y)\nu(dy)=g(x)$.}

\noindent This is true for almost every $x$, so

\Centerline{$\iint f(x,y)\nu(dy)\mu(dx)
=\int g\,d\mu=u=\int fd\lambda$.}

\noindent Thus $f\in\eusm L$, as claimed.\ \Qed

\medskip

\quad{\bf (iii)} The expression of the ideas in the next section of the
proof will go more smoothly if I introduce another term.   Write
$\Cal W$ for $\{W:W\subseteq X\times Y,\,\chi W\in\eusm L\}$.   Then

\inset{($\alpha$) if $W$, $W'\in\Cal W$ and $W\cap W'=\emptyset$,
$W\cup W'\in\Cal W$}

\noindent by (i), because $\chi(W\cup W')=\chi W+\chi W'$,

\inset{($\beta$) $\bigcup_{n\in\Bbb N}W_n\in\Cal W$ whenever
$\sequencen{W_n}$ is a non-decreasing sequence in $\Cal W$}

\noindent because $\sequencen{\chi W_n}\uparrow\chi W$, and we can use
(ii).

It is also helpful to note that, for any $W\subseteq X\times Y$ and any
$x\in X$, $\int\chi W(x,y)\nu(dy)=\nu W[\{x\}]$, at least whenever
$W[\{x\}]=\{y:(x,y)\in W\}$ is measured by $\nu$.   Moreover, because
$\lambda$ is complete, a set $W\subseteq X\times Y$ belongs to $\Lambda$
iff $\chi W$ is $\lambda$-virtually measurable iff
$\int\chi W\,d\lambda$ is defined in $[0,\infty]$, and in this case
$\lambda W=\int\chi W\,d\lambda$.

\medskip

\quad{\bf (iv)} Finally, we need to observe that, in appropriate
circumstances, the difference of two members of $\Cal W$ will belong to
$\Cal W$:  if $W$, $W'\in\Cal W$ and $W\subseteq W'$ and $\lambda
W'<\infty$, then $W'\setminus W\in\Cal W$.   \Prf\ We are supposing that
$g(x)=\int\chi W(x,y)\nu(dy)$ and $g'(x)=\int\chi W'(x,y)\nu(dy)$ are
defined for almost every $x$, and that $\int g\,d\mu=\lambda W$, $\int
g'd\mu=\lambda W'$.   Because $\lambda W'$ is finite, $g'$ must be
finite almost everywhere, and $D=\{x:x\in\dom g\cap\dom
g',\,g'(x)<\infty\}$ is conegligible.   Now, for any $x\in D$, both
$g(x)$ and $g'(x)$ are finite, so

\Centerline{$y\mapsto\chi(W'\setminus W)(x,y)=\chi W'(x,y)-\chi W(x,y)$}

\noindent is the difference of two integrable functions, and

$$\eqalign{\int\chi(W'\setminus W)(x,y)\nu(dy)
&=\int\chi W'(x,y)-\chi W(x,y)\nu(dy)\cr
&=\int\chi W'(x,y)\nu(dy)-\int\chi W(x,y)\nu(dy)
=g'(x)-g(x).\cr}$$

\noindent Accordingly

\Centerline{$\iint\chi(W'\setminus W)(x,y)\nu(dy)\mu(dx)
=\int g'(x)-g(x)\mu(dx)
=\lambda W'-\lambda W
=\lambda(W'\setminus W)$,}

\noindent and $W'\setminus W$ belongs to $\Cal W$.\ \Qed

(Of course the argument just above can be shortened by a few words if we
allow ourselves to assume that $\mu$ and $\nu$ are totally finite,
since then $g(x)$ and $g'(x)$ will be finite whenever they are defined;
but the key idea, that the difference of integrable functions is
integrable, is unchanged.)

\medskip

{\bf (b)} Now let us examine the class $\Cal W$, assuming that $\mu$ and
$\nu$ are totally finite.

\medskip

\quad{\bf (i)} $E\times F\in\Cal W$ for all $E\in\Sigma$, $F\in\Tau$.
\Prf\ $\lambda(E\times F)=\mu E\cdot\nu F$ (251J),  and

\Centerline{$\int\chi(E\times F)(x,y)\nu(dy)=\nu F\,\chi E(x)$}

\noindent for each $x$, so

$$\eqalign{\iint\chi(E\times F)(x,y)\nu(dy)\mu(dx)
&=\int(\nu F\,\chi E(x))\mu(dx)
=\mu E\cdot\nu F\cr
&=\lambda(E\times F)
=\int\chi(E\times F)d\lambda. \text{ \Qed}\cr}$$

\medskip

\quad{\bf (ii)} Let $\Cal C$ be $\{E\times F:E\in\Sigma,\,F\in\Tau\}$.
Then $\Cal C$ is closed under finite intersections (because
$(E\times F)\cap(E'\times F')=(E\cap E')\times(F\cap F')$)
and is included in
$\Cal W$.   In particular, $X\times Y\in\Cal W$.   But this, together
with (a-iv) and (a-iii-$\beta$) above, means that $\Cal W$ is a Dynkin
class (definition:  136A), so includes the $\sigma$-algebra of subsets
of $X\times Y$
generated by $\Cal C$, by the Monotone Class Theorem (136B);  that is,
$\Cal W\supseteq\Sigma\tensorhat\Tau$ (definition:  251D).

\medskip

\quad{\bf (iii)} Next, $W\in\Cal W$ whenever $W\subseteq X\times Y$ is
$\lambda$-negligible.   \Prf\ By 251Ib, there is a
$V\in\Sigma\tensorhat\Tau$ such that $V\subseteq(X\times Y)\setminus W$
and $\lambda V=\lambda((X\times Y)\setminus W)$.   Because
$\lambda(X\times Y)=\mu X\cdot\nu Y$ is finite, $V'=(X\times Y)\setminus
V$ is $\lambda$-negligible, and we have $W\subseteq
V'\in\Sigma\tensorhat\Tau$.   Consequently

\Centerline{$0=\lambda V'=\iint\chi V'(x,y)\nu(dy)\mu(dx)$.}

\noindent But this means that

\Centerline{$D=\{x:\int\chi V'(x,y)\nu(dy)$ is defined and equal
to $0\}$}

\noindent is conegligible.   If $x\in D$, then we must have
$\chi V'(x,y)=0$ for $\nu$-almost every $y$, that is, $V'[\{x\}]$ is
negligible;  in which case $W[\{x\}]\subseteq V'[\{x\}]$ also is
negligible, and $\int\chi W(x,y)\nu(dy)=0$.   And this is true for every
$x\in D$, so $\int\chi W(x,y)\nu(dy)$ is defined and equal to $0$ for
almost every $x$, and

\Centerline{$\iint\chi W(x,y)\nu(dy)\mu(dx)=0=\lambda W$,}

\noindent as required.\ \Qed

\medskip

\quad{\bf (iv)} It follows that $\Lambda\subseteq\Cal W$.   \Prf\ If
$W\in\Lambda$, then, by 251Ib again, there is a
$V\in\Sigma\tensorhat\Tau$ such that $V\subseteq W$ and
$\lambda V=\lambda W$, so that $\lambda(W\setminus V)=0$.   Now
$V\in\Cal W$ by
(ii) and $W\setminus V\in\Cal W$ by (iii), so $W\in\Cal W$ by
(a-iii-$\alpha$).\ \Qed

\medskip

{\bf (c)} I return to the class $\eusm L$.

\medskip

\quad{\bf (i)} If $f\in\eusm L$ and $g$ is a $[0,\infty]$-valued
function defined and equal to $f\,\,\lambda$-a.e., then $g\in\eusm L$.
\Prf\ Set

\Centerline{$W=(X\times Y)\setminus\{(x,y):(x,y)\in\dom f\cap\dom g$,
$f(x,y)=g(x,y)\}$,}

\noindent so that $\lambda W=0$.   (Remember that $\lambda$ is
complete.)   By (b), $\iint\chi W(x,y)\nu(dy)\mu(dx)=0$, that is,
$W[\{x\}]$ is $\nu$-negligible for $\mu$-almost every $x$.   Let $D$ be
$\{x:x\in X,\,W[\{x\}]$ is $\nu$-negligible$\}$.   Then $D$ is
$\mu$-conegligible.   If $x\in D$, then

\Centerline{$W[\{x\}]=Y\setminus\{y:(x,y)\in\dom f\cap\dom g$,
$f(x,y)=g(x,y)\}$}

\noindent is negligible, so that $\int f(x,y)\nu(dy)=\int g(x,y)\nu(dy)$
if either is defined.   Thus the functions
\discrcenter{468pt}{$x\mapsto\int f(x,y)\nu(dy)$,
\ifdim\pagewidth>467pt\quad\fi
$x\mapsto\int g(x,y)\nu(dy)$ }are equal almost everywhere, and

\Centerline{$\iint g(x,y)\nu(dy)\mu(dx)=\iint
f(x,y)\nu(dy)\mu(dx)
=\int fd\lambda=\int g\,d\lambda$,}

\noindent so that $g\in\eusm L$.\ \Qed

\medskip

\quad{\bf (ii)} Now let $f$ be any non-negative function such that
$\int fd\lambda$ is defined in $[0,\infty]$.  Then $f\in\eusm L$.
\Prf\ For $k$, $n\in\Bbb N$ set

\Centerline{$W_{nk}=\{(x,y):(x,y)\in\dom f,\,f(x,y)\ge 2^{-n}k\}$.}

\noindent Because $\lambda$ is complete and $f$ is $\lambda$-virtually
measurable and $\dom f$ is conegligible, every $W_{nk}$ belongs to
$\Lambda$, so $\chi W_{nk}\in\eusm L$, by (b).   Set
$f_n=\sum_{k=1}^{4^n}2^{-n}\chi W_{nk}$, so that

$$\eqalign{f_n(x,y)
&=2^{-n}k\text{ if }k\le 4^n\text{ and }2^{-n}k
                   \le f(x,y)<2^{-n}(k+1),\cr
&=2^n\text{ if }f(x,y)\ge 2^n,\cr
&=0\text{ if }(x,y)\in(X\times Y)\setminus\dom f.\cr}$$

\noindent By (a-i), $f_n\in\eusm L$ for every $n\in\Bbb N$, while
$\sequencen{f_n}$ is non-decreasing, so $f'=\sup_{n\in\Bbb N}f_n\in\eusm
L$, by (a-ii).   But $f\eae f'$, so $f\in\eusm L$, by (i) just above.\
\Qed

\medskip

\quad{\bf (iii)} Finally, let $f$ be any $[-\infty,\infty]$-valued
function such that $\int fd\lambda$ is defined in $[-\infty,\infty]$.
Then $\int f^+d\lambda$, $\int f^-d\lambda$ are both defined and at most
one is infinite.   By (ii), both $f^+$ and $f^-$ belong to $\eusm L$.
Set $g(x)=\int f^+(x,y)\nu(dy)$, $h(x)=\int f^-(x,y)\nu(dy)$ whenever
these are defined;  then $\int g\,d\mu=\int f^+d\lambda$ and $\int
h\,d\mu=\int f^-d\lambda$ are both defined in $[0,\infty]$.

Suppose first that $\int f^-d\lambda$ is finite.   Then $\int h\,d\mu$
is finite, so $h$ must be finite $\mu$-almost everywhere;  set

\Centerline{$D=\{x:x\in\dom g\cap\dom h,\,h(x)<\infty\}$.}

\noindent For any $x\in D$, $\int f^+(x,y)\nu(dy)$ and
$\int f^-(x,y)\nu(dy)$ are defined in $[0,\infty]$, and the latter is
finite;  so

\Centerline{$\int f(x,y)\nu(dy)
=\int f^+(x,y)\nu(dy)-\int f^-(x,y)\nu(dy)
=g(x)-h(x)$}

\noindent is defined in $\ocint{-\infty,\infty}$.   Because $D$ is
conegligible,

$$\eqalign{\iint f(x,y)\nu(dy)\mu(dx)
&=\int g(x)-h(x)\mu(dx)
=\int g\,d\mu-\int h\,d\mu\cr
&=\int f^+d\lambda-\int f^-d\lambda
=\int fd\lambda,\cr}$$

\noindent as required.

Thus we have the result when $\int f^-d\lambda$ is finite.   Similarly,
or by applying the argument above to $-f$, we see that
$\iint f(x,y)\nu(dy)\mu(dx)=\int fd\lambda$
if $\int f^+d\lambda$ is finite.

Thus the theorem is proved, at least when $\mu$ and $\nu$ are totally
finite.

\medskip

{\bf (b*)} The only point in the argument above where we needed to know
anything special about the measures $\mu$ and $\nu$ was in part (b),
when showing that $\Lambda\subseteq\Cal W$.   I now return to this point
under the hypotheses of the theorem as stated, that $\nu$ is
$\sigma$-finite and $\mu$ is either strictly localizable or complete and
locally determined.

\medskip

\quad{\bf (i)} It will be helpful to note that the completion $\hat\mu$
of $\mu$ (212C) is identical with its c.l.d.\ version $\tilde\mu$
(213E).   \Prf\ If $\mu$ is strictly localizable, then
$\hat\mu=\tilde\mu$ by 213Ha.   If $\mu$ is complete and locally
determined, then $\hat\mu=\mu=\tilde\mu$ (212D, 213Hf).\ \Qed

\medskip

\quad{\bf (ii)}  Write $\Sigma^f=\{G:G\in\Sigma,\,\mu G<\infty\}$,
$\Tau^f=\{H:H\in\Tau,\,\nu H<\infty\}$.   For $G\in\Sigma^f$,
$H\in\Tau^f$ let $\mu_G$, $\nu_H$ and $\lambda_{G\times H}$ be the
subspace measures on $G$, $H$ and $G\times H$ respectively;  then
$\lambda_{G\times H}$ is the c.l.d.\ product measure of $\mu_G$ and
$\nu_H$ (251Q(ii-$\alpha$)).   Now $W\cap(G\times H)\in\Cal W$ for every
$W\in\Lambda$.   \Prf\ $W\cap(G\times H)$ belongs to the domain of
$\lambda_{G\times H}$, so by (b) of this proof, applied to the totally
finite measures $\mu_G$ and $\nu_H$,

$$\eqalignno{\lambda(W\cap(G\times H))
&=\lambda_{G\times H}(W\cap(G\times H))\cr
&=\int_G\int_H\chi(W\cap(G\times H))(x,y)\nu_H(dy)\mu_G(dx)\cr
&=\int_G\int_Y\chi(W\cap(G\times H))(x,y)\nu(dy)\mu_G(dx)\cr
\displaycause{because $\chi(W\cap(G\times H))(x,y)=0$ if
$y\in Y\setminus H$, so we can use 131E}
&=\int_X\int_Y\chi(W\cap(G\times H))(x,y)\nu(dy)\mu(dx)\cr}$$

\noindent by 131E again, because
$\int_Y\chi(W\cap(G\times H))(x,y)\nu(dy)=0$ if
$x\in X\setminus G$.   So $W\cap(G\times H)\in\Cal W$.\ \Qed

\medskip

\quad{\bf (iii)} In fact, $W\in\Cal W$ for every $W\in\Lambda$.   \Prf\
Remember that we are supposing that $\nu$ is
$\sigma$-finite.   Let $\sequencen{Y_n}$ be a non-decreasing sequence in
$\Tau^f$ covering $Y$, and for each $n\in\Bbb N$ set
$W_n=W\cap(X\times Y_n)$, $g_n(x)=\int\chi W_n(x,y)\nu(dy)$ whenever
this is defined.   For any $G\in\Sigma^f$,

\Centerline{$\int_Gg_nd\mu
=\iint\chi(W\cap(G\times Y_n))(x,y)\nu(dy)\mu(dx)$}

\noindent is defined and equal to $\lambda(W\cap(G\times Y_n))$, by
(ii).   But this means, first, that $G\setminus\dom g_n$ is negligible,
that is, that $\hat\mu(G\setminus\dom g_n)=0$.   Since this is so
whenever $\mu G$ is finite, $\tilde\mu(X\setminus\dom g_n)=0$, and
$g_n$ is defined
$\tilde\mu$-a.e.;  but $\tilde\mu=\hat\mu$, so $g_n$ is defined
$\hat\mu$-a.e., that is, $\mu$-a.e.\ (212Eb).   Next, if we set
$E_{na}=\{x:x\in\dom g_n,\,g_n(x)\ge a\}$ for $a\in\Bbb R$, then
$E_{na}\cap G\in\hat\Sigma$ whenever $G\in\Sigma^f$, where $\hat\Sigma$
is the domain of $\hat\mu$;  by the definition in 213D, $E_{na}$ is
measured by $\tilde\mu=\hat\mu$.   As $a$ is arbitrary, $g_n$ is
$\mu$-virtually measurable (212Fa).

We can therefore speak of $\int g_nd\mu$.   Now

$$\eqalignno{\iint\chi W_n(x,y)\nu(dy)\mu(dx)
&=\int g_nd\mu
=\sup_{G\in\Sigma^f}\int_Gg_n\cr
\displaycause{213B, because $\mu$ is semi-finite}
&=\sup_{G\in\Sigma^f}\lambda(W\cap(G\times Y_n))
=\lambda(W\cap(X\times Y_n))\cr}$$

\noindent by the definition in 251F.   Thus
$W\cap(X\times Y_n)\in\Cal W$.

This is true for every $n\in\Bbb N$.   Because $\sequencen{Y_n}\uparrow
Y$, $W\in\Cal W$, by (a-iii-$\beta$).\ \Qed

\medskip

\quad{\bf (iv)} We can therefore return to part (c) of the argument
above and conclude as before.
}%end of proof of 252B

\leader{252C}{}\cmmnt{ The theorem above is of course asymmetric, in
that different hypotheses are imposed on the two factor measures $\mu$
and $\nu$.   If we want a `symmetric' theorem we have to suppose that
they are both $\sigma$-finite, as follows.

\medskip

\noindent}{\bf Corollary} Let $(X,\Sigma,\mu)$ and $(Y,\Tau,\nu)$ be two
$\sigma$-finite measure spaces, and $\lambda$ the c.l.d.\ product
measure on $X\times Y$.   If $f$ is $\lambda$-integrable, then
$\iint f(x,y)\nu(dy)\mu(dx)$ and $\iint f(x,y)\mu(dx)\nu(dy)$ are defined, finite and equal.

\proof{ Since $\mu$ and $\nu$ are surely strictly localizable (211Lc),
we can apply 252B from either side to conclude that

\Centerline{$\iint f(x,y)\nu(dy)\mu(dx)
=\int f\,d\lambda
=\iint f(x,y)\mu(dx)\nu(dy)$.}
}%end of proof of 252C

\leader{252D}{}\cmmnt{ So many applications of Fubini's theorem are to
indicator functions that I take a few lines to spell out the form
which 252B takes in this case, as in parts (b)-(b*) of the proof there.

\medskip

\noindent}{\bf Corollary} Let $(X,\Sigma,\mu)$ and $(Y,\Tau,\nu)$ be
measure spaces and $\lambda$ the c.l.d.\ product measure on
$X\times Y$.   Suppose that $\nu$ is $\sigma$-finite and that $\mu$ is
either strictly localizable or complete and locally determined.

(i) If $W\in\dom\lambda$, then $\int\nu^*W[\{x\}]\mu(dx)$ is defined in
$[0,\infty]$ and equal to $\lambda W$.

(ii) If $\nu$ is complete, we can write $\int\nu W[\{x\}]\mu(dx)$ in
place of $\int\nu^*W[\{x\}]\mu(dx)$.

\proof{ The point is just that $\int\chi W(x,y)\nu(dy)=\hat\nu W[\{x\}]$
whenever either is defined, where $\hat\nu$ is the completion of $\nu$
(212F).   Now 252B tells us that

\Centerline{$\lambda W=\iint\chi W(x,y)\nu(dy)\mu(dx)
=\int\hat\nu W[\{x\}]\mu(dx)$.}

\noindent We always have $\hat\nu W[\{x\}]=\nu^*W[\{x\}]$, by the
definition of $\hat\nu$ (212C);  and if $\nu$ is complete, then
$\hat\nu=\nu$ so $\lambda W=\int\nu W[\{x\}]\mu(dx)$.
}%end of proof of 252D

\leader{252E}{Corollary} Let $(X,\Sigma,\mu)$ and $(Y,\Tau,\nu)$ be
measure spaces, with c.l.d.\ product $(X\times Y,\Lambda,\lambda)$.
Suppose that $\nu$ is $\sigma$-finite and that $\mu$ has locally
determined negligible sets\cmmnt{ (213I)}.   Then if $f$ is a
$\Lambda$-measurable real-valued function defined on a subset of
$X\times Y$, $y\mapsto f(x,y)$ is
$\nu$-virtually measurable for $\mu$-almost every $x\in X$.

\proof{ Let $\tilde f$ be a $\Lambda$-measurable extension of $f$ to a
real-valued function defined everywhere in $X\times Y$ (121I), and set
$\tilde f_x(y)=\tilde f(x,y)$ for all $x\in X$, $y\in Y$,

\Centerline{$D=\{x:x\in X,\,\tilde f_x$ is $\nu$-virtually
measurable$\}$.}

If $G\in\Sigma$ and $\mu G<\infty$, then $G\setminus D$ is negligible.
\Prf\ Let $\sequencen{Y_n}$ be a non-decreasing sequence of sets
of finite measure covering $Y$ respectively, and set

$$\eqalign{\tilde f_n(x,y)
&=\tilde f(x,y)\text{ if }x\in G,\,y\in Y_n
  \text{ and }|\tilde f(x,y)|\le n,\cr
&=0\text{ for other }x\in X\times Y.\cr}$$

\noindent Then each $\tilde f_n$ is
$\lambda$-integrable, being bounded and $\Lambda$-measurable and zero
off $G\times Y_n$.
Consequently, setting $\tilde f_{nx}(y)=\tilde f_n(x,y)$,

\Centerline{$\int(\int\tilde f_{nx}d\nu)\mu(dx)$ exists
$=\int\tilde f_nd\lambda$.}

\noindent But this surely means that $\tilde f_{nx}$ is
$\nu$-integrable, therefore $\nu$-virtually measurable, for almost
every $x\in X$.   Set

\Centerline{$D_n=\{x:x\in X,\,\tilde f_{nx}$ is $\nu$-virtually
measurable$\}$;}

\noindent then every $D_n$ is $\mu$-conegligible, so
$\bigcap_{n\in\Bbb N}D_n$ is conegligible.   But for any
$x\in G\cap\bigcap_{n\in\Bbb N}D_n$,
$\tilde f_x=\lim_{n\to\infty}\tilde f_{nx}$ is $\nu$-virtually
measurable.   Thus
$G\setminus D\subseteq X\setminus\bigcap_{n\in\Bbb N}D_n$ is
negligible.\ \Qed

This is true whenever $\mu G<\infty$.   As $G$ is arbitrary and $\mu$ has
locally determined negligible sets, $D$ is conegligible.   But, for any
$x\in D$, $y\mapsto f(x,y)$ is a restriction of $\tilde f_x$ and must be
$\nu$-virtually measurable.
}%end of proof of 252E

\leader{252F}{}\cmmnt{ As a further corollary we can get some useful
information about the c.l.d.\ product measure for arbitrary measure
spaces.

\medskip

\noindent}{\bf Corollary} Let $(X,\Sigma,\mu)$ and $(Y,\Tau,\nu)$ be two
measure spaces, $\lambda$ the c.l.d.\ product measure on
$X\times Y$, and $\Lambda$ its domain.   Let $W\in\Lambda$ be such
that the vertical section $W[\{x\}]$ is $\nu$-negligible
for $\mu$-almost every $x\in X$.   Then $\lambda W=0$.

\proof{ Take $E\in\Sigma$, $F\in\Tau$ of finite measure.   Let
$\lambda_{E\times F}$ be the subspace measure on $E\times F$.   By
251Q(ii-$\alpha$) again,
this is just the product of the subspace measures $\mu_E$ and $\nu_F$.
We know that $W\cap(E\times F)$ is measured by $\lambda_{E\times F}$.
At the same time, the vertical section $(W\cap(E\times
F))[\{x\}]=W[\{x\}]\cap F$ is $\nu_F$-negligible for $\mu_E$-almost
every $x\in E$.   Applying 252B to $\mu_E$ and $\nu_F$ and
$\chi(W\cap(E\times F))$,

\Centerline{$\lambda(W\cap(E\times F))
=\lambda_{E\times F}(W\cap(E\times F))
=\int_E\nu_F(W[\{x\}]\cap F)\mu_E(dx)=0$.}

\noindent But looking at the definition in 251F, we see that this means
that $\lambda W=0$, as claimed.
}%end of proof of 252F

\leader{252G}{}\cmmnt{ Theorem 252B and its corollaries depend on the
factor measures $\mu$ and $\nu$ belonging to restricted classes.   There
is a partial result which applies to all c.l.d.\ product measures, as
follows.

\medskip

\noindent}{\bf Tonelli's theorem} Let $(X,\Sigma,\mu)$ and
$(Y,\Tau,\nu)$ be
measure spaces, and $(X\times Y,\Lambda,\lambda)$ their c.l.d.\ product.
Let $f$ be a $\Lambda$-measurable $[-\infty,\infty]$-valued function
defined on a member of $\Lambda$, and suppose that either
$\iint|f(x,y)|\mu(dx)\nu(dy)$ or $\iint|f(x,y)|\nu(dy)\mu(dx)$ exists in
$\Bbb R$.   Then $f$ is $\lambda$-integrable.

\proof{ Because the construction of the product measure is symmetric in
the two factors, it is enough to consider the case in which $\iint
|f(x,y)|\nu(dy)\mu(dx)$ is defined and finite, as the same ideas will
surely deal with the other case also.

\medskip

{\bf (a)} The first step is to check that $f$ is defined and finite
$\lambda$-a.e.   \Prf\ Set $W=\{(x,y):(x,y)\in\dom f,\,f(x,y)$ is
finite$\}$.   Then $W\in\Lambda$.   The hypothesis

\Centerline{$\iint|f(x,y)|\nu(dy)\mu(dx)$ is defined and finite}

\noindent includes the assertion

\Centerline{$\int|f(x,y)|\nu(dy)$ is defined and finite for
$\mu$-almost every $x$,}

\noindent which implies that

\Centerline{for $\mu$-almost every $x$, $f(x,y)$ is defined and finite
for $\nu$-almost every $y$;}

\noindent that is, that

\Centerline{for $\mu$-almost every $x$, $W[\{x\}]$ is
$\nu$-conegligible.}

\noindent But by 252F this implies
that $(X\times Y)\setminus W$ is $\lambda$-negligible, as required.\
\Qed

\medskip

{\bf (b)} Let $h$ be any non-negative $\lambda$-simple
function such that $h\le|f|\,\,\lambda$-a.e.   Then
\ifdim\pagewidth>467pt$\int h$ cannot be greater than
$\iint|f(x,y)|\nu(dy)\mu(dx)$.
\else $\int h$ is at most $\iint|f(x,y)|\nu(dy)\mu(dx)$.\fi
\Prf\ Set

\Centerline{$W=\{(x,y):(x,y)\in\dom f,\,h(x,y)\le|f(x,y)|\}$,
\quad$h'=h\times\chi W$;}

\noindent then $h'$ is a simple function and $h'\eae h$.   Express $h'$
as $\sum_{i=0}^na_i\chi W_i$ where $a_i\ge 0$ and $\lambda W_i<\infty$
for each $i$.
Let $\epsilon>0$.   For each $i\le n$ there are $E_i\in\Sigma$,
$F_i\in\Tau$ such that $\mu E_i<\infty$, $\nu F_i<\infty$ and
$\lambda(W_i\cap(E_i\times F_i))\ge\lambda W_i-\epsilon$.   Set
$E=\bigcup_{i\le n}E_i$ and $F=\bigcup_{i\le n}F_i$.   Consider the
subspace measures $\mu_E$ and $\nu_F$ and their product
$\lambda_{E\times F}$ on $E\times F$;  then $\lambda_{E\times F}$ is the
subspace measure on $E\times F$ defined from $\lambda$ (251Q(ii-$\alpha$)
once more).
Accordingly, applying 252B to the product $\mu_E\times\nu_F$,

\Centerline{$\int_{E\times F}h'\,d\lambda
=\int_{E\times F}h'\,d\lambda_{E\times F}
=\int_E\int_Fh'(x,y)\nu_F(dy)\mu_E(dx)$.}

For any $x$, we know that $h'(x,y)\le|f(x,y)|$ whenever $f(x,y)$ is
defined.   So we can be sure that

\Centerline{$\int_Fh'(x,y)\nu_F(dy)
=\int h'(x,y)\chi F(y)\nu(dy)
\le\int|f(x,y)|\nu(dy)$}

\noindent at least whenever $\int_Fh'(x,y)\nu_F(dy)$ and
$\int|f(x,y)|\nu(dy)$ are both defined, which is the case for almost
every $x\in E$.   Consequently

$$\eqalign{\int_{E\times F}h'\,d\lambda
&=\int_E\int_Fh'(x,y)\nu_F(dy)\mu_E(dx)\cr
&\le\int_E\int|f(x,y)|\nu(dy)\mu(dx)
\le\iint|f(x,y)|\nu(dy)\mu(dx).\cr}$$

On the other hand,

$$\eqalign{\int h'\,d\lambda-\int_{E\times F}h'\,d\lambda
&=\sum_{i=0}^na_i\lambda(W_i\setminus(E\times F))\cr
&\le\sum_{i=0}^na_i\lambda(W_i\setminus(E_i\times F_i))
\le\epsilon\sum_{i=0}^na_i.\cr}$$

\noindent So

\Centerline{$\int h\,d\lambda
=\int h'd\lambda
\le\iint|f(x,y)|\nu(dy)\mu(dx)$$+\epsilon\sum_{i=0}^na_i$.}

\noindent As $\epsilon$ is arbitrary,
$\int h\,d\lambda
\le\iint|f(x,y)|\nu(dy)\mu(dx)$, as claimed.\ \Qed

\medskip

{\bf (c)} This is true whenever $h$ is a $\lambda$-simple function
less than or equal to $|f|\,\,\lambda$-a.e.   But $|f|$ is
$\Lambda$-measurable and $\lambda$ is semi-finite (251Ic), so this is
enough to ensure that $|f|$ is $\lambda$-integrable (213B), which
(because $f$ is supposed to be $\Lambda$-measurable) in
turn implies that $f$ is $\lambda$-integrable.
}%end of proof of 252G

\leader{252H}{\bf Corollary} Let $(X,\Sigma,\mu)$ and $(Y,\Tau,\nu)$ be
$\sigma$-finite measure spaces, $\lambda$ the c.l.d.\ product measure on
$X\times Y$, and $\Lambda$ its domain.

(a) Let $f$ be a
$\Lambda$-measurable $[-\infty,\infty]$-valued function defined on a member of
$\Lambda$.   Then if one of

\Centerline{$\int_{X\times Y}|f(x,y)|\lambda(d(x,y))$,
\quad$\int_Y\int_X|f(x,y)|\mu(dx)\nu(dy)$,
\quad$\int_X\int_Y|f(x,y)|\nu(dy)\mu(dx)$}

\noindent exists in $\Bbb R$, so do the other two, and in this case

\Centerline{$\int_{X\times Y}f(x,y)\lambda(d(x,y))
=\int_Y\int_Xf(x,y)\mu(dx)\nu(dy)
=\int_X\int_Yf(x,y)\nu(dy)\mu(dx)$.}

(b)\dvAnew{2013} Let $f$ be a
$\Lambda$-measurable $[0,\infty]$-valued function defined on a member of
$\Lambda$.   Then

\Centerline{$\int_{X\times Y}f(x,y)\lambda(d(x,y))
=\int_Y\int_Xf(x,y)\mu(dx)\nu(dy)
=\int_X\int_Yf(x,y)\nu(dy)\mu(dx)$}

\noindent in the sense that if one of the integrals is defined in
$[0,\infty]$ so are the other two, and all three are then equal.

\proof{{\bf (a)(i)} Suppose that $\int|f|d\lambda$ is finite.
Because both $\mu$ and $\nu$ are $\sigma$-finite, 252B tells us that

\Centerline{$\iint|f(x,y)|\mu(dx)\nu(dy)$,
\quad$\iint|f(x,y)|\nu(dy)\mu(dx)$}

\noindent both exist and are equal to
$\int|f|d\lambda$, while

\Centerline{$\iint f(x,y)\mu(dx)\nu(dy)$,
\quad$\iint f(x,y)\nu(dy)\mu(dx)$}

\noindent both exist and are equal to $\int fd\lambda$.

\medskip

\quad{\bf (ii)} Now suppose that $\iint|f(x,y)|\nu(dy)\mu(dx)$ exists in
$\Bbb R$.   Then 252G tells us that $|f|$ is
$\lambda$-integrable, so we can
use (i) to complete the argument.   Exchanging the coordinates, the same
argument applies if $\iint|f(x,y)|\mu(dx)\nu(dy)$ exists in $\Bbb R$.

\medskip

{\bf (b)(i)} If $\int fd\lambda$ is defined, the result is immediate from
252B.

\medskip

\quad{\bf (ii)} Suppose that $\iint_Xf(x,y)\nu(dy)\mu(dx)$ is defined.
As in part (a) of the proof of 252G, but this time setting
$W=\dom f$, we see that $W\in\Lambda$ and that $W[\{x\}]$ is
$\nu$-conegligible for $\mu$-almost every $x$, so that $W$ is
$\lambda$-conegligible.   Since $f$ is non-negative, $\Lambda$-measurable
and defined almost everywhere, $\int fd\lambda$ is defined in $[0,\infty]$
and we are in case (i).
}%end of proof of 252H

\leader{252I}{Corollary} Let $(X,\Sigma,\mu)$ and $(Y,\Tau,\nu)$ be
measure spaces, $\lambda$ the c.l.d.\ product measure on $X\times
Y$, and $\Lambda$ its domain.   Take $W\in\Lambda$.   If
either of the integrals

\Centerline{$\int\mu^*W^{-1}[\{y\}]\nu(dy)$,
\quad$\int\nu^*W[\{x\}]\mu(dx)$}

\noindent exists and is finite, then $\lambda W<\infty$.

\proof{ Apply 252G with $f=\chi W$, remembering that

\Centerline{$\mu^*W^{-1}[\{y\}]=\int\chi W(x,y)\mu(dx)$,
\quad$\nu^*W[\{x\}]=\int\chi W(x,y)\nu(dy)$}

\noindent whenever the integrals are defined, as in the proof of 252D.
}%end of proof of 252I

\cmmnt{
\leader{252J}{Remarks} 252H is the basic form of Fubini's theorem;  it
is not a
coincidence that most authors avoid non-$\sigma$-finite spaces in this
context.    The next two examples exhibit some of the
difficulties which can arise if we leave the familiar territory of
more-or-less Borel measurable  functions on $\sigma$-finite spaces.
The first is a classic.
}%end of comment

\leader{252K}{Example} Let $(X,\Sigma,\mu)$ be $[0,1]$ with Lebesgue
measure, and let $(Y,\Tau,\nu)$ be $[0,1]$ with counting measure.
\cmmnt{

\spheader 252Ka} Consider the set

\Centerline{$W=\{(t,t):t\in[0,1]\}\subseteq X\times Y$.}

\cmmnt{
\noindent We observe that $W$ is expressible as

\Centerline{$\bigcap_{n\in\Bbb N}\bigcup_{k=0}^n
[\Bover{k}{n+1},\Bover{k+1}{n+1}]\times[\Bover{k}{n+1},
\Bover{k+1}{n+1}]\in\Sigma\tensorhat\Tau$.}

\noindent If we look at the sections

\Centerline{$W^{-1}[\{t\}]=W[\{t\}]=\{t\}$}

\noindent for $t\in [0,1]$, we have
}

\ifwithproofs
\Centerline{$\iint\chi W(x,y)\mu(dx)\nu(dy)
=\int\mu W^{-1}[\{y\}]\nu(dy)=\int 0\,\nu(dy)=0$,}

\Centerline{$\iint\chi W(x,y)\nu(dy)\mu(dx)
=\int\nu W[\{x\}]\mu(dx)=\int 1\,\mu(dx)=1$,}
\else
\Centerline{$\iint\chi W(x,y)\mu(dx)\nu(dy)
=0$,}

\Centerline{$\iint\chi W(x,y)\nu(dy)\mu(dx)
=1$,}
\fi

\noindent so the two repeated integrals differ.
\cmmnt{It is therefore not
generally possible to reverse the order of repeated integration, even
for a non-negative measurable function in which both repeated integrals
exist and are finite.
}%end of comment

\cmmnt{
\spheader 252Kb Because the set $W$ of part (a) actually belongs
to $\Sigma\tensorhat\Tau$, we know that it is measured by the c.l.d.\
product measure $\lambda$, and 252F (applied with the coordinates
reversed) tells us that $\lambda W=0$.
}%end of comment

\cmmnt{
\spheader 252Kc It is in fact easy to give a
full description of $\lambda$.

\medskip

\quad{\bf (i)} The point is that a set $W\subseteq[0,1]\times[0,1]$
belongs to the domain $\Lambda$ of $\lambda$ iff every horizontal
section $W^{-1}[\{y\}]$ is Lebesgue measurable.   \prooflet{\Prf\
($\alpha$) If $W\in\Lambda$, then, for every $b\in[0,1]$,
$\lambda([0,1]\times\{b\})$ is finite, so $W\cap([0,1]\times\{b\})$ is a
set of finite measure, and

\Centerline{$\lambda(W\cap([0,1]\times\{b\}))
=\int\mu(W\cap([0,1]\times\{b\}))^{-1}[\{y\}]\nu(dy)
=\mu W^{-1}[\{b\}]$}

\noindent by 252D, because $\mu$ is $\sigma$-finite, $\nu$ is both
strictly localizable and complete and locally determined, and

$$\eqalign{(W\cap([0,1]\times\{b\}))^{-1}[\{y\}]
&=W^{-1}[\{b\}]\text{ if }y=b,\cr
&=\emptyset\text{ otherwise.}\cr}$$

\noindent As $b$ is arbitrary, every horizontal section of $W$ is
measurable.   ($\beta$) If every horizontal section of $W$ is
measurable, let $F\subseteq[0,1]$ be any set of finite measure for
$\nu$;  then $F$ is finite, so

\Centerline{$W\cap([0,1]\times F)
=\bigcup_{y\in F}W^{-1}[\{y\}]\times\{y\}
\in\Sigma\tensorhat\Tau\subseteq\Lambda$.}

\noindent But it follows that $W$ itself belongs to $\Lambda$, by
251H.\ \Qed
}%end of prooflet

\medskip

\quad{\bf (ii)} Now some of the same calculations show that for every
$W\in\Lambda$,

\Centerline{$\lambda W=\sum_{y\in [0,1]}\mu W^{-1}[\{y\}]$.}

\prooflet{\noindent\Prf\ For any finite $F\subseteq[0,1]$,

$$\eqalign{\lambda(W\cap([0,1]\times F))
&=\int\mu(W\cap([0,1]\times F))^{-1}[\{y\}]\nu(dy)\cr
&=\int_F\mu W^{-1}[\{y\}]\nu(dy)
=\sum_{y\in F}\mu W^{-1}[\{y\}].\cr}$$

\noindent So

\Centerline{$\lambda W=\sup_{F\subseteq[0,1]\text{ is finite}}
\sum_{y\in F}\mu W^{-1}[\{y\}]=\sum_{y\in [0,1]}\mu W^{-1}[\{y\}]$.
\Qed}
}%end of prooflet
}%end of comment

\cmmnt{
\leader{252L}{Example} For the second example, I turn to a problem that
can arise if we neglect to check that a function is measurable as a
function of two variables.

Let $(X,\Sigma,\mu)=(Y,\Tau,\nu)$ be $\omega_1$, the first uncountable
ordinal (2A1Fc), with the countable-cocountable measure (211R).   Set

\Centerline{$W=\{(\xi,\eta):\xi\le\eta<\omega_1\}\subseteq X\times Y$.}

\noindent Then all the horizontal sections
$W^{-1}[\{\eta\}]=\{\xi:\xi\le\eta\}$ are countable, so

\Centerline{$\int\mu W^{-1}[\{\eta\}]\nu(d\eta)
=\int 0\,\nu(d\eta)
=0$,}

\noindent while all the vertical sections
$W[\{\xi\}]=\{\eta:\xi\le\eta<\omega_1\}$ are cocountable, so

\Centerline{$\int\nu W[\{\xi\}]\mu(d\xi)
=\int 1\,\mu(d\xi)
=1$.}

\noindent Because the two repeated integrals are different, they cannot
both be equal to the measure of $W$, and the sole resolution is to say
that $W$ is not measured by the product measure.
}%end of comment

\cmmnt{
\leader{252M}{Remark} A third kind of difficulty in the formula

\Centerline{$\iint f(x,y)dxdy=\iint f(x,y)dydx$}

\noindent can arise even on probability spaces with
$\Sigma\tensorhat\Tau$-measurable real-valued functions defined
everywhere if we neglect to check that $f$ is integrable with respect to
the product measure.   In 252H, we do need the hypothesis that one of

\Centerline{$\int_{X\times Y}|f(x,y)|\lambda(d(x,y))$,
\quad$\int_Y\int_X|f(x,y)|\mu(dx)\nu(dy)$,
\quad$\int_X\int_Y|f(x,y)|\nu(dy)\mu(dx)$}

\noindent is finite.   For examples to show this, see 252Xf and 252Xg.
}%end of comment

\leader{252N}{Integration through ordinate sets I:  Proposition} Let
$(X,\Sigma,\mu)$ be a complete locally determined measure space, and
$\lambda$ the c.l.d.\ product measure on $X\times\Bbb R$, where $\Bbb R$
is given Lebesgue measure;  write $\Lambda$ for the domain of $\lambda$.
For any
$[0,\infty]$-valued function $f$ defined on a conegligible subset of
$X$, write $\Omega_f$, $\Omega'_f$ for the {\bf ordinate sets}

\Centerline{$\Omega_f=\{(x,a):x\in\dom f,\,0\le a\le f(x)\}
\subseteq X\times\Bbb R$,}

\Centerline{$\Omega'_f=\{(x,a):x\in\dom f,\,0\le a<f(x)\}
\subseteq X\times\Bbb R$.}

\noindent Then

\Centerline{$\lambda\Omega_f=\lambda\Omega'_f=\int fd\mu$}

\noindent in the sense that if one of these is defined in $[0,\infty]$,
so are the other two, and they are equal.

\proof{{\bf (a)} If $\Omega_f\in\Lambda$, then

\Centerline{$\int f(x)\mu(dx)
=\int\nu\{y:(x,y)\in\Omega_f\}\mu(dx)
=\lambda\Omega_f$}

\noindent by 252D, writing $\mu$ for Lebesgue measure, because $f$ is
defined almost everywhere.   Similarly, if $\Omega'_f\in\Lambda$,

\Centerline{$\int f(x)\mu(dx)
=\int\nu\{y:(x,y)\in\Omega'_f\}\mu(dx)
=\lambda\Omega'_f$.}

\medskip

{\bf (b)} If $\int fd\mu$ is defined, then $f$ is $\mu$-virtually
measurable, therefore measurable (because $\mu$ is complete);
again because $\mu$ is complete, $\dom f\in\Sigma$.   So

\Centerline{$\Omega'_f
=\bigcup_{q\in\Bbb Q,q>0}\{x:x\in\dom f,\,f(x)>q\}\times[0,q]$,}

\Centerline{$\Omega_f
=\bigcap_{n\ge 1}\bigcup_{q\in\Bbb Q,q>0}
  \{x:x\in\dom f,\,f(x)\ge q-\Bover1n\}\times[0,q]$}

\noindent belong to $\Lambda$, so that $\lambda\Omega_f$ and
$\lambda\Omega'_f$ are defined.   Now both are equal to
$\int fd\mu$, by (a).
}%end of proof of 252N

\leader{252O}{Integration through ordinate sets II:  Proposition} Let
$(X,\Sigma,\mu)$ be a measure space, and $f$ a non-negative $\mu$-virtually
measurable function defined on a conegligible subset of $X$.   Then

\Centerline{$\int fd\mu
=\int_0^{\infty}\mu^*\{x:x\in\dom f,\,f(x)\ge t\}dt
=\int_0^{\infty}\mu^*\{x:x\in\dom f,\,f(x)>t\}dt$}

\noindent in $[0,\infty]$, where the integrals $\int\ldots dt$ are taken
with respect to Lebesgue measure.

\proof{ Completing $\mu$ does not change the integral of $f$ or the outer
measure $\mu^*$ (212Fb, 212Ea), so we may suppose that $\mu$ is complete,
in which case $\dom f$ and $f$ will be measurable.
For $n$, $k\in\Bbb N$ set $E_{nk}=\{x:x\in\dom f,\,f(x)>2^{-n}k\}$,
$g_n(x)=2^{-n}\sum_{k=1}^{4^n}\chi E_{nk}$.   Then $\sequencen{g_n}$ is
a non-decreasing sequence of measurable functions converging to $f$ at
every point of $\dom f$, so $\int fd\mu=\lim_{n\to\infty}\int g_nd\mu$
and $\mu\{x:f(x)>t\}=\lim_{n\to\infty}\mu\{x:g_n(x)>t\}$ for every
$t\ge 0$;  consequently

\Centerline{$\int_0^{\infty}\mu\{x:f(x)>t\}dt
=\lim_{n\to\infty}\int_0^{\infty}\mu\{x:g_n(x)>t\}dt$.}

\noindent On the other hand,
$\mu\{x:g_n(x)>t\}=\mu E_{nk}$ if $1\le k\le 4^n$ and
$2^{-n}(k-1)\le t<2^{-n}k$, $0$ if $t\ge 2^n$, so that

\Centerline{$\int_0^{\infty}\mu\{x:g_n(x)>t\}dt$
$=\sum_{k=1}^{4^n}2^{-n}\mu E_{nk}$
$=\int g_n\,d\mu$,}

\noindent for every $n\in\Bbb N$.   So
$\int_0^{\infty}\mu\{x:f(x)>t\}dt=\int fd\mu$.

Now $\mu\{x:f(x)\ge t\}=\mu\{x:f(x)>t\}$ for almost all $t$.
\Prf\ Set $C=\{t:\mu\{x:f(x)>t\}<\infty\}$, $h(t)=\mu\{x:f(x)>t\}$ for
$t\in C$.   If $C$ is not empty,
$h:C\to\coint{0,\infty}$ is monotonic, so is continuous
almost everywhere in $C$ (222A).   But at any point of
$C\setminus\{\inf C\}$ at which $h$ is continuous,

\Centerline{$\mu\{x:f(x)\ge t\}=\lim_{s\uparrow t}\mu\{x:f(x)>s\}
=\mu\{x:f(x)>t\}$.}

\noindent So we have the result, since
$\mu\{x:f(x)\ge t\}=\mu\{x:f(x)>t\}=\infty$ for any
$t\in\coint{0,\infty}\setminus C$.\ \Qed

Accordingly $\int_0^{\infty}\mu\{x:f(x)\ge t\}dt$ is also equal to
$\int fd\mu$.
}%end of proof of 252O

\leader{*252P}{}\cmmnt{ If we work through the ideas of 252B for
$\Sigma\tensorhat\Tau$-measurable functions, we get the following, which
is sometimes useful.

\medskip

\noindent}{\bf Proposition} Let $(X,\Sigma,\mu)$ be a measure space, and
$(Y,\Tau,\nu)$ a $\sigma$-finite measure space.   Then for any
$\Sigma\tensorhat\Tau$-measurable function
$f:X\times Y\to[0,\infty]$,
$x\mapsto\int f(x,y)\nu(dy):X\to[0,\infty]$ is $\Sigma$-measurable.
If $\mu$ is semi-finite,
$\iint f(x,y)\nu(dy)\mu(dx)\discretionary{}{}{}=\int fd\lambda$,
where $\lambda$ is the c.l.d.\ product measure on $X\times Y$.

\proof{{\bf (a)} Let $\sequencen{Y_n}$ be a non-decreasing sequence of
subsets of $Y$ of finite measure with union $Y$.   Set

$$\eqalign{\Cal A
&=\{W:W\subseteq X\times Y,\,W[\{x\}]\in\Tau\text{ for every }x\in X,\cr
&\qquad\qquad\qquad\qquad
x\mapsto\nu(Y_n\cap W[\{x\}])\text{ is }\Sigma\text{-measurable for
every }n\in\Bbb N\}.\cr}$$

\noindent Then $\Cal A$ is a Dynkin class of subsets of $X\times Y$
including $\{E\times F:E\in\Sigma,\,F\in\Tau\}$, so includes
$\Sigma\tensorhat\Tau$, by the Monotone Class Theorem again (136B).

This means that if $W\in\Sigma\tensorhat\Tau$, then

\Centerline{$\mu W[\{x\}]=\sup_{n\in\Bbb N}\nu(Y_n\cap W[\{x\}])$}

\noindent is defined for every $x\in X$ and is a $\Sigma$-measurable
function of $x$.

\medskip

{\bf (b)} Now, for $n$, $k\in\Bbb N$, set

\Centerline{$W_{nk}=\{(x,y):f(x,y)\ge 2^{-n}k\}$,
\quad$g_n=\sum_{k=1}^{4^n}2^{-n}\chi W_{nk}$.}

\noindent Then if we set

\Centerline{$h_n(x)
=\int g_n(x,y)\nu(dy)$$=\sum_{k=1}^{4^n}2^{-n}\nu W_{nk}[\{x\}]$}

\noindent for $n\in\Bbb N$ and $x\in X$, $h_n:X\to[0,\infty]$ is
$\Sigma$-measurable, and

\Centerline{$\lim_{n\to\infty}h_n(x)
=\int(\lim_{n\to\infty}g_n(x,y))\nu(dy)=\int f(x,y)\nu(dy)$}

\noindent for every $x$, because $\sequencen{g_n(x,y)}$ is a
non-decreasing sequence with limit $f(x,y)$ for all $x\in X$, $y\in Y$.
So $x\mapsto\int f(x,y)\nu(dy)$ is defined everywhere in $X$ and is
$\Sigma$-measurable.

\medskip

{\bf (c)} If $E\subseteq X$ is measurable and has finite measure, then
$\int_E\int f(x,y)\nu(dy)\mu(dx)=\int_{E\times Y}fd\lambda$, applying
252B to the product of the subspace measure $\mu_E$ and $\nu$ (and using
251Q to check that the product of $\mu_E$ and $\nu$ is the subspace
measure on $E\times Y$).   Now if $\lambda W$ is defined and finite,
there must be a non-decreasing sequence $\sequencen{E_n}$ of subsets of
$X$ of finite measure such that
$\lambda W=\sup_{n\in\Bbb N}\lambda(W\cap(E_n\times Y))$, so that
$W\setminus\bigcup_{n\in\Bbb N}(E_n\times Y)$ is negligible, and

$$\eqalignno{\int_Wfd\lambda
&=\lim_{n\to\infty}\int_{W\cap(E_n\times Y)}fd\lambda\cr
\displaycause{by B.Levi's theorem applied to
$\sequencen{f\times\chi(W\cap(E_n\times Y))}$}
&\le\lim_{n\to\infty}\int_{E_n\times Y}fd\lambda
=\lim_{n\to\infty}\int_{E_n}\int f(x,y)\nu(dy)\mu(dx)\cr
&\le\iint f(x,y)\nu(dy)\mu(dx).\cr}$$

\noindent By 213B once more,

\Centerline{$\int fd\lambda
=\sup_{\lambda W<\infty}\int_Wfd\lambda
\le\iint f(x,y)\nu(dy)\mu(dx)$.}

\noindent But also, if $\mu$ is semi-finite,

\Centerline{$\iint f(x,y)\nu(dy)\mu(dx)
=\sup_{\mu E<\infty}\int_E\int f(x,y)\nu(dy)\mu(dx)
\le\int fd\lambda$,}

\noindent so $\int fd\lambda=\iint f(x,y)\nu(dy)\mu(dx)$, as claimed.
}%end of proof of 252P

\leader{252Q}{The volume of a ball}\dvro{ The Lebesgue measure of the
unit ball in $\BbbR^r$ is}
{ We now have all the essential
machinery to perform a little calculation which is, I suppose, desirable
simply as general knowledge:  the volume of the unit ball
$\{x:\|x\|\le 1\}=\{(\xi_1,\ldots,\xi_r):\sum_{i=1}^r\xi_i^2\le 1\}$ in
$\BbbR^r$.   In fact, from a theoretical point
of view, I think we could very nearly just call it $\beta_r$ and leave
it at that;  but since there is a general formula in terms of
$\beta_2=\pi$ and factorials, it seems shameful not to present it.   The
calculation has
nothing to do with Lebesgue integration, and I could dismiss it as mere
advanced calculus;  but since only a minority of mathematicians are now
taught calculus to this level with reasonable rigour before being
introduced to the Lebesgue integral, I do not doubt that many readers,
like myself, missed some of the subtleties involved.   I
therefore take the space to spell the details out in the style used
elsewhere in this volume, recognising that the machinery employed is a
good deal more elaborate than is really necessary for this result.

\header{252Qa}{\bf (a)} The first basic fact we need is that, for any
$n\ge 1$,

$$\eqalign{I_n=\int_{-\pi/2}^{\pi/2}\cos^nt\,dt
&=\Bover{(2k)!}{(2^kk!)^2}\pi\text{ if }n=2k\text{ is even},\cr
&=2\Bover{(2^kk!)^2}{(2k+1)!}\text{ if }n=2k+1\text{ is odd}.\cr}$$

\prooflet{\noindent \Prf\  For $n=0$, of course,

\Centerline{$I_0=\int_{-\pi/2}^{\pi/2}1\,dt=\pi
=\Bover{0!}{(2^00!)^2}\pi$,}

\noindent while for $n=1$ we have

\Centerline{$I_1=\sin\bover{\pi}2-\sin(-\bover{\pi}2)=2
=2\Bover{(2^00!)^2}{1!}$,}

\noindent using the Fundamental Theorem of Calculus (225L) and the fact
that $\sin'=\cos$ is bounded.   For the inductive step to $n+1\ge 2$, we
can use integration by parts (225F):

$$\eqalign{I_{n+1}
&=\int_{-\pi/2}^{\pi/2}\cos t\cos^nt\,dt\cr
&=\sin\Bover{\pi}2\cos^n\Bover{\pi}2
   -\sin(-\Bover{\pi}2)\cos^n(-\Bover{\pi}2)
   +\int_{-\pi/2}^{\pi/2}\sin t\cdot n\cos^{n-1}t\cdot\sin t\,dt\cr
&=n\int_{-\pi/2}^{\pi/2}(1-\cos^2t)\cos^{n-1}t\,dt
=n(I_{n-1}-I_{n+1}),\cr}$$

\noindent   so that $I_{n+1}=\Bover{n}{n+1}I_{n-1}$.
Now the given formulae follow by an easy induction.\ \Qed}

\spheader 252Qb The next result is that, for any $n\in\Bbb N$
and any $a\ge 0$,

\Centerline{$\int_{-a}^a(a^2-s^2)^{n/2}ds=I_{n+1}a^{n+1}$.}

\prooflet{\noindent\Prf\ Of course this is an integration by
substitution;  but the singularity of the integrand at $s=\pm a$
complicates the issue slightly.   I offer the following argument.   If
$a=0$ the
result is trivial;  take $a>0$.   For $-a\le b\le a$, set
$F(b)=\int_{-a}^b(a^2-s^2)^{n/2}ds$.   Because the integrand is
continuous, $F'(b)$ exists and is equal to $(a^2-b^2)^{n/2}$ for
$-a<b<a$ (222H).   Set $G(t)=F(a\sin t)$;  then $G$ is continuous and

\Centerline{$G'(t)=aF'(a\sin t)\cos t=a^{n+1}\cos^{n+1}t$}

\noindent for
$-\bover{\pi}2<t<\bover{\pi}2$.   Consequently

$$\eqalignno{\int_{-a}^a(a^2-s^2)^{n/2}ds
&=F(a)-F(-a)
=G(\Bover{\pi}2)-G(-\Bover{\pi}2)
=\int_{-\pi/2}^{\pi/2}G'(t)dt\cr
\noalign{\noindent (by 225L, as before)}
&=a^{n+1}I_{n+1},\cr}$$

\noindent as required.\ \Qed}

\spheader 252Qc Now at last we are ready for the balls
$B_r=\{x:x\in\BbbR^r,\,\|x\|\le 1\}$.   Let $\mu_r$ be Lebesgue
measure on $\BbbR^r$, and set $\beta_r=I_1I_2\ldots I_r$ for $r\ge 1$.
I claim that, writing

\Centerline{$B_r(a)=\{x:x\in\BbbR^r,\,\|x\|\le a\}$,}

\noindent we have $\mu_r(B_r(a))=\beta_ra^r$ for every $a\ge 0$.
\prooflet{\Prf\ Induce on $r$.   For $r=1$ we have $\beta_1=2$,
$B_1(a)=[-a,a]$, so the result is trivial.   For the inductive step to
$r+1$, we have

$$\eqalignno{\mu_{r+1}B_{r+1}(a)
&=\int\mu_r\{x:(x,t)\in B_{r+1}(a)\}dt \cr
\noalign{\noindent (putting 251N and 252D together, and using the fact
that $B_{r+1}(a)$ is closed, therefore measurable)}
&=\int_{-a}^a\mu_rB_r(\sqrt{a^2-t^2})dt\cr
\noalign{\noindent (because $(x,t)\in B_{r+1}(a)$ iff $|t|\le a$ and
$\|x\|\le\sqrt{a^2-t^2}$)}
&=\int_{-a}^a\beta_r(a^2-t^2)^{r/2}dt\cr
\noalign{\noindent (by the inductive hypothesis)}
&=\beta_ra^{r+1}I_{r+1}\cr
\noalign{\noindent (by (b) above)}
&=\beta_{r+1}a^{r+1}\cr}$$

\noindent (by the definition of $\beta_{r+1}$).   Thus the induction
continues.\ \Qed}

\spheader 252Qd In particular, the $r$-dimensional Lebesgue
measure of the $r$-dimensional ball $B_r=B_r(1)$ is just
$\beta_r=I_1\ldots I_r$.   Now an easy induction on $k$ shows that
}%end of \dvro

$$\eqalign{\beta_r
&=\Bover{1}{k!}\pi^k\text{ if }r=2k\text{ is even},\cr
&=\Bover{2^{2k+1}k!}{(2k+1)!}\pi^k\text{ if }r=2k+1\text{ is odd}.\cr}$$

\cmmnt{
\spheader 252Qe Note that in part (c) of the proof we saw that
$\{x:x\in\BbbR^r,\,\|x\|\le a\}$ has measure $\beta_ra^r$ for every
$a\ge 0$.

The formulae here are consistent with the assignation $\beta_0=1$;
which corresponds to saying that $\BbbR^0=\{\emptyset\}$, that
$\mu_0\BbbR^0=1$ and that $B_0=\{\emptyset\}$.   Taking $\mu_0\BbbR^0$ to
be $1$ is itself consistent with the idea that, following 251N, the product
measure $\mu_0\times\mu_r$ ought to match $\mu_{0+r}$ on $\BbbR^{0+r}$.
}%end of comment


\leader{252R}{Complex-valued functions}\cmmnt{ It is easy to apply the
results of 252B-252I
above to complex-valued functions, by considering their real and
imaginary parts.   Specifically:

\medskip

}{\bf (a)} Let $(X,\Sigma,\mu)$ and $(Y,\Tau,\nu)$ be
measure spaces, and $\lambda$ the c.l.d.\ product measure on $X\times
Y$.   Suppose that $\nu$ is $\sigma$-finite and that $\mu$ is either
strictly localizable or complete and locally determined.   Let $f$ be a
$\lambda$-integrable
complex-valued function.  Then $\iint f(x,y)\nu(dy)\mu(dx)$ is defined
and equal to $\int fd\lambda$.

\spheader 252Rb Let $(X,\Sigma,\mu)$ and $(Y,\Tau,\nu)$
be measure spaces, $\lambda$ the c.l.d.\ product measure on $X\times
Y$, and $\Lambda$ its domain.
Let $f$ be a $\Lambda$-measurable complex-valued function defined on a
member of $\Lambda$, and suppose that either
$\iint|f(x,y)|\mu(dx)\nu(dy)$ or $\iint|f(x,y)|\nu(dy)\mu(dx)$ is
defined and finite.   Then $f$ is $\lambda$-integrable.

\spheader 252Rc Let $(X,\Sigma,\mu)$ and $(Y,\Tau,\nu)$ be
$\sigma$-finite measure spaces, $\lambda$ the c.l.d.\ product
measure on $X\times Y$, and $\Lambda$ its domain.   Let $f$ be a
$\Lambda$-measurable complex-valued function defined on a member of
$\Lambda$.   Then if one of

\Centerline{$\int_{X\times Y}|f(x,y)|\lambda(d(x,y))$,
\quad$\int_Y\int_X|f(x,y)|\mu(dx)\nu(dy)$,
\quad$\int_X\int_Y|f(x,y)|\nu(dy)\mu(dx)$}

\noindent exists in $\Bbb R$, so do the other two, and in this case

\Centerline{$\int_{X\times Y}f(x,y)\lambda(d(x,y))
=\int_Y\int_Xf(x,y)\mu(dx)\nu(dy)
=\int_X\int_Yf(x,y)\nu(dy)\mu(dx)$.}

\exercises{
\leader{252X}{Basic exercises (a)}
%\spheader 252Xa
Let $(X,\Sigma,\mu)$ and $(Y,\Tau,\nu)$ be
measure spaces, and
$\lambda$ the c.l.d.\ product measure on $X\times Y$.   Let $f$ be a
$\lambda$-integrable real-valued function such that
$\int_{E\times F}f=0$ whenever $E\in\Sigma$, $F\in\Tau$, $\mu E<\infty$
and $\nu F<\infty$.   Show that $f=0\,\,\lambda$-a.e.   \Hint{use 251Ie
to show that $\int_Wf=0$ whenever $\lambda W<\infty$.}
%252B

\spheader 252Xb Let $f$, $g:\Bbb R\to\Bbb R$ be two
non-decreasing functions, and $\mu_f$, $\mu_g$ the associated
Lebesgue-Stieltjes measures (see 114Xa).   Set

\Centerline{$f(x^+)=\lim_{t\downarrow x}f(t)$,
\quad$f(x^-)=\lim_{t\uparrow x}f(t)$}

\noindent for each $x\in\Bbb R$, and define $g(x^+)$, $g(x^-)$
similarly.   Show that whenever $a\le b$ in $\Bbb R$,

$$\eqalign{\int_{[a,b]}f(x^-)\mu_g(dx)&+\int_{[a,b]}g(x^+)\mu_f(dx)
=g(b^+)f(b^+)-g(a^-)f(a^-)\cr
&=\int_{[a,b]}\Bover12(f(x^-)+f(x^+))\mu_g(dx)
  +\int_{[a,b]}\Bover12((g(x^-)+g(x^+))\mu_f(dx).\cr}$$

\noindent \Hint{find two expressions for
$(\mu_f\times\mu_g)\{(x,y):a\le x<y\le b\}$.}
%252C

\sqheader 252Xc Let $(X,\Sigma,\mu)$ and $(Y,\Tau,\nu)$ be
complete locally determined measure spaces, $\lambda$ the c.l.d.\
product measure on $X\times Y$, and $\Lambda$ its domain.   Suppose that
$A\subseteq X$ and
$B\subseteq Y$.   Show that $A\times B\in\Lambda$ iff
either $\mu A=0$ or $\nu B=0$ or $A\in\Sigma$ and $B\in\Tau$.
\Hint{if $B$ is not negligible and $A\times B\in\Lambda$, take $H$ such
that $\nu H<\infty$ and $B\cap H$ is not negligible.   Then
$W=A\times(B\cap H)$ is measured by $\mu\times\nu_H$, where $\nu_H$ is
the subspace measure on $H$.   Now apply 252D to $\mu$, $\nu_H$ and $W$
to see that $A\in\Sigma$.}
%252D

\sqheader 252Xd Let $(X_1,\Sigma_1,\mu_1)$,
$(X_2,\Sigma_2,\mu_2)$, $(X_2,\Sigma_3,\mu_3)$ be three $\sigma$-finite
measure spaces, and $f$ a real-valued function defined almost everywhere
on $X_1\times X_2\times X_3$ and $\Lambda$-measurable, where $\Lambda$
is the domain of the product
measure described in 251W or 251Xg.   Show that if
$\iiint |f(x_1,x_2,x_3)|dx_1dx_2dx_3$ is defined in $\Bbb R$, then
$\iiint f(x_1,x_2,x_3)dx_2dx_3dx_1$ and
$\iiint f(x_1,x_2,x_3)dx_3dx_1dx_2$
exist and are equal.
%252H

\spheader 252Xe Give an example of strictly localizable measure spaces
$(X,\Sigma,\mu)$, $(Y,\Tau,\nu)$ and a $W\in\Sigma\tensorhat\Tau$ such
that $x\mapsto\nu W[\{x\}]$ is not $\Sigma$-measurable.   \Hint{in
252Kb, try $Y$ a proper subset of $[0,1]$.}
%252K

\sqheader 252Xf Set $f(x,y)=\sin(x-y)$ if $0\le y\le x\le y+2\pi$, $0$
for other $x$, $y\in\BbbR^2$.   Show that $\iint f(x,y)dx\,dy=0$ and
$\iint f(x,y)dy\,dx=2\pi$, taking all integrals with respect to Lebesgue
measure.
%252M

\sqheader 252Xg Set $f(x,y)=\Bover{x^2-y^2}{(x^2+y^2)^2}$ for $x$,
$y\in\ocint{0,1}$.   Show that
$\int_0^1\int_0^1f(x,y)dydx=\Bover{\pi}4$,
\discrversionA{\break}{}$\int_0^1\int_0^1f(x,y)dxdy=-\Bover{\pi}4$.
%252M

\sqheader 252Xh Let $(X,\Sigma,\mu)$ and $(Y,\Tau,\nu)$ be
measure spaces, and $f$ a $\Sigma\tensorhat\Tau$-measurable function
defined on a subset of $X\times Y$.   Show that $y\mapsto f(x,y)$ is
$\Tau$-measurable for every $x\in X$.
%252P

\spheader 252Xi Let $r\ge 1$ be an integer, and write $\beta_r$ for the
Lebesgue measure of the unit ball in $\BbbR^r$.   Set
$g_r(t)=r\beta_rt^{r-1}$ for $t\ge 0$, $\phi(x)=\|x\|$ for
$x\in\BbbR^r$.   (i) Writing $\mu_r$ for Lebesgue measure on
$\Bbb R^r$, show that $\mu_r\phi^{-1}[E]=\int_Er\beta_rt^{r-1}\mu_1(dt)$
for every Lebesgue measurable set $E\subseteq\coint{0,\infty}$.   ({\it
Hint\/}:  start with intervals $E$, noting from 115Xe that
$\mu_r\{x:\|x\|\le a\}=\beta_ra^r$ for $a\ge 0$, and progress to open
sets, negligible sets and general measurable sets.)  (ii) Using 235R,
show that

$$\eqalign{\int e^{-\|x\|^2/2}\mu_r(dx)
&=r\beta_r\int_0^{\infty}t^{r-1}e^{-t^2/2}\mu_1(dt)
=2^{(r-2)/2}r\beta_r\Gamma(\Bover{r}2)\cr
&=2^{r/2}\beta_r\Gamma(1+\Bover{r}2)
=(\sqrt{2}\Gamma(\Bover12))^r\cr}$$

\noindent where $\Gamma$ is the $\Gamma$-function (225Xh).   (iii) Show
that

\Centerline{$2\Gamma(\bover12)^2=2\beta_2\Gamma(2)
=2\beta_2\int_0^{\infty}te^{-t^2/2}dt=2\pi$,}

\noindent and hence that
$\beta_r=\Bover{\pi^{r/2}}{\Gamma(1+\bover{r}2)}$ and
$\int_{-\infty}^{\infty}e^{-t^2/2}dt=\sqrt{2\pi}$.
%252Q

\leader{252Y}{Further exercises (a)}
%\spheader 252Ya
Let $(X,\Sigma,\mu)$ be a measure space.   Show that the
following are equiveridical:  (i) the completion of $\mu$ is locally
determined;  (ii) the completion of $\mu$ coincides with the c.l.d.\
version of $\mu$;  (iii) whenever $(Y,\Tau,\nu)$ is a $\sigma$-finite
measure space and $\lambda$ the c.l.d.\ product measure on $X\times Y$
and $f$ is a function such that $\int fd\lambda$ is defined in
$[-\infty,\infty]$, then $\iint f(x,y)\nu(dy)\mu(dx)$ is defined and
equal to $\int fd\lambda$.
%252B

\spheader 252Yb Let $(X,\Sigma,\mu)$ be a measure space.   Show that the
following are equiveridical:  (i) $\mu$ has locally determined
negligible
sets;  (ii) whenever $(Y,\Tau,\nu)$ is a $\sigma$-finite measure
space and $\lambda$ the c.l.d.\ product measure on $X\times Y$, then
$\iint f(x,y)\nu(dy)\mu(dx)$ is defined and equal to $\int fd\lambda$
for any $\lambda$-integrable function $f$.
%252Ya, 252B

\spheader 252Yc Let $(X,\Sigma,\mu)$ and $(Y,\Tau,\nu)$ be measure
spaces, and $\lambda_0$
the primitive product measure on $X\times Y$ (251C).   Let $f$ be any
$\lambda_0$-integrable real-valued function.   Show that
$\iint f(x,y)\nu(dy)\mu(dx)=\int fd\lambda_0$.   \Hint{show that there
are sequences $\sequencen{G_n}$, $\sequencen{H_n}$ of sets of finite
measure such that $f(x,y)$ is defined and equal to $0$ for every
$(x,y)\in(X\times Y)\setminus\bigcup_{n\in\Bbb N}G_n\times H_n$.}
%252B

\spheader 252Yd Let $(X,\Sigma,\mu)$ and
$(Y,\Tau,\nu)$ be
measure spaces;  let $\lambda_0$ be the primitive product measure on
$X\times Y$, and $\lambda$ the c.l.d.\ product measure.   Show that  if
$f$ is a $\lambda_0$-integrable real-valued function, it is
$\lambda$-integrable, and $\int fd\lambda=\int fd\lambda_0$.
%252B

\spheader 252Ye\dvAformerly{252Yj}
Let $(X,\Sigma,\mu)$ be a complete locally determined
measure space and $a<b$ in
$\Bbb R$, endowed with Lebesgue measure;  let $\Lambda$ be the domain of
the c.l.d.\ product measure $\lambda$ on $X\times[a,b]$.   Let
$f:X\times\ooint{a,b}\to\Bbb R$ be a $\Lambda$-measurable function such
that $t\mapsto f(x,t):[a,b]\to\Bbb R$ is continuous on $[a,b]$ and
differentiable on $\ooint{a,b}$ for every $x\in X$.   (i) Show that the
partial derivative $\Pd{f}{t}$ with respect to the second variable is
$\Lambda$-measurable.   (ii) Now suppose that $\Pd{f}{t}$ is
$\lambda$-integrable and that $\int f(x,t_0)\mu(dx)$ is defined and
finite for some $t_0\in\ooint{a,b}$.   Show that
$F(t)=\int f(x,t)\mu(dx)$ is defined in $\Bbb R$ for every $t\in[a,b]$,
that $F$ is absolutely continuous, and that
$F'(t)=\int\Pd{f}{t}(x,t)\mu(dx)$
for almost every $t\in\ooint{a,b}$.   ({\it Hint\/}:
$F(c)=F(a)+\int_{X\times[a,c]}\Pd{f}{t}d\lambda$ for every $c\in[a,b]$.)
%252C %225L

\spheader 252Yf Show that
$\Bover{\Gamma(a)\Gamma(b)}{\Gamma(a+b)}=\int_0^1t^{a-1}(1-t)^{b-1}dt$
for all $a$, $b>0$.   ({\it Hint\/}:  show that

\Centerline{$\int_0^{\infty}t^{a-1}\int_t^{\infty}
  e^{-x}(x-t)^{b-1}dxdt
=\int_0^{\infty}e^{-x}\int_0^xt^{a-1}(x-t)^{b-1}dtdx$.)}
%252C

\spheader 252Yg Let $(X,\Sigma,\mu)$ and $(Y,\Tau,\nu)$ be
$\sigma$-finite measure spaces and $\lambda$ the c.l.d.\ product measure
on $X\times Y$.   Suppose that
$f\in\eusm L^0(\lambda)$ and that $1<p<\infty$.   Show that
$(\int|\int f(x,y)dx|^pdy)^{1/p}\le\int(\int|f(x,y)|^pdy)^{1/p}dx$.
({\it Hint\/}:
set $q=\bover{p}{p-1}$ and consider the integral
$\int|f(x,y)g(y)|\lambda(d(x,y))$ for $g\in\eusm L^q(\nu)$, using 244K.)
%252C

\spheader 252Yh Let $\nu$ be Lebesgue measure on $\coint{0,\infty}$;
suppose that $f\in\eusm L^p(\nu)$ where $1<p<\infty$.   Set
$F(y)=\Bover1y\int_0^yf$ for $y>0$.   Show that
$\|F\|_p\le\Bover{p}{p-1}\|f\|_p$.   ({\it Hint\/}:
$F(y)=\int_0^1f(xy)dx$;  use 252Yg with $X=[0,1]$,
$Y=\coint{0,\infty}$.)
%252Yg, 252C

\spheader 252Yi Show that if $p$ is any non-zero (real) polynomial in
$r$ variables, then $\{x:x\in\BbbR^r,\,p(x)=0\}$ is Lebesgue negligible.
%252D

\spheader 252Yj Let $(X,\Sigma,\mu)$ and $(Y,\Tau,\nu)$ be measure
spaces, with c.l.d.\ product $(X\times Y,\Lambda,\lambda)$.   Let $f$ be
a non-negative $\Lambda$-measurable real-valued
function defined on a $\lambda$-conegligible set, and suppose that

\Centerline{$
\overline{\intop}\bigl(\overline{\intop}f(x,y)\mu(dx)\bigr)
\nu(dy)$}

\noindent is finite.   Show that $f$ is $\lambda$-integrable.
%252G

\spheader 252Yk Let $(X,\Sigma,\mu)$ be the unit interval $[0,1]$ with
Lebesgue measure, and $(Y,\Tau,\nu)$ the interval with counting measure,
as in 252K;  let $\lambda_0$ be the primitive product measure on
$[0,1]^2$.   (i) Setting $\Delta=\{(t,t):t\in[0,1]\}$, show that
$\lambda_0\Delta=\infty$.   (ii) Show that
$\lambda_0$ is not semi-finite.   (iii) Show that if
$W\in\dom\lambda_0$, then $\lambda_0W=\sum_{y\in[0,1]}\mu W^{-1}[\{y\}]$
if there are a countable set $A\subseteq[0,1]$ and a Lebesgue negligible
set $E\subseteq[0,1]$ such that
$W\subseteq([0,1]\times A)\cup(E\times[0,1])$, $\infty$ otherwise.
%252K

\spheader 252Yl Let $(X,\Sigma,\mu)$ be a measure space, and
$\lambda_0$ the primitive product measure on
$X\times\Bbb R$, where $\Bbb R$ is given Lebesgue measure;  write
$\Lambda$ for its domain.   For any
$[0,\infty]$-valued function $f$ defined on a conegligible subset of
$X$, write $\Omega_f$, $\Omega'_f$ for the corresponding ordinate sets,
as in 252N.   Show that if any of $\lambda_0\Omega_f$,
$\lambda_0\Omega'_f$,
$\int fd\mu$ is defined and finite, so are the others, and all three are
equal.
%252B, 252N

\spheader 252Ym Let $(X,\Sigma,\mu)$ be a complete locally determined
measure space, and $f$ a non-negative function defined on a conegligible
subset of $X$.   Write $\Omega_f$, $\Omega'_f$ for the corresponding
ordinate sets, as in 252N.   Let
$\lambda$ be the c.l.d.\ product measure on $X\times\Bbb R$, where
$\Bbb R$ is given Lebesgue measure.   Show that
$\overline{\int}f\,d\mu=\lambda^*\Omega_f=\lambda^*\Omega'_f$.
%252N, 252B

\spheader 252Yn Let $(X,\Sigma,\mu)$ be a measure space and
$f:X\to\coint{0,\infty}$ a function.   Show that
$\overline{\int}fd\mu=\int_0^{\infty}\mu^*\{x:f(x)\ge t\}dt$.
%252O

\spheader 252Yo Let $(X,\Sigma,\mu)$ be a complete measure space and
write $\eusm M^{0,\infty}$ for the set
$\{f:f\in\eusm L^0(\mu),\,\mu\{x:|f(x)|\ge a\}$ is finite for some
$a\in\coint{0,\infty}\}$.   (i) Show that for each
$f\in\eusm M^{0,\infty}$
there is a non-increasing $f^*:\ooint{0,\infty}\to\Bbb R$ such that
$\mu_L\{t:f^*(t)\ge\alpha\}=\mu\{x:|f(x)|\ge\alpha\}$ for
every $\alpha>0$, writing $\mu_L$ for Lebesgue measure.   (ii) Show that
$\int_E|f|d\mu\le\int_0^{\mu E}f^*d\mu_L$ for every $E\in\Sigma$
(allowing $\infty$).   ({\it Hint\/}:  $(f\times\chi E)^*\le f^*$.)
(iii) Show that $\|f^*\|_p=\|f\|_p$ for every $p\in[1,\infty]$,
$f\in\eusm M^{0,\infty}$.   ({\it
Hint\/}:  $(|f|^p)^*=(f^*)^p$.)   (iv) Show that if $f$,
$g\in\eusm M^{0,\infty}$ then
$\int|f\times g|d\mu\le\int f^*\times g^*d\mu_L$.   ({\it
Hint\/}:  look at simple functions first.)   (v) Show that if $\mu$ is
atomless then $\int_0^af^*d\mu_L=\sup_{E\in\Sigma,\mu E\le a}\int_E|f|$
for every $a\ge 0$.   ({\it Hint\/}:  215D.)   (vi) Show that
$A\subseteq\eusm L^1(\mu)$ is uniformly integrable iff $\{f^*:f\in A\}$
is uniformly integrable in $\eusm L^1(\mu_L)$.   ($f^*$ is called the
{\bf decreasing rearrangement} of $f$.)
%252O

\spheader 252Yp Let $(X,\Sigma,\mu)$ be a complete locally
determined measure space, and write $\nu$ for Lebesgue measure on
$[0,1]$.   Show
that the c.l.d.\ product measure $\lambda$ on $X\times[0,1]$ is
localizable iff $\mu$ is localizable.   ({\it Hints\/}: (i) if
$\Cal E\subseteq\Sigma$, show that $F\in\Sigma$ is an essential supremum
for $\Cal E$ in $\Sigma$ iff $F\times[0,1]$ is an essential supremum for
$\{E\times[0,1]:E\in\Cal E\}$ in $\Lambda=\dom\lambda$.   (ii) For
$W\in\Lambda$, $n\in\Bbb N$, $k<2^n$ set

\Centerline{$W_{nk}
=\{x:x\in X,\,\nu^*\{t:(x,t)\in W,\,2^{-n}k\le t\le 2^{-n}(k+1)\}
\ge 2^{-n-1}\}$.}

\noindent Show that if $\Cal W\subseteq\Lambda$ and $F_{nk}$ is an
essential supremum for $\{W_{nk}:W\in\Cal W\}$ in $\Sigma$ for all $n$,
$k$, then

\Centerline{$\bigcup_{n\in\Bbb N}\bigcap_{m\ge n}
\bigcup_{k<2^{m}}F_{mk}\times[2^{-m}k,2^{-m}(k+1)]$}

\noindent is an essential supremum for $\Cal W$ in $\Lambda$.)
%252O

\spheader 252Yq Let $(X,\Sigma,\mu)$ be the space of Example
216D, and give Lebesgue measure to $[0,1]$.   Show that the c.l.d.\
product measure on $X\times [0,1]$ is complete, locally determined,
atomless and not localizable.
%252Yp 252O

\spheader 252Yr Let $(X,\Sigma,\mu)$ be a complete locally
determined measure space and $(Y,\Tau,\nu)$ a semi-finite measure space
with $\nu Y>0$.   Show that if the c.l.d.\ product measure on
$X\times Y$ is strictly localizable, then $\mu$ is strictly localizable.
({\it Hint\/}:  take $F\in\Tau$, $0<\nu F<\infty$.   Let
$\langle W_i\rangle_{i\in I}$ be a decomposition of $X\times Y$.   For
$i\in I$, $n\in\Bbb N$
set $E_{in}=\{x:\nu^*\{y:y\in F,\,(x,y)\in W_i\}\ge 2^{-n}\}$.   Apply
213Yf to $\{E_{in}:i\in I,\,n\in\Bbb N\}$.)
%252Yq 252O

\spheader 252Ys Let $(X,\Sigma,\mu)$ be the space of Example
216E, and give Lebesgue measure to $[0,1]$.   Show that the c.l.d.\
product measure on $X\times[0,1]$ is complete, locally determined,
atomless and localizable, but not strictly localizable.
%252Yr, 252O

\spheader 252Yt Let $(X,\Sigma,\mu)$ be a measure space and $f$ a
$\mu$-integrable complex-valued function.   For
$\alpha\in\ocint{-\pi,\pi}$
set $H_{\alpha}=\{x:x\in\dom f,\,\Real(e^{-i\alpha}f(x))>0\}$.   Show
that $\int_{-\pi}^{\pi}\Real(e^{-i\alpha}\int_{H_{\alpha}}f)d\alpha
\discrversionA{\break}{}=2\int|f|$, and hence that there is some
$\alpha$ such that
$|\int_{H_{\alpha}}f|\ge\Bover1{\pi}\int|f|$.   (Compare 246F.)
%246F, 252R

\spheader 252Yu Set $f(t)=t-\ln(t+1)$ for $t>-1$.   (i) Show that
$\Gamma(a+1)=a^{a+1}e^{-a}\int_{-1}^{\infty}e^{-af(u)}du$ for every
$a>0$.   \Hint{substitute $u=\bover{t}{a}-1$ in 225Xh(iii).}   (ii) Show
that there is a $\delta>0$ such that $f(t)\ge\bover13t^2$ for
$-1\le t\le\delta$.   (iii)
Setting $\alpha=\bover12f(\delta)$, show that (for $a\ge 1$)

\Centerline{$\sqrt{a}\int_{\delta}^{\infty}e^{-af(t)}dt
\le\sqrt{a}e^{-a\alpha}\int_0^{\infty}e^{-f(t)/2}dt\to 0$}

\noindent as $a\to\infty$.   (iv) Set $g_a(t)=e^{-af(t/\sqrt{a})}$ if
$-\sqrt{a}<t\le\delta\sqrt{a}$, $0$ otherwise.   Show that
$g_a(t)\le e^{-t^2/3}$ for all $a$, $t$ and that
$\lim_{a\to\infty}g_a(t)=e^{-t^2/2}$ for all $t$, so that

$$\eqalign{\lim_{a\to\infty}\Bover{e^a\Gamma(a+1)}{a^{a+\bover12}}
&=\lim_{a\to\infty}\sqrt{a}\int_{-1}^{\infty}e^{-af(t)}dt
=\lim_{a\to\infty}\sqrt{a}\int_{-1}^{\delta}e^{-af(t)}dt\cr
&=\lim_{a\to\infty}\int_{-\infty}^{\infty}g_a(t)dt
=\int_{-\infty}^{\infty}e^{-t^2/2}dt=\sqrt{2\pi}.\cr}$$

\noindent (v) Show that $\lim_{n\to\infty}\Bover{n!}{e^{-n}n^n\sqrt{n}}
=\sqrt{2\pi}$.   (This is {\bf Stirling's formula}.)
%252+

\spheader 252Yv Let $(X,\Sigma,\mu)$ be a complete locally determined
measure space and $f$, $g$ two real-valued, $\mu$-virtually measurable
functions defined almost everywhere in $X$.   (i) Let $\lambda$ be the
c.l.d.\ product of $\mu$ and Lebesgue measure on $\Bbb R$.
Setting $\Omega^*_f=\{(x,a):x\in\dom f$, $a\in\Bbb R$, $a\le f(x)\}$ and
$\Omega^*_g=\{(x,a):x\in\dom g$, $a\in\Bbb R$, $a\le g(x)\}$, show that
$\lambda(\Omega^*_f\setminus\Omega^*_g)=\int(f-g)^+d\mu$ and
$\lambda(\Omega^*\symmdiff\Omega^*_g)=\int|f-g|d\mu$.
(ii) Suppose that $\mu$ is $\sigma$-finite.   Show that

\Centerline{$\int|f-g|d\mu
=\int_{-\infty}^{\infty}
\mu(\{x:x\in\dom f\cap\dom g$, $(f(x)-a)(g(x)-a)<0\}da$.}

\noindent(iii) Suppose that $\mu$ is $\sigma$-finite, that $\Tau$ is a
$\sigma$-subalgebra of $\Sigma$, that $E\in\Sigma$ and that $g:X\to[0,1]$
is $\Tau$-measurable.   Show that there is an $F\in\Tau$ such that
$\mu(E\symmdiff F)\le\int|\chi E-g|d\mu$.
%252N
}%end of exercises

\endnotes{
\Notesheader{252} For a volume and a half now I have asked you to accept
the idea of integrating partially-defined functions, insisting that
sooner or later they would appear at the core of the subject.   The
moment has now come.   If we wish to apply Fubini's and Tonelli's
theorems in the most fundamental of all cases, with both factors equal
to Lebesgue measure on the unit interval, it is surely natural to
look at all functions which are integrable on the square for
two-dimensional Lebesgue measure.   Now two-dimensional Lebesgue measure
is a complete measure, so, in particular, assigns zero measure to any
set of the form $\{(x,b):x\in A\}$ or $\{(a,y):y\in A\}$, whether or not
the set $A$ is measured by one-dimensional measure.   Accordingly, if
$f$ is a function of two variables which is integrable for
two-dimensional Lebesgue measure, there is no reason why any particular
section $x\mapsto f(x,b)$ or $y\mapsto f(a,y)$ should be measurable, let
alone integrable.   Consequently, even if $f$ itself is defined
everywhere, the outer integral of $\iint f(x,y)dxdy$ is likely to be
applied to a function which is not defined for every $y$.
Let me remark that the problem does not concern `$\infty$';  the
awkward functions are those with sections so irregular that they cannot
be assigned an integral at all.

I have seen many approaches to this particular nettle, generally less
whole-hearted than the one I have determined on for this treatise.
Part of the
difficulty is that Fubini's theorem really is at the centre of measure
theory.   Over large parts of the subject, it is possible to assert that
a result is non-trivial if and only if it depends on Fubini's theorem.
I am therefore unwilling to insert any local fix, saying that `in
this chapter, we shall integrate functions which are not defined
everywhere';  before long, such a provision would have to be
interpolated into the preambles to half the best theorems, or an
explanation offered of why it wasn't necessary in their particular
contexts.   I suppose that one of the commonest responses is (like
{\smc Halmos 50}) to restrict attention to
$\Sigma\tensorhat\Tau$-measurable
functions, which eliminates measurability problems for the moment
(252Xh, 252P);   but
unhappily (or rather, to my mind, happily) there are crucial
applications in which the functions are not actually
$\Sigma\tensorhat\Tau$-measurable, but belong to some wider class, and
this restriction sooner or later leads to undignified contortions as we
are forced to adapt
limited results to unforeseen contexts.   Besides, it leaves unsaid the
really rather important information that if $f$ is a measurable function
of two variables then (under appropriate conditions) almost all its
sections are measurable (252E).

In 252B and its corollaries there is a clumsy restriction:  we assume
that one of the measures is $\sigma$-finite and the other is either
strictly localizable or complete and locally determined.   The obvious
question is, whether we need these hypotheses.   From 252K we see that
the hypothesis `$\sigma$-finite' on the second factor can certainly not
be abandoned, even when the first factor is a complete probability
measure.   The requirement `$\mu$ is either strictly localizable or
complete and locally determined' is in fact fractionally stronger than
what is needed, as well as disagreeably elaborate.   The `right'
hypothesis is that the completion of $\mu$ should be locally determined
(see 252Ya).   The point is that because the
product of two measures is the same as the product of their c.l.d.\
versions (251T), no theorem which leads from the product measure to the
factor measures can distinguish between a measure and its c.l.d.\
version;  so that, in 252B, we must expect to need $\mu$ and its c.l.d.\
version to give rise to the same integrals.   The proof of 252B would be
better focused if the hypothesis was simplified to `$\nu$ is
$\sigma$-finite and $\mu$ is complete and locally determined'.   But
this would just transfer part of the argument into the proof of 252C.

We also have to work a little harder in 252B in order to cover functions
and integrals taking the values $\pm\infty$.   Fubini's theorem is so
central to measure theory that I believe it is worth taking a bit of
extra trouble to state the results in maximal generality.   This is
especially important because we frequently apply it in multiply repeated
integrals, as in 252Xd, in which we have even less control than usual
over the intermediate functions to be integrated.

I have expressed all the main results of this section in terms of the
`c.l.d.'\ product measure.   In the case of $\sigma$-finite spaces, of
course, which is where the theory works best, we could just as well use
the `primitive' product measure.   Indeed, Fubini's theorem itself has a
version in terms of the primitive product measure which is rather more
elegant than 252B as stated (252Yc), and covers the great majority of
applications.   (Integrals with respect to the primitive and c.l.d.\
product measures are of course very closely related;  see 252Yd.)   But
we do sometimes need to look at
non-$\sigma$-finite spaces, and in these cases the asymmetric form in
252B is close to the best we can do.   Using the primitive product
measure does not help at all with the most substantial obstacle, the
phenomenon in 252K (see 252Yk).

The pre-calculus concept of an integral as `the area under a curve' is
given expression in 252N:  the integral of a non-negative function is
the measure of its ordinate set.   This is unsatisfactory as a
definition of the integral, not just because of the requirement that the
base space should be complete and locally determined (which can be dealt
with by using the primitive product measure, as in 252Yl), but because
the construction of the product measure involves integration (part (c)
of the proof of 251E).
The idea of 252N is to relate the measure of an ordinate set to the
integral of the measures of its vertical sections.   Curiously, if
instead we integrate the measures of its {\it horizontal} sections, as
in 252O, we get a more versatile result.   (Indeed this one does not
involve the concept of `product measure', and could have appeared at any
point after \S123.)   Note that the integral
$\int_0^{\infty}\ldots dt$ here is applied to a monotonic function, so
may be interpreted as an improper Riemann integral.   If you think you
know enough about the Riemann integral to
make this a tempting alternative to the construction in \S122, the
tricky bit now becomes the proof that the integral is additive.

A different line of argument is to use integration over sections to
define a product measure.   The difficulty with this approach is that
unless we take great care we may find ourselves with an asymmetric
construction.   My own view is that such an asymmetry is acceptable only
when there is no alternative.   But in Chapter 43 of Volume 4 I will
describe a couple of examples.

Of the two examples I give here, 252K is supposed to show that when I
call for $\sigma$-finite spaces they are really necessary, while 252L is
supposed to show that joint measurability is essential in Tonelli's
theorem and its corollaries.   The factor spaces in 252K, Lebesgue
measure and counting measure, are chosen to show that it is only the
lack of $\sigma$-finiteness that can be the problem;  they are otherwise
as regular as one can reasonably ask.   In 252L I have used the
countable-cocountable measure on $\omega_1$, which you may feel is fit
only for counter-examples;  and the question does arise, whether the
same phenomenon occurs with Lebesgue measure.   This leads into deep
water, and I will return to it in Chapter 53 of Volume 5.

I ought perhaps to note explicitly that in Fubini's theorem, we really
do need to have a function which is integrable for the product measure.
I include 252Xf and 252Xg to remind you that even in the best-regulated
circumstances,
the repeated integrals $\iint f\,dxdy$, $\iint f\,dydx$ may fail to be
equal if $f$ is not integrable as a function of two variables.

There are many ways to calculate the volume $\beta_r$ of an
$r$-dimensional ball;  the one I have used in 252Q follows a line that
would have been natural to me before I ever heard of measure theory.
In 252Xi I suggest another method.   The idea of
integration-by-substitution, used in part (b) of the argument for 252Q,
is there
supported by an ad hoc argument;  I will present a different, more
generally applicable, approach in Chapter 26.   Elsewhere (252Xi, 252Yf,
252Yh, 252Yu) I find myself taking for granted substitutions of the form
$t\mapsto at$, $t\mapsto a+t$, $t\mapsto t^2$;
for a systematic justification, see
\S263.   Of course an enormous number of other formulae of advanced
calculus are also based on repeated integration of one kind or another,
and I give a sample handful of such results (252Xb,
252Ye-252Yh, %252Ye 252Yf 252Yg 252Yh
252Yu).

}%end of notes

\discrpage

\leaveitout{%\spheader 252Y?
Let $(X,\Sigma,\mu)$ and $(Y,\Tau,\nu)$ be measure spaces, and
$\lambda_0$
the primitive product measure on $X\times Y$.   Let $f$ be a real-valued
function defined $\lambda_0$-a.e.\ on $X\times Y$.   Show that

\Centerline{$
\overline{\intop}\bigl(\overline{\intop}f(x,y)\mu(dx)\bigr)\nu(dy)
\le\overline{\intop}f\,d\lambda_0$.}
%252B
}%end of leaveitout

