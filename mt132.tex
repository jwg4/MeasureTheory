\frfilename{mt132.tex}
\versiondate{6.4.05}
\copyrightdate{1995}

\def\chaptername{Complements}
\def\sectionname{Outer measures from measures}

\newsection{132}

The next topic I wish to mention is a simple construction with
applications everywhere in measure theory.   With any measure there is
associated, in a straightforward way, a standard outer measure
(132A-132B).   If we start with Lebesgue measure we just return to
Lebesgue outer measure (132C).   I take the opportunity to
introduce the idea of `measurable envelope' (132D-132E).

\leader{132A}{Proposition} Let $(X,\Sigma,\mu)$ be a measure space.
Define $\mu^*:\Cal PX\to[0,\infty]$ by writing

\Centerline{$\mu^*A=\inf\{\mu E:E\in\Sigma,\,A\subseteq E\}$}

\noindent for every $A\subseteq X$.   Then

(a) for every $A\subseteq X$
there is an $E\in\Sigma$ such that $A\subseteq E$ and $\mu^*A=\mu E$;

(b) $\mu^*$ is an outer measure on $X$;

(c) $\mu^*E=\mu E$ for every $E\in\Sigma$;

(d) a set $A\subseteq X$ is $\mu$-negligible iff $\mu^*A=0$;

(e) $\mu^*(\bigcup_{n\in\Bbb N}A_n)=\lim_{n\to\infty}\mu^*A_n$ for
every non-decreasing sequence $\sequencen{A_n}$ of subsets of $X$;

(f) $\mu^*A=\mu^*(A\cap F)+\mu^*(A\setminus F)$ whenever
$A\subseteq X$ and $F\in\Sigma$.

\proof{{\bf (a)} For each $n\in\Bbb N$ we
can choose an $E_n\in\Sigma$ such that $A\subseteq E_n$ and
$\mu E_n\le\mu^*A+2^{-n}$;  now $E=\bigcap_{n\in\Bbb N}E_n\in\Sigma$,
$A\subseteq E$ and

\Centerline{$\mu^*A\le\mu E\le\inf_{n\in\Bbb N}\mu E_n\le\mu^*A$.}

\medskip

{\bf (b)}(i) $\mu^*\emptyset=\mu\emptyset=0$.   (ii) If
$A\subseteq B\subseteq X$ then $\{E:A\subseteq
E\in\Sigma\}\supseteq\{E:B\subseteq E\in\Sigma\}$ so
$\mu^*A\le\mu^*B$.
(iii) If $\sequencen{A_n}$ is any sequence in $\Cal PX$, then for
each $n\in\Bbb N$ there is an $E_n\in\Sigma$ such that
$A_n\subseteq E_n$ and $\mu E_n=\mu^*A_n$;  now
$\bigcup_{n\in\Bbb N}A_n\subseteq\bigcup_{n\in\Bbb N}E_n\in\Sigma$ so

\Centerline{$\mu^*(\bigcup_{n\in\Bbb N}A_n)
\le\mu(\bigcup_{n\in\Bbb N}E_n)
\le\sum_{n=0}^{\infty}\mu E_n=\sum_{n=0}^{\infty}\mu^*A_n$.}

\medskip

{\bf (c)} This is just because $\mu E\le\mu F$ whenever $E$,
$F\in\Sigma$ and $E\subseteq F$.

\medskip

{\bf (d)} By (a), $\mu^*A=0$ iff there is an $E\in\Sigma$ such
that $A\subseteq E$ and $\mu E=0$;  but this is the definition of
`negligible set'.

\medskip

{\bf (e)} Of
course $\sequencen{\mu^*A_n}$ is a non-decreasing sequence with limit
at most $\mu^*A$, writing $A=\bigcup_{n\in\Bbb N}A_n$, just because
$\mu^*B\le\mu^*C$ whenever $B\subseteq C\subseteq X$.    For each
$n\in\Bbb N$, let $E_n\in\Sigma$ be such that $A_n\subseteq E_n$ and
$\mu E_n=\mu^*A_n$.   Set $F_n=\bigcap_{m\ge n}E_m$ for each $n$;
then $\sequencen{F_n}$ is a non-decreasing sequence in $\Sigma$, and
$A_n\subseteq F_n\subseteq E_n$, so $\mu^*A_n=\mu F_n$ for each
$n\in\Bbb N$.   Set $F=\bigcup_{n\in\Bbb N}F_n$;  then $A\subseteq F$
so

\Centerline{$\mu^*A\le\mu F=\lim_{n\to\infty}\mu
F_n=\lim_{n\to\infty}\mu^*A_n$.}

\noindent Thus $\mu^*A=\lim_{n\to\infty}\mu^*A_n$, as claimed.

\medskip

{\bf (f)} Of course $\mu^*A\le\mu^*(A\cap F)+\mu^*(A\setminus F)$, by
(b).   On the other hand, there is an $E\in\Sigma$ such that
$A\subseteq E$ and $\mu E=\mu^*A$, by (a), and now
$A\cap F\subseteq E\cap F\in\Sigma$,
$A\setminus F\subseteq E\setminus F\in\Sigma$ so

\Centerline{$\mu^*(A\cap F)+\mu^*(A\setminus F)
\le\mu(E\cap F)+\mu(E\setminus F)=\mu E=\mu^*A$.}

}%end of proof of 132A

\leader{132B}{Definition} If $(X,\Sigma,\mu)$ is a measure space, I
will call $\mu^*$, as defined in 132A, {\bf the outer measure
defined from $\mu$}.

\cmmnt{\medskip

\noindent{\bf Remark} If we start with an outer measure $\theta$ on a
set $X$, construct a measure $\mu$ from $\theta$ by \Caratheodory's
method, and
then construct the outer measure $\mu^*$ from $\mu$, it is not
necessarily the case that $\mu^*=\theta$.   \prooflet{\Prf\ Take any
set $X$ with at least three members, and set $\theta A=0$ if
$A=\emptyset$, $1$ if $A=X$, $\bover12$ otherwise.
Then $\dom\mu=\{\emptyset,X\}$ and
$\mu^*A=1$ for every non-empty $A\subseteq X$.\ \Qed}

However, this problem does not arise with Lebesgue outer measure.   I
state the next proposition in terms of Lebesgue measure on $\BbbR^r$,
but if you skipped \S115 I hope that you will still be able to make
sense of this, and later results, in terms of Lebesgue measure on
$\Bbb R$, by setting $r=1$.
}%end of comment

\leader{132C}{Proposition} If $\theta$ is Lebesgue outer
measure on $\BbbR^r$
and $\mu$ is Lebesgue measure, then $\mu^*$, as defined in 132A, is
equal to $\theta$.

\proof{ Let $A\subseteq\BbbR^r$.

\medskip

{\bf (a)} If $E$ is measurable and
$A\subseteq E$, then $\theta A\le\theta E=\mu E$;  so
$\theta A\le\mu^*A$.

\medskip

{\bf (b)} If $\epsilon>0$, there is a sequence
$\sequencen{I_n}$ of half-open intervals, covering $A$, with
$\sum_{n=0}^{\infty}\mu I_n\le\theta A+\epsilon$ (using 114G/115G to
identify $\mu I_n$ with the volume $\lambda I_n$ used in the
definition of $\theta$), so

\Centerline{$\mu^*A\le\mu(\bigcup_{n\in\Bbb N}I_n)
\le\sum_{n=0}^{\infty}\mu I_n
\le\theta A+\epsilon$.}

\noindent As $\epsilon$ is arbitrary, $\mu^*A\le\theta A$.
}%end of proof of 132C

\cmmnt{\medskip

\noindent{\bf Remark}  Accordingly it will henceforth be unnecessary
to distinguish
$\theta$ from $\mu^*$ when speaking of `Lebesgue outer measure'.
(In the language of 132Xa below, Lebesgue outer measure is
`regular'.)   In
particular (using 132Aa), if $A\subseteq\BbbR^r$ there is a measurable
set $E\supseteq A$ such that $\mu E=\theta A$ (compare 134Fc).
}%end of comment

\leader{132D}{Measurable envelopes}
\cmmnt{The following is a useful concept in
this context.}   If $(X,\Sigma,\mu)$ is a measure space and
$A\subseteq X$, a {\bf measurable envelope} (or {\bf measurable
cover}) of $A$ is a set $E\in\Sigma$ such
that $A\subseteq E$ and $\mu(F\cap E)=\mu^*(F\cap A)$ for every
$F\in\Sigma$.   \cmmnt{In general, not every set in a measure space
has a measurable envelope (I suggest examples in 216Yc in Volume 2).
But we do have the following.}

\leader{132E}{Lemma} Let $(X,\Sigma,\mu)$ be a measure space.

(a) If $A\subseteq E\in\Sigma$, then $E$ is a measurable envelope of
$A$ iff $\mu F=0$ whenever $F\in\Sigma$ and $F\subseteq E\setminus A$.

(b) If $A\subseteq E\in\Sigma$ and $\mu E<\infty$ then $E$ is a
measurable envelope of $A$ iff $\mu E=\mu^*A$.

(c) If $E$ is a measurable envelope of $A$ and $H\in\Sigma$,
then $E\cap H$ is a measurable envelope of $A\cap H$.

(d) Let $\sequencen{A_n}$ be a sequence of subsets of $X$.   Suppose
that each $A_n$ has a measurable envelope $E_n$.   Then
$\bigcup_{n\in\Bbb N}E_n$ is
a measurable envelope of $\bigcup_{n\in\Bbb N}A_n$.

(e) If $A\subseteq X$ can be covered by a sequence of sets of finite
measure, then $A$ has a measurable envelope.

(f) In particular, if $\mu$ is Lebesgue measure on $\BbbR^r$, then
every subset of $\BbbR^r$ has a measurable envelope for $\mu$.

\proof{{\bf (a)} If $E$ is a measurable envelope of $A$, $F\in\Sigma$
and $F\subseteq E\setminus A$, then

\Centerline{$\mu F=\mu(F\cap E)=\mu^*(F\cap A)=0$.}

\noindent If $E$ is not a measurable envelope of $A$, there is an
$H\in\Sigma$ such that $\mu^*(A\cap H)<\mu(E\cap H)$.   Let
$G\in\Sigma$
be such that $A\cap H\subseteq G$ and $\mu G=\mu^*(A\cap H)$, and set
$F=E\cap H\setminus G$.   Since $\mu G<\mu(E\cap H)$, $\mu F>0$;   but
also $F\subseteq E$ and $F\cap A\subseteq H\cap A \setminus G$ is
empty.

\medskip

{\bf (b)} If $E$ is a measurable envelope of $A$ then we must have

\Centerline{$\mu^*A=\mu^*(A\cap E)=\mu(E\cap E)=\mu E$.}

\noindent If $\mu E=\mu^*A$, and $F\in\Sigma$ is a subset of
$E\setminus A$, then $A\subseteq E\setminus F$, so
$\mu(E\setminus F)=\mu E$;  because $\mu E$ is finite, it follows that
$\mu F=0$, so the condition of (a) is satisfied and $E$ is a
measurable envelope of $A$.

\medskip

{\bf (c)} If $F\in\Sigma$ and $F\subseteq E\cap H\setminus A$, then
$F\subseteq E\setminus A$, so $\mu F=0$, by (a);  as $F$ is arbitrary,
$E\cap H$ is a measurable envelope of $A\cap H$, by (a) again.

\medskip

{\bf (d)} Write $A$ for $\bigcup_{n\in\Bbb N}A_n$ and
$E$ for $\bigcup_{n\in\Bbb N}E_n$.
Then $A\subseteq E$.   If $F\in\Sigma$ and $F\subseteq E\setminus A$,
then, for every $n\in\Bbb N$, $F\cap E_n\subseteq E_n\setminus A_n$, so
$\mu(F\cap E_n)=0$, by (a).   Consequently
$F=\bigcup_{n\in\Bbb N}F\cap E_n$ is negligible;  as $F$ is arbitrary,
$E$ is a measurable envelope of $A$.

\medskip

{\bf (e)} Let $\sequencen{F_n}$ be a sequence of sets
of finite measure covering $A$.   For each $n\in\Bbb N$, let
$E_n\in\Sigma$ be such that $A\cap F_n\subseteq E_n$ and
$\mu E_n=\mu^*(A\cap F_n)$ (using 132Aa above);  by (b), $E_n$ is a
measurable envelope of $A\cap F_n$.   By (d), $\bigcup_{n\in\Bbb
N}E_n$ is a measurable envelope of $\bigcup_{n\in\Bbb N}A\cap F_n=A$.

\medskip

{\bf (f)} In the case of Lebesgue measure on $\BbbR^r$, of course, the
same sequence $\sequencen{B_n}$ will work for every $A$, if we take
$B_n$ to be the half-open interval $\coint{-\tbf{n},\tbf{n}}$ for each
$n\in\Bbb N$, writing $\tbf{n}=(n,\ldots,n)$ as in \S115.
}%end of proof of 132E

\leader{132F}{Full outer measure}\cmmnt{ This is a convenient
moment at which to introduce the following term.}  If $(X,\Sigma,\mu)$
is a measure space, a set $A\subseteq X$ is {\bf of full outer
measure}
or {\bf thick} if $X$ is a measurable envelope of $A$\cmmnt{;  that
is, if $\mu^*(F\cap A)=\mu F$ for every $F\in\Sigma$;  equivalently,
if $\mu F=0$ whenever $F\in\Sigma$ and $F\subseteq X\setminus A$.   If
$\mu X<\infty$, $A\subseteq X$ has full outer measure iff
$\mu^*A=\mu X$}.

\exercises{
\leader{132X}{Basic exercises $\pmb{>}$(a)}
%\spheader 132Xa
Let $X$ be a set and $\theta$ an outer measure on $X$;
let $\mu$ be the measure on $X$ defined by \Caratheodory's method from
$\theta$, and $\mu^*$ the outer measure defined from $\mu$ by the
construction of 132A.   (i) Show that $\mu^*A\ge\theta A$ for every
$A\subseteq X$.   (ii) $\theta$ is said to be a {\bf regular outer
measure} if $\theta=\mu^*$.
%Rao `Carathe\'odory regular'
Show that if there is any measure $\nu$
on $X$ such that $\theta=\nu^*$ then $\theta$ is regular.   (iii) Show
that if $\theta$ is regular and $\sequencen{A_n}$ is a non-decreasing
sequence of subsets of $X$, then
$\theta(\bigcup_{n\in\Bbb N}A_n)=\lim_{n\to\infty}\theta A_n$.
%132B

\spheader 132Xb Let $(X,\Sigma,\mu)$ be a measure space and $H$ any
member of $\Sigma$.   Let $\mu_H$ be the subspace measure on $H$
(131B) and $\mu^*$, $\mu_H^*$ the outer measures defined from $\mu$,
$\mu_H$ respectively.   Show that $\mu_H^*=\mu^*\restrp\Cal PH$.
%132B

\spheader 132Xc Give an example of a measure space $(X,\Sigma,\mu)$
such that the measure $\check\mu$ defined by \Caratheodory's method from
the outer measure $\mu^*$ is a proper extension of $\mu$.   \Hint{take
$\mu X=0$.}
%132B

\sqheader 132Xd Let $(X,\Sigma,\mu)$ be a measure space and $A$ a
subset of $X$.   Suppose that $\sequencen{E_n}$ is a sequence in $\Sigma$
such that $\sequencen{A\cap E_n}$ is disjoint.   Show that
$\mu^*(A\cap\bigcup_{n\in\Bbb N}E_n)=\sum_{n=0}^{\infty}\mu^*(A\cap E_n)$.
\Hint{replace $E_n$ by $E_n'=E_n\setminus\bigcup_{i<n}E_i$, and
use 132Ae-132Af.}
%132B

\spheader 132Xe Let $(X,\Sigma,\mu)$ be a measure space and
$\sequencen{A_n}$ any sequence of subsets of $X$.   Show that the
outer measure of $\bigcup_{n\in\Bbb N}\bigcap_{i\ge n}A_i$ is at most
$\liminf_{n\to\infty}\mu^*A_n$.
%132B

\spheader 132Xf Let $(X,\Sigma,\mu)$ be a measure space and suppose that
$A\subseteq B\subseteq X$ are such that $\mu^*A=\mu^*B<\infty$.   Show
that $\mu^*(A\cap E)=\mu^*(B\cap E)$ for every $E\in\Sigma$.   \Hint{a
measurable envelope of $B$ is a measurable envelope of $A$.}
%132E

\sqheader 132Xg Let $\nu_g$ be a Lebesgue-Stieltjes measure on $\Bbb R$,
constructed as in 114Xa from a non-decreasing function
$g:\Bbb R\to\Bbb R$.   Show that
(i) the outer measure $\nu_g^*$ derived from $\nu_g$ (132A)
coincides with the outer measure $\theta_g$ of 114Xa;
(ii) if $A\subseteq\Bbb R$ is any set, then $A$ has a
measurable envelope for the measure $\nu_g$.
%132E

\sqheader 132Xh Let $A\subseteq\BbbR^r$ be a set which is not
measured by Lebesgue measure $\mu$.   Show that there is a bounded
measurable set $E$ such that $\mu^*(E\cap A)=\mu^*(E\setminus A)=\mu E>0$.
\Hint{take $E=E'\cap E''\cap B$, where $E'$ is a measurable envelope
for $A$, $E''$ is a measurable envelope for $\BbbR^r\setminus A$, and
$B$ is a suitable bounded set.}
%132E

\spheader 132Xi Let $\mu$ be Lebesgue measure on $\BbbR^r$ and
$\Sigma$ its domain, and $f$ a real-valued function, defined on a subset
of $\BbbR^r$, which is not $\Sigma$-measurable.
Show that there are  $q<q'$ in $\Bbb Q$ and a bounded measurable set
$E$ such that

\Centerline{$\mu^*\{x:x\in E\cap\dom f,\,f(x)\le q\}
=\mu^*\{x:x\in E\cap\dom f,\,f(x)\ge q'\}=\mu E>0$.}

\noindent \Hint{take $E_q$, $E'_q$ to be measurable envelopes for
$\{x:f(x)\le q\}$, $\{x:f(x)>q\}$ for each $q$.   Find $q$ such that
$\mu(E_q\cap E'_q)>0$ and $q'$ such that $\mu(E_q\cap E'_{q'})>0$.}
%132E

\spheader 132Xj Check that you can do exercise 113Yc.
%132+

\spheader 132Xk Let $(X,\Sigma,\mu)$ be a measure space and $\mu^*$ the
outer measure defined from $\mu$.
Show that $\mu^*(A\cup B)+\mu^*(A\cap B)\le\mu^*A+\mu^*B$
for all $A$, $B\subseteq X$.
%132B

\leader{132Y}{Further exercises (a)}
%\spheader 132Ya
Let $(X,\Sigma,\mu)$ be a measure space and $\sequencen{f_n}$
a sequence of real-valued functions defined almost everywhere in $X$.
Suppose that $\sequencen{\epsilon_n}$ is a sequence of
non-negative real numbers such that

\Centerline{$\sum_{n=0}^{\infty}\epsilon_n<\infty$,
\quad
$\sum_{n=0}^{\infty}\mu^*\{x:|f_{n+1}(x)-f_n(x)|\ge\epsilon_n\}<\infty$.}

\noindent Show that $\lim_{n\to\infty}f_n$ is defined (as a
real-valued function) almost everywhere.
%132B

\spheader 132Yb Let $(X,\Sigma,\mu)$ be a measure space, $Y$ a set and
$f:X\to Y$ a function.   Let $\nu$ be the image measure $\mu f^{-1}$
(112Xf).   Show that $\nu^*f[A]\ge\mu^*A$ for every $A\subseteq X$.
%132B

\spheader 132Yc Let $(X,\Sigma,\mu)$ be a measure space with
$\mu X<\infty$.   Let $\sequencen{A_n}$ be a sequence of subsets of
$X$ such that $\bigcup_{n\in\Bbb N}A_n$ has full outer measure in $X$.
Show that there is a
partition $\sequencen{E_n}$ of $X$ into measurable sets such that
$\mu E_n=\mu^*(A_n\cap E_n)$ for every $n\in\Bbb N$.
%132F

\spheader 132Yd Let $(X,\Sigma,\mu)$ be a measure space and $\Cal A$ a
family of subsets of $X$ such that
$\bigcap_{n\in\Bbb N}A_n$ has full outer measure for every sequence
$\sequencen{A_n}$
in $\Cal A$.   Show that there is a measure $\nu$ on $X$, extending
$\mu$, such that every member of $\Cal A$ is $\nu$-conegligible.
%132F

\spheader 132Ye Check that you can do exercises 113Yg-113Yh.
%132+

\spheader 132Yf\dvAnew{2010} Let $(X,\Sigma,\mu)$ be a measure space.   
Show that
$\mu^*:\Cal PX\to[0,\infty]$ is {\bf alternating of all orders}, that is, 

\Centerline{$\sum_{J\subseteq I,\#(J)\text{ is even}}
   \mu^*(A\cup\bigcup_{i\in J}A_i)
\le\sum_{J\subseteq I,\#(J)\text{ is odd}}\mu^*(A\cup\bigcup_{i\in J}A_i)$}

\noindent whenever $I$ is a
non-empty finite set, $\familyiI{A_i}$ is a family of
subsets of $X$ and $A$ is another subset of $X$.   
%mt13bits

}%end of exercises

\endnotes{
\Notesheader{132} Almost the most fundamental fact in measure theory
is that in all important measure spaces there are non-measurable sets.
(For Lebesgue measure see 134B below.)   One can respond to this fact
in a variety of ways.   An approach which works quite well is just to
ignore it.   The point is that, for very deep reasons, the sets and
functions which arise in ordinary applications nearly always are
measurable, or can be made so by elementary manipulations;  the only
exceptions I know of in applied mathematics appear in generalized
control theory.   As a pure mathematician I am uncomfortable with such
an approach, and as a measure theorist I think it closes the door on
some of the most subtle ideas of the theory.   In this treatise,
therefore, non-measurable sets will always be present, if only
subliminally.   In this section I have described two of the basic
methods of dealing with them:  the move from a measure to an outer
measure, which at least assigns some sort of size to an arbitrary set,
and the idea of `measurable envelope', which (when defined) describes
the region in which the non-measurable set has to be taken into
account.   In both cases we seek to describe the non-measurable set from
the outside, so to speak.   There are no real difficulties, and the only
points to take note of are that (i) outside the boundary marked by
132Ee measurable envelopes need not exist (ii) \Caratheodory's construction
of a measure from an outer measure, and the construction here of an
outer measure from a measure, are closely related (132C, 132Xg, 113Yc,
132Xa(i)), but are not quite inverses of each other in general (132B,
132Xc).
}%end of notes

\discrpage

