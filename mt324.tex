\frfilename{mt324.tex}
\versiondate{9.2.98/27.2.04}
\copyrightdate{1998}
     
\def\chaptername{Measure algebras}
\def\sectionname{Homomorphisms}
     
\newsection{324}
     
In the course of Volume 2, I had occasion to remark that elementary
measure theory was unusual among abstract topics in pure mathematics in
not being dominated by any particular class of structure-preserving
operators.   We now come to what I think is one of the reasons for the
gap:  the most important operators of the theory are not between measure
spaces at all, but between their measure algebras.   In this section I
run through the most elementary facts about Boolean homomorphisms
between measure algebras.   I start with results on the construction of
such homomorphisms from functions between measure spaces (324A-324E),
then investigate continuity and order-continuity of homomorphisms
(324F-324H) before turning to measure-preserving homomorphisms
(324I-324O).
     
\leader{324A}{Theorem} Let $(X,\Sigma,\mu)$ and $(Y,\Tau,\nu)$
be measure spaces, and $(\frak A,\bar\mu)$, $(\frak B,\bar\nu)$ their
measure algebras.   Write $\hat\Sigma$ for the domain of the completion
$\hat\mu$ of $\mu$.   Let $D\subseteq X$ be a set of full outer
measure\cmmnt{ (definition: 132F)}, and let $\hat\Sigma_D$ be the
subspace $\sigma$-algebra on $D$ induced by $\hat\Sigma$.  Let
$\phi:D\to Y$ be a function such that
$\phi^{-1}[F]\in\hat\Sigma_D$ for every $F\in\Tau$ and
$\phi^{-1}[F]$ is $\mu$-negligible whenever $\nu F=0$.   Then
there is a sequentially order-continuous Boolean homomorphism 
$\pi:\frak B\to\frak A$ defined by the formula
     
\Centerline{$\pi F^{\ssbullet}=E^{\ssbullet}$ whenever $F\in\Tau$,
$E\in\Sigma$ and $(E\cap D)\symmdiff\phi^{-1}[F]$ is negligible.}
     
\proof{ Let $F\in\Tau$.  Then there is an $H\in\hat\Sigma$ such that
$H\cap D=\phi^{-1}[F]$;  now there is an $E\in\Sigma$ such that
$E\symmdiff H$ is negligible, so that $(E\cap D)\symmdiff\phi^{-1}[F]$
is negligible.   If $E_1$ is another member of $\Sigma$ such that
$(E_1\cap D)\symmdiff\phi^{-1}[F]$ is negligible, then $(E\symmdiff
E_1)\cap D$ is negligible, so is included in a negligible member $G$ of
$\Sigma$.   Since $(E\symmdiff E_1)\setminus G$ belongs to $\Sigma$ and
is disjoint from $D$, it is negligible;  accordingly $E\symmdiff E_1$ is
negligible and $E^{\ssbullet}=E_1^{\ssbullet}$ in $\frak A$.
     
What this means is that the formula offered defines a map $\pi:\frak
B\to\frak A$.   It is now easy to check that $\pi$ is a Boolean
homomorphism, because if
     
\Centerline{$(E\cap D)\symmdiff\phi^{-1}[F]$, \quad$(E'\cap
D)\symmdiff\phi^{-1}[F']$}
     
\noindent are negligible, so are
     
\Centerline{$((X\setminus E)\cap
D)\symmdiff\phi^{-1}[Y\setminus F]$,
\quad$((E\cup E')\cap D)\symmdiff\phi^{-1}[F\cup F']$.}
     
To see that $\pi$ is sequentially order-continuous, let
$\sequencen{b_n}$ be a sequence in $\frak B$.  For each $n$ we may
choose an $F_n\in\Tau$ such that $F_n^{\ssbullet}=b_n$, and
$E_n\in\Sigma$ such that $(E_n\cap D)\symmdiff\phi^{-1}[F_n]$ is
negligible;  now, setting $F=\bigcup_{n\in\Bbb N}F_n$,
$E=\bigcup_{n\in\Bbb N}E_n$,
     
\Centerline{$(E\cap D)\symmdiff\phi^{-1}[F]
\subseteq\bigcup_{n\in\Bbb N}(E_n\cap D)\symmdiff\phi^{-1}[F_n]$}
     
\noindent is negligible, so
     
\Centerline{$\pi(\sup_{n\in\Bbb N}b_n)=\pi(F^{\ssbullet})
=E^{\ssbullet}=\sup_{n\in\Bbb N}E_n^{\ssbullet}
=\sup_{n\in\Bbb N}\pi b_n$.}
     
\noindent (Recall that the maps $E\mapsto E^{\ssbullet}$, 
$F\mapsto F^{\ssbullet}$ are sequentially order-continuous, by 321H.)   
So $\pi$ is sequentially order-continuous (313L(c-iii)).
}%end of proof of 324A
     
     
\leader{324B}{Corollary} Let $(X,\Sigma,\mu)$ and $(Y,\Tau,\nu)$
be measure spaces, and $(\frak A,\bar\mu)$, $(\frak B,\bar\nu)$ their
measure algebras.   Let $\phi:X\to Y$ be a function such that
$\phi^{-1}[F]\in\Sigma$ for every $F\in\Tau$ and $\mu\phi^{-1}[F]=0$
whenever $\nu F=0$.   Then there is a sequentially order-continuous
Boolean homomorphism $\pi:\frak B\to\frak A$ defined by the formula
     
\Centerline{$\pi F^{\ssbullet}=(\phi^{-1}[F])^{\ssbullet}$ for every
$F\in\Tau$.}
     
     
\cmmnt{
\leader{324C}{Remarks (a)} In \S235 and elsewhere in Volume 2 I spent
a good deal of time on functions between measure spaces which satisfy
the conditions of 324A.   Indeed, I take the trouble to spell 324A out
in such generality just in order to catch these applications.   Some of
the results of the present chapter (322D, 322Jb) can also be regarded as
special cases of 324A.
     
\spheader 324Cb The question of which homomorphisms between the measure
algebras of
measure spaces $(X,\Sigma,\mu)$, $(Y,\Tau,\nu)$ can be realized
by functions between $X$ and $Y$ is important and deep;  I will return
to it in \S\S343-344.
     
\spheader 324Cc In the simplified context of 324B, I have actually
defined a contravariant functor;  the relevant facts are the following.
}%end of comment
     
\leader{324D}{Proposition} Let $(X,\Sigma,\mu)$, $(Y,\Tau,\nu)$
and $(Z,\Lambda,\lambda)$ be measure spaces, with measure algebras
$(\frak A,\bar\mu)$, $(\frak B,\bar\nu)$, $(\frak C,\bar\lambda)$.
Suppose that $\phi:X\to Y$ and $\psi:Y\to Z$ satisfy the conditions of
324B, that is,
     
\Centerline{$\phi^{-1}[F]\in\Sigma$ if $F\in\Tau$,
\quad$\mu\phi^{-1}[F]=0$ if $\nu F=0$,}
     
\Centerline{$\psi^{-1}[G]\in\Tau$ if $G\in\Lambda$,
\quad$\mu\psi^{-1}[G]=0$ if $\lambda G=0$.}
     
\noindent Let $\pi_{\phi}:\frak B\to\frak A$, $\pi_{\psi}:\frak
C\to\frak B$ be the corresponding homomorphisms.   Then $\psi\phi:X\to
Z$ is another map of the same type, and
$\pi_{\psi\phi}=\pi_{\phi}\pi_{\psi}:\frak C\to\frak A$.
     
\proof{ The necessary checks are all elementary.
}%end of proof of 324D
     
\leader{324E}{Stone \dvrocolon{spaces}}\cmmnt{ While in the context of
general measure
spaces the question of realizing homomorphisms is difficult, in the case
of the Stone representation it is relatively straightforward.
     
\medskip
     
\noindent}{\bf Proposition} Let $(\frak A,\bar\mu)$ and
$(\frak B,\bar\nu)$ be measure algebras, with Stone spaces $Z$ and $W$;  
let $\mu$, $\nu$ be the corresponding measures on $Z$ and
$W$\cmmnt{, as described in 321J-321K}, and $\Sigma$, $\Tau$ their domains.   If
$\pi:\frak B\to\frak A$ is any order-continuous Boolean homomorphism, let $\phi:Z\to W$ be
the corresponding continuous function\cmmnt{, as described in 312Q}.
Then $\phi^{-1}[F]\in\Sigma$ for every $F\in\Tau$, $\mu\phi^{-1}[F]=0$
whenever $\nu F=0$, and (writing $E^*$ for the member of $\frak A$
corresponding to $E\in\Sigma$) $\pi F^*=(\phi^{-1}[F])^*$ for every
$F\in\Tau$.
     
\proof{ Recall that $E^*=a$ iff
$E\symmdiff\widehat{a}$ is meager, where $\widehat{a}$ is the
open-and-closed subset of $Z$ corresponding to $a\in\frak A$.   In
particular, $\mu E=0$
iff $E$ is meager.   Now the point is that $\phi^{-1}[F]$ is nowhere
dense in $Z$ whenever $F$ is a nowhere dense subset of $W$, by 313R.
Consequently $\phi^{-1}[F]$ is meager whenever $F$ is meager in $W$,
since $F$ is then just a countable union of nowhere dense sets.
Thus we see already that $\mu\phi^{-1}[F]=0$ whenever $\nu F=0$.
If $F$ is any member of $\Tau$, there is an open-and-closed set $F_0$
such that $F\symmdiff F_0$ is meager;  now $\phi^{-1}[F_0]$ is
open-and-closed, so
$\phi^{-1}[F]=\phi^{-1}[F_0]\symmdiff\phi^{-1}[F\symmdiff F_0]$ belongs
to $\Sigma$.   Moreover, if $b\in \frak B$ is such that
$\widehat{b}=F_0$,
and $a=\pi b$, then $\widehat{a}=\phi^{-1}[F_0]$, so
     
\Centerline{$\pi F^*=\pi b=a=(\phi^{-1}[F_0])^*=(\phi^{-1}[F])^*$,}
     
\noindent as required.
}%end of proof of 324E
     
\leader{324F}{}\cmmnt{ I turn now to the behaviour of
order-continuous
homomorphisms between measure algebras.
     
\medskip
     
\noindent}{\bf Theorem} Let $(\frak A,\bar\mu)$ and $(\frak B,\bar\nu)$
be measure
algebras and $\pi:\frak A\to\frak B$ a Boolean homomorphism.
Give $\frak A$ and $\frak B$ their measure-algebra topologies and
uniformities\cmmnt{ (323Ab)}.
     
(a) $\pi$ is continuous iff it is continuous at $0$ iff it is uniformly
continuous.
     
(b) If $(\frak B,\bar\nu)$ is semi-finite and $\pi$ is continuous, then
it is order-continuous.
     
(c) If $(\frak A,\bar\mu)$ is semi-finite and $\pi$ is order-continuous,
then it is continuous.
     
\proof{ I use the notations $\frak A^f$, $\rho_a$ from 323A.
     
\medskip
     
{\bf (a)} Suppose that $\pi$ is continuous at $0$;   I seek to show that
it is uniformly continuous.  Take $b\in\frak B^f$ and $\epsilon>0$.
Then there are $a_0,\ldots,a_n\in\frak A^f$ and $\delta>0$ such that
     
\Centerline{$\bar\nu(b\Bcap \pi c)=\rho_b(\pi c,0)\le\epsilon$
whenever $\max_{i\le n}\rho_{a_i}(c,0)\le\delta$;}
     
\noindent setting $a=\sup_{i\le n}a_i$,
     
\Centerline{$\bar\nu(b\Bcap\pi c)\le\epsilon$ whenever $\bar\mu(a\Bcap
c)\le\delta$.}
     
Now suppose that $\rho_a(c,c')\le\delta$.   Then
$\bar\mu(a\Bcap(c\Bsymmdiff c'))\le\delta$, so
     
\Centerline{$\rho_b(\pi c,\pi c')
=\bar\nu(b\Bcap(\pi c\Bsymmdiff\pi c'))
=\bar\nu(b\Bcap\pi(c\Bsymmdiff c'))
\le\epsilon$.}
     
\noindent As $b$, $\epsilon$ are arbitrary, $\pi$ is uniformly
continuous.   The rest of the implications are elementary.
     
\medskip
     
{\bf (b)} Let $A$ be a non-empty downwards-directed set in $\frak A$
with infimum $0$.   Then $0\in\overline{A}$ (323D(b-ii));  because $\pi$
is continuous, $0\in\overline{\pi[A]}$.   \Quer\ If $b$ is a non-zero
lower bound for $\pi[A]$ in $\frak B$, then (because $(\frak B,\bar\nu)$
is semi-finite) there is a $c\Bsubseteq b$ with $0<\bar\nu c<\infty$;
now
     
\Centerline{$\rho_c(\pi a,0)=\bar\nu(c\Bcap\pi a)=\bar\nu c>0$}
     
\noindent for every $a\in A$, so $0\notin\overline{\pi[A]}$.\ \Bang
     
Thus $\inf\pi[A]=0$ in $\frak B$;  as $A$ is arbitrary, $\pi$ is
order-continuous (313L(b-ii)).
     
\medskip
     
{\bf (c)} By (a), it will be enough to show that $\pi$ is continuous at
$0$.   Let $b\in\frak B^f$, $\epsilon>0$.   \Quer\ Suppose, if possible,
that for every $a\in\frak A^f$, $\delta>0$ there is a $c\in\frak A$ such
that $\bar\mu(a\Bcap c)\le\delta$ but $\bar\nu(b\Bcap\pi c)\ge\epsilon$.
For
each $a\in\frak A^f$, $n\in\Bbb N$ choose $c_{an}$ such that
$\bar\mu(a\Bcap
c_{an})\le2^{-n}$ but $\bar\nu(b\Bcap\pi c_{an})\ge\epsilon$.   Set
$c_a=\inf_{n\in\Bbb N}\sup_{m\ge n}c_{am}$;  then
     
\Centerline{$\bar\mu(a\Bcap c_a)\le\inf_{n\in\Bbb
N}\sum_{m=n}^{\infty}\bar\mu(a\Bcap c_{an})=0$,}
     
\noindent so $c_a\Bcap a=0$.   On the other hand, because $\pi$ is
order-continuous, $\pi c_a=\inf_{n\in\Bbb N}\sup_{m\ge n}\pi c_{am}$,
so that
     
\Centerline{$\bar\nu(b\Bcap\pi
c_a)=\lim_{n\to\infty}\bar\nu(b\Bcap\sup_{m\ge
n}\pi c_{am})\ge\epsilon$.}
     
\noindent This shows that
     
\Centerline{$\rho_b(\pi(1\Bsetminus a),0)
=\bar\nu(b\Bcap\pi(1\Bsetminus a))
\ge\bar\nu(b\Bcap\pi c_a)\ge\epsilon$.}
     
\noindent But now observe that $A=\{1\Bsetminus a:a\in\frak A^f\}$ is a
downwards-directed subset of $\frak A$ with infimum $0$, because 
$(\frak A,\bar\mu)$ is semi-finite.   So $\pi[A]$ is downwards-directed and has infimum $0$, 
and $0$ must be in the closure of $\pi[A]$, by 323D(b-ii);
while we have just seen that $\rho_b(d,0)\ge\epsilon$ for every
$d\in\pi[A]$.   \Bang
     
Thus there must be $a\in\frak A^f$, $\delta>0$ such that
     
\Centerline{$\rho_b(\pi c,0)=\bar\nu(b\Bcap\pi c)\le\epsilon$}
     
\noindent whenever
     
\Centerline{$\rho_a(c,0)=\bar\mu(a\Bcap c)\le\delta$.}
     
\noindent As $b$, $\epsilon$ are arbitrary, $\pi$ is continuous at $0$
and therefore continuous.
}%end of proof of 324F
     
\leader{324G}{Corollary} If $(\frak A,\bar\mu)$ and $(\frak B,\bar\nu)$
are semi-finite measure algebras, a Boolean homomorphism 
$\pi:\frak A\to\frak B$ is continuous iff it is order-continuous.
     
\leader{324H}{Corollary} If $\frak A$ is a Boolean algebra and
$\bar\mu$,
$\bar\nu$ are two measures both rendering $\frak A$ a semi-finite
measure
algebra, then they endow $\frak A$ with the same uniformity
(and\cmmnt{, of
course,} the same topology).
     
\proof{ By 324G, the identity map from $\frak A$ to itself is continuous
whichever of the topologies we place on $\frak A$;  and by 324F it is
therefore uniformly continuous.
}%end of proof of 324H
     
\leader{324I}{Definition} Let $(\frak A,\bar\mu)$ and
$(\frak B,\bar\nu)$ be
measure algebras.   A Boolean homomorphism $\pi:\frak A\to\frak B$ is
{\bf measure-preserving} if $\bar\nu(\pi a)=\bar\mu a$ for every
$a\in\frak A$.
     
\leader{324J}{Proposition} Let $(\frak A,\bar\mu)$, $(\frak B,\bar\nu)$
and $(\frak C,\bar\lambda)$ be measure algebras, and
$\pi:\frak A\to\frak B$,
$\theta:\frak B\to\frak C$ measure-preserving Boolean homomorphisms.
Then $\theta\pi:\frak A\to\frak C$ is a measure-preserving Boolean
homomorphism.
     
\proof{ Elementary.
}%end of proof of 324J
     
\vleader{108pt}{324K}{Proposition} Let $(\frak A,\bar\mu)$ and
$(\frak B,\bar\nu)$ be measure algebras, and $\pi:\frak A\to\frak B$ a
measure-preserving Boolean homomorphism.
     
(a) $\pi$ is injective.
     
(b) $(\frak A,\bar\mu)$ is totally finite iff $(\frak B,\bar\nu)$ is,
and in this case $\pi$ is
order-continuous, therefore continuous, and $\pi[\frak A]$ is a closed
subalgebra of $\frak B$.
     
(c) If $(\frak A,\bar\mu)$ is semi-finite and $(\frak B,\bar\nu)$ is
$\sigma$-finite, then $(\frak A,\bar\mu)$ is $\sigma$-finite.
     
(d) If $(\frak A,\bar\mu)$ is $\sigma$-finite and $\pi$ is sequentially
order-continuous, then $(\frak B,\bar\nu)$ is $\sigma$-finite.
     
(e) If $(\frak A,\bar\mu)$ is semi-finite and $\pi$ is
order-continuous, then $(\frak B,\bar\nu)$ is semi-finite.
     
(f) If $(\frak A,\bar\mu)$ is atomless and semi-finite, and $\pi$ is
order-continuous, then $\frak B$ is atomless.
     
(g) If $\frak B$ is purely atomic and $(\frak A,\bar\mu)$ is
semi-finite, then $\frak A$ is purely atomic.
     
\proof{{\bf (a)} If $a\ne 0$ in $\frak A$, then 
$\bar\nu\pi a=\bar\mu a>0$ so $\pi a\ne 0$.   By 3A2Db, $\pi$ is injective.
     
\medskip
     
{\bf (b)} Because
     
\Centerline{$\bar\nu 1_{\frak B}=\bar\nu\pi 1_{\frak A}
=\bar\mu 1_{\frak A}$,}
     
\noindent $(\frak A,\bar\mu)$ is totally finite iff $(\frak B,\bar\nu)$
is.   Now suppose that $A\subseteq\frak A$ is downwards-directed and
non-empty and that $\inf A=0$.   Then
     
\Centerline{$\inf_{a\in A}\bar\nu\pi a=\inf_{a\in A}\bar\mu a=0$}
     
\noindent by 321F.   So $\bar\nu b=0$ for any lower bound $b$ of
$\pi[A]$, and $\inf\pi[A]=0$.   As $A$ is arbitrary, $\pi$ is
order-continuous.
     
By 324Fc, $\pi$ is continuous.   By 314Fa, $\pi[\frak A]$ is
order-closed in $\frak B$, that is, `closed' in the sense of 323I.
     
\medskip
     
{\bf (c)} I appeal to 322G.   If $C$ is a disjoint family in 
$\frak A\setminus\{0\}$, then 
$\langle\pi c\rangle_{c\in C}$ is a disjoint family in 
$\frak B\setminus\{0\}$,
so is countable, and $C$ must be countable, because $\pi$ is injective.
Thus $\frak A$ is ccc and
(being semi-finite) $(\frak A,\bar\mu)$ is $\sigma$-finite.
     
\medskip
     
{\bf (d)} Let $\sequencen{a_n}$ be a sequence in $\frak A$ such that
$\bar\mu a_n<\infty$ for every $n$ and $\sup_{n\in\Bbb N}a_n=1$.   Then
$\bar\nu\pi a_n<\infty$ for every $n$ and (because $\pi$ is sequentially
order-continuous) $\sup_{n\in\Bbb N}\pi a_n=1$, so $(\frak B,\bar\nu)$
is $\sigma$-finite.
     
\medskip
     
{\bf (e)} Setting $\frak A^f=\{a:\bar\mu a<\infty\}$, $\sup \frak
A^f=1$;
because $\pi$ is order-continuous, $\sup\pi[\frak A^f]=1$ in $\frak B$.
So if $\bar\nu b=\infty$, there is an $a\in\frak A^f$ such that $\pi
a\Bcap
b\ne 0$, and now $0<\bar\nu(b\Bcap\pi a)<\infty$.
     
\medskip
     
{\bf (f)} Take any non-zero $b\in\frak B$.   As in (e), there is an
$a\in\frak A$ such that $\bar\mu a<\infty$ and $\pi a\Bcap b\ne 0$.   If
$\pi a\Bcap b\ne b$, then surely $b$ is not an atom.   Otherwise, set
     
\Centerline{$C=\{c:c\in\frak A,\,c\Bsubseteq a,\,b\Bsubseteq\pi c\}$.}
     
\noindent Then $C$ is downwards-directed and contains $a$, so
$c_0=\inf C$ is defined in $\frak A$ (321F), and
     
\Centerline{$\bar\mu c_0=\inf_{c\in C}\bar\mu c\ge\bar\nu b>0$,}
     
\noindent so $c_0\ne 0$.   Because $\frak A$ is atomless, there is a
$d\Bsubseteq c_0$ such that neither $d$ nor $c_0\Bsetminus d$ is zero,
so that neither $c_0\Bsetminus d$ nor $d$ can belong to $C$.  But this
means that $b\Bcap\pi d$ and $b\Bcap\pi(c_0\Bsetminus d)$ are both
non-zero, so that again $b$ is not an atom.
As $b$ is arbitrary, $\frak B$ is atomless.
     
\medskip
     
{\bf (g)} Take any non-zero $a\in\frak A$.   Then there is an
$a'\Bsubseteq a$ such that $0<\bar\mu a'<\infty$.   Because $\frak B$ is
purely atomic, there is an atom $b$ of $\frak B$ with $b\Bsubseteq\pi
a'$.   Set
     
\Centerline{$C=\{c:c\in\frak A,\,c\Bsubseteq a',\,b\Bsubseteq\pi c\}$.}
     
\noindent Then $C$ is downwards-directed and contains $a'$, so $c_0=\inf
C$ is defined in $\frak A$, and
     
\Centerline{$\bar\mu c_0=\inf_{c\in C}\bar\mu c\ge\bar\nu b>0$,}
     
\noindent so $c_0\ne 0$.  If $d\Bsubseteq c_0$, then $b\Bcap \pi d$ must
be either $b$ or $0$.   If $b\Bcap\pi d=b$, then $d\in C$ and $d=c_0$.
If $b\Bcap\pi d=0$, then $c_0\Bsetminus d\in C$ and $d=0$.   Thus $c_0$
is an atom in $\frak A$.   As $a$ is arbitrary, $\frak A$ is purely
atomic.
}%end of proof of 324K
     
\leader{324L}{Corollary} Let $(\frak A,\bar\mu)$ be a totally finite
measure algebra, $(\frak B,\bar\nu)$ a measure algebra, and
$\pi:\frak A\to\frak B$
a measure-preserving homomorphism.   If $C\subseteq\frak A$ and
$\frak C$ is the closed subalgebra of $\frak A$ generated by $C$, then
$\pi[\frak C]$ is the closed subalgebra of $\frak B$ generated by
$\pi[C]$.
     
\proof{ This is a special case of 314H.
}%end of proof of 324L
     
\leader{324M}{Proposition} Let $(X,\Sigma,\mu)$ and $(Y,\Tau,\nu)$ be
measure spaces, with measure algebras $(\frak A,\bar\mu)$ and
$(\frak B,\bar\nu)$.   Let $\phi:X\to Y$ be \imp.   
Then we have a sequentially order-continuous measure-preserving Boolean
homomorphism $\pi:\frak B\to\frak A$ defined by setting
$\pi F^{\ssbullet}=\phi^{-1}[F]^{\ssbullet}$ for every $F\in\Tau$.
     
\proof{ This is immediate from 324B.
}%end of proof of 324M
     
\leader{324N}{Proposition} Let $(\frak A,\bar\mu)$ and
$(\frak B,\bar\nu)$ be measure algebras, with Stone spaces $Z$ and $W$;   
let $\mu$, $\nu$ be the corresponding measures on $Z$ and $W$.   If
$\pi:\frak B\to\frak A$ is an order-continuous
measure-preserving Boolean homomorphism, and $\phi:Z\to W$ the
corresponding continuous function, then $\phi$ is \imp.
     
\proof{ Use 324E.   In the notation there, if $F\in\Tau$, then
     
\Centerline{$\nu F=\bar\nu F^*=\bar\mu\pi F^*=\bar\mu\phi^{-1}[F]^*
=\mu\phi^{-1}[F]$.}
}%end of proof of 324N
     
\leader{324O}{Proposition} Let $(\frak A,\bar\mu)$ and
$(\frak B,\bar\nu)$ be
totally finite measure algebras, $\frak A_0$ a topologically dense
subalgebra of $\frak A$, and $\pi:\frak A_0\to\frak B$ a Boolean
homomorphism such that $\bar\nu\pi a=\bar\mu a$ for every
$a\in\frak A_0$.
Then $\pi$ has a unique extension to a measure-preserving homomorphism
from $\frak A$ to $\frak B$.
     
\proof{ Let $\rho$, $\sigma$ be the measure metrics on $\frak A$,
$\frak B$ respectively, as in 323Ad.   Then for any $a$, 
$a'\in\frak A_0$
     
\Centerline{$\sigma(\pi a,\pi a')=\bar\nu(\pi a\symmdiff\pi a')
=\bar\nu\pi(a\symmdiff a')=\bar\mu(a\symmdiff a')=\rho(a,a')$;}
     
\noindent that is, $\pi:\frak A_0\to\frak B$ is an isometry.   Because
$\frak A_0$ is dense in the metric space $(\frak A,\rho)$, while 
$\frak B$ is complete under $\sigma$ (323Gc), there is a unique continuous
function $\hat\pi:\frak A\to\frak B$ extending $\pi$ (3A4G).   Now
the operations
     
\Centerline{$(a,a')\mapsto \hat\pi(a\Bcup a')$,
\quad$(a,a')\mapsto \hat\pi
a\Bcup\hat\pi a':\frak A\times\frak A\to\frak B$,}
     
\noindent are continuous and agree on the dense subset 
$\frak A_0\times\frak A_0$ of $\frak A\times\frak A$;  because the topology of
$\frak B$ is Hausdorff, they agree on $\frak A\times\frak A$, that is,
$\hat\pi(a\Bcup a')=\hat\pi a\Bcup\hat\pi a'$ for all $a$, 
$a'\in\frak A$ (2A3Uc).   Similarly, the operations
     
\Centerline{$a\mapsto\hat\pi(1\Bsetminus a)$,
\quad$a\mapsto 1\Bsetminus\hat\pi a:\frak A\to\frak B$}
     
\noindent are continuous and agree on the dense subset $\frak A_0$ of
$\frak A$, so they agree on $\frak A$, that is, $\hat\pi(1\Bsetminus
a)=1\Bsetminus a$ for every $a\in\frak A$.   Thus $\hat\pi$ is a Boolean
homomorphism.   To see that it is measure-preserving, observe that
     
\Centerline{$a\mapsto\bar\mu a=\rho(a,0)$,
\quad$a\mapsto\bar\nu(\hat\pi
a)=\sigma(\hat\pi a,0):\frak A\to\Bbb R$}
     
\noindent are continuous and agree on $\frak A_0$, so agree on $\frak A$.
Finally, $\hat\pi$ is the only measure-preserving Boolean
homomorphism extending $\pi$, because any such map must be continuous
(324Kb), and $\hat\pi$ is the only continuous extension of $\pi$.
}%end of proof of 324O
     
\leader{*324P}{}\cmmnt{ The following fact will be applied in \S387,
by which time it will seem perfectly elementary;  for the moment, it may
be a useful exercise.
     
\medskip
     
\noindent}{\bf Proposition} Let $(\frak A,\bar\mu)$ and 
$(\frak B,\bar\nu)$ be totally finite measure algebras such that 
$\bar\mu 1=\bar\nu 1$.   Suppose that $A\subseteq\frak A$ and 
$\phi:A\to\frak B$
are such that $\bar\nu(\inf_{i\le n}\phi a_i)=\bar\mu(\inf_{i\le n}a_i)$
for all $a_0,\ldots,a_n\in A$.   Let $\frak C$ be the smallest closed
subalgebra of $\frak A$ including $A$.   Then $\phi$ has a unique
extension to a measure-preserving Boolean homomorphism from $\frak C$ to
$\frak B$.
     
\proof{{\bf (a)} Let $\Psi$ be the family of all functions $\psi$
extending $\phi$ and having the same properties;  that is, $\psi$ is a
function from a subset of $\frak A$ to $\frak B$, and
$\bar\nu(\inf_{i\le n}\psi a_i)=\bar\mu(\inf_{i\le n}a_i)$ for all
$a_0,\ldots,a_n\in\dom\psi$.   By Zorn's Lemma, $\Psi$ has a maximal
member $\theta$.   Write $D$ for the domain of $\theta$.
     
\medskip
     
{\bf (b)(i)} If $c$, $d\in D$ then $c\Bcap d\in D$.   \Prf\Quer\
Otherwise, set $D'=D\cup\{c\Bcap d\}$ and extend $\theta$ to
$\theta':D'\to\frak B$ by writing 
$\theta'(c\Bcap d)=\theta c\Bcap\theta d$.   
It is easy to check that $\theta'\in\Psi$, which is supposed to be
impossible.\ \Bang\Qed
     
Now
     
\Centerline{$\bar\nu(\theta c\Bcap\theta d\Bcap\theta(c\Bcap d))
=\bar\mu(c\Bcap d)
=\bar\nu(\theta c\Bcap\theta d)
=\bar\nu\theta(c\Bcap d)$,}
     
\noindent so $\theta(c\Bcap d)=\theta c\Bcap\theta d$.
     
\medskip
     
\quad{\bf (ii)} If $d\in D$ then $1\Bsetminus d\in D$.   \Prf\Quer\
Otherwise, set $D'=D\cup\{1\Bsetminus d\}$ and extend $\theta$ to $D'$
by writing $\theta'(1\Bsetminus d)=1\Bsetminus\theta d$.   Once again,
it is easy to check that $\theta'\in\Psi$, which is
impossible.\ \Bang\Qed
     
Consequently (since $D$ is certainly not empty, even if $C$ is), $D$ is
a subalgebra of $\frak A$ (312B(iii)).
     
\medskip
     
\quad{\bf (iii)} Since
     
\Centerline{$\bar\nu\theta 1=\bar\mu 1=\bar\nu 1$,}
     
\noindent $\theta 1=1$.   If $d\in D$ then
     
\Centerline{$\bar\nu\theta(1\Bsetminus d)
=\bar\mu(1\Bsetminus d)
=\bar\mu 1-\bar\mu d
=\bar\nu 1-\bar\nu\theta d
=\bar\nu(1\Bsetminus\theta d)$,}
     
\noindent while
     
\Centerline{$\bar\nu(\theta d\Bcap\theta(1\Bsetminus d))
=\bar\mu(d\Bcap(1\Bsetminus d))=0$,}
     
\noindent so $\theta d\Bcap\theta(1\Bsetminus d))=0$,
$\theta(1\Bsetminus d)\Bsubseteq 1\Bsetminus\theta d$ and
$\theta(1\Bsetminus d)$ must be equal to $1\Bsetminus\theta d$.
     
By 312H(ii), $\theta:D\to\frak B$ is a Boolean homomorphism.
     
\medskip
     
\quad{\bf (iv)} Let $\frak D$ be the topological closure of $D$ in
$\frak A$.   Then it is an order-closed subalgebra of $\frak A$ (323J),
so, with $\bar\mu\restrp\frak D$, is a totally finite measure algebra in
which $D$ is a topologically dense subalgebra.   By 324O, there is an
extension of $\theta$ to a measure-preserving Boolean homomorphism from
$\frak D$ to $\frak B$;  of course this extension belongs to $\Psi$, so
in fact $D=\frak D$ is a closed subalgebra of $\frak A$.
     
\medskip
     
{\bf (c)} Since $A\subseteq D$, $\frak C\subseteq\frak D$ and
$\phi_1=\theta\restrp\frak C$ is a suitable extension of $\phi$.
     
To see that $\phi_1$ is unique, let $\phi_2:\frak C\to\frak B$ be any
other measure-preserving Boolean homomorphism extending $\phi$.   Set
$C=\{a:\phi_1a=\phi_2a\}$;  then $C$ is a topologically closed
subalgebra of $\frak A$ including $A$, so is the whole of $\frak C$, and
$\phi_2=\phi_1$.
}%end of proof of 324P
     
\exercises{
\leader{324X}{Basic exercises (a)}
%\spheader 324Xa
Let $\frak A$ and $\frak B$ be Boolean algebras,
of which $\frak A$ is Dedekind $\sigma$-complete, and $\phi:\frak
A\to\frak B$ a sequentially order-continuous Boolean homomorphism.   Let
$I$ be an ideal of $\frak A$ included in the kernel of $\phi$.   Show
that we have a sequentially order-continuous Boolean homomorphism
$\pi:\frak A/I\to \frak B$ given by setting $\phi(a^{\ssbullet})=\phi a$
for every $a\in\frak A$.
%324B
     
\spheader 324Xb Let $(\frak A,\bar\mu)$ be a measure algebra, and
$\frak B$ a $\sigma$-subalgebra of $\frak A$.   Show
that {\it provided that $(\frak B,\bar\mu\restrp\frak B)$ is
semi-finite},
then the topology of $\frak B$ induced by $\bar\mu\restrp\frak B$ is just
the subspace topology induced by the topology of $\frak A$.   ({\it
Hint\/}:  apply 324Fc to the embedding $\frak B\embedsinto\frak A$.)
%324F
     
\spheader 324Xc Let $(X,\Sigma,\mu)$ be a measure space and
$(X,\tilde\Sigma,\tilde\mu)$ its c.l.d.\ version.   Let $\frak A$,
$\frak A_2$ be the corresponding measure algebras and 
$\pi:\frak A\to\frak A_2$ the canonical homomorphism (see 322Db).   Show that $\pi$ is topologically continuous.
%324F
     
\spheader 324Xd Let $(\frak A,\bar\mu)$ and 
$(\frak B,\bar\nu)$ be measure algebras, and $\pi:\frak A\to\frak B$ a 
bijective measure-preserving Boolean homomorphism.   Show that
$\pi^{-1}:\frak B\to\frak A$ is a measure-preserving homomorphism.
%324I
     
\spheader 324Xe Let $\bar\mu$ be counting measure on $\Cal P\Bbb N$.
Show that $(\Cal P\Bbb N,\bar\mu)$ is a $\sigma$-finite measure algebra.
Find a measure-preserving Boolean homomorphism from $\Cal P\Bbb N$ to
itself which is not sequentially order-continuous.
%324K
     
\leader{324Y}{Further exercises (a)} Let $\frak A$ and $\frak B$ be
Boolean algebras, of which $\frak A$ is Dedekind complete, and
$\phi:\frak A\to\frak B$ an order-continuous Boolean homomorphism.   Let
$I$ be an ideal of $\frak A$ included in the kernel of $\phi$.   Show
that we have an order-continuous Boolean homomorphism 
$\pi:\frak A/I\to \frak B$ given by setting $\phi(a^{\ssbullet})=\phi a$ for every
$a\in\frak A$.
%324B, 324Xa
     
\header{324Yb}{\bf (b)} Let $\frak A$ be a Dedekind
$\sigma$-complete Boolean algebra, and $Z$ its Stone space.   Write
$\Cal E$ for the algebra of open-and-closed subsets of $Z$, and $\Cal Z$
for the family of nowhere dense zero sets of $Z$;  let $\Cal Z_{\sigma}$
be the $\sigma$-ideal of subsets of $Z$ generated by $\Cal Z$.   Show
that $\Sigma=\{E\symmdiff U:E\in\Cal E,\,U\in\Cal Z_{\sigma}\}$ is a
$\sigma$-algebra of subsets of $Z$, and describe a canonical isomorphism
between $\Sigma/\Cal Z_{\sigma}$ and $\frak A$.
%324E
     
\header{324Yc}{\bf (c)} Let $\frak A$ and $\frak B$ be Dedekind
$\sigma$-complete Boolean algebras, with Stone spaces $Z$ and $W$.
Construct $\Cal Z_{\sigma}\subseteq\Sigma\subseteq\Cal PZ$ as in 324Yb,
and let $\Cal W_{\sigma}\subseteq\Tau\subseteq\Cal PW$ be the
corresponding structure defined from $\frak B$.   Let
$\pi:\frak B\to\frak A$ be a sequentially order-continuous 
Boolean homomorphism, and $\phi:Z\to W$ the corresponding continuous map.
Show that if $E^*\in\frak A$ corresponds to $E\in\Sigma$, then
$\pi F^*=\phi^{-1}[F]^*$ for every $F\in\Tau$.   \Hint{313Ye.}
%324E, 324Yb
     
\spheader 324Yd Let $\frak A$ be a Boolean algebra, $\frak B$ a ccc
Boolean algebra and $\pi:\frak A\to\frak B$ an injective Boolean
homomorphism.   Show that $\frak A$ is ccc.
%324K
     
\spheader 324Ye Let $\frak A$ be a Dedekind complete Boolean
algebra, $\frak B$ a Boolean algebra, and $\pi:\frak A\to\frak B$ an
order-continuous Boolean homomorphism.   Show that for every atom
$b\in\frak B$ there is an atom $a\in\frak A$ such that 
$\pi a\Bsupseteq b$.   
Hence show that if $\frak A$ is atomless so is $\frak B$, and that
if $\frak B$ is purely atomic and $\pi$ is injective then $\frak A$ is
purely atomic.
%324K
     
\spheader 324Yf Let $(\frak A,\bar\mu)$ and $(\frak B,\bar\nu)$ be
localizable measure algebras and $\frak A_0$ an order-dense subalgebra
of $\frak A$.
Suppose that $\pi:\frak A_0\to\frak B$ is an order-continuous Boolean
homomorphism such that $\bar\nu\pi a=\bar\mu a$ for every
$a\in\frak A_0$.  Show that $\pi$ has a unique extension to a
measure-preserving Boolean homomorphism from $\frak A$ to $\frak B$.
%324O
     
\spheader 324Yg\dvAnew{2009}
Let $(\frak A,\bar\mu)$ be the measure algebra of
Lebesgue measure on $[0,1]$.  (i) Show that there is an injective
order-preserving function $f:\frak A\to\Cal P\Bbb N$.   \Hint{take a
countable topologically dense subset $D$ of $\frak A$, and define
$f:\frak A\to\Cal P(D\times\Bbb Q)$ by setting
$f(a)=\{(d,q):\bar\mu(a\Bcap d)\ge q\}$.}   (ii) Show that there is an
order-preserving function $h:\Cal P\Bbb N\to\frak A$ such that
$h(f(a))=a$ for every $a\in\frak A$.   \Hint{set
$h(I)=\sup\{a:f(a)\subseteq I\}$.}   
     
\spheader 324Yh\dvAformerly{3{}24Yg} 
Let $(\frak A,\bar\mu)$ and $(\frak B,\bar\nu)$ be 
probability algebras, and $f:\frak A\to\frak B$ an isometry for the 
measure metrics.   Show that $a\mapsto f(a)\Bsymmdiff f(0)$ is a 
measure-preserving Boolean homomorphism.
     
}%end of exercises
     
\endnotes{
\Notesheader{324} If you examine the arguments of this section
carefully, you will see that rather little depends on the measures
named.   Really this material deals with structures $(X,\Sigma,\Cal I)$
where $X$ is a set, $\Sigma$ is a $\sigma$-ideal of subsets of $X$, and
$\Cal I$ is a $\sigma$-ideal of $\Sigma$, corresponding to the family of
measurable negligible sets.
In this abstract form it is natural to think in terms of
sequentially order-continuous homomorphisms, as in 324Yc.
I have stated 324E in terms of order-continuous homomorphisms just for a
slight gain in simplicity.   But in fact, when there is a difference, it
is likely that order-continuity, rather than sequential
order-continuity, will be the more significant condition.   Note that
when the domain algebra is $\sigma$-finite, it is ccc (322G),
so the two concepts coincide (316Fd).
     
Of course I need to refer to measures when looking at such concepts as
$\sigma$-finite measure algebra or measure-preserving homomorphism, but
even here the real ideas involved are such notions as
order-continuity and the countable chain condition, as you will see if
you work through 324K.   It is instructive to look at the translations
of these facts into the context of inverse-measure-preserving functions;
see 234B.
     
324H shows that we may speak of `the' topology and uniformity of a
Dedekind $\sigma$-complete Boolean algebra which carries any semi-finite
measure;  the topology of such an algebra is determined by its algebraic
structure.   Contrast this with the theory of normed spaces:  two Banach
spaces (e.g., $\ell^1$ and $\ell^2$) can be isomorphic as linear spaces,
both being of algebraic dimension $\frak c$, while they are not
isomorphic as topological linear spaces.   When we come to the theory of
ordered linear topological spaces, however, we shall again find
ourselves with operators whose algebraic properties guarantee continuity
(355C, 367O).
}%end of notes
     
\discrpage
 
