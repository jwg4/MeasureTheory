\frfilename{mt418.tex}
\versiondate{19.8.05}
\copyrightdate{2000}

\def\undphi{\underline{\phi}}

\def\chaptername{Topologies and measures I}
\def\sectionname{Measurable functions and almost continuous functions}

\newsection{418}

In this section I work through the basic properties of measurable and
almost continuous functions, as defined in 411L and 411M.   I give the
results in
the full generality allowed by the terminology so far introduced, but
most of the ideas are already required even if you are interested only
in Radon measure spaces as the domains of the functions involved.
Concerning the
codomains, however, there is a great difference between metrizable
spaces and others, and among metrizable spaces separability is of
essential importance.

I start with the elementary properties of measurable functions
(418A-418C) and almost continuous functions (418D).
Under mild conditions on
the domain space, almost continuous functions are measurable (418E);
for a separable metrizable codomain, we can expect that measurable
functions should be almost continuous (418J).   Before coming to this, I
spend a couple of paragraphs on image
measures:  a locally finite image
measure under a measurable function is Radon if the measure
on the domain is Radon and the function is almost continuous (418I).

418L-418Q are important results on expressing given Radon measures as
image measures associated with continuous functions, first dealing with
ordinary functions $f:X\to Y$ (418L) and then coming to Prokhorov's
theorem on projective limits of probability spaces (418M).

The machinery of the first part of the section can also be used to
investigate representations of vector-valued functions in terms of
product spaces (418R-418T).

\leader{418A}{Proposition} Let $X$ be a set, $\Sigma$ a $\sigma$-algebra
of subsets of $X$, $Y$ a
topological space and $f:X\to Y$ a measurable function.

(a) $f^{-1}[F]\in\Sigma$ for every Borel set $F\subseteq Y$.

(b) If $A\subseteq X$ is any set, endowed with the subspace
$\sigma$-algebra, then $f\restr A:A\to Y$ is measurable.

(c) Let $(Z,\frak T)$ be another topological space.   Then $gf:X\to Z$
is measurable for every Borel measurable function $g:Y\to Z$;  in
particular, for every continuous function $g:Y\to Z$.

\proof{{\bf (a)} The set $\{F:F\subseteq Y,\,f^{-1}[F]\in\Sigma\}$ is a
$\sigma$-algebra of subsets of $Y$ containing every open set, so
contains every Borel subset of $Y$.

\medskip

{\bf (b)} is obvious from the definition of `subspace $\sigma$-algebra'
(121A).

\medskip

{\bf (c)} If $H\subseteq Z$ is open, then $g^{-1}[H]$ is a Borel subset
of $Y$ so $(gf)^{-1}[H]=f^{-1}[g^{-1}[H]]$ belongs to $\Sigma$.
}%end of proof of 418A

\leader{418B}{Proposition} Let $X$ be a set and $\Sigma$ a
$\sigma$-algebra of subsets of $X$.

(a) If $Y$ is a metrizable space and $\sequencen{f_n}$ is a
sequence of measurable functions from $X$ to $Y$ such that
$f(x)=\lim_{n\to\infty}f_n(x)$ is defined in $Y$ for every $x\in X$,
then $f:X\to Y$ is measurable.

(b) If $Y$ is a topological space, $Z$ is a
separable metrizable space and $f:X\to Y$, $g:X\to Z$ are
functions, then $x\mapsto (f(x),g(x)):X\to Y\times Z$ is measurable iff
$f$ and $g$ are measurable.

(c) If $Y$ is a hereditarily Lindel\"of space, $\Cal U$ a
family of open sets generating its topology, and $f:X\to Y$ a
function such that $f^{-1}[U]\in\Sigma$ for every $U\in\Cal U$, then $f$
is measurable.

(d) If $\familyiI{Y_i}$ is a countable family of separable metrizable
spaces, with product $Y$, then a function $f:X\to Y$ is measurable iff
$\pi_if:X\to Y_i$ is measurable for every $i$, writing $\pi_i(y)=y(i)$
for $y\in Y$, $i\in\Bbb N$.

\proof{{\bf (a)} Let $\rho$ be a metric defining the topology of $Y$.
Let $G\subseteq Y$ be any open set, and for each $n\in\Bbb N$ set

\Centerline{$F_n=\{y:y\in Y,\,\rho(y,z)\ge 2^{-n}$ for every $z\in
Y\setminus G\}$.}

\noindent Then $F_n$ is closed, so $f_i^{-1}[F_n]\in\Sigma$ for every
$n$, $i\in\Bbb N$.   But this means that

\Centerline{$f^{-1}[G]
=\bigcup_{n\in\Bbb N}\bigcap_{i\ge n}f_i^{-1}[F_i]\in\Sigma$.}

\noindent As $G$ is arbitrary, $f$ is measurable.

\medskip

{\bf (b)(i)} The functions $(y,z)\mapsto y$, $(y,z)\mapsto z$ are
continuous, so if $x\mapsto(f(x),g(x))$ is measurable, so are $f$ and
$g$, by 418Ac.

\medskip

\quad{\bf (ii)} Now suppose that $f$ and $g$ are measurable, and that
$W\subseteq Y\times Z$ is open.   By 4A2P(a-i), the topology of $Z$ has
a countable base $\Cal H$;  let $\sequencen{H_n}$ be a sequence running
over $\Cal H\cup\{\emptyset\}$.   For each $n$, set

\Centerline{$G_n=\bigcup\{G:G\subseteq Y$ is open, $G\times H_n\subseteq
W\}$;}

\noindent then $G_n$ is open and $G_n\times H_n\subseteq W$.
Accordingly
$W\supseteq\bigcup_{n\in\Bbb N}G_n\times H_n$.   But in fact
$W=\bigcup_{n\in\Bbb N}G_n\times H_n$.   \Prf\ If $(y,z)\in W$, there
are open sets $G\subseteq Y$, $H\subseteq Z$ such that
$(y,z)\in G\times H\subseteq W$.   Now there is an $n\in\Bbb N$ such
that $z\in H_n\subseteq H$, in which case $G\times H_n\subseteq W$ and
$G\subseteq G_n$ and $(y,z)\in G_n\times H_n$.\ \QeD\

Accordingly

\Centerline{$\{x:(f(x),g(x))\in W\}
=\bigcup_{n\in\Bbb N}f^{-1}[G_n]\cap g^{-1}[H_n]\in\Sigma$.}

\noindent As $W$ is arbitrary, $x\mapsto (f(x),g(x))$ is measurable.

\medskip

{\bf (c)} This is just 4A3Db.

\medskip

{\bf (d)} If $f$ is measurable, so is every $\pi_if$, by 418Ac.   If
every $\pi_if$ is measurable, set

\Centerline{$\Cal U=\{\pi_i^{-1}[H]:i\in I,\,H\subseteq Y_i$ is
open$\}$.}

\noindent Then $\Cal U$ generates the topology of $Y$, and if
$U=\pi_i^{-1}[H]$ then $f^{-1}[U]=(\pi_if)^{-1}[H]$, so
$f^{-1}[U]\in\Sigma$ for every $U$.   Also $Y$ is hereditarily
Lindel\"of (4A2P(a-iii)), so $f$ is measurable, by (c).
}%end of proof of 418B

\leader{418C}{Proposition} Let $(X,\Sigma,\mu)$ be a measure space and
$Y$ a Polish space.   Let $\sequencen{f_n}$ be a
sequence of measurable functions from $X$ to $Y$.   Then

\Centerline{$\{x:x\in X,\,\lim_{n\to\infty}f_n(x)$ is defined in $Y\}$}

\noindent belongs to $\Sigma$.

\proof{(Compare 121H.) Let $\rho$ be a complete metric on $Y$ defining
the topology of $Y$.

\medskip

{\bf (a)} For $m$, $n\in\Bbb N$ and $\delta>0$, the set
$\{x:\rho(f_m(x),f_n(x))\le\delta\}$ belongs to $\Sigma$.   \Prf\ The
function $x\mapsto (f_m(x),f_n(x)):X\to Y^2$ is measurable, by 418Bb,
and the function $\rho:Y^2\to\Bbb R$ is continuous, so
$x\mapsto\rho(f_m(x),f_n(x))$ is measurable and
$\{x:\rho(f_m(x),f_n(x))\le\delta\}\in\Sigma$.\ \Qed

\medskip
{\bf (b)} Now $\sequencen{f_n(x)}$ is convergent iff it is Cauchy,
because
$Y$ is complete.   But

$$\eqalign{\{x:x\in X,\,&\sequencen{f_n(x)}\text{ is Cauchy}\}
=\bigcap_{n\in\Bbb N}\bigcup_{m\in\Bbb N}\bigcap_{i\ge m}
\{x:\rho(f_i(x),f_m(x))\le 2^{-n}\}\cr}$$

\noindent belongs to $\Sigma$.
}%end of proof of 418C

\leader{418D}{Proposition} Let $(X,\Sigma,\mu)$ be a measure space and
$\frak T$ a topology on $X$.

(a) Suppose that $Y$ is a topological space.   Then any continuous
function from $X$ to $Y$ is almost continuous.

(b) Suppose that $Y$ and $Z$ are topological spaces, $f:X\to Y$ is
almost continuous and $g:Y\to Z$ is continuous.   Then $gf:X\to Z$ is
almost continuous.

(c) Suppose that $(Y,\frak S,\Tau,\nu)$ is a $\sigma$-finite topological
measure space, $Z$ is a topological space, $g:Y\to Z$ is almost
continuous and $f:X\to Y$ is \imp\ and almost continuous.   Then
$gf:X\to Z$ is almost continuous.

(d) Suppose that $\mu$ is semi-finite, and that $\familyiI{Y_i}$ is a
countable family of topological spaces with product $Y$.   Then a
function $f:X\to Y$
is almost continuous iff $f_i=\pi_if$ is almost continuous for every
$i\in I$, writing $\pi_i(y)=y(i)$ for $i\in I$, $y\in Y$.

\proof{{\bf (a)} is trivial.

\medskip

{\bf (b)} The set $\{A:A\subseteq X,\,gf\restr A$ is continuous$\}$
includes
$\{A:A\subseteq X,\,f\restr A$ is continuous$\}$;  so if $\mu$ is inner
regular with respect to the latter, it is inner regular with respect to
the former.

\medskip

{\bf (c)} Take $E\in\Sigma$ and $\gamma<\mu E$;  take $\epsilon>0$.   We
have a cover of $Y$ by a non-decreasing sequence $\sequencen{Y_n}$ of
measurable sets of finite measure;  now $\sequencen{f^{-1}[Y_n]}$ is a
non-decreasing sequence covering $E$, so there is an $n\in\Bbb N$ such
that $\mu(E\cap f^{-1}[Y_n])\ge\gamma$.   Because $f$ is \imp, $E\cap
f^{-1}[Y_n]$ has finite measure.   Now we can find measurable sets
$F\subseteq Y_n$, $E_1\subseteq E\cap f^{-1}[Y_n]$ such that $f\restr
E_1$, $g\restr F$ are continuous and
$\nu F\ge\nu Y_n-\epsilon$, $\mu E_1\ge\mu(E\cap f^{-1}[Y_n]\setminus
E_1)-\epsilon$.   In this case $E_0=E_1\cap f^{-1}[F]$ has measure at
least $\gamma-2\epsilon$ and $gf\restr E_0$ is continuous.   As $E$,
$\gamma$ and $\epsilon$ are arbitrary, $gf$ is almost continuous.

\medskip

{\bf (d)(i)} If $f$ is almost continuous, every $f_i$ must be almost
continuous, by (b).

\medskip

\quad{\bf (ii)} Now suppose that every $f_i$ is almost continuous.
Take $E\in\Sigma$ and $\gamma<\mu E$.   There is an
$E_0\subseteq E$ such that $E_0\in\Sigma$ and $\gamma<\mu E_0<\infty$.
Let $\langle\epsilon_i\rangle_{i\in I}$ be a family of strictly positive
real numbers such that $\sum_{i\in I}\epsilon_i\le\mu E_0-\gamma$.   For
each $i\in I$ choose a measurable set $F_i\subseteq E_0$ such that
$\mu F_i\ge\mu E_0-\epsilon_i$ and $f_i\restr F_i$ is continuous.   Then
$F=E_0\cap\bigcap_{i\in I}F_i$ is a subset of $E$ with measure at least
$\gamma$, and $f\restr F$ is continuous because $f_i\restr F$ is
continuous for every $i$ (3A3Ib).
}%end of proof of 418D

\leader{418E}{Theorem} Let $(X,\frak T,\Sigma,\mu)$ be a complete
locally determined topological measure space, $Y$ a topological space,
and $f:X\to Y$ an almost continuous function.   Then $f$ is measurable.

\proof{ Set $\Cal K=\{K:K\in\Sigma,\,f\restr K$ is continuous$\}$;  then
$\mu$ is inner regular with respect to $\Cal K$.   If $H\subseteq Y$ is
open and $K\in\Cal K$, then $K\cap f^{-1}[H]$ is relatively open in $K$,
that is, there is an open set $G\subseteq X$ such that
$K\cap f^{-1}[H]=K\cap G$.
Because $\mu$ is a topological measure, $G\in\Sigma$ so
$K\cap f^{-1}[H]\in\Sigma$.   As $K$ is arbitrary, and $\mu$ is complete
and locally determined, $f^{-1}[H]\in\Sigma$ (412Ja).   As $H$ is
arbitrary, $f$ is measurable.
}%end of proof of 418E

\leader{418F}{Proposition} Let $(X,\frak T,\Sigma,\mu)$ be a
semi-finite topological measure space, $Y$ a metrizable space, and
$f:X\to Y$ a function.   Suppose there is a sequence $\sequencen{f_n}$
of almost continuous functions from $X$ to $Y$ such that
$f(x)=\lim_{n\to\infty}f_n(x)$ for almost every $x\in X$.   Then $f$ is
almost continuous.

\proof{ Suppose that $E\in\Sigma$ and that $\gamma<\mu E$, $\epsilon>0$.
Then there is a measurable set $F\subseteq E$ such that
$\gamma\le\mu F<\infty$;  discarding a negligible set if necessary, we
may arrange that $f(x)=\lim_{n\to\infty}f_n(x)$ for every $x\in F$.
Let $\rho$ be a metric on $Y$ defining its topology.   For each
$n\in\Bbb N$, let $F_n\subseteq F$ be a measurable set such that
$f_n\restr F_n$ is continuous and
$\mu(F_n\setminus F)\le 2^{-n}\epsilon$;  set
$G=\bigcap_{n\in\Bbb N}F_n$, so that $\mu G\ge\gamma-2\epsilon$ and
$f_n\restr G$ is continuous for every $n\in\Bbb N$.

For $m$, $n\in\Bbb N$, the functions $x\mapsto(f_m(x),f_n(x)):G\to Y^2$
and $x\mapsto\rho(f_m(x),f_n(x)):G\to\Bbb R$ are continuous, therefore
measurable, because $\mu$ is a topological measure.   Also
$\sequencen{f_n(x)}$ is a Cauchy sequence for every $x\in G$.   So if we
set $G_{kn}
=\{x:x\in G,\,\rho(f_i(x),f_j(x))\le 2^{-k}$ for all $i$, $j\ge n\}$,
$\sequencen{G_{kn}}$ is a non-decreasing sequence of measurable sets
with union $G$ for each $k\in\Bbb N$, and we can find a strictly
increasing sequence $\sequence{k}{n_k}$ such that $\mu(G\setminus
G_{kn_k})\le 2^{-k}\epsilon$ for every $k$.   Setting
$H=\bigcap_{k\in\Bbb N}G_{kn_k}$,
$\mu H\ge\mu G-2\epsilon\ge\gamma-4\epsilon$ and
$\rho(f_i(x),f_{n_k}(x))\le 2^{-k}$ whenever $x\in H$ and $i\ge n_k$;
consequently $\rho(f(x),f_{n_k}(x))\le 2^{-k}$ whenever $x\in H$ and
$k\in\Bbb N$.   But this means that $\sequence{k}{f_{n_k}}$ converges to
$f$ uniformly on $H$, while every $f_{n_k}$ is continuous on $H$, so
$f\restr H$ is continuous (3A3Nb).   And of course $H\subseteq E$.

As $E$, $\gamma$ and $\epsilon$ are arbitrary, $f$ is almost continuous.
}%end of proof of 418F
%4@43

\leader{418G}{Proposition} Let $(X,\frak T,\Sigma,\mu)$ be a
$\sigma$-finite quasi-Radon measure space, $Y$ a metrizable space and
$f:X\to Y$ an almost continuous function.   Then there is a conegligible
set $X_0\subseteq X$ such that $f[X_0]$ is separable.

\proof{{\bf (a)} Let $\Cal K$ be the family of self-supporting
measurable sets $K$ of finite measure such that $f\restr K$ is
continuous.   Then $\mu$ is inner regular with respect to $\Cal K$.
\Prf\ If $E\in\Sigma$ and $\gamma<\mu E$, there is an $F\in\Sigma$ such
that $F\subseteq E$ and $\gamma<\mu F<\infty$;  there is an $H\in\Sigma$
such that $H\subseteq F$, $\gamma\le\mu H$ and $f\restr H$ is
continuous;  and there is a measurable self-supporting $K\subseteq H$
with the same measure as $H$ (414F), in which case $K\in\Cal K$ and
$K\subseteq E$ and $\mu K\ge\gamma$.\ \Qed

\medskip

{\bf (b)} Now $f[K]$ is ccc for every $K\in\Cal K$.   \Prf\ If $\Cal G$
is a disjoint family of non-empty relatively open subsets of $f[K]$,
then $\family{G}{\Cal G}{K\cap f^{-1}[G]}$ is a disjoint family of
non-empty relatively open subsets of $K$, because $f\restr K$ is
continuous, and $\sum_{G\in\Cal G}\mu(K\cap f^{-1}[G])\le\mu K$.
Because $K$ is self-supporting, $\mu(K\cap f^{-1}[G])>0$ for every
$G\in\Cal G$;  because $\mu K$ is finite, $\Cal G$ is countable.   As
$\Cal G$ is arbitrary, $f[K]$ is ccc.\ \Qed

Because $Y$ is metrizable, $f[K]$ must be separable (4A2Pd).

\medskip

{\bf (c)} Because $\mu$ is $\sigma$-finite, there is a countable family
$\Cal L\subseteq\Cal K$ such that $X_0=\bigcup\Cal L$ is conegligible
(412Ic).   Now $f[X_0]=\bigcup_{L\in\Cal L}f[L]$ is a countable union of
separable spaces, so is separable (4A2B(e-i)).
}%end of proof of 418G

\leader{418H}{Proposition} (a) Let $X$ and $Y$ be topological spaces,
$\mu$ an effectively locally finite
$\tau$-additive measure on $X$, and $f:X\to Y$ an almost continuous
function.   Then the image measure $\mu f^{-1}$ is $\tau$-additive.

(b) Let $(X,\frak T,\Sigma,\mu)$ be a totally
finite quasi-Radon measure space, $(Y,\frak S)$ a regular topological
space, and $f:X\to Y$ an almost continuous function.   Then there is a
unique quasi-Radon measure $\nu$ on $Y$ such that $f$ is \imp\ for $\mu$
and $\nu$.

\proof{{\bf (a)} Let $\Cal H$ be an upwards-directed family of open
subsets of $Y$, all measured by $\mu f^{-1}$, and suppose that
$H^*=\bigcup\Cal H$ also is measurable.
Take any $\gamma<(\mu f^{-1})(H^*)=\mu f^{-1}[H^*]$.   Then there is a
measurable set $E\subseteq f^{-1}[H^*]$ such that $\mu E\ge\gamma$ and
$f\restr E$ is continuous.
Consider $\{E\cap f^{-1}[H]:H\in\Cal H\}$.   This is an upwards-directed
family of relatively open measurable subsets of $E$ with measurable
union $E$.   By 414K, the subspace measure on $E$ is $\tau$-additive, so

\Centerline{$\gamma\le\mu E
\le\sup_{H\in\Cal H}\mu(E\cap f^{-1}[H])
\le\sup_{H\in\Cal H}\mu f^{-1}[H]$.}

\noindent As $\gamma$ is arbitrary,
$\mu f^{-1}[H^*]\le\sup_{H\in\Cal H}\mu f^{-1}[H]$;  as $\Cal H$ is
arbitrary, $\mu f^{-1}$ is $\tau$-additive.

\medskip

{\bf (b)} By 418E, $f$ is measurable.   Let $\nu_0$ be the restriction
of $\mu f^{-1}$ to the Borel $\sigma$-algebra of $Y$;  by (a), $\nu_0$
is $\tau$-additive, and $f$ is \imp\ with respect to $\mu$ and $\nu_0$.
Because $Y$ is regular, the completion $\nu$ of $\nu_0$ is a quasi-Radon
measure (415Cb).   Because $\mu$ is complete, $f$ is still \imp\ with
respect to $\mu$ and $\nu$ (234Ba\formerly{2{}35Hc}).

To see that $\nu$ is unique, observe that its values on Borel sets are
determined by the requirement that $f$ be \imp, so that 415H gives the
result.
}%end of proof of 418H

\leader{418I}{}\cmmnt{ The next theorem is one of the central
properties of Radon measures.   I have already presented what amounts to
a special case in 256G.

\medskip

\noindent}{\bf Theorem} Let $(X,\frak T,\Sigma,\mu)$ be a Radon measure
space, $Y$ a Hausdorff space, and $f:X\to Y$ an almost continuous
function.   If the image measure $\nu=\mu f^{-1}$ is locally finite, it
is a Radon measure.

\proof{{\bf (a)} By 418E, $f$ is measurable, that is,
$f^{-1}[H]\in\Sigma$
for every open set $H\subseteq Y$;  but this means that the domain
$\Tau$ of $\nu$ contains every open set, and $\nu$ is a topological
measure.

\medskip

{\bf (b)} $\nu$ is inner regular with respect to the
compact sets.
\Prf\ If $F\in\Tau$ and $\nu F>0$, then $\mu f^{-1}[F]>0$, so there is
an $E\subseteq f^{-1}[F]$ such that $\mu E>0$ and $f\restr E$ is
continuous.
Next, there is a compact set $K\subseteq E$ such that $\mu K>0$.   In
this case, $L=f[K]$ is a compact subset of $F$, and

\Centerline{$\nu L=\mu f^{-1}[L]\ge\mu K>0$.}

\noindent By 412B, this is enough to prove that $\nu$ is tight.\ \QeD\
Note that because $\nu$ is
locally finite, $\nu L<\infty$ for every compact $L\subseteq Y$ (411Ga).

\medskip

{\bf (c)} Because $\mu$ is complete, so is $\nu$ (234Eb\formerly{2{}12Bd}).
Next, $\nu$ is locally
determined.   \Prf\ Suppose that $H\subseteq Y$ is such that
$H\cap F\in\Tau$ whenever $\nu F<\infty$.   Then, in particular,
$H\cap f[K]\in\Tau$ whenever $K\subseteq X$ is compact and $f\restr K$
is continuous.   But setting

\Centerline{$\Cal K=\{K:K\subseteq X$ is compact, $f\restr K$ is
continuous$\}$,}

\noindent $\mu$ is inner regular with respect to $\Cal K$ (412Ac).
And if $K\in\Cal K$,

\Centerline{$K\cap f^{-1}[H]=K\cap f^{-1}[H\cap f[K]]\in\Sigma$.}

\noindent Because $\mu$ is complete and locally determined, this is
enough to show that $f^{-1}[H]\in\Sigma$ (412Ja), that is, $H\in\Tau$.
As $H$ is arbitrary, $\nu$ is locally determined.\ \Qed

\medskip

{\bf (d)} Thus $\nu$ is a complete locally determined locally
finite topological measure which is inner regular with respect to the
compact sets;  that is, it is a Radon measure.
}%end of proof of 418I

\leader{418J}{Theorem} Let $(X,\Sigma,\mu)$ be a semi-finite measure
space and $\frak T$ a topology on $X$ such that $\mu$ is inner regular
with respect to the closed sets.   Suppose that $Y$ is a
second-countable space\cmmnt{ (for instance, $Y$ might be separable and 
metrizable),} and $f:X\to Y$ is measurable.   
Then $f$ is almost continuous.

\proof{ Let $\Cal H$ be a countable base for the topology of $Y$, and
$\sequencen{H_n}$ a sequence running over $\Cal H\cup\{\emptyset\}$.
Take $E\in\Sigma$ and $\gamma<\mu E$.   Choose $\sequencen{E_n}$
inductively, as
follows.   There is an $E_0\in\Sigma$ such that $E_0\subseteq E$ and
$\gamma<\mu E_0<\infty$.   Given $E_n\in\Sigma$ with
$\gamma<\mu E_n<\infty$, $E_n\setminus f^{-1}[H_n]\in\Sigma$,
so there is a closed set $F_n\in\Sigma$ such that

\Centerline{$F_n\subseteq E_n\setminus f^{-1}[H_n]$,
\quad$\mu((E_n\setminus f^{-1}[H_n])\setminus F_n)<\mu E_n-\gamma$;}

\noindent set $E_{n+1}=(E_n\cap f^{-1}[H_n])\cup F_n$, so that

\Centerline{$E_{n+1}\in\Sigma$,
\quad $E_{n+1}\subseteq E_n$,
\quad$\mu E_{n+1}>\gamma$,
\quad$E_{n+1}\setminus f^{-1}[H_n]=F_n$.}

\noindent Continue.

At the end of the induction, set $F=\bigcap_{n\in\Bbb N}E_n$.   Then
$F\subseteq E$, $\mu F\ge\gamma$, and for every $n\in\Bbb N$

\Centerline{$F\cap f^{-1}[H_n]=F\cap E_{n+1}\cap f^{-1}[H_n]
=F\setminus F_n$}

\noindent is relatively open in $F$.   It follows that $f\restr F$ is
continuous (4A2B(a-ii)).   As $E$, $\gamma$ are arbitrary, $f$ is almost
continuous.
}%end of proof of 418J

\cmmnt{\medskip

\noindent{\bf Remark} For variations on this idea, see 418Yg, 433E and
434Yb;  also 418Yh.
}

\leader{418K}{Corollary} Let $(X,\frak T,\Sigma,\mu)$ be a quasi-Radon
measure space and $Y$ a separable metrizable space.   Then a function
$f:X\to Y$ is measurable iff it is almost continuous.

\proof{ Put 418E and 418J together.
}%end of proof of 418K

\cmmnt{\medskip

\noindent{\bf Remark} This generalizes 256F.
}%end of comment

\leader{418L}{}\cmmnt{ In all the results above, the measure starts on
the left of the diagram $f:X\to Y$;  in 418H-418I,
it is transferred to an image measure on
$Y$.   If $X$ has enough compact sets, a measure can move in the reverse
direction, as follows.

\medskip

\noindent}{\bf Theorem} Let $(X,\frak T)$ be a Hausdorff space,
$(Y,\frak S,\Tau,\nu)$ a Radon measure space and $f:X\to Y$ a continuous
function such that whenever $F\in\Tau$ and $\nu F>0$ there is a compact
set $K\subseteq X$
such that $\nu(F\cap f[K])>0$.   Then there is a Radon measure $\mu$ on
$X$ such that $\nu$ is the image measure $\mu f^{-1}$ and the \imp\
function $f$ induces an isomorphism between the measure algebras of
$\nu$ and $\mu$.

\proof{{\bf (a)} Note first that $\nu$ is inner regular with respect to
$\Cal L=\{f[K]:K\in\Cal K\}$, where $\Cal K$ is the family of compact
subsets of $X$.   \Prf\ If $\nu F>0$, there is a $K\in\Cal K$ such that
$\nu(F\cap f[K])>0$;  now there is a closed set $F'\subseteq F\cap f[K]$
such that $\nu F'>0$, and $K'=K\cap f^{-1}[F']$ is compact, while
$f[K']\subseteq F$ has non-zero measure.   As $\Cal L$ is closed under
finite unions, this is enough to show that $\nu$ is inner regular with
respect to $\Cal L$ (412Aa).\ \Qed

\medskip

{\bf (b)} Consequently there is a disjoint set $\Cal L_0\subseteq\Cal L$
such that every non-negligible $F\in\Tau$ meets some member of $\Cal L_0$
in a non-negligible set (412Ib).   We can express $\Cal L_0$ as
$\{f[K]:K\in\Cal K_0\}$ where $\Cal K_0\subseteq\Cal K$ is disjoint.
Set $X_0=\bigcup\Cal K_0$.

\medskip

{\bf (c)} Set

\Centerline{$\Sigma_0=\{X_0\cap f^{-1}[F]:F\in\Tau\}$.}

\noindent Then $\Sigma_0$ is a $\sigma$-algebra of subsets of $X_0$.
If $F$, $F'\in\Tau$ and $\nu F\ne\nu F'$, then there must be some
$K\in\Cal
K_0$ such that $f[K]\cap(F\symmdiff F')\ne\emptyset$, so that
$X_0\cap f^{-1}[F]\ne X_0\cap f^{-1}[F']$;  we therefore have a
functional
$\mu_0:\Sigma_0\to[0,\infty]$ defined by setting
$\mu_0(X_0\cap f^{-1}[F])=\nu F$ whenever $F\in\Tau$.   It is easy to
check that $\mu_0$ is a measure on $X_0$.   Now $\mu_0$ is inner regular
with respect to $\Cal K$.   \Prf\ If $E\in\Sigma_0$ and $\mu E>0$, there
is an $F\in\Tau$ such that $E=X_0\cap f^{-1}[F]$ and $\nu F>0$.   There
are a $K\in\Cal K_0$ such that $\nu(F\cap f[K])>0$, and a closed set
$F'\subseteq F\cap f[K]$ such that $\nu F'>0$;  now
$K\cap f^{-1}[F']=X_0\cap f^{-1}[F']$ belongs to $\Sigma_0\cap\Cal K$,
is included in $E$ and has measure greater than $0$.   Because $\Cal K$
is closed under finite unions, this is enough to show that $\mu_0$ is
inner regular with respect to $\Cal K$.\ \Qed

\medskip

{\bf (d)} Set

\Centerline{$\Sigma_1=\{E:E\subseteq X,\,E\cap X_0\in\Sigma_0\}$,
\quad$\mu_1E=\mu_0(E\cap X_0)$ for every $E\in\Sigma_1$.}
\noindent Then $\mu_1$ is a measure on $X$ (being the image measure
$\mu_0\iota^{-1}$, where $\iota:X_0\to X$ is the identity map), and is
inner regular with respect to $\Cal K$.   If $F\in\Tau$, then

\Centerline{$\mu_1 f^{-1}[F]=\mu_0(X_0\cap f^{-1}[F])=\nu F$,}

\noindent so $f$ is \imp\ for $\mu_1$ and $\nu$.   Consequently $\mu_1$
is locally finite.   \Prf\ If $x\in X$, there is an open set $H\subseteq
Y$ such that $f(x)\in H$ and $\nu H<\infty$;  now $f^{-1}[H]$ is an open
subset of $X$ of finite measure containing $x$.\ \QeD\  In particular,
$\mu_1^*K<\infty$ for every compact $K\subseteq X$ (411Ga).

\medskip

{\bf (e)} By 413O, there is an extension of $\mu_1$ to a complete
locally determined measure $\mu$ on $X$ which is inner regular with
respect to $\Cal K$, defined on every member of $\Cal K$, and such that
whenever $E$
belongs to the domain $\Sigma$ of $\mu$ and $\mu E<\infty$, there is an
$E_1\in\Sigma_1$ such that $\mu(E\symmdiff E_1)=0$.   Now $\mu$ is
locally finite because $\mu_1$ is, so $\mu$ is a Radon measure;  and $f$
is \imp\ for $\mu$ and $\nu$ because it is \imp\ for $\mu_1$ and $\nu$.

\medskip

{\bf (f)} The image measure $\mu f^{-1}$ extends $\nu$, so is locally
finite, and is therefore a Radon measure (418I);  since it agrees with
$\nu$ on the compact subsets of $Y$, it must be identical with $\nu$.

\medskip

{\bf (g)} I have still to check that the corresponding
measure-preserving
homomorphism $\pi$ from the measure algebra $\frak B$ of $\nu$ to the
measure algebra $\frak A$ of $\mu$ is actually an isomorphism, that is,
is surjective.   If $a\in\frak A$ and $\bar\mu a<\infty$, we can find
$E\in\Sigma$ such that $E^{\ssbullet}=a$ and $E_1\in\Sigma_1$ such that
$\mu(E\symmdiff E_1)=0$.   Now $E_1\cap X_0=f^{-1}[F]\cap X_0$ for some
$F\in\Tau$;  but in this case

\Centerline{$\mu(E_1\symmdiff f^{-1}[F])
=\mu_1(E_1\symmdiff f^{-1}[F])=0$,
\quad$a=E_1^{\ssbullet}=(f^{-1}[F])^{\ssbullet}=\pi F^{\ssbullet}$.}

Accordingly $\pi[\frak B]$ includes $\{a:\bar\mu a<\infty\}$, and is
order-dense in $\frak A$.   But as $\pi$ is injective and $\frak B$ is
Dedekind complete (being the measure algebra of a Radon measure, which
is strictly localizable), it follows that $\pi[\frak B]=\frak A$
(314Ia).   Thus $\pi$ is an isomorphism, as required.
}%end of proof of 418L

\cmmnt{\medskip

\noindent{\bf Remarks} Of course this result is most commonly applied
when $X$ and $Y$ are both compact and $f$ is a surjection, in which case
the condition

\inset{(*) whenever $F\in\Tau$ and $\nu F>0$ there is a compact set
$K\subseteq X$ such that $\nu(F\cap f[K])>0$}

\noindent is trivially satisfied.

Evidently (*) is necessary if there is to be any Radon measure on $X$
for which $f$ is \imp, so in this sense the result is best possible.
In 433D, however, there is a version of the theorem in which $f$ is not
required to be continuous.
}%end of comment

\leader{418M}{Prokhorov's theorem} Suppose that $(I,\le)$,
$\familyiI{(X_i,\frak T_i,\Sigma_i,\mu_i)}$,
$\langle f_{ij}\rangle_{i\le j\in I}$,
$(X,\frak T)$ and $\familyiI{g_i}$ are such that

\inset{$(I,\le)$ is a non-empty upwards-directed partially ordered set,}

\inset{every $(X_i,\frak T_i,\Sigma_i,\mu_i)$ is a Radon probability
space,}

\inset{$f_{ij}:X_j\to X_i$ is an \imp\ function whenever $i\le j$ in
$I$,}

\inset{$(X,\frak T)$ is a Hausdorff space,}

\inset{$g_i:X\to X_i$ is a continuous function for every $i\in I$,}

\inset{$g_i=f_{ij}g_j$ whenever $i\le j$ in $I$.}

\noindent Suppose moreover that

\inset{for every $\epsilon>0$ there is a compact set
$K\subseteq X$ such that $\mu_ig_i[K]\ge 1-\epsilon$ for every $i\in
I$.}

\noindent Then there is a Radon probability measure $\mu$ on $X$ such
that every $g_i$ is \imp\ for $\mu$.   If the family $\familyiI{g_i}$
separates the points of $X$, then $\mu$ is uniquely defined.

\proof{{\bf (a)} Set

\Centerline{$\Tau=\{g_i^{-1}[E]:i\in I,\,E\in\Sigma_i\}\subseteq\Cal PX$.}

\noindent Then $\Tau$ is a subalgebra of $\Cal PX$.   \Prf\ (i) There is
an $i\in I$, so $\emptyset=g_i^{-1}[\emptyset]$ belongs to $\Tau$.
(ii) If $H\in\Tau$ there are $i\in I$, $E\in\Sigma_i$ such that
$H=g_i^{-1}[E]$;
now $X\setminus H=g_i^{-1}[X_i\setminus E]$ belongs to $\Tau$.   (iii)
If $G$, $H\in\Tau$, there are $i$, $j\in I$ and $E\in\Sigma_i$,
$F\in\Sigma_j$
such that $G=g_i^{-1}[E]$ and $H=g_j^{-1}[F]$.   Now $I$ is
upwards-directed, so there is a $k\in I$ such that $i\le k$ and $j\le
k$.   Because $f_{ik}$ and $f_{jk}$ are \imp, $f_{ik}^{-1}[E]$ and
$f_{jk}^{-1}[F]$ belong to $\Sigma_k$, so that

$$\eqalign{G\cap H
&=g_i^{-1}[E]\cap g_j^{-1}[F]
=(f_{ik}g_k)^{-1}[E]\cap (f_{jk}g_k)^{-1}[F]\cr
&=g_k^{-1}[f_{ik}^{-1}[E]\cap f_{jk}^{-1}[F]]
\in\Tau.  \text{ \Qed}\cr}$$

\medskip

{\bf (b)} There is an additive functional $\nu:\Tau\to[0,1]$ defined by
writing $\nu g_i^{-1}[E]=\mu_iE$ whenever $i\in I$ and $E\in\Sigma_i$.

\medskip

\Prf\ {\bf (i)} Suppose that $i$, $j\in I$ and $E\in\Sigma_i$,
$F\in\Sigma_j$ are such that
$g_i^{-1}[E]=g_j^{-1}[F]$.   Let $k\in I$ be such that $i\le k$ and
$j\le k$.   Then

\Centerline{$g_k^{-1}[f_{ik}^{-1}[E]\symmdiff f_{jk}^{-1}[F]]
=g_i^{-1}[E]\symmdiff g_j^{-1}[F]=\emptyset$,}

\noindent so $g_k[X]\cap(f_{ik}^{-1}[E]
\symmdiff f_{jk}^{-1}[F])=\emptyset$.   But now remember that for every
$\epsilon>0$ there is a set $K\subseteq X$ such that
$\mu_kg_k[K]\ge 1-\epsilon$.   This means that $\mu_kg_k[X]$ must be
$1$, so that $f_{ik}^{-1}[E]\symmdiff f_{jk}^{-1}[F]$ must be
negligible, and

\Centerline{$\mu_iE=\mu_kf_{ik}^{-1}[E]=\mu_kf_{jk}^{-1}[F]=\mu_jF$.}

\noindent Thus the proposed formula for $\nu$ defines a function on
$\Tau$.

\medskip

\quad{\bf (ii)} Now suppose that $G$, $H\in\Tau$ are disjoint.   Again,
take $i$, $j\in I$ and $E\in\Sigma_i$, $F\in\Sigma_j$ such that
$G=g_i^{-1}[E]$ and $H=g_j^{-1}[F]$, and $k\in I$ such that $i\le k$ and
$j\le k$.   Then

$$\eqalign{\nu G+\nu H
&=\mu_iE+\mu_jF
=\mu_kf_{ik}^{-1}[E]+\mu_kf_{jk}^{-1}[F]\cr
&=\mu_k(f_{ik}^{-1}[E]\cup f_{jk}^{-1}[F])
  +\mu_k(f_{ik}^{-1}[E]\cap f_{jk}^{-1}[F])\cr
&=\nu g_k^{-1}[f_{ik}^{-1}[E]\cup f_{jk}^{-1}[F]]
  +\nu g_k^{-1}[f_{ik}^{-1}[E]\cap f_{jk}^{-1}[F]]\cr
&=\nu(G\cup H)+\nu(G\cap H).\cr}$$

\noindent But as $\nu\emptyset$ is certainly $0$, we get
$\nu(G\cup H)=\nu G+\nu H$.   As $G$, $H$ are arbitrary, $\nu$ is
additive.\ \Qed

Note that $\nu X=1$.

\medskip

{\bf (c)} $\nu G=\sup\{\nu H:H\in\Tau$, $H\subseteq G$, $H$ is closed$\}$
for every $G\in\Tau$.   \Prf\ If
$\gamma<\nu G$, there are an $i\in I$ and an $E\in\Sigma_i$ such that
$G=g_i^{-1}[E]$.
In this case $\mu_iE=\nu G>\gamma$;  let $L\subseteq E$ be a compact set
such that
$\mu_iL\ge\gamma$;  then $H=g_i^{-1}[L]$ is a closed subset of $G$ and
$\nu H=\mu_iL\ge\gamma$.\ \Qed

$\nu X=\sup_{K\subseteq X\text{ is compact}}
  \inf_{G\in\Tau,G\supseteq K}\nu G$.   \Prf\
If $\epsilon>0$, there is a compact $K\subseteq X$ such that
$\mu_ig_i[K]\ge 1-\epsilon$
for every $i\in I$, by the final hypothesis of the theorem.   If
$G\in\Tau$ and $G\supseteq K$,
there are an $i\in I$ and an $E\in\Sigma_i$ such that $G=g_i^{-1}[E]$,
in which case
$g_i[K]\subseteq E$, so that

\Centerline{$\nu G=\mu_iE\ge\mu_ig_i[K]\ge 1-\epsilon$.}

\noindent Thus $\inf_{G\in\Tau,G\supseteq K}\nu G\ge 1-\epsilon$;  as
$\epsilon$ is arbitrary,
we have the result.\ \Qed

This means that the conditions of 416O are satisfied, and there is a
Radon measure $\mu$
extending $\nu$.   Of course this means that every $g_i$ is \imp.

\medskip

{\bf (d)} Now suppose that $\familyiI{g_i}$ separates the points of $X$.
Then whenever $K$, $L\subseteq X$ are disjoint there is an $i\in I$ such
that $g_i[K]\cap g_i[L]=\emptyset$.   \Prf\ Set
$V_i=\{(x,y):x\in K$, $y\in L$, $g_i(x)=g_i(y)\}$ for $i\in I$.
Because $g_i$ is continuous and $\frak T_i$ is Hausdorff, $V_i$ is closed.
If $i\le j$ in $I$, then $g_i=f_{ij}g_i$ so $V_j\subseteq V_i$;
accordingly $\familyiI{V_i}$ is downwards-directed.   Because
$\familyiI{g_i}$ separates the points of $X$, $\bigcap_{i\in I}V_i$ is
empty.   As $K\times L$ is compact, there is an $i\in I$ such that
$V_i=\emptyset$, that is, $g_i[K]$ and $g_i[L]$ are disjoint.\ \Qed

Let $\nu$ be any Radon probability
measure on $X$ such that $g_i$ is \imp\ for $\nu$ and $\mu_i$ for
every $i\in I$.   Let $K\subseteq X$ be compact.   \Quer\ If $\mu K<\nu K$
then there is a compact $L\subseteq X\setminus K$ such that
$\mu L+\nu K>1$.   Let $i\in I$ be such that $g_i[K]\cap g_i[L]=\emptyset$;
then

\Centerline{$1<\mu L+\nu K
\le\mu g_i^{-1}[g_i[L]]+\nu g_i^{-1}[g_i[K]]
=\mu_ig_i[L]+\mu_ig_i[K]
\le 1$,}

\noindent which is impossible.\ \BanG\   So $\nu K\le\mu K$.   Similarly,
$\mu K\le\nu K$.   By 416Eb, $\mu=\nu$.
Thus $\mu$ is uniquely determined.
}%end of proof of 418M

\cmmnt{
\leader{418N}{Remarks (a)} Taking $I$ to be a singleton,
we get a version of 418L in which $Y$ is a probability space, and
omitting the check that the function $g$ induces an isomorphism of the
measure algebras.
Taking $I$ to be the family of finite subsets of a set $T$, and every
$X_i$ to be a product $\prod_{t\in i}Z_t$ of Radon probability
spaces with its product Radon measure, we obtain a method of
constructing
products of arbitrary families of compact probability spaces from finite
products.

\spheader 418Nb In the hypotheses of 418M, I asked only that
the $f_{ij}$ should be measurable, and omitted any check on the
compositions
$f_{ij}f_{jk}$ when $i\le j\le k$.   But it is easy to see that the
$f_{ij}$ must in fact be almost continuous, and that $f_{ij}f_{jk}$ must
be equal almost everywhere to $f_{ik}$ (418Xt), just as in 418P below.

\spheader 418Nc In the theorem as written out above, the space
$X$ and the functions $g_i:X\to X_i$ are part of the data.   Of course
in many applications we start with a structure

\Centerline{$(\familyiI{(X_i,\frak T_i,\Sigma_i,\mu_i)},
\langle f_{ij}\rangle_{i\le j\in I})$,}

\noindent and the first step is to find a suitable $X$ and $g_i$, as in
418O and 418P.

\spheader 418Nd There are important questions concerning possible
relaxations of the hypotheses in 418M, especially in the special case
already mentioned, in which $X_i=\prod_{t\in i}Z_t$,
$f_{ij}(x)=x\restr i$ when $i\subseteq j\in[T]^{<\omega}$,
$X=\prod_{t\in T}Z_t$, and
$g_i(x)=x\restr i$ for $x\in X$ and $i\in I$, but there is no suggestion
that the $\mu_i$ are product measures.
For a case in which we can dispense with auxiliary topologies on the
$X_i$, see 451Yb.

\spheader 418Ne A typical class of applications of Prokhorov's
theorem is in the theory of stochastic processes, in which we have large
families $\family{t}{T}{X_t}$ of random variables;  for definiteness,
imagine that $T=\coint{0,\infty}$, so that we are looking at a system
evolving over time.   Not infrequently our intuition leads us to a clear
description of the joint distributions $\nu_J$ of finite subfamilies
$\family{t}{J}{X_t}$ without providing any suggestion of a measure space
on which the whole family $\family{t}{T}{X_t}$ might be defined.   (As I
tried to explain in the introduction to Chapter 27, probability spaces
themselves
are often very shadowy things in true probability theory.)   Each
$\nu_J$ can be thought of as a Radon measure on $\BbbR^J$, and for
$I\subseteq J\in[T]^{<\omega}$ we have a natural map
$f_{IJ}:\BbbR^J\to\BbbR^I$,
setting $f_{IJ}(y)=y\restr I$ for $y\in\BbbR^J$.   If our distributions
$\nu_J$ mean anything at all, every $f_{IJ}$ will surely be \imp;  this
is simply saying that $\nu_I$ is the joint distribution of a subfamily
of $\family{t}{J}{X_t}$.   If we can find a Hausdorff space $\Omega$ and
a continuous function $g:\Omega\to\BbbR^T$ such that, for every finite
$J\subseteq T$ and $\epsilon>0$, there is a compact set
$K\subseteq\Omega$
such that $\nu_Jg_J[K]\ge 1-\epsilon$ (where $g_J(x)=g(x)\restr J$),
then Prokhorov's theorem will give us a measure $\mu$ on $\Omega$ which
will then provide us with a suitable realization of $\family{t}{T}{X_t}$
as a family of random variables on a genuine probability space, writing
$X_t(\omega)=g(\omega)(t)$.   That they become continuous functions on a
Radon measure space is a valuable shield against irrelevant
complications.

Clearly, if this can be done at all it can be done with
$\Omega=\BbbR^T$;  but some of the central results of probability theory
are specifically concerned with the possibility of using other sets
$\Omega$ (e.g., $\Omega$ a set of \callal\ functions, as in 455H, or
continuous functions, as in 477B).

\spheader 418Nf In (e) above, we do always have the option of
regarding each $\nu_J$ as a measure on the compact space
$[-\infty,\infty]^J$.   In this case, by 418O or otherwise, we can be
sure of finding a measure
on $[-\infty,\infty]^T$ to support functions $X_t$, at the cost of
either allowing the values $\pm\infty$ or (as I should myself ordinarily
do) accepting that each $X_t$ would be undefined on a negligible set.
The advantage of this is just that it gives us confidence in applying
the Kolmogorov-Lebesgue theory to the whole family $\family{t}{T}{X_t}$
at once, rather than to finite or countable subfamilies.   For an
example of what can happen if we try to do similar things with
non-compact measures, see 419K.   For an example of the problems which
can arise with uncountable families, see 418Xu.
}%end of comment

\leader{418O}{}\cmmnt{ I
mention two cases in which we can be sure that the projective limit
$(X,\familyiI{g_i})$ required in Prokhorov's theorem will exist.

\medskip

\noindent}{\bf Proposition} Suppose that $(I,\le)$,
$\familyiI{(X_i,\frak T_i,\mu_i,\Sigma_i)}$
and $\langle f_{ij}\rangle_{i\le j\in I}$ are such that

\inset{$(I,\le)$ is a non-empty upwards-directed partially ordered set,}

\inset{every $(X_i,\frak T_i,\mu_i,\Sigma_i)$ is a compact Radon measure
space,}

\inset{$f_{ij}:X_j\to X_i$ is a continuous \imp\ function whenever
$i\le j$ in $I$,}

\inset{$f_{ij}f_{jk}=f_{ik}$ whenever $i\le j\le k$ in $I$.}

\noindent Then there are a compact Hausdorff space $X$ and a
family $\familyiI{g_i}$ such that
$I$, $\familyiI{X_i}$, $\langle f_{ij}\rangle_{i\le j\in I}$,
$X$ and $\familyiI{g_i}$ satisfy all the hypotheses of 418M.

\proof{ For each $i$, let $F_i$ be the support of $\mu_i$;  because $X_i$
is compact, so is $F_i$.   If $i\le j$,
then $F_i=\overline{f_{ij}[F_j]}=f_{ij}[F_j]$, by 411Ne.   Set

\Centerline{$X=\{x:x\in\prod_{i\in I}F_i,\,f_{ij}x(j)=x(i)$ whenever
$i\le j\in I\}$,}

\Centerline{$g_i(x)=x(i)$ for $x\in X$, $i\in I$.}

\noindent Of course $g_i=f_{ij}g_j$ whenever $i\le j$.   Also
$g_i[X]\supseteq F_i$
for every $i\in I$.   \Prf\ (i) If $X$ is empty, then there is some
$j\in I$ such that $F_j=\emptyset$.   In this case,
taking any $k$ greater than or equal to both $i$ and $j$,

\Centerline{$\mu_iX_i=\mu_kX_k=\mu_jX_j=0$}

\noindent and $F_i=\emptyset$, so we can stop.   (ii) Otherwise,
take any $y\in F_i$.   For each finite set $J\subseteq I$,

\Centerline{$H_J=\{x:x\in X,\,x(i)=y,\,f_{jk}x(k)=x(j)$
whenever $j\le k\in J\}$}

\noindent is a closed set.   $H_J$ is always non-empty, because if $k$
is an upper bound of $J\cup\{i\}$ there is
a $z\in F_k$ such that $f_{ik}(z)=y$, in which case $x\in H_J$ whenever
$x\in X$ and $x(j)=f_{jk}(z)$ for every $j\in J\cup\{i\}$.   Now
$\{H_J:J\in[I]^{<\omega}\}$ is a downwards-directed family of non-empty
closed sets in the compact space $\prod_{j\in I}F_j$, so has non-empty
intersection, and if $x$ is any point of the intersection then $x\in X$
and $g_i(x)=y$.\ \Qed

Accordingly $\mu_ig_i[X]=\mu_iX_i$ for every $i$;  as $X$ itself is
compact, the final condition of 418M is satisfied.
}%end of proof of 418O

\vleader{72pt}{418P}{Proposition}
Let $(I,\le)$, $\familyiI{(X_i,\frak T_i,\Sigma_i,\mu_i)}$ and
$\langle f_{ij}\rangle_{i\le j\in I}$ be such that

\inset{$(I,\le)$ is a
countable non-empty upwards-directed partially ordered set,}

\inset{every $(X_i,\frak
T_i,\Sigma_i,\mu_i)$ is a Radon probability space,}

\inset{$f_{ij}:X_j\to X_i$ is an \imp\ almost continuous function
whenever $i\le j$ in $I$,}

\inset{$f_{ij}f_{jk}=f_{ik}\,\,\mu_k$-a.e.\ whenever $i\le j\le k$ in
$I$.}

\noindent Then there are a Radon probability space $(X,\frak
T,\Sigma,\mu)$ and continuous \imp\ functions $g_i:X\to X_i$ such that
$g_i=f_{ij}g_j$ whenever $i\le j$ in $I$.

\proof{{\bf (a)} We can use nearly the same formula as in 418O:

\Centerline{$X=\{x:x\in\prod_{i\in I}X_i,\,f_{ij}x(j)=x(i)$ whenever
$i\le j\in I\}$,}

\Centerline{$g_i(x)=x(i)$ for $x\in X$, $i\in I$.}

\noindent As before, the consistency relation $g_i=f_{ij}g_j$ is a
trivial consequence of the definition of $X$.   For the rest, we have to
check that the final condition of 418M is satisfied.   Fix
$\epsilon\in\ooint{0,1}$.   Start by
taking a family $\langle\epsilon_{ij}\rangle_{i\le j\in I}$ of strictly
positive numbers such that $\sum_{i\le j\in
I}\epsilon_{ij}\le\bover12\epsilon$.   (This is where we need to know
that $I$ is countable.)    Set $\epsilon_j=\sum_{i\le j}\epsilon_{ij}$
for each $j$, so that $\sum_{j\in I}\epsilon_j\le\bover12\epsilon$.

For $i\le j\le k$ in $I$, set

\Centerline{$E_{ijk}=\{x:x\in X_k,\,f_{ik}(x)=f_{ij}f_{jk}(x)\}$,}

\noindent so that $E_{ijk}$ is $\mu_k$-conegligible;  set
$E_k=\bigcap_{i\le j\le k}E_{ijk}$, so that $E_k$ is
$\mu_k$-conegligible.   For $i\le j\in I$, choose compact sets
$K_{ij}\subseteq E_j$ such that
$\mu_jK_{ij}\ge 1-\epsilon_{ij}$ and $f_{ij}\restr K_{ij}$ is
continuous.   Now we seem to need a three-stage construction, as
follows:

\inset{for $j\in I$, set $K_j=\bigcap_{i\le j}K_{ij}$;}

\inset{for $j\in I$, set
$K^*_j=K_j\cap\bigcap_{i\le j}f_{ij}^{-1}[K_i]$;}

\inset{finally, set $K=X\cap\prod_{j\in I}K_j^*$.}

\noindent Let us trace the properties of these sets stage by stage.

\medskip

{\bf (b)} For each $j\in I$, $K_j\subseteq K_{jj}\subseteq E_j$ is
compact and

\Centerline{$\mu_j(X_j\setminus K_j)
\le\sum_{i\le j}\mu_j(X_j\setminus K_{ij})
\le\sum_{i\le j}\epsilon_{ij}
=\epsilon_j$,}

\noindent so that $\mu_jK_j\ge 1-\epsilon_j$.   Note that $f_{ik}$
agrees with $f_{ij}f_{jk}$ on $K_k$ whenever $i\le j\le k$, and that
$f_{ij}\restr K_j$ is continuous whenever $i\le j$.

\medskip

{\bf (c)} Every $K_j^*$ is compact, and if $i\le j\le k$ then $f_{ik}$
agrees with $f_{ij}f_{jk}$ on $K^*_k$, while $f_{ij}\restr K_j^*$ is
always continuous.   Also

$$\eqalign{\mu_j(X_j\setminus K_j^*)
&\le\mu_j(X_j\setminus K_j)
  +\sum_{i\le j}\mu_j(X_j\setminus f_{ij}^{-1}[K_i])\cr
&\le\epsilon_j+\sum_{i\le j}\epsilon_i
\le\epsilon,\cr}$$

\noindent so $\mu_jK_j^*\ge 1-\epsilon$, for every $j\in I$.

The point of moving from $K_j$ to $K^*_j$ is that
$f_{jk}[K_k^*]\subseteq K_j^*$ whenever $j\le k$ in $I$.   \Prf\
$K_k^*\subseteq f_{jk}^{-1}[K_j]$, so $f_{jk}[K_k^*]\subseteq K_j$.   If
$i\le j$, then

\Centerline{$K_k^*
=K_k^*\cap f_{ik}^{-1}[K_i]
=K_k^*\cap f_{jk}^{-1}[f_{ij}^{-1}[K_i]]$}

\noindent because $f_{ij}f_{jk}$ agrees with $f_{ik}$ on $K_k^*$.   So
$f_{jk}[K_k^*]\subseteq f_{ij}^{-1}[K_i]$.   As $i$ is arbitrary,
$f_{jk}[K_k^*]\subseteq K_j^*$.\ \Qed

Again because $f_{ik}$ agrees with $f_{ij}f_{jk}$ on $K_k^*$, we have
$f_{ik}[K_k^*]=f_{ij}[f_{jk}[K_k^*]]\subseteq f_{ij}[K_j^*]$ whenever
$i\le j\le k$.   And because $f_{ij}\restr K_j^*$ is always continuous,
all the sets $f_{ij}[K_j^*]$ are compact.

\medskip

{\bf (d)(i)}  $K$ is compact. \Prf\

\Centerline{$K=\{x:x\in\prod_{i\in I}K_i^*,\,f_{ij}x(j)=x(i)$ whenever
$i\le j\in I\}$}

\noindent is closed in $\prod_{i\in I}K_i^*$ because $f_{ij}\restr
K_j^*$ is always continuous (and every $X_i$ is Hausdorff).   Since
$\prod_{i\in I}K_i^*$ is compact, so is $K$.\ \Qed

\medskip

\quad{\bf (ii)} $\mu_ig_i[K]\ge 1-\epsilon$ for every $i\in I$.   \Prf\
By (c), $f_{ik}[K_k^*]\subseteq f_{ij}[K_j^*]$ whenever $i\le j\le k$.
So $\{f_{ij}[K_j^*]:j\ge i\}$ is a downwards-directed family of compact
sets;  write $L$ for their intersection.   Since

\Centerline{$\mu_if_{ij}[K_j^*]
=\mu_jf_{ij}^{-1}[f_{ij}[K_j^*]]\ge\mu_jK_j^*\ge 1-\epsilon$}

\noindent for every $j\ge i$, $\mu_iL\ge 1-\epsilon$ (414C).   If
$z\in L$, then for every $k\ge i$ the set

\Centerline{$F_k
=\{x:x\in\prod_{j\in I}K_j^*,\,x(k)=z$, $f_{jk}x(k)=x(j)$
  whenever $j\le k\}$}

\noindent is a closed set in $\prod_{j\in I}K_j^*$, while
$F_k\subseteq F_j$ when $j\le k$.   Also $F_k$ is non-empty, because there
is a $t\in K_k^*$ such that $f_{ik}(t)=z$, and now if we take any
$x\in\prod_{j\in I}K_j^*$ such that $x(j)=f_{jk}(t)$ for every $j\le k$,
we shall have $x\in F_k$.   So $\{F_k:k\ge i\}$ is a
downwards-directed family of non-empty closed sets in a compact space,
and has non-empty intersection.   But if $x\in\bigcap_{k\ge i}F_k$, then
$x\in K$ and $x(k)=z$, so $z\in g_i[K]$.   Thus $g_i[K]\supseteq L$ and
$\mu_ig_i[K]\ge 1-\epsilon$.\ \Qed

\medskip

{\bf (e)} As $\epsilon$ is arbitrary, the final condition of 418M is
satisfied.   But now 418M gives the result.
}%end of proof of 418P

\leader{418Q}{Corollary} Let
$\sequencen{(X_n,\frak T_n,\Sigma_n,\mu_n)}$
be a sequence of Radon probability spaces, and suppose we are given an
\imp\ almost continuous function $f_n:X_{n+1}\to X_n$ for each $n$.
Set

\Centerline{$X=\{x:x\in\prod_{n\in\Bbb N}X_n,\,f_n(x(n+1))=x(n)$ for
every $n\in\Bbb N\}$.}

\noindent Then there is a unique Radon probability measure $\mu$ on $X$
such that all the coordinate maps $x\mapsto x(n):X\to X_n$ are \imp.

\proof{ For $i\le j\in\Bbb N$, define $f_{ij}:X_j\to X_i$ by
writing

$$\eqalign{f_{ii}(x)&=x\text{ for every }x\in X_i,\cr
f_{i,j+1}&=f_{ij}f_j\text{ for every }j\ge i.\cr}$$

\noindent It is easy to check that $f_{ij}f_{jk}=f_{ik}$ whenever $i\le
j\le k$, and that every $f_{ij}$ is \imp\ and almost continuous (using
418Dc).   So we are exactly in the situation of 418P, and we know that
there is a Radon probability measure on $X$ for which every $g_i$ is
\imp;  moreover, the coordinate functionals $g_i$ separate the points of
$X$, so $\mu$ is unique.
}%end of proof of 418Q

\leader{418R}{}\cmmnt{ I turn now to a special kind of measurable
function, corresponding to a new view of product spaces.

\medskip

\noindent}{\bf Theorem} Let $X$ be a set, $\Sigma$ a $\sigma$-algebra of
subsets of $X$, and $(Y,\Tau,\nu)$ a $\sigma$-finite measure space.
Give $L^0(\nu)$ the topology of convergence in
measure\cmmnt{ (\S245)}.   Write $\eusm L^0(\Sigma\tensorhat\Tau)$ for
the space of $\Sigma\tensorhat\Tau$-measurable real-valued functions
on $X\times Y$\cmmnt{, where $\Sigma\tensorhat\Tau$ is  the
$\sigma$-algebra of subsets of $X\times Y$ generated by
$\{E\times F:E\in\Sigma,\,F\in\Tau\}$}. Then for a function
$f:X\to L^0(\nu)$ the following are equiveridical:

(i) $f[X]$ is separable and $f$ is measurable;

(ii) there is an $h\in\eusm L^0(\Sigma\tensorhat\Tau)$ such that
$f(x)=h_x^{\ssbullet}$ for every $x\in X$, where $h_x(y)=h(x,y)$ for
$x\in X$, $y\in Y$.

\proof{ Let $\sequencen{Y_n}$ be a
non-decreasing sequence of subsets of $Y$ of finite measure covering
$Y$.

\medskip

{\bf (a)(i)$\Rightarrow$(ii)} For each $n\in\Bbb N$, let $\rho_n$ be the
continuous pseudometric on $L^0(\nu)$ defined by saying that
$\rho_n(g_1^{\ssbullet},g_2^{\ssbullet})
=\int_{Y_n}\min(1,|g_1-g_2|)d\nu$
for $g_1$, $g_2\in\eusm L^0(\Tau)$, writing $\eusm L^0(\Tau)$ for the
space of $\Tau$-measurable real-valued functions on $Y$ (245A).   Then
$\{\rho_n:n\in\Bbb N\}$ defines the topology of $L^0(\nu)$ (see the
proof of 245Eb).
Because $f[X]$ is separable, there is a sequence $\sequence{k}{v_k}$ in
$L^0(\nu)$ such that $f[X]\subseteq\overline{\{v_k:k\in\Bbb N\}}$.   For
each $k$, choose $g_k\in\eusm L^0(\Tau)$ such that
$g_k^{\ssbullet}=v_k$.   For $n$, $k\in\Bbb N$ set

\Centerline{$E_{nk}=\{x:x\in X,\,\rho_n(f(x),v_k)<2^{-n}\}$,}

\Centerline{$H_{nk}=E_{nk}\setminus\bigcup_{i<k}E_{ni}$.}

\noindent Then every $E_{nk}$ belongs to $\Sigma$ (because $f$ is
measurable) and $\bigcup_{k\in\Bbb N}E_{nk}=X$ (because
$\{v_k:k\in\Bbb N\}$ is dense);
so $\sequence{k}{H_{nk}}$ is a partition of $X$ into measurable sets.
Set $h^{(n)}(x,y)=g_k(y)$ whenever $k\in\Bbb N$, $x\in H_{nk}$ and
$y\in Y$;  then $h^{(n)}\in\eusm L^0(\Sigma\tensorhat\Tau)$.

Fix $x\in X$ for the moment.   Then for each $n\in\Bbb N$ there is a
unique $k_n$ such that $x\in H_{nk_n}$, and
$\rho_n(f(x),v_{k_n})\le 2^{-n}$.   So if $n\le m$,

$$\eqalign{\int_{Y_n}\min(1,|h^{(m+1)}_x-h^{(m)}_x|)
&=\int_{Y_n}\min(1,|g_{k_{m+1}}-g_{k_m}|)
=\rho_n(g_{k_{m+1}}^{\ssbullet},g_{k_m}^{\ssbullet})\cr
&\le\rho_n(g_{k_{m+1}}^{\ssbullet},f(x))
 +\rho_n(f(x),g_{k_m}^{\ssbullet})\cr
&\le\rho_{m+1}(g_{k_{m+1}}^{\ssbullet},f(x))
 +\rho_m(f(x),g_{k_m}^{\ssbullet})
\le 3\cdot 2^{-m-1}.\cr}$$

\noindent But this means that
$\sum_{m=0}^{\infty}\int_{Y_n}\min(1,|h^{(m+1)}_x-h^{(m)}_x|)$ is
finite, so that $\sequence{m}{h^{(m)}_x}$ must be convergent almost
everywhere in $Y_n$.   As this is true for every $n$,
$\sequence{m}{h^{(m)}_x}$ is convergent a.e.\ on $Y$.   Moreover,

\Centerline{$\lim_{m\to\infty}(h^{(m)}_x)^{\ssbullet}
=\lim_{m\to\infty}g_{k_m}^{\ssbullet}
=f(x)$}

\noindent in $L^0(\nu)$.

Since this is true for every $x$,

\Centerline{$W=\{(x,y):\sequence{m}{h^{(m)}(x,y)}$ converges in
$\Bbb R\}$}

\noindent has conegligible vertical sections, while of course
$W\in\Sigma\tensorhat\Tau$ because every $h^{(m)}$ is
$\Sigma\tensorhat\Tau$-measurable (418C).   If we set
$h(x,y)=\lim_{m\to\infty}h^{(m)}(x,y)$ for $(x,y)\in W$, $0$ for other
$(x,y)\in X\times Y$, then $h\in\eusm L^0(\Sigma\tensorhat\Tau)$, while
(by 245Ca)

\Centerline{$h_x^{\ssbullet}
=\lim_{m\to\infty}(h^{(m)}_x)^{\ssbullet}=f(x)$}

\noindent in $L^0(\nu)$ for every $x\in X$.   So we have a suitable $h$.


\medskip

{\bf (b)(ii)$\Rightarrow$(i)} Let $\Phi$ be the set of those
$h\in\eusm L^0(\Sigma\tensorhat\Tau)$ such that (i) is satisfied;  that is,
$x\mapsto h_x^{\ssbullet}$ is measurable, and
$\{h_x^{\ssbullet}:x\in X\}$ is separable.

\medskip

\quad\grheada\ $\Phi$ is closed under addition.   \Prf\ If $h$,
$\tilde h$ belong to $\Phi$, set $A=\{h_x^{\ssbullet}:x\in X\}$,
$\tilde A=\{\tilde h_x^{\ssbullet}:x\in X\}$.   Then both $A$ and
$\tilde A$ are
separable metrizable spaces, so $A\times\tilde A$ is separable and
metrizable and
$x\mapsto(h_x^{\ssbullet},\tilde h_x^{\ssbullet}):X\to A\times\tilde A$
is measurable (418Bb).   But addition on $L^0(\nu)$ is
continuous (245Da), so

\Centerline{$x\mapsto h_x^{\ssbullet}+\tilde h_x^{\ssbullet}
=(h+\tilde h)_x^{\ssbullet}$}

\noindent is measurable (418Ac), and

\Centerline{$\{(h+\tilde h)_x^{\ssbullet}:x\in X\}
\subseteq\{u+\tilde u:u\in A,\,\tilde u\in\tilde A\}$}

\noindent is separable (4A2B(e-iii), 4A2P(a-iv)).
Thus $h+\tilde h\in\Phi$.\ \Qed

\medskip

\quad\grheadb\ $\Phi$ is closed under scalar multiplication, just
because $u\mapsto\alpha u:L^0(\nu)\to L^0(\nu)$ is always continuous.

\medskip

\quad\grheadc\ If $\sequencen{h^{(n)}}$ is a sequence in $\Phi$ and
$h(x,y)=\lim_{n\to\infty}h^{(n)}(x,y)$ for all $x\in X$, $y\in Y$, then
$h\in\Phi$.   \Prf\ Setting $A_n=\{(h^{(n)}_x)^{\ssbullet}:x\in X\}$ for
each $n$, then $A=\{h_x^{\ssbullet}:x\in X\}$ is included in
$\overline{\bigcup_{n\in\Bbb N}A_n}$, which is separable (4A2B(e-i)
again), so $A$ is separable (4A2P(a-iv) again);  moreover,
$h_x^{\ssbullet}=\lim_{n\to\infty}(h^{(n)}_x)^{\ssbullet}$ for every
$x\in X$, so $x\mapsto h_x^{\ssbullet}$ is measurable, by
418Ba.\ \Qed

\medskip

\quad\grheadd\ What this means is that if we set
$\Cal W=\{W:W\in\Sigma\tensorhat\Tau,\,\chi W\in\Phi\}$, then
$W\setminus W'\in\Cal W$ whenever $W$, $W'\in\Cal W$ and
$W'\subseteq W$, and
$\bigcup_{n\in\Bbb N}W_n\in\Cal W$ whenever $\sequencen{W_n}$ is a
non-decreasing sequence in $\Cal W$.   Also, it is easy to see that
$E\times F\in\Cal W$ whenever $E\in\Sigma$ and $F\in\Tau$.   By the
Monotone Class Theorem (136B), $\Cal W$ includes the $\sigma$-algebra
generated by $\{E\times F:E\in\Sigma,\,F\in\Tau\}$, that is, is equal to
$\Sigma\tensorhat\Tau$.   It follows at once, from ($\alpha$) and
($\beta$), that $\sum_{i=0}^n\alpha_i\chi W_i\in\Phi$ whenever
$W_0,\ldots,W_n\in\Sigma\tensorhat\Tau$ and
$\alpha_0,\ldots,\alpha_n\in\Bbb R$, and hence (using ($\gamma$)) that
$\eusm L^0(\Sigma\tensorhat\Tau)\subseteq\Phi$, which is what we had to
prove.
}%end of proof of 418R

\leader{418S}{Corollary} Let $(X,\Sigma,\mu)$ and $(Y,\Tau,\nu)$ be
$\sigma$-finite measure spaces with c.l.d.\ product
$(X\times Y,\Lambda,\lambda)$.   Give $L^0(\nu)$ the topology of
convergence in measure.   \cmmnt{Write $\eusm L^0(\lambda)$ for
the space
of $\Lambda$-measurable real-valued functions defined $\lambda$-a.e.\ on
$X\times Y$, as in 241A.}

(a) If $h\in\eusm L^0(\lambda)$, set $h_x(y)=h(x,y)$ whenever this is
defined.   Then

\Centerline{$\{x:f(x)=h_x^{\ssbullet}$ is defined in $L^0(\nu)\}$}

\noindent is $\mu$-conegligible, and includes a conegligible set $X_0$
such that $f:X_0\to L^0(\nu)$ is measurable and $f[X_0]$ is separable.

(b) If $f:X\to L^0(\nu)$ is measurable and there is a conegligible set
$X_0\subseteq X$ such that $f[X_0]$ is separable, then there is an
$h\in\eusm L^0(\lambda)$ such that $f(x)=h_x^{\ssbullet}$ for almost
every $x\in X$.

\proof{{\bf (a)} The point is that $\lambda$ is just the completion of
its restriction to $\Sigma\tensorhat\Tau$ (251K).   So there is a
conegligible set $W\in\Sigma\tensorhat\Tau$ such that $h\restrp W$ is
$\Sigma\tensorhat\Tau$-measurable (212Fa).   Setting
$\tilde h(x,y)=h(x,y)$ for $(x,y)\in W$, $0$ otherwise, and setting
$\tilde f(x)=\tilde h_x^{\ssbullet}$ for every $x\in X$, we see from
418R that
$\tilde f$ is measurable and that $\tilde f[X]$ is separable.   But 252D
tells us that

\Centerline{$X_0=\{x:((X\times Y)\setminus W)[\{x\}]$ is negligible$\}$}

\noindent is conegligible;  and if $x\in X_0$ then
$h_x=\tilde h_x\,\,\nu$-a.e., so that $f(x)$ is defined and equal to
$\tilde f(x)$.   This proves the result.

\medskip

{\bf (b)}  $f\restr X_0$ satisfies 418R(i).   So, setting $f_1(x)=f(x)$
for $x\in X_0$, $0$ otherwise, there is some $h\in\eusm
L^0(\Sigma\tensorhat\Tau)$ such that $f_1(x)=h_x^{\ssbullet}$ for every
$x$, so that $f(x)=h_x^{\ssbullet}$ for almost every $x$, and (ii) is
true.
}%end of proof of 418S

\leader{418T}{Corollary}\cmmnt{ ({\smc Mauldin \& Stone 81})} Let
$(Y,\Tau,\nu)$ be a
$\sigma$-finite measure space, and $(\frak B,\bar\nu)$ its measure
algebra,  with its measure-algebra topology\cmmnt{ (\S323)}.

(a) Let $X$ be a set, $\Sigma$ a
$\sigma$-algebra of subsets of $X$, and $f:X\to\frak B$ a function.
Then the following are equiveridical:

\quad(i) $f[X]$ is separable and $f$ is measurable;

\quad(ii) there is a $W\in\Sigma\tensorhat\Tau$ such that
$f(x)=W[\{x\}]^{\ssbullet}$ for every $x\in X$.

(b) Let $(X,\Sigma,\mu)$ be a $\sigma$-finite measure space and
$\Lambda$ the domain of the c.l.d.\ product measure $\lambda$ on
$X\times Y$.

\quad(i) Suppose that $\nu$ is complete.   If $W\in\Lambda$, then

\Centerline{$\{x:f(x)=W[\{x\}]^{\ssbullet}$ is defined in $\frak B\}$}

\noindent is $\mu$-conegligible, and includes a conegligible set $X_0$
such that $f:X_0\to \frak B$ is measurable and $f[X_0]$ is separable.

\quad(ii) If $f:X\to\frak B$ is measurable and there is a conegligible
set $X_0\subseteq X$ such that $f[X_0]$ is separable, then there is a
$W\in\Sigma\tensorhat\Tau$ such that $f(x)=W[\{x\}]^{\ssbullet}$ for
almost every $x\in X$.

\proof{ Everything follows directly from 418R and 418S if we observe
that $\frak B$ is homeomorphically embedded in $L^0(\nu)$ by the
function $F^{\ssbullet}\mapsto(\chi F)^{\ssbullet}$ for $F\in\Tau$
(323Xa, 367R).   We do need to check, for (i)$\Rightarrow$(ii) of part
(a), that if $h\in\eusm L^0(\Sigma\tensorhat\Tau)$ and $h_x^{\ssbullet}$
is always of the form $(\chi F)^{\ssbullet}$, then there is some
$W\in\Sigma\tensorhat\Tau$ such that $h_x^{\ssbullet}=(\chi
W[\{x\}])^{\ssbullet}$ for every $x$;  but of course this is true if we
just
take $W=\{(x,y):h(x,y)=1\}$.   Now (b-ii) follows from (a) just as 418Sb
followed from 418R.
}%end of proof of 418T

\leader{*418U}{Independent families of measurable functions}
\discrversionA{\footnote{New 2007}}{} In \S455 we
shall have occasion to look at independent families of random variables
taking values in spaces other than $\Bbb R$.   We can use the same
principle as in \S272:  a family $\familyiI{X_i}$ of random variables is
independent if $\familyiI{\Sigma_i}$ is independent, where $\Sigma_i$ is
the $\sigma$-subalgebra defined by $X_i$ for each $i$\cmmnt{ (272D)}.
\cmmnt{Of
course this depends on agreement about the definition of $\Sigma_i$.
The natural thing to do, in the context of this section, is to follow 272C,
as follows.   }Let $(X,\Sigma,\mu)$ be a probability space, $Y$ a
topological space, and $f$ a $Y$-valued
function defined on a conegligible subset $\dom f$ of $X$, which is
$\mu$-virtually measurable, that is, such that $f$ is measurable
with respect to the subspace $\sigma$-algebra on $\dom f$ induced by
$\hat\Sigma=\dom\hat\mu$, where $\hat\mu$ is the completion of $\mu$.
\cmmnt{(Note that if $Y$ is not second-countable this may not imply that
$f\restr D$ is $\Sigma$-measurable for a conegligible subset $D$ of $X$.)}
The `$\sigma$-algebra defined by $f$' will be

\Centerline{$\{f^{-1}[F]:F\in\Cal B(Y)\}
\cup\{(\Omega\setminus\dom f)\cup f^{-1}[F]:F\in\Cal B(Y)\}
\subseteq\hat\Sigma$,}

\noindent where $\Cal B(Y)$ is the Borel $\sigma$-algebra of $Y$\cmmnt{;
that is, the $\sigma$-algebra of subsets of $X$ generated by
$\{f^{-1}[G]:G\subseteq Y$ is open$\}$}.

Now, given a family $\familyiI{(f_i,Y_i)}$ where each $Y_i$ is a
topological space and each $f_i$ is a $\hat\Sigma$-measurable $Y_i$-valued
function defined on a conegligible subset of $X$, I will say that
$\familyiI{f_i}$ is {\bf independent} if $\familyiI{\Sigma_i}$ is
independent (with respect to $\hat\mu$),
where $\Sigma_i$ is the $\sigma$-algebra defined by $f_i$ for
each $i$.

\cmmnt{Corresponding to 272D, we can use the Monotone Class Theorem to
show that }$\familyiI{f_i}$ is independent iff

\Centerline{$\hat\mu(\bigcap_{j\le n}f_{i_j}^{-1}[G_j])
=\prod_{j\le n}\hat\mu f_{i_j}^{-1}[G_j]$}

\noindent whenever $i_0,\ldots,i_n\in I$ are
distinct and $G_j\subseteq Y_{i_j}$ is open for every $j\le n$.

\exercises{
\leader{418X}{Basic exercises $\pmb{>}$(a)}
%\spheader 418Xa
Let $(X,\Sigma,\mu)$ be a measure space, $Y$ a set and $h:X\to Y$ a
function;  give $Y$ the image measure $\mu h^{-1}$.   Show that for any
function $g$ from $Y$ to a topological space $Z$, $g$ is measurable iff
$gh:X\to Z$ is measurable.
%418A

\sqheader 418Xb Let $X$ be a set, $\Sigma$ a $\sigma$-algebra of subsets
of $X$, $\sequencen{Y_n}$
a sequence of topological spaces with product $Y$, and $f:X\to Y$ a
function.   Show that $f$ is measurable iff
$\psi_nf:X\to\prod_{i\le n}Y_i$ is measurable for every $n\in\Bbb N$,
where $\psi_n(y)=(y(0),\ldots,y(n))$
for $y\in Y$, $n\in\Bbb N$.
%418B

\spheader 418Xc Let $(X,\Sigma,\mu)$ be a semi-finite measure space,
$(Y,\frak S)$ a metrizable space, and
$\sequencen{f_n}$ a sequence of measurable functions from $X$ to $Y$
such that $\sequencen{f_n(x)}$ is convergent for almost every $x\in X$.
Show that $\mu$ is inner regular with respect to
$\{E:\sequencen{f_n\restr E}$ is uniformly convergent$\}$.
(Cf.\ 412Xt.)
%412-

\sqheader 418Xd  Set $Y=[0,1]^{[0,1]}$, with the product topology.   For
$n\in\Bbb N$ and $x\in [0,1]$ define $f_n(x)\in Y$ by saying that
$f_n(x)(t)=\max(0,1-2^n|x-t|)$ for $t\in[0,1]$.   Check that (i) each
$f_n$ is continuous, therefore measurable;  (ii)
$f(x)=\lim_{n\to\infty}f_n(x)$
is defined in $Y$ for every $x\in[0,1]$;  (iii) for each $t\in[0,1]$,
the coordinate functional $x\mapsto f(x)(t)$ is continuous except at
$t$, and in particular is almost continuous and measurable;  (iv)
$f\restr F$ is not
continuous for any infinite closed set $F\subseteq[0,1]$, and in
particular $f$ is not almost continuous;  (v) every subset of $[0,1]$ is
of the form
$f^{-1}[H]$ for some open set $H\subseteq Y$;  (vi) $f$ is not
measurable;
(vii) the image measure $\mu f^{-1}$, where $\mu$ is Lebesgue measure on
$[0,1]$, is neither a topological measure nor tight.
%418D

\spheader 418Xe Let $(X,\frak T,\Sigma,\mu)$ be a quasi-Radon measure
space, $Y$ a topological space, and $f:X\to Y$ a function.   Suppose
that for every $x\in X$ there is an open set $G$ containing $x$ such
that $f\restr G$ is almost continuous with respect to the subspace
measure on $G$.   Show that $f$ is almost continuous.
%418D

\spheader 418Xf For $i=1$, $2$ let $(X_i,\frak T_i,\Sigma_i,\mu_i)$ and
$(Y_i,\frak S_i,\Tau_i,\nu_i)$ be quasi-Radon measure spaces, and
$f_i:X_i\to Y_i$ an almost continuous \imp\ function.   Show that
$(x_1,x_2)\mapsto (f_1(x_1),f_2(x_2))$ is almost continuous and
\imp\ for the product topologies and quasi-Radon product measures.
%418D
%Is there an easy example to show that `almost continuous' is
%essential?

\spheader 418Xg Let $\familyiI{(X_i,\frak T_i,\Sigma_i,\mu_i)}$ and
$\familyiI{(Y_i,\frak T_i,\Sigma_i,\nu_i)}$ be two families of
topological
spaces with $\tau$-additive Borel probability measures, and let $\mu$,
$\nu$ be the $\tau$-additive product measures on $X=\prod_{i\in I}X_i$,
$Y=\prod_{i\in I}Y_i$.   Suppose that every $\nu_i$ is strictly
positive.
Show that if $f_i:X_i\to Y_i$ is almost continuous and \imp\ for each
$i$, then $x\mapsto\familyiI{f_i(x(i))}:X\to Y$ is \imp, but need not be
almost continuous.
%418D

\spheader 418Xh Let $(X,\frak T,\Sigma,\mu)$ and $(Y,\frak S,\Tau,\nu)$
be quasi-Radon measure spaces, $(Z,\frak U)$ a topological space and
$f:X\times Y\to Z$ a function which is almost continuous with respect to
the quasi-Radon product measure on $X\times Y$.   Suppose that $\nu$ is
$\sigma$-finite.   Show that $y\mapsto f(x,y)$ is almost continuous for
almost every $x\in X$.
%418D

\spheader 418Xi Let $(X,\frak T,\Sigma,\mu)$ be an effectively locally
finite $\tau$-additive topological measure space, $Y$ a topological
space and $f:X\to Y$ an almost continuous function.   (i) Show that the
image measure $\mu f^{-1}$ is $\tau$-additive.   (ii) Show that if $\mu$
is a totally finite quasi-Radon measure and the topology on $Y$ is
regular, then $\mu f^{-1}$ is quasi-Radon.
%418D

\spheader 418Xj Let $(X,\frak T,\Sigma,\mu)$ be a topological measure
space and $U$ a linear topological space.   Show that if $f:X\to U$ and
$g:X\to U$ are almost continuous, then $f+g:X\to U$ is almost
continuous.
%418D

\spheader 418Xk Let $(X,\frak T,\Sigma,\mu)$ and $(Y,\frak S,\Tau,\nu)$
be topological measure spaces, and $(Z,\frak U)$ a topological space;
let $f:X\to Y$ be almost continuous and \imp, and $g:Y\to Z$ almost
continuous.   Show that if {\it either} $\mu$ is a Radon measure
and $\nu$ is locally finite {\it or} $\mu$ is $\tau$-additive and
effectively locally finite and $\nu$ is effectively locally finite, then
$gf:X\to Z$ is almost continuous.   \Hint{show that if $\mu E>0$ there
is a set $F$ such that $\nu F<\infty$ and $\mu(E\cap f^{-1}[F])>0$.}
%418D

\spheader 418Xl Let $(X,\Sigma,\mu)$ be a complete strictly localizable
measure space, $\undphi:\Sigma\to\Sigma$ a lower density such that
$\undphi X=X$, and $\frak T$ the associated density topology on $X$
(414P).   Let
$f:X\to\Bbb R$ be a function.   Show that the following are
equiveridical:  (i) $f$ is
measurable;  (ii) $f$ is almost continuous;  (iii) $f$ is continuous at
almost every point;  (iv) there is a conegligible set $H\subseteq X$
such that $f\restr H$ is continuous.   (Cf.\ 414Xk.)
%418E

\spheader 418Xm Let $(X,\Sigma,\mu)$ be a complete strictly localizable
measure space, $\phi:\Sigma\to\Sigma$ a lifting, and $\frak S$ the
lifting topology on $X$ (414Q).   Let $f:X\to\Bbb R$ be a function.
Show that the
following are equiveridical:  (i) $f$ is measurable;  (ii) $f$ is almost
continuous;  (iii) there is a conegligible set $H\subseteq X$ such that
$f\restr H$ is continuous.   (Cf.\ 414Xr.)
%418E

\spheader 418Xn Let $(X,\frak T,\Sigma,\mu)$ be a quasi-Radon measure
space, $(Y,\frak S)$ a regular topological space and $f:X\to Y$ an
almost continuous function.   Show that there is a quasi-Radon measure
$\nu$ on $Y$ such that $f$ is \imp\ for $\mu$ and $\nu$ iff
$\bigcup\{f^{-1}[H]:H\subseteq Y$ is open, $\mu f^{-1}[H]<\infty\}$ is
conegligible in $X$.
%418H

\spheader 418Xo Let $(X,\frak T,\Sigma,\mu)$ be a Radon measure space,
$(Y,\frak S)$ and $(Z,\frak U)$ Hausdorff spaces, $f:X\to Y$ an almost
continuous function such that $\nu=\mu f^{-1}$ is locally finite, and
$g:Y\to Z$ a function.   Show that $g$ is almost continuous with respect
to $\nu$ iff $gf$ is almost continuous with respect to $\mu$.
%418I

\spheader 418Xp Let $(X,\frak T,\Sigma,\mu)$ and $(Y,\frak S,\Tau,\nu)$
be topological probability spaces, and $f:X\to Y$ a measurable function
such that $\mu f^{-1}[H]\ge\nu H$ for every $H\in\frak S$.
Show that (i) $\int gfd\mu=\int g\,d\nu$ for every $g\in C_b(Y)$ (ii)
$\mu f^{-1}[F]=\nu F$ for every Baire set $F\subseteq Y$ (iii) if $\mu$
is a Radon measure and $f$ is almost continuous, then
$\mu f^{-1}[F]=\nu F$ for every Borel set $F\subseteq Y$, so that if in
addition $\nu$ is complete and inner regular with respect to the Borel
sets then it is a Radon measure.
%418I

\spheader 418Xq Let $(X,\frak T,\Sigma,\mu)$ be a $\sigma$-finite
topological measure space in which the topology $\frak T$ is normal and
$\mu$ is outer regular with respect to the open sets.   Show that if
$f:X\to\Bbb R$ is a measurable function and $\epsilon>0$ there is a
continuous $g:X\to\Bbb R$ such that $\mu\{x:g(x)\ne f(x)\}\le\epsilon$.
%418J

\spheader 418Xr Let $X$ and $Y$ be Hausdorff spaces, $\nu$ a totally
finite Radon measure on $Y$, and $f:X\to Y$ an injective continuous
function.   Show that the following are equiveridical:  (i) there is a
Radon measure $\mu$ on $X$ such that $f$ is \imp;
(ii) $f[X]$ is conegligible and $f^{-1}:f[X]\to X$ is almost continuous.
%418L

\spheader 418Xs Let $(X,\frak T,\Sigma,\mu)$ and $(Y,\frak S,\Tau,\nu)$
be Radon measure spaces and $f:X\to Y$ an almost continuous \imp\
function.   Show that (i) $\mu_*A\le\nu_*f[A]$ for every $A\subseteq X$
(ii) $\nu$ is precisely the image measure $\mu f^{-1}$.
%418L

\spheader 418Xt In 418M, show that all the $f_{ij}$ must be almost
continuous.   Show that if $i\le j\le k$ then $f_{ij}f_{jk}=f_{ik}$
almost everywhere in $X_k$.
%418M

\sqheader 418Xu Let $\Cal I$ be the family of finite subsets of $[0,1]$,
and for each $I\in\Cal I$
let $(X_I,\frak T_I,\Sigma_I,\mu_I)$ be $[0,1]\setminus I$ with its
subspace topology and measure induced by Lebesgue measure.   For
$I\subseteq J\in\Cal I$ and $y\in X_J$ set $f_{IJ}(y)=y$.
Show that these $X_I$,
$f_{IJ}$ satisfy nearly all the hypotheses of 418O, but that there are
no $X$, $g_I$ which satisfy the hypotheses of 418M.
%418N

\spheader 418Xv Let $T$ be any set, and $X$ the set of total
orders on $T$.   (i) Regarding each member of $X$ as a subset of
$T\times T$, show that $X$ is a closed subset of $\Cal P(T\times T)$.
(ii) Show that there is a unique Radon measure $\mu$ on $X$ such that
$\Pr(t_1\le t_2\le\ldots\le t_n)=\Bover1{n!}$ for all distinct
$t_1,\ldots,t_n\in T$.   \Hint{for $I\in[T]^{<\omega}$, let $X_I$ be the
set of total orders on $I$ with the uniform probability measure
giving the same measure to each singleton;  show that the natural map
from $X_I$ to $X_J$ is \imp\ whenever $J\subseteq I$.}
%418N

\spheader 418Xw In 418Sb, suppose that $f_1:X\to L^0(\nu)$ and
$f_2:X\to L^0(\nu)$ correspond to $h_1$, $h_2\in\eusm L^0(\lambda)$.
Show that
$f_1(x)\le f_2(x)\,\,\mu$-a.e.($x$) iff $h_1\le h_2\,\,\lambda$-a.e.
Hence show that (if we assign appropriate algebraic operations to the
space of functions from $X$ to $L^0(\nu)$) we have an $f$-algebra
isomorphism between $L^0(\lambda)$ and the space of equivalence classes
of measurable functions from $X$ to $L^0(\nu)$ with separable ranges.
%418S

\spheader 418Xx Let $\mu$ be Lebesgue measure on $\BbbR^r$, where
$r\ge 1$, $X$ a Hausdorff space and $f:\BbbR^r\to X$ an almost continuous
function.
Show that for almost every $x\in\BbbR^r$ there is a measurable set
$E\subseteq\BbbR^r$ such that $x$ is a density point of $E$ and
$\lim_{y\in E,y\to x}f(y)=f(x)$.
%418D

\spheader 418Xy Let $X$ be a compact Hausdorff space.   Show that there is
an atomless Radon probability measure on $X$ iff $X$ is non-scattered.
%418L

\leader{418Y}{Further exercises (a)}
%\spheader 418Ya
Let $X$ be a set, $\Sigma$ a $\sigma$-algebra of subsets of $X$, $Y$ a
topological space and $f:X\to Y$ a function.   Set
$\Tau=\{F:F\subseteq Y,\,f^{-1}[F]\in\Sigma\}$.   Suppose that $Y$ is
hereditarily Lindel\"of and its topology is generated by some subset of
$\Tau$.   Show that $f$ is measurable.
%418B

\spheader 418Yb Let $(X,\Sigma,\mu)$ be a measure space, $Y$ and $Z$
topological spaces
and $f:X\to Y$, $g:X\to Z$ measurable functions.   Show that if $Z$ has
a countable network consisting of Borel sets (e.g., $Z$ is
second-countable, or $Z$ is regular and has a countable network), then
$x\mapsto (f(x),g(x)):X\to Y\times Z$ is measurable.
%418B
%what if $Z$ analytic?

\spheader 418Yc Let $X$ be a set, $\Sigma$ a $\sigma$-algebra of subsets
of $X$, and $\familyiI{Y_i}$ a countable family of topological spaces
with product $Y$.   Suppose that every $Y_i$ has a countable network,
and that $f:X\to Y$ is a function such that $\pi_if$ is measurable for
every $i\in I$, writing $\pi_i(y)=y(i)$.   Show that $f$ is measurable.
%418B %418Ya
%Y is hered Lind

\spheader 418Yd Find strictly localizable Hausdorff topological measure
spaces $(X,\frak T,\Sigma,\mu)$, $(Y,\frak S,\Tau,\nu)$ and
$(Z,\frak U,\Lambda,\lambda)$ and almost continuous \imp\ functions
$f:X\to Y$, $g:Y\to Z$ such that $gf$ is not almost continuous.
%418Xk 418D mt41bits

\spheader 418Ye Let $(X,\Sigma,\mu)$ be a $\sigma$-finite measure space
and $\frak T$ a topology on $X$ such that $\mu$ is effectively locally
finite and $\tau$-additive.   Let $Y$ be a topological space and
$f:X\to Y$ an almost continuous function.   Show that there is a
conegligible subset $X_0$ of $X$ such that $f[X_0]$ is ccc.
%418F

\spheader 418Yf Show that if $\mu$ is Lebesgue measure on $\Bbb R$,
$\frak T$ is the usual topology on $\Bbb R$ and
$\frak S$ is the right-facing Sorgenfrey topology, then the
identity map from $(\Bbb R,\frak T,\mu)$ to $(\Bbb R,\frak S)$ is
measurable, but not almost continuous, and the image
measure is not a Radon measure.
%418I

\spheader 418Yg Let $(X,\Sigma,\mu)$ be a semi-finite measure space and
$\frak T$ a topology on $X$ such that $\mu$ is inner regular with
respect to the closed sets.   Suppose that $Y$ is a topological space
with a countable network consisting of Borel sets, and that $f:X\to Y$
is measurable.   Show that $f$ is almost continuous.
%418J

\spheader 418Yh Find a topological probability space
$(X,\frak T,\Sigma,\mu)$ in which $\mu$ is inner regular with respect to
the closed sets, a topological space $Y$ with a countable network and a
measurable function $f:X\to Y$ which is not almost continuous.
%418J

\spheader 418Yi Let $(X,\Sigma,\mu)$ be a semi-finite measure space and
$\frak T$ a topology on $X$ such that $\mu$ is inner regular with
respect to the closed sets.   Let $A\subseteq L^{\infty}(\mu)$ be a
norm-compact set.   Show that there is a set $B$ of bounded real-valued
measurable functions on $X$ such that (i) $A=\{f^{\ssbullet}:f\in B\}$
(ii) $B$ is norm-compact in $\ell^{\infty}(X)$ (iii) $\mu$ is inner
regular with respect to
$\{E:f\restr E$ is continuous for every $f\in B\}$.
%418J

\spheader 418Yj Let $\mu$ be Lebesgue measure on $[0,1]$.   For
$t\in[0,1]$ set $u_t=\chi[0,t]^{\ssbullet}\in L^0(\mu)$.   Show that
$A=\{u_t:t\in[0,1]\}$ is norm-compact in $L^p(\mu)$ for every
$p\in\coint{1,\infty}$ and also compact for the topology of convergence
in measure on $L^0(\mu)$.   Show that if $B$ is a set of measurable
functions such that $A=\{f^{\ssbullet}:f\in B\}$ then $\mu$ is not inner
regular with respect to
$\{E:f\restr E$ is continuous for every $f\in B\}$.
%418Yi 418J

\spheader 418Yk Suppose that $(I,\le)$,
$\familyiI{(X_i,\frak T_i,\Sigma_i,\mu_i)}$ and
$\langle f_{ij}\rangle_{i\le j\in I}$ are such that
($\alpha$) $(I,\le)$ is a non-empty upwards-directed partially ordered
set ($\beta$) every $(X_i,\frak T_i,\Sigma_i,\mu_i)$ is a completely
regular Hausdorff quasi-Radon probability space
($\gamma$) $f_{ij}:X_j\to X_i$ is a
continuous \imp\ function whenever $i\le j$ in $I$
($\delta$) $f_{ij}f_{jk}=f_{ik}$ whenever $i\le j\le k$ in $I$.
Let $X'_i$ be the
support of $\mu_i$ for each $i$;  show that $f_{ij}[X'_j]$ is a dense
subset of $X'_i$ whenever $i\le j$.   Let $Z_i$ be the Stone-\v{C}ech
compactification of $X'_i$ and let $\tilde f_{ij}:Z_j\to Z_i$ be the
continuous extension of $f_{ij}\restr X'_j$ for $i\le j$;  let
$\tilde\mu_i$ be the Radon probability measure on $Z_i$ corresponding to
$\mu_i\restrp\Cal PX'_i$ (416V).   Show that $Z_i$, $\tilde f_{ij}$
satisfy the conditions of 418O, so that we have a projective limit $Z$,
$\familyiI{g_i}$, $\mu$ as in 418M.
%418M

\spheader 418Yl Suppose that $(I,\le)$,
$\familyiI{(X_i,\Sigma_i,\mu_i)}$, $X$, $\familyiI{f_i}$ and
$\langle f_{ij}\rangle_{i\le j\in I}$ are such that
($\alpha$) $(I,\le)$ is a non-empty upwards-directed partially ordered
set ($\beta$) every $(X_i,\Sigma_i,\mu_i)$ is a probability space
($\gamma$) $f_{ij}:X_j\to X_i$ is \imp\ whenever $i\le j$ in $I$
($\delta$) $f_{ij}f_{jk}=f_{ik}$ whenever $i\le j\le k$ ($\epsilon$)
$f_i:X\to X_i$ is a function for every $i\in I$ ($\zeta$)
$f_i=f_{ij}f_j$ whenever $i\le j$ ($\eta$) whenever $\sequencen{i_n}$,
$\sequencen{x_n}$ are such that $\sequencen{i_n}$ is a non-decreasing
sequence in $I$, $x_n\in X_{i_n}$ for every $n\in\Bbb N$ and
$f_{i_ni_{n+1}}(x_{n+1})=x_n$ for every $n\in\Bbb N$, then there is an
$x\in X$ such that $f_{i_n}(x)=x_n$ for every $n$.   Show that there is
a probability measure on $X$ such that every $f_i$ is \imp.
%418M

\spheader 418Ym Let $\mu$ be Lebesgue measure on $[0,1]$, and
$A\subseteq[0,1]$ a set with inner measure 0 and outer measure 1;  let
$\frak T$ be the usual topology on $[0,1]$.   Let $\Cal I$ be the family
of sets $I\subseteq A$ such that every point of $A$ has a neighbourhood
containing at most one point of $I$.   Show that
$\frak S=\{G\setminus I:G\in\frak T,\,I\in\Cal I\}$ is a topology on
$[0,1]$ with a countable network.   Show that the identity map from
$[0,1]$ to itself, regarded as a map from $([0,1],\frak T,\mu)$ to
$([0,1],\frak S)$, is measurable but not almost continuous.
%418Yg 418J out of order query

\spheader 418Yn Let $X$ be a set, $\Sigma$ a $\sigma$-algebra of subsets of
$X$ and $(Y,\Tau,\nu)$ a $\sigma$-finite measure space with countable
Maharam type.   (i) Let $f:X\to L^1(\nu)$ be a function such that
$x\mapsto\int_Ff(x)d\nu$ is $\Sigma$-measurable for every $F\in\Tau$.
Show that $f$ is $\Sigma$-measurable for the norm topology on $L^1(\nu)$.
(ii) Let $g:X\times Y\to\Bbb R$ be a function such that
$\int g(x,y)\nu(dy)$ is defined for every $x\in X$, and
$x\mapsto\int_Fg(x,y)\nu(dy)$ is $\Sigma$-measurable for every $F\in\Tau$.
Show that there is an $h\in\eusm L^0(\Sigma\otimes\Tau)$ such that, for
every $x\in X$, $g(x,y)=h(x,y)$ for $\nu$-almost every $y$.
%418R

\spheader 418Yo Let $(X,\Sigma,\mu)$ be a semi-finite measure
space and $\frak T$ a topology on $X$ such that $\mu$ is inner regular
with respect to the closed sets.   Suppose that $Y$ and $Z$ are
separable metrizable spaces,and
$f:X\times Y\to Z$ is a function such that $x\mapsto f(x,y)$ is measurable
for every $y\in Y$, and $y\mapsto f(x,y)$ is continuous for every $x\in X$.
Show that $\mu$ is inner regular
with respect to $\{F:F\subseteq X$, $f\restr F\times Y$ is continuous$\}$.
\Hint{Let $\rho$, $\sigma$ be metrics defining the topologies of
$Y$ and $Z$.   For $y\in Y$ and $n\in\Bbb N$ set
$g_{yn}(x)=\sup\{s:\sigma(f(x,y'),f(x,y))\le 2^{-n}$ whenever
$\rho(y',y)\le s\}$, $f_y(x)=f(x,y)$.
Show that if $D\subseteq Y$ is dense
and $F\subseteq X$ is such that $g_{yn}\restr F$ and $f_y\restr F$ are
continuous whenever $y\in D$ and $n\in\Bbb N$, then $f\restr F\times X$ is
continuous.}
%418J see Bouziad 07 out of order query

\spheader 418Yp\dvAnew{2009}
Use 418M and 418O to prove 328H.
%418O out of order query

\spheader 418Yq\dvAnew{2009}
Let $X$ be a set, $\Sigma$ a $\sigma$-algebra of
subsets of $X$, $(Y,\Tau,\nu)$ a $\sigma$-finite measure space and
$W\in\Sigma\tensorhat\Tau$.   Then there is a $V\subseteq W$ such that
$V\in\Sigma\tensorhat\Tau$, $W[\{x\}]\setminus V[\{x\}]$ is negligible
for every
$x\in X$, and $\bigcap_{x\in I}V[\{x\}]$ is either empty or non-negligible
for every finite $I\subseteq X$.
%418T n09131

\spheader 418Yr\dvAnew{2011}
Let $X$ be a compact Hausdorff space, $Y$ a Hausdorff space, $\nu$ a
Radon probability measure on $Y$ and $R\subseteq X\times Y$ a closed set 
such that $\nu^*R[X]=1$.   Show that there is a Radon 
probability measure $\mu$ on $X$ such that
$\mu R^{-1}[F]\ge\nu F$ for every closed set $F\subseteq Y$.
%418L out of order query
}%end of exercises

\cmmnt{\Notesheader{418}
The message of this section is that measurable functions are dangerous,
but that almost continuous functions behave themselves.   There are
two fundamental problems with measurable functions:  a function
$x\mapsto (f(x),g(x))$
may not be measurable when the components $f$ and $g$ are measurable
(419Xg), and an image measure under a measurable function can lose
tightness, even when both domain and
codomain are Radon measure spaces and the function is \imp\ (419Xh).
(This is the `image
measure catastrophe' mentioned in 235H and the notes to \S343.)
Consequently, as long as we are dealing with measurable functions, we
often have to impose strong conditions on the range
spaces -- commonly, we have to restrict ourselves to separable
metrizable spaces (418B, 418C), or something similar,
which indeed often means that a measurable function is actually almost
continuous (418J, 433E).   Indeed, for functions taking values in
metrizable spaces, `almost continuity' is very close to `measurable with
essentially separable range' (418G, 418J).   The condition `separable
and metrizable' is a
little stronger than is strictly necessary (418Yb, 418Yc, 418Yg), but
covers the principal applications other than 433E.   If we keep the
`metrizable' we can
very substantially relax the `separable' (438E, 438F), and it is in fact
the case that a measurable function from a Radon measure space to any
metrizable space is almost continuous (451T).   These extensions apply
equally to the results in 418R-418T (438Xi-438Xj, 451Xp).   But both
take us deeper into set theory than seems appropriate at the moment.

For almost continuous functions, the two problems mentioned above do not
arise (418Dd, 418I, 418Xs).   Indeed we rather expect almost continuous
functions to behave as if they were continuous.   But we still have to
be careful.   The limit of a sequence of almost continuous functions
need not be almost continuous (418Xd), unless the codomain is metrizable
(418F);  and if we have a function $f$ from a
topological measure space to an uncountable product of topological
spaces, it can happen that every coordinate of $f$ is an almost
continuous function
while $f$ is not (418Xd again).   But for many purposes, intuitions
gained from the study of measurable functions between Euclidean spaces
can be transferred to general almost continuous functions.

Theorems 418L and 418M are of a quite different kind, but seem to
belong here as well
as anywhere.   Even in the simplest application of 418L (when $Y=[0,1]$
with Lebesgue measure, and $X\subseteq[0,1]^2$ is a closed set meeting
every vertical line) it is not immediately obvious that there will be a
measure with the right projection onto the horizontal axis, though there
are at least two proofs which are easier than the general case treated
in 413N-413O-418L.

As I explain in 418N, the really interesting question concerning 418M is
when, given the projective system
\penalty-100$\familyiI{(X_i,\frak T_i,\Sigma_i,\mu_i)}$,
$\langle f_{ij}\rangle_{i\le j\in I}$, we can expect to find $X$ and
$\familyiI{g_i}$ satisfying the
rest of the hypotheses, and once past the elementary results 418O-418Q
this can be hard to determine.   I describe a method in 418Yk
which can sometimes be used, but (like the trick in 418Nf) it is too
easy and too abstract to be often illuminating.   See 454G below for
something rather deeper.

The results of 418R-418T stand somewhat aside from anything else
considered in this chapter, but they form part of an important
technique.   A special case has already been mentioned in 253Yg.   I do
not discuss vector-valued measurable functions in this book, except
incidentally, but 418R is one of the fundamental results on their
representation;  it means, for instance, that if $V$ is any of the
Banach function spaces of Chapter 36 we can expect to represent Bochner
integrable $V$-valued functions (253Yf) in terms of functions on product
spaces, because $V$ will be continuously embedded in an $L^0$ space
(367O).   The measure-algebra version in 418T will be useful in
Volume 5 when establishing relationships between properties of measure
spaces and corresponding properties of measure algebras.
}%end of notes

\discrpage

