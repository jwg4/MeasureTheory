\frfilename{mt328.tex}
\versiondate{2.6.09}
\copyrightdate{2008}

\def\chaptername{Measure algebras}
\def\sectionname{Reduced products and other constructions}

\newsection{*328\dvAnew{2008}}

I devote a section to some related constructions.   At the
end of \S315 %315R 315S
I mentioned projective and inductive limits of systems of
Boolean algebras with linking homomorphisms.   In the context of the
present chapter, we naturally ask whether similar constructions can be
found for probability algebras.   For projective limits there is no
difficulty (328I).   For inductive limits the situation is more complex
(328H).   Some ideas in Volume 5 will
depend on what I call `reduced products'
(328A-328F), %328A 328B 328C 328D 328E 328F
which also provide a route to 328H.   The same methods give a route to
a useful result relating measure-preserving Boolean homomorphisms on a
probability algebra to measure-preserving automorphisms on a larger
probability algebra (328J).

\leader{328A}{Construction}
Let $\familyiI{(\frak A_i,\bar\mu_i)}$ be a non-empty family of probability
algebras, and $\Cal F$ an ultrafilter on $I$.

\spheader 328Aa Set

\Centerline{$\Cal J
=\{\familyiI{a_i}:\familyiI{a_i}\in\prod_{i\in I}\frak A_i$,
  $\lim_{i\to\Cal F}\bar\mu_ia_i=0\}$.}

\noindent Then $\Cal J$ is an ideal in the simple
product Boolean algebra $\prod_{i\in I}\frak A_i$.
\prooflet{\Prf\ If $\familyiI{a_i}$ and $\familyiI{b_i}$ belong to
$\Cal J$, and $\familyiI{c_i}\in\prod_{i\in I}\frak A_i$ is such that
$\familyiI{c_i}\Bsubseteq\familyiI{a_i}\Bcup\familyiI{b_i}$, then
$c_i\Bsubseteq a_i\Bcup b_i$ for every $i$, so

\Centerline{$\lim_{i\to\Cal F}\bar\mu_ic_i
\le\lim_{i\to\Cal F}\bar\mu_ia_i+\bar\mu_ib_i
=\lim_{i\to\Cal F}\bar\mu_ia_i+\lim_{i\to\Cal F}\bar\mu_ib_i
=0$}

\noindent and $\familyiI{c_i}\in\Cal J$.   Of course
$\familyiI{0_{\frak A_i}}$ belongs to $\Cal J$, so
$\Cal J\normalsubgroup\prod_{i\in I}\frak A_i$.\ \Qed}

\spheader 328Ab Let $\frak A$ be the
quotient Boolean algebra $\prod_{i\in I}\frak A_i/\Cal J$.   Then we have a
functional $\bar\mu:\frak A\to[0,1]$ defined by saying that

\Centerline{$\bar\mu(\familyiI{a_i}^{\ssbullet})
=\lim_{i\to\Cal F}\bar\mu_ia_i$}

\noindent whenever $\familyiI{a_i}\in\prod_{i\in I}\frak A_i$.
\prooflet{\Prf\ If $\familyiI{a_i}$,
$\familyiI{b_i}\in\prod_{i\in I}\frak A_i$ and
$\familyiI{a_i}^{\ssbullet}=\familyiI{b_i}^{\ssbullet}$, then
$\familyiI{a_i\Bsymmdiff b_i}\in\Cal J$, so

\Centerline{$|\lim_{i\to\Cal F}\bar\mu_ia_i-\lim_{i\to\Cal F}\bar\mu_ib_i|
=\lim_{i\to\Cal F}|\bar\mu_ia_i-\bar\mu_ib_i|
\le\lim_{i\to\Cal F}\bar\mu_i(a_i\Bsymmdiff b_i)
=0$.  \Qed}}

\leader{328B}{Proposition}
Let $\familyiI{(\frak A_i,\bar\mu_i)}$ be a non-empty family of probability
algebras and $\Cal F$ an ultrafilter on $I$, and construct $\frak A$ and
$\bar\mu$ as in 328A.   Then $(\frak A,\bar\mu)$ is a probability algebra.

\proof{{\bf (a)} If $\familyiI{a_i}$,
$\familyiI{b_i}\in\prod_{i\in I}\frak A_i$ and
$\familyiI{a_i}^{\ssbullet}\Bcap\familyiI{b_i}^{\ssbullet}=0$, then
$\familyiI{a_i\cap b_i}\in\Cal J$, so

$$\eqalign{
\bar\mu(\familyiI{a_i}^{\ssbullet}\Bcup\familyiI{b_i}^{\ssbullet})
&=\bar\mu(\familyiI{a_i\Bcup b_i}^{\ssbullet})
=\lim_{i\to\Cal F}\bar\mu_i(a_i\Bcup b_i)\cr
&=\lim_{i\to\Cal F}\bar\mu_ia_i+\bar\mu_ib_i-\bar\mu_i(a_i\Bcap b_i)\cr
&=\lim_{i\to\Cal F}\bar\mu_ia_i
  +\lim_{i\to\Cal F}\bar\mu_ib_i
  -\lim_{i\to\Cal F}\bar\mu_i(a_i\Bcap b_i)\cr
&=\lim_{i\to\Cal F}\bar\mu_ia_i
  +\lim_{i\to\Cal F}\bar\mu_ib_i
=\bar\mu(\familyiI{a_i}^{\ssbullet})+\bar\mu(\familyiI{b_i}^{\ssbullet}).
\cr}$$

\noindent So $\bar\mu$ is additive.

\medskip

{\bf (b)}
$1_{\frak A}=\familyiI{1_{\frak A_i}}^{\ssbullet}$ so

\Centerline{$\bar\mu 1_{\frak A}
=\lim_{i\to\Cal F}\bar\mu_i1_{\frak A_i}=1$.}

\medskip

{\bf (c)} If $\familyiI{a_i}\in\prod_{i\in I}\frak A_i$ and
$\bar\mu(\familyiI{a_i}^{\ssbullet})=0$, then
$\familyiI{a_i}\in\Cal J$ and $\familyiI{a_i}^{\ssbullet}=0$;  thus
$\bar\mu$ is strictly positive.

\medskip

{\bf (d)} Suppose that $\sequencen{\tilde a_n}$ is a disjoint sequence in
$\frak A$.   Express each $\tilde a_n$ as $\familyiI{a_{ni}}^{\ssbullet}$
where
$a_{ni}\in\frak A_i$ for each $i$.   Set $b_{ni}=\sup_{m\le n}a_{mi}$ for
$n\in\Bbb N$ and $i\in I$;  then
$\familyiI{b_{ni}}^{\ssbullet}=\sup_{m\le n}\tilde a_m$ in $\frak A$.   Set

\Centerline{$\gamma=\sum_{n=0}^{\infty}\bar\mu\tilde a_n
=\sup_{n\in\Bbb N}\bar\mu(\familyiI{b_{ni}}^{\ssbullet})
=\sup_{n\in\Bbb N}\lim_{i\to\Cal F}\bar\mu_ib_{ni}$.}

\noindent Set $A_n=\{i:i\in I$, $\bar\mu_ib_{ni}\le\gamma+2^{-n}\}$ for
each $n\in\Bbb N$;  then $\sequencen{A_n}$ is a non-increasing sequence in
$\Cal F$, and $A_0=I$.   For $i\in I$ set

$$\eqalign{b_i
&=b_{ni}\text{ if }i\in A_n\setminus A_{n+1},\cr
&=\sup_{n\in\Bbb N}b_{ni}\text{ if }i\in\bigcap_{n\in\Bbb N}A_n.\cr}$$

\noindent Consider $\tilde b=\familyiI{b_i}^{\ssbullet}\in\frak A$.
For each
$n\in\Bbb N$, $\{i:a_{ni}\Bsubseteq b_i$, $\bar\mu b_i\le\gamma+2^{-n}\}$
includes $A_n\in\Cal F$, so
$\tilde a_n\Bsubseteq\tilde b$ for every $n$ and
$\bar\mu\tilde b\le\gamma$.

If $\tilde c\in\frak A$ is another upper bound for
$\{\tilde a_n:n\in\Bbb N\}$, then, using (a),

\Centerline{$\gamma=\sup_{n\in\Bbb N}\bar\mu(\sup_{m\le n}\tilde a_m)
\le\bar\mu(\tilde b\Bcap\tilde c)\le\bar\mu\tilde b\le\gamma$;}

\noindent so $\bar\mu(\tilde b\Bsetminus\tilde c)=0$ and
$\tilde b\Bsetminus\tilde c=0$, by (c).
Thus $\tilde b=\sup_{n\in\Bbb N}\tilde a_n$ in $\frak A$, while
$\bar\mu\tilde b=\sum_{n=0}^{\infty}\bar\mu\tilde a_n$.

\medskip

{\bf (e)} If $\sequencen{\tilde a_n}$ is any sequence in
$\frak A$, then (iv) tells us that
$\{\tilde a_n\Bsetminus\sup_{m<n}\tilde a_m:n\in\Bbb N\}$
has a supremum in $\frak A$, which is also the supremum of
$\{\tilde a_n:n\in\Bbb N\}$.   So $\frak A$ is Dedekind $\sigma$-complete.
Now (d) tells us also that $\bar\mu$ is countably additive, so that
$(\frak A,\bar\mu)$ is a probability algebra.
}%end of proof of 328B

\leader{328C}{Definition} In the context of 328A/328B,
I will call $(\frak A,\bar\mu)$ the {\bf probability
algebra reduced product} of $\familyiI{(\frak A_i,\bar\mu_i)}$
modulo $\Cal F$;  I will
sometimes write it as $\prod_{i\in I}(\frak A_i,\bar\mu_i)|\Cal F$.
\cmmnt{(There are dangers in this notation.   In 351M I will speak of
`reduced powers' $\BbbR^I|\Cal F$, and the rules will be significantly
different there.)}

If all the $(\frak A_i,\bar\mu_i)$ are the same, with common value
$(\frak B,\bar\nu)$, I will write
$(\frak B,\bar\nu)^I|\Cal F$ for
$\prod_{i\in I}(\frak A_i,\bar\mu_i)|\Cal F$, and call it the
{\bf probability algebra reduced power}.

\vleader{108pt}{328D}{Proposition} Let $I$ be a set,
$\familyiI{(\frak A_i,\bar\mu_i)}$,
$\familyiI{(\frak B_i,\bar\nu_i)}$ and
$\familyiI{(\frak C_i,\bar\lambda_i)}$ three families of probability
algebras, and $\Cal F$ an ultrafilter on $I$;  let
$(\frak A,\bar\mu)=\prod_{i\in I}(\frak A_i,\bar\mu_i)|\Cal F$,
$(\frak B,\bar\nu)=\prod_{i\in I}(\frak B_i,\bar\nu_i)|\Cal F$ and
$(\frak C,\bar\lambda)=\prod_{i\in I}(\frak C_i,\bar\lambda_i)|\Cal F$
be the corresponding reduced products.

(a) If $\pi_i:\frak A_i\to\frak B_i$ is a measure-preserving Boolean
homomorphism for each $i\in I$, we have a measure-preserving Boolean
homomorphism $\pi:\frak A\to\frak B$ given by saying that

\Centerline{$\pi(\familyiI{a_i}^{\ssbullet})
=\familyiI{\pi_ia_i}^{\ssbullet}$}

\noindent whenever $a_i\in\frak A_i$ for every $i\in I$.

(b) If, in addition, $\phi_i:\frak B_i\to\frak C_i$ is a measure-preserving
Boolean homomorphism for each $i\in I$, and $\phi:\frak B\to\frak C$ is
constructed as in (a), then $\phi\pi:\frak A\to\frak C$ corresponds to the
family $\familyiI{\phi_i\pi_i}$.

\proof{{\bf (a)} Following through the construction in 328A, we have ideals

\Centerline{$\Cal J=\{\familyiI{a_i}:\lim_{i\to\Cal F}\bar\mu_ia_i=0\}
\normalsubgroup\prod_{i\in I}\frak A_i$,}

\Centerline{$\Cal K=\{\familyiI{b_i}:\lim_{i\to\Cal F}\bar\nu_ib_i=0\}
\normalsubgroup\prod_{i\in I}\frak B_i$,}

\noindent and a Boolean homomorphism
$\hat\pi:\prod_{i\in I}\frak A_i\to\prod_{i\in I}\frak B_i$ given by
the formula $\hat\pi\familyiI{a_i}=\familyiI{\pi_ia_i}$ (use 315Bb).
Because the
homomorphisms $\pi_i$ are measure-preserving, $\hat\pi\pmb{a}\in\Cal K$
whenever $\pmb{a}\in\Cal J$.   Consequently we have a Boolean homomorphism
$\pi:\prod_{i\in I}\frak A_i/\Cal J\to\prod_{i\in I}\frak B_i/\Cal K$ given
by setting $\pi\pmb{a}^{\ssbullet}=(\hat\pi\pmb{a})^{\ssbullet}$ whenever
$\pmb{a}\in\prod_{i\in I}\frak A_i$ (3A2G).   And

\Centerline{$\bar\nu\pi(\familyiI{a_i}^{\ssbullet})
=\bar\nu(\familyiI{\pi_ia_i}^{\ssbullet})
=\lim_{i\to\Cal F}\bar\nu_i\pi_ia_i
=\lim_{i\to\Cal F}\bar\mu_ia_i
=\bar\mu(\familyiI{a_i}^{\ssbullet})$}

\noindent whenever $\familyiI{a_i}\in\prod_{i\in I}\frak A_i$, so
$\pi$ is measure-preserving.

\medskip

{\bf (b)} is now just a matter of writing the defining formulae out.
}%end of proof of 328D

\leader{328E}{Proposition}\dvAformerly{3{}28D}
Let $I$ be a non-empty set, $\le$ a reflexive
transitive relation on $I$, and $\Cal F$ an ultrafilter on $I$ such that
$\{j:j\in I$, $j\ge i\}$ belongs to $\Cal F$ for every $i\in I$.
Let $\familyiI{(\frak A_i,\bar\mu_i)}$ be a family of probability algebras,
and suppose
that we are given a family $\langle\pi_{ji}\rangle_{i\le j}$  such that

\inset{$\pi_{ji}$ is a measure-preserving Boolean
homomorphism from $\frak A_i$ to $\frak A_j$ whenever $i\le j$ in $I$,

$\pi_{ki}=\pi_{kj}\pi_{ji}$ whenever $i\le j\le k$ in $I$.}

\noindent Let $(\frak A,\bar\mu)$ be the probability algebra reduced
product $\prod_{i\in I}(\frak A_i,\bar\mu_i)|\Cal F$.

(a) For each $i\in I$ we have a measure-preserving Boolean
homomorphism $\pi_i:\frak A_i\to\frak A$ defined by saying that
$\pi_ia=\family{j}{I}{a_j}^{\ssbullet}$ whenever $a_j=\pi_{ji}a$ for every
$j\ge i$, and $\pi_i=\pi_j\pi_{ji}$ whenever $i\le j$ in $I$.

(b) $\familyiI{a_i}^{\ssbullet}\Bsubseteq\sup_{j\in A}\pi_ja_j$
whenever $\familyiI{a_i}\in\prod_{i\in I}\frak A_i$ and $A\in\Cal F$.

\proof{{\bf (a)} $\pi_i$ is well-defined because $\{j:j\ge i\}\in\Cal F$.
It is a measure-preserving Boolean homomorphism because every $\pi_{ji}$
is.   If $i\le j$ in $I$, $a\in\frak A_i$ and $a_k=\pi_{ki}a$ for every
$k\ge i$, then $a_k=\pi_{kj}\pi_{ji}a$ for every $k\ge j$, so
$\pi_j\pi_{ji}a=\family{k}{I}{a_k}^{\ssbullet}=\pi_ia$;  as $a$ is
arbitrary, $\pi_j\pi_{ji}=\pi_i$.

\medskip

{\bf (b)} Set $c=\sup_{j\in A}\pi_ja_j$ in $\frak A$.
For any $\epsilon>0$, there is a finite $K\subseteq A$ such that
$\bar\mu c\le\epsilon+\bar\mu(\sup_{j\in K}\pi_ja_j)$ (321C).
The set $B=\{k:k\in I$, $j\le k$ for every $j\in K\}$ belongs to $\Cal F$,
so is not empty;  fix $k\in B$, and set
$b=\sup_{j\in K}\pi_{kj}a_j\in\frak A_k$,

$$\eqalign{b_i
&=\pi_{ik}b\text{ if }i\ge k,\cr
&=0\text{ for other }i\in I.\cr}$$

\noindent Then

\Centerline{$\familyiI{b_i}^{\ssbullet}
=\pi_kb
=\pi_k(\sup_{j\in K}\pi_{kj}a_j)
=\sup_{j\in K}\pi_k\pi_{kj}a_j
=\sup_{j\in K}\pi_ja_j\Bsubseteq c$.}

\noindent If $i\in A$ and $i\ge k$, then

$$\eqalign{\bar\mu_i(a_i\Bsetminus b_i)
&=\bar\mu(\pi_ia_i\Bsetminus\pi_ib_i)
=\bar\mu(\pi_ia_i\Bsetminus\pi_i\pi_{ik}b)\cr
&=\bar\mu(\pi_ia_i\Bsetminus\pi_kb)
=\bar\mu(\pi_ia_i\Bsetminus\sup_{j\in K}\pi_ja_j)
\le\bar\mu(c\Bsetminus\sup_{j\in K}\pi_ja_j)
\le\epsilon\cr}$$

\noindent by the choice of $K$.   So

$$\eqalign{\bar\mu(\familyiI{a_i}^{\ssbullet}\Bsetminus c)
&\le\bar\mu(\familyiI{a_i}^{\ssbullet}\Bsetminus\familyiI{b_i}^{\ssbullet})
=\bar\mu(\familyiI{a_i\Bsetminus b_i}^{\ssbullet})\cr
&=\lim_{i\to\Cal F}\bar\mu_i(a_i\Bsetminus b_i)
\le\sup_{i\in A,i\ge k}\bar\mu_i(a_i\Bsetminus b_i)
\le\epsilon.\cr}$$

\noindent As $\epsilon$ is arbitrary,
$\familyiI{a_i}^{\ssbullet}\Bsubseteq c$.
}%end of proof of 328E

\leader{328F}{Corollary} Suppose that $\sequencen{(\frak A_n,\bar\mu_n)}$
is a sequence of
probability algebras, $\phi_n:\frak A_n\to\frak A_{n+1}$ is a
measure-preserving Boolean homomorphism for each $n$ and $\Cal F$ is a
non-principal ultrafilter on $\Bbb N$.   Let
$(\frak A,\bar\mu)$ be the probability algebra reduced product
$\prod_{n\in\Bbb N}(\frak A_n,\bar\mu_n)|\Cal F$.   Then we have canonical
measure-preserving Boolean homomorphisms $\pi_n:\frak A_n\to\frak A$ such
that
$\sequencen{a_n}^{\ssbullet}\Bsubseteq\sup_{n\in A}\pi_na_n$ whenever
$\sequencen{a_n}\in\prod_{n\in\Bbb N}\frak A_n$ and $A\in\Cal F$,
and $\pi_{n+1}\phi_n=\pi_n$ for every $n\in\Bbb N$.

\proof{ Apply 328E with $\pi_{ji}=\phi_{j-1}\ldots\phi_{i+1}\phi_i$
whenever $i<j$.
}%end of proof of 328F

\leader{328G}{Corollary}
Let $(\frak B,\bar\nu)$ be a probability algebra, $I$ a non-empty set, and
$\Cal F$ an ultrafilter on $I$.   Let $(\frak A,\bar\mu)$ be the
probability algebra reduced power $(\frak B,\bar\nu)^I|\Cal F$.

(a) We have a measure-preserving Boolean homomorphism
$\pi:\frak B\to\frak A$ defined by saying that
$\pi b=\familyiI{b}^{\ssbullet}$ for $b\in\frak B$.

(b)

\Centerline{$\familyiI{b_i}^{\ssbullet}
\Bsubseteq\sup_{j\in A}\pi b_j=\pi(\sup_{j\in A}b_j)$}

\noindent whenever $A\in\Cal F$ and $\familyiI{b_i}\in\frak B^I$.

\proof{ Apply 328E with $\le\mskip5mu=I\times I$ and
$\pi_{ji}$ the identity
operator on $\frak B$ for all $i$, $j\in I$.
}%end of proof of 328G

\leader{328H}{Proposition} Let $(I,\le)$ be an upwards-directed partially
ordered set, and $\familyiI{(\frak A_i,\bar\mu_i)}$
a family of probability algebras;  suppose
that $\pi_{ji}:\frak A_i\to\frak A_j$ is a measure-preserving
Boolean homomorphism whenever $i\le j$, and that
$\pi_{ki}=\pi_{kj}\pi_{ji}$ whenever $i\le j\le k$.
Then there are a probability algebra $(\frak C,\bar\lambda)$ and a family
$\familyiI{\pi_i}$ such that

\inset{$\pi_i:\frak A_i\to\frak C$ is a measure-preserving
Boolean homomorphism for each $i\in I$,

$\pi_i=\pi_j\pi_{ji}$ whenever $i\le j$,

$\{0,1\}\cup\bigcup_{i\in I}\pi_i[\frak A_i]$ is topologically dense in
$\frak C$,}

\noindent and whenever $(\frak B,\bar\nu)$, $\familyiI{\phi_i}$ are such
that

\inset{$(\frak B,\bar\nu)$ is a probability algebra,

$\phi_i:\frak A_i\to\frak B$ is a measure-preserving
Boolean homomorphism for each $i\in I$,

$\phi_i=\phi_j\pi_{ji}$ whenever $i\le j$,}

\noindent then there is a unique measure-preserving Boolean homomorphism
$\phi:\frak C\to\frak B$ such that $\phi\pi_i=\phi_i$ for every $i\in I$.

\proof{{\bf (a)}
If $I$ is empty the result is trivial (take $\frak C=\{0,1\}$);  so
let us suppose henceforth that $I\ne\emptyset$.   In this case,

\Centerline{$\{A:A\subseteq I$, there is some $i\in I$ such that $j\in A$
whenever $i\le j\}$}

\noindent is a filter on $I$, and is included in an ultrafilter $\Cal F$
say (2A1O).   Let $(\frak A,\bar\mu)$ be the reduced product
$\prod_{i\in I}(\frak A_i,\bar\mu_i)|\Cal F$.   Then we have for each
$i\in I$ a measure-preserving Boolean homomorphism
$\pi_i:\frak A_i\to\frak A$ such that
$\pi_i=\pi_j\pi_{ji}$ whenever $i\le j$ (328E).
If $i\le j$ in $I$, then
$\pi_i[\frak A_i]\subseteq\pi_j[\frak A_j]$;  because $(I,\le)$
is upwards-directed,
$\familyiI{\pi_i[\frak A_i]}$ is an upwards-directed family of
subalgebras of $\frak A$, and
$\frak D=\bigcup_{i\in I}\pi_i[\frak A_i]$ is a subalgebra of
$\frak A$;  let $\frak C$ be its closure (323J).   Set
$\bar\lambda=\bar\mu\restrp\frak C$, so that $(\frak C,\bar\lambda)$ is a
probability algebra, and $\pi_i:\frak A_i\to\frak C$ is a
measure-preserving Boolean homomorphism for each $i\in I$, with
$\pi_i=\pi_j\pi_{ji}$ whenever $i\le j$.

\medskip

{\bf (b)}
Now suppose that $\frak B$ and $\familyiI{\phi_i}$ are as declared.

\medskip

\quad{\bf (i)} Set

\Centerline{$\phi'=\{(\pi_ia,\phi_ia):i\in I$, $a\in\frak A_i\}
\subseteq\frak D\times\frak B$.}

\noindent Then $\phi'$ is (the graph of) a function from $\frak D$ to
$\frak B$.
\Prf\ If $c\in\frak D$, there is surely an $i\in I$ such that
$c\in\pi_i[\frak A_i]$, so that $(c,\phi_ia)\in\phi'$ for some
$a\in\frak A_i$.
If $(c,b)$ and $(c,b')$ belong to $\phi'$, there are $i$, $j\in I$ and
$a\in\frak A_i$, $a'\in\frak A_j$ such that

\Centerline{$\pi_ia=\pi_ja'=c$,
\quad$\phi_ia=b$,
\quad$\phi_ja'=b'$.}

\noindent Let $k\in I$ be such that $i\le k$ and $j\le k$;  then

\Centerline{$\pi_k\pi_{ki}a=\pi_ia=c=\pi_ja'=\pi_k\pi_{kj}a'$.}

\noindent As $\pi_k$ is measure-preserving, therefore injective,
$\pi_{ki}a=\pi_{kj}a'$, and

\Centerline{$b=\phi_ia=\phi_k\pi_{ki}a=\phi_k\pi_{kj}a'=\phi_ja'=b'$.}

\noindent So each element of $\frak D$ is the first member of exactly one
element of $\phi'$, and $\phi'$ is the graph of a function.\ \QeD\   Of
course the defining formula for $\phi'$ guarantees that
$\phi'\pi_i=\phi_i:\frak A_i\to\frak B$ for every $i\in I$.

\medskip

\quad{\bf (ii)} Next, $\phi':\frak D\to\frak B$
is a measure-preserving Boolean homomorphism.   \Prf\ If
$c$, $c'\in\frak D$ then there are $i$, $j\in I$ and $a\in\frak A_i$,
$a'\in\frak A_j$ such that $c=\pi_ia$ and $c'=\pi_ja'$.   Again take
$k\in I$ such that $i\le k$ and $j\le k$;  then

\Centerline{$c=\pi_k\pi_{ki}a$,
\quad$c'=\pi_k\pi_{kj}a'$,
\quad$\phi'c=\phi_k\pi_{ki}a$,
\quad$\phi'c'=\phi_k\pi_{kj}a'$.}

\noindent In this case, for either of the Boolean operations
$*=\Bsymmdiff$ or $*=\Bcap$, we have

$$\eqalign{\phi'c*\phi'c'
&=\phi_k\pi_{ki}a*\phi_k\pi_{kj}a'
=\phi_k(\pi_{ki}a*\pi_{kj}a')\cr
&=\phi'\pi_k(\pi_{ki}a*\pi_{kj}a')
=\phi'(\pi_k\pi_{ki}a*\pi_k\pi_{kj}a')
=\phi'(c*c').\cr}$$

\noindent As $c$, $c'$ and $*$ are arbitrary,
$\phi'$ is a ring homomorphism.   Moreover, in the same context,

\Centerline{$\bar\nu\phi'c=\bar\nu\phi_ia=\bar\mu_ia
=\bar\mu\pi_ia=\bar\lambda c$,}

\noindent so $\phi'$ is measure-preserving.
It follows that $\phi'1_{\frak C}=1_{\frak B}$,
and $\phi'$ is a Boolean homomorphism.\ \Qed

\medskip

\quad{\bf (iii)}
By 324O, there is a unique extension of $\phi'$ to a measure-preserving
Boolean homomorphism $\phi:\frak C\to\frak B$;  and of course we still have
$\phi\pi_i=\phi_i$ for every $i\in I$.

\medskip

\quad{\bf (iv)}
To see that $\phi$ is unique, take any measure-preserving Boolean
homomorphism $\tilde\phi:\frak C\to\frak B$ such that
$\tilde\phi\pi_i=\phi_i$ for every $i$.   Then $\tilde\phi$ must agree with
$\phi$ on $\pi_i[\frak A_i]$ for every $i$, so
$\tilde\phi\restrp\frak D=\phi\restrp\frak D$;  as $\frak D$ is
topologically dense in $\frak C$, $\tilde\phi=\phi$ (324O again).
}%end of proof of 328H

\leader{328I}{}\cmmnt{ For completeness, I spell out the relatively
elementary construction for projective limits.

\medskip

\noindent}{\bf Proposition} Let $(I,\le)$ be a non-empty
upwards-directed set, and $\familyiI{(\frak A_i,\bar\mu_i)}$
a family of probability algebras;
suppose that $\pi_{ij}:\frak A_j\to\frak A_i$ is a measure-preserving
Boolean homomorphism for $i\le j$ in $I$, and that
$\pi_{ij}\pi_{jk}=\pi_{ik}$ whenever $i\le j\le k$.
Then
there are a probability algebra $(\frak C,\bar\lambda)$ and a family
$\familyiI{\pi_i}$ such that

\inset{$\pi_i:\frak C\to\frak A_i$ is a measure-preserving
Boolean homomorphism for each $i\in I$,

$\pi_i=\pi_{ij}\pi_j$ whenever $i\le j$,}

\noindent and whenever $(\frak B,\bar\nu)$, $\familyiI{\phi_i}$
are such that

\inset{$(\frak B,\bar\nu)$ is a probability algebra,

$\phi_i:\frak B\to\frak A_i$ is a measure-preserving
Boolean homomorphism for each $i\in I$,

$\phi_i=\pi_{ij}\phi_j$ whenever $i\le j$,}

\noindent then there is a unique measure-preserving Boolean homomorphism
$\phi:\frak B\to\frak C$ such that $\pi_i\phi=\phi_i$ for every $i\in I$.

\proof{{\bf (a)} Let $\frak C\subseteq\prod_{i\in I}\frak A_i$ be the set

\Centerline{$\{\familyiI{a_i}:\pi_{ij}a(j)=a(i)$ whenever $i\le j$ in
$I\}$.}

\noindent Because every $\pi_{ij}$ is a Boolean homomorphism, $\frak C$ is
a subalgebra of $\prod_{i\in I}\frak A_i$;  taking
$\pi_j(\familyiI{a_i})=a_j$ whenever $\familyiI{a_i}\in\frak C$,
$\pi_j:\frak C\to\frak A_j$ is a Boolean homomorphism for every $j\in I$,
and $\pi_i=\pi_{ij}\pi_j$ whenever $i\le j$.

Because every $\pi_{ij}$ is order-continuous, $\frak C$ is an order-closed
subalgebra of $\prod_{i\in I}\frak A_i$, so is Dedekind complete.

\medskip

{\bf (b)} If $c=\familyiI{a_i}\in\frak C$, then

\Centerline{$\bar\mu_i\pi_ic=\bar\mu_ia_i=\bar\mu_i\pi_{ij}a_j
=\bar\mu_ja_j=\bar\mu_j\pi_jc$}

\noindent whenever $i\le j$ in $I$;  because $I$ is upwards-directed,
$\bar\mu_i\pi_ic=\bar\mu_j\pi_jc$ for all $i$, $j\in I$.   So we have a
functional $\bar\lambda:\frak C\to[0,1]$ defined by setting
$\bar\lambda c=\bar\mu_i\pi_ic$ whenever $c\in\frak C$ and $i\in I$.
Note that $1_{\frak C}=\familyiI{1_{\frak A_i}}$, so
$\bar\lambda 1_{\frak C}=\bar\mu_i1_{\frak A_i}=1$, for any $i\in I$.

If $\sequencen{c_n}$ is a disjoint sequence in $\frak C$ with supremum $c$,
then express each $c_n$ as $\familyiI{a_{ni}}$;  we must have
$c=\familyiI{\sup_{n\in\Bbb N}a_{ni}}$, so

\Centerline{$\bar\lambda c
=\bar\mu_i(\sup_{n\in\Bbb N}a_{ni})
=\sum_{n=0}^{\infty}\bar\mu_ia_{ni}
=\sum_{n=0}^{\infty}\bar\lambda c_n$}

\noindent for any $i\in I$.   Thus $\bar\lambda$ is countably additive.
If $c\in\frak C$ is non-zero, express it as $\familyiI{a_i}$;  there must
be an $i\in I$ such that $a_i\ne 0$, so that
$\bar\lambda c=\bar\mu_ia_i>0$.   Thus $\bar\lambda$ is strictly positive,
and $(\frak C,\bar\lambda)$ is a probability algebra.

\medskip

{\bf (c)} If $(\frak B,\bar\nu)$ is a probability algebra and
$\familyiI{\phi_i}$ is a family such that $\phi_i:\frak B\to\frak A_i$ is a
measure-preserving Boolean homomorphism and $\phi_i=\pi_{ij}\phi_j$
whenever $i\le j$ in $I$, set $\phi b=\familyiI{\phi_ib}$ for
$b\in\frak B$.   Then $\phi:\frak B\to\prod_{i\in I}\frak A_i$ is a Boolean
homomorphism;  also

\Centerline{$\pi_{ij}(\phi b)(j)=\pi_{ij}\phi_jb=\phi_ib=(\phi b)(i)$}

\noindent whenever $i\le j$ and $b\in\frak B$, so
$\phi[\frak B]\subseteq\frak C$, while $\pi_i\phi=\phi_i$ for every
$i\in I$.   And of course this uniquely determines $\phi$.   To see that
$\phi$ is measure-preserving, we have only to check that

\Centerline{$\bar\lambda\phi b
=\bar\mu_i\pi_i\phi b=\bar\mu_i\phi_ib=\bar\nu b$}

\noindent whenever $b\in\frak B$ and $i\in I$.
}%end of proof of 328I

\leader{328J}{}\cmmnt{ A different application of the method in 328A yields
the following result on commuting families of Boolean homomorphisms.

\medskip

\noindent}{\bf Theorem} Let $(\frak A,\bar\mu)$ be a probability
algebra, and $\Phi$
a family of measure-preserving Boolean homomorphisms from $\frak A$ to
itself such that $\phi\psi=\psi\phi$ for all $\phi$, $\psi\in\Phi$.   Then
there are a probability algebra $(\frak C,\bar\lambda)$, a
measure-preserving Boolean homomorphism $\pi:\frak A\to\frak C$
and a family $\family{\phi}{\Phi}{\tilde\phi}$ such that

(i) $\tilde\phi:\frak C\to\frak C$ is a measure-preserving Boolean
automorphism and $\tilde\phi\pi=\pi\phi$ for every $\phi\in\Phi$;

(ii) $(\phi\psi)\ssptilde=\tilde\phi\tilde\psi$ for all $\phi$,
$\psi\in\Phi$.

\proof{{\bf (a)} Let $\Psi$ be the set of all products
$\phi_0\phi_1\ldots\phi_n$ where $\phi_i\in\Phi\cup\{\iota\}$ for every
$i\le n$, $\iota$ here being the identity map from $\frak A$ to itself.
Then $\Psi$ is a family of measure-preserving Boolean homomorphisms from
$\frak A$ to itself, and $\phi\psi=\psi\phi\in\Psi$
for all $\phi$, $\psi\in\Psi$.

\medskip

{\bf (b)} For $\phi$, $\psi\in\Psi$, say that $\phi\le\psi$ if there is a
$\theta\in\Psi$ such that $\phi\theta=\psi$.   Then $\le$ is a reflexive
transitive relation on $\Psi$.   Note that if $\phi\le\psi$ in $\Psi$
then there is exactly one $\theta\in\Psi$ such that $\phi\theta=\psi$,
because $\phi$ is injective.   So we may define $\pi_{\psi,\phi}\in\Psi$
by saying that $\phi\pi_{\psi,\phi}=\psi$ whenever $\phi\le\psi$ in
$\Psi$;  that is,
$\pi_{\phi\psi,\phi}=\psi$ whenever $\phi$, $\psi\in\Psi$.
Observe that if $\phi\le\psi\le\theta$ in $\Psi$, then

\Centerline{$\phi\pi_{\psi,\phi}\pi_{\theta,\psi}
=\psi\pi_{\theta,\psi}=\theta=\phi\pi_{\theta,\phi}$,}

\noindent so

\Centerline{$\pi_{\theta,\phi}=\pi_{\psi,\phi}\pi_{\theta,\psi}
=\pi_{\theta,\psi}\pi_{\psi,\phi}$.}

\noindent Of course $\iota\le\phi$ for every
$\phi\in\Psi$.

\medskip

{\bf (c)} If $\phi_1$, $\phi_2\in\Psi$ then $\phi_1\le\phi_1\phi_2$ and
$\phi_2\le\phi_2\phi_1=\phi_1\phi_2$;  generally, if $D\subseteq\Psi$ is
finite, there is a $\psi\in\Psi$ such that $\phi\le\psi$ for every
$\phi\in D$.   Consequently

\Centerline{$\{A:A\subseteq\Psi$, there is some $\phi\in\Psi$
such that $\psi\in A$ whenever $\phi\le\psi\}$}

\noindent is a filter on $\Psi$, and is included in an ultrafilter
$\Cal F$ say.   Let $(\frak C_0,\bar\lambda_0)$ be the
probability algebra reduced power $(\frak A,\bar\mu)^{\Psi}|\Cal F$.
By 328E, we have for each $\phi\in\Psi$ a measure-preserving Boolean
homomorphism $\pi_{\phi}:\frak A\to\frak C_0$ defined by saying that
$\pi_{\phi}a=\family{\psi}{\Psi}{a_{\psi}}^{\ssbullet}$ if
$a_{\psi}=\pi_{\psi,\phi}a$ whenever $\phi\le\psi$ in $\Psi$, and
$\pi_{\phi}=\pi_{\psi}\pi_{\psi,\phi}$ whenever $\phi\le\psi$.
Re-interpreting this in terms of the definitions of $\le$ and
$\pi_{\psi,\phi}$, we have $\pi_{\phi}=\pi_{\phi\psi}\psi$ whenever
$\phi$, $\psi\in\Psi$.

\medskip

{\bf (d)} If $\phi$, $\psi$ in $\Psi$, then

\Centerline{$\pi_{\phi}[\frak A]\cup\pi_{\psi}[\frak A]
=\pi_{\phi\psi}[\psi[\frak A]]\cup\pi_{\psi\phi}[\phi[\frak A]]
\subseteq\pi_{\phi\psi}[\frak A]\cup\pi_{\psi\phi}[\frak A]
=\pi_{\phi\psi}[\frak A]$,}

\noindent which is a subalgebra of $\frak C_0$.
So $\frak D=\bigcup_{\phi\in\Psi}\pi_{\phi}[\frak A]$ is a
subalgebra of
$\frak C_0$, and its closure $\frak C$ is a closed subalgebra of
$\frak C_0$;  set $\bar\lambda=\bar\lambda_0\restrp\frak C$.   Then
$\pi=\pi_{\iota}:\frak A\to\frak C$ is a measure-preserving Boolean
homomorphism.

\medskip

{\bf (e)} If $\theta\in\Psi$, we have a measure-preserving Boolean
homomorphism $\hat\theta:\frak C\to\frak C$ defined by the formula

\Centerline{$\hat\theta(\family{\psi}{\Psi}{a_{\psi}}^{\ssbullet})
=\family{\psi}{\Psi}{\theta a_{\psi}}^{\ssbullet}$}

\noindent for every family
$\family{\psi}{\Psi}{a_{\psi}}$ in $\frak A$ (328Da);  and
%$(\theta\phi)\sphat=\hat\theta\hat\phi$
%$(\theta\phi)\mskip-3mu\hat{\phantom{x}}=\hat\theta\hat\phi$
$\widehat{\theta\phi}=\hat\theta\hat\phi$
for all $\theta$, $\phi\in\Psi$
(328Db).   Also $\hat\theta\pi_{\phi}=\pi_{\phi}\theta$ for every $\phi$,
$\theta\in\Psi$.   \Prf\ Let $a\in\frak A$.   Define
$\family{\psi}{\Psi}{a_{\psi}}$,
$\family{\psi}{\Psi}{a'_{\psi}}$ by setting

$$\eqalign{a_{\psi}
&=\pi_{\psi,\phi}a\text{ when }\phi\le\psi,\cr
&=0\text{ otherwise},\cr
a'_{\psi}&=\pi_{\psi,\phi}\theta a=\theta\pi_{\psi,\phi}a
   \text{ when }\phi\le\psi,\cr
&=0\text{ otherwise}.}$$

\noindent Then

\Centerline{$\pi_{\phi}a=\family{\psi}{\Psi}{a_{\psi}}^{\ssbullet}$,}

\Centerline{$\hat\theta\pi_{\phi}a
=\family{\psi}{\Psi}{\theta a_{\psi}}^{\ssbullet}
=\family{\psi}{\Psi}{a'_{\psi}}^{\ssbullet}
=\pi_{\phi}\theta a$.   \Qed}

\medskip

{\bf (f)} It follows that, for $\theta\in\Psi$,

\Centerline{$\hat\theta[\frak D]
=\bigcup_{\phi\in\Psi}\hat\theta[\pi_{\phi}[\frak A]]
=\bigcup_{\phi\in\Psi}\pi_{\phi}[\theta[\frak A]]
\subseteq\frak D$.}

\noindent But in fact $\hat\theta[\frak D]=\frak D$.   \Prf\ If
$d\in\frak D$, there are $\phi\in\Psi$ and $a\in\frak A$ such that
$\pi_{\phi}a=d$.   Now define

$$\eqalign{a_{\psi}
&=\pi_{\psi,\phi}a\text{ if }\phi\le\psi,\cr
&=0\text{ for other }\psi\in\Psi,\cr
a'_{\psi}
&=\pi_{\psi,\phi\theta}a\text{ if }\phi\theta\le\psi,\cr
&=0\text{ for other }\psi\in\Psi,\cr
d'
&=\pi_{\phi\theta}a=\family{\psi}{\Psi}{a'_{\psi}}^{\ssbullet}.\cr
}$$

\noindent In this case, if $\phi\theta\le\psi$,

\Centerline{$\phi\theta a'_{\psi}=\psi a$,
\quad$\phi a_{\psi}=\psi a$}

\noindent so $\theta a'_{\psi}=a_{\psi}$.   Consequently

$$\eqalignno{\hat\theta d'
&=\hat\theta(\family{\psi}{\Psi}{a'_{\psi}}^{\ssbullet})
=\family{\psi}{\Psi}{\theta a'_{\psi}}^{\ssbullet}
=\family{\psi}{\Psi}{a_{\psi}}^{\ssbullet}\cr
\displaycause{because $\{\psi:\phi\theta\le\psi\}\in\Cal F$}
&=d,\cr}$$

\noindent and $d=\hat\theta\pi_{\phi\theta}a\in\hat\theta[\frak D]$.\ \Qed

\medskip

{\bf (g)} Since $\hat\theta[\frak C]$ is a closed subalgebra of $\frak C_0$
(324Kb) in which $\hat\theta[\frak D]=\frak D$ is topologically dense
(3A3Eb), $\hat\theta[\frak C]=\frak C$.   Setting
$\tilde\theta=\hat\theta\restrp\frak C$, we see that
$\tilde\theta:\frak C\to\frak C$ is a surjective measure-preserving Boolean
homomorphism, so is a Boolean automorphism.   Since
%$(\phi\theta)\sphat=\hat\phi\hat\theta$,
%$(\phi\theta)\mskip-3mu\hat{\phantom{x}}=\hat\phi\hat\theta$,
$\widehat{\phi\theta}=\hat\phi\hat\theta$,
we have $(\phi\theta)\ssptilde=\tilde\phi\tilde\theta$ for all $\phi$,
$\theta\in\Psi$.

\medskip

{\bf (h)} Finally, as observed at the beginning of (e),

\Centerline{$\tilde\theta\pi
=\tilde\theta\pi_{\iota}
=\hat\theta\pi_{\iota}
=\pi_{\iota}\theta
=\pi\theta$}

\noindent for every $\theta\in\Psi$.   So
$(\frak C,\bar\lambda,\pi,\family{\theta}{\Phi}{\tilde\theta})$
has the required properties.
}%end of proof of 328J

\exercises{\leader{328X}{Basic exercises (a)}
%\spheader 328Xa
Write out a version of the proof of 328J adapted to the
case in which $\Phi=\{\phi\}$ is a singleton.   (This is an
abstract version of a construction known as the `natural extension' of
an \imp\ function;  see {\smc Petersen 83}, 1.3G.)
%328J

\spheader 328Xb Let $\nu_{\Bbb N}$ be the usual measure on
$X=\{0,1\}^{\Bbb N}$, and $(\frak B_{\Bbb N},\bar\nu_{\Bbb N})$
its measure algebra.
(i) Find \imp\ functions $f$, $g:X\to X$ such that $gf=g$ but
$f(x)\ne x$ for every $x\in X$.  \Hint{try
$g(x)(n)=x(n+1)$.}   (ii) Find measure-preserving Boolean homomorphisms
$\phi$, $\psi:\frak B_{\Bbb N}\to\frak B_{\Bbb N}$
such that $\phi\psi=\psi$ but $\phi$ is
not the identity.   (iii) In 328J, show that the hypothesis that members of
$\Phi$ commute cannot be omitted.
%328J

%\leader{328Y}{Further exercises (a)}
}%end of exercises

\endnotes{
\Notesheader{328} I have starred this section because it is far from the
main line of argument of the volume, and most readers should be moving on
to Maharam's theorem and the Lifting Theorem.   However the results here,
while natural enough, have some features which demand a little attention,
and it will be useful to be able to call on exact formulations of the
ideas.

The proof of 328H begins by taking an ultrafilter on $I$.   This ought to
ring bells.   It should be clear from the statement of the proposition that
$(\frak C,\bar\lambda,\familyiI{\pi_i})$ is determined up to isomorphism by
the properties declared here.   It cannot therefore depend on
which ultrafilter we pick, and there ought to be a construction not relying
on this approach (and, we can hope, not demanding any
application of the axiom of choice).   This is indeed the case, and
in 392Yd below I will sketch a method which can be adapted to give such a
proof.   Yet another proof of 328H is proposed in 418Yp in Volume 4.

The same remarks apply to the proof of 328J.   In the result as stated, I
have not imposed conditions on the structure
$(\frak C,\bar\lambda,\pi,\family{\phi}{\Phi}{\tilde\phi})$
sufficient to define it uniquely, but once again
it is not necessary to employ an ultrafilter, and in fact the filter

\Centerline{$\{A:A\subseteq\Psi$, there is some $\phi\in\Psi$
such that $\psi\in A$ whenever $\phi\le\psi\}$}

\noindent is already enough, if we take the trouble to move to the right
subalgebra of $\frak A^{\Psi}$ before taking the quotient algebra.
}%end of notes

\discrpage
