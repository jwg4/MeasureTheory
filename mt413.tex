\frfilename{mt413.tex}
\versiondate{25.2.05}
\copyrightdate{2000}

\def\chaptername{Topologies and measures I}
\def\sectionname{Inner measure constructions}

\newsection{413}

I now turn in a different direction, giving some basic results on
the construction of inner regular measures.   The first step is to
describe `inner measures' (413A) and a construction corresponding to the
\Caratheodory\ construction of measures from outer measures (413C).
Just as every measure gives rise to an outer measure, it gives rise to
an inner measure (413D).   Inner measures form an effective tool for
studying complete locally determined measures (413F).

The most substantial results of the section concern the construction of
measures as extensions of functionals defined on various classes
$\Cal K$ of sets.   Typically, $\Cal K$ is closed under finite unions
and countable
intersections, though it we can sometimes relax the hypotheses a bit.
The methods here make it possible to distinguish
arguments which produce finitely additive functionals (413H, 413N, 413P,
413Q) from
the succeeding steps to countably additive measures (413I, 413O, 413S).
413H-413M investigate conditions on a functional
$\phi:\Cal K\to\coint{0,\infty}$ sufficient to produce a measure
extending $\phi$, necessarily
unique, which is inner regular with respect to $\Cal K$ or
$\Cal K_{\delta}$, the
set of intersections of sequences in $\Cal K$.   413N-413O look instead
at functionals defined on sublattices of the class $\Cal K$ of interest,
and at sufficient conditions to ensure the existence of a measure, not
normally unique, defined on the whole of $\Cal K$, inner regular with
respect to
$\Cal K$ and extending the given functional.   Finally, 413P-413S are
concerned with majorizations rather than extensions;  we seek a measure
$\mu$ such that $\mu K\ge\lambda K$ for $K\in\Cal K$, while $\mu X$ is
as small as possible.

\leader{413A}{}\cmmnt{ I begin with some material from the exercises
of earlier volumes.

\medskip

\noindent}{\bf Definition} Let $X$ be a set.   An {\bf inner measure} on
$X$ is a functional $\phi:\Cal PX\to[0,\infty]$ such that

\inset{$\phi\emptyset=0$;}

\inset{($\alpha$) $\phi(A\cup B)\ge\phi A+\phi B$ for all disjoint $A$,
$B\subseteq X$;}

\inset{($\beta$) if $\sequencen{A_n}$ is a non-increasing sequence of
subsets of $X$ and $\phi A_0<\infty$ then
$\phi(\bigcap_{n\in\Bbb N}A_n)=\inf_{n\in\Bbb N}\phi A_n$;}

\inset{($*$) $\phi A=\sup\{\phi B:B\subseteq A,\,\phi B<\infty\}$ for
every $A\subseteq X$.}

\leader{413B}{}\cmmnt{ The following fact will be recognised as an
element of \Caratheodory's method.   There will be an application later
in which it
will be useful to know that it is not confined to proving countable
additivity.

\medskip

\noindent}{\bf Lemma} Let $X$ be a set and $\phi:X\to[0,\infty]$ any
functional such that $\phi\emptyset=0$.   Then

\Centerline{$\Sigma
=\{E:E\subseteq X,\,\phi A=\phi(A\cap E)+\phi(A\setminus E)$
for every $A\subseteq X\}$}

\noindent is an algebra of subsets of $X$, and
$\phi(E\cup F)=\phi E+\phi F$ for all disjoint $E$, $F\in\Sigma$.

\proof{ The symmetry of the definition of $\Sigma$ ensures that
$X\setminus E\in\Sigma$ whenever $E\in\Sigma$.   If $E$, $F\in\Sigma$ and
$A\subseteq X$, then

$$\eqalignno{\phi(A\cap(E\cup &F))+\phi(A\setminus(E\cup F))\cr
&=\phi(A\cap(E\cup F)\cap E)+\phi(A\cap(E\cup F)\setminus E)
   +\phi(A\setminus(E\cup F))\cr
&=\phi(A\cap E)+\phi((A\setminus E)\cap F)
   +\phi((A\setminus E)\setminus F)\cr
&=\phi(A\cap E)+\phi(A\setminus E)
=\phi A.\cr}$$

\noindent As $A$ is arbitrary, $E\cup F\in\Sigma$.   Finally, if
$A\subseteq X$,

\Centerline{$\phi(A\cap\emptyset)+\phi(A\setminus\emptyset)
=\phi\emptyset+\phi A=\phi A$}

\noindent because $\phi\emptyset=0$;  so $\emptyset\in\Sigma$.

Thus $\Sigma$ is an algebra of sets.   If $E$, $F\in\Sigma$ and
$E\cap F=\emptyset$, then

\Centerline{$\phi(E\cup F)
=\phi((E\cup F)\cap E)+\phi((E\cup F)\setminus E)=\phi E+\phi F$.}
}%end of proof of 413B

\leader{413C}{Measures from inner \dvrocolon{measures}}\cmmnt{ I come
now to a construction corresponding to \Caratheodory's method of
defining measures from outer measures.

\medskip

\noindent}{\bf Theorem} Let $X$ be a set and
$\phi:X\to[0,\infty]$ an inner measure.   Set

\Centerline{$\Sigma=\{E:E\subseteq X,\,\phi(A\cap E)+\phi(A\setminus
E)=\phi A$ for every $A\subseteq X\}$.}

\noindent Then $(X,\Sigma,\phi\restr\Sigma)$ is a complete measure
space.

\proof{ (Compare 113C.)

\medskip

{\bf (a)} The first step is to note that if $A\subseteq B\subseteq X$
then

\Centerline{$\phi B\ge\phi A+\phi(B\setminus A)\ge\phi A$.}

\noindent Next, a subset $E$ of $X$
belongs to $\Sigma$ iff $\phi A\le\phi(A\cap E)+\phi(A\setminus E)$
whenever $A\subseteq X$ and $\mu A<\infty$.
\Prf\ Of course any element of $\Sigma$ satisfies the condition.   If
$E$ satisfies the condition and $A\subseteq X$, then

$$\eqalign{\phi A
&=\sup\{\phi B:B\subseteq A,\,\phi B<\infty\}\cr
&\le\sup\{\phi(B\cap E)+\phi(B\setminus E):B\subseteq A\}\cr
&=\phi(A\cap E)+\phi(A\setminus E)
\le\phi A,\cr}$$

\noindent so $E\in\Sigma$.\ \Qed

\medskip

{\bf (b)} By 413B, $\Sigma$ is an algebra of subsets of $X$.   Now
suppose
that $\langle E_n\rangle_{n\in\Bbb N}$ is a non-decreasing
sequence in $\Sigma$, with union $E$.   If $A\subseteq X$ and
$\phi A<\infty$, then

\Centerline{$\phi(A\setminus E)=\inf_{n\in\Bbb N}\phi(A\setminus E_n)
=\lim_{n\to\infty}\phi(A\setminus E_n)$}

\noindent because $\sequencen{A\setminus E_n}$ is non-increasing and
$\phi(A\setminus E_0)$ is finite;  so

\Centerline{$\phi(A\cap E)+\phi(A\setminus E)
\ge\lim_{n\to\infty}\phi(A\cap E_n)+\phi(A\setminus E_n)
=\phi A$.}

\noindent By (a), $E\in\Sigma$.   So $\Sigma$ is a $\sigma$-algebra.

\medskip

{\bf (c)} If $E$, $F\in\Sigma$ and $E\cap F=\emptyset$ then
$\phi(E\cup F)=\phi E+\phi F$, by 413B.   If $\sequencen{E_n}$
is a disjoint sequence in $\Sigma$ with union $E$, then

\Centerline{$\mu E\ge\mu(\bigcup_{i\le n}E_i)=\sum_{i=0}^n\mu E_i$}

\noindent for every $n$, so $\mu E\ge\sum_{i=0}^{\infty}\mu E_i$.
\Quer\ If $\mu E>\sum_{i=0}^{\infty}\mu E_i$, there is an $A\subseteq E$
such that $\sum_{i=0}^{\infty}\mu E_i<\phi A<\infty$.   But now, setting
$F_n=\bigcup_{i\le n}E_i$ for each $n$, we have
$\lim_{n\to\infty}\phi(A\setminus F_n)=0$, so that

\Centerline{$\phi A
=\lim_{n\to\infty}\phi(A\cap F_n)+\phi(A\setminus F_n)
=\sum_{i=0}^{\infty}\phi(A\cap E_i)<\phi A$,}

\noindent which is absurd.\ \BanG\    Thus
$\mu E=\sum_{i=0}^{\infty}\mu E_i$.   As $\sequencen{E_n}$ is arbitrary,
$\mu$ is a measure.

\medskip

{\bf (d)} Finally, suppose that $B\subseteq E\in\Sigma$ and $\mu E=0$.
Then for any $A\subseteq X$ we must have

\Centerline{$\phi(A\cap B)+\phi(A\setminus B)\ge\phi(A\setminus E)
=\phi(A\cap E)+\phi(A\setminus E)=\phi A$,}

\noindent so $B\in\Sigma$.   Thus $\mu$ is complete.
}%end of proof of 413C

\cmmnt{\medskip

\noindent{\bf Remark} For a simple example see 213Yd\formerly{2{}13Yc}.
}%end of comment

\leader{413D}{The inner measure defined by a measure} Let
$(X,\Sigma,\mu)$ be any measure space.
\cmmnt{ Just as} $\mu$\cmmnt{ has an associated outer
measure $\mu^*$ defined by the formula

\Centerline{$\mu^*A=\inf\{\mu E:A\subseteq E\in\Sigma\}$}

\noindent (132A-132B), it} gives rise to an inner measure $\mu_*$
defined by the formula

\Centerline{$\mu_*A=\sup\{\mu E:E\in\Sigma^f,\,E\subseteq A\}$,}

\noindent where I write $\Sigma^f$ for $\{E:E\in\Sigma,\,\mu E<\infty\}$.
\prooflet{\Prf\ $\mu_*\emptyset=\mu\emptyset=0$.   ($\alpha$) If
$A\cap B=\emptyset$, and $E\subseteq A$, $F\subseteq B$ belong to
$\Sigma^f$, then
$E\cup F\subseteq A\cup B$ also has finite measure, so

\Centerline{$\mu_*(A\cup B)\ge\mu(E\cup F)=\mu E+\mu F$;}

\noindent taking the supremum over $E$ and $F$,
$\mu_*(A\cup B)\ge\mu_*A+\mu_*B$.   ($\beta$) If $\sequencen{A_n}$ is a
non-increasing sequence of sets with intersection $A$ and
$\mu_*A_0<\infty$, then for each $n\in\Bbb N$ we can find an
$E_n\subseteq A_n$ such that $\mu E_n\ge\mu_*A_n-2^{-n}$.   In this case,

\Centerline{$\mu(\bigcup_{m\in\Bbb N}E_m)
=\sup_{n\in\Bbb N}\mu(\bigcup_{m\le n}E_m)\le\mu_*A_0<\infty$.}

\noindent Set

\Centerline{$E=\bigcap_{n\in\Bbb N}\bigcup_{m\ge n}E_m\subseteq A$.}

\noindent Then $E\in\Sigma^f$, so

\Centerline{$\mu_*A\ge\mu E\ge\limsup_{n\to\infty}\mu E_n
=\lim_{n\to\infty}\mu_*A_n\ge\mu_*A$.}

\noindent ($*$) If $A\subseteq X$ and $\mu_*A=\infty$ then

\Centerline{$\sup\{\mu_*B:B\subseteq A,\,\mu_*B<\infty\}
\ge\sup\{\mu E:E\in\Sigma^f,\,E\subseteq A\}=\infty$.  \Qed}
}%end of prooflet

\cmmnt{\medskip

\noindent{\bf Warning} Many authors use the formula

\Centerline{$\mu_*A=\sup\{\mu E:A\supseteq E\in\Sigma\}$.}

\noindent In `ordinary' cases, when $(X,\Sigma,\mu)$ is semi-finite,
this agrees with my usage (413Ed);  but for non-semi-finite spaces there
is a difference.   See 413Yg.
}%end of comment

\vleader{72pt}{413E}{}\cmmnt{ I note the following simple facts
concerning inner measures defined from measures.

\medskip

\noindent}{\bf Proposition} Let $(X,\Sigma,\mu)$ be a measure space.
Write $\Sigma^f$ for $\{E:E\in\Sigma,\,\mu E<\infty\}$.

(a) For every $A\subseteq X$ there is an $E\in\Sigma$ such that
$E\subseteq A$ and $\mu E=\mu_*A$.

(b) $\mu_*A\le\mu^*A$ for every $A\subseteq X$.

(c) If $E\in\Sigma$ and $A\subseteq X$, then
$\mu_*(E\cap A)+\mu^*(E\setminus A)\le\mu E$, with equality if either
(i) $\mu E<\infty$ or (ii) $\mu$ is semi-finite.

(d) \cmmnt{In particular,} $\mu_*E\le\mu E$ for every $E\in\Sigma$, with
equality if
either $\mu E<\infty$ or $\mu$ is semi-finite.

(e) If $\mu$ is inner regular with respect to $\Cal K$, then
$\mu_*A=\sup\{\mu K:K\in\Cal K\cap\Sigma^f,\,K\subseteq A\}$ for every
$A\subseteq X$.

(f) If $A\subseteq X$ is such that $\mu_*A=\mu^*A<\infty$, then $A$ is
measured by the completion of $\mu$.

(g) If $\hat\mu$, $\tilde\mu$ are the completion and c.l.d.\ version of
$\mu$, then $\hat\mu_*=\tilde\mu_*=\mu_*$.

(h) If $(Y,\Tau,\nu)$ is another measure space, and $f:X\to Y$ is an
\imp\ function, then

\Centerline{$\mu^*(f^{-1}[B])\le\nu^*B$,
\quad$\mu_*(f^{-1}[B])\ge\nu_*B$}

\noindent for every $B\subseteq Y$, and

\Centerline{$\nu^*(f[A])\ge\mu^*A$}

\noindent for every $A\subseteq X$.

(i) Suppose that $\mu$ is semi-finite.   If $A\subseteq E\in\Sigma$,
then $E$ is a measurable envelope of $A$ iff $\mu_*(E\setminus A)=0$.

\proof{{\bf (a)} There is a sequence $\sequencen{E_n}$ in $\Sigma^f$
such that $E_n\subseteq A$ for each $n$ and
$\lim_{n\to\infty}\mu E_n=\mu_*A$;
now set $E=\bigcup_{n\in\Bbb N}E_n$.

\medskip

{\bf (b)} If $E\subseteq A\subseteq F$ we must have $\mu E\le\mu
F$.

\medskip

{\bf (c)} If $F\subseteq E\cap A$ and $F\in\Sigma^f$, then

\Centerline{$\mu F+\mu^*(E\setminus A)\le\mu F+\mu(E\setminus F)=\mu
E$;}

\noindent taking the supremum over $F$, $\mu_*(E\cap A)+\mu^*(E\setminus
A)\le\mu E$.   If $\mu E<\infty$, then

$$\eqalign{\mu_*(E\cap A)
&=\sup\{\mu F:F\in\Sigma,\,F\subseteq E\cap A\}\cr
&=\mu E-\inf\{\mu(E\setminus F):F\in\Sigma,\,F\subseteq E\cap A\}\cr
&=\mu E-\inf\{\mu F:F\in\Sigma,\,E\setminus A\subseteq F\subseteq E\}
=\mu E-\mu^*(E\setminus A).\cr}$$

\noindent If $\mu$ is semi-finite, then

$$\eqalign{\mu_*(E\cap A)+\mu^*(E\setminus A)
&\ge\sup\{\mu_*(F\cap A)+\mu^*(F\setminus A):F\in\Sigma^f,\,F\subseteq
E\}\cr
&=\sup\{\mu F:F\in\Sigma^f,\,F\subseteq E\}
=\mu E.\cr}$$

\medskip

{\bf (d)} Take $A=E$ in (c).

\medskip

{\bf (e)}

$$\eqalign{\mu_*A
&=\sup\{\mu E:E\in\Sigma^f,\,E\subseteq A\}\cr
&=\sup\{\mu K:K\in\Cal K\cap\Sigma,\,\exists\,E\in\Sigma^f,\,K\subseteq
E\subseteq A\}\cr
&=\sup\{\mu K:K\in\Cal K\cap\Sigma^f,\,K\subseteq A\}.\cr}$$

\medskip

{\bf (f)} By (a) above and 132Aa, there are $E$, $F\in\Sigma$ such that
$E\subseteq A\subseteq F$ and

\Centerline{$\mu E=\mu_*A=\mu^*A=\mu F<\infty$;  }

\noindent now $\mu(F\setminus E)=0$, so $F\setminus A$ and $A$ are
measured by the completion of $\mu$.

\medskip

{\bf (g)} Write $\check\mu$ for either $\hat\mu$ or $\tilde\mu$, and
$\check\Sigma$ for its domain, and let $A\subseteq X$.   (i) If
$\gamma<\mu_*A$, there is an $E\in\Sigma$ such that $E\subseteq A$ and
$\gamma\le\mu E<\infty$;  now $\check\mu E=\mu E$ (212D, 213Fa), so
$\check\mu_*A\ge\gamma$.   As $\gamma$ is arbitrary,
$\mu_*A\le\check\mu_*A$.   (ii) If $\gamma<\check\mu_*A$, there is an
$E\in\check\Sigma$ such that $E\subseteq A$ and
$\gamma\le\check\mu E<\infty$.   Now there is an $F\in\Sigma$ such that
$F\subseteq E$ and $\mu F=\check\mu E$ (212C, 213Fc), so that
$\mu_*A\ge\gamma$.   As $\gamma$ is arbitrary, $\mu_*A\ge\check\mu_*A$.

\medskip

{\bf (h)} This is elementary;  all we have to note is that if $F$,
$F'\in\Tau$ and $F\subseteq B\subseteq F'$, then
$f^{-1}[F]\subseteq f^{-1}[B]\subseteq f^{-1}[F']$, so that

\Centerline{$\nu F=\mu f^{-1}[F]\le\mu_*f^{-1}[B]
\le\mu^*f^{-1}[B]\le\mu f^{-1}[F']=\nu F'$.}

\noindent Now, for $A\subseteq X$,

\Centerline{$\mu^*A\le\mu^*(f^{-1}[f[A]])\le\nu^*(f[A])$.}

\medskip

{\bf (i)}(i) If $E$ is a measurable envelope of $A$ and $F\in\Sigma$ is
included in $E\setminus A$, then

\Centerline{$\mu F=\mu(F\cap E)=\mu^*(F\cap A)=0$;}

\noindent as $F$ is arbitrary, $\mu_*(E\setminus A)=0$.   (ii) If $E$ is
not a measurable envelope of $A$, there is an $F\in\Sigma$ such that
$\mu^*(F\cap A)<\mu(F\cap E)$.   Let $G\in\Sigma$ be such that
$F\cap A\subseteq G$ and $\mu G=\mu^*(F\cap A)$.   Then
$\mu(F\cap E\setminus G)>0$;  because $\mu$ is semi-finite,
$\mu_*(E\setminus A)\ge\mu_*(F\cap E\setminus G)>0$.
}%end of proof of 413E

\leader{413F}{}\cmmnt{ The language of 413D makes it easy to express
some useful facts about complete locally determined measure spaces,
complementing 412J.

\medskip

\noindent}{\bf Lemma} Let $(X,\Sigma,\mu)$ be a complete locally
determined measure space and $\Cal K$ a family of subsets of $X$ such
that $\mu$ is inner regular with respect to $\Cal K$.
Then for $E\subseteq X$ the following are equiveridical:

\quad(i) $E\in\Sigma$;

\quad(ii) $E\cap K\in\Sigma$ whenever $K\in\Sigma\cap\Cal K$;

\quad(iii) $\mu^*(K\cap E)+\mu^*(K\setminus E)=\mu^*K$ for every
$K\in\Cal K$;

\quad(iv) $\mu_*(K\cap E)+\mu_*(K\setminus E)=\mu_*K$ for every
$K\in\Cal K$;

\quad(v) $\mu^*(E\cap K)=\mu_*(E\cap K)$ for every
$K\in\Cal K\cap\Sigma$;

\quad(vi) $\min(\mu^*(K\cap E),\mu^*(K\setminus E))<\mu K$ whenever
$K\in\Cal K\cap\Sigma$ and $0<\mu K<\infty$;

\quad(vii) $\max(\mu_*(K\cap E),\mu_*(K\setminus E))>0$ whenever
$K\in\Cal K\cap\Sigma$ and $\mu K>0$.

\proof{{\bf (a)} Assume (i).   Then of course
$E\cap K\in\Sigma$ for every $K\in\Sigma\cap\Cal K$, and (ii) is true.
For any $K\in\Cal K$ there is an $F\in\Sigma$ such that $F\supseteq K$
and $\mu F=\mu^*K$ (132Aa);  now

\Centerline{$\mu^*K\le\mu^*(K\cap E)+\mu^*(K\setminus E)
\le\mu(F\cap E)+\mu(F\setminus E)=\mu F=\mu^*K$,}

\noindent so (iii) is true.   Next, for any $K\in\Cal K$,

$$\eqalignno{\mu_*(K\cap E)+\mu_*(K\setminus E)
&\le\mu_*K
=\sup\{\mu F:F\in\Sigma^f,\,F\subseteq K\}\cr
\noalign{\noindent (writing $\Sigma^f$ for
$\{F:F\in\Sigma,\,\mu F<\infty\}$)}
&=\sup\{\mu(F\cap E)+\mu(F\setminus E):F\in\Sigma^f,\,F\subseteq K\}\cr
&\le\mu_*(K\cap E)+\mu_*(K\setminus E).\cr}$$

\noindent So (iv) is true.   If $K\in\Cal K\cap\Sigma$, then

\Centerline{$\mu_*(E\cap K)
=\sup\{\mu F:F\in\Sigma^f,\,F\subseteq E\cap K\}
=\mu(E\cap K)=\mu^*(E\cap K)$}

\noindent because $\mu$ is semi-finite.   So (v) is true.   Since
(iii)$\Rightarrow$(vi) and (iv)$\Rightarrow$(vii), we see that all the
conditions are satisfied.

\medskip

{\bf (b)} Now suppose that $E\notin\Sigma$;  I have to show that
(ii)-(vii) are all false.   Because $\mu$ is locally determined, there
is an $F\in\Sigma^f$ such that $E\cap F\notin\Sigma$.
Take measurable envelopes $H$, $H'$ of $F\cap E$ and $F\setminus E$
respectively (132Ee).   Then
$F\setminus H'\subseteq F\cap E\subseteq F\cap H$, so

\Centerline{$G=(F\cap H)\setminus(F\setminus H')=F\cap H\cap H'$}

\noindent cannot be negligible.
Take $K\in\Cal K\cap\Sigma$ such that $K\subseteq G$ and $\mu K>0$.   As
$G\subseteq F$, $\mu K<\infty$.   Now

\Centerline{$\mu^*(K\cap E)=\mu^*(K\cap F\cap E)=\mu(K\cap H)=\mu K$,}

\Centerline{$\mu^*(K\setminus E)=\mu^*(K\cap F\setminus E)
=\mu(K\cap H')=\mu K$.}

\noindent But this means that

\Centerline{$\mu_*(K\cap E)=\mu K-\mu^*(K\setminus E)=0$,
\quad$\mu_*(K\setminus E)=\mu K-\mu^*(K\cap E)=0$}

\noindent by 413Ec.   Now we see that this $K$ witnesses that (ii)-(vii)
are all false.
}%end of proof of 413F

\leader{413G}{}\cmmnt{ The ideas of 413F can be used to give criteria
for measurability of real-valued functions.   I spell out one which is
particularly useful.

\medskip

\noindent}{\bf Lemma} Let $(X,\Sigma,\mu)$ be a complete locally
determined measure space and suppose that $\mu$ is inner regular with
respect to $\Cal K\subseteq\Sigma$.   Suppose that $f:X\to\Bbb R$ is a
function, and for $\alpha\in\Bbb R$ set
$E_{\alpha}=\{x:f(x)\le\alpha\}$, $F_{\alpha}=\{x:f(x)\ge\beta\}$.
Then $f$ is $\Sigma$-measurable iff

\Centerline{$\min(\mu^*(E_{\alpha}\cap K),\mu^*(F_{\beta}\cap K))
<\mu K$}

\noindent whenever $K\in\Cal K$, $0<\mu K<\infty$ and $\alpha<\beta$.

\proof{{\bf (a)} If $f$ is measurable, then

\Centerline{$\mu^*(E_{\alpha}\cap K)+\mu^*(F_{\beta}\cap K)
=\mu(E_{\alpha}\cap K)+\mu(F_{\beta}\cap K)
\le\mu K$}

\noindent whenever $K\in\Sigma$ and $\alpha<\beta$, so if
$0<\mu K<\infty$ then we must have
$\min(\mu^*(E_{\alpha}\cap K),\mu^*(F_{\beta}\cap K))<\mu K$.

\medskip

{\bf (b)} If $f$ is not measurable, then there is some $\alpha\in\Bbb R$
such that $E_{\alpha}$ is not measurable.   413F(vi) tells us that
there is a $K\in\Cal K$ such that $0<\mu K<\infty$ and
$\mu^*(E_{\alpha}\cap K)=\mu^*(K\setminus E_{\alpha})=\mu K$.   Note
that $K$ is a measurable envelope of $K\cap E_{\alpha}$ (132Eb).   Now
$\sequencen{K\cap F_{\alpha+2^{-n}}}$ is a non-decreasing sequence with
union $K\setminus E_{\alpha}$, so there is some $\beta>\alpha$ such that
$K\cap F_{\beta}$ is not negligible.   Let $H\subseteq K$ be a
measurable envelope of $K\cap F_{\beta}$, and $K'\in\Cal K$ such that
$K'\subseteq H$ and $\mu K'>0$;  then

\Centerline{$\mu^*(K'\cap E_{\alpha})
=\mu^*(K'\cap K\cap E_{\alpha})=\mu(K'\cap K)=\mu K'$,}

\Centerline{$\mu^*(K'\cap F_{\beta})
=\mu^*(K'\cap H\cap F_{\beta})=\mu(K'\cap H)=\mu K'$,}

\noindent so $K'$, $\alpha$ and $\beta$ witness that the condition is
not satisfied.
}%end of proof of 413G
%compare 213Ye

\leader{413H}{}\cmmnt{ Inner measure constructions based on 413C are
important
because they offer an efficient way of setting up measures which are
inner regular with respect to given families of sets.   Two of the
fundamental
results are 413I and 413J.   I proceed by means of a lemma on finitely
additive functionals.

\medskip

\noindent}{\bf Lemma} Let $X$ be a set and $\Cal K$ a family of subsets
of $X$ such that

\inset{$\emptyset\in\Cal K$,}

\inset{($\dagger$) $K\cup
K'\in\Cal K$ whenever $K$, $K'\in\Cal K$ are disjoint,}

\inset{($\ddagger$) $K\cap K'\in\Cal K$ for all $K$, $K'\in\Cal K$.}

\noindent Let $\phi_0:\Cal K\to\coint{0,\infty}$ be a
functional such that

\inset{($\alpha$) $\phi_0 K
=\phi_0 L+\sup\{\phi_0 K':K'\in\Cal K,\,K'\subseteq K\setminus L\}$
whenever $K$, $L\in\Cal K$ and $L\subseteq K$.}

\noindent Set

\Centerline{$\phi A=\sup\{\phi_0K:K\in\Cal K,\,K\subseteq A\}$ for
$A\subseteq X$,}

\Centerline{$\Sigma=\{E:E\subseteq X,\,\phi A=\phi(A\cap
E)+\phi(A\setminus
E)$ for every $A\subseteq X\}$.}

\noindent Then $\Sigma$ is an algebra of subsets of $X$, including
$\Cal K$,
and $\phi\restr\Sigma:\Sigma\to[0,\infty]$ is an additive functional
extending $\phi_0$.

\proof{{\bf (a)} To see that $\Sigma$ is an algebra of subsets and
$\phi\restr\Sigma$ is additive, all we need to know is that
$\phi\emptyset=0$ (413B);  and this is because, applying hypothesis
($\alpha$) with $K=L=\emptyset$,
$\phi_0\emptyset=\phi_0\emptyset+\phi_0\emptyset$, so
$\phi_0\emptyset=0$.
($\alpha$) also assures us that $\phi_0L\le\phi_0K$ whenever $K$,
$L\in\Cal
K$ and $L\subseteq K$, so $\phi K=\phi_0K$ for every $K\in\Cal K$.

\medskip

{\bf (b)} To check that $\Cal K\subseteq\Sigma$, we have a little more
work
to do.   First, observe that ($\dagger$) and ($\alpha$) together tell us
that $\phi_0(K\cup K')=\phi_0K+\phi_0K'$ for all disjoint $K$,
$K'\in\Cal
K$.   So if $A$, $B\subseteq X$ and $A\cap B=\emptyset$ then

$$\eqalign{\phi A+\phi B
&=\sup_{K\in\Cal K,K\subseteq A}
  \phi_0K+\sup_{L\in\Cal K,L\subseteq A}\phi_0L\cr
&=\sup_{K,L\in\Cal K,K\subseteq A,L\subseteq B}\phi_0(K\cup L)
\le\phi(A\cup B).\cr}$$

\medskip

{\bf (c)} $\Cal K\subseteq\Sigma$.   \Prf\ Take $K\in\Cal K$ and
$A\subseteq X$.   If $L\in\Cal K$ and $L\subseteq A$, then

\Centerline{$\phi_0L
=\phi_0(K\cap L)
  +\sup\{\phi_0L':L'\in\Cal K,\,L'\subseteq L\setminus K\}
\le\phi(A\cap K)+\phi(A\setminus K)$.}

\noindent (Note the use of the hypothesis ($\ddagger$).)   As $L$ is
arbitrary,
$\phi A\le\phi(A\cap K)+\phi(A\setminus K)$.   We already know that
$\phi(A\cap K)+\phi(A\setminus K)\le\phi A$;
as $A$ is arbitrary, $K\in\Sigma$.\ \Qed

This completes the proof.
}%end of proof of 413H

\vleader{60pt}{413I}{Theorem}\cmmnt{ ({\smc Tops{\o}e 70a})}
%Theorem 1 p 198
Let $X$ be a set and $\Cal K$ a family of subsets of $X$ such that

\inset{$\emptyset\in\Cal K$,}

\inset{($\dagger$) $K\cup
K'\in\Cal K$ whenever $K$, $K'\in\Cal K$ are disjoint,}

\inset{($\ddagger$) $\bigcap_{n\in\Bbb N}K_n\in\Cal K$ whenever
$\sequencen{K_n}$ is a sequence in $\Cal K$.}

\noindent Let $\phi_0:\Cal K\to\coint{0,\infty}$ be a
functional such that

\inset{($\alpha$) $\phi_0 K
=\phi_0 L+\sup\{\phi_0 K':K'\in\Cal K,\,K'\subseteq K\setminus L\}$
whenever $K$, $L\in\Cal K$ and $L\subseteq K$,}

\inset{($\beta$) $\inf_{n\in\Bbb N}\phi_0K_n=0$ whenever
$\sequencen{K_n}$ is
a non-increasing sequence in $\Cal K$ with empty intersection.}

\noindent Then there is a unique complete locally determined measure
$\mu$ on $X$ extending $\phi_0$ and inner regular with respect to
$\Cal K$.

\proof{{\bf (a)} Set

\Centerline{$\phi A=\sup\{\phi_0K:K\in\Cal K,\,K\subseteq A\}$ for
$A\subseteq X$,}

\Centerline{$\Sigma=\{E:E\subseteq X,\,\phi A=\phi(A\cap
E)+\phi(A\setminus
E)$ for every $A\subseteq X\}$.}

\noindent Then 413H tells us that $\Sigma$ is an algebra of subsets of
$X$,
including $\Cal K$, and $\mu=\phi\restr\Sigma$ is an additive functional
extending $\phi_0$.

\medskip

{\bf (b)} Now $\mu(\bigcap_{n\in\Bbb N}K_n)=\inf_{n\in\Bbb
N}\mu K_n$ whenever
$\sequencen{K_n}$ is a non-increasing sequence in $\Cal K$.   \Prf\ Set
$L=\bigcap_{n\in\Bbb N}K_n$.   Of course $\mu L\le\inf_{n\in\Bbb
N}\mu K_n$.   For the reverse inequality, take $\epsilon>0$.   Then
($\alpha$) tells us that there is a $K'\in\Cal K$ such that $K'\subseteq
K_0\setminus L$ and
$\mu K_0\le\mu L+\mu K'+\epsilon$.   Since $\sequencen{K_n\cap K'}$
is a non-increasing sequence in $\Cal K$ with empty intersection,
($\beta$)
tells us that there is an
$n\in\Bbb N$ such that $\mu(K_n\cap K')\le\epsilon$.   Now

$$\eqalign{\mu K_0-\mu L
&=\mu(K_0\setminus L)
=\mu(K_0\setminus(K'\cup L))+\mu K'\cr
&\le\epsilon+\mu(K_n\cap K')+\mu(K'\setminus K_n)
\le 2\epsilon+\mu(K_0\setminus K_n)
=2\epsilon+\mu K_0-\mu K_n.\cr}$$

\noindent (These calculations depend, of course, on the additivity of
$\mu$
and the finiteness of $\mu K_0$.)   So
$\mu L\ge\mu K_n-2\epsilon$.   As $\epsilon$ is arbitrary, $\mu
L=\inf_{n\in\Bbb N}\mu K_n$.\ \Qed

\medskip

{\bf (c)} If $\sequencen{A_n}$ is a non-increasing sequence of
subsets of $X$, with intersection $A$, and $\phi A_0<\infty$, then $\phi
A=\inf_{n\in\Bbb N}\phi A_n$.   \Prf\ Of course $\phi A\le\phi A_n$ for
every $n$.   Given $\epsilon>0$, then for each $n\in\Bbb N$ choose
$K_n\in\Cal K$ such that $K_n\subseteq A_n$ and
$\phi_0K_n\ge\phi A_n-2^{-n}\epsilon$ (this is where I use the
hypothesis
that $\phi A_0$ is finite);  set $L_n=\bigcap_{i\le n}K_i$ for each $n$,
and $L=\bigcap_{n\in\Bbb N}L_n$.   Then we have

$$\eqalign{\phi A_{n+1}-\mu L_{n+1}
&=\phi A_{n+1}-\mu(K_{n+1}\cap L_n)\cr
&=\phi A_{n+1}-\mu K_{n+1}-\mu L_n+\mu(K_{n+1}\cup L_n)\cr
&\le 2^{-n-1}\epsilon-\mu L_n+\phi A_n\cr}$$

\noindent because $K_{n+1}\subseteq A_{n+1}\subseteq A_n$ and
$L_n\subseteq
K_n\subseteq A_n$.   Inducing on $n$, we see that
$\mu L_n\ge\phi A_n-2\epsilon+2^{-n}\epsilon$ for every $n$.   So

\Centerline{$\phi A\ge\mu L=\inf_{n\in\Bbb N}\mu L_n\ge\inf_{n\in\Bbb
N}\phi A_n-2\epsilon$,}

\noindent using (b) above for the middle equality.   As $\epsilon$ is
arbitrary, $\phi A=\inf_{n\in\Bbb N}\phi A_n$.\ \Qed

\medskip

{\bf (d)} It follows that $\phi$ is an inner measure.
\Prf\ The arguments of parts (a) and (b) of the proof of 413H tell us
that
$\phi\emptyset=0$ and $\phi(A\cup B)\le\phi A+\phi B$ whenever $A$,
$B\subseteq X$ are disjoint.   We have just seen that
$\phi(\bigcap_{n\in\Bbb N}A_n)=\inf_{n\in\Bbb N}\phi A_n$ whenever
$\sequencen{A_n}$ is a non-increasing sequence of sets and $\phi
A_0<\infty$.
Finally, $\phi K=\phi_0K$ is finite for every $K\in\Cal
K$, so $\phi A=\sup\{\phi B:B\subseteq A,\,\phi B<\infty\}$ for every
$A\subseteq X$.   Putting these together, $\phi$ is an inner measure.\
\Qed

\medskip

{\bf (e)} So 413C tells us that $\mu$ is a complete measure, and of
course it is inner regular with respect to $\Cal K$, by the definition
of $\phi$.   It is semi-finite because $\mu K=\phi_0K$ is finite for
every $K\in\Cal K$.   Now suppose that
$E\subseteq X$ and that $E\cap F\in \Sigma$ whenever $\mu F<\infty$.
Take any $A\subseteq X$.   If $L\in\Cal K$ and $L\subseteq A$, we have
$L\in\Sigma$ and $\mu L<\infty$, so

\Centerline{$\phi_0L=\mu L=\mu(L\cap E)+\mu(L\setminus E)
=\phi(L\cap E)+\phi(L\setminus E)\le\phi(A\cap E)+\phi(A\setminus E)$;}

\noindent taking the supremum over $L$,
$\phi A\le\phi(A\cap E)+\phi(A\setminus E)$.   As $A$ is arbitrary,
$E\in\Sigma$;  as $E$ is arbitrary, $\mu$ is locally determined.

\medskip

{\bf (f)} Finally, $\mu$ is unique by 412L.
}%end of proof of 413I

\vleader{60pt}{413J}{Theorem} Let $X$ be a set and $\Cal K$ a family of subsets
of $X$ such that

\inset{$\emptyset\in\Cal K$,}

\inset{($\dagger$) $K\cup
K'\in\Cal K$ whenever $K$, $K'\in\Cal K$ are disjoint,}

\inset{($\ddagger$) $K\cap K'\in\Cal K$ whenever $K$, $K'\in\Cal K$.}

\noindent Let $\phi_0:\Cal K\to\coint{0,\infty}$ be a
functional such that

\inset{($\alpha$) $\phi_0 K
  =\phi_0 L+\sup\{\phi_0 K':K'\in\Cal K,\,K'\subseteq K\setminus L\}$
whenever $K$, $L\in\Cal K$ and $L\subseteq K$,}

\inset{($\beta$) $\inf_{n\in\Bbb N}\phi_0K_n=0$ whenever
$\sequencen{K_n}$ is
a non-increasing sequence in $\Cal K$ with empty intersection.}

\noindent Then there is a unique complete locally determined measure
$\mu$ on $X$ extending $\phi_0$ and inner regular with respect to $\Cal
K_{\delta}$, the family of sets expressible as intersections of
sequences in $\Cal K$.

\proof{{\bf (a)} Set

\Centerline{$\psi A=\sup\{\phi_0K:K\in\Cal K,\,K\subseteq A\}$ for
$A\subseteq X$,}

\Centerline{$\Tau
=\{E:E\subseteq X,\,\psi A=\psi (A\cap E)+\psi(A\setminus E)$
for every $A\subseteq X\}$.}

\noindent Then 413H tells us that $\Tau$ is an algebra of subsets of $X$,
including $\Cal K$, and $\nu =\psi \restrp\Tau$ is an additive functional
extending $\phi_0$.

\medskip

{\bf (b)} Write $\Tau^f$ for $\{E:E\in\Tau,\,\nu E<\infty\}$.
If $\sequencen{E_n}$ is a non-increasing sequence in $\Tau^f$ with empty
intersection, $\lim_{n\to\infty}\nu E_n=0$.   \Prf\
Given $\epsilon>0$, we can choose a sequence $\sequencen{K_n}$ in
$\Cal K$ such that $K_n\subseteq E_n$ and

\Centerline{$\nu K_n=\phi_0K_n\ge\nu E_n-2^{-n}\epsilon$}

\noindent for each $n$.   Set $L_n=\bigcap_{i\le n}K_i$ for each $n$;
then

\Centerline{$\lim_{n\to\infty}\nu L_n=\lim_{n\to\infty}\phi_0L_n=0$}

\noindent by hypothesis ($\beta$).   But also, for each $n$,

\Centerline{$\nu E_n
\le\nu L_n+\sum_{i=0}^n\nu(E_i\setminus K_i)
\le\nu L_n+2\epsilon$,}

\noindent because $\nu$ is additive and non-negative and $E_n\subseteq
L_n\cup\bigcup_{i\le n}(E_i\setminus K_i)$.
So $\limsup_{n\to\infty}\nu E_n\le 2\epsilon$;  as $\epsilon$ is
arbitrary, $\lim_{n\to\infty}\nu E_n=0$.\ \Qed

\medskip

{\bf (c)} Write $\Tau^f_{\delta}$ for the family of sets expressible as
intersections of sequences in $\Tau^f$, and for $H\in\Tau^f_{\delta}$
set $\phi_1H=\inf\{\nu E:H\subseteq E\in\Tau\}$.
Note that because $E\cap F\in\Tau^f$ whenever $E$, $F\in\Tau^f$, every
member of $\Tau^f_{\delta}$ can be expressed as the intersection of a
non-increasing sequence in $\Tau^f$.

\medskip

\quad{\bf (i)} If $\sequencen{E_n}$ is a non-increasing sequence in
$\Tau^f$ with intersection $H\in\Tau^f_{\delta}$, $\phi_1
H=\lim_{n\to\infty}\nu E_n$.   \Prf\ Of course

\Centerline{$\phi_1 H\le\inf_{n\in\Bbb N}\nu E_n
=\lim_{n\to\infty}\nu E_n$.}

\noindent On the other hand, if $H\subseteq E\in\Tau$, then
$\sequencen{E_n\setminus E}$ is a non-increasing sequence in $\Tau^f$
with empty intersection, and

\Centerline{$\nu E\ge\lim_{n\to\infty}\nu(E_n\cap E)
=\lim_{n\to\infty}\nu E_n-\lim_{n\to\infty}\nu(E_n\setminus E)
=\lim_{n\to\infty}\nu E_n$}

\noindent by (b) above.   As $E$ is arbitrary,
$\phi_1(\bigcap_{n\in\Bbb N}E_n)=\lim_{n\to\infty}\nu E_n$.\ \Qed

\medskip

\quad{\bf (ii)} Because $\Cal K\subseteq\Tau^f$,
$\Cal K_{\delta}\subseteq\Tau^f_{\delta}$.   Now for any
$H\in\Tau^f_{\delta}$,
$\phi_1H=\sup\{\phi_1L:L\in\Cal K_{\delta},\,L\subseteq H\}$.   \Prf\
Express $H$ as $\bigcap_{n\in\Bbb N}E_n$ where $\sequencen{E_n}$ is a
non-increasing sequence in $\Tau^f$.   Given $\epsilon>0$, we can choose
a sequence $\sequencen{K_n}$ in $\Cal K$ such that $K_n\subseteq E_n$ and
$\nu K_n\ge\nu E_n-2^{-n}\epsilon$ for each $n$.   Setting
$L_n=\bigcap_{i\le n}K_i$ for each $n$ and $L=\bigcap_{n\in\Bbb N}L_n$,
we have $L\in\Cal K_{\delta}$, $L\subseteq H$ and

\Centerline{$\phi_1H=\lim_{n\to\infty}\nu E_n
\le\lim_{n\to\infty}(\nu L_n+\sum_{i=0}^n\nu(E_i\setminus K_i))
\le\phi_1L+2\epsilon$.}

\noindent As $\epsilon$ is arbitrary, this gives the result.\ \Qed

\medskip

{\bf (d)} We find that $\Tau^f_{\delta}$ and $\phi_1$ satisfy the
conditions of
413I.   \Prf\ Of course $\emptyset\in\Tau^f_{\delta}$.   If $G$,
$H\in\Tau^f_{\delta}$ and $G\cap H=\emptyset$, express them as
$\bigcap_{n\in\Bbb N}E_n$, $\bigcap_{n\in\Bbb N}F_n$ where
$\sequencen{E_n}$, $\sequencen{F_n}$ are non-increasing sequences in
$\Tau^f$.   Then

\Centerline{$G\cup H=\bigcap_{n\in\Bbb N}E_n\cup F_n$}

\noindent belongs to $\Tau^f_{\delta}$, and

$$\eqalignno{\phi_1(G\cup H)
&=\lim_{n\to\infty}\nu(E_n\cup F_n)
=\lim_{n\to\infty}\nu E_n+\nu F_n-\nu(E_n\cap F_n)\cr
&=\lim_{n\to\infty}\nu E_n+\nu F_n\cr
\noalign{\noindent (by (b))}
&=\phi_1G+\phi_1H.\cr}$$

The definition of $\Tau^f_{\delta}$ as the set of intersections of
sequences
in $\Tau^f$ ensures that the intersection of any sequence in
$\Tau^f_{\delta}$ will belong to $\Tau^f_{\delta}$.

Now suppose that $G$, $H\in\Tau^f_{\delta}$ and that $G\subseteq H$.
Express them as intersections $\bigcap_{n\in\Bbb N}E_n$,
$\bigcap_{n\in\Bbb N}F_n$ of non-increasing sequences in $\Tau^f$, so that
$\phi_1G=\lim_{n\to\infty}\nu E_n$ and $\phi_1H=\lim_{n\to\infty}\nu F_n$.
For each $n$, set $H_n=\bigcap_{m\in\Bbb N}F_m\setminus E_n$, so that
$H_n\in\Tau^f_{\delta}$, $H_n\subseteq H\setminus G$, and

$$\eqalign{\phi_1H_n
&=\lim_{m\to\infty}\nu(F_m\setminus E_n)
=\lim_{m\to\infty}\nu F_m-\nu(F_m\cap E_n)\cr
&\ge\lim_{m\to\infty}\nu F_m-\nu E_n
=\phi_1H-\nu E_n.\cr}$$

\noindent Accordingly

\Centerline{$\sup\{\phi_1G':G'\in\Tau^f_{\delta},\,G'\subseteq
H\setminus
G\}\ge\sup_{n\in\Bbb N}\phi_1H-\nu E_n=\phi_1H-\phi_1G$.}

\noindent On the other hand, if $G'\in\Tau^f_{\delta}$ and $G'\subseteq
H\setminus G$, then

\Centerline{$\phi_1G+\phi_1G'=\phi_1(G\cup G')\le\phi_1H$}

\noindent because of course $\phi_1$ is non-decreasing, as well as being
additive on disjoint sets.   So

\Centerline{$\sup\{\phi_1G':G'\in\Tau^f_{\delta},\,G'\subseteq
H\setminus G\}=\phi_1H-\phi_1G$}

\noindent as required by condition ($\alpha$) of 413I.
Finally, suppose that $\sequencen{H_n}$ is a non-increasing sequence in
$\Tau^f_{\delta}$ with empty intersection.   For each $n\in\Bbb N$, let
$\sequence{i}{E_{ni}}$ be a non-increasing sequence in $\Tau^f$ with
intersection $H_n$, and set $F_m=\bigcap_{n\le m}E_{nn}$ for each $m$.
Then $\sequence{m}{F_m}$ is a non-increasing sequence in $\Tau^f$ with
empty intersection, while $H_m\subseteq F_m$ for each $m$, so

\Centerline{$\lim_{m\to\infty}\phi_1H_m
\le\lim_{m\to\infty}\nu F_m=0$.}

\noindent Thus condition 413I($\beta$) is satisfied, and we have the
full
list.\ \Qed

\medskip

{\bf (e)} By 413I, we have a complete locally determined measure $\mu$,
extending $\phi_1$, and inner regular with respect to $\Tau^f_{\delta}$.
Since $\phi_1K=\nu K=\phi_0K$ for $K\in\Cal K$, $\mu$ extends $\phi_0$.
If $G$ belongs to the domain of $\mu$, and $\gamma<\mu G$, there is an
$H\in\Tau^f_{\delta}$ such that $H\subseteq G$ and
$\gamma<\mu H=\phi_1H$;
by (c-ii), there is an $L\in\Cal K_{\delta}$ such that $L\subseteq H$
and $\gamma\le\phi_1L=\mu L$.   Thus $\mu$ is inner regular with respect
to $\Cal K_{\delta}$.
To see that $\mu$ is unique, observe that if $\mu'$ is any other measure
with these properties, and $L\in\Cal K_{\delta}$, then $L$ is
expressible as
$\bigcap_{n\in\Bbb N}K_n$ where $\sequencen{K_n}$ is a sequence in
$\Cal K$.   Now

\Centerline{$\mu L
=\lim_{n\to\infty}\mu(\bigcap_{i\le n}K_i)
=\lim_{n\to\infty}\phi_0(\bigcap_{i\le n}K_i)
=\mu'L$.}

\noindent So $\mu$ and $\mu'$ must agree on $\Cal K_{\delta}$, and by
412L they are identical.
}%end of proof of 413J

\leader{413K}{Corollary} (a) Let $X$ be a set, $\Sigma$ a subring of
$\Cal PX$, and $\nu:\Sigma\to\coint{0,\infty}$
a non-negative finitely additive functional such
that $\lim_{n\to\infty}\nu E_n=0$ whenever $\sequencen{E_n}$ is a
non-increasing sequence in $\Sigma$ with empty intersection.   Then
$\nu$ has a unique extension to a complete locally determined measure on
$X$ which is inner regular with respect to the family $\Sigma_{\delta}$
of intersections of sequences in $\Sigma$.

(b) Let $X$ be a set, $\Sigma$ a subalgebra of $\Cal PX$, and
$\nu:\Sigma\to\coint{0,\infty}$ a non-negative finitely additive functional such
that $\lim_{n\to\infty}\nu E_n=0$ whenever $\sequencen{E_n}$ is a
non-increasing sequence in $\Sigma$ with empty intersection.   Then
$\nu$ has a unique extension to a measure defined on the
$\sigma$-algebra of subsets of $X$ generated by $\Sigma$.

\proof{{\bf (a)} Take $\Sigma$, $\nu$ in place of $\Cal K$, $\phi_0$ in
413J.

\medskip

{\bf (b)} Let $\nu_1$ be the complete extension as in (a), and let
$\nu_1'$ be the restriction of $\nu_1$ to the $\sigma$-algebra $\Sigma'$
generated by $\Sigma$;  this is the extension required here.   To see
that $\nu_1'$ is unique, use the Monotone Class Theorem (136C).
}%end of proof of 413K

\cmmnt{\medskip

\noindent{\bf Remark} These are versions of the {\bf Hahn extension
theorem}.   You will sometimes see (b) above stated as `an
additive functional on an algebra of sets extends to a measure iff it is
countably additive'.   But this formulation depends on a different
interpretation
of the phrase `countably additive' from the one used in this book;  see
the note after the definition in 326I\formerly{3{}26E}.}

\leader{413L}{}\cmmnt{ It will be useful to have a definition extending
an idea in \S342.

\medskip

\noindent}{\bf Definition} A {\bf countably compact class}\cmmnt{ (or
{\bf semicompact paving})} is a family
$\Cal K$ of sets such that $\bigcap_{n\in\Bbb N}K_n\ne\emptyset$
whenever $\sequencen{K_n}$ is a sequence in $\Cal K$ such that
$\bigcap_{i\le n}K_i\ne\emptyset$ for every $n\in\Bbb N$.

\leader{413M}{Corollary} Let $X$ be a set and $\Cal K$ a countably
compact class of subsets of $X$ such that

\inset{$\emptyset\in\Cal K$,}

\inset{($\dagger$) $K\cup
K'\in\Cal K$ whenever $K$, $K'\in\Cal K$ are disjoint,}

\inset{($\ddagger$) $K\cap K'\in\Cal K$ whenever $K$, $K'\in\Cal K$.}

\noindent Let $\phi_0:\Cal K\to\coint{0,\infty}$ be a
functional such that

\inset{($\alpha$) $\phi_0 K
  =\phi_0 L+\sup\{\phi_0 K':K'\in\Cal K,\,K'\subseteq
K\setminus L\}$ whenever $K$, $L\in\Cal K$ and $L\subseteq K$.}

\noindent Then there is a unique complete locally determined measure
$\mu$ on $X$ extending $\phi_0$ and inner regular with respect to
$\Cal K_{\delta}$, the family of sets expressible as intersections of
sequences in $\Cal K$.

\proof{ The point is that the hypothesis ($\beta$) of 413J is
necessarily satisfied:  if $\sequencen{K_n}$ is
a non-increasing sequence in $\Cal K$ with empty intersection, then,
because $\Cal K$ is countably compact, there must be some $n$ such that
$K_n=\emptyset$.   Since hypothesis ($\alpha$) here is already enough to
ensure that $\phi_0\emptyset=0$ and $\phi_0 K\ge 0$ for every $K\in\Cal K$,
we must have $\inf_{n\in\Bbb N}\phi_0K_n=0$.   So we apply 413J to get
the result.
}%end of proof of 413M

\leader{413N}{}\cmmnt{ I now turn to constructions of a different
kind, being extension theorems in which the extension is not uniquely
defined.   Again I start with a theorem on finitely additive
functionals.

\wheader{413N}{4}{2}{2}{60pt}

\noindent}{\bf Theorem} Let $X$ be a set, $\Tau_0$ a subring of
$\Cal PX$,
and $\nu_0:\Tau_0\to\coint{0,\infty}$ a finitely additive functional.
Suppose that $\Cal K\subseteq\Cal PX$ is a family of sets such that

\inset{($\dagger$) $K\cup K'\in\Cal K$ whenever $K$, $K'\in\Cal K$ are disjoint,}

\inset{($\ddagger$) $K\cap K'\in\Cal K$ for all $K$, $K'\in\Cal K$,}

\inset{every member of $\Cal K$ is included in some
member of $\Tau_0$,}

\noindent and $\nu_0$ is inner regular with respect to $\Cal K$ in the
sense that

\inset{($\alpha$) $\nu_0E=\sup\{\nu_0K:K\in\Cal K\cap\Tau_0,\,K\subseteq
E\}$ for every $E\in\Tau_0$.}

\noindent Then $\nu_0$ has an extension to a
non-negative finitely additive functional $\nu_1$, defined on a subring
$\Tau_1$ of $\Cal PX$ including $\Tau_0\cup\Cal K$, inner regular with
respect to $\Cal K$, and such that whenever $E\in\Tau_1$ and $\epsilon>0$
there is an $E_0\in\Tau_0$ such that $\nu_1(E\symmdiff E_0)\le\epsilon$.

\proof{{\bf (a)} Let $P$ be the set of all non-negative additive
real-valued
functionals $\nu$, defined on subrings of $\Cal PX$, inner regular with
respect to $\Cal K$, and such that

\inset{($*$) whenever $E\in\dom\nu$ and $\epsilon>0$ there is an
$E_0\in\Tau_0$
such that $\nu(E\symmdiff E_0)\le\epsilon$.}

\noindent Order $P$ by extension of functions, so that $P$ is a
partially
ordered set.

\medskip

{\bf (b)} It will be convenient to borrow some notation from the theory
of countably additive functionals.   If $\Tau$ is a subring of $\Cal PX$
and $\nu:\Tau\to\coint{0,\infty}$ is a non-negative additive functional,
set

\Centerline{$\nu^*A=\inf\{\nu E:A\subseteq E\in\Tau\}$,
\quad$\nu_*A=\sup\{\nu E:A\supseteq E\in\Tau\}$}

\noindent for every $A\subseteq X$ (interpreting $\inf\emptyset$ as
$\infty$ if necessary).   Now if $A\subseteq X$ and $E$, $F\in\Tau$ are
disjoint,

\Centerline{$\nu^*(A\cap(E\cup F))=\nu^*(A\cap E)+\nu^*(A\cap F)$,}

\Centerline{$\nu_*(A\cap(E\cup F))=\nu_*(A\cap E)+\nu_*(A\cap F)$.}

$$\eqalignno{\text{\Prf\ }\nu^*(A\cap(E\cup F))
&=\inf\{\nu G:G\in\Tau,\,A\cap(E\cup F)\subseteq G\}\cr
&=\inf\{\nu G:G\in\Tau,\,A\cap(E\cup F)\subseteq G\subseteq E\cup F\}\cr
&=\inf\{\nu(G\cap E)+\nu(G\cap F):G\in\Tau,\,A\cap(E\cup F)\subseteq
G\subseteq E\cup F\}\cr
&=\inf\{\nu G_1+\nu G_2:G_1,\,G_2\in\Tau,\,A\cap E\subseteq G_1\subseteq
E,\,A\cap F\subseteq G_2\subseteq F\}\cr
&=\inf\{\nu G_1:G_1\in\Tau,\,A\cap E\subseteq G_1\subseteq E\}\cr
 &\qquad\qquad\qquad
  +\inf\{\nu G_2:G_2\in\Tau,\,A\cap F\subseteq G_2\subseteq F\}\cr
&=\nu^*(E\cap A)+\nu^*(F\cap A),\cr
\noalign{\vskip 12pt}
\nu_*(A\cap(E\cup F))
&=\sup\{\nu G:G\in\Tau,\,A\cap(E\cup F)\supseteq G\}\cr
&=\sup\{\nu(G\cap E)+\nu(G\cap F):G\in\Tau,\,A\cap(E\cup F)\supseteq
G\}\cr
&=\sup\{\nu G_1+\nu G_2:G_1,\,G_2\in\Tau,\,A\cap E\supseteq G_1,\,A\cap
F\supseteq G_2\}\cr
&=\sup\{\nu G_1:G_1\in\Tau,\,A\cap E\supseteq G_1\}\cr
  &\qquad\qquad\qquad+\sup\{\nu G_2:G_2\in\Tau,\,A\cap F\supseteq
G_2\}\cr
&=\nu_*(E\cap A)+\nu_*(F\cap A).   \text{ \Qed}\cr}$$

\medskip

{\bf (c)} The key to the proof is the following fact:  if $\nu\in P$ and
$M\in\Cal K$, there is a $\nuprime\in P$ such that $\nuprime$ extends $\nu$ and
$M\in\dom\nuprime$.   \Prf\ Set $\Tau=\dom\nu$, $\Tau'=\{(E\cap
M)\cup(F\setminus M):E,\,F\in\Tau\}$.   For $H\in\Tau'$, set
\Centerline{$\nuprime H=\nu^*(H\cap M)+\nu_*(H\setminus M)$.}

\noindent Now we have to check the following.

\medskip

\quad{\bf (i)} $\Tau'$ is a subring of $\Cal PX$, because if $E$, $F$,
$E'$, $F'\in\Tau$ then

\Centerline{$((E\cap M)\cup(F\setminus M))
  *((E'\cap M)\cup(F'\setminus M))
=((E*E')\cap M)\cup((F*F')\setminus M)$}

\noindent for both the Boolean operations $*=\symmdiff$ and $*=\cap$.
$\Tau'\supseteq\Tau$ because $E=(E\cap M)\cup(E\setminus M)$ for every
$E\in\Tau$.   (Cf.\ 312N.)   $M\in\Tau'$ because there is some
$E\in\Tau_0$ such that
$M\subseteq E$, so that $M=(E\cap M)\cup(\emptyset\setminus M)\in\Tau'$.

\medskip

\quad{\bf (ii)} $\nuprime$ is finite-valued because if $H=(E\cap
M)\cup(F\setminus M)$, where $E$, $F\in\Tau$, then $\nuprime H\le\nu E+\nu
F$.
If $H$, $H'\in\Tau$ are disjoint, they can be expressed as $(E\cap
M)\cup(F\setminus M)$, $(E'\cap M)\cup(F'\setminus M)$ where $E$, $F$,
$E'$,
$F'$ belong to $\Tau$;  replacing $E'$, $F'$ by $E'\setminus E$
and $F'\setminus F$ if necessary, we may suppose that $E\cap E'=F\cap
F'=\emptyset$.   Now

$$\eqalignno{\nuprime(H\cup H')
&=\nu^*((E\cup E')\cap M)+\nu_*((F\cup F')\cap(X\setminus M))\cr
&=\nu^*(E\cap M)+\nu^*(E'\cap M)
   +\nu_*(F\cap(X\setminus M))+\nu_*(F'\cap(X\setminus M))\cr
\noalign{\noindent (by (b) above)}
&=\nuprime H+\nuprime H'.\cr}$$

\noindent Thus $\nuprime$ is additive.

\medskip

\quad{\bf (iii)} If $E\in\Tau$, then

$$\eqalign{\nu_*(E\setminus M)
&=\sup\{\nu F:F\in\Tau,\,F\subseteq E\setminus M\}\cr
&=\sup\{\nu E-\nu(E\setminus F):F\in\Tau,\,F\subseteq E\setminus M\}\cr
&=\sup\{\nu E-\nu F:F\in\Tau,\,E\cap M\subseteq F\subseteq E\}\cr
&=\nu E-\inf\{\nu F:F\in\Tau,\,E\cap M\subseteq F\subseteq E\}
=\nu E-\nu^*(E\cap M).\cr}$$

\noindent So

\Centerline{$\nuprime E=\nu^*(E\cap M)+\nu_*(E\setminus M)=\nu E$.}

\noindent Thus $\nuprime$ extends $\nu$.

\medskip

\quad{\bf (iv)} If $H\in\Tau'$ and $\epsilon>0$, express $H$ as $(E\cap
M)\cup(F\setminus M)$, where $E$, $F\in\Tau$.   Then we can find
($\alpha$)
a $K\in\Cal K\cap\Tau$ such that $K\subseteq E$ and $\nu(E\setminus
K)\le\epsilon$ ($\beta$) an $F'\in\Tau$ such that $F'\subseteq
F\setminus
M$
and $\nu F'\ge\nu_*(F\setminus M)-\epsilon$ ($\gamma$) a $K'\in\Cal
K\cap\Tau$ such that $K'\subseteq F'$ and $\nu K'\ge\nu F'-\epsilon$.
Set
$L=(K\cap M)\cup K'\in\Tau'$;  by the hypotheses ($\dagger$) and
($\ddagger$), $L\in\Cal K$.   Now $L\subseteq H$ and

$$\eqalign{\nuprime L
&=\nuprime(K\cap M)+\nuprime K'
=\nuprime(E\cap M)-\nuprime((E\setminus K)\cap M)+\nu K'\cr
&=\nu^*(H\cap M)-\nu^*((E\setminus K)\cap M)+\nu K'
\ge\nu^*(H\cap M)-\nu(E\setminus K)+\nu F'-\epsilon\cr
&\ge\nu^*(H\cap M)+\nu_*(F\setminus M)-3\epsilon
=\nuprime H-3\epsilon.\cr}$$

\noindent As $H$ and $\epsilon$ are arbitrary, $\nu$ is inner regular
with
respect to $\Cal K$.

\medskip
\quad{\bf (v)} Finally, given $H\in\Tau'$ and $\epsilon>0$, take $E$,
$F\in\Tau$ such that $H\cap M\subseteq E$, $F\subseteq H\setminus M$,
$\nu
E\le\nu^*(H\cap M)+\epsilon$ and $\nu F\ge\nu_*(H\setminus M)-\epsilon$.
In this case,

\Centerline{$\nuprime(E\setminus(H\cap M))=\nuprime E-\nuprime(H\cap M)
=\nu E-\nu^*(H\cap M)\le\epsilon$,}

\Centerline{$\nuprime((H\setminus M)\setminus F)=\nuprime(H\setminus M)-\nuprime F
=\nu_*(H\setminus M)-\nu F\le\epsilon$.}

\noindent But as

\Centerline{$H\symmdiff(E\cup F)\subseteq(E\setminus(H\cap
M))\cup((H\setminus M)\setminus F)$,}

\noindent $\nuprime(H\symmdiff(E\cup F))\le 2\epsilon$.   Now $\nu$
satisfies
the condition ($*$), so there is an $E_0\in\Tau_0$ such that $\nu((E\cup
F)\symmdiff E_0)\le\epsilon$, and $\nuprime(H\symmdiff E_0)\le 3\epsilon$.
As
$H$ and $\epsilon$ are arbitrary, $\nuprime$ satisfies ($*$).

This completes the proof that $\nuprime$ is a member of $P$ extending
$\nu$.\ \Qed

\medskip

{\bf (d)} It is easy to check that if $Q\subseteq P$ is a
non-empty totally ordered subset, the smallest common extension $\nuprime$
of the functions in $Q$ belongs to $P$.   (To see that $\nuprime$ is inner
regular with respect to $\Cal K$, observe that if $E\in\dom\nuprime$ and
$\gamma<\nuprime E$,
there is some $\nu\in Q$ such that $E\in\dom\nu$;  now there is a
$K\in\Cal K\cap\dom\nu$ such that $K\subseteq E$ and $\nu K\ge\gamma$,
so that $K\in\Cal K\cap\dom\nuprime$ and $\nuprime K\ge\gamma$.)
And of course $P$ is
not empty, because $\nu_0\in P$.   So by Zorn's Lemma $P$ has a maximal
element $\nu_1$ say;  write $\Tau_1$ for the domain of $\nu_1$.   If
$M\in\Cal K$ there is an element of $P$, with a domain containing $M$,
extending $\nu_1$; as $\nu_1$ is maximal, this must be $\nu_1$ itself,
so $M\in\Tau_1$.   Thus
$\Cal K\subseteq\Tau_1$, and $\nu_1$ has all the required properties.
}%end of proof of 413N

\vleader{72pt}{413O}{Corollary} Let $(X,\Sigma_0,\mu_0)$ be a measure space and
$\Cal K$ a countably compact class of subsets of $X$ such that

\inset{($\dagger$) $K\cup K'\in\Cal K$ whenever $K$, $K'\in\Cal K$ are
disjoint,}

\inset{($\ddagger$) $\bigcap_{n\in\Bbb N}K_n\in\Cal K$ for every
sequence $\sequencen{K_n}$ in $\Cal K$,}

\inset{$\mu_0^*K<\infty$ for every $K\in\Cal K$,}

\inset{$\mu_0$ is inner regular with respect to $\Cal K$.}

\noindent Then $\mu_0$ has an extension to a complete locally determined
measure $\mu$, defined on every member of $\Cal K$, inner regular with
respect to $\Cal K$, and such that whenever $E\in\dom\mu$ and $\mu
E<\infty$
there is an $E_0\in\Sigma_0$ such that $\mu(E\symmdiff E_0)=0$.

\proof{{\bf (a)} Set $\Tau_0=\{E:E\in\Sigma_0,\,\mu_0E<\infty\}$,
$\nu_0=\mu_0\restrp\Tau_0$.   Then $\nu_0$, $\Tau_0$ satisfy the
conditions
of 413N;   take $\nu_1$, $\Tau_1$ as in 413N.   If $K$, $L\in\Cal K$ and
$L\subseteq K$, then

\Centerline{$\nu_1L+\sup\{\nu_1K':K'\in\Cal K,\,K'\subseteq K\setminus
L\}
=\nu_1L+\nu_1(K\setminus L)=\nu_1K$.}

\noindent So $\nu_1\restr\Cal K$ satisfies the conditions of 413M and
there is a complete locally determined measure $\mu$, extending
$\nu_1\restr\Cal
K$, and inner regular with respect to $\Cal K$.

\medskip

{\bf (b)} Write $\Sigma$ for the domain of $\mu$.   Then
$\Tau_1\subseteq\Sigma$.   \Prf\ If $E\in\Tau_1$ and $K\in\Cal K$,

$$\eqalign{\mu_*(K\cap E)&+\mu_*(K\setminus E)\cr
&\ge\sup\{\mu K':K'\in\Cal K,\,K'\subseteq K\cap E\}
  +\sup\{\mu K':K'\in\Cal K,\,K'\subseteq K\setminus E\}\cr
&=\sup\{\nu_1 K':K'\in\Cal K,\,K'\subseteq K\cap E\}
  +\sup\{\nu_1 K':K'\in\Cal K,\,K'\subseteq K\setminus E\}\cr
&=\nu_1(K\cap E)+\nu_1(K\setminus E)
=\nu_1 K
=\mu K.\cr}$$

\noindent By 413F(iv), $E\in\Sigma$.\ \QeD\  It follows at once that
$\mu$ extends $\nu_1$, since if $E\in\Tau_1$

\Centerline{$\nu_1E
=\sup\{\nu_1K:K\in\Cal K,\,K\subseteq E\}
=\sup\{\mu K:K\in\Cal K,\,K\subseteq E\}
=\mu E$.}

\medskip

{\bf (c)} In particular, $\mu$ agrees with $\mu_0$ on $\Tau_0$.   Now in
fact $\mu$ extends $\mu_0$.   \Prf\  Take $E\in\Sigma_0$.   If $K\in\Cal
K$, there is an $E_0\in\Sigma_0$ such that $K\subseteq E_0$ and
$\mu_0E_0<\infty$.   Since $E\cap E_0\in\Tau_0\subseteq\Sigma$, $E\cap
K=E\cap E_0\cap K\in\Sigma$.   As $K$ is arbitrary, $E\in\Sigma$, by
413F(ii).   Next, because every member of $\Cal K$ is included in a
member of $\Tau_0$,

$$\eqalign{\mu_0 E
&=\sup\{\mu_0 K:K\in\Cal K\cap\Sigma_0,\,K\subseteq E\}
=\sup\{\mu_0(E\cap E_0):E_0\in\Tau_0\}\cr
&=\sup\{\mu(E\cap E_0):E_0\in\Tau_0\}
=\sup\{\mu K:K\in\Cal K,\,K\subseteq E\}
=\mu E.  \text{ \Qed}\cr}$$

\medskip

{\bf (d)}
Finally, suppose that $E\in\Sigma$ and $\mu E<\infty$.   For each
$n\in\Bbb N$ we
can find $K_n\in\Cal K$ and $E_n\in\Sigma_0$ such that $K_n\subseteq E$,
$\mu(E\setminus K_n)\le 2^{-n}$ and $\nu_1(K_n\symmdiff E_n)\le 2^{-n}$.
In this case $\sum_{n=0}^{\infty}\mu(E_n\symmdiff E)<\infty$, so
$\mu(E\symmdiff E')=0$, where
$E'=\bigcup_{n\in\Bbb N}\bigcap_{m\ge n}E_m\in\Sigma_0$.

Thus $\mu$ has all the required properties.
}%end of proof of 413O

\leader{413P}{}\cmmnt{ I now describe an alternative route to some of
the applications of 413N.   As before, I do as much as possible in the
context of finitely additive functionals.

\medskip

\noindent}{\bf Lemma} Let $X$ be a set and $\Cal K$ a sublattice of
$\Cal PX$ containing $\emptyset$.
Let $\lambda:\Cal K\to\coint{0,\infty}$ be a bounded functional
such that

\Centerline{$\lambda\emptyset = 0$,
\quad$\lambda K\le\lambda K'$ whenever $K$, $K'\in\Cal K$ and
$K\subseteq K'$,}

\Centerline{$\lambda(K\cup K')+\lambda(K\cap K')\ge\lambda K+\lambda K'$
for all $K$, $K'\in\Cal K$.}

\noindent   Then there is a finitely additive functional
$\nu:\Cal PX\to\coint{0,\infty}$ such that

\Centerline{$\nu X=\sup_{K\in\Cal K}\lambda K$,
\quad$\nu K\ge\lambda K$ for every $K\in\Cal K$.}

\proof{{\bf (a)} Let us consider first the case in which $\Cal K$ is finite.   I
induce on $n=\#(\Cal K)$.   If $n=1$ then $\Cal K=\{\emptyset\}$ and $\nu$ must be the
zero functional.   For the inductive step to $n>1$,
let $K_0$ be a minimal member of $\Cal K\setminus\{\emptyset\}$.   If $K\in\Cal K$
then $K\cap K_0$ is a member of $\Cal K$ included in $K_0$, so is either empty or
$K_0$, that is, either $K\cap K_0=\emptyset$ or $K\supseteq K_0$.   Set
$Y=X\setminus K_0$ and $\Cal L=\{K\setminus K_0:K\in\Cal K\}$.   Then
$\Cal L$ is a sublattice of $\Cal PY$ containing $\emptyset$, and
$K\mapsto K\setminus K_0:\Cal K\to\Cal L$ is surjective but not injective, so
$\#(\Cal L)<n$.

For $L\in\Cal L$, observe that $L\cup K_0\in\Cal K$.
\Prf\ There is a $K\in\Cal K$ such that $L=K\setminus K_0$.   If $K$ is disjoint from
$K_0$, then $L\cup K_0=K\cup K_0$ belongs to $\Cal K$;  if $K$ includes $K_0$ then
$L\cup K_0=K$ belongs to $\Cal K$.\ \Qed

We can therefore define $\lambda':\Cal L\to\coint{0,\infty}$ by setting

\Centerline{$\lambda'L=\lambda(L\cup K_0)-\lambda K_0$}

\noindent for every $L\in\Cal L$.   Of course $\lambda'\emptyset=0$ and
$\lambda'L\le\lambda'L'$ whenevery $L\subseteq L'$.   If $L$, $L'\in\Cal L$ then

$$\eqalign{\lambda'(L\cup L')+\lambda'(L\cap L')
&=\lambda(L\cup L'\cup K_0)+\lambda((L\cap L')\cup K_0)-2\lambda K_0\cr
&=\lambda((L\cup K_0)\cup(L'\cup K_0))+\lambda((L\cup K_0)\cap(L'\cup K_0))
    -2\lambda K_0\cr
&\ge\lambda(L\cup K_0)+\lambda(L'\cup K_0)-2\lambda K_0
=\lambda'L+\lambda'L'.\cr}$$

\noindent So the hypotheses of this lemma are satisfied by $Y$, $\Cal L$ and
$\lambda'$, and by the inductive hypothesis there is a finitely additive functional
$\nuprime:\Cal PY\to\coint{0,\infty}$ such that

\Centerline{$\nuprime Y=\sup_{L\in\Cal L}\lambda'L$,
\quad$\nuprime L\ge\lambda'L$ for every $L\in\Cal L$.}

Fix any $x_0\in K_0$ and define $\nu:\Cal PX\to\coint{0,\infty}$ by setting

$$\eqalign{\nu A
&=\lambda K_0+\nuprime(A\cap Y)\text{ if }x_0\in A\subseteq X,\cr
&=\nuprime(A\cap Y)\text{ for other }A\subseteq X.\cr}$$

\noindent Then $\nu$ is additive.   If $K\in\Cal K$ is disjoint from $K_0$ then

\Centerline{$\nu K=\nuprime K\ge\lambda'K=\lambda(K\cup K_0)-\lambda K_0
\ge\lambda K-\lambda(K\cap K_0)=\lambda K$.}

\noindent If $K\in\Cal K$ includes $K_0$ then

\Centerline{$\nu K=\lambda K_0+\nuprime(K\setminus K_0)
\ge\lambda K_0+\lambda'(K\setminus K_0)
=\lambda K_0+\lambda K-\lambda K_0=\lambda K$.}

\noindent Finally,

$$\eqalign{\nu X
&=\lambda K_0+\nuprime Y
=\lambda K_0+\sup_{L\in\Cal L}\lambda'L\cr
&=\lambda K_0+\sup_{K\in\Cal K}(\lambda(K\cup K_0)-\lambda K_0)
=\sup_{K\in\Cal K}\lambda(K\cup K_0)
=\sup_{K\in\Cal K}\lambda K.\cr}$$

\noindent So $\nu$ has the required properties and the induction continues.

\medskip

{\bf (b)} For the general case, set $\gamma=\sup_{K\in\Cal K}\lambda K$.
We need to know that every finite subset of $\Cal K$ is included in a
finite sublattice of $\Cal K$;  this is because it is included in a
finite subalgebra $\Cal E$ of $\Cal PX$ and $\Cal K\cap\Cal E$ is a
sublattice.   Let $\Nu$ be the set of all finitely additive functionals
$\nu:\Cal PX\to[0,\gamma]$.   Then $\Nu$ is a closed subset of
$[0,\gamma]^{\Cal PX}$, so is compact.   For each $K\in\Cal K$ set
$\Nu_K=\{\nu:\nu\in\Nu$, $\nu K\ge\lambda K\}$.   Then $\Nu_K$ is a
closed subset of $\Nu$.   If $\Cal K_0\subseteq\Cal K$ is finite, there
is a finite sublattice $\Cal L$ of $\Cal K$ including
$\Cal K_0\cup\{\emptyset\}$, and now (a) tells us that there is a
$\nu\in\bigcap_{K\in\Cal L}\Nu_K$.   Thus $\{\Nu_K:K\in\Cal K\}$ has the
finite intersection property and there is a
$\nu\in\bigcap_{K\in\Cal K}\Nu_K$.   In this case,
$\nu:\Cal PX\to[0,\gamma]$ is a finitely additive functional dominating
$\lambda$;  it follows that $\nu X=\gamma$ and the proof is complete.
}%end of proof of 413P

\cmmnt{\medskip

\noindent{\bf Remark} If $P$ is a lattice, a function
$f:P\to\ocint{-\infty,\infty}$
such that $f(p\vee q)+f(p\wedge q)\ge f(p)+f(q)$ for all $p$, $q\in P$
is called {\bf supermodular}.
}%end of comment

\vleader{72pt}{413Q}{Theorem} Let $X$ be a set and $\Cal K$ a sublattice of
$\Cal PX$ containing $\emptyset$.
Let $\Sigma$ be the algebra of subsets of $X$ generated by $\Cal K$, and
$\nu_0:\Sigma\to\coint{0,\infty}$ a finitely additive functional.
Then there is a finitely additive functional
$\nu:\Sigma\to\coint{0,\infty}$ such that

(i) $\nu X=\sup_{K\in\Cal K}\nu_0K$,

(ii) $\nu K\ge\nu_0K$ for every $K\in\Cal K$,

(iii) $\nu$ is inner regular with respect to $\Cal K$
in the sense that $\nu E=\sup\{\nu K:K\in\Cal K,\,K\subseteq E\}$ for
every $E\in\Sigma$.

\proof{{\bf (a)} Set $\gamma=\sup_{K\in\Cal K}\nu_0K$.   Let $P$ be the
set of all functionals $\lambda:\Cal K\to[0,\gamma]$ such that

\Centerline{$\lambda K+\lambda K'\le\lambda(K\cup K')+\lambda(K\cap
K')$}

\noindent for every $K$, $K'\in\Cal K$.   Give $P$ the natural partial
order inherited from $\BbbR^{\Cal K}$.   Note that
$\nu_0\restr\Cal K$
belongs to $P$.   If $Q\subseteq P$ is non-empty and upwards-directed,
then $\sup Q$, taken in $\BbbR^{\Cal K}$, belongs to $P$;  so there is
a maximal $\lambda\in P$ such that $\nu_0\restr\Cal K\le\lambda$.   By
413P, there is a non-negative additive functional $\nu$ on $\Cal PX$ such that
$\nu K\ge\lambda K$ for every $K\in\Cal K$ and $\nu X=\gamma$.   Since
$\nu\restr\Cal K$ also belongs to $P$, we
must have $\nu K=\lambda K$ for every $K\in\Cal K$.

\medskip

{\bf (b)} Now for any $K_0\in\Cal K$,

\Centerline{$\nu K_0
  +\sup\{\nu L:L\in\Cal K,\,L\subseteq X\setminus K_0\}
=\gamma$.}

\medskip

\Prf\ {\bf (i)} Set
$\Cal L=\{L:L\in\Cal K,\,L\subseteq X\setminus K_0\}$.
For $A\subseteq X$, set $\theta_0A=\sup_{L\in\Cal L}\nu(A\cap L)$.
Because
$\Cal L$ is upwards-directed, $\theta_0:\Cal PX\to\Bbb R$ is additive,
and
of course $0\le\theta_0\le\nu$.   Set $\theta_1=\nu-\theta_0$, so that
$\theta_1$ is another additive functional, and write

\Centerline{$\lambda'K=\theta_0K+\sup\{\theta_1M:M\in\Cal K,\,M\cap
K_0\subseteq K\}$}

\noindent for $K\in\Cal K$.

\medskip

\quad{\bf (ii)} If $K$, $K'\in\Cal K$ and $\epsilon>0$, there are $M$,
$M'\in\Cal K$ such that $M\cap K_0\subseteq K$, $M'\cap K_0\subseteq K'$
and

\Centerline{$\theta_0K+\theta_1M\ge\lambda'K-\epsilon$,
\quad$\theta_0K'+\theta_1M'\ge\lambda'K'-\epsilon$.}

\noindent Now

\Centerline{$M\cup M'\in\Cal K$,\quad $M\cap M'\in\Cal K$,}

\Centerline{$(M\cup M')\cap K_0\subseteq K\cup K'$,
\quad$(M\cap M')\cap K_0\subseteq K\cap K'$,}

\noindent so

$$\eqalign{\lambda'(K\cup K')+\lambda'(K\cap K')
&\ge\theta_0(K\cup K')+\theta_1(M\cup M')
  +\theta_0(K\cap K')+\theta_1(M\cap M')\cr
&=\theta_0K+\theta_1M+\theta_0K'+\theta_1M'
\ge\lambda'K+\lambda'K'-2\epsilon.\cr}$$

\noindent As $\epsilon$ is arbitrary, $\lambda'(K\cup K')+\lambda'(K\cap
K')\ge\lambda K+\lambda K'$.

\medskip

\quad{\bf (iii)} Suppose that $K$, $M\in\Cal K$ are such that $M\cap
K_0\subseteq K$.   If $L\in\Cal L$, then

$$\eqalign{\nu(K\cap L)+\theta_1M
&=\nu(K\cap L)+\nu M-\theta_0M\cr
&=\nu(M\cap K\cap L)+\nu(M\cup(K\cap L))-\theta_0M
\le\gamma\cr}$$

\noindent because $K\cap L\in\Cal L$;   taking the supremum over $L$ and
$M$, $\lambda'K\le\gamma$.   As $K$ is arbitrary, $\lambda'\in P$.

\medskip

\quad{\bf (iv)} If $K\in\Cal K$, then of course $K\cap K_0\subseteq K$,
so

\Centerline{$\lambda'K\ge\theta_0K+\theta_1K=\nu K=\lambda K$.}

\noindent Thus $\lambda'\ge\lambda$.   Because $\lambda$ is maximal,
$\lambda'=\lambda$.   But this means that

\Centerline{$\lambda K_0=\lambda'K_0
=\theta_0K_0+\sup\{\theta_1M:M\in\Cal K,\,M\cap K_0\subseteq K_0\}
=\sup_{M\in\Cal K}\theta_1M$.}

Now given $\epsilon>0$ there is an $M\in\Cal K$ such that

\Centerline{$\gamma-\epsilon\le\nu_0M\le\lambda M=\nu M$,}

\noindent so that

\Centerline{$\nu K_0=\lambda K_0\ge\theta_1M=\nu M-\theta_0M
\ge\gamma-\epsilon-\theta_0M
\ge\gamma-\epsilon-\sup_{L\in\Cal L}\nu L$,}

\noindent and $\nu K_0+\sup_{L\in\Cal L}\nu L\ge\gamma-\epsilon$.   As
$\epsilon$ is arbitrary, $\nu K_0+\sup_{L\in\Cal L}\nu L\ge\gamma$.
But of course $\nu K_0+\nu L\le\nu X=\gamma$ for every $L\in\Cal L$, so
$\nu K_0+\sup_{L\in\Cal L}\nu L=\gamma$, as claimed.\ \Qed

\medskip

{\bf (c)} It follows that if $K$, $L\in\Cal K$ and $L\subseteq K$,

\Centerline{$\nu K=\nu L+\sup\{\nu K':K'\in\Cal K,\,K'\subseteq
K\setminus L\}$.}

\noindent\Prf\ Because $\nu$ is additive and non-negative, we surely
have

\Centerline{$\nu K\ge\nu L+\sup\{\nu K':K'\in\Cal K,\,K'\subseteq
K\setminus L\}$.}

\noindent On the other hand, given $\epsilon>0$, there is an $M\in\Cal
K$ such that $M\subseteq X\setminus L$ and
$\nu L+\nu M\ge\gamma-\epsilon$, so that $M\cap K\in\Cal K$,
$M\cap K\subseteq K\setminus L$
and

\Centerline{$\nu L+\nu(M\cap K)=\nu L+\nu K+\nu M-\nu(M\cup K)
\ge\nu K+\gamma-\epsilon-\gamma=\nu K-\epsilon$.}

\noindent As $\epsilon$ is arbitrary,

\Centerline{$\nu K
\le\nu L+\sup\{\nu K':K'\in\Cal K,\,K'\subseteq K\setminus L\}$}

\noindent and we have equality.\ \Qed

\medskip

{\bf (d)} By 413H, we have an additive functional
$\nuprime:\Sigma\to\coint{0,\infty}$ such that
$\nuprime E=\sup\{\nu K:K\in\Cal K,\,K\subseteq E\}$ for every $E\in\Sigma$.
It is easy to show that
$\nuprime$ and $\nu$ must agree on $\Sigma$, but even without doing so we
can see that
$\nuprime$ has the properties (i)-(iii) required in the theorem.
}%end of proof of 413Q

\leader{413R}{}\cmmnt{ The following lemma on countably compact
classes, corresponding to 342Db, will be useful.

\medskip

\noindent}{\bf Lemma}\cmmnt{ ({\smc Marczewski 53})} Let $\Cal K$ be a
countably compact class of sets.
Then there is a countably compact class $\Cal K^*\supseteq\Cal K$ such
that $K\cup L\in\Cal K^*$ and $\bigcap_{n\in\Bbb N}K_n\in\Cal K^*$ whenever $K$, $L\in\Cal K^*$ and $\sequencen{K_n}$ is a sequence in $\Cal K^*$.

\proof{{\bf (a)} Write $\Cal K_s$ for $\{K_0\cup\ldots\cup
K_n:K_0,\ldots,K_n\in\Cal K\}$.   Then $\Cal K_s$ is countably compact.
\Prf\ Let $\sequencen{L_n}$ be a sequence in $\Cal K_s$ such that
$\bigcap_{i\le n}L_i\ne\emptyset$ for each $n\in\Bbb N$.   Then there is
an ultrafilter $\Cal F$ on $X=\bigcup\Cal K$ containing every $L_n$.
For each $n$, $L_n$ is a finite union of members of $\Cal K$, so there
must be a $K_n\in\Cal K$ such that $K_n\subseteq L_n$ and $K_n\in\Cal
F$.   Now $\bigcap_{i\le n}K_i\ne\emptyset$ for every $n$, so
$\bigcap_{n\in\Bbb N}K_n\ne\emptyset$ and $\bigcap_{n\in\Bbb
N}L_n\ne\emptyset$.   As $\sequencen{L_n}$ is arbitrary, $\Cal K_s$ is
countably compact.\ \Qed

Note that $L\cup L'\in\Cal K_s$ for all $L$, $L'\in\Cal K_s$.

\medskip

{\bf (b)} Write $\Cal K^*$ for

\Centerline{$\{\bigcap\Cal L_0:\Cal L_0\subseteq\Cal K_s$ is non-empty
and countable$\}$.}

\noindent Then $\Cal K^*$ is countably compact.   \Prf\ If
$\sequencen{M_n}$ is any sequence in $\Cal K^*$ such that
$\bigcap_{i\le n}M_i\ne\emptyset$ for every $n\in\Bbb N$, then for each
$n\in\Bbb N$
let $\Cal L_n\subseteq\Cal K_s$ be a countable non-empty set such that
$M_n=\bigcap\Cal L_n$.   Let $\sequencen{L_n}$ be a sequence running
over $\bigcup_{n\in\Bbb N}\Cal L_n$;  then
$\bigcap_{i\le n}L_i\ne\emptyset$ for every $n$, so
$\bigcap_{n\in\Bbb N}L_n=\bigcap_{n\in\Bbb N}M_n$ is non-empty.   As
$\sequencen{M_n}$ is arbitrary, $\Cal K^*$ is countably compact.\ \Qed

\medskip

{\bf (c)} Of course $\Cal K\subseteq\Cal K_s\subseteq\Cal K^*$.   It is
immediate from the definition of $\Cal K^*$ that it is closed under
countable intersections.   Finally, if $M_1$, $M_2\in\Cal K^*$, let
$\Cal L_1$, $\Cal L_2\subseteq\Cal K_s$ be countable sets such that
$M_1=\bigcap\Cal L_1$ and $M_2=\bigcap\Cal L_2$;  then
$\Cal L=\{L_1\cup L_2:L_1\in\Cal L_1,\,L_2\in\Cal L_2\}$ is a countable
subset of $\Cal K_s$, so $M_1\cup M_2=\bigcap\Cal L$ belongs to
$\Cal K^*$.
}%end of proof of 413R

\leader{413S}{Corollary} Let $X$ be a set and $\Cal K$ a countably
compact class of subsets of $X$.
Let $\Tau$ be a subalgebra of $\Cal PX$ and $\nu:\Tau\to\Bbb R$
a non-negative finitely additive functional.

(a) There is a complete measure $\mu$ on $X$ such that $\mu X\le\nu X$,
$\Cal K\subseteq\dom\mu$ and
$\mu K\ge\nu K$ for every $K\in\Cal K\cap\Tau$.

(b) If

\inset{($\dagger$) $K\cup K'\in\Cal K$ whenever $K$, $K'\in\Cal K$,}

\inset{($\ddagger$) $\bigcap_{n\in\Bbb N}K_n\in\Cal K$ for every
sequence $\sequencen{K_n}$ in $\Cal K$,}

\noindent we may arrange that $\mu$ is inner
regular with respect to $\Cal K$.

\proof{ By 413R, there is always a countably compact class
$\Cal K^*\supseteq\Cal K$ satisfying ($\dagger$) and ($\ddagger$);  for
case (b), take $\Cal K^*=\Cal K$.
By 391G, there is an extension of $\nu$ to a finitely additive
functional $\nuprime:\Cal PX\to\Bbb R$.
Let $\Tau_1$ be the subalgebra of $\Cal PX$ generated by $\Cal K^*$.
By 413Q, there is a non-negative additive functional
$\nu_1:\Tau_1\to\Bbb
R$ such that $\nu_1X\le\nuprime X=\nu X$, $\nu_1K\ge\nuprime K=\nu K$ for every
$K\in\Cal K^*\cap\Tau$ and $\nu_1E=\sup\{\nu_1K:K\in\Cal
K^*,\,K\subseteq
E\}$ for every $E\in\Tau_1$.   In particular, if $K$, $L\in\Cal K^*$,

\Centerline{$\nu_1L
   +\sup\{\nu_1K':K'\in\Cal K^*,\,K'\subseteq K\setminus L\}
=\nu_1L+\nu_1(K\setminus L)=\nu_1K$.}

\noindent So $\Cal K^*$ and $\nu_1\restr\Cal K^*$ satisfy the hypotheses
of 413M.   Accordingly we have a complete measure $\mu$ extending
$\nu_1\restr\Cal K^*$ and inner regular with respect to
$\Cal K^*=\Cal K^*_{\delta}$;  in
which case

\Centerline{$\mu K=\nu_1K\ge\nu K$}

\noindent for every $K\in\Cal K\cap\Tau$, and

\Centerline{$\mu X=\sup_{K\in\Cal K^*}\mu K
=\sup_{K\in\Cal K^*}\nu_1 K\le\nu_1X\le\nu X$,}

\noindent as required.
}%end of proof of 413S

\leader{413T}{}\cmmnt{ The following fact is interesting and not quite
obvious.

\medskip

\noindent}{\bf Proposition} Let $(X,\Sigma,\mu)$ be a complete totally
finite measure space, $(Y,\Tau,\nu)$ a measure space, and $\frak S$ a
Hausdorff topology on $Y$ such that $\nu$ is inner
regular with respect to the closed sets.   Let $\sequencen{f_n}$ be a
sequence of
\imp\ functions from $X$ to $Y$.   If $f(x)=\lim_{n\to\infty}f_n(x)$ is
defined in $Y$ for every $x\in X$, then $f$ is \imp.

\proof{ Let $\mu_*$ be the inner measure associated with $\mu$
(413D).   If $F\in\Tau$ is closed then
$\bigcap_{n\in\Bbb N}\bigcup_{m\ge n}f_m^{-1}[F]\subseteq f^{-1}[F]$, so

\Centerline{$\mu_*f^{-1}[F]
\ge\mu(\bigcap_{n\in\Bbb N}\bigcup_{m\ge n}f_m^{-1}[F])
\ge\liminf_{n\to\infty}\mu f_n^{-1}[F]=\nu F$.}

\noindent So if $H$ is any member of $\Tau$,

$$\eqalign{\mu_*f^{-1}[H]
&\ge\sup\{\mu_*f^{-1}[F]:
  F\in\Tau,\,F\subseteq H\text{ and }F\text{ is closed}\}\cr
&\ge\sup\{\nu F:
  F\in\Tau,\,F\subseteq H\text{ and }F\text{ is closed}\}
=\nu H.\cr}$$

\noindent Taking complements,

$$\eqalignno{\mu^*f^{-1}[H]
&=\mu X-\mu_*f^{-1}[Y\setminus H]\cr
\displaycause{413Ec}
&\le\nu Y-\nu(Y\setminus H)
=\nu H\cr}$$

\noindent (of course $\nu Y=\mu f_0^{-1}[Y]=\mu X$).   So
$\mu^*f^{-1}[H]=\mu_*f^{-1}[H]=\nu H$.   Because $\mu$ is complete,
$\mu f^{-1}[H]$ is defined and equal to $\nu H$ (413Ef).  As $H$ is
arbitrary, $f$ is \imp.
}%end of proof of 413T
%is this in the right place? query

%doesn't work for $\sigma$-finite case, try $X=Y=\Bbb R\cup\{\infty}$ and
% $f_n(x)=x+n$

\exercises{
\leader{413X}{Basic exercises (a)}
%\spheader 413Xa
Define $\phi:\Cal P\Bbb N\to\coint{0,\infty}$ by setting
$\phi A=0$ if $A$ is finite, $\infty$ otherwise.   Check that $\phi$
satisfies conditions ($\alpha$) and ($\beta$) of 413A, but that if we
attempt to reproduce the construction of 413C then we obtain
$\Sigma=\Cal P\Bbb N$ and $\mu=\phi$, so that $\mu$ is not countably
additive.
%413C

\spheader 413Xb Let $\phi_1$, $\phi_2$ be two inner measures on a set
$X$, inducing measures $\mu_1$ and $\mu_2$ by the method of 413C.   (i)
Show that $\phi=\phi_1+\phi_2$ is an inner measure.   (ii) Show that the
measure $\mu$
induced by $\phi$ extends the measure $\mu_1+\mu_2$ defined on
$\dom\mu_1\cap\dom\mu_2$.
%413C

\sqheader 413Xc Let $X$ be a set, $\phi$ an inner measure on $X$, and
$\mu$ the measure constructed from it by the method of 413C.   (i) Let
$Y$ be a subset of $X$.   Show that $\phi\restrp\Cal PY$ is an inner
measure on $Y$,
and that the measure on $Y$ defined from it extends the subspace measure
$\mu_Y$ induced on $Y$ by $\mu$.   (ii) Now suppose that $\phi X$ is
finite.   Let $Y$ be a set and $f:X\to Y$
a function.   Show that $B\mapsto\phi f^{-1}[B]$ is an inner measure on
$Y$, and that it defines a measure on $Y$ which extends the image
measure $\mu f^{-1}$.
%413C

\spheader 413Xd Let $(X,\Sigma,\mu)$ be a measure space.   Set
$\theta A=\bover12(\mu^*A+\mu_*A)$ for every $A\subseteq X$.   Show that
$\theta$ is
an outer measure on $X$, and that if $\mu$ is semi-finite then
the measure defined from $\theta$ by
\Caratheodory's method extends $\mu$.
%413D

\sqheader 413Xe Show that there is a partition $\sequencen{A_n}$ of
$[0,1]$ such that $\mu_*(\bigcup_{i\le n}A_i)=0$ for every $n$, where
$\mu_*$ is
Lebesgue inner measure.   \Hint{set $A_n=(A+q_n)\cap[0,1]$ where
$\sequencen{q_n}$ is an enumeration of $\Bbb Q$ and $A$ is a suitable
set;  cf.\ 134B.}
%413D

\spheader 413Xf Let $(X,\Sigma,\mu)$ be a measure space.   (i) Show that
$\mu_*\restr\Sigma$ is the semi-finite version $\mu_{\text{sf}}$ of $\mu$ as
constructed in 213Xc.   (ii) Show that if $A$ is any subset of $X$, and
$\Sigma_A$ the
subspace $\sigma$-algebra, then $\mu_*\restr\Sigma_A$ is a semi-finite
measure on $A$.
%413E

\sqheader 413Xg Let $(X,\Sigma,\mu)$ and $(Y,\Tau,\nu)$ be two measure
spaces, and $\lambda$ the c.l.d.\ product measure on $X\times Y$.   Show
that $\lambda_*(A\times B)=\mu_*A\cdot\nu_*B$ for all $A\subseteq X$ and
$B\subseteq Y$.   \Hint{use Fubini's theorem to show that
$\lambda_*(A\times B)\le\mu_*A\cdot\nu_*B$.}
%413E

\spheader 413Xh(i) Let $(X,\Sigma,\mu)$ be a $\sigma$-finite measure
space and $f:X\to\Bbb R$ a function such that $\overline{\int}fd\mu$ is
finite.   Show that for every $\epsilon>0$ there is a measure $\nu$ on
$X$ extending $\mu$ such that
$\underline{\int}fd\nu\ge\overline{\int}fd\mu-\epsilon$.   \Hint{133Ja,
215B(viii), 417Xa.}   (ii) Let $(X,\Sigma,\mu)$ be a totally finite
measure space and $f:X\to\Bbb R$ a bounded function.   Show that there is
a finitely additive functional $\nu:\Cal PX\to\coint{0,\infty}$,
extending $\mu$, such that $\dashint fd\nu$, defined as in 363L, is
equal to $\overline{\int}fd\mu$.
%413E

\spheader 413Xi Let $X$ be a set and $\mu$, $\nu$ two complete locally determined measures on $X$ with domains $\Sigma$, $\Tau$ respectively, both inner
regular with respect to $\Cal K\subseteq\Sigma\cap\Tau$.   Suppose that,
for $K\in\Cal K$, $\mu K=0$ iff $\nu K=0$.   Show that $\Sigma=\Tau$ and
that $\mu$ and $\nu$ have the same null ideals.
%413F

\sqheader 413Xj Let $(X,\Tau,\nu)$ be a measure space.   (i) Show that
the measure constructed by the method of 413C from the inner measure
$\nu_*$ is the c.l.d.\ version of $\nu$.   (ii) Set
$\Cal K=\{E:E\in\Tau,\,\nu E<\infty\}$, $\phi_0=\nu\restr\Cal K$.   Show
that $\Cal K$ and $\phi_0$ satisfy the conditions of 413I, and that the
measure constructed by the method there is again the c.l.d.\ version of
$\nu$.
%413I

\spheader 413Xk Let $(X,\Sigma,\mu)$ be a complete locally determined
measure space and $\Cal L$ a family of subsets of $X$ such that $\mu$ is
inner regular with respect to $\Cal L$.   Set
$\Cal K=\{K:K\in\Cal L\cap\Sigma,\,\mu K<\infty\}$ and
$\phi_0=\mu\restr\Cal K$.   Show that
$\Cal K$ and $\phi_0$ satisfy the conditions of 413J and that the measure
constructed from them by the method there is just $\mu$.
%413I

\sqheader 413Xl Let $\Cal K$ be the family of subsets of $\Bbb R$
expressible as disjoint finite unions of bounded closed intervals.
(i) Show from first principles that there is a unique functional
$\phi_0:\Cal K\to\coint{0,\infty}$ such that
$\phi_0[\alpha,\beta]=\beta-\alpha$
whenever $\alpha\le\beta$ and $\phi_0$ satisfies the conditions of 413J.
(ii) Show that the measure on $\Bbb R$ constructed from $\phi_0$ by the
method of 413J is Lebesgue measure.
%413J

\spheader 413Xm Let $X$ be a set, $\Sigma$ a subring of $\Cal PX$, and
$\nu:\Sigma\to\coint{0,\infty}$ a non-negative additive functional such
that $\lim_{n\to\infty}\nu E_n=0$ whenever $\sequencen{E_n}$ is a
non-increasing sequence in $\Sigma$ with empty intersection, as in 413K.
Define $\theta:\Cal PX\to[0,\infty]$ by setting

\Centerline{$\theta A=\inf\{\sum_{n=0}^{\infty}\nu E_n:\sequencen{E_n}$
is a sequence in $\Sigma$ covering $A\}$}

\noindent for $A\subseteq X$, interpreting $\inf\emptyset$ as $\infty$
if necessary.   Show that $\theta$ is an outer measure.   Let
$\mu_{\theta}$ be
the measure defined from $\theta$ by \Caratheodory's method.   Show
that the measure defined from $\nu$ by the process of 413K is the
c.l.d.\ version
of $\mu_{\theta}$.   \Hint{the c.l.d.\ version of $\mu_{\theta}$ is
inner regular with respect to $\Sigma_{\delta}$.}
%413K

\sqheader 413Xn Let $X$ be a set, $\Sigma$ a subring of $\Cal PX$, and
$\nu:\Sigma\to\coint{0,\infty}$ a non-negative additive functional.
Show that the following are equiveridical:  (i) $\nu$ has an extension to a measure on $X$;  (ii) $\lim_{n\to\infty}\nu E_n=0$ whenever
$\sequencen{E_n}$ is a
non-increasing sequence in $\Sigma$ with empty intersection;  (iii)
$\nu(\bigcup_{n\in\Bbb N}E_n)=\sum_{n=0}^{\infty}\nu E_n$ whenever
$\sequencen{E_n}$ is a disjoint sequence in $\Sigma$ such that
$\bigcup_{n\in\Bbb N}E_n\in\Sigma$.
%413K

\sqheader 413Xo Let $\sequencen{(X_n,\Sigma_n,\mu_n)}$ be a sequence of
probability spaces, and $\Cal F$ a non-principal ultrafilter on
$\Bbb N$.   For $x$, $y\in\prod_{n\in\Bbb N}X_n$, write $x\sim y$ if
$\{n:x(n)=y(n)\}\in\Cal F$.   (i) Show that $\sim$ is an equivalence
relation;  write $X$ for the set of equivalence classes, and
$x^{\ssbullet}\in X$ for the equivalence class of
$x\in\prod_{n\in\Bbb N}X_n$.  (Compare 351M.)   (ii) Let $\Sigma$ be the
set of subsets of $X$ expressible in the form
$Q(\sequencen{E_n})=\{x^{\ssbullet}:x\in\prod_{n\in\Bbb N}E_n\}$, where
$E_n\in\Sigma_n$ for each $n\in\Bbb N$.   Show that $\Sigma$ is an
algebra of subsets of $X$, and that there is a well-defined additive
functional $\nu:\Sigma\to[0,1]$ defined by setting
$\nu(Q(\sequencen{E_n}))=\lim_{n\to\Cal F}\mu_nE_n$.   (iii) Show that
for any non-decreasing sequence $\sequence{i}{H_i}$ in $\Sigma$ there is
an $H\in\Sigma$ such that $H\subseteq\bigcap_{i\in\Bbb N}H_i$ and $\nu
H=\lim_{n\to\infty}\nu H_n$.   \Hint{express each $H_i$ as
$Q(\sequencen{E_{in}})$.   Do this in such a way that
$E_{i+1,n}\subseteq
E_{in}$ for all $i$, $n$.   Take a decreasing sequence
$\sequence{i}{J_i}$
in $\Cal F$, with empty intersection, such that
$\nu H_i\le\mu E_{in}+2^{-i}$ for $n\in J_i$.   Set $E_n=E_{in}$ for
$n\in J_i\setminus J_{i+1}$.}   (iv) Show that there is a unique
extension of
$\nu$ to a complete probability measure $\mu$ on $X$ which is inner regular with
respect to $\Sigma$.   (This is a kind of {\bf Loeb measure}.)
%413K

\spheader 413Xp Let $\frak A$ be a Boolean algebra and
$K\subseteq\frak A$ a sublattice containing $0$.   Suppose that
$\lambda:K\to\coint{0,\infty}$
is a bounded functional such that $\lambda 0=0$,
$\lambda a\le\lambda a'$ whenever $a$, $a'\in K$ and $a\Bsubseteq a'$,
and $\lambda(a\Bcup a')+\lambda(a\Bcap a')\ge\lambda a+\lambda a'$ for
all $a$, $a'\in K$.   Show that there is a non-negative additive
functional $\nu:\frak A\to\Bbb R$ such that $\nu a\ge\lambda a$ for
every $a\in K$ and $\nu 1=\sup_{a\in K}\lambda a$.
%413P

\spheader 413Xq Let $X$ be a set and $\Cal K$ a sublattice of $\Cal PX$
containing $\emptyset$.   Let $\lambda:\Cal K\to\Bbb R$ be an
order-preserving function such that $\lambda\emptyset=0$ and
$\lambda(K\cup K')+\lambda(K\cap K')=\lambda K+\lambda K'$ for all $K$,
$K'\in\Cal K$.   Show that there is a non-negative additive functional
$\nu:\Cal PX\to[0,\infty]$ extending $\lambda$.   \Hint{start with the case
$X\in\Cal K$.}
(If $P$ is a lattice, a functional $f:P\to\Bbb R$ such that
$f(p\vee q)+f(p\wedge q)=f(p)+f(q)$ for all $p$, $q\in P$ is called
{\bf modular}.)
%413P

\spheader 413Xr Let $X$ be a set and $\Cal K$ a sublattice of
$\Cal PX$.   Let $\lambda:\Cal K\to[0,1]$ be a functional
such that

\Centerline{$\lambda K\le\lambda K'$ whenever $K$, $K'\in\Cal K$ and
$K\subseteq K'$,
\quad$\inf_{K\in\Cal K}\lambda K=0$,}

\Centerline{$\lambda(K\cup K')+\lambda(K\cap K')\le\lambda K+\lambda K'$
for all $K$, $K'\in\Cal K$}

\noindent Show that there is a finitely
additive functional $\nu:\Cal PX\to[0,1]$ such that

\Centerline{$\nu X=\sup_{K\in\Cal K}\lambda K$,
\quad$\nu K\le\lambda K$ for every $K\in\Cal K$.}
%413P

\spheader 413Xs Let $X$ be a set and $\Cal K$ a sublattice of $\Cal PX$
containing $\emptyset$.   Let $\lambda:\Cal K\to\coint{0,\infty}$ be such
that $\lambda K\le\sum_{n=0}^{\infty}(\lambda K_n-\lambda L_n)$ whenever
$K\in\Cal K$ and $\sequencen{K_n}$, $\sequencen{L_n}$ are sequences in
$\Cal K$ such that $L_n\subseteq K_n$ for every $n$ and
$\sequencen{K_n\setminus L_n}$ is a disjoint cover of
$K$.   Show that there is a measure on $X$ extending $\lambda$.
\Hint{Show that we can apply 413Xq.
Show that if $\Tau$ is the ring of subsets of $X$ generated by $\Cal K$,
every member of $\Tau$ is a finite union of differences of members of
$\Cal K$.   Now apply 413Ka.   See {\smc Kelley \& Srinivasan 71}.}
%413Xq 413K 413P
%Kelley & Srinivasan?

\leader{413Y}{Further exercises (a)}
%\spheader 413Ya
Give an example of two inner measures $\phi_1$, $\phi_2$ on a set $X$
such that the measure defined by $\phi_1+\phi_2$ strictly extends the
sum of the measures defined by $\phi_1$ and $\phi_2$.
%413C, 413Xb mt41bits

\spheader 413Yb Let $\langle(X_i,\Sigma_i,\mu_i)\rangle_{i\in I}$ be any
family of probability spaces, and $\lambda$ the product measure on
$X=\prod_{i\in I}X_i$.   Show that
$\lambda_*(\prod_{i\in I}A_i)\le\prod_{i\in I}(\mu_i)_*A_i$ whenever
$A_i\subseteq X_i$ for every $i$, with equality if $I$ is countable.
%413E

\spheader 413Yc Let $(X,\Sigma,\mu)$ be a totally finite measure space,
and $Z$ the Stone space of the Boolean algebra $\Sigma$.   For
$E\in\Sigma$ write $\widehat{E}$ for the corresponding open-and-closed
subset of $Z$.   Show that there is a unique function $f:X\to Z$ such
that $f^{-1}[\widehat{E}]=E$ for every $E\in\Sigma$.   Show that there
is a measure $\nu$ on $Z$, inner regular with respect to the
open-and-closed sets,
such that $f$ is \imp\ with respect to $\mu$ and $\nu$, and that $f$
represents an isomorphism between the measure algebras of $\mu$ and
$\nu$.   Use this construction to prove (vi)$\Rightarrow$(i) in Theorem
343B without appealing to the Lifting Theorem.
%413J

\spheader 413Yd Let $X$ be a set, $\Tau$ a subalgebra of $\Cal PX$, and
$\nu:\Tau\to\coint{0,\infty}$ a finitely additive functional.   Suppose
that there is a set $\Cal K\subseteq\Tau$, containing $\emptyset$, such
that (i) $\mu F=\sup\{\mu K:K\in\Cal K,\,K\subseteq F\}$ for every
$F\in\Tau$ (ii) $\Cal K$ is {\bf monocompact}, that is,
$\bigcap_{n\in\Bbb N}K_n\ne\emptyset$ for every non-increasing sequence
in $\Cal K$.   Show that $\nu$ extends to a measure on $X$.
%413K

\spheader 413Ye(i) Let $X$ be a topological space.   Show that the
family of closed countably compact subsets of $X$ is a countably compact
class.   (ii) Let $X$ be a Hausdorff space.   Show that the family of
sequentially compact subsets of $X$ is a countably compact class.
%413L

\spheader 413Yf\dvAnew{2008}
Let $\frak A$ be a Boolean algebra and $\nu$ a totally finite
submeasure on $\frak A$ which is {\it either} supermodular
{\it or} exhaustive and submodular.
Show that $\nu$ is uniformly exhaustive.
%413P  n04624  1K

\spheader 413Yg\dvAnew{2008}(i)
Find a measure space $(X,\Sigma,\mu)$, with $\mu X>0$,
and a sequence
$\sequencen{X_n}$ of subsets of $X$, covering $X$, such that
whenever $E\in\Sigma$, $n\in\Bbb N$ and $\mu E>0$, there is an $F\in\Sigma$
such that $F\subseteq E\setminus X_n$ and $\mu F=\mu E$.
(ii) For $A\subseteq X$ set
$\phi A=\sup\{\mu E:E\in\Sigma$, $E\subseteq A\}$.   Set
$\Tau=\{G:G\subseteq X$, $\phi A=\phi(A\cap G)+\phi(A\setminus G)$ for
every $A\subseteq X\}$.   Show that $\phi\restr\Tau$ is not a measure.
%413Xj 413I out of order query mt41bits

\spheader 413Yh\dvAnew{2010} Let $X$ be a set, $\Cal K$ a sublattice of
$\Cal PX$ containing $\emptyset$, and $f:\Cal K\to\Bbb R$ a modular
functional such that $f(\emptyset)=0$.   Show that there is an additive
functional $\nu:\Cal K\to\Bbb R$ extending $f$.
%413Xq 413P mt41bits
}%end of exercises

\cmmnt{\Notesheader{413}
I gave rather few methods of constructing measures in the first three
volumes of this treatise;  in the present volume I shall have to make up
for lost time.   In particular I used \Caratheodory's construction
for Lebesgue measure (Chapter 11), product measures (Chapter 25) and
Hausdorff measures (Chapter 26).   The first two, at least, can be
tackled in quite different ways if we choose.   The first alternative
approach I offer is the `inner measure' method of 413C.   Note the exact
definition in 413A;  I do not think it is an obvious one.   In
particular, while ($\alpha$) seems to have something to do with
subadditivity, and ($\beta$) is a kind of sequential order-continuity,
there is no straightforward way in which to associate an outer measure
with an inner measure, unless they both happen to be derived from
measures (132B, 413D), even when they are finite-valued;  and for an
inner measure which
is allowed to take the value $\infty$ we have to add the semi-finiteness
condition ($*$) of 413A (see 413Xa).

Once we have got these points right, however, we have a method which
rivals \Caratheodory's in scope, and in particular is especially well
adapted to the construction of inner regular measures.   As an almost
trivial example, we have a route to the c.l.d.\ version of a measure
$\mu$ (413Xj(i)), which can be derived from the inner measure $\mu_*$
defined from
$\mu$ (413D).   Henceforth $\mu_*$ will be a companion to the familiar
outer measure $\mu^*$, and many calculations will be a little easier
with both available, as in 413E-413F.

The intention behind 413I-413J is to find a minimal set of properties of
a functional $\phi_0$ which will ensure that it has an extension to a
measure. Indeed it is easy to see that, in the context of 413I, given a
family $\Cal K$ with the properties ($\dagger$) and ($\ddagger$) there,
a functional $\phi_0$ on $\Cal K$ can have an extension to an inner
regular measure iff it satisfies the conditions ($\alpha$) and
($\beta$), so in
this sense 413I is the best possible result.   Note that while
\Caratheodory's construction is liable to produce wildly infinite
measures (like Hausdorff measures, or primitive product measures), the
construction here always gives us locally determined measures, provided
only that $\phi_0$ is finite-valued.

We have to work rather hard for the step from 413I to 413J.   Of course
413I is a special case of 413J, and I could have saved a little space by
giving a direct proof of the latter result.   But I do not think that
this would have made it easier;  413J really does require an extra step,
because somehow we have to extend the functional $\phi_0$ from $\Cal K$
to $\Cal K_{\delta}$.   The method I have chosen uses 413B and 413H to
cast as much of the argument as possible into the context of algebras of
sets with additive functionals, where I hope the required manipulations
will seem natural.   (But perhaps I should insist that you must not take
them too much
for granted, as some of the time we have a finitely additive
functional taking infinite values, and must take care not to
subtract illegally, as well as not to take limits in the wrong places.)
Note that the progression $\phi_0\to\phi_1\to\mu$ in the proof of 413J
involves first an approximation from outside (if $K\in\Cal K_{\delta}$,
then $\phi_1K$ will be $\inf\{\phi_0K':K\subseteq K'\in\Cal K\}$) and
then an approximation from inside (if $E\in\Sigma$, then $\mu
E=\sup\{\phi_1K:K\in\Cal K_{\delta},\,K\subseteq E\}$).   The essential
difficulty in the proof is just that we have to take successive
non-exchangeable limits.
I have slipped 413K in as a corollary of 413J;  but it can be regarded
as one of the fundamental results of measure theory.   A non-negative
finitely additive functional $\nu$ on an algebra $\Sigma$ of sets can be
extended to
a countably additive measure iff it is `relatively countably additive'
in the sense
that $\nu(\bigcup_{n\in\Bbb N}E_n)=\sum_{n=0}^{\infty}\mu E_n$ whenever
$\sequencen{E_n}$ is a disjoint sequence in $\Sigma$ such that
$\bigcup_{n\in\Bbb N}E_n\in\Sigma$ (413Xn).   Of course the same result
can easily be got from an outer measure construction (413Xm).   Note
that the outer measure construction also has repeated limits, albeit
simpler ones:  in the formula

\Centerline{$\theta A=\inf\{\sum_{n=0}^{\infty}\nu E_n:\sequencen{E_n}$
is a sequence in $\Sigma$ covering $A\}$}

\noindent the sum $\sum_{n=0}^{\infty}\nu E_n=\sup_{n\in\Bbb
N}\sum_{i=0}^n\nu E_i$ can be regarded as a crude approximation from
inside, while the infimum
is an approximation from outside.   To get the result as stated in 413K,
of course, the outer measure construction needs a third limiting
process, to
obtain the c.l.d.\ version automatically provided by the inner measure
method, and the inner regularity with respect to $\Sigma_{\delta}$,
while easily checked, also demands a few words of argument.

Many applications of the method of 413I-413J pass through 413M;  if the
family $\Cal K$ is a countably compact class then the sequential
order-continuity hypothesis ($\beta$) of 413I or 413J becomes a
consequence
of the other hypotheses.   The essence of the method is the inner
regularity hypothesis ($\alpha$).   I have tried to use the labels
$\dagger$, $\ddagger$, $\alpha$ and $\beta$ consistently enough to
suggest the currents which I think are flowing in this material.

In 413N we strike out in a new direction.   The object here is to build
an extension which is not going to be unique, and for which choices will
have to be made.   As with any such argument, the trick is to specify
the allowable intermediate stages, that is, the partially ordered set
$P$ to which we shall apply Zorn's Lemma.   But here the form of the
theorem makes
it easy to guess what $P$ should be:  it is the set of functionals
satisfying the hypotheses of the theorem which have not wandered outside
the boundary set by the conclusion, that is, which satisfy the condition
($*$) of part (a) of the proof of 413N.   The finitistic nature of the
hypotheses makes it easy to check that totally ordered subsets of $P$
have upper bounds
(that is to say, if we did this by transfinite induction there would be
no problem at limit stages), and all we have to prove is that maximal
elements of $P$ are defined on adequately large domains;  which amounts
to showing
that a member of $P$ not defined on every element of $\Cal K$ has a
proper extension, that is, setting up a construction for the step to a
successor ordinal in the parallel transfinite induction (part (c) of the
proof).

Of course the principal applications of 413N in this book will be in the
context of countably additive functionals, as in 413O.

It is clear that 413N and 413Q overlap to some extent.   I include both
because they have different virtues.   413N provides actual extensions of
functionals in a way that 413Q, as given, does not;
but its chief advantage,
from the point of view of the work to come, is the approximation of
members of $\Tau_1$, in measure, by members of $\Tau_0$.   This will
eventually enable us to retain control of the Maharam types of measures
constructed by
the method of 413O.   In 413S we have a different kind of control;  we
can specify a lower bound for the measure of each member of our basic
class $\Cal K$, provided only that our specifications are consistent
with some {\it finitely} additive functional.

}%end of notes

\discrpage

