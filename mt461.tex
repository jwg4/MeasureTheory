\frfilename{mt461.tex}
\versiondate{9.7.08}
\copyrightdate{2000}

\def\chaptername{Pointwise compact sets of measurable functions}
\def\sectionname{Barycenters and Choquet's theorem}

\newsection{461}

One of the themes of this chapter will be the theory of measures on
linear spaces, and the first fundamental concept is that of `barycenter'
of a measure, its centre of mass (461Aa).   The elementary theory
(461B-461E) uses non-trivial results from the theory of locally convex
spaces\cmmnt{ (\S4A4)}, but is otherwise natural and straightforward.
It is not always easy to be sure whether a measure has a barycenter in a
given space, and I give a representative pair of results in this
direction (461F, 461H).   Deeper questions concern the existence and
nature of measures on a given compact set with a given barycenter.   The
Riesz representation theorem is enough to tell us just which points can
be barycenters of measures on compact sets (461I).   A new idea (461K-461L)
shows that the measures can be moved out towards the boundary of the
compact set.   We need a precise definition of `boundary';  the set of
extreme points seems to be the appropriate concept (461M).
In some important cases, such representing measures on boundaries are
unique (461P).   I append a result
identifying the extreme points of a particular class of compact convex sets
of measures (461Q-461R).


\leader{461A}{Definitions (a)} Let $X$ be a Hausdorff locally convex
linear topological space, and $\mu$ a probability measure on a subset
$A$ of $X$.   Then $x^*\in X$ is a {\bf barycenter}\cmmnt{ or
{\bf resultant}} of $\mu$ if
$\int_Ag\,d\mu$ is defined and equal to $g(x^*)$ for every $g\in X^*$.
\cmmnt{Because $X^*$ separates the points of $X$ (4A4Ec),} $\mu$ can
have at most one barycenter\cmmnt{, so we may speak of `the' barycenter
of $\mu$}.

\spheader 461Ab Let $X$ be any linear space over $\Bbb R$,
and $C\subseteq X$ a
convex set\cmmnt{ (definition:  2A5E)}.   Then a function
$f:C\to\Bbb R$ is {\bf convex} if
$f(tx+(1-t)y)\le tf(x)+(1-t)f(y)$ for all $x$,
$y\in C$ and $t\in[0,1]$.   \cmmnt{ (Compare 233G, 233Xd.)}

\spheader 461Ac\cmmnt{ The following elementary remark is useful.}
Let $X$ be a linear space over $\Bbb R$, $C\subseteq X$ a convex set,
and $f:C\to\Bbb R$ a function.   Then $f$ is convex iff the set
$\{(x,\alpha):x\in C$, $\alpha\ge f(x)\}$ is convex in
$X\times\Bbb R$\cmmnt{ (cf.\ 233Xd)}.

\vleader{72pt}{461B}{Proposition} Let $X$ and $Y$ be Hausdorff locally convex
linear topological spaces, and $T:X\to Y$ a continuous linear operator.
Suppose that $A\subseteq X$, $B\subseteq Y$ are such that
$T[A]\subseteq B$, and let $\mu$ be a probability measure on $A$ which
has a barycenter
$x^*$ in $X$.   Then $Tx^*$ is the barycenter of the image measure
$\mu T^{-1}$ on $B$.

\proof{ All we have to observe is that if $g\in Y^*$ then $gT\in X^*$
(4A4Bd), so that

\Centerline{$g(Tx^*)=\int_Ag(Tx)\mu(dx)=\int_Bg(y)\nu(dy)$}

\noindent by 235G\formerly{2{}35I}.
}%end of proof of 461B

\leader{461C}{Lemma} Let $X$ be a Hausdorff locally convex linear
topological space, $C$ a convex subset of $X$, and $f:C\to\Bbb R$ a lower
semi-continuous convex function.   If $x\in C$ and $\gamma<f(x)$, there
is a $g\in X^*$ such that
$g(y)+\gamma-g(x)\le f(y)$ for every $y\in C$.

\proof{ Let $D$ be the convex set
$\{(x,\alpha):x\in C$, $\alpha\ge f(x)\}$ in $X\times\Bbb R$ (461Ac).
Then the closure $\overline{D}$ of $D$ in $X\times\Bbb R$
is also convex (2A5Eb).   Now
$D$ is closed in $C\times\Bbb R$ (4A2B(d-i)), and $(x,\gamma)\notin D$,
so $(x,\gamma)\notin\overline{D}$.

Consequently there is a continuous linear functional
$h:X\times\Bbb R\to\Bbb R$ such that
$h(x,\gamma)<\inf_{w\in\overline{D}}h(w)$
(4A4Eb).   Now there are $g_0\in X^*$, $\beta\in\Bbb R$ such that
$h(y,\alpha)=g_0(y)+\beta\alpha$ for every $y\in X$ and $\alpha\in\Bbb R$
(4A4Be).   So we have

\Centerline{$g_0(x)+\beta\gamma=h(x,\gamma)
<h(y,f(y))=g_0(y)+\beta f(y)$}

\noindent for every $y\in C$.   In particular,
$g_0(x)+\beta\gamma<g_0(x)+\beta f(x)$, so $\beta>0$.   Setting
$g=-\bover1{\beta}g_0$,

\Centerline{$f(y)\ge\bover1{\beta}g_0(x)+\gamma-\bover1{\beta}g(y)
=g(y)+\gamma-g(x)$}

\noindent for every $y\in C$, as required.
}%end or proof of 461C

\leader{461D}{Theorem} Let $X$ be a Hausdorff locally convex linear
topological space,
$C\subseteq X$ a convex set and $\mu$ a probability measure
on a subset $A$ of $C$.   Suppose that $\mu$ has a barycenter $x^*$ in
$X$ which belongs to $C$.
Then $f(x^*)\le\underline{\int}_Af\,d\mu$ for every
lower semi-continuous convex function $f:C\to\Bbb R$.

\proof{ Take any $\gamma<f(x^*)$.   By 461C there
is a $g\in X^*$ such that
$g(y)+\gamma-g(x^*)\le f(y)$ for every $y\in C$.
Integrating with respect to $\mu$,

\Centerline{$\underline{\intop}_Afd\mu
\ge\gamma-g(x^*)+\int_Ag\,d\mu=\gamma$.}

\noindent As $\gamma$ is arbitrary, $f(x^*)\le\underline{\int}_Af\,d\mu$.
}%end of proof of 461D

\cmmnt{\medskip

\noindent{\bf Remark} Of course $\underline{\int}_Af\,d\mu$ might be infinite.
}%end of comment

\leader{461E}{Theorem} Let $X$ be a Hausdorff locally convex linear
topological space, and $\mu$ a probability measure on $X$ such that (i)
the domain of $\mu$ includes the cylindrical $\sigma$-algebra of $X$
(ii) there is a compact convex set $K\subseteq X$ such that $\mu^*K=1$.
Then $\mu$ has a barycenter in $X$, which belongs to $K$.

\proof{ If $g\in X^*$, then $\gamma_g=\sup_{x\in K}|g(x)|$ is finite,
and $\{x:|g(x)|\le\gamma_g\}$ is a measurable set including $K$, so must
be conegligible, and $\phi(g)=\int g\,d\mu$ is defined and finite.   Now
$\phi:X^*\to\Bbb R$ is a linear functional and
$\phi(g)\le\sup_{x\in K}g(x)$ for every $g\in X^*$;  because $K$ is
compact and convex, there is an $x_0\in K$ such that $\phi(g)=g(x_0)$
for every $g\in X^*$ (4A4Ef), so that $x_0$ is the barycenter of $\mu$ in
$X$.
}%end of proof of 461E

\leader{461F}{Theorem} Let $X$ be a complete locally convex
linear topological space, and $A\subseteq X$ a bounded set.   Let $\mu$
be a $\tau$-additive topological probability measure on $A$.   Then
$\mu$ has a barycenter in $X$.

\proof{{\bf (a)} If $g\in X^*$, then $g\restr A$ is continuous and
bounded (3A5N(b-v)), therefore $\mu$-integrable.   For each
neighbourhood $G$ of $0$ in $X$, set

\Centerline{$F_G
=\{y:y\in X,\,|g(y)-\int_Ag\,d\mu|\le 2\tau_G(g)$ for every
$g\in X^*\}$}

\noindent where $\tau_G(g)=\sup_{x\in G}g(x)$ for $g\in X^*$.   Then
$F_G$ is non-empty.   \Prf\ Set
$H=\interior(G\cap(-G))$, so that $H$ is an open neighbourhood of $0$.
Because $A$ is bounded, there is an $m\ge 1$ such that
$A\subseteq mH$.   The set $\{x+H:x\in A\}$ is an open cover of $A$, and
$\mu$ is a $\tau$-additive topological measure, so there are
$x_0,\ldots,x_n\in A$ such that
$\mu(A\setminus\bigcup_{i\le n}(x_i+H))\le\Bover1m$.   Set

\Centerline{$E_i=A\cap(x_i+H)\setminus\bigcup_{j<i}(x_j+H)$}

\noindent for $i\le n$, and $y=\sum_{i=0}^n(\mu E_i)x_i$;  set
$E=A\setminus\bigcup_{i\le n}(x_i+H)$.   Then, for any
$g\in X^*$,

$$\eqalignno{|g(y)-\int_Ag\,d\mu|
&\le\sum_{i=0}^n|\mu E_ig(x_i)-\int_{E_i}g\,d\mu|+\int_E|g|d\mu\cr
&\le\sum_{i=0}^n\int_{E_i}|g(x)-g(x_i)|\mu(dx)+m\tau_G(g)\mu E\cr
\displaycause{because $E\subseteq mH$, so $g(x)\le m\tau_G(g)$ and
$g(-x)\le m\tau_G(g)$ for every $x\in E$}
&\le\sum_{i=0}^n\tau_G(g)\mu E_i+m\tau_G(g)\mu E\cr
\displaycause{because if $i\le n$ and $x\in E_i$, then $x-x_i$ and
$x_i-x$ belong to $G$, so $|g(x)-g(x_i)|=|g(x-x_i)|\le\tau_G(g)$}
&\le 2\tau_G(g).\cr}$$

\noindent As $g$ is arbitrary, $y\in F_G$ and $F_G\ne\emptyset$.\ \Qed

\medskip

{\bf (c)} Since $F_{G\cap H}\subseteq F_G\cap F_H$ for all
neighbourhoods $G$ and $H$ of $0$, $\{F_G:G$ is a neighbourhood of $0\}$
is a filter base and generates a filter $\Cal F$ on $X$.   Now $\Cal F$
is Cauchy.   \Prf\ If $G$ is any neighbourhood of $0$, let
$G_1\subseteq G$ be a closed convex neighbourhood of $0$, and set
$H=\bover14G_1$.   \Quer\ If $y$, $y'\in F_H$ and $y-y'\notin G$, then
there is a $g\in X^*$ such that $g(y-y')>\tau_{G_1}(g)$ (4A4Eb again).   
But now

$$\eqalign{\tau_H(g)
&=\Bover14\tau_{G_1}(g)
<\Bover14(|g(y)-\int_Ag\,d\mu|+|g(y')-\int_Ag\,d\mu|)\cr
&\le\Bover14(2\tau_H(g)+2\tau_H(g)). \text{ \Bang}\cr}$$

\noindent This means that $F_H-F_H\subseteq G$;  as $G$ is arbitrary,
$\Cal F$ is Cauchy.\ \Qed

\medskip

{\bf (d)} Because $X$ is complete, $\Cal F$ has a limit $x^*$ say.
Take any $g\in X^*$.   \Quer\ If $g(x^*)\ne\int_Ag\,d\mu$, set
$G=\{x:|g(x)|\le\bover13|g(x^*)-\int_Ag\,d\mu|\}$.   Then $G$ is a
neighbourhood of $0$ in $X$, and

$$\eqalign{0
&<|g(x^*)-\int_Ag\,d\mu|
=\lim_{x\to\Cal F}|g(x)-\int_Ag\,d\mu|\cr
&\le\sup_{x\in F_G}|g(x)-\int_Ag\,d\mu|
\le 2\tau_G(g)
\le\Bover23|g(x^*)-\int_Ag\,d\mu|.  \text{ \Bang}\cr}$$

\noindent So $g(x^*)=\int_Ag\,d\mu$;  as $g$ is arbitrary, $x^*$ is the
barycenter of $\mu$.
}%end of proof of 461F

\leader{461G}{Lemma} Let $X$ be a normed space, and $\mu$ a probability
measure on $X$ such that every member of the dual $X^*$ of $X$ is
integrable.   Then $g\mapsto\int g\,d\mu:X^*\to\Bbb R$ is a bounded linear
functional on $X^*$.

\proof{ Replacing $\mu$ by its completion if necessary, we may suppose
that $\mu$ is complete, so that every member of $X^*$ is
$\Sigma$-measurable, where $\Sigma$ is the domain of $\mu$.   (The point
is that $\mu$ and its completion give rise to the same integrals, by
212Fb.)    Set $\phi(g)=\int g\,d\mu$ for $g\in X^*$.
For each $n\in\Bbb N$ let $E_n$ be a measurable envelope of
$B_n=\{x:\|x\|\le n\}$;  replacing $E_n$ by $\bigcap_{i\ge n}E_i$ if
necessary, we may suppose that $\sequencen{E_n}$ is non-decreasing.   If
$n\in\Bbb N$ and $g\in X^*$ then $\{x:x\in E_n,\,|g(x)|>n\|g\|\}$ is a
measurable subset of $E_n$ disjoint from $B_n$, so must be negligible,
and $|\int_{E_n}g|\le n\|g\|$.   We therefore have an element $\phi_n$
of $X^{**}$ defined by setting $\phi_n(g)=\int_{E_n}g$ for every $g\in
X^*$.   But also $\phi(g)=\lim_{n\to\infty}\phi_n(g)$ for every $g$,
because $\sequencen{E_n}$ is a non-decreasing sequence of measurable
sets with union $X$.   By the Uniform Boundedness Theorem (3A5Ha),
$\{\phi_n:n\in\Bbb N\}$ is bounded in $X^{**}$, and $\phi\in X^{**}$.
}%end of proof of 461G

\leader{461H}{Proposition} Let $X$ be a reflexive Banach space, and
$\mu$ a probability measure on $X$ such that every member of $X^*$ is
$\mu$-integrable.   Then $\mu$ has a barycenter in $X$.

\proof{ By 461G, $g\mapsto\int g\,d\mu$ is a bounded linear functional on
$X^*$;  but this means that it is represented by a member of $X$, which
is the barycenter of $\mu$.
}%end of proof of 461H

\leader{461I}{Theorem} Let $X$ be a Hausdorff locally convex linear
topological space, and $K\subseteq X$ a compact set.   Then the closed
convex hull of $K$ in $X$ is just the set of barycenters of Radon
probability measures on $K$.

\proof{{\bf (a)} If $\mu$ is a Radon probability measure on $K$ with
barycenter $x^*$, then

\Centerline{$g(x^*)=\int_Kg(x)\mu(dx)
\le\sup_{x\in K}g(x)\le\sup_{x\in\overline{\Gamma(K)}}g(x)$}

\noindent for every $g\in X^*$;
because $\overline{\Gamma(K)}$ is closed and convex, it must
contain $x^*$ (4A4Eb once more).

\medskip

{\bf (b)} Now suppose that $x^*\in\overline{\Gamma(K)}$.   Let
$W\subseteq C(X)$ be the set of functionals of the form
$g+\alpha\chi X$, where $g\in X^*$ and $\alpha\in\Bbb R$.   Set
$U=\{g\restr K:g\in W\}$, so that $U$ is a linear subspace of $C(K)$
containing $\chi K$.

If $g_1$, $g_2\in W$ and $g_1\restr K=g_2\restr K$, then
$\{x:g_1(x)=g_2(x)\}$ is a closed convex set including $K$, so contains
$x^*$, and $g_1(x^*)=g_2(x^*)$;  accordingly we have a functional
$\phi:U\to\Bbb R$ defined by setting $\phi(g\restr K)=g(x^*)$ for every
$g\in W$.   Of course $\phi$ is linear;  moreover,
$\phi(f)\le\sup_{x\in K}f(x)$ for every $f\in U$, by 4A4Eb
yet again.   Applying this to $\pm f$, we see
that $|\phi(f)|\le\|f\|_{\infty}$ for every $f\in U$.   We therefore
have an extension of $\phi$ to a continuous linear functional $\psi$ on
$C(K)$ such that $\|\psi\|\le 1$, by the Hahn-Banach theorem (3A5Ab).   Now

\Centerline{$\psi(\chi K)=\phi(\chi K)=\chi X(x^*)=1$;}

\noindent so if $0\le f\le\chi K$ then

\Centerline{$|1-\psi(f)|
=|\psi(\chi K-f)|\le \|\chi K-f\|_{\infty}\le 1$,}

\noindent and $\psi(f)\ge 0$.   It follows that $\psi(f)\ge 0$ for every
$f\in C(K)^+$.   But this means that there is a Radon probability
measure $\mu$ on
$K$ such that $\psi(f)=\int f\,d\mu$ for every $f\in C(K)$ (436J/436K).
As $\mu K=\psi(\chi K)=1$, $\mu$ is a probability measure;  and for any
$g\in X^*$

\Centerline{$\int_Kg\,d\mu=\psi(g\restr K)=\phi(g\restr K)=g(x^*)$,}

\noindent so $x^*$ is the barycenter of $\mu$, as required.
}%end of proof of 461I

\leader{461J}{Corollary:  \Krein's theorem} Let $X$ be a complete
Hausdorff locally convex linear topological space, and $K\subseteq X$ a
weakly compact set.   Then the closed convex hull $\overline{\Gamma(K)}$
of $K$ is weakly compact.

\proof{ Give $K$ the weak topology induced by $\frak T_s(X,X^*)$.   Let
$P$ be the set of Radon probability measures on $K$,
so that $P$ is compact in its narrow topology (437R(f-ii)).
By 461F, every $\mu\in P$ has a barycenter $b(\mu)$ in
$K$.   If $g\in X^*$, $g(b(\mu))=\int_Kg\,d\mu$, while $g\restr K$ is
continuous, so $\mu\mapsto\int_Kg\,d\mu$ is continuous, by 437Kc.
Accordingly $b:P\to X$ is continuous for the narrow topology on $P$
and the weak topology on $X$, and $b[P]$ is weakly compact.
But $b[P]$ is the weakly closed convex hull of $K$, by 461I applied to
the weak topology on $X$.   By 4A4Ed, $\Gamma(K)$ has the same closure
for the original topology of $X$ as it has for the weak topology, and
$\overline{\Gamma(K)}=b[P]$ is weakly compact.
}%end of proof of 461J

\leader{461K}{Lemma} Let $X$ be a Hausdorff locally convex linear
topological space, $K$ a compact convex subset of $X$, and $P$ the set of
Radon probability measures on $K$.   Define a relation $\preccurlyeq$ on
$P$ by saying that $\mu\preccurlyeq\nu$ if
$\int fd\mu\le\int fd\nu$ for every continuous convex function
$f:K\to\Bbb R$.

(a) $\preccurlyeq$ is a partial order on $P$.

(b) If $\mu\preccurlyeq\nu$ then
$\int fd\mu\le\int fd\nu$ for every lower semi-continuous convex function
$f:K\to\Bbb R$.

(c) If $\mu\preccurlyeq\nu$ then $\mu$ and $\nu$ have the same barycenter.

(d) If we give $P$ its narrow topology, then $\preccurlyeq$ is closed in
$P\times P$.

(e) For every $\mu\in P$ there is a $\preccurlyeq$-maximal $\nu\in P$
such that $\mu\preccurlyeq\nu$.

\proof{{\bf (a)} Write $\Psi$ for the set of continuous convex functions
from $K$ to $\Bbb R$.   Note that if $f$, $g\in\Psi$ and $\alpha\ge 0$ then
$\alpha f$, $f+g$ and $f\vee g$ all belong to $\Psi$.   Consequently
$\Psi-\Psi$ is a Riesz subspace of $C(K)$.   \Prf\ $\Psi-\Psi$ is a linear
subspace because $\Psi$ is closed under addition and multiplication by
positive scalars.   If $f$, $g\in\Psi$ then

\Centerline{$|f-g|=(f-g)\vee(g-f)=2(f\vee g)-(f+g)\in\Psi$;}

\noindent by 352Ic, $\Psi-\Psi$ is a Riesz subspace.\ \Qed

It follows that $\Psi-\Psi$ is $\|\,\|_{\infty}$-dense in $C(K)$.   \Prf\
Constant functions belong to $\Psi$, and if $x$, $y\in K$ are distinct
there is an $f\in X^*$ such that $f(x)\ne f(y)$, in which case
$f\restr K$ belongs to $\Psi$ and separates $y$ from $x$.   By the
Stone-Weierstrass theorem (281A), $\Psi-\Psi$ is dense.\ \Qed

The definition of $\preccurlyeq$ makes it plain that
it is reflexive and transitive.   But it is also antisymmetric.
\Prf\ If $\mu\preccurlyeq\nu$ and $\nu\preccurlyeq\mu$, then
$\int fd\mu=\int fd\nu$ for every $f\in\Psi$, therefore for every
$f\in\Psi-\Psi$, therefore for every $f\in C(K)$, and $\mu=\nu$
by 416E(b-v).\ \Qed

So $\preccurlyeq$ is a partial order.

\medskip

{\bf (b)(i)}
Now suppose that $f:K\to\Bbb R$ is a lower semi-continuous convex
function, and $x\in K$.
Then $f(x)=\sup\{g(x):g\in\Psi$, $g\le f\}$.   \Prf\ If $\gamma<f(x)$ there
is a $g\in X^*$ such that
$g(y)+\gamma-g(x)\le f(y)$ for every $y\in K$, by 461C.   Now $g\restr K$
belongs to $\Psi$, $g\restr K\le f$ and $(g\restr K)(x)=\gamma$.\ \Qed

\medskip

\quad{\bf (ii)} It follows that if $f:K\to\Bbb R$ is
lower semi-continuous and
convex, $\int fd\mu=\sup\{\int g\,d\mu:g\in\Psi$, $g\le f\}$ for every
$\mu\in P$.   \Prf\ Because $\Psi$ is closed under $\vee$,
$A=\{g:g\in\Psi$, $g\le f\}$ is upwards-directed.   Because $\mu$ is
$\tau$-additive and $f=\sup A$, $\int fd\mu=\sup_{g\in A}\int g\,d\mu$ by
414Ab.\ \Qed

So if $\mu$, $\nu\in P$ and $\mu\preccurlyeq\nu$, then

\Centerline{$\int fd\mu
=\sup_{g\in\Psi,g\le f}\int g\,d\mu
\le\sup_{g\in\Psi,g\le f}\int g\,d\nu
=\int fd\nu$;}

\noindent as $f$ is arbitrary, (b) is true.

\medskip

{\bf (c)} By 461E, applied to the Radon probability measure on $X$
extending $\mu$, $\mu$ has a barycenter $x\in K$.   If $g\in X^*$ then
$g\restr K$ belongs to $\Psi$, so $\int_Kg\,d\mu\le\int_Kg\,d\nu$;  but the
same applies to $-g$, so

\Centerline{$g(x)=\int_Kg\,d\mu=\int_Kg\,d\nu$.}

\noindent As $g$ is arbitrary, $x$ is also the barycenter of $\nu$.

\medskip

{\bf (d)} As noted in 437Kc, the narrow topology on $P$ corresponds to the
weak* topology on $C(K)^*$.   So all the functionals
$\mu\mapsto\int fd\mu$, for $f\in\Psi$, are continuous;  it follows at once
that $\preccurlyeq$ is a closed subset of $P\times P$.

\medskip

{\bf (e)} Any non-empty
upwards-directed $Q\subseteq P$ has an upper bound in $P$.   \Prf\ For
$\nu\in Q$ set $V_{\nu}=\{\lambda:\nu\preccurlyeq\lambda\}$.   Then every
$V_{\nu}$ is closed, and because $Q$ is upwards-directed the family
$\{V_{\nu}:\nu\in Q\}$ has the finite intersection property.   Because $P$
is compact (437R(f-ii) again), $\bigcap_{\nu\in Q}V_{\nu}$ is non-empty;
now any member of the
intersection is an upper bound of $Q$.\ \Qed

By Zorn's lemma, every member of $P$ is dominated by a maximal element of
$P$.
}%end of proof of 461K

\leader{461L}{Lemma} Let $X$ be a Hausdorff locally convex linear
topological space, $K$ a compact convex subset of $X$, and $P$ the set of
Radon probability measures on $K$.   Suppose that
$\mu\in P$ is maximal for the partial order $\preccurlyeq$ of 461K.

(a) $\mu(\bover12(M_1+M_2))=0$ whenever $M_1$, $M_2$ are disjoint closed
convex subsets of $K$.

(b) $\mu F=0$ whenever $F\subseteq K$ is a Baire set (for the
subspace topology of $K$) not containing any extreme point of $K$.

\proof{{\bf (a)} Set $M=\{\bover12(x+y):x\in M_1$, $y\in M_2\}$.
Set $q(x,y)=\bover12(x+y)$ for $x\in M_1$, $y\in M_2$, so that
$q:M_1\times M_2\to M$ is continuous.   Let $\mu_M$ be the
subspace measure on
$M$ induced by $\mu$, so that $\mu_M$ is a Radon measure on
$M$ (416Rb).   Let $\lambda$ be a Radon measure on $M_1\times M_2$ such
that $\mu_M=\lambda q^{-1}$ (418L), and define
$\psi:C(K)\to\Bbb R$ by writing

\Centerline{$\psi(f)
=\int_{K\setminus M}f\,d\mu
  +\int_{M_1\times M_2}\bover12(f(x)+f(y))\lambda(d(x,y))$.}

\noindent Then $\psi$ is linear, $\psi(f)\ge 0$ whenever $f\in C(K)^+$,
and

\Centerline{$\psi(\chi K)
=\mu(K\setminus M)+\lambda(M_1\times M_2)
=\mu(K\setminus M)+\mu_M(M)=1$.}

Let $\nu\in P$ be such that $\int fd\nu=\psi(f)$ for every $f\in C(K)$
(436J/436K again).   If $f:K\to\Bbb R$ is continuous and convex, then

$$\eqalignno{\int fd\nu
=\psi(f)
&=\int_{K\setminus M}f\,d\mu
  +\int_{M_1\times M_2}\Bover12(f(x)+f(y))\lambda(d(x,y))\cr
&\ge\int_{K\setminus M}f\,d\mu
  +\int_{M_1\times M_2}f(\Bover12(x+y))\lambda(d(x,y))\cr
&=\int_{K\setminus M}f\,d\mu
  +\int_{M_1\times M_2}fq\,d\lambda
=\int_{K\setminus M}f\,d\mu
  +\int_Mf\,d\mu_M\cr
\noalign{\noindent (235G)}
&=\int_Kf\times\chi(K\setminus M)d\mu
  +\int_Kf\times\chi M\,d\mu\cr
\noalign{\noindent (131Fa)}
&=\int_Kf\,d\mu.\cr}$$

\noindent So $\mu\preccurlyeq\nu$;  as we are assuming that $\mu$ is
maximal, $\mu=\nu$.

Because $M_1$ and $M_2$ are disjoint compact convex sets in
the Hausdorff locally convex space $X$, there are a $g\in X^*$ and an
$\alpha\in\Bbb R$ such that $g(x)<\alpha<g(y)$ whenever $x\in M_1$ and
$y\in M_2$ (4A4Ee).   Set
$f(x)=|g(x)-\alpha|$ for $x\in K$;  then $f$ is a continuous convex
function.   If $x\in M_1$ and $y\in M_2$, then

$$\eqalign{f(\Bover12(x+y))
&=|g(\Bover12(x+y))-\alpha|
=\Bover12|(g(x)-\alpha)+(g(y)-\alpha)|\cr
&<\Bover12(|g(x)-\alpha|+|g(y)-\alpha|)
=\Bover12(f(x)+f(y)).\cr}$$

\noindent Looking at the formulae above for $\psi(f)$, we see that we
have

\Centerline{$\int f\,d\nu
=\int_{K\setminus M}f\,d\mu
  +\int_{M_1\times M_2}\Bover12(f(x)+f(y))\lambda(d(x,y))$,}

\Centerline{$\int f\,d\mu
=\int_{K\setminus M}f\,d\mu
  +\int_{M_1\times M_2}f(\Bover12(x+y))\lambda(d(x,y))$.}

\noindent Since these are equal, and $f(\bover12(x+y))<\bover12(f(x)+f(y))$
for all $x\in M_1$ and $y\in M_2$, we must have
$\mu M=\lambda(M_1\times M_2)=0$, as required.

\medskip

{\bf (b)(i)} Consider first the case in which $F$ is a zero set for the
subspace topology.   Since $F\subseteq K$ is a closed
G$_{\delta}$ set in $K$, $K\setminus F$ is expressible as a union
$\bigcup_{n\in\Bbb N}F_n$ of compact sets.   For any $n\in\Bbb N$,
$z\in F$ and $y\in F_n$, there is a $g\in X^*$ such that $g(z)\ne g(y)$;
since $F$ and $F_n$ are compact, there is a finite set
$\Phi_n\subseteq X^*$ such that whenever $z\in F$ and $y\in F_n$
there is a $g\in\Phi_n$ such that $g(z)\ne g(y)$.   Set
$\Phi=\bigcup_{n\in\Bbb N}\Phi_n\cup\{0\}$;  then
$\Phi$ is countable;  let $\sequencen{g_n}$ be a sequence running over
$\Phi$, and define $T:X\to\BbbR^{\Bbb N}$ by setting $(Tx)(n)=g_n(x)$
for $n\in\Bbb N$, $x\in X$.   Then $T[F]\cap T[F_n]=\emptyset$ for every
$n$, so $F=K\cap T^{-1}[T[F]]$.

Now $T[F]$ is a compact subset of the metrizable
compact convex set $T[K]$,
and does not contain any extreme point of $T[K]$, by 4A4Gc.

Let $\Cal U$ be the set of convex open subsets of
$\BbbR^{\Bbb N}$.   Because the topology of $\BbbR^{\Bbb N}$ is
locally convex, $\Cal U$ is a base
for the topology of $\BbbR^{\Bbb N}$;  because it is separable and
metrizable, $\Cal U$ includes a countable base $\Cal U_0$ (4A2P(a-iii)),
and $\Cal V=\{T[K]\cap U:U\in\Cal U_0\}$ is a countable base
for the topology of $T[K]$ (4A2B(a-vi)).   Set
$\Cal M=\{K\cap T^{-1}[\overline{V}]:V\in\Cal V\}$, so that
$\Cal M$ is a countable family of closed convex subsets of $K$.

If $z\in F$, then there are distinct $u$, $v\in T[K]$ such that
$Tz=\bover12(u+v)$.   Now there must be $V$, $V'\in\Cal V$, with disjoint
closures, such that $u\in V$ and $v\in V'$, so that
$z\in\bover12(M+M')$, where $M=K\cap T^{-1}[\overline{V}]$ and
$M'=K\cap T^{-1}[\overline{V'}]$ are disjoint members of $\Cal M$.   Thus

\Centerline{$F\subseteq\bigcup\{\bover12(M+M'):
M$, $M'\in\Cal M$, $M\cap M'=\emptyset\}$.}

\noindent But (a) tells us that $\bover12(M+M')$ is
$\mu$-negligible whenever $M$, $M'\in\Cal M$ are disjoint.
As $\Cal M$ is countable, $\mu F=0$, as required.

\medskip

\quad{\bf (ii)} Now consider the Baire measure $\mu\restr\CalBa(K)$, where
$\CalBa(K)$ is the Baire $\sigma$-algebra of $K$.   This is inner regular
with respect to the zero sets (412D).   If $F\in\CalBa(K)$ contains no
extremal point of $K$, then (i) tells us that $\mu Z=0$ for every zero set
$Z\subseteq F$, so $\mu F$ must also be $0$.
}%end of proof of 461L

\leader{461M}{Theorem} Let $X$ be a Hausdorff locally convex linear
topological space, $K$ a compact convex subset of $X$ and $E$ the set of
extreme points of $K$.   Let $x\in X$.   Then there is a probability
measure $\mu$ on $E$ with barycenter $x$.
If $K$ is metrizable we can take $\mu$ to be a Radon measure.

\proof{ Let $P$ be the set of Radon probability measures on $K$
and $\preccurlyeq$ the partial order on $P$ described in 461K.
By 461Ke, there is a maximal element $\nu$ of $P$ such that
$\delta_x\preccurlyeq\nu_0$, where $\delta_x\in P$ is the Dirac measure on
$X$ concentrated at $x$.   By 461Kc, $x$ is the barycenter of $\nu$.

Let $\lambda=\nu\restr\CalBa(K)$ be the Baire measure
associated with $\nu$.   By 461Lb, $\lambda^*E=1$.   So the subspace
measure $\lambda_E$ on $E$ is a probability measure on $E$.   Let $\mu$ be
the completion of $\lambda_E$.

If $g\in X^*$ then
$g\restr K$ is continuous, therefore $\CalBa(K)$-measurable, so

$$\eqalignno{g(x)
&=\int_Kg\,d\nu
=\int_Kg\,d\lambda
=\int_Eg\,d\lambda_E\cr
\displaycause{214F}
&=\int_Eg\,d\mu\cr}$$

\noindent (212Fb).   So $x$ is the barycenter of $\mu$.

Now suppose that $K$ is metrizable.   In this case $E$ is a
G$_{\delta}$ set in $K$.   \Prf\ Let $\rho$ be a metric on $K$ inducing its
topology.   Then

\Centerline{$K\setminus E=\bigcup_{n\ge 1}\{tx+(1-t)y:x$, $y\in K$,
$t\in[2^{-n},1-2^{-n}]$, $\rho(x,y)\ge 2^{-n}\}$}

\noindent is K$_{\sigma}$, so its complement in $K$ is a G$_{\delta}$ set
in $K$.\ \QeD\   So $E$ is analytic (423Eb)
and $\mu$ is a Radon measure (433Cb).
}%end of proof of 461M

\leader{461N}{Lemma} Let $X$ be a Hausdorff locally convex linear
topological space, $K$ a compact convex subset of $X$, and $P$ the set of
Radon probability measures on $K$.   Let $E$ be
the set of extreme points of $K$ and suppose that $\mu\in P$ and
$\mu^*E=1$.   Then $\mu$ is maximal in $P$ for
the partial order $\preccurlyeq$ of 461K.

\proof{{\bf (a)} For $\mu\in P$ write $b(\mu)$ for the barycentre of $\mu$.
Then $b:P\to K$ is continuous for the narrow topology of $P$ and the weak
topology of $X$, as in 461J.    For $f\in C(K)$ define $\bar f:K\to\Bbb R$
by setting $\bar f(x)=\sup\{\int fd\mu:\mu\in P$, $b(\mu)=x\}$ for
$x\in K$.

\medskip

\quad{\bf (i)} Taking $\delta_x$ to be the Dirac measure on $X$
concentrated at $x$, we see that

\Centerline{$f(x)=\int fd\delta_x\le\bar f(x)$}

\noindent for any $x\in K$.

\medskip

\quad{\bf (ii)} For any $x\in K$ there is a $\mu\in P$ such that $b(\mu)=x$
and $\int fd\mu=\bar f(x)$.   \Prf\ The set $\{\mu:\mu\in P$, $b(\mu)=x\}$
is compact, so its continuous image $\{\int fd\mu:b(\mu)=x\}$ is compact
and contains its supremum.\ \Qed

$\bar f$ is upper semi-continuous.   \Prf\ For any
$\alpha\in\Bbb R$, the set

\Centerline{$\{(\mu,x):\mu\in P$, $x\in K$, $b(\mu)=x$,
$\int fd\mu\ge\alpha\}$}

\noindent is
compact, so its projection onto the second coordinate is closed;  but this
projection is just $\{x:\bar f(x)\ge\alpha\}$.\ \Qed

\medskip

\quad{\bf (iii)} $\bar f:K\to\Bbb R$ is concave.   \Prf\ Suppose that
$x$, $y\in K$ and $t\in[0,1]$.   Take $\mu$, $\nu\in P$ such that
$b(\mu)=x$, $\bar f(x)=\int fd\mu$, $b(\nu)=y$ and
$\bar f(y)=\int fd\nu$.   Set
$\lambda=t\mu+(1-t)\nu$.   Then

\Centerline{$\int_Kg\,d\lambda=t\int_Kg\,d\mu+(1-t)\int_Kg\,d\nu
=tg(x)+(1-t)g(y)=g(tx+(1-t)y)$}

\noindent for any $g\in X^*$, so $b(\lambda)=tx+(1-t)y$ and

\Centerline{$\bar f(tx+(1-t)y)
\ge\int fd\lambda
=t\int fd\mu+(1-t)\int fd\nu
=t\bar f(x)+(1-t)\bar f(y)$.}

\noindent As $x$, $y$ and $t$ are arbitrary, $\bar f$ is concave.\ \Qed

\medskip

\quad{\bf (iv)} If $x\in K$ and $\bar f(x)>f(x)$ then $x\notin E$.
\Prf\ There
is a $\mu\in P$ such that $b(\mu)=x$ and $\int fd\mu>f(x)$.   We cannot have
$\mu\{x\}=1$ because $\int fd\mu\ne f(x)$, so there is a point $y$ of the
support of $\mu$ such that $y\ne x$.   Let $g\in X^*$ be such that
$g(y)>g(x)$ and set $G=\{z:z\in K$, $g(z)>g(x)\}$, $t=\mu G$.
Then $t>0$;  also
$\int_Gg\,d\mu>tg(x)=t\int g\,d\mu$, so $t\ne 1$.
Define $\nu_1$, $\nu_2\in P$
by setting $\nu_1H=\bover1t\mu(G\cap H)$ whenever $H\subseteq K$ and
$\mu$ measures $G\cap H$, $\nu_2H=\bover1{1-t}\mu(H\setminus G)$
whenever $H\subseteq K$ and $\mu$ measures $H\setminus G$.   Let $x_1$,
$x_2$ be the barycenters of $\nu_1$, $\nu_2$ respectively.   For any
$h\in X^*$,

$$\eqalign{h(tx_1+(1-t)x_2)
&=th(x_1)+(1-t)h(x_2)
=t\int_Kh\,d\nu_1+(1-t)\int_Kh\,d\nu_2\cr
&=\int_Gh\,d\mu+\int_{K\setminus G}h\,d\mu
=h(x).\cr}$$

\noindent So $x=tx_1+(1-t)x_2$.    But both $x_1$ and $x_2$ belong to $K$
and $g(x_1)>g(x)$, so $x_1\ne x$, while $0<t<1$, so $x$ is not an extreme
point of $K$.\ \Qed


\medskip

{\bf (b)} Now take $\mu\in P$ such that $\mu^*E=1$, and $\nu\in P$ such that
$\mu\preccurlyeq\nu$.   For any convex
$f\in C(K)$, $f-\bar f$ is the sum of lower semi-continuous convex functions
so is lower semi-continuous, and
$\{x:f(x)=\bar f(x)\}=\{x:f(x)-\bar f(x)\ge 0\}$ is a G$_{\delta}$ set.
By (a-iv), it includes $E$, so $\int fd\mu=\int\bar fd\mu$.
In addition, $\int fd\mu\le\int fd\nu$ and
$\int\bar fd\nu\le\int\bar fd\mu$, by 461Kb applied to $-\bar f$.
But since $f\le\bar f$, we have

\Centerline{$\int fd\nu\le\int\bar fd\nu\le\int\bar fd\mu=\int fd\mu$;}

\noindent as $f$ is arbitrary, $\nu\preccurlyeq\mu$;
as $\nu$ is arbitrary, $\mu$ is maximal.
}%end of proof of 461N

\leader{461O}{Lemma} Suppose that $X$ is a Riesz space with a Hausdorff
locally convex linear space topology, and $K\subseteq X$ a
compact convex set such that every non-zero member of the positive cone
$X^+$ is uniquely expressible as $\alpha x$ for some $x\in K$ and
$\alpha\ge 0$.
Let $P$ be the set of Radon probability measures on $K$ and $\preccurlyeq$
the partial order described in 461K.   If $\mu$, $\nu\in P$ have the same
barycenter then they have a common upper bound in $P$.

\proof{{\bf (a)} If $X=\{0\}$ then $K$ is either $\{0\}$ or empty
and the result is immediate, so
henceforth suppose that $X$ is non-trivial.   Because each non-zero member
of $X^+$ is uniquely expressible as a multiple of a member of $K$, no
distinct members of $K\setminus\{0\}$ can be multiples of each other;  as
$K$ is convex, $0\notin K$.
If $z_0,\ldots,z_r\in K$,
$\gamma_0,\ldots,\gamma_r\ge 0$ and $z=\sum_{k=0}^r\gamma_kz_k$ belongs to
$K$, then
$\sum_{k=0}^r\gamma_k=1$.   \Prf\ Setting $\gamma=\sum_{k=0}^r\gamma_k$,
$\gamma\ne 0$ and
$\Bover1{\gamma}z=\sum_{k=0}^r\Bover{\gamma_r}{\gamma}z_k$ belongs to $K$;
accordingly $\Bover1{\gamma}z=z$ and $\gamma=1$.\ \Qed

\medskip

{\bf (b)} Take any $x\in K$, and let $P_x$ be the set of elements of $P$
with barycenter $x$.
Write $Q_x$ for the set of members of $P_x$ with finite support.
Then $Q_x$ is dense in $P_x$.   \Prf\ Suppose that
$\mu\in P_x$, $f_0,\ldots,f_n\in C(K)$ and $\epsilon>0$.   Then there is a
finite cover of $K$ by relatively open convex sets on each of which every
$f_i$ has
oscillation at most $\epsilon$;  so we have a partition $\Cal H$ of $K$ into
finitely many non-empty Borel sets $H$ such that every $f_i$ has
oscillation at most $\epsilon$ on the convex hull $\Gamma(H)$.
If $H\in\Cal H$ and $\mu H>0$, let $x_H$ be the barycentre of the measure
$\mu_H\in P$ where $\mu_HF=\Bover1{\mu H}\mu(F\cap H)$ whenever
$F\subseteq K$
and $\mu$ measures $F\cap H$.   For other $H\in\Cal H$, take any point $x_H$
of $H$.   In all cases, $x_H\in\overline{\Gamma(H)}$ so
$|f_i(y)-f_i(x_H)|\le\epsilon$ whenever $i\le n$ and $y\in H$.

Consider $\nu=\sum_{H\in\Cal H}\mu H\cdot\delta_{x_H}$, where
$\delta_{x_H}\in P$ is the Dirac measure on $K$ concentrated at $x_H$.
If $g\in X^*$ then

\Centerline{$\int_Kg\,d\nu
=\sum_{H\in\Cal H}\mu H\cdot g(x_H)
=\sum_{H\in\Cal H}\int_Hg\,d\mu
=\int_Kg\,d\mu
=g(x)$;}

\noindent as $g$ is arbitrary, $\nu\in P_x$ and $\nu\in Q_x$.   Next, for
$i\le n$,

$$\eqalign{|\int f_id\nu-\int f_id\mu|
&\le\sum_{H\in\Cal H}|f_i(x_H)\mu H-\int_Hf_id\mu|\cr
&\le\sum_{H\in\Cal H}\int_H|f_i(x_H)-f_i(y)|\mu(dy)
\le\sum_{H\in\Cal H}\epsilon\mu H
=\epsilon.\cr}$$

\noindent As $f_0,\ldots,f_n$ and $\epsilon$ are arbitrary, $Q_x$ is dense
in $P_x$.\ \Qed

\medskip

{\bf (c)} Suppose that $x\in K$ and $\mu$, $\nu\in Q_x$.
Then they have a common upper bound in $P$.   \Prf\
Express $\mu$, $\nu$ as $\sum_{i=0}^m\alpha_i\delta_{x_i}$,
$\sum_{j=0}^n\beta_j\delta_{y_j}$ respectively,
where all the $\alpha_i$ and
$\beta_j$ are strictly positive, all the $x_i$ and $y_j$ belong to $K$, and
$\sum_{i=0}^m\alpha_i=\sum_{j=0}^n\beta_j=1$.   If $g\in X^*$ then

\Centerline{$g(x)
=\int_Kg\,d\mu=\sum_{i=0}^m\alpha_ig(x_i)=\sum_{j=0}^n\beta_jg(y_j)$,}

\noindent so $x=\sum_{i=0}^m\alpha_ix_i=\sum_{j=0}^n\beta_jy_j$.   By the
decomposition theorem 352Fd there is a family
$\langle w_{ij}\rangle_{i\le m,j\le n}$ in $X^+$ such that
$\alpha_ix_i=\sum_{j=0}^nw_{ij}$ for every $i\le m$ and
$\beta_jy_j=\sum_{i=0}^nw_{ij}$ for every $j\le n$.   Each $w_{ij}$ is
expressible as $\gamma_{ij}z_{ij}$ where $z_{ij}\in K$ and
$\gamma_{ij}\ge 0$.   Now

\Centerline{$\sum_{j=0}^n\Bover{\gamma_{ij}}{\alpha_i}z_{ij}=x_i\in K$;}

\noindent by (a), $\sum_{j=0}^n\gamma_{ij}=\alpha_i$, for every $i\le m$.
Similarly, $\sum_{i=0}^m\gamma_{ij}=\beta_j$ for $j\le n$.   Of course this
means that $\sum_{i=0}^m\sum_{j=0}^n\gamma_{ij}=1$.

Set $\lambda=\sum_{i=0}^m\sum_{j=0}^n\gamma_{ij}\delta_{z_{ij}}\in P$.
If $f:K\to\Bbb R$ is continuous and convex,

$$\eqalign{\int fd\mu
&=\sum_{i=0}^m\alpha_if(x_i)
=\sum_{i=0}^m\alpha_if(\sum_{j=0}^n\bover{\gamma_{ij}}{\alpha_i}z_{ij})\cr
&\le\sum_{i=0}^m\alpha_i\sum_{j=0}^n\Bover{\gamma_{ij}}{\alpha_i}f(z_{ij})
=\sum_{i=0}^m\sum_{j=0}^n\gamma_{ij}f(z_{ij})
=\int fd\lambda.\cr}$$

\noindent So $\mu\preccurlyeq\lambda$.   Similarly,
$\nu\preccurlyeq\lambda$
and we have the required upper bound for $\{\mu,\nu\}$.\ \Qed

\medskip

{\bf (d)} Now consider

\Centerline{$\{(\mu,\nu,\lambda):\mu$, $\nu$, $\lambda\in P$,
$\mu\preccurlyeq\lambda$, $\nu\preccurlyeq\lambda\}$.}

\noindent This is a closed set in the compact set $P\times P\times P$
(461Kd), so its projection

\Centerline{$R=\{(\mu,\nu):\mu$, $\nu$ have a common upper bound in $P\}$}

\noindent is a closed set in $P\times P$.

If $x\in K$ then (c) tells us that $R$
includes $Q_x\times Q_x$, so (b) tells us that $R$ includes
$P_x\times P_x$.
Thus any two members of $P_x$ have a common upper bound in $P$, as
required.
}%end of proof of 461O

\leader{461P}{Theorem} Suppose that $X$ is a Riesz space with a Hausdorff
locally convex linear space topology, and $K\subseteq X$ a
metrizable compact convex set such that every non-zero member of the
positive cone
$X^+$ is uniquely expressible as $\alpha x$ for some $x\in K$ and
$\alpha\ge 0$.   Let $E$ be the set of extreme points of $K$, and $x$ any
point of $K$.   Then there is a unique Radon
probability measure $\mu$ on $E$ such that $x$ is the barycenter of $\mu$.

\proof{ By 461M, there is a Radon probability measure
$\mu$ on $E$ such that $x$ is the barycenter of $\mu$.
Suppose that $\mu_1$ is another measure with the same properties.   Let
$\nu$, $\nu_1$ be the Radon probability measures on $K$ extending $\mu$,
$\mu_1$ respectively.   Then $\nu$ and $\nu_1$ both have barycenter
$x$ and make $E$ conegligible.
By 461N, they are both maximal in $P$.   By 461O, they must have a
common upper bound in $P$, so they are equal.   But this means that
$\mu=\mu_1$.
}%end of proof of 461P

\leader{461Q}{}\cmmnt{ It is a nearly universal rule that when we
encounter a compact convex set we should try to identify its extreme
points.   I look now at some sets which arose naturally in \S437.

\medskip

\noindent}{\bf Proposition (a)}\dvAnew{2011}
Let $\frak A$ be a Dedekind
$\sigma$-complete Boolean algebra and $\pi:\frak A\to\frak A$ a
sequentially order-continuous Boolean
homomorphism.   Let $M_{\sigma}$ be the $L$-space of countably additive
real-valued functionals on $\frak A$, and $Q$ the set

\Centerline{$\{\nu:\nu\in M_{\sigma}$, $\nu\ge 0$, $\nu 1=1$,
$\nu\pi=\nu\}$.}

\noindent If $\nu\in Q$, then the following are equiveridical:
(i) $\nu$ is an extreme point of $Q$;  (ii) $\nu a\in\{0,1\}$
whenever $\pi a=a$;  (iii)
$\nu a\in\{0,1\}$ whenever $a\in\frak A$ is such that
$\nu(a\Bsymmdiff\pi a)=0$.

(b) Let $X$ be a set, $\Sigma$ a
$\sigma$-algebra of subsets of $X$, and $\phi:X\to X$ a
$(\Sigma,\Sigma)$-measurable function.   Let $M_{\sigma}$ be the
$L$-space of countably additive real-valued functionals on $\Sigma$, and
$Q\subseteq M_{\sigma}$ the set of probability measures with domain
$\Sigma$ for which $\phi$ is \imp.   If $\mu\in Q$, then $\mu$ is an
extreme point of $Q$ iff $\phi$ is ergodic with respect to
$\mu$\cmmnt{ (definition:  372Ob\formerly{3{}72Pb})}.

\proof{{\bf (a)} I ought to remark at once that because $\frak A$ is
Dedekind $\sigma$-complete, every countably additive functional on
$\frak A$ is bounded (326M\formerly{3{}26I}),
so that $M_{\sigma}$ is the $L$-space of
bounded countably additive functionals on $\frak A$, as studied in
362A-362B.

\medskip

\quad{\bf (i)$\Rightarrow$(ii)} Suppose that $\nu$ is an extreme point of
$Q$ and that $a\in\frak A$ is such that $\pi a=a$.   Set $\alpha=\nu a$.
\Quer\ If $0<\alpha<1$, define $\nu_1:\frak A\to\Bbb R$ by setting

\Centerline{$\nu_1b=\Bover1{\alpha}\nu(b\Bcap a)$}

\noindent for $b\in\frak A$.
Then $\nu_1$ is a non-negative countably additive
functional, and $\nu_11=1$.   Moreover, for any $b\in\frak A$,

\Centerline{$\nu_1\pi b=\Bover1{\alpha}\nu(\pi b\Bcap a)
=\Bover1{\alpha}\nu(\pi b\Bcap\pi a)
=\Bover1{\alpha}\nu\pi(b\Bcap a)
=\Bover1{\alpha}\nu(b\Bcap a)
=\nu_1b$,}

\noindent so $\nu_1\pi=\nu_1$ and $\nu_1\in Q$.   Since $\nu_1a=1$,
$\nu_1\ne\nu$.   Similarly,
$\nu_2\in Q$, where $\nu_2b=\Bover1{1-\alpha}\nu(b\Bsetminus a)$ for
$b\in\frak A$.   Now $\nu=\alpha\nu_1+(1-\alpha)\nu_2$ is a proper convex
combination of members of $Q$ and is not extreme.\ \BanG\   So
$\nu a\in\{0,1\}$;  as $a$ is arbitrary, (ii) is true.

\medskip

\quad{\bf (ii)$\Rightarrow$(iii)} Suppose that (ii) is true, and that
$a\in\frak A$ is such that $\nu(a\Bsymmdiff\pi a)=0$.   Then

\Centerline{$\nu(\pi^na\Bsymmdiff\pi^{n+1}a)
=\nu\pi^n(a\Bsymmdiff\pi a)=\nu(a\Bsymmdiff\pi a)=0$}

\noindent for every $n\in\Bbb N$, so $\nu(a\Bsymmdiff\pi^na)=0$ for every
$n\in\Bbb N$.   Set

\Centerline{$b_n=\sup_{m\ge n}\pi^ma$ for $n\in\Bbb N$,
\quad$b=\inf_{n\in\Bbb N}b_n$.}

\noindent Because $\pi$ is sequentially order-continuous and $\nu$ is
countably additive,

\Centerline{$\nu(a\Bsymmdiff b_n)=0$ for every $n\in\Bbb N$,
\quad$\nu(a\Bsymmdiff b)=0$.}

\noindent Now

\Centerline{$\pi b_n=\sup_{m\ge n}\pi^{m+1}a=b_{n+1}\Bsubseteq b_n$}

\noindent for every $n\in\Bbb N$, so

\Centerline{$\pi b=\inf_{n\in\Bbb N}\pi b_n=\inf_{n\in\Bbb N}b_{n+1}=b$.}

\noindent Consequently
$\nu a=\nu b\in\{0,1\}$.   As $a$ is arbitrary, (iii) is true.

\medskip

\quad{\bf(iii)$\Rightarrow$(i)} Suppose that (iii) is true, and that
$\nu=\bover12(\nu_1+\nu_2)$ where $\nu_1$, $\nu_2\in Q$.   For
$\alpha\ge 0$, set $\theta_{\alpha}=\nu_1-\alpha\nu\in M_{\sigma}$.
Then we have a corresponding
element $a_{\alpha}=\Bvalue{\theta_{\alpha}>0}$ in $\frak A$ such that

\Centerline{$\theta_{\alpha}c>0$ whenever $0\ne c\Bsubseteq a_{\alpha}$,
\quad$\theta_{\alpha}c\le 0$ whenever
$c\Bcap a_{\alpha}=0$}

\noindent(326S\formerly{3{}26O}).   Observe that if $b\in\frak A$, then

\Centerline{$\theta_{\alpha}b
=\theta_{\alpha}a_{\alpha}-\theta_{\alpha}(a_{\alpha}\Bsetminus b)
   +\theta_{\alpha}(b\Bsetminus a_{\alpha})
\le\theta_{\alpha}a_{\alpha}$,}

\Centerline{$\theta_{\alpha}\pi b
=\nu_1\pi b-\alpha\nu\pi b=\nu_1b-\alpha\nu b=\theta_{\alpha}b$.}

Now $\nu(a_{\alpha}\Bsetminus\pi a_{\alpha})=0$.
\Prf\Quer\ Otherwise, setting $c=a_{\alpha}\Bsetminus\pi a_{\alpha}$, we
must have $\theta_{\alpha}c>0$, so

\Centerline{$\theta_{\alpha}(c\Bcup\pi a_{\alpha})
>\theta_{\alpha}\pi a_{\alpha}
=\theta_{\alpha}a_{\alpha}$.  \Bang\Qed}

\noindent Consequently

\Centerline{$\nu(a_{\alpha}\Bsymmdiff\pi a_{\alpha})
=\nu\pi a_{\alpha}-\nu a_{\alpha}+2\nu(a_{\alpha}\Bsetminus\pi a_{\alpha})
=0$}

\noindent and $\nu a_{\alpha}\in\{0,1\}$.

If $\alpha\le\beta$ then $\theta_{\beta}\le\theta_{\alpha}$ and
$a_{\beta}\Bsubseteq a_{\alpha}$.   As $\theta_0=\nu_1$,
$\nu_1(1\Bsetminus a_0)=0$,
$\nu_1a_0=1$ and $\nu a_0\ge\bover12$;  accordingly $\nu a_0=1$.
As $\theta_2\le 0$, $a_2=0$ and $\nu a_2=0$.   So

\Centerline{$\beta=\sup\{\alpha:\nu a_{\alpha}=1\}
=\sup\{\alpha:\nu a_{\alpha}>0\}$}

\noindent is defined in $[0,2]$.

Now $\nu_1=\beta\nu$.   \Prf\ Let $c\in\frak A$.
\Quer\ If $\nu_1c>\beta\nu c$, take $\alpha>\beta$ such
that $\nu_1c>\alpha\nu c$.   Then $\nu a_{\alpha}=0$, but

\Centerline{$0<\theta_{\alpha}c\le\theta_{\alpha}a_{\alpha}
\le\nu_1a_{\alpha}\le 2\nu a_{\alpha}$.  \Bang}

\noindent \Quer\ If
$\nu_1c<\beta\nu c$, take $\alpha\in\coint{0,\beta}$ such that
$\nu_1c<\alpha\nu c$.   Then $\nu a_{\alpha}=1$ so
$\nu(1\Bsetminus a_{\alpha})=0$, but

\Centerline{$0>\theta_{\alpha}c\ge\theta_{\alpha}(c\Bsetminus a_{\alpha})
\ge -\alpha\nu(c\Bsetminus a_{\alpha})
\ge -\alpha\nu(1\Bsetminus a_{\alpha})$.  \Bang}

\noindent Thus $\nu_1c=\beta\nu c$;  as
$c$ is arbitrary, $\nu_1=\beta\nu$.\ \Qed

Accordingly $\nu_1$ is a multiple of $\nu$ and must be equal to $\nu$.
Similarly, $\nu_2=\nu$.   As $\nu_1$ and $\nu_2$ were arbitrary,
$\nu$ is an extreme point of $Q$.

\medskip

{\bf (b)} In (a), set $\frak A=\Sigma$ and $\pi E=\phi^{-1}[E]$ for
$E\in\Sigma$;  then `$\phi$ is ergodic with respect to $\mu$' corresponds
to condition (ii) of (a), so we have the result.
}%end of proof of 461Q

\leader{461R}{Corollary} Let $X$ be a compact Hausdorff space and
$\phi:X\to X$ a continuous function.   Let $Q$ be the non-empty
compact convex set of
Radon probability measures $\mu$ on $X$ such that $\phi$ is \imp\ for
$\mu$, with its narrow topology and the convex
structure defined by 234G and 234Xf.   Then the extreme points of
$Q$ are those for which $\phi$ is ergodic.

\proof{{\bf (a)} If $\mu_0\in Q$ is not extreme, let
$\Cal B=\Cal B(X)$ be the
Borel $\sigma$-algebra of $X$, so that $\phi$ is
$(\Cal B,\Cal B)$-measurable, and write $Q'$ for the set of
$\phi$-invariant Borel probability measures on $X$.   Then 461Q tells us
that the extreme points of $Q'$ are just the measures for which
$\phi$ is ergodic.   If $\mu\in Q$, then
$\mu\restr\Cal B\in Q'$, and of
course the function $\mu\mapsto\mu\restr\Cal B$ is injective (416Eb)
and preserves convex combinations.   So $\mu_0\restr\Cal B$ is not
extreme in $Q'$.   By 461Q, $\phi$ is not
$\mu_0\restr\Cal B$-ergodic, and therefore not $\mu_0$-ergodic.

\medskip

{\bf (b)} If $\mu_0\in Q$ and $\phi$ is not $\mu_0$-ergodic, let
$E\in\dom\mu_0$ be such that $0<\mu_0E<1$ and $\phi^{-1}[E]=E$.   Set
$\alpha=\mu_0E$ and $\beta=1-\alpha$, and let $\mu_1$, $\mu_2$ be the
indefinite-integral measures over $\mu_0$ defined by
$\Bover1{\alpha}\chi E$ and $\Bover1{\beta}\chi(X\setminus E)$.   Then
$\mu_1$ is a Radon probability measure on $X$ (416S), so the image
measure $\mu_1\phi^{-1}$ also is a Radon measure (418I).   The argument
of part (b) of the proof of 461Q tells us that $\mu_1\phi^{-1}$ agrees
with $\mu_1$ on Borel sets, so $\mu_1=\mu_1\phi^{-1}$ (416Eb) and
$\mu_1\in Q$.   Similarly, $\mu_2\in Q$, and
$\mu_0=\alpha\mu_1+\beta\mu_2$, so $\mu_0$ is not extreme in $Q$.
}%end of proof of 461R

\exercises{\leader{461X}{Basic exercises $\pmb{>}$(a)}
%\spheader 461Xa
Let $X$ be a Hausdorff locally convex linear topological space,
$C\subseteq X$ a convex set, and $g:C\to\Bbb R$ a function.   Show that
the following are equiveridical:  (i) $g$ is convex and lower
semi-continuous;  (ii) there are a non-empty set $D\subseteq X^*$ and a
family $\family{f}{D}{\beta_f}$ in $\Bbb R$ such that
$g(x)=\sup_{f\in D}f(x)+\beta_f$ for every $x\in C$.   (Compare 233Hb.)
%461C

\sqheader 461Xb Let $X$ be a Hausdorff locally convex linear topological
space, $K\subseteq X$ a compact convex set, and $x$ an extreme point of
$K$.   Let $\mu$ be a probability measure on $X$ such that $\mu^*K=1$
and $x$ is the barycenter of $\mu$.   (i) Show that $\{y:y\in K$, $f(y)\ne f(x)\}$ is
$\mu$-negligible for every $f\in X^*$.   \Hint{461E.}   (ii) Show that
if $\mu$ is a Radon measure then $\mu\{x\}=1$.

\spheader 461Xc For each $n\in\Bbb N$, define $e_n\in\pmb{c}_0$ by
saying that $e_n(n)=1$, $e_n(i)=0$ if $i\ne n$.   Let $\mu$ be the
point-supported Radon
probability measure on $\pmb{c}_0$ defined by saying that
$\mu E=\sum_{n=0}^{\infty}2^{-n-1}\chi E(2^ne_n)$ for every
$E\subseteq\pmb{c}_0$.   (i) Show that every member of $\pmb{c}_0^*$ is
$\mu$-integrable.   \Hint{$\pmb{c}_0^*$ can be identified with
$\ell^1$.}   (ii) Show that $\mu$ has no barycenter in $\pmb{c}_0$.
%461F

\spheader 461Xd Let $I$ be an uncountable set, and
$X=\{x:x\in\ell^{\infty}(I),\,\{i:x(i)\ne 0\}$ is countable$\}$.
(i) Show that $X$ is a closed linear subspace of $\ell^{\infty}(I)$.
(ii) Show
that there is a probability measure $\mu$ on $X$ such that ($\alpha$)
$\mu\{x:\|x\|\le 1\}$ is defined and equal to $1$ ($\beta$)
$\mu\{x:x(i)=1\}=1$ for every $i\in I$.
(iii) Show that $\int fd\mu$ is defined for every $f\in X^*$.   \Hint{for
any $f\in X^{\sim}$, there is a countable set $J\subseteq X$ such that
$f(x)=0$ whenever $x\restr J=0$, so that $f\eae f(\chi J)$.}   (iv)
Show that $\mu$ has no barycenter in $X$.
%461F

\sqheader 461Xe Let $X$ be a complete Hausdorff locally convex linear
topological
space, and $K\subseteq X$ a compact set.   Show that every extreme point
of $\overline{\Gamma(K)}$ belongs to $K$.   \Hint{show that it cannot be
the barycenter of any measure on $K$ which is not supported by a single
point.}
%461I

\sqheader 461Xf Let $X$ be a Hausdorff locally convex linear topological
space, and $K\subseteq X$ a metrizable compact set.   Show that
$\overline{\Gamma(K)}$ is metrizable.   \Hint{we may suppose that $X$ is
complete, so that $\overline{\Gamma(K)}$ is compact.   Show that
$\overline{\Gamma(K)}$ is a continuous image of the space of Radon
probability measures on $K$, and use 437Rf.}
%461J

\spheader 461Xg Let $X$ be a Hausdorff locally convex linear topological
space and $K\subseteq X$ a compact set.   Show that the Baire
$\sigma$-algebra of $K$ is just the subspace $\sigma$-algebra induced by
the cylindrical $\sigma$-algebra of $X$.
%461L

\sqheader 461Xh Let $X$ be a Hausdorff locally convex linear topological
space, and $K\subseteq X$ a compact convex set;  let $E$ be the set of
extreme points of $K$.   Let $\sequencen{f_n}$ be a sequence in $X^*$
such that $\sup_{x\in E,n\in\Bbb N}|f_n(x)|$ is finite and
$\lim_{n\to\infty}f_n(x)=0$ for every $x\in E$.   Show that
$\lim_{n\to\infty}f_n(x)=0$ for every $x\in K$.
%461M

\sqheader 461Xi Let $X$ be a Hausdorff locally convex linear topological
space, and $K\subseteq X$ a metrizable compact convex set.
Show that the algebra of Borel subsets of $K$ is just the
subspace algebra of the cylindrical $\sigma$-algebra of $X$.
%461M

\spheader 461Xj Let $X$ be a Hausdorff locally convex linear topological
space, and $K\subseteq X$ a compact set.   Let us say that a point $x$
of $K$ is {\bf extreme} if the only Radon probability measure on $K$
with barycenter $x$ is the Dirac measure on $K$ concentrated at $x$.
(Cf.\ 461Xb.)
(i) Show that if $X$ is complete, then $x\in X$ is an extreme point of
$K$ iff it is an extreme point of $\overline{\Gamma(K)}$.   (ii)
Writing $E$ for the set of extreme points of $K$, show that any point of
$K$ is the barycenter of some probability measure on $E$.   (iii) Show
that if $K$ is metrizable then $E$ is a G$_{\delta}$ subset of $K$ and
any point of $K$ is the barycenter of some Radon probability measure on
$E$.
%461M

\spheader 461Xk Let $G$ be an abelian group with identity $e$,
and $K$ the set of positive definite functions
$h:G\to\Bbb C$ such that $h(e)=1$.   (i) Show that $K$ is a compact convex
subset of $\Bbb C^G$.   (ii) Show that the extreme points of $K$ are just
the group homomorphisms from $G$ to $S^1$.
(iii) Show that $K$ generates the
positive cone of a Riesz space.   \Hint{445N.}
%461M

\spheader 461Xl Let $X$ be a compact metrizable space and $G$ a subgroup
of the group of autohomeomorphisms of $X$.   Let $M_{\sigma}$ be the space
of signed Borel measures on $X$ with its vague topology, and
$Q\subseteq M_{\sigma}$ the set of $G$-invariant Borel probability measures
on $X$.   Show that every member of $Q$ is uniquely expressible as the
barycenter of a Radon probability
measure on the set of extreme points of $Q$.
%461P

\spheader 461Xm Let $X$ be a set, $\Sigma$ a $\sigma$-algebra of subsets
of $X$, and $P$ the set of probability measures with domain $\Sigma$,
regarded as a
convex subset of the linear space of countably additive functionals on
$\Sigma$.   Show that $\mu\in P$ is an extreme point in $P$ iff it takes
only the values $0$ and $1$.
%461Q

\spheader 461Xn Let $\frak A$ be a Boolean
algebra and $M$ the $L$-space of bounded finitely additive functionals on
$\frak A$, and $\pi:\frak A\to\frak A$ a Boolean homomorphism.
(i) Show that $U=\{\nu:\nu\in M$, $\nu\pi=\nu\}$ is a closed Riesz subspace
of $M$.   (ii) Set $Q=\{\nu:\nu\in U$, $\nu\ge 0$, $\nu 1=1\}$.
Show that if $\mu$, $\nu$ are distinct extreme points of $Q$ then
$\mu\wedge\nu=0$.   (iii) Set $Q_{\sigma}=\{\nu:\nu\in Q$, $\nu$ is
countably additive$\}$.   Show that any extreme point of $Q_{\sigma}$ is an
extreme point of $Q$.
(iv) Set $Q_{\tau}=\{\nu:\nu\in Q$, $\nu$ is
countably additive$\}$.   Show that any extreme point of $Q_{\tau}$ is an
extreme point of $Q$.
%461Q

\sqheader 461Xo Set $S^1=\{z:z\in\Bbb C$, $|z|=1\}$, and let $w\in S^1$
be such that $w^n\ne 1$ for any integer $1$.   Define $\phi:S^1\to S^1$
by setting $\phi(z)=wz$ for every $z\in S^1$.   Show that the only Radon
probability measure on $S^1$ for which $\phi$ is \imp\ is the Haar
probability measure $\mu$ of $S^1$.   \Hint{use 281N to show that
$\lim_{n\to\infty}\bover1{n+1}\sum_{k=0}^nf(w^kz)=\int fd\mu$ for every
$f\in C(S^1)$;  now put 461R and 372H\formerly{3{}72I} together.}
%461R

\spheader 461Xp Set $\phi(x)=2\min(x,1-x)$ for $x\in[0,1]$ (cf.\
372Xp\formerly{3{}72Xm}).
Show that there are many point-supported Radon measures on $[0,1]$ for
which $\phi$ is \imp.
%461R   are there unexpected atomless measures?

\spheader 461Xq
Let $X$ and $Y$ be Hausdorff locally convex linear
topological spaces, $A\subseteq X$ a convex set and
$\phi:A\to Y$ a continuous
function such that $\phi[A]$ is bounded and
$\phi(tx+(1-t)y)=t\phi(x)+(1-t)\phi(y)$ for all $x$, $y\in A$ and
$t\in[0,1]$.   Let $\mu$ be a topological
probability measure on $A$ with a barycenter $x^*\in A$.
Show that $\phi(x^*)$ is the barycenter of the image measure
$\mu\phi^{-1}$ on $Y$.   \Hint{show first that if
$\familyiI{E_i}$ is a finite partition of $A$ into
non-empty convex sets measured by $\mu$,
$\alpha_i=\mu E_i$ for each $i\in I$ and
$C=\{\sum_{i\in I}\alpha_ix_i:x_i\in E_i$ for every $i\in I\}$, then
$x^*\in\overline{C}$.}
%461B

\leader{461Y}{Further exercises (a)}
%\spheader 461Ya
Let $X$ be a Hausdorff locally convex linear topological space.   (i)
Show that if $M_0,\ldots,M_n$ are non-empty compact convex subsets of
$X$ with empty
intersection, then there is a continuous convex function $g:X\to\Bbb R$
such that $g(\sum_{i=0}^n\alpha_ix_i)<\sum_{i=0}^n\alpha_ig(x_i)$
whenever $x_i\in M_i$ and
$\alpha_i>0$ for every $i\le n$.   (ii) Show that if $K\subseteq X$ is
compact and $x^*\in\overline{\Gamma(K)}$ then there is a Radon
probability measure $\mu$ on $K$,
with barycenter $x^*$, such that $\mu(\alpha_0M_0+\ldots+\alpha_nM_n)=0$
whenever $M_0,\ldots,M_n$ are compact convex subsets of $K$ with empty
intersection, $\alpha_i\ge 0$ for every
$i\le n$, and $\sum_{i=0}^n\alpha_i=1$.
%461L

\spheader 461Yb
In $\BbbR^{\coint{0,2}}$ let $K$ be the set of those functions $u$ such
that ($\alpha$) $0\le u(s)\le u(t)\le 1$ whenever $0\le s\le t\le 1$
($\beta$) $|u(t+1)|\le u(s')-u(s)$ whenever $0\le s<t<s'\le 1$.   (i)
Show that $K$ is a compact convex set.   (ii) Show that the set $E$ of
extreme points of $K$ is just the set of functions of the types
$\tbf{0}$, $\chi[0,1]$, $\chi[s,1]\pm\chi\{1+s\}$ and
$\chi\ocint{s,1}\pm\chi\{1+s\}$ for $0<s<1$.   (iii) Set $w(s)=s$ for
$s\in[0,1]$, $0$ for $s\in\ooint{1,2}$.   Show that if $\mu$ is any
Radon probability measure on $K$ with barycenter $w$ then $\mu E=0$.
%461M
%what about Borel measures?  when there is an \am cardinal I suppose

\spheader 461Yc Write $\nu_{\omega_1}$ for the usual measure on
$Z=\{0,1\}^{\omega_1}$.   Fix any $z_0\in Z$, and let $U$ be the linear
space $\{u:u\in C(Z)$, $u(z_0)=\int u\,d\nu_{\omega_1}\}$.   Let $X$ be the
Riesz space of signed tight Borel measures $\mu$ on $Z$ such that
$\mu\{z_0\}=0$,
with the topology generated by the functionals $\mu\mapsto\int u\,d\mu$ as
$u$ runs over $U$.    Let $K\subseteq X$ be the set of tight Borel
probability measures $\mu$ on $Z$ such that $\mu\{z_0\}=0$.
(i) Show that $K$
is compact and convex and that every member of $X^+\setminus\{0\}$ is
uniquely expressible as a positive multiple of a member of $K$.   (ii) Show
that the set $E$ of extreme points of $K$ can be identified, as topological
space, with $Z\setminus\{z_0\}$, so is a Borel subset of $K$ but not a
Baire subset.   (iii) Show that the restriction of $\nu_{\omega_1}$ to the
Borel $\sigma$-algebra of $Z$ is the barycenter of more than
one Baire measure on $E$.
%461P

\spheader 461Yd Let $X$ be a set, $\Sigma$ a
$\sigma$-algebra of subsets of $X$, and $\Phi$ a set of
$(\Sigma,\Sigma)$-measurable functions from $X$ to itself.   Let
$M_{\sigma}$ be the
$L$-space of countably additive real-valued functionals on $\Sigma$, and
$Q\subseteq M_{\sigma}$ the set of probability measures with domain
$\Sigma$ for which every member of $\Phi$ is \imp.   (i) Show that if
$\mu\in Q$, then $\mu$ is an extreme point of $Q$ iff $\mu E\in\{0,1\}$
whenever $E\in\Sigma$ and $\mu(E\symmdiff\phi^{-1}[E])=0$ for every
$\phi\in\Phi$.
(ii) Show that if $\mu\in Q$ and $\Phi$ is countable and commutative,
then $\mu$ is an extreme point of $Q$ iff $\mu E\in\{0,1\}$ whenever
$E\in\Sigma$ and $E=\phi^{-1}[E]$ for every $\phi\in\Phi$.
%461Q

\spheader 461Ye Let $X$ be a non-empty Hausdorff space, and define
$\phi:X^{\Bbb N}\to X^{\Bbb N}$ by setting $\phi(x)(n)=x(n+1)$ for
$x\in X^{\Bbb N}$ and $n\in\Bbb N$.
Let $Q$ be the set of Radon probability measures
on $X^{\Bbb N}$ for which $\phi$ is \imp.   Show that a Radon
probability measure $\lambda$ on $X^{\Bbb N}$ is an extreme point of
$Q$ iff it is a Radon product measure $\mu^{\Bbb N}$ for some Radon
probability measure $\mu$ on $X$.
%461R mt46bits

\spheader 461Yf Let $G$ be a topological group.   Show that the following
are equiveridical:  (i) $G$ is amenable;  (ii) whenever $X$ is a Hausdorff
locally convex linear topological space, and
$\action$ is a continuous action of $G$ on $X$ such that
$x\mapsto a\action x$ is a linear operator for every $a\in G$, and
$K\subseteq X$ is a
non-empty compact convex set such that $a\action x\in K$ whenever
$a\in G$ and $x\in K$, then there is an $x\in K$ such that
$a\action x=x$ for every $a\in G$;
(iii) whenever $X$ is a Hausdorff
locally convex linear topological space, $K\subseteq X$ is a non-empty
compact convex set, and
$\action$ is a continuous action of $G$ on $K$ such that
$a\action(tx+(1-t)y)=t\,a\action x+(1-t)a\action y$ whenever $a\in G$, $x$,
$y\in K$ and $t\in[0,1]$, then there is an $x\in K$ such that
$a\action x=x$ for every $a\in G$.
Use this to simplify parts of the proof of 449C.   \Hint{493B.}
%461Xq

\spheader 461Yg Let $G$ be an amenable topological group, and $\action$ an
action of $G$ on a reflexive Banach space $U$, continuous for the
given topology on $G$ and the weak topology of $U$, such that
$u\mapsto a\action u$ is a linear operator of
norm at most $1$ for every $a\in G$.   Set
$V=\{v:v\in U$, $a\action v=v$ for every $a\in G\}$.   Show that
$\{u+v-a\action u:u\in U$, $v\in V$, $a\in G\}$ is dense in $U$.
}%end of exercises

\endnotes{
\Notesheader{461} The results above are a little unusual in that we have
studied locally convex spaces for several pages without encountering two
topologies on the same space more than once (461J).
In fact some of the most interesting
properties of measures on locally convex spaces concern their
relationships with strong and weak topologies, but I defer these ideas
to later parts of the chapter.   For the moment, we just have the basic
results affirming (i) that barycenters exist (461E, 461F, 461H) (ii)
that points can be represented as barycenters (461I, 461M).   The
last two can be thought of as refinements of the
\Krein-Milman theorem.   Any compact convex set $K$ (in a locally
compact
Hausdorff space) is the closed convex hull of the set $E$ of its extreme
points.   By 461M, given $x\in K$, we can actually find a measure on $E$
with barycenter $x$;  and if $K$ is metrizable we can do this with a
Radon measure.   Of course the second part of 461M is a straightforward
consequence of the first.   But I do not know of
any proof of 461M which does not pass through 461K-461L.

\Krein's theorem (461J) is a fundamental result in the theory of linear
topological spaces.   The proof here, using the Riesz representation
theorem and vague topologies, is a version of the standard one (e.g.,
{\smc Bourbaki 87}, II.4.1), written out to be a little heavier in the
measure theory and a little lighter in the topological linear space
theory than is usual.   There are of course proofs which do not use
measure theory.

In \S437 I have already looked at an archetypal special case of 461I and
461M.   If $X$ is a compact Hausdorff space and
$P$ the compact convex set of Radon probability measures on $X$ with the
narrow (or vague) topology, then the set of extreme points of $P$ can be
identified with the set $\Delta$ of Dirac measures on $X$ (437S, 437Xt).
If we think of $P$ as a subset of $C(X)^*$ with the weak* topology, so
that the dual of the linear topological space $C(X)^*$ can be identified
with $C(X)$, then any $\mu\in P$ is the barycenter of a Baire
probability measure $\nu$ on
$\Delta$.   In fact (because $\Delta$ here is compact) $\mu$ is the
barycentre of a Radon measure
on $\Delta$, and this is just the image measure $\mu\delta^{-1}$.

I have put the phrase `Choquet's theorem' into the title of this section.
Actually it should perhaps be `first steps in Choquet theory', because
while the theory as a whole was dominated for many years by the work of
G.Choquet the exact attribution of the results
presented here is more complicated.
See {\smc Phelps 66} for a much fuller account.   But certainly both the
existence and uniqueness theorems 461M and 461P draw heavily on Choquet's
ideas.

Theorem 461P, demanding an excursion through 461N-461O, seems fairly hard
work for a relatively specialized result.   But it provides a unified
explanation for a good many apparently disparate phenomena.   Of course the
simplest example is when $X=C(Z)^*$ for some compact metrizable space $Z$
and $K$ is the set of positive linear functionals of norm $1$, so that $E$
can be identified with $Z$ and we find ourselves back with the Riesz
representation theorem.   A less familiar case already examined is in
461Xk.   At the next level we have such examples as 461Xl.

Another class of examples arising in \S437
is explored in 461Q-461R, 461Xm-461Xn and 461Yd-461Ye.
It is when we have an explicit listing of the extreme points,
as in 461Yb and 461Ye,
that we can begin to feel that we understand a compact convex set.
}%end of notes

\leaveitout{\Krein's theorem:\prooflet{(\Bourbaki, II.4.1:  \Schaefer,
II.4.3;  \Kothe, \S20.6.)}
%\NB, 5.7.5.
}%end of leaveitout

\discrpage

