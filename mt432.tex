\frfilename{mt432.tex} 
\versiondate{2.10.13} 
\copyrightdate{2008} 
      
\def\NN{\BbbN^{\Bbb N}} 
      
\def\chaptername{Topologies and measures II} 
\def\sectionname{K-analytic spaces} 
      
\newsection{432} 
      
I describe the basic measure-theoretic properties of K-analytic 
spaces\cmmnt{ (\S422)}.   I start with `elementary' results 
(432A-432C), assembling ideas from \S\S421, 422 and 431.   The main 
theorem of the section is 432D, one of the leading cases of the general 
extension theorem 416P.   An important corollary (432G) gives a 
sufficient condition for the existence of pull-back measures.   I 
briefly mention `capacities' (432J-432L). 
      
\leader{432A}{Proposition} Let $(X,\frak T,\Sigma,\mu)$ be a complete 
locally determined Hausdorff topological measure space.   Then every 
K-analytic subset of $X$ is measurable. 
      
\proof{ If $A\subseteq X$ is K-analytic, it is Souslin-F (422Ha), 
therefore measurable (431B). 
}%end of proof of 432A 
      
\leader{432B}{Theorem} Let $X$ be a K-analytic Hausdorff space, and 
$\mu$ a semi-finite topological measure on $X$.   Then 
      
\Centerline{$\mu X=\sup\{\mu K:K\subseteq X$ is compact$\}$.} 
      
\proof{ If $\gamma<\mu X$, there is an $E\in\dom\mu$ such that 
$\gamma<\mu E<\infty$;   set $\nu F=\mu(E\cap F)$ for every Borel set 
$F\subseteq X$, so that $\nu$ is a totally finite Borel measure on $X$, 
and $\nu X>\gamma$.   Let $\hat\nu$ be the completion of $\nu$.   Let 
$R\subseteq\NN\times X$ be an usco-compact relation such 
that $R[\NN]=X$.   Set $F_{\sigma}=\overline{R[I_{\sigma}]}$ for 
$\sigma\in S^*=\bigcup_{k\ge 1}\BbbN^k$, where 
$I_{\sigma}=\{\phi:\sigma\subseteq\phi\in\NN\}$.   Because $R$ is closed 
in $\NN\times X$ (422Da), $X$ is the kernel of the Souslin scheme 
$\family{\sigma}{S^*}{F_{\sigma}}$ (421I).   By 431D, there is a compact 
$L\subseteq\NN$ such that 
$\hat\nu(\bigcup_{\phi\in L}\bigcap_{n\in\Bbb N}F_{\phi\restr n}) 
\ge\gamma$. 
But, by 421I, this is just $\hat\nu(R[L])$;  and $R[L]$ is compact, by 
422D(e-i).   So $\mu R[L]$ is defined, with  
$\mu R[L]\ge\nu R[L]=\hat\nu R[L]$, and we have a compact subset of $X$ of measure at least 
$\gamma$.   As $\gamma$ is arbitrary, the theorem is proved. 
}%end of proof of 432B 
      
\leader{432C}{Proposition} Let $X$ be a Hausdorff space such that all 
its open sets are K-analytic, and $\mu$ a Borel measure on $X$. 
      
(a) If $\mu$ is semi-finite, it is tight. 
      
(b) If $\mu$ is locally finite, its completion is a Radon measure on 
$X$. 
      
\proof{{\bf (a)} By 422Hb, every open subset of $X$ is Souslin-F. 
Applying 421F to the family $\Cal E$ of closed 
subsets of $X$, we see that every Borel subset of $X$ is Souslin-F, 
therefore K-analytic (422Ha). 
Now suppose that $E\subseteq X$ is a Borel set.   Then the subspace 
measure $\mu_E$ is a semi-finite Borel measure on the 
K-analytic space $E$, so by 432B 
$\mu E=\sup_{K\subseteq E\text{ is compact}}\mu K$.   As $E$ is 
arbitrary, $\mu$ is inner regular with respect to the compact sets;  but we are supposing that $X$ is Hausdorff, so these are all closed, and $\mu$ is tight. 
      
\medskip 
      
{\bf (b)} Because $X$ is Lindel\"of (422Gg), $\mu$ is $\sigma$-finite 
(411Ge), therefore semi-finite.   So (a) tells us that $\mu$ is tight.    
By 416F, its c.l.d.\ version 
is a Radon measure.   But (because $\mu$ is $\sigma$-finite) this is 
just its completion (213Ha). 
}%end of proof of 432C 
      
\leader{432D}{Theorem}\cmmnt{ ({\smc Aldaz \& Render 00})} Let 
$X$ be a K-analytic Hausdorff space and $\mu$ a locally finite measure 
on $X$ which is inner regular with respect to the closed sets.   Then 
$\mu$ has an extension to a Radon measure on $X$.   In particular, $\mu$ 
is $\tau$-additive. 
      
\proof{ The point is that if $\mu E>0$ then there is a compact 
$K\subseteq E$ such that $\mu^*K>0$.   \Prf\ Write $\Sigma$ for the  
domain of $\mu$.   Take $\gamma<\mu E$.   Because $X$ is Lindel\"of  
(422Gg), $\mu$ is $\sigma$-finite (411Ge), therefore semi-finite.   Let 
$E'\subseteq E$ be such that 
$\gamma<\mu E'<\infty$.   Because $\mu$ is inner regular with respect to 
the closed sets, there is a closed set $F\subseteq E$ such that 
$\mu F>\gamma$.   $F$ is K-analytic (422Gf);  let 
$R\subseteq\NN\times F$ be an 
usco-compact relation such that $R[\NN]=F$.  For 
$\sigma\in S=\bigcup_{n\in\Bbb N}\BbbN^n$ set 
      
\Centerline{$A_{\sigma}=\{x:(\phi,x)\in R$ for some $\phi\in\NN$ such 
that $\phi(i)\le\sigma(i)$ for every $i<\#(\sigma)\}$.} 
      
\noindent Then $\sequence{i}{A_{\sigma^{\smallfrown}\fraction{i}}}$ is a 
non-decreasing sequence with union $A_{\sigma}$, so 
      
\Centerline{$\mu^*A_{\sigma} 
=\sup_{i\in\Bbb N}\mu^*A_{\sigma^{\smallfrown}\fraction{i}}$} 
      
\noindent for every $\sigma\in S$ (132Ae).   
We can therefore find a sequence $\psi\in\NN$ such that 
      
\Centerline{$\mu^*A_{\psi\restr n}>\gamma$} 
      
\noindent for every $n\in\Bbb N$.   Set 
      
\Centerline{$K 
=\{\phi:\phi\in\NN,\,\phi(i)\le\psi(i)$ for every $i\in\Bbb N\}$;} 
      
\noindent then $K=\prod_{n\in\Bbb N}(\psi(n)+1)$ is compact, so $R[K]$ 
is compact (422D(e-i)). 
      
\Quer\ Suppose, if possible, that $\mu^*R[K]<\gamma$.   Then there is an 
$H\in\Sigma$ such that $R[K]\subseteq H\subseteq F$ and 
$\mu(F\setminus H)>\mu F-\gamma$.   Because $\mu$ is inner regular with 
respect to the closed sets, there is a closed set $F'\in\Sigma$ such that 
$F'\subseteq F\setminus H$ and $\mu F'>\mu F-\gamma$.    Since 
$R[K]\cap F'=\emptyset$, $K\cap R^{-1}[F']=\emptyset$.   $R^{-1}[F']$ 
is closed, because $R$ is usco-compact, so there is some $n$ such that 
      
\Centerline{$L=\{\phi:\phi\in\NN$, $\phi\restr n=\phi'\restr n$ for some 
$\phi'\in K\}$} 
      
\noindent does not meet $R^{-1}[F']$ (4A2F(h-vi)), and 
$R[L]\cap F'=\emptyset$. 
But $L$ is just $\{\phi:\phi(i)\le\psi(i)$ for every $i<n\}$, so 
$R[L]=A_{\psi\restr n}$, and 
      
\Centerline{$\gamma<\mu^*A_{\psi\restr n}\le\mu(F\setminus F')<\gamma$,} 
      
\noindent which is absurd.\ \Bang 
      
Thus $\mu^*R[K]\ge\gamma$.   As $\gamma>0$, we have the 
result.\ \Qed 
      
Now the theorem follows at once from 416P(ii)$\Rightarrow$(i). 
}%end of proof of 432D 
      
\leader{432E}{Corollary} Let $X$ be a K-analytic Hausdorff space, and 
$\mu$ a locally finite quasi-Radon measure on $X$.   Then $\mu$ is a 
Radon measure. 
      
\proof{ By 432D, $\mu$ has an extension to a Radon measure $\mu'$.   But 
of course $\mu$ and $\mu'$ must coincide, by 415H or otherwise. 
}%end of proof of 432E 
      
\leader{432F}{Corollary} Let $X$ be a K-analytic Hausdorff space, 
and $\nu$ a locally finite Baire measure on $X$.   Then $\nu$ has an 
extension to a Radon measure on $X$;  in particular, it is 
$\tau$-additive.   If the topology of $X$ is 
regular, the extension is unique. 
      
\proof{ Because $X$ is Lindel\"of (422Gg), $\nu$ is $\sigma$-finite, 
therefore semi-finite;  by 412D, it is inner regular with 
respect to the closed sets.   So 432D tells us that it has an extension 
to a Radon measure on $X$.   Since the extension is $\tau$-additive, so 
is $\nu$. 
      
If $X$ is regular, then it must be completely regular (4A2H(b-i)), and 
the family $\Cal G$ of cozero sets is a base for the topology closed 
under finite unions.   If $\mu$, $\mu'$ are Radon measures extending 
$\nu$, they agree on $\Cal G$, and must be equal, by 415H(iv). 
}%end of proof of 432F 
      
\leader{432G}{Corollary} Let $X$ be a K-analytic Hausdorff space, $Y$ a 
Hausdorff space and $\nu$ a locally finite measure on $Y$ which is inner 
regular with respect to the closed sets.   Let $f:X\to Y$ be a 
continuous function such that $f[X]$ has full outer measure in $Y$. 
Then there is a Radon 
measure $\mu$ on $X$ such that $f$ is \imp\ for $\mu$ and $\nu$.   If 
$\nu$ is Radon, it is precisely the image measure $\mu f^{-1}$. 
      
\proof{{\bf (a)} Write $\Tau$ for the domain of $\nu$, and set 
$\Sigma_0=\{f^{-1}[F]:f\in\Tau\}$, so that $\Sigma_0$ is a 
$\sigma$-algebra of subsets of $X$, and we have a measure $\mu_0$ on $X$ 
defined by setting $\mu_0f^{-1}[F]=\nu F$ whenever $F\in\Tau$ 
(234F). 
      
\medskip 
      
{\bf (b)} If $E\in\Sigma_0$ and $\gamma<\mu_0E$, there is an $F\in\Tau$ 
such that $E=f^{-1}[F]$.   Now there is a closed set $F'\subseteq F$ 
such that $\nu F'\ge\gamma$.   Because $f$ is continuous, $f^{-1}[F']$ 
is closed, and we have $f^{-1}[F']\subseteq E$ and 
$\mu_0f^{-1}[F']\ge\gamma$.   As $E$ and $\gamma$ are arbitrary, $\mu_0$ 
is inner regular with respect to the closed sets. 
      
If $x\in X$, then (because $\nu$ is locally finite) there is an open 
set $H\subseteq Y$ such that $f(x)\in H$ and
$\nu^*H<\infty$;  as $f$ is of course \imp\
for $\mu_0$ and $\nu$, $\mu_0^*f^{-1}[H]\le\nu^*H$ (234B(f-i)) is finite,
while $f^{-1}[H]$ is an open set containing $x$.   Thus $\mu_0$ is locally
finite.
      
\medskip 
      
{\bf (c)} By 432D, there is a Radon measure $\mu$ on $X$ extending 
$\mu_0$.   Because $f$ is \imp\ for $\mu_0$ and $\nu$, it is surely 
\imp\ for $\mu$ and $\nu$. 
      
The image measure $\mu f^{-1}$ extends $\nu$, so must be locally finite; 
it is therefore a Radon measure (418I).   So if $\nu$ itself is a Radon 
measure, it must be identical with $\mu f^{-1}$, by 416Eb. 
}%end of proof of 432G 
      
\leader{432H}{Corollary} Suppose that $X$ is a set and that $\frak S$, 
$\frak T$ are Hausdorff topologies on $X$ such that $(X,\frak T)$ is 
K-analytic and $\frak S\subseteq\frak T$.   Then 
the totally finite Radon measures on $X$ are the same for $\frak S$ and 
$\frak T$. 
      
\proof{ Write $f$ for the identity function on $X$ regarded as a 
continuous function from $(X,\frak T)$ to $(X,\frak S)$.   If $\mu$ is a 
totally finite $\frak T$-Radon measure on $X$, then $\mu=\mu f^{-1}$ is 
$\frak S$-Radon, by 418I again.    
If $\nu$ is a totally finite $\frak S$-Radon 
measure on $X$, then 432G tells us that it is of the form 
$\mu=\mu f^{-1}$ for some $\frak T$-Radon measure $\mu$, that is, is 
itself $\frak T$-Radon. 
}%end of proof of 432H 
      
\leader{432I}{Corollary}\dvAnew{2010} 
Let $X$ be a K-analytic Hausdorff space, and 
$\Cal U$ a subbase for the topology of $X$.   Let $(Y,\Tau,\nu)$ be a 
complete totally finite measure space and $\phi:Y\to X$ a function such 
that $\phi^{-1}[U]\in\Tau$ for every $U\in\Cal U$.   Then there is a Radon 
measure $\mu$ on $X$ such that $\int fd\mu=\int f\phi\,d\nu$ for every 
bounded continuous $f:X\to\Bbb R$. 
 
\proof{{\bf (a)} Let $\nu\phi^{-1}$ be the image measure on $X$, and 
$\Sigma_0$ its domain.   Then $\Sigma_0$ is a  
$\sigma$-algebra of subsets of $X$ including $\Cal U$.   So if $x$, $y$ are 
distinct points of $X$, there are disjoint open sets $U$, $V\in\Sigma_0$ 
containing $x$, $y$ respectively.   \Prf\ Because $X$ is Hausdorff, there 
are disjoint open sets $U_0$ and $V_0$ such that $x\in U_0$ and $y\in V_0$. 
Because $\Sigma_0$ is closed under finite intersections and includes the 
subbase $\Cal U$, it includes a base for the topology of $X$ (4A2B(a-i)), 
and there are open sets $U$, $V\in\Sigma_0$ such that  
$x\in U\subseteq U_0$ and $y\in V\subseteq V_0$.\ \Qed 
 
Every 
cozero subset of $X$ belongs to $\Sigma_0$.   \Prf\ If $G\subseteq X$ is a 
cozero set, there is a sequence $\sequencen{F_n}$ of closed subsets of $X$ 
with union $G$.   For each $n$, $F_n$ and $X\setminus G$ are disjoint 
K-analytic subsets of $X$ (422Gf), so there is an $E_n\in\Sigma_0$ such that 
$F_n\subseteq E_n\subseteq G$ (422I).   Now $G=\bigcup_{n\in\Bbb N}E_n$ 
belongs to $\Sigma_0$.\ \Qed 
 
\medskip 
 
{\bf (b)} It follows that the Baire $\sigma$-algebra $\CalBa(X)$ of $X$ is 
included in $\Sigma_0$.   So $\mu_0=\nu\phi^{-1}\restr\CalBa(X)$  
is a Baire 
measure on $X$.   By 432F, $\mu_0$ has an extension to a Radon measure 
$\mu$ on $X$. 
 
If $f\in C_b(X)$, then $f$ is $\mu_0$-integrable;  since $\phi$ is \imp\ 
for $\nu$ and $\mu_0$, $\int f\phi\,d\nu$ is defined and equal to  
$\int fd\mu_0$ (235G).   Similarly $\int fd\mu=\int fd\mu_0$. 
So $\int fd\mu=\int f\phi\,d\nu$, as required. 
}%end of proof of 432I 
 
\leader{432J}{\dvrocolon{Capacitability}}\cmmnt{ The next theorem is 
not exactly measure theory as studied in most of this treatise;   
but it is clearly very close to the other ideas of this section, and it has 
important applications to measure theory in the narrow sense.    
      
\medskip 
      
\noindent}{\bf Definitions}\dvAformerly{4{}32I} 
Let $(X,\frak T)$ be a topological space.    
 
\spheader 432Ja A {\bf Choquet capacity} on $X$ is a function 
$c:\Cal PX\to[0,\infty]$ such that 
      
\quad(i) $c(A)\le c(B)$ whenever $A\subseteq B\subseteq X$; 
      
\quad(ii) $\lim_{n\to\infty}c(A_n)=c(A)$ whenever $\sequencen{A_n}$ is a 
non-decreasing sequence of subsets of $X$ with union $A$; 
      
\quad(iii) $c(K)=\inf\{c(G):G\supseteq K$ is open$\}$ for every compact set 
$K\subseteq X$. 
%Kechris demands c(K)<\infty for compact K 
 
\spheader 432Jb\dvAnew{2008}  
A Choquet capacity $c$ on $X$ is {\bf outer regular} if 
$c(A)=\inf\{c(G):G\supseteq A$ is open$\}$ for every $A\subseteq X$. 
 
\spheader 432Jc\dvAnew{2008}  
If $P$ is any lattice, an order-preserving 
function $c:P\to\ocint{-\infty,\infty}$ is 
{\bf submodular} if $c(p\wedge q)+c(p\vee q)\le c(p)+c(q)$ for all $p$, 
$q\in P$.   \cmmnt{The phrase `strongly subadditive' is used by many 
authors in similar contexts.} 
%"strongly subadditive" in Choquet 69, Zapletal p06, Kechris 95. 
 
      
\leader{432K}{Theorem}\cmmnt{ ({\smc Choquet 55})} Let $X$ be a 
Hausdorff space and $c$ a Choquet capacity on $X$.   If $A\subseteq X$ is 
K-analytic, then $c(A)=\sup\{c(K):K\subseteq A$ is compact$\}$. 
      
\proof{ Take $\gamma<c(A)$.   Let $R\subseteq\NN\times X$ be an 
usco-compact relation such that $R[\NN]=A$;  for 
$\sigma\in S=\bigcup_{n\in\Bbb N}\BbbN^n$ set 
      
\Centerline{$A_{\sigma}=\{x:(\phi,x)\in R$ for some $\phi\in\NN$ such 
that $\phi(i)\le\sigma(i)$ for every $i<\#(\sigma)\}$.} 
      
\noindent Then $\sequence{i}{A_{\sigma^{\smallfrown}\fraction{i}}}$ is a 
non-decreasing sequence with union $A_{\sigma}$, so 
      
\Centerline{$c(A_{\sigma})=\sup_{i\in\Bbb 
N}c(A_{\sigma^{\smallfrown}\fraction{i}})$} 
      
\noindent for every $\sigma\in S$.   We can therefore find a sequence 
$\psi\in\NN$ such that $c(A_{\psi\restr n})>\gamma$ for every 
$n\in\Bbb N$.   Set 
      
\Centerline{$K=\{\phi:\phi\in\NN,\,\phi(i)\le\psi(i)$ for every 
$i\in\Bbb N\}$;} 
      
\noindent then $K=\prod_{n\in\Bbb N}(\psi(n)+1)$ is compact, so $R[K]$ 
is compact (422D(e-i)). 
      
\Quer\ Suppose, if possible, that $c(R[K])<\gamma$.   Then, by (iii) of 
432J, there is an open set $G\supseteq R[K]$ such that $c(G)<\gamma$. 
Set $F=X\setminus G$, so that $F$ is closed and 
$K\cap R^{-1}[F]=\emptyset$.   $R^{-1}[F]$ is closed, because $R$ is 
usco-compact, so there is some $n$ such that 
      
\Centerline{$L=\{\phi:\phi\in\NN$, $\phi\restr n=\phi'\restr n$ for some 
$\phi'\in K\}$} 
      
\noindent does not meet $R^{-1}[F]$ (4A2F(h-vi) again), and 
$R[L]\cap F=\emptyset$, that is, $R[L]\subseteq G$.   But $L$ is just 
$\{\phi:\phi(i)\le\psi(i)$ for every $i<n\}$, so 
$R[L]=A_{\psi\restr n}$, and 
      
\Centerline{$\gamma<c(A_{\psi\restr n})\le c(G)<\gamma$,} 
      
\noindent which is absurd.\ \Bang 
      
Thus $c(R[K])\ge\gamma$.   As $\gamma$ is arbitrary and $R[K]$ is 
compact, we have the result. 
}%end of proof of 432K 
 
\leader{432L}{Proposition}\dvAnew{2008} 
Let $(X,\frak T)$ be a topological space. 
 
(a) Let $c_0:\frak T\to[0,\infty]$ be a functional such that  
 
\inset{$c_0(G)\le c_0(H)$ whenever $G$, $H\in\frak T$ and $G\subseteq H$; 
 
$c_0$ is submodular; 
 
$c_0(\bigcup_{n\in\Bbb N}G_n)=\lim_{n\to\infty}c_0(G_n)$ for every 
non-decreasing sequence $\sequencen{G_n}$ in $\frak T$.} 
 
\noindent Then $c_0$ has a unique extension to an outer regular  
Choquet capacity $c$ on $X$, and $c$ is submodular. 
 
(b) Suppose that $X$ is regular.   Let $\Cal K$ be the family of compact 
subsets of $X$, and $c_1:\Cal K\to[0,\infty]$ a functional such that 
 
\inset{$c_1$ is submodular; 
 
$c_1(K) 
=\inf_{G\in\frak T,G\supseteq K}\sup_{L\in\Cal K,L\subseteq G}c_1(L)$ 
for every $K\in\Cal K$.} 
 
\noindent Then $c_1$ has a unique extension to an outer regular 
Choquet capacity $c$ on $X$ such that  
 
\Centerline{$c(G)=\sup\{c(K):K\subseteq G$ is compact$\}$  
for every open $G\subseteq X$,} 
 
\noindent and $c$ is submodular. 
 
\proof{{\bf (a)} For $A\subseteq X$, set  
$c(A)=\inf\{c_0(G):A\subseteq G\in\frak T\}$.   Then  
$c:\Cal PX\to[0,\infty]$ extends $c_0$ because $c_0$ is order-preserving. 
Conditions (i) and (iii) of 
432J are obviously satisfied.   As for (ii), let $\sequencen{A_n}$ be a 
non-decreasing sequence of subsets of $X$ with union $A$.   Then 
$\lim_{n\to\infty}c(A_n)$ is defined and not greater than $c(A)$. 
If the limit is infinite, then certainly it is equal to 
$c(A)$.   Otherwise, take $\epsilon>0$.   For each $n\in\Bbb N$, choose 
an open set $G_n\supseteq A_n$ such that  
$c_0(G_n)\le c(A_n)+2^{-n}\epsilon$.   Set $H_n=\bigcup_{i\le n}G_i$ 
for $n\in\Bbb N$.   Then $c_0(H_n)\le c(A_n)+2\epsilon-2^{-n}\epsilon$ for 
every $n$.   \Prf\ Induce on $n$.   If $n=0$ then $H_0=G_0$ and the 
result is immediate.   For the inductive step to $n+1$, we have 
 
$$\eqalignno{c_0(H_{n+1})+c_0(H_n\cap G_{n+1}) 
&\le c_0(H_n)+c_0(G_{n+1})\cr 
\displaycause{because $c_0$ is submodular} 
&\le c(A_n)+2\epsilon-2^{-n}\epsilon+c(A_{n+1})+2^{-n-1}\epsilon\cr 
\displaycause{using the inductive hypothesis} 
&\le c_0(H_n\cap G_{n+1})+2\epsilon-2^{-n-1}\epsilon+c(A_{n+1});\cr}$$ 
 
\noindent as $c_0(H_n\cap G_{n+1})\le c_0(G_{n+1})$ is finite, 
$c_0(H_{n+1})\le 2\epsilon-2^{-n-1}\epsilon+c(A_{n+1})$, as required.\ \Qed 
 
Set $H=\bigcup_{n\in\Bbb N}H_n$.   As $\sequencen{H_n}$ is non-decreasing, 
 
\Centerline{$c(A)\le c_0(H)=\lim_{n\to\infty}c_0(H_n) 
\le\lim_{n\to\infty}c(A_n)+2\epsilon$.} 
 
\noindent As $\epsilon$ is arbitrary, $c(A)\le\lim_{n\to\infty}c(A_n)$ and 
the final condition of 432J is satisfied. 
 
Of course $c$ is the only Choquet capacity extending $c_0$ and outer 
regular with respect to the open sets. 
 
As for the submodularity of $c$, if $A$, $B\subseteq X$ and $\epsilon>0$, 
there are open sets $G\supseteq A$ and $H\supseteq B$ such that 
$c_0(G)+c_0(H)\le c(G)+c(H)+\epsilon$;  so that 
 
$$\eqalign{c(A\cup B)+c(A\cap B) 
&\le c_0(G\cup H)+c_0(G\cap H)\cr 
&\le c_0(G)+c_0(H) 
\le c(A)+c(B)+\epsilon.\cr}$$ 
 
\noindent As $\epsilon$ is arbitrary,  
$c(A\cup B)+c(A\cap B)\le c(A)+c(B)$, as required. 
 
\medskip 
 
{\bf (b)(i)} The key fact is this:  if $G$, $H\in\frak T$, 
$K$, $L\in\Cal K$, $K\subseteq G\cup H$ and $L\subseteq G\cap H$, then 
there are $K_1$, $L_1\in\Cal K$ such that $K_1\subseteq G$, 
$L_1\subseteq H$, $K\subseteq K_1\cup L_1$ and $L\subseteq K_1\cap L_1$. 
\Prf\ Because $(X,\frak T)$ is regular, there is an open set 
$G_1$ such that $K\setminus H\subseteq G_1$ and 
$\overline{G}_1\subseteq G$ (4A2F(h-ii)).   Set 
$K_1=(K\cap\overline{G}_1)\cup L$, 
$L_1=(K\setminus G_1)\cup L$;  these work.\ \Qed 
 
\medskip 
 
\quad{\bf (ii)} Define $c_0:\frak T\to[0,\infty]$ by setting 
$c_0(G)=\sup_{K\in\Cal K,K\subseteq G}c_1(K)$ for open $G\subseteq X$. 
Then $c_0$ satisfies the conditions of (a).   \Prf\ The first and third are 
elementary.   As for the second, if $G$, $H\in\frak T$ and 
$\gamma<c_0(G\cup H)+c_0(G\cap H)$, there are $K$, $L\in\Cal K$ such that 
$K\subseteq G\cup H$, $L\subseteq G\cap H$ and  
$\gamma\le c_1(K)+c_2(L)$.   Now (i) tells us that there are compact sets 
$K_1\subseteq G$ and $L_1\subseteq H$ such that $K\subseteq K_1\cup L_1$ 
and $L\subseteq K_1\cap L_1$, in which case 
 
$$\eqalign{c_0(G)+c_0(H) 
&\ge c_1(K_1)+c_1(L_1) 
\ge c_1(K_1\cup L_1)+c_1(K_1\cap L_1)\cr 
&\ge c_1(K)+c_1(L) 
\ge\gamma.\cr}$$ 
 
\noindent As $\gamma$ is arbitrary, 
$c_0(G)+c_0(H)\ge c_0(G\cup H)+c_0(G\cap H)$, as required.\ \Qed 
 
\medskip 
 
\quad{\bf (iii)} We therefore have a submodular outer regular 
Choquet capacity $c:\Cal PX\to[0,\infty]$ 
defined by setting $c(A)=\inf_{G\in\frak T,A\subseteq G}c_0(G)$ for every 
$A\subseteq X$.   From the second condition on $c_1$, we see that $c$ 
extends $c_1$.   Clearly $c$ satisfies the two regularity conditions, and 
is the only extension of $c_1$ which does so.    
}%end of proof of 432L 
      
\exercises{ 
\leader{432X}{Basic exercises (a)} 
%\spheader 432Xa 
Put 422Xf, 431Xb and 432D together to prove 432C. 
%432C, 432D 
%422Xf > Borel sets K-analytic [therefore Souslin-F].    
%431Xb > \mu is i.r. wrt closed sets. 
      
\spheader 432Xb Let $X$ be a K-analytic Hausdorff space, and $\mu$ a 
measure on $X$ which is outer regular with respect to the open sets. 
Show that $\mu X=\sup_{K\subseteq X\text{ is compact}}\mu^*K$. 
\Hint{see the proof of 432D.} 
%432D 
      
\sqheader 432Xc Let $X$ be a K-analytic Hausdorff space, and $\mu$ a 
semi-finite topological measure on $X$.   Show that if {\it either} 
$\mu$ is inner regular with respect to the closed sets %432D
{\it or} $X$ is 
regular and $\mu$ is a $\tau$-additive Borel measure, then $\mu$ is 
tight. %414M 
%432E 
      
\spheader 432Xd Use 422Gf, 432B and 416C to prove 432E. 
%432E 
%422Gf > closed sset of K-analytic is K-analytic 
%416C > quasi-Radon + \sup_K K^{\ssbullet}=1 => Radon      
 
\sqheader 432Xe Suppose that $X$ is a set and that $\frak S$, $\frak T$ 
are Hausdorff topologies on $X$ such that $(X,\frak T)$ is K-analytic 
and $\frak S\subseteq\frak T$.   Let $(Z,\frak U,\Tau,\nu)$ be a Radon 
measure space and $f:Z\to X$ a function which is almost continuous for 
$\frak U$ and $\frak S$.   Show that $f$ is almost continuous for 
$\frak U$ and $\frak T$.   \Hint{it is enough to consider totally finite 
$\nu$;  show that $\nu f^{-1}$ is $\frak T$-Radon, so is inner regular 
for $\{K:\frak T_K=\frak S_K\}$, writing $\frak T_K$ for the subspace 
topology induced by $\frak T$ on $K$.} 
%432H 
      
\spheader 432Xf Let $X$ be a topological space and $\mu$ a locally 
finite measure on $X$ which is inner regular with respect to the closed 
sets.   Show that $\mu^*$ is a Choquet capacity. 
%432J 
      
\spheader 432Xg Let $X$ be a topological space and $F$ a closed subset 
of $X$.   Define $c:\Cal PX\to\{0,1\}$ by setting $c(A)=1$ if $A$ meets 
$F$, $0$ otherwise.   Show that $c$ is a Choquet capacity on $X$. 
%432J 
      
\spheader 432Xh Let $X$ and $Y$ be Hausdorff spaces, and 
$R\subseteq X\times Y$ an usco-compact relation.   Show that if $c$ is a 
Choquet capacity on $Y$, then $A\mapsto c(R[A])$ is a Choquet
capacity on $X$. 
%432J 
      
\spheader 432Xi Use 432K and 432Xf to shorten the proof of 432D. 
%432K 
 
\spheader 432Xj 
Let $P$ be a lattice, and $c:P\to\coint{0,\infty}$ a submodular 
order-preserving functional.    
For $p$, $q\in P$ set $\rho(p,q)=2c(p\vee q)-c(p)-c(q)$.   Show that $\rho$ 
is a pseudometric. 
%432J out of order query 
      
\leader{432Y}{Further exercises (a)} 
%\spheader 432Ya 
Show that there are a K-analytic Hausdorff space $X$ and a probability 
measure $\mu$ on $X$ such that (i) $\mu$ is inner regular with respect 
to the Borel sets (ii) the domain of $\mu$ includes a base for the 
topology of $X$ (iii) every compact subset of $X$ is negligible.   Show 
that there is no extension of $\mu$ to a topological measure on $X$. 
%432B
%432D %mt43bits 
 
\spheader 432Yb\dvAnew{2011} 
Let $X$, $Y$ be Hausdorff spaces, $R\subseteq X\times Y$ an usco-compact 
relation and $\mu$ a Radon probability measure on $X$ such that  
$\mu_*R^{-1}[Y]=1$.   Show that there is a Radon probability measure on $Y$ 
such that $\nu_*R[A]\ge\mu_*A$ for every $A\subseteq X$. 
%418L applied to projection from  R  onto  R^{-1}[Y]
       
\leaveitout{\spheader 432Y? Let $X$ be a set with its discrete topology, 
and $\kappa$ a cardinal of uncountable cofinality.   Define 
$c:\Cal PX\to\{0,1\}$ by setting $c(A)=0$ if $\#(A)\le\kappa$, $1$ 
otherwise.   Show that $c$ is a Choquet capacity on $X$. 
%432J
}%end of leaveitout 
      
      
}%end of exercises 
      
\leaveitout{\leader{432Z}{Problem} Let $X$ be a K-analytic Hausdorff 
space and $\mu$ a probability measure on $X$ such that (i) the domain of 
$\mu$ includes a base for the topology of $X$ (ii) $\mu$ is inner 
regular with respect to the Borel sets.   Must there be a non-negligible 
compact set? 
 
$\mu$ measures every Baire set and 
there is a unique 
Radon measure agreeing with $\mu$ on the Baire sets, I think. 
}%end of leaveitout 
      
\cmmnt{\Notesheader{432} 
The measure-theoretic properties of K-analytic spaces can largely be 
summarised in the slogan `K-analytic spaces have lots of compact sets'. 
I said above that it is sometimes helpful to think of K-analytic spaces 
as an amalgam of compact Hausdorff spaces and Souslin-F subsets of 
$\Bbb R$.   For the former, it is obvious that they have many compact 
subsets; 
for the latter, it is not obvious, but is of course one of their 
fundamental properties, deducible from 422De.   432B and the proof of 432D (repeated in 432K) are typical manifestations of the phenomenon. 
The real point of these theorems is that we can extend a Borel or Baire 
measure to a Radon measure with no prior assumption of $\tau$-additivity 
(432F).   A Radon measure must be $\tau$-additive just because it is 
tight.   A (locally finite) 
Borel or Baire measure must be $\tau$-additive whenever the measurable 
open sets are K-analytic. 
      
The condition `every open set is K-analytic' in 432C is of course a very 
strong one in the context of compact Hausdorff spaces (422Xe).   But for 
analytic spaces it is automatically satisfied (423Eb), and 
that is the side on which the principal applications of 432C appear. 
      
The results which I call corollaries of 432D can mostly be proved by 
more direct methods (see 432Xd), but the line I choose here seems to be 
the most powerful technique.   Indeed it can be used to deal with 432C 
as well (432Xa). 
      
In \S434 I will discuss `universally measurable' sets in topological 
spaces.   In 
fact K-analytic sets are universally measurable in a particularly strong 
sense (432A).   The point here is that K-analyticity is intrinsic;  a 
K-analytic space is measurable whenever embedded as a subspace of a 
(complete locally determined) Hausdorff topological measure space. 
      
The theorems here touch on two phenomena of particular importance. 
First, in 432G we have an example of `pulling back' a measure, that 
is, we have a measure $\nu$ on a set $Y$ and a function $f:X\to Y$ and 
seek a Radon measure $\mu$ on $X$ such that $f$ is \imp, or, even 
better, such 
that $\nu=\mu f^{-1}$.   There was a similar result in 418L.   In both 
cases we have to suppose that $f$ is continuous and (in effect) that 
$\nu$ is a Radon measure.   (This is not part of the hypotheses of 432G, 
but of course it is an easy consequence of them, using 432B.)   In 418L, 
we need a special hypothesis to ensure that there are enough compact 
subsets of $X$ to carry an appropriate Radon measure;  in 432G, this is 
an automatic result of assuming that $X$ is K-analytic.   Both 418L and 
432G can be regarded as consequences of Henry's theorem (416N).   The 
difficulty arises from the requirement that $\mu$ should be a Radon 
measure;  if we do not insist on this there is a much simpler solution, 
since we need suppose only that $f[X]$ has full outer measure (234F). 
      
The next theme I wish to mention is a related one, the investigation of 
comparable topologies.   If $\frak S$ and $\frak T$ are (Hausdorff) 
topologies on a set $X$, and $\frak S$ is coarser than $\frak T$ (so 
that $(X,\frak S)$ is a continuous image of $(X,\frak T)$), then 
418I tells us that any totally finite $\frak T$-Radon 
measure is $\frak S$-Radon.   We very much want to know when the reverse 
is true, so that the (totally finite) Radon measures for the two 
topologies are the same.   432H provides one of the important cases in 
which this occurs.   The hypothesis `$(X,\frak T)$ is K-analytic' 
generalizes the alternative \lq$(X,\frak T)$ is compact';  in the latter 
case, $\frak S=\frak T$, so that the result is, from our point of view 
here, trivial.   (But from the point of view of elementary general 
topology, of course, it is one of the pivots of the theory of compact 
Hausdorff spaces.)   In a similar vein we have a variety of important 
topological consequences of the same hypotheses (422Yf, 423Fb). 
      
The paragraphs 432J-432L %432J 432K 432L
may appear to be no more that a minor 
extension of ideas already set out.   I ought therefore to say plainly 
that the topological and measure theory of K-analytic spaces have 
co-evolved with the notion of capacity, and that 432K (`K-analytic 
spaces are capacitable') is one of the cornerstones of a theory of which 
I am giving only a minuscule part.   For a idea of the vitality and 
scope of this theory, see {\smc Dellacherie 80}. 
}%end of notes 
      
\discrpage 
  
