%(If you can read TeX fluently, you may wish to fast-forward to a line
%beginning   %file starts here   or   \frfilename
%if one of these is present.)

%TeXfiles with this heading are designed for processing with AmSTeX,
%accessed through TeXfiles  amstex.tex (with amssym.tex  and
% amsppt.sty  and possibly  amsppt.sti)  during compilation.   
%If these are unavailable, a readable
%version can (I hope) be produced by deleting the  %  before
% \amstexsupportedfalse  below.

%Some special difficulties arise with the importation of  .ps  files
%in diagrams.   Look for  \picture  below for notes on these.

%The following \count registers are used for special purposes:
%   \count1=\stylenumber  to specify style
%   \count2=\footnotenumber  to count footnotes

%AmSTeX declaration
\newif\ifamstexsupported
     \amstexsupportedtrue
%     \amstexsupportedfalse               %use when AMSTeX unavailable
\newif\ifDHFamstexloaded        %for detection of a non-standard AMSTeX
     \DHFamstexloadedfalse    
\newif\ifDHFamspptloaded        
     \DHFamspptloadedfalse    
\newif\iffourteensavailable
     \fourteensavailabletrue
    \fourteensavailablefalse
\newif\ifathome
     \athometrue     %to cope with a quirk     of my son's Ghostscript
     \athomefalse
\newif\iffirstpage
     \firstpagetrue

%Style assignments
\countdef\stylenumber=1
\countdef\footnotenumber=2
\def\fullsize{\magnification=\magstep1
     \def\pagewidth{390pt}
     \def\pageheight{9truein}
     \hsize\pagewidth
     \vsize\pageheight
     \stylenumber=1}
\def\smallprint{\magnification=\magstep0
     \overfullrule0pt
     \def\pagewidth{468pt}
     \def\pageheight{9truein}
     \hsize\pagewidth
     \vsize\pageheight
     \stylenumber=2}
\def\printersize{\magnification=\magstep1
     \baselineskip24pt
     \overfullrule0pt
     \def\pagewidth{330pt}
     \def\pageheight{8.5truein}
     \hsize\pagewidth
     \vsize\pageheight
     \stylenumber=3}
\def\refsize{\magnification=\magstep1
     \overfullrule0pt
     \def\pagewidth{390pt}
     \def\pageheight{9truein}
     \hsize\pagewidth
     \vsize\pageheight
     \stylenumber=4}
\def\rrversion{\magnification=\magstep1
     \overfullrule0pt
     \def\pagewidth{330pt}
     \def\pageheight{9truein}
     \hsize\pagewidth
     \vsize\pageheight
     \stylenumber=5}
\def\bigsize{\magnification=\magstep2
     \overfullrule0pt
     \def\pagewidth{325pt}
     \def\pageheight{9truein}
     \hsize\pagewidth
     \vsize\pageheight
     \stylenumber=6}
\def\bbigsize{\magnification=\magstep3
     \overfullrule0pt
     \def\pagewidth{271pt}
     \def\pageheight{9truein}
     \hsize\pagewidth
     \vsize\pageheight
     \stylenumber=7}
\def\USsmallprint{\magnification=\magstep0
     \overfullrule0pt
     \def\pagewidth{482pt}
     \def\pageheight{8.7truein}
     \hsize\pagewidth
     \vsize\pageheight
     \stylenumber=8}
\def\halfmagsize{\magnification=\magstephalf
     \def\pagewidth{427pt}
     \def\pageheight{9truein}
     \hsize\pagewidth
     \vsize\pageheight
     \stylenumber=9}
           % register  stylenumber  is used to control the output format

%addresses, letterheads
\def\deptaddress{\hskip 3truein Department of Mathematical Sciences\par
   \noindent e-mail: {\tt fremdh}\hskip 2.9truein\today\par
   \medskip}
\long\def\univaddress#1{\hskip 3truein Department of Mathematical Sciences\par
   \hskip 3truein University of Essex\par
   \hskip 3truein Colchester CO4 3SQ\par
   \hskip 3truein England\par
   %\hskip 3truein van Vleck Hall\par
   %\hskip 3truein 480 Lincoln Drive\par
   %\hskip 3truein Madison\par
   %\hskip 3truein WI 53706-1388, U.S.A.\par
   \hskip 3.2truein#1\par
   \noindent e-mail: {\tt fremdh\@essex.ac.uk}
       %{\tt fremlin\@math.wisc.edu}
   \bigskip}
\long\def\letterhead#1#2#3{\univaddress{#1}
   {\parindent=0pt
    #2
   }
   \medskip

   \noindent Dear #3,
     }
\def\shortaddress{\noindent Departmentof Mathematical Sciences, University of Essex,
   Colchester CO4 3SQ, England\par
   \noindent e-mail: fremdh\@essex.ac.uk}
\def\valediction#1#2{\bigskip
   \hskip 3truein #1,
   \vskip 1truein
   \hskip 3.3truein #2
   \bigskip}

%picture importation
   %It is clear that there is no general standard.   The syntax I
   %use when calling for a  .ps  file is
   %    \def\Caption{<...>}
   %    \picture{<filename>}{<dim>}
   %which is supposed to import  <filename>.ps  at height  <dim> ,
   %centered, and adding a one-line centered caption  <...> .
   %The basic macro listed in  \def\picture  is that which
   %used to work at Essex.   We now use the macro in
   % \def\picturewithpsfig , which requires a previous input
   % psfig.tex  or  psfig.sty ;
   %if you interpolate  \atUEssex  early in your file it will
   %give you the latter.
   %   I find that  psfig  is erratic in its centering, especially
   %under magnification, so I have written a routine
   % \centerpicturewithpsfig  (which may be called from
   % \centerpicture ) with the syntax
   %    \centerpicture{<filename>}{<dim>}{<dim>}
   %the first dimension being width and the second height.
\def\picture#1#2{\bigskip
  \bigskip
  \immediate\write0{importing diagram #1.ps at height #2
         with dvitops}
  \vbox{\vskip #2
    \special{dvitops: import #1.ps \pagewidth #2}
    \Centerline{\Caption}
    }
   \medskip
   \def\Caption{}}
%an alternative version, without caption
\def\centerpicture#1#2#3{\bigskip
  \bigskip
  \immediate\write0{importing diagram #1.ps at width #2
         and height #3 with dvitops}
  \vbox{\vskip #3
    \special{dvitops: import #1.ps \pagewidth #3}
  }}
%to put pictures on same line
\def\sideshiftedpicture#1#2#3#4{\bigskip
   \bigskip
   \immediate\write0{importing diagram #1.ps at height #4
        and width #3 with dvitops}
   \vbox{\special{dvitops:  import #1.ps #3 #4}}
   \hskip #2 plus 0pt minus 0pt
   }
\def\startsideshiftedpicture{}
\newcount\scaleup
\newcount\scaledown
\scaleup=1
\scaledown=1

\def\scalesideshiftedpicture#1#2#3#4{
  \dimen4=#4
  \dimen3=#3
  \multiply\dimen4 by\scaleup
  \divide\dimen4 by \scaledown
  \multiply\dimen3 by\scaleup
  \divide\dimen3 by \scaledown
  \sideshiftedpicture{#1}{#2}{\dimen3}{\dimen4}}


%alternative picture importation code
\def\atUEssex{
   \input psfig.sty
   \def\picture##1##2{\picturewithpsfig{##1}{##2}}
   \def\sideshiftedpicture##1##2##3##4{
     \sideshiftedpicturewithpsfig{##1}{##2}{##3}{##4}}
   \def\centerpicture##1##2##3{
       \centerpicturewithpsfig{##1}{##2}{##3}}
   \def\startsideshiftedpicture{
       \sideshiftedpicture{empty}{1pt}{1pt}{1pt}}
   }
\def\athome{
   \input psfig.sty
   \def\picture##1##2{\picturewithpsfig{##1}{##2}}
   \def\sideshiftedpicture##1##2##3##4{
     \sideshiftedpicturewithpsfig{##1}{##2}{##3}{##4}}
   \def\centerpicture##1##2##3{
       \centerpicturewithpsfig{##1}{##2}{##3}}
   \def\startsideshiftedpicture{
       \sideshiftedpicture{empty}{1pt}{1pt}{1pt}}
   \voffset=1cm}
\def\atUWMadison{
   \fourteensavailablefalse
   \input psfig
   \def\picture##1##2{\picturewithpsfig{##1}{##2}}
   }
\def\picturewithpsfig#1#2{\bigskip
   \immediate\write0{importing diagram #1.ps at height #2 with psfig}
   \vbox{\hsize\pagewidth
%\centerline{\discrversionA{\hskip1truein}{}\psfig{figure=#1.ps,height=#2
   \centerline{\psfig{figure=#1.ps,height=#2
}}
   \medskip
   \centerline{\Caption}}
   \medskip
   \def\Caption{}}
\def\centerpicturewithpsfig#1#2#3{
  \immediate\write0{importing diagram #1.ps at width #2
         and height #3 with psfig}
  \dimen 3 = #2
  \dimen 4 = #3
  \dimen 2 = \pagewidth
  \advance\dimen2 by -\dimen3
  \divide\dimen2 by 2
  \discrversionB{\shrinkonestep}{}
  \discrversionH{\shrinkonestep}{}
  \discrversionF{\enlargeonestep}{}
  \discrversionG{\enlargeonestep\enlargeonestep}{}
  \dimen 6 = \dimen4
  \divide\dimen6 by 3
  \vskip\dimen6
  \ifdim\dimen2>0\hskip\dimen2\fi
     \vbox{\psfig{file=#1.ps,width=\dimen3,height=\dimen4}}
  \discrversionG{\vskip-\dimen6}{}
  }
\def\enlargeonestep{\dimen5 = 5truein
   \dimen6 = 7truein
   \ifdim\dimen3<\dimen5
      \ifdim\dimen4<\dimen6
         \divide\dimen3 by 5
         \multiply\dimen3 by 6
         \divide\dimen4 by 5
         \multiply\dimen4 by 6
      \fi\fi}
\def\shrinkonestep{\divide\dimen3 by 6
         \multiply\dimen3 by 5
         \divide\dimen4 by 6
         \multiply\dimen4 by 5}
\def\sideshiftedpicturewithpsfig#1#2#3#4{
   \immediate\write0{importing diagram #1.ps at height #4
      and width #3 with psfig}
   \dimen 4 = #4
   \dimen 3 = #3
   \discrversionB{\shrinkonestep}{}
   \discrversionH{\shrinkonestep}{}
   \discrversionF{\enlargeonestep}{}
   \discrversionG{\enlargeonestep\enlargeonestep}{}
%   \vbox{\psfig{figure=#1.ps,height=\dimen4,width=\dimen3}}
   \psfig{figure=#1.ps,height=\dimen4,width=\dimen3}
   \hskip #2 plus 0pt minus 0pt}
%alternative picture importation code;  I've been told about this
  %but have not used it successfully myself
%\def\Caption{}
%\input epsf
%\epsfverbosetrue
%\def\picture#1#2{\bigskip
%    \bigskip
%    \immediate\write0{importing diagram #1.ps at height #2 with epsf}
%    \epsfysize=#2
%    \centerline{\epsffile{#1.ps}}
%    \medskip
%    \Centerline{\Caption}
%    \medskip
%    }

%in addition it may be necessary to amend the imported  .ps  file
%by replacing the   %%BoundingBox   header by  %%BoundingBox:


%special fonts
\def\biggerscriptfonts{\scriptfont0=\eightrm
     \scriptfont1=\eighti
     \scriptfont2=\eightsy
     \scriptfont3=\eightex
     \scriptfont\bffam=\eightbf
     \scriptfont\msbfam=\eightmsb
     \scriptfont\eufmfam=\eighteufm
     }
\def\Loadcyrillic{\ifamstexsupported
      \input cyracc.def
      \font\tencyr=wncyr10
      \def\cyr{\tencyr\cyracc} \def\itcyr{\tenitcyr\cyracc}
    \else\def\cyr{\smc}\def\cdprime{u\i}\def\Cdprime{UI}\fi}
%http://wkweb3.cableinet.co.uk/linux/LDP/HOWTO/Cyrillic-HOWTO-7.html
%cyrillic.tex

\def\Loadfourteens{\iffourteensavailable
       \font\fourteenrm=cmr14
       \font\fourteenbf=cmbx14
       \font\fourteenit=cmsl14
   \else\font\fourteenrm=cmr10 scaled \magstep2
      \font\fourteenbf=cmbx10 scaled \magstep2
      \font\fourteenit=cmsl10 scaled \magstep2
   \fi}
\def\smallerfonts{\eightpoint
   \textfont\msbfam=\eightmsb
   \textfont\eufmfam=\eighteufm
   \textfont\bffam=\eightbf
   \textfont\eurmfam=\eighteurm
   \textfont\eusmfam=\eighteusm
   }
\font\twelveex=cmex10 scaled \magstep1


%Copyright declarations
\newcount\oldcopyrightdate
\newcount\newcopyrightdate
\oldcopyrightdate=0
\def\copyrightdate#1{\newcopyrightdate=#1
       \ifnum\newcopyrightdate=\oldcopyrightdate\else
       \oldcopyrightdate=#1
       \oldfootnote{}{\eightrm \copyright\ #1 D.~H.~Fremlin}
       \immediate\write0{Copyright #1}
       \fi}

%hyphenation
\hyphenation{equi-ver-id-ical}

%miscellaneous definitions
\def\Aut{\mathop{\text{Aut}}}
\def\backstep#1{\hskip -0.#1em plus 0 pt minus 0 pt}
\def\Bang{\vthsp\hbox{\bf X\hskip0.07em\llap X\hskip0.07em\llap X\rm}}
\def\BanG{\Bang{\hskip 1.111pt plus 3.333pt minus 0pt}}
%\def\BbbN{\mathchoice{\hbox{$\Bbb N$\hskip0.02em}}
%  {\hbox{$\Bbb N$\hskip0.02em}}
%  {\hbox{$\scriptstyle\Bbb N$\hskip0.04em}}
%  {\hbox{$\scriptscriptstyle\Bbb N$\hskip0.04em}}}
%\def\BbbQ{\mathchoice{\hbox{$\Bbb Q$\hskip0.02em}}
%  {\hbox{$\Bbb Q$\hskip0.02em}}
%  {\hbox{$\scriptstyle\Bbb Q$\hskip0.04em}}
%  {\hbox{$\scriptscriptstyle\Bbb Q$\hskip0.04em}}}
%\def\BbbR{\mathchoice{\hbox{$\Bbb R$\hskip0.02em}}
%  {\hbox{$\Bbb R$\hskip0.02em}}
%  {\hbox{$\scriptstyle\Bbb R$\hskip0.04em}}
%  {\hbox{$\scriptscriptstyle\Bbb R$\hskip0.04em}}}
%  %do not use with $\Bbb R^X$
%\def\BbbZ{\mathchoice{\hbox{$\Bbb Z$\hskip0.02em}}
%  {\hbox{$\Bbb Z$\hskip0.02em}}
%  {\hbox{$\scriptstyle\Bbb Z$\hskip0.04em}}
%  {\hbox{$\scriptscriptstyle\Bbb Z$\hskip0.04em}}}
\def\BbbN{\openBbb{N}}
\def\BbbQ{\openBbb{Q}}
\def\BbbR{\openBbb{R}}
  %do not use with $\Bbb R^X$  do not use in $_{\Bbb R^r}$
\def\BbbZ{\openBbb{Z}}
\def\Bbbone{\hbox{1\hskip0.13em\llap{1}}}
\def\biggerint{\textfont3=\twelveex\intop\nolimits}
  % you can get this size into a displayed formula with
  %  $$\hbox{$\biggerint_{\Cal S}\pmb{u}\,d\pmb{v}$}$$
  % but it may then seem cramped
\def\biggeriint{\textfont3=\twelveex\iint}
  % remarkable fact:  this seems to infect all \int within a given
  % $..$ ; but  $\biggerint$$\int$ seems to leave the $\int$ normal
  % really  \textfont3=\twelveex  is more reliable, since  \biggerint
  % as written gets turned off rather easily (e.g. by  \comment,
  % or by an extra  {...})
\def\bover#1#2{{{#1}\over{#2}}}
\def\Bover#1#2{\backstep4\hbox{
     \biggerscriptfonts${\shortstrut#1}\over{\shortstrut#2}$}}
  % when there are superscripts in the denominator, as in \Bover1{2^m}
  % or  \Bover1{\sqrt{2\pi}^r} , these can sometimes come out with better
  % horizontal alignment if you use  \Bover1{{\sshortstrut 2}^m}  or
  % \Bover1{{\sqrt{2\pi}}^r}
\def\bsover#1#2{\hbox{${#1}\over{#2}$}}
\def\bsp{/\penalty-100}
\def\Caption{}
\def\Center#1{

   \Centerline{#1}

   \noindent}
\def\Centerline#1{\vskip 5pt
     {\def\int{\text{}$$\textfont3=\twelveex\intop\nolimits}
      \def\dashint{\text{}$$\textfont3=\twelveex
         \mskip2mu\vrule height2.65pt width4.5pt depth-2.35pt\mskip-11mu
	 \intop\nolimits\mskip-2mu}
      \def\idashint{\text{}$$\int\mskip-7mu\dashint}	 
      \def\dashiint{\text{}$$\dashint\mskip-6mu\int}	 
      \def\dashidashint{\text{}$$\dashint\mskip-6mu\dashint}	 
      \def\underlineint{
           \text{}$$\textfont3=\twelveex\underline{\intop\nolimits}}
      \def\overlineint{
           \text{}$$\textfont3=\twelveex\overline{\intop\nolimits}}
      \def\iint{\text{}$$\textfont3=\twelveex
           \intop\nolimits\hskip-0.5em\intop\nolimits}
      \def\sum{\text{}$$\sumop}
      \def\prod{\text{}$$\prodop}
      \def\bigcap{\text{}$$\bigcapop}
      \def\bigcup{\text{}$$\bigcupop}
   \centerline{#1}}\vskip 5pt}
   \mathchardef\bigcapop="1354
   \mathchardef\bigcupop="1353
   \mathchardef\prodop="1351
   \mathchardef\sumop="1350
   %in  \Centerline{$\overline{}$, $\cmmnt{}$} must use \sumop etc
\def\cf{\mathop{\text{cf}}}
\def\circumflex#1{{\accent"5E #1}}
\def\closeplus{\hbox{+}}
\def\coint#1{\left[#1\right[}
\def\curl{\mathop{\text{curl}}}
\def\dashint{\mathchoice{-\mskip-20mu\intop\nolimits}
   {{\vrule height2.75pt width4.5pt depth-2.45pt\mskip-10mu\intop\nolimits}}
      {-\mskip-16.5mu\intop\nolimits} %not yet used
      {-\mskip-16.5mu\intop\nolimits}} %not yet used
\def\idashint{\mathchoice{\int\mskip-6mu\dashint}
      {\int\mskip-5mu\dashint}
      {\int\mskip-5mu\dashint} %not yet used
      {\int\mskip-5mu\dashint}} %not yet used
\def\dashiint{\mathchoice{\dashint\mskip-10mu\int}
      {\dashint\mskip-6mu\int}
      {\dashint\mskip-6mu\int} %not yet used
      {\dashint\mskip-6mu\int}} %not yet used
\def\dashidashint{\mathchoice{\dashint\mskip-10mu\dashint}
      {\dashint\mskip-6mu\dashint}
      {\dashint\mskip-6mu\dashint} %not yet used
      {\dashint\mskip-6mu\dashint}} %not yet used
\def\dashidashiint{\mathchoice{\dashint\mskip-8mu\dashint\mskip-10mu\int}
      {\dashint\mskip-6mu\int} %not yet used
      {\dashint\mskip-6mu\int} %not yet used
      {\dashint\mskip-6mu\int}} %not yet used
\def\dashiidashint{\mathchoice{\dashint\mskip-10mu\int\mskip-8mu\dashint}
      {\dashint\mskip-6mu\int} %not yet used
      {\dashint\mskip-6mu\int} %not yet used
      {\dashint\mskip-6mu\int}} %not yet used
\def\idashidashint{\mathchoice{\int\mskip-8mu\dashint\mskip-8mu\dashint}
      {\dashint\mskip-6mu\int} %not yet used
      {\dashint\mskip-6mu\int} %not yet used
      {\dashint\mskip-6mu\int}} %not yet used
\def\diam{\mathop{\text{diam}}}
\def\discrpage{\ifnum\stylenumber=1{\frnewpage}  %redefined in mt.tex
           \else\ifnum\stylenumber=3{\frnewpage}
           \else\ifnum\stylenumber=5{\frnewpage}
           \else\bigskip\fi\fi\fi}
\long\def\discrversionA#1#2{\ifnum\stylenumber=1#1\else#2\fi}
\long\def\discrversionB#1#2{\ifnum\stylenumber=2#1\else#2\fi}
\long\def\discrversionC#1#2{\ifnum\stylenumber=3#1\else#2\fi}
\long\def\discrversionD#1#2{\ifnum\stylenumber=4#1\else#2\fi}
\long\def\discrversionE#1#2{\ifnum\stylenumber=5#1\else#2\fi}
\long\def\discrversionF#1#2{\ifnum\stylenumber=6#1\else#2\fi}
\long\def\discrversionG#1#2{\ifnum\stylenumber=7#1\else#2\fi}
\long\def\discrversionH#1#2{\ifnum\stylenumber=8#1\else#2\fi}
           % facility for special interpolations in particular output
           % versions
\def\displaycause#1{\noalign{\noindent (#1)}}
\def\diverg{\mathop{\text{div}}}
\def\dom{\mathop{\text{dom}}}
\def\eae{=_{\text{a.e.}}}

\def\esssup{\mathop{\text{ess sup}}}
\mathchardef\Exists="3239

\def\family#1#2#3{\langle#3\rangle_{#1\in#2}}
\def\familyi#1#2{\family{i}{#1}{#2}}
\def\familyiI#1{\familyi{I}{#1}}

\def\frfilename#1{\ifnum\stylenumber=1{\noindent filename #1}
               \else\ifnum\stylenumber=6{\noindent{\fiverm filename #1}}
               \else\ifnum\stylenumber=7{\noindent{\fiverm filename #1}}
               \else\fi\fi\fi}
\def\Footnote#1#2{\footnote"#1"{#2}}
\mathchardef\Forall="3238
\def\frakc{\frak c\phantom{.}}
\def\frnewpage{\vskip \pageheight plus 0pt minus\pageheight
           \eject
           \gdef\topparagraph{}
           \gdef\bottomparagraph{}
           }
\def\frsmallcirc{{\hskip 1pt plus 0pt minus 0pt}
     {\raise 1pt\hbox{$\smallcirc$}}{\hskip 1pt plus 0pt minus 0pt}}

\def\geae{\ge_{\text{a.e.}}}
\def\grad{\mathop{\text{grad}}}
\def\grheada{{\bf ($\pmb{\alpha}$)}}
\def\grheadb{{\bf ($\pmb{\beta}$)}}
\def\grheadc{{\bf ($\pmb{\gamma}$)}}
\def\grheadd{{\bf ($\pmb{\delta}$)}}
\def\grheade{{\bf ($\pmb{\epsilon}$)}}
\def\grheadz{{\bf ($\pmb{\zeta}$)}}
\def\grv#1{{\accent"12 #1}}

\def\hfillall{\hskip\pagewidth plus0pt minus\pagewidth}

\def\Imag{\mathop{\Cal I\text{m}}}
\long\def\inset#1{{\narrower\narrower{\vskip 1pt plus 1pt minus 0pt}
     #1
     {\vskip 1pt plus 1pt minus 0pt}}}
\def\Int#1#2{\displaystyle{\int_{#1}^{#2}}}
\def\interior{\mathop{\text{int}}}

\def\leae{\le_{\text{a.e.}}}
\long\def\leaveitout#1{}
\def\LLcorner{\mathchoice{\,\hbox{\vrule height7pt width0.5pt depth0pt
       \vrule height0.5pt width5pt depth0pt}\,}
   {\,\hbox{\vrule height7pt width0.5pt depth0pt
       \vrule height0.5pt width5pt depth0pt}\,}
   {\,\hbox{\vrule height5pt width0.3pt depth0pt
       \vrule height0.3pt width4pt depth0pt}\,}
   {\,\hbox{\vrule height4pt width0.25pt depth0pt
       \vrule height0.25pt width3pt depth0pt}\,}
  }
\def\Ln{\mathop{\text{Ln}}}
\def\Loadfont#1{\font#1}
\def\loibr{\hbox{\hskip0.1em]\hskip0.05em}}
\def\Matrix#1{\pmatrix#1\endpmatrix}
\def\Mu{\text{M}}
\def\noamslogo{\def\amslogo{}}
\def\normalsubgroup{\vartriangleleft}
\def\Nu{\text{N}}
\def\ocint#1{\left]#1\right]}
\def\ofamily#1#2#3{\langle#3\rangle_{#1<#2}}
\let\oldcases=\cases
\let\oldfootnote=\footnote
\def\On{\mathop{\text{On}}}
\def\ooint#1{\left]#1\right[}
\def\openBbb#1{\Bbb#1\mskip1mu}
\def\pd#1#2{\bover{\partial#1}{\partial#2}}
\def\Pd#1#2{\Bover{\partial#1}{\partial#2}}
\def\pdd#1#2{{{\partial^2#1}\over{\partial#2^2}}}
\def\Pdd#1#2{\Bover{\partial^2#1}{\partial#2^2}}
\def\pde#1#2#3{{{\partial^2#1}\over{\partial#2\partial#3}}}
\def\Pde#1#2#3{\Bover{\partial^2#1}{\partial#2\partial#3}}
\def\Prf{\vthsp\hbox{\bf P\hskip0.07em\llap P\hskip0.07em\llap P\rm}}
\def\pushbottom#1{\hbox{\vrule height0pt depth#1 width0pt}}
\def\Qed{\vthsp\hbox{\bf Q\hskip0.07em\llap Q\hskip0.07em\llap Q\rm}}
\def\QeD{\Qed{\hskip 1.111pt plus 3.333pt minus 0pt}}
\def\Quer{\vthsp\hbox{\bf ?\hskip0.05em\llap ?\hskip0.05em\llap
   ?\thinspace\rm}}
\def\query{\discrversionA{\immediate\write0{query}
    \global\advance\footnotenumber by 1
    \oldfootnote{$^{\the\footnotenumber}$}{query}}{}}
\def\Real{\mathop{\Cal R\text{e}}}
\def\restr{\mathchoice
  {\hbox{$\restriction$}\mskip 1mu}
  {\hbox{$\restriction$}\mskip 1mu}
  {\hbox{$\scriptstyle\restriction$}\mskip 1mu}
  {\hbox{$\scriptscriptstyle\restriction$}\mskip 1mu}}
\def\restrp{\mathchoice
  {\hbox{$\restriction$}\mskip 2mu}
  {\hbox{$\restriction$}\mskip 2mu}
  {\mskip-1mu\hbox{$\scriptstyle\restriction$}\mskip 2mu}
  {\mskip-1mu\hbox{$\scriptscriptstyle\restriction$}\mskip 1mu}}
    %use in \restrp V \restrp W \restrp Y
    % \restrp\Cal P \restrp\Bbb R \restrp\frak
    % \restrp\Omega \restrp\Tau \restrp\Upsilon
    % \restrp[ \restrp\widehat{}
\def\Rho{\text{P}}
\def\roibr{\hbox{\hskip0.05em[\hskip0.1em}}
\def\RoverC{\hbox{\biggerscriptfonts${{\Bbb R}\atop{\Bbb C}}$}}
\def\sequence#1#2{\family{#1}{\Bbb N}{#2}}
\def\sequencen#1{\sequence{n}{#1}}
\def\setover#1#2{\overset{#1}\to{#2}}
\def\shortstrut{\hbox{\vrule height7.5pt depth2.5pt width0pt}}
\def\sshortstrut{\hbox{\vrule height5.8pt depth2.5pt width0pt}}
\def\smallcirc{{\scriptstyle\circ}}
\def\ssbullet{{\scriptscriptstyle\bullet}}
\def\symmdiff{\triangle}
\def\thicken#1{#1\hskip0.03em\llap #1\hskip0.03em\llap
   #1\hskip0.03em\llap #1}
\def\Tau{{\text{T}}}
\def\tbf#1{\text{\bf #1}}
\def\terror#1#2#3{#1:  `#2' for `#3'}
\def\today{\number\day.\number\month.\number\year}
\def\UEMDRR{University of Essex Mathematics Department Research Report}
\def\UEMDWP{University of Essex Mathematics Department Working Paper}
\def\versiondate#1{
    \immediate\write0{Version of #1}
    \def\vsdate{#1}
               \ifnum\stylenumber=1\hfill Version of #1
                              \medskip
               \else\ifnum\stylenumber=2\hfill Version of #1
                              \medskip
               \else\ifnum\stylenumber=6\hfill{\fiverm Version of #1}
                              \medskip
               \else\ifnum\stylenumber=7\hfill{\fiverm Version of #1}
                              \medskip
               \else\ifnum\stylenumber=8\hfill Version of #1
                              \medskip
               \else\fi\fi\fi\fi\fi}
\def\versionhead{\ifnum\stylenumber=3 \hfill PRINTER'S VERSION

                        \bigskip
                   \else\ifnum\stylenumber=4 \hfill REFEREE'S VERSION

                        \bigskip
                   \else \fi\fi}
\def\versionref{\ifnum\stylenumber=1\workingref
           \else\ifnum\stylenumber=2\workingref
           \else\ifnum\stylenumber=8\workingref
           \else\printerref\fi\fi\fi}
\def\vthsp{\hskip0.001em}

%Missing AmSTeX can be coped with by using the following lines:

\ifamstexsupported\else
   \immediate\write0{AmSTeX-unsupported version of  fremtex.tex }
   \fourteensavailablefalse
   \def\@{@}
   \def\\{\cr}
   \def\AmSTeX{AmS\TeX}
   \def\Bbb#1{\underline{\hbox{\bf #1}}}
   \def\biggerscriptfonts{}
   \def\bumpeq{=_{\text{approx}}}
   \def\Cal#1{{\cal#1}}
   \def\curlyeqprec{\preceq}
   \def\eightit{\it}
   \def\eightrm{\rm}
   \def\eightsmc{\smc}
   \def\eurm#1{\underline{\cal#1}}
   \def\Footnote#1#2{\footnote#1#2}
   \def\frac#1#2{{#1}\over{#2}}
   \def\frak#1{{\hbox{\bf #1\hskip0.1em\llap #1}}}
   \def\frnewpage{\vfill\eject}
   \def\iint{\int\!\!\!\int}
   \def\iiint{\int\!\!\!\int\!\!\!\int}
   \def\iiiint{\int\!\!\!\int\!\!\!\int\!\!\!\int}
   \def\llcorner{\lfloor}
   \def\loadeurm{}
   \def\loadeusm{}
%   \def\loadeusmx#1{\def\eusm#1{\underline{\cal#1}}}
   \def\ltimes{|\hskip-0.5em\times}
   \def\eusm#1{\underline{\cal#1}}
   \def\ltimes{|\times}
   \def\Loadfont#1{}
   \def\Matrix#1{\pmatrix{#1}}
   \def\noamslogo{}
   \def\nrightarrow{\not\rightarrow}
   \def\preccurlyeq{\preceq}
   \def\query{\discrversionA{\immediate\write0{query}
                             \footnote{$^*$}{query}}{}}
   \def\restriction{|}
   \def\rtimes{\times|}
   \def\setover#1#2{#2^{#1}}
   \def\shortstrut{\strut}
   \def\smallerfonts{\eightrm}
   \def\smallfrown{\wedge}
   \def\smc{}
   \def\spcheck{\check{\phantom{x}}}
   \def\sphat{\hat{\phantom{x}}}
   \def\square{\mathchoice\sqr64\sqr64\sqr43\sqr33}
   \def\sqr#1#2{{\vcenter{\vbox{\hrule height.#2pt
          \hbox{\vrule width.#2pt height#1pt \kern#1pt
                  \vrule width.#2pt}
          \hrule height.#2pt}}}}
   \def\pmb#1{\underline{#1}}
   \def\text#1{\hbox{#1}}
   \def\therefore{\ssbullet\raise.9ex\hbox{$\ssbullet$}\ssbullet}
   \def\twelvebf{\bf}
   \def\vartriangleleft{\triangleleft}
   \def\Vdash{|\vdash}
   \def\vDash{|=}
   \fi

%End of lines for missing AmSTeX

\ifamstexsupported\input amstex
%              \documentstyle{amsppt}
               \input amsppt.sty
               \PSAMSFonts
	       \tenpoint
               %\font\eightmsb=msym8
               \font\eightmsb=msbm10
               \font\eighteufm=eufm10 at 8 pt
               \font\eighteurm=eurm8
               \font\eighteusm=eusm8
               \font\varseveneufm=eufm10 at 7 pt
               \scriptfont\eufmfam=\varseveneufm
   \def\=#1{{\accent"16 #1}}
   \gdef\nopagenumbers{\NoPageNumbers}
   \else\fi

\newcount\footmarkcount
\ifDHFamstexloaded
    \loadmsBm
    \def\eightBbb#1{{\fam\msBfam#1}}
    \loadmsFm
    \def\eightfrak#1{{\fam\msFfam#1}}
  \else
    \def\eightBbb#1{{\scriptstyle\Bbb#1}}
    \def\eightfrak#1{{\scriptstyle\frak#1}}
  \fi
\ifathome\athome\fi

%interword glue (\tenrm standard):
%  (space) 3.333 + 1.667 - 1.111
%  (,)     3.333 + 2.083 - 0.889
%  (.)     4.444 + 5.    - 0.370

%very odd fact:  \discrversionA{2.}{}  doesn't work,
%   \discrversionA{ 2.}{} does

%The false line feed introduced by my Fortran programs, leading to
%a surplus  ae  at the top of the next page, seems to be associated
%with  Fk(\032)3081  or something similar in the PostScript file.


