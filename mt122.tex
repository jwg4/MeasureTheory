\frfilename{mt122.tex}
\versiondate{4.1.04}
\copyrightdate{1994}

\def\chaptername{Integration}
\def\sectionname{Definition of the integral}

\newsection{122}

I set out the definition of ordinary integration for real-valued
functions defined on an arbitrary measure space, with its most basic
properties.

\leader{122A}{Definitions}  Let $(X,\Sigma,\mu)$ be a measure space.

\header{122Aa}{\bf (a)} For any set $A\subseteq X$, I write $\chi A$
for
the {\bf characteristic function} of $A$, the function
from $X$ to $\{0,1\}$ given by setting $\chi A(x)=1$ if $x\in A$, $0$
if
$x\in X\setminus A$.   \cmmnt{(Of course this notation depends on it
being understood which is the `universal' set $X$ under consideration;
perhaps I should call it the `characteristic function of $A$ as a
subset of $X$'.)}
\cmmnt{Observe that} $\chi A$ is $\Sigma$-measurable\cmmnt{, in
the
sense of 121C above,} iff $A\in\Sigma$\cmmnt{ (because
$A=\{x:\chi A(x)>0\}$)}.

\header{122Ab}{\bf (b)} Now a {\bf simple} function on $X$ is a
function
of the form
$\sum_{i=0}^n a_i\chi E_i$, where $E_0,\ldots,E_n$ are measurable
sets of finite measure and $a_0,\ldots, a_n$ belong to
$\Bbb R$.   \cmmnt{{\bf Warning!} Some authors allow arbitrary sets
$E_i$, so that a simple function on $X$ is any function taking only
finitely many values.}

\leader{122B}{Lemma} Let $(X,\Sigma,\mu)$ be a measure space.

(a) Every simple function on $X$ is measurable.

(b) If $f$, $g:X\to\Bbb R$ are simple functions, so is $f+g$.

(c) If $f:X\to \Bbb R$ is a simple function and $c\in\Bbb R$, then
$cf:X\to\Bbb R$ is a simple function.

(d) The constant zero function is simple.

\proof{ (a) comes from the facts that $\chi E$ is
measurable for measurable $E$, and that sums and scalar multiples of
measurable functions are measurable (121Eb-121Ec).  (b)-(d) are
trivial.
}%end of proof of 122B

\vleader{42pt}{122C}{Lemma} Let $(X,\Sigma,\mu)$ be a measure space.

(a) If $E_0,\ldots,E_n$ are measurable sets of finite measure,
there are disjoint measurable sets $G_0,\ldots,G_m$ of finite measure
such that each $E_i$ is expressible as a union of some of the $G_j$.

(b) If $f:X\to\Bbb R$ is a simple function, it is expressible in the
form $\sum_{j=0}^mb_j\chi G_j$ where $G_0,\ldots,G_m$ are disjoint
measurable sets of finite measure.

(c) If $E_0,\ldots,E_n$ are measurable sets of finite measure, and
$a_0,\ldots,a_n\in\Bbb R$ are such that $\sum_{i=0}^na_i\chi E_i(x)\ge
0$ for every $x\in X$, then $\sum_{i=0}^na_i\mu E_i\ge 0$.

\proof{{\bf (a)} Set $m=2^{n+1}-2$, and enumerate the
non-empty subsets of $\{0,\ldots,n\}$ as $I_0,\ldots,I_m$.   For each
$j\le m$, set

\Centerline{$G_j
=\bigcap_{i\in I_j}E_i\setminus\bigcup_{i\le n,i\notin I_j}E_i$.}

\noindent   Then every $G_j$ is a measurable set, being obtained from
finitely many measurable sets by the operations $\cup$, $\cap$ and
$\setminus$, and has finite measure, because $I_j\ne\emptyset$ and
$G_j\subseteq E_i$ if $i\in I_j$.   Moreover, the $G_j$ are disjoint,
for if $i\in I_j\setminus I_k$ then $G_j\subseteq E_i$ and $G_k\cap
E_i=\emptyset$.   Finally, if $k\le n$ and $x\in E_k$, there is a
$j\le
m$ such that $I_j=\{i:i\le n,\,x\in E_i\}$, and in this case $x\in
G_j\subseteq E_k$;  thus  $E_k$ is the union of those $G_j$ which it
includes.

\medskip

{\bf (b)} Express $f$ as $\sum_{i=0}^na_i\chi E_i$ where
$E_0,\ldots,E_n$ are measurable sets of finite measure and
$a_0,\ldots,a_n$ are real numbers.   Let $G_0,\ldots,G_m$ be disjoint
measurable sets of finite measure such that every $E_i$ is expressible
as a union of appropriate $G_j$.   Set $c_{ij}=1$ if $G_j\subseteq
E_i$,
$0$ otherwise, so that, because the $G_j$ are disjoint, $\chi
E_i=\sum_{j=0}^mc_{ij}\chi G_j$ for each $i$.      Then

\Centerline{$f=\sum_{i=0}^na_i\chi E_i
=\sum_{i=0}^n\sum_{j=0}^ma_ic_{ij}\chi G_j
=\sum_{j=0}^mb_j\chi G_j$,}

\noindent setting $b_j=\sum_{i=0}^na_ic_{ij}$ for each $j\le m$.

\medskip

{\bf (c)} Set $f=\sum_{i=0}^na_i\chi E_i$, and take $G_j$, $c_{ij}$,
$b_j$ as in  (b).   Then $b_j\mu G_j\ge 0$ for every $j$.   \Prf\ If
$G_j=\emptyset$, this is trivial.  Otherwise, let $x\in G_j$;  then

\Centerline{$0\le f(x)=\sum_{i=0}^nb_i\chi G_i(x)=b_j\chi
G_j(x)=b_j$,}

\noindent so again $b_j\mu G_j\ge 0$.\ \Qed\    Next, because the
$G_j$
are disjoint,

\Centerline{$\mu E_i=\sum_{j=0}^mc_{ij}\mu G_j$}

\noindent for each $i$, so

\Centerline{$\sum_{i=0}^na_i\mu E_i
=\sum_{i=0}^n\sum_{j=0}^ma_ic_{ij}\mu G_j
=\sum_{j=0}^mb_j\mu G_j
\ge 0$,}

\noindent as required.
}%end of proof of 122C

\leader{122D}{Corollary} Let $(X,\Sigma,\mu)$ be a measure space.   If

\Centerline{$\sum_{i=0}^m a_i\chi E_i
=\sum_{j=0}^n b_j\chi F_j$,}

\noindent where all the $E_i$ and $F_j$ are
measurable sets of finite measure and the $ a_i$, $ b_j$ are real
numbers, then

\Centerline{$\sum_{i=0}^m a_i\mu E_i=\sum_{j=0}^n b_j\mu F_j$.}

\proof{ Apply 122Cc to
$\sum_{i=0}^m a_i\chi E_i+\sum_{j=0}^n(-b_j)\chi F_j$ to see that
$\sum_{i=0}^ma_i\mu E_i-\sum_{j=0}^nb_j\mu F_j\ge 0$;   now reverse
the
roles of the two sums to get the opposite inequality.
}%end of proof of 122D


\leader{122E}{Definition} Let $(X,\Sigma,\mu)$ be a measure space.
Then we
may define the {\bf integral} $\int f$ of $f$, for simple functions
$f:X\to\Bbb R$, by saying that
$\int f=\sum_{i=0}^m a_i\mu E_i$ whenever $f=\sum_{i=0}^m a_i\chi
E_i$ and every $E_i$ is a measurable set of finite measure\cmmnt{;
122D promises us that it won't matter which representation of $f$ we pick
on for the calculation}.

\leader{122F}{Proposition} Let $(X,\Sigma,\mu)$ be a measure space.

(a) If $f$, $g:X\to\Bbb R$ are simple functions, then $f+g$ is a
simple
function and $\int f+g=\int f+\int g$.

(b) If $f$ is a simple function and $c\in\Bbb R$, then $cf$ is a
simple
function and $\int cf=c\int f$.

(c)  If $f$, $g$ are simple functions and $f(x)\le g(x)$ for every
$x\in
X$, then $\int f\le \int g$.

\proof{ (a) and (b) are immediate from the formula given
for $\int f$ in 122E.   As for (c), observe that $g-f$ is a
non-negative
simple function, so that $\int g-f\ge 0$, by 122Cc;  but this means
that
$\int g-\int f\ge 0$.
}%end of proof of 122F


\vleader{72pt}{122G}{Lemma} Let $(X,\Sigma,\mu)$ be a measure space.
If $\langle f_n\rangle_{n\in\Bbb N}$ is a sequence of simple functions
which is non-decreasing (in the sense that $f_n(x)\le f_{n+1}(x)$ for
every $n\in\Bbb N$, $x\in X$) and $f$ is a simple function such that
$f(x)\le\sup_{n\in\Bbb N}f_n(x)$ for almost every $x\in X$ (allowing
$\sup_{n\in\Bbb N}f_n(x)=\infty$ in this formula), then
$\int f\le\sup_{n\in\Bbb N}\int f_n$.

\proof{  Note that $f-f_0$ is a simple function, so
$H=\{x:(f-f_0)(x)\ne 0\}$ is a finite union of sets of finite measure,
and $\mu H<\infty$;  also $f-f_0$ is bounded, so there is an $M\ge 0$
such that $(f-f_0)(x)\le M$ for every $x\in X$.

Let $\epsilon>0$.   For each $n\in\Bbb N$, set
$H_n=\{x:(f-f_n)(x)\ge\epsilon\}$.   Then each $H_n$ is measurable (by
121E), and $\langle H_n\rangle_{n\in\Bbb N}$ is a non-increasing
sequence of sets with intersection

\Centerline{$\bigcap_{n\in\Bbb N}H_n
=\{x:f(x)\ge\epsilon+\sup_{n\in\Bbb N}f_n(x)\}
\subseteq\{x:f(x)>\sup_{n\in\Bbb N}f_n(x)\}$.}

\noindent Because $f(x)\le\sup_{n\in\Bbb N}f_n(x)$ for almost every
$x$,
$\{x:f(x)>\sup_{n\in\Bbb N}f_n(x)\}$ and $\bigcap_{n\in\Bbb N}H_n$ are
negligible.   Also $\mu H_0<\infty$, because
$H_0\subseteq H$.
Consequently

\Centerline{$\lim_{n\to\infty}\mu H_n=\mu(\bigcap_{n\in\Bbb N}H_n)=0$}

\noindent (112Cf).   Let $n$ be so large that $\mu H_n\le\epsilon$.

Consider the simple function $g=f_n+\epsilon\chi H+M\chi H_n$.   Then
$f\le g$, so

\Centerline{$\int f\le\int g=\int f_n+\epsilon\mu H+M\mu H_n
\le\int f_n+\epsilon(M+\mu H)$.}

\noindent As $\epsilon$ is arbitrary,
$\int f\le\sup_{n\in\Bbb N}\int f_n$.
}%end of proof of 122G


\leader{122H}{Definition} Let $(X,\Sigma,\mu)$ be a measure space.
For the rest of this section, I will write $U$ for the set of
functions $f$ such that

\quad(i) the domain of $f$ is a conegligible subset of $X$ and
$f(x)\in\coint{0,\infty}$ for each $x\in\dom f$,

\quad(ii) there is a non-decreasing sequence $\sequencen{f_n}$ of
non-negative
simple functions such that $\sup_{n\in\Bbb N}\int f_n<\infty$ and
$\lim_{n\to\infty}f_n(x)\penalty-100=f(x)$ for almost
every $x\in X$.

\leader{122I}{Lemma} If $f$ and $\langle f_n\rangle_{n\in\Bbb N}$ are
as
in 122H, then

\Centerline{$\sup_{n\in\Bbb N}\int f_n
=\sup\{\int g:g$ is a simple function and $g\leae f\}$.}

\proof{ Of course

\Centerline{$\sup_{n\in\Bbb N}\int f_n
\le \sup\{\int g:g$ is a simple function and $g\leae f\}$}

\noindent because $f_n\leae f$  for each $n$.   On the other hand, if
$g$ is a simple function and $g\leae f$, then
$g(x)\le\sup_{n\in\Bbb N}f_n(x)$ for almost every $x$, so
$\int g\le\sup_{n\in\Bbb N}\int f_n$ by 122G.   Thus

\Centerline{$\sup_{n\in\Bbb N}\int f_n
\ge \sup\{\int g:g$ is a simple function and $g\leae f\}$,}

\noindent as required.
}%end of proof of 122I


\leader{122J}{Lemma} Let $(X,\Sigma,\mu)$ be a measure
space\cmmnt{, and define $U$ as in 122H}.

(a) If $f$ is a function defined on a conegligible subset of $X$ and
taking values in $\coint{0,\infty}$, then $f\in U$ iff
there is a conegligible measurable set $E\subseteq\dom f$ such that

\inset{($\alpha$)
$f\restr E$ is measurable,

($\beta$) for every $\epsilon>0$,
$\mu\{x:x\in E,\,f(x)\ge\epsilon\}<\infty$,

($\gamma$)
$\sup\{\int g:g$ is a simple function, $g\leae f\}<\infty$.}

(b) Suppose that $f\in U$ and that $h$ is a function defined on a
conegligible subset of $X$ and taking values in $\coint{0,\infty}$.
Suppose that $h\leae f$ and there is a conegligible
$F\subseteq X$ such that $h\restr F$ is measurable.   Then $h\in U$.

\proof{{\bf (a)(i)} Suppose that $f\in U$.   Then there is an
non-decreasing sequence $\langle f_n\rangle_{n\in\Bbb N}$ of
non-negative
simple functions such that $f\eae\lim_{n\to\infty}f_n$ and
$\sup_{n\in\Bbb N}\int f_n=c<\infty$.   The set
$\{x:f(x)=\lim_{n\to\infty}f_n(x)\}$ is conegligible, so includes a
measurable conegligible set $E$ say.   Now
$f\restr E=(\lim_{n\to\infty}f_n)\restr E$ is measurable, by 121Fa
and 121Eh; thus ($\alpha$) is satisfied.   Next, given $\epsilon>0$,
set $H_n=\{x:x\in E,\,f_n(x)\ge{1\over 2}\epsilon\}$;  then
$f_n\ge{1\over 2}\epsilon\chi H_n$, so

\Centerline{$\Bover12\epsilon\mu H_n
=\int\bover12\epsilon\chi H_n\le\int f_n\le c$,}

\noindent for each $n$.   Now $\langle H_n\rangle_{n\in\Bbb N}$ is
non-decreasing, so

\Centerline{$\mu(\bigcup_{n\in\Bbb N}H_n)
=\sup_{n\in\Bbb N}\mu H_n\le 2c/\epsilon$,}

\noindent by 112Ce.   Accordingly

\Centerline{$\mu\{x:x\in E,\,f(x)\ge\epsilon\}
\le\mu(\bigcup_{n\in\Bbb N}H_n)\le 2c/\epsilon<\infty$.}

\noindent As $\epsilon$ is arbitrary, ($\beta$) is satisfied.
Finally, ($\gamma$) is satisfied by 122I.

\medskip

\quad{\bf (ii)} Now suppose that the conditions ($\alpha$)-($\gamma$)
are satisfied.   Take an appropriate conegligible $E\in\Sigma$, and
for
each $n\in\Bbb N$ define $f_n:X\to\Bbb R$ by setting

$$\eqalign{f_n(x)&=2^{-n}k\text{ if }x\in E,\, 0\le k<4^n,\,
2^{-n}k\le f(x)<2^{-n}(k+1),\cr
&=0\text{ if }x\in X\setminus E,\cr
&=2^n\text{ if }x\in E\text{ and }f(x)\ge 2^n.\cr}$$

\noindent Then $f_n$ is a non-negative simple function, being
expressible as

\Centerline{$f_n
=\sum_{k=1}^{4^n}2^{-n}\chi\{x:x\in E,\,f(x)\ge 2^{-n}k\}$;}

\noindent all the sets $\{x:x\in E,\,f(x)\ge 2^{-n}k\}$ being
measurable
(because $f\restr E$ is measurable) and of finite measure, by
($\beta$).
Also it is easy to see that $\langle f_n\rangle_{n\in\Bbb N}$ is an
non-decreasing sequence which converges to $f$ at every point of $E$,
so
that $f\eae\lim_{n\to\infty}f_n$.   Finally,

\Centerline{$\lim_{n\to\infty}\int f_n
=\sup_{n\in\Bbb N}\int f_n\le\sup\{\int g:g\le f\text{ is simple}\}
<\infty$,}

\noindent by ($\gamma$).   So $f\in U$.

\medskip

{\bf (b)} Let $E$ be a set as in (a).   The sets $E$, $F$ and
$\{x:h(x)\le f(x)\}$ are all conegligible, so there is a conegligible
measurable set $E'$ included in their intersection.   Now
$E'\subseteq\dom h$, $h\restr E'$ is measurable,

\Centerline{$\mu\{x:x\in E',\,h(x)\ge\epsilon\}
\le\mu\{x:x\in E,\,f(x)\ge\epsilon\}<\infty$}

\noindent for
every $\epsilon>0$, and

\Centerline{$\sup\{\int g:g\text{ is simple},\, g\leae h\}
\le\sup\{\int g:g\text{ is simple},\,g\leae f\}<\infty$.}

\noindent So $h\in U$.
}%end of proof of 122J

\leader{122K}{Definition} Let $(X,\Sigma,\mu)$ be a measure
space\cmmnt{, and define $U$ as in 122H}.   For $f\in U$, set

\Centerline{$\int f
=\sup\{\int g:g$ is a simple function and $g\leae f\}$.}

\cmmnt{\noindent By 122I, we see that
$\int f=\lim_{n\to\infty}\int f_n$ whenever $\sequencen{f_n}$ is a
non-decreasing sequence
of simple functions converging to $f$ almost everywhere;  in
particular,
if $f\in U$ is itself a simple function, then $\int f$, as defined
here,
agrees with the original definition of $\int f$ in 122E, since we may
take $f_n=f$ for every $n$.}



\leader{122L}{Lemma} Let $(X,\Sigma,\mu)$ be a measure space.

(a) If $f$, $g\in U$ then $f+g\in U$ and $\int f+g=\int f+\int g$.

(b) If $f\in U$ and $c\ge 0$ then $cf\in U$ and $\int cf=c\int f$.

(c) If $f$, $g\in U$ and $f\leae g$ then $\int f\le \int g$.

(d) If $f\in U$ and $g$ is a function with domain a conegligible
subset
of $X$, taking values in $\coint{0,\infty}$, and equal to $f$ almost
everywhere, then $g\in U$ and $\int g=\int f$.

(e) If $f_1$, $g_1$, $f_2$, $g_2\in U$ and $f_1-f_2=g_1-g_2$, then
$\int
f_1-\int f_2=\int g_1-\int g_2$.

\proof{{\bf (a)} We know that there are non-decreasing sequences
$\langle f_n\rangle_{n\in\Bbb N}$, $\langle g_n\rangle_{n\in\Bbb N}$
of
non-negative simple functions such that $f\eae\lim_{n\to\infty}f_n$,
$g\eae\lim_{n\to\infty}g_n$, $\sup_{n\in\Bbb N}\int f_n<\infty$ and
$\sup_{n\in\Bbb N}\int g_n<\infty$.   Now $\langle
f_n+g_n\rangle_{n\in\Bbb N}$ is a non-decreasing sequence of simple
functions converging to $f+g$ a.e., and

\Centerline{$\sup_{n\in\Bbb N}\int f_n+g_n
=\lim_{n\to\infty}\int f_n+g_n
=\lim_{n\to\infty}\int f_n+\lim_{n\to\infty}\int g_n=\int f+\int g$.}

\noindent   Accordingly $f+g\in U$, and also, as remarked in 122K,

\Centerline{$\int f+g=\lim_{n\to\infty}\int f_n+g_n=\int f+\int g$.}

\medskip

{\bf (b)} We know that there is a non-decreasing sequence
$\langle f_n\rangle_{n\in\Bbb N}$ of
non-negative simple functions such that $f\eae\lim_{n\to\infty}f_n$
and $\sup_{n\in\Bbb N}\int f_n<\infty$.   Now
$\langle cf_n\rangle_{n\in\Bbb N}$ is a non-decreasing sequence of
simple functions converging to $cf$ a.e., and

\Centerline{$\sup_{n\in\Bbb N}\int cf_n
=\lim_{n\to\infty}\int cf_n=c\lim_{n\to\infty}\int f_n=c\int f$.}

\noindent Accordingly $cf\in U$, and also, as remarked in 122K,

\Centerline{$\int cf=\lim_{n\to\infty}\int cf_n=c\int f$.}

\medskip

{\bf (c)} This is obvious from 122K.

\medskip

{\bf (d)} If $\langle f_n\rangle_{n\in\Bbb N}$ is a non-decreasing
sequence of simple functions converging to $f$ a.e., then it also
converges to $g$ a.e.;  so $g\in U$ and

\Centerline{$\int g=\lim_{n\to\infty}\int f_n=\int f$.}

\medskip

{\bf (e)} By (a), $f_1+g_2$ and $f_2+g_1$ both belong to $U$.   Also,
they are equal at any point at which all four functions are defined,
which is almost everywhere.   So

\Centerline{$\int f_1+\int g_2=\int f_1+g_2
=\int f_2+g_1=\int f_2+\int g_1$,}

\noindent using (a) and (d).   Shifting $\int g_2$ and $\int f_2$
across the equation, we have the result.
}%end of proof of 122L


\leader{122M}{Definition} Let $(X,\Sigma,\mu)$ be a measure space.
\cmmnt{Define $U$ as in 122H.}   A real-valued function $f$ is {\bf
integrable}, or {\bf integrable over
$X$}, or {\bf $\mu$-integrable over $X$}, if it is expressible as
$f_1-f_2$ with $f_1$, $f_2\in U$, and in this case its {\bf integral}
is

\Centerline{$\int f=\int f_1-\int f_2$.}

\cmmnt{
\leader{122N}{Remarks (a)} We see from 122Le that the
integral
$\int f$ is uniquely defined  by the
formula in 122M.   Secondly, if $f\in U$, then $f=f-\tbf{0}$ is
integrable, and the
integral here agrees with that defined in 122K.   Finally, if $f$ is a
simple function, then it can be expressed as $f_1-f_2$ where $f_1$,
$f_2$ are non-negative simple functions (if $f=\sum_{i=0}^na_i\chi
E_i$, where each $E_i$ is measurable and of finite measure, set

\Centerline{$f_1=\sum_{i=0}^na_i^+\chi E_i$,
\quad$f_2=\sum_{i=0}^na_i^-\chi E_i$,}

\noindent writing $a_i^+=\max(a_i,0)$, $a_i^-=\max(-a_i,0)$);  so that

\Centerline{$\int f=\int f_1-\int f_2=$$\sum_{i=0}^na_i\mu E_i$,}

\noindent and the definition of 122M is consistent with the definition
of 122E.


\header{122Nb}{\bf (b)} Alternative notations which I will use for
$\int
f$ are $\int_Xf$,
$\int fd\mu$, $\int f(x)\mu(dx)$, $\int f(x)dx$, $\int_Xf(x)\mu(dx)$,
etc., according to which aspects of the context seem due for emphasis.

When $\mu$ is Lebesgue measure on $\Bbb R$ or $\BbbR^r$ we say that
$\int f$ is the {\bf Lebesgue integral} of $f$, and that $f$ is {\bf
Lebesgue integrable} if this is defined.

\header{122Nc}{\bf (c)} Note that when I say, in 122M, that `$f$ can
be
expressed as $f_1-f_2$', I mean
to interpret $f_1-f_2$ according to the rules set out in 121E, so that
$\dom f$ must be $\dom(f_1-f_2)=\dom f_1\cap\dom f_2$, and is surely
conegligible.
}%end of comment

\leader{122O}{Theorem} Let $(X,\Sigma,\mu)$ be a measure space.

(a) If $f$ and $g$ are integrable over $X$
then $f+g$ is integrable and $\int f+g=\int f+\int g$.

(b) If $f$ is integrable over $X$ and $c\in\Bbb R$ then $cf$
is integrable and $\int cf=c\int f$.

(c) If $f$ is integrable over $X$ and $f\ge 0$ a.e.\ then
$\int f\ge 0$.

(d) If $f$ and $g$ are integrable over $X$ and $f\leae g$
then $\int f\le\int g$.

\proof{{\bf (a)} Express $f$ as $f_1-f_2$ and $g$ as $g_1-g_2$
where $f_1$, $f_2$, $g_1$ and $g_2$ belong to $U$, as defined in 122H.
Then $f+g=(f_1+g_1)-(f_2+g_2)$ is integrable because $U$ is closed
under addition (122La), and

\Centerline{$\int f+g=\int f_1+g_1-\int f_2+g_2
=\int f_1+\int g_1-\int f_2-\int g_2=\int f+\int g$.}

\medskip

{\bf (b)} Express $f$ as $f_1-f_2$ where $f_1$, $f_2$ belong to $U$.
If $c\ge 0$ then $cf=cf_1-cf_2$ is integrable because $U$ is closed
under multiplication by non-negative scalars (122Lb), and

\Centerline{$\int cf=\int cf_1-\int cf_2=c\int f_1-c\int f_2=c\int
f$.}

\noindent   If $c=-1$ then $-f=f_2-f_1$ is integrable and

\Centerline{$\int (-f)=\int f_2-\int f_1=-\int f$.}

\noindent    Putting
these together we get the result for $c<0$.

\medskip

{\bf (c)} Express $f$ as $f_1-f_2$ where $f_1$, $f_2\in U$.   Then
$f_2\leae f_1$, so $\int f_2\le\int f_1$ (122Lc), and $\int f\ge 0$.

\medskip

{\bf (d)} Apply (c) to $g-f$.
}%end of proof of 122O


\leader{122P}{Theorem} Let $(X,\Sigma,\mu)$ be a measure space and
$f$ a
real-valued function defined on a conegligible subset of $X$.   Then
the following are equiveridical:

\quad(i)  $f$ is integrable;

\quad(ii) $|f|\in U$\cmmnt{, as defined in 122H,} and there is a
conegligible set $E\subseteq X$ such that $f\restr E$ is measurable;

\quad(iii) there are a $g\in U$ and a
conegligible set $E\subseteq X$  such that $|f|\leae g$
and $f\restr E$ is measurable.

\proof{{\bf (i)$\Rightarrow$(iii)} Suppose that $f$ is
integrable.   Let $f_1$,
$f_2\in U$ be such that $f=f_1-f_2$.   Then there are conegligible
sets $E_1$, $E_2$ such that $f_1\restr E_1$ and $f_2\restr
E_2$ are measurable;  set $E=E_1\cap E_2$, so that $E$ also is a
conegligible set.   Now $f\restr E=f_1\restr E-f_2\restr E$
is measurable.   Next, $f_1+f_2\in U$ (122La) and $|f|(x)\le
f_1(x)+f_2(x)$ for every $x\in\dom f$, so we may take $g=f_1+f_2$.

\medskip

{\bf (iii)$\Rightarrow$(ii)} If $f\restr E$ is measurable, so is
$|f|\restr E=|f\restr E|$ (121Eg);  so if $g\in U$ and
$|f|\leae g$, then $|f|\in U$ by 122Jb.

\medskip

{\bf (ii)$\Rightarrow$(i)} Suppose that $f$ satisfies the conditions
of
(ii).   Set
$f^+={1\over 2}(|f|+f)$ and $f^-={1\over 2}(|f|-f)$.    Of course
$|f|\restr E$, $f^+\restr E$ and $f^-\restr E$ are all measurable.
Also $0\le f^+(x)\le |f|(x)$ and $0\le f^-(x)\le |f|(x)$ for every
$x\in\dom f$, while $|f|\in U$ by hypothesis, so $f^+$ and $f^-$
belong
to $U$ by 122Jb.  Finally, $f=f^+-f^-$, so $f$ is integrable.
}%end of proof of 122P

\leader{122Q}{Remark} The condition `there is a conegligible set $E$
such that $f\restr E$ is measurable' recurs so often that I think it
worth having a phrase for it;  I will call such functions {\bf
virtually
measurable}, or {\bf $\mu$-virtually measurable} if it seems necessary
to specify the measure.

\leader{122R}{Corollary} Let $(X,\Sigma,\mu)$ be a measure space.

(a) A non-negative real-valued function, defined on a subset of $X$,
is integrable iff it belongs to $U$\cmmnt{, as defined in 122H}.

(b) If $f$ is integrable over $X$ and $h$ is a real-valued function,
defined on a conegligible subset of $X$ and equal to $f$ almost
everywhere, then $h$ is integrable, with $\int h=\int f$.

(c) If $f$ is integrable over $X$, $f\ge 0$ a.e.\ and $\int f\le 0$,
then $f=0$ a.e.

(d) If $f$ and $g$ are integrable over $X$, $f\leae g$ and
$\int g\le\int f$, then $f\eae g$.

(e) If $f$ is integrable over $X$, so is $|f|$, and
$|\int f|\le\int|f|$.

\proof{{\bf (a)} If $f$ is integrable then $f=|f|\in U$, by
122P(ii).   If $f\in U$ then $f=f-\tbf{0}$ is integrable, by 122M.

\medskip

{\bf (b)} Let $E$, $F$ be conegligible sets such that $f\restr E$ is
measurable and $h\restr F=f\restr F$;  then $E\cap F$ is conegligible
and $h\restr(E\cap F)=(f\restr E)\restr F$ is measurable.   Next,
there
is a $g\in U$ such that $|f|\leae g$, and of course $|h|\leae g$.
So $h$ is integrable by 122P(iii).   By 122Od, applied to $f$ and $h$
and then to $h$ and $f$, $\int h=\int f$.

\medskip

{\bf (c)} \Quer\ Suppose, if possible, otherwise.   Let $E\subseteq X$
be a conegligible set such that $f\restr E$ is measurable (122P(ii)),
and $E'\subseteq E\cap\dom f$ a conegligible measurable set.   Then
$F=\{x:x\in E',\,f(x)>0\}$ must be non-negligible.   Set
$F_k=\{x:x\in
E',\,f(x)\ge 2^{-k}\}$ for each $k\in\Bbb N$, so that
$F=\bigcup_{k\in\Bbb N}F_k$ and there is a $k$ such that $\mu F_k>0$.
But consider $g=2^{-k}\chi F_k$.   Because $f\ge 0$ a.e.\ and
$f\ge 2^{-k}$ on $F_k$, $f\geae g$, so that

\Centerline{$ 0<2^{-k}\mu F_k=\int g\le\int f$,}

\noindent by 122Od;  which is impossible.  \Bang

\medskip

{\bf (d)} Apply (c) to $g-f$.

\medskip

{\bf (e)} By (i)$\Rightarrow$(ii) of 122P, $|f|$ is integrable.   Now
$f^+={1\over 2}(|f|+f)$ and $f^-={1\over 2}(|f|-f)$ are both
integrable
and non-negative, so have non-negative integrals, and

\Centerline{$|\int f|=|\int f^+-\int f^-|\le\int f^++\int
f^-=\int|f|$.}
}%end of proof of 122R

\exercises{
\leader{122X}{Basic exercises (a)} Let $(X,\Sigma,\mu)$ be a measure
space.   (i)
Show that if $f:X\to\Bbb R$ is simple so is $|f|$, setting
$|f|(x)=|f(x)|$ for
$x\in\dom f=X$.  (ii) Show that if $f$, $g:X\to\Bbb R$ are simple
functions so are $f\vee g$ and $f\wedge g$, as defined in 121Xb.
%122D

\sqheader 122Xb Let $(X,\Sigma,\mu)$ be a measure space and $f$
a real-valued function which is integrable over $X$.   Show that for
every $\epsilon>0$ there is a simple function $g:X\to\Bbb R$ such that
$\int|f-g|\le\epsilon$.   \Hint{consider non-negative $f$ first.}
%122O

\spheader 122Xc Let $(X,\Sigma,\mu)$ be a measure space, and
write $\eusm L^1$ for the set of all real-valued functions which are
integrable over $X$.   Let $\Phi\subseteq\eusm L^1$ be such that

\inset{(i) $\chi E\in\Phi$ whenever $E\in\Sigma$ and $\mu E<\infty$;

(ii) $f+g\in\Phi$ for all $f$, $g\in\Phi$;

(iii) $cf\in\Phi$ whenever $c\in\Bbb R$, $f\in\Phi$;

(iv) $f\in\Phi$ whenever $f\in\eusm L^1$ is such that there is a
non-decreasing sequence $\sequencen{f_n}$ in $\Phi$ for which
$\lim_{n\to\infty}f_n=f$ almost everywhere.}

\noindent Show that $\Phi=\eusm L^1$.
%122O

\sqheader 122Xd Let $\mu$ be counting measure on $\Bbb N$ (112Bd).
Show that a function $f:\Bbb N\to\Bbb R$ (that is, a sequence
$\sequencen{f(n)}$) is $\mu$-integrable iff it is absolutely summable,
and in this case

\Centerline{$\int fd\mu=\int_{\Bbb N}f(n)\mu(dn)
=\sum_{n=0}^{\infty}f(n)$.}
%122O

\sqheader 122Xe Let $(X,\Sigma,\mu)$ be a measure space and $f$, $g$
two
virtually measurable real-valued functions defined on subsets of $X$.
(i) Show that $f+g$, $f\times g$ and $f/g$, defined as in 121E, are
all
virtually measurable.   (ii) Show that if $h$ is a Borel
measurable real-valued function defined on any subset of $\Bbb R$,
then
the composition $hf$ is virtually measurable.
%122Q

\sqheader 122Xf Let $(X,\Sigma,\mu)$ be a measure space and
$\sequencen{f_n}$ a sequence of virtually measurable real-valued
functions defined on subsets of $X$.   Show that
$\lim_{n\to\infty}f_n$,
$\sup_{n\in\Bbb N}f_n$, $\inf_{n\in\Bbb N}f_n$,
$\limsup_{n\to\infty}f_n$
and $\liminf_{n\to\infty}f_n$, defined as in 121F, are virtually
measurable.
%122Q

\sqheader 122Xg Let $(X,\Sigma,\mu)$ be a measure space and $f$,
$g$ real-valued functions which are integrable over $X$.   Show that
$f\wedge g$ and $f\vee g$, as defined in 121Xb, are integrable.
%122R

\sqheader 122Xh Let $(X,\Sigma,\mu)$ be a measure space, $f$ a
real-valued function which is integrable over $X$, and $g$ a bounded
real-valued virtually measurable function defined on a conegligible
subset of $X$.    Show that $f\times g$, defined as in 121Ed, is
integrable.
%122R

%maybe better in 122Y?
\spheader 122Xi Let $X$ be a set, $\Sigma$ a $\sigma$-algebra of
subsets
of $X$, and $\mu_1$, $\mu_2$ two measures with domain $\Sigma$.   Set
$\mu E=\mu_1E+\mu_2E$ for $E\in\Sigma$.
Show that for any
real-valued function $f$ defined on a subset of $X$,
$\int fd\mu=\int fd\mu_1+\int fd\mu_2$ in the sense that if one side
is defined as a real
number so is the other, and they are then equal.   \Hint{($\alpha$)
Check that a subset of $X$ is $\mu$-conegligible iff it is
$\mu_i$-conegligible for
both $i$.  ($\beta$) Check the result for simple functions $f$.
($\gamma$) Now consider general non-negative $f$.}

\leader{122Y}{Further exercises (a)} Let $(X,\Sigma,\mu)$ be a
`complete' measure space, that is, one
in which all negligible sets are measurable (see, for instance,
113Xa). Show that if
$f$ is a virtually measurable real-valued function defined on a subset
of $X$, then $f$ is measurable.   Use this fact to find appropriate
simplifications of 122J and 122P for such spaces $(X,\Sigma,\mu)$.
%122P

\spheader 122Yb Write $\eusm L^1$ for the set of all Lebesgue
integrable real-valued functions on
$\Bbb R$.   Let $\Phi\subseteq\eusm L^1$ be such that

\inset{(i) $\chi I\in\Phi$ whenever $I$ is a bounded half-open
interval
in $\Bbb R$;

(ii) $f+g\in\Phi$ for all $f$, $g\in\Phi$;

(iii) $cf\in\Phi$ whenever $c\in\Bbb R$, $f\in\Phi$;

(iv) $f\in\Phi$ whenever $f\in\eusm L^1$ is such that there is a
non-decreasing sequence $\sequencen{f_n}$ in $\Phi$ for which
$\lim_{n\to\infty}f_n=f$ almost everywhere.}

\noindent Show that $\Phi=\eusm L^1$.   \Hint{show that ($\alpha$)
$\chi E\in\Phi$ whenever $E$ is expressible as the union of finitely
many half-open intervals ($\beta$) $\chi E\in\Phi$ whenever $E$ has
finite measure and is expressible as the union of a sequence of
half-open
intervals ($\gamma$) $\chi E\in\Phi$ whenever $E$ is measurable and
has
finite measure.}
%122O, 122Xc

\spheader 122Yc Let $X$ be any set, and let $\mu$ be counting
measure on $X$.   Let $f:X\to\Bbb R$ be a function;  set
$f^+(x)=\max(0,f(x))$, $f^-(x)=\max(0,-f(x))$ for $x\in X$.   Show
that the following are equiveridical:  (i) $\int fd\mu$ exists in $\Bbb R$,
and is equal to $s$;  (ii) for every $\epsilon>0$ there is a finite
$K\subseteq X$ such that $|s-\sum_{i\in I}f(i)|\le\epsilon$ whenever
$I\subseteq X$ is
a finite set including $K$ (iii) $\sum_{x\in X}f^+(x)$ and
$\sum_{x\in X}f^-(x)$, defined as in 112Bd, are finite, and
$s=\sum_{x\in X}f^+(x)-\sum_{x\in X}f^-(x)$.
%122Xd, 122O

\spheader 122Yd Let $(X,\Sigma,\mu)$ be a measure space.   Let
us say that a function $g:X\to\Bbb R$ is {\bf quasi-simple} if it is
expressible as $\sum_{i=0}^{\infty}a_i\chi G_i$, where
$\sequence{i}{G_i}$ is a partition of $X$ into measurable sets,
$\sequence{i}{a_i}$ is a sequence in $\Bbb R$, and
$\sum_{i=0}^{\infty}|a_i|\mu G_i<\infty$, counting $0\cdot\infty$ as
$0$, so that there can be $G_i$ of infinite measure provided that the
corresponding $a_i$ are zero.

\quad(i) Show that if $g$ and $h$ are quasi-simple functions so are
$g+h$, $|g|$ and $cg$, for any $c\in \Bbb R$.   \Hint{for $g+h$ you
will need 111F(b-ii) or its equivalent.}

\quad(ii) Show from first principles (I mean, without using anything
later than 122F in this chapter) that if
$g=\sum_{i=0}^{\infty}a_i\chi G_i$ and
$h=\sum_{i=0}^{\infty}b_i\chi H_i$ are quasi-simple
functions, and  $g\leae h$, then
$\sum_{i=0}^{\infty}a_i\mu G_i\le\sum_{i=0}^{\infty}b_i\mu H_i$.

\quad{(iii)} Hence show that we have a functional $I$ defined by
saying that $I(g)=\sum_{i=0}^{\infty}a_i\mu G_i$ whenever $g$ is a
quasi-simple
function represented as $\sum_{i=0}^{\infty}a_i\chi G_i$.

\quad{(iv)} Show that if $g$ and $h$ are quasi-simple functions and
$c\in\Bbb R$,
then $I(g+h)=I(g)+I(h)$ and $I(cg)=cI(g)$, and that $I(g)\le I(h)$ if
$g\leae h$.

\quad{(v)} Show that if $g$ is a quasi-simple function then $g$ is
integrable and $\int g=I(g)$.   (I do now allow you to use
122G-122R.)

\quad{(vi)} Show that a real-valued function $f$, defined on a
conegligible subset of $X$, is integrable iff for every $\epsilon>0$
there are quasi-simple functions $g$, $h$ such that $g\leae f\leae h$
and $I(h)-I(g)\le\epsilon$.
%122R

\spheader 122Ye Let $\mu$ be Lebesgue measure on $\Bbb R$.   Let
us say (for this exercise only) that a real-valued function $g$ with
$\dom
g\subseteq\Bbb R$ is `pseudo-simple' if it is expressible as
$\sum_{i=0}^{\infty}a_i\chi J_i$, where $\sequence{i}{J_i}$ is a
sequence
of bounded half-open intervals ({\it not} necessarily disjoint) and
$\sum_{i=0}^{\infty}|a_i|\mu J_i<\infty$.    (Interpret the infinite
sum
$\sum_{i=0}^{\infty}a_i\chi J_i$ as in 121F, so that

\Centerline{$\dom(\sum_{i=0}^{\infty}a_i\chi J_i)
=\{x:\lim_{n\to\infty}\sum_{i=0}^na_i(\chi J_i)(x)
  \text{ exists in }\Bbb R\}$.)}

\quad(i) Show that if $g$, $h$ are pseudo-simple functions so are
$g+h$
and $cg$, for any $c\in \Bbb R$.

\quad(ii) Show that if $g$ is a pseudo-simple function then $g$ is
integrable.

\quad(iii) Show that a real-valued function $f$, defined on a
conegligible subset of $\Bbb R$, is integrable iff for every
$\epsilon>0$
there are pseudo-simple functions $g$, $h$ such that $g\leae f\leae h$
and $\int h-g\,d\mu\le\epsilon$.   \Hint{Take $\Phi$ to be the set
of integrable functions with this property, and show that it satisfies
the
conditions of 122Yb.}
%122Yd, 122R

\spheader 122Yf Repeat 122Yb and 122Ye for Lebesgue measure on
$\BbbR^r$,
where $r>1$.
%122Yd, 122Yb, 122Ye, 122R

\spheader 122Yg Let $(X,\Sigma,\mu)$ be a measure space, and
assume that there is at least one partition of $X$ into infinitely
many
non-empty measurable sets.   Let $f$ be a real-valued function
defined on a conegligible subset of $X$, and $a\in\Bbb R$.   Show that
the following are equiveridical:

\quad (i) $f$ is integrable, with $\int f=a$;

\quad (ii) for every $\epsilon>0$ there is a partition
$\sequencen{E_n}$
of $X$ into non-empty measurable sets such that

\Centerline{$\sum_{n=0}^{\infty}|f(t_n)|\mu E_n<\infty$,
\quad$|a-\sum_{n=0}^{\infty}f(t_n)\mu E_n|\le\epsilon$}

\noindent whenever $\sequencen{t_n}$ is a sequence such that
$t_n\in E_n\cap\dom f$ for every $n$.   (As usual, take
$0\cdot\infty=0$ in these formulae.)  \Hint{use 122Yd.}
%122Yd, 122R

\spheader 122Yh Find a re-formulation of (g) which covers the case
of measure spaces which can {\it not} be partitioned into sequences of
non-empty measurable sets.
%122Yg, 122Yd, 122R

\spheader 122Yi Let $X$ be a set, $\Sigma$ a $\sigma$-algebra of
subsets
of $X$, and $\sequencen{\mu_n}$ a sequence of measures with domain
$\Sigma$.   Set $\mu E=\sum_{n=0}^{\infty}\mu_nE$ for $E\in\Sigma$.
(i) Show that $\mu$ is a measure.
(ii) Show that for any real-valued function $f$ defined on a subset of $X$,
$f$ is $\mu$-integrable iff it is $\mu_n$-integrable for every $n$ and
$\sum_{n=0}^{\infty}\int|f|d\mu_n$ is finite, and that then
$\int fd\mu=\sum_{n=0}^{\infty}\int fd\mu_n$.
%122Xi

\spheader 122Yj Let $X$ be a set, $\Sigma$ a $\sigma$-algebra of
subsets
of $X$, and $\langle\mu_i\rangle_{i\in I}$ a family of measures with
domain $\Sigma$.
Set $\mu E=\sum_{i\in I}\mu_iE$ for $E\in\Sigma$.
(i) Show that $\mu$ is a measure.
(ii) Show that for any $\Sigma$-measurable function $f:X\to\Bbb R$,
$f$ is $\mu$-integrable iff it is $\mu_i$-integrable for every $i$ and
$\sum_{i\in I}\int|f|d\mu_i$ is finite.
%122Yi, 122Xi
}%end of exercises

\endnotes{
\Notesheader{122} Just as in \S121, some extra
technical problems are caused by my insistence on trying to integrate
(i) functions which are not defined on the whole of the measure space
under consideration (ii) functions which are not, strictly speaking,
measurable, but are only measurable on some conegligible set.   There
is
nothing in the present section to justify either of these
elaborations.
In the next section, however, we shall be looking at the limits of
sequences of functions, and these limits need not be defined at every
point;  and the examples in which the limits are not everywhere
defined are not in any sense pathological, but are central to the most
important applications of the theory.

The question of integrating not-quite-measurable functions is more
disputable.   I will discuss this point further after formally
introducing
`complete' measure spaces in Chapter 21.   For the moment, I will
say only that I think it is worth taking the trouble to have a
formalisation which integrates as many functions as is reasonably
possible;  the original point of the Lebesgue integral being, in part,
that it enables us to integrate more functions than its predecessors.

The definition of `integration' here proceeds in three
distinguishable stages:  (i) integration of simple functions
(122A-122G);  (ii) integration of non-negative functions (122H-122L);
(iii) integration of general real-valued functions (122M-122R).   I
have
taken each stage slowly, passing to non-negative integrable functions
only when I have a full set of the requisite lemmas on simple
functions,
for instance.   There are, of course, innumerable alternative routes;
see, for instance, 122Yd, which offers a definition using two steps
rather than three.   I
prefer the longer, gentler climb partly because (to my eye) it gives a
clearer view of the ideas and partly because it corresponds to an
almost
canonical method of proving properties of integrable functions:  we
prove them first for simple functions, then for non-negative
integrable
functions, then for general integrable functions.   (The hint I give
for
122Yb conforms to this philosophy.   See also 122Xc;  but I
do not give this as a formally expressed theorem, because the exact
order of proof varies from case to case, and I think it is best
remembered as a method of attack rather than as a specific result to
apply.)

You have a right to feel that this section has been singularly
abstract,
and gives very little idea of which of your favourite functions are
likely to be integrable, let alone what the integrals are.   I hope
that
Chapter 13 will provide some help in this direction, though I have to
say that for really useful methods for calculating integrals we must
wait for Chapters 22, 25 and 26 in the next volume.    If you want to
know the true centre of the arguments of this section, I would myself
locate it in 122G, 122H and 122K.   The point is that the ideas of
122A-122F apply to a much wider class of structures $(X,\Sigma,\mu)$,
because they involve only operations on finitely many members of
$\Sigma$ at a time;  there is no mention of sequences of sets.   The
key
that makes all the rest possible is 122G, which is founded on 112Cf.
And after 122G-122K, the rest of the section, although by no means
elementary, really is no more than a careful series of checks to
ensure
that the functional defined in 122K behaves as we expect it to.

Many of the results of this section (including the key one, 122G) will
be superseded by stronger results in the following section.   But I
should remark on Lemma 122Ja, which will periodically recur as a most
useful criterion for integrability of non-negative functions (see
122Ra).

There is another point about the standard integral as defined here.   It
is an `absolute' integral, meaning that if $f$ is integrable so is
$|f|$ (122P).   This means that although the Lebesgue integral extends
the `proper' Riemann integral (see 134K below), there are functions
with
finite `improper' Riemann integrals which are not Lebesgue integrable;
a typical example is $f(x)=\bover{\sin x}x$, where
$\lim_{a\to\infty}\int_0^af$ exists in $\Bbb R$, while
$\lim_{a\to\infty}\int_0^a|f|=\infty$, so that $f$ is not integrable,
in the sense defined here, over the whole interval $\ooint{0,\infty}$.
(For full proofs of these assertions, see 283D and 282Xm in Volume
2.)   If you have encountered the theory of `absolutely' and
`conditionally'
summable series, you will be aware that the latter can exhibit very
confusing behaviour, and will appreciate that restricting the notion of
`integrable' to mean `absolutely integrable' is a great convenience.

Indeed, it is more than just a convenience;  it is necessary to make the
definition work at the level of abstraction used in this chapter.
Consider the example of counting measure $\mu$ on $\Bbb N$ (112Bd,
122Xd).   The structure $(\Bbb N,\Cal P\Bbb N,\mu)$ is invariant under
permutations;
that is, $\mu(\pi[A])=\mu A$ for every $A\subseteq\Bbb N$ and every
permutation $\pi:\Bbb N\to\Bbb N$.   Consequently, {\it any} definition
of integration which depends only on the structure
$(\Bbb N,\Cal P\Bbb N,\mu)$ must also be invariant under
permutations, that is,

\Centerline{$\int f(\pi(n))\mu(dn)=\int f(n)\mu(dn)$}

\noindent for any integrable function $f$ and any permutation $\pi$.
But of course (as I hope you have been told) a series
$\sequencen{f(n)}$ such that
$\sum_{n=0}^{\infty}f(\pi(n))=\sum_{n=0}^{\infty}f(n)\in\Bbb R$ for
any permutation $\pi$ must be absolutely summable.   Thus if we are to
define an integral on an abstract measure space $(X,\Sigma,\mu)$ in
terms depending only on $\Sigma$ and $\mu$, we are nearly inevitably
forced to define an absolute integral.

Naturally there are important contexts in which this restriction is an
embarrassment, and in which some kind of `improper' integral seems
appropriate.   A typical one is the theory of Fourier transforms, in
which we find ourselves looking at $\lim_{a\to\infty}\int_{-a}^af$ in
place of $\int_{-\infty}^{\infty}f$ (see \S283).   A vast number of
more or less abstract forms of improper integral have been proposed;
 many are interesting and some are important;  but none rivals the
`standard'
integral as described in this chapter.   (For an attempt at a
systematic examination of a particular class of such improper
integrals, see Chapter 48 in Volume 4.)

Much less work has been done on the integration of non-measurable
functions -- to speak more exactly, of functions which are not equal
almost everywhere to a measurable integrable function.   I am sure that
this is simply because there are too few important problems to show us
which way to turn.   In 134C below I mention the question of whether
there is {\it any} non-measurable real-valued function on $\Bbb R$.
The standard answer is `yes', but no such function can possibly arise
as a result of any ordinary construction.    Consequently the
majority of
questions concerning non-measurable functions appear in very special
contexts, and so far I have seen none which gives any useful hint of
what a generally appropriate extension of the notion of
`integrability' might be.
}%end of notes

\discrpage

