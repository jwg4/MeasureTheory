\frfilename{mt535.tex}
\versiondate{1.6.11}
\copyrightdate{2003}

\def\chaptername{Topologies and measures III}
\def\sectionname{Liftings}

\newsection{535}

I introduced the Lifting Theorem\cmmnt{ (\S341)} as one of the
fundamental facts
about complete strictly localizable measure spaces.   Of course we can
always complete a measure space and thereby in effect obtain a lifting
for any $\sigma$-finite measure.   For the applications of the Lifting
Theorem in \S\S452-453 this procedure is natural and effective;  and
generally in this treatise I have taken the view that one should work
with completed measures unless there is some strong reason not to.   But
I have also embraced the principle of maximal convenient generality,
seeking formulations which will exhibit the full force of each idea in
the context appropriate to that idea, uncluttered by the special
features of intended applications.   So the question of when, and why,
liftings for incomplete measures can be found is one which automatically
arises.   It turns out to be a fruitful question, in the sense that it
leads us to new arguments, even though the answers so far available are
unsatisfying.

As usual, much of what we want to know depends on the behaviour of the
usual measures on powers of $\{0,1\}$ (535B).   An old argument relying
on the continuum hypothesis shows that Lebesgue measure can have a Borel
lifting;  this has been usefully refined, and I give a strong version in
535D-535E.
We know that we cannot expect to have translation-invariant Borel
liftings\cmmnt{ (345F)}, but strong Borel liftings are possible
(535H-535I), and in some cases can be built from Borel liftings
(535J-535N).   %535J 535N

For certain applications in functional analysis, we are more interested
in liftings for $L^{\infty}$ spaces than in liftings for measure algebras;
and it is sometimes sufficient to have a `linear lifting', not
necessarily corresponding to a lifting in the strict sense (535O,
535P).   I give a couple of paragraphs to linear liftings because in
some ways they are easier to handle and it is conceivable that they are
relevant to the main outstanding problem (535Zf).

\leader{535A}{Notation (a)}\cmmnt{ The most interesting questions to
be examined in this section can be phrased in the following language.}
If $(X,\Sigma,\mu)$ is a measure space and $\frak T$ a topology on $X$,
\cmmnt{I will say that} a {\bf Borel lifting} of $\mu$ is a lifting which takes
values in the Borel $\sigma$-algebra $\Cal B(X)$ of $X$.   \cmmnt{(As
usual, I will use the word `lifting' indifferently for homomorphisms
from $\Sigma$ to itself, or from $\frak A$ to $\Sigma$, where $\frak A$
is the measure algebra of $\mu$.   Of course a homomorphism
$\theta:\frak A\to\Sigma$ is a Borel lifting iff the corresponding
homomorphism $E\mapsto\theta E^{\ssbullet}:\Sigma\to\Sigma$ is a Borel
lifting.)}   Similarly, a {\bf Baire lifting} of $\mu$ is a lifting
which takes values in the Baire $\sigma$-algebra $\CalBa(X)$ of $X$.

\spheader 535Ab I remark at once that if $(X,\frak T,\Sigma,\mu)$ is a
topological measure space and $\phi:\Sigma\to\Cal B(X)$ is a Borel
lifting for $\mu$, then $\phi\restr\Cal B(X)$ is a lifting for the Borel
measure $\mu\restr\Cal B(X)$.   Conversely, if
$\phi':\Cal B(X)\to\Cal B(X)$ is a lifting for $\mu\restr\Cal B(X)$, and
if for every $E\in\Sigma$ there is a Borel set $E'$ such that
$E\symmdiff E'$ is negligible, then $\phi'$ extends uniquely to a Borel
lifting $\phi$ of $\mu$.

In the same way, any Baire lifting for a measure $\mu$ which measures
every zero set will give\cmmnt{ us} a lifting for $\mu\restr\CalBa(X)$;  and a lifting for $\mu\restr\CalBa(X)$ will correspond to a Baire lifting for $\mu$ if\cmmnt{, for instance,} $\mu$ is completion regular\cmmnt{, as in 535B below}.

\spheader 535Ac \cmmnt{As in Chapter 52,} I will say that, for any set
$I$, $\nu_I$ is the usual measure on $\{0,1\}^I$ and $\frak B_I$ its
measure algebra.

\leader{535B}{Proposition} Let $(X,\Sigma,\mu)$ be a strictly
localizable measure space with non-zero measure.   Suppose that
$\nu_{\kappa}$ has a Baire lifting\cmmnt{ (that is,
$\nu_{\kappa}\restr\CalBa(\{0,1\}^{\kappa})$ has a lifting)} for every
infinite cardinal $\kappa$
such that the Maharam-type-$\kappa$ component of the measure algebra of
$\mu$ is non-zero.   Then $\mu$ has a lifting.

\proof{ Write $(\frak A,\bar\mu)$ for the measure algebra of $\mu$.

\medskip

{\bf (a)} Suppose first that $\mu$ is a Maharam-type-homogeneous
probability measure.   In this case $\frak A$ is either $\{0,1\}$ or
isomorphic to $\frak B_{\kappa}$ for some infinite $\kappa$.   The case
$\frak A=\{0,1\}$ is trivial, as we can set $\phi E=\emptyset$ if
$E\in\Sigma$ is negligible, $\phi E=X$ if $E\in\Sigma$ is conegligible.
Otherwise, $\frak A$ is
$\tau$-generated by a stochastically independent family
$\ofamily{\xi}{\kappa}{e_{\xi}}$ of elements of measure $\bover12$.
For each $\xi<\kappa$, choose $E_{\xi}\in\Sigma$ such that
$E_{\xi}^{\ssbullet}=e_{\xi}$, and define $f:X\to\{0,1\}^{\kappa}$ by
setting $f(x)(\xi)=\chi E_{\xi}(x)$ for $x\in X$ and $\xi<\kappa$.
Then $\{F:F\subseteq\{0,1\}^{\kappa}$, $\nu F$ and $\mu f^{-1}[F]$ are
defined and equal$\}$ is a Dynkin class containing all the measurable
cylinders in $\{0,1\}^{\kappa}$, so includes
$\CalBa_{\kappa}=\CalBa(\{0,1\}^{\kappa})$,
and $f$ is \imp\ for $\mu$ and
$\nuprime_{\kappa}=\nu_{\kappa}\restr\CalBa_{\kappa}$.   Note that
$\frak B_{\kappa}$ can be identified with the measure algebra of
$\nuprime_{\kappa}$ (put 415E and
322Da together, or see 415Xp).   So we have an induced
measure-preserving Boolean homomorphism $\pi:\frak B_{\kappa}\to\frak A$
defined by setting $\pi F^{\ssbullet}=f^{-1}[F]^{\ssbullet}$ for every
$F\in\CalBa_{\kappa}$.   Since $\pi[\frak B_{\kappa}]$ is an
order-closed subalgebra of $\frak A$ (324Kb) containing every $e_{\xi}$,
it is the whole of $\frak A$.

We are supposing that there is a lifting
$\theta:\frak B_{\kappa}\to\CalBa_{\kappa}$ of
$\nu_{\kappa}$.   Define
$\theta_1:\frak A\to\Sigma$ by setting
$\theta_1a=f^{-1}[\theta\pi^{-1}a]$ for every $a\in\frak A$;  then
$\theta_1$ is a Boolean homomorphism because $\theta$ and $\pi^{-1}$
are, and

\Centerline{$(\theta_1a)^{\ssbullet}=\pi((\theta\pi^{-1}a)^{\ssbullet})
=\pi\pi^{-1}a=a$}

\noindent for every $a\in\frak A$, so $\theta_1$ is a lifting for $\mu$.

\medskip

{\bf (b)} It follows at once that if $\mu$ is any non-zero totally
finite Maharam-type-homogeneous measure, then it will have a lifting, as
we can apply (a) to a scalar multiple of $\mu$.   Now consider the
general case.   Let $\Cal K$ be the family of measurable subsets $K$ of
$X$ such that the subspace measure $\mu_K$ is non-zero, totally finite
and Maharam-type-homogeneous.   Then $\mu$ is inner regular with respect
to $\Cal K$, by Maharam's theorem (332B).   By 412Ia, there is a
decomposition $\familyiI{X_i}$ of $X$ such that at most
one $X_i$ does not belong to $\Cal K$, and that exceptional one, if any,
is negligible;  adding a trivial element $X_k=\emptyset$ if necessary,
we may suppose that there is exactly one $k\in I$ such that
$\mu X_k=\emptyset$.   For each $i\in I\setminus\{k\}$, let $\mu_i$ be
the subspace measure on $X_i$, and $\Sigma_i$ its domain;  then $\mu_i$
has a lifting $\phi_i:\Sigma_i\to\Sigma_i$.   (The point is that if the
Maharam type $\kappa$ of $\mu_i$ is infinite, then the
Maharam-type-$\kappa$ component of $\frak A$ includes $X_i^{\ssbullet}$
and is non-zero, so our hypothesis tells us that $\nu_{\kappa}$ has a
Baire lifting.)   At this point, recall that we are also supposing that
$\mu X>0$, so there is some $j\in I\setminus\{k\}$;  fix $z\in X_j$, and
define $\phi:\Sigma\to\Cal PX$ by setting

$$\eqalign{\phi E&=\bigcup_{i\in I\setminus\{k\}}\phi_i(E\cap X_i)
  \text{ if }z\notin\phi_j(E\cap X_j),\cr
&=X_k\cup\bigcup_{i\in I\setminus\{k\}}\phi_i(E\cap X_i)
  \text{ if }z\in\phi_j(E\cap X_j).\cr}$$

\noindent Then $\phi$ is a lifting for $\mu$.   \Prf\ It is a Boolean
homomorphism because every $\phi_i$ is.   If $E\in\Sigma$, then
$X_i\cap\phi E=\phi_i(E\cap X_i)$ if $i\in I\setminus\{k\}$, and is
either $X_k$ or $\emptyset$ if $i=k$;  in any case, it belongs to
$\Sigma_i$;  as $\familyiI{X_i}$ is a decomposition for $\mu$,
$\phi E\in\Sigma$.   Also

\Centerline{$\mu(E\symmdiff\phi E)
\le\mu X_k+\sum_{i\in I\setminus\{k\}}
  \mu_i((E\cap X_i)\symmdiff\phi_i(E\cap X_i))
=0$.}

\noindent Finally, if $\mu E=0$, then $\mu_i(E\cap X_i)=0$ and
$\phi_i(E\cap X_i)=\emptyset$ for every $i\in I\setminus\{k\}$, so
$\phi E=\emptyset$.\ \Qed
}%end of proof of 535B

\leader{535C}{Proposition} If $\lambda$ and $\kappa$ are cardinals with
$\lambda=\lambda^{\omega}\le\kappa$, and $\nu_{\kappa}$ has a Baire
lifting, then $\nu_{\lambda}$ has a Baire lifting.

\proof{ If $\lambda$ is finite, the result is trivial, so we may suppose
that $\lambda\ge\omega$ (and therefore that $\lambda\ge\frak c$).   For
$I\subseteq\kappa$, write $\CalBa_I$ for the Baire
$\sigma$-algebra of $\{0,1\}^I$ and $\Tau_I$ for the family of those
$E\in\CalBa_{\kappa}$ which are determined by coordinates in $I$.
Set $\pi_I(x)=x\restr I$ for every $x\in\{0,1\}^{\kappa}$;  then
$H\mapsto\pi_I^{-1}[H]$ is a Boolean isomorphism between $\CalBa_I$ and
$\Tau_I$, with inverse $E\mapsto\pi_I[E]$.   \Prf\ Because $\pi_I$ is
continuous,
$\pi_I^{-1}[H]\in\CalBa_{\kappa}$ for every $H\in\CalBa_I$.   Of course
$H\mapsto\pi_I^{-1}[H]$ is a Boolean homomorphism, and it is injective
because $\pi_I$ is surjective.   Identifying $\{0,1\}^{\kappa}$ with
$\{0,1\}^I\times\{0,1\}^{\kappa\setminus I}$, we have a function
$h:\{0,1\}^I\to\{0,1\}^{\kappa}$ defined by setting $h(v)=(v,\tbf{0})$
for $v\in\{0,1\}^I$.   This is continuous, therefore
$(\CalBa_I,\CalBa_{\kappa})$-measurable.   If $E\in\Tau_I$, then
$E=\pi_I^{-1}[\pi_I[E]]=\pi_I^{-1}[h^{-1}[E]]$;   so
$H\mapsto\pi_I^{-1}[H]$ is surjective and is an isomorphism.\ \QeD\

Consequently $\#(\Tau_I)\le\frak c$ for every countable
$I\subseteq\kappa$
(4A1O, because $\CalBa_I$ is $\sigma$-generated by the cylinder sets, by
4A3Na).   For any $I$, $\Tau_I=\bigcup_{J\in[I]^{\le\omega}}\Tau_J$,
because every member of $\CalBa_I$ is determined by coordinates in a
countable set (4A3Nb).   So
$\#(\Tau_I)\le\max(\frak c,\#(I)^{\omega})=\lambda$ whenever
$I\subseteq\kappa$ and $\#(I)=\lambda$.

Let $\phi$ be a Baire lifting for $\nu_{\kappa}$.
Choose a non-decreasing family $\ofamily{\xi}{\omega_1}{J_{\xi}}$ in
$[\kappa]^{\lambda}$ such that $J_0=\lambda$ and
$\phi E\in\Tau_{J_{\xi+1}}$ whenever $\xi<\omega_1$ and
$E\in\Tau_{J_{\xi}}$.   Set $J=\bigcup_{\xi<\omega_1}J_{\xi}$;  then
$\Tau_J=\bigcup_{\xi<\omega_1}\Tau_{J_{\xi}}$, so $\phi E\in\Tau_J$ for
every $E\in\Tau_J$.

We therefore have a Boolean homomorphism $\phi_1:\CalBa_J\to\CalBa_J$
defined by setting $\phi_1H=\pi_J[\phi(\pi_J^{-1}[H])]$ for every
$H\in\CalBa_J$.   If $\nu_JH=0$, then $\nu_{\kappa}\pi_J^{-1}[H]=0$ and
$\phi_1H=\phi(\pi_J^{-1}[H])=0$.   For any $H\in\CalBa_J$,

\Centerline{$\pi_J^{-1}[H\symmdiff\phi_1H]
=\pi_J^{-1}[H]\symmdiff\phi(\pi_J^{-1}[H])$}

\noindent is $\nu_{\kappa}$-negligible, so $H\symmdiff\phi_1H$ is
$\nu_J$-negligible.   Thus $\phi_1$ is a lifting for
$\nu_J\restr\CalBa_J$.   As $\nu_J\restr\CalBa_J$ is isomorphic to
$\nu_{\lambda}\restr\CalBa_{\lambda}$, the latter also has a lifting.
As $\nu_{\lambda}$ is completion regular (416U), the measure algebra of
$\nu_{\lambda}\restr\CalBa_{\lambda}$ can be identified with
$\frak B_{\lambda}$, and we can interpret a lifting for
$\nu_{\lambda}\restr\CalBa_{\lambda}$ as a Baire lifting for its
completion $\nu_{\lambda}$.
}%end of proof of 535C

\leader{535D}{}\cmmnt{ The following result covers most of the cases
in which non-complete probability measures are known to have liftings.

\medskip

\noindent}{\bf Theorem} Let $(X,\Sigma,\mu)$ be a measure space such
that $\mu X>0$, and suppose that its measure algebra is tightly
$\omega_1$-filtered\cmmnt{ (definition:  511Di)}.   Then $\mu$ has a
lifting.

\proof{ This is a special case of 518L.
}%end of proof of 535D

\leader{535E}{Proposition} Suppose that $\frak c\le\omega_2$ and
the Freese-Nation number $\FN(\Cal P\Bbb N)$ is $\omega_1$.

(a) If $\frak A$ is a measurable
algebra with cardinal at most $\omega_2$, it is tightly
$\omega_1$-filtered.

(b) ({\smc Mokobodzki 7?})
% Corollary 19
Let $(X,\Sigma,\mu)$ be a
$\sigma$-finite measure space with non-zero measure and
Maharam type at most $\omega_2$.

\quad(i) $\mu$ has a lifting.

\quad(ii) If $\frak T$ is a topology on $X$ such that $\mu$ is inner
regular with respect to the Borel sets, then $\mu$ has a Borel lifting.

\quad(iii) If $\frak T$ is a topology on $X$ such that $\mu$ is inner
regular with respect to the zero sets, then $\mu$ has a Baire lifting.

\proof{{\bf (a)} By 524O(b-iii), $\FN(\frak A)\le\omega_1$, so 518M gives
the result.

\medskip

{\bf (b)(i)} By 514De, the measure algebra of $\mu$ has
cardinal at most

\Centerline{$\omega_2^{\omega}=\max(\frak c,\omega_2)\le\omega_2$}

\noindent (5A1E(e-iii)).   So we can put (a) and 535D together.

\medskip

\quad{\bf (ii)} Because $\mu$ is $\sigma$-finite and inner regular with
respect to the Borel sets, every measurable set can be expressed as the
union of a Borel set and a negligible set.   By (i),
$\mu\restr\Cal B(X)$ has a lifting, which can be interpreted as a Borel
lifting for $\mu$, as in 535Ab.

\medskip

\quad{\bf (iii)} As (ii), but with $\CalBa(X)$ in place of $\Cal B(X)$.
}%end of proof of 535E

\leader{535F}{}\cmmnt{ Using the continuum hypothesis, we can go a little
farther with ideas from 341J.

\wheader{535F}{6}{2}{2}{48pt}

\noindent}{\bf Proposition}
Let $(X,\Sigma,\mu)$ be a measure space such that
$\mu X>0$ and $\#(\frak A)\le\omega_1$,
where $\frak A$ is the measure algebra of $\mu$, and suppose that
$\undtheta:\frak A\to\Sigma$ is such that

\Centerline{$\undtheta 0=\emptyset$,
\quad$\undtheta(a\Bcap b)=\undtheta a\cap\undtheta b$ for all $a$, $b\in\frak A$,
\quad$(\undtheta a)^{\ssbullet}\Bsubseteq a$ for every $a\in\frak A$.}

\noindent Then $\mu$ has a lifting $\theta:\frak A\to\Sigma$ such that
$\theta E^{\ssbullet}\supseteq\undtheta E$ for every $E\in\Sigma$.

\proof{{\bf (a)}
Adjusting $\undtheta 1$ if necessary, we can suppose that $\undtheta 1=X$.
Note that $\undtheta a\subseteq\undtheta b$ whenever $a\Bsubseteq b$ in $\frak A$.
Let $\ofamily{\xi}{\omega_1}{a_{\xi}}$ be a family running over $\frak A$,
and for $\alpha\le\omega_1$ let $\frak C_{\alpha}$ be the subalgebra of
$\frak A$ generated by $\{a_{\xi}:\xi<\alpha\}$.   Define Boolean
homomorphisms $\theta_{\alpha}:\frak C_{\alpha}\to\Sigma$ inductively, as
follows.   The inductive hypothesis will be that
$(\theta_{\alpha}c)^{\ssbullet}=c$ and $\theta_{\alpha}c\supseteq\undtheta c$
for every $c\in\frak C_{\alpha}$, while $\theta_{\alpha}$ extends
$\theta_{\beta}$ for every $\beta\le\alpha$.
Start with $\theta_00=\emptyset$, $\theta_01=X$.

\medskip

{\bf (b)} Given $\theta_{\alpha}$, where $\alpha<\omega_1$, set

\Centerline{$F
=\bigcup\{\undtheta(c\Bcup a_{\alpha})\Bsetminus\theta_{\alpha}c:
  c\in\frak C_{\alpha}\}$,}

\Centerline{$G
=\bigcup\{\undtheta(c\Bcup(1\Bsetminus a_{\alpha}))\Bsetminus\theta_{\alpha}c:
  c\in\frak C_{\alpha}\}$.}

\noindent Because $\frak C_{\alpha}$ is countable, $F$ and
$G$ belong to $\Sigma$.   If $c\in\frak C_{\alpha}$, then

\Centerline{$(\undtheta(c\Bcup a_{\alpha})\setminus\theta_{\alpha}c)^{\ssbullet}
=\undtheta(c\Bcup a_{\alpha})^{\ssbullet}\Bsetminus c
\Bsubseteq(c\Bcup a_{\alpha})\Bsetminus c
\Bsubseteq a_{\alpha}$,}

\noindent so $F^{\ssbullet}\Bsubseteq a_{\alpha}$;  similarly,
$G^{\ssbullet}\Bsubseteq 1\Bsetminus a_{\alpha}$.
Next, $F\cap G=\emptyset$.   \Prf\ If $b$,
$c\in\frak C_{\alpha}$, then

$$\eqalign{(\undtheta(b\Bcup a_{\alpha})\setminus\theta_{\alpha}b)
   \cap(\undtheta(c\Bcup(1\Bsetminus a_{\alpha}))\setminus\theta_{\alpha}c)
&=\undtheta((b\Bcup a_{\alpha})\Bcap(c\Bcup(1\Bsetminus a_{\alpha})))
   \setminus(\theta_{\alpha}b\Bcup\theta_{\alpha}c)\cr
&\subseteq\undtheta(b\Bcup c)
   \setminus\theta_{\alpha}(b\Bcup c)
=\emptyset.  \text{ \Qed}\cr}$$

\noindent Choose any $E\in\Sigma$ such that $E^{\ssbullet}=a_{\alpha}$ and
set $E_{\alpha}=(E\cup F)\setminus G$;  then
$E_{\alpha}^{\ssbullet}=a_{\alpha}$, $F\subseteq E_{\alpha}$ and
$G\cap E_{\alpha}=\emptyset$.

If $c\in\frak C_{\alpha}$ and $c\Bsubseteq a_{\alpha}$, then
$\undtheta((1\Bsetminus c)\Bcup a_{\alpha})=\undtheta 1=X$, so

\Centerline{$\theta_{\alpha}c
=\undtheta((1\Bsetminus c)\Bcup a_{\alpha})
  \setminus\theta_{\alpha}(1\Bsetminus c)
\subseteq F\subseteq E_{\alpha}$.}

\noindent Similarly, if $c\in\frak C_{\alpha}$ and
$c\Bcap a_{\alpha}=0$, then

\Centerline{$\theta_{\alpha}c
=\undtheta((1\Bsetminus c)\Bcup(1\Bsetminus
a_{\alpha}))\setminus\theta_{\alpha}(1\Bsetminus c)
\subseteq G$}

\noindent is disjoint from $E_{\alpha}$.   We can therefore define a
Boolean homomorphism $\theta_{\alpha+1}:\frak C_{\alpha+1}\to\Sigma$ by
setting

\Centerline{$\theta_{\alpha+1}((b\Bcap a_{\alpha})
  \Bcup(c\Bsetminus a_{\alpha}))
=(\theta_{\alpha}b\cap E_{\alpha})
  \cup(\theta_{\alpha}c\setminus E_{\alpha})$}

\noindent for all $b$, $c\in\frak C_{\alpha}$ (312O), and
$\theta_{\alpha+1}$ will extend $\theta_{\beta}$ for every
$\beta\le\alpha+1$.
Because 
$(\theta_{\alpha+1}a_{\alpha})^{\ssbullet}=E_{\alpha}^{\ssbullet}
=a_{\alpha}$ and
$\theta_{\alpha+1}c=\theta_{\alpha}c$ for every $c\in\frak C_{\alpha}$,
$(\theta_{\alpha+1}a)^{\ssbullet}=a$ for every $a\in\frak C_{\alpha+1}$.

I have still to check the other part of the inductive hypothesis.   If $b$,
$c\in\frak C_{\alpha}$, then

$$\eqalign{
\undtheta((b\Bcap a_{\alpha})\Bcup(c\Bsetminus a_{\alpha}))
&=\undtheta((b\Bcup c)\Bcap(c\Bcup a_{\alpha})
   \Bcap(b\Bcup(1\Bsetminus a_{\alpha})))\cr
&=\undtheta(b\Bcup c)\cap\undtheta(c\Bcup a_{\alpha})
   \cap\undtheta(b\Bcup(1\Bsetminus a_{\alpha}))\cr
&\subseteq\theta_{\alpha}(b\Bcup c)\cap(F\cup\theta_{\alpha}c)
   \cap(G\cup\theta_{\alpha}b)\cr
&\subseteq\theta_{\alpha+1}(b\Bcup c)
  \cap(\theta_{\alpha+1}a_{\alpha}\cup\theta_{\alpha+1}c)
  \cap(\theta_{\alpha+1}(1\Bsetminus a_{\alpha})\cup\theta_{\alpha+1}b)\cr
&=\theta_{\alpha+1}((b\Bcap a_{\alpha})\Bcup(c\Bsetminus a_{\alpha})),
\cr}$$

\noindent which is what we need to know.

\medskip

{\bf (c)} For non-zero limit ordinals $\alpha\le\omega_1$, we have
$\frak C_{\alpha}=\bigcup_{\beta<\alpha}\frak C_{\beta}$ so we can, and
must, take $\theta_{\alpha}=\bigcup_{\beta<\alpha}\theta_{\beta}$.

At the end of the induction, $\theta_{\omega_1}:\frak A\to\Sigma$ is an
appropriate lifting.
}%end of proof of 535F

\leader{535G}{Corollary}\cmmnt{ (see {\smc Neumann 31})} % p. 111
Suppose that $\frak c=\omega_1$.   Then for any integer $r\ge 1$ there is
a Borel lifting $\theta$ of Lebesgue measure  on $\Bbb R^r$ such that
$x\in\theta E^{\ssbullet}$ whenever $E\subseteq\Bbb R^r$ is a Borel set and
$x$ is a density point of $E$.

\proof{ In 535F, let $\undtheta$ be lower Lebesgue density (341E), 
interpreted
as a function from the Lebesgue measure algebra to the Borel
$\sigma$-algebra.   We need to check that $\undtheta E^{\ssbullet}$ is indeed
always a Borel set;  this is because

\Centerline{$\undtheta E^{\ssbullet}=\intstar E
=\{x:\lim_{n\to\infty}\Bover{\mu(E\cap B(x,2^{-n}))}{\mu B(x,2^{-n})}=1\}$}

\noindent and the functions $x\mapsto\mu(E\cap B(x,2^{-n}))$ are all
continuous (use 443B).
}%end of proof of 535G

\leader{535H}{}\cmmnt{ Again using the continuum hypothesis, we have
some results on `strong' liftings, as described in \S453.

\medskip

\noindent}{\bf Theorem} Let
$(X,\frak T,\Sigma,\mu)$ be an effectively
locally finite $\tau$-additive topological measure space with measure
algebra $\frak A$.   If $\#(\frak A)\le\add\mu$ and
$\mu$ is strictly positive, then $\mu$ has a strong lifting.

\wheader{535H}{0}{0}{0}{36pt}

\proof{{\bf (a)} For each $a\in\frak A$, set

\Centerline{$\overline{a}=\bigcap\{F:F\subseteq X$ is closed,
$F^{\ssbullet}\supseteq a\}$.}

\noindent Then $\overline{a}$ is closed and
$\overline{a}^{\ssbullet}\Bsupseteq a$ (414Ac).   If $a$, $b\in\frak A$,
then $\overline{a\Bcup b}=\overline{a}\cup\overline{b}$.   \Prf\ Of
course $\overline{a\Bcup b}\supseteq\overline{a}\cup\overline{b}$,
because the operation $\overline{\phantom{a}}$ is order-preserving.   On
the other hand, $\overline{a}\cup\overline{b}$ is a closed set and
$(\overline{a}\cup\overline{b})^{\ssbullet}\Bsupseteq a\Bcup b$, so
$\overline{a}\cup\overline{b}\supseteq\overline{a\Bcup b}$.\ \Qed

For a subalgebra $\frak B$ of $\frak A$, say that a function
$\theta:\frak B\to\Sigma$ is `potentially a strong lifting' if it is a
Boolean homomorphism and $(\theta b)^{\ssbullet}=b$ and
$\theta b\subseteq\overline{b}$ for every $b\in\frak B$.

\medskip

{\bf (b)} (The key.)  Suppose that $\frak B$ is a subalgebra of
$\frak A$, with cardinal less than $\add\mu$, and $c\in\frak A$;  let
$\frak B_1$ be the subalgebra of $\frak A$ generated by
$\frak B\cup\{c\}$.   If $\theta:\frak B\to\Sigma$ is potentially a
strong lifting, then it has an extension $\theta_1:\frak B_1\to\Sigma$
which is also potentially a strong lifting.

\Prf\ Set

\Centerline{$C_0=\bigcup\{\theta a:a\in\frak B,\,a\Bsubseteq c\}$,}

\Centerline{$D_0=\bigcap\{\theta b:b\in\frak B,\,c\Bsubseteq b\}$,}

\Centerline{$C_1
=\bigcup\{\theta a\setminus\overline{a\Bsetminus c}:a\in\frak B\}$,}

\Centerline{$D_1
=\bigcap\{(X\setminus\theta b)\cup\overline{b\Bcap c}:b\in\frak B\}$.}

\noindent Fix $E_0\in\Sigma$ such that $E_0^{\ssbullet}=c$.

If $a$, $a'$, $b$, $b'\in\frak B$ and $a'\Bsubseteq c\Bsubseteq b'$,
then

\Centerline{$a'\Bsubseteq b'$, so $\theta a'\subseteq\theta b'$;}

\Centerline{$\theta a'\Bcap\theta b=\theta(a'\Bcap b)
\subseteq\overline{a'\Bcap b}\subseteq\overline{b\Bcap c}$,
so
$\theta a'\subseteq(X\setminus\theta b)\cup\overline{b\Bcap c}$;}

\Centerline{$\theta a\setminus\theta b'=\theta(a\Bsetminus b')
\subseteq\overline{a\Bsetminus b'}\subseteq\overline{a\Bsetminus c}$,
so
$\theta a\setminus\overline{a\Bsetminus c}\subseteq\theta b'$;}

\Centerline{$\theta a\cap\theta b
=\theta(a\Bcap b)\subseteq\overline{a\Bcap b}
=\overline{a\Bcap b\Bcap c}\cup\overline{a\Bcap b\Bsetminus c}
\subseteq\overline{a\Bsetminus c}\cup\overline{b\Bcap c}$,}

\noindent so

\Centerline{$\theta a\setminus\overline{a\Bsetminus c}
\subseteq(X\setminus\theta b)\cup\overline{b\Bcap c}$.}

\noindent This shows that $C_0\cup C_1\subseteq D_0\cap D_1$.   At the
same time,

\Centerline{$E_0^{\ssbullet}=c\Bsupseteq a'$, so
$\theta a'\setminus E_0$ is negligible;}

\Centerline{$E_0^{\ssbullet}=c\Bsubseteq b'$, so $E_0\setminus\theta b'$
is negligible;}

\Centerline{$(E_0\cup\overline{a\Bsetminus c})^{\ssbullet}
\Bsupseteq c\Bcup(a\Bsetminus c)\Bsupseteq a=(\theta a)^{\ssbullet}$}

\noindent so
$(\theta a\setminus\overline{a\Bcap c})\setminus E_0$ is negligible;

\Centerline{$E_0^{\ssbullet}=c\Bsubseteq(1\Bsetminus b)\Bcup(b\Bcap c)
\Bsubseteq(X\Bsetminus\theta b)^{\ssbullet}
  \Bcup\overline{b\Bcap c}^{\ssbullet}$,}

\noindent so $E_0\setminus((X\setminus\theta b)\cup\overline{b\Bcap c})$
is negligible.   Because $\#(\frak B)<\add\mu$,
$(C_0\cup C_1)\setminus E_0$ and $E_0\setminus(D_0\cap D_1)$ are
measurable and negligible.

If we set

\Centerline{$E=(E_0\cup C_0\cup C_1)\cap(D_0\cap D_1)$,}

\noindent then $E\in\Sigma$, $E^{\ssbullet}=c$ and
$C_0\cup C_1\subseteq E_0\subseteq D_0\cap D_1$.   So we can set
$\theta_1c=E$ to define a homomorphism from $\frak B_1$ to $\Sigma$
(312O again), and we shall have $(\theta_1d)^{\ssbullet}=d$ for every
$d\in\frak B_1$.

We must check that $\theta_1d\subseteq\overline{d}$ for every
$d\in\frak B_1$.   Now $d$ is expressible as
$(b\Bcap c)\Bcup(a\Bsetminus c)$ for some $a$, $b\in\frak B$, and in this
case

\Centerline{$\theta b\cap E
\subseteq\theta b\cap((X\setminus\theta b)\cup\overline{b\Bcap c})
\subseteq\overline{b\Bcap c}$,}

\Centerline{$\theta a\setminus E
\subseteq\theta a\setminus(\theta a\setminus\overline{a\Bsetminus c})
\subseteq\overline{a\Bsetminus c}$,}

\noindent so

\Centerline{$\theta_1d=(\theta b\cap E)\cup(\theta a\setminus E)
\subseteq\overline{b\Bcap c}\cup\overline{a\Bsetminus c}
=\overline{d}$.}

\noindent So $\theta_1$ is a potential strong lifting, as required.\
\Qed

\wheader{535H}{4}{2}{2}{36pt}

{\bf (c)} Enumerate $\frak A$ as $\family{\xi}{\kappa}{a_{\xi}}$ where
$\kappa\le\add\mu$, and for $\alpha\le\kappa$ let $\frak B_{\alpha}$ be
the subalgebra of $\frak A$ generated by $\{a_{\xi}:\xi<\alpha\}$.
Then (b) tells us that we can choose inductively a family
$\ofamily{\alpha}{\kappa}{\theta_{\alpha}}$ such that
$\theta_{\alpha}:\frak B_{\alpha}\to\Sigma$ is a potential strong
lifting and $\theta_{\alpha+1}$ extends $\theta_{\alpha}$ for each
$\alpha<\kappa$.   (At non-zero limit ordinals $\alpha$,
$\frak B_{\alpha}=\bigcup_{\xi<\alpha}\frak B_{\xi}$ so we can take
$\theta_{\alpha}$ to be the common extension of
$\bigcup_{\xi<\alpha}\theta_{\xi}$.   We need to know that $\mu$ is
strictly positive in order to be sure that $\overline{1}=X$, so that we
can take $\theta_01=X$.)   In this way we obtain a lifting
$\theta=\theta_{\kappa}$ of $\mu$.   Also
$\theta a\subseteq\overline{a}$ for every $a\in\frak A$.   Looking at
this from the other side, if $F\subseteq X$ is closed then
$\overline{F^{\ssbullet}}\subseteq F$ so
$\theta(F^{\ssbullet})\subseteq F$, and $\theta$ is a strong lifting.
}%end of proof of 535H

\leader{535I}{Corollary}\cmmnt{ (see {\smc Mokobodzki 75})}
%Theorem 5
Suppose that $\frak c=\omega_1$.   Let $(X,\frak T,\Sigma,\mu)$ be a
strictly positive $\sigma$-finite quasi-Radon measure space with Maharam
type at most $\omega_1=\frak c$.   Then $\mu$ has a strong Borel
lifting.

%any hope of strong Baire lifting?  or of getting to $\omega_2$?

\proof{ Because $\mu$ is $\sigma$-finite, its measure algebra $\frak A$
is ccc, and has size at most $\frak c^{\omega}=\omega_1$;  so we can
apply 535H to $\mu\restr\Cal B(X)$.
}%end of proof of 535I

\leader{535J}{}\cmmnt{ Under certain conditions, we can deduce the
existence of a strong lifting from the existence of a lifting.   The
basic case is the following.

\medskip

\noindent}{\bf Lemma} Let $(X,\frak T,\Sigma,\mu)$ be a completely
regular totally finite topological measure space with a Borel lifting
$\phi$.   Suppose that $K\subseteq X$ is a self-supporting set of
non-zero measure, homeomorphic to $\{0,1\}^{\Bbb N}$, such that
$K\cap G\subseteq\phi G$
for every open set $G\subseteq X$.   Then the subspace measure $\mu_K$
has a strong Borel lifting.

\proof{{\bf (a)} Taking $\Cal E$ to be the algebra of
relatively open-and-closed subsets of $K$, we have a Boolean
homomorphism $\psi_0:\Cal E\to\Cal B(X)$ such that
$E\subseteq\interior\psi_0E$ for every $E\in\Cal E$.   \Prf\ We have a
Boolean-independent sequence $\sequencen{E_n}$ in $\Cal E$ which generates
$\Cal E$ and separates the points of $K$.   Because every member
of $\Cal E$ is compact, we can choose for each $n\in\Bbb N$ an open
$H_n\subseteq X$ such that $E_n=K\cap H_n=K\cap\overline{H}_n$.
Define $h:X\to K$ by saying that, for every $n\in\Bbb N$
and $x\in X$, $h(x)\in E_n$ iff $x\in H_n$.
Define $\psi_0:\Cal E\to\Cal B(X)$ by setting $\psi_0E=h^{-1}[E]$ for
$E\in\Cal E$.   Then $\psi_0$ is a Boolean homomorphism.   The set

\Centerline{$\{E:E\in\Cal E$, $E\subseteq\interior\psi_0E$,
$K\setminus E\subseteq\interior\psi_0(K\setminus E)\}$}

\noindent is a subalgebra of $\Cal E$ containing every $E_n$, so is the
whole of $\Cal E$, and $\psi_0$ has the required property.\ \Qed

\medskip

{\bf (b)} Let $\frak A$ be the measure algebra of $\mu$, and
$\theta:\frak A\to\Cal B(X)$ the lifting corresponding to $\phi$.   Set
$\psi_1E=(\psi_0E)^{\ssbullet}$ for $E\in\Cal E$, so that
$\psi_1:\Cal E\to\frak A$ is a Boolean homomorphism.   Let
$\Cal I$ be the null ideal of $\mu_K$.   Because $K$ is
self-supporting, $\Cal E\cap\Cal I=\{\emptyset\}$.   Taking
$\Cal E'=\{E\symmdiff F:E\in\Cal E$, $F\in\Cal I\}$, $\Cal E'$ is a
subalgebra of $\Cal PK$, and we have a Boolean homomorphism
$\psi':\Cal E'\to\Cal E$ defined by setting $\psi'(E\symmdiff F)=E$
whenever $E\in\Cal E$ and $F\in\Cal I$;  set $\psi_1'=\psi_1\psi'$, so
that $\psi_1':\Cal E'\to\frak A$ is a Boolean homomorphism extending
$\psi_1$, and $\psi_1'F=0$ whenever $F\in\Cal I$.   Because $\mu$ is
totally finite, $\frak A$ is Dedekind complete, and there is a Boolean
homomorphism $\tilde\psi_1:\Cal PK\to\frak A$ extending $\psi_1'$
(314K).   Now set

\Centerline{$\phi_1E
=K\cap(\phi E\cup(\theta\tilde\psi_1E\setminus\phi K))$}

\noindent for every measurable $E\subseteq K$.   Then $\phi_1$ is a
strong lifting for $\mu_K$.    \Prf\ $\phi\restr\Sigma_K$
is a Boolean homomorphism from the domain $\Sigma_K$ of $\mu_K$ to
$\Cal B(\phi K)$, while $E\mapsto\theta\tilde\psi_1E\setminus\phi K$ is
a Boolean homomorphism from $\Sigma_K$ to $\Cal B(X\setminus\phi K)$;
putting these together, $\phi_1$ is a Boolean homomorphism from
$\Sigma_K$ to $\Cal B(K)$.   If $E\in\Sigma_K$, then
$E\symmdiff(K\cap\phi E)$ and
$K\setminus\phi K$ are negligible, so
$E\symmdiff\phi_1E$ is negligible.   If $E\in\Sigma_K$ is negligible,
then $\phi E=\emptyset$, $\psi_1'E=0$ and $\phi_1E$ is empty.   
Thus $\phi_1$ is a lifting for $\mu_K$.   Morover, if $E\in\Cal E$, set
$G=\interior\psi_0E$, so that $E=K\cap G$.   In this case,

\Centerline{$E\subseteq\phi G=\theta G^{\ssbullet}
\subseteq\theta(\psi_0E)^{\ssbullet}=\theta\psi_1E
=\theta\tilde\psi_1E$,}

\noindent while

\Centerline{$E\cap\phi K\subseteq\phi G\cap\phi K=\phi E$;}

\noindent so $E\subseteq\phi_1E$.   So if $V\subseteq K$ is relatively
open,

\Centerline{$V=\bigcup\{E:E\in\Cal E$, $E\subseteq V\}
\subseteq\bigcup\{\phi_1E:E\in\Cal E$, $E\subseteq V\}
\subseteq\phi_1V$.}

\noindent Thus $\phi_1$ is strong.\ \Qed
}%end of proof of 535J

\leader{535K}{Lemma} Let $X$ be a metrizable space, $\mu$ an atomless
Radon measure on $X$ and $\nu$ an atomless strictly positive Radon
measure on $\{0,1\}^{\Bbb N}$.   Let $\Cal K$ be the family of those
subsets $K$ of $X$ such that $K$, with the subspace topology and
measure, is isomorphic to $\{0,1\}^{\Bbb N}$ with its usual topology and
a scalar multiple of $\nu$.   Then $\mu$ is inner regular with respect
to $\Cal K$.

\proof{{\bf (a)} It will be helpful to note that if $E\in\dom\mu$ and
$\gamma<\mu E$ there is a compact set $K\subseteq E$ such that
$\mu K=\gamma$.   \Prf\ Let $\sequencen{\gamma_n}$ be a strictly
decreasing sequence with $\gamma_0<\mu E$ and $\inf_{n\in\Bbb
N}\gamma_n=\gamma$.   Choose $\sequencen{K_n}$, $\sequencen{E_n}$
inductively as follows.   $E_0=E$.   Given that $\mu E_n>\gamma_n$, let
$K_n\subseteq E_n$ be a compact set such that $\mu K_n\ge\gamma_n$;  now
let $E_{n+1}$ be a measurable set with measure $\gamma_n$ (215D, because
$\mu$ is atomless).   At the end of the induction, set
$K=\bigcap_{n\in\Bbb N}K_n$.\ \Qed

\medskip

{\bf (b)} Now for the main argument.   Suppose that $0\le\gamma<\mu E$.
Let $\sequencen{\gamma_n}$ be a strictly decreasing sequence with
$\gamma_0<\mu E$ and $\inf_{n\in\Bbb N}\gamma_n=\gamma$.   Set
$\gamma'_n=\bover12(\gamma_n+\gamma_{n+1})$ for each $n$.   For
$\sigma\in\bigcup_{n\in\Bbb N}\{0,1\}^n$, set
$I_{\sigma}=\{z:\sigma\subseteq z\in\{0,1\}^{\Bbb N}\}$.
Let $K_0$ be a compact subset of $E$ of measure $\gamma_0$;  because $X$
is metrizable, $K_0$ is second-countable; let $\sequencen{V_n}$ run over
a base for the topology of $K_0$.   Choose $\sequencen{m(n)}$ and
$L_{\sigma}$, for $\sigma\in\{0,1\}^{m(n)}$, as follows.   Start with
$m(0)=0$ and $L_{\emptyset}=K_0$.   Given that
$\family{\sigma}{\{0,1\}^{m(n)}}{L_{\sigma}}$ is a disjoint family of
compact subsets of $X$ with $\mu L_{\sigma}=\gamma_n\nu I_{\sigma}$ for
every $\sigma\in\{0,1\}^{m(n)}$, let $m(n+1)>m(n)$ be so large that
$\gamma_{n+1}\nu I_{\tau}<(\gamma_n-\gamma_{n+1})\nu I_{\sigma}$
whenever $\sigma\in\{0,1\}^{m(n)}$ and $\tau\in\{0,1\}^{m(n+1)}$.
(This is where we need to know that $\nu$ is atomless and strictly
positive.)    Now, for each $\sigma\in\{0,1\}^{m(n)}$, enumerate
$\{\tau:\sigma\subseteq\tau\in\{0,1\}^{m(n+1)}\}$ as
$\ofamily{i}{2^{m(n+1)-m(n)}}{\tau(\sigma,i)}$.   Choose inductively
disjoint compact sets $L_{\tau(\sigma,i)}\subseteq L_{\sigma}$, for
$i<2^{m(n+1)-m(n)}$, in such a way that
$\mu L_{\tau(\sigma,i)}=\gamma_{n+1}\nu I_{\tau(\sigma,i)}$ and
$L_{\tau(\sigma,i)}$ is always either included in $V_n$ or disjoint from
it;  this will be possible because when we come to choose
$L_{\tau(\sigma,i)}$, the measure of the set
$F=L_{\sigma}\setminus\bigcup_{j<i}L_{\tau(\sigma,j)}$ available will be

$$\eqalign{\gamma_n\nu I_{\sigma}
   -\sum_{j<i}\gamma_{n+1}\nu I_{\tau(\sigma,j)}
&\ge(\gamma_n-\gamma_{n+1})\nu I_{\sigma}
  +\gamma_{n+1}\nu I_{\tau(\sigma,i)}\cr
&>2\gamma_{n+1}\nu I_{\tau(\sigma,i)},\cr}$$

\noindent so at least one of $F\cap V_n$, $F\setminus V_n$ will be of
measure greater than $\gamma_{n+1}\nu I_{\tau(\sigma,i)}$.
Continue.

Set $K_n=\bigcup\{L_{\sigma}:\sigma\in\{0,1\}^{m(n)}\}$ for each
$n\in\Bbb N$, and $K=\bigcap_{n\in\Bbb N}K_n$.
The construction ensures that whenever $n\le k$,
$\sigma\in\{0,1\}^{m(n)}$, $\tau\in\{0,1\}^{m(k)}$ and
$\sigma\subseteq\tau$, then $L_{\tau}\subseteq L_{\sigma}$.   We
therefore have a function $f:K\to\{0,1\}^{\Bbb N}$ defined by saying
that $f(x)\restr m(n)=\sigma$ whenever $n\in\Bbb N$,
$\sigma\in\{0,1\}^{m(n)}$ and $x\in K\cap L_{\sigma}$.   Because all the
$L_{\sigma}$ are compact, $f$ is continuous.   But it is also
injective.   \Prf\ If $x$, $y\in K$ are different, there is an
$n\in\Bbb N$ such that $x\in V_n$ and $y\notin V_n$;  now 
$f(x)\restr m(n+1)\ne f(y)\restr m(n+1)$.\ \Qed

For any $n\in\Bbb N$, $\sigma\in\{0,1\}^{m(n)}$ and $k\ge n$,

\Centerline{$\mu(\bigcup\{L_{\tau}:
  \sigma\subseteq\tau\in\{0,1\}^{m(k)}\})
=\sum_{\sigma\subseteq\tau\in\{0,1\}^{m(k)}}\gamma_k\nu I_{\tau}
=\gamma_k\nu I_{\sigma}$.}

\noindent So

\Centerline{$\mu(f^{-1}[I_{\sigma}])
=\inf_{k\ge n}\gamma_k\nu I_{\sigma}=\gamma\nu I_{\sigma}$.}

\noindent Thus the Radon measure $\mu f^{-1}$ on $\{0,1\}^{\Bbb N}$
agrees with the Radon measure $\gamma\nu$ on
$\{I_{\sigma}:\sigma\in\bigcup_{n\in\Bbb N}\{0,1\}^{m(n)}\}$;  as this
is a base for the topology of $\{0,1\}^{\Bbb N}$ closed under finite
intersections, $\mu f^{-1}$ and $\gamma\nu$ are identical (415H(v)).
Once again because $\nu$ is strictly positive, $f$ is surjective and is
a homeomorphism.   So $f$ witnesses that $K\in\Cal K$.
As $E$ and $\gamma$ are arbitrary, $\mu$ is inner regular with respect
to $\Cal K$.
}%end of proof of 535K

\leader{535L}{Lemma} (a) If $(X,\frak T)$ is a separable metrizable
space, there is a zero-dimensional separable metrizable topology
$\frak S$ on $X$, finer than $\frak T$, with the same Borel sets as
$\frak T$, such that $\frak T$ is a $\pi$-base for $\frak S$.

(b) If $X$ is a non-empty zero-dimensional separable metrizable space
without isolated points, it is homeomorphic to a dense subset of
$\{0,1\}^{\Bbb N}$.

(c) Any completely regular space of size less than $\frak c$
is zero-dimensional.

\proof{{\bf (a)} Enumerate a countable base for $\frak T$ as
$\sequencen{U_n}$.   Define a sequence $\sequencen{\frak S_n}$ of
topologies on $X$ by saying that $\frak S_0=\frak T$ and that
$\frak S_{n+1}$ is the topology on $X$ generated by
$\frak S_n\cup\{V_n\}$, where $V_n$ is the closure of $U_n$ for
$\frak S_n$.   Inducing on $n$, we see that $\frak S_n$ is
second-countable and has the same Borel sets as $\frak T$, for every
$n$.   So taking $\frak S$ to be the topology generated by
$\bigcup_{n\in\Bbb N}\frak S_n$ (that is, the topology generated by
$\{U_n:n\in\Bbb N\}\cup\{V_n:n\in\Bbb N\}$), this also is
second-countable and has the same Borel sets as $\frak T$.   Each $V_n$
is open for $\frak S_{n+1}$ and closed for $\frak S_n$, so is
open-and-closed for $\frak S$.   Moreover, since

\Centerline{$U_n
=\bigcup\{U_m:m\in\Bbb N$, $\overline{U}^{\frak T}_m\subseteq U_n\}
=\bigcup\{V_m:m\in\Bbb N$, $V_m\subseteq U_n\}$}

\noindent for each $n$, $\{V_n:n\in\Bbb N\}$ is a base for $\frak S$
consisting of open-and-closed sets for $\frak S$, and $\frak S$ is
zero-dimensional.   Finally, observe that if $V_n$ is not empty, then
$V_n\supseteq U_n\ne\emptyset$, so $\frak T\supseteq\{U_n:n\in\Bbb N\}$
is a $\pi$-base for $\frak S$.

\medskip

{\bf (b)} The family $\Cal E_0$ of open-and-closed subsets of $X$ is a
base for the topology of $X$, so includes a countable base $\Cal U$
(4A2P(a-iii)).   Because $X$ has no isolated points, the subalgebra
$\Cal E_1$ of $\Cal E_0$ generated by $\Cal U$ is countable, atomless
and non-trivial, and must be isomorphic to the algebra $\Cal E$ of
open-and-closed subsets of $\{0,1\}^{\Bbb N}$
(316M).   Let
$\pi:\Cal E\to\Cal E_1$ be an isomorphism.   Then we have a function
$f:X\to\{0,1\}^{\Bbb N}$ defined by saying that, for $E\in\Cal E$,
$f(x)\in E$ iff $x\in\pi E$.   Because $\pi E\ne\emptyset$ for every
non-empty $E\in\Cal E$, $f[X]$ is dense in $\{0,1\}^{\Bbb N}$.
Because $\{f^{-1}[E]:E\in\Cal E\}=\Cal E_1\supseteq\Cal U$ is a base
for the topology of $X$, $f$ is a homeomorphism between $X$ and
$f[X]$.

\medskip

{\bf (c)} If $X$ is a completely regular space and $\#(X)<\frak c$,
$G\subseteq X$ is open and $x\in G$, let $f:X\to[0,1]$ be a continuous
function such that $f(x)=1$ and $f(y)=0$ for $y\in X\setminus G$.
Because $\#(X)<\frak c$, there is an $\alpha\in[0,1]\setminus f[X]$, and
now $\{y:f(x)>\alpha\}=\{y:f(x)\ge\alpha\}$ is an open-and-closed set
containing $x$ and included in $G$.   As $x$ and $G$ are arbitrary, $X$
is zero-dimensional.
}%end of proof of 535L

\leader{535M}{Lemma} Suppose that there is a Borel probability measure
on $\{0,1\}^{\Bbb N}$ with a strong lifting.   Then whenever $X$ is a
separable metrizable space and $D\subseteq X$ is a dense set, there is a
Boolean homomorphism $\phi$ from $\Cal PD$ to the Borel $\sigma$-algebra
$\Cal B(X)$ of $X$ such that
$\phi A\subseteq\overline{A}$ for every $A\subseteq D$.

\proof{{\bf case 1} Suppose that $X$ is countable.   Then it is
zero-dimensional (535Lc), so has a base $\Cal U$ consisting of
open-and-closed sets;  let $\Cal E$ be the algebra of sets generated by
$\Cal U$.   For $E\in\Cal E$ set $\pi E=E\cap D$;   then $\pi$ is an
isomorphism between $\Cal E$ and a subalgebra $\Cal E'$ of $\Cal PD$.
Because $\Cal B(X)=\Cal PX$ is Dedekind complete, the Boolean homomorphism
$\pi^{-1}:\Cal E'\to\Cal E$ extends to a Boolean homomorphism
$\phi:\Cal PD\to\Cal PX=\Cal B(X)$ (314K again).   If $A\subseteq D$ and
$x\in X\setminus\overline{A}$, then there is a $U\in\Cal U$ such that
$x\in U$ and $A\cap U=\emptyset$, in which case

\Centerline{$\phi A\subseteq\pi^{-1}(D\setminus U)=X\setminus U$}

\noindent does not contain $x$.   As $x$ is arbitrary,
$\phi A\subseteq\overline{A}$;  as $A$ is arbitrary, $\phi$ has the
required property.

\medskip

{\bf case 2} Suppose that $X$ is zero-dimensional and has no isolated
points.   If $X$ is empty the result is trivial;  otherwise, by 535Lb,
we may suppose that $X$ is a dense subset of $\{0,1\}^{\Bbb N}$.   This
time, let $\Cal E$ be the algebra of open-and-closed subsets of
$\{0,1\}^{\Bbb N}$.   For $E\in\Cal E$, set
$\pi E=E\cap D$.   Because $D$ is dense in $X$ and therefore in
$\{0,1\}^{\Bbb N}$, $\pi$ is an isomorphism between $\Cal E$ and a
subalgebra $\Cal E'$ of $\Cal PD$.    Fix a Borel probability measure
$\mu$ on $\{0,1\}^{\Bbb N}$ with a strong lifting $\theta$,
and let $\frak A$ be the measure algebra of
$\mu$.   Then $A\mapsto(\pi^{-1}A)^{\ssbullet}$ is a Boolean
homomorphism from $\Cal E'$ to $\frak A$;  because $\frak A$ is Dedekind
complete, it extends to a Boolean homomorphism $\psi:\Cal PD\to\frak A$.
For $E\subseteq\{0,1\}^{\Bbb N}$, set $\tilde\pi E=E\cap X$.
Then $\phi=\tilde\pi\theta\psi$ is a Boolean homomorphism from
$\Cal PD$ to $\Cal B(X)$.   If $A\subseteq D$ and
$x\in X\setminus\overline{A}$, then there is an $E\in\Cal E$ such that
$x\in E$ and $A\cap E=\emptyset$, in which case

\Centerline{$\phi A\subseteq\theta\psi A
\subseteq\theta\psi(D\setminus E)
=\theta(\{0,1\}^{\Bbb N}\setminus E)^{\ssbullet}
=\{0,1\}^{\Bbb N}\setminus E$,}

\noindent and $x\notin\phi A$.   As $x$ and $A$ are arbitrary, $\phi$ is
a suitable homomorphism.

\medskip

{\bf case 3} Suppose that $X$ has no isolated points.   Write $\frak T$
for the given topology on $X$.   By 535La, there is a finer
zero-dimensional separable metrizable topology $\frak S$ on $X$, with
the same Borel sets, such that $\frak T$ is a $\pi$-base for $\frak S$.
If $V\in\frak S$ is non-empty, there is a non-empty $U\in\frak T$ such
that $U\subseteq V$, and $D\cap V\supseteq D\cap U$ is non-empty;  so
$D$ is $\frak S$-dense.   By case 2, there is a Boolean homomorphism
$\phi:\Cal PD\to\Cal B(X,\frak S)$ such that
$\phi A\subseteq\overline{A}^{\frak S}$ for every $A\subseteq D$.   As
$\Cal B(X,\frak S)=\Cal B(X,\frak T)$, and
$\overline{A}^{\frak S}\subseteq\overline{A}^{\frak T}$ for every
$A\subseteq X$, this $\phi$ satisfies the conditions required.

\medskip

{\bf general case} In general, let $\Cal G$ be the family of countable
open subsets of $X$, and $G_0=\bigcup\Cal G$;  because $X$ is separable
and metrizable, therefore hereditarily Lindel\"of, $G_0$ is countable.
Set $Z=X\setminus G_0$, and let $D_0$ be a countable dense subset of
$Z$;  set $Y=D\cup G_0\cup D_0$.   By case 1, there is a Boolean
homomorphism $\phi_0:\Cal PD\to\Cal PY$ such that
$\phi_0A\subseteq\overline{A}$ for every $A\subseteq D$.   By case 3,
there is a Boolean homomorphism $\phi_1:\Cal P(Y\cap Z)\to\Cal B(Z)$
such that $\phi_1B\subseteq\overline{B}$ for every $B\subseteq Y\cap Z$.
Now set

\Centerline{$\phi A=(\phi_0A\setminus Z)\cup\phi_1(Z\cap\phi_0A)$}

\noindent for every $A\subseteq D$.   Then $\phi$ is a Boolean
homomorphism from $\Cal PD$ to $\Cal B(X)$;  and if
$A\subseteq D$, then

\Centerline{$\phi A
\subseteq\phi_0A\cup\phi_1(Z\cap\phi_0A)
\subseteq\overline{A}\cup\overline{Z\cap\overline{A}}
=\overline{A}$,}

\noindent so in this case also we have a homomorphism of the kind we
need.
}%end of proof of 535M

\leader{535N}{Theorem} Suppose there is a metrizable space $X$ with
a non-zero atomless semi-finite
tight Borel measure $\mu$ which has a lifting.
Then whenever $Y$ is a metrizable space and $\nu$ is a strictly positive
$\sigma$-finite Borel measure on $Y$, $\nu$ has a strong lifting.

\proof{{\bf (a)} Let $\phi$ be a lifting for $\mu$.   Then there is
a Borel set $E\subseteq X$, of non-zero finite measure, such that
$E\cap G\subseteq\phi G$ for every open $G\subseteq X$.
\Prf\ Let $L_0\subseteq X$ be a compact set of
non-zero measure;  then $L_0$ has a countable base $\Cal U$;  set
$E=L_0\cap\phi L_0\setminus\bigcup_{U\in\Cal U}(U\symmdiff\phi U)$,
so that $\mu E=\mu L_0\in\ooint{0,\infty}$.   If $G\subseteq X$ is open
and $x\in E\cap G$, then there is a
$U\in\Cal U$ such that $x\in U\subseteq G$.   Since
$x\in E\cap U$, $x\in\phi U\subseteq\phi G$.   As $x$ and $G$ are
arbitrary, we have an appropriate $E$.\ \Qed

\medskip

{\bf (b)} Let $\lambda$ be any strictly positive atomless Radon measure
on $\{0,1\}^{\Bbb N}$.   There is a compact set $K\subseteq E$ such that
$K$, with its induced topology and measure, is isomorphic to
$\{0,1\}^{\Bbb N}$ with its usual topology and a non-zero multiple of
$\lambda$, by 535K.   In particular, $K$ is self-supporting.   By 535J,
the subspace measure on $K$ has a strong Borel lifting.   It follows at
once that $\lambda$ has a strong Borel lifting.

\medskip

{\bf (c)} Refining (b) slightly, we see that if
$Y\subseteq\{0,1\}^{\Bbb N}$ is a dense set and $\lambda$ is a strictly
positive atomless totally finite Borel measure on $Y$, then $\lambda$
has a strong lifting.   \Prf\ There is a Radon measure $\nu$ on
$\{0,1\}^{\Bbb N}$ such that $\nu E=\lambda(Y\cap E)$ for every Borel
set $E\subseteq\{0,1\}^{\Bbb N}$ (416F);  
because $\lambda$ is atomless, so is
$\nu$;  because $\lambda$ is strictly positive and $Y$ is dense, $\nu$
is strictly positive.   So $\nu$ has a strong Borel lifting $\psi_0$
say.   If $E$, $F\in\Cal B(\{0,1\}^{\Bbb N})$ and $E\cap Y=F\cap Y$,
then $\nu(E\symmdiff F)=0$ and $\psi_0E=\psi_0F$;  we therefore have a
Boolean homomorphism $\psi:\Cal B(Y)\to\Cal B(Y)$ defined by setting
$\psi(E\cap Y)=Y\cap\psi_0E$ for every Borel set
$E\subseteq\{0,1\}^{\Bbb N}$.   It is easy to check that $\psi$ is a
lifting for $\lambda$, and it is strong because if
$G\subseteq\{0,1\}^{\Bbb N}$ is open then
$\psi(Y\cap G)=Y\cap\psi_0G\subseteq Y\cap G$.\ \Qed

\medskip

{\bf (d)} If $(Y,\frak S)$ is a separable metrizable space with a
strictly positive atomless totally finite Borel measure $\nu$, then $\nu$
has a strong lifting.   \Prf\ If $Y=\emptyset$ the
result is trivial.   Otherwise, by 535La, there is a finer separable
metrizable topology $\frak S'$ on $Y$ with the same Borel sets such that
$\frak S$ is a $\pi$-base for $\frak S'$.   Because $\frak S$ and
$\frak S'$ have the same Borel sets, $\nu$ is a Borel measure for
$\frak S'$;  because every non-empty $\frak S'$-open set includes a
non-empty $\frak S$-open set, $\nu$ is strictly positive for $\frak S'$;
because $\nu$ is atomless, $Y$ has no $\frak S'$-isolated points.
By 535Lb, $(Y,\frak S')$ is homeomorphic to a dense subset of
$\{0,1\}^{\Bbb N}$;  by (c) above, $\nu$ has a lifting $\phi$ which is
strong with respect to the topology $\frak S'$.    But now $\phi$ is
still strong with respect to the coarser topology $\frak S$.\ \Qed

\medskip

{\bf (e)} Now suppose that $Y$ is a separable metrizable space with a
strictly positive totally finite Borel measure $\nu$.   Then $\nu$ has a
strong lifting.   \Prf\ The set $D=\{y:\nu\{y\}>0\}$ is countable.   If
$D$ is empty, then the result is immediate from (d) applied to a scalar
multiple of $\nu$.  (If $\nu Y=0$ then $Y=\emptyset$ and the result is
trivial.)    Otherwise, let
$\nu_{Y\setminus D}$ be the subspace measure;  then $\nu_{Y\setminus D}$
is a totally finite Borel measure on $Y\setminus D$, and is zero on
singletons, so must be atomless.   Because $Y\setminus D$ is
hereditarily Lindel\"of, $\nu_{Y\setminus D}$ is $\tau$-additive;  let
$Z$ be its support, and $\nu_Z$ the subspace measure on $Z$.   Then
$\nu_Z$ has a strong Borel lifting $\psi_0$, by (d) again.   Next, $Z$
is relatively closed
in $Y\setminus D$, so is expressible as $F\setminus D$ for some closed
set $F\subseteq Y$.   If $x\in Y\setminus F$ and $G$ is an open set
containing $x$, then $G'=G\setminus F$ is a non-empty open set, so has
non-zero measure, while
$\nu_{Y\setminus D}(G'\setminus D)=0$;  accordingly
$G'\cap D\ne\emptyset$.   This shows that
$Y\setminus F\subseteq\overline{D}$ so $D$ is dense in $Y\setminus Z$.
Now 535M (with (b) above) tells us that there is a Boolean homomorphism
$\psi_1:\Cal PD\to\Cal B(Y\setminus Z)$ such that
$\psi_1A\subseteq\overline{A}$ for every $A\subseteq D$.   Define
$\psi:\Cal B(Y)\to\Cal B(Y)$ by setting

\Centerline{$\psi E
=\psi_0(E\cap Z)\cup(E\cap D)\cup(\psi_1(E\cap D)\setminus D)$}

\noindent for every Borel set $E\subseteq Y$.   $\psi$ is a Boolean
homomorphism because $\psi_0$ and $\psi_1$ are.   If $\nu E=0$, then
$\nu_Z(E\cap Z)=0$ and $E\cap D=\emptyset$, so $\psi E=\emptyset$.   
For any $E\in\Cal B(Y)$, $\psi_0(E\cap Z)\symmdiff(E\cap Z)$ and
$Y\setminus(D\cup Z)$ are negligible, so $E\symmdiff\psi E$ is
negligible.   Thus $\psi$ is a lifting for $\nu$.   Finally, for any $E$,

\Centerline{$\psi E
\subseteq\psi_0(E\cap Z)\cup(E\cap D)\cup\psi_1(E\cap D)
\subseteq\overline{E}$,}

\noindent so $\psi$ is a strong lifting.\ \Qed

\medskip

{\bf (f)} Finally, if $Y$ is a metrizable space and $\nu$ is a strictly
positive $\sigma$-finite Borel measure on $Y$, then $Y$ must be ccc,
therefore separable;  and there is a totally finite Borel measure
$\nuprime$ with the same null ideal as $\nu$, so that $\nuprime$ has a
strong lifting, by (e), which is also a strong lifting for $\nu$.
}%end of proof of 535N

%{\smc Burke 93}, 3.6 gives a version with an incorrect proof

\leader{535O}{Linear liftings} Let $(X,\Sigma,\mu)$ be a measure space,
with measure algebra $\frak A$.   Write
$\eusm L^{\infty}(\Sigma)$ for the space of bounded $\Sigma$-measurable
real-valued functions on $X$.  A {\bf linear lifting} for $\mu$ is

\inset{{\it either} a positive linear operator
$T:L^{\infty}(\mu)\to\eusm L^{\infty}(\Sigma)$ such that
$T(\chi X^{\ssbullet})=\chi X$ and
$(Tu)^{\ssbullet}=u$ for every $u\in L^{\infty}(\mu)$

{\it or} a positive linear operator
$S:\eusm L^{\infty}(\Sigma)\to\eusm L^{\infty}(\Sigma)$ such that
$S(\chi X)=\chi X$, $Sf=0$ whenever $f=0$ a.e.\ and $Sf\eae f$ for every
$f\in\eusm L^{\infty}(\Sigma)$.}

\cmmnt{\noindent As with liftings (see 341A-341B) we have a direct
correspondence between the two kinds of linear operator;  given $T$ as
in the first formulation, we can set $Sf=T(f^{\ssbullet})$ for every
$f\in\eusm L^{\infty}(\Sigma)$;  given $S$ as in the second formulation,
we can set $T(f^{\ssbullet})=Sf$ for every
$f\in\eusm L^{\infty}(\Sigma)$.
}%end of comment

If $\theta:\frak A\to\Sigma$ is a lifting for $\mu$, then we have a
corresponding Riesz homomorphism
$T:L^{\infty}(\frak A)\to\eusm L^{\infty}(\Sigma)$ such that
$T(\chi a)=\chi(\theta a)$ for every $a\in\frak A$\cmmnt{, by 363F}.
Identifying $L^{\infty}(\frak A)$ with $L^{\infty}(\mu)$\cmmnt{ as in
363I}, \cmmnt{we
see that} $T$ can be regarded as a linear lifting.   \cmmnt{Of course
the associated linear operator from
$\eusm L^{\infty}(\Sigma)$ to itself is the operator derived \cmmnt{by
the process of 363F} from the Boolean homomorphism
$E\mapsto\theta E^{\ssbullet}:\Sigma\to\Sigma$.}

\cmmnt{As in 535Aa,} I will say that a {\bf Borel linear lifting} is a
linear lifting such that all its values are Borel measurable functions;
\cmmnt{similarly,}
a {\bf Baire linear lifting} is a linear lifting such that all its
values are Baire measurable functions.

\leader{535P}{}\cmmnt{ I give a sample result to show that for some
purposes linear liftings are adequate.

\medskip

\noindent}{\bf Proposition} Let $(X,\Sigma,\mu)$ be a countably compact
measure space such that $\Sigma$ is countably generated, $(Y,\Tau,\nu)$
a $\sigma$-finite measure space with a linear lifting, and
$f:X\to Y$ an \imp\ function.   Then there is a disintegration
$\family{y}{Y}{\mu_y}$ of $\mu$ over $\nu$, consistent with $f$, such
that $y\mapsto\mu_yE$ is a $\Tau$-measurable function for every
$E\in\Sigma$.

\proof{ I use the method of 452H-452I.

\medskip

{\bf (a)} Suppose first that $\mu$ and $\nu$ are probability measures.
Let $S:L^{\infty}(\nu)\to\eusm L^{\infty}(\Tau)$ be a linear lifting for
$\nu$.   Let $T:L^{\infty}(\mu)\to L^{\infty}(\nu)$ be the positive
linear operator defined by saying that $\int_FTu=\int_{f^{-1}[F]}u$
whenever $u\in L^{\infty}(\mu)$ and $F\in\Tau$ (as in part (a) of the
proof of 452I).   For $y\in Y$ and $E\in\Sigma$, set

\Centerline{$\psi_yE=(ST(\chi E^{\ssbullet}))(y)$}

\noindent as in part (b) of the proof of 452H.   Because $\mu$ is
countably compact, we can use the argument of 452H to see that we have a
family $\family{y}{Y}{\mu'_y}$ of totally finite measures on $X$ such
that, for any $E\in\Sigma$, $\mu'_yE=\psi_yE$ for almost every $y\in Y$.

Let $\Cal H$ be a countable subalgebra of $\Sigma$ such that $\Sigma$ is
the $\sigma$-algebra of sets generated by $\Cal H$.   Set
$Y_0=\{y:\mu'_yH=\psi_yH$ for every $H\in\Cal H\}$, so that $Y_0$ is
conegligible;  let $Y_1\subseteq Y_0$ be a measurable conegligible set;
set $\mu_y=\mu'_y$ for $y\in Y_1$, and take $\mu_y$ to be the zero
measure on $X$ for $y\in Y\setminus Y_1$.   If $H\in\Cal H$, then

\Centerline{$\mu_yH=\psi_yH=ST(\chi H^{\ssbullet})(y)$}

\noindent for every $y\in Y_1$, so $y\mapsto\mu_yH$ is
$\Tau$-measurable;  also, of course,

\Centerline{$\int_F\mu_yH\nu(dy)
=\int_FST(\chi H^{\ssbullet})d\nu
=\int_FT(\chi H^{\ssbullet})
=\int_{f^{-1}[F]}\chi H^{\ssbullet}
=\mu(H\cap f^{-1}[F])$.}

Now consider the family $\Cal E$ of those $E\in\Sigma$ such that
$y\mapsto\mu_yE$ is $\Tau$-measurable and
$\int_F\mu_yE\nu(dy)=\mu(E\cap f^{-1}[F])$ for every $F\in\Tau$.   This
is a Dynkin class including $\Cal H$, so is the whole of
$\Sigma$;  which is what we need to know.

\medskip

{\bf (b)} In general, if $\nu Y=0$, the result is trivial.   Otherwise,
apply (a) to a suitable pair of indefinite-integral measures over $\mu$
and $\nu$, as in part (c) of the proof of 452I.
}%end of proof of 535P

\leader{535Q}{Proposition} Let $(X,\Sigma,\mu)$ and $(Y,\Tau,\nu)$ be
probability spaces, and $\lambda$ the c.l.d.\ product measure on
$X\times Y$.   Suppose that $\lambda\restr\Sigma\tensorhat\Tau$ has a
linear lifting.   Then $\mu$ has a linear lifting.

\proof{ Let $S:\eusm L^{\infty}(\Sigma\tensorhat\Tau)
\to\eusm L^{\infty}(\Sigma\tensorhat\Tau)$ be a linear lifting for
$\lambda\restr\Sigma\tensorhat\Tau$.   For
$h\in\eusm L^{\infty}(\Sigma\tensorhat\Tau)$, set
$(Uh)(x)=\int h(x,y)\nu(dy)$ for every $x\in X$;  by 252P, $Uh$ is
well-defined and is $\Sigma$-measurable.   Now $U$ is a positive linear
operator from $\eusm L^{\infty}(\Sigma\tensorhat\Tau)$ to
$\eusm L^{\infty}(\Sigma)$, and $U(\chi(X\times Y))=\chi X$, because
$\nu Y=1$.   Note that

\Centerline{$\int|Uh|d\mu\le\int U|h|d\mu
=\iint|h(x,y)|\nu(dy)\mu(dx)=\int|h|d\lambda$}

\noindent for every $h\in\eusm L^{\infty}(\Sigma\tensorhat\Tau)$ (252P
again).
Next, for $f\in\eusm L^{\infty}(\Sigma)$ set $(Vf)(x,y)=f(x)$ for every
$x\in X$ and $y\in Y$, so that $V$ is a positive linear operator from
$\eusm L^{\infty}(\Sigma)$ to
$\eusm L^{\infty}(\Sigma\tensorhat\Tau)$.

Consider $S_1=USV:\eusm L^{\infty}(\Sigma)\to\eusm L^{\infty}(\Sigma)$.
This is a positive linear operator and $S_1(\chi X)=\chi X$.   If
$f\in\eusm L^{\infty}(\Sigma)$ and $f=0\,\,\mu$-a.e., then
$Vf=0\,\,\lambda$-a.e.\ and $SVf=0$, so $S_1f=0$.   For any
$f\in\eusm L^{\infty}(\Sigma)$,

\Centerline{$\int|f-S_1f|d\mu=\int|f-USVf|d\mu
=\int|UVf-USVf|d\mu\le\int|Vf-SVf|d\lambda=0$,}

\noindent so $f\eae S_1f$;  thus $S_1$ is a linear lifting for $\mu$.
}%end of proof of 535Q

\leader{535R}{Proposition} Write $\nu_{\omega}^2$ for the usual measure on $(\{0,1\}^{\omega})^2$, and $\Tau_{\omega}^{(2)}$ for its domain.
Suppose that $\nu_{\kappa}$ has a Baire linear lifting for some
$\kappa\ge\frak c^{++}$.   Then there is a Borel linear lifting $S$ for
$\nu_{\omega}^2$ which respects coordinates in the sense that if
$f\in\eusm L^{\infty}(\Tau_{\omega}^{(2)})$ is determined by a single
coordinate, then $Sf$ is determined by the same coordinate.

\proof{ Because $(\{0,1\}^{\kappa},\nu_{\kappa})$ is isomorphic, as
topological measure space, to
$(\{0,1\}^{\kappa\times\omega},\nu_{\kappa\times\omega})$, the latter
has a Baire linear lifting $S_0$ say.   For $I\subseteq\kappa$, let
$\Tau_I$ be the $\sigma$-algebra of Baire subsets of
$\{0,1\}^{\kappa\times\omega}$ determined by coordinates in
$I\times\omega$.   Then $\#(\Tau_I)\le\frak c$ whenever
$\#(I)\le\frak c$.   Also $\CalBa(\{0,1\}^{\kappa\times\omega})
=\bigcup\{\Tau_I:I\in[\kappa]^{\le\omega}\}$ (4A3N).   It follows that
for every $\xi<\kappa$ there is a set $I_{\xi}\subseteq\kappa$, of size
at most $\frak c$, such that $\xi\in I_{\xi}$ and $S_0(\chi E)$ is
$\Tau_{I_{\xi}}$-measurable whenever $E\in\Tau_{I_{\xi}}$;  so that
$S_0f$ is $\Tau_{I_{\xi}}$-measurable whenever
$f:\{0,1\}^{\kappa\times\omega}\to\Bbb R$ is bounded and
$\Tau_{I_{\xi}}$-measurable.

Because $\kappa\ge\frak c^{++}$, there are $\xi$, $\eta<\kappa$ such
that $\xi\notin I_{\eta}$ and $\eta\notin I_{\xi}$ (5A1I(a-iii)).   Set
$J=\{\xi\}\times\omega$, $K=\{\eta\}\times\omega$ and
$L=(\kappa\times\omega)\setminus(J\cup K)$, so that
$\{0,1\}^{\kappa\times\omega}$ can be identified with
$\{0,1\}^{J\cup K}\times\{0,1\}^L$ and
$\CalBa(\{0,1\}^{\kappa\times\omega})$
with $\CalBa(\{0,1\}^{J\cup K})\tensorhat\CalBa(\{0,1\}^L)$.   Set
$(Vf)(w,z)=f(w)$ when $f:\{0,1\}^{J\cup K}\to\Bbb R$ is a function,
$w\in\{0,1\}^{J\cup K}$ and $z\in\{0,1\}^L$;   and
$(Uh)(w)=\int h(w,z)\nu_L(dz)$ when
$h:\{0,1\}^{\kappa\times\omega}\to\Bbb R$ is a bounded Baire measurable
function and $w\in\{0,1\}^{J\cup K}$.
Then $S_1=US_0V$ is a Baire linear lifting for $\nu_{J\cup K}$, just as
in 535Q.   Moreover, if $f:\{0,1\}^{J\cup K}\to\Bbb R$ is a bounded
Baire measurable function determined by coordinates in $J$, in the sense
that $f(x,y)=f(x,y')$ whenever $x\in\{0,1\}^J$ and $y$,
$y'\in\{0,1\}^K$, then $S_1f$ is determined by coordinates in $J$.
\Prf\ $Vf$ is determined by coordinates in $J$, so $S_0Vf$ is determined
by coordinates in $I_{\xi}\times\omega$;  since
$K\cap(I_{\xi}\times\omega)$ is empty, $S_0Vf(x,y,z)=S_0Vf(x,y',z)$ for
all $x\in\{0,1\}^J$, $z\in\{0,1\}^L$ and $y$, $y'\in\{0,1\}^K$.   It
follows at once that

\Centerline{$S_1f(x,y)=\int S_0Vf(x,y,z)\nu_L(dz)
=\int S_0Vf(x,y',z)\nu_L(dz)=S_1f(x,y')$}

\noindent whenever $x\in\{0,1\}^J$ and $y$, $y'\in\{0,1\}^K$.\ \QeD\
Similarly, if
$f:\{0,1\}^{J\cup K}\to\Bbb R$ is a bounded Baire measurable function
determined by coordinates in $K$, then $S_1f$ is determined by
coordinates in $K$.

Now we can
transfer $S_1$ from $\{0,1\}^{J\cup K}\cong\{0,1\}^J\times\{0,1\}^K$ to
$(\{0,1\}^{\omega})^2$, and we shall obtain a Baire (or Borel)
linear lifting $S$ for $\nu_{\omega}^2$ which respects coordinates.
}%end of proof of 535R

\exercises{\leader{535X}{Basic exercises (a)}
%\spheader 535Xa
Let $(X,\Sigma,\mu)$ be a measure space with a lifting,
and $A$ any subset of $X$.   Show that if $A$ has a measurable envelope
then the subspace measure $\mu_A$ has a lifting.   \Hint{322I.}
%-

\spheader 535Xb Let $\familyiI{(X_i,\Sigma_i,\mu_i)}$ be a family of
measure spaces, with $\mu_iX_i>0$ for every $i\in I$, and
$(X,\Sigma,\mu)$ their direct sum.   Show that $\mu$ has a lifting iff
every $\mu_i$ has a lifting.
%-

\spheader 535Xc Let $\frak A$ be a Boolean algebra and $I$ a proper
ideal of $\frak A$.   Suppose that $\sup A$ is defined in $\frak A$ and
belongs to $I$ whenever $A\subseteq I$ and $\#(A)<\#(\frak A)$.
Show that there is a Boolean homomorphism $\theta:\frak A/I\to\frak A$
such that $(\theta b)^{\ssbullet}=b$ for every $b\in\frak A/I$.
\Hint{enumerate $\frak A$ as $\{a_{\xi}:\xi<\kappa\}$;  let
$\frak C_{\xi}$ be the subalgebra of $\frak A/I$ generated by
$\{a_{\eta}^{\ssbullet}:\eta<\xi\}$;  construct
$\theta\restrp\frak C_{\xi}$ inductively by choosing
$\theta a_{\xi}^{\ssbullet}$ appropriately.}
%535D

\spheader 535Xd Let $\frak A$ be a Dedekind $\sigma$-complete Boolean
algebra and $I$ a proper ideal of $\frak A$.   Show that if the quotient
Boolean algebra $\frak A/I$ is tightly $\omega_1$-filtered, then
there is a Boolean homomorphism $\theta:\frak A/I\to\frak A$ such that
$(\theta b)^{\ssbullet}=b$ for every $b\in\frak A/I$.
%535D

\spheader 535Xe Let $\frak A$ be a tightly $\omega_1$-filtered Boolean
algebra, $\frak B$ a Dedekind $\sigma$-complete Boolean
algebra and $\frak A_0$ a countable subalgebra of $\frak A$.   Show that
every Boolean homomorphism from $\frak A_0$ to $\frak B$ extends to a
Boolean homomorphism from $\frak A$ to $\frak B$.
%535D

\spheader 535Xf Let $\frak A$, $\frak B$ be 
Boolean algebras such that $\sup A$ is defined in $\frak A$ whenever
$A\subseteq\frak A$ and $\#(A)<\#(\frak B)$,
and $\pi:\frak A\to\frak B$ a surjective 
Boolean homomorphism.   Suppose that
$\undtheta:\frak B\to\frak A$ is such that $\undtheta 0=0$, 
$\pi\undtheta b\Bsubseteq b$ for every $b\in\frak B$
and $\undtheta(b\Bcap c)=\undtheta b\Bcap\undtheta c$ for all 
$b$, $c\in\frak B$.   Show that there is a Boolean homomorphism
$\theta:\frak B\to\frak A$ such that $\undtheta b\Bsubseteq\theta b$ and
$\pi\theta b=b$ for every $b\in\frak B$.
%535F

\spheader 535Xg Suppose that $\frak c\le\omega_2$ and
$\FN(\Cal P\Bbb N)=\omega_1$.   Show that $\nu_{\kappa}$ has a strong
Baire lifting whenever $\kappa\le\omega_2$.   \Hint{let
$\ofamily{\xi}{\kappa}{e_{\xi}}$ be the standard generating family for
$\frak B_{\kappa}$.   Show that there is a tight $\omega_1$-filtration
$\ofamily{\eta}{\zeta}{a_{\eta}}$ of
$\frak B_{\kappa}$ such that for every $\xi<\kappa$ there is an
$\eta<\zeta$ such that the closed subalgebras generated by
$\{e_{\delta}:\delta<\xi\}$ and $\{a_{\delta}:\delta<\eta\}$ are the
same and $e_{\xi}=a_{\eta}$.}
%535I method of 518M does the job 

\spheader 535Xh Suppose that $\frak c\le\omega_2$ and
$\FN(\Cal P\Bbb N)=\omega_1$.   Show that
whenever $X$ is a separable metrizable space and $D\subseteq X$ is a
dense set, there is a Boolean homomorphism $\phi:\Cal PD\to\Cal B(X)$
such that $\phi A\subseteq\overline{A}$ for every $A\subseteq D$.
%535M 535Xe

\spheader 535Xi Let $(X,\Sigma,\mu)$ be a measure space.   Show that a
linear lifting $S:\eusm L^{\infty}(\Sigma)\to\eusm L^{\infty}(\Sigma)$
of $\mu$ corresponds to a lifting iff it is `multiplicative', that is,
$S(f\times g)=Sf\times Sg$ for all $f$, $g\in\eusm L^{\infty}(\Sigma)$.
%535O

\spheader 535Xj Let $(X,\Sigma,\mu)$ be a strictly localizable measure
space with non-zero measure.   Suppose that $\nu_{\kappa}$ has a Baire
linear lifting for every infinite cardinal $\kappa$ such that the
Maharam-type-$\kappa$ component of the measure algebra of $\mu$ is
non-zero.   Show that $\mu$ has a linear lifting.
%535O 535B

\spheader 535Xk Let $(X,\Sigma,\mu)$ be a probability space such that
whenever $\Cal E\subseteq\Sigma$, $\#(\Cal E)\le\frak c$ and
$\bigcup\Cal E$ is negligible, then $\bigcup\Cal E\in\Sigma$.   Show
that $\mu$ has a linear lifting.   \Hint{363Yf.}
%535O

\spheader 535Xl Let $(Y,\Tau,\nu)$ be a $\sigma$-finite measure space
with a linear lifting, $Z$ a set, $\Upsilon$ a countably generated
$\sigma$-algebra of subsets of $Z$, and $\mu$ a measure with domain
$\Tau\tensorhat\Upsilon$ such that $\nu$ is the marginal measure of
$\mu$ on $Y$ and the marginal measure of $\mu$ on $Z$ is countably compact.
Show that there is a family
$\family{y}{Y}{\mu_y}$ of measures with domain $\Upsilon$ such that
$y\mapsto\mu_yH$ is a $\Tau$-measurable function for every
$H\in\Upsilon$ and $\mu W=\int\mu_yW[\{y\}]\nu(dy)$ for every
$W\in\Tau\tensorhat\Upsilon$.
%535P 452M

\spheader 535Xm Let $(X,\frak T,\Sigma,\mu)$ and $(Y,\frak S,\Tau,\nu)$
be $\tau$-additive topological probability spaces, and $\lambda$ the
$\tau$-additive product measure on $X\times Y$ (417G).   Suppose that
$\lambda$ has a Borel linear lifting and that $\mu$ is inner regular
with respect to the Borel sets.   Show that $\mu$ has a Borel linear
lifting.
%

\leader{535Y}{Further exercises (a)}
%\spheader 535Ya
Suppose that we are provided with a bijection between
$\Cal B(\Bbb R)$ and $\omega_1$, but are otherwise not permitted to use
the axiom of choice.   Show that we can construct a Borel lifting for
Lebesgue measure.
%535D

\spheader 535Yb Suppose that for every cardinal $\kappa$ there is a
Baire linear lifting for $\nu_{\kappa}$.   Show that for every
$n\in\Bbb N$ there is a Borel linear lifting $S$ for Lebesgue measure on
$[0,1]^n$ which ($\alpha$) respects coordinates in the sense that if
$f:[0,1]^n\to\Bbb R$ is a bounded measurable function determined by
coordinates in
$I\subseteq n$, then $Sf$ also is determined by coordinates in $I$ 
($\beta$) is symmetric in the sense that if $\rho:n\to n$ is any 
permutation and
$(\hat\rho f)(x)=f(x\rho)$ for $x\in[0,1]^n$ and $f:[0,1]^n\to\Bbb R$,
then $S$ commutes with $\hat\rho$.
\Hint{5A1Ib.} 
%535O

%\spheader ????b For any $n\in\Bbb N$ there is an infinite $\kappa$
%such that for $f:[\kappa]^n\to[\kappa]^{\le\frakc}$ there is an
%$I\in[\kappa]^{n+1}$ such that $I\cap f(J)\subseteq J$ for every
%$J\in[I]^n$.

\spheader 535Yc Let $(X,\Sigma,\mu)$ be a countably compact
measure space, $(Y,\Tau,\nu)$ a $\sigma$-finite measure space with a
linear lifting, and
$f:X\to Y$ an \imp\ function.   Suppose there is a family
$\Cal H\subseteq\Sigma$ such that $\Sigma$ is the $\sigma$-algebra of
sets generated by $\Cal H$ and $\#(\Cal H)<\add\nu$.   Show that there
is a disintegration
$\family{y}{Y}{\mu_y}$ of $\mu$ over $\nu$, consistent with $f$, such
that $y\mapsto\mu_yE$ is a $\Tau$-measurable function for every
$E\in\Sigma$.
%535P

\spheader 535Yd\dvAnew{2009}
({\smc T\"ornquist 11}) Let $(X,\Sigma,\mu)$ be a countably separated
perfect complete strictly localizable measure space,
$\frak A$ its measure algebra and $G$ a subgroup of $\Aut\frak A$ of
cardinal at most $\min(\add\Cal N,\frak p)$, where $\Cal N$ is the
null ideal of Lebesgue measure on $\Bbb R$.   Show that there is an action
$\action$ of $G$ on $X$ such that
$\pi\action E=\{\pi\action x:x\in E\}$ belongs to $\Sigma$ and
$(\pi\action E)^{\ssbullet}=\pi(E^{\ssbullet})$ whenever
$\pi\in G$ and $E\in\Sigma$.   \Hint{344C, 425Ya.}
%425Ba mt53bits
}%end of exercises

\leader{535Z}{Problems (a)} Can it be that every probability space has a
lifting?

\cmmnt{By 535B, it is enough to consider
$(\{0,1\}^{\kappa},\vthsp\CalBa(\{0,1\}^{\kappa}),
\vthsp\nu_{\kappa}\restr\CalBa(\{0,1\}^{\kappa}))$ where $\kappa$ is a
cardinal.
Since Mokobodzki's theorem (535Eb) deals with $\kappa\le\omega_2$ when
$\frak c=\omega_1$, the key case to consider seems to be
$\kappa=\omega_3$.
}%end of comment

\spheader 535Zb Suppose that $\frak c\ge\omega_3$.
Does $\nu_{\omega}$ have a Borel lifting?

\cmmnt{It is known to be relatively consistent with ZFC to suppose
that $\frak c=\omega_2$ and that $\FN(\Cal P\Bbb N)=\omega_1$ (554G-554H).
In this case
$\nu_{\omega}$ has a Borel lifting (535E(b-ii)).   But if
$\frak c\ge\omega_3$ then $\frak B_{\omega}$ is not tightly
$\omega_1$-filtered (518S).
}%end of comment

\spheader 535Zc (A.H.Stone) Can there be a countable ordinal $\zeta$ and
a lifting $\phi$
of $\nu_{\omega}$ such that $\phi E$ is a Borel set,
with Baire class at most $\zeta$, for every Borel set
$E\subseteq\{0,1\}^{\omega}$?

\cmmnt{The point of this question is that while, subject to the
continuum hypothesis, we can almost write down a formula for a Borel
lifting for Lebesgue measure (535Ya), the method gives no control over
the Baire classes of the sets constructed.
}%end of comment

\spheader 535Zd Can there be a strictly positive Radon probability
measure of countable Maharam type which does not have a strong lifting?
\cmmnt{(See 453G, 453N, 535I, 535Xg.)}
%535I

\discrversionA{535Zg\query\ Must there be a Boolean homomorphism
$\phi:\Cal P\Bbb Q\to\Cal B(\Bbb R)$ such that
$\phi A\subseteq\overline{A}$ for every $A\subseteq\Bbb Q$?}{}
%535M

\spheader 535Ze Is there a probability space which has a linear lifting
but no lifting?
%535O

\spheader 535Zf Can there be a Borel linear lifting for the usual measure
on $(\{0,1\}^{\omega})^2$ which respects coordinates in the sense of 535R?
%535R

\cmmnt{It seems possible that there is a proof in ZFC that there is no
such lifting;  in which case 535R shows that we should have a negative
answer to (a).
}%end of comment

\endnotes{
\Notesheader{535} For a fuller account of this topic, see {\smc Burke
93}.

{\smc Neumann \& Stone 35} used a direct construction along the lines of
535Xc to show that if the continuum hypothesis is true then Lebesgue
measure has a Borel lifting.   The method works equally well for
$\nu_{\omega_1}$, but for $\nu_{\omega_2}$ we need a further idea from
{\smc Mokobodzki 7?}; %\query;
the version I give here is based on {\smc Geschke 02},
itself derived at some remove from 
{\smc Carlson Frankiewicz \& Zbierski 94}, 
who showed that we could have a Borel
lifting for Lebesgue measure in a model in which the continuum hypothesis
is false (554I).

It is not a surprise that there should be a model of set theory in which
Lebesgue measure has no Borel lifting.   Nor is it a surprise that the
first such model should have been found by S.Shelah ({\smc Shelah 83}).
What does remain surprising is that in most of the vast number of models
of set theory which have been studied, we do not know whether there is
such a lifting.   Only in the familiar case $\frak c=\omega_1$, the
special combination $\frak c=\omega_2=\FN(\Cal P\Bbb N)^+$ (535E),
and in variations of Shelah's
model, do we have definite information.   It remains possible that in
any model in which $\frak m>\omega_1$ or $\frak c=\omega_3$ there is no
Borel lifting for Lebesgue measure.   When we leave the real line, the
position is even more open;  conceivably it is relatively consistent
with ZFC to suppose that every probability space has a lifting, and at
least equally believably it is a theorem of ZFC that $\nu_{\omega_3}$
does not have a Baire lifting.

From 535I we see that $\omega_2$ appears in Losert's example (453N) for
a good reason.   Once again, it seems to be unknown whether it is
consistent to suppose that there is a (completed) strictly positive
Radon probability measure with countable Maharam type which has no
strong lifting (535Zd).   When we come to look for strong Borel
liftings, we have some useful information in the separable metrizable
case (535N).   The result is natural enough.   We are used to supposing
that Polish spaces are all very much the same, and that point-supported
measures are trivial.   But because the concept of `strong' lifting is
topological, and cannot easily be reduced to the Borel structure, we
have to work a bit;  and it seems also that
point-supported measures need care (535M).
%535Zg\query

`Linear liftings' (535O-535R) %535O 535P 535Q 535R
remain poor relations.   I give them house room here partly for
completeness and partly because of a slender hope that they will lead us
to a solution of 535Za.   Of course the match between $\omega_3$
in 535Za and $\frak c^{++}$ in 535R may show only a temporarily
coincidental frontier of ignorance.
{\smc Burke \& Shelah 92} have shown that it is relatively
consistent with ZFC to suppose that $\nu_{\omega}$ has no Borel linear
lifting.

}%end of notes

\discrpage

