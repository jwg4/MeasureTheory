\frfilename{mt212.tex}
\versiondate{10.9.04}

\def\chaptername{Taxonomy of measure spaces}
\def\sectionname{Complete spaces}

\newsection{212}

In the next two sections of this chapter I give brief accounts of the
theory of measure spaces possessing certain of the properties described
in \S211.   I begin with `completeness'.   I give the elementary
facts about complete measure spaces in 212A-212B;  then I turn to the notion of
`completion' of a measure (212C) and its relationships with the other
concepts of measure theory introduced so far (212D-212G).

\leader{212A}{Proposition} Any measure space constructed
by \Caratheodory's method is complete.

\proof{ Recall that `\Caratheodory's method' starts from an arbitrary
outer measure $\theta:\Cal PX\to[0,\infty]$ and sets

\Centerline{$\Sigma=\{E:E\subseteq X,\,\theta A
=\theta(A\cap E)+\theta(A\setminus E)$ for every $A\subseteq X\}$,
\quad$\mu=\theta\restr\Sigma$}

\noindent (113C).    In this case, if $B\subseteq E\in\Sigma$ and
$\mu E=0$, then $\theta B=\theta E=0$ (113A(ii)), so for any
$A\subseteq X$ we have

\Centerline{$\theta(A\cap B)+\theta(A\setminus B)
=\theta(A\setminus B)\le\theta A
\le\theta(A\cap B)+\theta(A\setminus B)$,}

\noindent and $B\in\Sigma$.
}%end of proof of 212A

\leader{212B}{Proposition} (a) If $(X,\Sigma,\mu)$ is a complete measure
space, then any conegligible subset of $X$ is measurable.

(b) Let $(X,\Sigma,\mu)$ be a complete measure space, and $f$ a
$[-\infty,\infty]$-valued function defined on a subset of $X$.   If $f$ is
virtually measurable\cmmnt{ (that is, there is a
conegligible set $E\subseteq X$ such that $f\restr E$ is measurable)},
then $f$ is measurable.

(c) Let $(X,\Sigma,\mu)$ be a complete measure space, and $f$ a
real-valued function defined on a conegligible subset of $X$.   Then the
following are equiveridical, that is, if one is true so are the others:

\quad (i) $f$ is integrable;

\quad(ii) $f$ is measurable and $|f|$ is integrable;

\quad(iii) $f$ is measurable and there is an integrable function $g$
such that $|f|\leae g$.

\proof{{\bf (a)} If $E$ is conegligible, then $X\setminus E$ is
negligible, therefore measurable, and $E$ is measurable.

\medskip

{\bf (b)} Let $a\in\Bbb R$.   Then there is an $H\in\Sigma$ such that


\Centerline{$\{x:(f\restr E)(x)\le a\}=H\cap\dom(f\restr E)
=H\cap E\cap\dom f$.}

\noindent Now $F=\{x:x\in\dom f\setminus E,\,f(x)\le a\}$ is a subset of
the negligible set $X\setminus E$, so is measurable, and

\Centerline{$\{x:f(x)\le a\}=(F\cup H)\cap\dom f\in\Sigma_{\dom f}$,}

\noindent writing $\Sigma_D=\{D\cap E:E\in\Sigma\}$, as in 121A.   As
$a$ is arbitrary, $f$ is measurable (135E).

\medskip

{\bf (c)(i)$\Rightarrow$(ii)} If $f$ is integrable, then by 122P $f$ is
virtually measurable and by 122Re $|f|$ is integrable.   By (b) here,
$f$ is measurable, so (ii) is true.

\medskip

\quad{\bf (ii)$\Rightarrow$(iii)} is trivial.

\medskip

\quad{\bf (iii)$\Rightarrow$(i)} If $f$ is measurable and $g$ is integrable
and $|f|\leae g$, then $f$ is virtually measurable, $|g|$ is integrable and
$|f|\leae|g|$, so 122P tells us that $f$ is integrable.
}%end of proof of 212B

\leader{212C}{The completion of a measure} Let $(X,\Sigma,\mu)$ be any
measure space.

\header{212Ca}{\bf (a)} Let $\hat\Sigma$ be the family of those sets
$E\subseteq X$ such that there are $E'$, $E''\in\Sigma$ with
$E'\subseteq E\subseteq E''$ and $\mu(E''\setminus E')=0$.   Then
$\hat\Sigma$ is a $\sigma$-algebra of subsets of $X$.   \prooflet{\Prf\
{(i)} Of course $\emptyset$ belongs to
$\hat\Sigma$, because we can take $E'=E''=\emptyset$.  {(ii)} If
$E\in\hat\Sigma$, take $E'$, $E''\in\Sigma$ such that
$E'\subseteq E\subseteq E''$ and $\mu(E''\setminus E')=0$.
Then

\Centerline{$X\setminus E''\subseteq X\setminus E
\subseteq X\setminus E'$,
\quad $\mu((X\setminus E')\setminus(X\setminus E''))
=\mu(E''\setminus E')=0$,}

\noindent so $X\setminus E\in\hat\Sigma$. {(iii)} If
$\sequencen{E_n}$ is a sequence in
$\hat\Sigma$, then for each $n$ choose $E'_n$, $E_n''\in\Sigma$ such
that $E'_n\subseteq E_n\subseteq E_n''$ and
$\mu(E_n''\setminus E'_n)=0$.   Set

\Centerline{$E=\bigcup_{n\in\Bbb N}E_n$,
\quad$E'=\bigcup_{n\in\Bbb N}E'_n$,
\quad$E''=\bigcup_{n\in\Bbb N}E_n''$;}

\noindent  then $E'\subseteq E\subseteq E''$ and
$E''\setminus E'\subseteq\bigcup_{n\in\Bbb N}(E_n''\setminus E'_n)$ is
negligible, so $E\in\hat\Sigma$.  \Qed}

\spheader 212Cb For $E\in\hat\Sigma$, set

\Centerline{$\hat\mu E=\mu^*E
=\min\{\mu F:E\subseteq F\in\Sigma\}$\dvro{.}{}}

\cmmnt{
\noindent (132A).   It is worth remarking at once that if $E\in\hat\Sigma$,
$E'$, $E''\in\Sigma$, $E'\subseteq E\subseteq E''$ and
$\mu(E''\setminus  E')=0$, then $\mu E'=\hat\mu E=\mu E''$;  this is
because

\Centerline{$\mu E'=\mu^*E'\le\mu^*E\le\mu^*E''=\mu E''
=\mu E'+\mu(E''\setminus E)=\mu E'$}

\noindent (recalling from 132A, or noting now, that $\mu^*A\le\mu^*B$
whenever $A\subseteq B\subseteq X$, and that $\mu^*$ agrees with $\mu$
on $\Sigma$).
}%end of comment

\header{212Cc}{\bf (c)}\cmmnt{ We now find that}
$(X,\hat\Sigma,\hat\mu)$ is a
measure space.   \prooflet{\Prf\ (i) Of course $\hat\mu$, like $\mu$,
takes values in $[0,\infty]$.   (ii)  $\hat\mu\emptyset=\mu\emptyset=0$.
(iii)  Let $\sequencen{E_n}$ be a disjoint sequence in
$\hat\Sigma$, with union $E$.  For each $n\in\Bbb N$ choose $E_n'$,
$E_n''\in\Sigma$ such that $E_n'\subseteq E_n\subseteq E_n''$ and
$\mu(E_n''\setminus E_n')=0$.    Set $E'=\bigcup_{n\in\Bbb N}E_n'$,
$E''=\bigcup_{n\in\Bbb N}E_n''$.   Then (as in (a-iii) above)
$E'\subseteq E\subseteq E''$ and $\mu(E''\setminus E')=0$, so

\Centerline{$\hat\mu E=\mu E'=\sum_{n=0}^{\infty}\mu E_n'
=\sum_{n=0}^{\infty}\hat\mu E_n$}

\noindent because $\sequencen{E_n'}$, like $\sequencen{E_n}$, is
disjoint. \Qed}

\header{212Cd}{\bf (d)}\cmmnt{ The measure space}
$(X,\hat\Sigma,\hat\mu)$
is called the {\bf completion} of the measure space $(X,\Sigma,\mu)$;
\cmmnt{equally,} I will call $\hat\mu$ the {\bf completion} of $\mu$,
and occasionally\cmmnt{ (if it seems plain which null ideal is
under consideration)} I will call $\hat\Sigma$ the completion of
$\Sigma$.   Members of $\hat\Sigma$ are sometimes called
{\bf $\mu$-measurable}.

\leader{212D}{}\cmmnt{There is something I had better check at once.

\medskip

\noindent}{\bf Proposition} Let $(X,\Sigma,\mu)$ be any measure space.
Then $(X,\hat\Sigma,\hat\mu)$, as defined in 212C, is a complete measure
space and $\hat\mu$ is an extension of $\mu$;  and
$(X,\hat\Sigma,\hat\mu)=(X,\Sigma,\mu)$ iff $(X,\Sigma,\mu)$ is
complete.

\proof{{\bf (a)} Suppose that $A\subseteq E\in\hat\Sigma$ and
$\hat\mu E=0$.   Then (by 212Cb) there is an $E''\in\Sigma$ such that
$E\subseteq E''$ and $\mu E''=0$.   Accordingly we have

\Centerline{$\emptyset\subseteq A\subseteq E''$,\quad
$\mu(E''\setminus\emptyset)=0$,}

\noindent so $A\in\hat\Sigma$.   As $A$ is arbitrary, $\hat\mu$ is
complete.

\medskip

{\bf (b)} If $E\in\Sigma$, then of course $E\in\hat\Sigma$, because
$E\subseteq E\subseteq E$ and $\mu(E\setminus E)=0$;  and $\hat\mu
E=\mu^*E=\mu E$.   Thus $\Sigma\subseteq\hat\Sigma$ and $\hat\mu$
extends $\mu$.

\medskip

{\bf (c)} If $\mu=\hat\mu$ then of course
$\mu$ must be complete.   If $\mu$ is complete,
and $E\in\hat\Sigma$, then there are $E'$, $E''\in\Sigma$ such that
$E'\subseteq E\subseteq E''$ and $\mu(E''\setminus E')=0$.   But now
$E\setminus E'\subseteq E''\setminus E'$, so (because $(X,\Sigma,\mu)$
is complete) $E\setminus E'\in\Sigma$ and $E=E'\cup(E\setminus
E')\in\Sigma$.   As $E$ is arbitrary, $\hat\Sigma\subseteq\Sigma$ and
$\hat\Sigma=\Sigma$ and $\mu=\hat\mu$.
}%end of proof of 212D

\vleader{72pt}{212E}{}\cmmnt{The importance of this construction is such that
it is worth spelling out some further elementary properties.

\medskip

\noindent}{\bf Proposition} Let $(X,\Sigma,\mu)$ be a measure space, and
$(X,\hat\Sigma,\hat\mu)$ its completion.

(a) The outer measures $\hat\mu^*$, $\mu^*$ defined from $\hat\mu$ and
$\mu$ coincide.

(b) $\mu$, $\hat\mu$ give rise to the same negligible and conegligible
sets and the same sets of full outer measure.

(c) $\hat\mu$ is the only measure with domain $\hat\Sigma$ which agrees
with $\mu$ on $\Sigma$.

(d) A subset of $X$ belongs to $\hat\Sigma$ iff it is expressible as
$F\symmdiff A$ where $F\in\Sigma$ and $A$ is $\mu$-negligible.

\proof{{\bf (a)} Take any $A\subseteq X$.   (i) If
$A\subseteq F\in\Sigma$, then $F\in\hat\Sigma$ and $\mu F=\hat\mu F$, so

\Centerline{$\hat\mu^*A\le\hat\mu F=\mu F$;}

\noindent as $F$ is arbitrary, $\hat\mu^*A\le\mu^*A$.   (ii) If
$A\subseteq E\in\hat\Sigma$, there is an $E''\in\Sigma$ such that
$E\subseteq E''$ and $\mu E''=\hat\mu E$, so

\Centerline{$\mu^*A\le\mu E''=\hat\mu E$;}

\noindent as $E$ is arbitrary, $\mu^*A\le\hat\mu^*A$.

\medskip
{\bf (b)} Now, for $A\subseteq X$,

\Centerline{$A$ is $\mu$-negligible$\,\iff\mu^*A=0
\iff\hat\mu^*A=0\iff A$ is $\hat\mu$-negligible,}

$$\eqalign{A\text{ is }\mu&\text{-conegligible}
\,\iff\,\mu^*(X\setminus A)=0\cr
&\iff\,\hat\mu^*(X\setminus A)=0
\iff A\text{ is }\hat\mu\text{-conegligible}.\cr}$$

\noindent If $A$ has full outer measure for $\mu$, $F\in\hat\Sigma$ and $F\cap A=\emptyset$, then there is an $F'\in\Sigma$ such that $F'\subseteq F$ and $\mu F'=\hat\mu F$;  as $F'\cap A=\emptyset$,
$F'$ is $\mu$-negligible and $F$ is $\hat\mu$-negligible;  as $F$ is arbitrary, $A$ has full outer measure for $\hat\mu$.   In the other direction, of course, if $A$ has full outer measure for
$\hat\mu$ then

\Centerline{$\mu^*(F\cap A)=\hat\mu^*(F\cap A)=\hat\mu F=\mu F$}

\noindent for every $F\in\Sigma$, so $A$ has full outer measure for $\mu$.

\medskip

{\bf (c)} If $\tilde\mu$ is any measure with domain $\hat\Sigma$
extending $\mu$, we must have

\Centerline{$\mu E'\le\tilde\mu E\le\mu E''$,
\quad $\mu E'=\hat\mu E=\mu E''$,}

\noindent so $\tilde\mu E=\hat\mu E$, whenever $E'$, $E''\in\Sigma$,
$E'\subseteq E\subseteq E''$ and $\mu(E''\setminus E')=0$.

\medskip

{\bf (d)(i)} If $E\in\hat\Sigma$, take $E'$, $E''\in\Sigma$ such that
$E'\subseteq E\subseteq E''$ and $\mu(E''\setminus E')=0$.   Then
$E\setminus E'\subseteq E''\setminus E'$, so $E\setminus E'$ is
$\mu$-negligible, and $E=E'\symmdiff(E\setminus E')$ is the symmetric
difference of a member of $\Sigma$ and a negligible set.

\medskip

\quad{\bf (ii)} If $E=F\symmdiff A$, where $F\in\Sigma$ and
$A$ is $\mu$-negligible, take $G\in\Sigma$ such that $\mu G=0$ and
$A\subseteq G$;  then $F\setminus G\subseteq E\subseteq F\cup G$ and
$\mu((F\cup G)\setminus(F\setminus G))=\mu G=0$, so $E\in\hat\Sigma$.
}%end of proof of 212E

\vleader{60pt}{212F}{}\cmmnt{Now let us consider integration with respect to
the completion of a measure.

\medskip

\noindent}{\bf Proposition} Let $(X,\Sigma,\mu)$ be a measure space and
$(X,\hat\Sigma,\hat\mu)$ its completion.

(a) A $[-\infty,\infty]$-valued function $f$ defined on a subset of $X$
is $\hat\Sigma$-measurable iff it is $\mu$-virtually measurable.

(b) Let $f$ be a $[-\infty,\infty]$-valued function defined on a subset
of $X$.   Then
$\int fd\mu=\int fd\hat\mu$ if either is defined in $[-\infty,\infty]$;
in particular, $f$ is $\mu$-integrable iff it is $\hat\mu$-integrable.

\proof{{\bf (a)(i)}  Suppose that $f$ is a $[-\infty,\infty]$-valued
$\hat\Sigma$-measurable function.    For $q\in\Bbb Q$ let
$E_q\in\hat\Sigma$ be such that $\{x:f(x)\le q\}=\dom f\cap E_q$, and
choose $E_q'$, $E_q''\in\Sigma$ such that
$E_q'\subseteq E_q\subseteq E_q''$ and $\mu(E_q''\setminus E_q')=0$.
Set $H=X\setminus\bigcup_{q\in\Bbb Q}(E_q''\setminus E_q')$;  then $H$
is $\mu$-conegligible.   For $a\in\Bbb R$ set

\Centerline{$G_a=\bigcup_{q\in\Bbb Q,q<a}E_q'\in\Sigma$;}

\noindent  then

\Centerline{$\{x:x\in\dom(f\restr H),\,(f\restr H)(x)<a\}
=G_a\cap\dom(f\restr H)$.}

\noindent This shows that $f\restr H$ is
$\Sigma$-measurable, so that $f$ is $\mu$-virtually measurable.

\medskip

\quad{\bf (ii)} If $f$ is $\mu$-virtually measurable, then there is a
$\mu$-conegligible set $H\subseteq X$ such that $f\restr H$ is
$\Sigma$-measurable.   Since $\Sigma\subseteq\hat\Sigma$, $f\restr H$ is
also $\hat\Sigma$-measurable.   And $H$ is $\hat\mu$-conegligible, by
212Eb.   But this means that $f$ is $\hat\mu$-virtually measurable,
therefore $\hat\Sigma$-measurable, by 212Bb.

\medskip

{\bf (b)(i)}  Let $f:D\to[-\infty,\infty]$ be a function, where
$D\subseteq X$.
If either of $\int fd\mu$, $\int fd\hat\mu$ is defined in
$[-\infty,\infty]$, then $f$ is virtually measurable, and defined almost
everywhere, for one of the appropriate measures, and therefore for both
(putting (a) above together with 212Bb).

\medskip

\quad{\bf (ii)} Now suppose that $f$ is non-negative and
integrable either with respect to $\mu$ or with respect to $\hat\mu$.
Let $E\in\Sigma$ be a conegligible set included in $\dom f$ such
that $f\restr E$ is $\Sigma$-measurable.    For $n\in\Bbb N$, $k\ge 1$ set

\Centerline{$E_{nk}=\{x:x\in E$, $f(x)\ge 2^{-n}k\}$;}

\noindent then each $E_{nk}$ belongs to $\Sigma$ and is of finite
measure for both $\mu$ and $\hat\mu$.   (If $f$ is $\mu$-integrable,

\Centerline{$\hat\mu E_{nk}=\mu E_{nk}\le 2^n\int fd\mu$;}

\noindent if $f$ is $\hat\mu$-integrable,

\Centerline{$\mu E_{nk}=\hat\mu E_{nk}\le 2^n\int fd\hat\mu$.)}

\noindent So

\Centerline{$f_n=\sum_{k=1}^{4^n}2^{-n}\chi E_{nk}$}

\noindent is both $\mu$-simple and $\hat\mu$-simple, and
$\int f_nd\mu=\int f_nd\hat\mu$.   Observe that, for $x\in E$,

$$\eqalign{f_n(x)
&=2^{-n}k\text{ if }k<4^n\text{ and }2^{-n}k\le f(x)<2^{-n}(k+1),\cr
&=2^n\text{ if }f(x)\ge 2^n.\cr}$$

\noindent Thus $\sequencen{f_n}$ is a non-decreasing
sequence of functions converging to $f$ at every point of $E$, that is,
both $\mu$-almost everywhere and $\hat\mu$-almost everywhere.   So we
have, for any $c\in\Bbb R$,

$$\eqalign{\int fd\mu=c
&\iff\lim_{n\to\infty}\int f_nd\mu=c\cr
&\iff\lim_{n\to\infty}\int f_nd\hat\mu=c
\iff\int fd\hat\mu=c.\cr}$$

\medskip

\quad{\bf (iii)} As for infinite integrals, recall that for a non-negative
function I write `$\int f=\infty$' just when $f$ is defined almost
everywhere, is virtually measurable, and is not integrable.   So (i) and
(ii) together show that $\int fd\mu=\int fd\hat\mu$ whenever $f$ is
non-negative and either integral is defined in $[0,\infty]$.

\medskip

\quad{\bf (iv)} Since both $\mu$, $\hat\mu$ agree that $\int f$ is to be
interpreted as $\int f^+-\int f^-$ just when this can be defined in
$[-\infty,\infty]$, writing $f^+(x)=\max(f(x),0)$,
$f^-(x)=\max(-f(x),0)$
for $x\in\dom f$, the result for general real-valued $f$ follows at
once.
}%end of proof of 212F

\leader{212G}{}\cmmnt{I turn now to the question of the effect of the
construction on the properties listed in
211B-211K. %211B 211C 211D 211E 211F 211G 211H 211I 211J 211K
\medskip

\noindent}{\bf Proposition} Let $(X,\Sigma,\mu)$ be a measure space, and
$(X,\hat\Sigma,\hat\mu)$ its completion.

(a) $(X,\hat\Sigma,\hat\mu)$ is a probability space, or totally finite,
or $\sigma$-finite, or semi-finite, or localizable,  iff
$(X,\Sigma,\mu)$ is.

(b) $(X,\hat\Sigma,\hat\mu)$ is strictly localizable if $(X,\Sigma,\mu)$
is, and any decomposition of $X$ for $\mu$ is a decomposition for
$\hat\mu$.

(c) A set $H\in\hat\Sigma$ is an atom for $\hat\mu$ iff there is an
$E\in\Sigma$ such that $E$ is an atom for $\mu$ and $\hat\mu(H\symmdiff
E)=0$.

(d) $(X,\hat\Sigma,\hat\mu)$ is atomless or purely atomic iff
$(X,\Sigma,\mu)$ is.

\proof{{\bf (a)(i)} Because $\hat\mu X=\mu X$,
$(X,\hat\Sigma,\hat\mu)$ is a
probability space, or totally finite, iff $(X,\Sigma,\mu)$ is.

\medskip

\quad{\bf (ii)}\grheada\ If $(X,\Sigma,\mu)$ is $\sigma$-finite,
there is a sequence $\sequencen{E_n}$, covering $X$, with $\mu
E_n<\infty$
for each $n$.   Now $\hat\mu E_n<\infty$ for each $n$, so
$(X,\hat\Sigma,\hat\mu)$ is $\sigma$-finite.

\medskip

\qquad\grheadb\ If $(X,\hat\Sigma,\hat\mu)$ is $\sigma$-finite,
there is a sequence $\sequencen{E_n}$, covering $X$, with
$\hat\mu E_n<\infty$ for each $n$.   Now we can find, for each $n$, an
$E_n''\in\Sigma$ such that $\mu E_n''<\infty$ and $E_n\subseteq E_n''$;
so that $\sequencen{E_n''}$ witnesses that $(X,\Sigma,\mu)$ is
$\sigma$-finite.

\medskip

\quad{\bf (iii)}\grheada\ If $(X,\Sigma,\mu)$ is semi-finite and
$\hat\mu E=\infty$, then there is an $E'\in\Sigma$ such
that $E'\subseteq E$ and $\mu E'=\infty$.  Next, there is an
$F\in\Sigma$
such that $F\subseteq E'$ and $0<\mu F<\infty$.   Of course we now have
$F\in\hat\Sigma$, $F\subseteq E$ and $0<\hat\mu F<\infty$.   As $E$ is
arbitrary, $(X,\hat\Sigma,\mu)$ is semi-finite.

\medskip

\qquad\grheadb\ If $(X,\hat\Sigma,\hat\mu)$ is semi-finite
and $\mu E=\infty$, then $\hat\mu E=\infty$, so there is an $F\subseteq
E$
such that $0<\hat\mu F<\infty$.   Next, there is an $F'\in\Sigma$ such
that $F'\subseteq F$ and $\mu F'=\hat\mu F$.   Of course we now have
$F'\subseteq E$ and $0<\mu F'<\infty$.   As $E$ is arbitrary,
$(X,\Sigma,\mu)$ is semi-finite.

\medskip

\quad{\bf (iv)}\grheada\ If $(X,\Sigma,\mu)$ is localizable and
$\Cal E\subseteq \hat\Sigma$, then set

\Centerline{$\Cal F=\{F:F\in\Sigma,\,\exists\enskip
E\in\Cal E,\, F\subseteq E\}$.}

\noindent Let $H$ be an essential supremum of
$\Cal F$ in $\Sigma$, as in 211G.

If $E\in\Cal E$, there is an $E'\in\Sigma$ such that $E'\subseteq E$ and
$E\setminus E'$ is negligible;  now $E'\in\Cal F$, so

\Centerline{$\hat\mu(E\setminus H)
\le\hat\mu(E\setminus E')+\mu(E'\setminus H)=0$.}

\noindent If $G\in\hat\Sigma$ and
$\hat\mu(E\setminus G)=0$ for every $E\in\Cal E$, let $G''\in\Sigma$ be
such that $G\subseteq G''$ and $\hat\mu(G''\setminus G)=0$;  then, for
any $F\in\Cal F$, there is an
$E\in\Cal E$ including $F$, so that

\Centerline{$\mu(F\setminus G'')\le\hat\mu(E\setminus G)=0.$}

\noindent As $F$ is arbitrary, $\mu(H\setminus G'')=0$ and
$\hat\mu(H\setminus G)=0$.   This shows that $H$ is an essential
supremum of $\Cal E$ in $\hat\Sigma$.   As $\Cal E$ is arbitrary,
$(X,\hat\Sigma,\hat\mu)$ is localizable.

\medskip

\qquad\grheadb\ Suppose that $(X,\hat\Sigma,\hat\mu)$ is
localizable and that $\Cal E\subseteq\Sigma$.   Working in
$(X,\hat\Sigma,\hat\mu)$, let $H$ be an essential supremum for $\Cal E$
in $\hat\Sigma$.   Let $H'\in\Sigma$ be such that $H'\subseteq H$ and
$\hat\mu(H\setminus H')=0$.   Then

\Centerline{$\mu(E\setminus H')
\le\hat\mu(E\setminus H)+\hat\mu(H\setminus H')=0$}

\noindent for every $E\in\Cal E$;  while if $G\in\Sigma$ and
$\mu(E\setminus G)=0$ for every $E\in\Cal E$, we must have

\Centerline{$\mu(H'\setminus G)\le\hat\mu(H\setminus G)=0$.}

\noindent   Thus $H'$ is an essential supremum of $\Cal E$ in $\Sigma$.
As $\Cal E$ is arbitrary, $(X,\Sigma,\mu)$ is localizable.

\medskip

{\bf (b)} Let $\langle X_i\rangle_{i\in I}$ be a decomposition of $X$
for $\mu$, as in 211E.    Of course it is a partition of $X$ into
sets of finite $\hat\mu$-measure.   If $H\subseteq X$ and
$H\cap X_i\in\hat\Sigma$ for every $i$, choose for each $i\in I$ sets $E_i'$, $E_i''\in\Sigma$ such that

\Centerline{$E_i'\subseteq H\cap X_i\subseteq E_i''$,\quad
$\mu(E_i''\setminus E_i')=0$.}

\noindent   Set $E'=\bigcup_{i\in I}E_i'$,
$E''=\bigcup_{i\in I}(E_i''\cap X_i)$.   Then $E'\cap X_i=E_i'$,
$E''\cap X_i=E_i''\cap X_i$ for
each $i$, so $E'$ and $E''$ belong to $\Sigma$.   Also

\Centerline{$\mu(E''\setminus E')
=\sum_{i\in I}\mu(E_i''\cap X_i\setminus E_i')=0$.}

\noindent As $E'\subseteq H\subseteq E''$, $H\in\hat\Sigma$ and

\Centerline{$\hat\mu H=\mu E'
=\sum_{i\in I}\mu E'_i=\sum_{i\in I}\hat\mu(H\cap X_i)$.}

\noindent As $H$ is arbitrary, $\langle X_i\rangle_{i\in I}$
is a decomposition of $X$ for $\hat\mu$.

Accordingly, $(X,\hat\Sigma,\hat\mu)$ is strictly localizable if such a
decomposition exists, which is so if $(X,\Sigma,\mu)$ is strictly
localizable.

\medskip

{\bf (c)-(d)(i)} Suppose that $E\in\hat\Sigma$ is an atom for $\hat\mu$.
Let $E'\in\Sigma$ be such that $E'\subseteq E$ and
$\hat\mu(E\setminus E')=0$.   Then $\mu E'=\hat\mu E>0$.   If $F\in\Sigma$ and $F\subseteq E'$, then $F\subseteq E$, so either
$\mu F=\hat\mu F=0$ or $\mu(E'\setminus F)=\hat\mu(E\setminus F)=0$.   As
$F$ is arbitrary, $E'$ is an atom for $\mu$, and
$\hat\mu(E\symmdiff E')=\hat\mu(E\setminus E')=0$.

\medskip

\quad{\bf (ii)} Suppose that $E\in\Sigma$ is an atom for $\mu$, and that
$H\in\hat\Sigma$ is such that $\hat\mu(H\symmdiff E)=0$.   Then
$\hat\mu H=\mu E>0$.   If $F\in\hat\Sigma$ and $F\subseteq H$, let
$F'\subseteq F$ be such that $F'\in\Sigma$ and
$\hat\mu(F\setminus F')=0$.   Then
$E\cap F'\subseteq E$ and $\hat\mu(F\symmdiff(E\cap F'))=0$, so either
$\hat\mu F=\mu(E\cap F')=0$ or
$\hat\mu(H\setminus F)=\mu(E\setminus F')=0$.   As $F$ is arbitrary, $H$
is an atom for $\hat\mu$.

\medskip

\quad{\bf (iii)} It follows at once that $(X,\hat\Sigma,\hat\mu)$ is
atomless iff $(X,\Sigma,\mu)$ is.

\medskip

\quad{\bf (iv)}\grheada\ On the other hand, if $(X,\Sigma,\mu)$
is purely atomic and $\hat\mu H>0$, there is an $E\in\Sigma$ such that
$E\subseteq H$ and $\mu E>0$, and an atom $F$ for $\mu$ such that
$F\subseteq E$;  but $F$ is also an atom for $\hat\mu$.   As $H$ is
arbitrary, $(X,\hat\Sigma,\hat\mu)$ is purely atomic.

\medskip

\qquad\grheadb\ And if $(X,\hat\Sigma,\hat\mu)$ is purely
atomic and $\mu E>0$, then there is an $H\subseteq E$ which is an atom
for
$\hat\mu$;  now let $F\in\Sigma$ be such that $F\subseteq H$ and
$\hat\mu(H\setminus F)=0$, so that $F$ is an atom for $\mu$ and
$F\subseteq E$.   As $E$ is arbitary, $(X,\Sigma,\mu)$ is purely atomic.
}%end of proof of 212G

\exercises{
\leader{212X}{Basic exercises $\pmb{>}$(a)} 
%\spheader 212Xa
Let $(X,\Sigma,\mu)$ be a
complete measure space.   Suppose that $A\subseteq E\in\Sigma$ and that
$\mu^*A+\mu^*(E\setminus A)=\mu E<\infty$.   Show that $A\in\Sigma$.

\sqheader 212Xb Let $\mu$ and $\nu$ be two measures on a set $X$, with
completions $\hat\mu$ and $\hat\nu$.   Show that the following are
equiveridical:  (i) the outer measures $\mu^*$, $\nu^*$ defined from $\mu$
and $\nu$ coincide;  (ii) $\hat\mu E=\hat\nu E$ whenever either is
defined and finite;   (iii) $\int fd\mu=\int fd\nu$ whenever $f$ is a
real-valued function such that either integral is defined and finite.
\Hint{for (i)$\Rightarrow$(ii), if $\hat\mu E<\infty$, take a measurable
envelope $F$ of $E$ for $\nu$ and calculate
$\nu^*E+\nu^*(F\setminus E)$.}

\spheader 212Xc Let $\mu$ be the restriction of Lebesgue measure
to the Borel $\sigma$-algebra of $\Bbb R$, as in 211P.   Show that its
completion is Lebesgue measure itself.   \Hint{134F.}

\spheader 212Xd Repeat 212Xc for (i) Lebesgue measure on
$\BbbR^r$ (ii) Lebesgue-Stieltjes measures on $\Bbb R$ (114Xa).

\spheader 212Xe
Let $X$ be a set and $\Sigma$ a $\sigma$-algebra
of subsets of $X$.   Let $\Cal I$ be a $\sigma$-ideal of subsets of $X$
(112Db).
(i) Show that $\Sigma_1=\{E\symmdiff A:E\in\Sigma,\,A\in\Cal I\}$ is a
$\sigma$-algebra of subsets of $X$.
(ii) Let $\Sigma_2$ be the family of sets $E\subseteq X$ such
that there are $E'$, $E''\in\Sigma$ with $E'\subseteq E\subseteq E''$
and $E''\setminus E'\in\Cal I$.   Show that $\Sigma_2$ is a
$\sigma$-algebra of subsets of $X$ and that $\Sigma_2\subseteq\Sigma_1$.
(iii) Show that $\Sigma_2=\Sigma_1$ iff every member of
$\Cal I$ is included in a member of $\Sigma\cap\Cal I$.

\spheader 212Xf
Let $(X,\Sigma,\mu)$ be a measure space, $Y$ any set and
$\phi:X\to Y$ a function.   Set $\theta B=\mu^*\phi^{-1}[B]$ for every
$B\subseteq Y$.   (i) Show that $\theta$ is an outer measure on $Y$.
(ii) Let $\nu$ be the measure defined from $\theta$ by \Caratheodory's
method, and $\Tau$ its domain.   Show that if $C\subseteq Y$ and
$\phi^{-1}[C]\in\Sigma$ then $C\in\Tau$.   (iii) Suppose that
$(X,\Sigma,\mu)$ is complete
and totally finite.   Show that $\nu$ is the image measure
$\mu\phi^{-1}$.

\spheader 212Xg
Let $g$, $h$ be two non-decreasing functions from
$\Bbb R$ to itself, and $\mu_g$, $\mu_h$ the associated
Lebesgue-Stieltjes
measures.   Show that a real-valued function $f$ defined on a subset of
$\Bbb R$ is $\mu_{g+h}$-integrable iff it is both
$\mu_g$-integrable and $\mu_h$-integrable, and that then
$\int fd\mu_{g+h}=\int fd\mu_g+\int fd\mu_h$.   \Hint{114Yb}.

\spheader 212Xh Let $(X,\Sigma,\mu)$ be a measure 
space, and $\Cal I$ a
$\sigma$-ideal of subsets of $X$;  set
$\Sigma_1=\{E\symmdiff A:E\in\Sigma,\,A\in\Cal I\}$, as in 212Xe.
Show that if every member of $\Sigma\cap\Cal I$
is $\mu$-negligible, then there is a unique extension of $\mu$ to a
measure $\mu_1$ with domain $\Sigma_1$ such that $\mu_1A=0$ for every
$A\in\Cal I$.
%212C

\spheader 212Xi\dvAnew{2009}
Let $(X,\Sigma,\mu)$ be a complete measure space such that
$\mu X>0$, $Y$ a set, $f:X\to Y$ a function and $\mu f^{-1}$ the image
measure on $Y$.   Show that if $\Cal F$ is the filter of
$\mu$-conegligible subsets of $X$, then the image filter $f[[\Cal F]]$
(2A1Ib) is the filter of $\mu f^{-1}$-conegligible subsets of $Y$.

\spheader 212Xj\dvAnew{2010}
Let $(X,\Sigma,\mu)$ be a complete measure space and $f:X\to\Bbb R$ a
function such that $\overline{\int}fd\mu<\infty$.   Show that there is a
measurable function $g:X\to\Bbb R$ such that $f(x)\le g(x)$ for every 
$x\in X$ and $\int g\,d\mu=\overline{\int}fd\mu$.

\leader{212Y}{Further exercises (a)}
%\spheader 212Ya
Let $X$ be a set and $\phi$ an inner measure on $X$,
that is, a functional from $\Cal PX$ to $[0,\infty]$ such that

\inset{$\phi\emptyset=0$,}
\inset{$\phi(A\cup B)\ge\phi A+\phi B$ if $A\cap B=\emptyset$,}

\inset{$\phi(\bigcap_{n\in\Bbb N}A_n)=\lim_{n\to\infty}\phi A_n$
whenever $\sequencen{A_n}$ is a non-increasing sequence of subsets of
$X$ and $\phi A_0<\infty$,}

\inset{if $\phi A=\infty$, $a\in\Bbb R$ there is a $B\subseteq A$ such
that $a\le\phi B<\infty$.}

\noindent Let $\mu$ be the measure defined from $\phi$, that is,
$\mu=\phi\restr\Sigma$, where

\Centerline{$\Sigma=\{E:\phi(A)
  =\phi(A\cap E)+\phi(A\setminus E)\Forall A\subseteq X\}$}

\noindent (113Yg).   Show that $\mu$ must be complete.
}%end of exercises

\endnotes{
\Notesheader{212} The process of completion is so natural, and so
universally applicable,
and so convenient, that over large parts of measure theory it is
reasonable to use only complete measure spaces.   Indeed many authors so
phrase their definitions that, explicitly or implicitly, only complete
measure spaces are considered.   In this treatise I avoid taking quite
such a large step, even though it would simplify the statements of many
of the theorems in this volume (for instance).   I did take the trouble, in Volume 1, to give a definition of `integrable function' which, in effect, looks at integrability with respect to the
completion of a measure (212Fb).   There are non-complete
measure spaces which are worthy of study (for example, the restriction
of Lebesgue measure to the Borel $\sigma$-algebra of $\Bbb R$ -- see
211P), and some
interesting questions to be dealt with in Volumes 3 and 5 apply to
them.   At the cost of rather a lot of verbal manoeuvres, therefore, I
prefer to write theorems out in a form in which they can be applied to
arbitrary measure spaces, without assuming completeness.   But it would
be reasonable, and indeed would sharpen your technique, if you regularly
sought the alternative formulations which become natural if you are
interested only in complete spaces.
}

\discrpage


