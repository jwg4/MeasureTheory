\frfilename{mt32.tex} 
\versiondate{21.7.11} 
\copyrightdate{2008} 
 
\def\chaptername{Measure algebras} 
\def\sectionname{Introduction} 
 
\newchapter{32} 
 
I now come to the real work of this volume, the study of the Boolean 
algebras of equivalence classes of measurable sets.   In this chapter I 
work through the `elementary' theory, defining this to consist of the 
parts which do not depend on Maharam's theorem or the lifting theorem or 
non-trivial set theory. 
 
\S321 gives the definition of `measure algebra', and relates this idea 
to its origin as the quotient of a $\sigma$-algebra of measurable sets 
by a $\sigma$-ideal of negligible sets, both in its elementary 
properties (following those of measure spaces treated in \S112) and in 
an appropriate version of the Stone representation. 
\S322 deals with the classification of measure algebras according to the 
scheme already developed in \S211 for measure spaces.   \S323 discusses 
the standard topology and uniformity of a measure algebra.   \S324 
contains results 
concerning Boolean homomorphisms between measure algebras, with the 
relationships between topological continuity, order-continuity and 
preservation of measure.   \S325 is devoted to the measure algebras of 
product measures, and their abstract characterization as completed free
products.    
\S\S326-327 address the properties of additive functionals on Boolean 
algebras, generalizing the ideas of Chapter 23.   Finally, \S328 looks at 
`reduced products' of probability algebras and some related constructions, 
including inductive limits. 
 
\discrpage 
 
 
