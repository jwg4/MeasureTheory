\frfilename{mt523.tex}
\versiondate{23.6.10}
\copyrightdate{2005}

\def\chaptername{Cardinal functions of measure theory}
\def\sectionname{The measure of $\{0,1\}^I$}

\newsection{523}

\def\headlinesectionname{The measure of
$\{0,1{\delimiter"5267309 }^I$}

In \S522 I tried to give an account of current knowledge concerning
the most important cardinals associated with Lebesgue measure.   The
next step is to investigate the usual measure $\nu_I$ on $\{0,1\}^I$
for an arbitrary set $I$.   Here I discuss the cardinals
associated with these measures.   Obviously they depend only on $\#(I)$,
and are trivial if $I$ is finite.   I start with the basic diagram
relating the cardinal functions
of $\nu_{\kappa}$ and $\nu_{\lambda}$ for different cardinals $\kappa$
and $\lambda$ (523B).   I take the opportunity to mention some simple facts
about the measures $\nu_I$ (523C-523D). %523C 523D
Then I look at additivities (523E), covering numbers (523F-523G),
uniformities (523H-523L), %523H 523I 523J 523K 523L
shrinking numbers (523M) and cofinalities (523N).   I end with a
description of these cardinals under the generalized continuum
hypothesis (523P).

\leader{523A}{Notation} For any measure $\mu$, write $\Cal N(\mu)$ for
the null ideal of $\mu$.   For any set $I$, I will write $\nu_I$ for
the usual measure on $\{0,1\}^I$ and $\Cal N_I=\Cal N(\nu_I)$ for its
null ideal.   \cmmnt{Recall that
$(\{0,1\}^{\omega},\Cal N_{\omega})$ is isomorphic to $(\Bbb R,\Cal N)$,
where $\Cal N$ is the Lebesgue null ideal (522Wa).}

\leader{523B}{The basic diagram} Suppose that $\kappa$ and $\lambda$ are
infinite cardinals, with $\kappa\le\lambda$.   Then we have the
following
diagram\cmmnt{ for the additivity, covering number, uniformity,
shrinking number and cofinality of the ideals $\Cal N_{\kappa}$ and
$\Cal N_{\lambda}$}:

\def\tmphrule{\hskip0.4em\raise
2.5pt\hbox{\leaders\hrule\hskip1.7em\hfil}\hskip0.4em}
\def\tmpstrut{\vrule height10.5pt depth5.5pt width0pt}

$$\vbox{\offinterlineskip
\halign{\hfil#\hfil&\hfil#\hfil&\hfil#\hfil&\hfil#\hfil
  &\hfil#\hfil&\hfil#\hfil&\hfil#\hfil
  &\hfil#\hfil&\hfil#\hfil&\hfil#\hfil
  &\hfil#\hfil&\hfil#\hfil&\hfil#\hfil&\hfil#\hfil\cr
&\tmpstrut&$\cov\Cal N_{\lambda}$&\tmphrule
  &$\cov\Cal N_{\kappa}$&\tmphrule
  &$\cf\Cal N_{\kappa}$&\tmphrule&$\cf\Cal N_{\lambda}$
  &\tmphrule&$\lambda^{\omega}$\cr
&\tmpstrut&\vrule&&\vrule&&\vrule&&\vrule\cr
&\tmpstrut&\vrule&&\vrule&&$\shr\Cal N_{\kappa}$&\tmphrule
  &$\shr\Cal N_{\lambda}$\cr
&\tmpstrut&\vrule&&\vrule&&\vrule&&\vrule\cr
$\tmpstrut\omega_1$&\tmphrule&$\add\Cal N_{\lambda}$&\tmphrule
  &$\add\Cal N_{\kappa}$&\tmphrule
  &$\non\Cal N_{\kappa}$&\tmphrule&$\non\Cal N_{\lambda}$\cr
}}$$

\noindent (As in 522B, the cardinals here increase from bottom
left to top right.)

\proof{ For the inequalities relating two cardinals associated with the
same ideal, see 511Jc;  all we need to know is that $\Cal N_{\kappa}$
and $\Cal N_{\lambda}$ are proper ideals containing singletons.   For
the inequalities relating the cardinal functions of the two different
ideals, use 521H;  $\nu_{\kappa}$ is the image of $\nu_{\lambda}$ under
the map $x\mapsto x\restr\kappa:\{0,1\}^{\lambda}\to\{0,1\}^{\kappa}$,
by 254Oa.   Of course $\omega_1\le\add\Cal N_{\lambda}$.   I leave the
final inequality $\cf\Cal N_{\lambda}\le\lambda^{\omega}$ for the
moment, since this will be part of Theorem 523N below.
}%end of proof of 523B

\leader{523C}{}\cmmnt{ In the next few paragraphs I will say what is
known about the cardinals here.   It will be convenient to begin with
two easy lemmas.

\medskip

\noindent}{\bf Lemma} Let $I$ be any set, and $\Cal J$ a family of
subsets of $I$ such that every countable subset of $I$ is included in
some member of $\Cal J$.   Then a subset $A$ of $\{0,1\}^I$ belongs to
$\Cal N_I$ iff there is some $J\in\Cal J$ such that
$\{x\restr J:x\in A\}\in\Cal N_J$.

\proof{ For $J\subseteq I$, $x\in\{0,1\}^I$ set
$\pi_J(x)=x\restr J\in\{0,1\}^J$.
Then $\nu_J$ is the image measure $\nu_I\pi_J^{-1}$ (254Oa),
so $A\in\Cal N_I$ whenever there
is some $J\in\Cal J$ such that $\pi_J[A]\in\Cal N_J$.   On the other
hand, if $A\in\Cal N_I$, there is a countable set $K\subseteq I$ such
that $\pi_K[A]\in\Cal N_K$ (254Od).   Now there is a $J\in\Cal J$ such
that $K\subseteq J$, so that
$\pi_J^{-1}[\pi_J[A]]\subseteq\pi_K^{-1}[\pi_K[A]]\in\Cal N_I$ and
$\pi_J[A]\in\Cal N_J$.
}%end of proof of 523C

\leader{523D}{}\cmmnt{ Because the measures $\nu_I$ are homogeneous in a
strong sense, we have the following facts which are occasionally useful.

\medskip

\noindent}{\bf Proposition} Let $\kappa$ be an infinite cardinal, and
$\Tau$ the domain of $\nu_{\kappa}$.
For $A\subseteq\{0,1\}^{\kappa}$ write $\Tau_A$ for the subspace
$\sigma$-algebra on $A$.

(a) If $E\subseteq\{0,1\}^{\kappa}$ is measurable and not negligible,
then $(E,\Tau_E,\Cal N_{\kappa}\cap\Cal PE)$ is
isomorphic to $(\{0,1\}^{\kappa},\Tau,\Cal N_{\kappa})$.

(b) If $\Cal E\subseteq\Cal N_{\kappa}$ and
$\#(\Cal E)<\cov\Cal N_{\kappa}$, then $(\nu_{\kappa})_*(\bigcup\Cal E)=0$.

(c) If $A\subseteq\{0,1\}^{\kappa}$ is non-negligible,
then there is a set $B\subseteq\{0,1\}^{\kappa}$, of full outer measure,
such that $(A,\Tau_A,\Cal N_{\kappa}\cap\Cal PA)$ is
isomorphic to $(B,\Tau_B,\Cal N_{\kappa}\cap\Cal PB)$.

(d) There is a set
$A\subseteq\{0,1\}^{\kappa}$ with cardinal $\non\Cal N_{\kappa}$ which has
full outer measure.
%old 523Ge

\proof{{\bf (a)} In fact the subspace measure on $E$ is isomorphic to
a scalar multiple of $\nu_I$ (344L). 

\medskip

{\bf (b)} \Quer\ Otherwise, let $F\subseteq\bigcup\Cal E$ be a
non-negligible measurable set;  then $\{F\cap E:E\in\Cal E\}$ witnesses
that $\cov(F,\Cal N_{\kappa}\cap\Cal PF)<\cov\Cal N_{\kappa}$, which
contradicts (a).\ \Bang

\medskip

{\bf (c)} Let $E$ be a measurable envelope of $A$.   By (a), there is a
bijection $f:E\to\{0,1\}^{\kappa}$ which is an isomorphism of the
structures $(E,\Tau_E,\Cal N_{\kappa}\cap\Cal PE)$ and
$(\{0,1\}^{\kappa},\Tau,\Cal N_{\kappa})$.   Set $B=f[A]$.
Then $f\restr A$ is an isomorphism of the structures
$(A,\Tau_A,\Cal N_{\kappa}\cap\Cal PA)$ and
$(B,\Tau_B,\Cal N_{\kappa}\cap\Cal PB)$.   Moreover, since $A$ meets every
member of $\Tau_E\setminus\Cal N_{\kappa}$, $B$ meets every member of
$\Tau\setminus\Cal N_{\kappa}$, that is, $B$ has full outer measure.

\medskip

{\bf (d)} Let $A_0\subseteq\{0,1\}^{\kappa}$ be a non-negligible set of
cardinal $\non\Cal N_{\kappa}$.   By (c), there is a set $A$ of full outer
measure which is isomorphic to $A_0$ in the sense described there;  in
particular, $\#(A)=\non\Cal N_{\kappa}$.
}%end of proof of 523D

\leader{523E}{Additivities}\cmmnt{ Because the function
$\kappa\mapsto\add\Cal N_{\kappa}$ is non-increasing, it must stabilize,
that is, there is some first $\kappa_a$ such that
$\add\Cal N_{\kappa}=\add\Cal N_{\kappa_a}$ for every
$\kappa\ge\kappa_a$.   But
in fact it stabilizes almost immediately.}   If $\kappa$ is any
uncountable cardinal, then
$\add\Cal N_{\kappa}=\add\nu_{\kappa}=\omega_1$\prooflet{, by 521Jb}.
\cmmnt{ Thus among the additivities $\add\Cal N_{\kappa}$, only
$\add\Cal N_{\omega}=\add\Cal N$, the additivity of Lebesgue
measure, can have any surprises for us.}

\leader{523F}{Covering numbers} Still on the left-hand side of
the diagram, we\cmmnt{ again} have a non-increasing function
$\kappa\mapsto\cov\Cal N_{\kappa}$, and a critical value $\kappa_c$
after which it is constant.   \cmmnt{We can locate this value to some
extent through the following simple fact.  If
$\theta\dvro{}{\mskip5mu=\cov\Cal N_{\kappa_c}}
=\min\{\cov\Cal N_{\kappa}:\kappa$ is a cardinal$\}$, then
$\cov\Cal N_{\theta}=\theta$.   \prooflet{\Prf\ Let $\kappa$ be such
that $\cov\Cal N_{\kappa}=\theta$.   For $I\subseteq\kappa$, set
$\pi_I(x)=x\restr I$ for $x\in\{0,1\}^{\kappa}$.   Let
$\Cal E\subseteq\Cal N_{\kappa}$ be a cover of $\{0,1\}^{\kappa}$ of
cardinality $\theta$.   For each $E\in\Cal E$, let $J_E\subseteq\kappa$
be a countable set such that $\pi_{J_E}[E]\in\Cal N_{J_E}$.   Set
$I=\bigcup_{E\in\Cal E}J_E$,
so that $\#(I)\le\theta$ and $\pi_I[E]\in\Cal N_I$ for every
$E\in\Cal E$.   Then
$\{\pi_I[E]:E\in\Cal E\}$ is a cover of $\{0,1\}^I$ by at most
$\cov\Cal N_{\kappa}$ sets, and
$\cov\Cal N_I\le\cov\Cal N_{\kappa}$.   Since $(\{0,1\}^I,\Cal N_I)$ is
isomorphic to $(\{0,1\}^{\#(I)},\Cal N_{\#(I)})$, we also have

\Centerline{$\cov\Cal N_{\theta}\le\cov\Cal N_{\#(I)}
\le\cov\Cal N_{\kappa}\le\cov\Cal N_{\theta}$,}

\noindent and $\cov\Cal N_{\theta}=\cov\Cal N_{\kappa}=\theta$.\ \Qed}

What this means is that}

\Centerline{$\omega\le\kappa_c\le\cov\Cal N_{\kappa_c}
\le\cov\Cal N_{\omega}=\cov\Cal N\le\frak c$.}

\noindent Another way of putting the same idea is to say that

\Centerline{if $\lambda$ is a cardinal such that
$\cov\Cal N_{\lambda}\ge\lambda$ then $\cov\Cal N_{\kappa}\ge\lambda$ for
every $\kappa$\dvro{.}{}}

\prooflet{\noindent (since $\theta\ge\lambda$).}

%See {\smc Brendle 06},
%Theorem 3:  Con($\frak c=\omega_2$ or more, $\clubsuit$
%  + $\cov\Cal N_{\omega} = \frak c$ with a Sierpinski set?
% now $\stick$ already gives $\cov\Cal N_{\omega_1}=\omega_1$

%Theorem 5.8??
% consistency of  \cov\Cal N=\cov\Cal N_{\omega_1}=\frak c  large
%    and          \cov\Cal N_{\omega_2}=\omega_2

\leader{523G}{}\cmmnt{ When the additivity of Lebesgue measure is
large we have a further constraint on covering numbers.

\medskip

\noindent}{\bf Proposition}\cmmnt{ ({\smc Kraszewski 01})} If
$\kappa$ is an infinite cardinal and $\cov\Cal N_{\kappa}<\add\Cal N$,
then $\cov\Cal N_{\kappa}\le\cff[\kappa]^{\le\omega}$.

\proof{ Let $\Cal E$ be a subset of $\Cal N_{\kappa}$ of size
$\cov\Cal N_{\kappa}$ with union $\{0,1\}^{\kappa}$, and $\Cal J$ a
cofinal subset of $[\kappa]^{\omega}$ of size
$\cff[\kappa]^{\le\omega}$.   For $J\in\Cal J$ and $x\in\{0,1\}^{\kappa}$
set $\pi_J(x)=x\restr J$, so that $\pi_J:\{0,1\}^{\kappa}\to\{0,1\}^J$
is \imp.   For $J\in\Cal J$ set
$\Cal E_J=\{E:E\in\Cal E$, $\pi_J[E]\in\Cal N_J\}$,
$H_J=\bigcup\Cal E_J$.   Since

\Centerline{$\#(\Cal E_J)\le\#(\Cal E)
=\cov\Cal N_{\kappa}<\add\Cal N=\add\Cal N_{\omega}=\add\Cal N_J$,}

\noindent $F_J=\bigcup\{\pi_J[E]:E\in\Cal E_J\}\in\Cal N_J$ and
$H_J\subseteq\pi_J^{-1}[F_J]\in\Cal N_{\kappa}$.   Since
$\bigcup_{J\in\Cal J}\Cal E_J=\Cal E$ (523C) covers $\{0,1\}^{\kappa}$,
$\{H_J:J\in\Cal J\}$ covers $\{0,1\}^{\kappa}$ and
$\cov\Cal N_{\kappa}\le\#(\Cal J)=\cff[\kappa]^{\le\omega}$.
}%end of proof of 523G

%Question:  can we have  \add\Cal N_{\omega}>\omega_1  but  Stick ?

\leader{523H}{\dvrocolon{Uniformities}}\cmmnt{ On the other side of the
diagram we have non-decreasing functions.   To get upper bounds for
$\non\Cal N_{\kappa}$ we have the following method.

\medskip

\noindent}{\bf Lemma}\cmmnt{ ({\smc Kraszewski 01})} Suppose that
$I$ is a set and $F$ a family of functions
with domain $I$ such that for every countable $J\subseteq I$ there
is an $f\in F$ such that $f\restr J$ is injective.   Then

\Centerline{$\non\Cal N_I
\le\max(\#(F),\sup_{f\in F}\non\Cal N_{f[I]})$.}

\proof{ If $I$ is finite the result is trivial.
Otherwise, for each $f\in F$ take a non-negligible subset $A_f$ of
$\{0,1\}^{f[I]}$ of size $\non\Cal N_{f[I]}$.   For $y\in A_f$
set $x_{fy}=yf\in\{0,1\}^I$.    Set
$A=\{x_{fy}:f\in F,\,y\in A_f\}$.   \Quer\ If
$A\in\Cal N_I$, there is a countable set $J\subseteq I$
such that $\{x\restr J:x\in A\}\in\Cal N_J$.   Let $f\in F$ be such that
$f\restr J$ is injective.   Then we have a
function $\phi:\{0,1\}^{f[I]}\to\{0,1\}^J$ defined by saying that
$\phi(z)=zf\restr J$ for every $z\in\{0,1\}^{f[I]}$, and (because
$f\restr J$ is injective) $\phi$ is
\imp\ for $\nu_{f[I]}$ and $\nu_J$, so $\phi[A_f]$ cannot be
$\nu_J$-negligible.   But if $y\in A_f$ then
$\phi(y)(\xi)=y(f(\xi))=x_{fy}(\xi)$ for every $\xi\in J$, so
$\phi[A_f]\subseteq\{x\restr J:x\in A\}$, which is supposed to be
negligible.\ \Bang

Thus $A$ is not negligible, and

\Centerline{$\non\Cal N_I\le\#(A)
\le\max(\omega,\#(F),\sup_{f\in F}\#(A_f))
=\max(\#(F),\sup_{f\in F}\non\Cal N_{f[I]})$}

\noindent because we are supposing that $I$ is infinite, so there is
some $f\in F$ such that $f[I]$ is infinite.
}%end of proof of 523H

\leader{523I}{Theorem}\dvArevised{2010}
(a) For any cardinal $\kappa$,

\quad(i) $\non\Cal N_{\kappa}
  \le\max(\non\Cal N,\cff[\kappa]^{\le\omega})$,

\quad(ii) $\non\Cal N_{2^{\kappa}}
\le\max(\frak c,\cff[\kappa]^{\le\omega})$,

\def\tinyplus{{\vrule height 1.75pt depth -1.45pt width 2.5pt}
\hskip-1.3pt{\vrule height 2.5pt depth 0pt width 0.3pt}}

\quad(iii)
$\non\Cal N_{2^{\kappa^{\tinyplus}}}
\le\max(\kappa^+,\non\Cal N_{2^{\kappa}})$.

(b) If $\cf\kappa>\omega$, then
$\non\Cal N_{\kappa^+}
\le\max(\cf\kappa,\sup_{\lambda<\kappa}\non\Cal N_{\lambda})$.

\proof{{\bf (a)(i)} Let $\Cal J\subseteq[\kappa]^{\le\omega}$ be a cofinal set
of size $\cff[\kappa]^{\le\omega}$, and for $J\in\Cal J$ let
$f_J$ be the identity function on $J$.   Applying 523H with
$F=\{f_J:J\in\Cal J\}$ we get

$$\eqalign{\non\Cal N_{\kappa}
&\le\max(\#(\Cal J),\sup_{J\in\Cal J}\non\Cal N_J)\cr
&\le\max(\non\Cal N_{\omega},\cff[\kappa]^{\le\omega})
=\max(\non\Cal N,\cff[\kappa]^{\le\omega}).\cr}$$

\medskip

\quad{\bf (ii)} Take $\Cal J$ as in (i).
This time, for $J\in\Cal J$, define
$f_J:\Cal P\kappa\to\Cal PJ$ by setting $f_J(A)=A\cap J$ for every
$A\subseteq\kappa$.   Applying 523H with $F=\{f_J:J\in\Cal J\}$ we get

$$\eqalign{\non\Cal N_{2^{\kappa}}
&=\non\Cal N_{\Cal P\kappa}
\le\max(\#(\Cal J),\sup_{J\in\Cal J}\non\Cal N_{\Cal PJ})
\le\max(\cff[\kappa]^{\le\omega},\non\Cal N_{\frak c})\cr
&\le\max(\cff[\kappa]^{\le\omega},\non\Cal N,\cff[\frak c]^{\le\omega})
=\max(\cff[\kappa]^{\le\omega},\frak c)\cr}$$

\noindent (5A1E(c-ii)).

\medskip

\quad{\bf (iii)} Set $f_{\xi}(A)=A\cap\xi$ for $\xi<\kappa^+$ and
$A\subseteq\kappa^+$.   If
$\Cal J\subseteq\Cal P\kappa^+$ is countable, there is a $\xi<\kappa^+$
such that $A\cap\xi\ne A'\cap\xi$ for all distinct $A$, $A'\in\Cal J$,
that is, $f_{\xi}\restr\Cal J$ is injective.   So 523H tells us that

$$\eqalign{\non\Cal N_{2^{\kappa^{\tinyplus}}}
&=\non\Cal N_{\Cal P(\kappa^+)}
\le\max(\kappa^+,\sup_{\xi<\kappa^+}\non\Cal N_{f_{\xi}[\kappa^+]})\cr
&\le\max(\kappa^+,\sup_{\xi<\kappa^+}\non\Cal N_{\Cal P\xi})
=\max(\kappa^+,\non\Cal N_{2^{\kappa}}).\cr}$$

\medskip

{\bf (b)(i)} If $\kappa=\theta^+$ then
$\non\Cal N_{\kappa}\le\max(\kappa,\non\Cal N_{\theta})$.
\Prf\ For each $\xi<\kappa$ let $f_{\xi}:\kappa\to\theta$ be a function
which is injective on $\xi$, and set $F=\{f_{\xi}:\xi<\kappa\}$.   By 523H,

\Centerline{$\non\Cal N_{\kappa}
\le\max(\kappa,\sup_{\xi<\kappa}\non\Cal N_{f_{\xi}[\kappa]})
\le\max(\kappa,\non\Cal N_{\theta})$.  \Qed}

\noindent In fact
$\non\Cal N_{\kappa^+}\le\max(\kappa,\non\Cal N_{\theta})$.   \Prf\
Choose an injective function $h_{\zeta}:\zeta\to\kappa$ for each
$\zeta<\kappa^+$.   For $\xi<\kappa$ define
$f_{\xi}:\kappa^+\to\kappa$ by saying that

\Centerline{$f_{\xi}(\zeta)
=\min(\kappa\setminus\{f_{\xi}(\eta):
  \eta<\zeta$, $h_{\zeta}(\eta)\le\xi\})$}

\noindent for $\zeta<\kappa^+$.   If
$J\subseteq\kappa^+$ is countable, then
$\xi=\sup_{\eta,\zeta\in J,\eta<\zeta}h_{\zeta}(\eta)$ is less than
$\kappa$, and
$f_{\xi}(\eta)\ne f_{\xi}(\zeta)$ for all distinct $\eta$, $\zeta\in J$.
Applying 523H with $F=\{f_{\xi}:\xi<\kappa\}$, we get

\Centerline{$\non\Cal N_{\kappa^+}
\le\max(\kappa,\non\Cal N_{\kappa})
\le\max(\kappa,\non\Cal N_{\theta})$.  \Qed}

So $\non\Cal N_{\kappa^+}
\le\max(\cf\kappa,\sup_{\lambda<\kappa}\non\Cal N_{\lambda})$ if $\kappa$
is an infinite successor cardinal.

\medskip

\quad{\bf (ii)} Now suppose that $\kappa$ is an uncountable limit cardinal
with uncountable cofinality.   Again choose an injective function
$h_{\zeta}:\zeta\to\kappa$ for each $\zeta<\kappa^+$.
This time, let $K\subseteq\kappa$ be a cofinal set of size
$\cf\kappa$ consisting of cardinals, and for $\lambda\in K$ define
$f_{\lambda}:\kappa^+\to\lambda^+$ by the formula

\Centerline{$f_{\lambda}(\zeta)
=\min\{\lambda^+\setminus\{f_{\lambda}(\eta):
  \eta<\zeta$, $h_{\zeta}(\eta)\le\lambda\})$}

\noindent for $\zeta<\kappa^+$.   If
$J\subseteq\kappa^+$ is countable, then there is a $\lambda\in K$ such that
$\lambda\ge\sup_{\eta,\zeta\in J,\eta<\zeta}h_{\zeta}(\eta)$, and
$f_{\lambda}(\eta)\ne f_{\lambda}(\zeta)$ for all distinct
$\eta$, $\zeta\in J$.
Applying 523H with $F=\{f_{\lambda}:\lambda\in K\}$, we get

\Centerline{$\non\Cal N_{\kappa^+}
\le\max(\#(F),\sup_{f\in F}\non\Cal N_{f[\kappa^+]})
\le\max(\cf\kappa,\sup_{\lambda<\kappa}\non\Cal N_{\lambda})$.}
}%end of proof of 523I

\leader{523J}{Corollary}\cmmnt{ ({\smc Kraszewski 01})}
(a) $\non\Cal N_{\omega_2}=\non\Cal N_{\omega_1}=\non\Cal N$.

(b) For any $n\in\Bbb N$,
$\non\Cal N_{\omega_{n+1}}\le\max(\omega_n,\non\Cal N)$.

(c) $\non\Cal N_{2^{\omega_1}}=\non\Cal N_{\frakc}$.

(d) If $n\in\Bbb N$ then
$\non\Cal N_{2^{\omega_n}}\le\max(\omega_n,\non\Cal N_{\frakc})$.

\proof{{\bf (a)} We have

$$\eqalignno{\non\Cal N&=\non\Cal N_{\omega}
\le\non\Cal N_{\omega_1}\le\non\Cal N_{\omega_2}\cr
\displaycause{523B}
&\le\max(\omega_1,\non\Cal N_{\omega})\cr
\displaycause{523Ib}
&=\non\Cal N.\cr}$$

\medskip

{\bf (b)} Induce on $n$, using 523Ib for the inductive step.

\medskip

{\bf (c)} By 523I(a-iii),
$\non\Cal N_{2^{\omega_1}}\le\max(\omega_1,\non\Cal N_{\frakc})$;
since

\Centerline{$\omega_1\le\non\Cal N_{\frakc}
\le\non\Cal N_{2^{\omega_1}}$,}

\noindent we have the result.

\medskip

{\bf (d)} Induce on $n$, using 523I(a-iii) for the inductive step.
}%end of proof of 523J

\leader{523K}{Corollary}\cmmnt{ ({\smc Burke n05})}
For any sets $I$, $K$ let
$\Upsilon_{\omega}(I,K)$ be the least cardinal of any family $F$ of
functions from $I$ to $K$ such that for every countable $J\subseteq I$
there is an $f\in F$ which is injective on $J$.   (If
$\#(K)<\min(\omega,\#(I))$ take $\Upsilon_{\omega}(I,K)=\infty$.)   Then

(a) $\non\Cal N_I\le\max(\Upsilon_{\omega}(I,K),\non\Cal N_K)$ for all sets
$I$ and $K$;

(b) if $\kappa\ge\frak c$ is a cardinal, then
$\non\Cal N_{\kappa}
=\max(\Upsilon_{\omega}(\kappa,\frak c),\non\Cal N_{\frak c})$.

\proof{{\bf (a)} This is just a slightly weaker version of 523H.

\medskip

{\bf (b)} The point is that
$\Upsilon_{\omega}(\kappa,\frak c)\le\non\Cal N_{\kappa}$.   \Prf\
Let $A\subseteq\{0,1\}^{\kappa\times\omega}$ be a non-negligible set of
cardinal $\non\Cal N_{\kappa\times\omega}$.
For $x\in\{0,1\}^{\kappa\times\omega}$ define
$f_x:\kappa\to\{0,1\}^{\omega}$ by
setting $f_x(\xi)=\sequencen{x(\xi,n)}$ for each $\xi<\kappa$.
If $\xi$, $\eta<\kappa$ are distinct, then $\{x:f_x(\xi)=f_x(\eta)\}$ is
negligible, so if $J\subseteq\kappa$ is countable then
$\{x:f_x\restr J$ is injective$\}$ is conegligible and meets $A$.
Accordingly $\{f_x:x\in A\}$ witnesses that

\Centerline{$\Upsilon_{\omega}(\kappa,\frak c)
=\Upsilon_{\omega}(\kappa,\{0,1\}^{\omega})
\le\#(A)=\non\Cal N_{\kappa\times\omega}
=\non\Cal N_{\kappa}$.  \Qed}

Since we already know that

\Centerline{$\non\Cal N_{\frak c}\le\non\Cal N_{\kappa}
\le\max(\Upsilon_{\omega}(\kappa,\frak c),\non\Cal N_{\frak c})$,}

\noindent we have the result.
}%end of proof of 523K

\leader{523L}{}\cmmnt{ On the other side we can find lower bounds which
give a notion of the rate of growth of the numbers $\non\Cal N_{\kappa}$
as $\kappa$ increases.

\medskip

\noindent}{\bf Proposition} (a) If $\lambda$ and $\kappa$ are
infinite cardinals with $\kappa>2^{\lambda}$, then
$\non\Cal N_{\kappa}>\lambda$.
%old 523Gc

(b) If $\kappa$ is a strong limit cardinal of countable
cofinality then $\non\Cal N_{\kappa}>\kappa$.
%old 523Gd

\proof{{\bf (a)} Let
$A\subseteq\{0,1\}^{\kappa}$ be any set of size at most $\lambda$.   For
$\xi<\kappa$ set $B_{\xi}=\{x:x\in A,\,x(\xi)=1\}$.   Because
$\kappa>2^{\#(A)}$, there is some $B\subseteq A$ such that
$I=\{\xi:B_{\xi}=B\}$ is infinite.    But what this means is that if
$\xi\in I$ then $x(\xi)=1$ for every $\xi\in B$ and $x(\xi)=0$ for every
$x\in A\setminus B$, and
$A\subseteq\{x:x$ is constant on $I\}$ is negligible.   As $A$ is
arbitrary, $\non\Cal N_{\kappa}>\lambda$.

\medskip

{\bf (d)} By (a),
$\non\Cal N_{\kappa}>\lambda$ for every $\lambda<\kappa$, so
$\non\Cal N_{\kappa}\ge\kappa$;  but also
$\cf(\non\Cal N_{\kappa})\ge\add\Cal N_{\kappa}$ (513C(b-ii)), so
$\non\Cal N_{\kappa}$ has uncountable cofinality and must be greater
than $\kappa$.
}%end of proof of 523L

\leader{523M}{Shrinking \dvrocolon{numbers}}\cmmnt{ As with
$\non\Cal N_{\ssbullet}$, the functions
$\kappa\mapsto\shr\Cal N_{\kappa}$ and $\kappa\mapsto\shr\Cal N_{\kappa}$
are non-decreasing, by 521Hb.
Some of the ideas used in 523I can be adapted to this context, but the
pattern as a whole is rather different.

\medskip

\noindent}{\bf Proposition} (a)(i) For any non-zero
cardinals $\kappa$ and $\lambda$,

\Centerline{$\shr\Cal N_{\kappa}
\le\max(\covSh(\kappa,\lambda,\omega_1,2),
  \sup_{\theta<\lambda}\shr\Cal N_{\theta})$.}

\quad(ii) For any infinite cardinal $\kappa$,
$\shr\Cal N_{\kappa}
\le\max(\shr\Cal N,\cff[\kappa]^{\le\omega})$.

\quad(iii) If $\cf\kappa>\omega$, then
$\shr\Cal N_{\kappa}
\le\max(\kappa,\sup_{\theta<\kappa}\shr\Cal N_{\theta})$.

(b) For any infinite cardinal $\kappa$,

\quad(i) $\shr\Cal N_{\kappa}\ge\kappa$;

\quad(ii) $\cf(\shr\Cal N_{\kappa})>\omega$;

\quad(iii) $\cf(\shr^+\Cal N_{\kappa})>\kappa$.

\cmmnt{\medskip

\noindent{\bf Remark} For the definition of $\covSh$, see 5A2Da.}

\proof{{\bf (a)(i)}
If $\covSh(\kappa,\lambda,\omega_1,2)=\infty$ or $\kappa$ is finite
this is trivial.   Otherwise, $\lambda\ge\omega_1$.
Take a non-negligible $A\subseteq\{0,1\}^{\kappa}$.
Let $\Cal J\subseteq[\kappa]^{<\lambda}$ be a
set of size $\covSh(\kappa,\lambda,\omega_1,2)$ such that for every
$I\in[\kappa]^{<\omega_1}$ there is a $\Cal D\in[\Cal J]^{<2}$ such that
$I\subseteq\bigcup\Cal D$, that is, there is a $J\in\Cal J$ such that
$I\subseteq J$.   For each $J\in\Cal J$, $A_J=\{x\restr J:x\in A\}$ is
non-negligible;  let $B_J\subseteq A_J$ be a non-negligible
set of size at most $\shr\Cal N_J$.   Let $B\subseteq A$ be a set of size
at most $\max(\omega,\#(\Cal J),\sup_{J\in\Cal J}\shr\Cal N_J)$ such that
$B_J\subseteq\{x\restr J:x\in B\}$ for every $J\in\Cal J$.   If
$I\subseteq\kappa$ is countable, there is a $J\in\Cal J$ such that
$I\subseteq J$, so $\{x\restr I:x\in B\}\supseteq\{y\restr I:y\in B_J\}$ is
non-negligible;  it follows that $B$ is non-negligible, while
$\#(B)\le\max(\covSh(\kappa,\lambda,\omega_1,2),
  \sup_{\theta<\lambda}\shr\Cal N_{\theta})$.

\medskip

\quad{\bf (ii)} Taking $\lambda=\omega_1$ in (i),

\Centerline{$\shr\Cal N_{\kappa}
\le\max(\covSh(\kappa,\omega_1,\omega_1,2),\shr\Cal N_{\omega})
=\max(\cff[\kappa]^{\le\omega},\shr\Cal N)$.}

\medskip

\quad{\bf(iii)} Take $\lambda=\kappa$ in (i);  as
$[\kappa]^{\le\omega}=\bigcup_{\xi<\kappa}[\xi]^{\le\omega}$,

\Centerline{$\shr\Cal N_{\kappa}
\le\max(\covSh(\kappa,\kappa,\omega_1,2),
\sup_{\theta<\kappa}\shr\Cal N_{\theta})
=\max(\kappa,\sup_{\theta<\kappa}\shr\Cal N_{\theta})$.}

\medskip

{\bf (b)(i)} Induce on $\kappa$.   If $\kappa=\omega$
the result is trivial.   For the inductive step to $\kappa^+$, consider the
set

\Centerline{$A
=\{x:x\in\{0,1\}^{\kappa^+},\,\Exists\xi<\kappa^+,\,x(\eta)=0$ for every
$\eta\ge\xi\}$.}

\noindent Then the only set which includes $A$ and is determined by a
countable set of coordinates is $\{0,1\}^{\kappa^+}$, so $A$ has full
outer measure.   On the other hand, if $B\subseteq A$ and
$\#(B)\le\kappa$, then there is some $\zeta<\kappa^+$ such that
$x(\xi)=0$ for every $x\in B$ and every $\xi\ge\zeta$, so $B$ is
negligible.   Thus $A$ witnesses that
$\shr\Cal N_{\kappa^+}\ge\kappa^+$.   Because
$\kappa\mapsto\shr\Cal N_{\kappa}$ is non-decreasing (523B),
the inductive step
to limit cardinals $\kappa$ is trivial.

\medskip

\quad{\bf (ii)} \Quer\ Now suppose, if possible, that
$\cf(\shr\Cal N_{\kappa})=\omega$.
Then there is a sequence $\sequencen{\lambda_n}$ of cardinals
less than $\shr\Cal N_{\kappa}$ with supremum $\shr\Cal N_{\kappa}$.
For each $n\in\Bbb N$ set $I_n=\kappa\times\{n\}$, and let
$A_n\subseteq\{0,1\}^{I_n}$ be a
non-negligible set such that every non-negligible
subset of $A_n$ has more than $\lambda_n$ members.
By 523Dc, there is a set $B_n\subseteq\{0,1\}^{I_n}$ of full outer
measure such that every non-negligible
subset of $B_n$ has more than $\lambda_n$ members.   Set

\Centerline{$B
=\{x:x\in\{0,1\}^{\kappa\times\Bbb N}$, $x\restr I_n\in B_n$
for every $n\in\Bbb N\}$.}

\noindent Then the natural isomorphism between
$\{0,1\}^{\kappa\times\Bbb N}$ and $\prod_{n\in\Bbb N}\{0,1\}^{I_n}$
identifies $B$ with $\prod_{n\in\Bbb N}B_n$, so $B$ has full outer
measure in $\{0,1\}^{\kappa\times\Bbb N}$ (254Lb).   There must therefore
be a set $C\subseteq B$, of non-zero measure, such that
$\#(C)\le\shr\Cal N_{\kappa}$.   Express $C$ as
$\bigcup_{n\in\Bbb N}C_n$ where $\#(C_n)\le\lambda_n$ for every
$n$.   Then there is an $n\in\Bbb N$ such that $C_n$ is not negligible,
in which case $D_n=\{x\restr I_n:x\in C_n\}$ is non-negligible.   But
$D_n\subseteq B_n$ and $\#(D_n)\le\lambda_n$, so this is impossible.\ \Bang

\medskip

\quad{\bf (iii)} The argument of (i) shows that if $\kappa$ is a successor
cardinal, then $\shr^+\Cal N_{\kappa}>\kappa$.   So we need consider only
the case in which $\kappa$ is a limit cardinal.
\Quer\ If $\cf(\shr^+\Cal N_{\kappa})\le\kappa$, then
there is a family $\ofamily{\xi}{\kappa}{\lambda_{\xi}}$ of cardinals
less than $\shr^+\Cal N_{\kappa}$ with supremum $\shr^+\Cal N_{\kappa}$.
I use the same method as in (ii).
For each $\xi<\kappa$ set $I_{\xi}=\kappa\times\{\xi\}$, and let
$B_{\xi}\subseteq\{0,1\}^{I_{\xi}}$ be a
set of full outer measure such that every non-negligible
subset of $B_{\xi}$ has at least $\lambda_{\xi}$ members.  Set

\Centerline{$B
=\{x:x\in\{0,1\}^{\kappa\times\kappa}$, $x\restr I_{\xi}\in B_{\xi}$
for every $\xi<\kappa$.}

\noindent Then $B$ has full outer
measure in $\{0,1\}^{\kappa\times\kappa}$.   There must therefore
be a set $C\subseteq B$, of non-zero measure, such that
$\#(C)<\shr^+\Cal N_{\kappa}$.   Let $\xi<\kappa$ be such that
$\#(C)<\lambda_{\xi}$.   Then
$D=\{x\restr I_{\xi}:x\in C\}$ is non-negligible.   But
$D\subseteq B_{\xi}$ and $\#(D_{\xi})<\lambda_{\xi}$, so this is
impossible.\ \Bang

}%end of proof of 523M

\leader{523N}{\dvrocolon{Cofinalities}}\cmmnt{ For the cardinals
$\cf\Cal N_{\kappa}$ the pattern from 523I(a-i) and 523Mb continues, and
indeed we have an exact formula.

\medskip

\noindent}{\bf Theorem} For any infinite cardinal $\kappa$,

\Centerline{$\kappa\le\cf\Cal N_{\kappa}
=\max(\cf\Cal N,\cff[\kappa]^{\le\omega})
\le\kappa^{\omega}$.}

\proof{{\bf (a)} $\cf\Cal N_{\kappa}
\le\max(\cf\Cal N,\cff[\kappa]^{\le\omega})$.
\Prf\ Let $\Cal J$ be a
cofinal family in $[\kappa]^{\omega}$ with cardinal
$\cff[\kappa]^{\le\omega}$.   For each $J\in\Cal J$,
write $\pi_J(x)=x\restr J$ for $x\in\{0,1\}^{\kappa}$.   Let $\Cal E_J$
be a cofinal subset of
$\Cal N_J$ with cardinal $\cf\Cal N_J=\cf\Cal N_{\omega}=\cf\Cal N$.
Consider
$\Cal E=\{\pi_J^{-1}[E]:J\in\Cal J,\,E\in\Cal E_J\}$.   By 523C,
$\Cal E$ is cofinal with $\Cal N_{\kappa}$, so that

\Centerline{$\cf\Cal N_{\kappa}\le\#(\Cal E)
\le\max(\cf\Cal N,\cff[\kappa]^{\le\omega})$.
\Qed}

\medskip

{\bf (b)} We know that
$\cff[\kappa]^{\le\omega}\le\cf\Cal N_{\kappa}$ (521Jb) and that
$\cf\Cal N=\cf\Cal N_{\omega}\le\cf\Cal N_{\kappa}$ (523B).   So
$\cf\Cal N_{\kappa}=\max(\cf\Cal N,\cff[\kappa]^{\le\omega})$.

\medskip

{\bf (c)} For the inequalities, note that if $\kappa$ is uncountable
then

\Centerline{$\cff[\kappa]^{\le\omega}
\ge\cov(\kappa,[\kappa]^{\le\omega})=\kappa$.}

\noindent On the other side,
$\cf\Cal N\le\frak c\le\kappa^{\omega}$ and
$\cff[\kappa]^{\le\omega}\le\#([\kappa]^{\le\omega})=\kappa^{\omega}$.
}%end of proof of 523N

\cmmnt{
\leader{523O}{Cofinalities of the cardinals} In 523Mb I have
shown that $\shr\Cal N_{\kappa}$ has uncountable cofinality for infinite
$\kappa$, and rather more about $\shr^+\Cal N_{\kappa}$.
From 513Cb we have a little information concerning the cofinalities of
$\add\Cal N_{\kappa}$, $\cov\Cal N_{\kappa}$, $\non\Cal N_{\kappa}$ and
$\cf\Cal N_{\kappa}$;  but except when $\kappa=\omega$ we learn only
that $\cf\Cal N_{\kappa}$ and $\non\Cal N_{\kappa}$ have uncountable
cofinality, and that if $\cov\Cal N_{\kappa}=\cf\Cal N_{\kappa}$ then
their common cofinality is at least $\non\Cal N_{\kappa}$.   This last
remark can apply only to `small' $\kappa$, since
$\cf\Cal N_{\kappa}\ge\kappa$ (if $\kappa$ is infinite) and
$\cov\Cal N_{\kappa}\le\cov\Cal N$.}

\leader{523P}{The generalized continuum \dvrocolon{hypothesis}}\cmmnt{
In this chapter I am trying to present arguments in forms which show
their full strength and are not tied to particular axioms beyond those
of ZFC.   However it is perhaps worth mentioning that in one of the
standard universes the pattern is particularly simple.

\medskip

\noindent}{\bf Proposition} Suppose that the generalized continuum
hypothesis is true.   Then, for any infinite cardinal $\kappa$,

$$\eqalign{\add\Cal N_{\kappa}=\add\nu_{\kappa}
=\cov\Cal N_{\kappa}=\omega_1;\cr}$$

$$\eqalign{\non\Cal N_{\kappa}&=\lambda
  \text{ if }\kappa=\lambda^+\text{ where }\cf\lambda>\omega,\cr
&=\kappa^+\text{ if }\cf\kappa=\omega,\cr
&=\kappa\text{ otherwise};\cr}$$

$$\eqalign{\shr\Cal N_{\kappa}=\cf\Cal N_{\kappa}
&=\kappa^+\text{ if }\cf\kappa=\omega,\cr
&=\kappa\text{ otherwise};\cr}$$

$$\eqalign{\shr^+\Cal N_{\kappa}=(\shr\Cal N_{\kappa})^+
&=\kappa^{++}\text{ if }\cf\kappa=\omega,\cr
&=\kappa^+\text{ otherwise}.\cr}$$

\proof{ Since

\Centerline{$\omega_1\le\add\Cal N_{\kappa}=\add\nu_{\kappa}
\le\cov\Cal N_{\kappa}\le\cov\Cal N\le\frak c=\omega_1$,}

\noindent the additivity and covering number are always
$\omega_1$.

If $\cf\kappa=\omega$, then

\Centerline{$\kappa^+\le\shr\Cal N_{\kappa}
\le\cf\Cal N_{\kappa}=\max(\omega_1,\cff[\kappa]^{\le\omega})
\le 2^{\kappa}=\kappa^+$}

\noindent by 523M(b-ii) and 523N.   If $\cf\kappa>\omega$ then

\Centerline{$\kappa\le\shr\Cal N_{\kappa}\le\cf\Cal N_{\kappa}\le\kappa$}

\noindent by 523M(b-i), 523N and 5A6Ab.   This deals with
$\shr\Cal N_{\kappa}$ and $\cf\Cal N_{\kappa}$.   For the augmented
shrinking numbers, we know that
$\kappa<\shr^+\Cal N_{\kappa}\le(\shr\Cal N_{\kappa})^+$
(523M(b-iii)), with equality if $\shr\Cal N_{\kappa}$ is a successor
cardinal, so in the present case we must have
$\shr^+\Cal N_{\kappa}=(\shr\Cal N_{\kappa})^+$.

As for $\non\Cal N_{\kappa}$, if $\kappa=\lambda^+$ where
$\cf\lambda>\omega$, then $\kappa>2^{\theta}$ for every
$\theta<\lambda$, so we have

\Centerline{$\lambda\le\non\Cal N_{\kappa}
\le\max(\frak c,\cff[\lambda]^{\omega})=\lambda$}

\noindent (523I(a-ii), 523La, 5A6Ab).   If $\kappa=\lambda^+$ where
$\cf\lambda=\omega$, then

\Centerline{$\lambda\le\non\Cal N_{\kappa}
\le\lambda^{\omega}\le 2^{\lambda}=\kappa$;}

\noindent but as $\non\Cal N_{\kappa}$ has uncountable cofinality
(513C(b-ii) again), $\non\Cal N_{\kappa}$ must be $\kappa$.
If $\kappa$ is
a limit cardinal, then $\kappa>2^{\theta}$ for every $\theta<\kappa$, so

\Centerline{$\kappa\le\non\Cal N_{\kappa}
\le\max(\omega_1,\cff[\kappa]^{\le\omega})$;}

\noindent if $\cf\kappa>\omega$ this is already enough to show that
$\non\Cal N_{\kappa}=\kappa$;  if $\cf\kappa=\omega$ then
$\non\Cal N_{\kappa}$ cannot be $\kappa$ so must be
$\kappa^+=\kappa^{\omega}$.
}%end of proof of 523P

\exercises{\leader{523X}{Basic exercises (a)}
%\spheader 523Xa
Show that

\Centerline{$(\Cal N_{\kappa},\not\ni,\{0,1\}^{\kappa})
\prGT([\kappa]^{\le\omega},\subseteq,[\kappa]^{\le\omega})
  \ltimes(\Cal N,\not\ni,\Bbb R)$}

\noindent for every infinite cardinal $\kappa$.   (See 512I for the
definition of $\ltimes$.)   Use this to prove 523I(a-i).
%523I

\spheader 523Xb Let $\kappa$ be an infinite cardinal, and $\Cal J$ a
family of subsets of $\kappa$ such that every countable subset of
$\kappa$ is included in some member of $\Cal J$.   Show that
$\non\Cal N_{\kappa}
\le\max(\#(\Cal J),\sup_{J\in\Cal J}\non\Cal N_J)$,
$\non\Cal N_{\Cal P\kappa}
\le\max(\#(\Cal J),\sup_{J\in\Cal J}\non\Cal N_{\Cal PJ})$\dvAnew{2010},
$\shr\Cal N_{\kappa}
\le\max(\#(\Cal J),\penalty-100\sup_{J\in\Cal J}\shr\Cal N_J)$ and
$\cf\Cal N_{\kappa}
\le\max(\#(\Cal J),\penalty-100\sup_{J\in\Cal J}\cf\Cal N_J)$.
%523M

\spheader 523Xc Show that

\Centerline{$(\Cal N_{\kappa},\subseteq,\Cal N_{\kappa})
\prGT([\kappa]^{\le\omega},\subseteq,[\kappa]^{\le\omega})
\ltimes(\Cal N,\subseteq,\Cal N)$}

\noindent for every infinite cardinal $\kappa$.   Use this to prove
523N.
%523N  523Xa

\spheader 523Xd Let $(X,\Sigma,\mu)$ be any probability space, and for a
set $I$ write $\Cal N(\mu^I)$ for the null ideal of the product measure
on $X^I$.   Show that all the results of
523E-523J, %523E 523F 523G 523H 523I 523J
523L-523N %523L 523M 523N
are valid with
$\Cal N(\mu^I)$ in place of $\Cal N_I$ and $\Cal N(\mu^{\omega})$
in place of $\Cal N$, except that

----- in 523F we can no longer be sure that
$\cov\Cal N(\mu^{\omega})\le\frak c$;

----- in 523I(a-ii) we need to write
`$\non\Cal N(\mu^{2^{\kappa}})
\le\max(\frak c,\non\Cal N(\mu^{\omega}),\cff[\kappa]^{\le\omega})$';

----- in 523L and 523Mb
we have to assume that the measure algebra of $\mu$ is not
$\{0,1\}$, so that the product measure $\mu^{\Bbb N}$ is atomless;

----- in 523N we can no longer be sure that
$\cf\Cal N(\mu^{\omega})\le\kappa^{\omega}$.
%523N


\leader{523Y}{Further exercises (a)}
%\spheader 523Ya
Set $\frak A=\Cal P\Bbb R/\Cal N$.   (i) Show that
$\frak c\le c(\frak A)\le\pi(\frak A)\le 2^{\shr\Cal N}$.
(ii) Show that if
$\shr^+\Cal N\ge\frak c$ and $2^{\lambda}\le\frak c$ for every
$\lambda<\frak c$, then $c(\frak A)=2^{\frak c}$.
%523M

\spheader 523Yb Let $\kappa$ be an infinite cardinal.   Show that there is
a family $\Cal J\subseteq[\kappa]^{\le\omega}$ such that
$\#(\Cal J)\le\shr\Cal N_{\kappa}$ and every infinite subset of $\kappa$
meets some member of $\Cal J$ in an infinite set.
%523M mt52bits

\spheader 523Yc Suppose that $\kappa\ge\omega$ and that
$[\kappa]^{\le\omega}$ has bursting number at most $\add\Cal N$.   Show
that $\Cal N_{\kappa}\equivT[\kappa]^{\le\omega}\times\Cal N$.
\Hint{{\smc Fremlin 91.}.}
%523N

\spheader 523Yd Show that

\Centerline{$(\omega_1,\le,\omega_1)\ltimes(\omega_1,\le,\omega_1)
\not\preccurlyeq_{\text{GT}}
(\omega_1,\le,\omega_1)\times(\omega_1,\le,\omega_1)$.}
%523Xc 523Yc 523N mt52bits

\spheader 523Ye For infinite cardinals $\kappa$, write $\Cal M_{\kappa}$
for the ideal of meager subsets of $\{0,1\}^{\kappa}$.   Show that under
the same conventions as in 522B and 523B we have the diagrams

$$\vbox{\offinterlineskip
\halign{\hfil#\hfil&\hfil#\hfil&\hfil#\hfil&\hfil#\hfil
  &\hfil#\hfil&\hfil#\hfil&\hfil#\hfil
  &\hfil#\hfil&\hfil#\hfil&\hfil#\hfil
  &\hfil#\hfil&\hfil#\hfil&\hfil#\hfil&\hfil#\hfil\cr
&\tmpstrut&$\cov\Cal M_{\lambda}$&\tmphrule
  &$\cov\Cal M_{\kappa}$&\tmphrule
  &$\cf\Cal M_{\kappa}$&\tmphrule&$\cf\Cal M_{\lambda}$
  &\tmphrule&$\lambda^{\omega}$\cr
&\tmpstrut&\vrule&&\vrule&&\vrule&&\vrule\cr
&\tmpstrut&\vrule&&\vrule&&$\shr\Cal M_{\kappa}$&\tmphrule
  &$\shr\Cal M_{\lambda}$\cr
&\tmpstrut&\vrule&&\vrule&&\vrule&&\vrule\cr
$\tmpstrut\omega_1$&\tmphrule&$\add\Cal M_{\lambda}$&\tmphrule
  &$\add\Cal M_{\kappa}$&\tmphrule
  &$\non\Cal M_{\kappa}$&\tmphrule&$\non\Cal M_{\lambda}$\cr
}}$$

\noindent and

$$\vbox{\offinterlineskip
\halign{\hfil#\hfil&\hfil#\hfil&\hfil#\hfil&\hfil#\hfil
  &\hfil#\hfil&\hfil#\hfil&\hfil#\hfil
  &\hfil#\hfil&\hfil#\hfil&\hfil#\hfil&\hfil#\hfil\cr
&\tmpstrut&$\cov\Cal N_{\kappa}$&\tmphrule&$\non\Cal M_{\kappa}$
  &\tmphrule
  &$\cf\Cal M_{\kappa}$&\tmphrule&$\cf\Cal N_{\kappa}$
  &\tmphrule&$\kappa^{\omega}$\cr
&\tmpstrut&\vrule&&\vrule&&\vrule&&\vrule\cr
$\tmpstrut\omega_1$&\tmphrule&$\add\Cal N_{\kappa}$&\tmphrule
  &$\add\Cal M_{\kappa}$&\tmphrule
  &$\cov\Cal M_{\kappa}$&\tmphrule&$\non\Cal N_{\kappa}$\cr
}}$$

\noindent whenever $\omega\le\kappa\le\lambda$.   Show moreover that all
the results of
523E-523P %523E 523F 523G 523H 523I 523J 523L 523M 523N 523O 523P
have parallel forms referring to $\Cal M_{\kappa}$.
%523P

\spheader 523Yf In the language of 523Ye, show that (i)
$\frak m_{\text{pc}\omega_1}\le\cov\Cal M_{\kappa}$ for every infinite
$\kappa$ (ii) $\frak m_{\sigma\text{-linked}}\le\cov\Cal N_{\kappa}$ if
$\omega\le\kappa\le\frak c$.   \Hint{$\omega_1$ is a precaliber of
$\RO(\{0,1\}^{\kappa})$, and the measure algebra of $\nu_{\kappa}$
is $\sigma$-linked if $\kappa\le\frak c$.}
%523Ye 523P

\spheader 523Yg Show that Ostaszewski's $\clubsuit$ implies that
$\cov\Cal N_{\omega_1}=\cov\Cal M_{\omega_1}=\omega_1$.
%523Ye 523P
}%end of exercises

\leader{523Z}{Problem} Is there a proof in ZFC that
$\shr\Cal N_{\kappa}\ge\cff[\kappa]^{\le\omega}$ for every cardinal
$\kappa$?
%523M

\endnotes{
\Notesheader{523} The basic diagram 523B is natural and easy to
establish.   Of course it leaves a great deal of room, especially on the
right-hand side, where we have the increasing functions
$\non\Cal N_{\ssbullet}$, $\shr\Cal N_{\ssbullet}$ and
$\cf\Cal N_{\ssbullet}$, and rather weak constraints

\Centerline{$\lambda<\non\Cal N_{\kappa}\le\shr\Cal N_{\kappa}
\le\cf\Cal N_{\kappa}\le\kappa^{\omega}$ whenever $2^{\lambda}<\kappa$}

\noindent to control them.   However the generalized continuum
hypothesis is sufficient to determine exact values for all the cardinals
considered here (523P).

The combinatorics of $\cff[\kappa]^{\le\omega}$ and
almost-disjoint families of functions are extremely complex, and depend
in surprising ways on special axioms;  I think it possible that
the results of 523I-523J can be usefully extended.
%Kraszewski 3.18
However 523N at least reduces the measure-theoretic problem of
determining $\cf\Cal N_{\kappa}$ to a standard, if difficult, question
in infinitary combinatorics.   I do not know if there are corresponding
results concerning $\non\Cal N_{\kappa}$ and $\shr\Cal N_{\kappa}$ (see
523Kb and 523Z).

All the ideas in this section up to and including 523P can be applied to
ideals of meager sets (523Ye) and indeed to other classes of ideals
satisfying the fundamental lemma 523C; see {\smc Kraszewski 01}.
}%end of notes

\discrpage

