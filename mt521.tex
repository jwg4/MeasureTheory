\frfilename{mt521.tex}
\versiondate{3.3.14}
\copyrightdate{2007}

\def\chaptername{Cardinal functions of measure theory}
\def\sectionname{Basic theory}

\def\magnitude{\mathop{\text{mag}}}

\newsection{521}

In the first half of this section (down to 521L) I collect facts
about the cardinal
functions $\add$, $\cf$, $\non$, $\cov$, $\shr$ and $\shr^+$ when applied
to the null ideal $\Cal N(\mu)$ of a measure $\mu$, and also the
$\pi$-weight of a measure.
In particular I look at their relations with the
constructions introduced earlier in this treatise:
measure algebras and function
spaces (521B), subspace measures (521F), direct sums (521G), \imp\
functions and image measures (521H), products (521J), perfect measures
(521K) and compact measures (521L).   The list is long just because I
have four volumes' worth of miscellaneous concepts to examine;  nearly all
the individual arguments are elementary.

In the second half of the section, I give
a handful of easy results which may clarify some
patterns from earlier volumes.   In 521M-521P %521M 521N 521O 521P
I look again at `strict localizability' as considered in Chapter 21,
importing the concept of `magnitude' of a measure space from \S332,
hoping to throw light on the examples of \S216.
In 521E I consider the topological densities of measure algebras.
In 521R-521S I
explore possibilities for the `countably separated' measure spaces of
\S\S343-344, examining in particular their Maharam types.   Finally,
in 521T, I review some measures which arose in \S464 while analyzing the
$L$-space $\ell^{\infty}(I)^*$.

\leader{521A}{Proposition} Let $(X,\Sigma,\mu)$ be a measure space.

(a) If $\Cal E\subseteq\Sigma$ and $\#(\Cal E)<\add\mu$ then
$\bigcup\Cal E\in\Sigma$ and

\Centerline{$\mu(\bigcup\Cal E)
=\sup\{\mu(\bigcup\Cal E_0):\Cal E_0\subseteq\Cal E$ is finite$\}$.}

(b) $\omega_1\le\add\mu\le\add\Cal N(\mu)$.

(c) If $\mu$ is the measure defined by \Caratheodory's method from
an outer measure $\theta$ on $X$, then $\add\mu=\add\Cal N(\mu)$.

(d) If $\mu$ is complete and locally determined,
$\add\mu=\add\Cal N(\mu)$.

\proof{{\bf (a)} Induce on $\#(\Cal E)$.   If $\Cal E$ is finite, the
result is trivial.   For the inductive step to
$\#(\Cal E)=\kappa\ge\omega$, enumerate $\Cal E$ as
$\ofamily{\xi}{\kappa}{E_{\xi}}$.   For each $\xi<\kappa$, set
$H_{\xi}=E_{\xi}\setminus\bigcup_{\eta<\xi}E_{\eta}$ for each
$\xi<\kappa$.   Then the inductive hypothesis tells us that
$H_{\xi}\in\Sigma$ for every $\xi$.   Set $E=\bigcup\Cal
E=\bigcup_{\xi<\kappa}H_{\xi}$;  because
$\ofamily{\xi}{\kappa}{H_{\xi}}$ is
disjoint, and $\kappa<\add\mu$, $E\in\Sigma$ and

$$\mu E
=\sum_{\xi<\kappa}\mu H_{\xi}
=\sup_{I\subseteq\kappa\text{ is finite}}
  \mu(\bigcup_{\xi\in I}H_{\xi})
\le\sup_{\Cal E_0\subseteq\Cal E\text{ is finite}}\mu(\bigcup\Cal E)
\le\mu E.$$

\medskip

{\bf (b)} By the definition of `measure' (112A), $\mu$ is
$\omega_1$-additive.   Suppose that $\Cal A\subseteq\Cal N(\mu)$ and
$\#(\Cal A)<\add\mu$.   For each $A\in\Cal A$, choose a measurable
negligible $E_A\supseteq A$.   Then (a) tells us that
$E=\bigcup_{A\in\Cal A}E_A$ has measure zero, so
$\bigcup\Cal A\subseteq E$ is negligible.   As $\Cal A$ is arbitrary,
$\add\Cal N(\mu)\ge\add\mu$.

\medskip

{\bf (c)} Now suppose that $\mu$ is defined by \Caratheodory's method
from $\theta$.
Let $\familyiI{E_i}$ be a disjoint family in $\Sigma$, where
$\#(I)<\add\Cal N(\mu)$, with union $E$.

Let $A\subseteq X$ be any set.   Then
$\theta(A\cap E)=\sum_{i\in I}\theta(A\cap E_i)$.   \Prf\ Of course

$$\eqalignno{\theta(A\cap E)
&\ge\sup_{J\subseteq I\text{ is finite}}
  \theta(A\cap\bigcup_{i\in J}E_i)
=\sup_{J\subseteq I\text{ is finite}}
  \sum_{i\in J}\theta(A\cap E_i)\cr
\displaycause{induce on $\#(J)$, using the fact that
$\theta B=\theta(B\cap E_i)+\theta(B\setminus E_i)$ for every
$B\subseteq X$ and $i\in J$}
&=\sum_{i\in I}\theta(A\cap E_i).\cr}$$

\noindent If $\sum_{i\in I}\theta(A\cap E_i)$ is infinite, we can stop.
Otherwise, recalling that $\Cal N(\mu)=\theta^{-1}[\{0\}]$,
$J=\{i:A\cap E_i\notin\Cal N(\mu)\}$ is countable,
and $\bigcup_{i\in I\setminus J}A\cap E_i$ is negligible, because
$\#(I)<\add\Cal N(\mu)$;  so

\Centerline{$\theta(A\cap E)=\theta(A\cap\bigcup_{i\in J}E_i)
\le\sum_{i\in J}\theta(A\cap E_i)=\sum_{i\in I}\theta(A\cap E_i)$}

\noindent and we have equality.\  \Qed

It follows that
$\theta(A\cap E)+\theta(A\setminus E)\le\theta A$.   \Prf\ For any finite
$J\subseteq I$,

$$\eqalign{\theta(A\setminus E)+\sum_{i\in J}\theta(A\cap E_i)
&=\theta(A\setminus E)+\theta(A\cap\bigcup_{i\in J}E_i)\cr
&\le\theta(A\setminus\bigcup_{i\in J}E_i)+\theta(A\cap\bigcup_{i\in J}E_i)
=\theta A.\cr}$$

\noindent Taking the supremum over $J$, we have the result.\ \Qed

As $A$ is arbitrary, $E\in\Sigma$;  and setting $A=E$, we see that
$\mu E=\sum_{i\in I}\mu E_i$.   As $\familyiI{E_i}$ is arbitrary,
$\add\mu\ge\add\Cal N(\mu)$ and the two additivities are equal.

\medskip

{\bf (d)} Now this follows immediately from (c), by 213C.
}%end of proof of 521A

\leader{521B}{Proposition} Let $(X,\Sigma,\mu)$ be a measure space
and $(\frak A,\bar\mu)$ its measure algebra.

(a) If $\Cal E\subseteq\Sigma$ and $\#(\Cal E)<\add\mu$, then
$(\bigcup\Cal E)^{\ssbullet}=\sup_{E\in\Cal E}E^{\ssbullet}$ and
$(X\cap\bigcap\Cal E)^{\ssbullet}=\inf_{E\in\Cal E}E^{\ssbullet}$
in $\frak A$.

(b) Suppose that $A\subseteq[-\infty,\infty]^X$ is a non-empty family of
$\Sigma$-measurable functions with $\#(A)<\add\mu$, and that
$g(x)=\sup_{f\in A}f(x)$ in $[-\infty,\infty]$ for every $f\in A$.
Then $g$ is $\Sigma$-measurable.

(c) Write $\eusm L^0$ for the family of $\mu$-virtually measurable
real-valued functions defined almost everywhere in $X$, and $L^0$ for
the corresponding space of equivalence classes, as in \S241.   Suppose
that $A\subseteq\eusm L^0$ is such that $0<\#(A)<\add\mu$ and
$\{f^{\ssbullet}:f\in A\}$ is bounded above in $L^0$.   Set
$g(x)=\sup_{f\in A}f(x)$ whenever this is defined in $\Bbb R$;  then
$g\in\eusm L^0$ and $g^{\ssbullet}=\sup_{f\in A}f^{\ssbullet}$ in $L^0$.

\allowmorestretch{500}{
(d)(i) If, in (b), $A$ consists of non-negative integrable functions and
is upwards-directed, then $\int g\,d\mu=\sup_{f\in A}\int fd\mu$.
}

\quad(ii) If, in (b), $f_1\wedge f_2=0$ a.e.\ for all distinct $f_1$,
$f_2\in A$, then $\int g\,d\mu=\sum_{f\in A}\int fd\mu$.

\proof{{\bf (a)} As in 521Aa, $\bigcup\Cal E\in\Sigma$, and of course
$(\bigcup\Cal E)^{\ssbullet}$ is an upper bound for
$\{E^{\ssbullet}:E\in\Cal E\}$.   If $F\in\Sigma$ and $F^{\ssbullet}$ is an
upper bound for $\{E^{\ssbullet}:E\in\Cal E\}$, then, applying 521Aa to
$\{E\setminus F:E\in\Cal E\}$, we see that $\bigcup\Cal E\setminus F$ is
negligible, so $(\bigcup\Cal E)^{\ssbullet}\Bsubseteq F^{\ssbullet}$.
Thus $(\bigcup\Cal E)^{\ssbullet}$ is the least upper bound of
$\{E^{\ssbullet}:E\in\Cal E\}$.

Applying this to $\{X\setminus E:E\in\Cal E\}$ we see that
$(X\cap\bigcap\Cal E)^{\ssbullet}=\inf_{E\in\Cal E}E^{\ssbullet}$.

\medskip

{\bf (b)} For any $\alpha\in\Bbb R$,

\Centerline{$\{x:g(x)>\alpha\}
=\bigcup_{f\in A}\{x:f(x)>\alpha\}\in\Sigma$}

\noindent by 521Aa.

\medskip

{\bf (c)} Take any $h\in\eusm L^0$ such that
$f^{\ssbullet}\le h^{\ssbullet}$ for every $f\in A$.   For each
$f\in A$, let $E_f$ be a conegligible measurable subset of
$\{x:x\in\dom f\cap\dom h$, $f(x)\le h(x)\}$ such that $f\restr E_f$ is
measurable.   Set $E=\bigcap_{f\in A}E_f$;  then $E$ is measurable and
$g$ is defined everywhere in $E$ and $g\restr E$ is measurable (as in
(b)).    Also $E$ is conegligible, so
$g\in\eusm L^0$, and of course $f^{\ssbullet}\le g^{\ssbullet}$ for
every $f\in A$, while $g^{\ssbullet}\le h^{\ssbullet}$.   But this
argument works for every $h$ such that $h^{\ssbullet}$ is an upper bound
for $\{f^{\ssbullet}:f\in A\}$, so $g^{\ssbullet}$ must be actually the
supremum of $\{f^{\ssbullet}:f\in A\}$.

\medskip

{\bf (d)(i)} If $\sup_{f\in A}\int fd\mu$ is infinite, this is trivial.
Otherwise, $\{f^{\ssbullet}:f\in A\}$ is bounded above in $L^1$ and
therefore in $L^0$.   By (c), $g^{\ssbullet}$ is its supremum in $L^0$,
therefore in $L^1$;  so

\Centerline{$\int g=\int g^{\ssbullet}
=\sup_{f\in A}\int f^{\ssbullet}=\sup_{f\in A}\int f$,}

\noindent as in 365Df.

\medskip

\quad{\bf (ii)} Apply (i) to $A^*=\{\sup I:I\in[A]^{<\omega}\}$.
}%end of proof of 521B

\leader{521C}{}\cmmnt{ Just because null ideals are $\sigma$-ideals of
sets, we can read off some of the elementary properties of their cardinal
functions from 511J.   But the presence of a measure gives us a new way
to use shrinking numbers, which will be useful later.

\medskip

\noindent}{\bf Proposition} Let $(X,\Sigma,\mu)$ be a measure space, and
$A\subseteq X$.

(a) If $\gamma<\mu^*A$ there is a $B\subseteq A$
such that $\#(B)<\shr^+\Cal N(\mu)$ and $\mu^*B>\gamma$.

(b) There is a $B\subseteq A$ such that 
$\#(B)\le\max(\omega,\shr\Cal N(\mu))$ and $\mu^*B=\mu^*A$.

%exercise on  \non^+Cal N(\mu) ?
% on \omega_1-sat (X,\Sigma,\Cal I) ?

\proof{{\bf (a)} Set $\kappa=\shr^+\Cal N(\mu)$.
Let $\Cal E$ be the
family of those measurable subsets of $X$ such that there is a
$B\in[A\cap E]^{<\kappa}$ with $\mu^*B=\mu E$.   Then $\Cal E$ is
closed under finite unions (132Ed).   \Quer\ If $\mu^*B\le\gamma$
for every $B\in[A]^{<\kappa}$, then $\mu E\le\gamma$ for every
$E\in\Cal E$.   By 215Ab,
there is a non-decreasing sequence $\sequencen{E_n}$ in $\Cal E$ such
that $E\setminus\bigcup_{n\in\Bbb N}E_n$ is negligible for every
$E\in\Cal E$.   Now $\mu(\bigcup_{n\in\Bbb N}E_n)\le\gamma<\mu^*A$
and $A'=A\setminus\bigcup_{n\in\Bbb N}E_n$ is not negligible.   Let
$B\in[A']^{<\kappa}$ be a non-negligible set.    Then $\mu^*B\le\gamma$ is
finite, so $B$ has a measurable envelope $F$ (132Ee),
which belongs to $\Cal E$;
but $F\setminus\bigcup_{n\in\Bbb N}E_n\supseteq B$
is not negligible.\ \BanG\
So we have a $B\in[A]^{<\kappa}$ with $\mu^*B>\gamma$, as required.

\medskip

{\bf (b)} If $\mu^*A=0$ take $B=\emptyset$.   Otherwise, let
$\sequencen{\gamma_n}$ be a sequence in $\coint{0,\mu^*A}$ with supremum
$\mu^*A$.   For each $n\in\Bbb N$, (a) tells us that there is a set
$B_n\subseteq A$ such that $\#(B_n)\le\shr(\mu)$ and 
$\mu^*B_n>\gamma_n$;  set $B=\bigcup_{n\in\Bbb N}B_n$.
}%end of proof of 521C

\leader{521D}{Proposition}\dvAnew{2014}
Let $(X,\Sigma,\mu)$ be a measure space and
$(\frak A,\bar\mu)$ its measure algebra.

(a) $\pi(\frak A)\le\pi(\mu)\le\max(\pi(\frak A),\cf\Cal N(\mu))$\cmmnt{
(definitions:  511Dc, 511Gb)}.

(b) If $\mu X>0$, then $\non\Cal N(\mu)\le\pi(\mu)$.

(c) If $(X,\Sigma,\mu)$
has locally determined negligible sets\cmmnt{ (definition:
213I)}, then $\shr\Cal N(\mu)\le\pi(\mu)$.

(d)\dvAformerly{5{}25B} Suppose that
there is a topology $\frak T$ on $X$ such that
$(X,\frak T,\Sigma,\mu)$ is a
quasi-Radon measure space.   Then, writing $\frak A^+$ for
$\frak A\setminus\{0\}$, the
partially ordered sets $(\Sigma\setminus\Cal N(\mu),\supseteq)$
and $(\frak A^+,\Bsupseteqshort)$ are Tukey equivalent and
$\pi(\mu)=\pi(\frak A)$.

\proof{ Let $\Cal H\subseteq\Sigma\setminus\Cal N(\mu)$ be a
coinitial set of size $\pi(\mu)$.

\medskip

{\bf (a)(i)} If $a\in\frak A$ is non-zero, there is an
$E\in\Sigma$ such that $E^{\ssbullet}=a$, and now $E$ is not negligible, so
there is an $H\in\Cal H$ such that $H\subseteq E$ and
$0\ne H^{\ssbullet}\Bsubseteq a$.   Thus $\{H^{\ssbullet}:H\in\Cal H\}$ is
coinitial with $\frak A^+$ and witnesses that
$\pi(\frak A)\le\#(\Cal H)=\pi(\mu)$.

\medskip

\quad{\bf (ii)} Let $B\subseteq\frak A^+$ be a coinitial set
of size $\pi(\frak A)$, and $\Cal E$ a cofinal subset of $\Cal N(\mu)$ of
size $\cf\Cal N(\mu)$.   For $b\in B$, let $F_b\in\Sigma$ be such that
$F_b^{\ssbullet}=b$, and consider
$\Cal G=\{F_b\setminus E:b\in B$, $E\in\Cal E\}$.   Then
$\Cal G\subseteq\Sigma\setminus\Cal N(\mu)$ is coinitial with
$\Sigma\setminus\Cal N(\mu)$.   \Prf\ If $\mu F>0$, there is a $b\in B$
such that $b\Bsubseteq F^{\ssbullet}$.   In this case, $F_b\setminus F$ is
negligible, so there is an $E\in\Cal E$ such that
$F_b\setminus F\subseteq E$ and $F\supseteq F_b\setminus E\in\Cal G$.\ \Qed

It follows that $\pi(\mu)\le\#(\Cal G)\le\#(B\times\Cal E)$
is at most the cardinal product
$\pi(\frak A)\cdot\cf\Cal N(\mu)\le\max(\omega,\pi(\frak A),\cf\Cal N(\mu))$.
But if $\cf\Cal N(\mu)$ is finite it is $1$, so in fact
$\pi(\mu)\le\pi(\frak A)\cdot\cf\Cal N(\mu)=\max(\pi(\frak A),\cf\Cal N(\mu))$.

\medskip

{\bf (b)} For each $H\in\Cal H$ choose $x_H\in\Cal H$.   Then
$A=\{x_H:H\in\Cal H\}$ must meet every non-negligible measurable set,
so (as $\mu X>0$) cannot itself be negligible.   Thus

\Centerline{$\non\Cal N(\mu)\le\#(A)\le\#(\Cal H)=\pi(\mu)$.}

\medskip

{\bf (c)} Suppose that $B\subseteq X$ is non-negligible.
Because $(X,\Sigma,\mu)$ has locally determined
negligible sets there is an $E\in\Sigma$ such that $\mu E>0$ and
$B\cap E$ is not negligible, and now $B\cap E$ has a measurable envelope
$F$ say (132Ee again).   Set
$\Cal H'=\{H:H\in\Cal H$, $B\cap H\ne\emptyset\}$ and for
$H\in\Cal H'$ choose $x_H\in B\cap H$;  set $A=\{x_H:H\in\Cal H'\}$, so
that $A\subseteq B$ and $\#(A)\le\pi(\mu)$.
\Quer\ If $A$ is negligible, then $F\setminus A$ includes a non-negligible
measurable set so includes a member $H$ of $\Cal H$.   As $\mu H>0$ and
$F$ is a measurable envelope of $B$, $H$ meets $B$ and belongs to
$\Cal H'$, and $x_H\in A\cap H$.\ \BanG\  Thus $A$ is not negligible.
As $B$ is arbitrary, $\shr\Cal N(\mu)\le\pi(\mu)$.

\medskip

{\bf (d)} For $E\in\Sigma\setminus\Cal N(\mu)$ let $F_E$ be a closed
non-negligible subset of $E$ and set
$\phi(E)=F_E^{\ssbullet}\in\frak A^+$;  for $a\in\frak A^+$, let
$\psi(a)$ be a self-supporting measurable set such that
$\psi(a)^{\ssbullet}=a$ (414F).   Then if $\phi(E)\Bsupseteq a$,
$\psi(a)\setminus F_E$ is negligible so
$E\supseteq F_E\supseteq\psi(a)$.   Thus $(\phi,\psi)$ is a Galois-Tukey
connection and
$(\Sigma\setminus\Cal N(\mu),\supseteq,\Sigma\setminus\Cal N(\mu))
\prGT(\frak A^+,\Bsupseteqshort,\frak A^+)$.

Moreover, if $\psi(a)\supseteq E$, then $a\Bsupseteq\phi(E)$, so
$(\psi,\phi)$ also is a Galois-Tukey connection and
$(\frak A^+,\Bsupseteqshort,\frak A^+)\prGT
(\Sigma\setminus\Cal N(\mu),\discretionary{}{}{}\supseteq\nobreak,
\discretionary{}{}{}\Sigma\setminus\Cal N(\mu))$.

Thus $(\Sigma\setminus\Cal N(\mu),\supseteq,\Sigma\setminus\Cal N(\mu))
\equivGT(\frak A^+,\Bsupseteqshort,\frak A^+)$, that is,
$(\Sigma\setminus\Cal N(\mu),\supseteq)
\equivT(\frak A^+,\Bsupseteqshort)$.   By 513E(e-i), inverted,

\Centerline{$\pi(\mu)=\ci(\Sigma\setminus\Cal N(\mu))
=\ci(\frak A^+)=\pi(\frak A)$.}
}%end of proof of 521D

\leader{521E}{}\cmmnt{ It will be useful later in the chapter to be able to
calculate the topological density of measure-algebra topologies.

\medskip

\noindent}{\bf Proposition} Let $(\frak A,\bar\mu)$ be a semi-finite
measure algebra.

(a) Give $\frak A$ its measure-algebra topology\cmmnt{ (323A)}.

\quad(i) If $\frak B$ is a subalgebra of $\frak A$, it is topologically
dense iff it $\tau$-generates $\frak A$\cmmnt{, that is, $\frak A$ is the
order-closed subalgebra of itself generated by $\frak B$}.

\quad(ii) If
$\frak A$ is finite, then its topological density is $\#(\frak A)$;  if
$\frak A$ is infinite, its topological density is equal to its Maharam
type $\tau(\frak A)$.

(b) Let $\frak A^f$ be the set of elements of $\frak A$ with
finite measure, with its strong measure-algebra
topology\cmmnt{ (323Ad)}.   Then the
topological density of $\frak A^f$ is $\#(\frak A^f)=\#(\frak A)$ if
$\frak A$
is finite, and $\max(c(\frak A),\tau(\frak A))$ if $\frak A$ is infinite.

\proof{{\bf (a)(i)}\grheada\ Suppose that $\frak B$ is topologically dense.
Let $\frak C$ be the order-closed subalgebra
of $\frak A$ generated by $\frak B$.
If $a\in\frak A^f$ and $c\in\frak A$,
there is a $b\in\frak C$ such that $b\Bcap a=c\Bcap a$.   \Prf\ For each
$n\in\Bbb N$, there is an
$a_n\in\frak B$ such that $\bar\mu(a\Bcap(a_n\Bsymmdiff c))\le 2^{-n}$.
Set $b=\inf_{n\in\Bbb N}\sup_{m\ge n}a_m\in\frak C$;  then
$b\cap a=c\Bcap a$ (apply 323F to $\sequencen{a\Bcap a_n}$).\ \Qed

It follows that $\frak A^f\subseteq\frak C$.   \Prf\ If $c\in\frak A^f$,
then whenever $c\Bsubseteq a\in\frak A^f$ there is a $b_a\in\frak C$
such that $b_a\Bcap a=c$.   Now (because $\bar\mu$ is semi-finite)
$c=\inf\{b_a:c\Bsubseteq a\in\frak A^f\}\in\frak C$.\ \Qed

Finally, again because $\bar\mu$ is semi-finite,

\Centerline{$c=\sup\{a:a\in\frak A^f,\,a\Bsubseteq c\}\in\frak C$}

\noindent for every $c\in\frak A$, and $\frak A=\frak C$.  Thus
$\frak B\,\,\tau$-generates $\frak A$.

\medskip

\qquad\grheadb\ Suppose that $\frak B\,\,\tau$-generates $\frak A$.
Then the topological closure of $\frak B$ is order-closed (323D(c-i)) and
a subalgebra (323B), so must be $\frak A$, and $\frak B$ is topologically
dense.

\medskip

\quad{\bf (ii)}\grheada\
If $\frak A$ is finite, this is trivial, just because
the measure-algebra topology is Hausdorff (323Ga).   So let us
henceforth suppose that $\frak A$ is infinite, so that both
$\tau(\frak A)$ and the topological density $d_{\frak T}(\frak A)$ of
$\frak A$ are infinite.

\medskip

\qquad\grheadb\
Let $A\subseteq\frak A$ be a set with cardinal $\tau(\frak A)$
which $\tau$-generates $\frak A$, and let $\frak B$ be the subalgebra of
$\frak A$ generated by $A$.   Then
$\#(\frak B)=\#(A)=\tau(\frak A)$ (331Gc), and
$\frak B$ is topologically dense in $\frak A$, by (i);  so
$d_{\frak T}(\frak A)\le\tau(\frak A)$.

\medskip

\qquad\grheadc\ Let $A\subseteq\frak A$ be a topologically dense set
with cardinal $d_{\frak T}(\frak A)$, and $\frak B$ the subalgebra of
$\frak A$ generated by $A$.   Then $\frak B$ is topologically dense, so it
$\tau$-generates $\frak A$, and

\Centerline{$\tau(\frak A)\le\#(\frak B)=\#(A)=d_{\frak T}(\frak A)$;}

\noindent with ($\beta$), this means that we have equality, as claimed.

\medskip

{\bf (b)(i)} The case of finite $\frak A$ is again trivial;  suppose that
$\frak A$ is infinite.   Let $\familyiI{a_i}$ be a partition of unity in
$\frak A$ consisting of non-zero elements of finite measure.

\medskip

\quad{\bf (ii)} The topological density $d_{\text{top}}(\frak A^f)$ 
is at most $\max(c(\frak A),\tau(\frak A))$.
\Prf\ For each $i$, the topological density of $\frak A_{a_i}$, with its
measure-algebra topology, is at most
$\max(\omega,\tau(\frak A_{a_i}))\le\tau(\frak A)$ ((a) above and 514Ed);
let $B_i\subseteq\frak A_{a_i}$ be a dense subset of this size or less.   
Set $B=\bigcup_{i\in I}B_i$, $D=\{\sup B':B'\in[B]^{<\omega}\}$.
Then the metric closure $\overline{D}$ of $D$ in $\frak A^f$ is closed
under $\Bcup$ and includes $\frak A_{a_i}$ for every $i$.   If now
$a\in\frak A^f$, $a=\sup_{i\in I}a\Bcap a_i\in\overline{D}$.   So

\Centerline{$d_{\text{top}}(\frak A^f)\le\#(D)
\le\max(\omega,\#(I),\tau(\frak A))
\le\max(c(\frak A),\tau(\frak A))$.  \Qed}

\medskip

\quad{\bf (iii)} $c(\frak A)\le d_{\text{top}}(\frak A^f)$.   \Prf\ Let
$\family{j}{J}{b_j}$
be any disjoint family in $\frak A^+$.
For each $j$, let $b_j'\Bsubseteq b_j$ be a non-zero element of non-zero
finite measure.   Set
$G_j=\{a:a\in\frak A^f$, $\bar\mu(a\Bsymmdiff b'_j)<\bar\mu b'_j\}$ for
$j\in J$.   Then $\family{j}{J}{G_j}$ is a disjoint family of non-empty
open sets in $\frak A^f$, so $\#(J)\le d_{\text{top}}(\frak A^f)$ (5A4Ba).   As
$\family{j}{J}{b_j}$ is arbitrary, 
$c(\frak A)\le d_{\text{top}}(\frak A^f)$.\ \Qed

\medskip

\quad{\bf (iv)} $\tau(\frak A)\le d_{\text{top}}(\frak A^f)$.   \Prf\
Let $A\subseteq\frak A^f$ be a dense set of size 
$d_{\text{top}}(\frak A^f)$.  Let $\frak B$ be the order-closed
subalgebra of $\frak A$ generated by $B=A\cup\{a_i:i\in I\}$.   For any
$i\in I$, set
$A_i=\{a\Bcap a_i:a\in A\}$.   Now $A_i$ is topologically dense in
$\frak A_{a_i}$ (use 3A3Eb), so the order-closed subalgebra of
$\frak A_{a_i}$ it
generates is the whole of $\frak A_{a_i}$ (323H);  by 314H,
$\frak A_{a_i}=\{b\Bcap a_i:b\in\frak B\}$.   As
$a_i\in A\subseteq\frak B$, $\frak B$ includes $\frak A_{a_i}$.   As
$\sup_{i\in I}a_i=1$, $\frak B=\frak A$.   Thus

\Centerline{$\tau(\frak A)\le\#(B)
\le\max(\omega,\#(I),d_{\text{top}}(\frak A^f))
=d_{\text{top}}(\frak A^f)$}

\noindent (using (iii) for the last equality).\ \Qed

\medskip

\quad{\bf (v)} Putting these together, we have the result.
}%end of proof of 521E

\leader{521F}{Proposition} Let $(X,\Sigma,\mu)$ be a measure space,
$A$ a subset of $X$ and $\mu_A$ the subspace measure on $A$.

(a) $\Cal N(\mu_A)\prT\Cal N(\mu)$, so
$\add\Cal N(\mu_A)\ge\add\Cal N(\mu)$ and
$\cf\Cal N(\mu_A)\le\cf\Cal N(\mu)$.

(b) $(A,\in,\penalty-100\Cal N(\mu_A))\prGT(X,\in,\Cal N(\mu))$, so
$\non\Cal N(\mu_A)\ge\non\Cal N(\mu)$ and
$\cov\Cal N(\mu_A)\le\cov\Cal N(\mu)$.

(c) $\add\mu_A\ge\add\mu$.

(d) $\shr\Cal N(\mu_A)\le\shr\Cal N(\mu)$
and $\shr^+\Cal N(\mu_A)\le\shr^+\Cal N(\mu)$.

(e)\dvAnew{2014} If either $A\in\Sigma$ or $(X,\Sigma,\mu)$
has locally determined negligible sets, $\pi(\mu_A)\le\pi(\mu)$.

(f)\dvAnew{2014} If $\mu_A$ is semi-finite, then
$\tau(\mu_A)\le\tau(\mu)$.

\proof{{\bf (a)}
Because $\Cal N(\mu_A)=\Cal PA\cap\Cal N(\mu)$ (214Cb),
the embedding $\Cal N(\mu_A)\embedsinto\Cal N(\mu)$ is a
Tukey function, and $\Cal N(\mu_A)\prT\Cal N(\mu)$.
By 513Ee, $\add\Cal N(\mu_A)\ge\add\Cal N(\mu)$ and
$\cf\Cal N(\mu_A)\le\cf\Cal N(\mu)$.

\medskip

{\bf (b)} Next, setting $\phi(x)=x$ for $x\in A$ and $\psi(F)=F\cap A$ for
$F\in\Cal N(\mu)$, $(\phi,\psi)$ witnesses that
$(A,\in,\Cal N(\mu_A))\prGT(X,\in,\Cal N(\mu))$.
By 512D and 512Ed,

\Centerline{$\non\Cal N(\mu_A)=\add(A,\in,\Cal N(\mu_A))
\ge\add(X,\in,\Cal N(\mu))=\non\Cal N(\mu)$,}

\Centerline{$\cov\Cal N(\mu_A)=\cov(A,\in,\Cal N(\mu_A))
\le\cov(X,\in,\Cal N(\mu))=\cov\Cal N(\mu)$.}

\medskip

{\bf (c)} If $\ofamily{\xi}{\kappa}{F_{\xi}}$ is a disjoint family in
$\Sigma_A=\dom\mu_A$, where $\kappa<\add\mu$, then for each $\xi<\kappa$ we
have an $E_{\xi}\in\Sigma$ such that $F_{\xi}=A\cap E_{\xi}$ and
$\mu_AF_{\xi}=\mu E_{\xi}$ (214Ca).   Set
$E'_{\xi}=E_{\xi}\setminus\bigcup_{\eta<\xi}E_{\eta}$ for $\xi<\kappa$;
then $E'_{\xi}\in\Sigma$ for each $\xi$, and
$\ofamily{\xi}{\kappa}{E'_{\xi}}$ is disjoint.   Set
$E=\bigcup_{\xi<\kappa}E'_{\xi}=\bigcup_{\xi<\kappa}E_{\xi}$ and
$F=A\cap E=\bigcup_{\xi<\kappa}F_{\xi}$.   Then

\Centerline{$\sum_{\xi<\kappa}\mu_AF_{\xi}
\le\mu_AF
\le\mu E
=\sum_{\xi<\kappa}\mu E'_{\xi}
\le\sum_{\xi<\kappa}\mu E_{\xi}
=\sum_{\xi<\kappa}\mu_AF_{\xi}$,}

\noindent and we have equality.   As
$\ofamily{\xi}{\kappa}{F_{\xi}}$ is arbitrary, $\add\mu_A\ge\add\mu$.

\medskip

{\bf (d)}
If $B\in\Cal PA\setminus\Cal N(\mu_A)$, there is a $C\subseteq B$ such that
$C\notin\Cal N(\mu)$ and $\#(C)\le\shr\Cal N(\mu)$ (resp.\
$\#(C)<\shr^+\Cal N(\mu))$;  now
$C\notin\Cal N(\mu_A)$;  as $B$ is arbitrary,
$\shr\Cal N(\mu_A)\le\shr\Cal N(\mu)$ (resp.\
$\shr^+\Cal N(\mu_A)\le\shr^+\Cal N(\mu)$).

\medskip

{\bf (e)} Let $\Cal H\subseteq\Sigma\setminus\Cal N(\mu)$ be a coinitial
set of size $\pi(\mu)$.   Set
$\Cal G=\{H\cap A:H\in\Cal H\}\setminus\Cal N(\mu)$.   Then
$\mu_AG$ is defined and non-zero for every $G\in\Cal G$.
Now $\Cal G$ is coinitial with $\dom\mu_A\setminus\Cal N(\mu_A)$.
\Prf\ If
$\mu_AB>0$, there is an $E\in\Sigma$ such that $B=E\cap A$.   If
$A\in\Sigma$, then $B\in\Sigma$ and there is an $H\in\Cal H$ such that
$H\subseteq B$, while of course $H\in\Cal G$.   If $(X,\Sigma,\mu)$ has
locally determined negligible sets, then, as in the proof of 521Dc,
there is a non-negligible set $F\in\Sigma$ which is a measurable envelope
of a subset of $B$.   Now there is an $H\in\Cal H$ included in
$F\cap E$, in which case $H\cap A$ is included in $B$
and belongs to $\Cal G$.\ \QeD\   So

\Centerline{$\pi(\mu_A)\le\#(\Cal G)\le\#(\Cal H)=\pi(\mu)$.}

\medskip

{\bf (f)} Writing $\frak A$, $\frak A_A$ for the measure algebras of
$\mu$ and $\mu_A$, we have a Boolean homomorphism
$\pi:\frak A\to\frak A_A$ defined by saying that
$\pi E^{\ssbullet}=(F\cap A)^{\ssbullet}$ for every $E\in\Sigma$.
(The point is just that $F\cap A\in\Cal N(\mu_A)$ whenever
$F\in\Cal N(\mu)$.)   Now $\pi$ is order-continuous.   \Prf\ Suppose that
$C\subseteq\frak A$ is non-empty and downwards-directed and
$\inf C=0$ in $\frak A$.   \Quer\ If
$b\in\frak A_A$ is a non-zero lower bound of $\pi[C]$, then, because
$\nu_A$ is semi-finite, there is a $G\in\dom\mu_A$ such that
$0<\mu_AG<\infty$ and $G^{\ssbullet}\Bsubseteq b$.   Let $E\in\Sigma$ be
such that $G=E\cap A$ and $\mu E=\mu_AG$ (214Ca).   Then $E^{\ssbullet}$
cannot be a lower bound of $C$;  let $a\in C$ be such that
$E^{\ssbullet}\Bsetminus a\ne 0$.   In this case, there is an $F\in\Sigma$
such that $F\subseteq E$ and
$F^{\ssbullet}=E^{\ssbullet}\Bsetminus a$, so that
$\pi F^{\ssbullet}$ is disjoint from $\pi a\Bsupseteq b$, and
$F\cap G=(F\cap A)\cap G$ must be negligible.   We know that $\mu F>0$,
so $\mu(E\setminus F)<\mu E$;  but also $G\setminus(E\setminus F)$ is
negligible, so

\Centerline{$\mu_AG=\mu^*G\le\mu(E\setminus F)<\mu E=\mu_AG$.  \Bang}

\noindent It follows that $\inf\pi[C]=0$ in $\frak A_A$;  as
$C$ is arbitrary, $\pi$ is order-continuous.\ \Qed

Now let $B\subseteq\frak A$ be such that $B\,\,\tau$-generates $\frak A$
and $\#(B)=\tau(\frak A)$.   Writing $\frak B$ for the
order-closed subalgebra of $\frak A_A$ generated by $\pi[B]$,
we see that $\pi^{-1}[\frak B]$ is an order-closed subalgebra of
$\frak A$ including $B$, so must be the whole of $\frak A$, and
$\frak A_A=\pi[\frak A]=\frak B$.   Accordingly

\Centerline{$\tau(\mu_A)=\tau(\frak A_A)\le\#(\pi[B])
\le\#(B)=\tau(\frak A)=\tau(\mu)$,}

\noindent as claimed.
}%end of proof of 521F

\leader{521G}{Proposition} Let $\familyiI{(X_i,\Sigma_i,\mu_i)}$ be a
non-empty
family of measure spaces with direct sum $(X,\Sigma,\mu)$.   Then

\Centerline{$\add\Cal N(\mu)=\min_{i\in I}\add\Cal N(\mu_i)$,
\quad$\add\mu=\min_{i\in I}\add\mu_i$,}

\Centerline{$\cov\Cal N(\mu)=\sup_{i\in I}\cov\Cal N(\mu_i)$,
\quad$\non\Cal N(\mu)=\min_{i\in I}\non\Cal N(\mu_i)$,}

\Centerline{$\shr\Cal N(\mu)=\sup_{i\in I}\shr\Cal N(\mu_i)$,
\quad$\shr^+\Cal N(\mu)=\sup_{i\in I}\shr^+\Cal N(\mu_i)$}

\noindent and $\pi(\mu)$ is the cardinal sum
$\sum_{i\in I}\pi(\mu_i)$\dvAnew{2014}.
If $I$ is finite, then

\Centerline{$\cf\Cal N(\mu)=\max_{i\in I}\cf\Cal N(\mu_i)$.}

\proof{ Concerning each of
$\add\Cal N(\mu)$, $\add\mu$, $\cov\Cal N(\mu)$,
$\non\Cal N(\mu)$, $\shr\Cal N(\mu)$ and $\shr^+\Cal N(\mu)$,
521F provides an inequality in one direction.   The reverse
inequalities are equally straightforward, especially if we note that
$\Cal N(\mu)\cong\prod_{i\in I}\Cal N(\mu_i)$, so that
512Hc is relevant.

As for $\pi(\mu)$, if for each $i\in I$ we choose a coinitial set
$\Cal H_i$ of $\Sigma_i\setminus\Cal N(\mu_i)$ of size $\pi(\mu_i)$, then

\Centerline{$\Cal H=\{H\times\{i\}:i\in I$, $H\in\Cal H_i\}$}

\noindent is coinitial with $\Sigma\setminus\Cal N(\mu)$ and witnesses that
$\pi(\mu)\le\sum_{i\in I}\pi(\mu_i)$.   (As in 214L, I am thinking of
$X$ as $\bigcup_{i\in I}X_i\times\{i\}$.)   Conversely, if $\Cal H$ is
coinitial with $\Sigma\setminus\Cal N(\mu)$ and for each $i\in I$ we set
$\Cal H_i=\{H:H\times\{i\}\in\Cal H\}$, we shall have $\Cal H_i$ coinitial
with $\Sigma_i\setminus\Cal N(\mu_i)$, so that

\Centerline{$\sum_{i\in I}\pi(\mu_i)\le\sum_{i\in I}\#(\Cal H_i)
\le\#(\Cal H)=\pi(\mu)$}

\noindent and $\pi(\mu)=\sum_{i\in I}\pi(\mu_i)$.
}%end of proof of 521G

\leader{521H}{Proposition} Let $(X,\Sigma,\mu)$ and $(Y,\Tau,\nu)$ be
measure spaces, and $f:X\to Y$ an \imp\ function.

(a)(i)
$(X,\in,\Cal N(\mu))\discretionary{}{}{}\prGT(Y,\in,\Cal N(\nu))$, so
$\non\Cal N(\mu)\ge\non\Cal N(\nu)$ and
$\cov\Cal N(\mu)\le\cov\Cal N(\nu)$.

\quad(ii)\dvAnew{2014} If there is a topology on $Y$ such that $\nu$ is a
topological measure
inner regular with respect to the closed sets, then $\pi(\nu)\le\pi(\mu)$.

\quad(iii)\dvAnew{2014} If $\nu$ is $\sigma$-finite, then
$\tau(\nu)\le\tau(\mu)$.

(b) If $\nu$ is the image measure $\mu f^{-1}$, then
$\add\nu\ge\add\mu$.   If, moreover, $\mu$ is complete,
$\Cal N(\nu)\prT\Cal N(\mu)$, so $\add\Cal N(\mu)\le\add\Cal N(\nu)$ and
$\cf\Cal N(\mu)\ge\cf\Cal N(\nu)$;  also
$\shr\Cal N(\mu)\ge\shr\Cal N(\nu)$
and $\shr^+\Cal N(\mu)\ge\shr^+\Cal N(\nu)$.

\proof{{\bf (a)(i)} Set $\psi(F)=f^{-1}[F]$ for
$F\in\Cal N(\nu)$.   Then $(f,\psi)$ is a Galois-Tukey connection from
$(X,\in,\Cal N(\mu))$ to $(Y,\in,\Cal N(\nu))$, so

\Centerline{$\cov\Cal N(\mu)=\cov(X,\in,\Cal N(\mu))
\le\cov(Y,\in,\Cal N(\nu))=\cov\Cal N(\nu)$,}

\Centerline{$\non\Cal N(\mu)=\add(X,\in,\Cal N(\mu))
\ge\add(Y,\in,\Cal N(\nu))=\non\Cal N(\nu)$}

\noindent (512D, 512Ed again).

\medskip

\quad{\bf (ii)} Let $\Cal H$ be a coinitial subset of
$\Sigma\setminus\Cal N(\mu)$ of size $\pi(\mu)$.   Set
$\Cal G=\{\overline{f[H]}:H\in\Cal H\}$.   Because $\nu$ is a topological
measure, $\Cal G\subseteq\Tau$;  and if $H\in\Cal H$, then

\Centerline{$\nu\overline{f[H]}=\mu f^{-1}[\overline{f[H]}]
\ge\mu H>0$,}

\noindent so $\Cal G\subseteq\Tau\setminus\Cal N(\nu)$.   If
$F\in\Tau\setminus\Cal N(\nu)$, there is a closed set $F'\subseteq F$ such
that $0<\nu F'=\mu f^{-1}[F']$;  there is an $H\in\Cal H$ such that
$H\subseteq f^{-1}[F']$;  now $G=\overline{f[H]}$ belongs to $\Cal G$ and
is included in $F'\subseteq F$.   So $\Cal G$ is coinitial with
$\Tau\setminus\Cal N(\nu)$ and

\Centerline{$\pi(\nu)\le\#(\Cal G)\le\#(\Cal H)=\pi(\mu)$.}

\medskip

\quad{\bf (iii)} Let $(\frak A,\bar\mu)$ and $(\frak B,\bar\nu)$ be the
measure algebras of $\mu$, $\nu$ respectively.   Then we have a
sequentially order-continuous
measure-preserving Boolean homomorphism $\pi:\frak B\to\frak A$ defined by
setting $\pi F^{\ssbullet}=f^{-1}[F]^{\ssbullet}$ for every $F\in\dom\nu$
(324M).   If $\frak A$ is finite then $\frak B$ must be finite with
$\#(\frak B)\le\#(\frak A)$, and consequently
$\tau(\frak B)\le\tau(\frak A)$ (331Xc, or otherwise).
So let us suppose that $\frak A$ is infinite.

Writing $\frak A^f$, $\frak B^f$ for the respective ideals of
elements of finite measure, $\pi\restr\frak B^f$ is a function from
$\frak B^f$ to $\frak A^f$ which is an isometry for the measure metrics on
$\frak B^f$ and $\frak A^f$.   So the topological density 
$d_{\text{top}}(\frak B^f)$ is
equal to $d_{\text{top}}(\pi[\frak B^f])$ and less than or equal to 
$d_{\text{top}}(\frak A^f)$ (5A4B(h-ii)).

\woddheader{521H}{0}{0}{0}{72pt}
Observe next that $(\frak B,\bar\nu)$ is $\sigma$-finite because $\nu$ is,
and that $(\frak A,\bar\mu)$ therefore also is (324Kd).   So we get

$$\eqalignno{\tau(\nu)
&=\tau(\frak B)
\le\max(\omega,c(\frak B),\tau(\frak B))
=\max(\omega,d_{\text{top}}(\frak B^f))\cr
\displaycause{521Eb}
&\le\max(\omega,d_{\text{top}}(\frak A^f))
=\max(\omega,c(\frak A),\tau(\frak A))
=\tau(\frak A)
=\tau(\mu),\cr}$$

\noindent as required.

\medskip

{\bf (b)} If $\ofamily{\xi}{\kappa}{F_{\xi}}$ is a disjoint family in
$\Tau$, where $\kappa<\add\mu$, then
$\ofamily{\xi}{\kappa}{f^{-1}[F_{\xi}]}$ is a disjoint family in $\Sigma$,
so

$$\nu(\bigcup_{\xi<\kappa}F_{\xi})
=\mu f^{-1}[\bigcup_{\xi<\kappa}F_{\xi}]
=\mu(\bigcup_{\xi<\kappa}f^{-1}[F_{\xi}]
=\sum_{\xi<\kappa}\mu f^{-1}[F_{\xi}]
=\sum_{\xi<\kappa}\nu F_{\xi}.$$

\noindent
As $\ofamily{\xi}{\kappa}{F_{\xi}}$ is arbitrary, $\add\nu\ge\add\mu$.

Now suppose that $\mu$ is complete.   In this case, $F\in\Tau$
whenever $F\subseteq Y$ and $f^{-1}[F]\in\Cal N(\mu)$, so that
$\Cal N(\nu)$ is precisely
$\{F:F\subseteq Y,\,f^{-1}[F]\in\Cal N(\mu)\}$.   It is now easy to
check that $F\mapsto f^{-1}[F]:\Cal N(\nu)\to\Cal N(\mu)$ is a Tukey
function.   So $\add\Cal N(\nu)\ge\add\Cal N(\mu)$
and $\cf\Cal N(\nu)\le\cf\Cal N(\mu)$, by 513Ee again.

Take any non-negligible $A\subseteq Y$.   Then
$f^{-1}[A]\notin\Cal N(\mu)$, so there is a set $B\subseteq f^{-1}[A]$
such that $\#(B)\le\shr\Cal N(\mu)$ and $B\notin\Cal N(\mu)$.   In this
case, $f[B]\subseteq A$, $f[B]\notin\Cal N(\nu)$ and
$\#(f[B])\le\shr\Cal N(\mu)$.   As $A$ is arbitrary,
$\shr\Cal N(\nu)\le\shr\Cal N(\mu)$.
The same argument, with $<$ instead of $\le$ at appropriate points, shows
that $\shr^+\Cal N(\nu)\le\shr^+\Cal N(\mu)$.
}%end of proof of 521H

\leader{521I}{Corollary} Let $(X,\Sigma,\mu)$ be an atomless strictly
localizable measure space.   Then $\non\Cal N(\mu)\ge\non\Cal N$ and
$\cov\Cal N(\mu)\le\cov\Cal N$, where $\Cal N$ is the null ideal of
Lebesgue measure on $\Bbb R$.

\proof{{\bf (a)} If $\mu X=0$ this is trivial.

\medskip

{\bf (b)} If $0<\mu X<\infty$, let $\nu$ be the completion of the
normalized measure
$\Bover1{\mu X}\mu$.   Then $\nu$ is complete and atomless,
so by 343Cb there is an \imp\ function from $(X,\nu)$ to
$([0,1],\mu_1)$, where $\mu_1$ is Lebesgue measure on $[0,1]$.
Also $\Cal N(\nu)=\Cal N(\mu)$.   By 521Ha,
$\non\Cal N(\nu)\ge\non\Cal N(\mu_1)$ and
$\cov\Cal N(\nu)\le\cov\Cal N(\mu_1)$.
Now $([0,1],\Cal N(\mu_1))$ is isomorphic to
$(\Bbb R,\Cal N)$.   \Prf\ Take a bijection $h:\Bbb R\to[0,1]$
such that $h(x)=\bover12(1+\tanh x)$ for $x\in\Bbb R\setminus\Bbb Q$;  then
$h$ is a suitable isomorphism.\ \QeD\
So $\non\Cal N(\mu)=\non\Cal N(\nu)\ge\non\Cal N$ and
$\cov\Cal N(\mu)\le\cov\Cal N$.

\medskip

{\bf (c)} If $X$ has infinite measure, let $\familyiI{X_i}$ be a
decomposition of $X$ into sets of finite measure.   For each $i\in I$
let $\mu_i$ be the subspace measure on $X_i$.
Then every $\mu_i$ is atomless, so, putting (b) and 521G together,

\Centerline{$\non\Cal N(\mu)
=\min_{i\in I}\non\Cal N(\mu_i)\ge\non\Cal N$,}

\Centerline{$\cov\Cal N(\mu)
=\sup_{i\in I}\cov\Cal N(\mu_i)\le\cov\Cal N$.}
}%end of proof of 521I

\leader{521J}{}\cmmnt{ For product spaces the situation is more
complicated, because the product measure introduces `new' negligible sets
which are not directly definable in terms of the null ideals of the
factors.   In the next three sections, however, we shall find out
quite a lot about the cardinal functions of Radon measures,
and this information, when it comes, can be used to give results about
general products of probability measures.

\medskip

\noindent}{\bf Proposition} Let $\familyiI{(X_i,\Sigma_i,\mu_i)}$ be a
non-empty family of probability spaces with product $(X,\Sigma,\mu)$.

(a)

\Centerline{$\non\Cal N(\mu)\ge\sup_{i\in I}\non\Cal N(\mu_i)$,
\quad$\cov\Cal N(\mu)\le\min_{i\in I}\cov\Cal N(\mu_i)$,}

\Centerline{$\add\mu=\add\Cal N(\mu)\le\min_{i\in I}\add\Cal N(\mu_i)$,
\quad$\cf\Cal N(\mu)\ge\sup_{i\in I}\cf\Cal N(\mu_i)$,}

\Centerline{$\shr\Cal N(\mu)\ge\sup_{i\in I}\shr\Cal N(\mu_i)$,
\quad$\shr^+\Cal N(\mu)\ge\sup_{i\in I}\shr^+\Cal N(\mu_i)$,}

\Centerline{$\pi(\mu)\ge\sup_{i\in I}\pi(\mu_i)$\dvAnew{2014}.}

(b) Set $\kappa=\#(\{i:i\in I$, $\Sigma_i\ne\{\emptyset,X_i\}\}$.
Then $[\kappa]^{\le\omega}\prT\Cal N(\mu)$;  consequently
$\add\mu=\add\Cal N(\mu)$ is $\omega_1$ if $\kappa$ is uncountable,
while $\cf\Cal N(\mu)$ is at least $\cff[\kappa]^{\le\omega}$.

(c) Now suppose that $I$ is countable and that
we have for each $i\in I$ a probability space
$(Y_i,\Tau_i,\nu_i)$ and an \imp\ function
$f_i:X_i\to Y_i$ which represents an isomorphism of the measure algebras of
$\mu_i$ and $\nu_i$.   Let $(Y,\Tau,\nu)$ be the product of
$\familyiI{(Y_i,\Tau_i,\nu_i)}$.   Then

\Centerline{$\Cal N(\mu)\prT\Cal N(\nu)\times\prod_{i\in I}\Cal N(\mu_i)$.}

\noindent Consequently

\Centerline{$\add\Cal N(\mu)
\ge\min(\add\Cal N(\nu),\min_{i\in I}\add\Cal N(\mu_i))$,}

\noindent and if $I$ is finite

\Centerline{$\cf\Cal N(\mu)
\le\max(\cf\Cal N(\nu),\max_{i\in I}\cf\Cal N(\mu_i))$.}

(d) If $I$ is finite, then

\Centerline{$\non\Cal N(\mu)=\max_{i\in I}\non\Cal N(\mu_i)$,
\quad$\cov\Cal N(\mu)=\min_{i\in I}\cov\Cal N(\mu_i)$.}

\proof{{\bf (a)} Note that $\add\mu=\add\Cal N(\mu)$ by 521Ad.   Now
with one exception the inequalities are immediate if we
apply 521H to the canonical maps from $X$ to $X_i$.   The odd one out is
the last, because we do not have a simple general result concerning the
$\pi$-weight of an image measure.   But in the present case we can argue as
follows.   Let $\Cal H\subseteq\Sigma\setminus\Cal N(\mu)$ be a coinitial
set of size $\pi(\mu)$, and take $i\in I$.   Then we can identify
$(X,\Sigma,\mu)$ with $(X',\Sigma',\mu')\times(X_i,\Sigma_i,\mu_i)$
where $(X',\Sigma',\mu')$ is the product of the family
$\langle(X_j,\Sigma_j,\mu_j)\rangle_{j\in I,j\ne i}$ (254N).
If $H\in\Cal H$, $\mu(X\setminus H)
=\int\mu_i^*(X_i\setminus H[\{x'\}])\mu'(dx')$ (252D) is less than
$1$, so there is an $x'_H\in X'$ such that
$\mu_i^*(X_i\setminus H[\{x'_H\}])<1$, $(\mu_i)_*H[\{x'_H\}]>0$
and there is a
$G_H\in\Sigma_i\setminus\Cal N(\mu_i)$ such that
$G_H\subseteq H[\{x'_H\}]$.   Set $\Cal G=\{G_H:H\in\Cal H\}$.   If
$\mu_iF>0$, then $\mu(X'\times F)>0$ and there is
an $H\in\Cal H$ included in $X'\times F$;  in which case
$G_H\subseteq H[\{x'_H\}]\subseteq F$.   So $\Cal G$ is coinitial with
$\Sigma_i\setminus\Cal N(\mu_i)$ and

\Centerline{$\pi(\mu_i)\le\#(\Cal G)\le\#(\Cal H)\le\pi(\mu)$.}

\noindent As $i$ is arbitrary, $\sup_{i\in I}\pi(\mu_i)\le\pi(\mu)$, as
claimed.

\medskip

{\bf (b)(i)} If $\kappa\le\omega$ then the constant function with value
$\emptyset$ is a Tukey function from $[\kappa]^{\le\omega}$ to
$\Cal N(\mu)$.   Otherwise, set
$J=\{i:i\in I$, $\Sigma_i\ne\{\emptyset,X_i\}\}$ and for $i\in J$ choose
a non-empty $C_i\in\Sigma_i$ such that $\mu_iC_i\le\bover12$.   Index $J$
as $\langle i_{\xi n}\rangle_{\xi<\kappa,n\in\Bbb N}$ and for $\xi<\kappa$
set $E_{\xi}=\{x:x(i_{\xi n})\in C_{i_{\xi n}}$ for every $n\in\Bbb N\}$,
so that $\mu E_{\xi}=\prod_{n\in\Bbb N}\mu_{i_{\xi n}}C_{i_{\xi n}}=0$.
Define $\phi:[\kappa]^{\le\omega}\to\Cal N(\mu)$ by setting
$\phi K=\bigcup_{\xi\in K}E_{\xi}$ for countable $K\subseteq\kappa$.
Then $\phi$ is a Tukey function.   \Prf\ If $E\in\Cal N(\mu)$, there is a
negligible $E'\supseteq E$ which is determined by a countable set $I'$ of
coordinates (254Oc).   Set
$L=\{\xi:\xi<\kappa$, $i_{\xi n}\in I'$ for some $n\in\Bbb N\}$;  then $L$
is countable.   If $\xi<\kappa$ and $E_{\xi}\subseteq E'$, $E_{\xi}$ is
determined by coordinates in $\{i_{\xi n}:n\in\Bbb N\}$;  as neither
$E_{\xi}$ nor $X\setminus E'$ is empty, this must meet $I'$, and
$\xi\in L$.   So $\{K:K\in[\kappa]^{\le\omega}$, $\phi K\subseteq E\}$ is
bounded above by $L\in[\kappa]^{\le\omega}$.   As $E$ is arbitrary, $\phi$
is a Tukey function.\ \QeD\   Accordingly
$[\kappa]^{\le\omega}\prT\Cal N(\mu)$.

\medskip

\quad{\bf (ii)} It follows that
$\add\Cal N(\mu)\le\add[\kappa]^{\le\omega}\le\omega_1$ if $\kappa$ is
uncountable, and that $\cf\Cal N(\mu)\ge\cff[\kappa]^{\le\omega}$.

\medskip

{\bf (c)(i)} Recall that any \imp\ function $f$ between measure spaces
induces a measure-preserving Boolean homomorphism
$F^{\ssbullet}\mapsto(f^{-1}[F])^{\ssbullet}$ between the measure algebras
(324M).
For $i\in I$ and $C\in\Sigma_i$ choose $\psi_i(C)\in\Tau_i$ such that
$(f_i^{-1}[\psi_i(C)])^{\ssbullet}=C^{\ssbullet}$ in the measure algebra of
$\mu_i$, that is, $C\symmdiff f_i^{-1}[\psi_i(C)]\in\Cal N(\mu_i)$.
Next, for $E\in\Cal N(\mu)$ and $n\in\Bbb N$, choose a family
$\langle C_{Enmi}\rangle_{m\in\Bbb N,i\in I}$ such that
$C_{Enmi}\in\Sigma_i$ for every $m\in\Bbb N$ and $i\in I$,
$E\subseteq\bigcup_{m\in\Bbb N}\prod_{i\in I}C_{Enmi}$, and
$\sum_{m\in\Bbb N}\prod_{i\in I}\mu_iC_{Enmi}\le 2^{-n}$;  set

\Centerline{$\phi(E)
=\bigl(\bigcap_{n\in\Bbb N}\bigcup_{m\in\Bbb N}
   \prod_{i\in I}\psi_i(C_{Enmi}),
\familyiI{\bigcup_{m,n\in\Bbb N}
(C_{Enmi}\setminus f_i^{-1}[\psi_i(C_{Enmi})])}\bigr)$.}

\noindent Because

$$\eqalign{
\nu(\bigcap_{n\in\Bbb N}\bigcup_{m\in\Bbb N}\prod_{i\in I}\psi_i(C_{Enmi}))
&\le\inf_{n\in\Bbb N}\sum_{m\in\Bbb N}\prod_{i\in I}
   \nu_i\psi_i(C_{Enmi})\cr
&=\inf_{n\in\Bbb N}\sum_{m\in\Bbb N}\prod_{i\in I}
   \mu_iC_{Enmi}
=0,\cr}$$

\noindent $\phi$ is a function from $\Cal N(\nu)$ to
$\Cal N(\nu)\times\prod_{i\in I}\Cal N(\mu_i)$.

Now $\phi$ is a Tukey function.   \Prf\ Suppose that $W\in\Cal N(\nu)$ and
that $E_i\in\Cal N(\mu_i)$ for every $i\in I$.   Define $f:X\to Y$ by
setting $f(x)=\familyiI{f_i(x(i))}$ for $x\in X$;
then $f$ is \imp\ (254H).   So
$V=f^{-1}[W]\cup\bigcup_{i\in I}\{x:x(i)\in E_i\}$ is negligible.   (This
is where we need to know that $I$ is countable.)   Suppose that
$E\in\Cal N(\mu)$ is such that
$\phi(E)\le(W,\familyiI{E_i})$;  take $x\in E$ such that
$x(i)\notin E_i$ for every $i\in I$, and $n\in\Bbb N$.   Then there is an
$m\in\Bbb N$ such that $x\in\prod_{i\in I}C_{Enmi}$.   For each $i\in I$,

\Centerline{$C_{Enmi}\setminus f_i^{-1}[\psi_i(C_{Enmi})]\subseteq E_i$,
\quad$x(i)\in C_{Enmi}\setminus E_i$,}

\noindent so $f_i(x(i))\in\psi_i(C_{Enmi})$;  thus
$f(x)\in\prod_{i\in I}\psi_i(C_{Enmi})$.   As $n$ is arbitrary,

\Centerline{$f(x)
\in\bigcap_{n\in\Bbb N}\bigcup_{m\in\Bbb N}\prod_{i\in I}\psi_i(C_{Enmi})
\subseteq W$}

\noindent and $x\in V$.   As $x$ is arbitrary, $E\subseteq V$.   As
$(W,\familyiI{E_i})$ is arbitrary, $\phi$ is a Tukey function.\ \Qed

So $\Cal N(\mu)\prT\Cal N(\nu)\times\prod_{i\in I}\Cal N(\mu_i)$.

\medskip

\quad{\bf (ii)} Accordingly

\Centerline{$\add\Cal N(\mu)
\ge\add(\Cal N(\nu)\times\prod_{i\in I}\Cal N(\mu_i))
=\min(\add\Cal N(\nu),\min_{i\in I}\add\Cal N(\mu_i))$}

\noindent and

\Centerline{$\cf\Cal N(\mu)
\le\cf(\Cal N(\nu)\times\prod_{i\in I}\Cal N(\mu_i))
=\max(\cf\Cal N(\nu),\max_{i\in I}\cf\Cal N(\mu_i))$}

\noindent if $I$ is finite.

\medskip

{\bf (d)(i)} For each $i\in I$ let $A_i\subseteq X_i$ be a non-negligible
set of size $\non\Cal N(\mu_i)$.   Then $A=\prod_{i\in I}A_i$ is not
negligible (251Wm), while
$\#(A)\le\max(\omega,\max_{i\in I}\non\Cal N(\mu_i))$.   If all the
$\non\Cal N(\mu_i)$ are finite, then they are all equal to $1$, and $A$ is
a singleton.   So we must in any case have
$\#(A)=\max_{i\in I}\non\Cal N(\mu_i)$, and
$\non\Cal N(\mu)\le\max_{i\in I}\non\Cal N(\mu_i)$.   By (a), we have
equality.

\medskip

\quad{\bf (ii)} Suppose that $I=\{0,1\}$, and that $\Cal E$ is a cover of
$X=X_0\times X_1$ by negligible sets.   For each $E\in\Cal E$, set
$C_E=\{x:x\in X_0$, $E[\{x\}]\notin\Cal N(\mu_1)\}$;  then $C_E$ is
negligible.   If $\#(\Cal E)<\cov\Cal N(\mu_0)$, then there is an
$x\in X_0\setminus\bigcup_{E\in\Cal E}C_E$;  in which case
$\{E[\{x\}]:E\in\Cal E\}$ witnesses that $\cov\Cal N(\mu_1)\le\#(\Cal E)$.
So $\#(\Cal E)$ must be at least
$\min(\cov\Cal N(\mu_0),\cov\Cal N(\mu_1))$.   As $\Cal E$ is arbitrary,
$\cov\Cal N(\mu)\ge\min(\cov\Cal N(\mu_0),\cov\Cal N(\mu_1))$.

Now an induction on $\#(I)$ (using the associative law 254N) shows that
$\cov\Cal N(\mu)\ge\min_{i\in I}\cov\Cal N(\mu_i)$ whenever $I$ is finite.
Using (a) again, we have equality here also.
}%end of proof of 521J

\cmmnt{\medskip

\noindent{\bf Remark} The simplest applications of (c) here will be when
the $\mu_i$ are \Mth, so that we can take the $\nu_i$ to be the usual
measures on powers $\{0,1\}^{\kappa_i}$ of $\{0,1\}$, and $\nu$ will be
isomorphic to the
usual measure on $\{0,1\}^{\kappa}$ where $\kappa$ is the cardinal sum
$\sum_{i\in I}\kappa_i$.   The cardinal functions of these measures are
dealt with in \S523.   For non-homogeneous $\mu_i$ we shall still be able
to arrange for the $\nu_i$ to be completion regular Radon measures on
dyadic spaces, so that the product measure $\nu$ is again a Radon measure
$Y$ (532F), and (once we have identified its measure algebra -- see 334E,
334Ya) approachable by the methods of \S524.
}%end of comment

\leader{521K}{}\cmmnt{ I turn now to `perfect' and `compact' measure
spaces.    (See \S451 for the basic theory of these.)

\medskip

\noindent}{\bf Proposition} Let $(X,\Sigma,\mu)$ be a perfect
semi-finite measure space which is not purely atomic.   Then

\Centerline{$\add\Cal N(\mu)\le\add\Cal N$,
\quad$\cf\Cal N(\mu)\ge\cf\Cal N$,}

\Centerline{$\shr\Cal N(\mu)\ge\shr\Cal N$,
\quad$\shr^+\Cal N(\mu)\ge\shr^+\Cal N$,
\quad$\pi(\mu)\ge\pi(\mu_L)$\dvAnew{2014}}

\noindent where $\Cal N$ is the null ideal of
Lebesgue measure on $\Bbb R$ and $\mu_L$ is Lebesgue measure on $\Bbb R$.

\proof{{\bf (a)} Suppose first that $\mu$ is a complete atomless
probability measure.   Then there is a function $f:X\to[0,1]$ which is
\imp\ for $\mu$ and Lebesgue measure $\mu_1$ on $[0,1]$ (343Cb again);
and in fact
$\mu_1$ is the image measure $\mu f^{-1}$.   \Prf\ By
451O, $\mu f^{-1}$ is
a Radon measure;  since it extends $\mu_1$ it must actually be equal to
$\mu_1$, by 415H.\ \QeD\
So $\add\Cal N(\mu)\le\add\Cal N(\mu_1)$,
$\cf\Cal N(\mu)\ge\cf\Cal N(\mu_1)$,
$\shr\Cal N(\mu)\ge\shr\Cal N(\mu_1)$,
and $\shr^+\Cal N(\mu)\ge\shr^+\Cal N(\mu_1)$ and $\pi(\mu)\ge\pi(\mu_1)$,
by 521H.
As in the proof of 521I, $([0,1],\Cal N(\mu_1))$ is isomorphic to
$(\Bbb R,\Cal N)$.   Of course $\mu_1$ is not isomorphic to $\mu_L$.
But $\mu_L$ is isomorphic to a direct sum of countably many copies of
$\mu_1$, so by 521G we know that $\pi(\mu_L)$ is the cardinal product
$\omega\cdot\pi(\mu_1)$;  as $\pi(\mu_1)$ is surely infinite, this is
$\pi(\mu_1)$ again.   So we have the result in the special case.

\medskip

{\bf (b)} Now suppose that $(X,\Sigma,\mu)$ is any semi-finite perfect
measure space which is not purely atomic.   Then the completion
$\hat\mu$ of $\mu$ is still a semi-finite perfect measure which is not
purely atomic (212Gd, 451G(c-i)), and $\Cal N(\mu)=\Cal N(\hat\mu)$
(212Eb).
Because $\hat\mu$ is semi-finite and not purely atomic, there is a set
$E\in\Sigma$ of non-zero finite measure such that the subspace measure
$\hat\mu_E$ is atomless.   Set $\nu=\Bover1{\mu E}\hat\mu_E$, so that
$\nu$ is an atomless complete perfect probability measure on $E$, while
$\Cal N(\nu)=\Cal N(\mu_E)$.   Putting (a) together with 521F, we get

\Centerline{$\add\Cal N(\mu)=\add\Cal N(\hat\mu)
\le\add\Cal N(\hat\mu_E)=\add\Cal N(\nu)\le\add\Cal N$}

\noindent and similarly for $\cf$, $\shr$ and $\shr^+$.
}%end of proof of 521K

\leader{521L}{Proposition} (a) Let $(X,\Sigma,\mu)$ be a strictly
localizable measure space and $(Y,\Tau,\nu)$ a
locally compact semi-finite
measure space, and suppose that they have isomorphic measure algebras.
Then $(X,\in,\Cal N(\mu))\prGT(Y,\in,\Cal N(\nu))$;  consequently
$\cov\Cal N(\mu)\le\cov\Cal N(\nu)$ and
$\non\Cal N(\nu)\le\non\Cal N(\mu)$.

(b) Let $(X,\Sigma,\mu)$ be a \Mth\ compact
probability space with Maharam type $\kappa$.   Then
$\cov\Cal N(\mu)\penalty-100=\cov\Cal N_{\kappa}$
and $\non\Cal N(\mu)=\non\Cal N_{\kappa}$, where $\Cal N_{\kappa}$ is the
null ideal of the usual measure\cmmnt{ $\nu_{\kappa}$} on
$\{0,1\}^{\kappa}$.

(c) Let $(X,\Sigma,\mu)$ be a compact strictly localizable measure space
with measure algebra $\frak A$.   Then

\Centerline{$d(\frak A)=\min\{\#(A):A\subseteq X$ has full outer
measure$\}$.}

\proof{{\bf (a)} This follows immediately from 521Ha, because by 343B
there is an \imp\ function from $X$ to $Y$.

\medskip

{\bf (b)} The point is that $\nu_{\kappa}$ is a compact measure
(342Jd, 451Ja), so that we can apply (a) in both directions to see that
$\cov\Cal N(\mu)=\cov\Cal N_{\kappa}$ and
$\non\Cal N(\mu)=\non\Cal N_{\kappa}$.

\medskip

{\bf (c)} The case $\mu X=0$ is trivial;  suppose that $\mu X>0$.   Let
$\Cal K$ be a compact class such that $\mu$ is inner regular with
respect to $\Cal K$.

\medskip

\quad{\bf (i)} Suppose that $\ofamily{\xi}{d(\frak A)}{C_{\xi}}$ is a
family of centered sets in $\frak A$ covering $\frak A^+$.
For each $\xi<d(\frak A)$, set
$\Cal K_{\xi}=\{K:K\in\Cal K\cap\Sigma$, $K^{\ssbullet}\in C_{\xi}\}$;
then $\Cal K_{\xi}$ has the finite intersection property so there is a
point $x_{\xi}\in X\cap\bigcap\Cal K_{\xi}$.   Set
$A=\{x_{\xi}:\xi<d(\frak A)\}$.   If $K\in\Cal K\cap\Sigma$ and
$K\cap A=\emptyset$, then $K\notin\bigcup_{\xi<d(\frak A)}\Cal K_{\xi}$
so $K^{\ssbullet}=0$;  it follows that every measurable subset of
$X\setminus A$ is negligible and $A$ has full outer measure, while
$\#(A)\le d(\frak A)$.

\medskip

\quad{\bf (ii)} Let $\hat\mu$ be the completion of $\mu$, $\hat\Sigma$
its domain and $\theta:\frak A\to\hat\Sigma$ a lifting (341K, 212Gb,
322Da).   Take any $A\subseteq X$ of full outer measure for $\mu$;  then
it also has full outer measure for $\hat\mu$ (212Eb).   For $x\in A$,
set $C_x=\{a:a\in\frak A$, $x\in\theta a\}$;  then $\family{x}{A}{C_x}$
is a family of centered sets in $\frak A$ with union
$\frak A^+$, so $d(\frak A)\le\#(A)$.
}%end of proof of 521L

\leader{521M}{Proposition}\dvArevised{2010}
Let $(X,\Sigma,\mu)$ be a complete locally
determined measure space of magnitude at most $\add\mu$.   Then it is
strictly localizable.

\proof{ Write $\kappa$ for $\add\mu$.
Let $(\frak A,\bar\mu)$ be the measure algebra of $\mu$.   Then
there is a partition of unity $D\subseteq\frak A$ consisting of
elements
of finite measure;  as $\#(D)\le c(\frak A)\le\kappa$, there is a
family $\ofamily{\xi}{\kappa}{a_{\xi}}$ running over $D\cup\{0\}$.
For each $\xi<\kappa$, choose $E_{\xi}\in\Sigma$ such that
$E_{\xi}^{\ssbullet}=a_{\xi}$, and set
$F_{\xi}=E_{\xi}\setminus\bigcup_{\eta<\xi}E_{\eta}$.
Because $E_{\xi}\setminus F_{\xi}=\bigcup_{\eta<\xi}E_{\xi}\cap E_{\eta}$
is the union of fewer than $\add\mu$ negligible sets, it is negligible, and
$F_{\xi}\in\Sigma$, with $F_{\xi}^{\ssbullet}=a_{\xi}$.
Now
$\ofamily{\xi}{\kappa}{F_{\xi}}$ is a disjoint family of sets of
finite measure.   If $E\in\Sigma$ and $\mu E>0$, there is some
$\xi<\kappa$ such that $E^{\ssbullet}\Bcap a_{\xi}\ne 0$, and now
$\mu(E\cap F_{\xi})>0$.
Thus $\ofamily{\xi}{\kappa}{F_{\xi}}$ satisfies the condition of
213Oa, and $\mu$ is strictly localizable.
}%end of proof of 521M

\leader{521N}{Proposition} Let $(X,\Sigma,\mu)$ be a complete locally
determined localizable measure space of magnitude at most $\frak c$.
Then it is strictly localizable.

\proof{ Again let $(\frak A,\bar\mu)$ be the measure algebra of $\mu$,
and take a partition of unity $D\subseteq\frak A$ consisting of
elements
of finite measure;  as $\#(D)\le c(\frak A)\le\frak c$, there is an
injective function $h:D\to\Cal P\Bbb N$.   This time, because $\frak A$
is Dedekind complete, we can set $b_n=\sup\{d:d\in D$, $n\in h(d)\}$ for
each $n\in\Bbb N$.   If $d\in D$, then
$d=\inf_{n\in h(d)}b_n\Bsetminus\sup_{n\in\Bbb N\setminus h(d)}b_n$.
So if we choose $E_n\in\Sigma$ such that $E_n^{\ssbullet}=b_n$ for
each $n$, and set $F_d
=\bigcap_{n\in h(d)}E_n\setminus\bigcup_{n\in\Bbb N\setminus h(d)}E_n$
for $d\in D$, $\family{d}{D}{F_d}$ will be a disjoint family in
$\Sigma$
and $F_d^{\ssbullet}=d$ for every $d$.   Now $\mu F_d=\bar\mu d$ is
always finite;  and if $E\in\Sigma$ is non-negligible, there is a
$d\in D$ such that $0\ne\bar\mu(E^{\ssbullet}\Bcap d)=\mu(E\cap F_d)$.
Thus $\family{d}{D}{F_d}$ satisfies the condition of 213Oa, and $\mu$
is strictly localizable.
}%end of proof of 521N

\leader{521O}{Proposition} (a) If $(X,\Sigma,\mu)$ is a semi-finite
measure space, its magnitude is at most $\max(\omega,2^{\#(X)})$.

(b) If $(X,\Sigma,\mu)$ is a strictly localizable measure space, its
magnitude is at most $\max(\omega,\penalty-100\#(X))$.

(c) There is an infinite semi-finite measure space $(X,\Sigma,\mu)$
with magnitude $2^{\#(X)}$.

\proof{{\bf (a)-(b)} These are elementary.   If $(X,\Sigma,\mu)$ is
semi-finite, with measure algebra $\frak A$, then

\Centerline{$c(\frak A)\le\#(\frak A)\le\#(\Sigma)
\le\#(\Cal PX)=2^{\#(X)}$.}

\noindent If $\mu$ is strictly localizable, with decomposition
$\familyiI{X_i}$, then $\familyiI{X_i^{\ssbullet}}$ is a partition of
unity in $\frak A$ consisting of elements of finite measure, so

\Centerline{$c(\frak A)
\le\max(\omega,\#(\{i:i\in I$, $\mu X_i>0\}))\le\max(\omega,\#(X))$}

\noindent by 332E.

\medskip

{\bf (c)} Let $\ofamily{\xi}{\omega_1}{X_{\xi}}$ be a disjoint family
of sets such that
$\#(X_{\xi})=\#(\Cal P(\bigcup_{\eta<\xi}X_{\eta}))$ for every
$\xi<\omega_1$;  for each $\xi$, let
$h_{\xi}:\Cal P(\bigcup_{\eta<\xi}X_{\eta})\to X_{\xi}$ be an
injection.
Set $X=\bigcup_{\xi<\omega_1}X_{\xi}$.   For $A\subseteq X$ define
$f_A:\omega_1\to X$ by setting
$f(\xi)=h_{\xi}(A\cap\bigcup_{\eta<\xi}X_{\eta})$ for each $\xi$;  let
$J_A$ be $f_A[\omega_1]$ and $\mu_A$ the countable-cocountable measure
on $J_A$.    Observe that $\#(J_A)=\omega_1$ for every $A\subseteq X$,
and that if $A$, $B\subseteq X$ are distinct then $J_A\cap J_B$ is
countable.   So if we set $\mu E=\sum_{A\subseteq X}\mu_A(E\cap J_A)$
whenever $E\subseteq X$ is such that $E\cap J_A$ is countable or
cocountable in $J_A$ for every $A$, then $\mu$ will be a complete locally
determined measure on $X$.   Since $\mu J_A=1$ and $\mu(J_A\cap J_B)=0$
whenever $A$, $B\subseteq X$ are distinct, $\mu$ has magnitude
$2^{\#(X)}$.
}%end of proof of 521O

\leader{521P}{Proposition} (a) If $2^{\lambda}<2^{\kappa}$ whenever
$\frak c\le\lambda<\kappa$ and $\cf\lambda>\omega$, then
the magnitude\cmmnt{ $\magnitude\mu$} of $\mu$ is at most
$\max(\omega,\#(X))$ for every localizable measure space $(X,\Sigma,\mu)$.

(b) Suppose that $2^{\frak c}=2^{\frak c^+}$.   Then there is a
localizable measure space $(Y,\Tau,\nu)$ with $\#(Y)=\frak c$ and
$\magnitude\nu=\frak c^+$.

\medskip

\noindent{\bf Remark}\cmmnt{ \TeX, for once, is obscure;}
$2^{\frak c^+}$ here is $\#(\Cal P(\frak c^+))$\cmmnt{, not
$(2^{\frak c})^+$}.

\proof{{\bf (a)} If $\magnitude\mu\le\omega$ we can stop.   Otherwise,
set $\kappa=\magnitude\mu$.   Let $(\frak A,\bar\mu)$
be the measure algebra of $\mu$, so that $\kappa=c(\frak A)$ and there
is a disjoint family $\ofamily{\xi}{\kappa}{a_{\xi}}$ in
$\frak A^+$ (332F).   If $\tilde\mu$ is the c.l.d.\
version of $\mu$, we can identify
$\frak A$ with the measure algebra of $\tilde\mu$ (322Db).

\medskip

\quad{\bf case 1} If $\kappa\le\frak c$, $\tilde\mu$ is strictly
localizable (521N), so has a lifting $\theta$ (341K again);  but now
$\ofamily{\xi}{\kappa}{\theta a_{\xi}}$ is a disjoint family of
non-empty subsets of $X$, so $\#(X)\ge\kappa$.

\medskip

\quad{\bf case 2} If $\kappa>\frak c$, of course $X$ is uncountable
(521Oa).   For $\xi<\kappa$, choose $E_{\xi}\in\Sigma$ such that
$E_{\xi}^{\ssbullet}=a_{\xi}$.
\Quer\ If $\#(X)<\kappa$, there is a set $Y\subseteq X$ such that
$\#(Y)$ has uncountable cofinality and $I_Y=\{\xi:\xi<\kappa$,
$\mu^*(E_{\xi}\cap Y)>0\}$ has cardinal greater than
$\max(\frak c,\#(X))$.   \Prf\ If $\cf(\#(X))$ is uncountable, take
$Y=X$.   Otherwise, let $\sequencen{Y_n}$ be an increasing sequence of
subsets of $X$, with union $X$, such that $\#(Y_n)$ is an uncountable
successor cardinal less than $\#(X)$ for every $n$.   If $\xi<\kappa$,
there is some $n$ such that $E_{\xi}\cap Y_n$ is non-negligible, that
is, $\xi\in I_{Y_n}$.   So the non-decreasing sequence
$\sequencen{I_{Y_n}}$ has union $\kappa$, and
there is some $n\in\Bbb N$ such that
$\#(I_{Y_n})>\max(\frak c,\#(X))$.   Now we can take $Y=Y_n$.\ \Qed

For every $J\subseteq I_Y$, set $b_J=\sup_{\xi\in J}a_\xi$ and let
$F_J\in\Sigma$ be such that $F_J^{\ssbullet}=b_J$.   If $J$,
$K\subseteq I_Y$ are distinct, there is a $\xi\in J\symmdiff K$, in
which case
$a_\xi\Bsubseteq b_J\Bsymmdiff b_K$, $E_\xi\setminus(F_J\symmdiff F_K)$
is negligible and $Y\cap(F_J\symmdiff F_K)$ is non-empty.   Thus
$J\mapsto Y\cap F_J:\Cal PI_Y\to\Cal PY$ is injective, and

\Centerline{$2^{\#(I_Y)}\le 2^{\#(Y)}\le 2^{\#(X)}\le 2^{\#(I_Y)}$.}

\noindent Setting $\lambda=\max(\frak c,\#(Y))$, $\kappa'=\#(I_Y)$ we
now have $\frak c\le\lambda<\kappa'$, $\cf\lambda>\omega$ and
$2^{\lambda}=2^{\kappa'}$, which is supposed to be impossible.\ \Bang

So in this case also we have $\#(X)\ge\kappa$.

\medskip

{\bf (b)(i)} Set $I=\Cal P\frak c^+$ and
$X=\{0,1\}^I\cong\{0,1\}^{2^{\frak c}}$.   Putting 5A4Be and 5A4C(a-ii)
together, we see that there is a set $Y\subseteq X$, with cardinal
at most $\frak c^{\omega}=\frak c$, which meets every non-empty
G$_{\delta}$ subset of $X$.   In particular, if $\Cal K\subseteq I$ is
countable and $x\in X$ there is a $y\in Y$ such that
$y\restr\Cal K=x\restr\Cal K$.

\medskip

\quad{\bf (ii)} Let $\mu$ be the complete locally determined
localizable
measure on $X$ described in 216E, with $C=\frak c^+$.   Then $Y$ has
full outer measure in $X$.   \Prf\ (I follow the notation and argument
of 216E.)   If
$\mu E>0$, then, by the argument of part (g) of the proof of 216E,
there are a $\gamma<\frak c^+$ and a $K\in[I]^{\le\omega}$ such that
$F_{\gamma K}\subseteq E$, where
$F_{\gamma K}=\{x:x\restr K=x_{\gamma}\restr K\}$.   But $Y$ was
chosen to meet every such set.   As $E$ is arbitrary, $Y$ has full outer
measure.\ \Qed

\medskip

\quad{\bf (iii)} $\magnitude\mu=\frak c^+$.   \Prf\ In the language of
216E, we have a family $\ofamily{\gamma}{\frakc^+}{G_{\{\gamma\}}}$ of
$\mu$-atoms of measure $1$, each pair with negligible intersection,
and every non-negligible measurable set meets some $G_{\{\gamma\}}$ in a
non-negligible set.\ \Qed

\medskip

\quad{\bf (iv)} Now let $\nu$ be the subspace measure on $Y$.   By
214Ie, $\nu$ is complete, locally determined and localizable.   By
322I, we can identify the measure algebras of $\mu$ and $\nu$, so
$\magnitude\nu=\magnitude\mu=\frak c^+$, while $\#(Y)=\frak c$.
}%end of proof of 521P


\leader{521Q}{Free \dvrocolon{products}}\cmmnt{ We have some simple
calculations associated with the measure algebra free products of \S325.

\medskip

\noindent}{\bf Proposition} (a) Let $(\frak A,\bar\mu)$ and
$(\frak B,\bar\nu)$ be semi-finite measure algebras and
$(\frak C,\bar\lambda)$ their localizable measure algebra free product.
Then

\Centerline{$c(\frak C)\le\max(\omega,c(\frak A),c(\frak B))$,}

\Centerline{$\tau(\frak C)\le\max(\omega,\tau(\frak A),\tau(\frak B))$.}

\wheader{521Q}{0}{0}{0}{36pt}
(b) Let $\familyiI{(\frak A_i,\bar\mu_i)}$ be a family of probability
algebras, and $(\frak C,\bar\lambda)$ their probability algebra free
product.   Then

\Centerline{$\tau(\frak C)
\le\max(\omega,\#(I),\sup_{i\in I}\tau(\frak A_i))$.}

\proof{{\bf (a)(i)} Let $A\subseteq\frak A$, $B\subseteq\frak B$ be
partitions of unity consisting of elements of
finite measure (322Ea).
Then $C=\{a\otimes b:a\in A$, $b\in B\}$ is a disjoint family in
$\frak C$, and

$$\eqalignno{\sup C
&=\sup\{(a\otimes 1)\Bcap(1\otimes b):a\in A,\,b\in B\}
=(\sup_{a\in A}a\otimes 1)\Bcap(\sup_{b\in B}1\otimes b)\cr
\displaycause{313Bc}
&=(\sup A\otimes 1)\Bcap(1\otimes\sup B)\cr
\displaycause{325Da}
&=(1\otimes 1)\Bcap(1\otimes 1)
=1;\cr}$$

\noindent that is, $C$ is a partition of unity.   As every member of $C$
has finite measure,

\Centerline{$c(\frak C)
\le\max(\omega,\#(C))
=\max(\omega,\#(A),\#(B))
=\max(\omega,c(\frak A),c(\frak B))$}

\noindent by 332E.

\medskip

\quad{\bf (ii)} As for Maharam types, I am just repeating the result stated
and proved in 334B.

\medskip

{\bf (b)} This is 334D.
}%end of proof of 521Q

\leader{521R}{Proposition} If $(X,\Sigma,\mu)$ is any measure space,
its Maharam type is at most $2^{\#(X)}$.

\proof{ If $\frak A$ is the measure algebra of $\mu$,

\Centerline{$\tau(\frak A)\le\#(\frak A)\le\#(\Sigma)\le\#(\Cal PX)
=2^{\#(X)}$.}

}%end of proof of 521R

\leader{521S}{Proposition} (a) A countably separated measure space
has Maharam type at most $2^{\frakc}$.

(b) There is a countably separated quasi-Radon probability space with
Maharam type $2^{\frakc}$.

(c) A countably separated semi-finite measure space has magnitude at
most $2^{\frakc}$.

(d) There is a countably separated semi-finite measure space with
magnitude $2^{\frakc}$.

\proof{ Set $\kappa=2^{\frakc}$.

\medskip

{\bf (a)} If $(X,\Sigma,\mu)$ is countably separated, there is an
injective function from $X$ to $\Bbb R$ (343E), so $\#(X)\le\frak c$;
now use 521R.

\medskip

{\bf (b)} As in (b-i) of the proof of 521P, there is a set
$Y\subseteq X=\{0,1\}^{\kappa}$, with cardinal
$\frak c$, which meets every non-empty G$_{\delta}$ subset of $X$, and
therefore has full outer measure for the usual measure
$\nu_{\kappa}$ of $X$.

In $[0,1]$ let $\family{y}{Y}{C_y}$
be a disjoint family of sets of full outer measure for Lebesgue
measure $\mu_1$ on $[0,1]$ (419I), and set
$C=\{(y,t):y\in Y,\,t\in C_y\}\subseteq Z=X\times[0,1]$.   Now $C$ has
full outer measure for the product measure $\lambda$ on $Z$.   \Prf\
Suppose that $W\subseteq Z$ and $\lambda W>0$.   Then
$\int\mu_1 W[\{x\}]\nu_{\kappa}(dx)>0$ (252D), so
$\{x:\mu_1W[\{x\}]>0\}$ has non-zero measure and meets $Y$.   Taking
$y\in Y$ such that $\mu_1W[\{y\}]>0$, $\{y\}\times(C_y\cap W[\{y\}])$
is a non-empty subset of $C\cap W$.\ \QeD\

The measure algebra $\frak A$ of the subspace measure $\lambda_C$ on $C$
can therefore be identified with the measure algebra of $\lambda$ (322Jb),
and has Maharam type $\kappa$.   Because $\family{y}{Y}{C_y}$ is disjoint,
each horizontal section of $C$ is a singleton and $C$ is separated by the
measurable sets $C\cap(X\times[0,q])$ for $q\in\Bbb Q\cap[0,1]$.
Thus $\lambda_C$ is countably separated.

If we give $Z$ the product topology, then $\lambda$ is a Radon measure
(417T, or otherwise), so $\lambda_C$ is quasi-Radon for the subspace
topology (415B).

\medskip

{\bf (c)} As in (a), $\#(X)\le\frak c$, so we can use 521Oa.

\medskip

{\bf (d)(i)} The first step is to build a measure space of magnitude
$2^{\frakc}$ and cardinal $\frak c$, as follows.
Let $h:\frak c\to([\frak c]^{\le\omega})^2$ be a surjection;
take its two components to be $h_1$ and $h_2$.   For $D\subseteq\frak c$
set $F_D=\{\xi:\xi<\frak c$, $h_2(\xi)=D\cap h_1(\xi)\}$.   For
$I\in[\frak c]^{\le\omega}$ set
$A_I=\{\xi:\xi<\frak c$, $I\not\subseteq h_1(\xi)\}$, and set
$\Cal A=\bigcup\{\Cal PA_I:I\in[\frak c]^{\le\omega}\}$;  note that
$\Cal A$ is a $\sigma$-ideal of subsets of $\frak c$.

If $D\subseteq\frak c$, $F_D\notin\Cal A$.   \Prf\ If
$I\in[\frak c]^{\le\omega}$, there is a $\xi<\frak c$ such that
$h(\xi)=(I,I\cap D)$,   Now $\xi\in F_D\setminus A_I$;  as $I$ is
arbitrary, $F_D\notin\Cal A$.\ \QeD\  So we can define a measure $\nu_D$
on $\frak c$ by saying that

$$\eqalign{\nu_D(E)
&=1\text{ if }E\subseteq\frak c\text{ and }F_D\setminus E\in\Cal A,\cr
&=0\text{ if }E\subseteq\frak c\text{ and }F_D\cap E\in\Cal A,\cr
&\text{undefined otherwise},\cr}$$

\noindent and $\nu_DF_D=1$.

If $D$, $D'\subseteq\frak c$ are distinct, $F_D\cap F_{D'}\in\Cal A$.
\Prf\ Take $\eta\in D\symmdiff D'$.   If $\xi\in F_D\cap F_{D'}$, then
$D\cap h_1(\xi)=h_2(\xi)=D'\cap h_1(\xi)$, so $\eta\notin h_1(\xi)$ and
$\xi\in A_{\{\eta\}}$.   Thus
$F_D\cap F_{D'}\subseteq A_{\{\eta\}}\in\Cal A$.\ \Qed

So if we set $\nu=\sum_{D\subseteq\frak c}\nu_D$, as defined in
234G, $\nu$ is a measure on
$\frak c$ such that $\nu F_D=1$ and $\nu(F_D\cap F_{D'})=0$ for all
distinct $D$, $D'\subseteq\frak c$.   Also $\nu$ is semi-finite, because if
$\nu E>0$ there is a $D\subseteq\frak c$ such that $\nu_DE>0$, in which
case $\nu(E\cap F_D)=\nu_D(E\cap F_D)=1$.   So $\nu$ is a semi-finite
measure on $\frak c$ with magnitude $\#(\Cal P\frak c)=2^{\frakc}$.
Because every $\nu_D$ is complete, so is $\nu$ (234Ha).

\medskip

\quad{\bf (ii)} As in (b),
let $\ofamily{\xi}{\frak c}{C_{\xi}}$ be a disjoint family of
subsets of $[0,1]$ all with full outer measure for Lebesgue measure
$\mu_1$.   Set $Z=\frak c\times[0,1]$ with its c.l.d.\ product measure
$\lambda=\nu\times\mu_1$, and
$C=\{(\xi,t):\xi<\frak c$, $t\in C_{\xi}\}\subseteq Z$.   Then
$C$ has full outer measure, by the argument of (b) above.   So, as in
(b), the measure algebra $\frak A$ of the subspace measure $\lambda_C$
on $C$ can be identified with the measure algebra of
$\lambda$.   The map $E\mapsto C\cap(E\times[0,1])$ induces a
measure-preserving homomorphism from the measure algebra of $\nu$ to
$\frak A$, so $\magnitude\lambda_C=c(\frak A)$ is at least $2^{\frakc}$;
by (c), it is exactly $2^{\frakc}$.
Also as in (b), $\lambda_C$ is countably separated.
}%end of proof of 521S

\leader{521T}{}\cmmnt{ In \S464 I looked at the $L$-space $M$ of
bounded additive functionals on $\Cal PI$ for infinite sets $I$, of
which $I=\Bbb N$ is of course by far the most important, and found a
band decomposition of $M$ as
$M_{\tau}\oplus(M_{\text{m}}\cap M_{\tau}^{\perp})
\oplus M_{\text{pnm}}$,
where $M_{\tau}$ consists of  the `completely additive' functionals
(and may be identified with $\ell^1(I)$), $M_{\text{m}}$ consists of the
`measurable' functionals (that is, those integrated by the usual
measure
on $\Cal PI$), and $M_{\text{pnm}}=M_{\text{m}}^{\perp}$ consists of
the `purely non-measurable' functionals.
Any non-negative functional $\theta\in M$ can be identified with a Radon
measure $\mu_{\theta}$ on the Stone-\v{C}ech compactification $\beta I$
(464P).   The purely atomic measures on $I$ correspond to members of
$M_{\tau}$.   Among the others, the general rule is that `simple'
measures must correspond to functionals in $M_{\text{pnm}}$;  see 464Pa
and 464Xa.   The next proposition, strengthening 464Qb, shows that
this rule is followed by Maharam types.

\medskip

\noindent}{\bf Proposition}\dvArevised{2014}
Let $I$ be a set, and suppose that a
non-zero $\theta\in(M_{\text{m}}\cap M_{\tau}^{\perp})^+$, as defined
in \S464, corresponds to the Radon measure $\mu_{\theta}$ on $\beta I$.
Let $\nu$ be the usual measure on $\Cal PI$.   Then
the Maharam type of $\mu_{\theta}$ is at least $\cov\Cal N(\nu)$.

\proof{ Of course $I$ has to be infinite, since not every
additive functional on $\Cal PI$ is completely additive;  so
$\cov\Cal N(\nu)$ is not $\infty$.   By 464Qc, we know that

\Centerline{$\{(a,b):a$, $b\subseteq I$,
$\theta a=\Bover12\theta I$,
$\theta(a\cap b)=\Bover14\theta I\}$}

\noindent is conegligible for the product measure $\nu\times\nu$ on
$(\Cal PI)^2$.   Set

\Centerline{$A_0=\{a:a\subseteq I$, $\theta a=\Bover12$,
$\{b:\theta(a\cap b)=\Bover14\theta I\}$ is
$\nu$-conegligible$\}$;}

\noindent then $A_0$ is $\nu$-conegligible.   Now take a set
$A\subseteq A_0$ which is maximal subject to the requirement that
$\theta(a\cap b)=\bover14\theta I$ for all distinct $a$, $b\in A$.   If
$a$ is any subset of $I$, then either $a\in A$, or $a\notin A_0$, or
there is a $b\in A\setminus\{a\}$
such that $\theta(a\cap b)\ne\bover14\theta I$;  so
$\Cal PI$ is the union of

\Centerline{$(\Cal PI\setminus A_0)
  \cup\bigcup_{b\in A}\{a:\theta(a\cap b)\ne\Bover14\theta I\}$}

\noindent and $\cov\Cal N(\nu)\le 1+\#(A)$.
As $\cov\Cal N(\nu)$ is surely
infinite, it is in fact less than or equal to $\#(A)$.

Now consider the open-and-closed sets $\widehat{a}\subseteq\beta I$ for
$a\in A$.   If $a$, $b\in A$ are distinct,

\Centerline{$\mu_{\theta}(\widehat{a}\symmdiff\widehat{b})
=\mu_{\theta}(\widehat{a\symmdiff b})
=\theta(a\symmdiff b)=\bover12\theta I>0$.}

\noindent So in the measure algebra $\frak A$ of $\mu_{\theta}$,
$\{\widehat{a}^{\ssbullet}:a\in A\}$ is a discrete set of size
at least $\cov\Cal N(\nu)$, and the topological density of
$\frak A$ is at least $\cov\Cal N(\nu)$ (5A4B(h-ii) again).
By 521E, $\tau(\mu_{\theta})=\tau(\frak A)\ge\cov\Cal N(\nu)$.
}%end of proof of 521T

\exercises{\leader{521X}{Basic exercises (a)}
%\spheader 521Xa
Let $\Cal B(\Bbb R)$ be the Borel $\sigma$-algebra of $\Bbb R$,
and $\mu$ the restriction of Lebesgue measure to $\Cal B(\Bbb R)$.
Show that $\add\mu=\omega_1$.   \Hint{if $\frak c=\omega_1$,
$[\Bbb R]^{\omega_1}\not\subseteq\Cal B(\Bbb R)$;  if $\frak c>\omega_1$,
$[\Bbb R]^{\omega_1}\cap\Cal B(\Bbb R)=\emptyset$;
or use 423L and 423P.}
%521A

\spheader 521Xb Let $(X,\Sigma,\mu)$ be a semi-finite
locally compact measure
space.   Show that $\add\mu$ is the least cardinal of any set
$\Cal E\subseteq\Sigma$ such that $\bigcup\Cal E\notin\Sigma$, or
$\infty$ if there is no such $\Cal E$.   \Hint{451Q.}
%521A+

\spheader 521Xc\dvAnew{2010} Let $(X,\Sigma,\mu)$ be a complete locally
determined measure space, and $\kappa$ a cardinal such that
$\kappa<\cov\Cal N(\mu_E)$ for every non-negligible measurable set
$E\subseteq X$, writing $\mu_E$ for the subspace measure.
Suppose that $A\subseteq X$ is such that both $A$ and
$X\setminus A$ are expressible as the union of at most $\kappa$ members of
$\Sigma$.   Show that $A\in\Sigma$.
%521B+

\sqheader 521Xd\dvAnew{2013}(i) Find a probability space $(X,\Sigma,\mu)$,
with measure algebra $\frak A$, such that $\pi(\frak A)<\pi(\mu)$.
%\pi(\frak A)=1$
(ii) Find a probability space $(X,\Sigma,\mu)$, with null ideal
$\Cal N(\mu)$, such that $\cf\Cal N(\mu)<\pi(\mu)$.
%\cf\Cal N=1
(iii) Find a probability space $(X,\Sigma,\mu)$
such that $\pi(\mu)<\cf\Cal N(\mu)$.   \Hint{513X(q-iii).}
%direct sum with $\cf\Cal N(\mu_n)=\pi(\mu_n)=\omega_n$.
%521D

\spheader 521Xe\dvArevised{2014}
Let $(X,\Sigma,\mu)$ be a measure space and $\nu$ an
indefinite-integral measure over $\mu$.   Show that
$\add\Cal N(\nu)\ge\add\Cal N(\mu)$, $\cf\Cal N(\nu)\le\cf\Cal N(\mu)$,
$\non\Cal N(\nu)\ge\non\Cal N(\mu)$, $\cov\Cal N(\nu)\le\cov\Cal N(\mu)$,
$\shr\Cal N(\nu)\le\shr\Cal N(\mu)$,
$\shr^+\Cal N(\nu)\le\shr^+\Cal N(\mu)$, $\pi(\nu)\le\pi(\mu)$,
$\tau(\nu)\le\tau(\mu)$.
%322K 521F

\spheader 521Xf\dvAnew{2013} Let $(X,\Sigma,\mu)$ be a
semi-finite measure space which is not purely atomic.   Show that
$\pi(\mu)\ge\pi(\mu_L)$, where $\mu_L$ is Lebesgue measure on $\Bbb R$.
%521H

\spheader 521Xg Let $(X,\Sigma,\mu)$ be an atomless measure space with
locally determined negligible sets (definition: 213I).   Show that
$\non\Cal N(\mu)\ge\non\Cal N$, where $\Cal N$ is the null ideal of
Lebesgue measure.
%521I

\spheader 521Xh Let $(X,\Sigma,\mu)$ and $(Y,\Tau,\nu)$ be complete
locally determined measure spaces, neither of measure $0$,
and $\mu\times\nu$ the c.l.d.\ product measure on $X\times Y$.   Show that

\Centerline{$\non\Cal N(\mu\times\nu)
=\max(\non\Cal N(\mu),\non\Cal N(\nu))$,}

\Centerline{$\cov\Cal N(\mu\times\nu)
\le\min(\cov\Cal N(\mu),\cov\Cal N(\nu))$}

\noindent with equality if either $\mu$ or $\nu$ is strictly localizable,

\Centerline{$\add(\mu\times\nu)=\add\Cal N(\mu\times\nu)
\le\min(\add\Cal N(\mu),\add\Cal N(\nu))$,}

\Centerline{$\cf\Cal N(\mu\times\nu)
\ge\max(\cf\Cal N(\mu),\cf\Cal N(\nu))$,}

\Centerline{$\shr\Cal N(\mu\times\nu)
\ge\max(\shr\Cal N(\mu),\shr\Cal N(\nu))$,}

\Centerline{$\shr^+\Cal N(\mu\times\nu)
\ge\max(\shr^+\Cal N(\mu),\shr^+\Cal N(\nu))$,}

\Centerline{$\pi(\mu\times\nu)\ge\max(\pi(\mu),\pi(\nu))$.\dvAnew{2013}}
%521J

\spheader 521Xi Let $(X,\Sigma,\mu)$ be a probability space, and
$\mu^{\Bbb N}$ the product measure on $X^{\Bbb N}$.
(i) Show that $X$ has a set of full
outer measure of size at most $\non\Cal N(\mu^{\Bbb N})$.
(ii) Show that if $\Cal A\subseteq\Sigma\setminus\Cal N(\mu)$ and
$\#(\Cal A)<\cov\Cal N(\mu^{\Bbb N})$, then there is a countable set
which meets every member of $\Cal A$.
%521J

\spheader 521Xj Show that the direct sum of $\frak c$ or fewer
countably separated measure spaces is countably separated.
%521S

\spheader 521Xk Show that $2^{\frak c}<2^{\frak c^+}$ iff every
countably separated complete locally determined localizable measure
space is strictly localizable.  \Hint{521P, 521S, 252Yp.}
%521S 521P mt52bits

\spheader 521Xl Show that if $(X,\Sigma,\mu)$ is a purely atomic
countably separated semi-finite measure space then its magnitude is at
most $\max(\omega,\#(X))$ and its Maharam type is countable.
%521S

\spheader 521Xm Suppose that $2^{\kappa}\le\frak c$ for every
$\kappa<\frak c$.   Show that there is a countably separated
semi-finite measure space with magnitude $2^{\frakc}$.
%521Sd mt52bits

\leader{521Y}{Further exercises (a)}
%\spheader 521Ya
Find a probability space $(X,\Sigma,\mu)$, a set $Y$ and
a function $f:X\to Y$ such that, setting $\nu=\mu f^{-1}$,
$\add\Cal N(\mu)>\add\Cal N(\nu)$, $\cf\Cal N(\mu)<\cf\Cal N(\nu)$,
$\shr\Cal N(\mu)<\shr\Cal N(\nu)$ and $\pi(\mu)<\pi(\nu)$.
%521H  mt52bits

\spheader 521Yb Find a strictly localizable measure space
$(X,\Sigma,\mu)$, a set $Y$, and a
function $f:X\to Y$ such that, setting $\nu=\mu f^{-1}$, $\nu$ is
semi-finite and $\tau(\mu)<\tau(\nu)$.
%521H mt52bits

\spheader 521Yc Let $(X,\Sigma,\mu)$ and $(Y,\Tau,\nu)$ be localizable
measure spaces, and suppose that
\penalty-100$\max(\magnitude(\nu),\discretionary{}{}{}\tau(\nu))\le\frak c$.
Show that the c.l.d.\
product measure on $X\times Y$ is localizable.
%521N mt52bits
%what if just  \tau(\nu)\le\frak c ?

\spheader 521Yd Show that there is a probability space
$(X,\Sigma,\mu)$ with Maharam type greater than $\#(X)$.   \Hint{523Ib.}
%521R

\spheader 521Ye
Let $\kappa$ be an infinite cardinal.   Let us say that a measure
space $(X,\Sigma,\mu)$ is {\bf $\kappa$-separated} if there is a family
$\Cal E\subseteq\Sigma$, with cardinal at most $\kappa$, separating the
points of $X$.
(i) Show that there is a disjoint family $\Cal A$ of
subsets of $\{0,1\}^{\kappa}$, all of full outer measure for the usual
measure of $\{0,1\}^{\kappa}$, such that $\#(\Cal A)=2^{\kappa}$.
(ii) Show that every $\kappa$-separated measure space has Maharam type
at most $2^{2^{\kappa}}$,  %521Sa
and that there is a $\kappa$-separated quasi-Radon
probability space with Maharam type $2^{2^{\kappa}}$.   %521Sb
(iii) Show
that every semi-finite $\kappa$-separated measure space has magnitude at
most $2^{2^{\kappa}}$, %521Sc
and that there is a semi-finite $\kappa$-separated
measure space with magnitude greater than $2^{\kappa}$.   %521Sd
(iv) Suppose that $\frak c\le\kappa\le\lambda$ and
$2^{\kappa}=2^{\lambda}$.   Show that the usual measure on
$\{0,1\}^{\lambda}$ is $\kappa$-separated.  %mt52bits
\leaveitout{?? (v) Show that for every
uncountable cardinal $\kappa$ there is a compact $\kappa$-separated
Maharam-type-homogeneous Radon
probability space with Maharam type $\kappa$ which is not isomorphic to
$\{0,1\}^{\kappa}$ with its usual measure.  See \S344 notes.}
%521S
}%end of exercises

\leaveitout{
%Problem:  which of the following are possible?
%   \add\Cal N(\mu\times\nu) < \min(\add\Cal N(\mu),\add\Cal N(\nu))
%   \cf\Cal N(\mu\times\nu) > \max(\cf\Cal N(\mu),\cf\Cal N(\nu))
%   \shr\Cal N(\mu\times\nu) > \max(\shr\Cal N(\mu),\shr\Cal N(\nu))

\leader{}{Problem} Must there be a family
$\ofamily{\xi}{\omega_1}{\Cal F_{\xi}}$ of distinct ultrafilters on
$\Bbb N$ such that $\bigcap_{\xi<\omega_1}\Cal F_{\xi}$ is measurable?
}%end of leaveitout

\endnotes{\Notesheader{521}
The cardinal functions of an ideal can be thought of as measures of the
`complexity' of that ideal.   In a measure space, it is natural to suppose
that a subspace measure (at least, on a measurable subspace) will be
`simpler' than the original measure;  in 521F we see that the additivity
and uniformity tend to
rise and the covering number, cofinality, shrinking
number and $\pi$-weight tend to fall.
Similarly, an image measure ought to be simpler than
its parent;  but here, while additivity rises and cofinality and shrinking
number fall, uniformity falls and covering number rises (521H).
Also there is a trap if the original measure is not complete (521Ya), and
$\pi$-weight is more complicated (521H(a-ii)).   There is a similar
problem concerning topological $\pi$-weight, which led to the concept
of network weight (5A4Ai, 5A4Bc);  and just as network weight matches
topological weight for compact Hausdorff spaces (5A4C(a-i)), an appropriate
hypothesis on our measures can make their $\pi$-weights more coherent
(521H(a-ii)).

Direct sums should not be more complex than their most complex component;
521G confirms this prejudice except in respect of cofinality.   Since we
are looking, in effect, at the cofinality of a product of partially ordered
sets, we can expect at least as many difficulties as are to be found in pcf
theory (\S5A2).   We should like to be able to bound the complexity of a
product in terms of the complexities of the factors;  here there seem to be
some interesting questions, and 521J and 521Xh are, I hope, only a start.

Consider the statement

\Centerline{($\dagger$) `$\magnitude\mu\le\#(X)$ for every localizable
measure space $(X,\Sigma,\mu)$'.}

\noindent From 521P we see that the generalized continuum hypothesis
implies ($\dagger$), and also that there are simple models of set
theory in which ($\dagger$) is false ({\smc Kunen 80}, VIII.4.7;
{\smc Jech 03}, 15.18).   I do
not know whether there is a natural combinatorial statement
equiveridical with ($\dagger$).   If we amend ($\dagger$) to

\Centerline{`$\magnitude\mu\le\#(X)$ for every countably separated
localizable measure space $(X,\Sigma,\mu)$'}

\noindent we find ourselves with a statement equiveridical with
`$2^{\frak c}=2^{\frak c^+}$' (cf.\ 521Xk).

I give space to `countably separated' measures because these can be
identified with the topological measures on subsets of $\Bbb R$, and I
do not think it is immediately apparent just how complicated these can
be.   In fact, as shown by the proofs of parts (b) and (d) in 521S,
most
of the phenomena which can arise in any measure space with cardinal less
than or equal to $\frak c$ can appear in countably separated measure
spaces.   In 521Sb I add `quasi-Radon' to show that the very strong
restrictions on countably separated Radon probability measures (522Wa)
depend on their perfectness, not on their $\tau$-additivity.

The constructions in 521Oc and 521Sd both depend on almost-disjoint
families of sets.   Those described here are elementary.   In many
models of set theory, we have much more striking results, of which
521Xm is a simple example.

Some new considerations intrude rather abruptly
in 521T, but the argument here is both elementary and important, quite
apart from its use in helping us to understand the classification
scheme in \S464.
}%end of notes

\discrpage

