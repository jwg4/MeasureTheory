\frfilename{mt5.tex} 
\versiondate{9.1.15} 
 
\noindent{\bf Introduction to Volume 5} 
 
\medskip 
      
For the final volume of this treatise, I have collected results which 
demand more sophisticated set theory than elsewhere.   The line is not 
sharp, but typically we are much closer to questions which are undecidable 
in ZFC.   Only in Chapter 55 are these brought to the forefront of the 
discussion, but elsewhere much of the work depends on formulations 
carefully chosen to express, as arguments in ZFC, ideas which arose in 
contexts in which some special axiom -- Martin's axiom, for instance -- was 
being assumed.   This has forced the development of concepts -- e.g., 
cardinal functions of structures -- which have taken on vigorous lives of  
their own, and 
which stand outside the territory marked by the techniques of earlier 
volumes. 
 
In terms of the classification I have used elsewhere, this volume has  
one preparatory chapter and five working chapters.   There is practically 
no measure theory in Chapter 51, which is an introduction to some of the 
methods which have been devised to make sense of abstract analysis in 
the vast range of alternative mathematical 
worlds which have become open to us in the last fifty years. 
It is centered on a study of partially ordered sets, which provide a 
language in which many of the most important principles can be expressed. 
Chapter 52 looks at manifestations of these ideas in measure theory.   In 
Chapter 53 I continue the work of Volumes 3 and 4, examining questions 
which arise more or less naturally if we approach the topics of those 
volumes with the new techniques. 
 
The Banach-Ulam problem got a mention in Volume 2, a paragraph in Volume 3 
and a section in Volume 4;  
%232Hc 363S 438
at last, in Chapter 54 of the present volume, I try to give a proper 
account of the extraordinary ideas to which it has led.   I have 
regretfully abandoned the idea of describing even a representative sample 
of the forcing models which have been devised to show that 
measure-theoretic propositions are consistent, but in Chapter 55 I set out 
some of the basic properties of random real forcing.   Finally, in Chapter 
56, I look at what measure theory becomes in ZF alone, with countable 
or dependent choice, and with the axiom of determinacy. 
 
While I should like to believe that most of the material of this volume 
will be accessible to those who have learnt measure theory from other 
sources, it has obviously been written with earlier volumes constantly in 
mind, and I have to advise you to make sure that Volumes 3 and 4, at 
least, will be available in case of need.   Apart from these, I do of 
course assume that readers will be at ease with modern set theory.   It is 
not so much that I demand a vast amount of knowledge -- \S\S5A1-5A2 have a 
good many proofs to help cover any gaps -- as that I present arguments 
without much consideration for the inexperienced, and some of them may be 
indigestible at first if you have not cut your teeth on  
{\smc Just \& Weese 96} or {\smc Jech 78}.   What you do not need is any 
prior knowledge of forcing.   But of course for Chapter 55 you will have to 
take a proper introduction to forcing, e.g., {\smc Kunen 80}, in parallel 
with \S5A3, since nothing here will make sense without an 
acquaintance with forcing languages and the fundamental theorem of forcing. 
 
\bigskip

\noindent{\bf Note on second printing}

There has been the usual crop of errors (most, but not all, minor) to be
corrected, and I have added a few new results.   The most important is
P.Larson's proof that it is relatively consistent with ZFC to suppose that
there is no medial limit.   %538Sb
In the process of preparing new editions of Volumes 1-4, I have I hope
covered all the items listed in the old \S5A6 
(`Later editions only'), which I have therefore dropped, even though
there are one or two further entries under this heading.   As before,
these can be found on the Web edition at
{\tt http://www.essex.ac.uk/maths{\bsp}people/fremlin/mtcont.htm}.    
% 421Cf 367Rd
