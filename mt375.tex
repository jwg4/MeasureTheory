\frfilename{mt375.tex}
\versiondate{30.1.10}
\copyrightdate{1996}

\def\ssbhm{$\sigma$-\vthsp{}subhomomorphism}

\def\chaptername{Linear operators between function spaces}
\def\sectionname{Kwapien's theorem}

\newsection{375}

In \S368 and the first part of \S369 I examined maps from various types
of Riesz space into $L^0$
spaces.   There are equally striking results about maps out of $L^0$
spaces.   I start with some relatively elementary facts about positive
linear operators from $L^0$ spaces to Archimedean Riesz spaces in
general (375A-375D), and then turn to a remarkable analysis, due
essentially to S.Kwapien, of the positive linear operators from a
general $L^0$ space
to the $L^0$ space of a semi-finite measure algebra (375J), with a
couple of simple corollaries.

\leader{375A}{Theorem} Let $\frak A$ be a Dedekind $\sigma$-complete
Boolean algebra and $W$ an Archimedean Riesz space.   If
$T:L^0(\frak A)\to W$ is a positive linear operator, it is sequentially
order-continuous.

\proof{{\bf (a)} The first step is to observe that if $\sequencen{u_n}$
is any non-increasing sequence in $L^0=L^0(\frak A)$ with infimum $0$,
and $\epsilon>0$, then $\{n(u_n-\epsilon u_0):n\in\Bbb N\}$ is bounded
above in $L^0$.   \Prf\ For $k\in\Bbb N$ set
$a_k=\sup_{n\in\Bbb N}\Bvalue{n(u_n-\epsilon u_0)>k}$;  set
$a=\inf_{k\in\Bbb N}a_k$.   \Quer\ Suppose, if possible, that $a\ne 0$.
Because $u_n\le u_0$, $n(u_n-\epsilon u_0)\le nu_0$ for every $n$ and

\Centerline{$a\Bsubseteq a_0\Bsubseteq\Bvalue{u_0>0}
=\Bvalue{\epsilon u_0>0}
=\sup_{n\in\Bbb N}\Bvalue{\epsilon u_0-u_n>0}$.}

\noindent So there is some $m\in\Bbb N$ such that
$a'=a\Bcap\Bvalue{\epsilon u_0-u_m>0}\ne 0$.   Now, for any $n\ge m$,
any $k\in\Bbb N$,

\Centerline{$a'\Bcap\Bvalue{n(u_n-\epsilon u_0)>k}
\Bsubseteq\Bvalue{\epsilon u_0-u_m>0}\Bcap\Bvalue{u_m-\epsilon u_0>0}
=0$.}

\noindent But $a'\Bsubseteq\sup_{n\in\Bbb N}\Bvalue{n(u_n-\epsilon
u_0)>k}$, so in fact

\Centerline{$a'\Bsubseteq\sup_{n\le m}\Bvalue{n(u_n-\epsilon
u_0)>k}=\Bvalue{v>k}$,}

\noindent where $v=\sup_{n\le m}n(u_n-\epsilon u_0)$.   And
this means that $\inf_{k\in\Bbb N}\Bvalue{v>k}\Bsupseteq a'\ne 0$, which
is impossible.\   \Bang\   Accordingly $a=0$;  by 364L(a-i),
$\{n(u_n-\epsilon u_0):n\in\Bbb N\}$ is bounded above.\   \Qed

\medskip

{\bf (b)} Now suppose that $\sequencen{u_n}$ is a
non-increasing sequence in $L^0$ with infimum $0$, and that $w\in W$ is
a lower bound for $\{Tu_n:n\in\Bbb N\}$.
Take any $\epsilon>0$.   By (a), $\{n(u_n-\epsilon u_0):n\in\Bbb N\}$
has an upper bound $v$ in $L^0$.   Because $T$ is positive,

\Centerline{$w\le Tu_n=T(u_n-\epsilon u_0)+T(\epsilon u_0)\le
T(\Bover1nv)+T(\epsilon u_0)=\Bover1nTv+\epsilon Tu_0$}

\noindent for every $n\ge 1$.   Because $W$ is Archimedean,
$w\le\epsilon Tu_0$.   But this is true for every $\epsilon>0$, so
(again because $W$ is Archimedean) $w\le 0$.   As $w$ is arbitrary,
$\inf_{n\in\Bbb N}Tu_n=0$.   As $\sequencen{u_n}$ is arbitrary, $T$ is
sequentially order-continuous (351Gb).
}%end of proof of 375A

\leader{375B}{Proposition} Let $\frak A$ be an atomless Dedekind
$\sigma$-complete Boolean algebra.   Then $L^0(\frak A)^{\times}=\{0\}$.

\proof{ \Quer\ Suppose, if possible, that $h:L^0(\frak A)\to\Bbb R$ is a
non-zero order-continuous positive linear functional.   Then there is a
$u>0$ in $L^0$ such that $h(v)>0$ whenever $0<v\le u$ (356H).
Because $\frak A$ is atomless, there is a disjoint sequence
$\sequencen{a_n}$ such that $a_n\Bsubseteq\Bvalue{u>0}$ for each $n$, so
that $u_n=u\times\chi a_n>0$, while $u_m\wedge u_n=0$ if $m\ne n$.   Now
however

\Centerline{$v=\sup_{n\in\Bbb N}\Bover{n}{h(u_n)}u_n$}

\noindent is defined in $L^0$, by 368K, and $h(v)\ge n$ for every $n$,
which is impossible.\ \Bang
}%end of proof of 375B

\leader{375C}{Theorem} Let $\frak A$ be a Dedekind complete Boolean
algebra, $W$ an Archimedean Riesz space, and $T:L^0(\frak A)\to W$ an
order-continuous Riesz homomorphism.   Then $V=T[L^0(\frak A)]$ is an
order-closed Riesz subspace of $W$.

\proof{ The kernel $U$ of $T$ is a band in $L^0=L^0(\frak A)$
(352Oe), and must be a projection band (353I), because $L^0$ is Dedekind
complete (364M).   Since $U+U^{\perp}=L^0$, $T[U]+T[U^{\perp}]=V$, that
is, $T[U^{\perp}]=V$;  since $U\cap U^{\perp}=\{0\}$, $T$ is an
isomorphism between $U^{\perp}$ and $V$.   Now
suppose that $A\subseteq V$ is upwards-directed and has a least upper
bound $w\in W$.   Then $B=\{u:u\in U^{\perp},\,Tu\in A\}$ is
upwards-directed and $T[B]=A$.   The point is that $B$ is bounded above
in $L^0$.   \Prf\Quer\ If not, then $\{u^+:u\in B\}$ cannot be bounded
above, so there is a $u_0>0$ in $L^0$ such that
$nu_0=\sup_{u\in B}nu_0\wedge u^+$
for every $n\in\Bbb N$ (368A).   Since $B\subseteq U^{\perp}$,
$u_0\in U^{\perp}$ and $Tu_0>0$.   But now, because $T$ is an
order-continuous Riesz homomorphism,

\Centerline{$nTu_0
=\sup_{u\in B}T(nu_0\wedge u^+)
=\sup_{v\in A}nTu_0\wedge v^+\le w^+$}

\noindent for every $n\in\Bbb N$, which is impossible.\   \Bang\Qed\

Set $u^*=\sup B$;  then $Tu^*=\sup A=w$ and $w\in V$.   As $A$ is
arbitrary, $V$ is order-closed.
}%end of proof of 375C

\leader{375D}{Corollary} Let $W$ be a Riesz space and $V$
an order-dense Riesz subspace which is isomorphic to $L^0(\frak A)$ for
some Dedekind complete Boolean algebra $\frak A$.   Then $V=W$.

\proof{ By 353Q, $W$ is Archimedean.
So we can apply 375C to an isomorphism $T:L^0(\frak A)\to V$ to see that
$V$ is order-closed in $W$.
}%end of proof of 375D

\leader{375E}{Theorem}\dvAnew{2010}
Let $(\frak A,\bar\mu)$ be a semi-finite measure
algebra, $(\frak B,\bar\nu)$ any measure algebra, and
$T:L^0(\frak A)\to L^0(\frak B)$ an order-continuous positive linear
operator.   Then $T$ is continuous for the topologies of convergence in
measure.

\proof{ \Quer\ Otherwise, we can find $w\in L^0(\frak A)$,
$b\in\frak B^f$ and $\epsilon>0$
such that whenever $a\in\frak A^f$ and $\delta>0$ there is a
$u\in L^0(\frak A)$ such that
$\bar\mu(a\Bcap\Bvalue{|u-w|>\delta})\le\delta$
and $\bar\nu(b\Bcap\Bvalue{|Tu-Tw|>\epsilon})\ge\epsilon$ (367M, 2A3H).
%4A5Fa would help here
Of course it follows that whenever $a\in\frak A^f$ and $\delta>0$ there
is a $u\in L^0(\frak A)$ such that
$\bar\mu(a\Bcap\Bvalue{|u|>\delta})\le\delta$
and $\bar\nu(b\Bcap\Bvalue{|Tu|>\epsilon})\ge\epsilon$ (367M).   Choose
$\sequencen{a_n}$ and $\sequencen{u_n}$ inductively, as follows.   $a_0=0$.
Given that $a_n\in\frak A^f$, let $u_n\in L^0(\frak A)$ be such that
$\bar\mu(a_n\Bcap\Bvalue{|u_n|>2^{-n}})\le 2^{-n}$ and
$\bar\nu(b\Bcap\Bvalue{|Tu_n|>\epsilon})\ge\epsilon$.
Of course it follows that
$\bar\nu(b\Bcap\Bvalue{T|u_n|>\epsilon})\ge\epsilon$.  Because
$(\frak A,\bar\mu)$ is semi-finite,
$|u_n|=\sup_{a\in\frak A^f}|u_n|\times\chi a)$;  because $T$ is
order-continuous,
$T|u_n|=\sup_{a\in\frak A^f}T(|u_n|\times\chi a)$, and we can find
$a_{n+1}\in\frak A^f$ such that
$\bar\nu(b\Bcap\Bvalue{T(|u_n|\times\chi a_{n+1})>\epsilon})
\ge\bover12\epsilon$.   Enlarging $a_{n+1}$ if necessary, arrange that
$a_{n+1}\Bsupseteq a_n$.   Continue.

At the end of the induction, set $v_n=2^n|u_n|\times\chi a_{n+1}$;
then $\bar\mu(a_n\Bcap\Bvalue{v_n>1})\le 2^{-n}$, for each $n\in\Bbb N$.
It follows that $\{v_n:n\in\Bbb N\}$ is bounded above.   \Prf\ For
$k\in\Bbb N$,  set $c_k=\sup_{n\in\Bbb N}\Bvalue{v_n>k}$.   Then
$c_k\Bsubseteq\sup_{n\in\Bbb N}a_n$.   If $n\in\Bbb N$ and $\delta>0$, let
$m\ge n$ be such that $2^{-m+1}\le\delta$, and $k\ge 1$ such that
$\bar\mu(a_n\Bcap\Bvalue{\sup_{m<n}v_m>k})\le\delta$.   Then

\Centerline{$\bar\mu(a_n\Bcap c_k)
\le\bar\mu(a_n\Bcap\Bvalue{\sup_{m<n}v_m>k})
  +\sum_{i=m}^{\infty}\bar\mu(a_i\Bcap\Bvalue{v_i>1})
\le 2\delta$.}

\noindent As $\delta$ is arbitrary, $a_n\Bcap\inf_{k\in\Bbb N}c_k=0$;
as $n$ is arbitrary, $\inf_{k\in\Bbb N}c_k=0$;  by 364L(a-i) again,
$\{v_n:n\in\Bbb N\}$ is bounded above.\ \Qed

Set $v=\sup_{n\in\Bbb N}v_n$.   Then $2^{-n}v\ge|u_n|\times\chi a_{n+1}$,
so $2^{-n}Tv\ge T(|u_n|\times\chi a_{n+1})$ and
$\bar\nu(b\Bcap\Bvalue{2^{-n}Tv>\epsilon})\ge\bover12\epsilon$, for each
$n\in\Bbb N$.   But $\inf_{n\in\Bbb N}2^{-n}Tv=0$, so
$\inf_{n\in\Bbb N}\Bvalue{2^{-n}Tv>\epsilon}=0$ (364L(b-ii)) and
$\inf_{n\in\Bbb N}\bar\nu(b\Bcap\Bvalue{2^{-n}Tv>\epsilon})=0$.\ \Bang

So we have the result.
}

\leader{375F}{}\cmmnt{ I come now to the deepest result of this
section, concerning positive linear operators from $L^0(\frak A)$ to
$L^0(\frak B)$ where $\frak B$ is a measure algebra.   I approach
through a couple of lemmas which are striking enough in their own right.

The following temporary definition will be useful.

\medskip

\noindent}{\bf Definition} Let $\frak A$ and $\frak B$ be Boolean
algebras.   I will say that a function $\phi:\frak A\to\frak B$ is a
{\bf \ssbhm} if

\inset{$\phi(a\Bcup a')=\phi(a)\Bcup\phi(a')$ for all $a$,
$a'\in\frak A$,}

\inset{$\inf_{n\in\Bbb N}\phi(a_n)=0$ whenever $\sequencen{a_n}$ is a
non-increasing sequence in $\frak A$ with infimum $0$.}

\cmmnt{\noindent Now we have the following easy facts.}

\leader{375G}{Lemma} Let $\frak A$ and $\frak B$ be Boolean algebras and
$\phi:\frak A\to\frak B$ a \ssbhm.

(a) $\phi(0)=0$, $\phi(a)\Bsubseteq\phi(a')$ whenever $a\Bsubseteq a'$,
and $\phi(a)\Bsetminus\phi(a')\Bsubseteq\phi(a\Bsetminus a')$ for every
$a$, $a'\in\frak A$.

(b) If $\bar\mu$, $\bar\nu$ are measures such that $(\frak A,\bar\mu)$
and $(\frak B,\bar\nu)$ are totally finite measure algebras, then for
every $\epsilon>0$ there is a $\delta>0$ such that
$\bar\nu\phi(a)\le\epsilon$ whenever $\bar\mu a\le\delta$.

\proof{{\bf (a)} This is elementary.   Set every $a_n=0$ in the second
clause of the definition 375F to see that $\phi(0)=0$.   The other two
parts are immediate consequences of the first clause.

\medskip

{\bf (b)} (Compare 232Ba, 327Bb.)   \Quer\ Suppose, if possible,
otherwise.   Then for every $n\in\Bbb N$ there is an $a_n\in\frak A$
such that $\bar\mu a_n\le 2^{-n}$ and $\bar\nu\phi(a_n)\ge\epsilon$.
Set $c_n=\sup_{i\ge n}a_i$ for each $n$;  then $\sequencen{c_n}$ is
non-increasing and has infimum $0$ (since $\bar\mu c_n\le 2^{-n+1}$ for
each $n$), but $\bar\nu\phi(c_n)\ge\epsilon$ for every $n$, so
$\inf_{n\in\Bbb N}\phi c_n$ cannot be $0$.\ \Bang
}%end of proof of 375G

\leader{375H}{Lemma} Let $(\frak A,\bar\mu)$ and $(\frak B,\bar\nu)$ be
totally finite measure algebras and $\phi:\frak A\to\frak B$ a
\ssbhm.   Then for every non-zero $b_0\in\frak B$
there are a non-zero $b\Bsubseteq b_0$ and an $m\in\Bbb N$ such that
$b\Bcap\inf_{j\le m}\phi(a_j)=0$ whenever $a_0,\ldots,a_m\in\frak A$ are
disjoint.

\proof{{\bf (a)}  Suppose first that $\frak A$ is atomless and that
$\bar\mu 1=1$.

Set $\epsilon=\bover15\bar\nu b_0$ and let $m\ge 1$ be such that
$\bar\nu\phi(a)\le\epsilon$ whenever $\bar\mu a\le\bover1m$ (375Gb).
We need to know that
$(1-\bover1m)^m\le\bover12$;   this is because (if $m\ge 2$)
$\ln m-\ln(m-1)\ge\bover1m$, so $m\ln(1-\bover1m)\le-1\le-\ln 2$.

Set

\Centerline{$C=\{\inf_{j\le m}\phi(a_j):a_0,\ldots,a_m\in\frak A$ are
disjoint$\}$.}

\noindent\Quer\ Suppose, if possible, that $b_0\Bsubseteq\sup C$.   Then
there are $c_0,\ldots,c_k\in C$ such that
$\bar\nu(b_0\Bcap\sup_{i\le k}c_i)\discretionary{}{}{}\ge 4\epsilon$.
For each $i\le k$
choose disjoint $a_{i0},\ldots,a_{im}\in\frak A$ such that
$c_i=\inf_{j\le m}\phi(a_{ij})$.   Let $D$ be the set of atoms of the
finite subalgebra
of $\frak A$ generated by $\{a_{ij}:i\le k,\,j\le m\}$, so that $D$ is a
finite partition of unity in $\frak A$, and every $a_{ij}$ is the join
of the members of $D$ it includes.   Set $p=\#(D)$, and for each
$d\in D$ take a maximal disjoint set
$E_d\subseteq\{e:e\Bsubseteq d,\,\bar\mu e=\bover1{pm}\}$, so that
$\bar\mu(d\Bsetminus\sup E_d)<\bover1{pm}$;  set

\Centerline{$d^*=1\Bsetminus\sup(\bigcup_{d\in D}E_d)
=\sup_{d\in D}(d\Bsetminus\sup E_d)$,}

\noindent so that $\bar\mu d^*$ is a multiple of $\bover1{pm}$ and is
less than $\bover1m$.   Let $E^*$ be a disjoint set of elements of
measure $\bover1{pm}$ with union $d^*$, and take
$E=E^*\cup\bigcup_{d\in D}E_d$, so that $E$ is a partition of unity in
$\frak A$, $\bar\mu
e=\bover1{pm}$ for every $e\in E$, and $a_{ij}\Bsetminus d^*$ is the
join of the members of $E$ it includes for every $i\le k$ and $j\le m$.

Set

\Centerline{$\Cal K=\{K:K\subseteq E,\,\#(K)=p\}$,
\quad $M=\#(\Cal K)=\Bover{(mp)!}{p!(mp-p)!}$.}

\noindent For every $K\in\Cal K$, $\bar\mu(\sup K)=\bover1m$ so
$\bar\nu\phi(\sup K)\le\epsilon$.   So if we set

\Centerline{$v=\sum_{K\in\Cal K}\chi\phi(\sup K)$,}

\noindent $\int v\le\epsilon M$.   On the other hand,

\Centerline{$\bar\nu(b_0\Bcap\sup_{i\le k}c_i)\ge 4\epsilon$,
\quad$\bar\nu\phi(d^*)\le\epsilon$,}

\noindent so $\bar\nu b_1\ge 3\epsilon$, where

\Centerline{$b_1=b_0\Bcap\sup_{i\le k}c_i\Bsetminus\phi(d^*)$.}

\noindent Accordingly $\int v\le\bover13M\bar\nu b_1$ and

\Centerline{$b_2=b_1\Bcap\Bvalue{v<\bover12M}$}

\noindent is non-zero.

Because $b_2\Bsubseteq b_1$, there is an $i\le k$ such that $b_2\Bcap
c_i\ne 0$.   Now

\Centerline{$b_2\Bcap c_i
\Bsubseteq c_i\Bsetminus\phi(d^*)
=\inf_{j\le m}\phi(a_{ij})\Bsetminus\phi(d^*)
\Bsubseteq\inf_{j\le m}\phi(a_{ij}\Bsetminus d^*)$.}

\noindent But every $a_{ij}\Bsetminus d^*$ is the join of the members of
$E$ it includes, so

$$\eqalign{b_2\Bcap c_i
&\Bsubseteq\inf_{j\le m}\phi(a_{ij}\Bsetminus d^*)
\Bsubseteq\inf_{j\le m}\phi(\sup\{e:e\in E,\,e\Bsubseteq a_{ij}\})\cr
&=\inf_{j\le m}\sup\{\phi(e):e\in E,\,e\Bsubseteq a_{ij}\}\cr
&=\sup\{\inf_{j\le m}\phi(e_j):e_0,\ldots,e_m\in E
   \text{ and }e_j\Bsubseteq a_{ij}\text{ for every }j\}.\cr}$$

\noindent So there are $e_0,\ldots,e_m\in E$ such that $e_j\Bsubseteq
a_{ij}$ for each $j$ and $b_3=b_2\Bcap\inf_{j\le m}\phi(e_j)\ne 0$.
Because $a_{i0},\ldots,a_{im}$ are disjoint, $e_0,\ldots,e_m$ are
distinct;  set $J=\{e_0,\ldots,e_m\}$.   Then whenever $K\in\Cal K$ and
$K\cap J\ne\emptyset$, $b_3\Bsubseteq\phi(\sup K)$.

So let us calculate the size of $\Cal K_1=\{K:K\in\Cal K,\,K\cap
J\ne\emptyset\}$.   This is

$$\eqalign{M-\Bover{(mp-m-1)!}{p!(mp-p-m-1)!}
&=M\bigl(1-\Bover{(mp-p)(mp-p-1)\ldots(mp-p-m)}
  {mp(mp-1)\ldots(mp-m)}\bigr)\cr
&\ge M\bigl(1-(\Bover{mp-p}{mp})^{m+1}\bigr)
\ge\Bover12M.\cr}$$

\noindent But this means that $b_3\Bsubseteq\Bvalue{v\ge\bover12M}$,
while also $b_3\Bsubseteq\Bvalue{v<\bover12M}$;  which is surely
impossible.\  \Bang

Accordingly $b_0\notBsubseteq\sup C$, and we can take
$b=b_0\Bsetminus\sup C$.

\medskip

{\bf (b)} Now for the general case.   Let $A$ be the set of atoms of
$\frak A$, and set $d=1\Bsetminus\sup A$.   Then the principal
ideal $\frak A_d$ is
atomless, so there are a non-zero $b_1\Bsubseteq b_0$ and an
$n\in\Bbb N$ such that $b_1\Bcap\inf_{j\le n}\phi(a_j)=0$ whenever
$a_0,\ldots,a_n\in\frak A_d$ are disjoint.   \Prf\ If $\bar\mu d>0$ this
follows from (a), if we apply it to $\phi\restrp\frak A_d$ and
$(\bar\mu d)^{-1}\bar\mu\restrp\frak A_d$.   If $\bar\mu d=0$ then we can
just take $b_1=b_0$ and $n=0$.\  \Qed

Let $\delta>0$ be such that $\bar\nu\phi(a)<\bar\nu b_1$ whenever
$\bar\mu a\le\delta$.   Let $A_1\subseteq A$ be a finite set such that
$\bar\mu(\sup A_1)\ge\bar\mu(\sup A)-\delta$, and set $r=\#(A)$,
$d^*=\sup(A\Bsetminus A_1)$.   Then $\bar\mu d^*\le\delta$ so
$b=b_1\Bsetminus\phi(d^*)\ne 0$.   Try $m=n+r$.   If $a_0,\ldots,a_m$
are disjoint, then at most $r$ of them can meet $\sup A_1$, so
(re-ordering if necessary) we can suppose that $a_0,\ldots,a_n$ are
disjoint from $\sup A_1$, in which case $a_j\Bsetminus d^*\Bsubseteq d$
for each $j\le m$.   But in this case (because $b\Bcap\phi(d^*)=0$)

\Centerline{$b\Bcap\inf_{j\le m}\phi(a_j)
\Bsubseteq b\Bcap\inf_{j\le n}\phi(a_j)
=b\Bcap\inf_{j\le n}\phi(a_j\Bcap d)
=0$}

\noindent by the choice of $n$ and $b_1$.

Thus in the general case also we can find appropriate $b$ and $m$.
}%end of proof of 375H

\leader{375I}{Lemma} Let $(\frak A,\bar\mu)$ and $(\frak B,\bar\nu)$ be
totally finite measure algebras and $\phi:\frak A\to\frak B$ a
\ssbhm.   Then for every non-zero $b_0\in\frak B$
there are a non-zero $b\Bsubseteq b_0$ and a finite partition of unity
$C\subseteq\frak A$ such that $a\mapsto b\Bcap\phi(a\Bcap c)$ is a ring
homomorphism for every $c\in C$.

\proof{ By 375H, we can find $b_1$, $m$ such that
$0\ne b_1\Bsubseteq b_0$ and $b_1\Bcap\inf_{j\le m}\phi(a_j)=0$ whenever
$a_0,\ldots,a_m\in\frak A$ are disjoint.   Do this with the smallest
possible $m$.   If $m=0$ then $b_1\Bcap\phi(1)=0$, so we can take
$b=b_1$, $C=\{1\}$.   Otherwise, because $m$ is minimal, there must be
disjoint $c_1,\ldots,c_m\in\frak A$ such that
$b=b_1\Bcap\inf_{1\le j\le m}\phi(c_j)\ne 0$.   Set
$c_0=1\Bsetminus\sup_{1\le j\le m}c_j$,
$C=\{c_0,c_1,\ldots,c_m\}$;  then $C$ is a partition of unity in
$\frak A$.   Set $\pi_j(a)=b\Bcap\phi(a\Bcap c_j)$ for each
$a\in\frak A$ and $j\le m$.   Then we always have
$\pi_j(a\Bcup a')=\pi_j(a)\Bcup\pi_j(a')$ for all $a$, $a'\in\frak A$,
because $\phi$ is a subhomomorphism.

To see that every $\pi_j$ is a ring homomorphism, we need only check
that $\pi_j(a\Bcap a')=0$ whenever $a\Bcap a'=0$.   (Compare
312H(iv).)   In the case $j=0$, we actually have $\pi_0(a)=0$ for
every $a$, because
$b\Bcap\phi(c_0)=b_1\Bcap\inf_{0\le j\le m}\phi(c_j)=0$ by the choice of
$b_1$ and $m$.   When $1\le j\le m$, if $a\Bcap a'=0$, then

\Centerline{$\pi_j(a)\Bcap\pi_j(a')
=b_1\Bcap\inf_{1\le i\le m,i\ne j}\phi(c_j)\Bcap\phi(a)\Bcap\phi(a')$}

\noindent is again $0$, because $a$, $a'$,
$c_1,\ldots,c_{j-1},c_{j+1},\ldots,c_m$ are disjoint.   So we have a
suitable pair $b$, $C$.
}%end of proof of 375I

\leader{375J}{Theorem} Let $\frak A$ be any Dedekind $\sigma$-complete
Boolean algebra and $(\frak B,\bar\nu)$ a semi-finite measure algebra.
Let $T:L^0(\frak A)\to L^0(\frak B)$ be a positive linear operator.
Then we can find $B$, $\langle A_b\rangle_{b\in B}$ such that $B$ is a
partition of unity in $\frak B$, each $A_b$ is a finite partition of
unity in $\frak A$, and $u\mapsto T(u\times\chi a)\times\chi b$ is a
Riesz homomorphism whenever $b\in B$ and $a\in A_b$.

\proof{{\bf (a)} Write $B^*$ for the set of potential members of $B$;
that is, the set of those $b\in \frak B$ such that there is a finite
partition of unity $A\subseteq\frak A$ such that $T_{ab}$ is a Riesz
homomorphism for every $a\in A$, writing $T_{ab}(u)=T(u\times\chi
a)\times\chi b$.   If I can show that $B^*$ is order-dense in $\frak B$,
this will suffice, since there will then be a partition of unity
$B\subseteq B^*$.

\medskip

{\bf (b)} So let $b_0$ be any non-zero member of $\frak B$;  I seek a
non-zero member of $B^*$ included in $b_0$.   Of course there is a
non-zero $b_1\Bsubseteq b_0$ with $\bar\nu b_1<\infty$.   Let $\gamma>0$
be such that $b_2=b_1\Bcap\Bvalue{T(\chi 1)\le\gamma}$ is non-zero.
Define $\mu:\frak A\to\coint{0,\infty}$ by setting
$\mu a=\int_{b_2}T(\chi a)$
for every $a\in\frak A$.   Then $\mu$ is countably additive, because
$\chi$, $T$ and $\int$ are all additive and sequentially
order-continuous (using 375A).   Set $\Cal N=\{a:\mu a=0\}$;  then
$\Cal N$ is a $\sigma$-ideal of $\frak A$, and $(\frak C,\bar\mu)$ is a
totally finite measure algebra, where $\frak C=\frak A/\Cal N$ and
$\bar\mu a^{\ssbullet}=\mu a$ for every $a\in\frak A$ (just as in 321H).

\medskip

{\bf (c)} We have a function $\phi$ from $\frak C$ to the principal
ideal $\frak B_{b_2}$ defined by saying that
$\phi a^{\ssbullet}=b_2\Bcap\Bvalue{T(\chi a)>0}$ for every
$a\in\frak A$.   \Prf\ If $a_1$, $a_2\in\frak A$ are such that
$a_1^{\ssbullet}=a_2^{\ssbullet}$ in $\frak C$, this means that
$a_1\Bsymmdiff a_2\in\Cal N$;  now

$$\eqalign{\Bvalue{T(\chi a_1)>0}\Bsymmdiff\Bvalue{T(\chi a_2)>0}
&\Bsubseteq\Bvalue{|T(\chi a_1)-T(\chi a_2)|>0}\cr
&\Bsubseteq\Bvalue{T(|\chi a_1-\chi a_2|)>0}
=\Bvalue{T\chi(a_1\Bsymmdiff a_2)>0}\cr}$$

\noindent is disjoint from $b_2$ because $\int_{b_2}T\chi(a_1\Bsymmdiff
a_2)=0$.   Accordingly $b_2\Bcap\Bvalue{T(\chi a_1)>0}
=b_2\Bcap\Bvalue{T(\chi a_2)>0}$ and we can take this common value for
$\phi(a_1^{\ssbullet})=\phi(a_2^{\ssbullet})$.\  \Qed

\medskip

{\bf (d)} Now $\phi$ is a \ssbhm.   \Prf\ (i) For any
$a_1$, $a_2\in\frak A$ we have

\Centerline{$\Bvalue{T\chi(a_1\Bcup a_2)>0}
=\Bvalue{T(\chi a_1)>0}\Bcup\Bvalue{T(\chi a_2)>0}$}

\noindent because

\Centerline{$T(\chi a_1)\vee T(\chi a_2)\le T\chi(a_1\Bcup a_2)
\le T(\chi a_1)+T(\chi a_2)$.}

\noindent So $\phi(c_1\Bcup c_2)=\phi(c_1)\Bcup\phi(c_2)$ for all $c_1$,
$c_2\in\frak C$.   (ii) If $\sequencen{c_n}$ is a non-increasing
sequence in $\frak C$ with infimum $0$, choose $a_n\in\frak A$ such that
$a_n^{\ssbullet}=c_n$ for each $n$, and set
$\tilde a_n=\inf_{i\le n}a_i\Bsetminus\inf_{i\in\Bbb N}a_i$ for each
$n$;  then $\tilde a_n^{\ssbullet}=c_n$ so
$\phi(c_n)=\Bvalue{T(\chi\tilde a_n)>0}$ for
each $n$, while $\sequencen{\tilde a_n}$ is non-increasing and
$\inf_{n\in\Bbb N}\tilde a_n=0$.   \Quer\ Suppose, if possible, that
$b'=\inf_{n\in\Bbb N}\phi(c_n)\ne 0$;  set
$\epsilon=\bover12\bar\nu b'$.   Then
$\bar\nu(b_2\Bcap\Bvalue{T(\chi\tilde a_n)>0})\ge 2\epsilon$ for every
$n\in\Bbb N$.   For each $n$, take $\alpha_n>0$ such that
$\bar\nu(b_2\Bcap\Bvalue{T(\chi\tilde a_n)>\alpha_n})\ge\epsilon$.
Then $u=\sup_{n\in\Bbb N}n\alpha_n^{-1}\chi\tilde a_n$ is defined in
$L^0(\frak A)$ (because
$\sup_{n\in\Bbb N}\Bvalue{n\alpha_n^{-1}\chi\tilde a_n>k}
\subseteq\tilde a_m$ if $k\ge\max_{i\le m}i\alpha_i^{-1}$, so
$\inf_{k\in\Bbb N}\sup_{n\in\Bbb N}
\Bvalue{n\alpha_n^{-1}\chi\tilde a_n>k}=0$).   But now

\Centerline{$\bar\nu(b_2\Bcap\Bvalue{Tu>n})
\ge\bar\nu(b_2\Bcap\Bvalue{T(\chi\tilde a_n)>\alpha_n})\ge\epsilon$}

\noindent for every $n$, so $\inf_{n\in\Bbb N}\Bvalue{Tu>n}\ne 0$, which
is impossible.\ \BanG\   Thus $\inf_{n\in\Bbb N}\phi(c_n)=0$;  as
$\sequencen{c_n}$ is arbitrary, $\phi$ is a \ssbhm.\
\Qed

\medskip

{\bf (e)} By 375I, there are a non-zero $b\in\frak B_{b_2}$ and a finite
partition of unity $C\subseteq\frak C$ such that $d\mapsto
b\Bcap\phi(d\Bcap c)$ is a ring homomorphism for every $c\in C$.   There
is a partition of unity $A\subseteq\frak A$, of the same size as $C$,
such that $C=\{a^{\ssbullet}:a\in A\}$.   Now $T_{ab}$ is a Riesz
homomorphism for every $a\in A$.   \Prf\ It is surely a positive linear
operator.   If $u_1$, $u_2\in L^0(\frak A)$ and $u_1\wedge u_2=0$, set
$e_i=\Bvalue{u_i>0}$ for each $i$, so that $e_1\Bcap e_2=0$.   Observe
that $u_i=\sup_{n\in\Bbb N}u_i\wedge n\chi e_i$, so that

\Centerline{$\Bvalue{T_{ab}u_i>0}=\sup_{n\in\Bbb
N}\Bvalue{T_{ab}(u_i\wedge n\chi e_i)>0}\Bsubseteq\Bvalue{T_{ab}(\chi
e_i)>0}=b\Bcap\Bvalue{T\chi(e_i\Bcap a)>0}$}

\noindent for both $i$ (of course $T_{ab}$, like $T$, is sequentially
order-continuous).   But this means that

$$\eqalign{\Bvalue{T_{ab}u_1>0}\Bcap\Bvalue{T_{ab}u_2>0}
&\Bsubseteq b\Bcap\Bvalue{T\chi(e_1\Bcap
a)>0}\Bcap\Bvalue{T\chi(e_2\Bcap a)>0}\cr
&=b\Bcap\phi(e_1^{\ssbullet}\Bcap
   a^{\ssbullet})\Bcap\phi(e_2^{\ssbullet}\Bcap a^{\ssbullet})
=0\cr}$$

\noindent because $a^{\ssbullet}\in C$, so
$d\mapsto b\Bcap\phi(d\Bcap a^{\ssbullet})$ is a ring homomorphism, while
$e_1^{\ssbullet}\Bcap e_2^{\ssbullet}=0$.   So
$T_{ab}u_1\wedge T_{ab}u_2=0$.   As $u_1$ and $u_2$ are arbitrary,
$T_{ab}$ is a Riesz homomorphism (352G(iv)).\ \Qed

\medskip

{\bf (f)} Thus $b\in B^*$.   As $b_0$ is arbitrary, $B^*$ is
order-dense, and we're home.
}%end of proof of 375J

\leader{375K}{Corollary} Let $\frak A$ be a Dedekind $\sigma$-complete
Boolean algebra and $U$ a Dedekind complete Riesz space such that
$U^{\times}$ separates the points of $U$.   If $T:L^0(\frak A)\to U$ is
a positive linear operator, there is a sequence $\sequencen{T_n}$ of
Riesz homomorphisms from $L^0(\frak A)$ to $U$ such that
$T=\sum_{n=0}^{\infty}T_n$, in the sense that $Tu=\sup_{n\in\Bbb
N}\sum_{i=0}^nT_iu$ for every $u\ge 0$ in $L^0(\frak A)$.

\proof{ By 369A, $U$ can be embedded as an order-dense Riesz subspace of
$L^0(\frak B)$ for some localizable measure algebra $(\frak B,\bar\nu)$;
being Dedekind complete, it is solid in $L^0(\frak B)$ (353K).   Regard
$T$ as an operator from $L^0(\frak A)$ to $L^0(\frak B)$, and take $B$,
$\langle A_b\rangle_{b\in B}$ as in 375J.   Note that $L^0(\frak B)$ can
be identified with $\prod_{b\in B}L^0(\frak B_b)$ (364R, 322L).   For
each $b\in B$ let $f_b:A_b\to\Bbb N$ be an injection.   If $b\in B$ and
$n\in f_b[A_b]$, set $T_{nb}(u)=\chi b\times T(u\times\chi f_b^{-1}(n))$;
otherwise set $T_{nb}=0$.   Then $T_{nb}:L^0(\frak A)\to L^0(\frak B_b)$
is a Riesz homomorphism;   because $A_b$ is a finite partition of unity,
$\sum_{n=0}^{\infty}T_{nb}u=\chi b\times Tu$
for every $u\in L^0(\frak A)$.   But this means that if we set
$T_nu=\langle T_{nb}u\rangle_{b\in B}$,

\Centerline{$T_n:L^0(\frak A)\to\prod_{b\in
B}L^0(\frak B_b)\cong L^0(\frak B)$}

\noindent is a Riesz homomorphism for each
$n$;  and $T=\sum_{n=0}^{\infty}T_n$.   Of course every $T_n$ is an
operator from $L^0(\frak A)$ to $U$ because $|T_nu|\le T|u|\in U$ for
every $u\in L^0(\frak A)$.
}%end of proof of 375K

\leader{375L}{Corollary} (a) If $\frak A$ is a Dedekind
$\sigma$-complete Boolean algebra, $(\frak B,\bar\nu)$ is a semi-finite
measure algebra, and there is any non-zero positive linear operator from
$L^0(\frak A)$ to $L^0(\frak B)$, then there is a non-trivial
sequentially order-continuous ring homomorphism from $\frak A$ to
$\frak B$.

(b) If $(\frak A,\bar\mu)$ and $(\frak B,\bar\nu)$ are homogeneous
probability algebras and $\tau(\frak A)>\tau(\frak B)$, then
$\eurm L^{\sim}(L^0(\frak A);L^0(\frak B))=\{0\}$.

\proof{{\bf (a)} It is probably quickest to look at the proof of 375J:
starting from a non-zero positive linear operator $T:L^0(\frak A)\to
L^0(\frak B)$, we move to a non-zero \ssbhm\
$\phi:\frak A/\Cal N\to\frak B$ and thence to a non-zero ring
homomorphism from $\frak A/\Cal N$ to $\frak B$, corresponding to a
non-zero ring homomorphism from $\frak A$ to $\frak B$, which is
sequentially order-continuous because it is dominated by $\phi$.
Alternatively, quoting 375J, we have a non-zero Riesz
homomorphism $T_1:L^0(\frak A)\to L^0(\frak B)$, and it is easy to check
that $a\mapsto\Bvalue{T(\chi a)>0}$ is a non-zero sequentially
order-continuous ring homomorphism.

\medskip

{\bf (b)} Use (a) and 331J.
}%end of proof of 375L

\exercises{
\leader{375X}{Basic exercises (a)}
%\spheader 375Xa
Let $\frak A$ be a Dedekind complete Boolean algebra and $W$ an
Archimedean Riesz space.   Let $T:L^0(\frak A)\to W$ be a positive
linear operator.   Show that $T$ is order-continuous iff
$T\chi:\frak A\to W$ is order-continuous.
%375A

\spheader 375Xb Let $\frak A$ be an atomless Dedekind $\sigma$-complete
Boolean algebra and $W$ a Banach lattice.   Show that the only
order-continuous positive linear operator from $L^0(\frak A)$ to $W$ is
the zero operator.
%375B

\spheader 375Xc Let $\frak A$ be a Dedekind complete Boolean algebra and
$W$ a Riesz space.   Let $T:L^0(\frak A)\to W$ be an
order-continuous Riesz homomorphism such that $T[L^0(\frak A)]$ is
order-dense in $W$.   Show that $T$ is surjective.
%375D

\sqheader 375Xd Let $\frak A$ and $\frak B$ be Boolean algebras and
$\phi:\frak A\to\frak B$ a \ssbhm\ as defined in 375F.
Show that $\phi$ is sequentially order-continuous.
%375F

\sqheader 375Xe Let $\frak A$ be the measure algebra of Lebesgue measure
on $[0,1]$ and $\frak G$ the regular open algebra of $\Bbb R$.  (i) Show
that there is no non-zero positive linear operator from $L^0(\frak G)$
to $L^0(\frak A)$.   \Hint{suppose $T:L^0(\frak G)\to L^0(\frak A)$ were
such an operator.   Reduce to the case $T(\chi 1)\le\chi 1$.   Let
$\sequencen{b_n}$ enumerate
an order-dense subset of $\frak G$ (316Yo).   For each $n\in\Bbb N$ take
non-zero $b'_n\Bsubseteq b_n$ such that
$\int T(\chi b'_n)\le 2^{-n-2}\int T(\chi 1)$ and consider
$T\chi(\sup_{n\in\Bbb N}b'_n)$.  See also 375Yf-375Ye.}
(ii) Show that there is no non-zero
positive linear operator from $L^0(\frak A)$ to $L^0(\frak G)$.
\Hint{suppose $T:L^0(\frak A)\to L^0(\frak G)$ were such an operator.
For each $n\in \Bbb N$ choose $a_n\in\frak A$,
$\alpha_n>0$ such that $\bar\mu a_n\le 2^{-n}$  and if
$b_n\Bsubseteq\Bvalue{T(\chi 1)>0}$ then
$b_n\Bcap\Bvalue{T(\chi a_n)>\alpha_n}\ne 0$.   Consider $Tu$ where
$u=\sum_{n=0}^{\infty}n\alpha_n^{-1}\chi a_n$.}
%375J

\spheader 375Xf In 375K, show that for any $u\in L^0(\frak A)$

\Centerline{$\inf_{n\in\Bbb N}\sup_{m\ge n}
\Bvalue{|Tu-\sum_{i=0}^mT_iu|>0}=0$.}
%375K

\sqheader 375Xg Prove directly, without quoting
375F-375L, that if $\frak A$ is a Dedekind $\sigma$-complete Boolean
algebra then every positive linear functional from $L^0(\frak A)$ to
$\Bbb R$ is a finite sum of Riesz homomorphisms.
%375L

\spheader 375Xh Let $\frak A$ and $\frak B$ be Dedekind $\sigma$-complete
Boolean algebras, and $T:L^0(\frak A)\to L^0(\frak B)$ a Riesz
homomorphism.   Show that there are a
sequentially order-continuous ring homomorphism $\pi:\frak A\to\frak B$ and
a $w\in L^0(\frak B)^+$ such that $Tu=w\times T_{\pi}u$ for every
$u\in L^0(\frak A)$, where $T_{\pi}:L^0(\frak A)\to L^0(\frak B)$ is
defined as in 364Yg.
%375L

\leader{375Y}{Further exercises (a)}\dvAnew{2010}
%\spheader 375Ya
Let $\frak A$ and $\frak B$ be Dedekind $\sigma$-complete
Boolean algebras, and $T:L^0(\frak A)\to L^0(\frak B)$
a linear operator.   (i) Show that if $T$ is order-bounded, then
$\sequencen{Tu_n}$ order*-converges to $0$ in
$L^0(\frak B)$ (definition: 367A)
whenever $\sequencen{u_n}$ order*-converges to $0$ in $L^0(\frak A)$.
(ii) Show that if $\frak B$ is ccc and \wsid\ and
$\sequencen{Tu_n}$ order*-converges to $0$ in $L^0(\frak B)$
whenever $\sequencen{u_n}$ order*-converges to $0$ in $L^0(\frak A)$, then
$T$ is order-bounded.
%375A mt37bits

\spheader 375Yb
Show that the following are equiveridical:  (i) there is a purely atomic
probability space $(X,\Sigma,\mu)$
such that $\Sigma=\Cal PX$ and $\mu\{x\}=0$ for every $x\in X$;  (ii)
there are a set $X$ and a Riesz homomorphism $f:\Bbb R^X\to\Bbb R$ which
is not order-continuous;   (iii) there are a Dedekind complete Boolean
algebra $\frak A$ and a positive linear operator
$f:L^0(\frak A)\to\Bbb R$ which is not order-continuous;  (iv) there are
a Dedekind complete Boolean algebra $\frak A$ and a sequentially
order-continuous Boolean homomorphism $\pi:\frak A\to\{0,1\}$ which is
not order-continuous;  (v) there are a Dedekind complete Riesz space $U$
and a sequentially order-continuous Riesz homomorphism $f:U\to\Bbb R$
which is not order-continuous;
*(vi) there are an atomless Dedekind complete Boolean algebra $\frak A$
and a sequentially
order-continuous Boolean homomorphism $\pi:\frak A\to\{0,1\}$ which is
not order-continuous.   (Compare 363S.)
%375B mt37bits

\spheader 375Yc Give an example of an atomless Dedekind
$\sigma$-complete Boolean algebra $\frak A$ such that
$L^0(\frak A)^{\sim}\ne\{0\}$.
%375B

\spheader 375Yd\discrversionA{\footnote{Reversed 2010.}}{}
Let $\frak A$ be the measure algebra of Lebesgue measure on $[0,1]$, and
set $L^0=L^0(\frak A)$.   Show that there is a positive linear operator
$T:L^0\to L^0$ such that $T[L^0]$ is not order-closed in $L^0$.
%375C

\spheader 375Ye\dvAnew{2010}
Show that the following are equiveridical:
(i) there is a probability space $(X,\Sigma,\mu)$
such that $\Sigma=\Cal PX$ and $\mu\{x\}=0$ for every $x\in X$;
(ii) there are localizable measure algebras $(\frak A,\bar\mu)$ and
$(\frak B,\bar\nu)$ and a positive linear operator
$T:L^0(\frak A)\to L^0(\frak B)$
which is not order-continuous.
%375Yb 375E

\spheader 375Yf
Let $\frak A$, $\frak B$ be Dedekind $\sigma$-complete Boolean algebras
of which $\frak B$ is weakly $\sigma$-distributive.   Let
$T:L^0(\frak A)\to L^0(\frak B)$ be a positive linear operator.   Show
that $a\mapsto\Bvalue{T(\chi a)>0}:\frak A\to\frak B$ is a
\ssbhm.
%375J

\spheader 375Yg\dvAnew{2010}
Let $\frak A$, $\frak B$ be Dedekind $\sigma$-complete Boolean algebras
of which $\frak B$ is weakly $\sigma$-distributive.   Let
$\phi:\frak A\to\frak B$ be a \ssbhm\ such that $\pi a\ne 0$ whenever
$a\in\frak A\setminus\{0\}$.   Show that $\frak A$ is weakly
$\sigma$-distributive.
%375Yf 375J

\spheader 375Yh Let $\frak A$ and $\frak B$ be Dedekind complete Boolean
algebras, and $\phi:\frak A\to\frak B$ a \ssbhm\ such that
$\phi 1_{\frak A}=1_{\frak B}$.   Show that there
is a sequentially order-continuous Boolean homomorphism
$\pi:\frak A\to\frak B$ such that $\pi a\Bsubseteq\phi a$ for every
$a\in\frak A$.
%375L

\spheader 375Yi Let $\frak G$ be the regular open algebra of $\Bbb R$,
and $L^0=L^0(\frak G)$.   Give an example of a non-zero positive linear
operator $T:L^0\to L^0$ such that there is no non-zero Riesz homomorphism
$S:L^0\to L^0$ with $S\le T$.
%375L mt37bits
}%end of exercises

\leader{375Z}{Problem} Let $\frak G$ be the regular open algebra of
$\Bbb R$, and $L^0=L^0(\frak G)$.   If $T:L^0\to L^0$ is a positive
linear operator, must $T[L^0]$ be order-closed?

\endnotes{
\Notesheader{375} Both this section, and the earlier work on linear
operators into $L^0$ spaces, can be regarded as describing different
aspects of a single fact:  $L^0$ spaces are very large.   The most
explicit statements of this principle are 368E and 375D:  every
Archimedean Riesz space can be embedded into a Dedekind complete $L^0$
space, but no such $L^0$ space can be properly embedded as an
order-dense Riesz subspace of any other Archimedean Riesz space.
Consequently there are many maps into $L^0$ spaces (368B).   But by the
same token there are few
maps out of them (375B, 375Lb), and those which do exist have a variety
of special properties (375A, 375J).

The original version of Kwapien's theorem ({\smc Kwapien 73}) was the
special case of 375J in which $\frak A$ is the Lebesgue measure algebra.
The ideas of the proof here are mostly taken
from {\smc Kalton Peck \& Roberts 84}.   I have based my account on
the concept of `subhomomorphism' (375F);  this seems to be an
effective tool when $\frak B$ is \wsid\ (375Yf),
but less useful in other cases.   The case $\frak B=\{0,1\}$,
$L^0(\frak B)\cong\Bbb R$ is not entirely trivial and is worth working
through on its own (375Xg).
}%end of notes

\discrpage


%\showhyphens{\ssbhm}
