
\wheader{}{10}{4}{4}{100pt}

Chapter 54:  Real-valued-measurable cardinals
%page-ends checked 8.1.15

\chapintrosection{15.4.05}{}{330}

\section{541}{Saturated ideals}{10.12.12}{}{330}
{$\kappa$-saturated $\kappa^+$-additive ideals;  $\kappa$-saturated
$\kappa$-additive ideals;  $\Tr_{\Cal{I}}(X;Y)$;  normal ideals;
$\kappa$-saturated normal ideals;  two-valued-measurable and weakly
compact cardinals;  the Tarski-Solovay dichotomy;
$\covSh(2^{\gamma},\kappa,\delta^+,\delta)$.}

\section{542}{Quasi-measurable cardinals}{8.7.13}{}{341}
{Definition and basic properties;  $\omega_1$-saturated $\sigma$-ideals;
and pcf theory;  and cardinal arithmetic;
cardinals of quotient algebras;  cofinality of $[\kappa]^{<\theta}$;
cofinality of product partial orders.}

\section{543}{The Gitik-Shelah theorem}{11.11.13}{}{346}
{Real-valued-measurable and atomlessly-measurable cardinals;  Ulam's
dichotomy;  a Fubini inequality;  Maharam types of witnessing
probabilities;  compact measures, inverse-measure-preserving functions and
extensions of measures.}

\section{544}{Measure theory with an atomlessly-measurable cardinal}
{31.12.13}{}{353}
{Covering numbers of null ideals;  repeated integrals;
measure-precalibers;  functions
from $[\kappa]^{<\omega}$ to null ideals;  Sierpi\'nski sets;
uniformities of null ideals;  weakly $\Pi^1_1$-indescribable cardinals;
Cicho\'n's diagram.}

\section{545}{PMEA and NMA}{10.2.14}{}{364}
{The product measure extension axiom;  the normal measure axiom;  Boolean
algebras with many measurable subalgebras.}

\section{546}{Power set $\sigma$-quotient algebras}{10.5.14}{}{366}
{Power set $\sigma$-quotient algebras;  harmless algebras and
skew products of ideals;  the Gitik-Shelah theorem for
category algebras;  the category algebra $\frak{G}_{\omega}$
of $\{0,1\}^{\omega}$;  completed free products of probability algebras 
with $\frak{G}_{\omega}$.}

\section{547}{Disjoint refinements of sequences of sets}{26.8.14}{}{383}
{Refining a sequence of sets to a disjoint sequence without changing outer
measures;  other results on simultaneous partitions.}

\wheader{}{10}{4}{4}{100pt}

Chapter 55:  Possible worlds

\chapintrosection{7.12.07}{}{391}

\section{551}{Forcing with quotient algebras}{2.12.13}{}{391}
{Measurable spaces with negligibles;  associated forcing notions;  
representing names for members of $\{0,1\}^I$;  representing names for
Baire sets in $\{0,1\}^I$;  the usual measure on $\{0,1\}^I$;  
re-interpreting Baire sets in the forcing model;  representing Baire
measurable functions;  representing measure algebras;  iterated forcing;
extending filters.}

\section{552}{Random reals I}{29.1.14}{}{409}
{Random real forcing notions;  calculating $2^{\kappa}$;  $\frak{b}$ and
$\frak{d}$;  preservation of outer measure;  Sierpi\'nski sets;  
cardinal functions of the usual measure on $\{0,1\}^{\lambda}$;  
Carlson's theorem on extending measures;  iterated random real forcing.}

\section{553}{Random reals II}{3.5.14}{}{430}
{Rothberger's property;  non-scattered compact sets;  Haydon's property;
rapid $p$-point ultrafilters;  products of ccc partially ordered sets;  
Aronszajn and Souslin
trees;  medial limits;  universally measurable sets.}

\section{554}{Cohen reals}{2.9.14}{}{450}
{Calculating $2^{\kappa}$;  Lusin sets;  precaliber pairs of measure
algebras;  Freese-Nation numbers;  Borel liftings for Lebesgue measure.}

\section{555}{Solovay's construction of real-valued-measurable cardinals}
{12.4.08}{}{456}
{Measurable cardinals are quasi-measurable after ccc forcing,
real-valued-measurable after random real forcing;
Maha\discretionary{-}{}{}ram-type-homogeneity;  
covering number of product measure;  power set
$\sigma$-quotient algebras can have countable centering number or Maharam
type;  supercompact cardinals and the normal measure axiom.}

\section{556}{Forcing with Boolean subalgebras}{3.1.15}{}{470}
{Forcing names over a Boolean subalgebra;  
Boolean operations, ring homomorphisms;  when the
subalgebra is regularly embedded;  upper bounds, suprema, saturation,
Maharam type;  quotient forcing;
Dedekind completeness;  $L^0$;  probability algebras;
relatively independent subalgebras;  strong law of large numbers;  Dye's
theorem;  Kawada's theorem;  the Dedekind completion of the asymptotic
density algebra.}

\wheader{}{10}{4}{4}{108pt}

Chapter 56:  Choice and Determinacy
%page-ends checked 10.1.15

\chapintrosection{10.7.08}{}{502}

\section{561}{Analysis without choice}{8.9.13}{}{502}
{Elementary facts;  Tychonoff's theorem;  Baire's theorem;  Stone's
theorem;  Haar measure;  Kakutani's representation of $L$-spaces;
Hilbert space.}

\section{562}{Borel codes}{20.10.13}{}{513}
{Coding sets with trees;  codable Borel sets;
in a Polish space, a set is analytic and coanalytic iff it is a 
codable Borel set;  resolvable sets
are self-coding;  codable families of codable sets;  codable Borel
functions, codable Borel equivalence;  real-valued functions;  codable
families of codable functions;  codable Baire sets and functions for
general topological spaces.}

\section{563}{Borel measures without choice}{3.12.13}{}{533}
{Borel-coded measures on second-countable spaces;  construction of
measures;  inner and outer regularity;
analytic sets are universally measurable;   Baire-coded measures
on general topological spaces;  measure algebras.}

\section{564}{Integration without choice}{9.2.14}{}{543}
{Integration with respect to Baire-coded measures;  convergence theorems
for codable sequences of functions;  Riesz representation theorem;  when
$L^1$ is a Banach space;  Radon-Nikod\'ym theorem;  conditional
expectations;  products of measures on second-countable spaces.}

\section{565}{Lebesgue measure without choice}{25.4.14}{}{559}
{Construction of Lebesgue measure as a Borel-coded measure;  Vitali's
theorem;  Fundamental Theorem of Calculus;  Hausdorff measures as
Borel-coded measures.}

\section{566}{Countable choice}{22.8.14}{}{570}
{Basic measure theory survives;  exhaustion;  $\sigma$-finite spaces and
algebras;  atomless countably additive functionals;  Vitali's theorem;
bounded additive functionals;  infinite products without DC;  topological
product measures;  the Loomis-Sikorski theorem;
the usual measure on $\{0,1\}^{\Bbb{N}}$ and its measure
algebra;  weak compactness;  automorphisms of measurable algebras;  Baire
$\sigma$-algebras;  dependent choice.}

\section{567}{Determinacy}{31.10.14}{}{587}
{Infinite games;  closed games are determined;  the axiom of determinacy;
AC($\Bbb{R};\omega$);  universal measurability and the Baire property;
automatic continuity of group homomorphisms and linear operators;
countable additivity of functionals;  reflexivity of $L$-spaces;
$\omega_1$ is two-valued-measurable;  surjections from $\Cal{P}\Bbb{N}$
onto ordinals;  two-valued-measurable cardinals and determinacy in ZFC;
measurability of PCA sets.
}

\wheader{}{10}{4}{4}{100pt}

Appendix to Volume 5
  %page-ends checked 8.1.15

\chapintrosection{10.12.07}{}{600}

\section{5A1}{Set theory}{23.2.11}{}{600}
{Ordinal and cardinal arithmetic;  trees;  cofinalities;
$\Delta$-systems and free sets;  partition calculus;
transversals.}

\section{5A2}{Pcf theory}{6.6.11}{}{285}
{Reduced products of partially ordered sets;
cofinalities of reduced products;  $\covSh(\alpha,\beta,\gamma,\delta)$;
$\Theta(\alpha,\gamma)$.}

\section{5A3}{Forcing}{18.4.08}{}{608}
{Forcing notions;  forcing languages;  the forcing relation;  the forcing
theorem;  Boolean truth values;  names for functions;
regular open algebras;  discriminating
names;  $L^0$ and names for real numbers;  forcing with Boolean algebras; 
ordinals and cardinals;  iterated forcing;  Martin's axiom;  countably
closed forcings.}

\section{5A4}{General topology}{15.8.13}{}{629}
{Cardinal functions;  compactness;  Vietoris topologies;  
category and the Baire property;  normal and paracompact spaces.}

\section{5A5}{Real analysis}{3.10.13}{}{636}
{Real-entire functions.}

\section{5A6}{Special axioms}{4.8.10}{}{636}
{GCH, \VeqL, $0^{\sharp}$ and the Covering Lemma, squares,
Chang's transfer principle, Todor\v{c}evi\'c's $p$-ideal dichotomy, the
filter dichotomy.}

\wheader{}{10}{2}{2}{100pt} 

References for Volume 5 \vtmpb{6.9.13}\pagereference{}{641} 
  %page-ends checked 10.1.15
  
\medskip 

Index to Volumes 1-5
  %page-ends checked 10.1.15
     
\qquad Principal topics and results \pagereference{}{647}
     
\qquad General index \pagereference{}{661}
     
%Lulu 2015:  411 pages
