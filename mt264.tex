\frfilename{mt264.tex}
\versiondate{12.5.03}
\copyrightdate{1994}

\def\chaptername{Change of variable in the integral}
\def\sectionname{Hausdorff measures}

\newsection{264}

The next topic I wish to approach is the question of `surface
measure';  a useful example to bear in mind throughout this section and
the next is the notion of area for regions on the sphere, but any other
smoothly curved two-dimensional surface in three-dimensional space will
serve equally well.   It is I think more than plausible that our
intuitive concepts of `area' for such surfaces should correspond to
appropriate measures.   But formalizing this intuition is
non-trivial, especially if we seek the generality that simple geometric
ideas lead us to;  I mean, not contenting ourselves with arguments that
depend on the special nature of the sphere, for instance, to describe
spherical surface area.   I divide the problem into two parts.   In this
section I will describe a construction which enables us to define the
$r$-dimensional measure of an $r$-dimensional surface -- among other
things -- in $s$-dimensional space.   In the next section I will set out
the basic theorems making it possible to calculate these measures
effectively in the leading cases.

\leader{264A}{Definitions} Let $s\ge 1$ be an integer, and $r>0$.
\cmmnt{(I am primarily concerned with integral $r$,
but will not insist on this until it becomes necessary, since there are
some very interesting ideas which involve non-integral `dimension'
$r$.)}   For any $A\subseteq \BbbR^s$, $\delta>0$ set

$$\eqalign{\theta_{r\delta}A
=\inf\{\sum_{n=0}^{\infty}(\diam A_n)^r:\sequencen{A_n}
&\text{ is a sequence of subsets of }\BbbR^s\text{ covering }A,\cr
&\qquad\qquad\qquad\quad\diam A_n\le\delta\text{ for every }
  n\in\Bbb N\}.\cr}$$

\noindent   It is convenient in this context to
say that $\diam\emptyset=0$.
Now set

\Centerline{$\theta_rA=\sup_{\delta>0}\theta_{r\delta}A$;}

\noindent $\theta_r$ is {\bf $r$-dimensional Hausdorff outer measure} on
$\BbbR^s$.

\leader{264B}{}\cmmnt{ Of course we must immediately check the
following:

\medskip

\noindent}{\bf Lemma} $\theta_r$, as defined in 264A, is always an outer
measure.

\proof{ You should be used to these arguments by now, but there is an
extra step in this one, so I spell out the details.

\medskip

{\bf (a)} Interpreting the diameter of the empty set as $0$, we have
$\theta_{r\delta}\emptyset=0$ for every $\delta>0$, so
$\theta_r\emptyset=0$.

\medskip

{\bf (b)} If $A\subseteq B\subseteq\BbbR^s$, then every sequence
covering $B$
also covers $A$, so $\theta_{r\delta}A\le\theta_{r\delta}B$ for every
$\delta$ and $\theta_rA\le\theta_rB$.

\medskip

{\bf (c)} Let $\sequencen{A_n}$ be a sequence of subsets of $\BbbR^s$
with union $A$, and take any  $a<\theta_rA$.   Then there is a $\delta>0$
such that $a\le\theta_{r\delta}A$.   Now
$\theta_{r\delta}A\le\sum_{n=0}^{\infty}\theta_{r\delta}(A_n)$.   \Prf\
Let $\epsilon>0$, and for each $n\in\Bbb N$ choose a sequence
$\sequence{m}{A_{nm}}$ of sets, covering $A_n$, with 
$\diam A_{nm}\le\delta$ for every $m$ and 
$\sum_{m=0}^{\infty}(\diam A_{nm})^r\le\theta_{r\delta}A_n+2^{-n}\epsilon$.
Then $\langle A_{nm}\rangle_{m,n\in\Bbb N}$ is a cover of $A$ by 
countably many sets of diameter at most $\delta$, so

\Centerline{$\theta_{r\delta}A
\le\sum_{n=0}^{\infty}\sum_{m=0}^{\infty}(\diam A_{nm})^r
\le\sum_{n=0}^{\infty}\theta_{r\delta}A_n+2^{-n}\epsilon
=2\epsilon+\sum_{n=0}^{\infty}\theta_{r\delta}A_n$.}

\noindent As $\epsilon$ is arbitrary, we have the result.\ \Qed

Accordingly

\Centerline{$a\le\theta_{r\delta}A
\le\sum_{n=0}^{\infty}\theta_{r\delta}A_n
\le\sum_{n=0}^{\infty}\theta_rA_n$.}

\noindent As $a$ is arbitrary,

\Centerline{$\theta_rA\le\sum_{n=0}^{\infty}\theta_rA_n$;}

\noindent as $\sequencen{A_n}$ is arbitrary, $\theta_r$ is an outer
measure.
}%end of proof of 264B

\leader{264C}{Definition} If $s\ge 1$ is an integer, and $r>0$, then
{\bf Hausdorff $r$-dimensional measure} on $\BbbR^s$ is the measure
$\mu_{Hr}$ on $\BbbR^s$ defined by \Caratheodory's method from the outer
measure $\theta_r$ of 264A-264B.

\leader{264D}{Remarks}\cmmnt{ {\bf (a)} It is important to note that
the sets used in
the definition of the $\theta_{r\delta}$ need not be balls;  even in
$\BbbR^2$ not every set $A$ can be covered by a ball of the same
diameter as $A$.

\header{264Db}}{\bf (b)} In the definitions above I require $r>0$.   It
is sometimes appropriate to take $\mu_{H0}$ to be counting measure.
\cmmnt{This
is nearly the result of applying the formulae above with $r=0$, but
there can be difficulties if we interpret them over-literally.}

\header{264Dc}{\bf (c)} All Hausdorff measures must be
complete\cmmnt{, because
they are defined by \Caratheodory's method (212A)}.   For $r>0$, they
are atomless\cmmnt{ (264Yg)}.   \cmmnt{In terms of the other
criteria of \S211,
however, they are very ill-behaved;  for instance, if $r$, $s$ are
integers and $1\le r<s$, then $\mu_{Hr}$ on $\BbbR^s$ is not
semi-finite.   (I will give a proof of this in 439H in Volume 4.)
Nevertheless, they do have some striking properties which make them
reasonably tractable.}

\cmmnt{
\header{264Dd}{\bf (d)} In 264A, note that
$\theta_{r\delta}A\le\theta_{r\delta'}A$ when $0<\delta'\le\delta$;
consequently, for instance,
$\theta_rA=\lim_{n\to\infty}\theta_{r,2^{-n}}A$.   I have allowed
arbitrary sets $A_n$ in the covers, but it makes no difference if we
restrict our attention to covers consisting
of open sets or of closed sets (264Xc).
}%end of comment

\leader{264E}{Theorem} Let $s\ge 1$ be an integer, and $r\ge 0$;
let $\mu_{Hr}$ be Hausdorff $r$-dimensional measure on $\BbbR^s$, and
$\Sigma_{Hr}$ its domain.   Then every Borel subset of $\BbbR^s$ belongs
to $\Sigma_{Hr}$.

\proof{ This is trivial if $r=0$;  so suppose henceforth that $r>0$.

\medskip

{\bf (a)} The first step is to note that if $A$, $B$ are subsets of
$\Bbb R^s$
and $\eta>0$ is such that $\|x-y\|\ge\eta$ for all $x\in A$, $y\in B$,
then $\theta_r(A\cup B)=\theta_rA+\theta_rB$, where $\theta_r$ is
$r$-dimensional Hausdorff outer measure on $\BbbR^s$.   \Prf\ Of course
$\theta_r(A\cup B)\le\theta_rA+\theta_rB$, because $\theta_r$ is an
outer measure.   For the reverse inequality, we may suppose that
$\theta_r(A\cup B)<\infty$, so that $\theta_rA$ and $\theta_rB$ are both
finite.   Let $\epsilon>0$ and let $\delta_1$, $\delta_2>0$ be such that

\Centerline{$\theta_rA+\theta_rB\le
\theta_{r\delta_1}A+\theta_{r\delta_2}B+\epsilon$.}

\noindent Set $\delta=\min(\delta_1,\delta_2,\bover12\eta)>0$ and let
$\sequencen{A_n}$ be a sequence of sets of diameter at most $\delta$,
covering $A\cup B$, and such that
$\sum_{n=0}^{\infty}(\diam A_n)^r\le\theta_{r\delta}(A\cup B)+\epsilon$.
Set

\Centerline{$K=\{n:A_n\cap A\ne\emptyset\}$,
\quad$L=\{n:A_n\cap B\ne\emptyset\}$.}

\noindent Because

\Centerline{$\|x-y\|\ge\eta>\diam A_n$}

\noindent whenever $x\in A$, $y\in B$ and $n\in\Bbb N$, $K\cap
L=\emptyset$;  and of course $A\subseteq\bigcup_{n\in K}A_k$,
$B\subseteq\bigcup_{n\in L}A_n$.   Consequently

$$\eqalign{\theta_rA+\theta_rB
&\le\epsilon+\theta_{r\delta_1}A+\theta_{r\delta_2}B\cr
&\le\epsilon+\sum_{n\in K}(\diam A_n)^r+\sum_{n\in L}(\diam A_n)^r\cr
&\le\epsilon+\sum_{n=0}^{\infty}(\diam A_n)^r
\le2\epsilon+\theta_{r\delta}(A\cup B)
\le 2\epsilon+\theta_r(A\cup B).\cr}$$

\noindent As $\epsilon$ is arbitrary, $\theta_r(A\cup
B)\ge\theta_rA+\theta_rB$, as required.\ \Qed

\medskip

{\bf (b)} It follows that
$\theta_rA=\theta_r(A\cap G)+\theta_r(A\setminus G)$ whenever
$A\subseteq \BbbR^s$ and $G$ is open.
\Prf\ As usual, it is enough to consider the case $\theta_rA<\infty$
and to show that in this case
$\theta_r(A\cap G)+\theta_r(A\setminus G)\le\theta_rA$.   Set

\Centerline{$A_n=\{x:x\in A,\,\|x-y\|\ge 2^{-n}$ for every
$y\in A\setminus G\}$,}

\Centerline{$B_0=A_0$,
\quad$B_n=A_n\setminus A_{n-1}$ for $n>1$.}

\noindent Observe that $A_n\subseteq A_{n+1}$ for every $n$ and
$\bigcup_{n\in\Bbb N}A_n=\bigcup_{n\in\Bbb N}B_n=A\cap G$.   The point
is that if $m$, $n\in\Bbb N$ and $n\ge m+2$, and if $x\in B_m$ and
$y\in B_n$, then there is a $z\in A\setminus G$ such that
$\|y-z\|<2^{-n+1}\le 2^{-m-1}$, while $\|x-z\|$ must be at least
$2^{-m}$, so $\|x-y\|\ge\|x-z\|-\|y-z\|\ge 2^{-m-1}$.   It follows
that for any $k\ge 0$

\Centerline{$\sum_{m=0}^k\theta_rB_{2m}
=\theta_r(\bigcup_{m\le k}B_{2m})\le\theta_r(A\cap G)<\infty$,}

\Centerline{$\sum_{m=0}^k\theta_rB_{2m+1}
=\theta_r(\bigcup_{m\le k}B_{2m+1})\le\theta_r(A\cap G)<\infty$,}

\noindent (inducing on $k$, using (a) above for the inductive step).
Consequently $\sum_{n=0}^{\infty}\theta_rB_n<\infty$.

But now, given $\epsilon>0$, there is an $m$ such that
$\sum_{n=m}^{\infty}\theta_rB_m\le\epsilon$, so that

$$\eqalignno{\theta_r(A\cap G)+\theta_r(A\setminus G)
&\le\theta_rA_m+\sum_{n=m}^{\infty}\theta_rB_n+\theta_r(A\setminus G)\cr
&\le\epsilon+\theta_rA_m+\theta_r(A\setminus G)
=\epsilon+\theta_r(A_m\cup(A\setminus G))\cr
\noalign{\noindent (by (a) again, since $\|x-y\|\ge 2^{-m}$ for
$x\in A_m$, $y\in A\setminus G$)}
&\le\epsilon+\theta_rA.\cr}$$

\noindent As $\epsilon$ is arbitrary,
$\theta_r(A\cap G)+\theta_r(A\setminus G)\le\theta_rA$,
as required.\ \Qed

\medskip

{\bf (c)} Part (b) shows exactly that open sets belong to $\Sigma_{Hr}$.
It follows at once that the Borel $\sigma$-algebra of $\BbbR^s$ is
included in $\Sigma_{Hr}$, as claimed.
}%end of proof of 264E

\leader{264F}{Proposition} Let $s\ge 1$ be an integer, and $r>0$;
let $\theta_r$ be $r$-dimensional Hausdorff outer measure on $\BbbR^s$,
and write  $\mu_{Hr}$ for $r$-dimensional Hausdorff measure on
$\BbbR^s$, $\Sigma_{Hr}$ for its domain.   Then

(a) for every $A\subseteq \BbbR^s$ there is a Borel set $E\supseteq A$
such that $\mu_{Hr}E=\theta_rA$;

(b) $\theta_r=\mu^*_{Hr}$, the outer measure defined from $\mu_{Hr}$;

(c) if $E\in\Sigma_{Hr}$ is expressible as a countable union of sets of
finite measure,
there are Borel sets $E'$, $E''$ such that
$E'\subseteq E\subseteq E''$ and $\mu_{Hr}(E''\setminus E')=0$.

\proof{{\bf (a)} If $\theta_rA=\infty$ this is trivial -- take
$E=\BbbR^s$.
Otherwise, for each $n\in\Bbb N$ choose a sequence
$\sequence{m}{A_{nm}}$ of sets of diameter at most $2^{-n}$, covering
$A$, and such that
$\sum_{m=0}^{\infty}(\diam A_{nm})^r\le\theta_{r,2^{-n}}A+2^{-n}$.   Set
$F_{nm}=\overline{A}_{nm}$,   $E=\bigcap_{n\in\Bbb N}\bigcup_{m\in\Bbb
N}F_{nm}$;  then $E$ is a Borel set in $\BbbR^s$.   Of course

\Centerline{$A\subseteq\bigcap_{n\in\Bbb N}\bigcup_{m\in\Bbb N}A_{mn}
\subseteq\bigcap_{n\in\Bbb N}\bigcup_{m\in\Bbb N}F_{nm}=E$.}

For any $n\in\Bbb N$,

\Centerline{$\diam F_{nm}=\diam A_{nm}\le 2^{-n}$
for every $m\in\Bbb N$,}

\Centerline{$\sum_{m=0}^{\infty}(\diam F_{nm})^r
=\sum_{m=0}^{\infty}(\diam A_{nm})^r
\le\theta_{r,2^{-n}}A+2^{-n}$,}

\noindent so

\Centerline{$\theta_{r,2^{-n}}E\le\theta_{r,2^{-n}}A+2^{-n}$.}

\noindent Letting $n\to\infty$,

\Centerline{$\theta_rE=\lim_{n\to\infty}\theta_{r,2^{-n}}E
\le\lim_{n\to\infty}\theta_{r,2^{-n}}A+2^{-n}
=\theta_rA$;}

\noindent of course it follows that $\theta_rA=\theta_rE$, because
$A\subseteq E$.    Now by 264E we know that $E\in\Sigma_{Hr}$, so we can
write $\mu_{Hr}E$ in place of $\theta_rE$.

\medskip

{\bf (b)} This follows at once, because we have

\Centerline{$\mu_{Hr}^*A
=\inf\{\mu_{Hr}E:E\in\Sigma_{Hr},\,A\subseteq E\}
=\inf\{\theta_rE:E\in\Sigma_{Hr},\,A\subseteq E\}
\ge\theta_rA$}

\noindent for every $A\subseteq \BbbR^s$.   On the other hand, if
$A\subseteq\BbbR^s$, we have a Borel set $E\supseteq A$ such that
$\theta_rA=\mu_{Hr}E$, so that $\mu_{Hr}^*A\le\mu_{Hr}E=\theta_rA$.

\medskip

{\bf (c)(i)} Suppose first that $\mu_{Hr}E<\infty$.   By (a), there are
Borel sets $E''\supseteq E$, $H\supseteq E''\setminus E$ such that
$\mu_{Hr}E''=\theta_rE$,

\Centerline{$\mu_{Hr}H=\theta_r(E''\setminus E)
=\mu_{Hr}(E''\setminus E)
=\mu_{Hr}E''-\mu_{Hr}E
=\mu_{Hr}E''-\theta_rE
=0$.}

\noindent So setting $E'=E''\setminus H$, we obtain a Borel set
included in $E$, and

\Centerline{$\mu_{Hr}(E''\setminus E')\le\mu_{Hr}H=0$.}

\medskip

\quad{\bf (ii)} For the general case, express $E$ as $\bigcup_{n\in\Bbb
N}E_n$ where $\mu_{Hr}E_n<\infty$ for each $n$;  take Borel sets $E'_n$,
$E_n''$ such that $E_n'\subseteq E_n\subseteq E_n''$ and
$\mu_{Hr}(E_n''\setminus E_n')=0$ for each $n$;  and set
$E'=\bigcup_{n\in\Bbb N}E'_n$, $E''=\bigcup_{n\in\Bbb N}E_n''$.
}%end of proof of 264F

\leader{264G}{Lipschitz functions}\dvro{\bf: }{ The definition of
Hausdorff measure is exactly adapted to the following result,
corresponding to 262D.

\wheader{264G}{6}{2}{2}{48pt}

\noindent}{\bf Proposition} Let $m$, $s\ge 1$ be integers, and
$\phi:D\to\BbbR^s$ a $\gamma$-Lipschitz function, where $D$ is a subset
of $\BbbR^m$.   Then for any $A\subseteq D$ and $r\ge 0$,

\Centerline{$\mu_{Hr}^*(\phi[A])\le\gamma^r\mu_{Hr}^*A$}

\noindent for every $A\subseteq D$, writing $\mu_{Hr}$ for
$r$-dimensional Hausdorff outer measure on either $\BbbR^m$ or
$\BbbR^s$.

\proof{{\bf (a)} The case $r=0$ is trivial, since then $\gamma^r=1$ and
$\mu_{Hr}^*A=\mu_{H0}A=\#(A)$ if $A$ is finite, $\infty$
otherwise, while $\#(\phi[A])\le\#(A)$.

\medskip

{\bf (b)} If $r>0$, then take any $\delta>0$.   Set
$\eta=\delta/(1+\gamma)$ and consider $\theta_{r\eta}:\Cal
P\BbbR^m\to[0,\infty]$, defined as in 264A.   We know from 264Fb
that

\Centerline{$\mu_{Hr}^*A=\theta_rA\ge\theta_{r\eta}A$,}

\noindent so there is a sequence $\sequencen{A_n}$ of sets, all of
diameter at most $\eta$, covering $A$, with
$\sum_{n=0}^{\infty}(\diam A_n)^r\le\mu_{Hr}^*A+\delta$.
Now $\phi[A]\subseteq\bigcup_{n\in\Bbb N}\phi[A_n\cap D]$ and

\Centerline{$\diam \phi[A_n\cap D]
\le\gamma\diam A_n\le\gamma\eta\le\delta$}

\noindent for every $n$.   Consequently

\Centerline{$\theta_{r\delta}(\phi[A])\le\sum_{n=0}^{\infty}
(\diam \phi[A_n])^r
\le\sum_{n=0}^{\infty}\gamma^r(\diam A_n)^r
\le\gamma^r(\mu_{Hr}^*A+\delta)$,}

\noindent and

\Centerline{$\mu_{Hr}^*(\phi[A])
=\lim_{\delta\downarrow 0}\theta_{r\delta}(\phi[A])
\le\gamma^r\mu_{Hr}^*A$,}

\noindent as claimed.
}%end of proof of 264G

\leader{264H}{}\cmmnt{ The next step is to relate $r$-dimensional
Hausdorff
measure on $\BbbR^r$ to Lebesgue measure on $\BbbR^r$.   The basic
fact we need is the following, which is even more important for the idea
in its proof than for the result.

\wheader{264H}{6}{2}{2}{48pt}\noindent}{\bf Theorem}
%hold this line to get spacing right
Let $r\ge 1$ be an integer, and $A$ a bounded
subset of $\BbbR^r$;  write $\mu_r$ for Lebesgue measure on $\BbbR^r$
and $d=\diam A$.   Then

\Centerline{$\mu_r^*(A)\le\mu_rB(\tbf{0},\Bover{d}2)=2^{-r}\beta_rd^r$,}

\noindent where $B(\tbf{0},\bover{d}2)$ is the ball with centre
$\tbf{0}$ and
diameter $d$, so that $B(\tbf{0},1)$ is the unit ball in $\BbbR^r$, and
has measure

$$\eqalign{\beta_r
&=\Bover{1}{k!}\pi^k\text{ if }r=2k\text{ is even},\cr
&=\Bover{2^{2k+1}k!}{(2k+1)!}\pi^k\text{ if }r=2k+1\text{ is odd}.\cr}$$

\proof{{\bf (a)} For the calculation of $\beta_r$, see 252Q or
252Xi.

\medskip

{\bf (b)} The case $r=1$ is elementary, for in this case $A$ is included
in an interval of length $\diam A$, so that $\mu_1^*A\le\diam A$.   So
henceforth let us suppose that $r\ge 2$.

\medskip

{\bf (c)} For $1\le i\le r$ let $S_i:\BbbR^r\to\BbbR^r$ be reflection
in the $i$th coordinate, so that
$S_ix=(\xi_1,\ldots,
\ifdim\pagewidth>467pt\penalty-100\fi
\xi_{i-1},-\xi_i,\xi_{i+1},
\ifdim\pagewidth>467pt\penalty-50\fi
\ldots,
\ifdim\pagewidth>467pt\penalty-10\fi
\xi_r)$ for every
$x=(\xi_1,\ldots,\xi_r)\in \BbbR^r$.   Let us say that a set
$C\subseteq\BbbR^r$ is {\bf symmetric in coordinates in $J$}, where
$J\subseteq\{1,\ldots,r\}$, if $S_i[C]=C$ for $i\in J$.   Now the centre
of the argument is the following fact:  if $C\subseteq\Bbb R$ is a
bounded set which is symmetric in coordinates in $J$, where $J$ is a
proper subset of $\{1,\ldots,r\}$, and $j\in\{1,\ldots,r\}\setminus J$,
then there is a set $D$, symmetric in coordinates in $J\cup\{j\}$,
such that $\diam D\le\diam C$ and $\mu_r^*C\le\mu^*_rD$.


\Prf\ {\bf (i)} Because Lebesgue measure is invariant under permutation
of coordinates, it is enough to deal with the case $j=r$.   Start by
writing $F=\overline{C}$, so that $\diam F=\diam C$ and
$\mu_rF\ge\mu_r^*C$.   Note that because $S_i$ is a homeomorphism for
every $i$,

\Centerline{$S_i[F]=S_i[\overline{C}]=\overline{S_i[C]}=\overline{C}=F$}

\noindent for $i\in J$, and $F$ is symmetric in coordinates in $J$.

For $y=(\eta_1,\ldots,\eta_{r-1})\in\BbbR^{r-1}$, set

\Centerline{$F_y=\{\xi:(\eta_1,\ldots,\eta_{r-1},\xi)\in F\}$,\quad
$f(y)=\mu_1F_y$,}

\noindent where $\mu_1$ is Lebesgue measure on $\Bbb R$.
Set

\Centerline{$D=\{(y,\xi):y\in\BbbR^{r-1},\,|\xi|<\Bover12f(y)\}
\subseteq\BbbR^r$.}

\medskip

\quad{\bf (ii)} If $H\subseteq\BbbR^r$ is measurable and $H\supseteq D$,
then, writing $\mu_{r-1}$ for Lebesgue measure on $\BbbR^{r-1}$, we have

$$\eqalignno{\mu_rH
&=\int\mu_1\{\xi:(y,\xi)\in H\}\mu_{r-1}(dy)\cr
\noalign{\noindent (using 251N and 252D)}
&\ge\int\mu_1\{\xi:(y,\xi)\in D\}\mu_{r-1}(dy)
=\int f(y)\mu_{r-1}(dy)\cr
&=\int\mu_1\{\xi:(y,\xi)\in F\}\mu_{r-1}(dy)
=\mu_rF
\ge\mu^*_rC.\cr}$$

\noindent As $H$ is arbitrary, $\mu_r^*D\ge\mu_r^*C$.

\medskip

\quad{\bf (iii)} The next step is to check that $\diam D\le\diam C$.
If $x$, $x'\in D$, express them as $(y,\xi_r)$ and $(y',\xi_r')$.
Because $F$ is a bounded closed set in $\BbbR^r$, $F_y$ and $F_{y'}$ are
bounded closed subsets of $\Bbb R$.   Also both $f(y)$ and $f(y')$ must
be
greater than $0$, so that $F_y$, $F_{y'}$ are both non-empty.
Consequently

\Centerline{$\alpha=\inf F_y$,\quad $\beta=\sup F_y$,
\quad$\alpha'=\inf F_{y'}$,
\quad$\beta'=\sup F_{y'}$}

\noindent are all defined in $\Bbb R$, and $\alpha$, $\beta\in F_y$,
while
$\alpha'$ and $\beta'$ belong to $F_{y'}$.   We have

$$\eqalign{|\xi_r-\xi_r'|
&\le|\xi_r|+|\xi'_r|
<\Bover12f(y)+\Bover12f(y')\cr
&=\Bover12(\mu_1F_y+\mu_1F_{y'})
\le\Bover12(\beta-\alpha+\beta'-\alpha')
\le\max(\beta'-\alpha,\beta-\alpha').\cr}$$

\noindent So taking $(\xi,\xi')$ to be one of $(\alpha,\beta')$ or
$(\beta,\alpha')$, we can find $\xi\in F_y$, $\xi'\in F_{y'}$ such that
$|\xi-\xi'|\ge|\xi_r-\xi'_r|$.   Now $z=(y,\xi)$, $z'=(y',\xi')$ both
belong to $F$, so

\Centerline{$\|x-x'\|^2=\|y-y'\|^2+|\xi_r-\xi_r'|^2
\le\|y-y'\|^2+|\xi-\xi'|^2
=\|z-z'\|^2
\le(\diam F)^2$,}

\noindent and $\|x-x'\|\le\diam F$.   As $x$ and $x'$ are arbitrary,
$\diam D\le\diam F=\diam C$, as claimed.

\medskip

\quad{\bf (iv)} Evidently $S_r[D]=D$.   Moreover, if $i\in J$, then
(interpreting $S_i$ as an operator on $\BbbR^{r-1}$)

\Centerline{$F_{S_i(y)}=F_y$ for every $y\in\BbbR^{r-1}$,}

\noindent so $f(S_i(y))=f(y)$ and, for $\xi\in \Bbb R$,
$y\in\BbbR^{r-1}$,

\Centerline{$(y,\xi)\in D\,\iff\,|\xi|<\bover12 f(y)\,
\iff\,|\xi|<\bover12f(S_i(y))
\,\iff\,(S_i(y),\xi)\in D$,}

\noindent so that $S_i[D]=D$.   Thus $D$ is symmetric in coordinates in
$J\cup\{r\}$.\ \Qed

\medskip

{\bf (d)} The rest is easy.   Starting from any bounded
$A\subseteq\BbbR^r$, set $A_0=A$ and construct inductively
$A_1,\ldots,A_r$ such that

\Centerline{$d=\diam A=\diam A_0\ge\diam A_1\ge\ldots\ge\diam A_r$,}

\Centerline{$\mu_r^*A=\mu_r^*A_0\le\ldots\le\mu_r^*A_r$,}

\Centerline{$A_j$ is symmetric in coordinates in $\{1,\ldots,j\}$ for
every $j\le r$.}

\noindent At the end, we have $A_r$ symmetric in coordinates in
$\{1,\ldots,r\}$.   But this means that if $x\in A_r$ then

\Centerline{$-x=S_1S_2\ldots S_rx\in A_r$,}

\noindent so that

\Centerline{$\|x\|=\Bover12\|x-(-x)\|
\le\Bover12\diam A_r\le\Bover{d}2$.}

\noindent Thus $A_r\subseteq B(\tbf{0},\bover{d}2)$, and

\Centerline{$\mu_r^*A\le\mu_r^*A_r\le\mu_rB(\tbf{0},\Bover{d}2)$,}

\noindent as claimed.
}%end of proof of 264H


\leader{264I}{Theorem} Let $r\ge 1$ be an integer;  let $\mu$ be
Lebesgue measure on $\BbbR^r$, and let $\mu_{Hr}$ be
$r$-dimensional Hausdorff measure on $\BbbR^r$.   Then $\mu$ and
$\mu_{Hr}$ have the same measurable sets and

\Centerline{$\mu E=2^{-r}\beta_r\mu_{Hr}E$}

\noindent for every measurable set $E\subseteq\BbbR^r$, where
$\beta_r=\mu B(\tbf{0},1)$, so that the normalizing factor is

$$\eqalign{2^{-r}\beta_r
&=\Bover{1}{2^{2k}k!}\pi^k\text{ if }r=2k\text{ is even},\cr
&=\Bover{k!}{(2k+1)!}\pi^k\text{ if }r=2k+1\text{ is odd}.\cr}$$

\proof{{\bf (a)} Of course if $B=B(x,\alpha)$ is any ball of radius
$\alpha$,

\Centerline{$2^{-r}\beta_r(\diam B)^r
=\beta_r\alpha^r=\mu B$.}

\medskip

{\bf (b)} The point is that $\mu^*=2^{-r}\beta_r\mu_{Hr}^*$.   \Prf\
Let $A\subseteq\BbbR^r$.

\medskip

\quad{\bf (i)} Let $\delta$, $\epsilon>0$.   By
261F, there is a sequence $\sequencen{B_n}$ of  balls, all of diameter
at most $\delta$, such that $A\subseteq\bigcup_{n\in\Bbb N}B_n$ and
$\sum_{n=0}^{\infty}\mu B_n\le\mu^*A+\epsilon$.   Now, defining
$\theta_{r\delta}$ as in 264A,

\Centerline{$2^{-r}\beta_r\theta_{r\delta}(A)
\le 2^{-r}\beta_r\sum_{n=0}^{\infty}(\diam B_n)^r
=\sum_{n=0}^{\infty}\mu B_n
\le\mu^*A+\epsilon$.}

\noindent Letting $\delta\downarrow 0$,

\Centerline{$2^{-r}\beta_r\mu^*_{Hr}A\le\mu^*A+\epsilon$.}

\noindent As $\epsilon$ is arbitrary,
$2^{-r}\beta_r\mu_{Hr}^*A\le\mu^*A$.

\medskip

\quad{\bf (ii)} Let $\epsilon>0$.   Then there is a sequence
$\sequencen{A_n}$ of sets of diameter at most $1$ such that
$A\subseteq\bigcup_{n\in\Bbb N}A_n$ and $\sum_{n=0}^{\infty}(\diam
A_n)^r\le\theta_{r1}A+\epsilon$, so that

\Centerline{$\mu^*A\le\sum_{n=0}^{\infty}\mu^*A_n
\le\sum_{n=0}^{\infty}2^{-r}\beta_r(\diam A_n)^r
\le 2^{-r}\beta_r(\theta_{r1}A+\epsilon)
\le 2^{-r}\beta_r(\mu_{Hr}^*A+\epsilon)$}

\noindent by 264H.   As $\epsilon$ is arbitrary,
$\mu^*A\le 2^{-r}\beta_r\mu_{Hr}^*A$.\ \Qed

\medskip

{\bf (c)} Because $\mu$, $\mu_{Hr}$ are the measures defined from
their respective outer measures by \Caratheodory's method, it follows
at once that $\mu=2^{-r}\beta_r\mu_{Hr}$ in the strict sense required.
}%end of proof of 264I

\leader{*264J}{The Cantor \dvrocolon{set}}\cmmnt{ I remarked in 264A
that fractional
`dimensions' $r$ were of interest.   I have no space for these here, and
they are off the main lines of this volume, but I will give one result
for its intrinsic interest.

\medskip

\noindent}{\bf Proposition} Let $C$ be the Cantor set in $[0,1]$.   Set
$r=\ln 2/\ln 3$.   Then the $r$-dimensional Hausdorff measure of $C$ is
$1$.

\proof{{\bf (a)} Recall that $C=\bigcap_{n\in\Bbb N}C_n$, where each
$C_n$ consists of $2^n$ closed intervals of length $3^{-n}$, and
$C_{n+1}$ is obtained from $C_n$ by deleting the middle (open) third of
each interval of $C_n$.   (See 134G.)    Because $C$ is
closed, $\mu_{Hr}C$ is defined (264E).   Note that $3^r=2$.

\medskip

{\bf (b)} If $\delta>0$, take $n$ such that $3^{-n}\le\delta$;  then $C$
can be covered by $2^n$ intervals of diameter $3^{-n}$, so

\Centerline{$\theta_{r\delta}C\le 2^n(3^{-n})^r=1$.}

\noindent Consequently

\Centerline{$\mu_{Hr}C=\mu_{Hr}^*C
=\lim_{\delta\downarrow 0}\theta_{r\delta}C\le 1$.}

\medskip

{\bf (c)} We need the following elementary fact:  if $\alpha$, $\beta$,
$\gamma\ge 0$ and $\max(\alpha,\gamma)\le\beta$, then
$\alpha^r+\gamma^r\le(\alpha+\beta+\gamma)^r$.   \Prf\ Because
$0<r\le 1$,

\Centerline{$\xi\mapsto(\xi+\eta)^r-\xi^r
=r\biggerint_0^{\eta}(\xi+\zeta)^{r-1}d\zeta$}

\noindent is non-increasing for every $\eta\ge 0$.
Consequently

$$\eqalign{(\alpha+\beta+\gamma)^r-\alpha^r-\gamma^r
&\ge (\beta+\beta+\gamma)^r-\beta^r-\gamma^r\cr
&\ge(\beta+\beta+\beta)^r-\beta^r-\beta^r
=\beta^r(3^r-2)
=0,\cr}$$

\noindent as required.\ \Qed

\medskip

{\bf (d)} Now suppose that $I\subseteq\Bbb R$ is any interval, and
$m\in\Bbb N$;  write $j_m(I)$ for the number of the intervals composing
$C_m$ which are included in $I$.   Then $2^{-m}j_m(I)\le(\diam I)^r$.
\Prf\ If $I$ does not meet $C_m$, this is trivial.   Otherwise, induce
on

\Centerline{$l=\min\{i:I$ meets only one of the intervals composing
$C_{m-i}\}$.}

\noindent If $l=0$, so that $I$ meets only one of the intervals
composing $C_m$, then $j_m(I)\le 1$, and if $j_m(I)=1$ then
$\diam I\ge 3^{-m}$ so $(\diam I)^r\ge 2^{-m}$;  thus the induction
starts.   For
the inductive step to $l>1$, let $J$ be the interval of
$C_{m-l}$ which meets $I$, and $J'$, $J''$ the two intervals of
$C_{m-l+1}$ included in $J$, so that $I$ meets both $J'$ and $J''$, and


\Centerline{$j_m(I)=j_m(I\cap J)=j_m(I\cap J')+j_m(I\cap J'')$.}

\noindent By the inductive hypothesis,

\Centerline{$(\diam(I\cap J'))^r+(\diam(I\cap J''))^r
\ge 2^{-m}j_m(I\cap J')+2^{-m}j_m(I\cap J'')
=2^{-m}j_m(I)$.}

\noindent On the other hand, by (c),

$$\eqalign{(\diam(I\cap J'))^r+(\diam(I\cap J''))^r
&\le(\diam(I\cap J')+3^{-m+l-1}+\diam(I\cap J''))^r\cr
&=(\diam(I\cap J))^r
\le(\diam I)^r\cr}$$

\noindent because $J'$, $J''$ both have diameter at most $3^{-(m-l+1)}$,
the length of the interval between them.
Thus the induction continues.\ \Qed

\medskip

{\bf (e)} Now suppose that $\epsilon>0$.   Then there is a sequence
$\sequencen{A_n}$ of sets, covering $C$, such that
\discrcenter{468pt}{$\sum_{n=0}^{\infty}(\diam A_n)^r
<\mu_{Hr}C+\epsilon$.  }Take $\eta_n>0$ such that
$\sum_{n=0}^{\infty}(\diam A_n+\eta_n)^r\le\mu_{Hr}C+\epsilon$, and for
each $n$ take an open
interval $I_n\supseteq A_n$ of length at most $\diam A_n+\eta_n$ and
with neither endpoint belonging to $C$;  this is possible because $C$
does not include any non-trivial interval.   Now
$C\subseteq\bigcup_{n\in\Bbb N}I_n$;  because $C$ is compact, there is a
$k\in\Bbb N$ such that $C\subseteq\bigcup_{n\le k}I_n$.   Next, there is
an $m\in\Bbb N$ such that no endpoint of any $I_n$, for $n\le k$,
belongs to $C_m$.   Consequently each of the intervals composing $C_m$
must be included in some $I_n$, and
(in the terminology of (d) above) $\sum_{n=0}^kj_m(I_n)\ge 2^m$.
Accordingly

\Centerline{$1
\le\sum_{n=0}^k2^{-m}j_m(I_n)
\le \sum_{n=0}^k(\diam I_n)^r
\le \sum_{n=0}^{\infty}(\diam A_n+\eta_n)^r
\le\mu_{Hr}C+\epsilon$.}

\noindent As $\epsilon$ is arbitrary, $\mu_{Hr}C\ge 1$, as required.
}%end of proof of 264J

\leader{*264K}{General metric spaces}\cmmnt{ While this chapter deals
exclusively with Euclidean spaces, readers familiar with the general
theory
of metric spaces may find the nature of the theory clearer if they use
the language of metric spaces in the basic definitions and results.   I
therefore repeat the definition here, and spell out the corresponding
results in the exercises 264Yb-264Yl.

} Let $(X,\rho)$ be a metric space, and $r>0$.   For any $A\subseteq X$,
$\delta>0$ set

$$\eqalign{\theta_{r\delta}A
=\inf\{\sum_{n=0}^{\infty}(\diam A_n)^r:\sequencen{A_n}
&\text{ is a sequence of subsets of }X\text{ covering }A,\cr
&\qquad\qquad\qquad\quad\diam A_n\le\delta\text{ for every }
  n\in\Bbb N\},\cr}$$

\noindent interpreting the diameter of the empty set as $0$, and
$\inf\emptyset$ as $\infty$, so that $\theta_{r\delta}A=\infty$ if $A$
cannot be covered by a sequence of sets of diameter at most $\delta$.
Say
that $\theta_rA=\sup_{\delta>0}\theta_{r\delta}A$ is the
{\bf $r$-dimensional Hausdorff outer measure} of $A$, and take the
measure $\mu_{Hr}$ defined by \Caratheodory's method from this outer measure to be {\bf $r$-dimensional Hausdorff measure} on $X$.

\exercises{
\leader{264X}{Basic exercises $\pmb{>}$(a)}
%\spheader 264Xa
Show that all the functions $\theta_{r\delta}$ of 264A are outer
measures.   Show that in that context, $\theta_{r\delta}(A)=0$ iff
$\theta_r(A)=0$, for any $\delta>0$ and any $A\subseteq\BbbR^s$.

\spheader 264Xb Let $s\ge 1$ be an integer, and $\theta$ an
outer measure on $\BbbR^s$ such that $\theta(A\cup B)=\theta A+\theta B$
whenever $A$, $B$ are non-empty subsets of $\BbbR^s$ and
$\inf_{x\in A,y\in B}\|x-y\|>0$.   Show that every Borel subset of
$\BbbR^s$ is measured by the
measure defined from $\theta$ by \Caratheodory's method.

\sqheader 264Xc Let $s\ge 1$ be an integer and $r>0$;
define $\theta_{r\delta}$ as in 264A.   Show that for any
$A\subseteq\BbbR^s$, $\delta>0$,

$$\eqalign{\theta_{r\delta}A
&=\inf\{\sum_{n=0}^{\infty}(\diam F_n)^r:\sequencen{F_n}
 \text{ is a sequence of closed subsets of }X\cr
&\qquad\qquad\qquad\qquad\qquad\qquad\text{ covering }A,\diam
F_n\le\delta
  \text{ for every }n\in\Bbb N\}\cr
&=\inf\{\sum_{n=0}^{\infty}(\diam G_n)^r:\sequencen{G_n}
  \text{ is a sequence of open subsets of }X\cr
&\qquad\qquad\qquad\qquad\qquad\qquad\text{ covering }A,\diam
G_n\le\delta
  \text{ for every }n\in\Bbb N\}.\cr}$$

\sqheader 264Xd Let $s\ge 1$ be an integer and $r\ge 0$;
let $\mu_{Hr}$ be $r$-dimensional Hausdorff measure on $\BbbR^s$.
Show that for every $A\subseteq \BbbR^s$ there is a
G$_{\delta}$ set (that is, a set expressible as the intersection of a
sequence of open sets) $H\supseteq A$ such that $\mu_{Hr}H=\mu^*_{Hr}A$.
({\it Hint\/}:  use 264Xc.)

\sqheader 264Xe Let $s\ge 1$ be an integer, and $0\le r<r'$.   Show that
if $A\subseteq\BbbR^s$ and the $r$-dimensional
Hausdorff outer measure $\mu_{Hr}^*A$ of $A$ is finite, then
$\mu_{Hr'}^*A$ must be zero.

\spheader 264Xf(i) Suppose that $f:[a,b]\to\Bbb R$ has graph
$\Gamma_f\subseteq\BbbR^2$, where $a\le b$ in $\Bbb R$.   Show that the
outer measure $\mu_{H1}^*(\Gamma_f)$ of $\Gamma$ for one-dimensional
Hausdorff measure on $\BbbR^2$ is at most $b-a+\Var_{[a,b]}(f)$.
\Hint{if $f$ has finite variation, show that
$\diam(\Gamma_{f\restr\ooint{t,u}})
\le u-t+\Var_{\ooint{t,u}}(f)$;  then use 224E.}  (ii) Let
$f:[0,1]\to[0,1]$ be the Cantor function (134H).   Show that
$\mu_{H1}(\Gamma_f)=2$.   \Hint{264G.}
%264G

\spheader 264Xg\dvAnew{2009.} In 264A, show that

$$\eqalign{\theta_{r\delta}A
=\inf\{\sum_{n=0}^{\infty}(\diam A_n)^r:\sequencen{A_n}
&\text{ is a sequence of convex sets covering }A,\cr
&\qquad\qquad\qquad\quad\diam A_n\le\delta\text{ for every }
  n\in\Bbb N\}\cr}$$

\noindent for any $A\subseteq\BbbR^s$.
%264A

\leader{264Y}{Further exercises (a)}
%\spheader 264Ya
Let $\theta_{11}$ be the outer measure on $\BbbR^2$ defined in 264A,
with
$r=\delta=1$, and $\mu_{11}$ the measure derived from $\theta_{11}$ by
\Caratheodory's method, $\Sigma_{11}$ its domain.   Show that any set in
$\Sigma_{11}$ is either negligible or conegligible.

\spheader 264Yb Let $(X,\rho)$ be a metric space and
$r\ge 0$.   Show that if $A\subseteq X$ and $\mu_{Hr}^*A<\infty$,
then $A$ is separable.

\spheader 264Yc Let $(X,\rho)$ be a metric space, and $\theta$ an
outer measure on $X$ such that $\theta(A\cup B)=\theta A+\theta B$
whenever $A$, $B$ are non-empty subsets of $X$ and $\inf_{x\in A,y\in
B}\rho(x,y)>0$.   (Such an outer measure is called a {\bf metric outer
measure}.)    Show that every open subset of $X$ is measured by the
measure defined from $\theta$ by \Caratheodory's method.

\spheader 264Yd Let $(X,\rho)$ be a metric space and $r>0$;
define $\theta_{r\delta}$ as in 264K.    Show that for any
$A\subseteq X$,

$$\eqalign{\mu_{Hr}^*A
&=\sup_{\delta>0}\inf\{\sum_{n=0}^{\infty}(\diam F_n)^r:\sequencen{F_n}
  \text{ is a sequence of closed subsets of }X\cr
&\qquad\qquad\qquad\qquad\qquad\qquad\text{ covering }A,\,\diam
F_n\le\delta
  \text{ for every }n\in\Bbb N\}\cr
&=\sup_{\delta>0}\inf\{\sum_{n=0}^{\infty}(\diam G_n)^r:\sequencen{G_n}
  \text{ is a sequence of open subsets of }X\cr
&\qquad\qquad\qquad\qquad\qquad\qquad\text{ covering }A,\,\diam
G_n\le\delta
  \text{ for every }n\in\Bbb N\}.\cr}$$
\spheader 264Ye Let $(X,\rho)$ be a metric space and $r\ge 0$;
let $\mu_{Hr}$ be $r$-dimensional Hausdorff measure on $X$.
Show that for every $A\subseteq X$ there is a
G$_{\delta}$ set $H\supseteq A$ such that $\mu_{Hr}H=\mu^*_{Hr}A$ is the
$r$-dimensional Hausdorff outer measure of $A$.

\spheader 264Yf Let $(X,\rho)$ be a metric space and $r\ge 0$;
let $Y$ be any subset of $X$, and give $Y$ its induced metric $\rho_Y$.
(i) Show that the $r$-dimensional Hausdorff outer measure
$\mu^{(Y)*}_{Hr}$ on $Y$ is just the restriction to $\Cal PY$ of the
outer measure $\mu_{Hr}^*$ on $X$.   (ii) Show that if {\it either}
$\mu_{Hr}^*Y<\infty$ {\it or} $\mu_{Hr}$ measures $Y$ then
$r$-dimensional Hausdorff measure $\mu^{(Y)}_{Hr}$ on
$Y$ is just the subspace measure on $Y$ induced by the measure
$\mu_{Hr}$ on $X$.

\spheader 264Yg Let $(X,\rho)$ be a metric space
and $r>0$.   Show that $r$-dimensional Hausdorff measure on $X$ is
atomless.   ({\it Hint\/}: Let $E\in\dom\mu_{Hr}$.   (i) If
$E$ is not separable, there is an open set $G$ such that $E\cap G$ and
$E\setminus G$ are both non-separable, therefore both non-negligible.
(ii) If there is an $x\in E$ such that $\mu_{Hr}(E\cap B(x,\delta))>0$
for
every $\delta>0$, then one of these sets has non-negligible complement
in
$E$.    (iii) Otherwise, $\mu_{Hr}E=0$.)

\spheader 264Yh  Let $(X,\rho)$ be a metric space and $r\ge 0$;
let $\mu_{Hr}$ be
$r$-dimensional Hausdorff measure on $X$.   Show that if
$\mu_{Hr}E<\infty$
then $\mu_{Hr}E=\sup\{\mu_{Hr}F:F\subseteq E$ is closed and totally
bounded$\}$.   ({\it Hint\/}: given $\epsilon>0$, use 264Yd to find a
closed totally bounded set $F$ such that $\mu_{Hr}(F\setminus E)=0$
and $\mu_{Hr}(E\setminus F)\le\epsilon$, and now apply 264Ye to
$F\setminus E$.)

\spheader 264Yi  Let $(X,\rho)$ be a complete metric space and $r\ge 0$;
let $\mu_{Hr}$ be
$r$-dimensional Hausdorff measure on $X$.   Show that if
$\mu_{Hr}E<\infty$
then $\mu_{Hr}E=\sup\{\mu_{Hr}F:F\subseteq E$ is compact$\}$.

\spheader 264Yj Let $(X,\rho)$ and $(Y,\sigma)$ be metric spaces.   If
$D\subseteq X$ and $\phi:D\to Y$ is a function, then $\phi$ is {\bf
$\gamma$-Lipschitz} if
$\sigma(\phi(x),\phi(x'))\le\gamma\rho(x,x')$ for every $x$, $x'\in D$.
(i) Show that in this case, if $r\ge 0$,
$\mu_{Hr}^*(\phi[A])\le\gamma^r\mu_{Hr}^*A$ for every $A\subseteq D$,
writing $\mu_{Hr}^*$ for $r$-dimensional Hausdorff outer measure on
either $X$ or $Y$.   (ii) Show that if $X$ is complete and $\mu_{Hr}E$
is defined
and finite, then $\mu_{Hr}(\phi[E])$ is defined.   ({\it Hint\/}:
264Yi.)

\spheader 264Yk Let $(X,\rho)$ be a metric space, and for $r\ge 0$ let
$\mu_{Hr}$ be Hausdorff $r$-dimensional measure on $X$.   Show that
there is a unique $\Delta=\Delta(X)\in[0,\infty]$ such that
$\mu_{Hr}X=\infty$ if
$r\in\coint{0,\Delta}$, $0$ if $r\in\ooint{\Delta,\infty}$.

\spheader 264Yl Let $(X,\rho)$ be a metric space and $\phi:I\to X$ a
continuous function, where $I\subseteq\Bbb R$ is an interval.   Write
$\mu_{H1}$ for one-dimensional Hausdorff measure on $X$.   Show that

\Centerline{$\mu_{H1}(\phi[I])
\le\sup\{\sum_{i=1}^n\rho(\phi(t_i),\phi(t_{i-1})):t_0,\ldots,t_n\in
I,\,t_0\le\ldots\le t_n\}$,}

\noindent the length of the curve $\phi$, with equality if $\phi$ is
injective.


\spheader 264Ym Set $r=\ln 2/\ln 3$, as in 264J, and write
$\mu_{Hr}$ for $r$-dimensional Hausdorff measure on the Cantor set $C$.
Let $\lambda$ be the usual measure on $\{0,1\}^{\Bbb N}$ (254J).
Define $\phi:\{0,1\}^{\Bbb N}\to C$ by setting
$\phi(x)=\bover23\sum_{n=0}^{\infty}3^{-n}x(n)$ for
$x\in\{0,1\}^{\Bbb N}$.   Show that $\phi$ is an isomorphism between
$(\{0,1\}^{\Bbb N},\lambda)$ and $(C,\mu_{Hr})$, so that $\mu_{Hr}$ is the
subspace measure on $C$ induced by `Cantor measure' as defined in 256Hc.

\spheader 264Yn Set $r=\ln 2/\ln 3$ and write
$\mu_{Hr}$ for $r$-dimensional Hausdorff measure on the Cantor set $C$.
Let $f:[0,1]\to[0,1]$ be the Cantor function and let $\mu$ be
Lebesgue measure on $\Bbb R$.   Show that $\mu f[E]=\mu_{Hr}E$ for every
$E\in\dom\mu_{Hr}$ and $\mu_{Hr}(C\cap f^{-1}[F])=\mu F$ for every
Lebesgue measurable set $F\subseteq[0,1]$.

\spheader 264Yo Let $(X,\rho)$ be a metric space and
$h:\coint{0,\infty}\to\coint{0,\infty}$ a non-decreasing function.   For
$A\subseteq X$ set

$$\eqalign{\theta_hA
&=\sup_{\delta>0}\inf\{\sum_{n=0}^{\infty}h(\diam A_n):\sequencen{A_n}
  \text{ is a sequence of subsets of }X\cr
&\qquad\qquad\qquad\qquad\qquad\qquad\text{ covering }A,\,\diam
A_n\le\delta
  \text{ for every }n\in\Bbb N\},\cr}$$

\noindent interpreting $\diam\emptyset$ as $0$, $\inf\emptyset$ as
$\infty$ as usual.   Show that $\theta_h$ is an outer measure on $X$.
State and prove theorems corresponding to 264E and 264F.   Look through
264X and 264Y for further results which might be generalizable, perhaps
on the assumption that $h$ is continuous on the right.

\spheader 264Yp Let $(X,\rho)$ be a metric space.   Let us say that if
$a<b$ in $\Bbb R$ and $f:[a,b]\to X$ is a function, then $f$ is {\bf
absolutely continuous} if for every $\epsilon>0$ there is a $\delta>0$
such that $\sum_{i=1}^n\rho(f(a_i),f(b_i))\le\epsilon$ whenever
$a\le a_0\le b_0\le\ldots\le a_n\le b_n\le b$ and
$\sum_{i=0}^nb_i-a_i\le\delta$.   Show that $f:[a,b]\to X$ is absolutely
continuous iff it is continuous and of bounded variation (in the sense
of 224Ye) and $\mu_{H1}f[A]=0$ whenever $A\subseteq[a,b]$ is Lebesgue
negligible, where $\mu_{H1}$ is 1-dimensional Hausdorff measure on $X$.
(Compare 225M.)
Show that in this case $\mu_{H1}f[\,[a,b]\,]<\infty$.
%-  mt26bits

\spheader 264Yq Let $s\ge 1$ be an integer, and $r\in\coint{1,\infty}$.   
For $x$, $y\in\BbbR^s$ set $\rho(x,y)=\|x-y\|^{s/r}$.   (i) Show that 
$\rho$ is a metric on $\BbbR^s$ inducing the Euclidean topology.   (ii) 
Let $\mu_{Hr}$ be the associated $r$-dimensional Hausdorff measure.   
Show that $\mu_{Hr}B(\tbf{0},1)=2^s$.
%264I out of order query
}%end of exercises

\endnotes{
\Notesheader{264} In this section we have come to the next step in
`geometric measure theory'.  I am taking this very slowly, because there
are real difficulties in the subject, and for the purposes of this
volume we do not need to master very much of it.   The idea here is to
find a definition of $r$-dimensional Lebesgue measure which will be
`geometric' in the strict sense, that is, dependent only on the metric
structure of $\BbbR^r$, and therefore applicable to sets which have a
metric structure but no linear structure.   As has happened before, the
definition of Hausdorff measure from an outer measure gives no
problems -- the only new idea in 264A-264C is that of using a supremum
$\theta_r=\sup_{\delta>0}\theta_{r\delta}$ of outer measures -- and the
difficult part is proving that our new measure has any useful
properties.   Concerning the properties of Hausdorff measure, there are
two essential objectives;  first, to check that these measures, in
general, share a reasonable proportion of the properties of Lebesgue
measure;  and second, to justify the term `$r$-dimensional measure' by
relating Hausdorff $r$-dimensional measure on $\BbbR^r$ to Lebesgue
measure on $\BbbR^r$.

As for the properties of general Hausdorff measures, we have to go
rather carefully.   I do not give counter-examples here because they
involve concepts which belong to Volumes 4 and 5 rather than this
volume, but I must warn you to expect the worst.   However, we do at
least have open sets measurable, so that all Borel sets are measurable
(264E).   The outer measure of a set $A$ can be defined in terms of the
Borel sets including $A$ (264Fa), though not in general in terms of the
open sets including $A$;  but the measure of a measurable set  $E$ is
not necessarily the supremum of the measures of the Borel sets included
in $E$, unless $E$ has finite measure (264Fc).   We do find that the
outer measure $\theta_r$ defined in 264A is the outer measure defined
from $\mu_{Hr}$ (264Fb), so that the phrase
`$r$-dimensional Hausdorff outer measure' is unambiguous.   A crucial
property of Lebesgue measure is the fact that the measure of a
measurable set $E$ is the supremum of the measures of the compact
subsets
of $E$;  this is not generally shared by Hausdorff measures, but is
valid for sets $E$ of finite measure in complete spaces (264Yi).
Concerning subspaces, there are no problems with the outer measures, and
for sets of finite measure the subspace measures are also consistent
(264Yf).   Because Hausdorff measure is defined in metric terms, it
behaves regularly for Lipschitz maps (264G);  one of the most
natural classes of functions to consider when studying metric spaces is
that of $1$-Lipschitz functions, so that (in the language of 264G)
$\mu_{Hr}^*\phi[A]\le \mu_{Hr}^*A$ for every $A$.

The second essential feature of Hausdorff measure, its relation with
Lebesgue measure in the appropriate dimension, is Theorem 264I.
Because both Hausdorff measure and Lebesgue measure are
translation-invariant, this can be proved by relatively elementary
means, except for the evaluation of the normalizing constant;  all we
need to know is that $\mu\coint{0,1}^r=1$ and
$\mu_{Hr}\coint{0,1}^r$ are both finite and non-zero, and this is
straightforward.   (The arguments of part (a) of the proof of 261F are
relevant.)   For the purposes of this chapter, we do not I think have to
know the value of the constant;  but I cannot leave it unsettled, and
therefore give Theorem 264H, the {\bf isodiametric inequality}, to show
that it is just the Lebesgue measure of an $r$-dimensional ball
of diameter $1$, as one would hope and expect.   The critical step in
the argument of 264H is in part (c) of the proof.   This is called
`Steiner symmetrization';  the idea is that given a set $A$, we
transform
$A$ through a series of steps, at each stage lowering, or at least not
increasing, its diameter, and raising, or at least not decreasing, its
outer measure, progressively making $A$ more symmetric, until at the end
we have a set which is sufficiently constrained to be amenable.   The
particular symmetrization operation used in this proof is important
enough;  but the idea of progressive regularization of an object is one
of the most powerful methods in measure theory, and you should give all
your attention to mastering any example you encounter.   In my
experience, the idea is principally useful when seeking an inequality
involving disparate quantities -- in the present example, the diameter
and volume of a set.

Of course it is awkward having two measures on $\BbbR^r$, differing by
a constant multiple, and for the purposes of the next section it would
actually have been a little more convenient to follow {\smc Federer 69}
in using `normalized Hausdorff measure' $2^{-r}\beta_r\mu_{Hr}$.
(For non-integral $r$, we could take
$\beta_r=\pi^{r/2}/\Gamma(1+\bover{r}2)$, as suggested in
252Xi.)   However, I believe this to be a minority position, and
the striking example of Hausdorff measure on the Cantor set (264J,
264Ym-264Yn) looks much better in the non-normalized version.

Hausdorff ($\ln 2/\ln 3$)-dimensional measure on the Cantor set is of
course but one, perhaps the easiest, of a large class of examples.
Because the Hausdorff
$r$-dimensional outer measure of a set $A$, regarded as a function of
$r$, behaves dramatically (falling from $\infty$ to $0$) at a certain
critical
value $\Delta(A)$ (see 264Xe, 264Yk), it gives us a metric space
invariant
of $A$;  $\Delta(A)$ is the {\bf Hausdorff dimension} of $A$.
Evidently the Hausdorff dimension of $C$ is $\ln 2/\ln 3$, while that of
$r$-dimensional Euclidean space is $r$.
}%end of notes

\discrpage


