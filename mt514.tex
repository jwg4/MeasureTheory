\frfilename{mt514.tex}
\versiondate{16.5.14}
\copyrightdate{2003}

\def\nw{\mathop{\text{nw}}}
\def\triplepc#1#2#3{(#1,#2,\hbox{$<$}#3)}
\def\congforc{\cong_{\text{forcing}}}
\def\chaptername{Cardinal functions}
\def\sectionname{Boolean algebras}

\newsection{514}

The cardinal functions of Boolean algebras and topological spaces are
intimately entwined;  necessarily so, because we have a functorial
connexion between Boolean algebras and zero-dimensional compact
Hausdorff spaces\cmmnt{ (312Q)}.   In this section I run through the
elementary ideas.   In 514D-514E I list properties of cardinal
functions of
Boolean algebras, corresponding to the relatively familiar results in
5A4B for topological spaces;
Stone spaces (514B), regular open algebras (514H) and category algebras
(514I) provide links of
different kinds between the two theories.   It turns out that some of
the most important features of the cofinal structure of a
pre-ordered set can also be described in terms of its
`up-topology' (514L-514M) and the associated regular open algebra
(514N-514S).
I conclude with a brief note on finite-support products (514T-514U).

\leader{514A}{}\cmmnt{ I put a special property of locally compact
spaces into the language of this chapter.

\medskip

\noindent}{\bf Lemma} Let $(X,\frak T)$ be a topological space.
Then $d(X)\ge\ddownarrow(\frak T\setminus\{\emptyset\})$.
If $X$ is locally compact and Hausdorff, then
$d(X)=\ddownarrow(\frak T\setminus\{\emptyset\})$.

\proof{ If $x\in X$, then $\{G:x\in G\in\frak T\}$ is downwards-centered in
$\frak T\setminus\{\emptyset\}$.   So

\Centerline{$\ddownarrow(\frak T\setminus\{\emptyset\})
\le\cov(\frak T\setminus\{\emptyset\},\ni,X)=d(X)$.}

Now suppose that $X$ is locally compact and Hausdorff.   Set
$\kappa=\ddownarrow(\frak T\setminus\{\emptyset\})$,
and let $\ofamily{\xi}{\kappa}{\Cal H_{\xi}}$ be a cover of
$\frak T\setminus\{\emptyset\}$ by downwards-centered sets.   For
$\xi<\kappa$ set
$F_{\xi}=\bigcap\{\overline{H}:H\in\Cal H_{\xi}\}$, and let
$D\subseteq X$ be a set with cardinal at most $\kappa$ such that
$D\cap F_{\xi}\ne\emptyset$ whenever $\xi<\kappa$ and
$F_{\xi}\ne\emptyset$.    If $G\subseteq X$ is a non-empty open set,
then there is a non-empty relatively compact open set $H_0$ such that
$\overline{H}_0\subseteq G$ (recall that $X$, being locally compact and
Hausdorff, is certainly regular).   There is some $\xi<\kappa$ such that
$H_0\in\Cal H_{\xi}$;  because $\{\overline{H}:H\in\Cal H_{\xi}\}$ is a
family of closed sets with the finite intersection property containing
the compact set $\overline{H}_0$, its intersection $F_{\xi}$ is not
empty.   Also $F_{\xi}\subseteq\overline{H}_0\subseteq G$, so $D\cap
G\supseteq D\cap F_{\xi}$ is non-empty.   As $G$ is arbitrary, $D$ is
dense, and $d(X)\le\#(D)\le\kappa$.   We know already that
$\kappa\le d(X)$, so they are equal.
}%end of proof of 514A

\leader{514B}{Stone \dvrocolon{spaces}}\cmmnt{ Necessarily, any
cardinal function $\zeta$ of topological spaces corresponds to a
cardinal function $\tilde{\zeta}$ of Boolean algebras, taking
$\tilde{\zeta}(\frak A)=\zeta(Z)$ where $Z$ is the Stone space of
$\frak A$.   Working through the functions described in 5A4A and
511D, we have the following results.

\medskip

\noindent}{\bf Theorem}
Let $\frak A$ be any Boolean algebra and $Z$ its Stone space.   For
$a\in\frak A$ let $\widehat{a}$ be the corresponding open-and-closed
subset of $Z$.

(a) $\#(\frak A)$ is $2^{w(Z)}=2^{\#(Z)}$ if $\frak A$ is finite, $w(Z)$
otherwise.

(b) $\sat(\frak A)=\sat(Z)$, $c(\frak A)=c(Z)$.

(c) $\pi(\frak A)=\pi(Z)$.

(d) $d(\frak A)=d(Z)$.

(e) Let $\CalNwd(Z)$ be the ideal of nowhere dense subsets of $Z$.   Then
$\wdistr(\frak A)=\add\CalNwd(Z)$.

\proof{ Let $\Cal E$ be the algebra of open-and-closed subsets of $Z$,
so that $a\mapsto\widehat{a}$ is an isomorphism from $\frak A$ to
$\Cal E$.   The essential fact here is that
$\Cal E\setminus\{\emptyset\}$ is coinitial with
$\frak T\setminus\{\emptyset\}$, where $\frak T$ is the topology of $Z$,
so that (writing $\frak A^+$ for $\frak A\setminus\{0\}$, as usual)

\Centerline{$(\frak A^+,\Bsupseteqshort,\frak A^+)
\cong(\Cal E\setminus\{\emptyset\},\supseteq,
  \Cal E\setminus\{\emptyset\})
\equivGT(\frak T\setminus\{\emptyset\},\supseteq,
  \frak T\setminus\{\emptyset\})$}

\noindent by 513E(d-ii), inverted.

\medskip

{\bf (a)} If $\frak A$ is finite, so is $Z$, and in
this case $\frak A\cong\Cal PZ$ has cardinal $2^{\#(Z)}$.   If
$\frak A$ is infinite, so are $Z$ and $w(Z)$.   Because
$\Cal E$ is a base for the topology of $Z$,
$w(Z)\le\#(\Cal E)=\#(\frak A)$.   On the other hand, let $\Cal U$ be a
base for the topology of $Z$ with $\#(\Cal U)=w(Z)$.   Then every member
of $\Cal E$ is expressible as the union of a finite subset of $\Cal U$,
so

\Centerline{$\#(\frak A)=\#(\Cal E)
\le\#([\Cal U]^{<\omega})\le\max(\omega,\#(\Cal U))=w(Z)$.}

\medskip

{\bf (b)-(c)}

\Centerline{$c(\frak A)=c(\Cal E)
=\cdownarrow(\Cal E\setminus\{\emptyset\})
=\cdownarrow(\frak T\setminus\{\emptyset\})
=c(Z)$,}

\Centerline{$\sat(\frak A)=\sat(\Cal E)
=\sat^{\downarrow}(\Cal E\setminus\{\emptyset\})
=\sat^{\downarrow}(\frak T\setminus\{\emptyset\})
=\sat(Z)$,}

\Centerline{$\pi(\frak A)=\pi(\Cal E)
=\ci(\Cal E\setminus\{\emptyset\})
=\ci(\frak T\setminus\{\emptyset\})
=\pi(Z)$}

\noindent using 513Gb, inverted, to move between
$\Cal E\setminus\{\emptyset\}$ and $\frak T\setminus\{\emptyset\}$.

\medskip

{\bf (d)}

$$\eqalignno{d(\frak A)
&=\ddownarrow(\frak A^+)
=\ddownarrow(\Cal E\setminus\{\emptyset\})
=\ddownarrow(\frak T\setminus\{\emptyset\})\cr
\displaycause{513Gd, inverted}
&=d(Z)\cr}$$

\noindent because $Z$ is compact and Hausdorff (514A).

\medskip

{\bf (e)} Let $(\Pou(\frak A),\sqsubseteq^*)$
be the pre-ordered set of partitions of unity in $\frak A$ as described in
512Ee.   For $C\in\Pou(\frak A)$, set

\Centerline{$f(C)=Z\setminus\bigcup_{c\in C}\widehat{c}$.}

\noindent Then $f(C)\in\CalNwd(Z)$.   \Prf\Quer\ Otherwise, since $f(C)$ is
certainly closed, its interior is non-empty, and there is a non-zero
$a\in\frak A$ such that $\widehat{a}\subseteq f(C)$;  but in this case
$a\Bcap c=0$ for every $c\in C$ and $C$ is not a partition of unity.\
\Bang\Qed

If $C$, $D\in\Pou(\frak A)$ and $C\sqsubseteq^*D$ then
$f(C)\subseteq f(D)$.
\Prf\ If $d\in D$, $C_0=\{c:c\in C$, $c\Bcap d\ne 0\}$ is finite and
$d\Bsubseteq\sup C_0$;  so
$\widehat{d}\subseteq\bigcup_{c\in C_0}\widehat{c}$ is disjoint from
$f(C)$.   Thus $Z\setminus f(D)\subseteq Z\setminus f(C)$ and
$f(C)\subseteq f(D)$.\ \Qed

If $C$, $D\in\Pou(\frak A)$ and $f(C)\subseteq f(D)$ then
$C\sqsubseteq^*D$.   \Prf\ If $d\in D$ then the compact set $\widehat{d}$
is included in the open set $\bigcup_{c\in C}\widehat{c}$.   So there is a
finite set $C_0\subseteq C$ such that
$\widehat{d}\subseteq\bigcup_{c\in C_0}\widehat{c}$ and
$\{c:c\in C$, $d\Bcap c\ne 0\}\subseteq C_0$ is finite.\ \Qed

$f[\Pou(\frak A)]$ is cofinal with $\CalNwd(Z)$.
\Prf\ If $F\in\CalNwd(Z)$, let
$C\subseteq\frak A$ be a maximal disjoint set such that
$F\cap\widehat{c}=\emptyset$ for every $c\in C$.   \Quer\ If $C$ is not a
partition of unity in $\frak A$, let $a\in\frak A^+$ be such that
$a\Bcap c=0$ for every $c\in C$.   Then $\widehat{a}\setminus F$ is a
non-empty open set, so there is a non-zero $b\in\frak A$ such that
$\widehat{b}\subseteq\widehat{a}\setminus F$;  in which case we ought to
have added $b$ to $C$.\ \BanG\  So $C\in\Pou(\frak A)$ and
$F\subseteq f(C)$.\ \Qed

By 513E(d-i), $\Pou(\frak A)$ and $\CalNwd(Z)$ are Tukey equivalent, and

\Centerline{$\add\CalNwd(Z)=\add\Pou(\frak A)=\wdistr(\frak A)$}

\noindent as remarked in 512Ee.
}%end of proof of 514B

\leader{514C}{}\cmmnt{ I begin the detailed study of cardinal
functions of Boolean algebras with two elementary remarks.

\medskip

\noindent}{\bf Lemma} Let $\frak A$ be a Boolean algebra.

(a) $d(\frak A)$ is the smallest cardinal $\kappa$ such that $\frak A$
is isomorphic, as Boolean algebra, to a subalgebra of $\Cal P\kappa$.

(b) $\link(\frak A)$ is the smallest cardinal $\kappa$ such that
$\frak A$ is isomorphic, as partially ordered set, to a subset of
$\Cal P\kappa$.

\proof{{\bf (a)(i)} If we have an isomorphism $\pi$ from $\frak A$ to a
subalgebra of $\Cal P\kappa$, then $A_{\xi}=\{a:\xi\in\pi a\}$ is
centered for each $\xi<\kappa$, and
$\bigcup_{\xi<\kappa}A_{\xi}=\frak A^+$;  so
$d(\frak A)\le\kappa$.

\medskip

\quad{\bf (ii)} Let $Z$ be the Stone space of $\frak A$, and for
$a\in\frak A$ let $\widehat{a}\subseteq Z$ be the corresponding
open-and-closed set.   There is a dense set $D\subseteq Z$ of size
$d(\frak A)$ (514Bd), and $a\mapsto D\cap\widehat{a}:\frak A\to\Cal PD$
is an injective Boolean homomorphism;  so $\frak A$ is isomorphic to a
subalgebra of $\Cal PD\cong\Cal P(d(\frak A))$.

\medskip

{\bf (b)(i)} $\kappa\le\link(\frak A)$.   \Prf\ Let
$\ofamily{\xi}{\link(\frak A)}{A_{\xi}}$ be a family of linked subsets
of $\frak A^+$ covering $\frak A^+$.   Set
$A'_{\xi}=\{b:\,\Exists a\in A_{\xi},\,b\Bsupseteq a\}$;  then each
$A'_{\xi}$ is still linked in $\frak A$.   Define
$h:\frak A\to\Cal P\kappa$ by setting $h(a)=\{\xi:a\in A'_{\xi}\}$.
Then $h$ is order-preserving.   Now if $a$, $b\in\frak A$ and
$a\notBsubseteq b$, there is a $\xi<\kappa$ such that
$a\Bsetminus b\in A_{\xi}$, in which case $\xi\in h(a)\setminus h(b)$.
Thus $h$ is an embedding and $\kappa\le\link(\frak A)$.\ \Qed

\medskip

\quad{\bf (ii)} $\link(\frak A)\le\kappa$.   \Prf\ Let
$h:\frak A\to\Cal P\kappa$ be an order-isomorphism between $\frak A$ and
a subset of $\Cal P\kappa$.   For each $\xi$, set

\Centerline{$A_{\xi}
=\{a:a\in\frak A,\,\xi\in h(a)\setminus h(1\Bsetminus a)\}$.}

\noindent If $a$, $b\in A_{\xi}$ then
$\xi\in h(b)\setminus h(1\Bsetminus a)$ so
$b\notBsubseteq 1\Bsetminus a$ and $a\Bcap b\ne 0$;  thus $A_{\xi}$ is
linked.   If $a\in\frak A^+$ then $a\notBsubseteq 1\Bsetminus a$ so
$h(a)\not\subseteq h(1\Bsetminus a)$ and there is a $\xi<\kappa$ such
that $\xi\in h(a)\setminus h(1\Bsetminus a)$;  thus
$\frak A^+=\bigcup_{\xi<\kappa}A_{\xi}$ and
$\link(\frak A)\le\kappa$.\ \Qed
}%end of proof of 514C

\leader{514D}{Theorem} Let $\frak A$ be a Boolean algebra.

(a)

\Centerline{$c(\frak A)\le\link(\frak A)\le d(\frak A)
\le\pi(\frak A)\le\#(\frak A)\le 2^{\link(\frak A)}$,
\quad$\tau(\frak A)\le\pi(\frak A)$,}

\noindent $\sat(\frak A)=c(\frak A)^+$ unless $\sat(\frak A)$ is weakly
inaccessible, in which case $\sat(\frak A)=c(\frak A)$.

(b) If $A\subseteq\frak A$, there is a $B\in[A]^{<\sat(\frak A)}$ with
the same upper bounds as $A$;  \cmmnt{similarly,} there is a
$B\in[A]^{<\sat(\frak A)}$ with the same lower bounds as $A$.

(c) $\link_{c(\frak A)}(\frak A)=\link_{<\sat(\frak A)}(\frak A)
=\pi(\frak A)$.

(d) If $\frak A$ is not purely atomic,
$\wdistr(\frak A)\le\min(d(\frak A),2^{\tau(\frak A)})$ is a regular
infinite cardinal.

(e) $\#(\frak A)
\le\max(4,\sup_{\lambda<\sat(\frak A)}\tau(\frak A)^{\lambda})$\cmmnt{,
where $\tau(\frak A)^{\lambda}$ is the cardinal power}.

\proof{ Let $Z$ be the Stone space of $\frak A$;  for $a\in\frak A$, let
$\widehat{a}\subseteq Z$ be the corresponding open-and-closed set.

\medskip

{\bf (a)} This is mostly a repetition of 511Ia.
By 514Cb, $\#(\frak A)\le 2^{\link(\frak A)}$.
By 513Bc, inverted, and the definitions in 511Db,

\Centerline{$\sat(\frak A)=\sat^{\downarrow}(\frak A^+)
=c^{\downarrow}(\frak A^+)^+=c(\frak A)^+$}

\noindent unless $\sat(\frak A)$ is a regular uncountable limit cardinal,
that is, is weakly inaccessible, and otherwise $\sat(\frak A)=c(\frak A)$.
(See also 5A4Ba.)

\medskip

{\bf (b)} By 5A4Bd, applied to $\{\widehat{a}:a\in A\}$, there is a
$B\in[A]^{<\sat(\frak A)}$ such that
$\overline{\bigcup_{b\in B}\widehat{b}}
=\overline{\bigcup_{a\in A}\widehat{a}}$.   Now if $c$ is an upper bound
of $B$, then $\widehat{c}$ is a closed set including $\widehat{b}$ for
every $b\in B$, so it also includes $\widehat{a}$ for every $a\in A$,
and $c$ is an upper bound of $A$.

Applying this to $\{1\Bsetminus a:a\in A\}$ we see that there is a set
$B'\in[A]^{<\sat(\frak A)}$ with the same lower bounds as $A$.

\medskip

{\bf (c)} Set $\kappa=\link_{<\sat(\frak A)}(\frak A)$.   By 511Ia,
$\kappa\le\link_{c(\frak A)}(\frak A)\le\pi(\frak A)$.
On the other hand, if $A\subseteq\frak A^+$ is
$\hbox{$<$}\sat(\frak A)$-linked,
it has a lower bound in $\frak A^+$.   \Prf\ By (b), there is a set
$B\subseteq A$, with the same lower bounds as $A$, such that
$\#(B)<\sat(\frak A)$.   Now $B$ has a
non-zero lower bound because $A$ is $\hbox{$<$}\sat(\frak A)$-linked, so
$A$ also has a non-zero lower bound.\ \QeD\
We have a cover $\ofamily{\xi}{\kappa}{A_{\xi}}$ of $\frak A^+$ by
$\hbox{$<$}\sat(\frak A)$-linked sets;  each $A_{\xi}$ has a non-zero
lower bound $a_{\xi}$ say;  and $\{a_{\xi}:\xi<\kappa\}$ is a $\pi$-base
for
$\frak A$, so $\pi(\frak A)\le\kappa$.

\medskip

{\bf (d)(i)} Because $\wdistr(\frak A)=\add\CalNwd(Z)$, where
$\CalNwd(Z)$ is
the ideal of nowhere dense subsets of $Z$ (514Be), and is not $\infty$
(511Ie), it must be a regular infinite cardinal (513C(a-i)).   (Or argue
directly from 511Df.)

\medskip

\quad{\bf (ii)} As for the upper bound for $\wdistr(\frak A)$, suppose
that $a\in\frak A^+$ includes no atom and that
$D\in[\frak A]^{\tau(\frak A)}\,\,\tau$-generates $\frak A$.   Since
$\frak A$ and $\tau(\frak A)$ are surely infinite, the subalgebra
$\frak B$ of $\frak A$ generated by $D\cup\{a\}$ is still of size
$\tau(\frak A)$ (331Gc).   For $B\subseteq\frak B$ set
$E_B=Z\cap\bigcap_{b\in B}\widehat{b}$, and set
$\Cal C=\{B:B\subseteq\frak B,\,E_B$ is nowhere dense$\}$.   Then
$\bigcup_{B\in\Cal C}E_B\supseteq\widehat{a}$.   \Prf\ Take any
$z\in\widehat{a}$.   Set $B=\{b:b\in\frak B,\,z\in\widehat{b}\}$.
\Quer\ If $E_B$ has non-empty interior, it includes $\widehat{c}$ for
some non-zero $c\Bsubseteq a$.   But now, for any $d\in D$, either
$d\in B$ and $c\Bsubseteq d$, or $1\Bsetminus d\in B$ and $c\Bcap d=0$.
So the order-closed subalgebra
$\{d:$ either $c\Bsubseteq d$ or $c\Bcap d=0\}$ includes $D$ and must be
the whole of $\frak A$, and $c\Bsubseteq a$ is an atom.\ \BanG\  So
$\interior E_B=\emptyset$, $B\in\Cal C$ and $z\in E_B$.   As $z$ is
arbitrary, $\widehat{a}\subseteq\bigcup_{B\in\Cal C}E_B$.\ \Qed

By 514Be, with 514Bd,

\Centerline{$\wdistr(\frak A)\le\#(\Cal C)\le 2^{\#(\frak B)}
=2^{\tau(\frak A)}$.}

At the same time, if $Y\subseteq Z$ is any dense set of size $d(Z)$,
then $\{\{y\}:y\in Y\cap\widehat{a}\}$ is a family of nowhere dense sets
with no upper bound in the ideal of nowhere dense subsets of $Z$;  so
514Be also tells us that

\Centerline{$\wdistr(\frak A)\le\#(Y\cap\widehat{a})\le d(Z)
=d(\frak A)$.}

\medskip

{\bf (e)} (Compare 4A1O.)   Set
$\kappa=\sup_{\lambda<\sat(\frak A)}\tau(\frak A)^{\lambda}$.   If
$\#(\frak A)>4$ then $\tau(\frak A)\ge 2$ so
$\kappa\ge\sup_{\lambda<\sat(\frak A)}2^{\lambda}$, and the
result is immediate from 511Ic if $\frak A$ is finite.   If $\frak A$ is
infinite, so is $\sat(\frak A)$, while $\lambda<\kappa$ for every
$\lambda<\sat(\frak A)$, so $\sat(\frak A)\le\kappa$.   Let
$D\subseteq\frak A$ be a set with cardinal $\tau(\frak A)$ which
$\tau$-generates $\frak A$.   Define
$\ofamily{\xi}{\kappa}{D_{\xi}}$ inductively by setting

\Centerline{$D_0=D$,
\quad$D_{\xi}=\{1\Bsetminus a:a\in\frak A,\,a=\sup C$ for some
$C\subseteq\bigcup_{\eta<\xi}D_{\eta}\}$}

\noindent for $\xi<\kappa$.   Then
$\#(D_{\xi})\le\kappa$ for every $\xi<\kappa$.   \Prf\ The
point is that, by (b), every member of $D_{\xi}$ is expressible in the
form $1\Bsetminus\sup C$ for some
$C\in[\bigcup_{\eta<\xi}D_{\eta}]^{<\sat(\frak A)}$.   But the inductive
hypothesis tells us that $\bigcup_{\eta<\xi}D_{\eta}$ has cardinal at
most $\kappa$, so the number of its subsets of
cardinal less than $\sat(\frak A)$ is also $\kappa$ (5A1Ef, because
$\sat(\frak A)$ is regular), and
$\#(D_{\xi})\le\kappa$.\ \Qed

At the end of the induction, set
$\frak B=\bigcup_{\xi<\sat(\frak A)}D_{\xi}$.
Then $1\Bsetminus(b\Bcup b')\in\frak B$ for every $b$, $b'\in\frak B$,
so $\frak B$ is a subalgebra of $\frak A$.   Also it is order-closed.
\Prf\ If $B\subseteq\frak B$ has a supremum $a\in\frak A$, there is a
$C\subseteq B$ such that $\#(C)<\sat(\frak A)$ and $a=\sup C$.   Now
there must be some set $J\subseteq\sat(\frak A)$ such that
$\#(J)<\sat(\frak A)$ and $C\subseteq\bigcup_{\eta\in J}D_{\eta}$.
Since $\sat(\frak A)$ is
regular (513Bb), $\zeta=\sup J$ is less than $\sat(\frak A)$.
Now $1\Bsetminus a\in D_{\zeta+1}$ and $a\in\frak B$.\ \Qed

By the choice of $D$, $\frak B=\frak A$, so
$\#(\frak A)=\#(\frak B)\le\kappa$.
}%end of proof of 514D

\vleader{108pt}{514E}{Subalgebras, homomorphic images, products:
Theorem} Let $\frak A$ be a Boolean algebra.

(a) If $\frak B$ is a subalgebra of $\frak A$, then

\Centerline{$\sat(\frak B)\le\sat(\frak A)$,
\quad$c(\frak B)\le c(\frak A)$,}

\Centerline{$\link_{<\kappa}(\frak B)\le\link_{<\kappa}(\frak A)$}

\noindent for every $\kappa\le\omega$\cmmnt{, in particular},

\Centerline{$d(\frak B)\le d(\frak A)$,
\quad$\link(\frak B)\le\link(\frak A)$.}

(b) If $\frak B$ is a regularly embedded subalgebra of $\frak A$
then\cmmnt{, in addition,}
$\link_{<\kappa}(\frak B)\le\link_{<\kappa}(\frak A)$ for
$\kappa>\omega$, $\pi(\frak B)\le\pi(\frak A)$ and
$\wdistr(\frak A)\le\wdistr(\frak B)$.

(c) If $\frak B$ is a Boolean algebra and $\phi:\frak A\to\frak B$ is a
surjective order-continuous Boolean homomorphism, then

\Centerline{$\sat(\frak B)\le\sat(\frak A)$,
\quad$c(\frak B)\le c(\frak A)$,
\quad$\pi(\frak B)\le\pi(\frak A)$,}

\Centerline{$\link_{<\kappa}(\frak B)\le\link_{<\kappa}(\frak A)$
for every cardinal $\kappa$,}

\Centerline{$d(\frak B)\le d(\frak A)$,
\quad$\link(\frak B)\le\link(\frak A)$,}

\cmmnt{\noindent and also}

\Centerline{$\wdistr(\frak A)\le\wdistr(\frak B)$,
\quad$\tau(\frak B)\le\tau(\frak A)$.}

(d) If $\frak B$ is a principal ideal of $\frak A$, then

\Centerline{$\sat(\frak B)\le\sat(\frak A)$,
\quad$c(\frak B)\le c(\frak A)$,
\quad$\pi(\frak B)\le\pi(\frak A)$,}

\Centerline{$\link_{<\kappa}(\frak B)\le\link_{<\kappa}(\frak A)$
for every $\kappa$,}

\Centerline{$d(\frak B)\le d(\frak A)$,
\quad$\link(\frak B)\le\link(\frak A)$;}

\cmmnt{\noindent moreover,}

\Centerline{\quad$\wdistr(\frak A)\le\wdistr(\frak B)$,
\quad$\tau(\frak B)\le\tau(\frak A)$.}

(e) If $\frak B$ is an order-dense subalgebra of $\frak A$ then

\Centerline{$\sat(\frak B)=\sat(\frak A)$,
\quad$c(\frak B)=c(\frak A)$,
\quad$\pi(\frak B)=\pi(\frak A)$,}

\Centerline{$\link_{<\kappa}(\frak B)=\link_{<\kappa}(\frak A)$
for every $\kappa$,}

\Centerline{$d(\frak B)=d(\frak A)$,
\quad$\link(\frak B)=\link(\frak A)$,}

\cmmnt{\noindent and finally}

\Centerline{$\wdistr(\frak B)=\wdistr(\frak A)$,
\quad$\tau(\frak A)\le\tau(\frak B)$.}
%query:  is $\tau(\frak B)\le\max(c(\frak A),\tau(\frak A))$?

(f) If $\frak A$ is the simple product of a family
$\familyiI{\frak A_i}$ of Boolean algebras, then

\Centerline{$\tau(\frak A)
\le\max(\omega,\sup_{i\in I}\tau(\frak A_i),
   \min\{\lambda:\#(I)\le 2^{\lambda}\})$,}

\Centerline{$\sat(\frak A)
\le\max(\omega,\#(I)^+,\sup_{i\in I}\sat(\frak A_i))$,}

\Centerline{$c(\frak A)
\le\max(\omega,\#(I),\sup_{i\in I}c(\frak A_i))$,}

\Centerline{$\pi(\frak A)
\le\max(\omega,\#(I),\sup_{i\in I}\pi(\frak A_i))$,}

\Centerline{$\link_{<\kappa}(\frak A)
\le\max(\omega,\#(I),\sup_{i\in I}\link_{<\kappa}(\frak A_i))$
for every $\kappa$,}

\Centerline{$\link(\frak A)
\le\max(\omega,\#(I),\sup_{i\in I}\link(\frak A_i))$,}

\Centerline{$d(\frak A)
\le\max(\omega,\#(I),\sup_{i\in I}d(\frak A_i))$,}

\cmmnt{\noindent and}

\Centerline{$\wdistr(\frak A)=\min_{i\in I}\wdistr(\frak A_i)$.}

\proof{ Write $Z$ for the Stone space of $\frak A$.

\medskip

{\bf (a)} Any disjoint subset of $\frak B^+$ is a disjoint subset of
$\frak A^+$, so $\sat(\frak B)\le\sat(\frak A)$ and
$c(\frak B)\le c(\frak A)$.   If $\kappa\le\omega$ and $\Cal A$ is
a cover of $\frak A^+$ by sets which are downwards
$\hbox{$<$}\kappa$-linked in $\frak A^+$, then $A\cap\frak B$ is
downwards
$\hbox{$<$}\kappa$-linked in $\frak B^+$ for each $A\in\Cal A$, so
$\link_{<\kappa}(\frak B)\le\link_{<\kappa}(\frak A)$.

\medskip

{\bf (b)} For each non-zero $a\in\frak A$, the set
$B_a=\{b:b\in\frak B,\,a\Bsubseteq b\}$ does not have infimum $0$ in
$\frak A$ so cannot have infimum $0$ in $\frak B$;  let
$\psi(a)\in\frak B^+$ be a lower bound for $B_a$.   If now we set
$\phi(b)=b$ for $b\in\frak B$, $(\phi,\psi)$ is a
Galois-Tukey connection from
$(\frak B^+,\Bsupseteqshort,\frak B^+)$ to
$(\frak A^+,\Bsupseteqshort,\frak A^+)$.   It follows at once that

\Centerline{$\link_{<\kappa}(\frak B)
=\link_{<\kappa}(\frak B^+,\Bsupseteqshort,\frak B^+)
\le\link_{<\kappa}(\frak A^+,\Bsupseteqshort,\frak A^+)
=\link_{<\kappa}(\frak A)$}

\noindent for arbitrary $\kappa$ (512Dd), and that

\Centerline{$\pi(\frak B)
=\cov(\frak B^+,\Bsupseteqshort,\frak B^+)
\le\cov(\frak A^+,\Bsupseteqshort,\frak A^+)
=\pi(\frak A)$}

\noindent (512Da).

Now suppose that $\kappa<\wdistr(\frak A)$ and that
$\ofamily{\xi}{\kappa}{B_{\xi}}$ is a family of partitions of unity in
$\frak B$.   Then

\Centerline{$D=\{d:d\in\frak B$, $\{b:b\in B_{\xi}$, $b\Bcap d\ne 0\}$
is finite for every $\xi<\kappa\}$}

\noindent is order-dense in $\frak B$.   \Prf\ Take any non-zero
$d\in\frak B$.   $\sup B_{\xi}=1$ in $\frak A$, that is, $B_{\xi}$ is
still a partition of unity in $\frak A$, for each $\xi$.   So there
is a partition $A$ of unity in $\frak A$ such that
$\{b:b\in B_{\xi}$, $b\Bcap a\ne 0\}$ is finite for every $\xi<\kappa$
and $a\in A$.   Let $a\in A$ be such that $d\Bcap a\ne 0$, and set
$e_{\xi}=\sup\{b:b\in B_{\xi}$, $b\Bcap a\ne 0\}$ for each
$\xi<\kappa$.   Then $a\Bsubseteq e_{\xi}\in\frak B$ for each $\xi$.
This means that $\{d\}\cup\{e_{\xi}:\xi<\kappa\}$ has a
non-zero lower bound $d\Bcap a$ in $\frak A$;  as $\frak B$ is regularly
embedded in $\frak A$, there is a non-zero $d'\Bsubseteq d$ which is
also a lower bound for $\{e_{\xi}:\xi<\kappa\}$.   But this means that
$d'\in D$.   As $d$ is arbitrary, $D$ is order-dense in $\frak B$.\ \Qed

There is therefore a partition of unity included in $D$.   As
$\ofamily{\xi}{\kappa}{B_{\xi}}$ is arbitrary,
$\wdistr(\frak B)\ge\wdistr(\frak A)$.

\medskip

{\bf (c)(i)} For any $b\in\frak B^+$ there is a $\psi(b)\in\frak A^+$ such
that $\phi\psi(b)\Bsubseteq b$ and $a=0$ whenever
$a\Bsubseteq\psi(b)$ and $\phi a=0$.   \Prf\ Consider
$D=\{d:d\in\frak A$, $\phi d\supseteq b\}$.   This is a
non-empty downwards-directed subset of $\frak A$ and $b$ is a non-zero
lower bound of $\phi[D]$.   As $\phi$ is supposed to be order-continuous,
$D$ must have a non-zero lower bound in $\frak A$;  let $\psi(b)$ be such a
lower bound.   Since there is a $d\in\frak A$ such that $\phi d=b$, and now
$d\in D$, we must have $\phi\psi(b)\Bsubseteq\phi d=b$.   If
$a\Bsubseteq\psi(b)$ and $\phi a=0$, then
$\phi(1\Bsubseteq a)=1\Bsupseteq b$, $1\Bsetminus a\in D$ and
$a\Bsubseteq\psi(b)\Bsubseteq 1\Bsetminus a$, so $a=0$.\ \Qed

\medskip

\quad{\bf (ii)} If $\kappa=\sat(\frak A)$ and
$\ofamily{\xi}{\kappa}{b_{\xi}}$ is a family in $\frak B^+$, then
$\ofamily{\xi}{\kappa}{\psi(b_{\xi})}$ is a family in $\frak A^+$ and there
are distinct $\xi$, $\eta<\kappa$ such that
$a=\psi(b_{\xi})\Bcap\psi(b_{\eta})$ is non-zero.   Now

\Centerline{$0\ne\phi a
\Bsubseteq\phi\psi(b_{xi})\Bcap\phi\psi(b_{\eta})
\Bsubseteq b_{\xi}\Bcap b_{\eta}$.}

\noindent As $\ofamily{\xi}{\kappa}{b_{\xi}}$ is arbitrary,
$\sat(\frak B)\le\sat(\frak A)$.   By a similar argument, or using 514Da,
we see that $c(\frak B)\le c(\frak A)$.

\medskip

\quad{\bf (iii)} Let $A$ be a coinitial subset of $\frak A^+$ of
cardinal $\pi(\frak A)$.   Set $B=\phi[A]\setminus\{0\}$.   If
$b\in\frak B^+$, there is an $a\in A$ such that $a\Bsubseteq\psi(b)$, and
now $\phi a\in B$ and $\phi a\Bsubseteq\phi\psi(b)\Bsubseteq b$.   So $B$
is cofinal with $\frak B^+$ and

\Centerline{$\pi(\frak B)\le\#(B)\le\#(A)=\pi(\frak A)$.}

\medskip

\quad{\bf (iv)}
Let $W$ be the Stone space of $\frak B$.   Write $\CalNwd(Z)$,
$\CalNwd(W)$ for
the ideals of nowhere dense subsets of $Z$ and $W$, so that
$\add\CalNwd(Z)=\wdistr(\frak A)$ and $\add\CalNwd(W)=\wdistr(\frak B)$
(514Be).   Corresponding to
$\phi:\frak A\to\frak B$ we have an injective continuous function
$\theta:W\to Z$ such that $\theta^{-1}[E]\in\CalNwd(W)$ for every
$E\in\CalNwd(Z)$ (312Sb, 313R).   Also $\theta[F]\in\CalNwd(Z)$ for every
$F\in\CalNwd(W)$.   \Prf\Quer\ Otherwise, because $\theta[\overline{F}]$ is
compact, therefore closed, there is a non-empty open set
$G\subseteq\theta[\overline{F}]$.   Now $\theta^{-1}[G]$ is a non-empty
open set, and is included in $\overline{F}$, because $\theta$ is
injective;  but this is impossible.\ \Bang\QeD\  So if
$\Cal J_0\subseteq\CalNwd(W)$ and $\#(\Cal J_0)<\wdistr(\frak A)$,
$E=\bigcup\{\theta[F]:F\in\Cal J_0\}$ belongs to $\CalNwd(Z)$ and
$\bigcup\Cal J_0\subseteq\theta^{-1}[E]$ belongs to $\CalNwd(W)$.   This
shows that $\add\CalNwd(W)\ge\add\CalNwd(Z)$, so that
$\wdistr(\frak B)\ge\wdistr(\frak A)$.

\medskip

\quad{\bf (v)} As for $\tau(\frak B)$, we have only to recall that if
$D\subseteq\frak A$ is a $\tau$-generating set of size $\tau(\frak A)$,
the order-closed subalgebra of $\frak B$ generated by $\phi[D]$ includes
$\phi[\frak A]=\frak B$ (313Mb), and

\Centerline{$\tau(\frak B)\le\#(\phi[D])\le\tau(\frak A)$.}

\medskip

{\bf (d)} If $\frak B$ is the principal ideal generated by $b$, then
$a\mapsto a\Bcap b:\frak A\to\frak B$ is an order-continuous surjection,
so we can repeat the list in (c).

\medskip

{\bf (e)(i)} Because $\frak B^+$ is coinitial with
$\frak A^+$ we can use 513Gc, inverted, to see that

\Centerline{$\sat(\frak B)=\sat^{\downarrow}(\frak B^+)
=\sat^{\downarrow}(\frak A^+)=\sat(\frak A)$,
\quad$c(\frak B)=c(\frak A)$,}

\Centerline{$\pi(\frak B)=\ci(\frak B^+)
=\ci(\frak A^+)=\pi(\frak A)$,}

\Centerline{$\link_{<\kappa}(\frak B)
=\link_{<\kappa}^{\downarrow}(\frak B^+)
=\link_{<\kappa}^{\downarrow}(\frak A^+)
=\link_{<\kappa}(\frak A)$.}

\medskip

\quad{\bf (ii)} From (b) we know that
$\wdistr(\frak B)\ge\wdistr(\frak A)$.   For the reverse inequality,
suppose that $\kappa<\wdistr(\frak B)$ and that
$\ofamily{\xi}{\kappa}{A_{\xi}}$ is any family of partitions of unity in
$\frak A$.    For each $\xi<\kappa$ set
$B_{\xi}=\{b:b\in\frak B,\,\Exists a\in A_{\xi},\,b\Bsubseteq a\}$.
Then $B_{\xi}$ is order-dense in $\frak B$ and includes a partition of
unity $B'_{\xi}$ (313K).
Now there is a partition $C$ of unity in $\frak B$
such that $D'_{\xi c}=\{b:b\in B'_{\xi}$, $b\Bcap c\ne 0\}$ is finite
for any $\xi<\kappa$ and $c\in C$.   $C$ is still a partition of unity
in $\frak A$, and $D_{\xi c}=\{a:a\in A_{\xi}$, $a\Bcap c\ne 0\}$ is
finite for every $c\in C$ and $\xi<\kappa$.   (For if $a$, $a'$ are
distinct elements of $D_{\xi c}$, then $\{b:b\in D'_{\xi c}$,
$b\Bsubseteq a\}$ and $\{b:b\in D'_{\xi c}$, $b\Bsubseteq a'\}$ are
disjoint and not empty.)   As
$\ofamily{\xi}{\kappa}{A_{\xi}}$ is arbitrary,
$\wdistr(\frak B)\le\wdistr(\frak A)$.

\medskip

\quad{\bf (iii)} If $D\subseteq\frak B\,\,\tau$-generates $\frak B$,
then $D$ also $\tau$-generates $\frak A$.   \Prf\ Applying 313Mb to the
identity map from $\frak B$ to $\frak A$, we see that the
order-closed subalgebra $\frak D$ of $\frak A$ generated by $D$ includes
$\frak B$;  but as any member of $\frak A$ is the supremum of a subset
of $\frak B$, $\frak D=\frak A$.\ \QeD\   So
$\tau(\frak A)\le\tau(\frak B)$.

\medskip

{\bf (f)} We can identify each $\frak A_i$ with the principal ideal of
$\frak A$ generated by an element $a_i$, where $\familyiI{a_i}$ is a
partition of unity in $\frak A$ (315E).   If
\discrcenter{390pt}{$\kappa=\max(\omega,\sup_{i\in I}\tau(\frak A_i),
   \min\{\lambda:\#(I)\le 2^{\lambda}\})$, }then for each $i\in I$
choose $\ofamily{\xi}{\kappa}{a_{i\xi}}$ in $\frak A_i$ such that
$\{a_{i\xi}:\xi<\kappa\}\,\,\tau$-generates $\frak A_i$, and let
$\phi:I\to\Cal P\kappa$ be injective.   For $\xi<\kappa$, set

\Centerline{$b_{\xi}=\sup_{i\in I}a_{i\xi}$,
\quad$c_{\xi}=\sup_{i\in\phi(\xi)}a_i$.}

\noindent Let $\frak B$ be the order-closed subalgebra of $\frak A$
generated by $\{b_{\xi}:\xi<\kappa\}\cup\{c_{\xi}:\xi<\kappa\}$.
Then

\Centerline{$a_i
=\inf\{c_{\xi}:\xi\in\phi(i)\}
  \Bsetminus\sup\{c_{\xi}:\xi\in\kappa\setminus\phi(i)\}
\in\frak B$}

\noindent for each $i$.   Because $\{b:b\in\frak B,\,b\Bsubseteq a_i\}$
is an order-closed subalgebra of $\frak A_i$ containing
$b_{\xi}\cap a_i=a_{i\xi}$ for every $\xi<\kappa$, it is the whole of
$\frak A_i$, so $\frak A_i\subseteq\frak B$ for every $i\in I$.   It
follows at once that $\frak B=\frak A$, so that
$\tau(\frak A)\le\kappa$.

The other parts are all elementary.
}%end of proof of 514E

\leader{514F}{}\cmmnt{ For measure algebras, Maharam type is not only
the cardinal function which gives most information, but is also,
as a rule, easy to
calculate.   For other Boolean algebras, it may not be obvious what the
Maharam type is.   The following result sometimes helps.

\medskip

\noindent}{\bf Proposition} Let $\frak A$ be a Dedekind complete Boolean
algebra, and $\langle a_{ij}\rangle_{i\in I,j\in J}$ a
$\tau$-generating family in $\frak A$ such that

\Centerline{$\family{j}{J}{a_{ij}}$ is disjoint for
every $i\in I$,
\quad$\sup_{i\in I}a_{ij}=1$ for every $j\in J$.}

\noindent Then $\tau(\frak A)\le\max(\omega,\#(I))$.
%528R

\proof{{\bf (a)} We may suppose that $J=\kappa$ is a cardinal.
For $i$, $j\in I$ set

\Centerline{$a^*_i=\sup_{\xi<\kappa}a_{i\xi}$,
\quad$b_{ij}=\sup_{\xi<\eta<\kappa}a_{i\eta}\Bcap a_{j\xi}$.}

Then

\Centerline{$\sup_{\eta\le\zeta}a_{i\eta}
=a^*_i\Bsetminus\sup_{j\in I}(b_{ij}\setminus\sup_{\xi<\zeta}a_{j\xi})$}

\noindent whenever $i\in I$ and $\zeta<\kappa$.
\Prf\ (i) If $\eta\le\zeta$ and $j\in I$, then $a_{i\eta}\Bsubseteq a^*_i$ and

$$\eqalignno{a_{i\eta}\Bcap b_{ij}
&=\sup_{\xi<\theta<\kappa}a_{i\eta}\Bcap a_{i\theta}\Bcap a_{j\xi}
=\sup_{\xi<\eta}a_{i\eta}\Bcap a_{j\xi}\cr
\displaycause{because $\ofamily{\theta}{\kappa}{a_{i\theta}}$ is disjoint}
&\Bsubseteq\sup_{\xi<\zeta}a_{j\xi},\cr}$$

\noindent so

\Centerline{$\sup_{\eta\le\zeta}a_{i\eta}
\Bsubseteq a^*_i
   \Bsetminus\sup_{j\in I}(b_{ij}\setminus\sup_{\xi<\zeta}a_{j\xi})$.}

\noindent(ii) If $0\ne c\Bsubseteq a_i^*$ and $c\Bcap a_{i\eta}=0$ for every
$\eta\le\zeta$, there are
an $\eta>\zeta$ such that $c'=c\Bcap a_{i\eta}$ is non-zero, and
a $j\in I$ such that $c''=c'\Bcap a_{j\zeta}$ is non-zero.   In this case
$c''\Bsubseteq b_{ij}\Bsetminus\sup_{\xi<\zeta}a_{j\xi}$, so
$c\Bcap b_{ij}\Bsetminus\sup_{\xi<\zeta}a_{j\xi}\ne 0$.   Accordingly

\Centerline{$\sup_{\eta\le\zeta}a_{i\eta}
\Bsupseteq a^*_i
  \Bsetminus\sup_{j\in I}(b_{ij}\setminus\sup_{\xi<\zeta}a_{j\xi})$.
\Qed}

\medskip

{\bf (b)} Let $\frak B$ be the order-closed subalgebra of $\frak A$
generated by $\{a^*_i:i\in I\}\cup\{b_{ij}:i$, $j\in I\}$.
Using (a) for the inductive step, we see that
$\sup_{\xi\le\zeta}a_{i\xi}\in\frak B$
for every $i\in I$ and $\zeta<\kappa$.   Consequently
$a_{i\zeta}=\sup_{\xi\le\zeta}a_{i\xi}\Bsetminus\sup_{\xi<\zeta}a_{i\xi}$
belongs
to $\frak B$ whenever $i\in I$ and $\zeta<\kappa$, and $\frak A=\frak B$ is
$\tau$-generated by $\{a^*_i:i\in I\}\cup\{b_{ij}:i$, $j\in I\}$,
so has Maharam type at most $\max(\omega,\#(I))$.
}%end of proof of 514F

\leader{514G}{Order-preserving functions of Boolean algebras (a)} Let
$F$ be an ordinal function of Boolean algebras, that is, a function
defined on the class of Boolean algebras, taking ordinal values, and
such that $F(\frak A)=F(\frak B)$ whenever $\frak A$ and $\frak B$ are
isomorphic.   We say that $F$ is {\bf order-preserving} if
$F(\frak B)\le F(\frak A)$ whenever $\frak B$ is a principal ideal of
$\frak A$.   \cmmnt{It is easy to check that all the
cardinal functions defined in 511D are
order-preserving;  see 514Ed.}   Now a Boolean algebra $\frak A$ is
{\bf $F$-homogeneous} if $F(\frak B)=F(\frak A)$ for every non-zero
principal ideal $\frak B$ of $\frak A$.   Of course any principal ideal
of an $F$-homogeneous Boolean algebra is again $F$-homogeneous.

\cmmnt{We have already seen `\Mth' algebras in
Chapter 33.   I mention {\bf cellularity-homogeneous} algebras
as a class which will be used later.   The proof of the Erd\H{o}s-Tarski
theorem in 513Bb is based on the idea of upwards-saturation-homogeneous
partially ordered set.   Of course all the most important ordinal
functions of Boolean algebras actually take cardinal values.}

\spheader 514Gb If $F$ is any order-preserving ordinal function of
Boolean algebras, and $\frak A$ is a Boolean algebra,
then\cmmnt{ (writing $\frak A_a$ for the principal ideal generated by
$a$)} $\{a:a\in\frak A,\,\frak A_a$ is $F$-homogeneous$\}$ is
order-dense in $\frak A$.   \prooflet{\Prf\ If $a\in\frak A^+$, set
$\xi=\min\{F(\frak A_b):0\ne b\Bsubseteq a\}$, and let $b$ be such that
$0\ne b\Bsubseteq a$ and $F(\frak A_b)=\xi$;  then $\frak A_b$ is
$F$-homogeneous.\ \QeD}   So if $\frak A$ is a Dedekind complete Boolean
algebra, it is isomorphic to a simple product of $F$-homogeneous Boolean
algebras.   \cmmnt{(Argue as in the proof of 332B.)}

\spheader 514Gc Similarly, if $F_0,\ldots,F_n$ are order-preserving
ordinal functions of Boolean algebras, and $\frak A$ is any Boolean
algebra, then $\{a:\frak A_a$ is $F_i$-homogeneous for every $i\le n\}$
is order-dense in $\frak A$;  and if $\frak A$ is Dedekind complete, it
is isomorphic to a simple product of Boolean algebras all of which are
$F_i$-homogeneous for every $i\le n$.

\cmmnt{\spheader 514Gd Of course any Boolean algebra which is
homogeneous in the full sense (316N) is $F$-homogeneous for every
function $F$ of Boolean algebras.   Maharam's theorem tells us that
a Maharam-type-homogeneous {\it measurable} algebra is homogeneous
(331N).}

\leader{514H}{Regular open algebras:
Proposition} Let $(X,\frak T)$ be a topological space and
$\RO(X)$ its regular open algebra\cmmnt{ (314O {\it et seq.})}.

(a)(i) $(\RO(X)^+,\supseteq,\RO(X)^+)
\prGT(\frak T\setminus\{\emptyset\},\supseteq,
\frak T\setminus\{\emptyset\})$.

\quad (ii) If $X$ is regular,
$(\RO(X)^+,\supseteq,\RO(X)^+)
\equivGT(\frak T\setminus\{\emptyset\},\supseteq,
\frak T\setminus\{\emptyset\})$.

(b)(i) $\sat(\RO(X))=\sat(X)$, $c(\RO(X))=c(X)$, $\pi(\RO(X))\le\pi(X)$ and
$d(\RO(X))\le d(X)$.

\quad(ii) If $X$ is regular, $\pi(\RO(X))=\pi(X)$.

\quad(iii) If $X$ is locally compact and Hausdorff, $d(\RO(X))=d(X)$.

(c) Let $\CalNwd(X)$ be the ideal of nowhere dense subsets of $X$.

\quad(i) If $X$ is regular, $\wdistr(\RO(X))\le\add\CalNwd(X)$.

\quad(ii) If $X$ is locally compact and Hausdorff,
$\wdistr(\RO(X))=\add\CalNwd(X)$.

(d) If $Y\subseteq X$ is dense, then $G\mapsto G\cap Y$ is a Boolean
isomorphism from $\RO(X)$ to $\RO(Y)$.

\proof{{\bf (a)} For $G\in\frak T\setminus\{\emptyset\}$, set
$\psi(G)=\interior\overline{G}$.   If we set $\phi(G)=G$ for
$G\in\RO(X)^+$, then $(\phi,\psi)$ is a Galois-Tukey connection from
$(\RO(X)^+,\supseteq,\RO(X)^+)$ to
$(\frak T\setminus\{\emptyset\},\supseteq,
\frak T\setminus\{\emptyset\})$.

If $X$ is regular, then $\RO(X)^+$ is coinitial with
$\frak T\setminus\{\emptyset\}$, so 513Ed, inverted, shows that they are
equivalent.

\medskip

{\bf (b)(i)} Any disjoint family in $\RO(X)^+$ is a disjoint family of
non-empty open subsets of $X$, so $c(\RO(X))\le c(X)$ and
$\sat(\RO(X))\le\sat(X)$.   On the other hand, if $\Cal G$ is a disjoint
family of non-empty open subsets of $X$, then
$\family{G}{\Cal G}{\interior\overline{G}}$ is a disjoint family in
$\RO(X)^+$, so $c(X)\le c(\RO(X))$ and $\sat(\RO(X))\le\sat(X)$.

By (a) and 513Ee, inverted,

\Centerline{$\pi(\RO(X))=\ci(\RO(X)^+)
\le\ci(\frak T\setminus\{\emptyset\})=\pi(X)$,}

\Centerline{$d(\RO(X))
=\ddownarrow(\RO(X)^+)
\le \ddownarrow(\frak T\setminus\{\emptyset\}
\le d(X)$}

\noindent by 514A.

\medskip

\quad{\bf (ii)} If $X$ is regular, $\RO(X)^+$ is coinitial
with $\frak T\setminus\{\emptyset\}$, so
$\ci(\RO(X)^+)=\ci(\frak T\setminus\{\emptyset\})$ and
$\pi(\RO(X))=\pi(X)$.

\medskip

\quad{\bf (iii)} If $X$
is locally compact and Hausdorff it is also regular, so $\RO(X)^+$ is
coinitial with $\frak T\setminus\{\emptyset\}$, and

$$\eqalignno{d(X)
&=\ddownarrow(\frak T\setminus\{\emptyset\})\cr
\displaycause{514A}
&=\ddownarrow(\RO(X)^+)
=d(\RO(X)).\cr}$$

\medskip

{\bf (c)(i)} Let $\ofamily{\xi}{\kappa}{E_{\xi}}$ be a family of nowhere
dense sets in $X$, where $\kappa<\wdistr(\RO(X))$.   For each
$\xi<\kappa$, set
$\Cal G_{\xi}=\{G:G\in\RO(X)$, $\overline{G}\cap E_{\xi}=\emptyset\}$.
Then $\Cal G_{\xi}$ is upwards-directed, and
$\bigcup\Cal G_{\xi}=X\setminus\overline{E}_{\xi}$, because any point of
$X\setminus\overline{E}_{\xi}$
belongs to a regular open set with closure disjoint from $E_{\xi}$.
But this means
that $\sup\Cal G_{\xi}=X$ in $\RO(X)$ (314P), and there is a partition
$\Cal G'_{\xi}$ of unity included in $\Cal G_{\xi}$.   Because
$\kappa<\wdistr(\RO(X))$, there is a partition $\Cal H$ of unity in
$\RO(X)$ such that $\{G:G\in\Cal G'_{\xi}$, $G\cap H\ne\emptyset\}$ is
finite for each $\xi$ and $H\in\Cal H$.   It follows that
$H\subseteq\bigcup\{\overline{G}:G\in\Cal G'_{\xi}\}$ is disjoint from
$E_{\xi}$ whenever $\xi<\kappa$ and $H\in\Cal H$.   Accordingly
$\bigcup_{\xi<\kappa}E_{\xi}$ is disjoint from $\bigcup\Cal H$ and is
nowhere dense.   As
$\ofamily{\xi}{\kappa}{E_{\xi}}$ is arbitrary,
$\add\CalNwd(X)\ge\wdistr(\RO(X))$.

\medskip

\quad{\bf (ii)} If $X$ is locally compact and Hausdorff,
suppose that $\kappa<\add\CalNwd(X)$ and that
$\ofamily{\xi}{\kappa}{\Cal G_{\xi}}$ is a family of partitions of unity
in $\RO(X)$.   Then $E_{\xi}=X\setminus\bigcup\Cal G_{\xi}$ is a
nowhere dense closed set for each $\xi$ (314P again).   So
$E=\bigcup_{\xi<\kappa}E_{\xi}$ is nowhere dense.   Set

\Centerline{$\Cal U=\{U:U\subseteq X$ is open,
$\overline{U}\subseteq X\setminus E$ is compact$\}$;}

\noindent then $\Cal U$ is an upwards-directed family with union
$X\setminus\overline{E}$, so includes a partition $\Cal G$ of unity.
But if $H\in\Cal G$ and $\xi<\kappa$, $\overline{H}$ is a
compact set disjoint from $E_{\xi}$, so must be included in the union of
some finite subfamily from
$\Cal G_{\xi}$, and $\{G:G\in\Cal G_{\xi}$, $G\cap H\ne\emptyset\}$ is
finite.   As
$\ofamily{\xi}{\kappa}{\Cal G_{\xi}}$ is arbitrary,
$\wdistr(\RO(X))\ge\add\CalNwd(X)$ and we have equality.

\medskip

{\bf (d)} If $Y\subseteq X$ is dense, and we write
$\interior_Y$, $\overline{\phantom{W}}^{(Y)}$ for interior and closure
in the subspace topology of $Y$, we have

\Centerline{$\interior_Y\overline{G\cap Y}^{(Y)}
=\interior_Y(Y\cap\overline{G\cap Y})
=\interior_Y(Y\cap\overline{G})
=Y\cap\interior\overline{G}$}

\noindent for every open set $G\subseteq X$.   Let $f:Y\to X$ be the
identity map.   Then $f$ is continuous and
$f^{-1}[M]=Y\cap M$ is nowhere dense in $Y$ whenever $M\subseteq X$ is
nowhere dense in $X$, so we have a corresponding Boolean homomorphism
$\pi:\RO(X)\to\RO(Y)$ defined by setting

\Centerline{$\pi G
=\interior_Y\overline{f^{-1}[G]}^{(Y)}
=\interior_Y\overline{G\cap Y}^{(Y)}
=Y\cap\interior\overline{G}
=G\cap Y$}

\noindent for every $G\in\RO(X)$ (314Ra).   Because $Y$ is dense,
$\pi G\ne\emptyset$ for every non-empty $G$, and $\pi$ is injective.
If $H\in\RO(Y)\setminus\{\emptyset\}$, then there is an open set
$G\subseteq X$ such that $H=G\cap Y$, so that

\Centerline{$\pi(\interior\overline{G})=Y\cap\interior\overline{G}
=\interior_Y\overline{G\cap Y}^{(Y)}=H$;}

\noindent thus $\pi$ is surjective and is an isomorphism.
}%end of proof of 514H

\leader{514I}{Category algebras} \cmmnt{For many topological spaces,
their regular open algebras can be understood better through their
expressions as quotients of Baire-property algebras.   It is time I
brought this approach into the main line of the argument.

\medskip

}{\bf (a)} Let $X$ be a topological space, and $\Cal M$ the
$\sigma$-ideal of meager subsets of $X$.   Recall that the
Baire-property algebra of $X$ is the $\sigma$-algebra
$\widehat{\Cal B}=\{G\symmdiff A:G\subseteq X$ is open,
$A\in\Cal M\}$, and that the category algebra of $X$ is
the quotient Boolean algebra $\frak G=\widehat{\Cal B}/\Cal M$ (4A3Q).
Note that if $G\subseteq X$ is any open set,
then\cmmnt{ $\overline{G}\setminus G$ and
$\overline{G}\setminus\interior\overline{G}$ are nowhere dense, so}

\Centerline{$G^{\ssbullet}=\overline{G}^{\ssbullet}
=(\interior\overline{G})^{\ssbullet}$}

\noindent in $\frak G$.

\spheader 514Ib For $G\in\RO(X)$, set $\pi G=G^{\ssbullet}\in\frak G$.
Then $\pi:\RO(X)\to\frak G$ is an order-continuous surjective Boolean
homomorphism.   \prooflet{\Prf\ (i) If $G$, $H\in\RO(X)$, then

\Centerline{$G\hskip.2em\Bcapshort_{\RO(X)}\hskip.2em H=G\cap H$,
\quad$X\hskip.2em\Bsetminusshort_{\RO(X)}\hskip.2em
G=X\setminus\overline{G}$,}

\noindent (314P), so

\Centerline{$(G\hskip.2em\Bcapshort_{\RO(X)}\hskip.2em H)^{\ssbullet}
=(G\cap H)^{\ssbullet}=G^{\ssbullet}\Bcap H^{\ssbullet}$,}

\Centerline{$(X\hskip.2em\Bsetminusshort_{\RO(X)}\hskip.2em
G)^{\ssbullet}
=(X\setminus\overline{G})^{\ssbullet}
=1\Bsetminus\overline{G}^{\ssbullet}
=1\Bsetminus G^{\ssbullet}$.}

\noindent By 312H(ii), this is enough to show that
$\pi$ is a Boolean homomorphism.   (ii) If $E\in\widehat{\Cal B}$, let
$G_0\subseteq X$ be an open set such that $G_0\symmdiff E\in\Cal M$;
then $G=\interior\overline{G}_0$ belongs to $\RO(X)$ and

\Centerline{$\pi G=G^{\ssbullet}=G_0^{\ssbullet}=E^{\ssbullet}$.}

\noindent Thus $\pi$ is surjective.   (iii) There is a regular open set
$W$ such that $X\setminus W$ is meager and every non-empty open subset
of $W$ is non-meager (4A3Ra);  now the kernel of $\pi$ is just
$\{G:G\in\RO(X)$, $G\cap W=\emptyset\}$ which has a largest member
$\interior(X\setminus W)$.   This shows that the kernel of $\pi$ is
order-closed, so that $\pi$ is order-continuous (313P(a-ii)).\ \Qed}

\spheader 514Ic\cmmnt{ From the last part of the proof of (b), we see
that the kernel of $\pi$ is the principal ideal of $\RO(X)$ generated
by $X\setminus\overline{W}$, so that in fact} $\pi$ includes an isomorphism
between\dvro{ a
principal ideal of $\RO(X)$}{ the complementary principal ideal
generated by $W$} and $\frak G$.\cmmnt{

In particular, being isomorphic to a principal ideal in the
Dedekind complete Boolean algebra $\RO(X)$,} $\frak G$ is Dedekind
complete\cmmnt{ (314Xd, 314Ea)}.

\spheader 514Id It is useful to know that if $G\subseteq X$ is open,
then the category algebra of $G$ can be identified with the principal
ideal of $\frak G$ generated by $G^{\ssbullet}$\prooflet{;  this is
because a subset of $G$ is nowhere dense regarded as a subset of $G$ iff
it is nowhere dense regarded as a subset of $X$, so that
$\Cal M\cap\Cal PG$ is exactly the ideal of meager subsets of $G$ for
the subspace topology, while the Borel $\sigma$-algebra of $G$ is
$\{G\cap E:E\subseteq X$ is Borel$\}$ (4A3Ca)}.

\spheader 514Ie\cmmnt{ Recall from 431Fa that every $A\subseteq X$ has a
Baire-property envelope, that is,
a set $E\in\widehat{\Cal B}$ such that $A\subseteq E$ and
$E\setminus F$ is meager whenever $F\in\widehat{\Cal B}$ and
$A\subseteq F$.}
If $\sequencen{A_n}$ is any sequence of subsets of $X$, and $E_n$ is a
Baire-property envelope of $A_n$ for each $n$,
then $E=\bigcup_{n\in\Bbb N}E_n$ is a
Baire-property envelope of $A=\bigcup_{n\in\Bbb N}A_n$.
\prooflet{\Prf\ Of course $A\subseteq E\in\widehat{\Cal B}$.
If $A\subseteq F\in\widehat{\Cal B}$, then
$A_n\subseteq F$ for every $n$, so $E_n\setminus F$ is meager for every
$n$ and $E\setminus F$ is meager.\ \Qed}

If $A\subseteq X$, we can define $\psi(A)\in\frak G$ by setting
$\psi(A)=\inf\{F^{\ssbullet}:A\subseteq F\in\widehat{\Cal B}\}$\cmmnt{,
because $\frak G$ is Dedekind complete}.
\cmmnt{Note that} $\psi(A)=E^{\ssbullet}$ for any Baire-property envelope
$E$ of $A$.   It follows that
$\psi(\bigcup_{n\in\Bbb N}A_n)=\sup_{n\in\Bbb N}\psi(A_n)$ for any sequence
$\sequencen{A_n}$ of subsets of $X$;  also $\psi(A)=0$ in $\frak G$ iff $A$
is meager.

\spheader 514If\dvro{ When}{ The construction here is most useful
when} $X$ is a
Baire space,\cmmnt{ so that no non-empty open set is meager,}
$\pi$\cmmnt{ is injective and} is an isomorphism between $\RO(X)$ and
$\frak G$.

\spheader 514Ig If $X$ is a zero-dimensional
space, then the algebra $\Cal E$ of open-and-closed sets in $X$ is
an order-dense subalgebra of $\RO(X)$, so that $\RO(X)$ can be identified
with the Dedekind completion of $\Cal E$;  and if $X$ is a
zero-dimensional compact Hausdorff space, then the category algebra of $X$
can\cmmnt{ equally} be
identified with the Dedekind completion of $\Cal E$.

\spheader 514Ih\cmmnt{ Finally,} I note that if $X$ is an extremally
disconnected compact Hausdorff space, so that its algebra $\Cal E$ of
open-and-closed sets is already Dedekind complete\cmmnt{ (314S)},
then $\Cal E=\RO(X)$.   So if $X$ is the Stone space of a Dedekind complete
Boolean algebra $\frak A$, we have a Boolean isomorphism
$a\mapsto\widehat{a}^{\ssbullet}$ from $\frak A$ to $\frak G$, writing
$\widehat{a}$ for the open-and-closed subset of $X$ corresponding to
$a\in\frak A$.

\leader{514J}{}\cmmnt{ Now we have the following.

\medskip

\noindent}{\bf Proposition} Let $X$ be a topological space and $\frak C$
its category algebra.

(a) $\sat(\frak C)\le\sat(X)$, $c(\frak C)\le c(X)$,
$\pi(\frak C)\le\pi(X)$ and $d(\frak C)\le d(X)$.

(b) If $X$ is a Baire space, $\sat(\frak C)=\sat(X)$ and
$c(\frak C)=c(X)$.

(c) If $X$ is regular, $\wdistr(\frak C)\le\add\CalNwd(X)$, where
$\CalNwd(X)$ is the ideal of nowhere dense subsets of $X$.

\proof{ All we need to know is that $\frak C$ is isomorphic to a
principal ideal of $\RO(X)$, which is the whole of $\RO(X)$ if $X$ is a
Baire space (514Ic, 514If), and apply 514H and 514Ed.
}%end of proof of 514J

\leader{514K}{}\cmmnt{ Later in this volume, we shall see that the
Lebesgue measure algebra, in particular, can have weak distributivity
large compared with its cellularity and its Maharam type.   For such
algebras the following result gives us significant information.

\medskip

\noindent}{\bf Proposition} Let $\frak A$ be a Boolean algebra such that
$\sat(\frak A)\le\wdistr(\frak A)$.   Then whenever $A\subseteq\frak A$
and $\#(A)<\wdistr(\frak A)$ there is a set $C\subseteq\frak A$ such
that $\#(C)\le\max(c(\frak A),\tau(\frak A))$ and
$a=\sup\{c:c\in C,\,c\Bsubseteq a\}$ for every $a\in A$.

\proof{{\bf (a)} If $\frak A$ is finite we can take $C$ to be the set of
atoms of $\frak A$;  so let us henceforth suppose that $\frak A$ is
infinite.   Let $D\subseteq\frak A$ be a $\tau$-generating set of
cardinal $\tau(\frak A)$, and $\frak D$ the subalgebra of $\frak A$
generated by $D$, so that (because $\frak A$ is infinite)
$\#(\frak D)=\tau(\frak A)$.   For any $a\in\frak A$, write

\Centerline{$Q(a)
=\{b:b\in\frak A,\,\Exists\,d\in\frak D,\,(a\Bsymmdiff d)\Bcap b=0\}$,}

\Centerline{$\Cal E(a)=\{B:B$ is a maximal antichain, $\sup B'\in Q(a)$
for every finite $B'\subseteq B\}$.}

\noindent Now the first fact to establish is that
$\Cal E(a)\ne\emptyset$ for any $a\in\frak A$.

\Prf\ Set $E=\{a:\Cal E(a)\ne\emptyset\}$.   Then $\frak D\subseteq E$,
because $1\in Q(d)$ and $\{1\}\in\Cal E(d)$ for every $d\in\frak D$.
If $a\in E$, then $Q(1\Bsetminus a)=Q(a)$ (because
$1\Bsetminus d\in\frak D$ for every $d\in\frak D$), so
$\Cal E(1\Bsetminus a)=\Cal E(a)$ is non-empty, and $1\Bsetminus a\in E$.
If $F\subseteq E$ is
non-empty and has supremum $a\in\frak A$, then there is a non-empty set
$F_0\subseteq F$, still with supremum $a$, such that
$\#(F_0)<\sat(\frak A)$ (514Db).   For each $c\in F_0$ choose
$B_c\in\Cal E(c)$.   Because
$\#(F_0)<\wdistr(\frak A)$, there is a maximal antichain
$B\subseteq\frak A$ such that $\{e:e\in B_c$, $e\Bcap b\ne 0\}$ is
finite for every $c\in F_0$.   If $B'\subseteq B$ is finite and
$c\in F_0$, then $\sup B'\subseteq\sup B'_c$ where
$B'_c=\{e:e\in B_c$, $e\Bcap\sup B'\ne 0\}$, so $\sup B'\in Q(c)$.
Set

\Centerline{$\tilde D=\{b:$ there are $b'\in B$ and $c\in F_0$ such that
$b\Bsubseteq b'\Bsetminus(a\Bsetminus c)\}$.}

\noindent Because $\sup F_0=a$ and $\sup B=1$, $\sup\tilde D=1$ and
there is a maximal antichain $\tilde B\subseteq\tilde D$.
If $B'\subseteq\tilde B$
is finite, with supremum $b^*$, there are $c_0,\ldots,c_n\in F_0$ such
that $b^*$ is disjoint from $a\Bsetminus\sup_{i\le n}c_i$;  also
$b^*\in Q(c_i)$ for each $i$.   So we can find $d_i\in\frak D$ such that
$c_i\Bsymmdiff d_i$ is disjoint from $b^*$ for each $i\le n$;
accordingly $c\Bsymmdiff d$ is disjoint from $b^*$, where
$c=\sup_{i\le n}c_i$ and $d=\sup_{i\le n}d_i$, and

\Centerline{$a\Bsymmdiff d
\Bsubseteq(a\Bsymmdiff c)\Bcup(c\Bsymmdiff d)
\Bsubseteq(a\Bsetminus c)\Bcup(c\Bsymmdiff d)
\Bsubseteq 1\Bsetminus b^*$,}

\noindent while $d\in\frak D$.    This shows that
$b^*\in Q(a)$;  as $B'$ is arbitrary, $\tilde B\in\Cal E(a)$ and $a\in E$.

This shows that $E$ is closed under complements and arbitrary suprema.
It is therefore an order-closed subalgebra of $\frak A$ (312B(iii),
313E(a-i));  since it
includes $\frak D$, it is the whole of $\frak A$, which is what we need
to know.\ \Qed

\medskip

{\bf (b)} Now turn to the given set $A$.   For each $a\in A$ choose
$B_a\in\Cal E(a)$.   Then there is a maximal antichain $B$ such that
$\{e:e\in B_a$, $e\Bcap b\ne 0\}$ is finite for every $b\in B$ and
$a\in A$.   Of course $\#(B)<\sat(\frak A)$.   Set
$C=\{d\Bcap b:d\in\frak D$, $b\in B\}$.   Then

\Centerline{$\#(C)\le\max(\omega,\#(B),\#(\frak D))
\le\max(c(\frak A),\tau(\frak A))$.}

\noindent\Quer\ Suppose that $a\in A$ is not the supremum of
$C'=\{c:c\in C$, $c\Bsubseteq a\}$.   Let $a'\Bsubseteq a$ be non-zero
and disjoint from every member of $C'$.   Then there is a $b\in B$ such
that $b\Bcap a'\ne 0$.   As $b$ is covered by finitely many members of
$B_a$ it belongs to $Q(a)$, and there is a $d\in D$ such that
$(a\Bsymmdiff d)\Bcap b=0$;  which means that

\Centerline{$0\ne a'\Bcap b\Bsubseteq a\Bcap b=d\Bcap b$,}

\noindent while $d\Bcap b\in C$.   Thus $d\Bcap b\in C'$;  but $a'$ is
supposed to be disjoint from every member of $C'$.\ \Bang

Thus $C$ has the properties we need.
}%end of proof of 514K

\leader{514L}{The regular open algebra of a pre-ordered
\dvrocolon{set}}\cmmnt{ Many important features of pre-ordered
sets, at least in those aspects which are of concern to us here, can be
related to the regular open algebras of suitable topologies.

\medskip

\noindent}{\bf Definitions (a)} For any pre-ordered set $P$, a subset
$G$ of $P$ is {\bf up-open} if $\coint{p,\infty}\subseteq G$ whenever
$p\in G$.   The family of such sets is a topology on $P$, the
{\bf up-topology}.   Similarly, the {\bf down-topology} of $P$ is the
family of
{\bf down-open} sets $H$ such that $p\le q\in H\Rightarrow p\in H$.
\cmmnt{Note that} $G\subseteq P$ is up-open iff it is closed for the
down-topology\cmmnt{, and vice versa.   In particular, the intersection of
any non-empty family of up-open sets is again up-open.}.

\spheader 514Lb I will write $\RO^{\uparrow}(P)$ for the regular open
algebra of $P$ when $P$ is given its up-topology, and
$\RO^{\downarrow}(P)$ for the regular open
algebra of $P$ when $P$ is given its down-topology.

\leader{514M}{}\cmmnt{These up- and down-topologies, entirely
unrelated to the usual `order topology' on a totally ordered set (4A2A)
and the ideas of order-convergence considered in Volume 3, take a bit of
getting used to.   Their characteristic property is that every point $p$
has a smallest neighbourhood $\coint{p,\infty}$;  see 514Xj.   I begin
with an elementary lemma for practice.

\medskip

\noindent}{\bf Lemma} Let $P$ be a pre-ordered set endowed with
its up-topology.

(a)(i) For any $A\subseteq P$,
$\overline{A}=\{p:A\cap\coint{p,\infty}\ne\emptyset\}$.

\quad(ii) For any $p\in P$, $\overline{\coint{p,\infty}}$ is the set of
elements of $P$ which are compatible upwards with $p$.

\quad(iii) For any $p$, $q\in P$, the following are equiveridical:
($\alpha$) $q\in\interior\overline{\coint{p,\infty}}$;  ($\beta$) every
member of $\coint{q,\infty}$ is compatible upwards with $p$;  ($\gamma$)
$q$ is incompatible upwards with every $r\in P$ which is incompatible
upwards with $p$.

(b) A subset of $P$ is dense iff it is cofinal.

(c) If $Q$ is another pre-ordered set with its up-topology,
a function $f:P\to Q$ is continuous iff it is order-preserving.

(d)(i) A subset $G$ of $P$ is a regular open set iff

\Centerline{$G=\{p:G\cap\coint{q,\infty}\ne\emptyset$ for every
$q\ge p\}$.}

\quad(ii)\dvAnew{2011} If $\Cal G$ is a
non-empty family of regular open subsets of
$P$, then $\bigcap\Cal G$ is a regular open subset of $P$, and is
$\inf\Cal G$ in the regular open algebra $\RO^{\uparrow}(P)$.

(e) $P$ is separative upwards iff all the sets $\coint{p,\infty}$ are
regular open sets.

(f) If $P$ is separative upwards and $A\subseteq P$ has a supremum $p$ in
$P$, then $\coint{p,\infty}=\inf_{q\in A}\coint{q,\infty}$ in
$\RO^{\uparrow}(P)$.

\proof{{\bf (a)} For (i), we need only note that $\coint{p,\infty}$ is
the smallest open set containing $p$.   Now (ii) amounts to a
restatement of the definition of `compatible upwards'.   As for (iii),

$$\eqalignno{q\in\interior\overline{\coint{p,\infty}}
&\iff\coint{q,\infty}\subseteq\overline{\coint{p,\infty}}\cr
&\iff\coint{q',\infty}\cap\coint{p,\infty}\ne\emptyset
  \text{ for every }q'\ge q\cr
\displaycause{by (i)}
&\iff\coint{q,\infty}\cap\coint{r,\infty}=\emptyset
  \text{ whenever }\coint{r,\infty}\cap\coint{p,\infty}=\emptyset\cr}$$

\noindent because

\Centerline{$P\setminus\overline{\coint{p,\infty}}
=\bigcup\{\coint{r,\infty}:
   \coint{r,\infty}\cap\coint{p,\infty}=\emptyset\}$.}

\medskip

{\bf (b)} $\Cal U=\{\coint{p,\infty}:p\in P\}$ is a base for the
up-topology, so a subset of $P$ is dense iff it meets every member of
$\Cal U$;  but this is the same thing as saying that it is cofinal.

\medskip

{\bf (c)} If $f$ is order-preserving and $H\subseteq Q$ is up-open, then

\Centerline{$p'\ge p\in f^{-1}[H]\Longrightarrow f(p')\ge f(p)\in H
\Longrightarrow f(p')\in H$,}

\noindent so $f^{-1}[H]$ is up-open;  as $H$ is arbitrary, $f$ is
continuous.   If $f$ is continuous and $p\le p'$ in $P$, then
$H=\coint{f(p),\infty}$ is up-open, so $f^{-1}[H]$ is up-open and must
contain $p'$, that is, $f(p')\ge f(p)$;  as $p$ and $p'$ are arbitrary,
$f$ is order-preserving.

\medskip

{\bf (d)(i)} For any set $A\subseteq P$,

\Centerline{$\{p:A\cap\coint{q,\infty}\ne\emptyset
  \text{ for every }q\ge p\}
=\{p:\coint{p,\infty}\subseteq\overline A\}
=\interior\overline{A}$}

\noindent (using (a)).

\medskip

\quad{\bf (ii)} As noted in 514L, $\bigcap\Cal G$ is open, so is equal
to its interior;  but 314P tells us that $\interior\bigcap\Cal G$ is
$\inf\Cal G$ in $\RO(P)$.

\medskip

{\bf (e)}

$$\eqalignno{&P\text{ is separative upwards}
\cr&\mskip80mu
\Longleftrightarrow\,
   \Forall\,p,\,q\in P,\text{ either }p\le q
   \text{ or }\Exists\,r,\,r\ge q,
     \,\coint{r,\infty}\cap\coint{p,\infty}=\emptyset
\Displaycause{511Bk}
\mskip80mu
\Longleftrightarrow\,\Forall\,p,\,q\in P,\text{ either }q\in\coint{p,\infty}
  \text{ or }q\notin\interior\overline{\coint{p,\infty}}
\Displaycause{(a-iii) above}
\mskip80mu
\Longleftrightarrow\,\Forall\,p\in P,\,\interior\overline{\coint{p,\infty}}
   \subseteq\coint{p,\infty}
\cr&\mskip80mu
\Longleftrightarrow\,\Forall\,p\in P,\,\coint{p,\infty}
  \text{ is a regular open set}.
\cr}$$

\medskip

{\bf (f)} $\coint{p,\infty}$ is actually the intersection
$\bigcap_{q\in A}\coint{q,\infty}$.
}%end of proof of 514M

\leader{514N}{Proposition} Let $(P,\le)$ be a pre-ordered set, and
write $\frak T^{\uparrow}$ for the up-topology of $P$ and
$\RO^{\uparrow}(P)$ for the regular open algebra of
$(P,\frak T^{\uparrow})$.

(a) $(\RO^{\uparrow}(P)^+,\supseteq,\RO^{\uparrow}(P)^+)
\prGT(\frak T^{\uparrow}\setminus\{\emptyset\},\supseteq,
  \frak T^{\uparrow}\setminus\{\emptyset\})
\equivGT(P,\le,P)$.   If $P$ is separative upwards, then
$(\RO^{\uparrow}(P)^+,\supseteq,\penalty-100\RO^{\uparrow}(P)^+)
\equivGT(P,\le,P)$.

(b) $\pi(\RO^{\uparrow}(P))\le\pi(P,\frak T^{\uparrow})
=d(P,\frak T^{\uparrow})=\cf P$.   If $P$ is separative upwards,
then we have equality.

(c) $\sat^{\uparrow}(P,\le)
=\sat(P,\frak T^{\uparrow})=\sat(\RO^{\uparrow}(P))$ and
$c^{\uparrow}(P,\le)
=c(P,\frak T^{\uparrow})=c(\RO^{\uparrow}(P))$.

(d) For any cardinal $\kappa$,

\Centerline{$\link_{<\kappa}(\RO^{\uparrow}(P))
\le\link_{<\kappa}^{\uparrow}(P,\le)$,}

\noindent with equality if {\it either} $P$ is separative upwards {\it or}
$\kappa\le\omega$.   In particular, we always have

\Centerline{$\link^{\uparrow}(P,\le)=\link(\RO^{\uparrow}(P))$,
\quad$\duparrow(P,\le)=d(\RO^{\uparrow}(P))$.}

(e) If $Q\subseteq P$ is cofinal, then
$\RO^{\uparrow}(Q)\cong\RO^{\uparrow}(P)$.

(f) If $A\subseteq P$ is a maximal up-antichain, then
$\RO^{\uparrow}(P)\cong\prod_{a\in A}\RO^{\uparrow}(\coint{a,\infty})$.

(g) If $\tilde P$ is the partially ordered set of equivalence classes
associated with $P$, then $\RO^{\uparrow}(\tilde P)\cong\RO^{\uparrow}(P)$.

%pre-orders needed in 5A3O

\proof{{\bf (a)} By 514Ha,

\Centerline{$(\RO^{\uparrow}(P)^+,\supseteq,\RO^{\uparrow}(P)^+)
\prGT(\frak T^{\uparrow}\setminus\{\emptyset\},\supseteq,
  \frak T^{\uparrow}\setminus\{\emptyset\})$.}

\noindent Next, observe that $\Cal U=\{\coint{p,\infty}:p\in P\}$ is a
base for $\frak T^{\uparrow}$, so that

\Centerline{$(\frak T^{\uparrow}\setminus\{\emptyset\},\supseteq,
  \frak T^{\uparrow}\setminus\{\emptyset\})
\equivGT(\Cal U,\supseteq,\Cal U)$}

\noindent by 513Ed (inverted, as usual).
If we set $\phi(p)=\coint{p,\infty}$ for $p\in P$,
and choose $\psi(U)\in P$ such that $U=\coint{\psi(U),\infty}$ for
$U\in\Cal U$, then $(\phi,\psi)$ is a Galois-Tukey connection from
$(P,\le,P)$ to $(\Cal U,\supseteq,\Cal U)$, while $(\psi,\phi)$ is a
Galois-Tukey connection in the reverse direction;  so
$(P,\le)\equivGT(\Cal U,\supseteq)$.

If $P$ is separative upwards, then $\Cal U$ is included in
$\RO^{\uparrow}(P)$ (514Me) and is coinitial with $\RO^{\uparrow}(P)^+$, so

\Centerline{$(\RO^{\uparrow}(P)^+,\supseteq,\RO^{\uparrow}(P)^+)
\equivGT(\Cal U,\supseteq,\Cal U)\equivGT(P,\le,P)$.}

\medskip

{\bf (b)} Now

$$\eqalignno{\pi(\RO^{\uparrow}(P))
&\le\pi(P,\frak T^{\uparrow})\cr
\displaycause{514H(b-i)}
&=\ci(\frak T^{\uparrow}\setminus\{\emptyset\})
=\ci\Cal U
=\cf P,\cr}$$

\noindent  defining $\Cal U$ as in (a) above.
By 514Mb, $\cf P=d(P,\frak T^{\uparrow})$.   If $P$
is separative upwards, then $\pi(\RO^{\uparrow}(P))=\cf P$ because
$(\RO^{\uparrow}(P)^+,\supseteq,\RO^{\uparrow}(P)^+)\equivGT(P,\le,P)$.

\medskip

{\bf (c)} Similarly, again using 514H(b-i), and with 512Dc at the last
step,

\Centerline{$\sat(\RO^{\uparrow}(P))
=\sat(P,\frak T^{\uparrow})
=\sat^{\downarrow}(\frak T^{\uparrow}\setminus\{\emptyset\})
=\sat^{\downarrow}(\Cal U)=\sat^{\uparrow}(P)$.}

\noindent Now we saw in 514Da and 513Bc that cellularity is determined by
saturation both for partially ordered sets and for Boolean algebras, so
$c(\RO^{\uparrow}(P))=c^{\uparrow}(P)$.   (Of course this is easily shown
by a direct argument.)

\wheader{514N}{6}{2}{2}{48pt}

{\bf (d)} Using (a) and 512Dd, we see that

$$\eqalign{\link_{<\kappa}(\RO^{\uparrow}(P))
&=\link_{<\kappa}(\RO^{\uparrow}(P)^+,\supseteq,
  \RO^{\uparrow}(P)^+)\cr
&\le\link_{<\kappa}(P,\le,P)
=\link_{<\kappa}^{\uparrow}(P,\le),\cr}$$

\noindent with equality if $P$ is separative upwards.   For other $P$, if
$\kappa\le\omega$, set $\lambda=\link_{<\kappa}(\RO^{\uparrow}(P))$ and
let $\ofamily{\xi}{\lambda}{\Cal H_{\xi}}$ be a cover of
$\RO^{\uparrow}(P)^+$ by $\hbox{$<$}\kappa$-linked sets.   Set
$A_{\xi}=\{p:\interior\overline{\coint{p,\infty}}\in\Cal H_{\xi}\}$ for
each $\xi<\kappa$.   Then any $A_{\xi}$ is
upwards-$\hbox{$<$}\kappa$-linked in $P$.   \Prf\Quer\ Otherwise, there
is an $I\in[A_{\xi}]^{<\kappa}$ which has no upper bound in $P$, that
is, $\bigcap_{p\in I}\coint{p,\infty}=\emptyset$.   Now

\Centerline{$\bigcap_{p\in I}\interior\overline{\coint{p,\infty}}
\subseteq
\bigcup_{i\in I}(\overline{\coint{p,\infty}}
  \setminus\coint{p,\infty})$}

\noindent is an open set covered by finitely many nowhere dense sets and
is therefore empty, so we have a finite
subset of $\Cal H_{\xi}$ with empty intersection.\ \Bang\QeD\    So
$\ofamily{\xi}{\lambda}{A_{\xi}}$ witnesses that
$\link_{<\kappa}^{\uparrow}(P,\le)\le\lambda$ and again we have
equality.   In particular,

\Centerline{$\link(\RO^{\uparrow}(P))=\link_{<3}(\RO^{\uparrow}(P))
=\link_{<3}^{\uparrow}(P,\le)=\link^{\uparrow}(P,\le)$,}

\Centerline{$d(\RO^{\uparrow}(P))=\link_{<\omega}(\RO^{\uparrow}(P))
=\link_{<\omega}^{\uparrow}(P,\le)=\duparrow(P,\le)$.}

\medskip

{\bf (e)} Put 514Mb and 514Hd together.

\medskip

{\bf (f)} Because $A$ is an up-antichain,
$\family{a}{A}{\coint{a,\infty}}$ is a disjoint family of open sets in
$P$;  because $A$ is maximal, $\bigcup_{a\in A}\coint{a,\infty}$ is
cofinal, therefore dense.   So 315H gives the result.

\medskip

{\bf (g)} Let $Q\subseteq P$ be a set meeting each equivalence class in
just one point, so that $q\mapsto q^{\ssbullet}:Q\to\tilde P$ is a
bijection.   Then $Q$ is cofinal with $P$, while with its subspace ordering
$Q$ is isomorphic to $\tilde P$.  So

\Centerline{$\RO^{\uparrow}(\tilde P)
\cong\RO^{\uparrow}(Q)\cong\RO^{\uparrow}(P)$}

\noindent by (e).
}%end of proof of 514N

\leader{514O}{}\cmmnt{ Of course we very much want to be able to
recognise cases in which two partially ordered sets have isomorphic
regular open algebras;  and it is also important to know when one
$\RO^{\uparrow}(P)$ can be regularly embedded in another.   The next
four results give some of the known sufficient conditions for these.

\medskip

\wheader{514O}{0}{0}{0}{48pt}
\noindent}{\bf Proposition} Suppose that $P$ and $Q$ are pre-ordered sets
and $f:P\to Q$ is an order-preserving function such that
$f^{-1}[Q_0]$ is cofinal with $P$ for every up-open cofinal
$Q_0\subseteq Q$.   Then there is an order-continuous Boolean
homomorphism $\pi:\RO^{\uparrow}(Q)\to\RO^{\uparrow}(P)$ defined by setting
$\pi H=\interior\overline{f^{-1}[H]}$ (taking the closure and interior with
respect to the up-topology on $P$) for every
$H\in\RO^{\uparrow}(Q)$.   If
$f[P]$ is cofinal with $Q$ then $\pi$ is injective, so is a regular
embedding of $\RO^{\uparrow}(Q)$ in $\RO^{\uparrow}(P)$.

\proof{ By 514Mc, $f$ is continuous for the up-topologies.   Moreover,
$f^{-1}[M]$ is nowhere dense in $P$ whenever $M\subseteq Q$ is nowhere
dense in $Q$.   \Prf\ $Q_0=Q\setminus\overline{M}$ is up-open and dense,
therefore cofinal (514Mb), so $f^{-1}[Q_0]$ is up-open and dense, and
$f^{-1}[M]\subseteq P\setminus f^{-1}[Q_0]$ is nowhere dense.\ \Qed

By 314Ra again, there is an order-continuous Boolean
homomorphism $\pi:\RO^{\uparrow}(Q)\to\RO^{\uparrow}(P)$ defined by
setting $\pi H=\interior\overline{f^{-1}[H]}$ for every
$H\in\RO^{\uparrow}(Q)$.   Now

$$\eqalign{f[P]\text{ is cofinal}
&\iff f[P]\text{ is dense}\cr
&\mskip9mu\Longrightarrow\mskip5mu f[P]\cap H\ne\emptyset
  \text{ for every }H\in\RO^{\uparrow}(Q)\setminus\{\emptyset\}\cr
&\iff f^{-1}[H]\ne\emptyset
  \text{ for every }H\in\RO^{\uparrow}(Q)\setminus\{\emptyset\}\cr
&\iff\pi H\ne\emptyset
  \text{ for every }H\in\RO^{\uparrow}(Q)\setminus\{\emptyset\}
\iff\pi\text{ is injective}.\cr}$$

\noindent So in this case $\pi$ is a regular embedding of
$\RO^{\uparrow}(Q)$ in $\RO^{\uparrow}(P)$.
}%end of proof of 514O

\leader{514P}{Corollary} Suppose that $P$ and $Q$ are pre-ordered sets,
that $f:P\to Q$ is an order-preserving function
and whenever $p\in P$, $q\in Q$ and $f(p)\le q$, there is a $p'\ge p$ such
that $f(p')\ge q$.    If $f[P]$ is {\it either} cofinal with $Q$ {\it or}
coinitial with $Q$, then $\RO^{\uparrow}(Q)$ can be regularly
embedded in $\RO^{\uparrow}(P)$.

\proof{ If $Q_0\subseteq Q$ is up-open and cofinal, then $f^{-1}[Q_0]$ is
cofinal with $P$.   \Prf\ Take any $p\in P$.   Then there are a $q\in Q_0$
such that $q\ge f(p)$ and a $p'\ge p$ such that $f(p')\ge q$;  as $Q_0$ is
up-open, $p'\in f^{-1}[Q_0]$;  as $p$ is arbitrary, $f^{-1}[Q_0]$ is
cofinal.\ \QeD\  So if $f[P]$ is cofinal with $Q$, we can use 514O.   On
the other hand, if $f[P]$ is coinitial
with $Q$ it is also cofinal with $Q$.
\Prf\ For $q\in Q$ there is a $p\in P$ such that $f(p)\le q$;
now our main hypothesis tells us that there is a $p'\in P$ such that
$f(p')\ge q$.\ \QeD\  So we have the result in this case also.
}%end of proof of 514P

\vleader{48pt}{514Q}{Proposition} Let $P$ and $Q$ be pre-ordered sets,
endowed with their up-topologies, and $f:P\to Q$ a function such that

\inset{\noindent whenever $A\subseteq P$ is a maximal up-antichain then
$f\restr A$ is injective and $f[A]$ is a maximal up-antichain in $Q$.}

\noindent Then there is an injective order-continuous Boolean
homomorphism $\pi:\RO^{\uparrow}(P)\to\RO^{\uparrow}(Q)$ defined by
setting $\pi(\interior\overline{\coint{p,\infty}})
=\interior\overline{\coint{f(p),\infty}}$ for every $p\in P$.   In
particular, $\RO^{\uparrow}(P)$ can be regularly embedded in
$\RO^{\uparrow}(Q)$.   If $f[P]$ is cofinal with $Q$, then $\pi$ is an
isomorphism.

\proof{{\bf (a)} For $p\in P$, set
$H_p=\interior\overline{\coint{f(p),\infty}}\in\RO^{\uparrow}(Q)$.   If
$A\subseteq P$ is a maximal up-antichain,
$\family{p}{A}{\coint{f(p),\infty}}$ is a disjoint family of up-open
subsets of $Q$ with dense union, so $\family{p}{A}{H_p}$ is a partition
of unity in $\RO^{\uparrow}(Q)$.   It follows that $\family{p}{A}{H_p}$
must be disjoint for every up-antichain $A\subseteq P$.   Moreover, if
$p_0\in P$ and $p_1\in\interior\overline{\coint{p_0,\infty}}$ in $P$,
we have a maximal up-antichain $A$ containing $p_0$, and
$A'=(A\setminus\{p_0\})\cup\{p_1\}$ is an up-antichain;  as
$H_{p_1}\cap\bigcup_{p\in A,p\ne p_0}H_p=\emptyset$, $H_{p_1}$ must be
included in $H_{p_0}$.

\medskip

{\bf (b)} For $G\in\RO^{\uparrow}(P)$, set $\pi G=\sup\{H_p:p\in G\}$,
the supremum being taken in $\RO^{\uparrow}(Q)$.   If $G$,
$G'\in\RO^{\uparrow}(P)$ are disjoint, then $p$ and $p'$ are
incompatible upwards, so $H_p$ and $H_{p'}$ are disjoint, whenever
$p\in G$ and $p'\in G'$;  accordingly $\pi G$ and $\pi G'$ must be
disjoint.

\medskip

{\bf (c)} If $p\in P$, then of course
$H_p\subseteq\pi(\interior\overline{\coint{p,\infty}})$.
On the other hand, if
$p'\in\interior\overline{\coint{p,\infty}}$, then we saw in (a) that
$H_{p'}\subseteq H_p$, so that $\pi(\interior\overline{\coint{p,\infty}})$
must be exactly $H_p$.

\medskip

{\bf (d)} If $\Cal G\subseteq\RO^{\uparrow}(P)$ has supremum $G_0$ in
$\RO^{\uparrow}(P)$, $\pi G_0=\sup_{G\in\Cal G}\pi G$ in
$\RO^{\uparrow}(Q)$.   \Prf\ Of course $\pi G_0\supseteq\pi G$ for every
$G\in\Cal G$.   Let $A$ be maximal among the up-antichains included in
$\bigcup\Cal G$, and extend $A$ to a maximal up-antichain
$A'\subseteq P$.   Then $\family{p}{A'}{H_p}$ is a partition of unity in
$\RO^{\uparrow}(Q)$, so $H=\sup_{p\in A}H_p$ and
$H'=\sup_{p\in A'\setminus A}H_p$ are complementary elements of
$\RO^{\uparrow}(Q)$.   For every $p\in A$ there is a $G\in\Cal G$ with
$p\in G$, so that $H_p\subseteq\pi G$;  accordingly
$H\subseteq\sup_{G\in\Cal G}\pi G$.   On the other hand, take any
$p\in G_0$.   By the maximality of $A$,
$G\cap\coint{p',\infty}=\emptyset$ for every
$p'\in A'\setminus A$ and $G\in\Cal G$, so
$\coint{p,\infty}\cap\coint{p',\infty}
\subseteq G_0\cap\coint{p',\infty}=\emptyset$ for every
$p'\in A'\setminus A$ and
$H_p\cap H_{p'}=\emptyset$ for every $p'\in A'\setminus A$, that is,
$H_p\cap H'=\emptyset$ and $H_p\subseteq H$.   As $p$ is arbitrary,

\Centerline{$\pi G_0\subseteq H\subseteq\sup_{G\in\Cal G}\pi G
\subseteq\pi G_0$}

\noindent and we have equality.\ \Qed

\medskip

{\bf (e)} Now we see that

\Centerline{$\pi\emptyset=\emptyset$,}

\Centerline{$\pi P=Q$}

\noindent (because if we take any maximal up-antichain $A\subseteq P$,
$\pi P$ includes $\sup_{p\in A}H_p$),

\Centerline{$\pi G\cap\pi H=\emptyset$ whenever $G$,
$H\in\RO^{\uparrow}(P)$ and $G\cap H=\emptyset$,}

\Centerline{$\pi(\sup\Cal G)
=\sup\pi[\Cal G]$ for every $\Cal G\subseteq\RO^{\uparrow}(P)$.}

\noindent By 312H(iv), $\pi$ is a Boolean homomorphism, and by
313L(b-iv) it is order-continuous.   Finally, $\pi G\ne\emptyset$
whenever $G\in\RO^{\uparrow}(P)\setminus\{\emptyset\}$, so $\pi$ is
injective and is a regular embedding.

\medskip

{\bf (f)} If $f[P]$ is cofinal with $Q$, then $\pi[\RO^{\uparrow}(P)]$
is order-dense in $\RO^{\uparrow}(Q)$.   \Prf\ Let
$H\in\RO^{\uparrow}(Q)$ be non-empty.   As $f[P]$ is dense, there is a
$p\in P$ such that $f(p)\in H$.   Now

\Centerline{$\emptyset\ne\pi(\interior\overline{\coint{p,\infty}})
=\interior\overline{\coint{f(p),\infty}}
\subseteq\interior\overline{H}=H$;}

\noindent as $H$ is arbitrary, we have the result.\ \QeD\   By 314Ia,
$\pi$ is an isomorphism.   This completes the proof.
}%end of proof of 514Q

\leader{514R}{Corollary} Let $P$ and $Q$ be pre-ordered sets.
Suppose that there is a function $f:P\to Q$ such that $f[P]$ is cofinal
with $Q$ and, for $p$, $p'\in P$, $p$ and $p'$ are compatible upwards in
$P$ iff $f(p)$ and $f(p')$ are compatible upwards in $Q$.   Then
$\RO^{\uparrow}(P)\cong\RO^{\uparrow}(Q)$.

\proof{ The point is that $f$ satisfies the condition of 514Q.   \Prf\
Suppose that $A\subseteq P$ is a maximal up-antichain.   If $p$, $p'$
are distinct elements of $A$, then $p$ and $p'$ are incompatible upwards
in $P$, so $f(p)$ and $f(p')$ are incompatible upwards in $Q$.   This
shows simultaneously that $f\restr A$ is injective and that $f[A]$ is an
up-antichain in $Q$.   If $q$ is any element of $Q$, there is a $p\in P$
such that $f(p)\ge q$;  now there must be a $p'\in A$ such that $p'$ is
compatible upwards with $p$, in which case $f(p')$ is compatible upwards
with $f(p)$ and therefore with $q$.   So $f[A]$ is maximal;  as $A$ is
arbitrary, we have the result.\ \Qed

So 514Q gives the result.
}%end of proof of 514R

\vleader{72pt}{514S}{Proposition}
(a) Let $\frak A$ be a Dedekind complete Boolean algebra and $P$ a
pre-ordered set.   Suppose that we have a function $f:P\to\frak A^+$ such
that, for $p$, $q\in P$,

\Centerline{$f(p)\Bsubseteq f(q)$ whenever $p\le q$,}

\Centerline{$f(p)\Bcap f(q)=0$ whenever $p$ and $q$ are incompatible
downwards in $P$,}

\Centerline{$f[P]$ is order-dense in $\frak A$.}

\noindent Then $\RO^{\downarrow}(P)\cong\frak A$.

(b) Let $\frak A$ be a Dedekind complete Boolean algebra and
$D\subseteq\frak A$ an order-dense set not containing $0$.    Give $D$
the ordering $\Bsubseteq$, and write $\RO^{\downarrow}(D)$ for the
regular open algebra of $D$ with its
down-topology.   Then $\RO^{\downarrow}(D)\cong\frak A$.

(c) Let $(X,\frak T)$ be a topological space and $P$ a pre-ordered set.
Suppose we have a function $g:P\to\frak T\setminus\{\emptyset\}$ such that,
for $p$, $q\in P$,

\Centerline{$g(p)\subseteq g(q)$ whenever $p\le q$,}

\Centerline{$g(p)\cap g(q)=\emptyset$ whenever $p$ and $q$ are incompatible
downwards in $P$,}

\Centerline{$g[P]$ is a $\pi$-base for $\frak T$.}

\noindent Then $\RO^{\downarrow}(P)\cong\RO(X)$.

(d) Let $(X,\frak T)$ be a topological space and
$\Cal U$ a $\pi$-base for the topology of $X$ not containing
$\{\emptyset\}$.   Give $\Cal U$ the ordering $\subseteq$.
Then $\RO^{\downarrow}(\Cal U)\cong\RO(X)$.

\proof{{\bf (a)(i)}
The key is the following fact:  if $p\in P$, $a\in\frak A$ and
$a\Bcap f(p)\ne 0$, then there is a $q\le p$ such that $f(q)\Bsubseteq a$.
\Prf\ There is a $q_0\in P$ such that $f(q_0)\Bsubseteq a\Bcap f(p)$.
Now $q_0$ and $p$ cannot be incompatible downwards, so there is a
$q\in\ocint{-\infty,q_0}\cap\ocint{-\infty,p}$, and in this case
$f(q)\Bsubseteq f(q_0)\Bsubseteq a$.\ \Qed

\medskip

\quad{\bf (ii)} For $G\in\RO^{\downarrow}(P)$, set $\pi G=\sup f[G]$ in
$\frak A$.   Then $\pi:\RO^{\downarrow}(P)\to\frak A$ is order-preserving.
Of course $\pi(\emptyset)=0$.

$\pi(G\cap H)=\pi G\Bcap\pi H$ for all $G$, $H\in\RO^{\downarrow}(P)$.
\Prf\ Because $\pi$ is order-preserving,
$\pi(G\cap H)\Bsubseteq\pi G\Bcap\pi H$.   \Quer\ If
$a=\pi G\Bcap\pi H\Bsetminus\pi(G\cap H)\ne 0$, take $p\in G$ such that
$a\Bcap f(p)\ne 0$;  then there is a $p'\le p$ such that
$f(p')\Bsubseteq a$.   Next, there must be a $q\in H$ such that
$f(q)\Bcap f(p')\ne 0$, and a $q'\le q$ such that $f(q')\Bsubseteq f(p')$.
But now $q'\in\ocint{-\infty,p}\cap\ocint{-\infty,q}\subseteq G\cap H$,
so $f(q')\Bsubseteq\pi(G\cap H)$;  while at the same time
$f(q')\Bsubseteq a$.\ \BanG\
Thus $\pi(G\cap H)=\pi G\Bcap\pi H$.\ \Qed

$\pi(P\setminus\overline{G})=1\Bsetminus\pi G$ for every
$G\in\RO^{\downarrow}(P)$.   \Prf\ (Perhaps I should say that
$\overline{G}$ here is the closure of $G$ for the down-topology of $P$.)
Set $H=P\setminus\overline{G}$.   Then
$\pi G\Bcap\pi H=\pi(G\cap H)=0$
by what we have just seen.   \Quer\ If
$a=1\Bsetminus(\pi G\Bcup\pi H)$ is non-zero, let
$p_0\in P$ be such that $f(p_0)\Bsubseteq a$.   Then $\ocint{-\infty,p_0}$
is a non-empty open set so must meet one of $G$, $H$.   But if
$p\in G\cup H$ then $f(p_0)\Bcap f(p)=0$ so $p_0$ and $p$ are incompatible
downwards and, in particular, $p\not\le p_0$.\ \Bang\Qed

So $\pi$ is a Boolean homomorphism,
%313H(ii) again
and it is injective because
$\pi G\Bsupseteq f(p)\ne 0$ whenever $p\in G\in\RO^{\downarrow}(P)$.
Finally, $\pi$ is surjective.   \Prf\ If $a\in\frak A$, set
$G=\{p:f(p)\Bsubseteq a\}$.   Then $G$ is down-open.   If $q\notin G$,
$f(q)\Bsetminus a\ne 0$, so there is a $q_1\le q$ such that
$f(q_1)\Bcap a=0$ and $\ocint{-\infty,q_1}$ does not meet $G$;  accordingly
$\ocint{-\infty,q}\not\subseteq\overline{G}$
and $q\notin\interior\overline{G}$.   So $G\in\RO^{\downarrow}(P)$.
Because $f[P]$ is order-dense, $a=\sup f[G]=\pi G$ belongs to
$\pi[\RO^{\downarrow}(P)]$.\ \Qed

Thus we have an isomorphism between $\RO^{\downarrow}(P)$ and $\frak A$.

\medskip

{\bf (b)} Apply (a) to the identity map from $D$ to $\frak A$.

\medskip

{\bf (c)} Apply (a) to the map
$p\mapsto\interior\overline{g(p)}:P\to\RO(X)$.

\medskip

{\bf (d)} Apply (c) to the identity function from $\Cal U$ to $\frak T$.
}%end of proof of 514S

\leader{514T}{Finite-support \dvrocolon{products}}\cmmnt{ At many
points in this chapter we find ourselves seeking to relate partially
ordered sets to Boolean algebras and topological spaces.   In 511D
and 512Eb
I sought to describe the cardinal functions of topological spaces and
Boolean algebras in terms of naturally associated partially ordered
sets, and in 514L and 514N of this section I described constructions of
topologies and Boolean algebras from partial orders.   One of the
most important constructions of general topology is that of `product'.
The matching construction in Boolean algebra is that of `free product'
(315I).   I now come to the corresponding idea for partial orders.

\medskip

\wheader{514T}{0}{0}{0}{60pt}
\noindent}{\bf Definition} Let $\familyiI{P_i}$ be a family of non-empty
partially ordered sets.   The {\bf upwards finite-support product}
$\bigotimes^{\uparrow}_{i\in I}P_i$ of $\familyiI{P_i}$ is the set
$\bigcup\{\prod_{i\in J}P_i:J\in[I]^{<\omega}\}$, ordered by saying that
$p\le q$ iff $\dom p\subseteq\dom q$ and $p(i)\le q(i)$ for every
$i\in\dom p$.   Similarly, the {\bf downwards finite-support product}
$\bigotimes^{\downarrow}_{i\in I}P_i$ of $\familyiI{P_i}$ is\cmmnt{
the same set} $\bigcup\{\prod_{i\in J}P_i:J\in[I]^{<\omega}\}$\cmmnt{,
but} ordered by saying that $p\le q$ iff $\dom q\subseteq\dom p$ and
$p(i)\le q(i)$ for every $i\in\dom q$.

\vleader{48pt}{514U}{Proposition} Let $\familyiI{P_i}$ be a family of
non-empty partially ordered sets, with upwards finite-support product
$P=\bigotimes^{\uparrow}_{i\in I}P_i$.

(a) The regular open algebra $\RO^{\uparrow}(P)$ is isomorphic to the
regular open algebra of $P^*=\prod_{i\in I}P_i$ when every $P_i$ is
given its up-topology.

(b) If $I$ is finite, $P^*$ is a cofinal subset of $P$, and the ordering
of $P^*$, regarded as a subset of $P$, is the usual product partial
order on $P^*$.

(c) If $Q_i\subseteq P_i$ is cofinal for each $i\in I$, then
$\bigcup_{J\in[I]^{<\omega}}\prod_{i\in J}Q_i$ is cofinal with $P$.   So
\ifnum\stylenumber=12$\cf P\le\max(\omega,\#(I),\sup_{i\in I}\cf P_i)$.
\else$\cf P$ is at most $\max(\omega,\#(I),\sup_{i\in I}\cf P_i)$.\fi

(d) $c^{\uparrow}(P)
=\sup_{J\in[I]^{<\omega}}c^{\uparrow}(\prod_{i\in J}P_i)$.

\proof{{\bf (a)} For $p\in P$, set

\Centerline{$G_p=\{q:q\in P^*,\,q(i)\ge p(i)$ for every $i\in\dom p\}$.}

\noindent Then $G_p$ is a non-empty open set in $P^*$.   If $p\le p'$ in
$P$, then $G_p\supseteq G_{p'}$.   If $p$, $p'\in P$ are
incompatible upwards in $P$, there must be an $i\in\dom p\cap\dom p'$ such
that $p(i)$ and $p'(i)$ are incompatible upwards in $P_i$, in which case
$G_p\cap G_{p'}$ is empty.
If $V\subseteq P^*$ is a non-empty open set, take any $q\in V$.   There
is a finite set $J\subseteq I$ such that
$V\supseteq\{q':q'\in P^*,\,q'(i)\ge q(i)$ for every $i\in J\}$.
Set $p=q\restr J$;  then $G_p\subseteq V$.   So $p\mapsto G_p$ satisfies
the conditions of 514Sc, inverted, and $\RO^{\uparrow}(P)$ is isomorphic to
$\RO(P^*)$.

\medskip

{\bf (b)-(c)} These are immediate from the definition of the ordering of
$P$.   For the estimate of the cofinality of $P$, just take cofinal sets
$Q_i\subseteq P_i$ such that $\#(Q_i)=\cf P_i$ for each $i$, and
estimate $\#(\bigcup_{J\in[I]^{<\omega}}\prod_{i\in J}Q_i)$.

\medskip

{\bf (d)} We have

$$\eqalignno{c^{\uparrow}(P)
&=c(\RO^{\uparrow}(P))\cr
\displaycause{514Nc}
&=c(\RO^{\uparrow}(P^*))\cr
\displaycause{(a) above)}
&=c(P^*)\cr
\displaycause{514Hb}
&=\sup_{J\in[I]^{<\omega}}c(\prod_{i\in J}P_i)\cr
\displaycause{5A4Be, here taking the product topology on
$\prod_{i\in I}P_i$}
&=\sup_{J\in[I]^{<\omega}}c^{\uparrow}(\prod_{i\in J}P_i)\cr}$$

\noindent because if $J$ is finite then the up-topology
$\frak T^{\uparrow}_J$ on
$\prod_{i\in J}P_i$ is just the product of the up-topologies on the
$P_i$, so we can use the identification of
$c(\prod_{i\in J}P_i,\frak T^{\uparrow}_J)$ with
$c^{\uparrow}(\prod_{i\in J}P_i)$.
}%end of proof of 514U

\exercises{\leader{514X}{Basic exercises (a)}
%\spheader 514Xa
Let $\frak A$ be a Boolean algebra.   Show that
$\link_{<\kappa}(\frak A)=\pi(\frak A)$ for any
$\kappa\ge\sat(\frak A)$.
%514D

\spheader 514Xb Let $\familyiI{\frak A_i}$ be a family of Boolean
algebras and $\frak A$ their free product.   Show that
\ifnum\stylenumber=12

\Centerline{$d(\frak A)\le\max(\omega,\#(I),\sup_{i\in I}d(\frak A_i))$,
\quad$\pi(\frak A)\le\max(\omega,\#(I),\sup_{i\in I}\pi(\frak A_i))$,
\quad$c(\frak A)\le\max(\omega,\sup_{i\in I}2^{c(\frak A_i)})$.}

\noindent\Hint{5A1Ga.}
\else
$d(\frak A)\le\max(\omega,\#(I),\sup_{i\in I}d(\frak A_i))$,
%$d(\frak A)$ is at most 
%$\max(\omega,\#(I),\penalty-100\sup_{i\in I}d(\frak A_i))$,
$\pi(\frak A)\le\max(\omega,\#(I),\sup_{i\in I}\pi(\frak A_i))$,
$c(\frak A)\le\max(\omega,\sup_{i\in I}2^{c(\frak A_i)})$.
\Hint{5A1Ga.}
\fi
%514D

\spheader 514Xc Let $\frak A$ be a Boolean algebra and $\frak B$
a chargeable Boolean algebra (definition: 391Bb).
Suppose that $\frak A\setminus\{1\}\prT\frak B\setminus\{1\}$.
Show that $\frak A$ is chargeable.   \Hint{391J.}
%514D

\spheader 514Xd Let $\kappa$ be a cardinal and $\frak A$ a Boolean algebra
of size at most $2^{\kappa}$.   (i) Show that $\frak A$ is a homomorphic
image of a $\kappa$-centered Boolean algebra.   \Hint{if $\kappa$ is
infinite, $\{0,1\}^{2^{\kappa}}$ has density $\kappa$.}   (ii) Show that if
$\frak A$ is Dedekind complete it is a homomorphic image of $\Cal P\kappa$.
\Hint{514Ca, 314K.}
%514C 514D

\spheader 514Xe Let $\frak A$ be a Boolean algebra and $\frak B$ {\it
either} a
regularly embedded subalgebra of $\frak A$ {\it or} a quotient $\frak A/I$
where $I$ is an order-closed ideal in $\frak A$.
Let $\Pou(\frak A)$, $\Pou(\frak B)$ be
the pre-ordered sets of partitions of unity in $\frak A$, $\frak B$
respectively (512Ee).   Show that $\Pou(\frak B)\prT\Pou(\frak A)$, and
hence that $\wdistr(\frak B)\ge\wdistr(\frak A)$.
%514E

\spheader 514Xf\dvAnew{2011}
Let $X$ be a set.   Show that $\tau(\Cal PX)$ is the
least cardinal $\lambda$ such that $\#(X)\le 2^{\lambda}$.
%514E

\spheader 514Xg Let $\Sigma$ be the countable-cocountable algebra of
$\omega_1$.   Show that $\Sigma$ is an order-dense subalgebra of
$\Cal P\omega_1$, that $\tau(\Sigma)=\omega_1$, and that
$\tau(\Cal P\omega_1)=\omega$.
%514Xf 514E

\def\hc{\mathop{\text{hc}}\nolimits}
\spheader 514Xh\dvAnew{2011}
For a Boolean algebra $\frak A$, write
$\hc(\frak A)
=\min\{c(\frak B):\frak B$ is a non-zero principal ideal of $\frak A\}$,
counting $\min\emptyset$ as $\infty$.   (i) Show that if $\frak B$ is a
regularly embedded subalgebra of $\frak A$, then
$\hc(\frak B)\le\hc(\frak A)$.
(ii) Show that if $\frak B$ is a Boolean algebra and there is a surjective
order-continuous Boolean homomorphism from $\frak A$ onto $\frak B$, then
$\hc(\frak B)\le\hc(\frak A)$.   (iii) Show that if $\frak B$ is a
principal ideal of $\frak A$ then $\hc(\frak B)\ge\hc(\frak A)$.
(iv) Show that if $\frak B$ is an
order-dense subalgebra of $\frak A$ then $\hc(\frak B)=\hc(\frak A)$.
(v) Show that if $\frak A$ is the simple product of a family
$\familyiI{\frak A_i}$ of Boolean algebras, then
$\hc(\frak A)=\min_{i\in I}\hc(\frak A_i)$.
%514E

\sqheader 514Xi(i) ({\smc Solovay 66})
Let $I$ be any set, with its discrete topology, and
$X=I^{\Bbb N}$ with the product topology.
Show that $\tau(\RO(X))=\omega$.
(ii) Show that the subalgebra $\frak B$ of
$\RO(X)$ generated by $\{\{x:x(n)=i\}:n\in\Bbb N,\,i\in I\}$ is
an order-dense subalgebra of $\RO(X)$ and that $\tau(\frak B)\ge\#(I)$
if $\#(I)>1$.
%514D 514F

\spheader 514Xj Let $(X,\frak T)$ be a T$_0$ topological space.   Show
that we have a partial order on $X$ defined by saying that $x\le y$ iff
$x\in\overline{\{y\}}$.   Show that $\frak T$ is the up-topology on $X$
iff the family of $\frak T$-closed sets is a topology.
%514L

\spheader 514Xk Let $P$ be a partially ordered set.   Show that a subset
of $P$ is a regular open set for the up-topology iff it is of the form
$\bigcap_{q\in A}
\{p:p\in P,\,\coint{p,\infty}\cap\coint{q,\infty}=\emptyset\}$ for some
set $A\subseteq P$.
%514N

\sqheader 514Xl Rewrite the statement and proof of the Erd\H{o}s-Tarski
theorem (513Bb) (i) in terms of topological spaces (ii) in terms of
Boolean algebras.
%514N

\spheader 514Xm Find partially ordered sets $P$ and $Q$ such that the
regular open algebras of $P$ and $Q$ for their up-topologies are
isomorphic, but $\add P\ne\add Q$ and $\cf P\ne\cf Q$.
%514N

\spheader 514Xn Let $P$ be a non-empty partially ordered set such that
its regular open algebra $\RO^{\uparrow}(P)$ for the up-topology is
atomless, and let $Q$ be a set of the same size as $P$ with the trivial
partial order in which $q\le q'$ iff $q=q'$.   Show that $Q$ and the
product partially ordered set $P\times Q$ are Tukey equivalent but
$\RO^{\uparrow}(P\times Q)$ is atomless, while $\RO^{\uparrow}(Q)$ is
purely atomic.
%514N

\spheader 514Xo Let $P$ be a partially ordered set and $\kappa$
an infinite cardinal.   Show that $\kappa<\wdistr(\RO^{\uparrow}(P))$
iff for every family $\ofamily{\xi}{\kappa}{Q_{\xi}}$ of cofinal subsets
of $P$ there is a cofinal $Q\subseteq P$ such that for every $q\in Q$ and
$\xi<\kappa$ there is an $I\in[Q_{\xi}]^{<\omega}$ such that for every
$p\ge q$ there is an $r\in I$ which is compatible upwards with $p$.
%514N

\spheader 514Xp Suppose that $P$ is a
partially ordered set and that $A\subseteq P$ is such that

\Centerline{$Q=\{q:q\in P$, $q=\sup\{a:a\in A$, $a\le q\}\}$}

\noindent is cofinal with $P$.   Show that if $P$ is separative upwards,
then $\tau(\RO^{\uparrow}(P))\le\#(A)$.
%514N
%query do we need "separative"?

\spheader 514Xq Let $\frak A$ be the measure algebra of Lebesgue
measure.   Show that the simple products $\{0,1\}\times\frak A$ and
$\Cal P\Bbb N\times\frak A$ are not isomorphic, but that each can be
regularly embedded in the other.
%514Q

\spheader 514Xr Let $(X,\frak T)$ and $(Y,\frak S)$ be topological
spaces.   Suppose we have a $\pi$-base $\Cal U$ for $\frak T$ and a
function $f:\Cal U\to\frak S$ such that $f[\Cal U]$ is a $\pi$-base for
$\frak S$ and, for $U$, $U'\in\Cal U$, $U\cap U'=\emptyset$ iff
$f(U)\cap f(U')=\emptyset$.   Show that $\RO(X)\cong\RO(Y)$.
%514R

\spheader 514Xs Let $\familyiI{P_i}$ be a family of non-empty partially
ordered sets and $\family{j}{J}{I_j}$ a partition (that is, disjoint
cover) of $I$.   Show that
the upwards finite-support product $\bigotimes^{\uparrow}_{i\in I}P_i$
can be naturally identified with
$\bigotimes^{\uparrow}_{j\in J}\bigotimes^{\uparrow}_{i\in I_j}P_i$.
%514T

\leader{514Y}{Further exercises (a)}
%\spheader 514Ya
For a partially ordered set $P$, its {\bf
order-dimension} is the smallest cardinal $\kappa$ such that $P$ is
isomorphic, as partially ordered set, to a subset of a product
$\prod_{\xi<\kappa}X_{\xi}$ where every $X_{\xi}$ is a totally ordered
set (and the product is given its product partial order, as in 315C).
Show that the order-dimension of a Boolean algebra $\frak A$ is
$\link(\frak A)$.
%514C mt51bits

\spheader 514Yb Show that $\Cal P\Bbb N$ has a subalgebra with
uncountable $\pi$-weight.   \Hint{515H.}
%514D

\spheader 514Yc Let $\frak A$ be a Boolean algebra such that
$c(\frak A)\ne 1$, and $A$ a subset of $\frak A$.   
Show that there is a $B\in[A]^{\le c(\frak A)}$
with the same upper and lower bounds as $A$.
%514D

\spheader 514Yd Show that for any cardinal $\kappa$ there are a ccc
Boolean algebra $\frak A$ and an ideal $\Cal I$ of $\frak A$ such that
$c(\frak A/\Cal I)=\kappa$.
%514E 514Xi mt51bits

\spheader 514Ye Let $\frak A$ be a Dedekind complete Boolean algebra, not
$\{0\}$, $\Bbb P$ the forcing notion
$(\frak A^+,\Bsubseteqshort,1,\downarrow)$ (5A3M), and $\kappa$ a cardinal.
Show that the following are equiveridical:  (i) there is an
atomless order-closed subalgebra of $\frak A$ with Maharam type at most
$\kappa$;  (ii)
$\VVdP\,\Cal P\check\kappa\ne(\Cal P\kappa)\var2spcheck$.
%514F mt51bits

\spheader 514Yf Let $\frak A$ be a Boolean algebra and $\kappa$ a
cardinal.   I will say that $\frak A$ has the 
{\bf\hbox{$<$}$\kappa$-interpolation property}
if whenever $A$, $B\subseteq\frak A$, $a\Bsubseteq b$ whenever $a\in A$ and
$b\in B$, and $\#(A\cup B)<\kappa$, then there is a $c\in\frak A$ such that
$a\Bsubseteq c\Bsubseteq b$ for every $a\in A$, $b\in B$.
(Thus the $\sigma$-interpolation property of 466G is the
{\bf\hbox{$<$}$\omega_1$}-interpolation property.)
(i) Suppose that $\frak A$ has the \hbox{$<$}$\kappa$-interpolation
property and $I$ is an ideal of $\frak A$ such that $\kappa\le(\add I)^+$.
Show that the quotient $\frak A/I$ has the 
\hbox{$<$}$\kappa$-interpolation property.
(ii) Suppose that $\frak A$ has the \hbox{$<$}$\kappa$-interpolation
property,
$\frak B$ is a Boolean algebra of size at most $\kappa$, $\frak C$ is a
subalgebra of $\frak B$ and $\phi:\frak C\to\frak A$ is a
Boolean homomorphism.
Show that $\phi$ has an extension to a Boolean homomorphism from $\frak B$ to
$\frak A$.   (Compare 314K.)
(iii)\dvAnew{2010} Show that if $\frak A$ has the
\hbox{$<$}$\sat(\frak A)$-interpolation property it is Dedekind complete.
%514+ 466G 515K
}%end of exercises

\endnotes{
\Notesheader{514}
With any mathematical object, the set-theorist's first concern is simply
to establish its cardinality.   There is therefore a natural distinction
to make between cardinal invariants which control the cardinality of a
space, as linking number, centering number and $\pi$-weight do for
Boolean algebras
(514Da), and others, like weak distributivity, which are measures of
some kind of complexity not directly linked with cardinality.
Observe that for general Boolean algebras $\frak A$ not even
the cellularity is controlled by the Maharam type (514Xi);  in fact, of
the cardinals here, only $\wdistr(\frak A)$ is controlled by
$\tau(\frak A)$ alone (514Dd).   Maharam
type and cellularity together control the size of the algebra (514De),
and for measurable algebras, of course,
Maharam type almost completely determines the algebra and even the
measure (see Chapter 33).

I use the language of Galois-Tukey connections in many of the proofs of
this section.   This is not because there is any real need for it (there
is no depth to any of the results I quote) but because I think that it
shows some common strands running through a rather long list of facts.
Also it points up the proofs which are {\it not} reducible to simple
applications of ideas in \S512;  for instance, those relating to weak
distributivity.   And, finally, it will provide useful practice for the
ideas of Chapter 52.

I have deliberately arranged the lists of cardinal functions of
topological spaces and Boolean algebras in such a way that the
cardinals of Boolean algebras and their Stone spaces will naturally
correspond.   There are of course important exceptions.   The Maharam
type of a Boolean algebra, and the tightness of a topological space, do
not seem to have significant natural analogues in the other category.
Note that the correspondences depend to a significant degree on the
compactness of Stone spaces.   This is perhaps more important than their
zero-dimensionality.   The point about the open-and-closed algebra of a
zero-dimensional space is that it is order-dense in the regular open
algebra, and that our cardinal functions of Boolean algebras are nearly
all unchanged by Dedekind completion (514Ee).   For arbitrary
topological spaces, we can still investigate their regular open
algebras, and we find that the cardinal functions of a regular open
algebra are much more closely related to those of the topological space
if the space is locally compact (514A, 514H(b)-(c)).

You will not be surprised to recognise some of the results and arguments
of this section as direct generalizations of special cases already
treated;  thus 316B becomes 514Bb, 316E (or 215B(iv)) becomes 514Db,
316I becomes 514Be and 4A1O becomes 514De.

I have to admit that there are rather more pages than ideas in this
section.   What it is really here for is to provide a compendium of
useful facts in the language which I wish to use in the rest of the
volume.   Perhaps I should say `languages', because much of the space is
taken up by repeating results in three forms, as they apply to partially
ordered sets, to Boolean algebras and to topological spaces.   The point
is of course that we frequently find that a fact which is obvious in one
of its three manifestations is a surprise in another.   And some care is
needed in the translations.   The theory of finite-support products of
partially ordered sets (514T-514U), for instance, is supposed to mimic
the theory of products of topological spaces.   But actually it reflects
the theory of $\pi$-bases of topologies rather than the theory of
spaces-with-points.   And while we have straightforward functors between
the {\it categories} of Boolean algebras and topological spaces, with
Boolean homomorphisms corresponding to continuous functions (312Q-312S),
such results as we have concerning functions between partially ordered
sets and their actions on the corresponding
regular open algebras are partial and
delicate (514O-514R).  %514O 514Q 514R

The Tukey classification (513D) and the
regular open algebras of 514N are both attempts to reduce the
multitudinous variety of partially ordered sets to relatively coherent
schemes.   They carry rather different information;  the Tukey
classification tells us about additivity and cofinality (513E) and
precalibers (516C below), while the regular open algebra determines
linking numbers (514N).   It is easy to find partially ordered sets with
the same regular open algebras but different additivity and cofinality
(514Xm), or with the same Tukey classification but different regular
open algebras (514Xn).   The regular open algebras
studied here are primarily of interest in relation to the use of
partially ordered sets in the theory of
forcing;  I hope to return to such questions later in this volume.

%514F relevant only when we have uncountable cellularity

Of the cardinal functions of Boolean algebras defined in \S511, I have
not mentioned Martin numbers or Freese-Nation numbers.   These will
be dealt with at length in \S\S517-518.
}%end of notes

\discrpage

