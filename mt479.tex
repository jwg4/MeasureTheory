\frfilename{mt479.tex}
\versiondate{15.2.10/28.4.10}

\def\hp{\mathop{\text{hp}}\nolimits}
\def\rdtf{rapidly decreasing test function}
\def\varinnerprod#1#2{#1\dotproduct#2}
\def\zz{}
\def\trs{^{\top}}


\def\chaptername{Geometric measure theory}
\def\sectionname{Newtonian capacity}

\newsection{479}
I end the chapter with a sketch of fragments of the theory of Newtonian
capacity.   I introduce equilibrium measures as integrals
of harmonic measures (479B);  this gives a quick definition of capacity
(479C), with a substantial number of basic properties (479D, 479E),
including its extendability to a Choquet capacity (479Ed).
I give sufficient fragments of the theory of Newtonian potentials (479F,
479J) and harmonic analysis (479G, 479I) to support the classical
definitions of capacity and equilibrium measures in terms of potential and
energy (479K, 479N).   The method demands some Fourier analysis extending
that of Chapter 28 (479H).
479P is a portmanteau theorem on generalized equilibrium measures and
potentials with an exact description of the latter in terms of outer
Brownian hitting probabilities.
I continue with notes on
capacity and Hausdorff measure (479Q), self-intersecting Brownian
paths (479R) and an example of a discontinuous equilibrium potential
(479S).   Yet another definition of capacity, for
compact sets, can be formulated in terms of gradients of potential
functions (479U);  this leads to a simple inequality relating
capacity to Lebesgue measure (479V).   The section ends with an alternative
description of capacity in terms of a measure on the family of closed
subsets of $\BbbR^r$ (479W).

\leader{479A}{Notation} In this section, unless otherwise stated,
$r$ will be a fixed integer greater than or equal to $3$.
\cmmnt{As in \S478, }$\mu$ will be Lebesgue measure on $\BbbR^r$,
and $\beta_r$ the measure of\cmmnt{ the unit ball} $B(\tbf{0},1)$;
$\nu$ will be normalized $(r-1)$-dimensional Hausdorff
measure on $\BbbR^r$\cmmnt{, so that
$\nu(\partial B(\tbf{0},1))=r\beta_r$}.

Recall that if $\zeta$ is a measure on a space $X$, and
$E\in\dom\zeta$, then $\zeta\LLcorner E$ is defined by saying that
$(\zeta\LLcorner E)(F)=\zeta(E\cap F)$ whenever $F\subseteq X$ and
$\zeta$ measures $E\cap F$\cmmnt{ (234M\formerly{2{}34E})}.
If $\zeta$ is a Radon measure, so is
$\zeta\LLcorner E$\cmmnt{ (416S)}.

\cmmnt{As in \S478, }$\Omega$ will be $C(\coint{0,\infty};\BbbR^r)_0$,
with the topology of
uniform convergence on compact sets;
$\mu_W$ will be Wiener measure on $\Omega$.
Recall that the Brownian hitting probability $\hp(D)$
of a set $D\subseteq\BbbR^r$
is $\mu_W\{\omega:\omega^{-1}[D]\ne\emptyset\}$ if this is
defined, and that for any $D\subseteq\BbbR^r$ the outer
Brownian hitting probability is
$\hp^*(D)=\mu_W^*\{\omega:\omega^{-1}[D]\ne\emptyset\}$\cmmnt{ (477Ia)}.
%this is $\min\{\hp(E):E\supseteq D$ is G$_{\delta}\}$.

If $x\in\BbbR^r$ and $A\subseteq\BbbR^r$ is an analytic set, $\mu_x^{(A)}$
will be the harmonic measure for arrivals in $A$ from $x$\cmmnt{ (478P)};
note that $\mu_x^{(A)}(\BbbR^r)=\mu_x^{(A)}(\partial A)=\hp(A-x)$.

I will write $\rhotv$ for the total variation metric on the space
$M^+_{\text{R}}(\BbbR^r)$ of totally finite Radon measures on $\BbbR^r$, so
that

\Centerline{$\rhotv(\lambda,\zeta)
=\sup_{E,F\subseteq\BbbR^r\text{ are Borel}}
 \lambda E-\zeta E-\lambda F+\zeta F$}

\noindent for $\lambda$,
$\zeta\in M^+_{\text{R}}(\BbbR^r)$\cmmnt{ (437Qa)}.

\leader{479B}{Theorem} Let $A\subseteq\BbbR^r$ be a bounded analytic set.

\quad(i) There is a Radon measure $\lambda_A$ on $\BbbR^r$, with support
included in $\partial A$, defined by saying that
\discretionary{}{}{}$\family{x}{\partial B(\tbf{0},\gamma)}
{\Bover1{r\beta_r\gamma}\mu_x^{(A)}}$ is a
disintegration of $\lambda_A$ over the subspace measure
$\nu_{\partial B(\tbf{0},\gamma)}$ whenever $\gamma>0$ and
$\overline{A}\subseteq\interior B(\tbf{0},\gamma)$.

\quad(ii) $\lambda_A$ is the limit
$\lim_{\|x\|\to\infty}\|x\|^{r-2}\mu_x^{(A)}$ for
the total variation metric on $M^+_{\text{R}}(\BbbR^r)$.

\leaveitout{Whenever $\gamma>0$ is such that
$\overline{A}\subseteq\interior B(\tbf{0},\gamma)$, $M>1$ and $\|x\|=M\gamma$,
then

\Centerline{$|\lambda_AE-\|x\|^{r-2}\mu_x^{(A)}E|
\le\gamma^{r-2}
  \bigl(|\Bover{M^r}{(M-1)^r}-1|+\Bover{M^{r-2}}{(M-1)^r}\bigr)$}

\noindent for every Borel set $E\subseteq\BbbR^r$.}

\proof{{\bf (a)} Suppose that $\gamma>0$ is such that
$\overline{A}\subseteq\interior B(\tbf{0},\gamma)$.   By 478T,
$x\mapsto\mu^{(A)}_x(E):\partial B(\tbf{0},\gamma)\to\coint{0,\infty}$ is
continuous for every Borel set $E\subseteq\BbbR^r$,
and $\mu^{(A)}_x(\BbbR^r\setminus\overline{A})=0$ for every $x$, so
$\{\mu^{(A)}_x:x\in\partial B(\tbf{0},\gamma)\}$ is uniformly tight.
By 452Da, we have a unique
totally finite Radon measure $\zeta_{\gamma}$ such that
$\family{x}{\partial B(\tbf{0},\gamma)}{\Bover1{r\beta_r\gamma}\mu_x^{(A)}}$ is a
disintegration of $\zeta_{\gamma}$ over the subspace measure
$\nu_{\partial B(\tbf{0},\gamma)}$.   Since $\BbbR^r\setminus\partial A$ is
$\mu_x^{(A)}$-negligible for every $x\in\partial B(\tbf{0},\gamma)$ (478Pa),
it is also $\zeta_{\gamma}$-negligible.

\medskip

{\bf (b)} Now suppose that $\overline{A}\subseteq\interior B(\tbf{0},\gamma)$ and
that $\|x\|=M\gamma$, where $M>1$.   Then 478R tells us that
$\family{y}{\BbbR^r}{\mu^{(A)}_y}$ is a disintegration of
$\mu_x^{(A)}$ over $\mu_x^{(B(\tbf{0},\gamma))}$.  So, for any Borel set
$E\subseteq\BbbR^r$,

$$\eqalignno{|\zeta_{\gamma}E-\|x\|^{r-2}\mu_x^{(A)}E|
&=\Bover1{r\beta_r\gamma}\bigl|\int_S\mu_y^{(A)}E\,\nu(dy)
  -\|x\|^{r-2}\int_{\BbbR^r}\mu_y^{(A)}E\,\mu_x^{(B(\tbf{0},\gamma))}(dy)\bigr|
  \cr
\displaycause{where $S=\partial B(\tbf{0},\gamma)$}
&=\Bover1{r\beta_r\gamma}\bigl|\int_S\mu_y^{(A)}E\,\nu(dy)
  -\|x\|^{r-2}\int_S\mu_y^{(A)}E\,\mu_x^{(S)}(dy)\bigr|\cr
\displaycause{478Pa}
&=\Bover1{r\beta_r\gamma}\bigl|\int_S\mu_y^{(A)}E\,\nu(dy)
  -\|x\|^{r-2}\int_S\Bover{\|x\|^2-\gamma^2}{r\beta_r\gamma\|x-y\|^r}
     \mu_y^{(A)}E\,\nu(dy)\bigr|\cr
\displaycause{478Q}
&\le\Bover1{r\beta_r\gamma}\int_S
  |1-\|x\|^{r-2}\,\Bover{\|x\|^2-\gamma^2}{\|x-y\|^r}|
  \mu_y^{(A)}E\nu(dy)\cr
&\le\Bover{\nu S}{r\beta_r\gamma}
  \sup_{y\in S}|1-\|x\|^{r-2}\,\Bover{\|x\|^2-\gamma^2}{\|x-y\|^r}|\cr
&\le\gamma^{r-2}
  \sup_{y\in S}(|1-\Bover{\|x\|^r}{\|x-y\|^r}|
     +\Bover{\gamma^2\|x\|^{r-2}}{\|x-y\|^r})
  \cr
&=\gamma^{r-2}
  \bigl(|\Bover{M^r\gamma^r}{(M\gamma-\gamma)^r}-1|
     +\Bover{\gamma^rM^{r-2}}{\gamma^r(M-1)^r}\bigr)
  \cr
&=\gamma^{r-2}\bigl(|\Bover{M^{r-2}}{(M-1)^r}-1|
  +\Bover{M^r}{(M-1)^r}\bigr).\cr}$$

\medskip

{\bf (c)} Accordingly

\Centerline{$\rhotv(\zeta_{\gamma},\|x\|^{r-2}\mu_x^{(A)})
\le 2\gamma^{r-2}
   \bigl(|\Bover{M^r}{(M-1)^r}-1|+\Bover{M^{r-2}}{(M-1)^r}\bigr)$}

\noindent whenever $\gamma>0$,
$\overline{A}\subseteq\interior B(\tbf{0},\gamma)$ and
$\|x\|=M\gamma>\gamma$;  so that

\Centerline{$\zeta_{\gamma}=\lim_{\|x\|\to\infty}\|x\|^{r-2}\mu_x^{(A)}$}

\noindent for the total variation metric whenever $\gamma>0$ and
$\overline{A}\subseteq\interior B(\tbf{0},\gamma)$.   We can therefore write
$\lambda_A$ for the limit, and both (i) and (ii) will be true, since I have
already checked that $\supp(\zeta_{\gamma})\subseteq\partial A$ for all
large $\gamma$.
}%end of proof of 479B

\leader{479C}{Definitions (a)(i)} In the context of 479B, I will call
$\lambda_A$ the {\bf equilibrium measure} of the bounded analytic set
$A$, and $\lambda_A\BbbR^r\cmmnt{\mskip5mu=\lambda_A(\partial A)}$ the
{\bf Newtonian capacity} $\capacity A$ of $A$.

\medskip

\quad{\bf (ii)} For any $D\subseteq\BbbR^r$, its {\bf Choquet-Newton
capacity} will be

\Centerline{$c(D)
=\inf_{G\supseteq D\text{ is open}}\sup_{K\subseteq G\text{ is compact}}
\capacity K$.}

\noindent\cmmnt{(I will confirm in 479Ed below
that $c$ is in fact a capacity as defined in \S432.) }Sets with zero
Choquet-Newton capacity are called {\bf polar}.

\spheader 479Cb If $\zeta$ is a Radon
measure on $\BbbR^r$, the {\bf Newtonian potential} associated with
$\zeta$ is
the function $W_{\zeta}:\BbbR^r\to[0,\infty]$ defined by the formula

\Centerline{$W_{\zeta}(x)=\int_{\BbbR^r}\Bover1{\|y-x\|^{r-2}}\zeta(dy)$}

\noindent for $x\in\BbbR^r$.   The {\bf energy} of $\zeta$ is now

\Centerline{$\energy(\zeta)=\int W_{\zeta}d\zeta
=\int_{\BbbR^r}\int_{\BbbR^r}\Bover1{\|x-y\|^{r-2}}\zeta(dy)\zeta(dx)$.}

\noindent If $A$ is a bounded analytic subset of $\BbbR^r$, the potential
$\tilde W_A=W_{\lambda_A}$ is the {\bf equilibrium potential} of $A$.

\cmmnt{(In 479P below I will describe constructions of equilibrium
measures and
potentials for arbitrary subsets $D$ of $\BbbR^r$ such that $c(D)$ is
finite.)}

\spheader 479Cc If $\zeta$ is a Radon measure on $\BbbR^r$,
I will write $U_{\zeta}$ for the
{\bf ($r-1$)-potential} of $\zeta$, defined by saying that
$U_{\zeta}(x)
=\biggerint_{\BbbR^r}\Bover1{\|x-y\|^{r-1}}\zeta(dy)\in[0,\infty]$ for
$x\in\BbbR^r$.

\leader{479D}{}\cmmnt{ The machinery in Theorem 479B gives an efficient
method of approaching
several fundamental properties of equilibrium measures.   I start with
some elementary calculations.

\medskip

\noindent}{\bf Proposition}
(a) For any $\gamma>0$ and $z\in\BbbR^r$,
the Newtonian capacity of
$B(z,\gamma)$ is $\gamma^{r-2}$, the equilibrium measure
of $B(z,\gamma)$ is
$\Bover1{r\beta_r\gamma}\nu\LLcorner\partial B(z,\gamma)$, and the
equilibrium potential of $B(z,\gamma)$ is given by

\Centerline{$\tilde W_{B(z,\gamma)}(x)
=\min(1,\Bover{\gamma^{r-2}}{\|x-z\|^{r-2}})$}

\noindent for every $x\in\BbbR^r$.

(b) Let $A\subseteq\BbbR^r$ be a bounded analytic
set with equilibrium measure $\lambda_A$ and equilibrium potential
$\tilde W_A$.

\quad(i) $\tilde W_A(x)\le 1$ for every $x\in\BbbR^r$.

\quad(ii) If $B\subseteq A$ is another analytic set,
$\tilde W_B\le\tilde W_A$.

\quad(iii) $\tilde W_A(x)=1$ for every $x\in\interior A$.

(c) Let $A$, $B\subseteq\BbbR^r$ be bounded analytic sets.

\quad(i)\cmmnt{ Defining $+$ and $\le$ as in 234G\formerly{1{}12Xe} and
234P,} $\lambda_{A\cup B}\le\lambda_A+\lambda_B$.

\quad(ii) $\lambda_AB\le\capacity B$.

\proof{{\bf (a)} For $x\in\BbbR^r\setminus B(z,\gamma)$,
$\mu_x^{(B(z,\gamma))}$ is the indefinite-integral measure over
$\nu\LLcorner\partial B(z,\gamma)$ defined by the function
$y\mapsto\Bover{\|x-z\|^2-\gamma^2}{r\beta_r\gamma\|x-y\|^r}$ (478Qc).
So $\|x\|^{r-2}\mu_x^{(B(z,\gamma))}$ is the indefinite-integral measure
defined by

\Centerline{$y\mapsto f_x(y)
=\Bover{\|x\|^{r-2}(\|x-z\|^2-\gamma^2)}{r\beta_r\gamma\|x-y\|^r}$.}

\noindent As $\|x\|\to\infty$, $f_x(y)\to\Bover1{r\beta_r\gamma}$
uniformly for $y\in\partial B(z,\gamma)$, so
$\lambda_{B(\tbf{0},\gamma)}
=\lim_{\|x\|\to\infty}\|x\|^{r-2}\mu_x^{(B(z,\gamma))}$ is
$\Bover1{r\beta_r\gamma}\nu\LLcorner\partial B(z,\gamma)$.
Consequently the capacity of $B(z,\gamma)$ is
$\Bover1{r\beta_r\gamma}\nu(\partial B(z,\gamma))=\gamma^{r-2}$, and the
equilibrium potential is

$$\eqalignno{\tilde W_{B(z,\gamma)}(x)
&=\Bover1{r\beta_r\gamma}\int_{\partial B(z,\gamma)}
   \Bover1{\|y-x\|^{r-2}}\nu(dy)\cr
&=\Bover1{r\beta_r\gamma}\int_{\partial B(\tbf{0},\gamma)}
   \Bover1{\|y+z-x\|^{r-2}}\nu(dy)
=\Bover1{r\beta_r\gamma}
  \cdot\Bover{\nu(\partial B(\tbf{0},\gamma))}{\max(\gamma,\|x-z\|)^{r-2}}\cr
\displaycause{478Ga}
&=\min(1,\Bover{\gamma^{r-2}}{\|x-z\|^{r-2}}).\cr}$$

\medskip

{\bf (b)(i)} Let $\gamma>0$ be such that
$\overline{A}\subseteq\interior B(\tbf{0},\gamma)$.   Then

$$\eqalignno{\tilde W_A(x)
&=\int_{\BbbR^r}\Bover1{\|x-y\|^{r-2}}\lambda_A(dy)
=\Bover1{r\beta_r\gamma}\int_{\partial B(\tbf{0},\gamma)}\int_{\BbbR^r}
  \Bover1{\|x-y\|^{r-2}}\mu_z^{(A)}(dy)\nu(dz)\cr
\displaycause{452F}
&\le\Bover1{r\beta_r\gamma}\int_{\partial B(\tbf{0},\gamma)}
  \Bover1{\|x-z\|^{r-2}}\nu(dz)\cr
\displaycause{478Pb, 478H}
&=\Bover{\nu(\partial B(\tbf{0},\gamma))}{r\beta_r\gamma}
  \Bover1{\max(\gamma,\|x\|)^{r-2}}\cr
\displaycause{478Ga}
&\le 1.\cr}$$

\medskip

\quad{\bf (ii)} Let $\gamma>0$ be such that
$\overline{A}\subseteq\interior B(\tbf{0},\gamma)$.   Then

$$\eqalignno{\tilde W_B(x)
&=\int_{\BbbR^r}\Bover1{\|x-y\|^{r-2}}\lambda_B(dy)\cr
&=\Bover1{r\beta_r\gamma}\int_{\partial B(\tbf{0},\gamma)}
   \int_{\BbbR^r}\Bover1{\|x-y\|^{r-2}}\mu_z^{(B)}(dy)\nu(dz)\cr
&=\Bover1{r\beta_r\gamma}\int_{\partial B(\tbf{0},\gamma)}
   \int_{\BbbR^r}\int_{\BbbR^r}\Bover1{\|x-y\|^{r-2}}
   \mu_w^{(B)}(dy)\mu_z^{(A)}(dw)\nu(dz)\cr
\displaycause{because $\family{w}{\BbbR^r}{\mu_w^{(B)}}$ is a
disintegration of $\mu_z^{(B)}$ over $\mu_z^{(A)}$ for every $z$, by
478R}
&\le\Bover1{r\beta_r\gamma}\int_{\partial B(\tbf{0},\gamma)}
   \int_{\BbbR^r}\Bover1{\|x-w\|^{r-2}}\mu_z^{(A)}(dw)\nu(dz)\cr
\displaycause{by 478Pb, because $y\mapsto\Bover1{\|x-y\|^{r-2}}$ is
continuous and superharmonic}
&=\tilde W_A(x).\cr}$$

\medskip

\quad{\bf (iii)} If $x\in\interior A$, there is a $\gamma>0$ such that
$B(x,\gamma)\subseteq A$;  now, putting (a) and (ii) above together,

\Centerline{$\tilde W_A(x)\ge\tilde W_{B(x,\gamma)}(x)=1$.}

\noindent Since we know from (i) that $\tilde W_A(x)\le 1$, we have
equality.

\medskip

{\bf (c)(i)} Suppose that $K\subseteq\BbbR^r$ is
compact  and that $x\in\BbbR^r$.
Let $\tau_A$, $\tau_B$ and $\tau_{A\cup B}$ be the Brownian
hitting times to $A-x$, $B-x$ and $(A\cup B)-x$ respectively.   Then
$\tau_{A\cup B}=\tau_A\wedge\tau_B$.   Now

$$\eqalign{\mu^{(A\cup B)}_x(K)
&=\mu_W\{\omega:\tau_{A\cup B}(\omega)<\infty,\,
   x+\omega(\tau_{A\cup B}(\omega))\in K\}\cr
&\le\mu_W\{\omega:\tau_{A\cup B}(\omega)=\tau_A(\omega)<\infty,\,
      x+\omega(\tau_A(\omega))\in K\}\cr
&\mskip150mu
  +\mu_W\{\omega:\tau_{A\cup B}(\omega)=\tau_B(\omega)<\infty,\,
      x+\omega(\tau_B(\omega))\in K\}\cr
&\le\mu^{(A)}_x(K)+\mu^{(B)}_x(K).\cr}$$

\noindent Multiplying by $\|x\|^{r-2}$ and letting $\|x\|\to\infty$,

\Centerline{$\lambda_{A\cup B}(K)\le\lambda_AK+\lambda_BK
=(\lambda_A+\lambda_B)(K)$}

\noindent for every $K$, which is the criterion of 416E(a-ii).

\medskip

\quad{\bf (ii)} For any $x\in\BbbR^r$,

$$\eqalignno{\mu^{(A)}_x(B)
&=\mu_W\{\omega:\tau_A(\omega)<\infty,\,x+\omega(\tau_A(\omega))\in B\}\cr
\displaycause{where $\tau_A(\omega)$ is the Brownian hitting time
to $A-x$}
&\le\mu_W\{\omega:\omega^{-1}[B-x]\ne\emptyset\}
=\mu^{(B)}_x(\BbbR^r).\cr}$$

\noindent Multiplying by $\|x\|^{r-2}$ and taking the limit as
$\|x\|\to\infty$, $\lambda_AB\le\capacity B$.
}%end of proof of 479D

\leader{479E}{Theorem} (a) Newtonian capacity $\capacity$ is
submodular\cmmnt{ (definition:  432Jc)}.

(b) Suppose that $\sequencen{A_n}$ is a
non-decreasing sequence of analytic subsets of
$\BbbR^r$ with bounded union $A$.

\quad(i) The equilibrium measure $\lambda_A$ is the limit
$\lim_{n\to\infty}\lambda_{A_n}$ for the narrow topology on the space
$M^+_{\text{R}}(\BbbR^r)$ of
totally finite Radon measures on $\BbbR^r$.

\quad(ii) $\capacity A=\lim_{n\to\infty}\capacity A_n$.

\quad(iii) The equilibrium potential
$\tilde W_A$ is
$\lim_{n\to\infty}\tilde W_{A_n}=\sup_{n\in\Bbb N}\tilde W_{A_n}$.

(c) Suppose that $\sequencen{A_n}$ is a non-increasing sequence of bounded
analytic subsets of $\BbbR^r$ such that
$\bigcap_{n\in\Bbb N}A_n=\bigcap_{n\in\Bbb N}\overline{A_n}=A$ say.

\quad(i) $\lambda_A$ is the limit
$\lim_{n\to\infty}\lambda_{A_n}$ for the narrow topology on
$M^+_{\text{R}}(\BbbR^r)$.

\quad(ii) $\capacity A=\lim_{n\to\infty}\capacity A_n$.

(d)(i) Choquet-Newton capacity $c:\Cal P\BbbR^r\to[0,\infty]$ is the unique
outer regular Choquet capacity on $\BbbR^r$ extending $\capacity$.

\quad(ii) $c$ is submodular.

\quad(iii) $c(A)=\sup\{\capacity K:K\subseteq A\text{ is compact}\}$
for every analytic set $A\subseteq\BbbR^r$.

\proof{{\bf (a)} Let $A$, $B\subseteq\BbbR^r$ be bounded analytic sets.
If $x\in\BbbR^r$, then

\Centerline{$\hp((A\cup B)-x)+\hp((A\cap B)-x)
\le\hp(A-x)+\hp(B-x)$.}

\noindent\Prf\ For $C\subseteq\BbbR^r$ set

\Centerline{$H_C=\{\omega:\omega\in\Omega$,
there is some $t\ge 0$ such that $x+\omega(t)\in C\}$,}

\noindent so that if $C$ is an analytic set,
$\hp(C-x)=\mu_WH_C$.   Then

\Centerline{$H_{A\cup B}=H_A\cup H_B$,
\quad$H_{A\cap B}\subseteq H_A\cap H_B$,}

\noindent so

$$\eqalign{\hp((A\cup B)-x)+\hp((A\cap B)-x)
&=\mu_WH_{A\cup B}+\mu_WH_{A\cap B}\cr
&\le\mu_W(H_A\cup H_B)+\mu_W(H_A\cap H_B)\cr
&=\mu_WH_A+\mu_WH_B\cr
&=\hp(A-x)+\hp(B-x).  \text{ \Qed}\cr}$$

\noindent Consequently

$$\eqalign{\capacity(A\cup B)+\capacity(A\cap B)
&=\lambda_{A\cup B}(\BbbR^r)+\lambda_{A\cap B}(\BbbR^r)\cr
&=\lim_{\|x\|\to\infty}\|x\|^{r-2}
  (\hp((A\cup B)-x)+\hp((A\cap B)-x)\cr
&\le\lim_{\|x\|\to\infty}\|x\|^{r-2}
  (\hp(A-x)+\hp(B-x))\cr
&=\capacity A+\capacity B.\cr}$$

\noindent As $A$ and $B$ are arbitrary, $\capacity$ is submodular.

\medskip

{\bf (b)(i)} Let $f:\BbbR^r\to\Bbb R$ be any bounded continuous function.
For any $x\in\BbbR^r$,
$\int fd\mu_x^{(A)}=\lim_{n\to\infty}\int fd\mu_x^{(A_n)}$.
\Prf\ Let $\tau$, $\tau_n$ be the Brownian hitting times to $A-x$, $A_n-x$
respectively.   Observe that $\sequencen{\tau_n(\omega)}$ is non-increasing
and

\Centerline{$\tau(\omega)=\inf\{t:x+\omega(t)\in\bigcup_{n\in\Bbb N}A_n\}
=\lim_{n\to\infty}\tau_n(\omega)$}

\noindent for every $\omega\in\Omega$.
Set $H=\{\omega:\tau(\omega)<\infty\}$,
$H_n=\{\omega:\tau_n(\omega)<\infty\}$.   Then $\sequencen{H_n}$ is a
non-decreasing sequence with union $H$, and for $\omega\in H$

\Centerline{$f(x+\tau(\omega))=\lim_{n\to\infty}f(x+\tau_n(\omega))$}

\noindent because $f$ and $\omega$ are continuous.   Accordingly

$$\eqalign{\int fd\mu^{(A)}_x
&=\int_Hf(x+\omega(\tau(\omega)))\cr
&=\lim_{n\to\infty}\int_{H_n}f(x+\omega(\tau_n(\omega)))
=\lim_{n\to\infty}\int fd\mu^{(A_n)}_x. \text{ \Qed}\cr}$$

\woddheader{479E}{}{}{}{48pt}

Taking $\gamma>0$ such that $\overline{A}\subseteq\interior B(\tbf{0},\gamma)$,

$$\eqalignno{\int fd\lambda_A
&=\Bover1{r\beta_r\gamma}
   \int_{\partial B(\tbf{0},\gamma)}\int fd\mu_x^{(A)}\nu(dx)\cr
\displaycause{452F}
&=\Bover1{r\beta_r\gamma}\int_{\partial B(\tbf{0},\gamma)}
   \lim_{n\to\infty}\int fd\mu_x^{(A_n)}\nu(dx)\cr\cr
&=\lim_{n\to\infty}\Bover1{r\beta_r\gamma}
   \int_{\partial B(\tbf{0},\gamma)}\int fd\mu_x^{(A_n)}\nu(dx)
=\lim_{n\to\infty}\int fd\lambda_{A_n}.\cr}$$

\noindent As $f$ is arbitrary, $\lambda_A=\lim_{n\to\infty}\lambda_{A_n}$
for the narrow topology (437Kc).

\medskip

\quad{\bf (ii)} Taking $f=\chi\BbbR^r$ in (i), we see that
$\capacity A=\lim_{n\to\infty}\capacity A_n$.

\medskip

\quad{\bf (iii)} For any $x\in\BbbR^r$,

\Centerline{$\tilde W_A(x)
=\int\Bover1{\|y-x\|^{r-2}}\lambda_A(dy)
\le\liminf_{n\to\infty}\int\Bover1{\|y-x\|^{r-2}}\lambda_{A_n}(dy)$}

\noindent because $y\mapsto\Bover1{\|y-x\|^{r-2}}$ is non-negative and
continuous (437Jg).   As $\tilde W_{A_n}(x)\le\tilde W_A(x)$
for every $n$ (479D(b-ii)),
$\tilde W_A(x)=\lim_{n\to\infty}\tilde W_{A_n}(x)
=\sup_{n\in\Bbb N}\tilde W_{A_n}(x)$.

\medskip

{\bf (c)} Most of the ideas of (b) still work.   Again take
$f\in C_b(\BbbR^r)$.   Then
$\int fd\mu_x^{(A)}=\lim_{n\to\infty}\int fd\mu_x^{(A_n)}$
for any $x\in\BbbR^r$.
\Prf\ As before, let $\tau$, $\tau_n$ be the Brownian hitting times to
$A-x$, $A_n-x$
respectively.   This time, $\sequencen{\tau_n}$ is non-decreasing.
Let $\Omega'$ be the conegligible subset of $\Omega$ consisting of those
functions $\omega$ such that $\lim_{t\to\infty}\|\omega(t)\|=\infty$.
If $\omega\in\Omega'$ and $t=\lim_{n\to\infty}\tau_n(\omega)$ is finite,
then for every $n\in\Bbb N$ there is a $t_n\le t+2^{-n}$ such that
$x+\omega(t_n)\in A_n$.   Let $s\in[0,t]$ be a cluster point of
$\sequencen{t_n}$;  then $x+\omega(s)$ is a cluster point of
$\sequencen{x+\omega(t_n)}$, so belongs to
$\bigcap_{n\in\Bbb N}\overline{A}_n=A$, and $\tau(\omega)\le s\le t$.
Since $\tau(\omega)\ge\tau(\omega_n)$ for every $n$,
we have $\tau(\omega)=\lim_{n\to\infty}\tau(\omega_n)$.

Setting $H=\{\omega:\tau(\omega)<\infty\}$ and
$H_n=\{\omega:\tau_n(\omega)<\infty\}$, $\sequencen{H_n}$ is a
non-increasing sequence with intersection $H$, and for $\omega\in H$

\Centerline{$f(x+\tau(\omega))=\lim_{n\to\infty}f(x+\tau_n(\omega))$.}

\noindent So once again

$$\eqalign{\int fd\mu^{(A)}_x
&=\int_Hf(x+\omega(\tau(\omega)))\mu_W(d\omega)\cr
&=\lim_{n\to\infty}\int_{H_n}f(x+\omega(\tau_n(\omega)))\mu_W(d\omega)
=\lim_{n\to\infty}\int fd\mu^{(A_n)}_x. \text{ \Qed}\cr}$$

\noindent The rest of the argument follows (b-i) and (b-ii) unchanged.

\medskip

{\bf (d)(i)} I seek to apply 432Lb.

\medskip

\qquad\grheada\ Let $\Cal K$ be the family of compact
subsets of $\BbbR^r$ and set $c_1=\capacity\restr\Cal K$.   By
(a), $c_1$ is submodular.
If $G\subseteq\BbbR^r$ is a bounded open set, then it is
expressible as the union of a
non-decreasing sequence of compact sets, so by (b-ii) we
have $\capacity G=\sup\{c_1(L):L\in\Cal K$, $L\subseteq G\}$;  and if
$K\in\Cal K$, there is a non-increasing sequence $\sequencen{G_n}$ of
bounded open sets such that
$K=\bigcap_{n\in\Bbb N}G_n=\bigcap_{n\in\Bbb N}\overline{G}_n$, and now
$c_1(K)=\lim_{n\to\infty}\capacity G_n$, by (c-ii).   But this means that

\Centerline{$c_1(K)
\le\inf_{G\supseteq K\text{ is open}}
  \sup_{L\subseteq G\text{ is compact}}c_1(L)
\le\inf_{n\in\Bbb N}\capacity G_n
=c_1(K)$.}

\noindent So all the conditions of 432Lb are satisfied, and $c$, as defined
in 479C(a-ii), is the unique extension of $c_1$
to an outer regular Choquet capacity on $\BbbR^r$.

\medskip

\qquad\grheadb\ Now $c(A)=\capacity A$ for every bounded analytic set
$A\subseteq\BbbR^r$.   \Prf\

$$\eqalignno{c(A)
&=\sup_{K\subseteq A\text{ is compact}}c(K)\cr
\displaycause{432K}
&=\sup_{K\subseteq A\text{ is compact}}\capacity K
\le\capacity A
\le\inf_{G\supseteq A\text{ is bounded and open}}\capacity G\cr
&=\mathop{\text{inf}\hbox{\vrule depth2pt width0pt}}
  _{G\supseteq A\text{ is open}}
   \,\,\sup_{L\subseteq G\text{ is compact}}c(L)
=c(A).  \text{ \Qed}\cr}$$

\noindent So $c$ extends $\capacity$, as claimed, and must be the unique
outer regular Choquet capacity doing so.

\medskip

\quad{\bf (ii)-(iii)} By 432Lb, $c$ is submodular;  and (iii) is covered by
the argument in (i-$\beta$).
}%end of proof of 479E

\leader{479F}{}\cmmnt{ I now wish to describe an entirely different
characterization of the capacity and equilibrium measure of a compact set,
which demands a substantial investment in harmonic analysis (down to
479I) and an excursion into Fourier analysis (479H).   I begin with general
remarks about Newtonian potentials.

\medskip

\noindent}{\bf Theorem} Let $\zeta$ be a totally finite Radon measure on
$\BbbR^r$, and set $G=\BbbR^r\setminus\supp\zeta$\cmmnt{, where
$\supp\zeta$ is the support of $\zeta$ (411Nd)}.   Let $W_{\zeta}$ be the
Newtonian potential associated with $\zeta$.

(a) $W_{\zeta}:\BbbR^r\to[0,\infty]$ is lower semi-continuous,
and $W_{\zeta}\restr G:G\to\coint{0,\infty}$ is continuous.

(b) $W_{\zeta}$ is superharmonic, and $W_{\zeta}\restr G$ is harmonic.

(c) $W_{\zeta}$ is locally $\mu$-integrable;  in particular, it is finite
$\mu$-a.e.

(d) If $\zeta$ has compact support, then
$\zeta\BbbR^r=\lim_{\|x\|\to\infty}\|x\|^{r-2}W_{\zeta}(x)$.

(e) If $W_{\zeta}\restr\supp\zeta$ is continuous then $W_{\zeta}$ is
continuous.

(f) If $K$ is a compact set such that $W_{\zeta}\restr K$ is continuous
and finite-valued then $W_{\zeta\LLcorner K}$ is continuous.

(g) If $W_{\zeta}$ is finite $\zeta$-a.e.\ and
$f:\BbbR^r\to[0,\infty]$ is a lower semi-continuous superharmonic
function such that $f\ge W_{\zeta}\,\,\zeta$-a.e., then $f\ge W_{\zeta}$.
%index as `domination principle'

(h) If $\zeta'$ is another Radon measure on $\BbbR^r$ and
$\zeta'\le\zeta$, then $W_{\zeta'}\le W_{\zeta}$ and
$\energy(\zeta')\le\energy(\zeta)$.

%If $h:\BbbR^r\to[0,\infty]$ a Lebesgue integrable function, then
%$W_{\zeta*h\mu}=W_{\zeta}*h$, where $h\mu$ is the indefinite-integral
%measure over $\mu$ defined by $h$ and
%the convolution $\zeta*h\mu$ is defined as in 444A or 257A.
%see 479Q

\proof{{\bf (a)}
If $\sequencen{x_n}$ is a convergent sequence in
$\BbbR^r$ with limit $x$, then
$\Bover1{\|y-x\|^{r-2}}=\lim_{n\to\infty}\Bover1{\|y-x_n\|^{r-2}}$
for every $y$
(counting $\Bover10$ as $\infty$, as usual), so that
$W_{\zeta}(x)\le\liminf_{n\to\infty}W_{\zeta}(x_n)$, by Fatou's Lemma.   As $x$ and
$\sequencen{x_n}$ are arbitrary, $W_{\zeta}$ is lower semi-continuous.

If $x\in G$, then
$\Bover1{\|y-x_n\|^{r-2}}\le\Bover2{\rho(x,\supp\zeta)^{r-2}}$ for all $n$
large enough and all $y\in\supp\zeta$, so Lebesgue's Dominated Convergence Theorem
tells us that $W_{\zeta}(x)=\lim_{n\to\infty}W_{\zeta}(x_n)$ and that
$W_{\zeta}$ is continuous at $x$, as well as finite-valued there.

\medskip

{\bf (b)} If $x\in\BbbR^r$ and $\delta>0$, then

$$\eqalign{\Bover1{\nu(\partial B(x,\delta))}\int_{\partial B(x,\delta)}
  W_{\zeta}d\nu
&=\Bover1{\nu(\partial B(x,\delta))}\int_{\partial B(x,\delta)}
  \int_{\BbbR^r}\Bover1{\|z-y\|^{r-2}}\zeta(dz)\nu(dy)\cr
&=\int_{\BbbR^r}\Bover1{\nu(\partial B(x,\delta))}
  \int_{\partial B(x,\delta)}\Bover1{\|z-y\|^{r-2}}\nu(dy)\zeta(dz)\cr
&=\int_{\BbbR^r}\Bover1{\nu(\partial B(\tbf{0},\delta))}
  \int_{\partial B(\tbf{0},\delta)}\Bover1{\|z-x-y\|^{r-2}}\nu(dy)\zeta(dz)\cr
&\ge\int_{\BbbR^r}\Bover1{\|z-x\|^{r-2}}\zeta(dz)
=W_{\zeta}(x)\cr}$$

\noindent (478Ga) with equality if $B(x,\delta)$ does not meet
$\supp\zeta$, since then $z-x\notin B(\tbf{0},\delta)$ and

\Centerline{$\Bover1{\nu(\partial B(x,\delta))}
  \int_{\partial B(x,\delta)}\Bover1{\|z-x-y\|^{r-2}}\nu(dy)
=\Bover1{\|z-x\|^{r-2}}$}

\noindent for $\zeta$-almost every $z$.

\medskip

{\bf (c)} For any $\gamma>0$ and $y\in\BbbR^r$,
$\biggerint_{B(\tbf{0},\gamma)}\Bover1{\|y-x\|^{r-2}}\mu(dx)
\le\Bover12r\beta_r\gamma^2$ (478Gc), so

$$\eqalign{\int_{B(\tbf{0},\gamma)}W_{\zeta}d\mu
&=\int_{B(\tbf{0},\gamma)}\int_{\BbbR^r}\Bover1{\|y-x\|^{r-2}}\zeta(dy)\mu(dx)\cr
&=\int_{\BbbR^r}\int_{B(\tbf{0},\gamma)}\Bover1{\|y-x\|^{r-2}}\mu(dx)\zeta(dy)
\le\Bover12r\beta_r\gamma^2\zeta\BbbR^r\cr}$$

\noindent is finite.

\medskip

{\bf (d)} If $\zeta$ has compact support, there is an $\gamma>0$ such that
$\supp\zeta\subseteq B(\tbf{0},\gamma)$.
In this case, for $\|x\|>\gamma$, we have

\Centerline{$\Bover{\|x\|^{r-2}}{(\|x\|+\gamma)^{r-2}}\zeta\BbbR^r
\le\int_{\BbbR^r}\Bover{\|x\|^{r-2}}{\|x-y\|^{r-2}}\zeta(dy)
=W_{\zeta}(x)\|x\|^{r-2}
\le\Bover{\|x\|^{r-2}}{(\|x\|-\gamma)^{r-2}}\zeta\BbbR^r$}

\noindent so all the terms converge to $\zeta\BbbR^r$ as
$\|x\|\to\infty$.

\medskip

{\bf (e)} Since $W_{\zeta}$ is lower semi-continuous, it will be
enough to show that $H=\{x:W_{\zeta}(x)<\gamma\}$ is open for every
$\gamma\in\Bbb R$.   Take $x_0\in H$.
If $x_0\notin\supp\zeta$ then $W_{\zeta}$ is continuous at $x_0$,
by 479Fa, and
$H$ is certainly a neighbourhood of $x_0$.   If $x_0\in\supp\zeta$
take $\eta\in\ooint{0,2^{-r}(\gamma-W_{\zeta}(x_0))}$.
Because $W_{\zeta}(x_0)$ is finite, $\zeta\{x_0\}=0$;
because $W_{\zeta}\restr\supp\zeta$ is continuous, there is a $\delta>0$
such that
$\biggerint_{B(x_0,\delta)}\Bover1{\|x_0-y\|^{r-2}}\zeta(dy)\le\eta$ and
$|W_{\zeta}(x)-W_{\zeta}(x_0)|\le\eta$ whenever
$x\in B(x_0,\delta)\cap\supp\zeta$.   Let
$\delta'\in\ooint{0,\delta}$ be such that

\Centerline{$\bigl|\Bover1{\|x-y\|^{r-2}}-\Bover1{\|x_0-y\|^{r-2}}\bigr|
   \le\Bover{\eta}{\zeta\BbbR^r}$}

\noindent whenever $\|x-x_0\|\le\delta'$ and $\|y-x_0\|\ge\delta$;
then

\Centerline{$
|\int_{\BbbR^r\setminus B(x_0,\delta)}\Bover1{\|x-y\|^{r-2}}\zeta(dy)
  -\int_{\BbbR^r\setminus B(x_0,\delta)}\Bover1{\|x_0-y\|^{r-2}}\zeta(dy)|
\le\eta$}

\noindent whenever $x\in B(x_0,\delta')$.

Take $x\in B(x_0,\bover12\delta')$, and let
$z\in B(x_0,\delta)\cap\supp\zeta$ be such that
$\|x-z\|=\rho(x,B(x_0,\delta)\cap\supp\zeta)$.
We have $\|x-z\|\le\|x-x_0\|$ so $\|z-x_0\|\le 2\|x-x_0\|\le\delta'$.
If $y\in B(x_0,\delta)\cap\supp\zeta$, then
$\|y-z\|\le\|x-y\|+\|x-z\|\le 2\|x-y\|$;  so

$$\eqalign{\int_{B(x_0,\delta)}\Bover1{\|x-y\|^{r-2}}\zeta(dy)
&\le 2^{r-2}\int_{B(x_0,\delta)}\Bover1{\|z-y\|^{r-2}}\zeta(dy)\cr
&=2^{r-2}(W_{\zeta}(z)
  -\int_{\BbbR^r\setminus B(x_0,\delta)}\Bover1{\|z-y\|^{r-2}}\zeta(dy))\cr
&\le 2^{r-2}(2\eta+W_{\zeta}(x_0)
  -\int_{\BbbR^r\setminus B(x_0,\delta)}
     \Bover1{\|x_0-y\|^{r-2}}\zeta(dy))\cr
&=2^{r-1}\eta+2\int_{B(x_0,\delta)}\Bover1{\|x_0-y\|^{r-2}}\zeta(dy)\cr
&\le 2^{r-1}\eta+2\eta\le(2^r-1)\eta,\cr}$$

$$\eqalign{W_{\zeta}(x)
&\le (2^r-1)\eta+\int_{\BbbR^r\setminus B(x_0,\delta)}
  \Bover1{\|x-y\|^{r-2}}\zeta(dy)\cr
&\le 2^r\eta+\int_{\BbbR^r\setminus B(x_0,\delta)}
  \Bover1{\|x_0-y\|^{r-2}}\zeta(dy)
\le 2^r\eta+W_{\zeta}(x_0)
<\gamma.\cr}$$

\noindent Thus $B(x_0,\bover12\delta')\subseteq H$ and again $H$ is a
neighbourhood of $x_0$.   As $x_0$ is arbitrary, $H$ is open;  as
$\gamma$ is arbitrary, $W_{\zeta}$ is continuous.

\medskip

{\bf (f)} Setting $H=\BbbR^r\setminus K$,
$\zeta=\zeta\LLcorner K+\zeta\LLcorner H$, so
$W_{\zeta}=W_{\zeta\LLcorner K}+W_{\zeta\LLcorner H}$
(234Hc\formerly{2{}12Xh}).   Now
$W_{\zeta\LLcorner K}$ and $W_{\zeta\LLcorner H}$ are both lower
semi-continuous and non-negative,
so if $W_{\zeta}\restr K$ is continuous and finite-valued
then $W_{\zeta\LLcorner K}\restr K$
is continuous (4A2B(d-ix)).   Since $\supp(\zeta\LLcorner K)\subseteq K$,
(e) tells us that $W_{\zeta\LLcorner K}$ is continuous.

\medskip

{\bf (g)} \Quer\ Suppose that $f(x_0)<W_{\zeta}(x_0)$.   Since
$\{x:f(x)\ge W_{\zeta}(x)$, $W_{\zeta}(x)<\infty\}$ is
$\zeta$-conegligible, and $W_{\zeta}$ is measurable therefore
$\zeta$-almost continuous (418J), there is a compact set $K$ such that
$W_{\zeta}(x)<\infty$ and $W_{\zeta}(x)\le f(x)$ for every $x\in K$,
$W_{\zeta}\restr K$ is continuous and
$\biggerint_K\Bover1{\|x_0-y\|^{r-2}}\zeta(dy)>f(x_0)$.   Set
$\zeta'=\zeta\LLcorner K$.   By (f), $W_{\zeta'}$ is
continuous;  $W_{\zeta'}(x_0)>f(x_0)$;  and
$f(x)\ge W_{\zeta}(x)\ge W_{\zeta'}(x)$ for every
$x\in K\supseteq\supp\zeta'$.

Set $g=f-W_{\zeta'}$ and $\alpha=\inf_{x\in\BbbR^r}g(x)<0$.
Because $f$ is lower semi-continuous and $W_{\zeta'}$ is continuous,
$g$ is lower semi-continuous;  because $\zeta'$ has compact
support, $\lim_{\|x\|\to\infty}W_{\zeta'}(x)=0$ ((d) above) and
$\liminf_{\|x\|\to\infty}g(x)\ge 0$;  so
$L=\{x:g(x)=\alpha\}$ is non-empty and compact.   Note that
$L$ is disjoint from $K$.
Let $x_1\in L$ be a point of maximum norm.   Then $x_1\notin K$, while
$W_{\zeta'}\restr\BbbR^r\setminus K$ is harmonic ((b) above).
Let $\delta>0$
be such that $B(x_1,\delta)\cap K=\emptyset$.   Then we have

\Centerline{$\Bover1{\nu(\partial B(x_1,\delta))}
   \int_{\partial B(x_1,\delta)}g\,d\nu
>\alpha$}

\noindent because $g(x)\ge\alpha$ for every $x$ and
$g(x)>\alpha$ whenever $x\ne x_1$ and
$(x-x_1)\dotproduct x_1\ge 0$.   But we also have

$$\eqalign{\Bover1{\nu(\partial B(x_1,\delta))}
   \int_{\partial B(x_1,\delta)}g\,d\nu
&=\Bover1{\nu(\partial B(x_1,\delta))}
   \int_{\partial B(x_1,\delta)}f\,d\nu
   -\Bover1{\nu(\partial B(x_1,\delta))}
     \int_{\partial B(x_1,\delta)}W_{\zeta'}\,d\nu\cr
&\le f(x_1)-W_{\zeta'}(x_1)
=\alpha,\cr}$$

\noindent which is impossible.\ \Bang

\medskip

{\bf (h)} By 234Qc, $W_{\zeta'}\le W_{\zeta}$;  so

\Centerline{$\energy(\zeta')=\int W_{\zeta'}d\zeta'
\le\int W_{\zeta}d\zeta'\le\int W_{\zeta}d\zeta=\energy(\zeta)$.}

}%end of proof of 479F

\leader{479G}{}\cmmnt{ At this point
I embark on an extended parenthesis, down to 479I, covering
some essential material from harmonic analysis and Fourier analysis.
The methods here apply equally to the cases $r=1$ and $r=2$.

\medskip

\noindent}{\bf Lemma} (In this result, $r$ may be any integer greater than
or equal to $1$.)   For $\alpha\in\Bbb R$, set
$k_{\alpha}(x)=\Bover1{\|x\|^{\alpha}}$ for $x\in\BbbR^r\setminus\{0\}$.
If $\alpha<r$, $\beta<r$ and $\alpha+\beta>r$, then
$k_{\alpha+\beta-r}$ is a constant multiple of the convolution
$k_{\alpha}*k_{\beta}$\cmmnt{ (definition:  255E, 444O)}.

\proof{{\bf (a)} First note that

\Centerline{$\int_{B(\tbf{0},1)}k_{\alpha}(x)\mu(dx)
=r\beta_r\int_0^1\Bover{t^{r-1}}{t^{\alpha}}dt
=\Bover{r\beta_r}{r-\alpha}$}

\noindent is finite.   Consequently $k_{\alpha}$ is expressible as the sum
of an integrable function and a bounded function, and in particular is
locally integrable.

\medskip

{\bf (b)} For $x\in\BbbR^r$, set
$f(x)=\int_{\BbbR^r}k_{\alpha}(x-y)k_{\beta}(x-y)\mu(dy)\in[0,\infty]$.
If $e\in\BbbR^r$ is a unit vector, then $f(e)$ is finite.
\Prf\ For any $y\in\BbbR^r$,
at least one of $\|e-y\|$, $\|e+y\|$ is greater than or equal to $1$, so
$k_{\alpha}(e-y)k_{\beta}(e+y)\le k_{\alpha}(e-y)+k_{\beta}(e+y)$.
Consequently

\Centerline{$\int_{B(\tbf{0},2)}k_{\alpha}(e-y)k_{\alpha}(e+y)\mu(dy)
\le\int_{B(\tbf{0},2)}k_{\alpha}(e-y)+k_{\alpha}(e+y)\mu(dy)$}

\noindent is finite.   On the other hand, if $\|y\|\ge 2$,
$\|e-y\|$ and $\|e-y\|$ are both at least $\bover12\|y\|$, so

$$\eqalign{\int_{\BbbR^r\setminus B(\tbf{0},2)}
  k_{\alpha}(e-y)k_{\beta}(e+y)\mu(dy)
&\le\int_{\BbbR^r\setminus B(\tbf{0},2)}\Bover{2^{\alpha}}{\|y\|^{\alpha}}
  \cdot\Bover{2^{\beta}}{\|y\|^{\beta}}\mu(dy)\cr
&=2^{\alpha+\beta}r\beta_r\int_2^{\infty}
   \Bover{t^{r-1}}{t^{\alpha+\beta}}dt
=\Bover{2^rr\beta_r}{\alpha+\beta-r}\cr}$$

\noindent is finite.   Putting these together,
$f(e)=\int_{\BbbR^r}k_{\alpha}(e-y)k_{\beta}(e-y)\mu(dy)$ is finite.\
\Qed

\medskip

{\bf (c)} If $e$, $e'\in\BbbR^r$ are unit vectors, then $f(e)=f(e')$.
\Prf\ Let $T:\BbbR^r\to\BbbR^r$ be an orthogonal transformation such that
$Te=e'$.   Then

$$\eqalignno{f(e')
&=\int_{\BbbR^r}k_{\alpha}(Te-y)k_{\beta}(Te-y)\mu(dy)
=\int_{\BbbR^r}k_{\alpha}(Te-Ty)k_{\beta}(Te-Ty)\mu(dy)\cr
\displaycause{because $T$ is an automorphism of $(\BbbR^r,\mu)$}
&=\int_{\BbbR^r}k_{\alpha}(e-y)k_{\beta}(e-y)\mu(dy)\cr
\displaycause{because $k_{\alpha}(x)$, $k_{\beta}(x)$ are functions of
$\|x\|$}
&=f(e).  \text{ \Qed}\cr}$$

Let $c$ be the constant value of $f(e)$ for $\|e\|=1$.

\woddheader{479G}{4}{2}{2}{72pt}

{\bf (d)} If $x\in\BbbR^r\setminus\{0\}$,
$f(x)=\Bover{c}{\|x\|^{\alpha+\beta-r}}$.   \Prf\ Set $t=\|x\|$,
$e=\bover1tx$.   Then

$$\eqalignno{f(x)
&=\int_{\BbbR^r}k_{\alpha}(te-y)k_{\beta}(te-y)\mu(dy)
=\int_{\BbbR^r}t^rk_{\alpha}(te-tz)k_{\beta}(te-tz)\mu(dz)\cr
\displaycause{substituting $y=tz$}
&=t^{r-\alpha-\beta}\int_{\BbbR^r}k_{\alpha}(e-z)k_{\beta}(e-z)\mu(dz)
=\Bover{c}{\|x\|^{\alpha+\beta-r}}. \text{ \Qed}\cr}$$

\medskip

{\bf (e)} If $x\in\BbbR^r\setminus\{0\}$,
$(k_{\alpha}*k_{\beta})(x)=2^{\alpha+\beta-r}ck_{\alpha+\beta-r}(x)$.
\Prf\ Set $z=\bover12x$.   Then

$$\eqalign{(k_{\alpha}*k_{\beta})(x)
&=\int_{\BbbR^r}\Bover1{\|x-y\|^{\alpha}\|y\|^{\beta}}\mu(dy)
=\int_{\BbbR^r}\Bover1{\|x-y-z\|^{\alpha}\|y+z\|^{\beta}}\mu(dy)\cr
&=\int_{\BbbR^r}\Bover1{\|z-y\|^{\alpha}\|z+y\|^{\beta}}\mu(dy)
=\Bover{c}{\|z\|^{\alpha+\beta-r}}
=2^{\alpha+\beta-r}ck_{\alpha+\beta-r}(x).  \text{ \Qed}\cr}$$

\medskip

{\bf (f)} Of course it is of no importance what happens at $0$, but for
completeness:  $k_{\alpha+\beta-r}$ is declared to be undefined there, and
$\biggerint_{\BbbR^r}\Bover1{\|y\|^{\alpha}\|y\|^{\beta}}\mu(dy)$ is
infinite for
any $\alpha$ and $\beta$, so $k_{\alpha}*k_{\beta}$ also is undefined at
$0$ on the convention of 255E or 444O.   Thus we have
$k_{\alpha}*k_{\beta}=2^{\alpha+\beta-r}ck_{\alpha+\beta-r}$ in the strict
sense.
}%end of proof of 479G

\medskip

\noindent{\bf Remark}\cmmnt{ The functions $k_{\alpha}$ are called
{\bf Riesz kernels}.   It will be helpful later to have a name
for the constant arising here in a special case.}   If $r\ge 3$,
I will take $c_r>0$
to be the constant such that $c_rk_{r-2}=k_{r-1}*k_{r-1}$.

\leader{479H}{}\cmmnt{ Now for some Fourier analysis which wasn't
quite reached in Chapter 28.   In the following, I will define the Fourier
transform $\varhatf$ and the inverse Fourier transform
$\varcheckf$, for $\mu$-measurable complex-valued functions $f$ defined
$\mu$-almost everywhere in $\BbbR^r$, as in 283W and 284W, and
$\varhat{\zeta}$, for a totally finite Radon measure $\zeta$ on $\BbbR^r$,
by the formula offered in 285Ya for probability measures.   The convolution
$\zeta*f$ of a measure and a function will be defined as in 444H.   Thus
the basic formulae are

\Centerline{$\varhatf(y)=\Bover1{(\sqrt{2\pi})^r}
  \int_{\BbbR^r}e^{-iy\dotproduct x}f(x)\mu(dx)$,
\quad$\varcheckf(y)=\Bover1{(\sqrt{2\pi})^r}
  \int_{\BbbR^r}e^{iy\dotproduct x}f(x)\mu(dx)$}

\noindent for $\mu$-integrable $f$,

\Centerline{$\varhat{\zeta}(y)=\Bover1{(\sqrt{2\pi})^r}
  \int_{\BbbR^r}e^{-iy\dotproduct x}\zeta(dx)$,
\quad$(\zeta*f)(x)
  =\int_{\BbbR^r}f(x-y)\zeta(dy)$.}

\medskip

\noindent}{\bf Theorem} (In this result, $r$ may be any integer greater
than or equal to $1$.)   Let $\zeta$ be a totally finite Radon measure on
$\BbbR^r$\dvro{ and $\varhat{\zeta}$ its Fourier transform}{}.

(a) If $f\in\eusm L^1_{\Bbb C}(\mu)$, then $\zeta*f$ is
$\mu$-integrable and
$(\zeta*f)\varsphat=(\sqrt{2\pi})^r\varhat{\zeta}\times\varhatf$.

(b) If $\zeta$ has compact support and
$h:\BbbR^r\to\Bbb C$ is a rapidly decreasing test
function\cmmnt{ (284Wa)},
then $\zeta*h$ and $h\times\varhat{\zeta}$ are rapidly decreasing test
functions.

(c) Suppose that $f$ is a tempered function on $\BbbR^r$\cmmnt{ (284Wa)}.
If {\it either} $\zeta$ has compact support {\it or} $f$ is expressible as
the sum of a $\mu$-integrable function and a bounded function, then
$\zeta*f$ is defined $\mu$-almost everywhere and is a tempered function.

(d) Suppose that $f$, $g$ are tempered functions on
$\BbbR^r$ such that $g$ represents the Fourier transform
of $f$\cmmnt{ (284Wd)}.   If {\it either} $\zeta$ has compact support
{\it or} $f$ is expressible as the sum of a bounded function and a
$\mu$-integrable function, then
$(\sqrt{2\pi})^r\varhat{\zeta}\times g$ represents the Fourier transform of
$\zeta*f$.

\proof{{\bf (a)(i)} To begin with, suppose that $f$ is real-valued and
non-negative.   As in \S444, I will write $f\mu$ for the
indefinite-integral measure defined by $f$ over $\mu$.
By 444K, $\zeta*f$ is $\mu$-integrable and $(\zeta*f)\mu=\zeta*f\mu$.

As the formula used here for $\varhat{\zeta}$ does not
quite match that of 445C, whatever parametrization we use for the
characters of the topological group $\BbbR^r$, I had better not try to
quote Chapter 44 when discussing Fourier transforms.
Going back to first principles,

$$\eqalignno{(\zeta*f)\varsphat(y)
&=\Bover1{(\sqrt{2\pi})^r}\int e^{-iy\dotproduct x}(\zeta*f)(x)\mu(dx)
=\Bover1{(\sqrt{2\pi})^r}\int e^{-iy\dotproduct x}(\zeta*f)\mu(dx)\cr
&=\Bover1{(\sqrt{2\pi})^r}\int e^{-iy\dotproduct x}(\zeta*f\mu)(dx)
=\Bover1{(\sqrt{2\pi})^r}\iint e^{-iy\dotproduct(x+z)}
  \zeta(dz)(f\mu)(dx)\cr
\displaycause{444C}
&=\Bover1{(\sqrt{2\pi})^r}\int e^{-iy\dotproduct z}\zeta(dz)
   \int e^{-iy\dotproduct x}(f\mu)(dx)\cr
&=\varhat{\zeta}(y)\int e^{-iy\dotproduct x}f(x)\mu(dx)
=(\sqrt{2\pi})^r\varhat{\zeta}(y)\varhatf(y)\cr}$$

\noindent for every $y\in\BbbR^r$.

\medskip

\quad{\bf (ii)} For general integrable complex-valued functions $f$,
apply (i) to the positive and negative parts of the real and imaginary
parts of $f$.

\medskip

{\bf (b)(i)} Because $h$ is continuous, so is $\zeta*h$ (444Ib).
If $j<r$, then, as in 123D,

\Centerline{$(\Pd{}{\xi_j}(\zeta*h))(x)
=\Pd{}{\xi_j}\int h(x-y)\zeta(dy)
=\int\Pd{}{\xi_j}h(x-y)\zeta(dy)
=(\zeta*\Pd{h}{\xi_j})(x)$}

\noindent because $\Pd{h}{\xi_j}$ is bounded.
Since $\Pd{h}{\xi_j}$ is again a \rdtf,
we can repeat this process
to see that $\zeta*h$ is smooth.   Next, let $\gamma>0$ be such that
the support of $\zeta$ is included in $B(\tbf{0},\gamma)$.
If $k\in\Bbb N$, then
$M=\sup_{x\in\BbbR^r}(\gamma^k+\|x\|^k)|h(x)|$ is finite, and

\Centerline{$\|x\|^k|h(y)|\le(\|y\|+\gamma)^k|h(y)|
\le 2^kM$}

\noindent whenever $\|x-y\|\le\gamma$.   So

\Centerline{$\|x\|^k|(\zeta*h)(x)|
\le\zeta\BbbR^r\cdot\|x\|^k\sup_{\|y-x\|\le\gamma}|h(y)|
\le 2^kM\zeta\BbbR^r$}

\noindent for every $x\in\BbbR^r$.   Applying this to all the partial
derivatives of $h$, we see that $\zeta*h$ is a \rdtf.

\medskip

\quad{\bf (ii)} Again suppose that $j<r$.
Because $\zeta$ has compact support, $\int\|x\|\zeta(dx)$ is finite, so
$\Pd{}{\eta_j}\varhat{\zeta}(y)$ is defined and equal to
$-i\int\xi_je^{-iy\dotproduct x}\zeta(dx)$ for every $y\in\BbbR^r$
(cf.\ 285Fd).   More generally, whenever $j_1,\ldots,j_m<r$,

\Centerline{$
\Bover{\partial^m}{\partial\eta_{j_1}\ldots\partial\eta_{j_m}}
  \varhat{\zeta}(y)
=(-i)^m\int\xi_{j_1}\ldots\xi_{j_m}e^{-iy\dotproduct x}\zeta(dx)$,}

\noindent so all the partial derivatives of $\varhat{\zeta}$
are defined everywhere and bounded.   It follows that $\varhat{\zeta}$ is
smooth and $h\times\varhat{\zeta}$ is a \rdtf.

\medskip

{\bf (c)(i)} To begin with, suppose that $f$ is real and non-negative, and
that $\zeta$ has compact support.
Set $f_n=f\times\chi B(\tbf{0},n)$ for each $n\in\Bbb N$.   Then $f_n$ is
integrable, so $\zeta*f_n$ is defined $\mu$-a.e.;  also
$\sequencen{\zeta*f_n}$ is non-decreasing, and
$(\zeta*f)(x)=\sup_{n\in\Bbb N}(\zeta*f_n)(x)$ whenever the latter is
defined and finite.

Let $\gamma\ge 1$ be such that the support of $\zeta$ is included in
$B(\tbf{0},\gamma)$, and let $k\in\Bbb N$ be such that
$\biggerint_{\BbbR^r}\Bover1{1+\|x\|^k}f(x)\mu(dx)$ is finite.
If $y\in B(\tbf{0},\gamma)$ and $x\in\BbbR^r$, then
$\|x\|\le 2\max(\gamma,\|x+y\|)$, so
$\Bover1{1+\|x+y\|^k}\le\Bover{M}{1+\|x\|^k}$, where $M=1+2^k\gamma^k$.
It follows that

$$\eqalign{\int_{\BbbR^r}\Bover1{1+\|x\|^k}(\zeta*f_n)(x)\mu(dx)
&=\int_{\BbbR^r}\int_{B(\tbf{0},\gamma)}\Bover1{1+\|x\|^k}
  f_n(x-y)\zeta(dy)\mu(dx)\cr
&=\int_{B(\tbf{0},\gamma)}\int_{\BbbR^r}\Bover1{1+\|x\|^k}
  f_n(x-y)\mu(dx)\zeta(dy)\cr
&=\int_{B(\tbf{0},\gamma)}\int_{\BbbR^r}\Bover1{1+\|x+y\|^k}
  f_n(x)\mu(dx)\zeta(dy)\cr
&\le M\int_{B(\tbf{0},\gamma)}\int_{\BbbR^r}\Bover1{1+\|x\|^k}
  f(x)\mu(dx)\zeta(dy)}$$

\noindent for every $n\in\Bbb N$.   Consequently
$\biggerint_{\BbbR^r}\Bover1{1+\|x\|^k}(\zeta*f)(x)\mu(dx)$ is defined and
finite.

\medskip

\quad{\bf (ii)} Now suppose that $f$ is expressible as $f_1+f_{\infty}$,
where $f_1$ is $\mu$-integrable, $f_{\infty}$ is bounded and both are
real-valued and non-negative.
Adjusting $f_1$ and $f_2$ on a $\mu$-negligible set if necessary, we can
suppose that $f_{\infty}$ is Borel measurable and defined everywhere on
$\BbbR^r$.
By (a), $\zeta*f_1$ is defined and $\mu$-integrable.   Next,
$\zeta*f_{\infty}$
is defined everywhere, is bounded, and is Borel measurable (444Ia).    So
$\zeta*f\eae\zeta*f_1+\zeta*f_{\infty}$ is the sum of
a $\mu$-integrable function and a bounded Borel measurable function, and is
tempered.

\medskip

\quad{\bf (iii)} These arguments deal with the case in which $f\ge 0$.
For the general case, apply (i) or (ii) to the four parts of $f$,
as in (a-ii).

\medskip

{\bf (d)(i)} Suppose to begin with that $\zeta$ has compact support.
Let $h$ be a \rdtf.
Set $\Reverse{h}(x)=h(-x)$ for every $x\in\BbbR^r$.   Then $\Reverse{h}$
is a \rdtf, and

\Centerline{$(\zeta*\Reverse{h})(-x)
=\int\Reverse{h}(-x-y)\zeta(dy)
=\int h(x+y)\zeta(dy)$}

\noindent for every $x\in\BbbR^r$.   Accordingly

$$\eqalignno{\int(\zeta*f)\times h\,d\mu
&=\iint h(x)f(x-y)\zeta(dy)\mu(dx)\cr
&=\iint h(x)f(x-y)\mu(dx)\zeta(dy)\cr
\displaycause{because $\zeta*|f|$ is tempered, so
$\iint|h(x)f(x-y)|\zeta(dy)\mu(dx)=\int|h|\times(\zeta*|f|)d\mu$ is finite}
&=\iint h(x+y)f(x)\mu(dx)\zeta(dy)\cr
&=\iint h(x+y)f(x)\zeta(dy)\mu(dx)\cr
\displaycause{because $\iint|h(x+y)f(x)|\mu(dx)\zeta(dy)
=\iint|h(x)f(x-y)|\mu(dx)\zeta(dy)$ is finite}
&=\int f\times(\zeta*\Reverse{h})^{\scriptscriptstyle\leftrightarrow}
  d\mu
=\int g\times((\zeta*\Reverse{h})^{\scriptscriptstyle\leftrightarrow})
  \varspcheck d\mu\cr
\displaycause{because $\zeta*\Reverse{h}$ and
$(\zeta*\Reverse{h})^{\scriptscriptstyle\leftrightarrow}$ are \rdtf s,
by (b)}
&=\int g\times(\zeta*\Reverse{h})\varsphat d\mu
=(\sqrt{2\pi})^r\int g\times\varhat{\zeta}\times(\Reverse{h})\varsphat
  d\mu\cr
\displaycause{by (a)}
&=(\sqrt{2\pi})^r\int g\times\varhat{\zeta}\times\varcheck{h}\,d\mu.\cr}$$

\noindent As $h$ is arbitrary, $(\sqrt{2\pi})^rg\times\varhat{\zeta}$
represents the Fourier transform of $\zeta*f$.

\medskip

\quad{\bf (ii)} Now suppose that $f$ is expressible as $f_1+f_{\infty}$
where $f_1$ is $\mu$-integrable and $f_{\infty}$ is bounded.   By (c),
$\zeta*|f|$ is defined almost everywhere and is a tempered function.
Set $\zeta_n=\zeta\LLcorner B(\tbf{0},n)$ for each $n$.   Then
$(\sqrt{2\pi})^rg\times\varhat{\zeta}_n$ represents the Fourier transform
of $\zeta_n*f$, for each $n$.   Now $\sequencen{\varhat{\zeta}_n}$
converges uniformly to $\varhat{\zeta}$, and $\sequencen{\zeta_n*f}$
converges to $\zeta*f$ at every point at which $\zeta*|f|$ is defined and
finite, which is $\mu$-almost everywhere.   So if $h$ is a \rdtf,

$$\eqalignno{\int h\times(\sqrt{2\pi})^rg\times\varhat{\zeta}
&=\lim_{n\to\infty}\int h\times(\sqrt{2\pi})^rg\times\varhat{\zeta}_n\cr
\displaycause{the convergence is dominated by the integrable function
$(\sqrt{2\pi})^r\zeta\BbbR^r\cdot|h\times g|$}
&=\lim_{n\to\infty}\int\varhat{h}\times(\zeta_n*f)
=\int\varhat{h}\times(\zeta*f)\cr}$$

\noindent(this convergence being dominated by the integrable function
$|\varhat{h}|\times(\zeta*|f|)$).   As $h$ is arbitrary,
$(\sqrt{2\pi})^rg\times\varhat{\zeta}$
represents the Fourier transform of $\zeta*f$.
}%end of proof of 479H

\vleader{60pt}{479I}{Proposition} (In this result,
$r$ may be any integer greater than or equal to $1$.)

(a) Suppose that $0<\alpha<r$.

\quad(i) There is a tempered function representing the
Fourier transform of $k_{\alpha}$.

\quad(ii) There is a measurable function $g_0$, defined almost everywhere
on $\coint{0,\infty}$, such that $y\mapsto g_0(\|y\|)$ represents the
Fourier transform of $k_{\alpha}$.

\quad(iii) In (ii),

\Centerline{$2^{\alpha/2}\Gamma(\bover{\alpha}2)
   \int_0^{\infty}t^{r-1}g_0(t)e^{-\epsilon t^2}dt
=2^{(r-\alpha)/2}\Gamma(\bover{r-\alpha}2)
   \int_0^{\infty}t^{\alpha-1}e^{-\epsilon t^2}dt$}

\noindent for every $\epsilon>0$.

\quad(iv) $2^{\alpha/2}\Gamma(\bover{\alpha}2)g_0(t)
=2^{(r-\alpha)/2}\Gamma(\bover{r-\alpha}2)t^{\alpha-r}$ for
almost every $t>0$.

\quad(v) $2^{(r-\alpha)/2}\Gamma(\bover{r-\alpha}2)k_{r-\alpha}$
represents the Fourier transform
of $2^{\alpha/2}\Gamma(\bover{\alpha}2)k_{\alpha}$.

(b) Suppose that
$\zeta_1$, $\zeta_2$ are totally finite Radon measures on $\BbbR^r$, and
$0<\alpha<r$.   If $\zeta_1*k_{\alpha}=\zeta_2*k_{\alpha}\,\,\mu$-a.e.,
then $\zeta_1=\zeta_2$.

\proof{{\bf (a)(i)} Set $\beta=\bover12(\alpha+r)$.
Then $k_{\beta}$ is expressible as $f_1+f_2$ where
$f_1$ is integrable and $f_2$ is square-integrable.
\Prf\

\Centerline{$\int_{B(\tbf{0},1)}k_{\beta}d\mu
=r\beta_r\int_0^1\Bover{t^{r-1}}{t^{\beta}}dt$}

\noindent is finite because $\beta<r$;

\Centerline{$\int_{\BbbR^r\setminus B(\tbf{0},1)}k_{\beta}^2d\mu
=r\beta_r\int_1^{\infty}\Bover{t^{r-1}}{t^{2\beta}}dt$}

\noindent is finite because $2\beta>r$.   So we can take
$f_1=k_{\alpha}\times\chi B(\tbf{0},1)$ and $f_2=k_{\alpha}-f_1$.\ \Qed

479G tells us that there is a constant $c$ such that

\Centerline{$k_{\alpha}
=ck_{\beta}*k_{\beta}
=c(f_1*f_1+2f_1*f_2+f_2*f_2)$.}

\noindent Now $f_1*f_1$ is integrable and $f_1*f_2$ is square-integrable
(444Ra), so both have Fourier transforms represented by tempered functions;
while the continuous function $f_2*f_2$ also has a Fourier transform
represented by an integrable function (284Wi).   Assembling these,
$k_{\alpha}$ has a Fourier transform represented by a tempered function.

\medskip

\quad{\bf (ii)} We can
therefore represent the Fourier transform of $k_{\alpha}$ by
the function $g$, where

\Centerline{$g(y)=\lim_{n\to\infty}\Bover1{(\sqrt{2\pi})^r}
  \int e^{-iy\dotproduct x}e^{-\|x\|^2/n}k_{\alpha}(x)\mu(dx)$}

\noindent is defined $\mu$-almost everywhere
(284M/284Wg).   Now suppose that $T:\BbbR^r\to\BbbR^r$ is any orthogonal
transformation, and that $y\in\dom g$.   Then

$$\eqalignno{g(y)
&=\lim_{n\to\infty}\Bover1{(\sqrt{2\pi})^r}
  \int e^{-iy\dotproduct x}e^{-\|x\|^2/n}k_{\alpha}(x)\mu(dx)\cr
&=\lim_{n\to\infty}\Bover1{(\sqrt{2\pi})^r}
  \int e^{-iy\dotproduct T\trs x}e^{-\|T\trs x\|^2/n}k_{\alpha}(T\trs x)
  \mu(dx)\cr
\displaycause{because the transpose $T\trs$ of $T$ 
acts as an automorphism of $(\BbbR^r,\mu)$}
&=\lim_{n\to\infty}\Bover1{(\sqrt{2\pi})^r}
  \int e^{-iTy\dotproduct x}e^{-\|x\|^2/n}k_{\alpha}(x)\mu(dx),\cr}$$

\noindent and $g(Ty)$ is defined and equal to $g(y)$.   So we can set
$g_0(t)=g(y)$ whenever $y\in\dom g$ and $\|y\|=t$, and we shall have
$y\mapsto g_0(\|y\|)$ representing the Fourier transform of $k_{\alpha}$.

\medskip

\quad{\bf (iii)} If $\epsilon>0$, then $x\mapsto e^{-\epsilon\|x\|^2}$ is a
\rdtf, and its Fourier transform is the function
$x\mapsto\Bover1{(\sqrt{2\epsilon})^r}e^{-\|x\|^2/4\epsilon}$
(283N/283Wi\formerly{2{}83We}).   We therefore have

\Centerline{$\int g_0(\|y\|)e^{-\epsilon\|y\|^2}\mu(dy)
=\Bover1{(\sqrt{2\epsilon})^r}\int k_{\alpha}(x)e^{-\|x\|^2/4\epsilon}
   \mu(dx)$,}

\noindent that is,

\Centerline{$r\beta_r\int_0^{\infty}t^{r-1}g_0(t)e^{-\epsilon t^2}dt
=\Bover{r\beta_r}{(\sqrt{2\epsilon})^r}\int_0^{\infty}
  \Bover{t^{r-1}}{t^{\alpha}}e^{-t^2/4\epsilon}dt$;}

\noindent simplifying,

$$\eqalignno{\int_0^{\infty}t^{r-1}g_0(t)e^{-\epsilon t^2}dt
&=\Bover1{(\sqrt{2\epsilon})^r}\int_0^{\infty}
  t^{r-1-\alpha}e^{-t^2/4\epsilon}dt\cr
&=\Bover{2^{r-\alpha}}{2\cdot 2^{r/2}\epsilon^{\alpha/2}}
  \int_0^{\infty}u^{(r-\alpha-2)/2}e^{-u}du\cr
\displaycause{substituting $u=t^2/4\epsilon$}
&=\Bover{2^{r-\alpha}}{2\cdot 2^{r/2}\epsilon^{\alpha/2}}
  \Gamma(\Bover{r-\alpha}2).\cr}$$

\noindent On the other hand,

\Centerline{$\int_0^{\infty}t^{\alpha-1}e^{-\epsilon t^2}dt
=\int_0^{\infty}
  \Bover{(\sqrt u)^{\alpha-2}}{2\epsilon^{\alpha/2}}
  e^{-u}du
=\Bover1{2\epsilon^{\alpha/2}}\Gamma(\Bover{\alpha}2)$.}

\noindent Putting these together,

\Centerline{$2^{\alpha/2}\Gamma(\bover{\alpha}2)
   \int_0^{\infty}t^{r-1}g_0(t)e^{-\epsilon t^2}dt
=2^{(r-\alpha)/2}\Gamma(\bover{r-\alpha}2)
   \int_0^{\infty}t^{(r-1)-(r-\alpha)}e^{-\epsilon t^2}dt$}

\noindent for every $\epsilon>0$.

\medskip

\quad{\bf (iv)} Set

\Centerline{$g_1(t)=t^{r-1}e^{-t^2}
  \bigl(2^{\alpha/2}\Gamma(\bover{\alpha}2)g_0(t)
    -2^{(r-\alpha)/2}\Gamma(\bover{r-\alpha}2)t^{\alpha-r}\bigr)$}

\noindent for $t>0$.   Then $g_1$ is integrable and
$\int_0^{\infty}g_1(t)e^{-\epsilon t^2}dt=0$ for every $\epsilon\ge 0$.
It follows that $g_1=0$ a.e.   \Prf\ Consider the linear span $A$ of the
functions $t\mapsto e^{-\epsilon t^2}$ for $\epsilon\ge 0$.   This is a
subalgebra of $C_b(\coint{0,\infty})$ containing the constant functions and
separating the points of $\coint{0,\infty}$.   It follows that for every
$\gamma\ge 0$, $\delta>0$ and $h\in C_b(\coint{0,\infty})$, there
is an $f\in A$ such that $|f(t)-h(t)|\le\delta$ for $t\in[0,\gamma]$ and
$\|f\|_{\infty}\le\|h\|_{\infty}$ (281E).   Since
$\int_0^{\infty}g_1\times f=0$, we must have

\Centerline{$|\int_0^{\infty}g_1\times h|
\le\delta\|g_1\|_1+2\|h\|_{\infty}\int_{\gamma}^{\infty}|g_1(t)|dt$.}

\noindent As $\delta$ and $\gamma$ are arbitrary,
$\int_0^{\infty}g_1\times h=0$;  as $h$ is arbitrary, $\int_0^ag_1=0$
for every $a\ge 0$, and $g_1$ must be zero almost
everywhere (222D).\ \Qed

Accordingly
$2^{\alpha/2}\Gamma(\bover{\alpha}2)g_0(t)
  =2^{(r-\alpha)/2}\Gamma(\bover{r-\alpha}2)t^{\alpha-r}$
for almost every $t\ge 0$.

\medskip

\quad{\bf (v)} Now

\Centerline{$y
\mapsto 2^{(r-\alpha)/2}\Gamma(\bover{r-\alpha}2)k_{r-\alpha}(y)
\eae 2^{\alpha/2}\Gamma(\bover{\alpha}2)g_0(\|y\|)$}

\noindent represents the Fourier transform of
$2^{\alpha/2}\Gamma(\bover{\alpha}2)k_{\alpha}$.

\medskip

{\bf (b)} By (a), the Fourier transform of $k_{\alpha}$ is represented by a
tempered function $g$ which is non-zero $\mu$-a.e.
As $k_{\alpha}$ is the sum of an integrable function and a bounded
function, 479Hd tells us that the Fourier
transform of $\zeta_1*k_{\alpha}$ is represented by
$(\sqrt{2\pi})^r\varhat{\zeta}_1\times g$;  and similarly for $\zeta_2$.
As $\zeta_1*k_{\alpha}\eae\zeta_2*k_{\alpha}$,
$\varhat{\zeta}_1\times g\eae\varhat{\zeta}_2\times g$ (284Ib) and
$\varhat{\zeta}_1\eae\varhat{\zeta}_2$.   Since $\varhat{\zeta}_1$ and
$\varhat{\zeta}_2$ are both continuous (285Fb), they are equal everywhere;
in particular,

\Centerline{$\zeta_1\BbbR^r=\varhat{\zeta}_1(0)
=\varhat{\zeta}_2(0)=\zeta_2\BbbR^r$.}

\noindent If $\zeta_1=\zeta_2$ is the zero measure, we can stop.
Otherwise, they can be expressed as $\gamma\zeta'_1$ and $\gamma\zeta'_2$
where $\zeta'_1$ and $\zeta'_2$ are probability measures and $\gamma>0$.
In this case, $\zeta'_1$ and $\zeta'_2$ have the same characteristic
function (285D) and must be equal (285M);  so $\zeta_1=\zeta_2$, as
claimed.
}%end of proof of 479I

\cmmnt{\medskip

\noindent{\bf Remark} The functions $\zeta*k_{\alpha}$ are called {\bf
Riesz potentials}.
}

\leader{479J}{}\cmmnt{ Now I return to the study of
Newtonian potential when $r\ge 3$.

\medskip

\noindent}{\bf Lemma} (a) Let $\zeta$ be a totally finite Radon measure on
$\BbbR^r$.   Let $U_{\zeta}$ be the $(r-1)$-potential of $\zeta$ and
$W_{\zeta}$ the Newtonian potential of $\zeta$;  let
$k_{r-1}$ and $k_{r-2}$ be the Riesz kernels.
Then $U_{\zeta}\eae\zeta*k_{r-1}$ and $W_{\zeta}\eae\zeta*k_{r-2}$.

(b) Let $\zeta$,
$\zeta_1$ and $\zeta_2$ be totally finite Radon measures on $\BbbR^r$.

\quad(i) $\int_{\BbbR^r}W_{\zeta_1}d\zeta_2
=\int_{\BbbR^r}W_{\zeta_2}d\zeta_1
=\Bover1{c_r}\int_{\BbbR^r}U_{\zeta_1}\times U_{\zeta_2}d\mu$\cmmnt{,
where $c_r$ is the constant of 479G}.

\quad(ii) The energy $\energy(\zeta)$ of $\zeta$
is $\Bover1{c_r}\|U_{\zeta}\|_2^2$, counting
$\|U_{\zeta}\|_2$ as $\infty$ if $U_{\zeta}\notin\eusm L^2(\mu)$.

\quad(iii) If $\zeta=\zeta_1+\zeta_2$
then $U_{\zeta}=U_{\zeta_1}+U_{\zeta_2}$ and
$W_{\zeta}=W_{\zeta_1}+W_{\zeta_2}$;  similarly,
$U_{\alpha\zeta}=\alpha U_{\zeta}$ and $W_{\alpha\zeta}=\alpha W_{\zeta}$
for $\alpha\ge 0$.

\quad(iv) If $U_{\zeta_1}=U_{\zeta_2}\,\,\mu$-a.e., then $\zeta_1=\zeta_2$.

\quad(v) If $W_{\zeta_1}=W_{\zeta_2}\,\,\mu$-a.e., then $\zeta_1=\zeta_2$.

\quad(vi) $\zeta\BbbR^r
=\lim_{\gamma\to\infty}\Bover1{r\beta_r\gamma}
  \int_{\partial B(0,\gamma)}W_{\zeta}d\nu$.

(c) Let $M^+_{\text{R}}(\BbbR^r)$
be the set of totally finite Radon measures on
$\BbbR^r$, with its narrow topology.   Then
$\energy:M^+_{\text{R}}(\BbbR^r)\to[0,\infty]$ is lower semi-continuous.

\proof{{\bf (a)} As $k_{r-1}$ and $k_{r-2}$
are both expressible as sums
of integrable functions and bounded functions, $\zeta*k_{r-1}$ and
$\zeta*k_{r-2}$ are both defined a.e.\ (479Hc);  and now we have only to
read the definitions to see that $U_{\zeta}$ and $W_{\zeta}$ are these
convolutions with the technical adjustment that they are permitted to take
the value $\infty$.

\medskip

{\bf (b)(i)} For any $x$, $y\in\BbbR^r$,

$$\eqalign{\Bover1{\|x-y\|^{r-2}}
&=k_{r-2}(x-y)
=\Bover1{c_r}(k_{r-1}*k_{r-1})(x-y)\cr
&=\Bover1{c_r}\int_{\BbbR^r}\Bover1{\|x-y-z\|^{r-1}\|z\|^{r-1}}\mu(dz)\cr
&=\Bover1{c_r}\int_{\BbbR^r}\Bover1{\|x-z\|^{r-1}\|z-y\|^{r-1}}\mu(dz)
=\Bover1{c_r}\int_{\BbbR^r}\Bover1{\|x-z\|^{r-1}\|y-z\|^{r-1}}\mu(dz).
  \cr}$$

\noindent So

$$\eqalignno{\int_{\BbbR^r}W_{\zeta_1}d\zeta_2
&=\int_{\BbbR^r}\int_{\BbbR^r}\Bover1{\|x-y\|^{r-2}}\zeta_1(dx)\zeta_2(dy)
  \cr
&=\Bover1{c_r}\int_{\BbbR^r}\int_{\BbbR^r}\int_{\BbbR^r}
  \Bover1{\|x-z\|^{r-1}\|y-z\|^{r-1}}\mu(dz)\zeta_1(dx)\zeta_2(dy)\cr
&=\Bover1{c_r}\int_{\BbbR^r}\int_{\BbbR^r}\int_{\BbbR^r}
  \Bover1{\|x-z\|^{r-1}\|y-z\|^{r-1}}\zeta_1(dx)\zeta_2(dy)\mu(dz)\cr
&=\Bover1{c_r}\int_{\BbbR^r}U_{\zeta_1}(z)U_{\zeta_2}(z)\mu(dz)
=\int_{\BbbR^r}U_{\zeta_1}\times U_{\zeta_2}d\mu.\cr}$$

\noindent Hence (or otherwise)

\Centerline{$\int_{\BbbR^r}W_{\zeta_2}d\zeta_1
=\Bover1{c_r}\int_{\BbbR^r}U_{\zeta_2}\times U_{\zeta_1}d\mu
=\int_{\BbbR^r}W_{\zeta_1}d\zeta_2$.}

\medskip

\quad{\bf (ii)} Take $\zeta_1=\zeta_2=\zeta$ in (i).

\medskip

\quad{\bf (iii)} This is immediate from
234Hc.

\medskip

\quad{\bf (iv)-(v)} Put (a) and 479Ib together.

\medskip

\quad{\bf (vi)} For any $\gamma>0$,

$$\eqalignno{\Bover1{r\beta_r\gamma}
  \int_{\partial B(0,\gamma)}W_{\zeta}d\nu
&=\int W_{\zeta}d\lambda_{B(0,\gamma)}\cr
\displaycause{479Da}
&=\int\tilde W_{B(0,\gamma)}d\zeta\cr
\displaycause{(i) above}
&=\int\min(1,\Bover{\gamma^{r-2}}{\|x-z\|^{r-2}})\zeta(dx)\cr
\displaycause{479Da again}
&\to\zeta\BbbR^r\cr}$$

\noindent as $\gamma\to\infty$.

\medskip

{\bf (c)} The map
$\zeta\mapsto\zeta\times\zeta:M^+_{\text{R}}(\BbbR^r)
\to M^+_{\text{R}}(\BbbR^r\times\BbbR^r)$ is
continuous, by 437Ma.   Next, the function
$\lambda\mapsto\biggerint_{\BbbR^r}\Bover1{\|x-y\|^{r-2}}\lambda(d(x,y)):
M^+_{\text{R}}(\BbbR^r\times\BbbR^r)\to[0,\infty]$ is
lower semi-continuous, by 437Jg again.
So $\energy$ is the composition of a lower semi-continuous function with
a continuous function, and is lower semi-continuous (4A2B(d-ii)).
}%end of proof of 479J

\leader{479K}{Lemma} Let $K\subseteq\BbbR^r$ be a compact set, with
equilibrium measure $\lambda_K$.
Then $\lambda_KK=\capacity K=\energy(\lambda_K)$, and if
$\zeta$ is any Radon measure on $\BbbR^r$ such that
$\zeta K\ge\capacity K\ge\energy(\zeta)$, $\zeta=\lambda_K$.

\proof{{\bf (a)} We know that $\lambda_KK=\lambda_K\BbbR^r=\capacity K$
(479C(a-i)).   So if $K$ has zero capacity then $\lambda_K$ is the zero
measure
and $\energy(\lambda_K)=0$;  also the only Radon measure on $\BbbR^r$
with zero energy is
$\lambda_K$, and we can stop.   So henceforth let us suppose that
$\capacity K>0$.

Set

\Centerline{$e=\inf\{\energy(\zeta):\zeta$ is a Radon
probability measure on $\BbbR^r$ such that $\zeta K\ge\capacity K\}$.}

\noindent Because $\tilde W_K(x)\le 1$ for every $x\in\BbbR^r$
(479D(b-i)),

\Centerline{$e\le\energy(\lambda_K)=\int\tilde W_Kd\lambda_K
\le\lambda_K\BbbR^r=\capacity K$}

\noindent is finite.

\medskip

{\bf (b)} Consider the set $Q$ of Radon measures $\zeta$ on $\BbbR^r$ such
that $\zeta K=\zeta\BbbR^r=\capacity K$.   With its narrow topology,
$Q$ is homeomorphic to the set of Radon measures on $K$ of magnitude
$\capacity K$, which is compact (437R(f-ii)).
Since $\energy:Q\to[0,\infty]$ is
lower semi-continuous (479Jc), there is a $\lambda\in Q$ with
energy $e$ (4A2B(d-viii)).

In fact there is exactly one such member of $Q$.   \Prf\
Suppose that $\zeta$ is any other member of $Q$ with energy $e$.
Write $u_{\zeta}$ for the
equivalence class of $U_{\zeta}$ in $L^2$.   Then
$\bover12(\zeta+\lambda)$ belongs to $Q$ and
$U_{\bover12(\zeta+\lambda)}=\bover12(U_{\zeta}+U_{\lambda})$
(479J(b-iii)).   So, defining $c_r$ as in 479G,

$$\eqalignno{e+\Bover1{c_r}\|u_{\zeta}-u_{\lambda}\|_2^2
&\le\energy(\Bover12(\zeta+\lambda))
  +\Bover1{4c_r}\innerprod{u_{\zeta}-u_{\lambda}}{u_{\zeta}-u_{\lambda}}\cr
&=\Bover1{4c_r}\innerprod{u_{\zeta}+u_{\lambda}}{u_{\zeta}+u_{\lambda}}
  +\Bover1{4c_r}\innerprod{u_{\zeta}-u_{\lambda}}{u_{\zeta}-u_{\lambda}}\cr
\displaycause{479J(b-ii)}
&=\Bover1{2c_r}(\|u_{\zeta}\|_2^2+\|u_{\lambda}\|_2^2)
=e.\cr}$$

\noindent It follows that $\|u_{\zeta}-u_{\lambda}\|_2=0$ and
$U_{\zeta}\eae U_{\lambda}$.   Consequently $\zeta=\lambda$
(479J(b-iv)).\ \Qed

\medskip

{\bf (c)(i)} If $\zeta$ is any Radon measure on $\BbbR^r$ with finite
energy, then $\int W_{\zeta}d\lambda\ge\Bover{e\,\zeta K}{\capacity K}$.
\Prf\ If $\zeta K=0$ this is trivial.   Otherwise, set
$\zeta'=\Bover{\capacity K}{\zeta K}\zeta\LLcorner K$.   Then
$\zeta'$ has finite energy (479Fh) and belongs to $Q$, so for any
$\alpha\in[0,1]$ we have $\alpha\zeta'+(1-\alpha)\lambda\in Q$, and

$$\eqalign{c_re
&\le c_r\energy(\alpha\zeta'+(1-\alpha)\lambda)
=\|\alpha u_{\zeta'}+(1-\alpha)u_{\lambda}\|_2^2\cr
&=\alpha^2\|u_{\zeta'}\|_2^2
  +2\alpha(1-\alpha)\innerprod{u_{\zeta'}}{u_{\lambda}}
  +(1-\alpha)^2\|u_{\lambda}\|_2^2\cr
&=\alpha^2\|u_{\zeta'}\|_2^2
  +2\alpha(1-\alpha)\innerprod{u_{\zeta'}}{u_{\lambda}}
  +(1-\alpha)^2c_re\cr
&=c_re
  +2\alpha(\innerprod{u_{\zeta'}}{u_{\lambda}}-c_re)
  +\alpha^2(\|u_{\zeta'}\|_2^2
    -2\innerprod{u_{\zeta'}}{u_{\lambda}}
    +c_re).\cr}$$

\noindent It follows that
$\innerprod{u_{\zeta'}}{u_{\lambda}}-c_re\ge 0$ and

$$\eqalignno{\int W_{\zeta}d\lambda
&\ge\int W_{\zeta\LLcorner K}d\lambda\cr
\displaycause{479Fh}
&=\Bover{\zeta K}{\capacity K}\int W_{\zeta'}d\lambda
=\Bover{\zeta K}{c_r\capacity K}\int U_{\zeta'}\times U_{\lambda}d\mu\cr
\displaycause{479J(b-i)}
&=\Bover{\zeta K}{c_r\capacity K}\innerprod{u_{\zeta'}}{u_{\zeta}}
\ge\Bover{\zeta K}{c_r\capacity K}c_re
=\Bover{e\,\zeta K}{\capacity K},\cr}$$

\noindent as claimed.\ \Qed

\medskip

\quad{\bf (ii)} If $\zeta$ is any Radon measure on $\BbbR^r$ with finite
energy, then $W_{\lambda}(x)\ge\Bover{e}{\capacity K}$ for $\zeta$-almost
every $x\in K$.   \Prf\Quer\ Otherwise, set
$E=\{x:x\in K$, $W_{\lambda}(x)<\Bover{e}{\capacity K}\}$,
and consider $\zeta'=\zeta\LLcorner E$.   Then

$$\eqalignno{\int W_{\zeta'}d\lambda
&=\int W_{\lambda}d\zeta'\cr
\displaycause{479J(b-i)}
&<\Bover{e}{\capacity K}\zeta'E
\le\Bover{e\zeta'K}{\capacity K},\cr}$$

\noindent contradicting (i).\ \Bang\Qed

\medskip

\quad{\bf (iii)} $W_{\lambda}(x)=\Bover{e}{\capacity K}$ for
$\lambda$-almost every $x\in K$.   \Prf\
Since $\lambda$ has finite energy,
(ii) tells us that $W_{\lambda}(x)\ge\Bover{e}{\capacity K}$
for $\lambda$-almost every $x\in K$.   Since

\Centerline{$\int_KW_{\lambda}d\lambda\le\int W_{\lambda}d\lambda
=e=\Bover{e}{\capacity K}\lambda K$,}

\noindent we must have $W_{\lambda}(x)=\Bover{e}{\capacity K}$ for
$\lambda$-almost every $x\in K$.\ \Qed

\medskip

\quad{\bf (iv)} Since $\lambda K=\lambda\BbbR^r$, 479Fg,
with $f$ the constant function with value $\Bover{e}{\capacity K}$,
tells us that $W_{\lambda}(x)\le\Bover{e}{\capacity K}$ for every
$x\in\BbbR^r$.

\medskip

{\bf (d)} For $x\in\BbbR^r\setminus K$,
$W_{\lambda}(x)\le\Bover{e}{\capacity K}\hp(K-x)$.   \Prf\
Set $G=\BbbR^r\setminus K$, and
let $\tau$ be the Brownian exit time from $G-x$.   Define
$f:\overline{G}^{\infty}\to[0,1]$ by setting

$$\eqalign{f(y)
&=0\text{ if }y\in\partial G=\partial K,\cr
&=\Bover{e}{\capacity K}-W_{\lambda}(y)\text{ if }y\in G,\cr
&=\Bover{e}{\capacity K}\text{ if }y=\infty.\cr}$$

\noindent Because $W_{\lambda}\restr G$ is continuous
and harmonic (479Fa), so is $f\restr G$.
Because $\lambda$ has compact support,
$\lim_{y\to\infty}W_{\lambda}(y)=0$ (479Fd), so $f$ is continuous
at $\infty$;
because $W_{\lambda}(y)\le\Bover{e}{\capacity K}$ for every $y$, $f$ is
lower semi-continuous.   So

$$\eqalignno{\Bover{e}{\capacity K}-W_{\lambda}(x)
&=f(x)
\ge\Expn(f(x+X_{\tau}))\cr
\displaycause{478O, because $r\ge 3$ and $\BbbR^r$ has few wandering paths}
&=\Bover{e}{\capacity K}\Pr(\tau=\infty)
=\Bover{e}{\capacity K}(1-\Pr(\tau<\infty)).\cr}$$

\noindent Thus $W_{\lambda}(x)$ is at most
$\Bover{e}{\capacity K}\Pr(\tau<\infty)$.    But $\Pr(\tau<\infty)$ is just
the Brownian hitting probability $\hp(K-x)$.\ \Qed

\medskip

{\bf (e)} $e=\energy(\lambda_K)$.   \Prf\

$$\eqalignno{e
&\le\energy(\lambda_K)
\le\capacity K\cr
\displaycause{(a) above}
&=\lambda K
=\lambda\BbbR^r
=\lim_{\|x\|\to\infty}\|x\|^{r-2}W_{\lambda}(x)\cr
\displaycause{479Fd}
&\le\Bover{e}{\capacity K}
   \liminf_{\|x\|\to\infty}\|x\|^{r-2}\hp(K-x)\cr
\displaycause{by (d) of this proof}
&=\Bover{e}{\capacity K}\cdot\capacity K\cr
\displaycause{479B(ii)}
&=e. \text{ \Qed}\cr}$$

\medskip

{\bf (f)} From this we see at once
that $\lambda=\lambda_K$ and $\capacity K=\energy(\lambda_K)$.
Now suppose that
$\zeta$ is a Radon measure on $\BbbR^r$ such that
$\zeta K\ge\capacity K\ge\energy(\zeta)$.
Set $\zeta'=\Bover{\capacity K}{\zeta K}\zeta\LLcorner K$;  then
$\zeta'\in Q$, while $\zeta'\le\zeta$, so

\Centerline{$e=\capacity K\ge\energy(\zeta)\ge\energy(\zeta')$}

\noindent by 479Fh.   It follows that $\zeta'=\lambda$ and
$\lambda\le\zeta$.   Accordingly $W_{\lambda}\le W_{\zeta}$ (479Fh again),

$$\eqalignno{\energy(\zeta)
&=\int W_{\zeta}d\zeta
\ge\int W_{\lambda}d\zeta
\ge\int_KW_{\lambda}d\zeta
\ge\zeta K\cr
\displaycause{(c-ii) above}
&\ge\capacity K
\ge\energy(\zeta),\cr}$$

\noindent and we have equality throughout.   Since $\lambda$ is non-zero
and the kernel $(x,y)\mapsto\Bover1{\|x-y\|^{r-2}}$ is strictly positive,
$W_{\lambda}$ is strictly positive.
It follows that $\zeta(\BbbR^r\setminus K)=0$ and $\zeta\in Q$;
consequently $\zeta=\lambda=\lambda_K$, as required.
}%end of proof of 479K

\leader{479L}{}\cmmnt{ I shall wish later to quote a couple of
the facts which
appeared in the course of the proof above, and I think it will be safer to
list them now.

\medskip

\noindent}{\bf Corollary} Let $K\subseteq\BbbR^r$ be a compact set with
equilibrium potential $\tilde W_K$.

(a) If $\zeta$ is any Radon measure on $\BbbR^r$ with finite energy,
then $\tilde W_K(x)=1$ for $\zeta$-almost every $x\in K$.

(b) If $\zeta$ is a Radon measure on $\BbbR^r$ such that $W_{\zeta}\le 1$
everywhere on $K$, $\zeta K\le\capacity K$.

(c) $\tilde W_K(x)\le\hp(K-x)$ for every $x\in\BbbR^r\setminus K$.

\proof{{\bf (a)(i)} Suppose first that $\capacity K>0$.
Working through the proof of 479K, we discover, in parts
(e)-(f) of the proof, that
$e=\capacity K$ and $\lambda=\lambda_K$,
so we just have to put (c-ii) of the proof together with 479D(b-i).

\medskip

\quad{\bf (ii)} If $\capacity K=0$, let $B$ be a non-trivial closed
ball disjoint from $A$, and consider $L=K\cup B$.   Then
$\capacity B=\capacity L$ (479Ea) and
$\lambda_LK=0$, by 479D(c-ii), so

\Centerline{$\lambda_LB=\lambda_LL=\capacity L=\energy(\lambda_L)
=\capacity B$}

\noindent and $\lambda_L=\lambda_B$ (479K).   Now
$\tilde W_L=1\,\,\zeta$-a.e.\ on $L$, while

\Centerline{$\tilde W_L(x)=\tilde W_B(x)<1$}

\noindent for every $x\in\BbbR^r\setminus B$ (479Da), and in particular
for every $x\in K$;  so $K$ must be $\zeta$-negligible.

\medskip

{\bf (b)} Set $\zeta'=\zeta\LLcorner K$;  then $W_{\zeta'}\le W_{\zeta}$,
so $\energy(\zeta')=\int_KW_{\zeta'}d\zeta'\le\zeta'K$ is finite.
By (a), $\tilde W_K\ge 1\,\,\zeta'$-a.e., so

$$\eqalignno{\zeta K
&=\zeta'K
\le\int\tilde W_Kd\zeta'
=\int W_{\zeta'}d\lambda_K\cr
\displaycause{479J(b-i)}
&\le\lambda_K\BbbR^r
=\capacity K.\cr}$$

\medskip

{\bf (c)} If $\capacity K=0$ then $\lambda_K$ is the zero measure and the
result is trivial.   Otherwise, again look at the proof of 479K;  in part
(d), we saw that $W_{\lambda}(x)\le\Bover{e}{\capacity K}\hp(K-x)$;  but we
now know that $e=\capacity K$ and $\lambda=\lambda_K$, so we get
$\tilde W_K(x)\le\hp(K-x)$, as claimed.
}%end of proof of 479L

\leader{479M}{}\cmmnt{ In 479Ed we saw that there is a natural extension
of Newtonian capacity to a Choquet capacity defined on every subset of
$\BbbR^r$.   However the importance of Newtonian capacity lies as much in
the equilibrium measures and potentials as in the simple quantity of
capacity itself, and the methods of 479B-479E %479B 479C 479D 479E
do not seem to yield these by
any direct method.   With the new ideas of 479K-479L, we can now
approach the problem of defining equilibrium measures for unbounded
analytic sets of finite capacity.

\wheader{479M}{4}{2}{2}{60pt}

\noindent}{\bf Lemma} Let $A\subseteq\BbbR^r$ be an analytic set with
finite Choquet-Newton capacity $c(A)$.

(a) $\lim_{\gamma\to\infty}c(A\setminus B(\tbf{0},\gamma))=0$.

(b) $\lambda_A=\lim_{\gamma\to\infty}\lambda_{A\cap B(\tbf{0},\gamma)}$
is defined for the total variation metric on the space
$M^+_{\text{R}}(\BbbR^r)$ of totally finite Radon measures on $\BbbR^r$.

(c)(i) $\lambda_A\BbbR^r=c(A)$.

\quad(ii) $\supp(\lambda_A)\subseteq\partial A$.

\quad(iii) If $B\subseteq\BbbR^r$ is another analytic set such that
$c(B)<\infty$, then $\lambda_{A\cup B}\le\lambda_A+\lambda_B$.

(d)(i) $\tilde W_A=W_{\lambda_A}$ is the limit
$\lim_{\gamma\to\infty}\tilde W_{A\cap B(\tbf{0},\gamma)}
=\sup_{\gamma\ge 0}\tilde W_{A\cap B(\tbf{0},\gamma)}$.

\quad(ii) $\tilde W_A(x)\le 1$ for every $x\in\BbbR^r$.

\quad(iii) If $\zeta$ is any Radon measure on $\BbbR^r$ with finite energy,
$\tilde W_A(x)=1$ for $\zeta$-almost every $x\in A$.

\quad(iv) $\energy(\lambda_A)=c(A)$.

\proof{{\bf (a)} \Quer\ Otherwise, set

\Centerline{$\alpha=\lim_{\gamma\to\infty}c(A\setminus B(\tbf{0},\gamma))
=\inf_{\gamma>0}c(A\setminus B(\tbf{0},\gamma))>0$.}

\noindent Set $\epsilon=\bover18\alpha$
and $\delta=\bover32\sqrt{\alpha}$.
Let $\gamma$ be such that
$c(A\setminus B(\tbf{0},\gamma))\le\alpha+\epsilon$, and let
$K\subseteq A\setminus B(\tbf{0},\gamma)$ be a compact set such that
$\capacity K\ge\alpha-\epsilon$ (479E(d-iii)).   Let $\gamma'$ be such that
$K\subseteq B(\tbf{0},\gamma')$, and
let $L\subseteq A\setminus B(\tbf{0},\gamma'+\delta)$ be a compact set such that
$\capacity L\ge\alpha-\epsilon$.

Set $\zeta=\bover23(\lambda_K+\lambda_L)$.   Then
$W_{\zeta}=\bover23(\tilde W_K+\tilde W_L)$.   If $x\in K$, then
$\|x-y\|\ge\delta$ for every $y\in L$, so

\Centerline{$\tilde W_L(x)
\le\Bover1{\delta^2}\lambda_LL\le\Bover{\alpha+\epsilon}{\delta^2}
=\Bover12$;}

\noindent similarly, $\tilde W_K(x)\le\Bover12$ for every $x\in L$.
So $W_{\zeta}(x)\le 1$ for every $x\in K\cup L$, and therefore for every
$x\in\BbbR^r$, by 479Fg.   But this means that

$$\eqalignno{c(A\setminus B(\tbf{0},\gamma))
&\ge\capacity(K\cup L)
\ge\zeta(K\cup L)\cr
\displaycause{479Lb}
&=\Bover23(\capacity K+\capacity L)
\ge\Bover43(\alpha-\epsilon)
>\alpha+\epsilon,\cr}$$

\noindent which is impossible.\ \Bang

\medskip

{\bf (b)} For $\gamma\ge 0$ set
$\alpha_{\gamma}=c(A\setminus B(\tbf{0},\gamma))$
and $\zeta_{\gamma}=\lambda_{A\cap B(\tbf{0},\gamma)}$.
If $0\le\gamma\le\gamma'$ and $E\subseteq\BbbR^r$ is Borel, then
$|\zeta_{\gamma}E-\zeta_{\gamma'}E|\le\alpha_{\gamma}$.
\Prf\

$$\eqalignno{\zeta_{\gamma'}E
&\le\zeta_{\gamma}E+\lambda_{E\cap B(\tbf{0},\gamma')\setminus B(\tbf{0},\gamma)}(K)\cr
\displaycause{479D(c-i)}
&\le\zeta_{\gamma}E+c(A\cap B(\tbf{0},\gamma')\setminus B(\tbf{0},\gamma))
\le\zeta_{\gamma}E+c(A\setminus B(\tbf{0},\gamma))
=\zeta_{\gamma}E+\alpha_{\gamma}.\cr}$$

\noindent On the other side we now have

$$\eqalign{\zeta_{\gamma}E
&=c(A\cap B(\tbf{0},\gamma))-\zeta_{\gamma}(\BbbR^r\setminus E)\cr
&\le c(A\cap B(\tbf{0},\gamma'))-\zeta_{\gamma'}(\BbbR^r\setminus E)
   +\alpha_{\gamma}
=\zeta_{\gamma'}E+\alpha_{\gamma}.\cr}$$

\noindent So
$|\zeta_{\gamma}E-\zeta_{\gamma'}E|\le\alpha_{\gamma}$.\ \QeD\   It follows
at once that $\rhotv(\zeta_{\gamma},\zeta_{\gamma'})\le 2\alpha_{\gamma}$.

Since $\lim_{\gamma\to\infty}\alpha_{\gamma}=0$, by (a),
$\sequencen{\zeta_n}$ is a Cauchy sequence for $\rhotv$.
As noted in 437Q(a-iii), $M^+_{\text{R}}(\BbbR^r)$ is complete, so
$\lim_{\gamma\to\infty}\lambda_{A\cap B(\tbf{0},\gamma)}
=\lim_{n\to\infty}\zeta_n$ is defined, and we have our measure $\lambda_A$.

\medskip

{\bf (c)(i)} Now

\Centerline{$\lambda_A\BbbR^r
=\lim_{n\to\infty}\lambda_{A\cap B(\tbf{0},n)}(\BbbR^r)
=\lim_{n\to\infty}c(A\cap B(\tbf{0},n))=c(A)$.}

\medskip

\quad{\bf (ii)} For any $\gamma\ge 0$,

$$\eqalignno{\lambda_{A\cap B(\tbf{0},\gamma)}(\BbbR^r\setminus\partial A)
&\le\lambda_{A\cap B(\tbf{0},\gamma)}(\partial B(\tbf{0},\gamma))\cr
\displaycause{because the support of
$\lambda_{A\cap B(\tbf{0},\gamma)}$ is included in
$\partial(A\cap B(\tbf{0},\gamma))\subseteq\partial A\cup\partial B(\tbf{0},\gamma)$}
&\le|\lambda_A(\partial B(\tbf{0},\gamma))
  -\lambda_{A\cap B(\tbf{0},\gamma)}(\partial B(\tbf{0},\gamma))|
  +\lambda_A(\partial B(\tbf{0},\gamma))\cr
&\le\rhotv(\lambda_A,\lambda_{A\cap B(\tbf{0},\gamma)})
  +\lambda_A(\partial B(\tbf{0},\gamma)).\cr}$$

\noindent So

$$\eqalign{\lambda_A(\BbbR^r\setminus\partial A)
&=\lim_{\gamma\to\infty}
  \lambda_{A\cap B(\tbf{0},\gamma)}(\BbbR^r\setminus\partial A)\cr
&\le\lim_{\gamma\to\infty}\rhotv(\lambda_A,\lambda_{A\cap B(\tbf{0},\gamma)})
  +\lim_{\gamma\to\infty}\lambda_A(\partial B(\tbf{0},\gamma))
=0.\cr}$$

\medskip

\quad{\bf (iii)} For any compact set $K\subseteq\BbbR^r$,

$$\eqalignno{\lambda_{A\cup B}(K)
&=\lim_{\gamma\to\infty}\lambda_{(A\cup B)\cap B(\tbf{0},\gamma)}(K)
\le\lim_{\gamma\to\infty}
  \lambda_{A\cap B(\tbf{0},\gamma)}(K)+\lambda_{B\cap B(\tbf{0},\gamma)}(K)\cr
\displaycause{479D(c-i)}
&=\lambda_A(K)+\lambda_B(K)
=(\lambda_A+\lambda_B)(K).\cr}$$

\noindent By 416Ea, $\lambda_{A\cup B}\le\lambda_A+\lambda_B$.

\medskip

{\bf (d)(i)} By 479D(b-ii), the supremum and the limit are the same.
Suppose that $x\in\BbbR^r$ and $\epsilon>0$.
Start with $\gamma>\|x\|$.   Since $\tilde W_{A\cap B(\tbf{0},\gamma)}(x)$ is
finite, there is a $\delta\in\ooint{0,\gamma-\|x\|}$ such that
$\biggerint_{B(\tbf{0},\delta)}\Bover1{\|x-y\|^{r-2}}
  \lambda_{A\cap B(\tbf{0},\gamma)}(dy)\le\epsilon$.
If $\gamma'\ge\gamma\ge 0$, then

\Centerline{$\lambda_{A\cap B(\tbf{0},\gamma')}
\le\lambda_{A\cap B(\tbf{0},\gamma)}
  +\lambda_{A\cap B(\tbf{0},\gamma')\setminus B(\tbf{0},\gamma)}$,}

\noindent so

$$\eqalignno{\int_{B(\tbf{0},\delta)}\Bover1{\|x-y\|^{r-2}}
   \lambda_{A\cap B(\tbf{0},\gamma')}(dy)
&\le\int_{B(\tbf{0},\delta)}\Bover1{\|x-y\|^{r-2}}
  \lambda_{A\cap B(\tbf{0},\gamma)}(dy)\cr
&\mskip75mu
  +\int_{B(\tbf{0},\delta)}\Bover1{\|x-y\|^{r-2}}
   \lambda_{A\cap B(\tbf{0},\gamma')\setminus B(\tbf{0},\gamma)}(dy)\cr
\displaycause{234Hc, 234Qc}
% if \mu\le\nu then \int fd\mu\le\int fd\nu whenever f\ge 0 and
% latter is defined in [0,\infty]
&=\int_{B(\tbf{0},\delta)}\Bover1{\|x-y\|^{r-2}}
   \lambda_{A\cap B(\tbf{0},\gamma)}(dy)\cr
\displaycause{because $\interior B(\tbf{0},\gamma)$ is
$\lambda_{A\cap B(\tbf{0},\gamma')\setminus B(\tbf{0},\gamma)}$-negligible}
&\le\epsilon.\cr}$$

\noindent So, setting $M=\Bover1{\delta^{r-2}}$,

\Centerline{$|\tilde W_{A\cap B(\tbf{0},\gamma')}(x)
  -\int\min(M,\Bover1{\|x-y\|^{r-2}})\lambda_{A\cap B(\tbf{0},\gamma')}(dy)|
\le\epsilon$.}

\noindent Using (c-ii) and (c-iii) to apply
the same argument with $A$ in place of $A\cap B(\tbf{0},\gamma')$, we get

\Centerline{$|\tilde W_A(x)
  -\int\min(M,\Bover1{\|x-y\|^{r-2}})\lambda_A(dy)|
\le\epsilon$.}

\noindent On the other hand,

\Centerline{$\int\min(M,\Bover1{\|x-y\|^{r-2}})\lambda_A(dy)
=\lim_{\gamma'\to\infty}\int\min(M,\Bover1{\|x-y\|^{r-2}})
  \lambda_{A\cap B(\tbf{0},\gamma')}(dy)$}

\noindent (437Q(a-ii)), so

\Centerline{$\limsup_{\gamma'\to\infty}|\tilde W_{A\cap B(\tbf{0},\gamma')}(x)
  -\tilde W_A(x)|\le 2\epsilon$.}

\noindent As $\epsilon$ is arbitrary,
$W_A(x)=\lim_{\gamma'\to\infty}W_{A\cap B(\tbf{0},\gamma')}(x)$, as claimed.

\medskip

\quad{\bf (ii)} It follows at once that $\tilde W_A\le 1$ everywhere.

\medskip

\quad{\bf (iii)} Write $E=\{x:x\in A,\,\tilde W_A(x)<1\}$, and let $\zeta$
be a Radon measure on $\BbbR^r$ of finite energy.   \Quer\ If
$\zeta E>0$, there is a compact set $K\subseteq E$ such that
$\zeta K>0$.   Now there is a $\gamma>0$ such that
$K\subseteq B(\tbf{0},\gamma)$, in which case

\Centerline{$\tilde W_K(x)\le\tilde W_{A\cap B(\tbf{0},\gamma)}(x)<1$}

\noindent for every $x\in K$, and $\zeta K=0$, by 479La.\ \BanG\  So
$\zeta E=0$, as required.

\medskip

\quad{\bf (iv)} By (ii) and (c-i),

\Centerline{$\energy(\lambda_A)=\int\tilde W_Ad\lambda_A
\le\lambda_A\BbbR^r=c(A)$.}

\noindent In the other direction, for any $\gamma\ge 0$,

$$\eqalignno{\energy(\lambda_A)
&=\int\tilde W_Ad\lambda_A
\ge\int\tilde W_{A\cap B(\tbf{0},\gamma)}d\lambda_A
=\int\tilde W_Ad\lambda_{A\cap B(\tbf{0},\gamma)}\cr
\displaycause{479J(b-i)}
&\ge\int\tilde W_{A\cap B(\tbf{0},\gamma)}d\lambda_{A\cap B(\tbf{0},\gamma)}
=c(A\cap B(\tbf{0},\gamma));\cr}$$

\noindent taking the limit as $\gamma\to\infty$,
$\energy(\lambda_A)\ge c(A)$ and we have equality.
}%end of proof of 479M

\leader{479N}{}\cmmnt{ We are ready to match the definitions
in 479C to some alternative definitions of capacity.

\medskip

\noindent}{\bf Theorem} Let $A\subseteq\BbbR^r$ be an analytic set with
finite Choquet-Newton capacity $c(A)$.

(a) Writing $W_{\zeta}$ for the Newtonian potential of a Radon measure
$\zeta$ on $\BbbR^r$,

\Centerline{$c(A)
=\sup\{\zeta A:\zeta\text{ is a Radon measure on }\BbbR^r,\,
  W_{\zeta}(x)\le 1\text{ for every }x\in\BbbR^r\}$;}

\noindent if $A$ is closed, the supremum is attained.

(b) $c(A)
=\inf\{\energy(\zeta):\zeta\text{ is a Radon measure on }\BbbR^r,\,
  \zeta A\ge c(A)\}$;
if $A$ is closed, the infimum is attained.

(c) If $A\ne\emptyset$, $c(A)
=\sup\{\Bover1{\energy(\zeta)}:\zeta$ is a Radon measure on
$\BbbR^r$ such that $\zeta A=1\}$, counting $\Bover1{\infty}$ as zero;
if $A$ is closed, the supremum is attained.

\proof{ Note first that if there is
a Radon measure $\zeta$ on $\BbbR^r$, with finite energy, such that
$\zeta A>0$, then $c(A)>0$.   \Prf\ By 479M(d-iii),
$\tilde W_A=1\,\,\zeta$-a.e.\ on $A$.   So $\tilde W_A$ cannot be
identically $0$, and $0<\lambda_A\BbbR^r=c(A)$, by 479M(c-i).\ \Qed

\medskip

{\bf (a)(i)} We know from 479E(d-i) and 479D(b-i) that

$$\eqalign{c(A)
&=\sup\{\capacity K:K\subseteq A\text{ is compact}\}
=\sup\{\lambda_KK:K\subseteq A\text{ is compact}\}\cr
&=\sup\{\lambda_KA:K\subseteq A\text{ is compact}\}
\le\sup\{\zeta A:W_{\zeta}\le\chi\BbbR^r\}.\cr}$$

\medskip

\quad{\bf (ii)} If $\zeta$ is a Radon measure on $\BbbR^r$ and
$W_{\zeta}\le\chi\BbbR^r$, then

$$\eqalignno{\zeta A
&=\sup_{K\subseteq A\text{ is compact}}\zeta K
\le\sup_{K\subseteq A\text{ is compact}}\capacity K\cr
\displaycause{479Lb}
&=c(A).\cr}$$

\noindent Thus $\sup\{\zeta A:W_{\zeta}\le\chi\BbbR^r\}\le c(A)$ and we
have equality.

\medskip

\quad{\bf (iii)} If $A$ is closed, then by 479M(c-ii)

\Centerline{$\lambda_AA=\lambda_A(\partial A)=\lambda_A\BbbR^r
=c(A)$}

\noindent so $\lambda_A$ witnesses that the supremum is attained.

\medskip

{\bf (b)(i)} \Quer\ Suppose, if possible, that there is a Radon
measure $\zeta$ on $\BbbR^r$ such
that $\zeta A\ge c(A)>\energy(\zeta)$.   Let $\alpha\in\ooint{0,1}$
be such that $\alpha^4c(A)\ge\energy(\zeta)$.   Since

\Centerline{$\zeta A=\sup\{\zeta K:K\subseteq A$ is compact$\}$,
\quad$c(A)=\sup\{\capacity K:K\subseteq A$ is compact$\}$,}

\noindent there is a compact $K\subseteq A$ such that
$\zeta K\ge\alpha\zeta A$ and $\capacity K>\alpha^2c(A)$.
Set $\zeta'=\Bover{\capacity K}{\zeta K}\zeta$.   Then

$$\eqalign{\energy(\zeta')
&=\bigl(\Bover{\capacity K}{\zeta K}\bigr)^2\energy(\zeta)
\le\bigl(\Bover{c(A)}{\alpha\zeta A}\bigr)^2\alpha^4c(A)\cr
&\le\alpha^2c(A)
<\capacity K
=\zeta'K;\cr}$$

\noindent which is impossible, by 479K.\ \Bang

So $c(A)\le\inf\{\energy(\zeta):\zeta A\ge c(A)\}$.

\medskip

\quad{\bf (ii)} Take any $\epsilon>0$.
Then there is a compact set $K\subseteq A$ such that
$(1+\epsilon)\capacity K\ge c(A)$.   Set
$\zeta=(1+\epsilon)\lambda_K$;  then

\Centerline{$\zeta A\ge c(A)$,
\quad$\energy(\zeta)=(1+\epsilon)^2\energy(\lambda_K)
=(1+\epsilon)^2\capacity K\le(1+\epsilon)^2c(A)$.}

\noindent As $\epsilon$ is arbitrary,
$c(A)\ge\inf\{\energy(\zeta):\zeta A\ge c(A)\}$
and we have equality.

\medskip

\quad{\bf (iii)} If $A$ is closed, then

\Centerline{$\lambda_AA=\lambda_A(\partial A)=\lambda_A\BbbR^r=c(A)$}

\noindent by 479M(c-i) and (c-ii), while $\energy(\lambda_A)=c(A)$ by
479M(d-iv).   So $\lambda_A$ witnesses that
$c(A)=\min\{\energy(\zeta):\zeta A\ge c(A)\}$.

\medskip

{\bf (c)(i)} Suppose that
$\zeta$ is a Radon measure on $\BbbR^r$
such that $\zeta A=1$.   If $\energy(\zeta)=\infty$ then of course
$\Bover1{\energy(\zeta)}\le c(A)$.   Otherwise, $c(A)>0$, as
remarked at the beginning of this part of the proof.   Set
$\zeta'=c(A)\zeta$.   By (b),

\Centerline{$c(A)\le\energy(\zeta')=c(A)^2\energy(\zeta)$,}

\noindent so $\Bover1{\energy(\zeta)}\le c(A)$.

Thus $\sup\{\Bover1{\energy(\zeta)}:\zeta A=\zeta\BbbR^r=1\}\le c(A)$.

\medskip

\quad{\bf (ii)} If $c(A)=0$ then the supremum is attained by any
Radon measure $\zeta$ such that $\zeta A=1$, so we can stop.
If $c(A)>0$, then for any $\alpha\in\ooint{0,1}$ there is a compact set
$K\subseteq A$ such that $\capacity K\ge\alpha c(A)$.   Set
$\zeta=\Bover1{\capacity K}\lambda_K$;  then

\Centerline{$\zeta K=\zeta\BbbR^r=\zeta A=1$}

\noindent and

\Centerline{$\Bover1{\energy(\zeta)}
=\Bover{(\capacity K)^2}{\energy(\lambda_K)}
=\capacity K\ge\alpha c(A)$.}

\noindent As $\alpha$ is arbitrary,
$c(A)\le\sup\{\Bover1{\energy(\zeta)}:\zeta A=\zeta\BbbR^r=1\}$ and
we have equality.

\medskip

\quad{\bf (iii)} If $A$ is closed and $c(A)>0$, then
$\zeta=\Bover1{c(A)}\lambda_A$ witnesses that the supremum is
attained, as in (b) above.
}%end of proof of 479N

\leader{479O}{Polar \dvrocolon{sets}}\cmmnt{ To make the final step, to
arbitrary sets with finite Choquet-Newton capacity, we seem to
need an alternative description of polar sets.

\medskip

\noindent}{\bf Proposition} For a set $D\subseteq\BbbR^r$,
the following are equiveridical:

(i) $D$ is polar\cmmnt{, that is, $c(D)=0$};

(ii) there is a totally finite Radon measure $\zeta$ on $\BbbR^r$ such that
its Newtonian potential $W_{\zeta}$ is infinite at every point of $D$;

(iii) there is an analytic set $E\supseteq D$ such that
$\zeta E=0$ whenever $\zeta$ is a Radon measure on $\BbbR^r$ with
finite energy.

\proof{{\bf (i)$\Rightarrow$(ii)} If (i) is true, then for each
$n\in\Bbb N$ there is a bounded open set
$G_n\supseteq D\cap B(\tbf{0},n)$ such that
$c(G_n)\le 2^{-n}$.   Try $\zeta=\sum_{n=0}^{\infty}\lambda_{G_n}$,
defining the sum as in 234G.   Then
$\zeta\BbbR^r=\sum_{n=0}^{\infty}c(G_n)$ is finite, and
$W_{\zeta}=\sum_{n=0}^{\infty}\tilde W_{G_n}$ (234Hc).
If $x\in D\cap B(\tbf{0},n)$, then $\tilde W_{G_m}(x)=1$ for every $m\ge n$
(479D(b-iii)), so $W_{\zeta}(x)=\infty$.   Thus $\zeta$ witnesses that (ii)
is true.

\medskip

{\bf (ii)$\Rightarrow$(iii)} Suppose that $\lambda$ is a totally finite
Radon measure such that $W_{\lambda}(x)=\infty$ for every $x\in D$.
Set $E=\{x:W_{\lambda}(x)=\infty\}$;  then $E$ is a G$_{\delta}$ set,
because $W_{\lambda}$ is lower semi-continuous (479Fa).
\Quer\ If there is a Radon
measure $\zeta$ on $\BbbR^r$, with finite energy, such that $\zeta E>0$,
let $K\subseteq E$ be a compact set such that $\zeta K>0$.   Set
$\zeta_1=\Bover1{\zeta K}\zeta\LLcorner K$;  then $\zeta_1$ has finite
energy and $\zeta_1K=1$, so $\capacity K\ge\Bover1{\energy(\zeta_1)}>0$,
by 479Nc.

Let $G\supseteq K$ be a bounded open set;  set
$\lambda_1=\lambda\LLcorner G$ and
$\lambda_2=\lambda\LLcorner(\BbbR^r\setminus G)$, so that
$\lambda=\lambda_1+\lambda_2$ and $W_{\lambda}=W_{\lambda_1}+W_{\lambda_2}$
(234Hc).   Since $W_{\lambda_2}(x)$ is finite for $x\in G$ (479Fa),
$W_{\lambda_1}(x)=\infty$ for every $x\in K$.   Let $\epsilon>0$ be such
that $\epsilon\lambda_1\BbbR^r<\capacity K$.   Then
$\epsilon W_{\lambda_1}$ is a lower
semi-continuous superharmonic function greater than or equal to
$\tilde W_K$ on $K\supseteq\supp(\lambda_K)$, so
$\epsilon W_{\lambda_1}\ge\tilde W_K$ everywhere (479Fg).   But this
means that

$$\eqalignno{\epsilon\lambda_1\BbbR^r
&=\epsilon\lim_{\|x\|\to\infty}\|x\|^{r-2}W_{\lambda_1}(x)\cr
\displaycause{479Fd}
&\ge\lim_{\|x\|\to\infty}\|x\|^{r-2}\tilde W_K(x)
=\lambda_K\BbbR^r
=\capacity K
>\epsilon\lambda_1\BbbR^r,\cr}$$

\noindent which is absurd.\ \Bang

So $E$ witnesses that (iii) is true.

\medskip

{\bf (iii)$\Rightarrow$(i)} Suppose that $E\supseteq D$ is analytic and
that $\zeta E=0$ whenever $\energy(\zeta)$ is finite.   If
$K\subseteq E$ is compact and $\zeta$ is a Radon probability measure on
$\BbbR^r$ such that $\zeta K=1$, then $\energy(\zeta)$ must be infinite;
by 479Nc, $\capacity K=0$.   As $K$ is arbitrary, $c(E)=0$ and
$c(D)=0$.
}%end of proof of 479O

\leader{479P}{}\cmmnt{ At last I come to my final extension of the
notions of equilibrium measure and potential, together with a direct
expression of the latter in terms of Brownian hitting probabilities.

\medskip

\noindent}{\bf Theorem} Let $D\subseteq\BbbR^r$ be a set with finite
Choquet-Newton capacity $c(D)$.

(a) There is a totally finite Radon
measure $\lambda_D$ on $\BbbR^r$ such that
$\lambda_D=\lambda_A$, as defined in 479Mb, whenever
$A\supseteq D$ is analytic and $c(A)=c(D)$.

(b) Write
$\tilde W_D=W_{\lambda_D}$ for the equilibrium potential
corresponding to the equilibrium measure $\lambda_D$.   Then
$\tilde W_D(x)=\hp^*((D\setminus\{x\})-x)$ for every $x\in\BbbR^r$.

(c)(i)($\alpha$) $\lambda_D\BbbR^r=c(D)$;

\qquad($\beta$) if $\zeta$ is any Radon measure on $\BbbR^r$ with finite energy,
$\tilde W_D(x)=1$ for $\zeta$-almost every $x\in D$;

\qquad($\gamma$) $\energy(\lambda_D)=c(D)$;

\qquad($\delta$) if $D'\subseteq D$ and $c(D')=c(D)$, then
$\lambda_{D'}=\lambda_D$.

\quad(ii) $\supp(\lambda_D)\subseteq\partial D$.

\quad(iii) For any $D'\subseteq\BbbR^r$ such that $c(D')<\infty$,

\qquad($\alpha$) $\lambda_D^*(D')\le c(D')$;

\qquad($\beta$) $\lambda_{D\cup D'}\le\lambda_D+\lambda_{D'}$;

\qquad($\gamma$) $\tilde W_{D\cap D'}+\tilde W_{D\cup D'}
\le\tilde W_D+\tilde W_{D'}$;

\qquad($\delta$) $\rhotv(\lambda_D,\lambda_{D'})\le 2c(D\symmdiff D')$.

\quad(iv) If $\sequencen{D_n}$ is a non-decreasing sequence of sets with
union $D$, then

\qquad($\alpha$) $\tilde W_D=\lim_{n\to\infty}\tilde W_{D_n}
=\sup_{n\in\Bbb N}\tilde W_{D_n}$;

\qquad($\beta$) $\sequencen{\lambda_{D_n}}\to\lambda_D$ for the narrow
topology on $M^+_{\text{R}}(\BbbR^r)$.

\quad(v) $c(D)
=\inf\{\zeta\BbbR^r:\zeta\text{ is a Radon measure on }\BbbR^r,\,
   W_{\zeta}\ge\chi D\}$

\quad$\phantom{\text{(v) }c(D)}
=\inf\{\energy(\zeta):\zeta\text{ is a Radon measure on }\BbbR^r,\,
   W_{\zeta}\ge\chi D\}$.

\quad(vi) Writing $\clstar D$ for the essential closure of $D$,
$c(\clstar D)\le c(D)$ and $\tilde W_{\clstar D}\le\tilde W_D$.

\quad(vii) Suppose that $f:D\to\BbbR^r$ is $\gamma$-Lipschitz, where
$\gamma\ge 0$.   Then $c(f[D])\le\gamma^{r-2}c(D)$.

\proof{{\bf (a)(i)} If $A$, $B\subseteq\BbbR^r$ are analytic sets,
$c(B)<\infty$ and $A\subseteq B$, then

$$\eqalignno{\tilde W_A
&=\sup_{n\in\Bbb N}\tilde W_{A\cap B(\tbf{0},n)}\cr
\displaycause{479M(d-i)}
&\le\sup_{n\in\Bbb N}\tilde W_{B\cap B(\tbf{0},n)}\cr
\displaycause{479D(b-ii)}
&=\tilde W_B.\cr}$$

\noindent If $c(A)=c(B)$, then $\lambda_A=\lambda_B$.   \Prf\

$$\eqalignno{c(A)
&=\energy(\lambda_A)\cr
\displaycause{479M(d-iv)}
&=\int\tilde W_Ad\lambda_A
\le\int\tilde W_Bd\lambda_A
=\int\tilde W_Ad\lambda_B\cr
\displaycause{479J(b-i)}
&\le\int\tilde W_Bd\lambda_A
=\energy(\lambda_B)
=c(B)
=\lambda_B\BbbR^r}$$

\noindent (479M(c-i)).   So we must have equality throughout, and
$\tilde W_A=\tilde W_B\,\,\lambda_B$-a.e.   By 479Fg,
$\tilde W_A\ge\tilde W_B$ everywhere and

\Centerline{$W_{\lambda_B}=\tilde W_B=\tilde W_A=W_{\lambda_A}$.}

\noindent By 479J(b-v), $\lambda_B=\lambda_A$. \Qed

\quad{\bf (ii)} Now consider the given set $D$.
By 479E(d-i), there is an analytic set $A\supseteq D$ such
that $c(A)=c(D)$.   If $B$ is another such set, then
$c(A\cap B)=c(A)=c(B)$, so $\lambda_{A\cap B}=\lambda_A=\lambda_B$.
We therefore have a common measure which we can take to be
$\lambda_D$.   Of course this agrees with 479Mb if $D$ itself is analytic,
and with 479B if $D$ is bounded and analytic.

\medskip

{\bf (b)} Write $h_D(x)$ for $\hp^*((D\setminus\{x\})-x)$.

\medskip

\quad{\bf (i)} To begin with, suppose that
$D=K$ is compact and that $x\notin K$, so that $h_D(x)=h_K(x)=\hp(K-x)$.

\medskip

\qquad\grheada\ $h_K(x)\ge\tilde W_K(x)$.   \Prf\ 479Lc.\ \Qed

\medskip

\qquad\grheadb\ In fact $h_K(x)=\tilde W_K(x)$.  \Prf\ Let $\epsilon>0$.
Set $E=\{y:y\in K$, $\tilde W_K(y)<1\}$.   Because $\tilde W_K$ is lower
semi-continuous, $E$ is an F$_{\sigma}$ set, therefore analytic;
by 479M(d-iii), $E$ satisfies condition (iii) of 479O, and is polar.
By 479O(ii), there is a totally finite Radon measure $\zeta$
on $\BbbR^r$ such that $W_{\zeta}(y)=\infty$ for every $y\in E$.
Let $H$ be a bounded open set, including $K$, such that
$x\notin\overline{H}$;  set $\zeta_1=\zeta\LLcorner H$ and
$\zeta_2=\zeta\LLcorner(\BbbR^r\setminus H)$.
Then $\zeta=\zeta_1+\zeta_2$, so
$W_{\zeta}=W_{\zeta_1}+W_{\zeta_2}$ (479J(b-iii)).
Since $H$ is open and $\zeta_2$-negligible, $W_{\zeta_2}(y)$ is
finite for every $y\in K$ (479Fa), and $W_{\zeta_1}(y)=\infty$ for every
$y\in E$;  while $W_{\zeta_1}(x)$ is finite because the support of
$\zeta_1$ is included in $\overline{H}$.

There is therefore an $\eta>0$ such that $\eta W_{\zeta_1}(x)\le\epsilon$.
Consider $\lambda=\lambda_K+\eta\zeta_1$.   We have
$W_{\lambda}(y)\ge 1$ for every $y\in K$, while $W_{\lambda}$ is
superharmonic and lower semi-continuous (479Fa, 479Fb);  as
the support of $\lambda$ is included in the compact set $\overline{H}$,
$\lim_{\|y\|\to\infty}W_{\lambda}(y)=0$ (479Fd).   Consequently

$$\eqalignno{h_K(x)
&=\mu_x^{(K)}(K)
\le\int W_{\lambda}d\mu_x^{(K)}
\le W_{\lambda}(x)\cr
\displaycause{478Pc, with $G=\BbbR^r\setminus K$}
&\le\tilde W_K(x)+\epsilon.\cr}$$

\noindent As $\epsilon$ is arbitrary, $h_K(x)\le\tilde W_K(x)$ and we
have equality.\ \Qed

\medskip

\quad{\bf (ii)} If $D=A$ is analytic,
note that $\capacity\{x\}=0$ (479Da, or otherwise), so
$c(A\setminus\{x\})=c(A)$, because $c$ is monotonic and submodular,
therefore subadditive (479E(d-ii)).   Now we know that

\Centerline{$h_A(x)
=\sup\{\hp(K-x):K\subseteq A\setminus\{x\}$ is compact$\}$}

\noindent (477Ie) and

\Centerline{$c(A\setminus\{x\})
=\sup\{\capacity K:K\subseteq A\setminus\{x\}$ is compact$\}$}

\noindent (479E(d-iii)).   So there is a non-decreasing sequence
$\sequencen{K_n}$ of compact subsets of $A\setminus\{x\}$ such that

\Centerline{$h_A(x)
=\sup_{n\in\Bbb N}\hp(K_n-x)$,
\quad$c(A\setminus\{x\})
=\sup_{n\in\Bbb N}\capacity K_n$.}

\noindent Set $E=\bigcup_{n\in\Bbb N}K_n$;  then $E\subseteq A$ and
$c(E)=c(A)$, so $\lambda_E=\lambda_A$ ((a-i) above) and
$\tilde W_E=\tilde W_A$.   Accordingly

$$\eqalignno{h_A(x)
&=\sup_{n\in\Bbb N}\hp(K_n-x)
=\sup_{n\in\Bbb N}\tilde W_{K_n}(x)\cr
\displaycause{(a-i) above}
&=\sup_{m,n\in\Bbb N}\tilde W_{K_n\cap B(\tbf{0},m)}(x)
=\sup_{m\in\Bbb N}\tilde W_{E\cap B(\tbf{0},m)}(x)\cr
\displaycause{apply 479E(b-iii) twice}
&=\tilde W_E(x)
=\tilde W_A(x).\cr}$$

\medskip

\quad{\bf (iii)} For the general case, note first that
$h_D\le\tilde W_D$.   \Prf\
There is a G$_{\delta}$ set $E\supseteq D$ such that $c(E)=c(D)$, so
$\lambda_E=\lambda_D$.   Now, for any $x\in\BbbR^r$,

\Centerline{$h_D(x)\le h_E(x)=\tilde W_E(x)=\tilde W_D(x)$,}

\noindent using (ii) for the central equality.\ \Qed

Equally, $h_D\ge\tilde W_D$.   \Prf\ If $x\in\BbbR^r$,
there is a G$_{\delta}$ set $H\supseteq(D\setminus\{x\})-x$ such that

\Centerline{$h_D(x)=\hp^*((D\setminus\{x\})-x)=\hp H$}

\noindent (477Id).   Set $A=(H+x)\cup\{x\}$;  then $A\supseteq D$ and

\Centerline{$h_D(x)=h_A(x)=\tilde W_A(x)\ge\tilde W_{A\cap E}(x)
=\tilde W_D(x)$,}

\noindent using (a-i) again for the inequality.\ \Qed

So $h_D=\tilde W_D$, as claimed.

\medskip

{\bf (c)} Fix an analytic set $A\supseteq D$ such that $c(A)=c(D)$;
replacing $A$ by $A\cap\overline{D}$ if necessary, we may suppose that
$A\subseteq\overline{D}$.   We have
$\lambda_D=\lambda_A$ and $\tilde W_D=\tilde W_A$.

\medskip

\quad{\bf (i)}\grheada\

\Centerline{$\lambda_D\BbbR^r=\lambda_A\BbbR^r=c(A)=c(D)$}

\noindent by 479M(c-i).

\medskip

\qquad\grheadb-\grheadc\ 479M(d-iii) tells us that
$\tilde W_D(x)=\tilde W_A(x)=1$ for $\zeta$-almost every $x\in A$, and
therefore for $\zeta$-almost every $x\in D$.
At the same time,

\Centerline{$\energy(\lambda_D)=\energy(\lambda_A)=c(A)=c(D)$}

\noindent by 479M(d-iv).

\medskip

\qquad\grheadd\ Of course $A\supseteq D'$ and $c(A)=c(D')$, so
$\lambda_{D'}=\lambda_A=\lambda_D$.

\medskip

\quad{\bf (ii)}
$\overline{A}=\overline{D}$ and $\interior A\supseteq\interior D$, so
$\partial A\subseteq\partial D$ and

\Centerline{$\lambda_D(\BbbR^r\setminus\partial D)
=\lambda_A(\BbbR^r\setminus\partial D)
\le\lambda_A(\BbbR^r\setminus\partial A)
=0$}

\noindent by 479M(c-ii).   As $\partial D$ is closed, it includes
$\supp(\lambda_D)$.

\medskip

\quad{\bf (iii)} Let $A'\supseteq D'$ be an analytic set such that
$c(A')=c(D')$.

\qquad\grheada\

$$\eqalignno{\lambda^*_D(D')
&\le\lambda_D(A')
=\lambda_A(A')
\le\sup_{m\in\Bbb N}\lambda_{A\cap B(\tbf{0},m)}(A')\cr
\displaycause{479Mb}
&=\sup_{m,n\in\Bbb N}
    \lambda_{A\cap B(\tbf{0},m)}(A'\cap B(\tbf{0},n))
\le\sup_{n\in\Bbb N}c(A'\cap B(\tbf{0},n))\cr
\displaycause{479D(c-ii)}
&=c(A')\cr
\displaycause{because $c$ is a capacity}
&=c(D').\cr}$$

\medskip

\qquad\grheadb\ Because $c$ is subadditive, we know that $c(D\cup D')$ is
finite.   Let $B\supseteq D\cup D'$ be an analytic set such that
$c(B)=c(D\cup D')$.   Then

$$\eqalignno{\lambda_{D\cup D'}
&=\lambda_{B\cap(A\cup A')}
\le\lambda_{B\cap A}+\lambda_{B\cap A'}\cr
\displaycause{479M(c-iii)}
&=\lambda_D+\lambda_{D'}.\cr}$$

\medskip

\qquad\grheadc\ This is immediate from (b) and the general fact that
$\zeta^*(U\cap V)+\zeta^*(U\cup V)\le\zeta^*U+\zeta^*V$ for any measure
$\zeta$ and any sets $U$ and $V$ (132Xk).

\medskip

\qquad\grheadd\ As usual, it will be enough to show that
$|\lambda_DE-\lambda_{D'}E|\le c(D\symmdiff D')$ for every Borel set
$E\subseteq\BbbR^r$;  by symmetry, all we need to check is that
$\lambda_{D'}E\le\lambda_DE+c(D\symmdiff D')$ for every Borel set $E$.
\Prf

\medskip

\qquad\quad{\bf case 1} Suppose that $D$ and $D'$ are both bounded Borel
sets.   Take $x\in\BbbR^r$, and let $\tau$, $\tau':\Omega\to[0,\infty]$
be the Brownian arrival times to $D-x$, $D'-x$ respectively.   Then

$$\eqalign{\mu_x^{(D')}E
&=\mu_W\{\omega:\tau'(\omega)<\infty,\,x+\omega(\tau'(\omega))\in E\}\cr
&\le\mu_W\{\omega:\tau(\omega)<\infty,\,x+\omega(\tau(\omega))\in E\}
   +\mu_W\{\omega:\tau(\omega)\ne\tau'(\omega)\}\cr
&\le\mu_x^{(D)}E+\mu_W\{\omega:\text{there is some }t\ge 0
   \text{ such that }x+\omega(t)\in D\symmdiff D'\}\cr
&=\mu_x^{(D)}E+\mu_x^{(D\symmdiff D')}\BbbR^r.\cr}$$

\noindent So

$$\eqalign{\lambda_{D'}E
&=\lim_{\|x\|\to\infty}\|x\|^{r-2}\mu_x^{(D')}E\cr
&\le\lim_{\|x\|\to\infty}\|x\|^{r-2}\mu_x^{(D)}E
   +\lim_{\|x\|\to\infty}\|x\|^{r-2}\mu_x^{(D\symmdiff D')}\BbbR^r
=\lambda_DE+c(D\symmdiff D').\cr}$$

\medskip

\qquad\quad{\bf case 2} Suppose that $D$, $D'$ are Borel sets, not
necessarity bounded.   Set $D_n=D\cap B(\tbf{0},n)$, $D'_n=D'\cap B(\tbf{0},n)$.
Then

$$\eqalignno{\lambda_{D'}E
&=\lim_{n\to\infty}\lambda_{D'_n}E\cr
\displaycause{479Mb}
&\le\lim_{n\to\infty}\lambda_{D_n}E
   +\lim_{n\to\infty}c(D_n\symmdiff D'_n)\cr
\displaycause{by case 1}
&=\lambda_DE
   +\lim_{n\to\infty}c((D\symmdiff D')\cap B(\tbf{0},n))
=\lambda_DE+c(D\symmdiff D')\cr}$$

\noindent because $c$ is a capacity.

\medskip

\qquad\quad{\bf case 3} In general, let $G\supseteq D$, $G'\supseteq D'$
and $H\supseteq D\symmdiff D'$ be G$_{\delta}$ sets such that
$c(G)=c(D)$, $c(G')=c(D')$ and $c(H)=c(D\symmdiff D')$.   Set

\Centerline{$G_1=G\cap(G'\cup H)$,
\quad$G'_1=G'\cap(G\cup H)$;}

\noindent these are Borel sets, while $D\subseteq G_1\subseteq G$,
$D'\subseteq G'_1\subseteq G'$ and $G_1\symmdiff G_2\subseteq H$.
So

$$\eqalignno{\lambda_{D'}E
&=\lambda_{G'_1}E
\le\lambda_{G_1}E+c(G\symmdiff G_1)\cr
\displaycause{by case 2}
&\le\lambda_DE+c(H)
=\lambda_DE+c(D\symmdiff D')\cr}$$

\noindent and we have the result in this case also.   So we're done.\ \Qed

\medskip

\quad{\bf (iv)}\grheada\ This follows immediately from (b) above.

\medskip

\qquad\grheadb\
Consider first the case in which every $D_n$ is analytic.
Returning to the proof of 479M, or putting 479Ma together
with (iii-$\delta$) here, we see that for any $m$, $n\in\Bbb N$ we shall
have

\Centerline{$\rhotv(\lambda_{D_n},\lambda_{D_n\cap B(\tbf{0},m)})
\le 2c(D_n\setminus B(\tbf{0},m))\le 2c(D\setminus B(\tbf{0},m))=2\alpha_m$}

\noindent say, and that $\lim_{m\to\infty}\alpha_m=0$.   So if
$G\subseteq\BbbR^r$ is any open set,

$$\eqalignno{\lambda_DG
&=\lim_{m\to\infty}\lambda_{D\cap B(\tbf{0},m)}G
\le\lim_{m\to\infty}\liminf_{n\to\infty}\lambda_{D_n\cap B(\tbf{0},m)}G\cr
\displaycause{479E(c-i)}
&\le\lim_{m\to\infty}\liminf_{n\to\infty}\lambda_{D_n}G+2\alpha_m
=\liminf_{n\to\infty}\lambda_{D_n}G.\cr}$$

\noindent Since we know also that

\Centerline{$\lambda_D\BbbR^r=c(D)=\lim_{n\to\infty}c(D_n)
=\lim_{n\to\infty}\lambda_{D_n}\BbbR^r$,}

\noindent $\sequencen{\lambda_{D_n}}\to\lambda_D$ for the narrow topology.

For the general case, take analytic sets $A_n\supseteq D_n$, $A\supseteq D$
such that $c(A_n)=c(D_n)$ for every $n$ and $c(A)=c(D)$.   Set
$A'_n=A\cap\bigcap_{m\ge n}A_m$ for each $n$, $A'=\bigcup_{n\in\Bbb N}A_n$;
then

\Centerline{$\sequencen{\lambda_{D_n}}
=\sequencen{\lambda_{A'_n}}\to\lambda_{A'}=\lambda_D$}

\noindent for the narrow topology.

\medskip

{\bf (v)} Let $Q$ be the set of Radon measures $\zeta$ on $\BbbR^r$ such
that $W_{\zeta}\ge\chi D$.

\medskip

\qquad\grheada\ I show first that
$\inf_{\zeta\in Q}\zeta\BbbR^r$ and $\inf_{\zeta\in Q}\energy(\zeta)\}$
are both less than or equal to $c(D)$.
\Prf\ Let $\epsilon>0$.   Because $c$ is outer regular (479E(d-i)),
there is an open set $G\supseteq D$ such that $c(G)\le c(D)+\epsilon$.
Set $\zeta=\lambda_G$.   Then

\Centerline{$W_{\zeta}=\tilde W_G\ge\chi G\ge\chi D$}

\noindent (479D(b-iii)), so $\zeta\in Q$, while

\Centerline{$\zeta\BbbR^r=\energy(\zeta)=c(G)\le c(D)+\epsilon$.  \Qed}

\medskip

\qquad\grheadb\ Now suppose that $\zeta\in Q$.   Then
$c(D)\le\min(\zeta\BbbR^r,\energy(\zeta))$.   \Prf\
Take any $\gamma<c(D)$ and $\epsilon>0$.
Let $A\supseteq D$ be an analytic set such that
$c(A)=c(D)$;  replacing $A$ by $\{x:x\in A$, $W_{\zeta}(x)\ge 1\}$ if
necessary, we can suppose that $W_{\zeta}\ge\chi A$.
For each $n\in\Bbb N$, let $\zeta_n$ be the totally finite measure
$(1+\epsilon)\zeta\LLcorner B(\tbf{0},n)$.
Then $\sequencen{W_{\zeta_n}}$ is non-decreasing
and has supremum $(1+\epsilon)W_{\zeta}$ (479M(d-i)),
so $A=\bigcup_{n\in\Bbb N}A_n$,
where $A_n=\{x:x\in A$, $W_{\zeta_n}(x)\ge 1\}$.   There are an
$n\in\Bbb N$ such that $c(A_n)>\gamma$ and a compact
$K\subseteq A_n$ such that $\capacity K\ge\gamma$ (432K).   Now
$W_{\zeta_n}\ge\tilde W_K\,\,\lambda_K$-a.e., so
$W_{\zeta_n}\ge\tilde W_K$ everywhere (479Fg) and

$$\eqalignno{\gamma
&\le\capacity K
=\int\tilde W_Kd\lambda_K
\le\int W_{\zeta_n}d\lambda_K
=\int\tilde W_Kd\zeta_n\cr
\displaycause{479J(b-i)}
&\le\int W_{\zeta_n}d\zeta_n
\le(1+\epsilon)\int W_{\zeta}d\zeta_n
\le(1+\epsilon)^2\int W_{\zeta}d\zeta
=(1+\epsilon)^2\energy(\zeta).\cr}$$

\noindent Moreover, 479J(c-vi), applied to $\zeta_n$ and $\lambda_K$,
tells us that

\Centerline{$\zeta\BbbR^r\ge\zeta_n\BbbR^r\ge\lambda_K\BbbR^r
=\capacity K\ge\gamma$.}

\noindent As $\gamma$ and $\epsilon$ are arbitrary,
$c(D)\le\min(\energy(\zeta),\zeta\BbbR^r)$, as claimed.\ \Qed

\medskip

\qquad\grheadc\ Putting these together, we see that
$c(D)=\inf_{\zeta\in Q}\zeta\BbbR^r=\inf_{\zeta\in Q}\energy(\zeta)$.

\medskip

\quad{\bf (vi)} If $x\in\clstar A$, then
$0\in\clstar((A\setminus\{x\})-x)$ and
$\tilde W_A(x)=\hp^*((A\setminus\{x\})-x)=1$ for every $x\in\clstar E$, by
478U and (b) above.   Now

$$\eqalignno{c(\clstar D)
&\le c(\clstar A)
\le\energy(\lambda_A)\cr
\displaycause{(v) above}
&=c(A)\cr
\displaycause{479M(d-iv)}
&=c(D).\cr}$$

\medskip

\quad{\bf (vii)}\grheada\ Consider first the case $D=A$, so
that $f[D]=f[A]$ is analytic.   We can suppose that $c(f[A])>0$,
in which case $A\ne\emptyset$ and $\gamma>0$.   Take any
$\epsilon>0$.   By 479Nc there is a Radon measure $\zeta$ on
$\BbbR^r$ such that $\zeta f[A]=1$ and
$c(f[A])\le\Bover{1+\epsilon}{\energy(\zeta)}$.   Applying 433D to
the subspace measure $\zeta_{f[A]}$, we see that there is a Radon
probability measure $\zeta'$ on $A$ such that $\zeta_{f[A]}$ is the image
measure $\zeta'f^{-1}$;  let $\lambda$ be the Radon probability measure on
$\BbbR^r$ extending $\zeta'$.   Then

$$\eqalignno{\energy(\zeta)
&=\int_{\BbbR^r}\int_{\BbbR^r}\Bover1{\|x-y\|^{r-2}}\zeta(dx)\zeta(dy)
\ge\int_{f[A]}\int_{f[A]}\Bover1{\|x-y\|^{r-2}}\zeta(dx)\zeta(dy)\cr
&=\int_A\int_{f[A]}\Bover1{\|x-f(v)\|^{r-2}}\zeta(dx)\zeta'(dv)
=\int_A\int_A\Bover1{\|f(u)-f(v)\|^{r-2}}\zeta'(du)\zeta'(dv)\cr
\displaycause{applying 235J\formerly{2{}35L} twice}
&\ge\int_A\int_A\Bover1{\gamma^{r-2}\|u-v\|^{r-2}}\zeta'(du)\zeta'(dv)\cr
&=\Bover1{\gamma^{r-2}}\int_{\BbbR^r}\int_{\BbbR^r}
   \Bover1{\|u-v\|^{r-2}}\lambda(du)\lambda(dv)
=\Bover1{\gamma^{r-2}}\energy(\lambda).\cr}$$

\noindent By 479Nc in the other direction,

\Centerline{$c(A)\ge\Bover1{\energy(\lambda)}
\ge\Bover1{\gamma^{r-2}\energy(\zeta)}
\ge\Bover1{(1+\epsilon)\gamma^{r-2}}c(f[A])$.}

\noindent As $\epsilon$ is arbitrary, $c(f[A])\le\gamma^{r-2}c(A)$.

\medskip

\qquad\grheadb\ In general, since $f:D\to\BbbR^r$ is certainly uniformly
continuous, it has a continuous extension $g:\overline{D}\to\BbbR^r$
(3A4G), which is still $\gamma$-Lipschitz.   Now ($\alpha$) tells us that

\Centerline{$c(f[D])\le c(g[A])
\le\gamma^{r-2}c(A)=\gamma^{r-2}c(D)$,}

\noindent as required.
}%end of proof of 479P

\leader{479Q}{Hausdorff measure:  Theorem}
For $s\in\ooint{0,\infty}$ let $\mu_{Hs}$ be
Hausdorff $s$-dimensional measure on $\BbbR^r$.
Let $D$ be any subset of $\BbbR^r$.

(a) If the Choquet-Newton capacity $c(D)$ is non-zero,
then $\mu^*_{H,r-2}D=\infty$.

(b) If $s>r-2$ and $\mu^*_{Hs}D>0$, then $c(D)>0$.
%Mattila 8.9

\proof{{\bf (a)} Let $E\supseteq D$ be a G$_{\delta}$ set such that
$\mu_{H,r-2}E=\mu^*_{H,r-2}D$ (471Db).   Then $c(E)>0$.
Let $K\subseteq E$ be a compact set such that $\capacity K>0$.   Then

\Centerline{$\capacity K
=\int_K\int_K\Bover1{\|x-y\|^{r-2}}\lambda_K(dx)\lambda_K(dy)$}

\noindent is finite and not $0$;  applying 471Tb to the subspace measure on
$K$,

\Centerline{$\infty=\mu_{H,r-2}K=\mu_{H,r-2}E=\mu^*_{H,r-2}D$.}

\medskip

{\bf (b)} Let $E\supseteq D$ be a G$_{\delta}$ set such that
$c(E)=c(D)$ (479E(d-i)).   Then $\mu_{Hs}E>0$.
By 471Ta, there is a non-zero topological measure $\zeta_0$ on $E$ such
that
$\biggerint_E\biggerint_E\Bover1{\|x-y\|^{r-2}}\zeta_0(dx)\zeta_0(dy)$
is finite.
Let $K\subseteq E$ be a compact set such that $\zeta_0K>0$, and let $\zeta$
be the Radon measure on $\BbbR^r$ such that $\zeta H=\zeta_0(K\cap H)$ for
every Borel set $H\subseteq\BbbR^r$;  then

\Centerline{$\energy(\zeta)
=\int_K\int_K\Bover1{\|x-y\|^{r-2}}\zeta_0(dx)\zeta_0(dy)$}

\noindent is finite, while $K$ is $\zeta$-conegligible.
By 479Nc (or, more directly, by the first remark in the proof of 479N),
$\capacity K>0$, so that $c(D)=c(E)\ge\capacity K>0$.
}%end of proof of 479Q

\leader{479R}{}\cmmnt{ I come to the promised difference between Brownian
motion in $\BbbR^3$ and in higher dimensions, following 478M.

\medskip

\noindent}{\bf Proposition} (a) Suppose that $r=3$.
Then almost every $\omega\in\Omega$ is not injective.

(b) If $r\ge 4$, then almost every $\omega\in\Omega$ is injective.

\proof{ In this proof, I will take $X_t(\omega)=\omega(t)$ for
$\omega\in\Omega$ and $t\ge 0$.

\medskip

{\bf (a)(i)} For $\omega\in\Omega$ set
$F_{\omega}=\{\omega(t):t\in[0,1]\}$.   Then $\capacity F_{\omega}>0$ for
$\mu_W$-almost every $\omega$.   \Prf\ By 477Lb,
$\mu_{H,3/2}F_{\omega}=\infty$ for almost every $\omega$.   For any such
$\omega$, there is a non-zero
Radon measure $\zeta_0$ on $F_{\omega}$ such that
$\int_{F_{\omega}}\int_{F_{\omega}}\Bover1{\|x-y\|}\zeta_0(dx)\zeta_0(dy)$
is finite (471Ta).   Let $\zeta$ be the Radon measure on $\BbbR^r$,
extending $\zeta_0$, for which $F_{\omega}$ is conegligible.   Then
$\zeta(F_{\omega})>0$ and $\energy(\zeta)<\infty$.   (This is where we need
to know that $r=3$.)   So $\capacity F_{\omega}>0$ (479Nc).\ \Qed

\medskip

\quad{\bf (ii)} Consider
$E_0=\{\omega:$ there are $s\le 1$ and
$t\ge 2$ such that $\omega(s)=\omega(t)\}$.   (This is an F$_{\sigma}$ set,
so is measurable.)   Take $\tau$ to be the stopping time with constant
value $2$ and $\phi_{\tau}:\Omega\times\Omega\to\Omega$ the corresponding
\imp\ function as in 477G;  set
$H=\{\omega:\omega\in\Omega$, $\omega(2)\notin F_{\omega}\}$.   Then

$$\eqalignno{\mu_WE_0
&=\int_{\Omega}\mu_W\{\omega':\phi_{\tau}(\omega,\omega')\in E_0\}
  \mu_W(d\omega)\cr
&=\int_{\Omega}\mu_W\{\omega':\text{ there is some }t\ge 0\text{ such that }
  \omega(2)+\omega'(t)\in F_{\omega}\}\mu_W(d\omega)\cr
&=\mu_W(\Omega\setminus H)
  +\int_H\tilde W_{F_{\omega}}(\omega(2))\mu_W(d\omega)\cr
\displaycause{479Pb}
&>0\cr}$$

\noindent because $\tilde W_{F_{\omega}}(\omega(2))>0$ whenever
$\omega\in H$ and $\capacity F_{\omega}>0$, which is so for almost every
$\omega\in H$.

\medskip

\quad{\bf (iii)} Now, setting

\Centerline{$E_n=\{\omega:$ there are $s\in[n,n+1]$ and
$t\ge n+2$ such that $\omega(s)=\omega(t)\}$,}

\noindent we have $\mu_WE_n=\mu_WE_0$ for every $n$, because
$\langle X_{s+n}-X_n\rangle_{s\ge 0}$ has the same distribution as
$\langle X_s\rangle_{s\ge 0}$.   So if
$E=\bigcap_{n\in\Bbb N}\bigcup_{m\ge n}E_m$, $\mu_WE>0$.   But $E$ belongs
to the tail $\sigma$-algebra
$\bigcap_{t\ge 0}\Tau_{\coint{t,\infty}}$,
so has measure either $0$ or $1$ (477Hd), and must be conegligible.
Since every $\omega\in E$ is self-intersecting, we see that almost every
Brownian path is self-intersecting.

\medskip

{\bf (b)(i)} Suppose that $q$, $q'\in\Bbb Q$ are such that
$0\le q<q'$.   This time, set
$F_{\omega}=\{\omega(t):t\in[0,q]\}$.   For almost every $\omega$,
$F_{\omega}$ has
zero two-dimensional Hausdorff measure (477La), so has zero
$(r-2)$-dimensional Hausdorff measure (because $r\ge 4$),
and therefore has zero capacity (479Qa).   Also

\Centerline{$\mu_W\{\omega:\omega(q')\in F_{\omega}\}
  =(\mu_W\times\mu_W)\{(\omega,\omega'):
     \omega'(q'-q)\in F_{\omega}-\omega(q)\}=0$}

\noindent because the distribution of $X_{q'-q}$ is absolutely continuous
with respect to Lebesgue measure and $\mu F_{\omega}=0$ for $\mu_W$-almost
every $\omega$.   But this means that

$$\eqalign{\mu_W\{\omega:&\text{ there is a }t\ge q'
  \text{ such that }\omega(t)\in F_{\omega}\}\cr
&=(\mu_W\times\mu_W)\{(\omega,\omega'):\text{there is a }t\ge 0
  \text{ such that }\omega'(t)\in F_{\omega}-\omega(q')\}\cr
&=\int_{\Omega}\tilde W_{F_{\omega}}(\omega(q'))\mu(d\omega)
=0,\cr}$$

\noindent that is,

\Centerline{$\{\omega:$ there are $s\le q$, $t\ge q'$ such that
$\omega(s)=\omega(t)\}$}

\noindent is negligible.   As $q$ and $q'$ are arbitrary, almost every
sample path is injective.
}%end of proof of 479R

\leader{479S}{}\cmmnt{ A famous classical problem concerned, in effect,
the continuity of potential functions, in particular the continuity of
functions of the form $\tilde W_K$.   I think
that even with the modern theory
as sketched above, this is not quite trivial, so I spell out an example.

\medskip

\noindent}{\bf Example} Suppose that $e\in\BbbR^r$ is a unit vector.   Then
there is a sequence $\sequencen{\delta_n}$ of strictly positive real
numbers such that the equilibrium potential
$\tilde W_K$ is discontinuous at $e$ whenever
$K\subseteq B(\tbf{0},1)$ is compact, $e\in\overline{\interior K}$ and
$\|x-te\|\le\delta_n$ whenever $n\in\Bbb N$, $t\in[1-2^{-n},1]$, $x\in K$
and $\|x\|=t$.

\proof{ For $n\in\Bbb N$, let $K_n$ be the line segment
$\{te:1-2^{-n}\le t\le 1-2^{-n-1}\}$.   Then the one-dimensional Hausdorff
measure of $K_n$ is finite, so $\capacity K_n=0$ (479Qa).
By 479E(c-ii), $\lim_{\delta\downarrow 0}\capacity(K_n+B(\tbf{0},\delta))=0$;
let $\delta_n\in\ooint{0,2^{-n-2}}$
be such that $\capacity(K_n+B(\tbf{0},\delta_n))\le 2^{-3n-6}$.
Setting $L_n=K_n+B(\tbf{0},\delta_n)$, the distance from $e$ to $L_n$ is at least
$2^{-n-2}$.   By 479Pb,

\Centerline{$\hp(L_n-e)
=\tilde W_{L_n}(e)
\le 4^{n+2}\lambda_{L_n}(\BbbR^r)
=4^{n+2}\capacity L_n
\le 2^{-n-2}$.}

Suppose that
$K\subseteq B(\tbf{0},1)$ is compact, $e\in\overline{\interior K}$ and
$\|x-te\|\le\delta_n$ whenever $n\in\Bbb N$, $t\in[1-2^{-n},1]$, $x\in K$
and $\|x\|=t$.   Then
$K\subseteq\bigcup_{n\in\Bbb N}L_n\cup\{e\}$.   Using the full strength of
479Pb,

\Centerline{$\tilde W_K(e)=\hp((K\setminus\{e\})-e)
\le\hp(\bigcup_{n\in\Bbb N}L_n-e)
\le\sum_{n=0}^{\infty}\hp(L_n-e)
\le\Bover12$.}

\noindent On the other hand, $\tilde W_K(x)=1$ for every
$x\in\interior K$ (479D(b-iii)), so $\tilde W_K$ is not continuous at
$e$.
}%end of proof of 479S

\leader{*479T}{}\cmmnt{ This concludes the main argument of the section,
which you may feel is quite enough.   However, there is an important
alternative method of calculating the capacity of a compact set,
based on gradients of potential functions (479U), and a couple of further
results are reasonably accessible (479V-479W) which reflect other concerns
of this volume.

\medskip

\noindent}{\bf Lemma} (a) If $g:\BbbR^r\to\Bbb R$ is a smooth function with
compact support,

\Centerline{$\int_{\BbbR^r}\Bover1{\|x-y\|^{r-2}}\nabla^2g\,d\mu
=-r(r-2)\beta_rg(x)$}

\noindent for every $x\in\BbbR^r$.
%this ought to work for general $r$

(b) Let $g$, $h:\BbbR^r\to\Bbb R$ be smooth functions with compact support.
Then

\Centerline{$\int_{\BbbR^r}h\times\nabla^2g\,d\mu
=\int_{\BbbR^r}g\times\nabla^2h
=-\int_{\BbbR^r}\varinnerprod{\grad h}{\grad g}\,d\mu$.}

(c) Let $\zeta$ be a totally finite Radon measure on $\BbbR^r$, and
$W_{\zeta}:\BbbR^r\to[0,\infty]$ the associated Newtonian potential.   Then
$\int_{\BbbR^r}W_{\zeta}\times\nabla^2g\,d\mu
=-r(r-2)\beta_r\int_{\BbbR^r}g\,d\zeta$ for every smooth
function $g:\BbbR^r\to\Bbb R$ with compact support.

(d) Let $\zeta$ be a totally finite Radon measure on $\BbbR^r$ such that
$W_{\zeta}$ is finite-valued everywhere and Lipschitz.   Then
$\int_{\BbbR^r}\varinnerprod{\grad f}{\grad W_{\zeta}}d\mu
=r(r-2)\beta_r\int_{\BbbR^r}f\,d\zeta$ for every Lipschitz function
$f:\BbbR^r\to\Bbb R$ with compact support.

%maybe can make this work with Lipschitz $f\in C_0(\BbbR^r)$
%what about $\energy(\zeta)$ and $\int\|\grad W_{\zeta}\|^2$?

(e) Let $K\subseteq\BbbR^r$ be a compact set, and $\epsilon>0$.   Then
there is a Radon measure $\zeta$ on $\BbbR^r$, with support included in
$K+B(\tbf{0},\epsilon)$, such that $W_{\zeta}$ is a smooth function with compact
support, $W_{\zeta}\ge\chi K$, $\zeta\BbbR^r\le\capacity K+\epsilon$ and

\Centerline{$\int_{\BbbR^r}\|\grad W_{\zeta}\|^2d\mu
=r(r-2)\beta_r\energy(\zeta)\le r(r-2)\beta_r\zeta\BbbR^r$.}

\proof{{\bf (a)(i)} Consider first the case $x=0$.
Setting $f(y)=\Bover1{\|y\|^{r-2}}$ for $y\ne 0$, we have
$\grad f(y)=-\Bover{r-2}{\|y\|^r}y$ and $(\nabla^2f)(y)=0$ for $y\ne 0$
(478Fa);   also $f$ is locally integrable, by 478Ga.
So $\int_{\BbbR^r}f\times\nabla^2g\,d\mu$ is well-defined.

Let $R>0$ be such that $g$ is zero outside $B(\tbf{0},R)$, and set
$M=\|\grad g\|_{\infty}$;
take $\epsilon\in\ooint{0,R}$.    Then

$$\eqalignno{\int_{\BbbR^r\setminus B(\tbf{0},\epsilon)}f\times\nabla^2g\,d\mu
&=\int_{B(\tbf{0},R)\setminus B(\tbf{0},\epsilon)}
  f\times\nabla^2g-g\times\nabla^2f\,d\mu\cr
&=\int_{B(\tbf{0},R)\setminus B(\tbf{0},\epsilon)}
  \diverg(f\times\grad g-g\times\grad f)d\mu\cr
\displaycause{use 474Bb}
&=\int_{\partial B(\tbf{0},R)}
  (\Bover1{\|y\|^{r-2}}\grad g(y)+\Bover{(r-2)g(y)}{\|y\|^r}y)
  \dotproduct\Bover{y}{\|y\|}\nu(dy)\cr
&\mskip100mu
 -\int_{\partial B(\tbf{0},\epsilon)}
  (\Bover1{\|y\|^{r-2}}\grad g(y)+\Bover{(r-2)g(y)}{\|y\|^r}y)
  \dotproduct\Bover{y}{\|y\|}\nu(dy)\cr
\displaycause{475Nc}
&=-\int_{\partial B(\tbf{0},\epsilon)}
  (\Bover1{\|y\|^{r-1}}y\dotproduct\grad g(y)
     +\Bover{(r-2)g(y)}{\|y\|^{r-1}})\nu(dy)\cr
&=-\Bover1{\epsilon^{r-1}}\int_{\partial B(\tbf{0},\epsilon)}
  (y\dotproduct\grad g(y)+(r-2)g(y))\nu(dy).
  \cr}$$

\noindent Now we have

\Centerline{$|\int_{\partial B(\tbf{0},\epsilon)}
  y\dotproduct\grad g(y)\nu(dy)|
\le\epsilon M\nu(\partial B(\tbf{0},\epsilon))\le r\beta_r\epsilon^rM$,}

\noindent so

$$\eqalign{
|\int_{\BbbR^r\setminus B(\tbf{0},\epsilon)}f\times&\nabla^2g\,d\mu
   +r(r-2)\beta_rg(0)|\cr
&\le r\beta_r\epsilon M
 +\Bover1{\epsilon^{r-1}}|r\beta_r\epsilon^{r-1}(r-2)g(0)
   -\int_{\partial B(\tbf{0},\epsilon)}(r-2)g(y)\nu(dy)|\cr
&\le r\beta_r\epsilon M
 +\Bover{r-2}{\epsilon^{r-1}}\int_{\partial B(\tbf{0},\epsilon)}
    |g(0)-g(y)|\nu(dy)\cr
&\le r\beta_r\epsilon M
 +r(r-2)\beta_r\sup_{y\in\partial B(\tbf{0},\epsilon)}|g(0)-g(y)|
\to 0\cr}$$

\noindent as $\epsilon\downarrow 0$;  that is,

\Centerline{$\int_{\BbbR^r}f\times\nabla^2g\,d\mu=-r(r-2)\beta_rg(0)$.}

\medskip

\quad{\bf (ii)} For the general case, apply (i) to the function
$y\mapsto g(x+y)$.

\medskip

{\bf (b)} Take $R>0$ so large that both $g$ and $h$ are zero outside
$B(\tbf{0},R)$, and $M\ge\max(\|\nabla^2g\|_{\infty},\|\nabla^2h\|_{\infty})$.

\medskip

\quad{\bf (i)} We have

$$\eqalignno{\int_{\BbbR^r}\int_{\BbbR^r}&\Bover1{\|x-y\|^{r-2}}
  |(\nabla^2g)(x)(\nabla^2h)(y)|\mu(dx)\mu(dy)\cr
&\le M^2\int_{B(\tbf{0},R)}\int_{B(\tbf{0},R)}\Bover1{\|x-y\|^{r-2}}\mu(dx)\mu(dy)
\le M^2\int_{B(\tbf{0},R)}\Bover12r\beta_rR^2\mu(dy)\cr
\displaycause{478Gc}
&<\infty.\cr}$$

\noindent So

$$\eqalignno{-r(r-2)\beta_r\int_{\BbbR^r}h\times\nabla^2g\,d\mu
&=\int_{\BbbR^r}\int_{\BbbR^r}\Bover1{\|x-y\|^{r-2}}
   (\nabla^2h)(x)(\nabla^2g)(y)\mu(dx)\mu(dy)\cr
\displaycause{by (a)}
&=\int_{\BbbR^r}\int_{\BbbR^r}\Bover1{\|x-y\|^{r-2}}
   (\nabla^2g)(y)(\nabla^2h)(x)\mu(dy)\mu(dx)\cr
&=-r(r-2)\beta_r\int_{\BbbR^r}g\times\nabla^2h\,d\mu.\cr}$$

\noindent Thus $\int_{\BbbR^r}g\times\nabla^2h\,d\mu
=\int_{\BbbR^r}h\times\nabla^2g\,d\mu$.

\medskip

\quad{\bf (ii)} By 473Bd, $\grad(g\times h)=g\times\grad h+h\times\grad g$,
so 474Bb tells us that
$\nabla^2(g\times h)
=2\varinnerprod{\grad g}{\grad h}+g\times\nabla^2h+h\times\nabla^2g$, and

$$\eqalign{\int_{\BbbR^r}\nabla^2(g\times h)d\mu
&=\int_{B(\tbf{0},R)}\nabla^2(g\times h)d\mu\cr
&=\int_{\partial B(\tbf{0},R)}\varinnerprod{\grad(g\times h)}{\Bover{x}{\|x\|}}
   \nu(dx)
=0,\cr}$$

\noindent so

\Centerline{$\int_{\BbbR^r}\varinnerprod{\grad g}{\grad h}\,d\mu
=-\Bover12\int_{\BbbR^r}g\times\nabla^2h+h\times\nabla^2g\,d\mu
=-\int_{\BbbR^r}g\times\nabla^2h\,d\mu$.}

\medskip

{\bf (c)} If $g(x)=0$ for $\|x\|\ge R$ and
$|(\nabla^2g)(x)|\le M$ for every $x$, then

\Centerline{$\int_{\BbbR^r}\Bover1{\|x-y\|^{r-2}}|(\nabla^2g)(x)|\mu(dx)
\le M\int_{B(\tbf{0},R)}\Bover1{\|x-y\|^{r-2}}\mu(dx)
\le\Bover12Mr\beta_rR^2$}

\noindent for every $y$ (478Gc again).
We can therefore apply (a) and
integrate with respect to $\zeta$ to see that

$$\eqalign{-r(r-2)\beta_r\int_{\BbbR^r}g\,d\zeta
&=\int_{\BbbR^r}\int_{\BbbR^r}
  \Bover1{\|x-y\|^{r-2}}(\nabla^2g)(x)\mu(dx)\zeta(dy)\cr
&=\int_{\BbbR^r}\int_{\BbbR^r}
  \Bover1{\|x-y\|^{r-2}}(\nabla^2g)(x)\zeta(dy)\mu(dx) \cr
&=\int_{\BbbR^r}W_{\zeta}(x)(\nabla^2g)(x)\mu(dx),\cr}$$

\noindent as required.

\medskip

{\bf (d)} Let $\sequencen{\tilde h_n}$ be the
smoothing sequence of 473E.

\medskip

\quad{\bf (i)} Suppose to begin with that $f$ is smooth.    For
$n\in\Bbb N$ set $g_n=\tilde h_n*W_{\zeta}$.   As $W_{\zeta}$ is
continuous, $\lim_{n\to\infty}g_n=W_{\zeta}$ (473Ec);  as
$\|W_{\zeta}\|_{\infty}\le 1$, $\|g_n\|_{\infty}\le 1$ for every $n$
(473Da).   Because $f$ has compact support,

\Centerline{$\lim_{n\to\infty}\int_{\BbbR^r}g_n\times\nabla^2f\,d\mu
=\int_{\BbbR^r}W_{\zeta}\times\nabla^2f\,d\mu$}

\noindent by the dominated convergence theorem.   Next,
$\grad g_n=\tilde h_n*\grad W_{\zeta}$ for each $n$ (473Dd).   As
$\grad W_{\zeta}$ is essentially bounded (473Cc), all its coordinates are
locally integrable, so
$\grad W_{\zeta}\eae\lim_{n\to\infty}\grad g_n$ (473Ee).   We therefore
have

$$\eqalignno{\int_{\BbbR^r}\varinnerprod{\grad f}{\grad W_{\zeta}}d\mu
&=\lim_{n\to\infty}\int_{\BbbR^r}
   \varinnerprod{\grad f}{\grad g_n}d\mu\cr
&=-\lim_{n\to\infty}\int_{\BbbR^r}g_n\times\nabla^2f\,d\mu\cr
\displaycause{(b) above}
&=-\int_{\BbbR^r}W_{\zeta}\times\nabla^2f\,d\mu
=r(r-2)\beta_r\int_{\BbbR^r}f\,d\zeta\cr}$$

\noindent by (c).

\medskip

\quad{\bf (ii)} For the general case, smooth on the other side, setting
$f_n=\tilde h_n*f$ for every $n$.   This time, $f_n\to f$ uniformly
(473Ed), so
$\int_{\BbbR^r}fd\zeta=\lim_{n\to\infty}\int_{\BbbR^r}f_nd\zeta$.
On the other hand, if $f$ is $M$-Lipschitz,
$\grad f_n=\tilde h_n*\grad f$ converges $\mu$-a.e.\ to
$\grad f$, and $\|\grad f_n\|_{\infty}$ is at most $M$ for every $n$;
also there is a bounded set outside which all
the $f_n$ and $\grad f_n$ are zero, and $\|\grad W_{\zeta}\|$ is bounded.
So

\Centerline{$\int_{\BbbR^r}\varinnerprod{\grad f}{\grad W_{\zeta}}d\mu
=\lim_{n\to\infty}\int_{\BbbR^r}
  \varinnerprod{\grad f_n}{\grad W_{\zeta}}d\mu$.}

\noindent
Applying (i) to each $f_n$ and taking the limit, we get the equality we
seek.

\medskip

{\bf (e)(i)} There is a compact set $L\subseteq K+B(\tbf{0},\bover{\epsilon}2)$
such that $K\subseteq\interior L$ and $\capacity L\le\capacity K+\epsilon$
(479Ed).   Let $n\in\Bbb N$ be such that
$\Bover1{n+1}\le\Bover{\epsilon}2$
and $K+B(\tbf{0},\bover1{n+1})\subseteq L$.   Set
$h=\lambda_L*\tilde h_n$, where $\tilde h_n$ is the
function of 473E, as before;  let $\zeta=h\mu$ be the corresponding
indefinite-integral measure over $\mu$.   Because $\tilde h_n$ is zero
outside $B(\tbf{0},\bover1{n+1})$
and the support of $\lambda_L$ is included in $L$,
the support of $\zeta$ is included in
$L+B(\tbf{0},\bover1{n+1})\subseteq K+B(\tbf{0},\epsilon)$.

\medskip

\quad{\bf (ii)} By 444Pa, we have

\Centerline{$W_{\zeta}=\zeta*k_{r-2}=(h\mu)*k_{r-2}=h*k_{r-2}$}

\noindent where $k_{r-2}$ is the Riesz kernel (479G).
Now $W_{\zeta}=\tilde W_L*\tilde h_n$.   \Prf\ For $m\in\Bbb N$, set
$f_m=k_{r-2}\times\chi B(\tbf{0},m)$, so that $f_m$ is $\mu$-integrable.
Observe that

$$\eqalign{\tilde W_L(x)
&=\int_{\BbbR^r} k_{r-2}(x-y)\lambda_L(dy)
=\int_{\BbbR^r}\lim_{m\to\infty}f_m(x-y)\lambda_L(dy)\cr
&=\lim_{m\to\infty}\int_{\BbbR^r} f_m(x-y)\lambda_L(dy)
=\lim_{m\to\infty}(\lambda_L*f_m)(x)\cr}$$

\noindent for each $x$;  moreover, because $\sequence{m}{f_m}$ is
non-decreasing, so is $\sequence{m}{\lambda_L*f_m}$.
For each $m$,

$$\eqalignno{h*f_m
&=(h\mu)*f_m
=(\lambda_L*\tilde h_n)\mu*f_m
=(\lambda_L*\tilde h_n\mu)*f_m\cr
\displaycause{444K}
&=\lambda_L*(\tilde h_n\mu*f_m)\cr
\displaycause{444Ic}
&=\lambda_L*(\tilde h_n*f_m)
=\lambda_L*(f_m*\tilde h_n)
=\lambda_L*(f_m\mu*\tilde h_n)\cr
&=(\lambda_L*f_m\mu)*\tilde h_n
=(\lambda_L*f_m)\mu*\tilde h_n
=(\lambda_L*f_m)*\tilde h_n.\cr}$$

\noindent Now, for each $x$,

$$\eqalign{W_{\zeta}(x)
&=\int_{\BbbR^r} h(x-y)k_{r-2}(y)\mu(dy)
=\lim_{m\to\infty}\int_{\BbbR^r} h(x-y)f_m(y)\mu(dy)\cr
&=\lim_{m\to\infty}(h*f_m)(x)
=\lim_{m\to\infty}((\lambda_L*f_m)*\tilde h_n)(x)\cr
&=\lim_{m\to\infty}\int_{\BbbR^r}(\lambda_L*f_m)(y)\tilde h_n(x-y)\mu(dy)
  \cr
&=\int_{\BbbR^r}\lim_{m\to\infty}(\lambda_L*f_m)(y)
   \tilde h_n(x-y)\mu(dy)\cr
&=\int_{\BbbR^r}\tilde W_L(y)\tilde h_n(x-y)\mu(dy)
=(\tilde W_L*\tilde h_n)(x).  \text{ \Qed}\cr}$$

\medskip

\quad{\bf (iii)} Since $\tilde W_L(x)=1$ whenever $x\in\interior L$
(479D(b-iii)), and
$x+y\in\interior L$ whenever $x\in K$ and $\tilde h_n(y)\ne 0$,
$W_{\zeta}(x)=1$ for every $x\in K$.   Because both $\tilde W_L$ and
$\tilde h_n$ have compact support, so does $W_{\zeta}$;
because $\tilde h_n$ is smooth, so is $W_{\zeta}$ (473De).

\medskip

\quad{\bf (iv)} Now

$$\eqalignno{\int_{\BbbR^r}\|\grad W_{\zeta}\|^2d\mu
&=-\int_{\BbbR^r}W_{\zeta}\times\nabla^2W_{\zeta}\,d\mu\cr
\displaycause{(b) above}
&=r(r-2)\beta_r\int_{\BbbR^r}W_{\zeta}\,d\zeta\cr
\displaycause{(c) above}
&=r(r-2)\beta_r\energy(\zeta)
\le r(r-2)\beta_r\zeta\BbbR^r\cr}$$

\noindent because
$\|W_{\zeta}\|_{\infty}\le\|\tilde W_L\|_{\infty}\|\tilde h_n\|_1\le 1$.

\medskip

\quad{\bf (v)} Finally,

$$\eqalign{\zeta\BbbR^r=(h\mu)\BbbR^r
&=(\lambda_L*\tilde h_n\mu)\BbbR^r
=\lambda_L\BbbR^r\cdot(\tilde h_n\mu)\BbbR^r\cr
&=\lambda_L\BbbR^r
=\capacity L
\le\epsilon+\capacity K.\cr}$$
}%end of proof of 479T

\leader{*479U}{Theorem} Let $K\subseteq\BbbR^r$ be compact,
and let $\Phi$ be
the set of Lipschitz functions $g:\BbbR^r\to\Bbb R$ such that
$g(x)\ge 1$ for every $x\in K$ and $\lim_{\|x\|\to\infty}g(x)=0$.   Then

$$\eqalign{r(r-2)\beta_r\capacity K
&=\inf\{\int_{\BbbR^r}\|\grad g\|^2d\mu:g\in\Phi\text{ is smooth and has
compact support}\}\cr
&=\inf\{\int_{\BbbR^r}\|\grad g\|^2d\mu:g\in\Phi\}.\cr}$$

\proof{{\bf (a)} By 479Te,

\Centerline{$\inf\{\int_{\BbbR^r}\|\grad g\|^2d\mu:g\in\Phi$
  is smooth and has compact support$\}
\le r(r-2)\beta_r\capacity K$.}

\medskip

{\bf (b)} Now suppose that $g\in\Phi$ is a smooth function with compact
support.   Then
$r(r-2)\beta_r\capacity K\le\int_{\BbbR^r}\|\grad g\|^2d\mu$.   \Prf\
Take any $\epsilon\in\ooint{0,1}$.
Then there is a $\delta>0$ such that $g(x)\ge 1-\epsilon$ for every
$x\in K+B(\tbf{0},\delta)$.   By 479Te, there is a Radon measure $\zeta$ on
$\BbbR^r$, with support included in $K+B(\tbf{0},\delta)$, such that $W_{\zeta}$
is smooth and has compact support, $W_{\zeta}\ge\chi K$,
$\zeta\BbbR^r\le\capacity K+\epsilon$ and

\Centerline{$\int_{\BbbR^r}\|\grad W_{\zeta}\|^2d\mu
=r(r-2)\beta_r\energy(\zeta)\le r(r-2)\beta_r\zeta\BbbR^r$.}

\noindent In this case,

$$\eqalignno{\int_{\BbbR^r}\varinnerprod{\grad g}{\grad W_{\zeta}}d\mu
&=r(r-2)\beta_r\int_{\BbbR^r}g\,d\zeta\cr
\displaycause{479Td}
&\ge(1-\epsilon)r(r-2)\beta_r\zeta\BbbR^r
\ge(1-\epsilon)\int_{\BbbR^r}\|\grad W_{\zeta}\|^2.\cr}$$

\noindent Setting $v=(1-\epsilon)\grad W_{\zeta}$, we have

\Centerline{$\int_{\BbbR^r}\varinnerprod{v}{\grad g}d\mu
\ge\int_{\BbbR^r}\|v\|^2d\mu$.}

\noindent But this means that

$$\eqalignno{\int_{\BbbR^r}\|\grad g\|^2d\mu
&=2\int_{\BbbR^r}\varinnerprod{v}{\grad g}d\mu-\int_{\BbbR^r}\|v\|^2d\mu
   +\int_{\BbbR^r}\|v-\grad g\|^2d\mu\cr
&\ge\int_{\BbbR^r}\|v\|^2d\mu
\ge(1-\epsilon)^2\int_{\BbbR^r}\|\grad W_{\zeta}\|^2d\mu\cr
&=(1-\epsilon)^2r(r-2)\beta_r\int_{\BbbR^r}W_{\zeta}d\zeta
\ge(1-\epsilon)^2r(r-2)\beta_r\int_{\BbbR^r}\tilde W_Kd\zeta\cr
\displaycause{because $W_{\zeta}\ge\tilde W_K$ on $K$, so
$W_{\zeta}\ge\tilde W_K$ everywhere, by 479Fg}
&=(1-\epsilon)^2r(r-2)\beta_r\int_{\BbbR^r}W_{\zeta}d\lambda_K\cr
\displaycause{479J(b-i)}
&\ge(1-\epsilon)^2r(r-2)\beta_r\int_{\BbbR^r}\tilde W_Kd\lambda_K
=(1-\epsilon)^2r(r-2)\beta_r\capacity K.\cr}$$

\noindent As $\epsilon$ is arbitrary,
$r(r-2)\beta_r\capacity K\le\int_{\BbbR^r}\|\grad g\|^2d\mu$.\ \Qed

\medskip

{\bf (c)} If $g\in\Phi$ has compact support, then
$\int_{\BbbR^r}\|\grad g\|^2d\mu\ge r(r-2)\beta_r\capacity K$.   \Prf\
Let $R>0$ be such that $g$ is zero outside $B(\tbf{0},R)$.   Let $M\ge 0$ be such
that $g$ is $M$-Lipschitz;   then
$\|\grad g(x)\|\le M$ for every $x\in\dom\grad g$ (473Cc).
Take any $\epsilon>0$.   As in 479T, let $\sequencen{\tilde h_n}$ be
the smoothing sequence of 473E.
For $n\in\Bbb N$, set $g_n=(1+\epsilon)\tilde h_n*g$.   Then
$\grad g_n=(1+\epsilon)\tilde h_n*\grad g$ (473Dd) and
$\|\grad g_n\|_{\infty}\le M(1+\epsilon)$ (473Da).   In the limit,
$(1+\epsilon)\grad g\eae\lim_{n\to\infty}\grad g_n$ (473Ee).

There is an $m\in\Bbb N$ such that $(1+\epsilon)g(x)\ge 1$ for every
$x\in K+B(\tbf{0},\bover1{m+1})$;  now if $n\ge m$,

\Centerline{$g_n(x)\ge(1+\epsilon)\inf_{\|y\|\le 1/(n+1)}g(x-y)\ge 1$}

\noindent for every $x\in K$.   So

$$\eqalignno{(1+\epsilon)^2\int_{\BbbR^r}\|\grad g\|^2d\mu
&=(1+\epsilon)^2\int_{B(\tbf{0},R+1)}\|\grad g\|^2d\mu\cr
&=\lim_{n\to\infty}\int_{B(\tbf{0},R+1)}\|\grad g_n\|^2d\mu\cr
\displaycause{by the dominated convergence theorem}
&=\lim_{n\to\infty}\int_{\BbbR^r}\|\grad g_n\|^2d\mu\cr
\displaycause{because every $g_n$ is zero outside $B(\tbf{0},R+1)$}
&\ge r(r-2)\beta_r\capacity K\cr}$$

\noindent (applying (b) to $g_n$ for $n\ge m$).
As $\epsilon$ is arbitrary,
$r(r-2)\beta_r\capacity K\le\int_{\BbbR^r}\|\grad g\|^2d\mu$.\ \Qed

\medskip

{\bf (d)} If $g\in\Phi$, then
$\int_{\BbbR^r}\|\grad g\|^2d\mu\ge r(r-2)\beta_r\capacity K$.   \Prf\
Let $\epsilon>0$.   Set
$g_1(x)=\max(0,(1+\epsilon)g(x)-\epsilon)$ for $x\in\BbbR^r$.
Then $g_1\in\Phi$ has compact support, and
$\|g_1(x)-g_1(y)\|\le(1+\epsilon)\|g(x)-g(y)\|$ for all $x$,
$y\in\Bbb R^r$, so
$\|\grad g_1(x)\|\le(1+\epsilon)\|\grad g(x)\|$ whenever both gradients are
defined.   Accordingly

\Centerline{$(1+\epsilon)^2\int_{\BbbR^r}\|\grad g\|^2d\mu
\ge\int_{\BbbR^r}\|\grad g_1\|^2d\mu
\ge r(r-2)\beta_r\capacity K$}

\noindent by (c).   As $\epsilon$ is arbitrary, we have the result.\ \Qed

\medskip

{\bf (e)} Putting (a) and (d) together, the theorem is proved.
}%end of proof of 479U

\leader{*479V}{}\cmmnt{ We are ready for another theorem
along the lines of 476H, this time relating capacity and Lebesgue
measure.

%what about capacity and r-1-dimensional measure?

\medskip

\noindent}{\bf Theorem} Let $D\subseteq\BbbR^r$ be a set of finite outer
Lebesgue measure, and $B_D$ the closed ball with centre $0$ and
the same outer measure as $D$.   Then
the Choquet-Newton capacity $c(D)$ of $D$ is at least $\capacity B_D$.

\proof{{\bf (a)} We need an elementary fact about gradients.   Suppose that
$f$, $g:\BbbR^r\to\Bbb R$ and $x\in\BbbR^r$ are such that $\grad f$,
$\grad g$, $\grad(f\vee g)$ and $\grad(f\wedge g)$ are all defined at $x$.
Then $\{\grad(f\vee g)(x),\grad(f\wedge g)(x)\}=\{\grad f(x),\grad g(x)\}$.
\Prf\ (i) If $f(x)>g(x)$ then (because $f$ and $g$ are both continuous at
$x$) we have $\grad(f\vee g)(x)=\grad f(x)$,
$\grad(f\wedge g)(x)=\grad g(x)$ and the result is immediate.   (ii) The
same argument applies if $f(x)<g(x)$.   (iii) If $f(x)=g(x)$, consider
$h=|f-g|=(f\vee g)-(f\wedge g)$.   Then $\grad h(x)$ is defined, and
$h(x)=0\le h(y)$ for every $y$.   So all the partial derivatives of $h$
have to be zero at $x$, and $\grad h(x)=0$, that is,
$\lim_{y\to x}\Bover1{\|y-x\|}h(y)=0$.   It follows at once that
$\grad f(x)=\grad g(x)$, and therefore both are equal to
$\grad(f\vee g)(x)$ and $\grad(f\wedge g)(x)$.   So again we have the
result.\ \Qed

\medskip

{\bf (b)} Now for a further clause to add to Lemma 476E.   Suppose that
$e\in S_{r-1}=\partial B(\tbf{0},1)$ and $\alpha\in\Bbb R$;
let $R=R_{e\alpha}$ be the
reflection in the plane $\{x:\varinnerprod{x}{e}=\alpha\}$, and
$\psi=\psi_{e\alpha}:\Cal P\BbbR^r\to\Cal P\BbbR^r$ the partial-reflection
operator of 476D-476E, that is,

\Centerline{$\psi(D)=(W\cap(D\cup R[D]))\cup(W'\cap D\cap R[D])$}

\noindent for $D\subseteq\BbbR^r$, where
$W=\{x:\varinnerprod{x}{e}\ge\alpha\}$ and
$W'=\{x:\varinnerprod{x}{e}\le\alpha\}$.   Then $c(\psi(D))\le c(D)$ for
every $D\subseteq\BbbR^r$.

\medskip

\Prf{\bf (i)} Suppose first that $D=K$ is compact.   Take any
$\gamma>r(r-2)\beta_r\capacity K$.   By 479U, there is a Lipschitz function
$f:\BbbR^r\to\Bbb R$ such that $f(x)\ge 1$ for every $x\in K$,
$\lim_{\|x\|\to\infty}f(x)=0$ and
$\int_{\BbbR^r}\|\grad f\|^2d\mu\le\gamma$.   Set $g=fR$.   Of course
$g$ is Lipschitz and
$\int_{\BbbR^r}\|\grad g\|^2d\mu=\int_{\BbbR^r}\|\grad f\|^2d\mu$.
Now $f\vee g$ and $f\wedge g$ are also Lipschitz, so for almost every
$x\in\BbbR^r$ all the gradients $\grad f(x)$, $\grad g(x)$,
$\grad(f\vee g)(x)$ and $\grad(f\wedge g)(x)$ are defined;  by (a),
$\|\grad f\|^2+\|\grad g\|^2
\eae\|\grad(f\vee g)\|^2+\|\grad(f\wedge g)\|^2$.

Now consider the function $h$ defined by saying that

$$\eqalign{h(x)
&=(f\vee g)(x)\text{ if }x\in W,\cr
&=(f\wedge g)(x)\text{ if }x\in W'.\cr}$$

\noindent $h$ is Lipschitz and $\lim_{\|x\|\to\infty}h(x)=0$;  also
$h(x)\ge 1$ for every $x\in\psi(K)$.   For $x\in W\setminus W'$,
$h(x)=(f\vee g)(x)$ and $h(Rx)=(f\wedge g)(x)$, so
$\{\grad h(x),\grad(hR)(x)\}=\{\grad(f\vee g)(x),\grad(f\wedge g)(x)\}$ if
the gradients are defined;  for $x\in W'\setminus W$,
$h(x)=(f\wedge g)(x)$ and $h(Rx)=(f\vee g)(x)$, so
$\{\grad h(x),\grad(hR)(x)\}=\{\grad(f\vee g)(x),\grad(f\wedge g)(x)\}$ if
the gradients are defined.   Accordingly

$$\eqalign{\|\grad h\|^2+\|\grad(hR)\|^2
&\eae\|\grad(f\vee g)\|^2+\|\grad(f\wedge g)\|^2\cr
&\eae\|\grad f\|^2+\|\grad g\|^2.\cr}$$

\noindent By 479U again,

$$\eqalign{r(r-2)\beta_r\capacity(\psi(K))
&\le\int_{\BbbR^r}\|\grad h\|^2d\mu
=\Bover12\int_{\BbbR^r}(\|\grad h\|^2+\|\grad(hR)\|^2)d\mu\cr
&=\Bover12\int_{\BbbR^r}(\|\grad f\|^2+\|\grad g\|^2)d\mu
\le\gamma.\cr}$$

\noindent As $\gamma$ is arbitrary, $\capacity(\psi(K))\le\capacity K$.

\medskip

\quad{\bf (ii)} Now suppose that $D=G$ is open.   Then there is a
non-decreasing sequence $\sequencen{K_n}$ of compact sets with union $G$,
and $\sequencen{\psi(K_n)}$ is a non-decreasing sequence with union
$\psi(G)$.   So

\Centerline{$c(\psi(G))=\sup_{n\in\Bbb N}\capacity(\psi(K_n))
\le\sup_{n\in\Bbb N}\capacity K_n=c(G)$.}

\medskip

\quad{\bf (iii)} Finally, for arbitrary $D\subseteq\BbbR^r$, take any
$\gamma>c(D)$.   Then
there is an open set $G$ such that $D\subseteq G$ and $c(G)\le\gamma$
(because $c$ is outer regular, see 479E(d-i)).   In this case,
$\psi(D)\subseteq\psi(G)$, so

\Centerline{$c(\psi(D))\le c(\psi(G))\le c(G)\le\gamma$.}

\noindent As $\gamma$ is arbitrary, $c(\psi(D))\le c(D)$ and we are done.\
\Qed

\medskip

{\bf (c)} Now suppose that $E$ is a bounded Lebesgue measurable subset of
$\BbbR^r$ with finite perimeter.

\medskip

\quad{\bf (i)} Let $M\ge 0$ be such that
$E\subseteq B(\tbf{0},M)$.   Consider

$$\eqalign{\Cal E
&=\{F:F\subseteq B(\tbf{0},M)\text{ is Lebesgue measurable},\cr
&\mskip100mu
\mu F=\mu E,\,\per F\le\per E,\,c(F)\le c(E)\}.\cr}$$

\noindent Then $\Cal E$ is compact for the topology $\frak T_m$ of
convergence in measure as described in 474T.   \Prf\ By 474T,

\Centerline{$\Cal E_1=\{F:F$ is Lebesgue measurable,
$\per F\le\per E\}$}

\noindent is compact.   So if $\sequencen{F_n}$ is any sequence in
$\Cal E$, it has a subsequence $\sequencen{F'_n}$ which is
$\frak T_m$-convergent to $F\in\Cal E_1$ say (4A2Le;  recall that, as noted
in the proof of 474T, $\frak T_m$ is pseudometrizable).
Taking a further subsequence
if necessary, we can suppose that
$\mu((F\symmdiff F'_n)\cap B(\tbf{0},M))\le 2^{-n}$ for every
$n\in\Bbb N$.   Set $F'=\bigcup_{m\in\Bbb N}\bigcap_{n\ge m}F'_n$.
Because every $F'_n$ is included in $B(\tbf{0},M)$,
$F'$ is a $\frak T_m$-limit of $\sequencen{F'_n}$.   So
$\mu F'=\lim_{n\to\infty}\mu F'_n=\mu E$, and

\Centerline{$c(F')
=\lim_{m\to\infty}c(\bigcap_{n\ge m}F'_n)\le c(E)$.}

\noindent Finally, $F'\symmdiff F$ is negligible, so
$\partstar F'=\partstar F$ and $\per(F')=\per F\le\per E$.   Thus
$F'\in\Cal E$.   As $\sequencen{F_n}$ is arbitrary,
$\Cal E$ is relatively compact, by 4A2Le in the opposite direction.\ \Qed

\medskip

\quad{\bf (ii)} Because $B(\tbf{0},M)$ is bounded,
the function $F\mapsto\int_F\|x\|\mu(dx):\Cal E\to\coint{0,\infty}$ is
continuous, and must attain its infimum at $H$ say.   Let $B_E$ be the ball
with centre $0$ and the same measure as $E$.   Then
$B_E\subseteq\clstar H$.   \Prf\ (Compare part (b) of the proof of
476H.)  \Quer\ Otherwise, take $z\in B_E\setminus\clstar H$.
Then

\Centerline{$\lim_{\delta>0}
  \Bover{\mu(B(z,\delta)\setminus H)}{\mu B(z,\delta)}=1$,
\quad$\lim_{\delta>0}
  \Bover{\mu(B(z,\delta)\setminus B_E)}{\mu B(z,\delta)}\le\Bover12$,}

\noindent so there is a $\delta>0$ such that
$\mu(B(z,\delta)\setminus B_E)<\mu(B(z,\delta)\setminus H)$ and
$\mu(B_E\setminus H)>0$.   Because

\Centerline{$\mu(\clstar H)=\mu H=\mu E=\mu B_E$,}

\noindent $\clstar H\setminus B_E$ is also non-negligible.
Take $x_1\in\clstar H\setminus B_E$ and $x_0\in B_E\setminus\clstar H$.
Then $\delta_0=\|x_1\|-\|x_0\|$ is greater than $0$.
Since

\Centerline{$\limsup_{\delta\downarrow 0}
  \Bover{\mu(H\cap B(x_1,\delta))}{\mu B(x_1,\delta)}
>0=\lim_{\delta\downarrow 0}
  \Bover{\mu(H\cap B(x_0,\delta))}{\mu B(x_0,\delta)}$,}

\noindent there is a $\delta\in\ooint{0,\bover12\delta_0}$ such that
$\mu(H\cap B(x_1,\delta))>\mu(H\cap B(x_0,\delta))$.   Now let $e$ be the
unit vector $\Bover1{\|x_0-x_1\|}(x_0-x_1)$, and set
$\alpha=\varinnerprod{e}{\bover12(x_0+x_1)}$.   Consider the reflection
$R=R_{e\alpha}$ and the operator $\psi=\psi_{e\alpha}$;  set $H_1=\psi(H)$
and let $\phi=\phi_{H_1}:H\to H_1$ be the function of 476E.   As
$\alpha<0$, $\|\phi(x)\|\le\|x\|$ for every $x\in H$;   moreover,
$R[B(x_1,\delta)]=B(x_0,\delta)$, so

\Centerline{$\{x:\|\phi(x)\|<\|x\|\}
\supseteq\{x:x\in B(x_1,\delta)\cap H$, $Rx\notin H\}$}

\noindent is not negligible.   So
$\int_{H_1}\|x\|\mu(dx)<\int_H\|x\|\mu(dx)$.   On the other hand, we
surely have $H_1\subseteq B(\tbf{0},M)$, $\mu H_1=\mu H=\mu E$ and
$\per H_1\le\per H\le\per E$ (476Ee);  and, finally,
$c(H_1)\le c(H)\le c(E)$, by (b) of this proof.   Thus $H_1\in\Cal E$ and
the functional $F\mapsto\int_F\|x\|\mu(dx)$ is not minimized at $H$.\
\Bang\Qed

\medskip

\quad{\bf (iii)} Accordingly

$$\eqalignno{\capacity B_E
&\le c(\clstar H)\le c(H)\cr
\displaycause{479P(c-vi)}
&\le c(E).\cr}$$

\medskip

{\bf (d)} Thus $\capacity B_E\le c(E)$ whenever $E\subseteq\BbbR^r$ is
Lebesgue
measurable, bounded and has finite perimeter.   Consequently
$\capacity B_K\le\capacity K$ for every compact set $K\subseteq\BbbR^r$.
\Prf\ If $\epsilon>0$, there is an open set $G\supseteq K$ such that
$c(G)\le\capacity K+\epsilon$.   Now there is a set $E$, a finite union
of balls,
such that $K\subseteq E\subseteq G$.   In this case, $E$ has finite
perimeter and is bounded, while of course $B_E\supseteq B_K$.   So

\Centerline{$\capacity B_K
\le\capacity B_E\le c(E)\le c(G)\le\capacity K+\epsilon$.}

\noindent As $\epsilon$ is arbitrary, $\capacity B_K\le\capacity K$.\ \Qed

It follows that $\capacity B_E\le c(E)$ for every measurable set
$E\subseteq\BbbR^r$ of finite measure.   \Prf\ If $K\subseteq E$
is compact, then $\capacity B_K\le\capacity K\le c(E)$.   But as
$\mu E=\sup\{\mu K:K\subseteq E$ is compact$\}$,
$\diam B_E=\sup\{\diam B_K:K\subseteq E$ is compact$\}$;
because capacity is a continuous function of radius (479Da),

$$\eqalign{\capacity B_E)
&=\sup\{\capacity B_K:K\subseteq E\text{ is compact}\}\cr
&\le\sup\{\capacity K:K\subseteq E\text{ is compact }\}
\le c(E).  \text{ \Qed}\cr}$$

Finally, if $D$ is any set of finite outer measure, there is a G$_{\delta}$
set $E\supseteq D$ such that $c(E)=c(D)$ and $\mu E=\mu^*D$, so that

\Centerline{$\capacity B_D=\capacity B_E\le c(E)=c(D)$,}

\noindent and we have the general result claimed.
}%end of proof of 479V

\leader{*479W}{}\cmmnt{ I conclude with an alternative representation of
Choquet-Newton capacity $c$ in terms of a measure on the space of closed
subsets of $\BbbR^r$.

\medskip

\noindent}{\bf Theorem}
Let $\Cal C^+$ be the family of non-empty closed subsets of $\BbbR^r$,
with its Fell topology\cmmnt{ (4A2T)}.   Then there is a unique Radon
measure $\theta$ on $\Cal C^+$ such that
$\theta^*\{C:C\in\Cal C^+$, $D\cap C\ne\emptyset\}$ is the Choquet-Newton
capacity $c(D)$ of $D$ for every $D\subseteq\BbbR^r$.

\proof{{\bf (a)} Recall that the Fell topology on
$\Cal C=\Cal C^+\cup\{\emptyset\}$ is compact (4A2T(b-iii)) and metrizable
(4A2Tf), so $\Cal C^+$ is locally compact and Polish.
For $D\subseteq\BbbR^r$, set
$\Psi D=\{C:C\in\Cal C^+$, $C\cap D\ne\emptyset\}$.   Of course
$\Psi(\bigcup\Cal A)=\bigcup_{D\in\Cal A}\Psi D$ for every family
$\Cal A$ of subsets of $\BbbR^r$.

\medskip

{\bf (b)} Let $\Omega'$ be the set of those $\omega\in\Omega$ such that
$\lim_{t\to\infty}\|\omega(t)\|=\infty$;  because $r\ge 3$, $\Omega'$ is
conegligible in $\Omega$ (478Md).   If $\omega\in\Omega'$, then
$\omega[\,\coint{0,\infty}\,]$ is closed.
For $x\in\BbbR^r$ and $\omega\in\Omega'$, set
$h_x(\omega)=x+\omega[\,\coint{0,\infty}\,]\in\Cal C^+$.   Then
$h_x:\Omega'\to\Cal C^+$ is Borel measurable.   \Prf\ ($\alpha$) If
$G\subseteq\BbbR^r$ is open, then

\Centerline{$\{\omega:\omega\in\Omega'$, $h_x(\omega)\cap G\ne\emptyset\}
=\bigcup_{t\ge 0}\{\omega:x+\omega(t)\in G\}$}

\noindent is relatively open in $\Omega'$.
($\beta$) If $K\subseteq\BbbR^r$ is compact,

\Centerline{$\{(\omega,t):x+\omega(t)\in K\}$}

\noindent is closed in $\Omega\times\coint{0,\infty}$, so its projection
$\{\omega:x+\omega[\,\coint{0,\infty}\,]\cap K\ne\emptyset\}$ is
F$_{\sigma}$, and
$\{\omega:\omega\in\Omega'$, $h_x(\omega)\cap K=\emptyset\}$ is a
G$_{\delta}$ set in $\Omega'$.   ($\gamma$)
Because $\Cal C^+$ is hereditarily Lindel\"of, this is enough to prove that
$h_x$ is Borel measurable (4A3Db).\ \Qed

\medskip

{\bf (c)} Let $\Tau$ be the ring of subsets of $\Cal C^+$ generated by
sets of the form $\Psi E$ where $E\subseteq\BbbR^r$ is bounded and is
either compact or open.   Then we have an additive functional
$\phi:\Tau\to\coint{0,\infty}$ such that
$\phi(\Psi K)=\capacity K$ for every compact set $K\subseteq\BbbR^r$.
\Prf\ For $x\in\BbbR^r$ let $h_x:\Omega'\to\Cal C^+$ be as in (b).
Then we have a corresponding scaled Radon
image measure $\phi_x=\|x\|^{r-2}(\mu_W)_{\Omega'}h_x^{-1}$ on $\Cal C^+$
(418I), defined by setting
$\phi_xH=\|x\|^{r-2}\mu_W\{\omega:x+\omega[\,\coint{0,\infty}\,]\in H\}$
whenever this is defined.   If $E\subseteq\BbbR^r$ is either compact or
open, then

\Centerline{$\{\omega:x+\omega[\,\coint{0,\infty}\,]\cap E\ne\emptyset\}$}

\noindent is F$_{\sigma}$ or open, respectively, so $\phi_x(\Psi E)$ is
defined;
accordingly $\phi_xH$ is defined for every $H\in\Tau$.   If
$\gamma>0$, $E\subseteq B(\tbf{0},\gamma)$ and
$\|x\|>\gamma$, then

\Centerline{$\phi_x^*(\Psi E)
\le\phi_x(\Psi(B(\tbf{0},\gamma)))
=\|x\|^{r-2}\hp(B(\tbf{0},\gamma)-x)
=\gamma^{r-2}$}

\noindent(478Qc).   So $\limsup_{\|x\|\to\infty}\phi_xH$ is finite for
every $H\in\Tau$.   Take an ultrafilter $\Cal F$ on $\BbbR^r$ containing
$\BbbR^r\setminus B(\tbf{0},\gamma)$ for every $\gamma>0$;  then
$\phi H=\lim_{x\to\Cal F}\phi_xH$ is defined in $\coint{0,\infty}$
for every $H\in\Tau$, and $\phi$ is additive.   If $E\subseteq\BbbR^r$ is
bounded and either compact or open, then

\Centerline{$\phi(\Psi E)
=\lim_{x\to\Cal F}\|x\|^{r-2}\hp(E-x)=c(E)$}

\noindent by 479B(ii).\ \Qed

\medskip

{\bf (d)} $\phi$ is inner regular with respect to the compact sets, in the
sense that $\phi H=\sup\{\phi L:L\in\Tau$ is compact, $L\subseteq H\}$
for every $H\in\Tau$.   \Prf\ Note first that the set

\Centerline{$\Cal H
=\{H:H\in\Tau$,
  $\phi H=\sup\{\phi L:L\in\Tau$ is compact, $L\subseteq H\}\}$}

\noindent is a sublattice of $\Tau$.   Suppose that $E$,
$H\subseteq\BbbR^r$ are bounded sets which are either compact or open, and
$\epsilon>0$.   Then there are a compact $K\subseteq E$ and a bounded open
$G\supseteq H$ such that $c(E)\le\epsilon+\capacity K$ and
$c(G)\le\epsilon+c(H)$ (479E).   Now $\Psi K$ is a closed
subset of $\Cal C$ included in $\Cal C^+$, so is a compact subset of
$\Cal C^+$, while $\Psi G$ is open;  thus
$L=\Psi K\setminus\Psi G$ is a compact subset of
$\Psi E\setminus\Psi H$, and of course $L\in\Tau$.   Now

$$\eqalign{\phi(\Psi E\setminus\Psi H)
&\le\phi L+\phi(\Psi E\setminus\Psi K)+\phi(\Psi G\setminus\Psi H)\cr
&=\phi L+\phi(\Psi E)-\phi(\Psi K)+\phi(\Psi G)-\phi(\Psi H)\cr
&=\phi L+c(E)-\capacity K+c(G)-c(H)
\le\phi L+2\epsilon.\cr}$$

\noindent As $\epsilon$ is arbitrary, $\Psi E\setminus\Psi H$ belongs to
$\Cal H$.

Since any member of $\Tau$ is expressible as a finite union of finite
intersections of sets of this kind, $\Tau\subseteq\Cal H$, as required.\
\Qed

\medskip

{\bf (e)} Let $\Cal L$ be the family of compact subsets of $\Cal C^+$.
If $L\in\Cal L$, it is a closed subset of $\Cal C$ not containing
$\emptyset$, so there must be a compact set $K\subseteq\BbbR^r$ such that
$L\subseteq\Psi K$.   Thus every member of $\Cal L$ is covered by a member
of $\Tau$, and we have a functional $\phi_1:\Cal L\to\coint{0,\infty}$
defined by setting $\phi_1L=\inf\{\phi E:E\in\Tau$, $L\subseteq E\}$ for
$L\in\Cal L$.

I seek to apply 413I.   Of course $\emptyset\in\Cal L$ and $\Cal L$ is
closed under finite disjoint unions and countable intersections;  moreover,
if $\sequencen{L_n}$ is a non-increasing sequence in $\Cal L$ with empty
intersection, one of the $L_n$ must be empty, so
$\inf_{n\in\Bbb N}\phi_1L_n=0$.   Now turn to condition ($\alpha$) of the
theorem:

\Centerline{$\phi_1L_1
=\phi_1L_0+\sup\{\phi_1L:L\in\Cal L$, $L\subseteq L_1\setminus L_0\}$
whenever $L_0$, $L_1\in\Cal L$ and $L_0\subseteq L_1$.}

\medskip

\quad{\bf (i)} If $L_0$, $L\in\Cal L$ are disjoint, then
$\phi_1(L_0\cup L)\ge\phi_1L_0+\phi_1L$.   \Prf\ The topology $\frak S$ of
$\Cal C^+$ is generated by sets of the form $\Psi G$, where
$G\subseteq\BbbR^r$ is open, and by sets of the form
$\Cal C^+\setminus\Psi K$, where $K\subseteq\BbbR^r$ is compact.
It is therefore generated by

\Centerline{$\{\Psi G\setminus\Psi K:G\subseteq\BbbR^r$ is bounded and
open, $K\subseteq\BbbR^r$ is compact$\}
\subseteq\Tau$.}

\noindent So $\frak S\cap\Tau$ is a base for $\frak S$ and disjoint
compact sets in $\Cal C^+$ can be separated by members of $\Tau$
(4A2F(h-i));  let $E_0\in\Tau$ be such that
$L_0\subseteq E_0\subseteq\Cal C^+\setminus L$.   Now if $E\in\Tau$ and
$E\supseteq L_0\cup L$,

\Centerline{$\phi E=\phi(E\cap E_0)+\phi(E\setminus E_0)
\ge\phi_1L_0+\phi_1L$;}

\noindent as $E$ is arbitrary,
$\phi_1(L_0\cup L)\ge\phi_1L_0+\phi_1L$.\ \Qed

\medskip

\quad{\bf (ii)} If $L_0$, $L_1\in\Cal L$, $L_0\subseteq L_1$ and
$\epsilon>0$, there is an $L\in\Cal L$ such that
$L\subseteq L_1\cup L_0$ and $\phi_1L_1\le\phi_1L_0+\phi_1L+3\epsilon$.
\Prf\ Let $E_0$, $E_1\in\Tau$ be such that $L_0\subseteq E_0$,
$L_1\subseteq E_1$ and $\phi E_0\le\phi_1L_0+\epsilon$.
By (d), there is an
$L'\in\Cal L\cap\Tau$ such that $L'\subseteq E_1\setminus E_0$ and
$\phi L'\ge\phi(E_1\setminus E_0)-\epsilon$.   Set $L=L'\cap L_1$.
Then $L\in\Cal L$ and $L\subseteq L_1\setminus L_0$.   Let $E\in\Tau$ be
such that $L\subseteq E$ and $\phi E\le\phi_1L+\epsilon$.   Then
$L_1\subseteq E_0\cup E\cup((E_1\setminus E_0)\setminus L')$, so

\Centerline{$\phi_1L_1
\le\phi E_0+\phi E+\phi(E_1\setminus E_0)-\phi L'
\le\phi_1L_1+\phi L+3\epsilon$,}

\noindent as required.\ \Qed

Putting this together with (i), the final condition of 413I is satisfied.

\medskip

{\bf (f)} We therefore have a complete locally determined measure $\theta$
on $\Cal C^+$ extending $\phi_1$ and inner regular with respect to
$\Cal L$.   For $E\subseteq\Cal C^+$, $\theta$ measures $E$ iff $\theta$
measures $E\cap L$ for every $L\in\Cal L$ (412Ja);
so $\theta$ measures all
closed subsets of $\Cal C^+$, and is a topological measure.   Of course
$\theta$ is inner regular with respect to the compact sets.   If
$C\in\Cal C^+$, there is a bounded open set $G\subseteq\BbbR^r$ meeting
$C$, and now $\Psi G$ is an open set containing $C$ and included in the
compact set $\Psi\overline{G}$;  accordingly

\Centerline{$\theta(\Psi G)\le\theta(\Psi\overline{G})
=\phi_1(\Psi\overline{G})=\phi(\Psi\overline{G})
=\capacity\overline{G}$}

\noindent is finite.   Thus $\theta$ is locally finite and is a Radon
measure.

\medskip

{\bf (g)} As in (f), we have

\Centerline{$\theta(\Psi K)=\phi_1(\Psi K)=\phi(\Psi K)=\capacity K$}

\noindent for every compact $K\subseteq\BbbR^r$.   Next,
$\theta(\Psi G)=c(G)$ for every open $G\subseteq\BbbR^r$.   \Prf\
If $G$ is bounded,

$$\eqalign{\theta(\Psi G)
&=\sup\{\theta L:L\subseteq\Psi G\text{ is compact}\}\cr
&=\sup\{\phi_1L:L\subseteq\Psi G\text{ is compact}\}
\le\phi(\Psi G)
=c(G)\cr
&=\sup\{\capacity K:K\subseteq G\text{ is compact}\}\cr
&=\sup\{\theta(\Psi K):K\subseteq G\text{ is compact}\}
\le\theta(\Psi G).\cr}$$

\noindent If $G$ is unbounded, then there is a non-decreasing sequence
$\sequencen{G_n}$ of bounded open sets with union $G$, so

\Centerline{$\theta(\Psi G)
=\theta(\bigcup_{n\in\Bbb N}\Psi G_n)
=\sup_{n\in\Bbb N}\theta(\Psi G_n)
=\sup_{n\in\Bbb N}c(G_n)
=c(G)$.  \Qed}

\medskip

{\bf (h)} Now suppose that $D\subseteq\BbbR^r$ is any bounded set.   We
have

\Centerline{$\theta^*(\Psi D)
\le\inf\{\theta(\Psi G):G\supseteq D\text{ is open}\}
=\inf\{c(G):G\supseteq D\text{ is open}\}
=c(D)$.}

\noindent\Quer\ Suppose, if possible, that $\theta^*(\Psi D)<c(D)$.
Let $G\supseteq D$ be a bounded open set.   Then there is a compact
$L\subseteq\Psi G\setminus\Psi D$ such that
$\theta L>\theta(\Psi G)-c(D)$.   Set $F=\bigcup_{C\in L}C$;  then $F$
is closed (4A2T(e-iii)) and disjoint from $D$, so $G\setminus F$ is open,
$D\subseteq G\setminus F$ and $\Psi(G\setminus F)$ is disjoint from $L$.
But this means that

\Centerline{$c(D)\le c(G\setminus F)=\theta(\Psi(G\setminus F))
\le\theta(\Psi G)-\theta L<c(D)$,}

\noindent which is absurd.\ \BanG\  So $\theta^*(\Psi D)=c(D)$.

If $D\subseteq\BbbR^r$ is any set, then it is expressible as the union of a
non-decreasing sequence $\sequencen{D_n}$ of bounded sets, so

\Centerline{$c(D)
=\lim_{n\to\infty}c(D_n)
=\lim_{n\to\infty}\theta^*(\Psi D_n)
=\theta^*(\bigcup_{n\in\Bbb N}\Psi D_n)
=\theta^*(\Psi D)$.}

\noindent Thus $\theta$ has all the properties declared.

\medskip

{\bf (i)} To see that $\theta$ is unique, consider the base $\Cal V$ for
the topology of $\Cal C^+$ consisting of sets of the form
$\bigcap_{i\in I}\Psi G_i\setminus\Psi K$ where $\familyiI{G_i}$ is a
non-empty finite family of bounded open sets in $\BbbR^r$ and
$K\subseteq\BbbR^r$ is compact.   The conditions that $\theta$ must satisfy
determine its value on any set of the form
$\Psi(G\cup K)=\Psi G\cup\Psi K$ where $G\subseteq\BbbR^r$ is open and
$K\subseteq\BbbR^r$ is compact, and therefore determine its values on
$\Cal V$.   By 415H(iv), $\theta$ is fixed by these.
}%end of proof of 479W

\exercises{\leader{479X}{Basic exercises (a)}
%\spheader 479Xa
Let $\zeta$ be a Radon measure on $\BbbR^r$.  Show that

\Centerline{$\zeta\BbbR^r
=\lim_{\gamma\to\infty}\Bover2{r\beta_r\gamma^2}
  \int_{B(\tbf{0},\gamma)}W_{\zeta}d\mu$.}
%479J out of order query

\sqheader 479Xb(i) Show directly from 479B-479C that Choquet-Newton
capacity $c$ is
invariant under isometries of $\BbbR^r$.
(ii) Show that $c(\alpha D)=\alpha^{r-2}c(D)$
whenever $\alpha\ge 0$ and $D\subseteq\BbbR^r$.
%479C

\spheader 479Xc Suppose that $\zeta_1$ and $\zeta_2$ are totally finite
Radon measures on $\BbbR^r$.
Show that $W_{\zeta_1*\zeta_2}=\zeta_1*W_{\zeta_2}=\zeta_2*W_{\zeta_1}$.
%479J

\spheader 479Xd Show that there is a closed set $F\subseteq\Bbb R^r$
such that $\hp(F)<1$ but $c(F)=\infty$.   \Hint{look at the proof of
479Ma.}
%479M

\spheader 479Xe Let $K\subseteq\BbbR^r$ be compact.  Show that
$\interior\{x:\tilde W_K(x)<1\}$ is the unbounded component of
$\BbbR^r\setminus\supp\lambda_K$.   \Hint{setting $L=\supp\lambda_K$,
show that $\capacity(\supp\lambda_K)=\capacity K$ so that
$\tilde W_K=\tilde W_L$.}
%479N

\spheader 479Xf Let $A\subseteq\BbbR^r$ be an analytic set such that
$c(A)<\infty$.   Show that
$\tilde W_A=\sup\{W_{\zeta}:\zeta$ is a Radon measure on $\BbbR^r$,
$A$ is $\zeta$-conegligible, $W_{\zeta}\le 1\}$.
%479N 479M query out of order

\sqheader 479Xg Show that there
is a universally negligible set $D\subseteq B(\tbf{0},1)$ such that
$c(D)=1$.   \Hint{use the ideas of 439F to find $D$ such that
$\{\|x\|:x\in D\}$ is universally negligible and
$x\mapsto\|x\|:D\to[0,1]$ is injective, but
$\nu^*\{\Bover{x}{\|x\|}:x\in D$, $\|x\|\ge 1-\delta\}=r\beta_r$
for every $\delta\in\ooint{0,1}$;  compute $c(D)$ with the aid of
479D, 479P(c-iii-$\alpha$) and 479P(c-vii).}
		
\spheader 479Xh Suppose that $D\subseteq D'\subseteq\BbbR^r$ and
$c(D')<\infty$.
Show that $\int fd\lambda_D\le\int fd\lambda_{D'}$
for every lower semi-continuous superharmonic $f:\BbbR^r\to[0,\infty]$.
%479P

%if W_{\zeta}\le W_{\zeta'} does it follow that
%\int fd\zeta\le\int fd\zeta'  for non-neg lsc superharmonic f ?
%yes I think, see 479qS in mt47bits.tex
%so in fact  \zeta*f\le\zeta'*f

\spheader 479Xi Let $D\subseteq\BbbR^r$ be a bounded set.   Show that
$c(D)=\lim_{\|x\|\to\infty}\|x\|^{r-2}\hp^*(D-x)$.  \Hint{477Id.}
%479P

\spheader 479Xj Let $T:\BbbR^r\to\BbbR^r$ be an isometry, and
$D\subseteq\BbbR^r$ a set such that $c(D)<\infty$.   Show that
$\tilde W_{T[D]}Tx)=\tilde W_D(x)$ for every $x\in\BbbR^r$ and that
$\lambda_{T[D]}$ is the image measure $\lambda_DT^{-1}$.
%479P

\spheader 479Xk Let $D\subseteq\BbbR^r$ be a set such that
$c(D)<\infty$, and $\alpha>0$.   Show that
$\tilde W_{\alpha D}(x)=\tilde W_D(\bover1{\alpha}x)$
for every $x\in\BbbR^r$, and that
$\lambda_{\alpha D}=\alpha^{r-2}\lambda_DT^{-1}$,
where $T(x)=\alpha x$ for $x\in\BbbR^r$.
%479P 479Xj

\spheader 479Xl Show that
$c(D)
=\inf\{\zeta\BbbR^r:\zeta$ is a Radon measure on $\BbbR^r$,
$W_{\zeta}\ge\chi D\}
=\inf\{\energy(\zeta):\zeta$ is a Radon measure on $\BbbR^r$,
$W_{\zeta}\ge\chi D\}$
for any $D\subseteq\BbbR^r$.
%479P

\spheader 479Xm Let $K\subseteq\BbbR^r$ be a compact set, with complement
$G$, and $\Phi$ the set of
continuous harmonic functions $f:G\to[0,1]$ such that
$\lim_{\|x\|\to\infty}f(x)=0$.   Show that
$\tilde W_K\restr G$ is the greatest element of $\Phi$.
\Hint{479Pb, 478Pc.}
%479P

\spheader 479Xn(i) Show that if $G$ is a convex open set then
$\hp(G-x)=1$ for every $x\in\overline{G}$.   (ii) Show that if
$D\subseteq\BbbR^r$ is a convex bounded set with non-empty
interior, then $\tilde W_D$ is continuous.
%479P

\spheader 479Xo Show that if $D$, $D'\subseteq\BbbR^r$
and $c(D\cup D')<\infty$, then
$\tilde W_{D\cap D'}+\tilde W_{D\cup D'}\le\tilde W_D+\tilde W_{D'}$.
%479P

\spheader 479Xp Let $D\subseteq\BbbR^r$ be a set such that
$c(D)<\infty$, and set $\tilde D=\{x:\tilde W_D(x)=1\}$.   Show that
(i) $D\setminus\tilde D$ is polar (ii)
$\lambda_{\tilde D}=\lambda_D$.   \Hint{reduce to the case in which $D=A$
is analytic;  use 479Fg to show that $\tilde W_{\tilde A}\le\tilde W_A$;
use 479J(b-v).}
%479P

\spheader 479Xq Let $\Cal A$ be the set of subsets of $\BbbR^r$
with finite Choquet-Newton capacity, and
$\rho$ the pseudometric $(D,D')\mapsto 2c(D\cup D')-c(D)-c(D')$ (432Xj).
(i) Show that $\|U_{\lambda_D}-U_{\lambda_{D'}}\|^2_2\le 2c_r\rho(D,D')$
for $D$, $D'\in\Cal A$.
(ii) Show that $\rho(D,D')=0$ iff $\lambda_D=\lambda_{D'}$.
%479P query out of order

\spheader 479Xr Suppose that $D\subseteq\BbbR^r$.   Show that the
following are equiveridical:  (i) $D$ is polar;  (ii) there is some
$x\in\BbbR^r$ such that $\hp(D-x)=0$;  (iii)
$\hp((D\setminus\{x\})-x)=0$ for every $x\in\BbbR^r$.
%479P
%if (i), enlarge $D$ to analytic polar, then $\lambda_D=0$ so
%$\tilde W_D=0$ so (iii).   If (ii), again can enlarge to analytic,
%and $\tilde W_D=0$ so (iii) and (i).
%if (iii), $D\ne\BbbR^r$ so (ii).

\spheader 479Xs Suppose that $\sequencen{D_n}$ is a non-increasing sequence
of subsets of $\BbbR^r$ such that $\inf_{n\in\Bbb N}c(D_n)$ is finite and
$\bigcap_{n\in\Bbb N}D_n=\bigcap_{n\in\Bbb N}\overline{D_n}=F$ say.   Show
that $\sequencen{\lambda_{D_n}}\to\lambda_D$ for the narrow topology, and
that $\sequencen{c(D_n)}\to c(F)$.   (Compare 479Ye.)
%479P 479Ec

\spheader 479Xt For $\omega\in\Omega$ set
$\tau(\omega)=\sup\{t:\|\omega(t)\|\le 1\}$.   (i) Show that 
$\tau:\Omega\to[0,\infty]$ is measurable.
(ii) Show that if $r\le 2$ then $\tau=\infty$ a.e.   (iii) Show that if
$r\ge 3$ then $\tau$ is not a stopping time.   (iv) Show that if
$3\le r\le 4$ then $\tau$ is finite a.e., but has infinite expectation.
(v) Show that if $r\ge 5$ then $\tau$ has finite expectation.
({\it Hint\/}:  show that if $r\ge 2$ then

\Centerline{$\Pr(\tau\ge t)
=\Bover1{(\sqrt{2\pi})^r}\int e^{-\|x\|^2/2}
\Bover1{\max(1,(\sqrt t)^{r-2}\|x\|^{r-2})}dx$.)}
%479R 479Da 479Pb {\smc Chung 95}, chap 11

\leader{479Y}{Further exercises (a)}%
%\spheader 479Ya
(i) Show that there is an open set $G\subseteq B(\tbf{0},1)$,
dense in $B(\tbf{0},1)$, such that $c(G)<1$.   (ii) Show that
$\capacity(\supp\lambda_G)=1$.
%479E

\spheader 479Yb In 479G, suppose that $0<\alpha<r$, $0<\beta<r$ and
$\alpha+\beta>r$.   Show that

$$k_{\alpha+\beta-r}
=\bover{\Gamma(r-\bover{\alpha+\beta}2)\Gamma(\bover{\alpha}2)
                  \Gamma(\bover{\beta}2)}
   {(\sqrt\pi)^r\Gamma(\bover{\alpha+\beta-r}2)\Gamma(\bover{r-\alpha}2)
                  \Gamma(\bover{r-\beta}2)}
   k_{\alpha}*k_{\beta}.$$
%479Ia

\spheader 479Yc Let $A\subseteq\BbbR^r$ be an analytic set with
$c(A)<\infty$.   (i) Show that for every $\gamma>0$ there is a Radon
measure $\zeta_{\gamma}$ on $\BbbR^r$ such that
$\family{x}{\partial B(\tbf{0},\gamma)}{\Bover1{r\beta_r\gamma}\mu^{(A)}_x}$ is a
disintegration of $\zeta_{\gamma}$ over the subspace measure
$\nu_{\partial B(\tbf{0},\gamma)}$.
(ii) Show that $\lim_{\gamma\to\infty}\zeta_{\gamma}=\lambda_A$ for the
total variation metric on $M^+_{\text{R}}(\BbbR^r)$.
%479M

\spheader 479Yd Set $c'(D)=\sup\{\zeta^*D:\zeta$ is a Radon measure on
$\BbbR^r$ such that $W_{\zeta}\le 1$ everywhere$\}$ for
$D\subseteq\BbbR^r$.   Show that $c'$ is a Choquet capacity on $\BbbR^r$,
extending Newtonian capacity for compact sets, which is different from
Choquet-Newton capacity.
%479N 479Xg 479P

\spheader 479Ye
Let $\sequencen{D_n}$ be a non-increasing sequence of
subsets of $\BbbR^r$ with finite Choquet-Newton capacity.
For each $n\in\Bbb N$, set $\tilde D_n=\{x:\tilde W_{D_n}(x)=1\}$, and set
$A=\bigcap_{n\in\Bbb N}\tilde D_n$.   Show that
$c(A)=\inf_{n\in\Bbb N}c(D_n)$ and that $\lambda_A$ is the limit of
$\sequencen{\lambda_{D_n}}$ for the narrow topology on
$M^+_{\text{R}}(\BbbR^r)$.
%479Xp 479P 478Yi

\spheader 479Yf Let $\Cal A$, $\rho$ be as in 479Xq, and let
$(\bar{\Cal A},\bar\rho)$ be the corresponding metric space, identifying
members of $\Cal A$
which are zero distance apart.   Show that $\bar{\Cal A}$ is complete.
%479Ye 479P

\spheader 479Yg
Let $E\subseteq\BbbR^r$ be a set of finite Lebesgue
measure, and $B_E$ the ball with centre $0$ and the same measure as $E$.
Show that $\energy(\mu\LLcorner E)\le\energy(\mu\LLcorner B_E)$.
%479V mt47bits

\spheader 479Yh Prove 479V from 479U and 476Yb.
%479V

\spheader 479Yi(i) Show that $c$ is alternating of all orders, that is,

\Centerline{$\sum_{J\subseteq I,\#(J)\text{ is even}}
   c(D\cup\bigcup_{i\in J}D_i)
\le\sum_{J\subseteq I,\#(J)\text{ is odd}}c(D\cup\bigcup_{i\in J}D_i)$}

\noindent whenever $I$ is a
non-empty finite set, $\familyiI{D_i}$ is a family of
subsets of $\BbbR^r$ and $D$ is another subset of
$\BbbR^r$.   (Cf.\ 132Yf.)
(ii) Show that if $c(D\cup\bigcup_{i\in I}D_i)<\infty$, then

\Centerline{$\sum_{J\subseteq I,\#(J)\text{ is even}}
   \tilde W_{D\cup\bigcupop_{i\in J}D_i}
\le\sum_{J\subseteq I,\#(J)\text{ is odd}}
   \tilde W_{D\cup\bigcupop_{i\in J}D_i}$.}
%479W

\spheader 479Yj Let us say that if $X$ is a Polish space, a set
$A\subseteq X$ is {\bf projectively universally measurable} if $W[A]$ is
universally measurable whenever $Y$ is a Polish space and
$W\subseteq X\times Y$ is analytic.   Show that we can replace
the word `analytic' by the phrase `projectively universally measurable' in
all the theorems of this section.
%479 notes

\spheader 479Yk Suppose that $A\subseteq\BbbR^r$ is analytic and non-empty,
and $x\in\BbbR^r$ is such that $\rho(x,A)=\delta>0$.
Show that $\energy(\mu^{(A)}_x)\le\Bover1{\delta^{r-2}}$.
%479C out of order query mt47bits

\def\clcap{\text{cl}_{\text{cap}}}
\spheader 479Yl Show that if $D\subseteq\BbbR^r$ and $c(D)<\infty$, then
$c(\{x:\tilde W_D(x)\ge\gamma\})\le\Bover1{\gamma}c(D)$ for every
$\gamma>0$.
%479P mt47bits

\spheader 479Ym
For a set $D\subseteq\BbbR^r$ with $c(D)<\infty$, set
$\clcap D=\{x:\tilde W_D(x)=1\}$.   Show that
$c(D)=c(\clcap D)=c(D\cup\clcap D)$.
%479P mt47bits 479Yl
}%end of exercises

\endnotes{
\Notesheader{479} Newtonian potential is another of the great concepts of
mathematics, and is one of the points at which physical problems and
intuitions have stimulated and illuminated the development of analysis.
As with all the best ideas of mathematics, there is more than one route to
it, and any proper understanding of it must include a matching of the
different approaches.   In the exposition here I start with a description
of equilibrium measures in terms of harmonic measures (479B), themselves
defined in 478P in terms of Brownian motion.   We are led quickly to
definitions of capacity and equilibrium potential (479C), with some
elementary properties (479D).   Moreover, some very striking further
results (479E, 479W) are already accessible.

However we are still rather
far from the original physical concept of `capacity' of a conductor.
If you have ever studied electrostatics, the ideas here may recall some
basic physical principles.   The kernel $x\mapsto\Bover1{\|x\|^{r-2}}$
represents the potential energy field of a point mass or charge;
the potential $W_{\zeta}$ represents the field due to a mass or
charge with distribution $\zeta$.   The capacity of a set $K$ is the
largest charge that can be put on $K$ without raising the potential of
any point above $1$ (479Na), and the infimum of the charges
which raise the potential of every point of $K$ to $1$ (479P(c-v)).
The result that $\lambda_K$ is supported by $\partial K$ (479B(i))
corresponds to the principle that the charge on a conductor always collects
on the surface of the conductor;  479D(b-iii)
corresponds to the principle that there is no electric field
inside a conductor.

At the same time, the
equilibrium measure is supposed to be the (unique) distribution of the
charge, which on physical grounds ought to be the distribution with least
energy, as in 479K.   To reach these ideas, it seems that we need to know
various non-trivial facts from classical analysis, which I set out in
479G-479I.  %479G 479H 479I
The deepest of these is in 479Ib:  for the Riesz kernels $k_{\alpha}$,
the convolution $\zeta*k_{\alpha}$ determines the totally finite Radon
measure $\zeta$.   I do not know of any way of establishing this except
through the Fourier analysis of 479H and the detailed calculations of 479G
and 479Ia.

The ideas here are connected in so many ways that there is no clear flow to
the logic, and we are more than usually in danger of using circular
arguments.   In my style of exposition, this complexity manifests itself in
an exceptional density of detailed back-references;  I hope that these
will enable you to check the proofs effectively.   On a larger scale,
the laborious zigzag
progression from the original notion of capacity of compact
sets, as in 479K and 479U, through bounded analytic sets (479B, 479E)
and general analytic sets (479M, 479N) to arbitrary sets (479P),
displays a choice of path to which there are surely many
alternatives.

Of course we cannot expect all the properties of
Newtonian capacity to have recognizable forms in such a general context
as that of 479P
(see 479Xg, which shows that we cannot hope to replicate the ideas of
479Na-479Nc), but the elementary results if 479D mostly extend (479Pc).
More importantly, we have a quite new characterization of
equilibrium potentials (479Pb).
With these techniques available, we can learn a good deal more about
Brownian motion.   479R is a curious
and striking fact to go with 477K, 477L and 478M.   It is not a surprise
that capacity and Hausdorff dimension should be linked, but it is notable
that the phase change is at dimension $r-2$ (479Q);  this goes naturally
with 479P(c-vii).   I know of no such dramatic difference between four and
five dimensions, but for some purposes 479Xt marks a significant change.

My treatment is an unconventional one, so perhaps I should indicate points
where you should expect other authors to diverge from it.
While the notions of Newtonian
capacity, equilibrium measure and equilibrium potential are solidly
established for compact sets in $\BbbR^r$ (at least up to scalar factors,
and for $r\ge 3$),
the extension to general bounded analytic sets is not I think standard.
(I try to signal this by writing $c(A)$ in place of $\capacity A$, after
479E, for sets which are not guaranteed to be compact, even when the
definition in 479C(a-i) is applicable.
The fact that this step gives very little extra trouble is a
demonstration of the power of the Brownian-motion approach.)   The
further extension of Newtonian
capacity, defined on compact sets, to a Choquet capacity, defined on every
subset of $\BbbR^r$ (479Ed), is surely not standard, which is why I
give the extension a different name.   (While Choquet certainly
considered the capacity which I here call `Choquet-Newton capacity', I
fear that the phrase has no real historical justification;
but I hope it will convey some of the right ideas.)
You may have noticed that I
give essentially nothing concerning differential equations, which have
traditionally been one of the central concerns of potential theory;  there
are hints in 479Xm and 479T.

A weakness in the formulae of 479B is that
they are not self-evidently translation-invariant.   Of course it is
easy to show that in fact we have an isometry-invariant construction
(479Xb), and this can also be seen from the descriptions of capacity
and equilibrium potentials in 479N and 479Pb.
Because the capacity $c$ is countably subadditive,
it is easy to build a dense open subset $G$ of $\BbbR^r$ such that
$c(G)$ is finite (see 479Ya), and for such a set we cannot ask that
$\lambda_G$ should be
describable as a limit of $\|x\|^{r-2}\mu_x^{(G)}$ as $\|x\|\to\infty$.
But if we start from 479B(i) rather than 479B(ii), we do have an averaged
form, with

\Centerline{$\lambda_A=\lim_{\gamma\to\infty}\Bover1{r\beta_r\gamma}
  \int_{\partial B(\tbf{0},\gamma)}\mu_x^{(A)}\nu(dx)$}

\noindent whenever $A$ is an analytic set and $c(A)<\infty$ (479Yc; see
also 479J(b-vi) and 479Xa).

The factor $(\sqrt{2\pi})^r$ in 479H repeats that of
283Wg\formerly{2{}83Wj}.   The
appearance of $\sqrt{2\pi}$ in 283M, but not 445G, is proof that
the conventions of Chapter 28 are not reconcilable with those of
\S445.

In 479O I describe one of the important notions of `small' set in
Euclidean space, to go with `negligible' and `meager'.   I have no space to
deal with it properly here,
but the applications in the proofs of 479P, 479R
and 479S will give an idea of its uses;  another is in 479Xm.   As another
example of the logical complexity of the patterns here, consider the
problem of either proving 479Pb without 479O, or extending 479O to cover
479Xr without passing through a version of 479P.

Quite a lot of the work here is caused by the need to accommodate
discontinuous equilibrium potentials (479S).   This has been an
important theme in general potential theory.   479Pb shows that
the problem is essentially geometric:  if a compact set $K$ has a
sufficiently narrow spike at $e$, then a Brownian path starting at $e$ can
easily fail to enter $K$ again.

As I have written the theory out, 479T-479U are rather separate from the
rest, being closer in spirit to the work of \S\S473-475.   They explore
some more of the basic principles of potential theory.   Note, in
particular, the formula of 479Ta, which amounts to saying that (under the
right conditions) a function $g$ is a multiple of $k_{r-2}*\nabla^2g$;
of course this can be thought of as a method of finding a
particular solution of the equation $\nabla^2g=f$;  equally, it gives an
approach to the problem of expressing a given $g$ as a potential
$W_{\zeta}$.   From 479Tb and 479Tc we see that in the sense of
distributional
derivatives we can think of $r(r-2)\beta_r\zeta$ as representing
the Laplacian $-\nabla^2W_{\zeta}$;  recall that as $W_{\zeta}$ is
superharmonic (479Fb), we expect $\nabla^2W_{\zeta}$ to be negative (478E).

I give a bit of space to 479V because it links the material here to that of
\S476, and this book is about such linkages, and because it supports
my thesis that capacity is a geometrical concept.
479W is characteristic of
Choquet capacities which are alternating of all orders (479Yi).
I spell it out
here because it calls on the Fell topology, which is important
elsewhere in this volume.

It is natural to ask which of the ideas here applying to analytic sets
can be extended to wider classes.   If you look back
to where analytic sets first entered the discussion, in the theory of
hitting times
(455M), you will see that we needed a class of universally measurable
sets which would be invariant under various operations, notably
projections (479Yj).   In Volume 5 we shall meet axiom systems in which
there are various interesting possibilities.

This section is firmly directed at Euclidean space of three or more
dimensions.
The harmonic and Fourier analysis of 479G-479I %479G 479H 479I
applies unchanged to dimensions $1$ and $2$;  so does 479Tb.
On the line, Brownian
hitting probabilities are trivial;  in the plane, they are very
different from hitting probabilities in higher dimensions, but still of
considerable interest.   Theorems 479B, 479E and 479W still work,
but `capacity', if defined by the formulation of 479Ca, is bounded by
$1$.   The geometric nature of
the results changes dramatically, and 479I cannot be applied in the
same way, since we no longer have $0<r-2$.
}%end of notes

\discrpage

