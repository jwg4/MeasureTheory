\frfilename{mt282.tex}
\versiondate{24.9.09}
\copyrightdate{1996}

\def\chaptername{Fourier analysis}
\def\sectionname{Fourier series}
\def\Var{\mathop{\text{Var}}}

\newsection{282}

Out of the enormous theory of Fourier series, I extract a few results
which may at least provide a basis for further study.   I give the
definitions of Fourier and Fej\'er sums (282A), with five of the most
important results concerning their convergence (282G, 282H, 282J, 282L,
282O).   On the way I include the Riemann-Lebesgue lemma (282E).   I end
by mentioning convolutions (282Q).

\vleader{48pt}{282A}{Definition} Let $f$ be an integrable complex-valued
function defined almost everywhere in $\ocint{-\pi,\pi}$.

\header{282Aa}{\bf (a)} The {\bf
Fourier coefficients} of $f$ are the complex numbers

$$c_k=\Bover1{2\pi}\int_{-\pi}^{\pi}f(x)e^{-ikx}dx$$

\noindent for $k\in\Bbb Z$.

\header{282Ab}{\bf (b)} The {\bf Fourier sums} of $f$ are the functions

$$s_n(x)=\sum_{k=-n}^nc_ke^{ikx}$$

\noindent for $x\in\ocint{-\pi,\pi}$, $n\in\Bbb N$.

\header{282Ac}{\bf (c)} The {\bf Fourier
series} of $f$ is the series $\sum_{k=-\infty}^{\infty}c_ke^{ikx}$,
or\cmmnt{ (because
we ordinarily consider the symmetric partial sums $s_n$)} the series
$c_0+\sum_{k=1}^{\infty}(c_ke^{ikx}+c_{-k}e^{-ikx})$.

\header{282Ad}{\bf (d)} The {\bf Fej\'er sums} of $f$ are the functions

$$\sigma_m=\Bover1{m+1}\sum_{n=0}^ms_n$$

\noindent for $m\in\Bbb N$.

\header{282Ae}{\bf (e)}\cmmnt{ It will be convenient to have a further
phrase available.}   If $f$ is
any function with $\dom f\subseteq\ocint{-\pi,\pi}$, its {\bf periodic
extension} is the function $\tilde f$, with domain
$\bigcup_{k\in\Bbb Z}(\dom f+2k\pi)$, such that $\tilde f(x)=f(x-2k\pi)$
whenever $k\in\Bbb Z$ and $x\in\dom f+2k\pi$.

\cmmnt{
\leader{282B}{Remarks} I have made two more or less arbitrary choices
here.

\header{282Ba}{\bf (a)} I have chosen to express Fourier series in their
`complex' form rather than
their `real' form.   From the point of view of pure measure theory
(and, indeed, from the point of view of the nineteenth-century origins
of the subject) there are gains in elegance from directing attention to
real functions $f$ and looking at the real coefficients

$$a_k=\Bover1{\pi}\int_{-\pi}^{\pi}f(x)\cos kx\,dx\text{ for }k\in\Bbb
N,$$

$$b_k=\Bover1{\pi}\int_{-\pi}^{\pi}f(x)\sin kx\,dx\text{ for }k\ge
1.$$

\noindent If we do this we have

\Centerline{$c_0=\Bover12a_0$,}

\noindent and for $k\ge 1$ we have

\Centerline{$c_k=\Bover12(a_k-ib_k)$,\quad $c_{-k}=\Bover12(a_k+ib_k)$,
\quad
$a_k=c_k+c_{-k}$,\quad $b_k=i(c_k-c_{-k})$,}

\noindent so that the Fourier sums become

$$s_n(x)=\Bover12a_0+\sum_{k=1}^na_k\cos kx+b_k\sin kx.$$

\noindent The advantage of this is that real functions $f$ correspond to
real coefficients $a_k$, $b_k$, so that it is obvious that if $f$ is
real-valued so are its Fourier and Fej\'er sums.   The disadvantages are
that we have to
use a variety of trigonometric equalities which are rather more
complicated than the properties of the complex exponential function
which they reflect, and that we are farther away from the natural
generalizations to locally compact abelian groups.   So both electrical
engineers and harmonic analysts tend to prefer the coefficients $c_k$.

\header{282Bb}{\bf (b)} I have taken the functions $f$ to be defined on
the interval $\ocint{-\pi,\pi}$ rather than on the circle
$S^1=\{z:z\in\Bbb C,\,|z|=1\}$.   There would be advantages in elegance
of language in using $S^1$, though I do not recall often seeing the
formula

\Centerline{$c_k=\int z^kf(z)dz$}

\noindent which is the natural translation of
$c_k=\bover1{2\pi}\int e^{ikx}f(x)dx$ under the substitution $x=\arg z$,
$dx=2\pi\nu(dz)$.   However, applications of the theory tend to deal
with periodic functions on the real line, so I work with
$\ocint{-\pi,\pi}$, and accept the fact that its group operation
$+_{2\pi}$, writing
$x+_{2\pi}y$ for whichever of $x+y$, $x+y+2\pi$, $x+y-2\pi$ belongs to
$\ocint{-\pi,\pi}$, is less familiar than multiplication on
$S^1$.

\header{282Bc}{\bf (c)} The remarks in (b) are supposed to remind you of
\S255.

\header{282Bd}{\bf (d)} Observe that if $f\eae g$ then $f$ and $g$ have
the same Fourier coefficients, Fourier sums and Fej\'er sums.   This
means that we could, if we wished, regard the $c_k$, $s_n$ and
$\sigma_m$ as associated with a member of $L_{\Bbb C}^1$, the space of
equivalence classes of integrable functions (\S242), rather than as
associated with a particular function $f$.   Since however the $s_n$ and
$\sigma_m$ appear as actual functions, and since many of the questions
we are interested in refer to their values at particular points, it is
more natural to express the theory in terms of integrable functions $f$
rather than in terms of members of $L_{\Bbb C}^1$.
}%end of comment

\leader{282C}{The problems (a)} Under what conditions, and in what
senses, do the Fourier and Fej\'er sums $s_n$ and $\sigma_m$ of a
function $f$ converge to $f$?

\header{282Cb}{\bf (b)} How do the properties of the double-ended
sequence $\langle c_k\rangle_{k\in\Bbb Z}$ reflect the properties of
$f$, and vice versa?

\cmmnt{\medskip

\noindent{\bf Remark} The theory of Fourier series has been one of the
leading topics of analysis for nearly two hundred years, and innumerable
further problems have contributed greatly to our understanding.   (For
instance:  can one characterize those sequences $\langle
c_k\rangle_{k\in\Bbb Z}$ which are the Fourier coefficients of some
integrable
function?)   But in this outline I will concentrate on the question (a)
above, with one and a half results (282K, 282Rb) addressing (b), which
will give us more than enough material to work on.

While most people would feel that the Fourier sums are somehow closer to
what we really want to know, it turns out that the Fej\'er sums are
easier to analyse, and there are advantages in dealing with them first.
So while you may wish to look ahead to the statements of 282J, 282L and
282O for an idea of where we are going, the first half of this section
will be largely about Fej\'er sums.   Note that in any case in which we
know that the Fourier sums converge (which is quite common;  see, for
instance, the examples in 282Xh and 282Xo), then if we know that the
Fej\'er sums converge to $f$, we can deduce that the Fourier sums also
do, by 273Ca.

The first step is a basic lemma showing that both the Fourier and
Fej\'er sums of a function $f$ can be thought of as convolutions of $f$
with kernels describable in terms of familiar functions.
}%end of comment

\leader{282D}{Lemma} Let $f$ be a complex-valued function which is
integrable over $\ocint{-\pi,\pi}$, and

\Centerline{$c_k=\bover1{2\pi}\int_{-\pi}^{\pi}f(x)e^{-ikx}dx$,
\quad$s_n(x)=\sum_{k=-n}^nc_ke^{ikx}$,
\quad$\sigma_m(x)=\bover1{m+1}\sum_{n=0}^ms_n(x)$}

\noindent its Fourier coefficients, Fourier sums and Fej\'er sums.
Write $\tilde f$ for the periodic extension of $f$\cmmnt{ (282Ae)}.   For
$m\in\Bbb N$, write

$$\psi_m(t)=\Bover{1-\cos(m+1)t}{2\pi(m+1)(1-\cos t)}$$

\noindent for $0<|t|\le\pi$.   \cmmnt{(If you like, you can set
$\psi_m(0)=\bover{m+1}{2\pi}$ to make $\psi_m$ continuous on
$[-\pi,\pi]$.)}

(a) For each $n\in\Bbb N$, $x\in\ocint{-\pi,\pi}$,

$$\eqalign{s_n(x)
&=\Bover1{2\pi}\int_{-\pi}^{\pi}f(t)
\Bover{\sin(n+{1\over 2}\pushbottom{3.5pt})(x-t)}
  {\sin{1\over 2}(x-t)}dt\cr
&=\Bover1{2\pi}\int_{-\pi}^{\pi}\tilde f(x+t)
\Bover{\sin(n+{1\over 2}\pushbottom{3.5pt})t}{\sin{1\over 2}t}dt\cr
&=\Bover1{2\pi}\int_{-\pi}^{\pi}f(x-_{2\pi}t)
\Bover{\sin(n+{1\over 2}\pushbottom{3.5pt})t}{\sin{1\over 2}t}dt,\cr}$$

\noindent writing $x-_{2\pi}t$ for whichever of $x-t$, $x-t-2\pi$,
$x-t+2\pi$ belongs to $\ocint{-\pi,\pi}$.

(b) For each $m\in\Bbb N$, $x\in\ocint{-\pi,\pi}$,

$$\eqalign{\sigma_m(x)
&=\int_{-\pi}^{\pi}\tilde f(x+t)\psi_m(t)dt\cr
&=\int_{0}^{\pi}(\tilde f(x+t)+\tilde f(x-t))\psi_m(t)dt\cr
&=\int_{-\pi}^{\pi}f(x-_{2\pi}t)\psi_m(t)dt.\cr}$$

(c) For any $n\in\Bbb N$,

$$\Bover1{2\pi}\int_{-\pi}^{0}
  \Bover{\sin(n+{1\over 2}\pushbottom{3.5pt})t}{\sin{1\over 2}t}dt
=\Bover1{2\pi}\int_{0}^{\pi}
\Bover{\sin(n+{1\over 2}\pushbottom{3.5pt})t}{\sin{1\over 2}t}dt
=\Bover12,
\quad\Bover1{2\pi}\int_{-\pi}^{\pi}
\Bover{\sin(n+{1\over 2}\pushbottom{3.5pt})t}{\sin{1\over 2}t}dt
=1.$$

(d) For any $m\in\Bbb N$,

\quad (i) $0\le\psi_m(t)\le\Bover{m+1}{2\pi}$ for every $t$;

\quad (ii) for any $\delta>0$, $\lim_{m\to\infty}\psi_m(t)=0$ uniformly
on $\{t:\delta\le|t|\le\pi\}$;

\quad (iii) $\int_{-\pi}^0\psi_m=\int_0^{\pi}\psi_m=\Bover12$,
\quad $\int_{-\pi}^{\pi}\psi_m=1$.


\proof{ Really all that these amount to is summing geometric series.

\medskip

{\bf (a)} For (a), we have

$$\eqalign{\sum_{k=-n}^ne^{-ikt}
&=\bover{e^{int}-e^{-i(n+1)t}}{1-e^{-it}}\cr
&=\bover{e^{i(n+{1\over 2})t}-e^{-i(n+{1\over 2})t}}
{e^{{1\over 2}it}-e^{-{1\over 2}it}}
=\bover{\sin(n+{1\over 2}\pushbottom{3.5pt})t}{\sin{1\over 2}t}.\cr}$$

\noindent So

$$\eqalign{s_n(x)
&=\sum_{k=-n}^nc_ke^{ikx}
=\Bover1{2\pi}\int_{-\pi}^{\pi}f(t)
  \bigl(\sum_{k=-n}^ne^{ik(x-t)}\bigr)dt\cr
&=\Bover1{2\pi}\int_{-\pi}^{\pi}f(t)
  \Bover{\sin(n+{1\over 2}\pushbottom{3.5pt})(x-t)}
     {\sin{1\over 2}(x-t)}dt
=\Bover1{2\pi}\int_{-\pi}^{\pi}\tilde f(t)
  \Bover{\sin(n+{1\over 2}\pushbottom{3.5pt})(x-t)}
     {\sin{1\over 2}(x-t)}dt\cr
&=\Bover1{2\pi}\int_{-\pi-x}^{\pi-x}\tilde f(x+t)
  \Bover{\sin(n+{1\over 2}\pushbottom{3.5pt})t}{\sin{1\over 2}t}dt
=\Bover1{2\pi}\int_{-\pi}^{\pi}\tilde f(x+t)
  \Bover{\sin(n+{1\over 2}\pushbottom{3.5pt})t}{\sin{1\over 2}t}dt\cr}$$

\noindent because $\tilde f$ and $t\mapsto
\bover{\sin(n+{1\over 2})t}{\sin{1\over 2}t}$ are periodic with period
$2\pi$, so that the integral from $-\pi-x$ to $-\pi$ must be the same as
the integral from $\pi-x$ to $\pi$.

For the expression in terms of $f(x-_{2\pi}t)$, we have

$$\eqalignno{s_n(x)
&=\Bover1{2\pi}\int_{-\pi}^{\pi}\tilde f(x+t)
\Bover{\sin(n+{1\over 2}\pushbottom{3.5pt})t}{\sin{1\over 2}t}dt
=\Bover1{2\pi}\int_{-\pi}^{\pi}\tilde f(x-t)
\Bover{\sin(n+{1\over 2}\pushbottom{3.5pt})(-t)}
   {\sin{1\over 2}(-t)}dt\cr
\noalign{\noindent (substituting $-t$ for $t$)}
&=\Bover1{2\pi}\int_{-\pi}^{\pi}f(x-_{2\pi}t)
\Bover{\sin(n+{1\over 2}\pushbottom{3.5pt})t}{\sin{1\over 2}t}dt\cr}$$

\noindent because (for $x$, $t\in\ocint{-\pi,\pi}$)
$f(x-_{2\pi}t)=\tilde f(x-t)$ whenever
either is defined, and $\sin$ is an odd function.

\medskip

{\bf (b)} In the same way, we have

$$\eqalign{\sum_{n=0}^m\sin(n+\Bover12)t
&=\Imag\bigl(\sum_{n=0}^me^{i(n+{1\over 2})t}\bigr)
=\Imag\bigl(e^{{1\over 2}it}\sum_{n=0}^me^{int}\bigr)\cr
&=\Imag\bigl(e^{{1\over 2}it}\bover{1-e^{i(m+1)t}}{1-e^{it}}\bigr)
=\Imag\bigl(\bover{1-e^{i(m+1)t}}{e^{-{1\over 2}it}
   -e^{{1\over2}it}}\bigr)\cr
&=\Imag\bigl(\bover{1-e^{i(m+1)t}}{-2i\sin{1\over 2}t}\bigr)
=\Imag\bigl(\bover{i(1-e^{i(m+1)t})}{2\sin{1\over 2}t}\bigr)\cr
&=\bover{1-\cos(m+1)t}{2\sin{1\over 2}t}.\cr}$$

\noindent So

\Centerline{$\sum_{n=0}^m
  \Bover{\sin(n+{1\over 2}\pushbottom{3.5pt})t}{\sin{1\over 2}t}
=\Bover{1-\cos(m+1)t}{2\sin^2{1\over 2}t}
=\Bover{1-\cos(m+1)t}{1-\cos t}
=2\pi(m+1)\psi_m(t)$.}

\noindent Accordingly,

$$\eqalignno{\sigma_m(x)
&={\Bover1{m+1}}\sum_{n=0}^ms_n(x)\cr
&={\Bover1{m+1}}\sum_{n=0}^m{\Bover1{2\pi}}\int_{-\pi}^{\pi}
   \tilde f(x+t){\Bover{\sin(n+{1\over
2}\pushbottom{3.5pt})t}{\sin{1\over 2}t}}dt\cr
&={\Bover1{2\pi}}\int_{-\pi}^{\pi}
   \tilde f(x+t)\bigl({\Bover1{m+1}}\sum_{n=0}^m
   {\Bover{\sin(n+{1\over 2}\pushbottom{3.5pt})t}{\sin{1\over 2}t}}
     \bigr)dt\cr
&=\int_{-\pi}^{\pi}\tilde f(x+t)\psi_m(t)dt
=\int_{-\pi}^{\pi}f(x-_{2\pi}t)\psi_m(t)dt\cr}$$

\noindent as in (a), because $\cos$ and $\psi_m$ are even functions.
For the same reason,

$$\int_{0}^{\pi}\tilde f(x-t)\psi_m(t)dt
=\int_{-\pi}^{0}\tilde f(x+t)\psi_m(t)dt,$$

\noindent so

$$\sigma_m(x)=\int_0^{\pi}(\tilde f(x+t)+\tilde f(x-t))\psi_m(t)dt.$$


\medskip

{\bf (c)} We need only look at where the formula
$\bover{\sin(n+{1\over 2})t}{\sin{1\over 2}t}$ came from to see that

$$\eqalign{\Bover1{2\pi}\int_I
  \Bover{\sin(n+{1\over 2}\pushbottom{3.5pt})t}{\sin{1\over 2}t}dt
&=\Bover1{2\pi}\int_I\sum_{k=-n}^ne^{ikt}dt\cr
&=\Bover1{2\pi}\int_I(1+2\sum_{k=1}^n\cos kt)dt
=\Bover12\cr}$$

\noindent for both $I=[-\pi,0]$ and $I=[0,\pi]$, because
$\int_I\cos kt\,dt=0$ for every $k\ne 0$.

\medskip

{\bf (d)(i)} $\psi_m(t)\ge 0$ for every $t$ because
$1-\cos(m+1)t$, $1-\cos t$ are always greater than or equal to $0$.
For the upper bound, we have, using the constructions in (a) and (b),

$$\bigl|\Bover{\sin(n+{1\over 2}\pushbottom{3.5pt})t}
   {\sin{1\over 2}t}\bigr|
=\bigl|\sum_{k=-n}^ne^{ikt}\bigr|
\le 2n+1$$

\noindent for every $n$, so

$$\eqalign{\psi_m(t)
&=\Bover1{2\pi(m+1)}\sum_{n=0}^m\Bover{\sin(n+{1\over
2}\pushbottom{3.5pt})t}{\sin{1\over 2}t}\cr
&\le\Bover1{2\pi(m+1)}\sum_{n=0}^m\,2n+1
=\Bover{m+1}{2\pi}.\cr}$$

\medskip

\quad{\bf (ii)} If $\delta\le|t|\le\pi$,

\Centerline{$\psi_m(t)\le\Bover1{\pi(m+1)(1-\cos t)}
\le\Bover1{\pi(m+1)(1-\cos\delta)}\to 0$}

\noindent as $m\to\infty$.

\medskip

\quad{\bf (iii)} also follows from the construction in (b), because

$$\int_I\psi_m
={\Bover1{2\pi(m+1)}}\sum_{n=0}^m
  \int_I\Bover{\sin(n+{1\over 2}\pushbottom{3.5pt})t}{\sin{1\over 2}t}dt
={\Bover1{m+1}}\sum_{n=0}^m\Bover12
=\Bover12$$

\noindent for both $I=[-\pi,0]$ and $I=[0,\pi]$, using (c).
}%end of proof of 282D

\cmmnt{\medskip

\noindent{\bf Remarks} For a discussion of substitution in integrals, if
you feel any need to justify the manipulations in part (a) of the proof,
see 263J.

The functions

\Centerline{$t\mapsto\Bover{\sin(n+{1\over
2}\pushbottom{3.5pt})t}{\sin{1\over 2}t}$,
\quad $t\mapsto \Bover{1-\cos(m+1)t}{(m+1)(1-\cos t)}$}

\noindent are called respectively the {\bf Dirichlet kernel} and the
{\bf Fej\'er kernel}.

I give the formulae in terms of $f(x-_{2\pi}t)$ in (a) and (b) in order
to provide a link with the work of 255O.
}

\leader{282E}{}\cmmnt{ The next step is a vital lemma, with a suitably
distinguished name which (you will be glad to know) reflects its
importance rather than its difficulty.

\medskip

\noindent}{\bf The Riemann-Lebesgue lemma} Let $f$ be a
complex-valued function which is integrable over $\Bbb R$.   Then

\Centerline{$\lim_{y\to\infty}\int f(x)e^{-iyx}dx
=\lim_{y\to-\infty}\int f(x)e^{-iyx}dx
=0$.}

\proof{{\bf (a)} Consider first the case in which $f=\chi\ooint{a,b}$,
where $a<b$.   Then

\Centerline{$|\int f(x)e^{-iyx}dx|
=|\int_a^{b}e^{-iyx}dx|
=|\Bover1{-iy}(e^{-iyb}-e^{-iya})|
\le\Bover2{|y|}$}

\noindent if $y\ne 0$.   So in this case the result is obvious.

\medskip

{\bf (b)} It follows at once that the result is true if $f$ is a
step-function with bounded support, that is, if there are $a_0\le
a_1\ldots\le a_n$ such that
$f$ is constant on every interval
$\ooint{a_{j-1},a_j}$ and zero outside $[a_0,a_n]$.

\medskip

{\bf (c)} Now, for a given integrable $f$ and $\epsilon>0$, there is
a step-function $g$ such that $\int|f-g|\le\epsilon$
(242Oa).   So

\Centerline{$|\int f(x)e^{-iyx}dx-\int g(x)e^{-iyx}dx|
\le\int|f(x)-g(x)|dx\le\epsilon$}

\noindent for every $y$, and

\Centerline{$\limsup_{y\to\infty}|\int f(x)e^{-iyx}dx|\le\epsilon$,}

\Centerline{$\limsup_{y\to-\infty}|\int f(x)e^{-iyx}dx|\le\epsilon$.}

\noindent As $\epsilon$ is arbitrary, we have the result.
}%end of proof of 282E

\leader{282F}{Corollary} (a) Let $f$ be a complex-valued
function which is integrable over $\ocint{-\pi,\pi}$, and
$\langle c_k\rangle_{k\in\Bbb Z}$
its sequence of Fourier coefficients.   Then
$\lim_{k\to\infty}c_k=\lim_{k\to-\infty}c_k=0$.

(b) Let $f$ be a complex-valued function which is integrable over
$\Bbb R$.   Then
\penalty -100
$\lim_{y\to\infty}\int f(x)\sin yx\,dx=0$.

\wheader{282F}{6}{2}{2}{36pt}

\proof{{\bf (a)} We need only identify

$$c_k=\Bover1{2\pi}\int_{-\pi}^{\pi}f(x)e^{-ikx}dx$$

\noindent with $\int g(x)e^{-ikx}dx$, where $g(x)=f(x)/2\pi$ for
$x\in\dom f$ and $0$ for $|x|>\pi$.

\medskip

{\bf (b)} This is just because

\Centerline{$\int f(x)\sin yx\,dx
=\Bover1{2i}(\int f(x)e^{iyx}dx-\int f(x)e^{-iyx}dx)$.}
}%end of proof of 282F

\leader{282G}{}\cmmnt{ We are now ready for theorems on the
convergence of Fej\'er sums.
I start with an easy one, almost a warming-up exercise.

\medskip

\noindent}{\bf Theorem} Let $f:\ocint{-\pi,\pi}\to\Bbb C$ be a
continuous function such that $\lim_{t\downarrow-\pi}f(t)=f(\pi)$.
Then its sequence $\sequence{m}{\sigma_m}$ of
Fej\'er sums converges uniformly to $f$ on $\ocint{-\pi,\pi}$.

\proof{   The conditions on $f$ amount just to saying that
its periodic extension $\tilde f$ is defined and continuous everywhere
on $\Bbb R$.   Consequently it is bounded and uniformly continuous on
any bounded interval, in particular, on the interval $[-2\pi,2\pi]$.
Set
$K=\sup_{|t|\le 2\pi}|\tilde f(t)|=\sup_{t\in\ocint{-\pi,\pi}}|f(t)|$.
Write

\Centerline{$\psi_m(t)=\Bover{1-\cos(m+1)t}{2\pi(m+1)(1-\cos t)}$}

\noindent for $m\in\Bbb N$,  $0<|t|\le\pi$, as in 282D.

Given $\epsilon>0$ we can find a $\delta\in\ocint{0,\pi}$ such that
$|\tilde f(x+t)-\tilde f(x)|\le\epsilon$ whenever $x\in[-\pi,\pi]$ and
$|t|\le\delta$.   Next, we can find an $m_0\in\Bbb N$ such that
$M_m\le\bover{\epsilon}{4\pi K}$ for every $m\ge m_0$, where
$M_m=\sup_{\delta\le|t|\le\pi}\psi_m(t)$ (282D(d-ii)).   Now suppose
that $m\ge m_0$ and $x\in\ocint{-\pi,\pi}$.   Set
$g(t)=\tilde f(x+t)-f(x)$ for $|t|\le\pi$.   Then $|g(t)|\le 2K$ for all
$t\in[-\pi,\pi]$ and $|g(t)|\le\epsilon$ if $|t|\le\delta$, so

$$\eqalign{\bigl|\int_{-\pi}^{\pi}g\times\psi_m\bigr|
&\le\int_{-\pi}^{-\delta}|g|\times\psi_m
  +\int_{-\delta}^{\delta}|g|\times\psi_m
  +\int_{\delta}^{\pi}|g|\times\psi_m\cr
&\le 2M_mK(\pi-\delta)
  +\epsilon\int_{-\delta}^{\delta}\psi_m
  +2M_mK(\pi-\delta)\cr
&\le 4\pi M_mK+\epsilon
\le 2\epsilon.\cr}$$

\noindent Consequently, using 282Db and 282D(d-iii),

$$|\sigma_m(x)-f(x)|
=|\int_{-\pi}^{\pi}(\tilde f(x+t)-f(x))\psi_m(t)dt|
\le 2\epsilon$$

\noindent for every $m\ge m_0$;  and this is true for every
$x\in\ocint{-\pi,\pi}$.   As $\epsilon$ is arbitrary,
$\sequence{m}{\sigma_m}$ converges to $f$ uniformly on
$\ocint{-\pi,\pi}$.
}%end of proof of 282G

\leader{282H}{}\cmmnt{ I come now to a theorem describing the
behaviour of the
Fej\'er sums of general functions $f$.   The hypothesis of the theorem
may take a little bit of digesting;  you can get an idea of its intended
scope by glancing at Corollary 282I.

\medskip

\noindent}{\bf Theorem} Let $f$ be a complex-valued function which is
integrable over $\ocint{-\pi,\pi}$, and $\sequence{m}{\sigma_m}$ its
sequence of Fej\'er sums.
Suppose that $x\in\ocint{-\pi,\pi}$ and $c\in\Bbb C$ are such that

$$\lim_{\delta\downarrow
0}\Bover1{\delta}\int_0^{\delta}|\tilde f(x+t)+\tilde f(x-t)-2c|dt=0,$$

\noindent writing $\tilde f$ for the periodic extension of $f$\cmmnt{, as
usual};  then $\lim_{m\to\infty}\sigma_m(x)=c$.

\proof{ Set $\phi(t)=|\tilde f(x+t)+\tilde f(x-t)-2c|$ when
this is defined, which is almost everywhere, and
$\Phi(t)=\int_0^t\phi$, which is defined for every $t\ge 0$,
because $\tilde f$ is integrable over $\ocint{-\pi,\pi}$ and therefore
over every bounded interval.

As in 282D, set

\Centerline{$\psi_m(t)=\Bover{1-\cos(m+1)t}{2\pi(m+1)(1-\cos t)}$}

\noindent for $m\in\Bbb N$, $0<|t|\le\pi$.    We have

$$|\sigma_m(x)-c|
=|\int_0^{\pi}(\tilde f(x+t)+\tilde f(x-t)-2c)\psi_m(t)dt|
\le\int_0^{\pi}\phi(t)\psi_m$$

\noindent by (b) and (d) of 282D.

Let $\epsilon>0$.   By hypothesis, $\lim_{t\downarrow 0}\Phi(t)/t=0$;
let $\delta\in\ocint{0,\pi}$ be such that $\Phi(t)\le\epsilon t$ for
every $t\in[0,\delta]$.   Take any $m\ge \pi/\delta$.   I break the
integral $\int_0^{\pi}\phi\times\psi_m$ up into three parts.

\medskip

{\bf (i)} For the integral from $0$ to $1/m$, we have

$$\int_0^{1/m}\phi\times\psi_m
\le\int_0^{1/m}\Bover{m+1}{2\pi}\phi
=\Bover{m+1}{2\pi}\Phi(\Bover1m)
\le\Bover{\epsilon(m+1)}{2\pi m}
\le\epsilon,$$

\noindent because $\psi_m(t)\le\bover{m+1}{2\pi}$ for every $t$
(282D(d-i)).

\medskip

{\bf (ii)} For the integral from $1/m$ to $\delta$, we have

$$\eqalignno{\int_{1/m}^{\delta}\phi\times\psi_m
&\le\bover1{2\pi(m+1)}\int_{1/m}^{\delta}\phi(t)\bover{1}{1-\cos t}dt
\le\bover{\pi}{4(m+1)}\int_{1/m}^{\delta}\bover{\phi(t)}{t^2}dt\cr
\noalign{\noindent (because $1-\cos t\ge\Bover{2t^2}{\pi^2}$ for
$|t|\le\pi$)}
&=\bover{\pi}{4(m+1)}\bigl(\bover{\Phi(\delta)}{\delta^2}
-\bover{\Phi({1\over m})}{({1\over m})^2}
+\int_{1/m}^{\delta}\bover{2\Phi(t)}{t^3}dt\bigr)\cr
\noalign{\noindent (integrating by parts -- see 225F)}
&\le\bover{\pi}{4(m+1)}\bigl(\bover{\epsilon}{\delta}
+\int_{1/m}^{\delta}\bover{2\epsilon}{t^2}dt\bigr)\cr
\noalign{\noindent (because $\Phi(t)\le\epsilon t$ for $0\le
t\le\delta$)}
&\le\bover{\pi}{4(m+1)}\bigl(\bover{\epsilon}{\delta}
   +2\epsilon m\bigr)
\le\bover{\pi\epsilon}{4(m+1)\delta}+\bover{\pi\epsilon}2
\le\bover{\epsilon}4+\bover{\pi\epsilon}2
\le 2\epsilon.\cr}$$

\medskip

{\bf (iii)} For the integral from $\delta$ to $\pi$, we have

$$\int_{\delta}^{\pi}\phi\times\psi_m
\le\int_{\delta}^{\pi}\Bover1{\pi(m+1)(1-\cos\delta)}\phi
\to 0\text{ as }m\to\infty$$

\noindent because $\phi$ is integrable over $[-\pi,\pi]$.   There must
therefore be an $m_0\in\Bbb N$ such that

$$\int_{\delta}^{\pi}\phi\times\psi_m\le\epsilon$$

\noindent for every $m\ge m_0$.

\medskip

Putting these together, we see that

$$\int_0^{\pi}\phi\times\psi_m
\le\epsilon+2\epsilon
+\epsilon= 4\epsilon$$

\noindent for every $m\ge \max(m_0,\bover{\pi}{\delta})$.   As
$\epsilon$ is arbitrary,
$\lim_{m\to\infty}\sigma_m(x)=c$, as claimed.
}%end of proof of 282H

\leader{282I}{Corollary} Let $f$ be a complex-valued
function which is integrable over $\ocint{-\pi,\pi}$, and
$\sequence{m}{\sigma_m}$ its
sequence of Fej\'er sums.

(a) $f(x)=\lim_{m\to\infty}\sigma_m(x)$ for almost every
$x\in\ocint{-\pi,\pi}$.

(b) $\lim_{m\to\infty}\int_{-\pi}^{\pi}|f-\sigma_m|=0$.

(c) If $g$ is another integrable function with the same Fourier
coefficients, then $f\eae g$.

(d) If $x\in\ooint{-\pi,\pi}$ is such that
$a=\lim_{t\in\dom f,t\uparrow
x}f(t)$ and $b=\lim_{t\in\dom f,t\downarrow x}f(t)$ are both defined in
$\Bbb C$, then

\Centerline{$\lim_{m\to\infty}\sigma_m(x)=\Bover12(a+b)$.}

(e) If $a=\lim_{t\in\dom f,t\uparrow \pi}f(t)$ and $b=\lim_{t\in\dom
f,t\downarrow -\pi}f(t)$ are both defined in $\Bbb C$, then

\Centerline{$\lim_{m\to\infty}\sigma_m(\pi)=\Bover12(a+b)$.}

(f) If $f$ is defined and continuous at $x\in\ooint{-\pi,\pi}$, then

\Centerline{$\lim_{m\to\infty}\sigma_m(x)=f(x)$.}

(g) If $\tilde f$, the periodic extension of $f$, is defined and
continuous at $\pi$, then

\Centerline{$\lim_{m\to\infty}\sigma_m(\pi)=f(\pi)$.}

\proof{{\bf (a)} We have only to recall that by 223D

$$\eqalign{\limsup_{\delta\downarrow 0}\Bover1{\delta}\int_0^{\delta}
&|f(x+t)+f(x-t)-2f(x)|dt\cr
&\le\limsup_{\delta\downarrow 0}
\Bover1{\delta}\bigl(\int_0^{\delta}|f(x+t)-f(x)|dt
+\int_0^{\delta}|f(x-t)-f(x)|dt\bigr)\cr
&=\limsup_{\delta\downarrow 0}
\Bover1{\delta}\int_{-\delta}^{\delta}|f(x+t)-f(x)|dt
=0\cr}$$

\noindent for almost every $x\in\ooint{-\pi,\pi}$.

\medskip

{\bf (b)} Next observe that, in the language of 255O,

\Centerline{$\sigma_m=f*\psi_m$,}

\noindent by the last formula in 282Db.   Consequently, by 255Od,

\Centerline{$\|\sigma_m\|_1\le\|f\|_1\|\psi_m\|_1$,}

\noindent writing $\|\sigma_m\|_1=\int_{-\pi}^{\pi}|\sigma_m|$.
But this means that we have

\Centerline{$f(x)
=\lim_{m\to\infty}\sigma_m(x)$ for almost every $x$,
\quad$\limsup_{m\to\infty}\|\sigma_m\|_1\le\|f\|_1$;}

\noindent and it follows from 245H that
$\lim_{m\to\infty}\|f-\sigma_m\|_1=0$.

\medskip

{\bf (c)} If $g$ has the same Fourier coefficients as $f$, then it has
the same Fourier and Fej\'er sums, so we have

\Centerline{$g(x)=\lim_{m\to\infty}\sigma_m(x)=f(x)$}

\noindent almost everywhere.

\medskip

{\bf (d)-(e)} Both of these amount to considering $x\in\ocint{-\pi,\pi}$
such that

\Centerline{$\lim_{t\in\dom\tilde f,t\uparrow x}\tilde f(t)=a$,\quad
$\lim_{t\in\dom\tilde f,t\downarrow x}\tilde f(t)=b$.}

\noindent Setting $c={1\over 2}(a+b)$,
$\phi(t)=|\tilde f(x+t)+\tilde f(x-t)-2c|$ whenever this is defined, we
have $\lim_{t\in\dom\phi,t\downarrow 0}\phi(t)=0$, so surely
$\lim_{\delta\downarrow 0}\bover1{\delta}\int_0^{\delta}\phi=0$, and the
theorem applies.

\medskip

{\bf (f)-(g)} are special cases of (d) and (e).
}%end of proof of 282I

\leader{282J}{}\cmmnt{ I now turn to conditions for the convergence of
Fourier sums.    Probably the easiest result -- one which
is both striking and satisfying -- is the following.

\medskip

\noindent}{\bf Theorem} Let $f$ be a complex-valued function which is
square-integrable over $\ocint{-\pi,\pi}$.   Let
$\langle c_k\rangle_{k\in\Bbb Z}$ be its Fourier coefficients and
$\sequencen{s_n}$ its Fourier sums\cmmnt{ (282A)}.   Then

(i) $\sum_{k=-\infty}^{\infty}|c_k|^2
=\Bover1{2\pi}\int_{-\pi}^{\pi}|f|^2$,

(ii) $\lim_{n\to\infty}\int_{-\pi}^{\pi}|f-s_n|^2=0$.

\proof{{\bf (a)} I recall some notation from 244N/244P.   Let
$\eusm L_{\Bbb C}^2$
be the space of square-integrable complex-valued functions on
$\ocint{-\pi,\pi}$.   For $g$, $h\in\eusm L_{\Bbb C}^2$, write

$$(g|h)=\int_{-\pi}^{\pi}g\times\bar h,\quad
\|g\|_2=\sqrt{(g|g)}.$$

\noindent Recall that $\|g+h\|_2\le\|g\|_2+\|h\|_2$ for all $g$,
$h\in\eusm L_{\Bbb C}^2$ (244Fb/244Pb).   For $k\in\Bbb Z$,
$x\in \ocint{-\pi,\pi}$ set $e_k(x)=e^{ikx}$, so that

$$(f|e_k)=\int_{-\pi}^{\pi}f(x)e^{-ikx}dx=2\pi c_k.$$

\noindent Moreover, if $|k|\le n$,

$$(s_n|e_k)=\sum_{j=-n}^nc_j\int_{-\pi}^{\pi}e^{ijx}e^{-ikx}dx
=2\pi c_k,$$

\noindent because

$$\eqalign{\int_{-\pi}^{\pi}e^{ijx}e^{-ikx}dx&=2\pi\text{ if }j=k,\cr
&=0\text{ if }j\ne k.\cr}$$

\noindent  So

\Centerline{$(f-s_n|e_k)=0$ whenever $|k|\le n$;}

\noindent in particular,

$$(f-s_n|s_n)=\sum_{k=-n}^n\bar c_k(f-s_n|e_k)=0$$

\noindent for every $n\in\Bbb N$.

\medskip

{\bf (b)} Fix $\epsilon>0$.   The next element of the proof is the fact
that there are $m\in\Bbb N$, $a_{-m},\ldots,a_m\in\Bbb C$ such that
$\|f-h\|_2\le\epsilon$, where $h=\sum_{k=-m}^ma_ke_k$.   \Prf\ By
244Hb/244Pb we know that there is a continuous function
$g:[-\pi,\pi]\to\Bbb C$ such that $\|f-g\|_2\le{{\epsilon}\over 3}$.
Next, modifying $g$ on a suitably short interval
$\ocint{\pi-\delta,\pi}$, we can find a continuous function
$g_1:[-\pi,\pi]\to\Bbb C$ such that $\|g-g_1\|_2\le{{\epsilon}\over 3}$
and $g_1(-\pi)=g_1(\pi)$.
(Set $M=\sup_{x\in[-\pi,\pi]}|g(x)|$, take $\delta\in\ocint{0,2\pi}$
such that $(2M)^2\delta\le(\epsilon/3)^2$, and set
$g_1(\pi-t\delta)=tg(\pi-\delta)+(1-t)g(-\pi)$ for $t\in[0,1]$.)
Either by
the Stone-Weierstrass theorem (281J), or by 282G above, there are
$a_{-m},\ldots,a_m$ such
that $|g_1(x)-\sum_{k=-m}^ma_ke^{ikx}|\le\Bover{\epsilon}{3\sqrt{2\pi}}$
for every $x\in[-\pi,\pi]$;  setting $h=\sum_{k=-m}^ma_ke_k$, we have
$\|g_1-h\|_2\le\bover13\epsilon$, so that

\Centerline{$\|f-h\|_2\le\|f-g\|_2+\|g-g_1\|_2+\|g_1-h\|_2
\le\epsilon$.   \Qed}

\medskip

{\bf (c)} Now take any $n\ge m$.   Then $s_n-h$ is a linear combination
of  $e_{-n},\ldots,e_n$, so $(f-s_n|s_n-h)=0$.   Consequently

$$\eqalign{\epsilon^2
&\ge(f-h|f-h)\cr
&=(f-s_n|f-s_n)+(f-s_n|s_n-h)+(s_n-h|f-s_n)+(s_n-h|s_n-h)\cr
&=\|f-s_n\|_2^2+\|s_n-h\|_2^2
\ge\|f-s_n\|_2^2.\cr}$$

\noindent Thus $\|f-s_n\|_2\le\epsilon$ for every $n\ge m$.   As
$\epsilon$ is arbitrary, $\lim_{n\to\infty}\|f-s_n\|_2^2=0$, which
proves (ii).

\medskip

{\bf (d)} As for (i), we have

$$\sum_{k=-n}^n|c_k|^2=\Bover1{2\pi}\sum_{k=-n}^n\bar c_k(s_n|e_k)
=\Bover1{2\pi}(s_n|s_n)=\Bover1{2\pi}\|s_n\|_2^2.$$

\noindent But of course

\Centerline{$\bigl|\|s_n\|_2-\|f\|_2\bigr|\le\|s_n-f\|_2\to 0$}

\noindent as $n\to\infty$, so

$$\sum_{k=-\infty}^{\infty}|c_k|^2
=\Bover1{2\pi}\lim_{n\to\infty}\|s_n\|_2^2=\Bover1{2\pi}\|f\|_2^2
=\Bover1{2\pi}\int_{-\pi}^{\pi}|f|^2,$$

\noindent as required.
}%end of proof of 282J

\leader{282K}{Corollary} Let $L_{\Bbb C}^2$ be the Hilbert space of
equivalence classes of square-integrable complex-valued functions on
$\ocint{-\pi,\pi}$, with the inner product

$$(f^{\ssbullet}|g^{\ssbullet})
=\int_{-\pi}^{\pi}f\times\bar g$$

\noindent and norm

$$\|f^{\ssbullet}\|_2
=\bigl(\int_{-\pi}^{\pi}|f|^2\bigr)^{1/2},$$

\noindent writing $f^{\ssbullet}\in L_{\Bbb C}^2$ for the equivalence
class of a
square-integrable function $f$.   Let $\ell_{\Bbb C}^2(\Bbb Z)$ be the
Hilbert space of
square-summable double-ended complex sequences, with the inner product

$$(\pmb{c}|\pmb{d})=\sum_{k=-\infty}^{\infty}c_k\bar d_k$$

\noindent and norm

$$\|\pmb{c}\|_2
=\bigl(\sum_{k=-\infty}^{\infty}|c_k|^2\bigr)^{1/2}$$

\noindent for $\pmb{c}=\langle c_k\rangle_{k\in\Bbb Z}$,
$\pmb{d}=\langle d_k\rangle_{k\in\Bbb Z}$ in $\ell_{\Bbb C}^2(\Bbb Z)$.
Then we have an inner-product-space isomorphism
$S:L_{\Bbb C}^2\to\ell_{\Bbb C}^2(\Bbb Z)$
defined by saying that

\Centerline{$S(f^{\ssbullet})(k)
=\Bover1{\sqrt{2\pi}}\int_{-\pi}^{\pi}f(x)e^{-ikx}dx$}

\noindent for every square-integrable function $f$ and every
$k\in\Bbb Z$.

\proof{{\bf (a)} As in 282J, write $\eusm L_{\Bbb C}^2$ for the space of
square-integrable functions.   If $f$, $g\in\eusm L_{\Bbb C}^2$ and
$f^{\ssbullet}=g^{\ssbullet}$, then $f\eae g$, so

\Centerline{$\Bover1{\sqrt{2\pi}}\int_{-\pi}^{\pi}f(x)e^{-ikx}dx
=\Bover1{\sqrt{2\pi}}\int_{-\pi}^{\pi}g(x)e^{-ikx}dx$}

\noindent for every $k\in\Bbb N$.    Thus $S$ is well-defined.

\medskip

{\bf (b)}  $S$ is linear.   \Prf\ This is elementary.   If $f$,
$g\in\eusm L_{\Bbb C}^2$ and $c\in\Bbb C$,

$$\eqalign{S(f^{\ssbullet}+g^{\ssbullet})(k)
&=\Bover1{\sqrt{2\pi}}\int_{-\pi}^{\pi}(f(x)+g(x))e^{-ikx}dx\cr
&=\Bover1{\sqrt{2\pi}}\int_{-\pi}^{\pi}f(x)e^{-ikx}dx
  +\Bover1{\sqrt{2\pi}}\int_{-\pi}^{\pi}g(x)e^{-ikx}dx\cr
&=S(f^{\ssbullet})(k)+S(g^{\ssbullet})(k)\cr}$$

\noindent for every $k\in\Bbb Z$, so that
$S(f^{\ssbullet}+g^{\ssbullet})=S(f^{\ssbullet})+S(g^{\ssbullet})$.
Similarly,

\Centerline{$S(cf^{\ssbullet})(k)
=\Bover1{\sqrt{2\pi}}\int_{-\pi}^{\pi}cf(x)e^{-ikx}dx
=\Bover{c}{\sqrt{2\pi}}\int_{-\pi}^{\pi}f(x)e^{-ikx}dx
=cS(f^{\ssbullet})(k)$}

\noindent for every $k\in\Bbb Z$, so that
$S(cf^{\ssbullet})=cS(f^{\ssbullet})$.   \Qed

\medskip

{\bf (c)} If $f\in\eusm L_{\Bbb C}^2$ has Fourier coefficients $c_k$,
then
$S(f^{\ssbullet})=\langle c_k\sqrt{ 2\pi}\rangle_{k\in\Bbb Z}$, so by 282J(i)

$$\|S(f^{\ssbullet})\|^2_2
=2\pi\sum_{k=-\infty}^{\infty}|c_k|^2
=\int_{-\pi}^{\pi}|f|^2
=\|f^{\ssbullet}\|_2^2.$$

\noindent Thus $Su\in\ell_{\Bbb C}^2(\Bbb Z)$ and $\|Su\|_2=\|u\|_2$ for
every $u\in L_{\Bbb C}^2$.   Because $S$ is linear and norm-preserving,
it is surely injective.

\medskip

{\bf (d)}  It now follows that $(Sv|Su)=(v|u)$ for every $u$, $v\in
L_{\Bbb C}^2$.   \Prf\ (This is of course a standard fact about Hilbert
spaces.)   We know that for any $t\in\Bbb R$

$$\eqalign{\|u\|_2^2+2\Real(e^{it}(v|u))+\|v\|_2^2
&=(u|u)+e^{it}(v|u)+e^{-it}(u|v)+(v|v)\cr
&=(u+e^{it}v|u+e^{it}v)\cr
&=\|u+e^{it}v\|_2^2
=\|S(u+e^{it}v)\|_2^2\cr
&=\|Su\|_2^2+2\Real(e^{it}(Sv|Su))+\|Sv\|_2^2\cr
&=\|u\|_2^2+2\Real(e^{it}(Sv|Su))+\|v\|_2^2,\cr}$$

\noindent so that $\Real(e^{it}(Sv|Su))=\Real(e^{it}(v|u))$.   As $t$ is
arbitrary, $(Sv|Su)=(v|u)$.   \Qed

\medskip

{\bf (e)}    Finally, $S$ is surjective.   \Prf\ Let
$\pmb{c}=\langle c_k\rangle_{k\in\Bbb Z}$ be any member of
$\ell_{\Bbb C}^2(\Bbb Z)$.   Set
$c^{(n)}_k=c_k$ if $|k|\le n$, $0$ otherwise, and
$\pmb{c}^{(n)}=\langle c^{(n)}_k\rangle_{k\in\Bbb N}$.   Consider

$$s_n=\sum_{k=-n}^nc_ke_k,\quad u_n=s_n^{\ssbullet}$$

\noindent where I write $e_k(x)=\bover1{\sqrt{2\pi}}e^{ikx}$ for
$x\in\ocint{-\pi,\pi}$.   Then $Su_n=\pmb{c}^{(n)}$, by the same
calculations as in part (a) of the proof of 282J.   Now

\Centerline{$\|\pmb{c}^{(n)}-\pmb{c}\|_2=\sqrt{\sumop_{|k|>n}|c_k|^2}
\to 0$}

\noindent as $n\to\infty$, so

\Centerline{$\|u_m-u_n\|_2=\|\pmb{c}^{(m)}-\pmb{c}^{(n)}\|_2
\to 0$}

\noindent as $m$, $n\to\infty$, and $\sequencen{u_n}$ is a Cauchy
sequence in $L_{\Bbb C}^2$.   Because $L_{\Bbb C}^2$ is complete
(244G/244Pb), $\sequencen{u_n}$ has a limit $u\in L_{\Bbb C}^2$, and now

\Centerline{$Su=\lim_{n\to\infty}Su_n=\lim_{n\to\infty}\pmb{c}^{(n)}
=\pmb{c}$.   \Qed}

Thus $S:L_{\Bbb C}^2\to\ell_{\Bbb C}^2(\Bbb Z)$ is an
inner-product-space isomorphism.
}%end of proof of 282K

\cmmnt{\medskip

\noindent{\bf Remark} In the language of Hilbert spaces, all that is
happening here is that $\langle e_k^{\ssbullet}\rangle_{k\in\Bbb Z}$ is
a `Hilbert space basis' or `complete orthonormal sequence' in
$L_{\Bbb C}^2$, which is matched by $S$ with the standard basis of
$\ell^2_{\Bbb C}(\Bbb Z)$.   The only step which calls on non-trivial
real analysis, as
opposed to the general theory of Hilbert spaces, is the check that the
linear subspace generated by $\{e_k^{\ssbullet}:k\in\Bbb Z\}$ is dense;
this is part (b) of the proof of 282J.

Observe that while $S:L^2\to\ell^2$ is readily described, its inverse is
more of a problem.   If $\pmb{c}\in\ell^2$, we should like to say that
$S^{-1}\pmb{c}$ is the equivalence class of $f$, where
$f(x)=\bover1{\sqrt{2\pi}}\sum_{k=-\infty}^{\infty}c_ke^{ikx}$ for every
$x$.   This works very well if $\{k:c_k\ne 0\}$ is finite, but for the
general case it is less clear how to interpret the sum.   It is in fact
the case that if $\pmb{c}\in\ell^2$ then

\Centerline{$g(x)
=\Bover1{\sqrt{2\pi}}\lim_{n\to\infty}\sum_{k=-n}^nc_ke^{ikx}$}

\noindent is defined for almost every $x\in\ocint{-\pi,\pi}$, and that
$S^{-1}\pmb{c}=g^{\ssbullet}$ in $L^2$;  this is, in effect,  Carleson's
theorem (286V).   A proof of Carleson's theorem is out of our reach for
the moment.   What is covered by the results of this section is that

\Centerline{$h(x)
=\Bover1{\sqrt{2\pi}}\lim_{m\to\infty}\Bover1{m+1}\sum_{n=0}^m
\sum_{k=-n}^nc_ke^{ikx}$}

\noindent is defined for almost every $x\in\ocint{-\pi,\pi}$, and that
$h^{\ssbullet}=S^{-1}\pmb{c}$.   (The point is that we know from the
result just proved that there is {\it some} square-integrable $f$ such
that $\pmb{c}$ is the sequence of Fourier coefficients of $f$;  now
282Ia declares that the Fej\'er sums of $f$ converge to $f$ almost
everywhere, that is, that $h\eae\bover1{\sqrt{2\pi}}f$.)
}%end of comment

\leader{282L}{}\cmmnt{ The next result is the easiest, and one of the
most useful, theorems concerning pointwise convergence of Fourier sums.

\medskip

\noindent}{\bf Theorem} Let $f$ be a complex-valued function which is
integrable over $\ocint{-\pi,\pi}$, and $\sequencen{s_n}$ its
sequence of Fourier sums.

(i) If $f$ is differentiable at $x\in\ooint{-\pi,\pi}$, then
$f(x)=\lim_{n\to\infty}s_n(x)$.

(ii) If the periodic extension $\tilde f$ of $f$ is differentiable at
$\pi$, then $f(\pi)=\lim_{n\to\infty}s_n(\pi)$.

\wheader{282L}{0}{0}{0}{60pt}

\proof{{\bf (a)} Take $x\in\ocint{-\pi,\pi}$ such that $\tilde
f$ is differentiable at $x$;  of course this covers both parts.    We
have

$$s_n(x)=\Bover1{2\pi}\int_{-\pi}^{\pi}
\Bover{\tilde f(x+t)}{\sin{1\over 2}t}\sin(n+\Bover12)t\,dt$$

\noindent for each $n$, by 282Da.

\medskip

{\bf (b)} Next,

$$\int_{-\pi}^{\pi}\Bover{\tilde f(x+t)-\tilde f(x)}{t}dt$$

\noindent exists in $\Bbb C$, because there is surely some
$\delta\in\ocint{0,\pi}$
such that $(\tilde f(x+t)-\tilde f(x))/t$ is bounded on
$\{t:0<|t|\le\delta\}$, while

$$\int_{-\pi}^{-\delta}
\Bover{\tilde f(x+t)-\tilde f(x)}{t}dt,
\quad\int_{\delta}^{\pi}\Bover{\tilde f(x+t)-\tilde f(x)}{t}dt$$

\noindent exist because $1/t$ is bounded on those intervals.   It
follows that

$$\int_{-\pi}^{\pi}
\Bover{\tilde f(x+t)-\tilde f(x)}{\sin{1\over 2}t}dt$$

\noindent exists, because $|t|\le\pi|\sin{1\over 2}t|$ if
$|t|\le\pi$.   So by the Riemann-Lebesgue lemma (282Fb),

$$\lim_{n\to\infty}\int_{-\pi}^{\pi}
\Bover{\tilde f(x+t)-\tilde f(x)}{\sin{1\over 2}t}\sin(n+\Bover12)t\,dt
=0.$$

\medskip

{\bf (c)} Because

$$\Bover1{2\pi}\int_{-\pi}^{\pi}\tilde f(x)
  \Bover{\sin(n+{1\over 2}\pushbottom{3.5pt})t}{\sin{1\over 2}t}dt
=\tilde f(x)$$

\noindent for every $n$ (282Dc),

$$s_n(x)
=\tilde f(x)+\Bover{1}{2\pi}\int_{-\pi}^{\pi}
  \Bover{\tilde f(x+t)-\tilde f(x)}{\sin{1\over 2}t}
  \sin(n+\Bover12)t\,dt
\to\tilde f(x)$$

\noindent as $n\to\infty$, as required.
}%end of proof of 282L

\leader{282M}{Lemma} Suppose that $f$ is a complex-valued
function, defined almost everywhere and of bounded variation on
$\ocint{-\pi,\pi}$.   Then $\sup_{k\in\Bbb Z}|kc_k|<\infty$, where $c_k$
is the $k$th Fourier coefficient of $f$, as in 282A.

\proof{ Set

\Centerline{$M
=\lim_{x\in\dom f,x\uparrow \pi}|f(x)|+\Var_{\ooint{-\pi,\pi}}(f)$.}

\noindent By 224J,

$$\eqalign{|kc_k|
&=\Bover{1}{2\pi}\bigl|\int_{-\pi}^{\pi}kf(t)e^{-ikt}dt\bigr|
\le\Bover1{2\pi}M
   \sup_{c\in[-\pi,\pi]}\bigl|\int_{-\pi}^cke^{-ikt}dt\bigr|\cr
&=\Bover{M}{2\pi}\sup_{c\in[-\pi,\pi]}|e^{-ikc}-e^{ik\pi}|
\le\Bover{M}{\pi}\cr}$$

\noindent for every $k$.
}%end of proof of 282M

\leader{282N}{}\cmmnt{ I give another lemma, extracting the technical
part of
the proof of the next theorem.   (Its most natural application is in
282Xn.)

\medskip

\noindent}{\bf Lemma} Let $\sequence{k}{d_k}$ be a complex sequence, and
set $t_n=\sum_{k=0}^nd_k$, $\tau_m=\bover1{m+1}\sum_{n=0}^mt_n$ for $n$,
$m\in\Bbb N$.   Suppose that $\sup_{k\in\Bbb N}|kd_k|=M<\infty$.   Then
for any $j\ge 1$ and any $c\in\Bbb C$,

\Centerline{$|t_n-c|\le\Bover{M}{j}+(2j+3)\sup_{m\ge n-n/j}|\tau_m-c|$}

\noindent for every $n\ge j^2$.

\proof{{\bf (a)} The first point to note is that for any $n$,
$n'\in\Bbb N$,

\Centerline{$|t_n-t_{n'}|\le\Bover{M|n-n'|}{1+\min(n,n')}$.}

\noindent\Prf\ If $n=n'$ this is trivial.   Suppose that $n'<n$.   Then

$$|t_n-t_{n'}|
=|\sum_{k=n'+1}^{n}d_k|
\le\sum_{k=n'+1}^n\Bover{M}{k}
\le\Bover{M(n-n')}{n'+1}
=\Bover{M|n-n'|}{1+\min(n',n)}.$$

\noindent Of course the case $n<n'$ is identical.   \Qed


\medskip

{\bf (b)} Now take any $n\ge j^2$.
Set $\eta=\sup_{m\ge n-n/j}|\tau_m-c|$.   Let $m\ge j$ be such that
$jm\le n<j(m+1)$;  then $n<jm+m$;  also

\Centerline{$n(1-\bover1{j})\le m(j+1)(1-{1\over j})\le mj$.}

\noindent    Set

$$\tau^*=\Bover1m\sum_{n'=jm+1}^{jm+m}t_{n'}
=\Bover{jm+m+1}{m}\tau_{jm+m}-\Bover{jm+1}{m}\tau_{jm}.$$

\noindent Then

$$\eqalign{|\tau^*-c|
&=|\Bover{jm+m+1}{m}\tau_{jm+m}-\Bover{jm+1}{m}\tau_{jm}-c|\cr
&=|\Bover{jm+m+1}{m}(\tau_{jm+m}-c)
   -\Bover{jm+1}{m}(\tau_{jm}-c)|\cr
&\le\Bover{jm+m+1}{m}\eta+\Bover{jm+1}{m}\eta
\le(2j+3)\eta.\cr}$$

\noindent On the other hand,

$$\eqalign{|\tau^*-t_n|
&=\bigl|\Bover1m\sum_{n'=jm+1}^{jm+m}(t_{n'}-t_n)\bigr|
\le\Bover1m\sum_{n'=jm+1}^{jm+m}\Bover{M|n-n'|}{1+\min(n,n')}\cr
&\le\Bover1m\sum_{n'=jm+1}^{jm+m}\Bover{Mm}{1+jm}
=\Bover{Mm}{1+jm}
\le\Bover{M}{j}.\cr}$$

\noindent Putting these together, we have

\Centerline{$|t_n-c|\le|t_n-\tau^*|+|\tau^*-c|
\le\Bover{M}{j}+(2j+3)\eta
=\Bover{M}{j}+(2j+3)\sup_{m\ge n-n/j}|\tau_m-c|$,}

\noindent as required.
}%end of proof of 282N

\leader{282O}{Theorem} Let $f$ be a complex-valued function of bounded
variation, defined almost everywhere in $\ocint{-\pi,\pi}$, and let
$\sequencen{s_n}$ be its sequence of Fourier sums.

(i) If $x\in\ooint{-\pi,\pi}$, then

\Centerline{$\lim_{n\to\infty}s_n(x)
=\Bover12(\lim_{t\in\dom f,t\uparrow x}f(t)
  +\lim_{t\in\dom f,t\downarrow x}f(t))$.}

(ii) $\lim_{n\to\infty}s_n(\pi)
=\Bover12(\lim_{t\in\dom f,t\uparrow\pi}f(t)
  +\lim_{t\in\dom f,t\downarrow-\pi}f(t))$.

(iii) If $f$ is defined throughout $\ocint{-\pi,\pi}$, is continuous,
and $\lim_{t\downarrow -\pi}f(t)=f(\pi)$, then $s_n(x)\to f(x)$
uniformly on $\ocint{-\pi,\pi}$.

\proof{{\bf (a)} Note first that 224F shows that the limits
$\lim_{t\in\dom f,t\downarrow x}f(t)$,
$\lim_{t\in\dom f,t\uparrow x}f(t)$ required in the formulae above
always exist.   We know also from 282M that $M=
\penalty-100
\sup_{k\in\Bbb Z}|kc_k|<\infty$, where $c_k$ is the $k$th
Fourier coefficient of $f$.

Take any $x\in\ocint{-\pi,\pi}$, and set

\Centerline{$c=\bover12(\lim_{t\in\dom f,t\uparrow x}\tilde
f(t)+\lim_{t\in\dom\tilde f,t\downarrow x}\tilde f(t))$,}

\noindent writing $\tilde f$ for the periodic extension of $f$, as
usual.   We know from 282Id-282Ie that
$c=\lim_{m\to\infty}\sigma_m(x)$, writing $\sigma_m$ for the Fej\'er
sums of $f$.   Let $\epsilon>0$.   Take any $j\ge\max(2,2M/\epsilon)$, and
$m_0\ge 1$ such that
$|\sigma_m(x)-c|\le\epsilon/(2j+3)$ for every $m\ge m_0$.

Now if $n\ge\max(j^2,2m_0)$, apply Lemma 282N with

\Centerline{$d_0=c_0$,
\quad$d_k=c_ke^{ikx}+c_{-k}e^{-ikx}$ for $k\ge 1$,}

\noindent so that $t_n=s_n(x)$, $\tau_m=\sigma_m(x)$ and $|kd_k|\le 2M$
for every $k$, $n$, $m\in\Bbb N$.   We have $n-n/j\ge{1\over 2}n\ge
m_0$, so

\Centerline{$\eta=\sup_{m\ge n-n/j}|\tau_m-c|
\le\sup_{m\ge m_0}|\tau_m-c|\le\Bover{\epsilon}{2j+3}$.}

\noindent So 282N tells us that

\Centerline{$|s_n(x)-c|=|t_n-c|\le
\Bover{2M}{j}+(2j+3)\sup_{m\ge n-n/j}|\tau_m-c|\le\epsilon+(2j+3)\eta
\le 2\epsilon$.}

\noindent As $\epsilon$ is arbitrary, $\lim_{n\to\infty}s_n(x)=c$, as
required.

\medskip

{\bf (b)} This proves (i) and (ii) of this theorem.
Finally, for (iii), observe that under these conditions $\sigma_m(x)\to
f(x)$ uniformly as $m\to\infty$, by 282G.   So given $\epsilon>0$
we choose $j\ge \max(2,2M/\epsilon)$ and $m_0\in\Bbb N$ such
that $|\sigma_m(x)-f(x)|\le\epsilon/(2j+3)$ whenever $m\ge m_0$ and
$x\in\ocint{-\pi,\pi}$.   By the same calculation as before,

\Centerline{$|s_n(x)-f(x)|\le 2\epsilon$}

\noindent for every $n\ge\max(j^2,2m_0)$ and every
$x\in\ocint{-\pi,\pi}$.   As $\epsilon$ is arbitrary,
$\lim_{n\to\infty}s_n(x)=f(x)$ uniformly for $x\in\ocint{-\pi,\pi}$.
}%end of proof of 282O

\leader{282P}{Corollary} Let $f$ be a complex-valued
function which is integrable over $\ocint{-\pi,\pi}$, and
$\sequencen{s_n}$ its sequence of
Fourier sums.

(i) Suppose that $x\in\ooint{-\pi,\pi}$ is such that $f$ is of bounded
variation on some neighbourhood of $x$.   Then

\Centerline{$\lim_{n\to\infty}s_n(x)=\Bover12(\lim_{t\in\dom f,t\uparrow
x}f(t)+\lim_{t\in\dom f,t\downarrow x}f(t))$.}

(ii) If there is a $\delta>0$ such that $f$ is of bounded variation on
both $\ocint{-\pi,-\pi+\delta}$ and $[\pi-\delta,\pi]$, then

\Centerline{$\lim_{n\to\infty}s_n(\pi)
=\Bover12(\lim_{t\in\dom f,t\uparrow \pi}f(t)
  +\lim_{t\in\dom f,t\downarrow-\pi}f(t))$.}

\proof{ In case (i), take $\delta>0$ such that $f$ is of
bounded variation on $[x-\delta,x+\delta]$ and set $f_1(t)=f(t)$ if
$x\in\dom f\cap[x-\delta,x+\delta]$, $0$ for other
$t\in\ocint{-\pi,\pi}$;   in case (ii), set $f_1(t)=f(t)$ if $t\in\dom
f$ and $|t|\ge\pi-\delta$, $0$ for other $t\in\ocint{-\pi,\pi}$, and say
that $x=\pi$.   In either case, $f_1$ is of bounded variation, so by
282O the Fourier sums $\sequencen{s'_n}$ of $f_1$ converge at
$x$ to the value given by the formulae above.   But now observe that,
writing $\tilde f$ and $\tilde f_1$ for the periodic extensions of $f$
and $f_1$, $\tilde f-\tilde f_1=0$ on a neighbourhood of $x$, so

$$\int_{-\pi}^{\pi}
  \Bover{\tilde f(x+t)-\tilde f_1(x+t)}{\sin{1\over 2}t}dt$$

\noindent exists in $\Bbb C$, and by 282Fb

$$\lim_{n\to\infty}\int_{-\pi}^{\pi}
 \bover{\tilde f(x+t)-\tilde f_1(x+t)}{\sin{1\over 2}t}
\sin(n+\Bover12)t\,dt=0,$$

\noindent that is, $\lim_{n\to\infty}s_n(x)-s'_n(x)=0$.   So
$\sequencen{s_n}$ also converges to the right limit.
}%end of proof of 282P

\leader{282Q}{}\cmmnt{ I cannot leave this section without mentioning
one of the most important facts about Fourier series, even though I have
no space here to discuss its consequences.

\medskip

\noindent}{\bf Theorem} Let $f$ and $g$ be complex-valued
functions which are integrable over $\ocint{-\pi,\pi}$, and
$\langle c_k\rangle_{k\in\Bbb N}$,
$\langle d_k\rangle_{k\in\Bbb N}$ their Fourier coefficients.   Let
$f*g$ be their convolution, defined by the formula

$$(f*g)(x)=\int_{-\pi}^{\pi}f(x-_{2\pi}t)g(t)dt
=\int_{-\pi}^{\pi}\tilde f(x-t)g(t)dt,$$

\noindent\cmmnt{as in 255O,} writing $\tilde f$ for the periodic
extension of $f$.   Then the Fourier coefficients of $f*g$
are $\langle 2\pi c_kd_k\rangle_{k\in\Bbb Z}$.

\proof{ By 255O(c-i),

$$\eqalign{\Bover1{2\pi}\int_{-\pi}^{\pi}(f*g)(x)e^{-ikx}dx
&=\Bover1{2\pi}\int_{-\pi}^{\pi}\int_{-\pi}^{\pi}
e^{-ik(t+u)}f(t)g(u)dtdu\cr
&=\Bover1{2\pi}\int_{-\pi}^{\pi}e^{-ikt}f(t)dt\int_{-\pi}^{\pi}
e^{-iku}g(u)du
=2\pi c_kd_k.\cr}$$
}%end of proof of 282Q

\leader{*282R}{}\cmmnt{ In my hurry to get to the theorems on
convergence of Fej\'er and Fourier sums, I have rather neglected the
elementary manipulations which are essential when applying the theory.
One basic result is the following.

\medskip

\noindent}{\bf Proposition} (a) Let $f:[-\pi,\pi]\to\Bbb C$ be an
absolutely continuous function such that $f(-\pi)=f(\pi)$, and
$\family{k}{\Bbb Z}{c_k}$ its sequence of Fourier coefficients.   Then
the Fourier coefficients of $f'$ are $\family{k}{\Bbb Z}{ikc_k}$.

(b)\discrversionA{\footnote{Corrected and refined 2009.}}{}
Let $f:\Bbb R\to\Bbb C$ be a differentiable function such that $f'$
is absolutely continuous on $[-\pi,\pi]$, and $f(\pi)=f(-\pi)$.
If $\family{k}{\Bbb Z}{c_k}$ are the Fourier coefficients of
$f\restr\ocint{-\pi,\pi}$, then
$\sum_{k=-\infty}^{\infty}|c_k|$ is finite.

\proof{{\bf (a)} By 225Cb, $f'$ is integrable over $[-\pi,\pi]$;  by
225E, $f$ is an indefinite integral of $f'$.   So 225F tells us that

\Centerline{$\int_{-\pi}^{\pi}f'(x)e^{-ikx}dx
=f(\pi)e^{-ik\pi}-f(-\pi)e^{ik\pi}+ik\int_{-\pi}^{\pi}f(x)e^{-ikx}dx
=ikc_k$}

\noindent for every $k\in\Bbb Z$.

\medskip

{\bf (b)(i)} Suppose first that $f'(\pi)=f'(-\pi)$.
By (a), applied twice, the Fourier coefficients of $f''$ are
$\family{k}{\Bbb Z}{-k^2c_k}$, so $\sup_{k\in\Bbb Z}k^2|c_k|$ is finite;
because $\sum_{k=1}^{\infty}\Bover1{k^2}<\infty$,
$\sum_{k=-\infty}^{\infty}|c_k|<\infty$.

\medskip

\quad{\bf (ii)} Next, suppose that $f(x)=x^2$ for every $x$.   Then, for
$k\ne 0$,

$$\eqalign{c_k
&=\Bover1{2\pi}\int x^2e^{-ikx}dx
=\Bover1{2\pi}\bigl(-\Bover1{ik}(\pi^2e^{-ik\pi}-\pi^2e^{ik\pi})
   +\int_{-\pi}^{\pi}\Bover{2x}{ik}e^{-ikx}dx\bigr)\cr
&=\Bover1{ik\pi}\bigl(-\Bover1{ik}(\pi e^{-ik\pi}+\pi e^{ik\pi})
   +\Bover1{ik}\int_{-\pi}^{\pi}e^{ikx}dx\bigr)
=\Bover{2}{k^2}(-1)^k,\cr}$$

\noindent so
$\sum_{k\in\Bbb Z}|c_k|\le|c_0|+4\sum_{k=1}^{\infty}\Bover1{k^2}$
is finite.

\medskip

\quad{\bf (iii)} In general, we can express $f$ as $f_1+cf_2$ where
$f_2(x)=x^2$ for every $x$, $c=\Bover1{4\pi}(f'(\pi)-f'(-\pi))$, and
$f_1$ satisfies the conditions of (i);  so that
$\family{k}{\Bbb Z}{c_k}$ is the sum of two summable sequences and is
itself summable.
}%end of proof of 282R

\exercises{
\leader{282X}{Basic exercises $\pmb{>}$(a)}
%\spheader 282Xa
Suppose that $\langle c_k\rangle_{k\in\Bbb N}$
is an absolutely summable double-ended sequence of complex numbers.
Show that $f(x)=\sum_{k=-\infty}^{\infty}c_ke^{ikx}$ exists for every
$x\in\Bbb R$, that $f$ is continuous and periodic, and that its Fourier
coefficients are the $c_k$.
%282A

\spheader 282Xc Set $\phi_n(t)=\bover2t\sin(n+\bover12t)$ for $t\ne 0$.
(This is sometimes called the {\bf modified Dirichlet kernel}.)
Show that for any integrable function $f$ on $\ocint{-\pi,\pi}$, with
Fourier sums $\sequencen{s_n}$ and periodic extension $\tilde f$,

\Centerline{$\lim_{n\to\infty}|s_n(x)-\bover1{2\pi}
  \int_{-\pi}^{\pi}\phi_n(t)\tilde f(x+t)dt|
=0$}

\noindent for every $x\in\ocint{-\pi,\pi}$.   \Hint{show that
$\bover2t-\bover{1}{\sin{1\over 2}t}$ is bounded, and use 282E.}
%282E

\spheader 282Xd Give a proof of 282Ib from 242O, 255O and 282G.
%282I

\spheader 282Xe Give another proof of 282Ic, based on 222D, 281J and an
idea in the proof of 242O instead of on 282H.
%282I

\spheader 282Xf Use the idea of 255Ya to shorten one of the steps in the
proof of 282H, taking
\discrcenter{468pt}{$g_m(t)
=\min(\bover{m+1}{2\pi},\bover{\pi}{4(m+1)t^2})$ }for $|t|\le\delta$, so
that $g_m\ge\psi_m$ on $[-\delta,\delta]$.
%282H

\sqheader 282Xg(i) Let $f$ be a real square-integrable function
on $\ocint{-\pi,\pi}$, and $\sequence{k}{a_k}$,
$\langle b_k\rangle_{k\ge 1}$ its real Fourier coefficients (282Ba).
Show that
$\bover12a_0^2+\sum_{k=1}^{\infty}(a_k^2+b_k^2)
=\bover1{\pi}\int_{-\pi}^{\pi}|f|^2$.  (ii) Show that
$f\mapsto(\sqrt{\bover{\pi}2}a_0,\sqrt{\pi}a_1,
\ifdim\pagewidth>467pt\penalty-100\fi
\sqrt{\pi}b_1,\ldots)$
defines an inner-product-space isomorphism between the real Hilbert
space $L^2_{\Bbb R}$ of equivalence classes of real square-integrable
functions on $\ocint{-\pi,\pi}$ and the real Hilbert space
$\ell^2_{\Bbb R}$ of square-summable sequences.
%282K

\spheader 282Xh Show that
$\bover{\pi}4=1-\bover13+\bover15-\bover17+\ldots$.   \Hint{find the
Fourier series of $f$ where $f(x)=x/|x|$, and compute the sum of the
series at $\bover{\pi}2$.   Of course there are other methods, e.g.,
examining the Taylor series for $\arctan\bover{\pi}4$.}
%282L

\spheader 282Xi Let $f$ be an integrable complex-valued function
on $\ocint{-\pi,\pi}$, and $\sequencen{s_n}$ its sequence of Fourier
sums.   Suppose that $x\in\ooint{-\pi,\pi}$, $a\in\Bbb C$ are such that
$\biggerint_{-\pi}^{\pi}\Bover{f(t)-a}{t-x}dt$ exists and is finite.
Show that $\lim_{n\to\infty}s_n(x)=a$.
Explain how this generalizes 282L.   What modification is appropriate to
obtain a limit $\lim_{n\to\infty}s_n(\pi)$?
%282L

\spheader 282Xj Suppose that $\alpha>0$, $K\ge 0$ and
$f:\ooint{-\pi,\pi}\to\Bbb C$ are such that
$|f(x)-f(y)|\le K|x-y|^{\alpha}$ for all $x$, $y\in\ooint{-\pi,\pi}$.
(Such functions are called {\bf H\"older continuous}.)
Show that the Fourier sums of $f$ converge to $f$ everywhere in
$\ooint{-\pi,\pi}$.   \Hint{use 282Xi.}  (Compare 282Yb.)
%282Xi 282L

\spheader 282Xk In 282L, show that it is enough if $\tilde f$ is
differentiable with respect to its domain at $x$ or $\pi$ (see 262Fb),
rather than differentiable in the strict sense.
%282L

\spheader 282Xl Show that $\lim_{a\to\infty}\int_0^a\bover{\sin t}tdt$
exists and is finite.   \Hint{use 224J to estimate
$\int_a^b\bover{\sin t}tdt$ for $0<a\le b$.}
%

\spheader 282Xm Show that
$\int_0^{\infty}\bover{|\sin t|}tdt=\infty$.   \Hint{show that
$\sup_{a\ge 0}|\int_1^a\bover{\cos 2t}tdt|<\infty$, and therefore that
$\sup_{a\ge 0}\int_1^a\bover{\sin^2t}{t}dt\penalty-100=\infty$.}
%282Xl

\sqheader 282Xn Let $\sequence{k}{d_k}$ be a sequence in $\Bbb C$ such
that $\sup_{k\in\Bbb N}|kd_k|<\infty$ and

\Centerline{$\lim_{m\to\infty}\Bover1{m+1}\sum_{n=0}^m\sum_{k=0}^nd_k
=c\in\Bbb C$.}

\noindent Show that $c=\sum_{k=0}^{\infty}d_k$.   \Hint{282N.}
%282N

\sqheader 282Xo Show that
$\sum_{n=1}^{\infty}\bover1{n^2}=\bover{\pi^2}6$.   \Hint{(b-ii) of the
proof of 282R.}
%282Q

\spheader 282Xp Let $f$ be an integrable
complex-valued function on $\ocint{-\pi,\pi}$, and $\sequencen{s_n}$ its
sequence of Fourier sums.    Suppose that $x\in\ooint{-\pi,\pi}$ is such
that

(i) there is an $a\in\Bbb C$ such that

\inset{{\it either}
$\int_{-\pi}^x\Bover{a-f(t)}{x-t}dt$ exists in $\Bbb C$}

\inset{{\it or} there is some $\delta>0$ such that $f$ is of bounded
variation on $[x-\delta,x]$, and $a=\lim_{t\in\dom f,t\uparrow x}f(t)$}

(ii) there is a $b\in\Bbb C$ such that

\inset{{\it either}
$\int_x^{\pi}\Bover{f(t)-b}{t-x}dt$ exists in $\Bbb C$}

\inset{{\it or} there is some $\delta>0$ such that $f$ is of bounded
variation on $[x,x+\delta]$, and $b=\lim_{t\in\dom f,t\downarrow
x}f(t)$.}

\noindent Show that $\lim_{n\to\infty}s_n(x)={1\over 2}(a+b)$.
What modification is appropriate to obtain a limit
$\lim_{n\to\infty}s_n(\pi)$?
%282Xi, 282O

\sqheader 282Xq Let $f$, $g$ be integrable complex-valued
functions on $\ocint{-\pi,\pi}$, and
$\pmb{c}=\langle c_k\rangle_{k\in\Bbb Z}$,
$\pmb{d}=\langle d_k\rangle_{k\in\Bbb Z}$
their sequences of Fourier coefficients.   Suppose that {\it either}
$\sum_{k=-\infty}^{\infty}|c_k|<\infty$ {\it or}
$\sum_{k=-\infty}^{\infty}|c_k|^2+|d_k|^2<\infty$.   Show that the
sequence of
Fourier coefficients of $f\times g$ is just the convolution
$\pmb{c}*\pmb{d}$ of $\pmb{c}$ and $\pmb{d}$ (255Xk).
%282Q

\spheader 282Xr In 282Ra, what happens if $f(\pi)\ne f(-\pi)$?
%282R

\spheader 282Xs  Suppose that $\langle c_k\rangle_{k\in\Bbb N}$
is a double-ended sequence of complex numbers such that
$\sum_{k=-\infty}^{\infty}|kc_k|<\infty$.   Show that
$f(x)=\sum_{k=-\infty}^{\infty}c_ke^{ikx}$ exists for every
$x\in\Bbb R$ and that $f$ is differentiable everywhere.
%282R

\spheader 282Xt Let $\langle c_k\rangle_{k\in\Bbb Z}$ be a
double-ended sequence of complex numbers such that
$\sup_{k\in\Bbb Z}|kc_k|<\infty$.   Show that there is a
square-integrable function $f$
on $\ocint{-\pi,\pi}$ such that the $c_k$ are the
Fourier coefficients of $f$, that $f$ is the limit almost everywhere of
its Fourier sums, and that $f*f*f$ is differentiable.   \Hint{use
282K to show that there is an $f$, and 282Xn to show that its Fourier
sums converge wherever its Fej\'er sums do;  use 282Q and 282Xs to
show that $f*f*f$ is differentiable.}
%282Xs, 282Q

\leader{282Y}{Further exercises (a)}
%\spheader 282Ya
Let $f$ be a non-negative integrable function on
$\ocint{-\pi,\pi}$, with Fourier coefficients
$\langle c_k\rangle_{k\in\Bbb Z}$.   Show that

\Centerline{$\sum_{j=0}^n\sum_{k=0}^na_j\bar a_kc_{j-k}\ge 0$}

\noindent for all complex numbers $a_0,\ldots,a_n$.   (See also 285Xu
below.)
%282A

\spheader 282Yb Let $f:\ocint{-\pi,\pi}\to\Bbb C$, $K\ge 0$, $\alpha>0$
be such that $|f(x)-f(y)|\le K|x-y|^{\alpha}$ for all $x$,
$y\in\ocint{-\pi,\pi}$.   Let $c_k$, $s_n$ be the Fourier coefficients
and sums of $f$.   (i) Show that
$\sup_{k\in\Bbb Z}|k|^{\alpha}|c_k|<\infty$.   \Hint{show that
$c_k=\bover1{4\pi}\int_{-\pi}^{\pi}(f(x)-\tilde f(x+\bover{\pi}k))
e^{-ikx}dx$.}   (ii) Show that if $f(\pi)=\lim_{x\downarrow-\pi}f(x)$
then $s_n\to f$ uniformly.   (Compare 282Xj.)
%282E

\spheader 282Yc Let $f$ be a measurable
complex-valued function on $\ocint{-\pi,\pi}$, and suppose that
$p\in\coint{1,\infty}$ is such that
$\int_{-\pi}^{\pi}|f|^p<\infty$.   Let
$\sequence{m}{\sigma_m}$ be the sequence of Fej\'er sums of $f$.   Show
that $\lim_{m\to\infty}\int_{-\pi}^{\pi}|f-\sigma_m|^p=0$.
\Hint{use 245Xl, 255Yk and the ideas in 282Ib.}
%282I

\spheader 282Yd Construct a continuous function
$h:[-\pi,\pi]\to\Bbb R$ such that $h(\pi)=h(-\pi)$ but the Fourier sums
of $h$ are unbounded at $0$, as follows.   Set
$\alpha(m,n)=\int_0^{\pi}
\bover{\sin(m+{1\over2})t\sin(n+{1\over 2})t}{\sin{1\over 2}t}dt$.
Show that $\lim_{n\to\infty}\alpha(m,n)=0$
for every $m$, but $\lim_{n\to\infty}\alpha(n,n)=\infty$.   Set
$h_0(x)=\sum_{k=0}^{\infty}\delta_k\sin(m_k+{1\over 2})x$ for
$0\le x\le\pi$, $0$ for $-\pi\le x\le 0$, where
$\delta_k>0$, $m_k\in\Bbb N$ are such that
($\alpha$) $\delta_k\le 2^{-k}$, $\delta_k|\alpha(m_k,m_n)|\le 2^{-k}$
for every $n<k$ (choosing $\delta_k$) ($\beta$)
$\delta_k\alpha(m_k,m_k)\ge k$, $\delta_n|\alpha(m_k,m_n)|\le 2^{-n}$
for every $n<k$ (choosing
$m_k$).   Now modify $h_0$ on $\coint{-\pi,0}$ by adding a function of
bounded variation.
%282*

\spheader 282Ye(i) Show that $\lim_{n\to\infty}\int_{-\pi}^{\pi}
|\bover{\sin(n+{1\over2})t}{\sin{1\over 2}t}|dt=\infty$.  \Hint{282Xm.}
(ii) Show that for any $\delta>0$ there are $n\in\Bbb N$, $f\ge 0$ such
that $\int_{-\pi}^{\pi}f\le\delta$, $\int_{-\pi}^{\pi}|s_n|\ge 1$, where
$s_n$ is the $n$th Fourier sum of $f$.   \Hint{take $n$ such that
$\Bover1{2\pi}\int_{-\pi}^{\pi}
|\bover{\sin(n+{1\over2})t}{\sin{1\over 2}t}|dt>\Bover1{\delta}$ and set
$f(x)=\Bover{\delta}{\eta}$ for $0\le x\le\eta$, $0$ otherwise, where
$\eta$ is small.}   (iii) Show that there is an integrable function
$f:\ocint{-\pi,\pi}\to\Bbb R$ such that $\sup_{n\in\Bbb N}\|s_n\|_1$ is
infinite, where $\sequencen{s_n}$ is the sequence of Fourier sums of
$f$.   \Hint{it helps to know the `Uniform Boundedness Theorem' of
functional analysis, but $f$ can also be constructed bare-handed by the
method of 282Yd.}
%282*

\spheader 282Yf Let $u:[-\pi,\pi]\to\Bbb R$ be an absolutely continuous
function such that $u(\pi)=u(-\pi)$ and $\int_{-\pi}^{\pi}u=0$.   Show
that $\|u\|_2\le\|u'\|_2$.   (This is {\bf Wirtinger's inequality}.)
%282J

\spheader 282Yg For $0\le r<1$, $t\in\Bbb R$ set
$A_r(t)=\Bover{1-r^2}{1-2r\cos t+r^2}$.
($A_r$ is the {\bf Poisson kernel};  see 478Xl\Latereditions\ in
Volume 4.)
(i) Show that $\Bover1{2\pi}\int_{-\pi}^{\pi}A_r=1$.
(ii) For a real function $f$ which is
integrable over $\ocint{-\pi,\pi}$, with real Fourier coefficients
$a_k$, $b_k$ (282Ba), set
$S_r(x)=\Bover12a_0+\sum_{k=1}^{\infty}r^k(a_k\cos kx+b_k\sin kx)$ for
$x\in\ocint{-\pi,\pi}$, $r\in\coint{0,1}$.   Show that
$S_r(x)=\Bover1{2\pi}\int_{-\pi}^{\pi}A_r(x-t)f(t)dt$ for every
$x\in\ocint{-\pi,\pi}$.   \Hint{$A_r(t)=1+2\sum_{n=1}^{\infty}r^n\cos nt$.}
(iii) Show that $\lim_{r\uparrow 1}S_r(x)=f(x)$ for every
$x\in\ooint{-\pi,\pi}$ which is in the Lebesgue set of $f$.
\Hint{223Yg.}  (iv) Show that
$\lim_{r\uparrow 1}\int_{-\pi}^{\pi}|S_r-f|=0$.   (v) Show that if $f$ is
defined everywhere on $\ocint{-\pi,\pi}$, is continuous, and
$f(\pi)=\lim_{x\downarrow-\pi}f(x)$, then
$\lim_{r\uparrow 1}\sup_{x\in\ocint{-\pi,\pi}}|S_r(x)-f(x)|=0$.
%282+
}%end of exercises

\endnotes{
\Notesheader{282} This has been a long section with a
potentially confusing collection of results, so perhaps I should
recapitulate.   Associated with any integrable function on
$\ocint{-\pi,\pi}$ we have the corresponding Fourier sums, being the
symmetric partial sums $\sum_{k=-n}^nc_ke^{ikx}$ of the complex series
$\sum_{k=-\infty}^{\infty}c_ke^{ikx}$, or, equally, the partial sums
${1\over 2}a_0+\sum_{k=1}^na_k\cos kx+b_k\sin kx$ of the real series
${1\over 2}a_0+\sum_{k=1}^{\infty}a_k\cos kx+b_k\sin kx$.   The Fourier
coefficients $c_k$, $a_k$, $b_k$ are the only natural ones, because
if the series is to converge with any regularity at all then

\Centerline{$\Bover1{2\pi}\int_{-\pi}^{\pi}
\bigl(\sum_{k=-\infty}^{\infty}c_ke^{ikx}\bigr)e^{-ilx}dx$}

\noindent ought to be simultaneously

\Centerline{$\sum_{k=-\infty}^{\infty}\Bover1{2\pi}\int_{-\pi}^{\pi}
c_ke^{ikx}e^{-ilx}dx=c_l$}

\noindent and

\Centerline{$\Bover1{2\pi}\int_{-\pi}^{\pi}f(x)e^{-ilx}dx$.}

\noindent (Compare the calculations in 282J.)   The effect of taking
Fej\'er sums $\sigma_m(x)$ rather than the Fourier sums $s_n(x)$ is to
smooth the sequence out;   recall that if $\lim_{n\to\infty}s_n(x)=c$
then $\lim_{m\to\infty}\sigma_m(x)=c$, by 273Ca in the last chapter.

Most of the work above is concerned with the question of when Fourier or
Fej\'er sums converge, in some sense, to the original function $f$.   As
has happened before, in \S245 and elsewhere, we have more than
one kind of convergence to consider.   {\it Norm} convergence, for
$\|\,\|_1$ or $\|\,\|_2$ or $\|\,\|_{\infty}$, is the simplest;  the
three theorems 282G, 282Ib and 282J at least are
relatively straightforward.   (I have given 282Ib as a corollary of
282Ia;  but there is an easier proof from 282G.   See 282Xd.)
Respectively, we have

\inset{if $f$ is continuous (and matches at $\pm\pi$, that is,
$f(\pi)=\lim_{t\downarrow-\pi}f(t)$) then $\sigma_m\to
f$ uniformly, that is, for $\|\,\|_{\infty}$ (282G);}

\inset{if $f$ is any integrable function, then $\sigma_m\to f$ for
$\|\,\|_1$ (282Ib);}

\inset{if $f$ is a square-integrable function, then $s_n\to f$ for
$\|\,\|_2$ (282J);}

\inset{if $f$ is continuous and of bounded
variation (and matches at $\pm\pi$), then $s_n\to f$ uniformly (282O).}

\noindent There are some similar results for other $\|\,\|_p$ (282Yc);
but note that the Fourier sums need not converge for $\|\,\|_1$ (282Ye).

{\it Pointwise} convergence is harder.   The results I give are

\inset{if $f$ is any integrable function, then $\sigma_m\to f$ almost
everywhere (282Ia);}

\noindent this relies on some careful calculations in 282H, and also on
the deep result 223D.   Next we have the results which look at the
average of the limits of $f$ from the two sides.   Suppose I write

\Centerline{$f^{\pm}(x)=\Bover12(\lim_{t\uparrow
x}f(t)+\lim_{t\downarrow x}f(t))$}

\noindent whenever this is defined, taking $f^{\pm}(\pi)
=\bover12(\lim_{t\uparrow\pi}f(t)+\lim_{t\downarrow-\pi}f(t))$.   Then
we have

\inset{if $f$ is any integrable function, $\sigma_m\to f^{\pm}$ wherever
$f^{\pm}$ is defined (282I);}

\inset{if $f$ is of bounded variation, $s_n\to f^{\pm}$ everywhere
(282O).}

\noindent Of course these apply at any point at which $f$ is continuous,
in which case $f(x)=f^{\pm}(x)$.
Yet another result of this type is

\inset{if $f$ is any integrable function, $s_n\to f$ at any point at
which $f$ is differentiable (282L);}

\noindent in fact, this can be usefully extended for very little extra
labour (282Xi, 282Xp).

I cannot leave this list without mentioning the theorem I have {\it not}
given.   This is {\bf Carleson's theorem}:

\inset{if $f$ is square-integrable, $s_n\to f$ almost everywhere}

\noindent ({\smc Carleson 66}).   I will come to this in \S286.   There
is an elementary special case in 282Xt.   The result is in fact valid
for many other $f$ (see the notes to \S286).

The next glaring lacuna in the exposition here is the absence of any
examples to show how far these results are best possible.   There is no
suggestion, indeed, that there are any natural necessary and sufficient
conditions for

\inset{$s_n\to f$ at every point.}

\noindent Nevertheless, we have to make an effort to find a continuous
function
for which this is not so, and the construction of an example by du
Bois-Reymond ({\smc Bois-Reymond 1876}) was an important moment in
the history of analysis, not least because it forced mathematicians to
realise that some comfortable assumptions about  the classification of
functions -- essentially, that functions are either `good' or so bad
that one needn't trouble with them -- were false.  The example is
instructive but I have had to omit it for lack of space;  I give an
outline of a possible method in 282Yd.   (You can find a detailed
construction in {\smc K\"orner 88}, chapter 18, and a proof that such a
function exists in {\smc Dudley 89}, 7.4.3.)    If you allow general
integrable functions, then you can do much better, or perhaps I should
say much worse;  there is an integrable $f$ such that $\sup_{n\in\Bbb
N}|s_n(x)|=\infty$ for every $x\in\ocint{-\pi,\pi}$
({\smc Kolmogorov 1926};  see {\smc Zygmund 59}, $\S\S$VIII.3-4).

In 282C I mentioned two types of problem.   The first -- when is a
Fourier
series summable? -- has at least been treated at length, even though I
cannot pretend to have given more than a sample of what is known.   The
second -- how do properties of the $c_k$ reflect properties of $f$? -- I
have hardly touched on.   I do give what seem to me to be the three most
important results in this area.   The first is

\inset{if $f$ and $g$ have the same Fourier coefficients, they are equal
almost everywhere (282Ic).}

\noindent This at least tells us that we ought in principle to be able
to learn almost anything about $f$ by looking at its Fourier series.
(For instance, 282Ya describes a necessary and sufficient condition for
$f$ to be non-negative almost everywhere.)   The second is

\inset{$f$ is square-integrable iff
$\sum_{k=-\infty}^{\infty}|c_k|^2<\infty$;}

\noindent in fact,

\inset{$\sum_{k=-\infty}^{\infty}|c_k|^2
=\Bover1{2\pi}\int_{\pi}^{\pi}|f|^2$
(282J).}

\noindent Of course this is fundamental, since it shows that Fourier
coefficients provide a natural Hilbert space isomorphism between $L^2$
and $\ell^2$ (282K).   I should perhaps remark that while the real
Hilbert spaces $L^2_{\Bbb R}$, $\ell^2_{\Bbb R}$ are isomorphic as inner
product spaces (282Xg), they are certianly not isomorphic as Banach
lattices;  for instance, $\ell^2_{\Bbb R}$ has `atomic' elements
$\pmb{c}$ such that if $0\le\pmb{d}\le\pmb{c}$ then $\pmb{d}$ is a
multiple of $\pmb{c}$, while $L^2_{\Bbb R}$ does not.   Perhaps even
more important is

\inset{the Fourier coefficients of a convolution $f*g$ are just a scalar
multiple of the
products of the Fourier coefficients of $f$ and $g$ (282Q);}

\noindent  but to use this effectively we need to study the Banach
algebra structure of $L^1$, and I have no choice but to abandon this
path immediately.   (It will form a conspicuous part of Chapter 44 in
Volume 4.)   282Xt gives an elementary
consequence, and 282Xq a very partial description of the
relationship between a product $f\times g$ of two functions and the
convolution product of their sequences of Fourier coefficients.

The Fej\'er sums considered in this section are one way of working
around the convergence difficulties associated with Fourier sums.   When we
come to look at Fourier transforms in the next two sections we shall need
some further manoeuvres.   A different type of smoothing is
obtained by using the Poisson kernel in place of the Dirichlet or Fej\'er
kernel (282Yg).

I end these notes with a remark on the number $2\pi$.   This enters
nearly every formula involving Fourier series, but could I think be
removed totally from the present section, at least, by re-normalizing
the measure of $\ocint{-\pi,\pi}$.   If instead of Lebesgue measure
$\mu$ we took the measure $\nu=\bover1{2\pi}\mu$ throughout, then every
$2\pi$ would disappear.   (Compare the remark in 282Bb concerning the
possibility of doing integrals over $S^1$.)   But I think most of us
would prefer to remember the location of a $2\pi$ in every formula than
to deal with an unfamiliar measure.
}%end of notes

\discrpage

