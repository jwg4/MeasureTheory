\frfilename{mt3a1.tex}
\versiondate{31.10.07}
\copyrightdate{1996}

\def\chaptername{Appendix}
\def\sectionname{Set Theory}
\def\Enderton{{\smc Enderton 77}}
\def\Halmos{{\smc Halmos 60}}
\def\Jech{{\smc Jech 03}}
\def\Krivine{{\smc Krivine 71}}
\def\Kunen{{\smc Kunen 80}}
\def\Lipschutz{{\smc Lipschutz 64}}

\newsection{3A1}
\cmmnt{
\leader{3A1A}{The axioms of set theory} This treatise is based
on arguments within, or in principle reducible to, `ZFC', meaning
`Zermelo-Fraenkel set theory, including the Axiom of Choice'.   For
discussions of this system, see, for instance, \Krivine, \Jech\ or
\Kunen.   As I remarked in \S2A1, I believe that it is helpful,
as a matter of general principle, to distinguish between results
dependent on the axiom of choice and those which can be proved without
it, or with some relatively weak axiom such as `countable choice'.
(See 134C.   I will go much more deeply into this in Chapter 56 of
Volume 5.)   In Volumes 1 and 2, such a
distinction is useful in appreciating the special features of different
ideas.   In the present volume, however, most of the principal theorems
require something close to the full axiom of choice, and there are few
areas where it seems at present appropriate to work with anything
weaker.   Indeed, at many points we shall approach questions which are,
or may be, undecidable in ZFC;  but with very few exceptions
I postpone discussion of these to
Volume 5.   In particular, I specifically exclude, for the time being,
results dependent on such axioms as the continuum hypothesis.
}%end of comment

\leader{3A1B}{Definition} Let $X$ be a set.   By an {\bf enumeration} of
$X$ I mean a bijection $f:\kappa\to X$ where
$\kappa=\#(X)$\cmmnt{is the initial ordinal equipollent with $X$
(2A1Kb)};  more often than not I shall express
such a function in the form $\langle x_{\xi}\rangle_{\xi<\kappa}$.  In
this case I say that the function $f$, or the family $\langle
x_{\xi}\rangle_{\xi<\kappa}$, {\bf enumerates} $X$.   \cmmnt{You will
see that I am tacitly assuming that $\#(X)$ is
always defined, that is, that the axiom of choice is true.}

\leader{3A1C}{Calculation of cardinalities}\cmmnt{ The following
formulae are basic.}

\spheader 3A1Ca For any sets $X$ and $Y$,
$\#(X\times Y)\le\max(\omega,\#(X),\#(Y))$.
\cmmnt{(\Enderton, p.\ 64;  \Jech,
p.\ 51;    \Krivine, p.\ 33;  \Kunen, 10.13.)}

\spheader 3A1Cb For any $r\in\Bbb N$ and any family
$\langle X_i\rangle_{i\le r}$ of sets,
$\#(\prod_{i=0}^rX_i)\le\max(\omega,\max_{i\le r}\#(X_i))$.
\prooflet{(Induce on $r$.)}

\spheader 3A1Cc For any family $\langle X_i\rangle_{i\in I}$ of sets,
$\#(\bigcup_{i\in I}X_i)\le\max(\omega,\#(I),\sup_{i\in I}\#(X_i))$.
\cmmnt{(\Jech, p.\ 52;  \Krivine, p.\ 33;  \Kunen, 10.21.)}

\spheader 3A1Cd For any set $X$,\cmmnt{ the
set} $[X]^{<\omega}$\cmmnt{ of finite subsets of $X$} has
cardinal at most $\max(\omega,\#(X))$.  \prooflet{(There is a
surjection from $\bigcup_{r\in\Bbb N}X^r$ onto $[X]^{<\omega}$.   For
the notation $[X]^{<\omega}$ see 3A1J below.)}

\leader{3A1D}{Cardinal exponentiation} For a cardinal $\kappa$, I write
$2^{\kappa}$ for $\#(\Cal P\kappa)$.   So
$2^{\omega}=\frak c$, and $\kappa^+\le 2^{\kappa}$ for every $\kappa$.
\cmmnt{(\Enderton, p.\ 132;  \Lipschutz, p.\ 139;  \Jech, p.\ 29;
\Krivine, p.\ 25;  \Halmos, p.\ 93.)}

\leader{3A1E}{Definition}\cmmnt{ The class of infinite initial
ordinals, or cardinals, is a subclass of the class $\text{On}$ of all
ordinals, so is itself well-ordered;  being unbounded, it is a proper
class;  consequently there is a unique increasing enumeration of it as
$\langle\omega_{\xi}\rangle_{\xi\in\text{On}}$.
We have} $\omega_0=\omega$, $\omega_{\xi+1}=\omega_{\xi}^+$ for every
$\xi$\cmmnt{ (compare 2A1Fc)},
$\omega_{\xi}=\bigcup_{\eta<\xi}\omega_{\eta}$
for non-zero limit ordinals $\xi$.  \cmmnt{(\Enderton, pp.\ 213-214;
\Jech, p.\ 30;  \Krivine, p.\ 31.)}

\leader{3A1F}{Cofinal sets (a)} If $P$ is any partially ordered
set\cmmnt{ (definition:  2A1Aa)}, a
subset $Q$ of $P$ is {\bf cofinal} with $P$ if
for every $p\in P$ there is a $q\in Q$ such that $p\le q$.

\spheader 3A1Fb If $P$ is any partially ordered set, the {\bf
cofinality} of $P$, $\cf P$, is the least cardinal of any cofinal
subset of $P$.   \cmmnt{Note that} $\cf P=0$ iff $P=\emptyset$,
and\cmmnt{ that} $\cf P=1$ iff $P$ has
a greatest element.

\spheader 3A1Fc Observe that if $P$ is upwards-directed and $\cf P$ is
finite, then $\cf P$ is either $0$ or $1$\cmmnt{;  for if $Q$ is a
finite, non-empty cofinal set then it has an upper bound, which must be
the greatest  element of $P$}.

\spheader 3A1Fd If $P$ is a totally ordered set of cofinality $\kappa$,
then there is a strictly increasing family $\langle
p_{\xi}\rangle_{\xi<\kappa}$ in $P$ such that $\{p_{\xi}:\xi<\kappa\}$
is cofinal with $P$.   \prooflet{\Prf\ If $\kappa=0$ then
$P=\emptyset$ and this is trivial.    Otherwise, let $Q$ be a cofinal
subset of $P$ of cardinal $\kappa$, and $\{q_{\xi}:\xi<\kappa\}$ an
enumeration of $Q$.   Define $\langle p_{\xi}\rangle_{\xi<\kappa}$
inductively, as follows.   Start with $p_0=q_0$.   Given $\langle
p_{\eta}\rangle_{\eta<\xi}$, where $\xi<\kappa$, then if
$p_{\eta}<q_{\xi}$ for every $\eta<\xi$, take $p_{\xi}=q_{\xi}$;
otherwise, because $\#(\xi)\le\xi<\kappa$, $\{p_{\eta}:\eta<\xi\}$
cannot be cofinal with $P$, so there is a $p_{\xi}\in P$ such that
$p_{\xi}\not\le p_{\eta}$ for every $\eta<\xi$, that is,
$p_{\eta}<p_{\xi}$ for every $\eta<\xi$.   Note that there is some
$\eta<\xi$ such that $q_{\xi}\le p_{\eta}$, so that $q_{\xi}\le
p_{\xi}$.   Continue.

Now $\langle p_{\xi}\rangle_{\xi<\kappa}$ is a strictly increasing
family in $P$ such that $q_{\xi}\le p_{\xi}$ for every $\xi$;   it
follows at once that $\{p_{\xi}:\xi<\kappa\}$ is cofinal with $P$.
\Qed}

\spheader 3A1Fe In particular, for a totally ordered set $P$,
$\cf P=\omega$ iff there is a cofinal strictly increasing sequence in
$P$.

\cmmnt{
\leader{3A1G}{Zorn's Lemma} In Volume 2 I used Zorn's Lemma
only once or twice, giving the arguments in detail.   In the present
volume I feel that continuing in such a manner would often be tedious;
but nevertheless the arguments are not always quite obvious, at least
until you have gained a good deal of experience.   I therefore take a
paragraph to comment on some of the standard forms in which they appear.

The statement of Zorn's Lemma, as quoted in 2A1M, refers to arbitrary
partially ordered sets $P$.   A large proportion of the applications can
in fact be represented more or less naturally by taking $P$ to be a
family $\frak P$ of sets ordered by $\subseteq$;  in such a case, it
will be sufficient to check that (i) $\frak P$ is not empty (ii)
$\bigcup\frak Q\in\frak P$ for every non-empty totally ordered $\frak
Q\subseteq\frak P$.   More often than not, this will in fact be true for
all non-empty upwards-directed sets $\frak Q\subseteq\frak P$, and the
line of the argument is sometimes clearer if phrased in this form.

Within this class of partially ordered sets, we can distinguish a
special subclass.   If $A$ is any set and $\perp$ any relation on $A$,
we can consider the collection $\frak P$ of sets $I\subseteq A$ such
that $a\perp b$ for all distinct $a$, $b\in I$.   In this case we need
look no farther before declaring `$\frak P$ has a maximal element';  for
$\emptyset$ necessarily belongs to $\frak P$, and if $\frak Q$ is any
upwards-directed subset of $\frak P$, then $\bigcup\frak Q\in\frak P$.
\prooflet{\Prf\ If $a$, $b$ are distinct elements of $\bigcup\frak Q$,
there are $I_1$, $I_2\in\frak Q$ such that $a\in I_1$, $b\in I_2$;
because $\frak Q$ is upwards-directed, there is an $I\in\frak Q$ such
that $I_1\cup I_2\subseteq I$, so that $a$, $b$ are distinct members of
$I\in\frak P$, and $a\perp b$.\ \QeD}   So $\bigcup\frak Q$ is an upper
bound of $\frak Q$ in $\frak P$;  as $\frak Q$ is arbitrary, $\frak P$
satisfies the conditions of Zorn's Lemma, and must have a maximal
element.

Another important type of partially ordered set in this context is a
family $\Phi$ of functions, ordered by saying that $f\le g$ if $g$ is an
extension of $f$.   In this case, for any non-empty upwards-directed
$\Psi\subseteq\Phi$, we shall have a function $h$ defined by saying that

\Centerline{$\dom h=\bigcup_{f\in\Psi}\dom f$,
\quad $h(x)=f(x)$ whenever $f\in\Psi$, $x\in\dom f$,}

\noindent and the usual attack is to seek to prove that any such $h$
belongs to $\Phi$.

I find that at least once I wish to use Zorn's Lemma `upside down':
that is, I have a non-empty partially ordered set $P$ in which every
non-empty totally ordered subset has a {\it lower} bound.   In this
case, of course, $P$ has a {\it minimal} element.   The point is that
the definition of `partial order' is symmetric, so that $(P,\ge)$ is a
partially ordered set whenever $(P,\le)$ is;  and we can seek to apply
Zorn's Lemma to either.
}%end of comment

\cmmnt{
\leader{3A1H}{Natural numbers and finite ordinals} I remarked
in 2A1De that the first few ordinals

\Centerline{$\emptyset$,
\quad$\{\emptyset\}$,
\quad$\{\emptyset,\{\emptyset\}\}$,
\quad$\{\emptyset,\{\emptyset\},\{\emptyset,\{\emptyset\}\}\}$,
\quad$\ldots$}

\noindent may be identified with the natural numbers $0,1,2,3,\ldots$;
the idea being that $n=\{0,1,\ldots,\penalty-100{n-}\penalty+100{1}\}$
is a set with $n$ elements.
If we do this, then the set $\Bbb N$ of natural numbers becomes
identified with the first infinite ordinal $\omega$.   This convention
makes it possible to present a number of arguments in a particularly
elegant form.   A typical example is in 344H.   There I wish to describe
an inductive construction for a family
$\langle K_{\sigma}\rangle_{\sigma\in S^*}$
where $S^*=\bigcup_{n\in\Bbb N}\{0,1\}^{n}$.   If we think of $n$ as the
set of its predecessors, then $\sigma\in\{0,1\}^n$ becomes a function from
$n$ to $\{0,1\}$;  since the set $n$ has just $n$ members, this
corresponds well to the idea of $\sigma$ as the list of its
$n$ coordinates, except that it would now be natural to list them as
$\sigma(0),\ldots,\sigma(n-1)$ rather than as
$x_1,\ldots,x_n$, which was
the language I favoured in Volume 2.   An extension of $\sigma$ to a member
of $\{0,1\}^{n+1}$ is of the form
$\tau=\sigma^{\smallfrown}\fraction{i}$ where
$\tau(k)=\sigma(k)$ for $k<n$ (`$\tau\restr n=\sigma$') and $\tau(n)=i$.
If $w\in\{0,1\}^{\Bbb N}$, then we can
identify the initial segment $(w(0),w(1),\ldots,w(n-1))$ of its first
$n$ coordinates with the restriction $w\restr n$ of $w$ to the set
$n=\{0,\ldots,n-1\}$.
}%end of comment

\leader{3A1I}{Definitions (a)} If $P$ and $Q$ are
lattices\cmmnt{ (2A1Ad)}, a {\bf lattice homomorphism} from $P$ to $Q$
is a function $f:P\to Q$ such that $f(p\wedge p')=f(p)\wedge f(p')$ and
$f(p\vee p')=f(p)\vee f(p')$ for all $p$, $p'\in P$.   Such a
homomorphism is\cmmnt{ surely}
order-preserving\cmmnt{ (313H)}\prooflet{, for if
$p\le p'$ in $P$ then $f(p')=f(p\vee p')=f(p)\vee f(p')$ and
$f(p)\le f(p')$}.

\spheader 3A1Ib If $P$ is a lattice, a {\bf sublattice} of $P$ is a set
$Q\subseteq P$ such that $p\vee q$ and $p\wedge q$ belong to $Q$ for all
$p$, $q\in Q$.

\spheader 3A1Ic{\bf (i)} A lattice $P$ is {\bf distributive} if

\Centerline{$(p\wedge q)\vee r=(p\vee r)\wedge(q\vee r)$,
\quad$(p\vee q)\wedge r=(p\wedge r)\vee(q\wedge r)$}

\noindent for all $p$, $q$, $r\in P$.

\medskip

\quad{\bf (ii)}\dvAnew{2010} In a distributive lattice we have a
{\bf median function} of three variables

\dvro{\Centerline{$\med(p,q,r)
=(p\wedge q)\vee(p\wedge r)\vee(q\wedge r)$.}}
{$$\eqalign{\med(p,q,r)
&=(p\wedge q)\vee(p\wedge r)\vee(q\wedge r)\cr
&=((p\wedge(q\vee r))\vee(q\wedge r)
=(p\vee q)\wedge(p\vee r)\wedge(q\vee r).\cr}$$}

\noindent If $P$ and $Q$ are distributive lattices and $f:P\to Q$ is a
lattice homomorphism, $f(\med(p,q,r))=\med(f(p),f(q),f(r))$ for all $p$,
$q$, $r\in P$.

\medskip

\quad{\bf (iii)}\dvAnew{2012}
If $P$ is a distributive lattice and $I\subseteq P$
is finite, then the sublattice of $P$ generated by $I$ is finite.
\prooflet{\Prf\ If $J=\{\sup I_0:\emptyset\ne I_0\subseteq I\}$ and
$K=\{\inf J_0:\emptyset\ne J_0\subseteq J\}$
then $K$ is a sublattice of $P$.\ \Qed}
%for 611C

\leader{3A1J}{Subsets of given size}\cmmnt{ The following concepts are
used often enough for a special notation to be helpful.}   If $X$ is a
set and $\kappa$ is a cardinal, write

\Centerline{$[X]^{\kappa}=\{A:A\subseteq X,\,\#(A)=\kappa\}$,}

\Centerline{$[X]^{\le\kappa}=\{A:A\subseteq X,\,\#(A)\le\kappa\}$,}

\Centerline{$[X]^{<\kappa}=\{A:A\subseteq X,\,\#(A)<\kappa\}$.}

\cmmnt{\noindent Thus

\Centerline{$[X]^0=[X]^{\le 0}=[X]^{<1}=\{\emptyset\}$,}

\noindent $[X]^2$ is the set of doubleton subsets of $X$,
$[X]^{<\omega}$ is the set of finite subsets of $X$, $[X]^{\le\omega}$
is the set of countable subsets of $X$, and so on.
}%end of comment

\leader{3A1K}{}\cmmnt{ The next result is one of the fundamental
theorems of combinatorics.
In this volume it is used in the proofs of Ornstein's theorem (\S387)
and the Kalton-Roberts theorem (\S392).

\medskip

\noindent}{\bf Hall's Marriage Lemma} Suppose that $X$ and $Y$ are
finite sets and $R\subseteq X\times Y$ is a relation such that
$\#(R[I])\ge\#(I)$ for every $I\subseteq X$.   Then there is an
injective function $f:X\to Y$ such that $(x,f(x))\in R$ for every
$x\in X$.

\cmmnt{\medskip

\noindent{\bf Remark} Recall that $R[I]$ is the set
$\{y:\,\Exists x\in I,\,(x,y)\in R\}$ (1A1Bc).    If we identify a function
with its graph, then `$(x,f(x))\in R$ for every
$x\in X$' becomes `$f\subseteq R$'.
}%end of comment

\proof{{\smc Bollob\'as 79}, p.\ 54, Theorem 7;  {\smc Anderson 87}, 2.2.1;  {\smc Bose \& Manvel 84}, \S10.2.
}%end of proof of 3A1K

\discrpage

