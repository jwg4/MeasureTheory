\frfilename{mt251.tex}
\versiondate{10.11.06}
\copyrightdate{2000}

\def\chaptername{Product measures}
\def\sectionname{Finite products}

\newsection{251}

The first construction to set up is the product of a pair of measure
spaces.    It turns out that there are already substantial technical
difficulties in the way of finding a canonical universally applicable
method.   I find myself therefore describing two related, but distinct,
constructions, the `primitive' and `c.l.d.' product measures (251C,
251F).   After listing the fundamental properties of the c.l.d\ product
measure (251I-251J), I work through the identification of the product of
Lebesgue measure with itself (251N) and a fairly thorough discussion of
subspaces (251O-251S).

\vleader{48pt}{251A}{Definition} Let $(X,\Sigma,\mu)$ and $(Y,\Tau,\nu)$
be two
measure spaces.   For $A\subseteq X\times Y$ set

\Centerline{$\theta A=\inf\{\sum_{n=0}^{\infty}\mu E_n\cdot\nu F_n:
  E_n\in\Sigma,\,F_n\in\Tau\,\Forall \,n\in\Bbb N,
  \,A\subseteq\bigcup_{n\in\Bbb N}E_n\times F_n\}$.}

\cmmnt{\medskip

\noindent{\bf Remark} In the products $\mu E_n\cdot\nu F_n$,
$0\cdot\infty$ is to be taken as $0$, as in \S135.
}

\leader{251B}{Lemma} In the context of 251A, $\theta$ is an outer
measure on $X\times Y$.

\proof{{\bf (a)} Setting $E_n=F_n=\emptyset$ for every
$n\in\Bbb N$, we see that $\theta\emptyset = 0$.

\medskip

{\bf (b)} If $A\subseteq B\subseteq X\times Y$, then whenever
$B\subseteq\bigcup_{n\in\Bbb N}E_n\times F_n$ we shall have
$A\subseteq\bigcup_{n\in\Bbb N}E_n\times F_n$;  so $\theta A\le\theta B$.

\medskip

{\bf (c)} Let $\sequencen{A_n}$ be a sequence of subsets of $X\times Y$,
with union $A$.
For any $\epsilon>0$, we may choose, for each $n\in\Bbb N$, sequences
$\sequence{m}{E_{nm}}$ in $\Sigma$ and $\sequence{m}{F_{nm}}$ in $\Tau$
such that $A_n\subseteq\bigcup_{m\in\Bbb N}E_{nm}\times F_{nm}$ and
$\sum_{m=0}^{\infty}\mu E_{nm}\cdot\nu F_{nm}
\le\theta A_n+2^{-n}\epsilon$.
Because $\Bbb N\times\Bbb N$ is countable,  we have a bijection
$k\mapsto(n_k,m_k):\Bbb N\to\Bbb N\times\Bbb N$, and now

\Centerline{$A\subseteq\bigcup_{n,m\in\Bbb N}E_{nm}\times F_{nm}
=\bigcup_{k\in\Bbb N}E_{n_km_k}\times F_{n_km_k}$,}

\noindent so that

$$\eqalign{\theta A&\le\sum_{k=0}^{\infty}
  \mu E_{n_km_k}\cdot\nu F_{n_km_k}
=\sum_{n=0}^{\infty}\sum_{m=0}^{\infty}\mu E_{nm}\cdot\nu F_{nm}\cr
&\le\sum_{n=0}^{\infty}\theta A_n+2^{-n}\epsilon
=2\epsilon+\sum_{n=0}^{\infty}\theta A_n.\cr}$$

\noindent As $\epsilon$ is arbitrary, $\theta
A\le\sum_{n=0}^{\infty}\theta A_n$.

As $\sequencen{A_n}$ is arbitrary, $\theta$ is an outer measure.
}

\leader{251C}{Definition} Let $(X,\Sigma,\mu)$ and $(Y,\Tau,\nu)$ be
measure spaces.   By the {\bf primitive product measure} on $X\times Y$
I shall mean the measure $\lambda_0$ derived by \Caratheodory's
method\cmmnt{ (113C)} from the outer measure $\theta$ defined in 251A.

\cmmnt{\medskip

\noindent{\bf Remark}
I ought to point out that there is no general agreement on what
`the' product measure on $X\times Y$ should be.   Indeed in 251F below I
will introduce an alternative one, and in the notes to this section I
will mention a third.
}

\leader{251D}{Definition}\cmmnt{ It is convenient to have a name for a
natural construction for $\sigma$-algebras.}   If $X$ and $Y$ are sets
with $\sigma$-algebras $\Sigma\subseteq\Cal PX$ and
$\Tau\subseteq\Cal PY$, I will write $\Sigma\tensorhat\Tau$ for the
$\sigma$-algebra of subsets of $X\times Y$ generated by
$\{E\times F:E\in\Sigma,\,F\in\Tau\}$.


\leader{251E}{Proposition} Let $(X,\Sigma,\mu)$ and $(Y,\Tau,\nu)$ be
measure spaces;  let $\lambda_0$  be the primitive product measure on
$X\times Y$, and $\Lambda$ its domain.
Then $\Sigma\tensorhat\Tau\subseteq\Lambda$  and
$\lambda_0(E\times F)=\mu E\cdot\nu F$ for all $E\in\Sigma$
and $F\in\Tau$.

\proof{{\bf (a)}   Suppose that $E\in\Sigma$ and $A\subseteq X\times Y$.
For any $\epsilon>0$, there are sequences $\sequencen{E_n}$ in $\Sigma$
and $\sequencen{F_n}$ in $\Tau$ such that
$A\subseteq\bigcup_{n\in\Bbb N}E_n\times F_n$ and
$\sum_{n=0}^{\infty}\mu E_n\cdot\nu F_n\le\theta A+\epsilon$.   Now

\Centerline{$A\cap(E\times Y)
\subseteq\bigcup_{n\in\Bbb N}(E_n\cap E)\times F_n$,
\quad$A\setminus(E\times Y)
\subseteq\bigcup_{n\in\Bbb N}(E_n\setminus E)\times F_n$,}

\noindent so

$$\eqalign{\theta(A\cap(E\times Y))+\theta(A\setminus(E\times Y))
&\le\sum_{n=0}^{\infty}\mu(E_n\cap E)\cdot\nu F_n
  +\sum_{n=0}^{\infty}\mu(E_n\setminus E)\cdot\nu F_n\cr
&=\sum_{n=0}^{\infty}\mu E_n\cdot\nu F_n
\le\theta A+\epsilon.\cr}$$

\noindent As $\epsilon$ is arbitrary,
$\theta(A\cap(E\times Y))+\theta(A\setminus(E\times Y))\le\theta A$.
And this is enough to ensure that $E\times Y\in\Lambda$ (see 113D).

\medskip

{\bf (b)} Similarly, $X\times F\in\Lambda$ for every $F\in\Tau$,
so $E\times F=(E\times Y)\cap(X\times F)\in\Lambda$ for every
$E\in\Sigma$, $F\in\Tau$.

Because $\Lambda$ is a $\sigma$-algebra, it must include the smallest
$\sigma$-algebra containing all the products $E\times F$, that is,
$\Lambda\supseteq\Sigma\tensorhat\Tau$.

\medskip


{\bf (c)} Take $E\in\Sigma$, $F\in\Tau$.   We know that $E\times
F\in\Lambda$;  setting $E_0=E$, $F_0=F$, $E_n=F_n=\emptyset$ for $n\ge
1$ in the definition of $\theta$, we have

\Centerline{$\lambda_0(E\times F)
=\theta(E\times F)\le\mu E\cdot\nu F$.}

We have come to the central idea of the construction.   In fact
$\theta(E\times F)=\mu E\cdot\nu F$.   \Prf\ Suppose that
$E\times F\subseteq\bigcup_{n\in\Bbb N}E_n\times F_n$ where
$E_n\in\Sigma$ and $F_n\in\Tau$ for every $n$.   Set
$u=\sum_{n=0}^{\infty}\mu E_n\cdot\nu F_n$.   If $u=\infty$ or $\mu E=0$
or $\nu F=0$ then of course $\mu E\cdot\nu F\le u$.    Otherwise, set

\Centerline{$I=\{n:n\in\Bbb N,\,\mu E_n=0\}$,
\quad$J=\{n:n\in\Bbb N,\,\nu F_n=0\}$,
\quad$K=\Bbb N\setminus(I\cup J)$,}

\Centerline{$E'=E\setminus\bigcup_{n\in I}E_n$,
\quad$F'=F\setminus\bigcup_{n\in J}F_n$.}

\noindent Then $\mu E'=\mu E$ and $\nu F'=\nu F$;
$E'\times F'\subseteq\bigcup_{n\in K}E_n\times F_n$;  and for $n\in K$,
$\mu E_n<\infty$ and $\nu F_n<\infty$, since
$\mu E_n\cdot\nu F_n\le u<\infty$ and
neither $\mu E_n$ nor $\nu F_n$ is zero.   Set

\Centerline{$f_n=\nu F_n\chi E_n:X\to\Bbb R$}

\noindent if $n\in K$, and $f_n=\tbf{0}:X\to\Bbb R$ if $n\in I\cup J$.
Then $f_n$ is a simple function and
$\int f_n=\nu F_n\mu E_n$ for $n\in K$, $0$ otherwise, so

\Centerline{$\sum_{n=0}^{\infty}$
$\textfont3=\twelveex\int f_n(x)\mu(dx)$
$=\sum_{n=0}^{\infty}\mu E_n\cdot\nu F_n
\le u$.}

\noindent By B.Levi's theorem (123A), applied to
$\sequencen{\sum_{k=0}^nf_k}$,
$g=\sum_{n=0}^{\infty}f_n$ is integrable and $\int g\,d\mu\le u$.
Write $E''$  for $\{x:x\in E',\,g(x)<\infty\}$, so that
$\mu E''=\mu E'=\mu E$.   Now take any $x\in E''$ and set
$K_x=\{n:n\in K,\,x\in E_n\}$.   Because
$E'\times F'\subseteq\bigcup_{n\in K}E_n\times F_n$,
$F'\subseteq\bigcup_{n\in K_x}F_n$ and

\Centerline{$\nu F=\nu F'\le\sum_{n\in K_x}\nu F_n
=\sum_{n=0}^{\infty}f_n(x)=g(x)$.}

\noindent Thus $g(x)\ge\nu F$ for every $x\in E''$.   We are
supposing that $0<\mu E=\mu E''$ and $0<\nu F$, so we must have
$\nu F<\infty$, $\mu E''<\infty$.   Now  $g\ge\nu F\chi E''$,  so

\Centerline{$\mu E\cdot\nu F=\mu E''\cdot\nu F
=\int\nu F\chi E''\le\int g\le u$
$=\sum_{n=0}^{\infty}\mu E_n\cdot\nu F_n$.}

\noindent As $\sequencen{E_n}$, $\sequencen{F_n}$ are arbitrary,
$\theta(E\times F)\ge\mu E\cdot\nu F$ and
$\theta(E\times F)=\mu E\cdot\nu F$.\ \Qed

Thus

\Centerline{$\lambda_0(E\times F)=\theta(E\times F)=\mu E\cdot\nu F$}

\noindent for all $E\in\Sigma$, $F\in\Tau$.
}%end of proof of 251E

\leaveitout{for (c), M.Burke offers:

$\chi E(x)\chi F(y)\le\sum_{n=0}^{\infty}\chi E_n(x)\chi F_n(y)$

by 135F-135G, integrating with respect to $y$, \query

$\chi E(x)\nu F\le\sum_{n=0}^{\infty}\chi E_n(x)\nu F_n$

and now

$\mu E\cdot\nu F\le\sum_{n=0}^{\infty}\mu E_n\cdot\nu F_n$.}


\leader{251F}{Definition} Let $(X,\Sigma,\mu)$ and $(Y,\Tau,\nu)$ be
measure spaces, and $\lambda_0$ the primitive product
measure\cmmnt{ defined in 251C}.
By the {\bf c.l.d.\ product measure} on $X\times Y$ I shall mean the
function $\lambda:\dom\lambda_0\to[0,\infty]$ defined by setting

\Centerline{$\lambda W=\sup\{\lambda_0(W\cap(E\times F)):E\in\Sigma,\,
F\in\Tau,\,\mu E<\infty,\,\nu F<\infty\}$}

\noindent for $W\in\dom\lambda_0$.

\leader{251G}{Remark}\cmmnt{ I had better show at once
that} $\lambda$ is a
measure.   \prooflet{\Prf\ Of course its domain $\Lambda=\dom\lambda_0$
is a $\sigma$-algebra,
and $\lambda\emptyset=\lambda_0\emptyset=0$.   If $\sequencen{W_n}$ is a
disjoint sequence in $\Lambda$, then for any
$E\in\Sigma$, $F\in\Tau$ of finite measure

\Centerline{$\lambda_0(\bigcup_{n\in\Bbb N}W_n\cap(E\times F))
=\sum_{n=0}^{\infty}\lambda_0(W_n\cap(E\times F))
\le\sum_{n=0}^{\infty}\lambda W_n$,}

\noindent so
$\lambda(\bigcup_{n\in\Bbb N}W_n)\le\sum_{n=0}^{\infty}\lambda W_n$.
On the other hand, if
$a<\sum_{n=0}^{\infty}\lambda W_n$, then we can find $m\in\Bbb N$ and
$a_0,\ldots,a_m$ such that $a\le\sum_{n=0}^ma_n$ and $a_n<\lambda W_n$
for each $n\le m$;  now there are $E_0,\ldots,E_m\in\Sigma$ and
$F_0,\ldots,F_m\in\Tau$, all of finite measure, such that
$a_n\le\lambda_0(W_n\cap(E_n\times F_n))$ for each $n$.   Setting
$E=\bigcup_{n\le m}E_n$ and $F=\bigcup_{n\le m}F_n$, we have
$\mu E<\infty$ and $\nu F<\infty$, so
$$\eqalign{\lambda(\bigcup_{n\in\Bbb N}W_n)
&\ge\lambda_0(\bigcup_{n\in\Bbb N}W_n\cap(E\times F))
=\sum_{n=0}^{\infty}\lambda_0(W_n\cap(E\times F))\cr
&\ge\sum_{n=0}^m\lambda_0(W_n\cap(E_n\times F_n))
\ge\sum_{n=0}^ma_n
\ge a.\cr}$$

\noindent As $a$ is arbitrary, $\lambda(\bigcup_{n\in\Bbb
N}W_n)\ge\sum_{n=0}^{\infty}\lambda W_n$ and $\lambda(\bigcup_{n\in\Bbb
N}W_n)=\sum_{n=0}^{\infty}\lambda W_n$.   As $\sequencen{W_n}$ is
arbitrary, $\lambda$ is a measure.\ \Qed}

\leader{251H}{}\cmmnt{ We need a simple property of the measure
$\lambda_0$.

\medskip

\noindent}{\bf Lemma} Let $(X,\Sigma,\mu)$ and $(Y,\Tau,\nu)$ be two
measure spaces;  let $\lambda_0$ be the primitive product measure on
$X\times Y$, and $\Lambda$ its domain.   If $H\subseteq X\times Y$ and
$H\cap(E\times F)\in\Lambda$ whenever
$\mu E<\infty$ and $\nu F<\infty$, then $H\in\Lambda$.

\proof{ Let $\theta$ be the outer measure described in 251A.   Suppose
that $A\subseteq X\times Y$ and $\theta A<\infty$.   Let
$\epsilon>0$.   Let $\sequencen{E_n}$, $\sequencen{F_n}$ be sequences in
$\Sigma$, $\Tau$ respectively such that $A\subseteq\bigcup_{n\in\Bbb
N}E_n\times F_n$ and $\sum_{n=0}^{\infty}\mu E_n\cdot\nu
F_n\le\theta A+\epsilon$.
Now, for each $n$, the product of the measures $\mu E_n$, $\nu E_n$ is
finite, so either one is zero or
both are finite.   If $\mu E_n=0$ or $\nu F_n=0$ then of course

\Centerline{$\mu E_n\cdot\nu F_n=0
=\theta((E_n\times F_n)\cap H)+\theta((E_n\times F_n)\setminus H)$.}

\noindent If $\mu E_n<\infty$ and $\nu F_n<\infty$ then

$$\eqalignno{\mu E_n\cdot\nu F_n
&=\lambda_0(E_n\times F_n)\cr
&=\lambda_0((E_n\times F_n)\cap H)+\lambda_0((E_n\times F_n)\setminus
H)\cr
&=\theta((E_n\times F_n)\cap H)+\theta((E_n\times F_n)\setminus H).\cr
}$$

\noindent Accordingly, because $\theta$ is an outer measure,

$$\eqalign{\theta(A\cap H)+\theta(A\setminus H)
&\le\sum_{n=0}^{\infty}\theta((E_n\times F_n)\cap H)
  +\sum_{n=0}^{\infty}\theta((E_n\times F_n)\setminus H)\cr
&=\sum_{n=0}^{\infty}\mu E_n\cdot\nu F_n
\le\theta A+\epsilon.\cr}$$

\noindent As $\epsilon$ is arbitrary, $\theta(A\cap H)+\theta
(A\setminus H)\le\theta A$.   As $A$ is arbitrary, $H\in\Lambda$.
}

\leader{251I}{}\cmmnt{ Now for the fundamental properties of the
c.l.d.\ product measure.

\medskip

\noindent}{\bf Theorem} Let $(X,\Sigma,\mu)$ and $(Y,\Tau,\nu)$ be
measure spaces;  let $\lambda$ be the c.l.d.\ product measure on
$X\times Y$, and $\Lambda$ its domain.   Then

(a) $\Sigma\tensorhat\Tau\subseteq\Lambda$ and
$\lambda(E\times F)=\mu E\cdot\nu F$ whenever $E\in\Sigma$, $F\in\Tau$
and $\mu E\cdot\nu F<\infty$;

(b) for every $W\in\Lambda$ there is a $V\in\Sigma\tensorhat\Tau$ such
that $V\subseteq W$ and $\lambda V=\lambda W$;

(c) $(X\times Y,\Lambda,\lambda)$ is complete and locally determined,
and in fact is the c.l.d.\ version of
$(X\times Y,\Lambda,\lambda_0)$\cmmnt{ as described in 213D-213E};  in
particular, $\lambda W=\lambda_0W$ whenever $\lambda_0W<\infty$;

(d) if $W\in\Lambda$ and $\lambda W>0$ then there are $E\in\Sigma$,
$F\in\Tau$ such that $\mu E<\infty$, $\nu F<\infty$ and
$\lambda(W\cap(E\times F))>0$;

(e) if $W\in\Lambda$ and $\lambda W<\infty$, then for every $\epsilon>0$
there are $E_0,\ldots,E_n\in\Sigma$, $F_0,\ldots,F_n\in\Tau$, all of
finite measure, such that
$\lambda(W\symmdiff\bigcup_{i\le n}(E_i\times F_i))\le\epsilon$.

\proof{ Take $\theta$ to be the outer measure of 251A and
$\lambda_0$ the primitive product measure of 251C.  Set
$\Sigma^f=\{E:E\in\Sigma,\,\mu E<\infty\}$ and
$\Tau^f=\{F:F\in\Tau,\,\nu F<\infty\}$.

\medskip

{\bf (a)} By 251E, $\Sigma\tensorhat\Tau\subseteq\Lambda$.   If
$E\in\Sigma$ and $F\in\Tau$ and $\mu E\cdot\nu F<\infty$, either
$\mu E\cdot\nu F=0$ and $\lambda(E\times F)=\lambda_0(E\times F)=0$ or
both $\mu E$ and $\nu F$ are finite and again
$\lambda(E\times F)=\lambda_0(E\times F)=\mu E\cdot\nu F$.

\medskip

{\bf (b)(i)} Take any $a<\lambda W$.   Then there are $E\in\Sigma^f$,
$F\in\Tau^f$ such that $\lambda_0(W\cap(E\times F))>a$ (251F);  now

$$\eqalign{\theta((E\times F)\setminus W)
&=\lambda_0((E\times F)\setminus W)\cr
&=\lambda_0(E\times F)-\lambda_0(W\cap(E\times F))
<\lambda_0(E\times F)-a.\cr}$$

\noindent Let $\sequencen{E_n}$, $\sequencen{F_n}$ be sequences in
$\Sigma$, $\Tau$ respectively such that $(E\times F)\setminus
W\subseteq\bigcup_{n\in\Bbb N}E_n\times F_n$ and $\sum_{n=0}^{\infty}\mu
E_n\cdot\nu F_n\le\lambda_0(E\times F)-a$.   Consider

\Centerline{$V=(E\times
F)\setminus\bigcup_{n\in\Bbb N}E_n\times F_n\in\Sigma\tensorhat\Tau$;}

\noindent then $V\subseteq W$, and

$$\eqalignno{\lambda V=\lambda_0V
&=\lambda_0(E\times F)-\lambda_0((E\times F)\setminus V)\cr
&\ge\lambda_0(E\times F)-\lambda_0(\bigcup_{n\in\Bbb N}E_n\times F_n)\cr
\noalign{\noindent (because $(E\times F)\setminus
V\subseteq\bigcup_{n\in\Bbb N}E_n\times F_n$)}
&\ge\lambda_0(E\times F)-\sum_{n=0}^{\infty}\mu E_n\cdot\nu F_n
\ge a\cr}$$

\noindent (by the choice of the $E_n$, $F_n$).

\medskip

\quad{\bf (ii)} Thus for every $a<\lambda W$ there is a
$V\in\Sigma\tensorhat\Tau$ such that $V\subseteq W$ and $\lambda V\ge
a$.   Now choose a sequence $\sequencen{a_n}$ strictly increasing to
$\lambda W$, and for each $a_n$ a corresponding $V_n$;  then
$V=\bigcup_{n\in\Bbb N}V_n$ belongs to the $\sigma$-algebra
$\Sigma\tensorhat\Tau$, is included in $W$, and has measure at least
$\sup_{n\in\Bbb N}\lambda V_n$ and at most $\lambda W$;  so $\lambda
V=\lambda W$, as required.

\medskip

{\bf (c)(i)} If $H\subseteq X\times Y$ is $\lambda$-negligible, there is
a $W\in\Lambda$ such that $H\subseteq W$ and $\lambda W=0$.   If
$E\in\Sigma$, $F\in\Tau$ are of finite measure,
$\lambda_0(W\cap(E\times  F))=0$;  but $\lambda_0$, being derived from
the outer measure $\theta$
by \Caratheodory's method, is complete (212A), so
$H\cap(E\times F)\in\Lambda$ and $\lambda_0(H\cap(E\times F))=0$.
Because $E$ and $F$
are arbitrary, $H\in\Lambda$, by 251H.   As $H$ is arbitrary, $\lambda$
is complete.

\medskip

\quad{\bf (ii)} If $W\in\Lambda$ and $\lambda W=\infty$, then there must
be $E\in\Sigma$, $F\in\Tau$ such that $\mu E<\infty$, $\nu F<\infty$ and
$\lambda_0(W\cap(E\times F))>0$;  now

\Centerline{$0<\lambda(W\cap(E\times F))\le\mu E\cdot\nu F<\infty$.}

\noindent Thus $\lambda$ is semi-finite.

\medskip

\quad{\bf (iii)} If $H\subseteq X\times Y$ and $H\cap W\in\Lambda$
whenever $\lambda W<\infty$, then, in particular, $H\cap(E\times
F)\in\Lambda$ whenever $\mu E<\infty$ and $\nu F<\infty$;  by 251H
again, $H\in\Lambda$.   Thus $\lambda$ is locally determined.

\medskip

\quad{\bf (iv)} If $W\in\Lambda$ and $\lambda_0W<\infty$, then we have
sequences $\sequencen{E_n}$ in $\Sigma$, $\sequencen{F_n}$ in $\Tau$
such
that $W\subseteq\bigcup_{n\in\Bbb N}(E_n\times F_n)$ and
$\sum_{n=0}^{\infty}\mu E_n\cdot\nu F_n<\infty$.   Set

\Centerline{$I=\{n:\mu E_n=\infty\}$,
\quad$J=\{n:\nu F_n=\infty\}$,
\quad$K=\Bbb N\setminus (I\cup J)$;}

\noindent then $\nu(\bigcup_{n\in I}F_n)=\mu(\bigcup_{n\in J}E_n)=0$, so
$\lambda_0(W\setminus W')=0$, where

\Centerline{$W'=W\cap\bigcup_{n\in K}(E_n\times F_n)
\supseteq W\setminus((\bigcup_{n\in J}E_n\times Y)
  \cup(X\times\bigcup_{n\in I}F_n))$.}

\noindent Now set $E'_n=\bigcup_{i\in K,i\le n}E_i$, $F'_n=\bigcup_{i\in
K,i\le n}F_i$ for each $n$.   We have $W'=\bigcup_{n\in\Bbb
N}W'\cap(E'_n\times F'_n)$, so

\Centerline{$\lambda
W\le\lambda_0W=\lambda_0W'=\lim_{n\to\infty}\lambda_0(W'\cap(E'_n\times
F'_n))
\le \lambda W'
\le\lambda W$,}

\noindent and $\lambda W=\lambda_0W$.

\medskip

\quad{\bf (v)} Following the terminology of 213D, let us write

\Centerline{$\tilde\Lambda=\{W:W\subseteq X\times Y,\,W\cap V\in\Lambda$
whenever $V\in\Lambda$ and $\lambda_0V<\infty\}$,}

\Centerline{$\tilde\lambda W
=\sup\{\lambda_0(W\cap V):V\in\Lambda$, $\lambda_0V<\infty\}$.}

\noindent Because $\lambda_0(E\times F)<\infty$ whenever $\mu E<\infty$
and $\nu F<\infty$, $\tilde\Lambda\subseteq\Lambda$ and
$\tilde\Lambda=\Lambda$.


Now for any $W\in\Lambda$ we have

$$\eqalignno{\tilde\lambda W
&=\sup\{\lambda_0(W\cap V):V\in\Lambda,\,\lambda_0V<\infty\}\cr
&\ge\sup\{\lambda_0(W\cap (E\times F)):E\in\Sigma^f,\,F\in\Tau^f\}\cr
&=\lambda W\cr
&\ge\sup\{\lambda(W\cap V):V\in\Lambda,\,\lambda_0V<\infty\}\cr
&=\sup\{\lambda_0(W\cap V):V\in\Lambda,\,\lambda_0V<\infty\},\cr}$$

\noindent using (iv) just above, so that $\lambda=\tilde\lambda$ is the
c.l.d.\ version of $\lambda_0$.

\medskip

{\bf (d)} If $W\in\Lambda$ and $\lambda W>0$, there are
$E\in\Sigma^f$ and $F\in\Tau^f$ such that
$\lambda(W\cap(E\times F))=\lambda_0(W\cap(E\times F))>0$.

\medskip

{\bf (e)} There are $E\in\Sigma^f$, $F\in\Tau^f$ such that
$\lambda_0(W\cap(E\times F))\ge\lambda W-\bover13\epsilon$;   set
$V_1=W\cap (E\times F)$;  then

\Centerline{$\lambda(W\setminus V_1)=\lambda W-\lambda V_1
=\lambda W-\lambda_0V_1\le\Bover13\epsilon$.}

\noindent   There
are sequences $\sequencen{E'_n}$ in $\Sigma$, $\sequencen{F'_n}$ in
$\Tau$ such that $V_1\subseteq\bigcup_{n\in\Bbb
N}E'_n\times F'_n$ and $\sum_{n=0}^{\infty}\mu E'_n\cdot\nu
F'_n\le\lambda_0V_1+\bover13\epsilon$.   Replacing $E'_n$, $F'_n$ by
$E'_n\cap E$, $F'_n\cap F$ if necessary, we may suppose that
$E'_n\in\Sigma^f$ and $F'_n\in\Tau^f$ for every $n$.   Set
$V_2=\bigcup_{n\in\Bbb N}E'_n\times F'_n$;  then

\Centerline{$\lambda(V_2\setminus V_1)
\le\lambda_0(V_2\setminus V_1)
\le\sum_{n=0}^{\infty}\mu E'_n\cdot\nu F'_n-\lambda_0V_1
\le\Bover13\epsilon$.}

\noindent   Let $m\in\Bbb N$ be such that
$\sum_{n=m+1}^{\infty}\mu E'_n\cdot\nu F'_n\le\bover13\epsilon$, and set

\Centerline{$V=\bigcup_{n=0}^mE'_n\times F'_n$.}

\noindent Then

\Centerline{$\lambda(V_2\setminus V)
\le\sum_{n=m+1}^{\infty}\mu E'_n\cdot\nu F'_n\le\Bover13\epsilon$.}

Putting these together, we have
$W\symmdiff V\subseteq(W\setminus V_1)\cup(V_2\setminus
V_1)\cup(V_2\setminus V)$, so

\Centerline{$\lambda(W\symmdiff V)
\le\lambda(W\setminus V_1)+\lambda(V_2\setminus
V_1)+\lambda(V_2\setminus
V)\le\Bover13\epsilon+\Bover13\epsilon+\Bover13\epsilon=\epsilon$.}

\noindent And $V$ is of the required form.
}

\leader{251J}{Proposition} If $(X,\Sigma,\mu)$ and $(Y,\Tau,\nu)$ are
semi-finite measure spaces and $\lambda$ is the c.l.d.\ product measure
on $X\times Y$, then $\lambda(E\times F)=\mu E\cdot\nu F$ for all
$E\in\Sigma$, $F\in\Tau$.

\proof{ Setting $\Sigma^f=\{E:E\in\Sigma,\,\mu E<\infty\}$,
$\Tau^f=\{F:F\in\Tau,\,\nu F<\infty\}$, we have

$$\eqalign{\lambda(E\times F)
&=\sup\{\lambda_0((E\cap E_0)\times(F\cap F_0)):
  E_0\in\Sigma^f,\,F_0\in\Tau^f\}\cr
&=\sup\{\mu(E\cap E_0)\cdot\nu(F\cap F_0)):
  E_0\in\Sigma^f,\,F_0\in\Tau^f\}\cr
&=\sup\{\mu(E\cap E_0):E_0\in\Sigma^f\}\cdot
  \sup\{\nu(F\cap F_0):F_0\in\Tau^f\}
=\mu E\cdot\nu F\cr}$$

\noindent (using 213A).
}%end of proof of 251J

\leader{251K}{$\sigma$-finite \dvrocolon{spaces}}\cmmnt{ Of course
most of the measure spaces we shall apply these results to are
$\sigma$-finite, and in this case there are some useful simplifications.

\medskip

\noindent}{\bf Proposition} Let $(X,\Sigma,\mu)$ and $(Y,\Tau,\nu)$ be
$\sigma$-finite measure spaces.   Then the c.l.d.\ product measure on
$X\times Y$ is equal to the primitive product measure, and is the
completion of its restriction to $\Sigma\tensorhat\Tau$;  moreover, this
common product measure is $\sigma$-finite.

\proof{ Write $\lambda_0$, $\lambda$ for the primitive and c.l.d.\
product measures, as usual, and $\Lambda$ for their domain.   Let
$\sequencen{E_n}$, $\sequencen{F_n}$ be non-decreasing sequences of sets
of finite measure covering $X$, $Y$ respectively (see 211D).

\medskip

{\bf (a)} For each $n\in\Bbb N$, $\lambda(E_n\times F_n)=\mu E_n\cdot\nu
F_n$ is finite, by 251Ia.   Since $X\times Y=\bigcup_{n\in\Bbb
N}E_n\times F_n$, $\lambda$ is $\sigma$-finite.

\medskip

{\bf (b)} For any $W\in\Lambda$,

\Centerline{$\lambda_0W=\lim_{n\to\infty}\lambda_0(W\cap(E_n\times
F_n))=\lim_{n\to\infty}\lambda(W\cap(E_n\times F_n))=\lambda W$.}

\noindent So $\lambda=\lambda_0$.

\medskip

{\bf (c)} Write $\lambda_{\Cal B}$ for the restriction of
$\lambda=\lambda_0$ to $\Sigma\tensorhat\Tau$, and
$\hat\lambda_{\Cal B}$ for its completion.

\medskip

\quad{\bf (i)} Suppose that $W\in\dom\hat\lambda_{\Cal B}$.   Then there
are $W'$, $W''\in\Sigma\tensorhat\Tau$ such that
$W'\subseteq W\subseteq W''$ and $\lambda_{\Cal B}(W''\setminus W')=0$
(212C).   In this case,
$\lambda(W''\setminus W')=0$;  as $\lambda$ is complete, $W\in\Lambda$
and

\Centerline{$\lambda W=\lambda W'=\lambda_{\Cal B}W'
=\hat\lambda_{\Cal B}W$.}

\noindent Thus $\lambda$ extends $\hat\lambda_{\Cal B}$.

\medskip

\quad{\bf (ii)} If $W\in\Lambda$, then there is a
$V\in\Sigma\tensorhat\Tau$ such that $V\subseteq W$ and
$\lambda(W\setminus V)=0$.   \Prf\ For each $n\in\Bbb N$ there is a
$V_n\in\Sigma\tensorhat\Tau$ such that
$V_n\subseteq W\cap(E_n\times F_n)$ and
$\lambda V_n=\lambda(W\cap(E_n\times F_n))$ (251Ib).   But as
$\lambda(E_n\times F_n)=\mu E_n\cdot\nu F_n$ is finite, this means that
$\lambda(W\cap(E_n\times F_n)\setminus V_n)=0$.   So if we set
$V=\bigcup_{n\in\Bbb N}V_n$, we shall have $V\in\Sigma\tensorhat\Tau$,
$V\subseteq W$ and

\Centerline{$W\setminus V
=\bigcup_{n\in\Bbb N}W\cap(E_n\times F_n)\setminus V
\subseteq\bigcup_{n\in\Bbb N}W\cap(E_n\times F_n)\setminus V_n$}

\noindent is $\lambda$-negligible.\ \Qed

Similarly, there is a $V'\in\Sigma\tensorhat\Tau$ such that
$V'\subseteq(X\times Y)\setminus W$ and
$\lambda(((X\times Y)\setminus W)\setminus V')=0$.   Setting $V''=(X\times Y)\setminus V'$,
$V''\in\Sigma\tensorhat\Tau$, $W\subseteq V''$ and
$\lambda(V''\setminus W)=0$.   So

\Centerline{$\lambda_{\Cal B}(V''\setminus V)
=\lambda(V''\setminus V)
=\lambda(V''\setminus W)+\lambda(W\setminus V)=0$,}

\noindent and $W$ is measured by $\hat\lambda_{\Cal B}$, with
$\hat\lambda_{\Cal B}W=\lambda_{\Cal B}V=\lambda W$.   As $W$ is
arbitrary, $\hat\lambda_{\Cal B}=\lambda$.
}%end of proof of 251K

\leader{*251L}{}\cmmnt{ The following result fits in naturally here;
I star it because it will be used seldom (there is a more important version
of the same idea in 254G) and the proof
can be skipped until you come to need it.

\medskip

\noindent}{\bf Proposition}
Let $(X_1,\Sigma_1,\mu_1)$, $(X_2,\Sigma_2,\mu_2)$,
$(Y_1,\Tau_1,\nu_1)$ and $(Y_2,\Tau_2,\nu_2)$ be $\sigma$-finite
measure spaces;  let
$\lambda_1$, $\lambda_2$ be the product measures on $X_1\times Y_1$,
$X_2\times Y_2$ respectively.   Suppose that $f:X_1\to X_2$ and
$g:Y_1\to Y_2$ are \imp\ functions, and that
$h(x,y)=(f(x),g(y))$ for $x\in X_1$, $y\in Y_1$.   Then $h$ is \imp.
%251K

\proof{ Write $\Lambda_1$, $\Lambda_2$ for the domains of $\lambda_1$,
$\lambda_2$ respectively.

\medskip

{\bf (a)} Suppose that $E\in\Sigma_2$ and $F\in\Tau_2$ have finite
measure.   Then $\lambda_1h^{-1}[W\cap(E\times F)]$ is defined and equal to
$\lambda_2(W\cap(E\times F))$ for every $W\in\Lambda_2$.   \Prf\

$$\eqalign{\lambda_1h^{-1}[E\times F]
&=\lambda_1(f^{-1}[E]\times g^{-1}[F])
=\mu_1f^{-1}[E]\cdot\nu_1g^{-1}[F]\cr
&=\mu_2E\cdot\nu_2F
=\lambda_2(E\times F)\cr}$$

\noindent by 251E/251J.\ \Qed

\medskip

{\bf (b)} Take $E_0\in\Sigma_2$ and $F_0\in\Tau_2$ of finite measure.
Let $\tilde\lambda_1$, $\tilde\lambda_2$ be the subspace measures on
$f^{-1}[E_0]\times g^{-1}[F_0]$ and $E_0\times F_0$ respectively.
Set $\tilde h=h\restr f^{-1}[E_0]\times g^{-1}[F_0]$, and write
$\iota$ for the identity map from $E_0\times F_0$ to $X_2\times Y_2$;  let
$\lambda=\tilde\lambda_1\tilde h^{-1}$ and
$\lambda'=\tilde\lambda_2\iota^{-1}$ be the
image measures on $X_2\times Y_2$.   Then (a) tells us that

$$\eqalign{\lambda(E\times F)
&=\lambda_1(h^{-1}[(E\cap E_0)\times(F\cap F_0)])\cr
&=\lambda_2((E\cap E_0)\times(F\cap F_0))
=\lambda'(E\times F)\cr}$$

\noindent whenever $E\in\Sigma_2$ and $F\in\Tau_2$.   By the Monotone Class
Theorem (136C), $\lambda$ and $\lambda'$ agree on
$\Sigma_2\tensorhat\Tau_2$, that is,
$\lambda_1(h^{-1}[W\cap(E_0\times F_0)])=\lambda_2(W\cap(E_0\times F_0))$
for every $W\in\Sigma_2\tensorhat\Tau_2$.

If $W$ is any member of $\Lambda_2$,
there are $W'$, $W''\in\Sigma_2\tensorhat\Tau_2$ such
that $W'\subseteq W\subseteq W''$ and $\lambda_2(W''\setminus W')=0$
(251K).   Now we must have

\Centerline{$h^{-1}[W'\cap(E_0\times F_0)]
\subseteq h^{-1}[W\cap(E_0\times F_0)]
\subseteq h^{-1}[W''\cap(E_0\times F_0)]$,}

\Centerline{$\lambda_1(h^{-1}[W''\cap(E_0\times F_0)]\setminus
h^{-1}[W'\cap(E_0\times F_0)])
=\lambda_2((W''\setminus W')\cap(E_0\times F_0))=0$;}

\noindent because $\lambda_1$ is complete,
$\lambda_1h^{-1}[W\cap(E_0\times F_0)]$ is defined and equal to

\Centerline{$\lambda_1h^{-1}[W'\cap(E_0\times F_0)]
=\lambda_2(W'\cap(E_0\times F_0))
=\lambda_2(W\cap(E_0\times F_0))$.}

\medskip

{\bf (c)} Now suppose that $\sequencen{E_n}$, $\sequencen{F_n}$ are
non-decreasing sequences of sets of finite measure with union $X_2$, $Y_2$
respectively.   If $W\in\Lambda_2$,

\Centerline{$\lambda_1h^{-1}[W]
=\sup_{n\in\Bbb N}\lambda_1h^{-1}[W\cap(E_n\times F_n)]
=\sup_{n\in\Bbb N}\lambda_2(W\cap(E_n\times F_n))
=\lambda_2W$.}

\noindent So $h$ is \imp, as claimed.
}%end of proof of 251L

\leader{251M}{}\cmmnt{ It is time that I gave some examples.   Of
course the central
example is Lebesgue measure.   In this case we have the only reasonable
result.   I pause to describe the leading example of the product
$\Sigma\tensorhat\Tau$ introduced in 251D.

\medskip

\noindent}{\bf Proposition} Let $r$, $s\ge 1$ be integers.   Then we
have a natural bijection $\phi:\BbbR^r\times\BbbR^s\to\BbbR^{r+s}$,
defined by setting

\Centerline{$\phi((\xi_1,\ldots,\xi_r),(\eta_1,\ldots,\eta_s))
=(\xi_1,\ldots,\xi_r,\eta_1,\ldots,\eta_s)$}

\noindent for $\xi_1,\ldots,\xi_r,\eta_1,\ldots,\eta_s\in\Bbb R$.   If
we write $\Cal B_r$, $\Cal B_s$ and $\Cal B_{r+s}$ for the Borel
$\sigma$-algebras of $\BbbR^r$, $\BbbR^s$ and $\BbbR^{r+s}$
respectively, then $\phi$ identifies
$\Cal B_{r+s}$ with $\Cal B_r\tensorhat\Cal B_s$.

\proof{{\bf (a)} Write $\Cal B$ for the $\sigma$-algebra
$\{\phi^{-1}[W]:W\in\Cal B_{r+s}\}$ copied onto $\BbbR^r\times\BbbR^s$
by the bijection $\phi$;   we are seeking to prove that
$\Cal B=\Cal B_r\tensorhat\Cal B_s$.   We have maps
$\pi_1:\BbbR^{r+s}\to\BbbR^r$, $\pi_2:\BbbR^{r+s}\to\BbbR^s$ defined by
setting $\pi_1(\phi(x,y))=x$, $\pi_2(\phi(x,y))=y$.   Each co-ordinate
of $\pi_1$ is continuous, therefore Borel measurable (121Db), so
$\pi_1^{-1}[E]\in\Cal B_{r+s}$ for every $E\in\Cal B_r$, by 121K.
Similarly,
$\pi_2^{-1}[F]\in\Cal B_{r+s}$ for every $F\in\Cal B_s$.   So
$\phi[E\times F]=\pi_1^{-1}[E]\cap\pi_1^{-1}[F]$ belongs to
$\Cal B_{r+s}$, that is, $E\times F\in\Cal B$, whenever $E\in\Cal B_r$
and $F\in\Cal B_s$.   Because $\Cal B$ is a
$\sigma$-algebra, $\Cal B_r\tensorhat\Cal B_s\subseteq\Cal B$.

\medskip

{\bf (b)} Now examine sets of the form

\Centerline{$\{(x,y):\xi_i\le\alpha\}
=\{x:\xi_i\le\alpha\}\times\BbbR^s$,}

\Centerline{$\{(x,y):\eta_j\le\alpha\}
=\BbbR^r\times\{y:\eta_j\le\alpha\}$}

\noindent for $\alpha\in\Bbb R$, $i\le r$ and $j\le s$, taking
$x=(\xi_1,\ldots,\xi_r)$ and $y=(\eta_1,\ldots,\eta_s)$.   All of these
belong to $\Cal B_r\tensorhat\Cal B_s$.   But the $\sigma$-algebra they
generate is just $\Cal B$, by 121J.   So
$\Cal B\subseteq\Cal B_r\tensorhat\Cal B_s$ and
$\Cal B=\Cal B_r\tensorhat\Cal B_s$.
}%end of proof of 251M

\leader{251N}{Theorem}  Let $r$, $s\ge 1$ be integers.   Then the
bijection $\phi:\BbbR^r\times\BbbR^s\to\BbbR^{r+s}$ described in 251M
identifies Lebesgue measure on $\BbbR^{r+s}$ with the
c.l.d.\ product $\lambda$
of Lebesgue measure on $\BbbR^r$ and Lebesgue measure on $\BbbR^s$.

\proof{ Write $\mu_r$,
$\mu_s$, $\mu_{r+s}$ for the three versions of Lebesgue measure,
$\mu^*_r$, $\mu^*_s$ and $\mu^*_{r+s}$ for the corresponding outer
measures, and
$\theta$ for the outer measure on $\BbbR^r\times\BbbR^s$ derived from
$\mu_r$ and $\mu_s$ by the formula of 251A.

\medskip

{\bf (a)} If $I\subseteq\BbbR^r$ and $J\subseteq \BbbR^s$ are
half-open intervals, then
$\phi[I\times J]\subseteq\BbbR^{r+s}$ is also a half-open interval, and

\Centerline{$\mu_{r+s}(\phi[I\times J])=\mu_rI\cdot\mu_sJ$;}

\noindent this is immediate from the definition of the Lebesgue measure
of an interval.   (I speak of `half-open' intervals here, that is,
intervals of the form
$\prod_{1\le j\le r}\coint{\alpha_j,\beta_j}$, because I used them in the
definition of Lebesgue measure in \S115.   If you prefer
to work with open intervals or closed intervals it makes no difference.)
Note also that every half-open interval in $\BbbR^{r+s}$ is expressible
as $\phi[I\times J]$ for suitable $I$, $J$.

\medskip

{\bf (b)} For any $A\subseteq\BbbR^{r+s}$,
$\theta(\phi^{-1}[A])\le\mu^*_{r+s}(A)$.   \Prf\ For any $\epsilon>0$,
there is a sequence $\sequencen{K_n}$ of half-open intervals in $\Bbb
R^{r+s}$ such that $A\subseteq\bigcup_{n\in\Bbb N}K_n$ and
$\sum_{n=0}^{\infty}\mu_{r+s}(K_n)\le\mu^*_{r+s}(A)+\epsilon$.
Express each $K_n$ as $\phi[I_n\times J_n]$, where $I_n$ and $J_n$ are
half-open intervals in $\BbbR^r$ and $\BbbR^s$ respectively;  then
$\phi^{-1}[A]\subseteq\bigcup_{n\in\Bbb N}I_n\times J_n$, so that

\Centerline{$\theta(\phi^{-1}[A])
\le\sum_{n=0}^{\infty}\mu_rI_n\cdot\mu_sJ_n
=\sum_{n=0}^{\infty}\mu_{r+s}(K_n)\le\mu^*_{r+s}(A)+\epsilon$.}

\noindent As $\epsilon$ is arbitrary, we have the result.\ \Qed

\medskip

{\bf (c)} If $E\subseteq\BbbR^r$ and $F\subseteq\BbbR^s$ are
measurable, then $\mu^*_{r+s}(\phi[E\times F])\le\mu_rE\cdot\mu_sF$.

\medskip

\Prf\ {\bf (i)} Consider first the case $\mu_rE<\infty$, $\mu_sF<\infty$.   In
this case, given $\epsilon>0$, there are sequences $\sequencen{I_n}$,
$\sequencen{J_n}$ of half-open intervals such that
$E\subseteq\bigcup_{n\in\Bbb N}I_n$,
$F\subseteq\bigcup_{n\in\Bbb N}F_n$,

\Centerline{$\sum_{n=0}^{\infty}\mu_rI_n\le\mu^*_rE+\epsilon
=\mu_rE+\epsilon$,}

\Centerline{$\sum_{n=0}^{\infty}\mu_sJ_n\le\mu^*_sF+\epsilon=\mu_sF
+\epsilon$.}

\noindent   Accordingly
$E\times F\subseteq\bigcup_{m,n\in\Bbb N}I_m\times J_n$ and
$\phi[E\times F]\subseteq\bigcup_{m,n\in\Bbb N}\phi[I_m\times J_n]$, so
that

$$\eqalign{\mu^*_{r+s}(\phi[E\times F])
&\le\sum_{m,n=0}^{\infty}\mu_{r+s}(\phi[I_m\times J_n])
=\sum_{m,n=0}^{\infty}\mu_rI_m\cdot\mu_sJ_n\cr
&=\sum_{m=0}^{\infty}\mu_rI_m\cdot\sum_{n=0}^{\infty}\mu_sJ_n
\le(\mu_rE+\epsilon)(\mu_sF+\epsilon).\cr}$$

\noindent As $\epsilon$ is arbitrary,
we have the result.

\medskip

\quad{\bf (ii)}
Next, if $\mu_rE=0$, there is a sequence $\sequencen{F_n}$ of sets
of finite measure covering $\BbbR^s\supseteq F$, so that

\Centerline{$\mu^*_{r+s}(\phi[E\times F])
\le\sum_{n=0}^{\infty}\mu^*_{r+s}(\phi[E\times F_n])
\le\sum_{n=0}^{\infty}\mu_rE\cdot\mu_sF_n
=0
=\mu_rE\cdot\mu_sF$.}

\noindent Similarly,
$\mu^*_{r+s}(\phi[E\times F])\le\mu_rE\cdot\mu_sF$ if $\mu_sF=0$.
The only remaining case is that in
which both of $\mu_rE$, $\mu_sF$ are strictly positive and one is
infinite;  but in this case $\mu_rE\cdot\mu_sF=\infty$, so surely
$\mu^*_{r+s}(\phi[E\times F])\le\mu_rE\cdot\mu_sF$.\ \Qed

\medskip

{\bf (d)} If $A\subseteq\BbbR^{r+s}$, then
$\mu^*_{r+s}(A)\le\theta(\phi^{-1}[A])$.   \Prf\ Given $\epsilon>0$,
there are sequences $\sequencen{E_n}$, $\sequencen{F_n}$ of measurable
sets in $\BbbR^r$, $\BbbR^s$ respectively such that
$\phi^{-1}[A]\subseteq\bigcup_{n\in\Bbb N}E_n\times F_n$ and
$\sum_{n=0}^{\infty}\mu_rE_n\cdot\mu_sF_n
\le\theta(\phi^{-1}[A])+\epsilon$.
Now $A\subseteq\bigcup_{n\in\Bbb N}\phi[E_n\times F_n]$, so

\Centerline{$\mu^*_{r+s}(A)
\le\sum_{n=0}^{\infty}\mu^*_{r+s}(\phi[E_n\times F_n])
\le\sum_{n=0}^{\infty}\mu_rE_n\cdot\mu_sF_n
\le\theta(\phi^{-1}[A])+\epsilon$.}

\noindent As $\epsilon$ is arbitrary, we have the result.\ \Qed

\medskip

{\bf (e)} Putting (c) and (d) together,  we have
$\theta(\phi^{-1}[A])=\mu^*_{r+s}(A)$ for every $A\subseteq\Bbb
R^{r+s}$.    Thus $\theta$ on $\BbbR^r\times\BbbR^s$ corresponds
exactly to $\mu^*_{r+s}$ on $\BbbR^{r+s}$.   So the associated
measures $\lambda_0$, $\mu_{r+s}$ must correspond in the same way,
writing $\lambda_0$ for the primitive product measure.   But 251K tells
us that $\lambda_0=\lambda$, so we have the result.
}%end of proof of 251N

\vleader{48pt}{251O}{}\cmmnt{ In fact, a large proportion of the
applications
of the constructions here are to subspaces of Euclidean space, rather
than to the whole product $\BbbR^r\times\BbbR^s$.   It would not have
been especially difficult to write 251N out to deal with
arbitrary subspaces, but I prefer to give a more general description of
the product of subspace measures, as I feel that it illuminates the
method.   I start with a straightforward result on strictly localizable
spaces.

\medskip

\noindent}{\bf Proposition}  Let $(X,\Sigma,\mu)$ and $(Y,\Tau,\nu)$ be
strictly localizable measure spaces.   Then the c.l.d.\ product measure
on $X\times Y$ is strictly localizable;  moreover, if
$\langle X_i\rangle_{i\in I}$ and $\langle Y_j\rangle_{j\in J}$ are
decompositions of $X$ and $Y$ respectively,
$\langle X_i\times Y_j\rangle_{(i,j)\in I\times J}$ is a decomposition of
$X\times Y$.

\proof{ Let $\langle X_i\rangle_{i\in I}$ and
$\langle Y_j\rangle_{j\in J}$ be decompositions of $X$, $Y$ respectively.
Then $\langle X_i\times Y_j\rangle_{(i,j)\in I\times J}$ is a partition of
$X\times Y$ into measurable sets of finite measure.   If
$W\subseteq X\times Y$ and $\lambda W>0$, there are sets $E\in\Sigma$,
$F\in\Tau$ such that $\mu E<\infty$, $\nu F<\infty$ and
$\lambda(W\cap(E\times F))>0$.   We know that
$\mu E=\sum_{i\in I}\mu(E\cap X_i)$ and
$\mu F=\sum_{j\in J}\mu(F\cap Y_j)$, so there must be finite sets
$I_0\subseteq I$, $J_0\subseteq J$ such that

\Centerline{$\mu E\cdot\nu F
  -(\sum_{i\in I_0}\mu(E\cap X_i))(\sum_{j\in J_0}\nu(F\cap Y_j))
<\lambda(W\cap(E\times F))$.}

\noindent Setting $E'=\bigcup_{i\in I_0}X_i$ and $F'=\bigcup_{j\in J_0}Y_j$
we have

\Centerline{$\lambda((E\times F)\setminus(E'\times F'))
=\lambda(E\times F)-\lambda((E\cap E')\times(F\cap
F'))<\lambda(W\cap(E\times F))$,}

\noindent so that $\lambda(W\cap(E'\times F'))>0$.   There must
therefore be some $i\in I_0$, $j\in J_0$ such that
$\lambda(W\cap(X_i\times Y_j))>0$.

This shows that $\{X_i\times Y_j:i\in I,\,j\in J\}$ satisfies the
criterion of 213O, so that $\lambda$, being complete and locally
determined, must be strictly localizable.   Because
$\langle X_i\times Y_j\rangle_{(i,j)\in I\times J}$ covers $X\times Y$,
it is actually a decomposition of $X\times Y$ (213Ob).
}

\leader{251P}{Lemma} Let $(X,\Sigma,\mu)$ and $(Y,\Tau,\nu)$ be measure
spaces, and $\lambda$ the c.l.d.\ product measure on
$X\times Y$.   Let $\lambda^*$ be the corresponding
outer measure\cmmnt{ (132B)}.   Then

\Centerline{$\lambda^*C=\sup\{\theta(C\cap(E\times
F)):E\in\Sigma,\,F\in\Tau,\,\mu E<\infty,\,\nu F<\infty\}$}

\noindent for every
$C\subseteq X\times Y$, where $\theta$ is the outer measure of 251A.

\proof{ Write $\Lambda$ for the domain of $\lambda$, $\Sigma^f$ for
$\{E:E\in\Sigma,\,\mu E<\infty\}$, $\Tau^f$ for $\{F:F\in\Tau,\,\nu
F<\infty\}$;  set $u=\sup\{\theta(C\cap(E\times F)):E\in\Sigma^f$,
$F\in\Tau^f\}$.

\medskip

{\bf (a)} If $C\subseteq W\in\Lambda$, $E\in\Sigma^f$ and $F\in\Tau^f$,
then

$$\eqalignno{\theta(C\cap(E\times F))
&\le\theta(W\cap(E\times F))
=\lambda_0(W\cap(E\times F))\cr
\noalign{\noindent (where $\lambda_0$ is the primitive product measure)}
&\le\lambda W.\cr}$$

\noindent As $E$ and $F$ are arbitrary, $u\le\lambda W$;  as $W$ is
arbitrary, $u\le\lambda^*C$.

\medskip

{\bf (b)} If $u=\infty$, then of course $\lambda^*C=u$.   Otherwise, let
$\sequencen{E_n}$, $\sequencen{F_n}$ be
sequences in $\Sigma^f$, $\Tau^f$ respectively such that

\Centerline{$u=\sup_{n\in\Bbb N}\theta(C\cap(E_n\times F_n))$.}

\noindent Consider $C'=C\setminus(\bigcup_{n\in\Bbb
N}E_n\times\bigcup_{n\in\Bbb
N}F_n)$.   If $E\in\Sigma^f$ and $F\in\Tau^f$, then for every $n\in\Bbb
N$ we have

$$\eqalignno{u
&\ge\theta(C\cap((E\cup E_n)\times(F\cup F_n)))\cr
&=\theta(C\cap((E\cup E_n)\times(F\cup F_n))\cap(E_n\times F_n))\cr
&{\hskip 10em}+\theta(C\cap((E\cup E_n)\times(F\cup F_n))
  \setminus(E_n\times F_n))\cr
\noalign{\noindent (because $E_n\times F_n\in\Lambda$, by 251E)}
&\ge\theta(C\cap(E_n\times F_n))
  +\theta(C'\cap(E\times F)).\cr}$$

\noindent  Taking the supremum of the right-hand expression as $n$
varies, we have $u\ge u+\theta(C'\cap(E\times F))$ so

\Centerline{$\lambda(C'\cap(E\times F))=\theta(C'\cap(E\times F))=0$.}

\noindent As $E$ and $F$ are arbitrary, $\lambda C'=0$.

But this means that

$$\eqalignno{\lambda^*C
&\le\lambda^*
  (C\cap(\bigcup_{n\in\Bbb N}E_n\times\bigcup_{n\in\Bbb N}F_n))
  +\lambda^*C'\cr
&=\lim_{n\to\infty}
  \lambda^*(C\cap(\bigcup_{i\le n}E_i\times\bigcup_{i\le n}F_i))\cr
\noalign{\noindent (using 132Ae)}
&\le u,\cr}$$

\noindent as required.
}%end of proof of 251P

\leader{251Q}{\bf Proposition}  Let $(X,\Sigma,\mu)$ and $(Y,\Tau,\nu)$
be measure spaces, and $A\subseteq X$, $B\subseteq Y$ subsets;  write
$\mu_A$, $\nu_B$ for the subspace measures on $A$, $B$ respectively.
Let $\lambda$ be the c.l.d.\ product measure on $X\times Y$, and
$\lambda^{\#}$
the subspace measure it induces on $A\times B$.   Let $\tilde\lambda$ be
the c.l.d.\ product measure of $\mu_A$ and $\nu_B$ on $A\times B$.   Then

(i) $\tilde\lambda$ extends $\lambda^{\#}$.

(ii) If

\inset{{\it either} ($\alpha$) $A\in\Sigma$ and $B\in\Tau$}

\inset{{\it or} ($\beta$) $A$ and $B$ can both be covered by sequences
of sets of finite measure}

\inset{{\it or} ($\gamma$) $\mu$ and $\nu$ are both strictly
localizable,}

\noindent then $\tilde\lambda=\lambda^{\#}$.

\proof{ Let $\theta$ be the outer measure on $X\times Y$ defined from
$\mu$ and $\nu$ by the formula of 251A, and $\tilde\theta$ the outer
measure on $A\times B$ similarly defined from $\mu_A$ and $\nu_B$.
Write $\Lambda$ for the domain of $\lambda$,
$\tilde\Lambda$ for the domain of $\tilde\lambda$, and
$\Lambda^{\#}=\{W\cap(A\times B):W\in\Lambda\}$ for
the domain of $\lambda^{\#}$.
Set $\Sigma^f=\{E:\mu E<\infty\}$, $\Tau^f=\{F:\nu F<\infty\}$.

\medskip

{\bf (a)} The first point to observe is that $\tilde\theta C=\theta C$
for every $C\subseteq A\times B$.   \Prf\ (i) If $\sequencen{E_n}$
and $\sequencen{F_n}$ are sequences in $\Sigma$, $\Tau$ respectively
such that $C\subseteq\bigcup_{n\in\Bbb N}E_n\times F_n$, then

\Centerline{$C=C\cap(A\times B)\subseteq\bigcup_{n\in\Bbb N}(E_n\cap
A)\times(F_n\cap B)$,}

\noindent so

$$\eqalign{\tilde\theta C
&\le\sum_{n=0}^{\infty}\mu_A(E_n\cap A)\cdot\nu_B(F_n\cap B)\cr
&=\sum_{n=0}^{\infty}\mu^*(E_n\cap A)\cdot\nu^*(F_n\cap B)
\le\sum_{n=0}^{\infty}\mu E_n\cdot\nu F_n.\cr}$$

\noindent As $\sequencen{E_n}$ and $\sequencen{F_n}$ are arbitrary,
$\tilde\theta C\le\theta C$.   (ii) If $\sequencen{\tilde E_n}$,
$\sequencen{\tilde F_n}$ are sequences in $\Sigma_A=\dom\mu_A$,
$\Tau_B=\dom\nu_B$ respectively such that
$C\subseteq\bigcup_{n\in\Bbb N}\tilde E_n\times\tilde F_n$, then for
each $n\in\Bbb N$ we can choose
$E_n\in\Sigma$, $F_n\in\Tau$ such that

\Centerline{$\tilde E_n\subseteq E_n$,\quad $\mu E_n
=\mu^*\tilde E_n=\mu_A\tilde E_n$,}

\Centerline{$\tilde F_n\subseteq F_n$,
\quad $\nu F_n=\nu^*\tilde F_n=\nu_B\tilde F_n$,}

\noindent and now

\Centerline{$\theta C\le\sum_{n=0}^{\infty}\mu E_n\cdot\nu F_n
=\sum_{n=0}^{\infty}\mu_A\tilde E_n\cdot\nu_B\tilde F_n$.}

\noindent As $\sequencen{\tilde E_n}$, $\sequencen{\tilde F_n}$ are
arbitrary, $\theta C\le\tilde\theta C$.\ \Qed

\medskip

{\bf (b)} It follows that $\Lambda^{\#}\subseteq\tilde\Lambda$.   \Prf\
Suppose that $V\in\Lambda^{\#}$ and that $C\subseteq A\times B$.   In
this
case there is a $W\in\Lambda$ such that $V=W\cap(A\times B)$.   So

\Centerline{$\tilde\theta(C\cap V)+\tilde\theta(C\setminus V)
=\theta(C\cap W)+\theta(C\setminus W)
=\theta C
=\tilde\theta C$.}

\noindent As $C$ is arbitrary, $V\in\tilde\Lambda$.\ \Qed

Accordingly, for $V\in\Lambda^{\#}$,

$$\eqalignno{\lambda^{\#}V
&=\lambda^*V
=\sup\{\theta(V\cap(E\times F)):E\in\Sigma^f,\,F\in\Tau^f\}\cr
&=\sup\{\theta(V\cap(\tilde E\times\tilde F)):\tilde
E\in\Sigma_A,\,\tilde F\in\Tau_B,\,\mu_A\tilde E<\infty,\,\nu_B\tilde
F<\infty\}\cr
&=\sup\{\tilde\theta(V\cap(\tilde E\times\tilde F)):\tilde
E\in\Sigma_A,\,\tilde F\in\Tau_B,\,\mu_A\tilde E<\infty,\,\nu_B\tilde
F<\infty\}
=\tilde\lambda V,\cr}$$

\noindent using 251P twice.

This proves part (i) of the proposition.

\medskip

{\bf (c)} The next thing to check is that if $V\in\tilde\Lambda$ and
$V\subseteq E\times F$ where $E\in\Sigma^f$ and $F\in\Tau^f$, then
$V\in\Lambda^{\#}$.   \Prf\ Let $W\subseteq E\times F$ be a measurable
envelope of $V$ with respect to $\lambda$ (132Ee).    Then

$$\eqalignno{\theta(W\cap(A\times B)\setminus V)
&=\tilde\theta(W\cap(A\times B)\setminus V)
=\tilde\lambda(W\cap(A\times B)\setminus V)\cr
\noalign{\noindent (because $W\cap(A\times
B)\in\Lambda^{\#}\subseteq\tilde\Lambda$, $V\in\tilde\Lambda$)}
&=\tilde\lambda(W\cap(A\times B))-\tilde\lambda V
=\tilde\theta(W\cap(A\times B))-\tilde\theta V\cr
&=\theta(W\cap(A\times B))-\theta V
=\lambda^*(W\cap(A\times B))-\lambda^* V\cr
&\le\lambda W-\lambda^* V
=0.\cr}$$

\noindent But this means that $W'=W\cap(A\times B)\setminus V\in\Lambda$
and $V=(A\times B)\cap(W\setminus W')$ belongs to $\Lambda^{\#}$.\ \Qed

\medskip

{\bf (d)} Now fix any $V\in\tilde\Lambda$, and look at the conditions
($\alpha$)-($\gamma$) of part (ii) of the proposition.

\medskip

\quad\grheada\ If $A\in\Sigma$ and $B\in\Tau$, and $C\subseteq X\times Y$,
then $A\times B\in\Lambda$ (251E), so

$$\eqalignno{\theta(C\cap V)+\theta(C\setminus V)
&=\theta(C\cap V)+\theta((C\setminus V)\cap(A\times B))
  +\theta((C\setminus V)\setminus(A\times B))\cr
&=\tilde\theta(C\cap V)+\tilde\theta(C\cap(A\times B)\setminus V)
  +\theta(C\setminus(A\times B))\cr
&=\tilde\theta(C\cap(A\times B))
  +\theta(C\setminus(A\times B))\cr
&=\theta(C\cap(A\times B))
  +\theta(C\setminus(A\times B))
=\theta C.\cr}$$

\noindent As $C$ is arbitrary, $V\in\Lambda$, so
$V=V\cap(A\times B)$ belongs to $\Lambda^{\#}$.

\medskip

\quad\grheadb\ If $A\subseteq\bigcup_{n\in\Bbb N}E_n$ and
$B\subseteq\bigcup_{n\in\Bbb N}F_n$ where all the $E_n$, $F_n$ are of
finite measure, then
$V=\bigcup_{m,n\in\Bbb N}V\cap(E_m\times F_n)\in\Lambda^{\#}$, by (c).

\medskip

\quad\grheadc\ If $\familyiI{X_i}$, $\family{j}{J}{Y_j}$ are
decompositions
of $X$, $Y$ respectively, then for each $i\in I$, $j\in J$ we have
$V\cap(X_i\times Y_j)\in\Lambda^{\#}$, that is, there is a
$W_{ij}\in\Lambda$ such that
$V\cap(X_i\times Y_j)=W_{ij}\cap(A\times B)$.
Now $\family{(i,j)}{I\times J}{X_i\times Y_j}$ is a decomposition of
$X\times Y$ for $\lambda$ (251O), so that

\Centerline{$W=\bigcup_{i\in I,j\in J}W_{ij}\cap(X_i\times Y_j)
\in\Lambda$,}

\noindent and $V=W\cap(A\times B)\in\Lambda^{\#}$.

\medskip

{\bf (e)} Thus any of the three conditions is sufficient to ensure that
$\tilde\Lambda=\Lambda^{\#}$, in which case (a) tells us that
$\tilde\lambda=\lambda^{\#}$.
}%end of proof of 251Q

\leader{251R}{Corollary} Let $r$, $s\ge 1$ be integers, and
$\phi:\Bbb R^r\times\BbbR^s\to\BbbR^{r+s}$ the natural bijection.   If
$A\subseteq\BbbR^r$ and $B\subseteq\BbbR^s$, then the restriction of
$\phi$ to $A\times B$ identifies the product of Lebesgue measure on $A$
and Lebesgue measure on $B$ with Lebesgue measure on $\phi[A\times
B]\subseteq\BbbR^{r+s}$.

\cmmnt{\medskip

\noindent{\bf Remark} Note that by `Lebesgue measure on $A$' I mean
the subspace measure $\mu_{rA}$ on $A$ induced by $r$-dimensional
Lebesgue measure $\mu_r$ on $\BbbR^r$, whether or not $A$ is itself a
measurable set.
}%end of comment

\proof{ By 251Q, using either of the conditions
(ii-$\beta$) or (ii-$\gamma$), the product measure $\tilde\lambda$ on
$A\times B$ is just the subspace measure $\lambda^{\#}$ on $A\times B$
induced by the
product measure $\lambda$ on $\BbbR^r\times\BbbR^s$.   But by 251N we
know that $\phi$ is an isomorphism between $(\BbbR^r\times\Bbb
R^s,\lambda)$ and $(\BbbR^{r+s},\mu_{r+s})$;    so it must also
identify $\tilde\lambda$ with the subspace measure on $\phi[A\times B]$.
}

\leader{251S}{Corollary} Let $(X,\Sigma,\mu)$ and $(Y,\Tau,\nu)$ be
measure spaces, and $\lambda$ the c.l.d.\ product
measure on $X\times Y$.   If
$A\subseteq X$ and $B\subseteq Y$ can be covered by sequences of sets of
finite measure, then $\lambda^*(A\times
B)=\mu^*A\cdot\nu^*B$.

\proof{ In the language of 251Q,

$$\eqalignno{\lambda^*(A\times B)
&=\lambda^{\#}(A\times B)
=\tilde\lambda(A\times B)
=\mu_AA\cdot\nu_BB\cr
\noalign{\noindent (by 251K and 251E)}
&=\mu^*A\cdot\nu^*B.\cr}$$
}%end of proof of 251S

\leader{251T}{}\cmmnt{ The next proposition gives an idea of how the
technical definitions here fit together.

\medskip

\noindent}{\bf Proposition} Let $(X,\Sigma,\mu)$ and $(Y,\Tau,\nu)$ be
measure spaces.   Write $(X,\hat\Sigma,\hat\mu)$ and
$(X,\tilde\Sigma,\tilde\mu)$ for the completion and c.l.d.\ version of
$(X,\Sigma,\mu)$\cmmnt{ (212C, 213E)}.   Let $\lambda$, $\hat\lambda$
and $\tilde\lambda$ be the three c.l.d.\ product measures on $X\times Y$
obtained from the pairs $(\mu,\nu)$, $(\hat\mu,\nu)$ and
$(\tilde\mu,\nu)$
of factor measures.   Then $\lambda=\hat\lambda=\tilde\lambda$.

\proof{ Write $\Lambda$, $\hat\Lambda$ and $\tilde\Lambda$ for the
domains
of $\lambda$, $\hat\lambda$, $\tilde\lambda$ respectively;  and
$\theta$,
$\hat\theta$, $\tilde\theta$ for the outer measures on $X\times Y$
obtained
by the formula of 251A from the three pairs of factor measures.

\medskip

{\bf (a)} If $E\in\Sigma$ and $\mu E<\infty$, then $\theta$,
$\hat\theta$
and $\tilde\theta$ agree on subsets of $E\times Y$.   \Prf\ Take
$A\subseteq E\times Y$ and $\epsilon>0$.

\medskip

\quad{\bf (i)} There are sequences $\sequencen{E_n}$ in $\Sigma$,
$\sequencen{F_n}$ in $\Tau$ such that
$A\subseteq\bigcup_{n\in\Bbb N}E_n\times F_n$ and
$\sum_{n=0}^{\infty}\mu E_n\cdot\nu F_n\le\theta A+\epsilon$.   Now
$\tilde\mu E_n\le\mu E_n$ for every $n$ (213Fb), so

\Centerline{$\tilde\theta A
\le\sum_{n=0}^{\infty}\tilde\mu E_n\cdot\nu F_n
\le\sum_{n=0}^{\infty}\mu E_n\cdot\nu F_n\le\theta A+\epsilon$.}

\medskip

\quad{\bf (ii)} There are sequences $\sequencen{\hat E_n}$ in
$\hat\Sigma$, $\sequencen{\hat F_n}$ in $\Tau$ such that
$A\subseteq\bigcup_{n\in\Bbb N}\hat E_n\times\hat F_n$ and
$\sum_{n=0}^{\infty}\hat\mu\hat E_n\cdot\nu\hat F_n
\le\hat\theta A+\epsilon$.   Now for each $n$ there is an
$E'_n\in\Sigma$ such that $\hat E_n\subseteq E'_n$ and
$\mu E'_n=\hat\mu\hat E_n$, so that

\Centerline{$\theta A\le\sum_{n=0}^{\infty}\mu E'_n\cdot\nu\hat F_n
=\sum_{n=0}^{\infty}\hat\mu\hat E_n\cdot\nu\hat F_n
\le\hat\theta A+\epsilon$.}

\medskip

\quad{\bf (iii)} There are sequences $\sequencen{\tilde E_n}$ in
$\tilde\Sigma$, $\sequencen{\tilde F_n}$ in $\Tau$ such that
$A\subseteq\bigcup_{n\in\Bbb N}\tilde E_n\times\tilde F_n$ and
$\sum_{n=0}^{\infty}\tilde\mu\tilde E_n\cdot\nu\tilde F_n
\le\tilde\theta A+\epsilon$.   Now for each $n$,
$\tilde E_n\cap E\in\hat\Sigma$, so

\Centerline{$\hat\theta A
\le\sum_{n=0}^{\infty}\hat\mu(\tilde E_n\cap E)\cdot\nu\tilde F_n
\le\sum_{n=0}^{\infty}\tilde\mu\tilde E_n\cdot\nu\tilde F_n
\le\tilde\theta A+\epsilon$.}

\medskip

\quad{\bf (iv)} Since $A$ and $\epsilon$ are arbitrary,
$\theta=\hat\theta=\tilde\theta$ on $\Cal P(E\times Y)$.\ \Qed

\medskip

{\bf (b)} Consequently, the outer measures $\lambda^*$, $\hat\lambda^*$
and $\tilde\lambda^*$ are identical.   \Prf\ Use 251P.   Take
$A\subseteq X\times Y$, $E\in\Sigma$, $\hat E\in\hat\Sigma$,
$\tilde E\in\tilde\Sigma$,
$F\in\Tau$ such that $\mu E$, $\hat\mu\hat E$, $\tilde\mu\tilde E$ and
$\nu F$ are all finite.   Then

\medskip

\quad{\bf (i)}

\Centerline{$\theta(A\cap(E\times F))=\hat\theta(A\cap(E\times F))
\le\hat\lambda^*A$,
\quad$\theta(A\cap(E\times F))=\tilde\theta(A\cap(E\times F))
\le\tilde\lambda^*A$}

\noindent because $\hat\mu E$ and $\tilde\mu E$ are both finite.

\medskip

\quad{\bf (ii)} There is an $E'\in\Sigma$ such that $\hat E\subseteq E'$
and $\mu E'<\infty$, so that

\Centerline{$\hat\theta(A\cap(\hat E\times F))
\le\hat\theta(A\cap(E'\times F))
=\theta(A\cap(E'\times F))
\le\lambda^*A$.}

\medskip

\quad{\bf (iii)} There is an $E''\in\Sigma$ such that
$E''\subseteq\tilde E$ and $\tilde\mu(\tilde E\setminus E'')=0$ (213Fc),
so that
$\tilde\theta((\tilde E\setminus E'')\times Y)=0$ and $\mu E''<\infty$;
accordingly

\Centerline{$\tilde\theta(A\cap(\tilde E\times F))
=\tilde\theta(A\cap(E''\times F))
=\theta(A\cap(E''\times F))
\le\lambda^*A$.}

\medskip

\quad{\bf (iv)} Taking the suprema over $E$, $\hat E$, $\tilde E$ and
$F$,
we get

\Centerline{$\lambda^*A\le\hat\lambda^*A$,
\quad$\lambda^*A\le\tilde\lambda^*A$,
\quad$\hat\lambda^*A\le\lambda^*A$,
\quad$\tilde\lambda^*A\le\lambda^*A$.}

\noindent As $A$ is arbitrary,
$\lambda^*=\hat\lambda^*=\tilde\lambda^*$.\ \Qed

\medskip

{\bf (c)} Now $\lambda$, $\hat\lambda$ and $\tilde\lambda$ are all
complete and locally determined, so by 213C are the measures
defined by
\Caratheodory's method from their own outer measures, and are therefore
identical.
}%end of proof of 251T

\leader{251U}{}\cmmnt{ It is `obvious', and an easy consequence of
theorems so far proved, that the set $\{(x,x):x\in\Bbb R\}$ is
negligible for Lebesgue measure on $\BbbR^2$.   The corresponding result
is true in the square of any {\it atomless} measure space.

\medskip

\noindent}{\bf Proposition} Let $(X,\Sigma,\mu)$ be an atomless measure
space, and let $\lambda$ be the c.l.d.\ measure on $X\times X$.   Then
$\Delta=\{(x,x):x\in X\}$ is $\lambda$-negligible.

\proof{ Let $E$, $F\in\Sigma$ be sets of finite measure, and
$n\in\Bbb N$.   Applying 215D repeatedly, we can find a disjoint family
$\ofamily{i}{n}{F_i}$ of measurable subsets of $F$ such that
$\mu F_i=\Bover{\mu F}{n+1}$ for each $i$;  setting
$F_n=F\setminus\bigcup_{i<n}F_i$, we also have
$\mu F_n=\Bover{\mu F}{n+1}$.   Now

\Centerline{$\Delta\cap(E\times F)
\subseteq\bigcup_{i\le n}(E\cap F_i)\times F_i$,}

\noindent so

\Centerline{$\lambda^*(\Delta\cap(E\times F))
\le\sum_{i=0}^n\mu(E\cap F_i)\cdot\mu F_i
=\Bover{\mu F}{n+1}\sum_{i=0}^n\mu(E\cap F_i)
\le\Bover1{n+1}\mu E\cdot\mu F$.}

\noindent As $n$ is arbitrary, $\lambda(\Delta\cap(E\times F))=0$;  as
$E$ and $F$ are arbitrary, $\lambda\Delta=0$.
}%end of proof of 251U

%\mu\times(\nu_1+\nu_2)=(\mu\times\nu_1)+(\mu\times\nu_2)
%may need restrictions?

\leader{*251W}{Products of more than two spaces}\cmmnt{ The whole of
this section can be repeated for arbitrary finite products.   The labour
is substantial but no new ideas are required.   By the time we need the
general construction in any formal way, it should come very naturally,
and I do not think it is necessary to work through the next
page before proceeding, especially as products of {\it probability}
spaces are dealt with in \S254.   However, for completeness, and to help
locate results when applications do appear, I list them here.   They do
of course constitute a very instructive set of exercises.   The most
important fragments are repeated in 251Xe-251Xf.

} Let $\familyiI{(X_i,\Sigma_i,\mu_i)}$ be a finite family of measure
spaces, and set $X=\prod_{i\in I}X_i$.   Write
$\Sigma_i^f=\{E:E\in\Sigma_i,\,\mu_iE<\infty\}$ for each $i\in I$.

\spheader 251Wa For $A\subseteq X$ set

$$\theta A=\inf\{\sum_{n=0}^{\infty}\prod_{i\in I}\mu_iE_{ni}:
E_{ni}\in\Sigma_i\Forall i\in I,\,n\in\Bbb N,\,
A\subseteq\bigcup_{n\in\Bbb N}\prod_{i\in I}E_{ni}\}.$$

\noindent Then $\theta$ is an outer measure on $X$.   Let $\lambda_0$ be
the measure on $X$ derived by \Caratheodory's method from $\theta$, and
$\Lambda$ its domain.

\spheader 251Wb If $\familyiI{X_i}$ is a finite family of sets and
$\Sigma_i$ is a $\sigma$-algebra of subsets of $X_i$ for each $i\in I$,
then $\Tensorhat_{i\in I}\Sigma_i$ is the $\sigma$-algebra of subsets of
$X=\prod_{i\in I}X_i$ generated by $\{\prod_{i\in I}E_i:E_i\in\Sigma_i$
for every $i\in I\}$.   \cmmnt{(For the corresponding construction
when $I$ is infinite, see 254E.)}

\spheader 251Wc $\lambda_0(\prod_{i\in I}E_i)$ is defined and equal to
$\prod_{i\in I}\mu_iE_i$ whenever $E_i\in\Sigma_i$ for each $i\in I$.

\spheader 251Wd The {\bf c.l.d.\ product measure} on $X$ is the measure
$\lambda$ defined by setting

\Centerline{$\lambda W
=\sup\{\lambda_0(W\cap\prod_{i\in I}E_i):
  E_i\in\Sigma_i^f$ for each $i\in I\}$}

\noindent for $W\in\Lambda$.   If $I$ is empty, \cmmnt{so that
$X=\{\emptyset\}$, then the appropriate convention is to} set
$\lambda X=1$.

\spheader 251We If $H\subseteq X$, then $H\in\Lambda$ iff
$H\cap\prod_{i\in I}E_i\in\Lambda$ whenever $E_i\in\Sigma_i^f$ for each
$i\in I$.

\spheader 251Wf(i) $\Tensorhat_{i\in I}\Sigma_i\subseteq\Lambda$ and
$\lambda(\prod_{i\in I}E_i)=\prod_{i\in I}\mu_iE_i$ whenever
$E_i\in\Sigma_i^f$ for each $i$.

\quad(ii) For every $W\in\Lambda$ there is a $V\in\Tensorhat_{i\in
I}\Sigma_i$ such that $V\subseteq W$ and $\lambda V=\lambda W$.

\quad(iii) $\lambda$ is complete and locally determined, and is the
c.l.d.\ version of $\lambda_0$.

\quad(iv) If $W\in\Lambda$ and $\lambda W>0$ then there are
$E_i\in\Sigma_i^f$, for $i\in I$, such that
$\lambda(W\cap\prod_{i\in I}E_i)>0$.

\quad(v) If $W\in\Lambda$ and $\lambda W<\infty$, then for every
$\epsilon>0$ there are $n\in\Bbb N$ and
$E_{0i},\ldots,E_{ni}\in\Sigma_i^f$, for each $i\in I$, such that
$\lambda(W\symmdiff\bigcup_{k\le n}\prod_{i\in I}E_{ki})\le\epsilon$.

\spheader 251Wg If each $\mu_i$ is $\sigma$-finite, so is $\lambda$, and
$\lambda=\lambda_0$ is the completion of its restriction to
$\Tensorhat_{i\in I}\Sigma_i$.

\spheader 251Wh If $\family{j}{J}{I_j}$ is any partition of $I$, then
$\lambda$ can be identified with the c.l.d.\ product of
$\family{j}{J}{\lambda_j}$, where $\lambda_j$ is the c.l.d.\ product of
$\family{i}{I_j}{\mu_i}$.   \cmmnt{(See the arguments in 251N and also
in 254N below.)}

\spheader 251Wi If $I=\{1,\ldots,n\}$ and each $\mu_i$ is Lebesgue
measure on $\Bbb R$, then $\lambda$ can be identified with Lebesgue
measure on $\BbbR^n$.

\spheader 251Wj If, for each $i\in I$, we have a decomposition
$\family{j}{J_i}{X_{ij}}$ of $X_i$, then
$\langle\prod_{i\in I}X_{i,f(i)}\rangle_{f\in\prod_{i\in I}J_i}$ is a
decomposition of $X$.

\spheader 251Wk For any $C\subseteq X$,

\Centerline{$\lambda^*C
=\sup\{\theta(C\cap\prod_{i\in I}E_i):
  E_i\in\Sigma_i^f$ for every $i\in I\}$.}

\spheader 251Wl Suppose that $A_i\subseteq X_i$ for each $i\in I$.
Write $\lambda^{\#}$ for the subspace measure on $A=\prod_{i\in I}A_i$,
and $\tilde\lambda$ for the c.l.d.\ product of the subspace measures on
the $A_i$.   Then $\tilde\lambda$ extends $\lambda^{\#}$, and if

\quad{\it either} $A_i\in\Sigma_i$ for every $i$

\quad{\it or} every $A_i$ can be covered by a sequence of sets of finite
measure

\quad{\it or} every $\mu_i$ is strictly localizable,

\noindent then $\tilde\lambda=\lambda^{\#}$.

\spheader 251Wm If $A_i\subseteq X_i$ can be covered by a sequence of
sets of finite measure for each $i\in I$, then
$\lambda^*(\prod_{i\in I}A_i)=\prod_{i\in I}\mu_i^*A_i$.

\spheader 251Wn Writing $\hat\mu_i$, $\tilde\mu_i$ for the completion
and c.l.d.\ version of each $\mu_i$, $\lambda$ is the c.l.d.\ product of
$\familyiI{\hat\mu_i}$ and also of $\familyiI{\tilde\mu_i}$.

\spheader 251Wo If all the $(X_i,\Sigma_i,\mu_i)$ are the same atomless
measure space $(X,\Sigma,\mu)$, then
$\{x:x\in X$, $i\mapsto x(i)$ is injective$\}$ is
$\lambda$-conegligible.

\spheader 251Wp Now suppose that we have another family
$\familyiI{(Y_i,\Tau_i,\nu_i)}$ of measure spaces, with product
$(Y,\Lambda',\lambda')$, and \imp\ functions $f_i:X_i\to Y_i$ for each $i$;
define $f:X\to Y$ by saying that $f(x)(i)=f_i(x(i))$ for $x\in X$ and
$i\in I$.   If all the $\nu_i$ are $\sigma$-finite, then $f$ is \imp\ for
$\lambda$ and $\lambda'$.

\exercises{
\leader{251X}{Basic exercises (a)}
%\spheader 251Xa
Let $X$ and $Y$ be sets, $\Cal A\subseteq\Cal PX$ and
$\Cal B\subseteq\Cal PY$.   Let $\Sigma$ be the $\sigma$-algebra of subsets
of $X$ generated by $\Cal A$ and $\Tau$ the $\sigma$-algebra of subsets of
$Y$ generated by $\Cal B$.   Show that $\Sigma\tensorhat\Tau$ is the
$\sigma$-algebra of subsets of $X\times Y$ generated by
$\{A\times B:A\in\Cal A$, $B\in\Cal B\}$.
%251D query out of order

\spheader 251Xb Let $(X,\Sigma,\mu)$ and
$(Y,\Tau,\nu)$ be
measure spaces;  let $\lambda_0$ be the primitive product measure on
$X\times Y$, and $\lambda$ the c.l.d.\ product measure.   Show that
$\lambda_0W<\infty$ iff $\lambda W<\infty$ and $W$ is included in a set
of the form

\Centerline{$(E\times Y)\cup (X\times F)\cup\bigcup_{n\in\Bbb
N}E_n\times F_n$}

\noindent where $\mu E=\nu F=0$ and $\mu E_n<\infty$, $\nu F_n<\infty$
for every $n$.
%251F

\sqheader 251Xc Show that if $X$ and $Y$ are any sets, with
their respective counting measures, then the primitive and c.l.d.\
product measures on $X\times Y$ are both counting measure on $X\times
Y$.
%251F

\spheader 251Xd Let $(X,\Sigma,\mu)$ and $(Y,\Tau,\nu)$ be
measure spaces;  let $\lambda_0$ be the primitive product measure on
$X\times Y$, and $\lambda$ the c.l.d.\ product measure.    Show that

$$\eqalign{\lambda_0&\text{ is locally determined}\cr
&\iff\,\lambda_0\text{ is semi-finite}\cr
&\iff\,\lambda_0=\lambda\cr
&\iff\,\lambda_0\text{ and }\lambda
  \text{ have the same negligible sets}.\cr}$$

\sqheader 251Xe (See 251W.)  Let
$\langle(X_i,\Sigma_i,\mu_i)\rangle_{i\in
I}$ be a family of measure spaces, where $I$ is a non-empty finite set.
Set $X=\prod_{i\in I}X_i$.   For $A\subseteq X$, set

\Centerline{$\theta A
=\inf\{\sum_{n=0}^{\infty}\prod_{i\in I}\mu_iE_{ni}:
E_{ni}\in\Sigma_i\Forall n\in\Bbb N,\,i\in
I,\,A\subseteq\bigcup_{n\in\Bbb N}\prod_{i\in I}E_{ni}\}$.}

\noindent Show that $\theta$ is an outer measure on $X$.   Let
$\lambda_0$ be the measure defined from $\theta$ by \Caratheodory's
method, and for $W\in\dom\lambda_0$ set

\Centerline{$\lambda W=\sup\{\lambda_0(W\cap\prod_{i\in I}E_i):
E_i\in\Sigma_i,\,\mu_iE_i<\infty$ for every $i\in I\}$.}

\noindent Show that $\lambda$ is a measure on $X$, and is the c.l.d.\
version of $\lambda_0$.
%251I

\sqheader 251Xf (See 251W.)  Let $I$ be a non-empty finite set and
$\langle(X_i,\Sigma_i,\mu_i)\rangle_{i\in I}$ a family of measure
spaces.   For non-empty $K\subseteq I$ set $X^{(K)}=\prod_{i\in K}X_i$
and let $\lambda_0^{(K)}$, $\lambda^{(K)}$ be the measures on $X^{(K)}$
constructed as in 251Xe.   Show that if $K$ is a non-empty proper
subset of $I$, then the natural bijection between $X^{(I)}$ and
$X^{(K)}\times X^{(I\setminus K)}$ identifies $\lambda_0^{(I)}$ with the
primitive product measure of $\lambda_0^{(K)}$ and
$\lambda_0^{(I\setminus K)}$, and $\lambda^{(I)}$ with the c.l.d.\
product measure of $\lambda^{(K)}$ and $\lambda^{(I\setminus K)}$.
%251Xe, 251I

\sqheader 251Xg Using 251Xe-251Xf above, or otherwise, show that if
$(X_1,\Sigma_1,\mu_1)$, $(X_2,\Sigma_2,\mu_2)$,
$(X_3,\Sigma_3,\mu_3)$ are measure spaces then the primitive and c.l.d.\
product measures
$\lambda_0$, $\lambda$ of $(X_1\times X_2)\times X_3$, constructed by
first taking the appropriate product measure on $X_1\times X_2$ and then
taking the product of this with the measure of $X_3$, are identified
with the corresponding product measures on $X_1\times(X_2\times X_3)$ by
the canonical bijection between the sets $(X_1\times X_2)\times X_3$ and
$X_1\times(X_2\times X_3)$.
%251Xf, 251Xe, 251I

\spheader 251Xh(i) What happens in 251Xe when $I$ is a
singleton?  (ii) Devise an appropriate convention to make 251Xe-251Xf
remain valid when one or more of the sets $I$, $K$, $I\setminus K$ there
is empty.
%251Xf, 251Xe, 251I

\sqheader 251Xi Let $(X,\Sigma,\mu)$ be a complete locally
determined measure space, and $I$ any non-empty set;  let $\nu$ be
counting measure on $I$.   Show that the c.l.d.\ product measure on
$X\times I$ is equal to (or at any rate identifiable with) the direct
sum measure of the family $\langle(X_i,\Sigma_i,\mu_i)\rangle_{i\in I}$,
if we set $(X_i,\Sigma_i,\mu_i)=(X,\Sigma,\mu)$ for every $i$.
%251I

\sqheader 251Xj Let
$\langle(X_i,\Sigma_i,\mu_i)\rangle_{i\in I}$ be a family of measure
spaces, with direct sum $(X,\Sigma,\mu)$
(214L).   Let $(Y,\Tau,\nu)$ be any measure space, and give $X\times Y$,
$X_i\times Y$ their c.l.d.\ product measures.   Show that the natural
bijection between $X\times Y$ and
$Z=\bigcup_{i\in I}(X_i\times Y)\times\{i\}$
is an isomorphism between the measure of $X\times Y$ and
the direct sum measure on $Z$.
%251Xi, 251I

\sqheader 251Xk Let $(X,\Sigma,\mu)$ be any measure space, and
$Y$ a singleton
set $\{y\}$;  let $\nu$ be the measure on $Y$ such that $\nu Y=1$.
Show that the natural bijection between $X\times\{y\}$ and $X$
identifies the primitive product measure on $X\times\{y\}$ with
$\check\mu$ as defined in 213Xa, and the c.l.d.\ product measure with
the c.l.d.\
version of $\mu$.   Explain how to put this together with 251Xg and
251Ic to prove 251T.
%251I

\sqheader 251Xl Let $(X,\Sigma,\mu)$ and $(Y,\Tau,\nu)$ be measure
spaces, and $\lambda$ the c.l.d.\ product measure on $X\times Y$.   Show
that $\lambda$ is the c.l.d.\ version of its restriction to
$\Sigma\tensorhat\Tau$.
%251K

\spheader 251Xm Let $(X,\Sigma,\mu)$ and $(Y,\Tau,\nu)$ be measure
spaces, with primitive and c.l.d.\ product measures $\lambda_0$,
$\lambda$.   Let
$\lambda_1$ be any measure with domain $\Sigma\tensorhat\Tau$ such that
$\lambda_1(E\times F)=\mu E\cdot\nu F$ whenever $E\in\Sigma$ and
$F\in\Tau$.   Show that $\lambda W\le\lambda_1W\le\lambda_0W$ for every
$W\in\Sigma\tensorhat\Tau$.
%251K

\spheader 251Xn Let $(X,\Sigma,\mu)$ and $(Y,\Tau,\nu)$ be two measure
spaces, and $\lambda_0$ the primitive product measure on $X\times Y$.
Show that the corresponding outer measure $\lambda_0^*$ is just the
outer measure $\theta$ of 251A.
%251P

\spheader 251Xo Let $(X,\Sigma,\mu)$ and $(Y,\Tau,\nu)$
be measure spaces, and $A\subseteq X$, $B\subseteq
Y$ subsets;  write $\mu_A$, $\nu_B$ for the subspace
measures.   Let $\lambda_0$ be the primitive
product measure on $X\times Y$, and $\lambda_0^{\#}$
the subspace measure it induces on $A\times B$.   Let $\tilde\lambda_0$
be the primitive product measure of
$\mu_A$ and $\nu_B$ on $A\times B$.   Show that $\tilde\lambda_0$
extends $\lambda_0^{\#}$.
Show that if
{\it either} ($\alpha$) $A\in\Sigma$ and $B\in\Tau$
{\it or} ($\beta$) $A$ and $B$ can both be covered by sequences of sets
of finite measure
{\it or} ($\gamma$) $\mu$ and $\nu$ are both strictly localizable,
then $\tilde\lambda_0=\lambda_0^{\#}$.
%251Q

\spheader 251Xp In 251Q, show that $\tilde\lambda$ and $\lambda^{\#}$
will have the same null ideals, even if none of the conditions of
251Q(ii) are satisfied.
%251Q query out of order

\spheader 251Xq Let $(X,\Sigma,\mu)$ and $(Y,\Tau,\nu)$ be any measure
spaces, and $\lambda_0$ the primitive product measure on $X\times Y$.
Show that $\lambda_0^*(A\times B)=\mu^*A\cdot\nu^*B$ for any
$A\subseteq X$ and $B\subseteq Y$.
%251S

\spheader 251Xr Let $(X,\Sigma,\mu)$ and $(Y,\Tau,\nu)$ be
measure spaces, and $\hat\mu$ the completion of $\mu$.
Show that $\mu$, $\nu$ and $\hat\mu$, $\nu$ have the same primitive
product measures.
%251T

\spheader 251Xs Let $(X,\Sigma,\mu)$ be a semi-finite measure space.
Show that $\mu$ is atomless iff the diagonal $\{(x,x):x\in X\}$ is
negligible for the c.l.d.\ product measure on $X\times X$.
%251U

\spheader 251Xt Let $(X,\Sigma,\mu)$ be an atomless measure
space, and $(Y,\Tau,\nu)$ any measure space.   Show that the c.l.d.\
product measure on $X\times Y$ is atomless.
%251+

\sqheader 251Xu Let $(X,\Sigma,\mu)$ and $(Y,\Tau,\nu)$ be measure
spaces, and $\lambda$ the c.l.d.\ product measure on $X\times Y$.
(i) Show that if $\mu$ and $\nu$ are purely atomic, so is $\lambda$.
(ii) Show that if $\mu$ and $\nu$ are point-supported, so is $\lambda$.
%251+  (ii) used in 491F

\leader{251Y}{Further exercises (a)}
%\spheader 251Ya
Let $X$, $Y$ be sets with $\sigma$-algebras of subsets $\Sigma$, $\Tau$.
Suppose that $h:X\times Y\to\Bbb R$ is $\Sigma\tensorhat\Tau$-measurable
and $\phi:X\to Y$ is $(\Sigma,\Tau)$-measurable (121Yc).   Show that
$x\mapsto h(x,\phi(x)):X\to\Bbb R$ is $\Sigma$-measurable.
%251D

\spheader 251Yb Show that there are
measure spaces $(X_1,\Sigma_1,\mu_1)$ and $(X_2,\Sigma_2,\mu_2)$,
a probability space $(Y,\Tau,\nu)$ and an \imp\ function $f:X_1\to X_2$
such that $h:X_1\times Y\to X_2\times Y$ is not \imp\ for the c.l.d.\
product measures on $X_1\times Y$ and $X_2\times Y$, where
$h(x,y)=(f(x),y)$ for $x\in X_1$ and $y\in Y$.
%251L

\spheader 251Yc Let $(X,\Sigma,\mu)$ be a complete
locally determined measure space with a subspace $A$ whose measure is
not locally determined (see 216Xb).      Set $Y=\{0\}$, $\nu Y=1$ and
consider the c.l.d.\ product measures on $X\times Y$ and $A\times Y$;
write $\Lambda$, $\tilde\Lambda$ for their domains.   Show that
$\tilde\Lambda$ properly includes $\{W\cap(A\times Y):W\in\Lambda\}$.
%251P

\spheader 251Yd Let $(X,\Sigma,\mu)$ be any measure space,
$(Y,\Tau,\nu)$ an atomless measure space, and $f:X\to Y$ a
$(\Sigma,\Tau)$-measurable function.   Show that $\{(x,f(x)):x\in X\}$
is negligible for the c.l.d.\ product measure on $X\times Y$.
%251U
}%end of exercises

\endnotes{
\Notesheader{251} There are real difficulties in deciding which
construction to declare as `the' product of two arbitrary measures.
My phrase `primitive product measure', and
notation $\lambda_0$, betray a bias;  my own preference is for the
c.l.d.\ product $\lambda$, for two principal reasons.   The first is
that $\lambda_0$ is likely to be `bad', in particular,
not semi-finite, even if $\mu$ and $\nu$ are `good' (251Xd, 252Yk),
while
$\lambda$ inherits some of the most important properties of $\mu$ and
$\nu$ (e.g., 251O);  the second is that in the case of topological
measure spaces $X$ and $Y$, there is often a canonical topological
measure on $X\times Y$, which is likely to be more closely related to
$\lambda$ than to $\lambda_0$.   But for elucidation of this point I
must ask you to wait until \S417 in Volume 4.

It would be possible to remove the `primitive' product measure entirely
from the exposition, or at least to relegate it to the exercises.   This
is indeed what I expect to do in the rest of this treatise, since (in my
view) all significant features of product measures on finitely many
factors can be expressed in terms of the c.l.d.\ product measure.   For
the first introduction to product measures, however, a direct approach
to the c.l.d.\ product measure (through the description of $\lambda^*$
in 251P, for instance) is an uncomfortably large bite, and I have some
sort of duty to present the most natural rival to the c.l.d.\ product
measure prominently enough for you to judge for yourself whether I am
right to dismiss it.   There certainly are results associated with the
primitive product measure (251Xn, 251Xq, 252Yc) which have an
agreeable simplicity.

The clash is avoided altogether, of course, if we specialize immediately
to $\sigma$-finite spaces, in which the two
constructions coincide (251K).   But even this does not solve all
problems.   There is a popular alternative measure often called `the'
product measure:  the restriction $\lambda_{0\Cal B}$ of $\lambda_0$ to
the $\sigma$-algebra $\Sigma\tensorhat\Tau$.   (See, for instance,
{\smc Halmos 50}.)   The advantage of this is that if a function $f$ on
$X\times Y$ is
$\Sigma\tensorhat\Tau$-measurable, then $x\mapsto f(x,y)$ is
$\Sigma$-measurable for every $y\in Y$.   (This is because

$$\{W:W\subseteq X\times Y,\,\{x:(x,y)\in
W\}\in\Sigma\,\Forall \,y\in Y\}$$

\noindent is a $\sigma$-algebra of subsets of $X\times Y$ containing
$E\times F$ whenever $E\in\Sigma$ and $F\in\Tau$, and therefore including
$\Sigma\tensorhat\Tau$.)   The primary objection, to my mind, is that
Lebesgue measure on $\BbbR^2$ is no longer `the' product of Lebesgue
measure on $\Bbb R$ with itself.   Generally, it is right to seek
measures which measure as many sets as possible, and I prefer to face up
to the technical problems (which I acknowledge are off-putting) by
seeking appropriate definitions on the approach to major theorems,
rather than rely on ad hoc fixes when the time comes to apply them.

I omit further examples of product measures for the
moment, because the investigation of particular examples will be much
easier with the aid of results from the next section.   Of course the
leading example, and the one which should come always to mind in
response to the words `product measure', is Lebesgue measure on
$\BbbR^2$, the case $r=s=1$ of 251N and 251R.   For an indication of
what can happen when
one of the factors is not $\sigma$-finite, you could look ahead to 252K.

I hope that you will see that the definition of the outer measure
$\theta$ in 251A corresponds to the standard definition of Lebesgue
outer measure, with `measurable rectangles' $E\times F$ taking the
place of intervals, and the functional $E\times F\mapsto\mu E\cdot\nu F$
taking the place of `length' or `volume' of an interval;
moreover, thinking of $E$ and $F$ as intervals, there is an obvious
relation between Lebesgue measure on $\BbbR^2$ and the product measure
on $\Bbb R\times\Bbb R$.   Of course an `obvious relationship' is not
the same thing as a proper theorem with exact hypotheses and
conclusions, but Theorem 251N is clearly central.   Long before that,
however, there is another parallel between the construction of 251A and
that of Lebesgue measure.   In both cases, the proof that we have an
outer measure comes directly from the defining formula (in 113Yd I gave
as an exercise a general result covering 251B), and consequently a very
general construction can lead us to a measure.   But the measure would
be of far less interest and value if it did not measure, and measure
correctly, the basic sets, in this case the measurable rectangles.
Thus 251E corresponds to the theorem that intervals are Lebesgue
measurable, with the right measure (114Db, 114G).   This is the real key
to the construction, and is one of the fundamental ideas of measure
theory.

Yet another parallel is in 251Xn;  the outer measure defining the
primitive product measure $\lambda_0$ is exactly equal to the outer
measure defined from $\lambda_0$.   I described the corresponding
phenomenon for Lebesgue measure in 132C.

Any construction which claims the title `canonical' must
satisfy a variety of natural requirements;  for instance, one expects
the canonical bijection between $X\times Y$ and $Y\times X$ to be an
isomorphism between the corresponding product measure spaces.
`Commutativity' of the product in this sense is I think obvious from
the definitions in 251A-251C.   It is obviously desirable -- not, I
think, obviously true -- that the product should be `associative' in
that the canonical bijection between $(X\times Y)\times Z$ and
$X\times(Y\times Z)$ should also be an isomorphism between the
corresponding products of product measures.   This is in fact valid for
both the primitive and c.l.d.\ product measures (251Wh, 251Xe-251Xg).

Working through the classification of measure spaces presented in
\S211, we find that the primitive product measure $\lambda_0$ of
arbitrary factor measures $\mu$, $\nu$ is complete, while the c.l.d.\
product measure $\lambda$ is always complete and locally determined.
$\lambda_0$ may not be
semi-finite, even if $\mu$ and $\nu$ are strictly localizable
(252Yk);  but $\lambda$ will be strictly localizable if $\mu$ and
$\nu$ are (251O).   Of course this is associated with the fact that the
c.l.d.\ product measure is distributive over direct sums (251Xj).   If
either $\mu$ or $\nu$ is atomless, so is $\lambda$ (251Xt).   Both
$\lambda$ and $\lambda_0$ are $\sigma$-finite if $\mu$
and $\nu$ are (251K).   It is possible for both $\mu$ and $\nu$ to be
localizable but $\lambda$ not (254U).

At least if you have worked through Chapter 21, you have now done enough
`pure' measure theory for this kind of investigation, however
straightforward, to raise a good many questions.   Apart from direct
sums, we also have the constructions of `completion', `subspace', `outer
measure' and (in particular) `c.l.d.\ version' to integrate into the new
ideas;  I offer some results in 251T and 251Xk.   Concerning
subspaces, some possibly surprising difficulties arise.   The problem is
that the
product measure on the product of two subspaces can have a larger domain
than one might expect.   I give a simple example in 251Yc and a more
elaborate one in 254Yg.   For strictly localizable spaces, there is
no problem (251Q);  but no other criterion drawn from the list of
properties considered in \S251 seems adequate to remove the
possibility of a disconcerting phenomenon.

}%end of notes

\discrpage

