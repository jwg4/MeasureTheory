\frfilename{mt5a1.tex}
\versiondate{23.2.11}
\copyrightdate{2006}

\def\chaptername{Appendix}
\def\sectionname{Set theory}

\long\def\doubleinset#1{\inset{\inset{\parindent=-20pt #1}}}
\def\Glsquare{{Gl}$\Square$}
\def\Sing{\mathop{\text{Sing}}}
\def\VVdash{\mskip5mu\vrule height 7.5pt depth 2.5pt width 0.5pt
  \mskip2.5mu\vrule height 7.5pt depth 2.5pt width 0.5pt
  \vrule height 2.75pt depth -2.25pt width 4pt\mskip2mu}
\def\VVdP{\VVdash_{\Bbb P}}

\def\JechM{{\smc Jech 03}} %........
\def\Jech{{\smc Jech 78}} %.....
%{\smc Devlin 84}} %..... serious problems according to ARDM
%{\smc Kanamori 03} %....
\def\Kunen{{\smc Kunen 80}} %...
%\def\DS{{\smc Drake \& Singh 96}} %...

\newsection{5A1}

As usual, I begin with set theory, continuing from \S\S2A1 and 4A1.
I start with definitions and
elementary remarks filling some minor gaps in the deliberately sketchy
accounts in the earlier volumes (5A1A).   I give a relatively solid
paragraph on cardinal arithmetic (5A1E), including an account of
cofinalities of ideals $[\kappa]^{\le\lambda}$.
5A1G-5A1J %5{}A1G 5A1H 5A1I 5A1J
are devoted to infinitary combinatorics, with the Erd\H{o}s-Rado theorem
and Hajnal's Free Set Theorem.   5A1K-5A1O %5{}A1K 5A1L 5A1M 5A1N 5A1O
deal with the existence of `transversals' of various kinds in spaces of
functions, that is, large sets of functions which are well separated on
some combinatorial criterion.

\leader{5A1A}{Order types (a)} If $X$ is a well-ordered set, its
{\bf order type}
$\otp X$ is the ordinal order-isomorphic to $X$\cmmnt{ (2A1Dg)}.

If $S$ is a set of ordinals, an ordinal-valued function $f$ with
domain $S$ is {\bf regressive} if $f(\xi)<\xi$ for every
$\xi\in S$\cmmnt{ (cf.\ 4A1Cc)}.

\spheader 5A1Ab The non-stationary ideal on a
cardinal $\kappa$ of uncountable cofinality
is $(\cf\kappa)$-additive\prooflet{, because the intersection of fewer than
$\cf\kappa$ closed cofinal sets is a closed cofinal set (4A1Bd)}.

\spheader 5A1Ac If $\kappa$ is a cardinal,
$\lambda<\cf\kappa$ is an infinite regular cardinal and $C\subseteq\kappa$
is a closed cofinal set, then
$S=\{\xi:\xi<\kappa$, $\cf(\xi\cap C)=\lambda\}$ is stationary in $\kappa$.
\prooflet{\Prf\
If $C'\subseteq\kappa$ is a closed cofinal set, let
$\ofamily{\xi}{\otp(C\cap C')}{\gamma_{\xi}}$ be the increasing
enumeration of $C\cap C'$.
Then $\otp(C\cap C')\ge\cf\kappa>\lambda$, so $\gamma_{\lambda}$ is
defined and belongs to $S\cap C'$.\ \Qed}

\spheader 5A1Ad If $\alpha$ is an ordinal, there is a closed cofinal set
$C\subseteq\alpha$ such that $\otp C=\cf\alpha$.   \prooflet{\Prf\ Set
$\kappa=\cf\alpha$ and let $\ofamily{\xi}{\kappa}{\alpha_{\xi}}$ enumerate
a cofinal subset of $\alpha$.   Set
$\gamma_{\xi}=\sup_{\eta<\xi}\alpha_{\eta}$ for $\xi<\kappa$;  then
$C=\{\gamma_{\xi}:\xi<\kappa\}$ is a closed cofinal set in $\alpha$.
Let $\ofamily{\xi}{\otp C}{\gamma'_{\xi}}$ be the increasing enumeration of
$C$;  then an induction on $\xi$ shows that $\gamma_{\xi}\le\gamma'_{\xi}$
for $\xi<\min(\kappa,\otp C)$, so that $\otp C\le\kappa$.   As $C$ is
cofinal with $\alpha$, $\#(C)\ge\kappa$ so $\otp C\ge\kappa$.\ \Qed}
%5{}A2B

\spheader 5A1Ae If $\alpha$ is an ordinal and $C\subseteq\alpha$ has
closure $\overline{C}$ for the order topology of $\alpha$, then
$\#(\overline{C})=\#(C)$.   \prooflet{\Prf\ If $C$ is finite this is
trivial.   Otherwise, for any $\beta\in\overline{C}\setminus\{\sup C\}$
set $f(\beta)=\min(C\setminus\beta)$.  If $\beta$,
$\beta'\in\overline{C}$
and $\beta<\beta'<\sup C$ there must be a $\gamma\in C$ such that
$\beta\le\gamma<\beta'$ and $f(\beta)\le\gamma<f(\beta')$ (4A2S(a-ii));
thus $f$ is injective.   So

\Centerline{$\#(\overline{C})
=\#(\overline{C}\setminus\{\sup C\})\le\#(C)\le\#(\overline{C})$.  \Qed}}

\vleader{72pt}{5A1B}{Ordinal arithmetic (a)}
For ordinals $\xi$, $\eta$ their
{\bf ordinal sum} $\xi+\eta$ is defined inductively by saying that

\inset{$\xi+0=\xi$,

$\xi+(\eta+1)=(\xi+\eta)+1$,

$\xi+\eta=\sup_{\zeta<\eta}\xi+\zeta$ for non-zero limit ordinals
$\eta$\dvro{.}{}}

\cmmnt{\noindent(\Kunen, I.7.18;  \Jech, p.\ 18;
\JechM, 2.18).}

\spheader 5A1Bb For ordinals
$\xi$, $\eta$ their {\bf ordinal product} $\xi\cdot\eta$ is defined
inductively by saying

\inset{$\xi\cdot 0=0$,

$\xi\cdot(\eta+1)$ is the ordinal sum $\xi\cdot\eta+\xi$,

$\xi\cdot\eta=\sup_{\zeta<\eta}\xi\cdot\zeta$ for non-zero limit ordinals
$\eta$\dvro{.}{}}

\noindent\cmmnt{(\Kunen, I.7.20;  \Jech, p.\ 19;
\JechM, 2.19). }\cmmnt{Note that }$0\cdot\eta=0$ and
$1\cdot\eta=\eta$ for every
$\eta$, and that $\sup_{\zeta\in A}\xi\cdot\zeta=\xi\cdot(\sup A)$ for
every $\xi$ and every
non-empty set $A$ of ordinals.   Ordinal multiplication is
associative\cmmnt{ (\Kunen, I.7.20;  \JechM, 2.21)}.

\spheader 5A1Bc For ordinals $\xi$, $\eta$ the {\bf ordinal power}
$\xi^{\eta}$ is defined inductively by saying that

\inset{$\xi^0=1$,

$\xi^{\eta+1}$ is the ordinal product $\xi^{\eta}\cdot\xi$,

$\xi^{\eta}=\sup_{\zeta<\eta}\xi^{\zeta}$ for non-zero limit ordinals
$\eta$\dvro{.}{}}

\noindent\cmmnt{(\Kunen, I.9.5;  \JechM, 2.20).
{\bf Warning!}  If $\xi$ and $\eta$
happen to be cardinals, this is quite different from the `cardinal power'
of 5A1E below.

}If $\xi$, $\eta$ are ordinals, $\eta\ne 0$ and $\eta$ is greater than or
equal to the ordinal product $\xi\cdot\eta$, then $\eta$ is at least the
ordinal power $\xi^{\omega}$.   \cmmnt{\Prf\ Note first that as
multiplication is
associative, we can induce on $n$ to show that $\xi\cdot\xi^n=\xi^{n+1}$
for every $n$.   Now we are supposing that
$\eta\ge 1=\xi^0$.   If $n\in\Bbb N$ and $\eta\ge\xi^n$, then

\Centerline{$\eta\ge\xi\cdot\eta\ge\xi\cdot\xi^n=\xi^{n+1}$.}

\noindent So $\eta\ge\xi^n$ for every $n$ and $\eta\ge\xi^{\omega}$.\ \Qed}

\leader{5A1C}{Well-founded sets (a)} A partially ordered set $P$ is {\bf
well-founded} if every non-empty $A\subseteq P$ has a minimal
element\cmmnt{, that
is, a $p\in A$ such that $q\not<p$ for every $q\in A$}.

\spheader 5A1Cb If $P$ is a well-founded partially ordered set,
we have a rank function $r:P\to\On$ defined by saying that

\Centerline{$r(p)=\sup\{r(q)+1:q<p\}$}

\noindent for every $p\in P$\cmmnt{ (\Kunen, III.5.7;
\Jech, p.\ 21;  \JechM, 2.27)}.

\spheader 5A1Cc A partially ordered set $P$ is well-founded iff there is no
sequence $\sequencen{p_n}$ in $P$ such that $p_{n+1}<p_n$ for every
$n\in\Bbb N$.   \prooflet{(If $A\subseteq P$ is non-empty and has no
minimal element, we can choose inductively a strictly decreasing sequence
in $A$.)}

\leader{5A1D}{Trees }\cmmnt{In \S421 I introduced trees of sequences.
For this volume a more abstract approach is useful.

\medskip

}{\bf (a)} A {\bf tree} is a partially
ordered set $T$ such that $\{s:s\in T$, $s\le t\}$ is well-ordered for
every $t\in T$\cmmnt{;  alternatively, a well-founded partially ordered
set such that $\{s:s\in T$, $s\le t\}$ is totally ordered for
every $t\in T$}.   In particular, $T$ has
a rank function $r:T\to\On$ defined by saying that

\Centerline{$r(t)=\otp\{s:s<t\}=\inf\{\xi:r(s)<\xi$ whenever $s<t\}$}

\noindent for every $t\in T$\cmmnt{ (5A1C)}.
\cmmnt{ (Try to avoid using this terminology
in the same sentence as that of 421Ne and 562A.)}

The {\bf levels} of $T$ are now the sets $\{t:r(t)=\xi\}$ for $\xi\in\On$.
The {\bf height} of $T$ is the least
ordinal $\zeta$ such that $r(t)<\zeta$ for
every $t\in T$.   A {\bf branch} of $T$ is a maximal totally ordered
subset.   A tree is {\bf well-pruned} if it has at most one minimal element
and whenever $s$, $t\in T$ and $r(s)<r(t)$, there is an $s'\ge s$ such that
$r(s')=r(t)$.
If $T$ is a tree, a {\bf subtree} of $T$ is a set
$T'\subseteq T$ such that $s\in T'$ whenever $s\le t\in T'$;  in this
case, the rank function of $T'$ is the restriction to $T'$ of the rank
function of $T$.

\spheader 5A1Db{\bf (i)} Let $T$ be a tree in which every level is
finite.   Then
$T$ has a branch meeting every level.   \prooflet{\Prf\ If $T$ is empty,
this is trivial.   Otherwise, let $r$ be the rank function of $T$,
and $\zeta>0$ the height of
$T$;  let $\Cal F$ an ultrafilter on $T$ containing $\{t:r(t)\ge\xi\}$ for
every $\xi<\zeta$.   Set $C=\{t:\coint{t,\infty}\in\Cal F\}$.
Any two elements of $C$ are upwards-compatible, so $C$ is totally ordered,
and $C$ meets every level of $T$;  so $C$ is a branch of the kind we seek.\
\Qed}

\medskip

\quad{\bf (ii)} Let $(T,\preccurlyeq')$ be a tree of height $\omega_1$ in
which every level is countable.    Then there is an ordering
$\preccurlyeq$ of $\omega_1$, included in the usual ordering $\le$ of
$\omega_1$, such that $(T,\preccurlyeq')$ is
isomorphic to $(\omega_1,\preccurlyeq)$.   \prooflet{\Prf\ Let
$\ofamily{\xi}{\omega_1}{T_{\xi}}$ be the levels of $T$.   Let
$\le'_{\xi}$ be a well-ordering of $T_{\xi}$ for each
$\xi<\omega_1$, and define $\le'$ on $T$ by saying that
$s\le't$ if either $r(s)<r(t)$ or $r(s)=r(t)=\xi$ and
$s\le'_{\xi}t$;  then $\le'$ is a well-ordering of $T$
of order type $\omega_1$.   Now the order-isomorphism between
$(T,\le')$ and $(\omega_1,\le)$ copies $\preccurlyeq'$ onto a tree ordering
of $\omega_1$, isomorphic to $\preccurlyeq'$, and included in the usual
ordering.\ \Qed}

\spheader 5A1Dc An {\bf Aronszajn tree} is a tree $T$ of height $\omega_1$
in which every branch and every level is countable.
An Aronszajn tree $T$ is {\bf special} if it is expressible as
$\bigcup_{n\in\Bbb N}A_n$ where no two elements of any $A_n$ are
comparable\cmmnt{, that is, every $A_n$ is an up-antichain}.

\spheader 5A1Dd(i) A {\bf Souslin tree} is a tree $T$ of height $\omega_1$
in which every branch and every up-antichain is countable.

\quad(ii) Every Souslin tree is a non-special Aronszajn tree.

\quad(iii) If $T$ is a Souslin tree,
it has a subtree which is a well-pruned
Souslin tree.   \prooflet{(\Kunen, II.5.11;  \Jech,
p.\ 218;  \JechM, 9.13.)}

\quad(iv) {\bf Souslin's hypothesis} is the assertion

\Centerline{(SH)\quad There are no Souslin trees.}

\leader{5A1E}{Cardinal arithmetic (a)(i)} An infinite cardinal which is
not regular\cmmnt{ (4A1Aa)} is {\bf singular}.
A cardinal $\kappa$ is a {\bf successor cardinal} if it is of the form
$\lambda^+$\cmmnt{ (2A1Fc)};  otherwise it is a {\bf limit cardinal}.
$\kappa$ is a
{\bf strong limit cardinal} if it is uncountable and $2^{\lambda}<\kappa$
for every $\lambda<\kappa$.   It 
is {\bf weakly inaccessible} if it is a regular uncountable limit
cardinal;  it is {\bf strongly inaccessible} if moreover it is a strong
limit cardinal.

\medskip

\quad{\bf (ii)} If $\kappa$ is a cardinal,
define $\kappa^{(+\xi)}$, for ordinals $\xi$, by setting

\Centerline{$\kappa^{(+0)}=\kappa$,
\quad$\kappa^{(+\xi)}=\sup_{\eta<\xi}(\kappa^{(+\eta)})^+$ if $\xi>0$,}

\noindent that is, $\kappa^{(+\xi)}=\omega_{\zeta+\xi}$ if
$\kappa=\omega_{\zeta}$.

\medskip

\spheader 5A1Eb{\bf (i)} If $\familyiI{\kappa_i}$ is a
family of cardinals, its {\bf cardinal sum} is
$\#(\{(i,\xi):i\in I$, $\xi<\kappa_i\})$, which is at most
$\max(\omega,\#(I),\sup_{i\in I}\kappa_i)$.

\medskip

\quad{\bf (ii)} For cardinals $\kappa$ and $\lambda$,
the {\bf cardinal product} $\kappa\cdot\lambda$ is
$\#(\kappa\times\lambda)\le\max(\omega,\kappa,\lambda)$.
%5{}12I

\medskip

\quad{\bf (iii)} If $\kappa$ and
$\lambda$ are cardinals there are two natural interpretations
of the formula $\kappa^{\lambda}$:  (i) the set of functions from $\lambda$
to $\kappa$ (ii) the cardinal of this set.   In this volume the latter will
be the usual one, but I will try to signal this by using the phrase {\bf
cardinal power}.   \cmmnt{Of course}
$2^{\lambda}$ is always the cardinal power;
the corresponding set of functions will be denoted
by $\{0,1\}^{\lambda}$.   \cmmnt{We could also think of $\kappa$ and
$\lambda$ as ordinals, and look at the ordinal power $\kappa^{\lambda}$ as
described in 5A1Bc;  but I think I do this exactly three times, 
all at the end of \S539.}

\spheader 5A1Ec{\bf (i)} The cardinal power
$\kappa^{\lambda}$ is at most $2^{\max(\omega,\kappa,\lambda)}$ for any
cardinals $\kappa$ and $\lambda$.
\prooflet{(The
set of functions from $\lambda$ to $\kappa$ is a subset
of $\Cal P(\lambda\times\kappa)$.)}

\medskip

\quad{\bf (ii)} $\frak c^{\omega}\cmmnt{=\#(((\{0,1\})^{\Bbb N})^{\Bbb N})
=\#(\{0,1\}^{\Bbb N\times\Bbb N})=\#(\{0,1\}^{\Bbb N})}=\frak c$.

%5{}21N

\spheader 5A1Ed $\cf 2^{\kappa}>\kappa$ for every infinite cardinal
$\kappa$.  \prooflet{(\JechM, 5.11;  \Jech, p.\ 46;
{\smc Erd\H{o}s Hajnal M\'at\'e \& Rado 84}, 6.9;  \Kunen, 10.41;  %\DS, 7.4.6(5);
{\smc Just \& Weese 97}, 11.2.24.   Compare (e-v) below.)}
%5{}22U

\spheader 5A1Ee{\bf (i)} If $\kappa$ and $\lambda$ are infinite cardinals,
then

$$\eqalign{\cff[\kappa]^{\le\lambda}
&=1\text{ if }\lambda\ge\kappa,\cr
&\ge\kappa\text{ if }\lambda<\kappa.\cr}$$

\medskip

\quad{\bf (ii)} Let $\kappa$, $\lambda$ and $\theta$ be infinite
cardinals such that $\theta\le\lambda\le\kappa$.   Then
$\cff[\kappa]^{\le\theta}
\le\max(\cff[\kappa]^{\le\lambda},\cff[\lambda]^{\le\theta})$.
\prooflet{\Prf\ Let $\Cal A\subseteq[\kappa]^{\lambda}$ be a cofinal set
of size $\cff[\kappa]^{\lambda}=\cff[\kappa]^{\le\lambda}$.   Then
$[\kappa]^{\le\theta}=\bigcup_{A\in\Cal A}[A]^{\le\theta}$, so

\Centerline{$\cff[\kappa]^{\le\theta}
\le\max(\#(\Cal A),\sup_{A\in\Cal A}\cff[A]^{\le\theta})
\le\max(\cff[\kappa]^{\le\lambda},\cff[\lambda]^{\le\theta})$.  \Qed}}

\medskip

\quad{\bf (iii)} Let $\kappa$ and $\lambda$ be infinite cardinals.   Then
the cardinal power $\kappa^{\lambda}$ is
$\max(2^{\lambda},\cff[\kappa]^{\le\lambda})$.   \prooflet{\Prf\
$\kappa^{\lambda}\ge 2^{\lambda}$ because $\kappa\ge 2$;
$\kappa^{\lambda}\ge\#([\kappa]^{\le\lambda})\ge\cff[\kappa]^{\le\lambda}$
because $f\mapsto f[\lambda]$ is a surjection from the family
$F$ of functions from $\lambda$ to $\kappa$ onto
$[\kappa]^{\le\lambda}\setminus\{\emptyset\}$.   In the other direction, if
$\kappa\le\lambda$ then $F\subseteq\Cal P(\lambda\times\kappa)$ so

\Centerline{$\kappa^{\lambda}
=\#(F)\le 2^{\lambda}=\max(2^{\lambda},\cff[\kappa]^{\le\lambda})$.}

\noindent
If $\lambda<\kappa$ let $\Cal A\subseteq[\kappa]^{\le\lambda}$ be a cofinal
family with cardinal $\cff[\kappa]^{\le\lambda}$;  then
$F=\bigcup_{A\in\Cal A}A^{\lambda}$ so

\Centerline{$\#(F)\le\max(\#(\Cal A),\sup_{A\in\Cal A}\#(A^{\lambda}))
=\max(\cff[\kappa]^{\le\lambda},2^{\lambda})$.  \Qed}}

\medskip

\quad{\bf (iv)} If $\lambda$ is an infinite cardinal and
$\lambda\le\kappa<\lambda^{(+\omega)}$, then
$\cff[\kappa]^{\le\omega}=\max(\kappa,\cff[\lambda]^{\le\omega})$.
\prooflet{\Prf\
Induce on $n$ to see that

$$\eqalign{\cff([\lambda^{(+n+1)}]^{\le\omega})
&=\cff(\bigcup_{\xi<\lambda^{(+n+1)}}[\xi]^{\le\omega})
\le\max(\lambda^{(+n+1)},\sup_{\xi<\lambda^{(+n+1)}}\cff[\xi]^{\le\omega})
  \cr
&\le\max(\lambda^{(+n+1)},\lambda^{(+n)},\cff[\lambda]^{\le\omega})
=\max(\lambda^{(+n+1)},\cff[\lambda]^{\le\omega}).
\text{ \Qed}\cr}$$

\noindent}Consequently the cardinal power $\kappa^{\omega}$ is
$\cmmnt{\max(\frak c,\kappa,\cff[\lambda]^{\le\omega})
=}\max(\kappa,\lambda^{\omega})$.

In particular, if $\omega_1\le\kappa<\omega_{\omega}$
then $\cff[\kappa]^{\le\omega}=\kappa$ and
$\kappa^{\omega}=\max(\frak c,\kappa)$.   \cmmnt{Similarly,}
$(\frak c^+)^{\omega}\cmmnt{\mskip5mu
  =\max(\frak c^+,\frak c)}=\frak c^+$,
$(\frak c^{++})^{\omega}=\frak c^{++}$.
%5{}22T 5{}23J

\medskip

\quad{\bf (v)} If $\kappa$ is a singular infinite cardinal,
then $\cf([\kappa]^{\le\cf\kappa})>\kappa$.
\prooflet{\Prf\ Set $\lambda=\cf\kappa$, and
let $\ofamily{\xi}{\lambda}{\kappa_{\xi}}$
be a strictly increasing
family of cardinals with supremum $\kappa$.   If
$\ofamily{\eta}{\kappa}{A_{\eta}}$ is a family in $[\kappa]^{\le\lambda}$,
then for each $\xi<\lambda$ take
$\alpha_{\xi}\in\kappa\setminus\bigcup_{\eta<\kappa_{\xi}}A_{\eta}$;
set $A=\{\alpha_{\xi}:\xi<\lambda\}\in[\kappa]^{\le\lambda}$;  then
$A\not\subseteq A_{\eta}$ for every $\eta<\kappa$.\ \Qed}
%5{}24P

\spheader 5A1Ef If $\lambda$ is a regular uncountable
cardinal, $\theta\ge 2$ is a cardinal and
$\kappa=\sup_{\delta<\lambda}\theta^{\delta}$, where
$\theta^{\delta}$ is the cardinal power, then

\Centerline{$\#([\kappa]^{<\lambda})
\cmmnt{\mskip5mu=\sup_{\delta<\lambda}\kappa^\delta}=\kappa$.}

\prooflet{\noindent\Prf\ Of course
$\kappa\le\#([\kappa]^{<\lambda})$ because $\lambda\ge 2$, while

\Centerline{$\#([\kappa]^{<\lambda})
=\#(\bigcup_{\delta<\lambda}[\kappa]^{\delta})
\le\max(\lambda,\omega,\sup_{\delta<\lambda}\#([\kappa]^{\delta}))
\le\sup_{\delta<\lambda}\kappa^{\delta}$}

\noindent because $\lambda\le\sup_{\delta<\lambda}2^{\delta}\le\kappa$.
If $\delta<\lambda$ then, because $\lambda$ is regular,

\Centerline{$\kappa^{\delta}
=(\sup_{\zeta<\lambda}\theta^{\zeta})^{\delta}
=\sup_{\zeta<\lambda}(\theta^{\zeta})^{\delta}
\le\sup_{\zeta<\lambda}\theta^{\max(\omega,\zeta,\delta)}
\le\kappa$.  \Qed}}
%5{}14D 5{}15H

In particular, if $\kappa$ is strongly inaccessible then 
$\kappa^{\delta}\le\kappa$ for every $\delta<\kappa$.   \prooflet{(Take
$\lambda=\kappa$ and $\theta=2$.)}

\spheader 5A1Eg Let $X$, $Y$ and $Z$ be sets, with $\#(X)\le 2^{\#(Z)}$
and $0<\#(Y)\le\#(Z)$.
Then there is a function $f:X\times Z^{\Bbb N}\to Y$ such that
whenever $\sequencen{x_n}$ is a sequence of distinct elements
of $X$ and $\sequencen{y_n}$ is a sequence in $Y$ there is
a $z\in Z^{\Bbb N}$ such that $f(x_n,z)=y_n$ for every
$n\in\Bbb N$.  \prooflet{\Prf\ We can suppose that $X\subseteq\Cal PZ$ and
$Y\subseteq Z$;  moreover, the case of finite $X$ is trivial, so we can
suppose that $Z$ is infinite.   For each countably infinite set
$I\subseteq Z$, (c-ii) above tells us that there is a surjection
$g_I:I^{\Bbb N}\to(\Cal PI)^{\Bbb N}\times I^{\Bbb N}$.
Now let $f:X\times Z^{\Bbb N}\to Y$ be such that

\inset{\noindent
whenever $z\in Z^{\Bbb N}$ is such that $I=z[\Bbb N]$ is
infinite, $g_I(z)=(\sequencen{a_n},\sequencen{y_n})$ and $x\in X$ is such
that there is just one $n$ for which $a_n=x\cap I$, then $f(x,z)=y_n$.}

\noindent In this case, if $\sequencen{x_n}$ is a sequence of distinct
members of $X$ and $\sequencen{y_n}$ is a sequence in $Y$, let $I$ be a
countably infinite subset of $Z$ containing every $y_n$ and such that
$x_m\cap I\ne x_n\cap I$ for $m<n$;  let $z\in I^{\Bbb N}$ be such that
$g_I(z)=(\sequencen{x_n\cap I},\sequencen{y_n})$;  we shall have
$f(x_n,z)=y_n$ for every $n$.\ \Qed}
%5{}25M

\spheader 5A1Eh If $\kappa$ is an infinite cardinal, then
$2^{\kappa}$ is at most the cardinal power
$(\sup_{\lambda<\kappa}2^{\lambda})^{\cf\kappa}$.   \prooflet{\Prf\
Let $\ofamily{\xi}{\cf\kappa}{\alpha_{\xi}}$ be a family in $\kappa$ with
supremum $\kappa$.   Set $D=\bigcup_{\alpha<\kappa}\Cal P\alpha$;  then

\Centerline{$\#(D)\le\max(\kappa,\sup_{\alpha<\kappa}2^{\#(\alpha)})
=\sup_{\lambda<\kappa}2^{\lambda}$.}

\noindent Let $F$ be the set of functions from $\cf\kappa$ to $D$;  we
have an injection $A\mapsto\ofamily{\xi}{\cf\kappa}{A\cap\alpha_{\xi}}$
from $\Cal P\kappa$ to $F$, so $2^{\kappa}\le\#(F)$.\ \Qed}

If $\omega\le\lambda<\kappa$ and
$2^{\theta}=2^{\lambda}$ for $\lambda\le\theta<\kappa$ but
$2^{\kappa}>2^{\lambda}$ then $\kappa$ is regular.   \prooflet{\Prf\
$2^{\kappa}\le(2^{\lambda})^{\cf\kappa}=2^{\max(\lambda,\cf\kappa)}$.\
\Qed}

\leader{5A1F}{Three fairly simple facts} (a) There is a family
$\langle a_I\rangle_{I\subseteq\Bbb N}$ of infinite subsets of $\Bbb N$
such that $a_I\cap a_J$ is finite whenever $I$, $J\subseteq\Bbb N$
are distinct.

(b) Let $X$ be a set,
$f:[X]^{<\omega}\to[X]^{\le\omega}$ a function, and $Y\subseteq X$.
Then there is a $Z\subseteq X$ such that $Y\subseteq Z$,
$f(I)\subseteq Z$ for every $I\in[Z]^{<\omega}$, and
$\#(Z)\le\max(\omega,\#(Y))$.

(c) Let $\kappa\ge\frak c$ be a cardinal and
$\Cal A$ a family of countable subsets of $\kappa$ such that $\#(\Cal A)$
is less than the cardinal power $\kappa^{\omega}$.   Then there is a
countably infinite $K\subseteq\kappa$ such that $I\cap K$ is finite for
every $I\in\Cal A$.

\proof{{\bf (a)} For each $n\in\Bbb N$, set
$K_n=\{i:2^n\le i<2^{n+1}\}$, and let $f_n:\Cal Pn\to K_n$ be a bijection;
set $a_I=\{f_n(I\cap n):n\in\Bbb N\}$.   (Or apply 5A1Mc below with
$\kappa=\omega$.)

\medskip

{\bf (b)} Define $\sequencen{Z_n}$ inductively
by setting $Z_0=Y$ and
$Z_{n+1}=Z_n\cup\bigcup\{f(I):I\in[Z_n]^{<\omega}\}$
for each $n$.   Then $\#(Z_n)\le\max(\omega,\#(Y))$ for each $n$, so
setting $Z=\bigcup_{n\in\Bbb N}Z_n$ we still have
$\#(Z)\le\max(\omega,\#(Y))$, while $f(I)\subseteq Z$ for every
$I\in[Z]^{<\omega}$.

\medskip

{\bf (c)} If $\#(\Cal A)<\kappa$ this is trivial, as we can take
$K\subseteq\kappa\setminus\bigcup\Cal A$.   Otherwise, let
$\lambda\le\kappa$ be the least cardinal such that
$\#(\Cal A)<\lambda^{\omega}$.   Then $\cf\lambda=\omega$.   \Prf\Quer\
Otherwise,

\Centerline{$\lambda^{\omega}
=\max(\lambda,\sup_{\theta<\lambda}\theta^{\omega})
\le\#(\Cal A)$.  \Bang\Qed}

\noindent Let $\sequencen{\lambda_n}$ be a strictly increasing sequence of
cardinals with supremum $\lambda$, starting from $\lambda_0=0$ and
$\lambda_1=\omega$ (of course $\lambda>\omega$ because
$\#(\Cal A)\ge\frak c$).   For $n\in\Bbb N$ let
$\phi_n:[n\times\lambda_n]^{<\omega}\to\lambda_{n+1}\setminus\lambda_n$ be
an injective function.   For $f:\Bbb N\to\lambda$ define
$C_f\subseteq\lambda$ by setting

\Centerline{$C_f=\{\phi_n(f\cap(n\times\lambda_n)):n\in\Bbb N\}$.}

\noindent If $f$, $g\in\lambda^{\Bbb N}$ are distinct, then there are an
$i\in\Bbb N$ such that $f(i)\ne g(i)$ and an $m>i$ such that both $f(i)$
and $g(i)$ are less than $\lambda_m$, so that
$f\cap(n\times\lambda_n)\ne g\cap(n\times\lambda_n)$ for every $n\ge m$ and
$C_f\cap C_g$ is finite.   It follows that for any $I\in\Cal A$ the set
$B_I=\{f:f\in\lambda^{\Bbb N}$, $C_f\cap I$ is infinite$\}$ has cardinal at
most $\frak c$.   Since
$\frak c\le\#(\Cal A)<\lambda^{\omega}$, there must be an
$f\in\lambda^{\Bbb N}$ such that $C_f\cap I$ is finite for every
$I\in\Cal A$, and we can set $K=C_f$.
}%end of proof of 5A1F

\leader{5A1G}{Partition calculus (a) The Erd\H{o}s-Rado theorem}
Let $\kappa$ be an infinite cardinal.   Set $\kappa_1=\kappa$,
$\kappa_{n+1}=2^{\kappa_n}$ for $n\ge 1$.   If $n\ge 1$,
$\#(B)\le\kappa$,
$\#(A)>\kappa_n$ and
$f:[A]^n\to B$ is a function, there is a
$C\in[A]^{\kappa^+}$ such that $f$ is constant on $[C]^n$.
\prooflet{({\smc Erd\H{o}s Hajnal M\'at\'e \& Rado 84}, 16.5;
{\smc Kanamori 03}, 7.3;  {\smc Just \& Weese 97}, 15.13.)}

\spheader 5A1Gb Let $\kappa$ be a cardinal of uncountable cofinality, and
$Q\subseteq[\kappa]^2$.   Then {\it either} there is a stationary
$A\subseteq\kappa$ such that $[A]^2\subseteq Q$ {\it or} there
is an infinite closed $B\subseteq\kappa$ such that
$[B]^2\cap Q=\emptyset$.
\prooflet{\Prf\ (Cf.\ {\smc Erd\H{o}s Hajnal M\'at\'e \& Rado 84}, 11.3.)   Let $C\subseteq\kappa$ be a closed
cofinal set with $\otp(C)=\cf\kappa$ (5A1Ad).
Let $S_0$ be $\{\alpha:\alpha\in C$, $\cf\alpha=\omega\}$, so that
$S_0$ is stationary (5A1Ac).   For each $\alpha\in S_0$ let
$\sequencen{f_{\alpha}(n)}$
be a strictly increasing sequence in $\alpha$ with supremum $\alpha$.
Set

\Centerline{$\Cal I_{\alpha}
=\{I:I\subseteq\alpha\cap C$, $[I\cup\{\alpha\}]^2\cap Q=\emptyset$,
$\#(I\cap f_{\alpha}(n))\le n$ for every $n\in\Bbb N\}$.}

\noindent Let $I_{\alpha}$ be a maximal member of $\Cal I_{\alpha}$.
If there is any $\alpha$ such that $I_{\alpha}$ is infinite, we have the
second alternative, witnessed by $B=I_{\alpha}\cup\{\alpha\}$,
and we can stop.   Otherwise, there is an $n\in\Bbb N$
such that $S_1=\{\alpha:\alpha\in S_0$,
$I_{\alpha}\subseteq f_{\alpha}(n)\}$ is stationary.   As
$f_{\alpha}(n)<\alpha$ for every $\alpha\in S_1$, the Pressing-Down
Lemma (4A1Cc) tells us that there is a $\gamma<\kappa$ such that
$S_2=\{\alpha:\alpha\in S_1$, $f_{\alpha}(n)=\gamma\}$ is stationary.
Because

\Centerline{$\#([\gamma\cap C]^{<\omega})\le\max(\omega,\#(\gamma\cap C))
<\cf\kappa$,}

\noindent there is an $I\subseteq\gamma\cap C$
such that $A=\{\alpha:\alpha\in S_2$, $I_{\alpha}=I\}$ is stationary.

\Quer\ Suppose, if possible, that $[A]^2\not\subseteq Q$.  Take
$\alpha$, $\beta\in A$ such that $\alpha<\beta$ and
$\{\alpha,\beta\}\notin Q$.   We know that
$[I\cup\{\alpha\}]^2$ and $[I\cup\{\beta\}]^2$ are both disjoint from $Q$.
So $[J\cup\{\beta\}]^2$ is disjoint from $Q$, where $J=I\cup\{\alpha\}$.
If $m\le n$,

\Centerline{$f_{\beta}(m)\le f_{\beta}(n)=\gamma
=f_{\alpha}(n)<\alpha$,}

\noindent so
$\#(J\cap f_{\beta}(m))=\#(I_{\beta}\cap f_{\beta}(m))\le m$;
while if $m>n$ then
$\#(J\cap f_{\beta}(m))\le\#(I_{\beta}\cap f_{\beta}(n))+1\le m$.
So $J\in\Cal I_{\beta}$;  but $J$ properly includes $I_{\beta}$, so this is
impossible.\ \Bang

Thus $[A]^2\subseteq Q$ and we have the first alternative.\ \Qed}
%5{}31S

\leader{5A1H}{$\Delta$-systems and free sets:  Proposition}
Let $\kappa$ and $\lambda$ be infinite cardinals
and $\ofamily{\xi}{\kappa}{I_{\xi}}$ a family of
sets of size less than $\lambda$.

(a) If $\cf\kappa>\lambda$, there are a
$\Gamma\in[\kappa]^{\kappa}$ and a set $J$ of
cardinal less than $\kappa$ such that $I_{\xi}\cap I_{\eta}\subseteq J$
for all distinct $\xi$, $\eta\in\Gamma$.
%5{}25J \family{\xi}{\Gamma}{I_{\xi}} is a Delta-nebula with
%  root-cover  J

(b) If $\kappa>\lambda$ is regular and
the cardinal power $\theta^{\delta}$ is less than $\kappa$ for every
$\theta<\kappa$ and $\delta<\lambda$, then
there is a $\Gamma'\in[\kappa]^{\kappa}$ such
that $\family{\xi}{\Gamma'}{I_{\xi}}$ is a $\Delta$-system.
%5{}25N

(c) If $\kappa>\lambda$ there is a $\Gamma''\in[\kappa]^{\kappa}$ such
that $\eta\notin I_{\xi}$ for any distinct $\xi$, $\eta\in\Gamma''$.
%5{}31K

\proof{{\bf (a)} \Quer\ Otherwise, choose
$\ofamily{\alpha}{\lambda}{\Gamma_{\alpha}}$ and
$\ofamily{\alpha}{\lambda}{J_{\alpha}}$ as follows.
$J_{\alpha}=\bigcup_{\beta<\alpha}\bigcup_{\xi\in\Gamma_{\beta}}I_{\xi}$.
Given $J_{\alpha}$, let $\Gamma_{\alpha}\subseteq\kappa$ be maximal subject
to the requirement that $I_{\xi}\cap I_{\eta}\subseteq J_{\alpha}$ for all
distinct $\xi$, $\eta\in\Gamma_{\alpha}$.   Then we see by induction that
$\#(J_{\alpha})<\kappa$ so $\#(\Gamma_{\alpha})<\kappa$ for every
$\alpha<\lambda$;  because $\cf\kappa>\lambda$,
$\bigcup_{\alpha<\lambda}\Gamma_{\alpha}$ cannot be the whole of $\kappa$.

Take any $\xi\in\kappa\setminus\bigcup_{\alpha<\lambda}\Gamma_{\alpha}$.
As $\#(I_{\xi})<\lambda$, there must be an $\alpha<\lambda$ such that
$I_{\xi}\cap J_{\alpha}=I_{\xi}\cap J_{\alpha+1}$.
As $\xi\notin\Gamma_{\alpha}$,
there is an $\eta\in\Gamma_{\alpha}$ such that
$I_{\xi}\cap I_{\eta}\not\subseteq J_{\alpha}$;  but now
$I_{\xi}\cap I_{\eta}\setminus J_{\alpha}
\subseteq I_{\xi}\cap J_{\alpha+1}\setminus J_{\alpha}$.\ \Bang

\medskip

{\bf (b)} Let $J$ and $\Gamma$ be as in (a).   Because $\cf\kappa>\lambda$,
there must be some cardinal $\delta<\lambda$ such that
$\Gamma_1=\{\xi:\xi\in\Gamma$, $\#(I_{\xi}\cap J)\le\delta\}$ has cardinal
$\kappa$.   Now
$\#([J]^{\le\delta})\le\#(J)^{\delta}<\cf\kappa$, so there must be a
$K\subseteq J$ such that
$\Gamma'=\{\xi:\xi\in\Gamma_1$, $I_{\xi}\cap J=K\}$
has cardinal $\kappa$;  and $\family{\xi}{\Gamma'}{I_{\xi}}$ is a
$\Delta$-system with root $K$.

\medskip

{\bf (c)} It is enough to consider the case in which $\xi\in I_{\xi}$ for
every $\xi<\kappa$.

\medskip

\quad{\bf (i)} If $\cf\kappa>\lambda$,
take $\Gamma$ and $J$ from (a).   Then we can choose
$\ofamily{\delta}{\kappa}{\xi_{\delta}}$ inductively so that

\Centerline{$\xi_{\delta}
\in\Gamma\setminus(J\cup\bigcup_{\beta<\delta}I_{\xi_{\beta}})$}

\noindent for every $\delta<\kappa$;  and $\{\xi_{\delta}:\delta<\kappa\}$
will serve for $\Gamma''$.

\medskip

\quad{\bf (ii)} If $\cf\kappa=\theta\le\lambda$, let
$\ofamily{\alpha}{\theta}{\kappa_{\alpha}}$ be a strictly increasing family
of regular cardinals with supremum $\kappa$, starting from
$\kappa_0\ge\lambda^{++}$.   For each $\alpha<\theta$, (i) tells us that there is
an $A_{\alpha}\in[\kappa_{\alpha}]^{\kappa_{\alpha}}$ such that
$\eta\notin I_{\xi}$ for any distinct $\xi$, $\eta\in A_{\alpha}$.   Set

\Centerline{$B_{\alpha}
=A_{\alpha}\setminus\bigcup_{\beta<\alpha}
  (B_{\beta}\cup\bigcup_{\xi\in A_{\beta}}I_{\xi})$;}

\noindent then $\#(B_{\alpha})=\kappa_{\alpha}$
for each $\alpha<\theta$.   Choose
$\langle C_{\alpha\gamma}\rangle_{\alpha<\theta,\gamma<\lambda^+}$ and
$\ofamily{\alpha}{\theta}{\zeta_{\alpha}}$
inductively, as follows.   Given that
$\langle C_{\beta\gamma}\rangle_{\beta<\alpha,\gamma<\lambda^+}$ is
disjoint, then for each $\xi\in B_{\alpha}$ there is a $\zeta<\lambda^+$ such
that $I_{\xi}\cap\bigcup_{\beta<\alpha}C_{\beta\gamma}$ is empty for every
$\gamma\ge\zeta$;  because $\lambda^+<\cf\kappa_{\alpha}$, there is a
$\zeta_{\alpha}<\lambda^+$ such that

\Centerline{$B'_{\alpha}=\{\xi:\xi\in B_{\alpha}$,
$I_{\xi}\cap C_{\beta\gamma}=\emptyset$ whenever $\beta<\alpha$ and
$\zeta_{\alpha}\le\gamma<\lambda^+\}$}

\noindent has cardinal $\kappa_{\alpha}$.   Let
$\ofamily{\gamma}{\lambda^+}{C_{\alpha\gamma}}$ be a partition of
$B'_{\alpha}$ into sets of size $\kappa_{\alpha}$, and continue.

At the end of the induction, $\gamma=\sup_{\alpha<\theta}\zeta_{\alpha}$ is
less than $\lambda^+$.   Set
$\Gamma''=\bigcup_{\alpha<\theta}C_{\alpha\gamma}$.   Then
$\#(\Gamma'')=\kappa$.   If $\xi$, $\eta$ are
distinct members of $\Gamma''$, let $\alpha$, $\beta<\theta$ be such that
$\xi\in C_{\alpha\gamma}$ and $\eta\in C_{\beta\gamma}$.   If
$\alpha<\beta$ then $\xi\in A_{\alpha}$ and $\eta\in B_{\beta}$ so
$\eta\notin I_{\xi}$.   If $\alpha=\beta$ then both $\xi$ and $\eta$ belong
to $A_{\alpha}$ so $\eta\notin I_{\xi}$.   If $\beta<\alpha$ then
$\eta\in C_{\beta\gamma}$ while $\gamma\ge\zeta_{\alpha}$ and
$\xi\in B'_{\alpha}$, so $\eta\notin I_{\xi}$.   So $\Gamma''$ will serve.
}%end of proof of 5A1H

\cmmnt{\medskip

\noindent{\bf Remark} (c) above is Hajnal's Free Set Theorem.}

\leader{5A1I}{Remarks}{\bf (a)}\cmmnt{ I spell out the applications of
these results which are used in this volume.}
Let $\kappa$ be an infinite cardinal and
$\ofamily{\xi}{\kappa}{I_{\xi}}$ a family of countable sets.

\quad(i) If $\cf\kappa\ge\omega_2$, there are a
$\Gamma\in[\kappa]^{\kappa}$ and a set $J$ with cardinal less than $\kappa$
such that $I_{\xi}\cap I_{\eta}\subseteq J$ for all distinct $\xi$,
$\eta\in\Gamma$.

\quad(ii) If $\kappa$ is infinite and regular and
the cardinal power $\lambda^{\omega}$ is less than $\kappa$ for every
$\lambda<\kappa$, there is a $\Gamma'\in[\kappa]^{\kappa}$ such
that $\family{\xi}{\Gamma'}{I_{\xi}}$ is a $\Delta$-system.
\prooflet{(Of course $\kappa$ cannot be $\omega_1$, so we can apply 5A1Hc
with $\lambda=\omega_1$.)}
%5{}25N

\quad(iii) If $\kappa\ge\omega_2$ there is a $\Gamma''\in[\kappa]^{\kappa}$ such
that $\eta\notin I_{\xi}$ for any distinct $\xi$, $\eta\in\Gamma''$.
%5{}31K

\spheader 5A1Ib\cmmnt{ If, in 5A1Hc,
we are willing to settle for a weaker result,
there is an easier proof which generalizes to more complex systems.}
Let $\lambda$ be an infinite cardinal.   Then there is a $\kappa_0$ such
that for every cardinal $\kappa\ge\kappa_0$, every $n\in\Bbb N$
and every function $f:[\kappa]^n\to[\kappa]^{<\lambda}$ there is an
$A\in[\kappa]^{\lambda^+}$ such that $\xi\notin f(I)$ whenever
$I\in[A]^n$ and $\xi\in A\setminus I$.
\prooflet{\Prf\ By the Erd\H{o}s-Rado theorem (5A1Ga),
there is a $\kappa_0$ such that for every $\kappa\ge\kappa_0$, $n\ge 1$ and
function $g:[\kappa]^n\to\Bbb N$ there is an $A\in[\kappa]^{\lambda^+}$
such that $g$ is constant on $[A]^n$.   Now, given $n\in\Bbb N$,
$\kappa\ge\kappa_0$
and $f:[\kappa]^n\to[\kappa]^{<\lambda}$, define $g:[\kappa]^{n+1}\to\Bbb N$
by saying that if $J=\{\xi_0,\ldots,\xi_n\}$ with
$\xi_0<\xi_1<\ldots<\xi_n$, then
$g(J)=\min(\{n+1\}\cup\{j:j\le n$, $\xi_j\in f(J\setminus\{\xi_j\})\}$.
Let $A\in[\kappa]^{\lambda^+}$ be such that $g$ is constant on
$[A]^{n+1}$.   We can suppose that $A$ has order type $\lambda^+$.
\Quer\ Suppose that the constant value of $g$ in $[A]^{n+1}$ is $j\le n$.
Let $B$ be the set of the first $\lambda$ members of $A$, $I_0$ the set of
the first $j$ members of $A$ and $I_1$ the set of the first $n-j$ members
of $A\setminus B$.   Then we have $g(I_0\cup\{\xi\}\cup I_1)=j$ for every
$\xi\in B\setminus I_0$, so that
$B\setminus I_0\subseteq f(I_0\cup I_1)$;  but
$\#(f(I_0\cup I_1))<\lambda$.\ \BanG\
So the constant value of $g$ on $[A]^{n+1}$ is $n+1$, and $A$ satisfies the
required condition.\ \Qed}

\spheader 5A1Ic\cmmnt{ In the same complex of ideas,
we have an elementary
fact about the case $\lambda<\kappa=\omega$.}   If $n\in\Bbb N$ and
$\sequence{i}{K_i}$ is a sequence in $[\Bbb N]^{\le n}$,
there is an infinite
$\Gamma\subseteq\Bbb N$ such that $\family{i}{\Gamma}{K_i}$ is a
$\Delta$-system.   \prooflet{\Prf\
Let $K\subseteq\Bbb N$ be a maximal set such that
$I=\{i:K\subseteq K_i\}$ is infinite;  then
$\{i:i\in I$, $K_i\cap L\ne\emptyset\}$ is finite for every finite
$L\subseteq\Bbb N\setminus K$, so we can choose $\Gamma$ inductively
by saying that
$\Gamma=\{i:i\in I$, $K_i\cap K_j=K$ whenever $j\in\Gamma\cap i\}$.\ \Qed}

\leader{5A1J}{Lemma} Suppose that $\theta$, $\lambda$ and $\kappa$
are cardinals, with $\theta<\lambda<\cf\kappa$, and that $S$
is a stationary subset of
$\kappa$.   Let $\langle I_{\xi}\rangle_{\xi\in S}$ be a family in
$[\lambda]^{\le\theta}$.
Then there is a set $M\subseteq\lambda$ such that
$\cf(\#(M))\le\theta$
and $\{\xi:\xi\in S,\,I_{\xi}\subseteq M\}$ is stationary in $\kappa$.
%5{}43 5A1M

\proof{ For $M\subseteq\lambda$, set
$S_M=\{\xi:\xi\in S,\,I_{\xi}\subseteq M\}$.
Let $M\subseteq\lambda$ be a set of minimal cardinality such
that $S_M$ is stationary in $\kappa$.   Set $\delta=\#(M)$.  \Quer\  If
$\cf\delta>\theta$, enumerate $M$ as
$\langle\alpha_{\eta}\rangle_{\eta<\delta}$. For each $\xi\in S_M$,
set $\beta_{\xi}=\sup\{\eta:\alpha_{\eta}\in I_{\xi}\}$;
because $\#(I_{\xi})\le\theta<\cf\delta$,
$\beta_{\xi}<\delta$.   Because $\delta\le\lambda<\cf\kappa$,
there is a $\beta<\delta$ such that
$S'=\{\xi:\xi\in S_M$, $\beta_{\xi}=\beta\}$ is stationary in $\kappa$
(5A1Ab).   Consider $M'=\{\alpha_{\eta}:\eta\le\beta\}$;  then
$\#(M')<\#(M)$  but $S_{M'}\supseteq S'$ so is stationary in
$\kappa$, contrary to the choice of $M$.\ \Bang

Thus $M$ and $S=S_M$ will serve.
}%end of proof of 5A1J

\leader{5A1K}{Lemma} Let $\familyiI{X_i}$ be a non-empty family of
infinite sets, with product $X$.   Then there is a
set $Y\subseteq X$, with $\#(Y)=\#(X)$, such that for every finite
$L\subseteq Y$ there is an $i\in I$ such that
$x(i)\ne y(i)$ for any distinct $x$, $y\in L$.
%5{}15G  sort of transversals

\proof{ Set $\kappa=\#(X)$.

\medskip

{\bf (a)} We can well-order $I$ in such a way that
$\#(X_i)\le\#(X_j)$ whenever $i\le j$ in $I$.   It will therefore be
enough to deal with the case in which $I=\delta$ is an ordinal and
$\#(X_{\alpha})\le\#(X_{\beta})$ whenever $\alpha\le\beta<\delta$.
I proceed by induction on $\delta$.

\medskip

{\bf (b)} If $\delta$ is finite then
$\kappa=\max_{\alpha<\delta}\#(X_{\alpha})$ and the result is trivial,
since we can take the $x_{\xi}$ to be all different at a single
coordinate.

\medskip

{\bf (c)} Suppose there is a $\gamma<\delta$ such that
$\#(\delta\setminus\gamma)<\#(\delta)$.  Then, in particular, the order
type of $\delta\setminus\gamma$ is less than the order type of
$\delta$.   Set $I_0=\gamma$, $I_1=\delta\setminus\gamma$ and
$Y_j=\prod_{\alpha\in I_j}X_{\alpha}$ for both $j$.   Then
$X\cong Y_0\times Y_1$, so $\kappa=\max(\#(Y_0),\#(Y_1))$;  say
$\kappa=\#(Y_j)$.   By the inductive hypothesis, there is a family
$\ofamily{\xi}{\kappa}{y_{\xi}}$ in $Y_j$ such that for any
$L\in[\kappa]^{<\omega}$ there is an $\alpha\in I_j$ such that
$\xi\mapsto y_{\xi}(\alpha):L\to X_{\alpha}$ is injective.   Taking
$x_{\xi}$ to be any member of $X$ extending $y_{\xi}$, for each
$\xi<\kappa$, we have a suitable family
$\ofamily{\xi}{\kappa}{x_{\xi}}$
in $X$, and the induction proceeds.

\medskip

{\bf (d)} Suppose that $\delta$ is infinite and that
$\#(\delta\setminus\gamma)=\#(\delta)=\lambda$ for every
$\gamma<\delta$.   Enumerate $[\delta]^{<\omega}$ as
$\ofamily{\zeta}{\lambda}{J_{\zeta}}$, and choose
$\ofamily{\zeta}{\lambda}{\alpha_{\zeta}}$ such that

\Centerline{$J_{\zeta}
\subseteq\alpha_{\zeta}\in\delta\setminus\{\alpha_{\eta}:\eta<\zeta\}$}

\noindent for each $\alpha<\lambda$.   We have

\Centerline{$\#(X_{\alpha_{\zeta}})
\ge\max(\omega,\sup_{\beta\in J_{\zeta}}\#(X_{\beta}))
\ge\#(\prod_{\beta\in J_{\zeta}}X_{\beta})$,}

\noindent so there is an injective function
$f_{\zeta}:\prod_{\beta\in J_{\zeta}}X_{\beta}\to X_{\alpha_{\zeta}}$
for each $\zeta<\lambda$.
Let $\ofamily{\xi}{\kappa}{z_{\xi}}$ be any enumeration of $X$.
Because all the $\alpha_{\zeta}$ are distinct, we can find
$x_{\xi}\in X$, for each $\xi<\kappa$, such that
$x_{\xi}(\alpha_{\zeta})=f_{\zeta}(z_{\xi}\restr J_{\zeta})$ for every
$\zeta$.   Now if $L\in[\kappa]^{<\omega}$ there must be a
$\zeta<\lambda$ such that
$z_{\xi}\restr J_{\zeta}\ne z_{\eta}\restr J_{\zeta}$
for any distinct $\xi$, $\eta\in L$;  so that
$\xi\mapsto x_{\xi}(\alpha_{\zeta})$ is injective on $L$.   Thus
$\ofamily{\xi}{\kappa}{x_{\xi}}$ is a suitable family in $X$ and the
induction proceeds in this case also.
}%end of proof of 5A1K

\leader{5A1L}{Definitions (a)} Let $X$ and $Y$ be sets and
$\Cal I$ an ideal of subsets of $X$.   Write
$\Tr_{\Cal I}(X;Y)$ for the {\bf transversal number}

\Centerline{$\sup\{\#(F):F\subseteq Y^X$,
$\{x:f(x)=g(x)\}\in\Cal I$ for all distinct $f,\,g\in F\}$.}

\spheader 5A1Lb Let $\kappa$ be a cardinal.  Write
$\Tr(\kappa)$ for

\Centerline{$\Tr_{[\kappa]^{<\kappa}}(\kappa;\kappa)
=\sup\{\#(F):F\subseteq\kappa^{\kappa}$,
$\#(f\cap g)<\kappa\text{ for all distinct }f,\,g\in F\}$.}

\leader{5A1M}{Lemma} (a) For any infinite cardinal $\kappa$,

\Centerline{$\kappa^+\le\Tr(\kappa)\le 2^{\kappa}$.}

(b) For any infinite cardinal $\kappa$,

\Centerline{$\max(\Tr(\kappa),\sup_{\delta<\kappa}2^{\delta})
\ge\min(2^{\kappa},\kappa^{(+\omega)})$.}

(c) If $\kappa$ is such that $2^{\delta}\le\kappa$ for
every $\delta<\kappa$, then $\Tr(\kappa)=2^{\kappa}$, and in fact there is
an $F\subseteq\kappa^{\kappa}$ such that $\#(F)=2^{\kappa}$ and
$\#(f\cap g)<\kappa\text{ for all distinct }f,\,g\in F$.

(d) If $X$ and $Y$ are sets and $\Cal I$ is a maximal proper
ideal of $\Cal PX$, then there is
an $F\subseteq Y^X$ such that $\#(F)=\Tr_{\Cal I}(X;Y)$ and
$\{x:f(x)=g(x)\}\in\Cal I$ for all distinct $f,\,g\in F$.

\proof{{\bf (a)} We can build inductively a family
$\langle f_{\alpha}\rangle_{\alpha<\kappa^+}$ in $\kappa^{\kappa}$, as
follows. Given $\langle f_{\alpha}\rangle_{\alpha<\beta}$, where
$\beta<\kappa^+$, let $\theta:\beta\to\kappa$ be any injection.
Now choose $f_{\beta}:\kappa\to\kappa$ so that

\Centerline{$f_{\beta}(\xi)\ne f_{\alpha}(\xi)$
whenever $\alpha<\beta$ and $\theta(\alpha)\le\xi$.}

\noindent This will mean that if $\alpha<\beta$, then

\Centerline{$\{\xi:f_{\alpha}(\xi)=f_{\beta}(\xi)\}
\subseteq\theta(\alpha)$}

\noindent has cardinal less than $\kappa$.   So at the
end of the induction, $F=\{f_{\alpha}:\alpha<\kappa^+\}$
will witness that $\Tr(\kappa)\ge\kappa^+$.   On the other hand,
$\Tr(\kappa)\le\#(\kappa^{\kappa})=2^{\kappa}$.

\medskip

{\bf (b)}\Quer\ If not, then take
$\lambda=\max(\Tr(\kappa),\sup_{\delta<\kappa}2^{\delta})
<\min(2^{\kappa},\kappa^{(+\omega)})$.
For each $\xi<\kappa$ take an
injective function $\phi_{\xi}:\Cal P\xi\to\lambda$.
Because $\lambda<2^{\kappa}$, we have an injective function
$h:\lambda^+\to\Cal P\kappa$.   For $\alpha<\lambda^+$ set
$g_{\alpha}(\xi)=\phi_{\xi}(h(\alpha)\cap\xi)$
for every $\xi<\kappa$;  then
$\langle g_{\alpha}\rangle_{\alpha<\lambda^+}$
is a family in $\lambda^{\kappa}$ such that
$\#(g_{\alpha}\cap g_{\beta})<\kappa$ whenever $\alpha\ne\beta$.

Apply 5A1J with $S=\lambda^+$,
$I_{\alpha}=g_{\alpha}[\kappa]$ to see that there is a set
$M\subseteq\lambda$ with $\cf(\#(M))\le\kappa$ and
$S_1=\{\alpha:\alpha<\lambda^+$, $g_{\alpha}[\kappa]\subseteq M\}$
stationary in $\lambda^+$.   Because
$\lambda<\kappa^{(+\omega)}$, we must have $\#(M)\le\kappa$.   If
$f:M\to\kappa$ is any injection,
$\langle fg_{\alpha}\rangle_{\alpha\in S}$ will witness
that $\Tr(\kappa)\ge\#(S_1)=\lambda^+$;  which is impossible.\ \Bang

\wheader{5A1M}{4}{2}{2}{36pt}

{\bf (c)} For each $\xi<\kappa$, let $\phi_{\xi}:\Cal P\xi\to\kappa$
be injective.   For $A\subseteq\kappa$, define
$f_A\in\kappa^{\kappa}$ by writing

\Centerline{$f_A(\xi)=\phi_{\xi}(A\cap\xi)$ for every $\xi<\kappa$.}

\noindent Then $F=\{f_A:A\subseteq\kappa\}$ has the required property, and
$\Tr(\kappa)\ge 2^{\kappa}$;  by (a), we have equality.

\medskip

{\bf (d)} Take any maximal set $F\subseteq Y^X$ such that
$\{x:f(x)=f'(x)\}\in\Cal I$ for all distinct $f$, $f'\in F$.   Then
$\#(F)=\Tr_{\Cal I}(X;Y)$.   \Prf\ Of course $\#(F)\le\Tr_{\Cal I}(X;Y)$.
Suppose that $G\subseteq Y^X$ is such that
$\{x:g(x)=g'(x)\}\in\Cal I$ for all distinct $g$, $g'\in G$.   For each
$g\in G$, there must be an $f_g\in F$ such that
$\{x:g(x)=f_g(x)\}\notin\Cal I$;  because $\Cal I$ is maximal,
$\{x:g(x)\ne f_g(x)\}\in\Cal I$.   If $g$, $h\in G$ are distinct,
then

\Centerline{$\{x:f_g(x)=g_h(x)\}
\subseteq\{x:f_g(x)\ne g(x)\}\cup\{x:g(x)=h(x)\}\cup\{x:h(x)\ne f_h(x)\}
\in\Cal I$}

\noindent so $f_g\ne f_h$.   Thus we have an injective function from $G$ to
$F$ and $\#(G)\le\#(F)$.   As $G$ is arbitrary,
$\Tr_{\Cal I}(X;Y)\le\#(F)$ and we have equality.\ \Qed
}%end of proof of 5A1M
%5{}42 %5{}38Xk

\leader{5A1N}{Almost-square-sequences:  Lemma}
Let $\kappa$, $\lambda$ be regular infinite cardinals, with
$\lambda>\max(\omega_1,\kappa)$.
Then we can find a stationary set $S\subseteq\lambda^+$ and a family
$\langle C_{\alpha}\rangle_{\alpha\in S}$ of sets such that

(i) for each $\alpha\in S$, $C_{\alpha}$ is a closed cofinal set in
$\alpha$ of order type $\kappa$;

(ii) if $\alpha$, $\beta\in S$ and $\gamma$ is a limit
point of both $C_{\alpha}$ and $C_{\beta}$ then
$C_{\alpha}\cap\gamma=C_{\beta}\cap\gamma$.

\proof{{\bf (a)} For each $\gamma<\lambda^+$ fix an injection
$f_{\gamma}:\gamma\to\lambda$.   Let $S_0$ be the set of ordinals
$\alpha<\lambda^+$ of cofinality $\kappa$;  then $S_0$ is stationary
in $\lambda^+$ (5A1Ac).   For each $\alpha\in S_0$ choose a non-decreasing
family $\langle N_{\alpha\delta}\rangle_{\delta<\lambda}$ of subsets of
$\lambda^+$ such that

\inset{($\alpha$) $N_{\alpha0}$ is a cofinal subset of
$\alpha$ with cardinal $\kappa$;

($\beta$) if $\delta<\lambda$ then

\Centerline{$N_{\alpha,\delta+1}
=\bigcup\{f_{\gamma}[N_{\alpha\delta}]\cup f_{\gamma}^{-1}[\delta]:
    \gamma\in N_{\alpha\delta}\}
  \cup\overline{N}_{\alpha\delta}\cup\delta$}

\noindent (taking the closure $\overline{N}_{\alpha\delta}$ in the
order topology of $\lambda^+$);

($\gamma$) if $\delta<\lambda$ is a non-zero limit
ordinal then
$N_{\alpha\delta}=\bigcup_{\delta'<\delta}N_{\alpha\delta'}$.}

\noindent Then
$\#(N_{\alpha\delta})\le\max(\kappa,\#(\delta))<\lambda$
for each $\delta<\lambda$ (using 5A1Ae).   Because $\lambda$ is regular,
$\sup(N_{\alpha\delta}\cap\lambda)<\lambda$ for every $\delta$.
It follows that
$\{\delta:\delta<\lambda$, $N_{\alpha\delta}\cap\lambda=\delta\}$ is a
closed cofinal set in $\lambda$, and in particular
contains an ordinal of cofinality $\omega_1$,
for every $\alpha\in S_0$.   Let $\delta<\lambda$ be
such that $\cf\delta=\omega_1$ and

\Centerline{$S_1=\{\alpha:\alpha\in S_0$,
$N_{\alpha\delta}\cap\lambda=\delta\}$}

\noindent is stationary in $\lambda^+$.   For $\alpha\in S_1$, set
$C^*_{\alpha}=\alpha\cap\overline{N}_{\alpha\delta}$;
then $C^*_{\alpha}$ is a closed cofinal set in $\alpha$ and
$\#(C^*_{\alpha})<\lambda$ so $\otp(C^*_{\alpha})<\lambda$.   Let
$\zeta<\lambda$ be such that

\Centerline{$S=\{\alpha:\alpha\in S_1$, $\otp(C^*_{\alpha})=\zeta\}$}

\noindent is stationary in $\lambda^+$.   Observe that
as $\cf C^*_{\alpha}=\cf\alpha=\kappa$ for each $\alpha\in S$,
$\cf\zeta=\kappa$.

\medskip

{\bf (b)} Take any closed cofinal set $C\subseteq\zeta$ of order
type $\kappa$ and for each $\alpha\in S$ let $C_{\alpha}$ be the image
of $C$ in $C^*_{\alpha}$ under the order-isomorphism between
$\zeta$ and $C^*_{\alpha}$.   Then $C_{\alpha}$ will be a closed
cofinal subset of $\alpha$ of order type $\kappa$.

I claim that if $\alpha$, $\beta\in S$ and $\gamma$
is a common limit point of $C_{\alpha}$, $C_{\beta}$ then
$C_{\alpha}\cap\gamma=C_{\beta}\cap\gamma$.

\Prf\ {\bf case 1} Suppose $\kappa=\omega$.   In this case the only
limit point of $C_{\alpha}$
will be $\alpha$ itself, and similarly for $\beta$, so that in this
case
we have $\alpha=\beta$ and there is nothing more to do.

{\bf case 2} Suppose $\cf\gamma=\omega<\kappa$.
Then $\gamma$ is a limit point of
$C_{\alpha}\subseteq C^*_{\alpha}\subseteq\overline{N}_{\alpha\delta}$, so
there is an increasing sequence in $N_{\alpha\delta}$ with supremum
$\gamma$;  as
$N_{\alpha\delta}=\bigcup_{\delta'<\delta}N_{\alpha\delta'}$ and
$\cf\delta=\omega_1$, this sequence lies entirely within
$N_{\alpha\delta'}$ for some $\delta'<\delta$, and
$\gamma\in\overline{N}_{\alpha\delta'}\subseteq N_{\alpha,\delta'+1}$.
Now, for $\delta'+1\le\xi<\delta$, $N_{\alpha,\xi+1}
\supseteq f_{\gamma}^{-1}[\xi]\cup f_{\gamma}[N_{\alpha\xi}]$;
consequently

\Centerline{$N_{\alpha\delta}\cap\gamma
=f_{\gamma}^{-1}[N_{\alpha\delta}\cap\lambda]
=f_{\gamma}^{-1}[\delta]$.}

\noindent Similarly,
$N_{\beta\delta}\cap\gamma=f_{\gamma}^{-1}[\delta]$.   Now

\Centerline{$C^*_{\alpha}\cap\gamma
=\overline{N}_{\alpha\delta}\cap\gamma
=\overline{f_{\gamma}^{-1}[\delta]}\cap\gamma
=C^*_{\beta}\cap\gamma$.}

\noindent Accordingly the increasing enumerations of $C^*_{\alpha}$ and
$C^*_{\beta}$ must agree on
$C^*_{\alpha}\cap\gamma=C^*_{\beta}\cap\gamma$, and
$C_{\alpha}\cap\gamma=C_{\beta}\cap\gamma$.

{\bf case 3} Suppose that $\cf\gamma>\omega$ and $\kappa>\omega$.
Because $\gamma=\sup(C_{\alpha}\cap\gamma)=\sup(C_{\beta}\cap\gamma)$,

\Centerline{$D
=\{\gamma':\gamma'<\gamma$ is a limit point of both $C_{\alpha}$ and
$C_{\beta}$, $\cf\gamma'=\omega\}$}

\noindent is cofinal with $\gamma$, and

\Centerline{$C_{\alpha}\cap\gamma
=\bigcup_{\gamma'\in D}C_{\alpha}\cap\gamma'=C_{\beta}\cap\gamma$,}

\noindent using case 2.\ \Qed

Thus $S$ and $\langle C_{\alpha}\rangle_{\alpha\in S}$ have
the required properties.
}%end of proof of 5A1N
%5{}A1O

\leader{5A1O}{Corollary} Let $\kappa$, $\lambda$ be regular infinite
cardinals with $\lambda>\max(\omega_1,\kappa)$.
Then we can find a stationary subset $S$ of $\lambda^+$ and a family
$\langle g_{\alpha}\rangle_{\alpha\in S}$ of functions
from $\kappa$ to $\lambda^+$ such that, for all distinct $\alpha$,
$\beta\in S$,

(i) $g_{\alpha}[\kappa]\subseteq\alpha$,

(ii) $\#(g_{\alpha}\cap g_{\beta})<\kappa$,

(iii) if $\theta<\kappa$ is a limit ordinal and
$g_{\alpha}(\theta)=g_{\beta}(\theta)$ then
$g_{\alpha}\restr\theta=g_{\beta}\restr\theta$.

\proof{ Take $\langle C_{\alpha}\rangle_{\alpha\in S}$ from
5A1N above and let $g_{\alpha}$ be the increasing
enumeration of $C_{\alpha}$.
}%end of proof of 5A1O
%5{}43E

%\exercises{\leader{5A1X}{Basic exercises (a)}

%\leader{5A1Y}{Further exercises (a)}

%}%end of exercises

%\endnotes{
%\Notesheader{5A1}

%}%end of notes

\discrpage

