\frfilename{mt465.tex}
\versiondate{28.1.06}
\copyrightdate{1999}

\def\chaptername{Pointwise compact sets of measurable functions}
\def\sectionname{Stable sets}

\newsection{465}

The structure of general pointwise compact sets of measurable functions
is complex and elusive.   One particular class of such sets, however, is
relatively easy to describe, and has a variety of remarkable properties,
some of them relevant to important questions arising in the theory of
empirical measures.   In this section I outline the theory of `stable'
sets of measurable functions from {\smc Talagrand 84} and 
{\smc Talagrand 87}.

The first steps are straightforward enough.   The definition of stable
set (465B) is not obvious, but given this the basic properties of stable
sets listed in 465C are natural and easy to check, and we come quickly
to the fact that (for complete locally determined spaces) pointwise
bounded stable sets are relatively pointwise compact sets of measurable
functions (465D).   A less transparent, but still fairly elementary,
argument leads to the next reason for looking at stable sets:  the
topology of pointwise convervence on a stable set is finer than the
topology of convergence in measure (465G).

At this point we come to a remarkable fact:  a uniformly bounded set $A$
of functions on a complete probability space is stable if and only if
certain laws of large numbers apply `nearly uniformly' on $A$.   These
laws are expressed in conditions (ii), (iv) and (v) of 465M.   For
singleton sets $A$, they can be thought of as versions of the strong
law of large numbers described in \S273.   To get the full strength of
465M a further idea in this direction needs to be added,
described in 465H here.

The theory of stable sets applies in the first place to sets of true
functions.   There is however a corresponding notion applicable in
function spaces, which I explore briefly in 465O-465R.   Finally, I
mention the idea of `R-stable' set (465S-465U), obtained by using
$\tau$-additive product measures instead of c.l.d.\ product measures in
the definition.

\leader{465A}{Notation}\cmmnt{ Throughout this section, I will use
the following notation.

\medskip

} {\bf (a)} If $X$ is a set and $\Sigma$ a $\sigma$-algebra of subsets of
$X$, I will write $\eusm L^0(\Sigma)$ for the space of
$\Sigma$-measurable functions from $X$ to $\Bbb R$\cmmnt{, as in
\S463}.

\spheader 465Ab I will identify $\Bbb N$ with the set of finite
ordinals, so that\cmmnt{ each $n\in\Bbb N$ is the set of its
predecessors, and} a power $X^n$ becomes identified with the set of
functions from $\{0,\ldots,n-1\}$ to $X$.

\spheader 465Ac If $(X,\Sigma,\mu)$ is any measure space, then for
finite sets $I$\cmmnt{ (in particular, if $I=k\in\Bbb N$)} I write
$\mu^I$ for the\cmmnt{ c.l.d.} product measure on $X^I$\cmmnt{, as
defined in 251W}.   \cmmnt{(For definiteness, let us take $\mu^0$ to
be the unique probability measure on $X^0=\{\emptyset\}$.)}   If
$(X,\Sigma,\mu)$ is a probability space, then for any set $I\,\,\mu^I$
is to be the product probability measure on $X^I$\cmmnt{, as defined
in \S254}.

\spheader 465Ad\cmmnt{ We shall have occasion to look at free powers
of algebras of sets.}   If $X$ is a set and $\Sigma$ is an algebra of
subsets of $X$, then for any set $I$ write $\bigotimes_I\Sigma$ for the
algebra of subsets of $X^I$ generated by sets of the form
$\{w:w(i)\in E\}$ where $i\in I$ and $E\in\Sigma$, and
$\Tensorhat_I\Sigma$ for the
$\sigma$-algebra generated by $\bigotimes_I\Sigma$.

\spheader 465Ae\cmmnt{ Now for a new idea, which will be used in
almost every paragraph of the section.}   If $X$ is a set,
$A\subseteq\Bbb R^X$\cmmnt{ a set of real-valued functions defined on
$X$}, $E\subseteq X$, $\alpha<\beta$ in $\Bbb R$ and $k\ge 1$, write

$$\eqalign{D_k(A,E,\alpha,\beta)
=\bigcup_{f\in A}\{w:w\in E^{2k},\,
&f(w(2i))\le\alpha,\cr
&f(w(2i+1))\ge\beta\text{ for }i<k\}.\cr}$$

\spheader 465Af\cmmnt{ In this context it will be useful to have a
special
notation.}   If $X$ is a set, $k\ge 1$, $u\in X^k$ and $v\in X^k$, then I
will write $u\#v=(u(0),v(0),u(1),v(1),\ldots,u(k-1),v(k-1))\in X^{2k}$.
Note that if $(X,\Sigma,\mu)$ is a measure space then
$(u,v)\mapsto u\#v$ is an isomorphism between the c.l.d.\ product
$(X^k,\mu^k)\times(X^k,\mu^k)$ and
$(X^{2k},\mu^{2k})$\cmmnt{ (see 251Wh)}.

\cmmnt{We are now ready for the main definition.}

\leader{465B}{Definition} Let $(X,\Sigma,\mu)$ be a semi-finite measure
space.   \cmmnt{Following {\smc Talagrand 84},} I say that a
set $A\subseteq\Bbb R^X$ is {\bf stable} if whenever $E\in\Sigma$,
$0<\mu
E<\infty$ and $\alpha<\beta$ in $\Bbb R$, there is some $k\ge 1$ such
that $(\mu^{2k})^*D_k(A,E,\alpha,\beta)<(\mu E)^{2k}$.

\cmmnt{\medskip

\noindent{\bf Remark} I hope that the next few results will show why
this concept is important.   It is worth noting at once that these sets
$D_k$ need not be measurable, and that some of the power of the
definition derives precisely from the fact that quite naturally arising
sets $A$ can give rise to non-measurable sets $D_k(A,E,\alpha,\beta)$.
If, however, the set $A$ is countable, then all the corresponding $D_k$
will be measurable;  this will be important in the results following
465R.
}%end of comment

\leader{465C}{}\cmmnt{ I start with a list of the `easy'
properties of stable sets, derivable more or less directly from the
definition.

\medskip

\noindent}{\bf Proposition} Let $(X,\Sigma,\mu)$ be a semi-finite
measure space.

(a) If $A\subseteq\Bbb R^X$ is stable, then any subset of $A$ is stable.

(b) If $A\subseteq\Bbb R^X$ is stable, then $\overline{A}$, the closure
of $A$ in $\Bbb R^X$ for the topology of pointwise convergence, is
stable.

\ifdim\pagewidth>467pt\fontdimen3\tenrm=2pt\fi
\ifdim\pagewidth>467pt\fontdimen4\tenrm=1.67pt\fi
(c) Suppose that $A\subseteq\Bbb R^X$, $E\in\Sigma$, $n\ge 1$ and
$\alpha<\beta$ are such that $0<\mu E<\infty$ and
$(\mu^{2n})^*D_n(A,E,\alpha,\beta)<(\mu E)^{2n}$.   Then
\fontdimen3\tenrm=1.67pt
\fontdimen4\tenrm=1.11pt

\Centerline{$\lim_{k\to\infty}
\Bover1{(\mu E)^{2k}}(\mu^{2k})^*D_k(A,E,\alpha,\beta)=0$.}

(d) If $A$, $B\subseteq\Bbb R^X$ are stable, then $A\cup B$ is stable.

(e) If $A\subseteq\Bbb R^X$ is stable, then $\gamma A=\{\gamma f:f\in
A\}$ is stable, for any $\gamma\in\Bbb R$.

(f) If $A\subseteq\Bbb R^X$ is stable and
$g\in\eusm L^0=\eusm L^0(\Sigma)$, then
$A+g=\{f+g:f\in A\}$ is stable.

(g) If $A\subseteq\eusm L^0$ is finite it is stable.

(h) If $A\subseteq\Bbb R^X$ is stable and $g\in\eusm L^0$, then
$A\times g=\{f\times g:f\in A\}$ is stable.

(i) If $\hat\mu$, $\tilde\mu$ are the completion and c.l.d.\ version of
$\mu$, then $A\subseteq\Bbb R^X$ is stable with respect to one of the
measures $\mu$, $\hat\mu$, $\tilde\mu$ iff it is stable with respect to
the others.

(j) Let $\nu$ be an indefinite-integral measure over
$\mu$\cmmnt{ (234J\formerly{2{}34B})}.
If $A\subseteq\Bbb R^X$ is stable with respect
to $\mu$, it is stable with respect to $\nu$ and with respect to
$\nu\restr\Sigma$.

(k) Let $h:\Bbb R\to\Bbb R$ be a continuous non-decreasing function.
If $A\subseteq\Bbb R^X$ is stable, so is $\{hf:f\in A\}$.

(l) If $A\subseteq\Bbb R^X$ is stable, so is
$\{f^+:f\in A\}\cup\{f^-:f\in A\}$.

(m) If $A\subseteq\BbbR^X$ is stable, and $Y\subseteq X$ is such that the
subspace measure $\mu_Y$ is semi-finite, then $A_Y=\{f\restr Y:f\in A\}$ is
stable in $\BbbR^Y$ with respect to the measure $\mu_Y$.

(n) A set $A\subseteq\BbbR^X$ is stable iff $A_E=\{f\restr E:f\in A\}$ is
stable in $\BbbR^E$ with respect to the subspace measure $\mu_E$ whenever
$E\in\Sigma$ has finite measure.

\proof{{\bf (a)} This is immediate from the definition in 465B, since
$D_k(B,E,\alpha,\beta)\subseteq D_k(A,E,\alpha,\beta)$ for all $k$, $E$,
$\alpha$, $\beta$ and $B\subseteq A$.

\medskip

{\bf (b)} Given $E$ such that $0<\mu E<\infty$, and $\alpha<\beta$, take
$\alpha'$, $\beta'$ such that $\alpha<\alpha'<\beta'<\beta$.
Then it is easy to see that $D_k(\overline{A},E,\alpha,\beta)\subseteq
D_k(A,E,\alpha',\beta')$, so

\Centerline{$(\mu^{2k})^*D_k(\overline{A},E,\alpha,\beta)
\le(\mu^{2k})^*D_k(A,E,\alpha',\beta')
<(\mu E)^{2k}$}

\noindent for some $k\ge 1$.   As $E$,
$\alpha$ and $\beta$ are arbitrary, $\overline{A}$ is stable.

\medskip

{\bf (c)} For any $m\ge 1$ and $l<n$, if we identify $X^{2(mn+l)}$ with
$(X^{2n})^m\times X^{2l}$, we see that $D_{mn+l}(A,E,\alpha,\beta)$
becomes identified
with a subset of $D_n(A,E,\alpha,\beta)^m\times E^{2l}$.   (If
$w\in D_{mn+l}(A,E,\alpha,\beta)$, there is an $f\in A$ such that
$f(w(2i))\le\alpha$, $f(w(2i+1))\ge\beta$ for $i<mn+l$.   Now
$(w(2rn),w(2rn+1),\ldots,w(2rn+2n-1))\in D_n(A,E,\alpha,\beta)$ for
$r<m$.)   So

$$\eqalignno{\Bover1{(\mu E)^{2(mn+l)}}
   (\mu^{2(mn+l)})^*&D_{mn+l}(A,E,\alpha,\beta)\cr
&\le\Bover1{(\mu E)^{2(mn+l)}}
   \bigl((\mu^{2n})^*D_n(A,E,\alpha,\beta)\bigr)^m(\mu E)^{2l}\cr
\displaycause{251Wm}
&=\bigl(\Bover1{(\mu E)^{2n}}(\mu^{2n})^*D_n(A,E,\alpha,\beta)\bigr)^m
\to 0\cr}$$

\noindent as $m\to\infty$.

\medskip

{\bf (d)} Note that $D_k(A\cup B,E,\alpha,\beta)
=D_k(A,E,\alpha,\beta)\cup D_k(B,E,\alpha,\beta)$
for all $E$, $k$, $\alpha$ and $\beta$.  Now, given that
$0<\mu E<\infty$ and $\alpha<\beta$, there are $m$, $n\ge 1$ such that
$(\mu^{2m})^*D_m(A,E,\alpha,\beta)<(\mu E)^{2m}$ and
$(\mu^{2n})^*D_n(A,E,\alpha,\beta)<(\mu E)^{2n}$.   So, by (c) above,

$$\eqalignno{\Bover1{(\mu E)^{2k}}(\mu^{2k})^*
  &D_k(A\cup B,E,\alpha,\beta)\cr
&\le\Bover1{(\mu E)^{2k}}\bigl((\mu^{2k})^*D_k(A,E,\alpha,\beta)
  +(\mu^{2k})^*D_k(B,E,\alpha,\beta)\bigr)
\to 0\cr}$$

\noindent as $k\to\infty$, and there is some $k$ such that
$(\mu^{2k})^*D_k(A\cup B,E,\alpha,\beta)<(\mu E)^{2k}$.   As $E$,
$\alpha$ and $\beta$ are arbitrary, $A\cup B$ is stable.

\medskip

{\bf (e)(i)} If $\gamma>0$,
$D_k(\gamma A,E,\alpha,\beta)=D_k(A,E,\alpha/\gamma,\beta/\gamma)$ for
all $k$, $E$, $\alpha$ and $\beta$, so the result is elementary.
Similarly, if $\gamma=0$, then $D_k(\gamma A,E,\alpha,\beta)=\emptyset$
whenever $k\ge 1$, $E\in\Sigma$ and $\alpha<\beta$, so again we see that
$\gamma A$ is stable.

\medskip

\quad{\bf (ii)} If
$\gamma=-1$ then, for any $k$, $E$, $\alpha$ and $\beta$,

$$\eqalign{D_k(-A,E,\alpha,\beta)
&=\bigcup_{f\in A}\{w:w\in E^{2k},\,f(w(2i))\ge-\alpha,\cr
&\mskip170mu f(w(2i+1))\le-\beta\Forall i<k\}\cr
&=\phi[D_k(A,E,-\beta,-\alpha)],\cr}$$

\noindent where $\phi:X^{2k}\to X^{2k}$ is the measure space
automorphism defined by setting

\Centerline{$\phi(w)=(w(1),w(0),w(3),w(2),\ldots,w(2k-1),w(2k-2))$}

\noindent for $w\in X^{2k}$.   So, given $E\in\Sigma$ and
$\alpha<\beta$, there is a $k\ge 1$ such that

\Centerline{$(\mu^{2k})^*D_k(-A,E,\alpha,\beta)
=(\mu^{2k})^*D_k(A,E,-\beta,-\alpha)
<(\mu E)^{2k}$.}

\noindent So $-A$ is stable.

\medskip

\quad{\bf (iii)} Together with (i) this shows that $\gamma A$ is stable
for every $\gamma\in\Bbb R$.

\medskip

{\bf (f)} Take $E$ such that $0<\mu E<\infty$, and $\alpha<\beta$.   Set
$\eta=\bover12(\beta-\alpha)>0$.   Then there is a $\gamma\in\Bbb R$
such that $F=\{x:x\in E,\,\gamma\le g(x)\le\gamma+\eta\}$ has
non-zero measure.   Set $\alpha'=\alpha-\gamma$,
$\beta'=\beta-\gamma-\eta$.   Then $D_k(A+g,F,\alpha,\beta)\subseteq
D_k(A,F,\alpha',\beta')$, while $\alpha'<\beta'$.   So if we take $k\ge
1$ such that
$(\mu^{2k})^*D_k(A,F,\alpha',\beta')<(\mu F)^{2k}$, then

\Centerline{$(\mu^{2k})^*D_k(A+g,E,\alpha,\beta)
\le(\mu^{2k})^*D_k(A+g,F,\alpha,\beta)+\mu^{2k}(E^{2k}\setminus F^{2k})
<(\mu E)^{2k}$.}

\medskip

{\bf (g)} If $A$ is empty, or contains only the constant function
$\tbf{0}$ with value $0$, this is trivial.   Now (f) tells us that
$\{g\}=\{\tbf{0}\}+g$ is stable for every $g\in\eusm L^0$, and from (d)
it follows that any finite subset of $\eusm L^0$ is stable.

\medskip

{\bf (h)} Let $E\in\Sigma$, $\alpha<\beta$ be such that
$0<\mu E<\infty$.   Set

\Centerline{$E_0=\{x:x\in E,\,g(x)=0\}$,
\quad$E_1=\{x:x\in E,\,g(x)>0\}$,}

\Centerline{$E_2=\{x:x\in E,\,g(x)<0\}$.}

\medskip

\quad{\bf (i)} Suppose that $\mu E_0>0$.   Then
$D_1(A\times g,E,\alpha,\beta)$ does not meet $E_0^2$, so
$\mu^*D_1(A\times g,E,\alpha,\beta)<\mu E$.

\medskip

\quad{\bf (ii)} Suppose that $\mu E_1>0$.   Let $\eta>0$ be such that

\Centerline{$\max(\alpha,\Bover{\alpha}{1+\eta})=\alpha'
<\beta'=\min(\beta,\Bover{\beta}{1+\eta})$.}

\noindent Let $\gamma>0$ be such that $\mu F>0$, where
$F=\{x:x\in E,\,\gamma\le g(x)\le\gamma(1+\eta)\}$.   If $x\in F$ then

\Centerline{$f(x)g(x)\le\alpha
\Longrightarrow f(x)\le\Bover{\alpha}{g(x)}\le\Bover{\alpha'}{\gamma}$,}

\Centerline{$f(x)g(x)\ge\beta
\Longrightarrow f(x)\ge\Bover{\beta}{g(x)}\ge\Bover{\beta'}{\gamma}$.}

\noindent So

\Centerline{$D_k(A\times g,F,\alpha,\beta)
\subseteq D_k(A,F,\Bover{\alpha'}{\gamma},\Bover{\beta'}{\gamma})$}

\noindent for every $k$.   Now, because $A$ is stable, there is some
$k\ge 1$ such that

\Centerline{$(\mu^{2k})^*D_k(A,F,\Bover{\alpha'}{\gamma},
\Bover{\beta'}{\gamma})<(\mu F)^{2k}$,}

\noindent and in this case
$(\mu^{2k})^*D_k(A\times g,E,\alpha,\beta)<(\mu E)^{2k}$, just as in the
argument for (f) above.

\medskip

\quad{\bf (iii)} If $\mu E_2>0$, then we know from (e) that $-A$ is
stable, so (ii) tells us that
there is a $k\ge 1$ such that
$(\mu^{2k})^*D_k((-A)\times(-g),E,\alpha,\beta)<(\mu E)^{2k}$.

Since one of these three cases must occur, and since $E$, $\alpha$ and
$\beta$ are arbitrary, $A\times g$ is stable.

\medskip

{\bf (i)} The product measures $\mu^{2k}$, $\hat\mu^{2k}$ and
$\tilde\mu^{2k}$ are all the same (251Wn), so this follows immediately
from the definition in 465B.

\medskip

{\bf (j)} Let $h$ be a Radon-Nikod\'ym derivative of $\nu$ with respect
to $\mu$ (234J).
Suppose that $0<\nu E<\infty$ and $\alpha<\beta$.
Then there is an $F\in\Sigma$ such that $F\subseteq E\cap\dom h$,
$h(x)>0$ for every $x\in F$, and $0<\mu F<\infty$.   There is a $k\ge 1$
such that $(\mu^{2k})^*D_k(A,F,\alpha,\beta)<(\mu F)^{2k}$, that is,
there is a $W\subseteq F^{2k}\setminus D_k(A,F,\alpha,\beta)$ such that
$\mu^{2k}W>0$.   In this case, $\nu^{2k}W>0$.   \Prf\ Set
$\tilde h(w)=\prod_{i=0}^{2k-1}h(w(i))$ for $w\in(\dom h)^{2k}$.   Then
$\nu^{2k}$ is
the indefinite integral of $\tilde h$ with respect to $\mu^{2k}$
(253I, extended by induction to the product of more than two factors),
and $\tilde h(w)>0$ for every $w\in F^{2k}$.\ \QeD\
Since $W\subseteq E^{2k}\setminus D_k(A,E,\alpha,\beta)$,
$(\nu^{2k})^*D_k(A,E,\alpha,\beta)<(\nu E)^{2k}$;  as $E$, $\alpha$ and
$\beta$ are arbitrary, $A$ is stable with respect to $\nu$.

Since $\nu$ is the completion of its restriction $\nu\restr\Sigma$
(234Lb\formerly{2{}34D}),
$A$ is also stable with respect to $\nu\restr\Sigma$, by (i).

\medskip

{\bf (k)} Write $B=\{hf:f\in A\}$.   Suppose that $0<\mu E<\infty$ and
$\alpha<\beta$.   If either $\alpha<h(\gamma)$ for every $\gamma\in\Bbb
R$ or $h(\gamma)<\beta$ for every $\gamma\in\Bbb R$,

\Centerline{$\mu^*D_1(B,E,\alpha,\beta)=\mu\emptyset=0<\mu E$.}

\noindent Otherwise, because $h$ is continuous, the Intermediate Value
Theorem tells us that there are $\alpha'<\beta'$ such that
$\alpha<h(\alpha')<h(\beta')<\beta$.   In this case
$D_k(B,E,\alpha,\beta)\subseteq D_k(A,E,\alpha',\beta')$ for every $k$.
Because $A$ is stable, there is some $k\ge 1$ such that
$(\mu^{2k})^*D_k(A,E,\alpha',\beta')<(\mu E)^{2k}$, so that
$(\mu^{2k})^*D_k(B,E,\alpha,\beta)<(\mu E)^{2k}$.  As $E$, $\alpha$ and
$\beta$ are arbitrary, $B$ is stable.

\medskip

{\bf (l)} From (k) we see that $\{f^+:f\in A\}$ is stable;   now from
(e) and (d) we see that $\{f^-:f\in A\}=\{f^+:f\in -A\}$ and
$\{f^+:f\in A\}\cup\{f^-:f\in A\}$ are stable.

\medskip

{\bf (m)} Writing $\Sigma_Y$ for the subspace $\sigma$-algebra, take
$F\in\Sigma_Y$ such that $\mu_YF=\mu^*F$ is finite and non-zero, and
$\alpha<\beta$ in $\Bbb R$.   Let $E\in\Sigma$ be a measurable envelope of
$F$.   Then there is a $k\ge 1$ such that
$(\mu^{2k})^*D_k(A,E,\alpha,\beta)<(\mu E)^{2k}$.   Consider
$D_k(A_Y,F,\alpha,\beta)=F^k\cap D_k(A,E,\alpha,\beta)$.   The identity map
from $F$ to $E$ is \imp\ for the subspace measures $\mu_F$ and $\mu_E$
(214Ce),
so the identity map from $F^{2k}$ to $E^{2k}$ is \imp\ for the product
measures $\mu_F^{2k}$ and $\mu_E^{2k}$ (apply 254H to
appropriate normalizations of
$\mu_E$, $\mu_F$).   Also  $\mu_E^{2k}$ is the subspace measure on $E^{2k}$
induced by $\mu^{2k}$ (251Wl), and similarly $\mu_F^{2k}$ is the subspace
measure on $F^{2k}$ induced by $\mu_Y^{2k}$, so

$$\eqalignno{(\mu_Y^{2k})^*D_k(A_Y,F,\alpha,\beta)
&=(\mu_F^{2k})^*D_k(A_Y,F,\alpha,\beta)
\le(\mu_E^{2k})^*D_k(A,E,\alpha,\beta)\cr
\displaycause{413Eh}
&=(\mu^{2k})^*D_k(A,E,\alpha,\beta)
<(\mu E)^{2k}
=(\mu_YF)^{2k}.\cr}$$

\noindent As $F$, $\alpha$ and $\beta$ are arbitrary, $A_Y$ is stable, as
claimed.

\medskip

{\bf (n)} If $A$ is stable, then (m) tells us that $A_E$ will be stable for
every $E\in\Sigma$.   Conversely, if $A_E$ is stable for every $E$ of
finite measure, take $E\in\Sigma$ such that $0<\mu E<\infty$ and
$\alpha<\beta$ in $\Bbb R$.   Then there is a $k\ge 1$ such that

\Centerline{$(\mu E)^{2k}
=(\mu_EE)^{2k}
>(\mu_E^{2k})^*D_k(A_E,E,\alpha,\beta)
=(\mu^{2k})^*D_k(A,E,\alpha,\beta)$.}

\noindent As $E$, $\alpha$ and $\beta$ are arbitrary, $A$ is stable.

}%end of proof of 465C

\leader{465D}{}\cmmnt{ Now for the first result connecting the notion
of `stable' set with the concerns of this chapter.

\medskip

\noindent}{\bf Proposition} Let $(X,\Sigma,\mu)$ be a complete locally
determined measure space, and $A\subseteq\Bbb R^X$ a stable set.

(a) $A\subseteq\eusm L^0\cmmnt{\mskip5mu =\eusm L^0(\Sigma)}$\cmmnt{
(that is, every member of $A$ is $\Sigma$-measurable)}.

(b) If $\{f(x):f\in A\}$ is bounded for each $x\in X$, then $A$ is
relatively compact in $\eusm L^0$ for
the topology of pointwise convergence.

\proof{{\bf (a)} \Quer\ Suppose, if possible, that there is a
non-measurable
$f\in A$.   Then there is an $\alpha\in\Bbb R$ such that
$D_0=\{x:f(x)>\alpha\}\notin\Sigma$.   Because $\mu$ is locally
determined, there is an $F_0\in\Sigma$ such that $\mu F_0<\infty$ and
$D_0\cap F_0\notin\Sigma$.  Let $F_1\subseteq F_0$ be a measurable
envelope of $D_0\cap F_0$ (132Ee).   Then $D_0\cap F_1=D_0\cap F_0$ is
not measurable;  because $\mu$ is complete, $F_1\setminus D_0$ cannot be
negligible.   Now $D_0=\bigcup_{n\in\Bbb N}\{x:f(x)\ge\alpha+2^{-n}\}$,
so there is some $\beta>\alpha$ such that
$D_1=F_1\cap\{x:f(x)\ge\beta\}$ is not negligible.   Let $E$ be a
measurable envelope of $D_1$.   Then, setting
$P=\{x:x\in E,\,f(x)\le\alpha\}$, $Q=\{x:x\in E,\,f(x)\ge\beta\}$ we
have $\mu^*P=\mu^*Q=\mu E>0$.

Now suppose that $k\ge 1$.   Then
$D_k(\{f\},E,\alpha,\beta)\supseteq(P\times Q)^k$, so

\Centerline{$(\mu^{2k})^*D_k(\{f\},E,\alpha,\beta)=(\mu^*P\cdot\mu^*Q)^k
=(\mu E)^{2k}$}

\noindent (251Wm again).   Since this is true for every $k$,
$\{f\}$ is not stable, and
(by 465Ca) $A$ cannot be stable;  which contradicts our hypothesis.\
\Bang

\medskip

{\bf (b)} Because $\{f(x):x\in A\}$ is bounded for each $x$,
$\overline{A}$, the closure of $A$ in $\Bbb R^X$, is compact for the
topology of pointwise convergence.   But $\overline{A}$ is stable, by
465Cb, so is included in $\eusm L^0$, by (a).
}%end of proof of 465D

\leader{465E}{The topology
$\frak T_s(\eusm L^2,\eusm L^2)$}\cmmnt{ Some of the arguments below
will rely on ideas of compactness in function spaces.   There are of
course many ways of expressing the method, but a reasonably accessible
one uses the Hilbert space $L^2$, as follows.}   Let $(X,\Sigma,\mu)$ be
any measure space.   Then $L^2=L^2(\mu)$ is a Hilbert space with a
corresponding weak topology
$\frak T_s(L^2,L^2)$\cmmnt{ defined by the functionals
$u\mapsto\innerprod{u}{v}$ for $v\in L^2$}.   In the present section it
will be more convenient to regard this as a topology
$\frak T_s(\eusm L^2,\eusm L^2)$ on the space $\eusm L^2=\eusm L^2(\mu)$ of square-integrable real-valued functions\cmmnt{, defined
by the functionals $f\mapsto\int f\times g$ for $g\in\eusm L^2$}.   The
essential fact we need is that norm-bounded sets are relatively weakly
compact.   \prooflet{In $L^2$, this is because Hilbert spaces are
reflexive (4A4Ka).   In $\eusm L^2$, given an ultrafilter $\Cal F$
containing a $\|\,\|_2$-bounded set $B\subseteq\eusm L^2$,
$v=\lim_{f\to\Cal F}f^{\ssbullet}$ must be defined in $L^2$ for
$\frak T_s(L^2,L^2)$, and now there is a $g\in\eusm L^2$ such that
$g^{\ssbullet}=v$;  in which case

\Centerline{$\lim_{f\to\Cal F}\int f\times h
=\lim_{f\to\Cal F}\innerprod{f^{\ssbullet}}{h^{\ssbullet}}
=\innerprod{g^{\ssbullet}}{h^{\ssbullet}}
=\int g\times h$}

\noindent for every $h\in\eusm L^2$.   Note that we are free to take $g$
to be a $\Sigma$-measurable function with domain $X$ (241Bk).}

\leader{465F}{Lemma} Let $(X,\Sigma,\mu)$ be a measure space, and
$B\subseteq\eusm L^2=\eusm L^2(\mu)$ a $\|\,\|_2$-bounded set.
Suppose that $h\in\eusm L^2$ belongs to the closure of $B$ for
$\frak T_s(\eusm L^2,\eusm L^2)$.   Then for any $\delta>0$ and $k\ge 1$
the set

$$\eqalign{W=\bigcup_{f\in B}\{w:w\in X^k,\,
&w(i)\in\dom f\cap\dom h\cr
&\text{and }f(w(i))\ge h(w(i))-\delta\text{ for every }i<k\}\cr}$$

\noindent is $\mu^k$-conegligible in $X^k$.

\proof{{\bf (a)} Since completing the measure $\mu$ does not change the
space $\eusm L^2$ (244Xa) nor the product measure $\mu^k$ (251Wn), we
may suppose that $\mu$ is complete.

\medskip

{\bf (b)} The first substantive fact to note is that there is a sequence
$\sequencen{f_n}$ in $B$ converging to $h$ for
$\frak T_s=\frak T_s(\eusm L^2,\eusm L^2)$.   \Prf\ Setting
$C=\{f^{\ssbullet}:f\in B\}$,
$C$ is a bounded set in $L^2=L^2(\mu)$ and $h^{\ssbullet}$ belongs to
the $\frak T_s(L^2,L^2)$-closure of $C$.   But $L^2$, being a normed
space, is angelic in its weak topology (462D), and $C$ is relatively
compact in $L^2$, so there is a sequence in $C$ converging to
$h^{\ssbullet}$.   We can represent this sequence as
$\sequencen{f_n^{\ssbullet}}$ where $f_n\in B$ for every $n$, and now
$\sequencen{f_n}\to h$ for $\frak T_s$.\ \Qed

\medskip

{\bf (c)} The second component of the proof is the following simple
idea.   Suppose that $\sequencen{E_n}$ is a sequence in $\Sigma$ such
that $\bigcap_{n\in I}E_n$ is negligible for every infinite set
$I\subseteq\Bbb N$.   For $m\ge 1$ and $I\subseteq\Bbb N$ set

\Centerline{$V_m(I)
=\bigcap_{n\in I}\{w:w\in X^m,\,\exists\,i<m,\,w(i)\in E_n\}$.}

\noindent Then $\mu^mV_m(I)=0$ for every infinite $I\subseteq\Bbb N$.
\Prf\ Induce on $m$.   For $m=1$ this is just the original hypothesis on
$\sequencen{E_n}$.   For the inductive step to $m+1$, identify
$\mu^{m+1}$ with the product of $\mu^m$ and $\mu$, and observe that

\Centerline{$V_{m+1}(I)^{-1}[\{x\}]
=\{w:(w,x)\in V_{m+1}(I)\}
=V_m(\{n:n\in I,\,x\notin E_n\})$}

\noindent for every $x\in X$.   Now, setting $I_x=\{n:n\in I,\,x\notin
E_n\}$, $F=\{x:I_x$ is finite$\}$, we have

\Centerline{$F=\bigcup_{r\in\Bbb N}\bigcap_{n\in I\setminus r}E_n$,}

\noindent so $F$ is negligible, while if $x\notin F$ then $V_m(I_x)$ is
negligible, by the inductive hypothesis.   But this means that almost
every horizontal section of $V_{m+1}(I)$ is negligible, and
$V_{m+1}(I)$, being measurable, must be negligible, by Fubini's theorem
(252F).   Thus the induction proceeds.\ \Qed

\medskip

{\bf (d)} Now let us return to the main line of the argument from (b).
For each $n\in\Bbb N$, set $E_n=\{x:x\in\dom f_n\cap\dom
h,\,f_n(x)<h(x)-\delta\}$.   If $I\subseteq\Bbb N$ is infinite, then
$\bigcap_{n\in I}E_n$ is negligible.   \Prf\ Setting
$G=\bigcap_{n\in I}E_n$, $\mu G$ is finite (because
$\delta\chi E_n\leae|h-f_n|$, so
$\mu E_n<\infty$ for every $n$) and $\int_Gf_n\le\int_Gh-\delta\mu G$
for every $n\in I$.   But $\lim_{n\to\infty}\int_Gf_n=\int_Gh$
and $I$ is infinite, so

\Centerline{$\delta\mu G\le\inf_{n\in I}|\int_Gh-\int_Gf_n|=0$.
\Qed}

By (c), it follows that

\Centerline{$V
=\bigcap_{n\in\Bbb N}\{w:w\in X^k,\,\exists\,i<k,\,w(i)\in E_n\}$}

\noindent is negligible.   But if we set
$Y=\bigcap_{n\in\Bbb N}\dom f_n\cap\dom h$, $Y$ is a conegligible subset
of $X$, $Y^k$ is a conegligible subset of $X^k$, and
$Y^k\setminus V\subseteq W$, so $W$ is conegligible, as required.
}%end of proof of 465F

\leader{465G}{Theorem} Let $(X,\Sigma,\mu)$ be a semi-finite measure
space, and $A\subseteq\eusm L^0\cmmnt{\mskip5mu =\eusm L^0(\Sigma)}$ a
stable
set of measurable functions.   Let $\frak T_p$ and $\frak T_m$ be the
topologies of pointwise convergence and convergence in
measure\cmmnt{, as in \S463}.   Then the identity map from $A$ to
itself is $(\frak T_p,\frak T_m)$-continuous.

\proof{\Quer\ Suppose, if possible, otherwise.

\medskip

{\bf (a)} We must have an $f_0\in A$, a set $F\in\Sigma$ of finite
measure, and an $\epsilon>0$ such that for every
$\frak T_p$-neighbourhood $U$ of $f_0$ there is an $f\in A\cap U$ such
that $\int\chi F\wedge|f-f_0|d\mu\ge 2\epsilon$.   Set

\Centerline{$B=\{\chi F\wedge(f-f_0)^+:f\in A\}
  \cup\{\chi F\wedge(f-f_0)^-:f\in A\}$.}

\noindent Then

\Centerline{$B=\{\chi F-(\chi F-(f-f_0)^+)^+:f\in A\}
  \cup\{\chi F-(\chi F-(f-f_0)^-)^+:f\in A\}$}

\noindent is stable, by 465Cf, 465Cl and 465Ce, used repeatedly.
Setting $B'=\{f:f\in B,\,\int f\ge\epsilon\}$, $B'$ is again stable
(465Ca).
Our hypothesis is that $f_0$ is in the $\frak T_p$-closure of

$$\eqalign{\{f:f\in A,\,\int\chi F\wedge|f-f_0|\ge 2\epsilon\}
&\subseteq\{f:f\in A,\,\int\chi F\wedge(f-f_0)^+\ge\epsilon\}\cr
&\mskip50mu\cup\{f:f\in A,\,\int\chi F\wedge(f-f_0)^-\ge\epsilon\};\cr}$$

\noindent since
$f\mapsto\chi F\wedge(f-f_0)^+$ and
$f\mapsto\chi F\wedge(f-f_0)^-$ are $\frak T_p$-continuous,
$0$ belongs to the $\frak T_p$-closure of $B'$.

\medskip

{\bf (b)} Let $\Cal F$ be an ultrafilter on $B'$ which
$\frak T_p$-converges to $0$.   Because $B'$ is $\|\,\|_2$-bounded
(since $0\le f\le\chi F$ for every $f\in B'$), $\Cal F$ also has a
$\frak T_s(\eusm L^2,\eusm L^2)$-limit $h$ say, as noted in 465E;  and
we can suppose that $h$ is measurable and defined everywhere.   We must
have

\Centerline{$\int_Fh=\lim_{f\to\Cal F}\int_Ff\ge\epsilon>0$.}

\noindent So there is a $\delta>0$ such that
$E=\{x:x\in F,\,h(x)\ge 3\delta\}$ has measure greater than $0$.

\medskip

{\bf (c)} Because $B'$ is stable, there must be some $k\ge 1$ such that
$(\mu^{2k})^*D_k(B',E,\delta,2\delta)<(\mu E)^{2k}$.   Let
$W\subseteq E^{2k}\setminus D_k(B',E,\delta,2\delta)$
be a measurable set of
positive measure.   By Fubini's theorem, there must be a $u\in X^k$
such that $\mu^kV$ is defined and greater than $0$, where

\Centerline{$V=\{v:v\in X^k,\,u\#v\in W\}$.}

\noindent Set $C=\{f:f\in B',\,f(u(i))\le\delta$ for every $i<k\}$;  then
$C\in\Cal F$, because $\Cal F\to 0$ for $\frak T_p$.   Accordingly $h$
belongs to the $\frak T_s(\eusm L^2,\eusm L^2)$-closure of $C$.   But
now 465F tells us that there must be a $v\in V$ and an $f\in C$ such
that $f(v(i))\ge h(v(i))-\delta$ for every $i<k$.

Consider $w=u\#v$.   We know that $w\in W$
(because $v\in V$), so, in particular, $w\in E^{2k}$ and
$h(w(i))\ge 3\delta$ for every $i<2k$;  accordingly

\Centerline{$f(w(2i+1))=f(v(i))\ge h(v(i))-\delta\ge2\delta$}

\noindent for every $i<k$.   On the other hand,
$f(w(2i))=f(u(i))\le\delta$ for every $i<k$, because $f\in C$.   But this
means that $f$ witnesses that $w\in D_k(B',E,\delta,2\delta)$, which is
supposed to be disjoint from $W$.\ \Bang

This contradiction shows that the theorem is true.
}%end of proof of 465G

\leader{465H}{}\cmmnt{ We shall need some interesting and important
general facts concerning powers of measures.   I start with an important
elaboration of the strong law of large numbers.

\medskip

\noindent}{\bf Theorem} Let $(X,\Sigma,\mu)$ be any probability space.
For $n\in\Bbb N$, write $\Lambda_n$ for the domain of the product
measure $\mu^n$.    For $w\in X^{\Bbb N}$, $k\ge 1$, $n\ge 1$ write
$\nu_{wk}$ for the probability measure with domain $\Cal PX$ defined by
writing

\Centerline{$\nu_{wk}(E)=\Bover1k\#(\{i:i<k,\,w(i)\in E\})$}

\noindent for $E\subseteq X$, and $\nu_{wk}^n$ for the corresponding
product measure on $X^n$.

Then whenever $n\ge 1$ and $f:X^n\to\Bbb R$ is bounded and
$\Lambda_n$-measurable, $\lim_{k\to\infty}\int fd\nu_{wk}^n$ exists, and
is equal to $\int fd\mu^n$, for $\mu^{\Bbb N}$-almost every
$w\in X^{\Bbb N}$.

\proof{{\bf (a)} For $k\ge n$ and $w\in X^{\Bbb N}$,  set

\Centerline{$h_k(w)=\Bover{(k-n)!}{k!}\sum_{\pi:n\to k\text{ is
injective}}f(w\pi)$.}

\noindent Setting $M=\sup_{v\in X^n}|f(v)|$, we have

$$\eqalign{\bigl|h_k(w)-\int fd\nu_{wk}^n\bigr|
&=\bigl|\Bover{(k-n)!}{k!}\sum_{\pi:n\to k\text{ is injective}}f(w\pi)
  -\Bover1{k^n}\sum_{\pi:n\to k}f(w\pi)\bigr|\cr
&\le M\bigl(\Bover{(k-n)!}{k!}-\Bover1{k^n}\bigr)
  +\Bover{M}{k^n}\bigl(k^n-\Bover{k!}{(k-n)!}\bigr)
\to 0\cr}$$

\noindent as $k\to\infty$, for every $w\in X^{\Bbb N}$.

\medskip

{\bf (b)} Write $\Lambda$ for the domain of $\mu^{\Bbb N}$, and for
$k\ge n$ set

\Centerline{$\Tau_k=\{W:W\in\Lambda,\,w\pi\in W$ whenever $w\in W$ and
$\pi\in S_k\}$,}

\noindent where $S_k$ is the set of permutations $\psi:\Bbb N\to\Bbb N$
such that $\psi(i)=i$ for $i\ge k$.   Then $\langle\Tau_k\rangle_{k\ge
n}$ is a non-increasing sequence of $\sigma$-subalgebras of $\Lambda$.
For any injective function $\pi:n\to k$ there are just $(k-n)!$
extensions of $\pi$ to a member of $S_k$.   So

\Centerline{$h_k(w)=\Bover1{k!}\sum_{\pi\in S_k}f(w\pi\restr n)$}

\noindent for every $w$.   Observe that $h_k(w\psi)=h_k(w)$ for every
$\psi\in S_k$, $w\in X^{\Bbb N}$, so $h_k$ is $\Tau_k$-measurable.   (Of
course we need to look back at the definition of $h_k$ to confirm that
it is $\Lambda$-measurable.)

\medskip

{\bf (c)} For any $k\ge n$, $h_k$ is a conditional expectation of $h_n$
on $\Tau_k$.   \Prf\ If $W\in\Tau_k$, then

\Centerline{$\int_Wh_k(w)dw
=\Bover1{k!}\sum_{\pi\in S_k}\int_Wf(w\pi\restr n)dw
=\Bover1{k!}\sum_{\pi\in S_k}\int_Wg(w\pi)dw$}

\noindent where $g(w)=f(w\restr n)$ for $w\in X^{\Bbb N}$.   Now
observe that for every $\pi\in S_k$ the map $w\mapsto w\pi$ is a measure
space automorphism of $X^{\Bbb N}$ which leaves $W$ unchanged, because
$W\in\Tau_k$;  so that $\int_Wg(w\pi)dw=\int_Wg(w)dw$, by 235Gc.   So
$\int_Wh_k=\int_Wg$.   But (since $W\in\Tau_k\subseteq\Tau_n$)
$\int_Wh_n$ is also equal to $\int_Wg$, and $\int_Wh_k=\int_Wh_n$.   As
$W$ is arbitrary, $h_k$ is a conditional expectation of $h_n$ on
$\Tau_k$.\ \Qed

\medskip

{\bf (d)} By the reverse martingale theorem (275K),
$h_{\infty}(w)=\lim_{k\to\infty}h_k(w)$ is defined for almost every
$w\in X^{\Bbb N}$.   Accordingly $\lim_{k\to\infty}\int fd\nu_{wk}^n$ is
defined for almost every $w$.

\medskip

{\bf (e)} To see that the limit is $\int fd\mu^n$, observe that if
$W\in\Tau_{\infty}=\bigcap_{k\in\Bbb N}\Tau_k$ then $\mu^{\Bbb N}W$ must
be either $0$ or $1$.   \Prf\ Set $\gamma=\mu^{\Bbb N}W$.   Let
$\epsilon>0$.   Then there is a $V\in\bigotimes_{\Bbb N}\Sigma$
(notation:  465Ad) such that
$\mu^{\Bbb N}(W\symmdiff V)\le\epsilon$ (254Fe).   There is some $k$
such that $V$
is determined by coordinates in $k$.   If we set $\pi(i)=2k-i$ for
$i<2k$, $i$ for $i\ge 2k$, then $V'=\{w\pi:w\in V\}$ is determined by
coordinates in $2k\setminus k$, so
$\mu^{\Bbb N}(V\cap V')=(\mu^{\Bbb N}V)^2$.    On the other hand,
because $W\in\Tau_{2k}$, the measure
space automorphism $w\mapsto w\pi$ does not move $W$, and
$\mu^{\Bbb N}(W\setminus V')=\mu^{\Bbb N}(W\setminus V)$.   Accordingly

$$\eqalign{\gamma
&=\mu^{\Bbb N}(W\cap W)
\le\mu^{\Bbb N}(V\cap V')+2\mu^{\Bbb N}(W\setminus V)
\le(\mu^{\Bbb N}V)^2+2\epsilon
\le(\gamma+\epsilon)^2+2\epsilon.\cr}$$

\noindent As $\epsilon$ is arbitrary, $\gamma\le\gamma^2$ and
$\gamma\in\{0,1\}$.\ \Qed

\medskip

{\bf (e)} Now 275K tells us that $h_{\infty}$ is
$\Tau_{\infty}$-measurable, therefore essentially constant, and must be
equal to its expectation almost everywhere.   But, setting
$W=X^{\Bbb N}$ in the proof of (c), we see that
$\int h_k=\int f(w\restr n)dw=\int fd\mu^n$ for every $k$, so

\Centerline{$\lim_{k\to\infty}\int fd\nu_{wk}^n
=h_{\infty}(w)
=\int fd\mu^n$}

\noindent for almost every $w$, as claimed.
}%end of proof of 465H

\leader{465I}{}\cmmnt{ Now for a string of lemmas, working towards the
portmanteau Theorem 465M.   The first is elementary.

\medskip

\noindent}{\bf Lemma} Let $X$ be a set, and $\Sigma$ a $\sigma$-algebra
of subsets of $X$.    For $w\in X^{\Bbb N}$ and $k\ge 1$, write
$\nu_{wk}$ for the probability measure
with domain $\Cal PX$ defined by writing

\Centerline{$\nu_{wk}(E)=\Bover1k\#(\{i:i<k,\,w(i)\in E\})$}

\noindent for $E\subseteq X$.   Then for any $k\in\Bbb N$ and any set
$I$, $w\mapsto\nu_{wk}^I(W)$ is
$\Tensorhat_{\Bbb N}\Sigma$-measurable\cmmnt{ (notation:  465Ad)} for
every $W\in\Tensorhat_I\Sigma$.

\proof{ Write $\Cal W$ for the set of subsets $W$ of $X^I$ such that
$w\mapsto\nu_{wk}^I(W)$ is $\Tensorhat_{\Bbb N}\Sigma$-measurable.
Then $X^I\in\Cal W$, $W'\setminus W\in\Cal W$ whenever $W$, $W'\in\Cal
W$ and $W\subseteq W'$, and $\bigcup_{n\in\Bbb N}W_n\in\Cal W$ whenever
$\sequencen{W_n}$ is a non-decreasing sequence in $\Cal W$.   Write
$\Cal V$ for the set of $\Sigma$-cylinders in $X^I$, that is, sets
expressible in the form $\{v:v(i)\in E_i$ for every $i\in J\}$, where
$J\subseteq I$ is finite and $E_i\in\Sigma$ for $i\in J$.   Then $\Cal
V\subseteq\Cal W$.   \Prf\ If $J\subseteq I$ is finite and
$E_i\in\Sigma$ for $i\in J$, then

\Centerline{$w\mapsto\nu_{wk}E_i=\Bover1k\sum_{j=0}^{k-1}\chi
E_i(w(j))$}

\noindent is $\Tensorhat_{\Bbb N}$-measurable for every $i\in J$.  So

\Centerline{$w\mapsto\nu_{wk}^I\{v:v(i)\in E_i\Forall i\in J\}
=\prod_{i\in J}\nu_{wk}E_i$}

\noindent is also measurable.\ \Qed

Because $V\cap V'\in\Cal V$ for all $V$, $V'\in\Cal V$, the Monotone
Class Theorem (136B) tells us that $\Cal W$ must include the
$\sigma$-algebra generated by $\Cal V$, which is $\Tensorhat_I\Sigma$.
}%end of proof of 465I

\leader{465J}{}\cmmnt{ The next three lemmas are specifically adapted
to the study of stable sets of functions.

\medskip

\noindent}{\bf Lemma} Let $(X,\Sigma,\mu)$ be a probability space.   For
any $n\in\Bbb N$ and $W\subseteq X^n$ I say that $W$ is {\bf symmetric} if
$w\pi\in W$ whenever $w\in W$ and $\pi:n\to n$ is a permutation.
Give each power $X^n$ its product measure $\mu^n$.

(a) Suppose that for each $n\ge 1$ we have a measurable set
$W_n\subseteq X^n$, and that
$W_{m+n}\subseteq W_m\times W_n$ for all $m$, $n\ge 1$, identifying
$X^{m+n}$ with $X^m\times X^n$.
Then $\lim_{n\to\infty}(\mu^nW_n)^{1/n}$ is defined and equal to
$\delta=\inf_{n\ge 1}(\mu^nW_n)^{1/n}$.

(b) Now suppose that
each $W_n$ is symmetric.   Then
there is an $E\in\Sigma$ such that $\mu E=\delta$ and $E^n\setminus W_n$
is negligible for every $n\in\Bbb N$.

(c) Next, let $\langle D_n\rangle_{n\ge 1}$ be a sequence of sets such
that

\inset{$D_n\subseteq X^n$ is symmetric for every $n\ge 1$,}

\inset{whenever $1\le m\le n$, $v\in D_n$ then $v\restr m\in D_m$.}

\noindent Then $\delta=\lim_{n\to\infty}((\mu^n)^*D_n)^{1/n}$ is defined
and there is an $E\in\Sigma$ such that $\mu E=\delta$ and
$(\mu^n)^*(D_n\cap E^n)=(\mu E)^n$ for every $n\in\Bbb N$.

\proof{{\bf (a)} For any $\eta>0$, there is an $m\ge 1$ such that
$\mu^mW_m\le(\delta+\eta)^m$.   If $n=mk+i$, where $k\ge 1$ and $i<m$,
then (identifying $X^n$ with $(X^m)^k\times X^i$)
$W_n\subseteq(W_m)^k\times X^i$, so

\Centerline{$\mu^nW_n\le(\delta+\eta)^{mk}
\le\gamma(\delta+\eta)^{mk+i}=\gamma(\delta+\eta)^n$,}

\noindent where

\Centerline{$\gamma=\max_{i<m}(\Bover1{\delta+\eta})^i$.}

\noindent So

\Centerline{$\limsup_{n\to\infty}(\mu^nW_n)^{1/n}
\le(\delta+\eta)\limsup_{n\to\infty}\gamma^{1/n}=\delta+\eta$.}

\noindent As $\eta$ is arbitrary,

\Centerline{$\delta\le\liminf_{n\to\infty}(\mu^nW_n)^{1/n}\le
\limsup_{n\to\infty}(\mu^nW_n)^{1/n}\le\delta$,}

\noindent  and $\lim_{n\to\infty}(\mu^nW_n)^{1/n}=\delta$.

\medskip

{\bf (b)}  It is enough to consider the case $\delta>0$.

\medskip

\quad{\bf (i)} Consider the family $\Bbb V$ of sequences
$\langle V_n\rangle_{n\ge 1}$ such that

\inset{for each $n\ge 1$, $V_n$ is a symmetric measurable subset of
$X^n$ and $\mu^nV_n\ge\delta^n$,}

\inset{if $1\le m\le n$ then $v\restr m\in V_m$ for every $v\in V_n$.}

\noindent Observe that $\pmb{W}=\langle W_n\rangle_{n\ge 1}\in\Bbb V$.
Order $\Bbb V$ by saying that
$\langle V_n\rangle_{n\ge 1}\le\langle V'_n\rangle_{n\ge 1}$ if $V_n\subseteq V'_n$ for every $n$.   For
$\pmb{V}=\langle V_n\rangle_{n\ge 1}$ in $\Bbb V$, set
$\theta(\pmb{V})=\sum_{n=1}^{\infty}2^{-n}\mu^nV_n$.
Any non-increasing sequence
$\sequence{k}{\langle V_{kn}\rangle_{n\ge 1}}$ in $\Bbb V$ has a lower bound $\langle\bigcap_{k\in\Bbb N}V_{kn}\rangle_{n\ge 1}$
in $\Bbb V$, so there must be a
$\pmb{W}'=\langle W'_n\rangle_{n\ge 1}\in\Bbb V$ such that $\pmb{W}'\le\pmb{W}$ and
$\theta(\pmb{V})=\theta(\pmb{W}')$ whenever
$\pmb{V}\in\Bbb V$ and $\pmb{V}\le\pmb{W}'$;  that is, whenever
$\langle V_n\rangle_{n\ge 1}\in\Bbb V$ and $V_n\subseteq W'_n$ for every
$n\ge 1$, then $\mu^nV_n=\mu^nW'_n$ for every $n$.

\medskip

\quad{\bf (ii)} For $x\in X$, $n\ge 1$ set
$V^{(x)}_{n}=\{w:(x,w)\in W'_{n+1}\}$.   Then $V^{(x)}_{n}$ is measurable for almost every $x$;
let $X_1\subseteq X$ be a conegligible set such that $V^{(x)}_n$ is
measurable for every $x\in X_1$ and every $n\ge 1$.   Every
$V^{(x)}_n$ is symmetric, and if $1\le m\le n$ and $v\in V^{(x)}_n$ then
$v\restr m\in V^{(x)}_m$.   It follows that if $m$, $n\ge 1$ then
$V^{(x)}_{m+n}$ becomes identified with a subset of $V^{(x)}_m\times
V^{(x)}_n$.

From (a) we see that $\delta_x=\lim_{n\to\infty}(\mu^nV_n^{(x)})^{1/n}$
is defined for every $x\in X_1$.   The map
$x\mapsto\mu^nV_n^{(x)}:X_1\to[0,1]$ is measurable for each
$n$, by Fubini's theorem (252D),
%do we need completeness?  \query
so $x\mapsto\delta_x$ is also measurable.   Since
$V_n^{(x)}\subseteq W'_n\subseteq W_n$ for every $x$ and $n$,
$\delta_x\le\delta$ for every $x\in X_1$.

\medskip

\quad{\bf (iii)} Set $E=\{x:x\in X_1,\,\delta_x=\delta\}$.   Then
$\mu E\ge\delta$.   \Prf\Quer\ Otherwise, there is some $\beta<\delta$
such that $\mu F<\beta$, where $F=\{x:x\in X_1,\,\delta_x\ge\beta\}$.
Now

$$\eqalign{X_1\setminus F
&=\{x:x\in X_1,\,\lim_{n\to\infty}(\mu^nV^{(x)}_n)^{1/n}<\beta\}\cr
&\subseteq\bigcup_{m\in\Bbb N}
  \bigcap_{n\ge m}\{x:x\in X_1,\,\mu^nV^{(x)}_n\le\beta^n\},\cr}$$

\noindent so there is some $m\in\Bbb N$ such that $\mu H\ge 1-\beta$,
where

\Centerline{$H
=\bigcap_{n\ge m}\{x:x\in X_1,\,\mu^nV^{(x)}_n\le\beta^n\}$.}

Set $\gamma_n=\mu^nW'_n\ge\delta^n$ for each $n$.   Then, for any
$n\ge m$,

$$\eqalign{\gamma_{n+1}
&=\mu^{n+1}W'_{n+1}
=\int_{X_1}\mu^nV^{(x)}_n\mu(dx)\cr
&=\int_H\mu^nV^{(x)}_n\mu(dx)
   +\int_{X_1\setminus H}\mu^nV^{(x)}_n\mu(dx)
\le\beta^n+\beta\gamma_n\cr}$$

\noindent because $V^{(x)}_n\subseteq W'_n$ for every $x$, and
$\mu(X_1\setminus H)\le\beta$.   An easy induction shows that
$\gamma_{m+k}\le k\beta^{m+k-1}+\beta^k\gamma_m$ for every $k\in\Bbb N$.
But this means that

\Centerline{$\delta^k=\delta^{-m}\delta^{m+k}\le\delta^{-m}\gamma_{m+k}
\le\beta^k\delta^{-m}(k\beta^{m-1}+\gamma_m)$}

\noindent for every $k$;  setting $\eta=(\delta-\beta)/\beta>0$,

\Centerline{$\Bover{k(k-1)}2\eta
\le(1+\eta)^k=\bigl(\Bover{\delta}{\beta}\bigr)^k
\le\delta^{-m}(k\beta^{m-1}+\gamma_m)$}

\noindent for every $k$, which is impossible.\ \Bang\Qed

\medskip

\quad{\bf (iv)} Next, for any $x\in E$,
$\pmb{V}^{(x)}=\langle V^{(x)}_n\rangle_{n\ge 1}\in\Bbb V$ and
$\pmb{V}^{(x)}\le\pmb{W}'$, so
$\mu^n(W'_n\setminus V^{(x)}_n)=0$ for every $n\ge 1$.   This means
that, given $n\ge 1$, every vertical section of
$(E\times W'_n)\setminus W'_{n+1}$ (regarded as a subset of
$X\times X^n$) is negligible;  so $(E\times W'_n)\setminus W'_{n+1}$ is
negligible.   We are assuming that $\delta>0$, so

\Centerline{$E\subseteq\{x:x\in X_1,\,V^{(x)}_1\ne\emptyset\}
\subseteq\{x:\{w:(x,w)\in W'_2\}\ne\emptyset\}
\subseteq W'_1$.}

\noindent Now a simple induction shows that $E^n\setminus W'_n$ is
negligible for every $n\ge 1$, so that $E^n\setminus W_n$ is negligible
for every $n$, and we have an appropriate $E$.   (Of
course $\mu E=\delta$ exactly, because $(\mu E)^n\le\mu W_n$ for every
$n$.)

\medskip

{\bf (c)} For each $n\in\Bbb N$ let $V_n$ be a measurable envelope of
$D_n$ in $X^n$.   Define $\langle W_n\rangle_{n\ge 1}$ inductively by
saying

\Centerline{$W_1=V_1$,}

$$\eqalign{W_{n+1}=\{w:w\in X^{n+1},\,&w\pi\restr n\in W_n,\cr
&w\pi\in V_{n+1}\text{ for every permutation }\pi:n+1\to n+1\}}$$

\noindent for each $n\ge 1$.   Then an easy induction on $n$ shows
that $W_n$ is measurable and symmetric and that $D_n\subseteq
W_n\subseteq V_n$, so that $W_n$ is a measurable envelope of $D_n$ and
$\mu^nW_n=(\mu^n)^*D_n$.

Now $\langle W_n\rangle_{n\ge 1}$ satisfies the hypotheses of (b), so

\Centerline{$\delta=\lim_{n\to\infty}(\mu^nW_n)^{1/n}
=\lim_{n\to\infty}((\mu^n)^*D_n)^{1/n}$}

\noindent is defined and there is a set
$E\in\Sigma$, of measure $\delta$, such that $E^n\setminus W_n$ is
negligible for every $n$;  but this means that

\Centerline{$(\mu^n)^*(E^n\cap D_n)=\mu^n(E^n\cap W_n)=\delta^n$}

\noindent for every $n$, as required.
}%end of proof of 465J

\leader{465K}{Lemma} Let $(X,\Sigma,\mu)$ be a complete probability
space, and $A\subseteq[0,1]^X$ a stable set.   Suppose that $\epsilon>0$
is such that $\int
fd\mu\le\epsilon^2$ for every $f\in A$.   Then there are an $n\ge 1$ and
a $W\in\Tensorhat_n\Sigma$\cmmnt{ (notation: 465Ad)} and a
$\gamma>\mu^nW$ such that $\int fd\nu\le 3\epsilon$ whenever $f\in A$
and $\nu$ is a probability measure on $X$
with domain including $\Sigma$ such that $\nu^nW\le\gamma$.

\proof{{\bf (a)} For $n\in\Bbb N$ write
$\tilde C_n=\bigcup_{f\in A}\{x:f(x)\ge\epsilon\}^n$.   Then
$\sequencen{\tilde C_n}$ satisfies
the conditions of 465Jc, so
$\delta=\lim_{n\to\infty}((\mu^n)^*\tilde C_n)^{1/n}$ is defined, and
there is an $E\in\Sigma$ such that $\mu E=\delta$ and
$(\mu^n)^*(E^n\cap\tilde C_n)=\delta^n$ for every
$n\in\Bbb N$.   Now, for $B\subseteq[0,1]^X$ and $n\in\Bbb N$, write
$C_n(B)=\bigcup_{f\in B}\{x:x\in E,\,f(x)\ge\epsilon\}^n$, so that
$C_n(A)=E^n\cap\tilde C_n$, and $(\mu^n)^*C_n(A)=\delta^n$ for every
$n$.   For any $B\subseteq[0,1]^X$, $\sequencen{C_n(B)}$ also satisfies
the conditions of 465Jc, and
$\delta_B=\lim_{n\to\infty}((\mu^n)^*C_n(B))^{1/n}$ is defined;  we have
$\delta_A=\delta$.

\medskip

{\bf (b)} If $B$, $B'\subseteq[0,1]^X$, then

\Centerline{$(\mu^n)^*C_n(B)\le(\mu^n)^*C_n(B\cup B')
\le(\mu^n)^*C_n(B)+(\mu^n)^*C_n(B')$}

\noindent for every $n$, so
$\delta_B\le\delta_{B\cup B'}\le\max(\delta_B,\delta_{B'})$.   It
follows that if

\Centerline{$\Cal G=\{G:G\subseteq[0,1]^X$ is $\frak T_p$-open,
$\delta_{G\cap A}<\delta\}$,}

\noindent where $\frak T_p$ is the usual topology of $[0,1]^X$, no
finite subfamily of $\Cal G$ can cover $A$.   Accordingly, since the
$\frak T_p$-closure $\overline{A}$ of $A$ is $\frak T_p$-compact, there
is an $h\in\overline{A}$ such that $\delta_{G\cap A}=\delta$ for every
$\frak T_p$-open set $G$ containing $h$.

\medskip

{\bf (c)} At this point recall that every function in $\overline{A}$ is
measurable (465Cb, 465Da) and that $\int:\overline{A}\to[0,1]$ is
$\frak T_p$-continuous (465G).   So $\int h\le\epsilon^2$ and
$\mu\{x:h(x)\ge\epsilon\}\le\epsilon$.

\Quer\ Suppose, if possible, that $\delta>\epsilon$.   Then there is
some $\eta>0$ such that $\mu F>0$, where
$F=\{x:x\in E,\,h(x)<\epsilon-\eta\}$.   For $k\in\Bbb N$, $u\in F^k$
set $G_u=\{f:f\in[0,1]^X,\,f(u(i))<\epsilon-\eta$ for every $i<k\}$.
Then $G_u$ is an open neighbourhood of $h$, so
$\delta_{G_u\cap A}=\delta$ and $(\mu^k)^*(C_k(G_u\cap A))\ge\delta^k$.
But because
$F\subseteq E$ and $C_k(G_u\cap A)\subseteq E^k$, this means that
$(\mu^k)^*(F^k\cap C_k(G_u\cap A))=(\mu F)^k$.

In the notation of 465Af,

\Centerline{$C_k(G_u\cap A)\cap F^k
\subseteq\{v:u\#v\in D_k(A,F,\epsilon-\eta,\epsilon)\}$}

\noindent for any $u\in F^k$.   So
$(\mu^k)^*\{v:u\#v\in D_k(A,F,\epsilon-\eta,\epsilon)\}=(\mu F)^k$
for every $u\in F^k$.   But this means that
$(\mu^{2k})^*D_k(A,F,\epsilon-\eta,\epsilon)=(\mu F)^{2k}$.   Since this
is so for every $k\ge 1$, $A$ is not stable.\
\Bang

\medskip

{\bf (d)} Thus $\delta\le\epsilon$.   There is therefore some $n\ge 1$
such that $(\mu^n)^*\tilde C_n<(2\epsilon)^n$.   Let
$W\in\Tensorhat_n\Sigma$ be a measurable envelope of $\tilde C_n$, and
try $\gamma=(2\epsilon)^n$.   If $\nu$ is any probability measure on $X$ with domain including $\Sigma$ such that $\nu^nW\le\gamma$, then for any
$f\in A$ we have

\Centerline{$\{x:f(x)\ge\epsilon\}^n\subseteq\tilde C_n\subseteq W$,
\quad$(\nu\{x:f(x)\ge\epsilon\})^n\le\nu^nW\le(2\epsilon)^n$,}

\noindent so that $\nu\{x:f(x)\ge\epsilon\}\le 2\epsilon$.   As
$0\le f(x)\le 1$ for every $x\in X$, $\int fd\nu\le3\epsilon$, as
required.
}%end of proof of 465K

\leader{465L}{Lemma} ({\smc Talagrand 87}) Let $(X,\Sigma,\mu)$ be a
complete probability space, and
$A\subseteq[0,1]^X$ a set which is not stable.   Then there are
measurable functions $h_0$, $h_1:X\to[0,1]$ such that
$\int h_0\,d\mu<\int h_1\,d\mu$ and $(\mu^{2k})^*\tilde D_k=1$ for every
$k\ge 1$, where

$$\eqalignno{\tilde D_k
=\bigcup_{f\in A}\{w:w\in X^{2k},\,&f(w(2i))\le h_0(w(2i)),\cr
&f(w(2i+1))\ge h_1(w(2i+1))\text{ for every }i<k\}.
&\cmmnt{(*)}\cr}$$

\proof{ The proof divides into two cases.

\medskip

{\bf case 1} Suppose that there is an ultrafilter $\Cal F$ on $A$ such
that the $\frak T_p$-limit $g_0$ of $\Cal F$ is not measurable.   Let
$h'_0$, $h_1':X\to[0,1]$ be measurable functions such that
$h'_0\le g_0\le h'_1$
and $\int h'_0=\underline{\int}g_0$, $\int h'_1=\overline{\int}g_0$
(133Ja).   Then $\delta=\bover15\int h'_1-h'_0>0$ (133Jd).   Set
$h_0=h'_0+2\delta\chi X$, $h_1=h'_1-2\delta\chi X$, so that
$\int h_0<\int h_1$.

Set $Q_0=\{x:x\in X,\,g_0(x)\le h_0(x)-\delta\}$.  Then $\mu^*Q_0=1$, by
133J(a-i).   Similarly, $\mu^*Q_1=1$, where
$Q_1=\{x:g_0(x)\ge h_1(x)+\delta\}$.

If $k\ge 1$, $u\in Q_0^k$ and $v\in Q_1^k$,
then there is an $f\in A$ such that
$|f(u(i))-g_0(u(i))|\le\delta$,
$|f(v(i))-g_0(v(i))|\le\delta$ for every $i<k$.   But this means that
$f(u(i))\le h_0(u(i))$ and $f(v(i))\ge h_1(v(i))$ for every $i<k$, and
$u\#v\in\tilde D_k$.   Thus

\Centerline{$\tilde D_k
\supseteq Q_0\times Q_1\times Q_0\times Q_1
  \times\ldots\times Q_0\times Q_1$,}

\noindent which has full outer measure, by 254L.   As $k$ is arbitrary,
we have found appropriate $h_0$, $h_1$ in this case.

\medskip

{\bf case 2} Now suppose that for every ultrafilter $\Cal F$ on $A$,
the $\frak T_p$-limit of $\Cal F$ is measurable.

\medskip

\quad{\bf (i)} We are supposing that $A$ is not stable, so there are
$E\in\Sigma$ and $\alpha<\beta$ such that
$\mu E>0$ and $(\mu^{2k})^*D_k(A,E,\alpha,\beta)\penalty-100=(\mu E)^{2k}$
for every
$k\ge 1$.   The first thing to note is that if $\Cal I$ is the set
of those $B\subseteq A$ for which there is some $k\in\Bbb N$ such that
$(\mu^{2k})^*D_k(B,E,\alpha,\beta)<(\mu E)^{2k}$, then $\Cal I$ is an
ideal of subsets of $A$.   \Prf\ Of course $\emptyset\in\Cal I$ and
$B\in\Cal I$ whenever $B\subseteq B'\in\Cal I$.   Also 465Cc tells us
that, if $B\in\Cal I$, then
$\lim_{k\to\infty}\Bover1{(\mu E)^{2k}}(\mu^{2k})^*D_k(B,E,\alpha,\beta)
=0$.   It follows easily (as in
the proof of 465Cd) that $B\cup B'\in\Cal I$ for all $B$,
$B'\in\Cal I$.\ \Qed

\medskip

\quad{\bf (ii)} $\Cal I$ is a proper ideal of subsets of $A$ (by the
choice of $E$, $k$, $\alpha$ and $\beta$), so there is an ultrafilter
$\Cal F$ on $A$ such that $\Cal F\cap\Cal I=\emptyset$.   Let $g_0$ be
the $\frak T_p$-limit of $\Cal F$.   Then $g_0$ is measurable.

Set $\delta=\bover13(\beta-\alpha)\mu E>0$, and define $h_0$, $h_1$ by
setting

$$\eqalign{h_0(x)&=\alpha\text{ if }x\in E,\cr
&=g_0(x)+\delta\text{ if }x\in X\setminus E,\cr
h_1(x)&=\beta\text{ if }x\in E,\cr
&=g_0(x)-\delta\text{ if }x\in X\setminus E.\cr}$$

\noindent Then

\Centerline{$\int h_1-\int h_0
\ge(\beta-\alpha)\mu E-2\delta\mu(X\setminus E)>0$.}

\medskip

\quad{\bf (iii)} We shall need to know a little more about sets of the
form $D_k(B,E,\alpha,\beta)$ for $B\in\Cal F$.   In fact, if
$B\subseteq A$ and $B\notin\Cal I$, then for any finite sets $I$ and $J$

\Centerline{$(\mu^I\times\mu^J)^*D_{I,J}(B,E,\alpha,\beta)
=(\mu E)^{\#(I)+\#(J)}$,}

\noindent where $D_{I,J}(B,E,\alpha,\beta)$ is

\Centerline{$\bigcup_{f\in B}\{(u,v):u\in E^I,\,v\in
E^J,\,f(u(i))\le\alpha$ for $i\in I$, $f(v(j))\ge\beta$ for $j\in J\}$.}

\noindent\Prf\ We may suppose that $I=k$ and $J=l$ where $k$, $l\in\Bbb
N$.   Take $m=\max(1,k,l)$.   Then we have an \imp\ map $\phi:X^{2m}\to
X^I\times X^J$ defined by saying that $\phi(w)=(u,v)$ where $u(i)=w(2i)$
for $i<k$ and $v(i)=w(2i+1)$ for $i<l$.   \Quer\ If
$(\mu^I\times\mu^J)^*D_{I,J}(B,E,\alpha,\beta)<(\mu E)^{k+l}$, there is
a non-negligible measurable set $V\subseteq (E^I\times E^J)\setminus
D_{I,J}(B,E,\alpha,\beta)$.   Now $\phi^{-1}[V]$ is
non-negligible and depends only on coordinates in
$\{2i:i<k\}\cup\{2i+1:i<l\}$, so

\Centerline{$\mu^{2m}(E^{2m}\cap\phi^{-1}[V])
=\mu^{2m}(\phi^{-1}[V])\cdot(\mu E)^{2m-k-l}>0$.}

\ifdim\pagewidth>467pt\fontdimen3\tenrm=2pt\fi
\ifdim\pagewidth>467pt\fontdimen4\tenrm=1.67pt\fi
\noindent But $\phi^{-1}[V]\cap D_m(B,E,\alpha,\beta)=\emptyset$, so
$(\mu^{2m})^*D_m(B,E,\alpha,\beta)<(\mu E)^{2m}$, and $B\in\Cal I$.\
\BanG\  So
$(\mu^I\times\mu^J)^*D_{I,J}(B,E,\alpha,\beta)
=(\mu E)^{k+l}$, as required.\ \Qed
\fontdimen3\tenrm=1.67pt
\fontdimen4\tenrm=1.11pt

\medskip

\quad{\bf (iv)} \Quer\ Suppose, if possible, that $k\ge 1$ is such that
$(\mu^{2k})^*\tilde D_k<1$, where $\tilde D_k$ is defined from $h_0$ and
$h_1$ by the formula (*) in the statement of the lemma.   Let
$W\subseteq X^{2k}$ be a
measurable set of positive measure disjoint from $\tilde D_k$.   For
$I$, $J\subseteq k$ write $W_{IJ}$ for

$$\eqalign{\{w:w\in W,\,
&w(2i)\in E\text{ for }i\in I,\,
  w(2i)\notin E\text{ for }i\in k\setminus I,\cr
&w(2i+1)\in E\text{ for }i\in J,\,
  w(2i+1)\notin E\text{ for }i\in k\setminus J\}.\cr}$$

\noindent Then there are $I$, $J\subseteq K$ such that
$\mu^{2k}W_{IJ}>0$.

We can identify $X^{2k}$ with
$X^I\times X^J\times X^{k\setminus I}\times X^{k\setminus J}$, matching
any $w\in X^{2k}$ with $(w_0,w_1,w_2,w_3)$ where

$$\eqalign{w_0(i)&=w(2i)\text{ for }i\in I,\cr
w_1(i)&=w(2i+1)\text{ for }i\in J,\cr
w_2(i)&=w(2i)\text{ for }i\in k\setminus I,\cr
w_3(i)&=w(2i+1)\text{ for }i\in k\setminus J.\cr}$$

\noindent Write $\tilde W$ for the image of $W_{IJ}$ under this
matching.
The condition $W_{IJ}\cap\tilde D_k=\emptyset$ translates into

\inset{($\dagger$) whenever $(w_0,w_1,w_2,w_3)\in\tilde W$, $f\in A$,

\quad either there is an $i\in I$ such that $f(w_0(i))>\alpha$

\quad or there is an $i\in J$ such that $f(w_1(i))<\beta$

\quad or there is an $i\in k\setminus I$ such that
$f(w_2(i))>g_0(w_2(i))+\delta$

\quad or there is an $i\in k\setminus J$ such that
$f(w_3(i))<g_0(w_3(i))-\delta$.}

\medskip

\quad{\bf (v)} By Fubini's theorem, applied to
$(X^I\times X^J)\times(X^{k\setminus I}\times X^{k\setminus J})$, we can
find $w_2\in X^{k\setminus I}$, $w_3\in X^{k\setminus J}$ such that
$(\mu^I\times\mu^J)(V)$ is defined and greater than $0$, where
$V=\{(w_0,w_1):(w_0,w_1,w_2,w_3)\in\tilde W\}$.   Set

$$\eqalign{B=\{f:f\in A,\,
&|f(w_2(i))-g_0(w_2(i))|\le\delta\text{ for }i\in k\setminus I,\cr
&|f(w_3(i))-g_0(w_3(i))|\le\delta\text{ for }i\in k\setminus J\}.\cr}$$

\noindent Then $B\in\Cal F$, because $\Cal F\to g_0$ for $\frak T_p$.
So $(\mu^I\times\mu^J)^*D_{I,J}(B,E,\alpha,\beta)
=(\mu E)^{\#(I)+\#(J)}$, by (iii) above.   Since
$W_{IJ}\subseteq W$, $V$ is included in $E^I\times E^J$
and meets
$D_{I,J}(B,E,\alpha,\beta)$;  that is, there are $f\in B$ and
$(w_0,w_1)\in V$ such that $f(w_0(i))\le\alpha$ for $i\in I$ and
$f(w_1(i))\ge\beta$ for $i\in J$.   But because $f\in B$ we also have
$f(w_2(i))\le g_0(w_2(i))+\delta$ for $i\in k\setminus I$ and
$f(w_3(i))\ge g_0(w_3(i))-\delta$ for $i\in k\setminus J$;  which
contradicts the list in ($\dagger$) above.\ \Bang

\medskip

\quad{\bf (vi)} Thus $(\mu^{2k})^*\tilde D_k=1$ for every $k$, and in
this case also we have an appropriate pair $h_0$, $h_1$.
}%end of proof of 465L

\vleader{108pt}{465M}{Theorem}\cmmnt{ ({\smc Talagrand 82},
{\smc Talagrand 87})} Let $(X,\Sigma,\mu)$ be a complete probability
space, and $A$ a non-empty uniformly bounded set of real-valued
functions defined on $X$.   Then the following are equiveridical:

(i) $A$ is stable.

(ii) Every function in $A$ is measurable, and
$\lim_{k\to\infty}\sup_{f\in A}
|\Bover1k\sum_{i=0}^{k-1}f(w(i))-\int f|=0$ for almost every
$w\in X^{\Bbb N}$.

(iii) Every function in $A$ is measurable, and for every $\epsilon>0$
there are a finite subalgebra $\Tau$ of $\Sigma$ in which every atom
is non-negligible and a sequence $\langle h_k\rangle_{k\ge 1}$ of
measurable functions on $X^{\Bbb N}$ such that

\Centerline{$h_k(w)\ge\sup_{f\in A}\Bover1k\sum_{i=0}^{k-1}
\bigl|f(w(i))-\Expn(f|\Tau)(w(i))\bigr|$}

\noindent for every $w\in X^{\Bbb N}$ and $k\ge 1$, and

\Centerline{$\limsup_{k\to\infty}h_k(w)\le\epsilon$}

\noindent for almost every $w\in X^{\Bbb N}$.   (Here
$\Expn(f|\Tau)$ is the\cmmnt{ (unique)} conditional expectation
of $f$ on $\Tau$.)

(iv) $\lim_{k,l\to\infty}\sup_{f\in A}
|\Bover1k\sum_{i=0}^{k-1}f(w(i))-\Bover1l\sum_{i=0}^{l-1}f(w(i))|=0$ for
almost every $w\in X^{\Bbb N}$.

(v) $\lim_{k,l\to\infty}\overline{\int}\sup_{f\in A}
|\Bover1k\sum_{i=0}^{k-1}f(w(i))-\Bover1l\sum_{i=0}^{l-1}f(w(i))|
 \mu^{\Bbb N}(dw)
=0$.

\proof{ All the statements (i)-(v) are unaffected by translations (by
constant functions) and
scalar multiplications of the set $A$, so it will be enough to consider
the case in which $A\subseteq[0,1]^X$.

As in 465I-465H, I write $\nu_{wk}(E)=\bover1k\#(\{i:i<k,\,w(i)\in E\})$
for $w\in X^{\Bbb N}$, $k\ge 1$ and $E\subseteq X$;  so for any function
$f:X\to\Bbb R$, $\int fd\nu_{wk}=\Bover1k\sum_{i=0}^{r-1}f(w(i))$.

\medskip

{\bf (a)(i)$\Rightarrow$(iii)}\grheada\ If $A$ is stable, then every
function in $A$ is measurable, by 465Da.   Let $\epsilon>0$.   Set
$\eta=\bover1{108}\epsilon^2>0$.   By 465Db, the $\frak T_p$-closure
$\overline{A}$ of $A$ in $\Bbb R^X$ is a $\frak T_p$-compact set of
measurable functions, and by 465G it is $\frak T_m$-compact;  because
$A$ is uniformly bounded, it must be totally bounded for the
pseudometric induced
by $\|\,\|_1$.   So there are $f_0,\ldots,f_m\in A$ such that for every
$f\in A$ there is an $i\le m$ such that $\|f-f_i\|_1\le\eta$.   Let
$\Tau_0$ be the finite subalgebra of $\Sigma$ generated by the sets
$\{x:j\eta\le f_i(x)<(j+1)\eta\}$ for $i\le m$ and $j\le\Bover1{\eta}$.
Then $\Tau_0$ may have negligible atoms, but if we absorb these into
non-negligible atoms we get a finite subalgebra $\Tau$ of $\Sigma$ such
that $|f_i(x)-\Expn(f_i|\Tau)(x)|\le\eta$ for almost every $x\in X$,
every $i\le m$.   (Because $\Tau$ is a finite algebra with
non-negligible atoms, two $\Tau$-measurable functions which are equal
almost everywhere must be identical, and we have unique conditional
expectations with respect to $\Tau$.)   Since
$\|\Expn(f|\Tau)-\Expn(g|\Tau)\|_1\le\|f-g\|_1$ for all integrable
functions $f$ and $g$ (242Je), $\|f-\Expn(f|\Tau)\|_1\le 3\eta$ for
every $f\in A$.

\medskip

\quad\grheadb\ Set $A'=\{f-\Expn(f|\Tau):f\in A\}$.   Then $A'$ is
stable.   \Prf\ Suppose that $\mu E>0$ and $\alpha<\beta$.   The set
$B=\{\Expn(f|\Tau):f\in A\}$ is a uniformly bounded subset of a
finite-dimensional space of functions, so is $\|\,\|_{\infty}$-compact.
So there are $g_0,\ldots,g_r\in B$ such that
$B\subseteq\bigcup_{i\le r}\{g:\|g-g_i\|_{\infty}
\le\bover13(\beta-\alpha)\}$.   By 465Cf and
465Cd, $C=\bigcup_{i\le r}A-g_i$ is stable.   So there is a $k\ge 1$
such that

\Centerline{$(\mu^{2k})^*D_k(C,E,\bover23\alpha+\bover13\beta,
\bover13\alpha+\bover23\beta)<(\mu E)^{2k}$.}

\noindent But for every $g\in A'$ there is an $h\in C$ such that
$\|g-h\|_{\infty}\le\bover13(\beta-\alpha)$, so

\Centerline{$D_k(A',E,\alpha,\beta)\subseteq
D_k(C,E,\bover23\alpha+\bover13\beta,\bover13\alpha+\bover23\beta)$,
\quad$(\mu^{2k})^*D_k(A',E,\alpha,\beta)<(\mu E)^{2k}$.}

\noindent As $E$, $\alpha$ and $\beta$ are arbitrary, $A'$ is stable.\
\Qed

\medskip

\quad\grheadc\ By 465Cl,

\Centerline{$A''=\{g^+:g\in A'\}\cup\{g^-:g\in A'\}$}

\noindent is stable.
By 465K there are an $n\ge 1$, a $W\in\Tensorhat_n\Sigma$
and a $\gamma>\mu^nW$ such that
$\int h\,d\nu\le 3\sqrt{3\eta}=\bover12\epsilon$
whenever $h\in A''$ and $\nu$ is a probability measure on $X$,
with domain including $\Sigma$, such that $\nu^nW\le\gamma$.   So
$\int|g|d\nu\le\epsilon$ whenever $g\in A'$ and $\nu$ is such a measure.
If $w\in X^{\Bbb N}$ and $k\in\Bbb N$,
$\nu_{wk}$ is a probability measure on $X$;  set
$q_k(w)=\nu_{wk}^n(W)$.
Applying 465H to the indicator function of $W$, we see that
$\lim_{k\to\infty}q_k(w)=\mu^nW$ for almost every $w\in X^{\Bbb N}$.
Also, because $W\in\Tensorhat_n\Sigma$, every $q_k$ is measurable, by
465I.

Set $h_k(w)=1$ if $q_k(w)>\gamma$, $\epsilon$ if $q_k(w)\le\gamma$.
Then every $h_k$ is measurable and $\lim_{k\to\infty}h_k(w)=\epsilon$
for almost every $w$.

For any $w\in X^{\Bbb N}$ and any $f\in A$, $g=f-\Expn(f|\Tau)\in A'$
and $\|g\|_{\infty}\le 1$.   So either $h_k(w)=1$ and certainly
$\int|g|d\nu_{wk}\le h_k(w)$, or $h_k(w)=\epsilon$,
$\nu_{wk}^n(W)\le\gamma$ and $\int|g|d\nu_{wk}\le\epsilon$.   Thus we
have

\Centerline{$\Bover1k\sum_{i=0}^{k-1}\bigl|f(w(i))
  -\Expn(f|\Tau)(w(i))\bigr|
=\int\bigl|f-\Expn(f|\Tau)\bigr|d\nu_{wk}\le h_k(w)$}

\noindent for every $w\in X^{\Bbb N}$ and every $f\in A$, as required by
(iii).

\medskip

{\bf (b)(iii)$\Rightarrow$(ii) \& (v)} Set

\Centerline{$g_k(w)
=\sup_{f\in A}|\Bover1k\sum_{i=0}^{k-1}f(w(i))-\int fd\mu|$,}

\Centerline{$g'_{kl}(w)
=\sup_{f\in A}|\Bover1k\sum_{i=0}^{k-1}f(w(i))
  -\Bover1l\sum_{i=0}^{l-1}f(w(i))|$,}

\noindent for $w\in X^{\Bbb N}$ and $k$, $l\ge 1$.
Let $\epsilon>0$.   Let $\Tau$ and $\langle h_k\rangle_{k\ge 1}$ be as in
(iii), and let $E_0,\ldots,E_r$ be the atoms of $\Tau$.   For
$w\in X^{\Bbb N}$, $k\ge 1$, $j\le r$ set
$q_{kj}(w)=|\mu E_j-\nu_{wk}E_j|$.   Then for any $f\in A$,
$\Expn(f|\Tau)$ is expressible as $\sum_{j=0}^r\alpha_j\chi E_j$ where
$\alpha_j\in[0,1]$ for every $j$ (remember that $A\subseteq[0,1]^X$), so

$$\eqalign{\bigl|\Bover1k\sum_{i=0}^{k-1}f(w(i))&-\int fd\mu\bigr|
=\bigl|\int fd\nu_{wk}-\int fd\mu\bigr|\cr
&\le\bigl|\int fd\nu_{wk}-\int\Expn(f|\Tau)d\nu_{wk}\bigr|
  +\bigl|\int\Expn(f|\Tau)d\nu_{wk}-\int\Expn(f|\Tau)d\mu\bigr|\cr
&\le\Bover1k\sum_{i=0}^{k-1}\bigl|f(w(i))-\Expn(f|\Tau)(w(i))\bigr|
  +\sum_{j=0}^r\alpha_j\bigl|\mu E_j-\nu_{wk}E_j\bigr|\cr
&\le h_k(w)+\sum_{j=0}^rq_{kj}(w),\cr
\bigl|\Bover1k\sum_{i=0}^{k-1}f(w(i))
  &-\Bover1l\sum_{i=0}^{l-1}f(w(i))\bigr|\cr
&\le h_k(w)+\sum_{j=0}^rq_{kj}(w)+h_l(w)+\sum_{j=0}^rq_{lj}(w).\cr}$$

\noindent Taking the supremum over $f$, we have

\Centerline{$g_k(w)\le
h_k(w)+\sum_{j=0}^rq_{kj}(w)$,}

\Centerline{$g'_{kl}(w)\le
h_k(w)+\sum_{j=0}^rq_{kj}(w)+\sum_{j=0}^rq_{lj}(w)+h_l(w)$.}

\noindent But, for each $j\le r$,
$\lim_{k\to\infty}q_{kj}(w)=0$ for almost every $w$, by 465H (or 273J)
applied to the indicator function of $E_j$.   So

\Centerline{$\limsup_{k\to\infty}g_k(w)
\le\limsup_{k\to\infty}h_k(w)\le\epsilon$}

\noindent for almost every $w$.   At the same time,

\Centerline{$
\limsup_{k,l\to\infty}\overlineint g'_{kl}
\le\limsup_{k,l\to\infty}\int h_k+\sum_{j=0}^r\int q_{kj}
  +\sum_{j=0}^r\int q_{lj}+\int h_l
\le 2\epsilon$.}

As $\epsilon$ is arbitrary,
$\{w:\limsup_{k\to\infty}g_k(w)\ge 2^{-i}\}$ is negligible for every
$i\in\Bbb N$, and $\lim_{k\to\infty}g_k=0$ almost everywhere, as
required, while equally $\lim_{k,l\to\infty}\overline{\int}g'_{kl}=0$.
Thus (ii) and (v) are true.

\wheader{465M}{4}{2}{2}{36pt}

{\bf (c)(ii)$\Rightarrow$(iv)} is trivial, since

\Centerline{$\bigl|\Bover1k\sum_{i=0}^{k-1}f(w(i))
  -\Bover1l\sum_{i=0}^{l-1}f(w(i))\bigr|
\le\bigl|\Bover1k\sum_{i=0}^{k-1}f(w(i))-\int f\bigr|
  +\bigl|\Bover1l\sum_{i=0}^{l-1}f(w(i))-\int f\bigr|$.}

\medskip

{\bf (d) not-(i)$\Rightarrow$not-(iv) \& not-(v)} Now suppose that $A$
is not stable.

\medskip

\quad\grheada\ In this case, by 465L, there are measurable functions
$h_0$, $h_1:X\to[0,1]$ such that $\int h_0d\mu<\int h_1d\mu$ and
$(\mu^{2k})^*\tilde D_k=1$ for every $k\in\Bbb N$, where

$$\eqalign{\tilde D_k
=\bigcup_{f\in A}\{w:w\in X^{2k},\,&f(w(2i))\le h_0(w(2i)),\cr
&f(w(2i+1))\ge h_1(w(2i+1))\text{ for every }i<k\}.\cr}$$

\medskip

\quad\grheadb\ Set $\delta=\bover14\int h_1-h_0>0$.   Let $k_0\ge 1$ be
so large that

\Centerline{$\mu^k\{w:w\in X^k,\,
|\int h_j-\Bover1k\sum_{i=0}^{k-1}h_j(w(i))|
\le\delta\}\ge\Bover12$}

\noindent for both $j$ and for every $k\ge k_0$ (273J or 465H.   The
point is that

$$\eqalign{\mu^k\{w:w\in X^k,\,
&|\int h_j-\Bover1k\sum_{i=0}^{k-1}h_j(w(i))|
\le\delta\}\cr
&=\mu^{\Bbb N}\{w:w\in X^{\Bbb N},\,
|\int h_j-\Bover1k\sum_{i=0}^{k-1}h_j(w(i))|
\le\delta\}
\to 1\text{ as }k\to\infty.)\cr}$$

\noindent Let $\langle k_n\rangle_{n\ge 1}$ be such that
$k_n\ge\Bover2{\delta}\sum_{i=0}^{n-1}k_i$ for every $n\ge 1$.   Since
$\delta\le\bover14$, every $k_n$ is at least as large as $k_0$.

\medskip

\quad\grheadc\ Set $m_n=2\sum_{i<n}k_i$ for each $n\in\Bbb N$.   Then we
have a measure space isomorphism
$\phi:\prod_{n\in\Bbb N}X^{2k_n}\to X^{\Bbb N}$ defined by setting

\Centerline{$\phi(w)(m_n+i)=w(n)(2i)$,
\quad$\phi(w)(m_n+k_n+i)=w(n)(2i+1)$}

\noindent for $n\in\Bbb N$ and $i<k_n$.   For each $n\in\Bbb N$,
$\tilde D_{k_n}$ has outer measure $1$ in $X^{2k_n}$, so
$\tilde D=\prod_{n\in\Bbb N}\tilde D_{k_n}$ has outer measure $1$ in
$\prod_{n\in\Bbb N}X^{2k_n}$, and $\phi[\tilde D]$ has outer measure $1$
in $X^{\Bbb N}$.   Note that $\phi[\tilde D]$ is just the set of
$w\in X^{\Bbb N}$ such that, for every $n\in\Bbb N$, there is an
$f\in A$ such that

$$\eqalign{f(w(i))&\le h_0(w(i))\text{ for }m_n\le i<m_n+k_n,\cr
f(w(i))&\ge h_1(w(i))\text{ for }m_n+k_n\le i<m_n+2k_n=m_{n+1}.\cr}$$

If we set

\Centerline{$V_{n0}=\{w:w\in X^{\Bbb N},\,
|\Bover1{k_n}\sum_{i=m_n}^{m_n+k_n-1}h_0(w(i))-\int h_0|
\le\delta\}$,}

\Centerline{$V_{n1}=\{w:w\in X^{\Bbb N},\,
|\Bover1{k_n}\sum_{i=m_n+k_n}^{m_n+2k_n-1}
h_1(w(i))-\int h_1|\le\delta\}$,}

\noindent every $V_{n0}$ and $V_{n1}$ has measure at least $\bover12$,
because $k_n\ge k_0$.

\medskip

\quad\grheadd\ Now suppose that $n\in\Bbb N$ and
$w\in V_{n0}\cap V_{n1}\cap\phi[\tilde D]$.   Then

$$\eqalign{\sup_{f\in A}\bigl|\Bover1{m_n+2k_n}
  \sum_{i=0}^{m_n+2k_n-1}&f(w(i))
  -\Bover1{m_n+k_n}\sum_{i=0}^{m_n+k_n-1}f(w(i))\bigr|
\ge\Bover{\delta}{(2+\delta)(1+\delta)}.\cr}$$

\noindent\Prf\ Since $w\in\phi[\tilde D]$, there must be an
$f\in A$ such that

$$\eqalign{f(w(i))&\le h_0(w(i))\text{ for }m_n\le i<m_n+k_n,\cr
f(w(i))&\ge h_1(w(i))\text{ for }m_n+k_n\le i<m_n+2k_n=m_{n+1}.\cr}$$

\noindent Set $s=\sum_{i=0}^{m_n-1}f(w(i))$,
$t=\sum_{i=m_n}^{m_n+k_n}f(w(i))$ and
$t'=\sum_{i=m_n+k_n}^{m_n+2k_n-1}f(w(i))$.   Then

$$\eqalign{\Bover1{m_n+2k_n}\sum_{i=0}^{m_n+2k_n-1}&f(w(i))
  -\Bover1{m_n+k_n}\sum_{i=0}^{m_n+k_n-1}f(w(i))\cr
&=\bover{s+t+t'}{m_n+2k_n}-\bover{s+t}{m_n+k_n}
=\bover{(t'-t-s)k_n+t'm_n}{(m_n+2k_n)(m_n+k_n)}\cr
&\ge\bover{(t'-t-m_n)k_n}{(m_n+2k_n)(m_n+k_n)}
\ge\bover{t'-t-\delta k_n}{(2+\delta)(1+\delta)k_n}\cr}$$

\noindent because $m_n\le\delta k_n$, by the choice of $k_n$.

To estimate $t$ and $t'$, we have

\Centerline{$t
=\sum_{i=m_n}^{m_n+k_n-1}f(w(i))
\le\sum_{i=m_n}^{m_n+k_n-1}h_0(w(i))
\le k_n(\int h_0+\delta)$}

\noindent because $w\in V_{n0}$.   On the other hand,

\Centerline{$t'
=\sum_{i=m_n+k_n}^{m_n+2k_n-1}f(w(i))
\ge\sum_{i=m_n+k_n}^{m_n+2k_n-1}h_1(w(i))
\ge k_n(\int h_1-\delta)$}

\noindent because $w\in V_{n1}$.   So

\Centerline{$t'-t\ge k_n(\int h_1-h_0-2\delta)=2k_n\delta$,}

\noindent and

$$\eqalign{\Bover1{m_n+2k_n}\sum_{i=0}^{m_n+2k_n-1}&f(w(i))
  -\Bover1{m_n+k_n}\sum_{i=0}^{m_n+k_n-1}f(w(i))\cr
&\ge\bover{2\delta k_n-\delta k_n}{(2+\delta)(1+\delta)k_n}
=\bover{\delta}{(2+\delta)(1+\delta)}.\text{ \Qed}\cr}$$

\medskip

\quad\grheade\ The $V_{n0}$ and $V_{n1}$ are all
independent.   So $\mu^{\Bbb N}(V_{n0}\cap V_{n1})\ge\bover14$ for every
$n$, and

\Centerline{$W=\{w:w\in X^{\Bbb N},\,w\in V_{n0}\cap V_{n1}$ for
infinitely many $n\}$}

\noindent has measure $1$ (by the Borel-Cantelli lemma, 273K, or
otherwise).
Accordingly $W\cap\phi[\tilde D]$ has outer measure $1$ in
$X^{\Bbb N}$.   But if $w\in W\cap\phi[\tilde D]$, then ($\delta$) tells
us that

$$\eqalign{\limsup_{k,l\to\infty}
\sup_{f\in A}\bigl|\Bover1k\sum_{i=0}^{k-1}f(w(i))
  -\Bover1l\sum_{i=0}^{l-1}f(w(i))\bigr|
\ge\Bover{\delta}{(2+\delta)(1+\delta)},\cr}$$

\noindent because there are infinitely many $n$ such that
$w\in V_{n0}\cap V_{n1}\cap\phi[\tilde D]$.   So (iv) must be false.

\medskip

\quad\grheadz\ We see also that, for any $n\in\Bbb N$,

%$$\eqalign{\overline{\int}g'_{m_n+k_n,m_n+2k_n}
%&\ge\Bover{\delta}{(2+\delta)(1+\delta)}
%  \mu^{\Bbb N}(V_{n0}\cap V_{n1})\cr
%&\ge\Bover{\delta}{4(2+\delta)(1+\delta)},\cr}$$

\Centerline{$\overlineint g'_{m_n+k_n,m_n+2k_n}
\ge\Bover{\delta}{(2+\delta)(1+\delta)}
  \mu^{\Bbb N}(V_{n0}\cap V_{n1})
\ge\Bover{\delta}{4(2+\delta)(1+\delta)}$,}

\noindent writing

\Centerline{$g'_{kl}(w)=\sup_{f\in A}
 |\Bover1k\sum_{i=0}^{k-1}f(w(i))-\Bover1l\sum_{i=0}^{l-1}f(w(i))|$}

\noindent for $k$, $l\in\Bbb N$.   So
$\limsup_{k,l\to\infty}\overline{\int}g'_{kl}>0$, and (v) is false.
}%end of proof of 465M

\cmmnt{\medskip

\noindent{\bf Remark} If $(X,\Sigma,\mu)$ is a probability space, a set
$A\subseteq\eusm L^1(\mu)$ is a {\bf Glivenko-Cantelli class} if
$\sup_{f\in A}|\Bover1k\sum_{i=0}^{k-1}\penalty-100f(w(i))-\int f|\to 0$ as
$k\to\infty$ for $\mu^{\Bbb N}$-almost every $w\in X^{\Bbb N}$.
Compare 273Xi.}

\leader{465N}{Theorem}
Let $(X,\Sigma,\mu)$ be a semi-finite measure space.

(a)\cmmnt{ ({\smc Talagrand 84})} %11-2-1
Let $A\subseteq\BbbR^X$ be a stable
set.   Suppose that there is a measurable function
$g:X\to\coint{0,\infty}$ such that $|f(x)|\le g(x)$ whenever $x\in X$ and
$f\in A$.   Then the convex hull $\Gamma(A)$ of $A$ in $\Bbb R^X$ is
stable.

(b) If $A\subseteq\BbbR^X$ is stable, then $|A|=\{|f|:f\in A\}$ is stable.

(c) Let $A$, $B\subseteq\Bbb R^X$ be two stable sets such that
$\{f(x):f\in A\cup B\}$ is bounded for every $x\in X$.   Then
$A+B=\{f_1+f_2:f_1\in A,\,f_2\in B\}$ is stable.

(d) Suppose that $\mu$ is complete and locally determined.
Let $A\subseteq\BbbR^X$ be a stable set such that
$\{f(x):f\in A\}$ is bounded for every $x\in X$.   Then $\Gamma(A)$ is
relatively compact in $\eusm L^0(\Sigma)$ for the topology of pointwise
convergence.

\proof{{\bf (a)(i)} Consider first the case in which $\mu X=1$ and
$A\subseteq[-1,1]^X$.   In this case,

\Centerline{$\sup_{f\in\Gamma(A)}|
  \Bover1k\sum_{i=0}^{k-1}f(w(i))-\int f|
=\sup_{f\in A}|\Bover1k\sum_{i=0}^{k-1}f(w(i))-\int f|$}

\noindent for every $k\ge 1$ and $w\in X^{\Bbb N}$.   So $\Gamma(A)$
satisfies condition (ii) of 465M whenever $A$ does, and we have the
result.

\medskip

\quad{\bf (ii)} Now suppose just that $\mu X=1$.   Set
$A'=\{\Bover{f}{g+\chi X}:f\in A\}$.   Then $A'$ is stable (465Ch), so
$\Gamma(A')$ is stable, by (i), and
$\Gamma(A)=\{f\times(g+\chi X):f\in A'\}$ is stable.

\medskip

\quad{\bf (iii)} If $\mu X=0$, the result is trivial.   If $\mu X<\infty$,
apply (ii) to a multiple of the measure $\mu$.   For the general case,
write $A_E=\{f\restr E:f\in A\}$ for $E\subseteq X$.   Then $A_E$ is stable
for the subspace measure on $E$, by 465Cm.
It follows that $\Gamma(A_E)$ is stable
whenever $\mu E<\infty$.   But $\Gamma(A_E)=(\Gamma(A))_E$, so 465Cn tells
us that $\Gamma(A)$ is stable.

\medskip

{\bf (b)(i)} I begin with a basic special case of (c).   If $A$,
$B\subseteq\BbbR^X$ are stable and uniformly bounded, then $A+B$ is stable.
\Prf\ Putting 465Cd, (a) of this theorem, 465Ce and 465Ca together, we see
that $A\cup B$, $\Gamma(A\cup B)$ and $A+B\subseteq 2\Gamma(A\cup B)$ are
stable.\ \Qed

\medskip

\quad{\bf (ii)} Adding this to 465Cl,
$|A|\subseteq\{f^+:f\in A\}+\{f^-:f\in A\}$ is stable
whenever $A\subseteq\BbbR^X$ is stable and uniformly bounded.

\medskip

\quad{\bf (iii)} For the general case, set $h(\alpha)=\tanh\alpha$ for
$\alpha\in\Bbb R$.   Then 465Ck and (ii) here tell us that if
$A\subseteq\BbbR^X$ is stable then

\Centerline{$\{hf:f\in A\}$,
\quad$\{|hf|:f\in A\}$,
\quad$\{h^{-1}|hf|:f\in A\}=|A|$}

\noindent are stable.

\medskip

{\bf (c)(i)} By 465Cn, it is enough to consider the case of totally finite
$\mu$, so let us suppose from now on that $\mu X<\infty$.   We may also
suppose that neither $A$ nor $B$ is empty;  finally, by 465Ci,
we can suppose that $\mu$ is complete, so that
$A\cup B\subseteq\eusm L^0(\Sigma)$ (465Da).

I introduce some temporary notation:  if $E\subseteq X$, $k\ge 1$,
$\epsilon>0$ and $A\subseteq\BbbR^X$, set

\Centerline{$\tilde D_k(A,E,\epsilon)
=\bigcup_{f\in A}\{u:u\in E^k$, $|f(u(i))|\ge\epsilon$ for $i<k\}$.}

\medskip

\quad{\bf (ii)} We need to know that if $A\subseteq\BbbR^X$ and every
$f\in A$ is zero a.e., then $A$ is stable iff whenever $E\in\Sigma$,
$0<\mu E<\infty$ and $\epsilon>0$ there is a $k\ge 1$ such that
$(\mu^k)^*\tilde D_k(A,E,\epsilon)<(\mu E)^k$.

\medskip

\quad\Prf\grheada\ If $A$ is stable, then $|A|$ is stable,
by (b), so if $0<\mu E<\infty$ and $\epsilon>0$ there is a $k\ge 1$ such
that $(\mu^{2k})^*D_{2k}(|A|,E,0,\epsilon)<(\mu E)^{2k}$.   Let
$W\in\Tensorhat_{2k}\Sigma$ be such that
$D_{2k}(|A|,E,0,\epsilon)\subseteq W\subseteq E^{2k}$
and $\mu^{2k}W<(\mu E)^{2k}$.   Because $(u,v)\mapsto u\#v$ is a measure
space isomorphism,

\Centerline{$\mu^{2k}W=\int\mu^k\{u:u\#v\in W\}\mu^k(dv)$,}

\noindent so if we set
$V=\{v:v\in E^k$, $\mu^k\{u:u\#v\in W\}=(\mu E)^k\}$ we must have
$\mu^kV<(\mu E)^k$.   If
$v\in\tilde D_k(A,E,\epsilon)$, there is an $f\in A$ such that
$|f(v(i))|\ge\epsilon$ for every $i<k$;  now

\Centerline{$\{u:u\#v\in W\}
\supseteq\{u:u\in E^k$, $f(u(i))=0$ for every $i<k\}$}

\noindent has measure $(\mu E)^k$, because $f=0$ a.e.   So
$\tilde D_k(A,E,\epsilon)\subseteq V$ and
$(\mu^k)^*\tilde D_k(A,E,\epsilon)<(\mu E)^k$.
As $E$ and $\epsilon$ are arbitrary, $A$ satisfies the
condition.

\medskip

\qquad\grheadb\ Now suppose that $A$ satisfies the condition.
Take $E\in\Sigma$ such
that $\mu E>0$, and $\alpha<\beta$ in $\Bbb R$.   If $\beta>0$, set
$\epsilon=\beta$;  otherwise, set $\epsilon=-\alpha$.   Then there is a
$k\in\Bbb N$ such that $(\mu^k)^*\tilde D_k(A,E,\epsilon)<(\mu E)^k$.
As $D_k(A,E,\alpha,\beta)$ is included in
$\{(u,v):u\in E^k$, $v\in\tilde D_k(A,E,\epsilon)\}$ (if $\beta>0$) or
$\{(u,v):u\in\tilde D_k(A,E,\epsilon)$, $v\in E^k\}$ (if $\beta\le 0$),
$(\mu^{2k})^*D_k(A,E,\alpha,\beta)<(\mu E)^{2k}$.   As $E$, $\alpha$ and
$\beta$ are arbitrary, $A$ is stable.\ \Qed

\medskip

\quad{\bf (iii)} Suppose that $A$ and $B$ are stable sets such that
$f=0$ a.e.\ for every $f\in A\cup B$.   Then $A+B$ is stable.   \Prf\ Set
$A'=\{|f|\wedge\chi X:f\in A\}$, $B'=\{|f|\wedge\chi X:f\in B\}$.   Then
$A'$ and $B'$ are stable, so $A'+B'$ is stable, by (b)(i) above.
But now observe that if $u\in\tilde D_k(A+B,E,\epsilon)$, where
$E\subseteq X$, $\epsilon>0$ and $k\ge 1$, then there are $f_1\in A$,
$f_2\in B$ such that $|f_1(u(i))+f_2(u(i))|\ge\epsilon$ for every $i<k$.
In this case, setting $f_j'=|f_j|\wedge\chi X$ for both $j$, $g=f_1'+f_2'$
belongs to $A'+B'$ and $g(u(i))\ge\min(1,\epsilon)$ for every $i<k$.
This shows that
$\tilde D_k(A+B,E,\epsilon)\subseteq\tilde D_k(A'+B',E,\min(1,\epsilon))$.
Also every function in either $A+B$ or $A'+B'$ is zero a.e.   So
(ii) tells us that $A+B$ also is stable.\ \Qed

\medskip

\quad{\bf (iv)} Suppose that $A$, $B\subseteq\BbbR^X$ are stable, that
$|f|\le\chi X$ for every $f\in A$, and that $g=0$ a.e.\ for every $g\in B$.
Then $A+B$ is stable.   \Prf\ For $g\in B$ set
$g'(x)=\med(-2,g(x),2)$
for $x\in X$;  set $B'=\{g':g\in B\}$.   Then $B'$ is stable, by 465Ck, and
both $A$ and $B'$ are uniformly bounded, so $A+B'$
is stable.   Take $E\in\Sigma$ such that $\mu E>0$, and
$\alpha<\beta$ in $\Bbb R$.

If $\beta>1$, then, by (ii), there is a $k\ge 1$ such that
$(\mu^k)^*\tilde D_k(B,E,\beta-1)<(\mu E)^k$.   Now if
$w\in D_k(A+B,E,\alpha,\beta)$ there are $f\in A$, $g\in B$ such that
$f(w(2i+1))+g(w(2i+1))\ge\beta$ for every $i<k$;  accordingly
$g(w(2i+1))\ge\beta-1$ for $i<k$ and $w=u\#v$ for some $u\in E^k$,
$v\in\tilde D_k(B,E,\beta-1)$.   So

\Centerline{$(\mu^{2k})^*D_k(A+B,E,\alpha,\beta)
\le(\mu E)^k\cdot(\mu^k)^*\tilde D(B,E,\beta-1)<(\mu E)^{2k}$.}

\noindent Similarly, if $\alpha<-1$, then

\Centerline{$(\mu^{2k})^*D_k(A+B,E,\alpha,\beta)
\le(\mu E)^k\cdot(\mu^k)^*\tilde D(B,E,-1-\alpha)<(\mu E)^{2k}$}

\noindent for some $k$.

On the other hand, if $-1\le\alpha<\beta\le 1$, there is a $k\ge 1$ such
that $(\mu^{2k})^*D_k(A+B',E,\alpha,\beta)<(\mu E)^{2k}$.   If now
$w\in D_k(A+B,E,\alpha,\beta)$, take $f\in A$ and $g\in B$ such that
$f(w(2i))+g(w(2i))\le\alpha$ and $f(w(2i+1))+g(w(2i+1))\ge\beta$ for $i<k$.
In this case, for each $i<k$,

\inset{----- either $g'(w(2i))\le g(w(2i))$ and
$f(w(2i))+g'(w(2i))\le\alpha$,
or $g'(w(2i))=-2$ and $f(w(2i))+g'(w(2i))\le-1\le\alpha$,

----- either $g'(w(2i+1))\ge g(w(2i+1))$ and
$f(w(2i+1))+g'(w(2i+1))\ge\beta$,
or $g'(w(2i+1))=2$ and $f(w(2i+1))+g'(w(2i+1))\ge 1\ge\beta$.}

\noindent So $w\in D_k(A+B',E,\alpha,\beta)$.   Accordingly
$(\mu^{2k})^*D_k(A+B,E,\alpha,\beta)<(\mu E)^{2k}$.

As $E$, $\alpha$ and $\beta$ are arbitrary, $A+B$ is stable.\ \Qed

\medskip

\quad{\bf (v)} Now suppose that $|f|\leae\chi X$ for every $f\in A\cup B$.
For $f\in A\cup B$ and $x\in X$, set $f_0(x)=\med(-1,f(x),1)$,
$f_1(x)=\max(0,f(x)-1)$ and $f_2(x)=\max(0,-1-f(x))$; then $f=f_0+f_1-f_2$,
$|f_0|\le\chi X$ and $f_1$, $f_2$ are zero a.e.   Also
$A_0=\{f_0:f\in A\}$, $A_1=\{f_1:f\in A\}$, $A_2=\{f_2:f\in A\}$,
$B_0=\{f_0:f\in B\}$, $B_1=\{f_1:f\in B\}$ and $B_2=\{f_2:f\in B\}$ are all
stable, by 465Ck.   Accordingly $A_0+B_0$ is stable, by (ii);
by (iii), $A_1-A_2+B_1-B_2$ is stable;  by (iv),

\Centerline{$A+B\subseteq A_0+B_0+A_1-A_2+B_1-B_2$}

\noindent is stable.

\medskip

\quad{\bf (vi)} Finally, turn to the hypothesis stated in the proposition:
that $A$ and $B$ are stable and pointwise bounded.   Let
$h:X\to\coint{0,\infty}$ be such that $|f(x)|\le h(x)$ for every
$f\in A\cup B$ and $x\in X$;  note that I do {\it not} assume here that $h$
is measurable.   However, we are supposing that $\mu$ is totally finite, so
there must be a sequence $\sequencen{f_n}$ in $A\cup B$ such that
$|f|\leae\sup_{n\in\Bbb N}|f_n|$ for every $f\in A\cup B$.   \Prf\ For each
$q\in\Bbb Q$, choose a countable set $C_q\subseteq A\cup B$ such that
$\{x:|f(x)|\ge q\}\setminus\bigcup_{g\in C_q}\{x:|g(x)|\ge q\}$ is
negligible for any $f\in A\cup B$
(215B(iv));  let $\sequencen{f_n}$ run over
$\bigcup_{q\in\Bbb Q}C_q$.\ \QeD\   Set
$h_1=\chi X+\sup_{n\in\Bbb N}|f_n|$;
then $h_1$ is finite-valued, strictly positive and measurable, and
$|f|\leae h_1$ for every $f\in A\cup B$.   By 465Ch, $A_1=\{f/h_1:f\in A\}$
and $B_1=\{f/h_1:f\in B\}$ are stable;  by (v) here, $A_1+A_2$ is stable;
by 465Ch again, $A+B=\{g\times h_1:g\in A_1+A_2\}$ is stable.   So we're
done.

\medskip

{\bf (d)} Since $A$ is pointwise
bounded, the closure $\overline{\Gamma(A)}$ of $\Gamma(A)$ in $\BbbR^X$
for the topology
of pointwise convergence is compact.   \Quer\ Suppose, if possible, that
there is a $g\in\overline{\Gamma(A)}\setminus\eusm L^0(\Sigma)$.
Then there must
be a measurable set $E$ of finite measure and $\alpha<\beta$ in $\Bbb R$
such that $\mu^*P=\mu^*Q=\mu E>0$, where

\Centerline{$P=\{x:x\in E$, $g(x)\le\alpha\}$,
\quad$Q=\{x:x\in E$, $g(x)\ge\beta\}$}

\noindent (see part (a) of the proof of 465D).   Set
$Y_n=\{x:x\in E$, $|f(x)|\le n$ for every $f\in A\}$;  then
$\sequencen{Y_n}$ is a non-decreasing sequence with union $E$, so there is
an $n\in\Bbb N$ such that $\mu^*(P\cap Y_n)$ and $\mu^*(Q\cap Y_n)$ are
both at least $\bover23\mu E$.   Let $F'$, $F''$ be measurable envelopes of
$P\cap Y_n$ and $Q\cap Y_n$ respectively, and $Y=F'\cap F''\cap Y_n$;  then

$$\eqalign{0
&<\mu(F'\cap F'')
=\mu^*(F'\cap F''\cap P\cap Y_n)\cr
&=\mu^*(F'\cap F''\cap Q\cap Y_n)
=\mu^*Y
=\mu^*(P\cap Y)
=\mu^*(Q\cap Y).\cr}$$

Let $\mu_Y$ be the subspace measure on $Y$ and $\Sigma_Y$ its domain,
and consider the set
$A_Y=\{f\restr Y:f\in A\}$.   With respect to the measure $\mu_Y$,
this is stable (465Cm).   Also it is uniformly bounded.   So $\Gamma(A_Y)$
is $\mu_Y$-stable, by (a) of this theorem.
As $\mu_Y$ is complete and totally finite, the closure
$\overline{\Gamma(A_Y)}$ for the topology of pointwise convergence in
$\BbbR^Y$ is included in $\eusm L^0(\Sigma_Y)$ (465Da).   Since
$f\mapsto f\restr Y:\BbbR^X\to\BbbR^Y$ is linear and continuous for the
topologies of pointwise convergence, $g\restr Y\in\overline{\Gamma(A_Y)}$
and $g\restr Y$ is $\Sigma_Y$-measurable.   But

\Centerline{$\mu_Y^*(P\cap Y)=\mu^*(P\cap Y)=\mu^*(Q\cap Y)
=\mu_Y^*(Q\cap Y)=\mu_YY\in\ooint{0,\infty}$}

\noindent (214Cd), so this is impossible.\ \Bang

Thus $\overline{\Gamma(A)}\subseteq\eusm L^0(\Sigma)$ and $\Gamma(A)$ is
relatively compact.
}%end of proof of 465N

\leader{465O}{Stable sets in $L^0$}\cmmnt{ The notion of `stability'
as defined in 465B is applicable only to true functions;  in such
examples as 465Xl, the irregularity of the set $A$ is erased entirely if
we look at its image in the space $L^0$ of equivalence classes of
measurable functions.   We do, however, have a corresponding concept for
subsets of function spaces, which can be expressed in the language of
\S325.}
If $(\frak A,\bar\mu)$ is a semi-finite measure algebra, and $k\ge 1$, I
write $(\Tensorhat_k\frak A,\bar\mu^k)$ for the localizable measure
algebra free product of $k$ copies of $(\frak A,\bar\mu)$\cmmnt{, as
described in 325H}.   If $Q\subseteq L^0(\frak A)$, $k\ge 1$,
$a\in\frak A$ has finite measure and $\alpha<\beta$ in $\Bbb R$, set

$$\eqalign{d_k(Q,a,\alpha,\beta)
=\sup_{v\in Q}\bigl((a\Bcap\Bvalue{v\le\alpha})
&\otimes(a\Bcap\Bvalue{v\ge\beta})
\otimes\ldots\cr
&\otimes(a\Bcap\Bvalue{v\le\alpha})\otimes(a\Bcap\Bvalue{v\ge\beta})
\bigr)\cr}
$$

\noindent in $\Tensorhat_{2k}\frak A$, taking $k$ repetitions of the
formula $(a\Bcap\Bvalue{v\le\alpha})
\otimes(a\Bcap\Bvalue{v\ge\beta})$\dvro{.}{ to match the corresponding
formula

\Centerline{$D_k(A,E,\alpha,\beta)
=\bigcup_{f\in A}
  ((E\cap\{x:f(x)\le\alpha\})\times(E\cap\{x:f(x)\ge\beta))^k$.}

\noindent (Note that the supremum $\sup_{v\in Q}\ldots$ is defined
because $a^{\otimes 2k}=a\otimes\ldots\otimes a$ has finite measure in
the measure
algebra $(\Tensorhat_{2k}\frak A,\bar\mu^{2k})$.   Of course I mean to
take $d_k(Q,a,\alpha,\beta)=0$ if $Q=\emptyset$.)
Now we can say that} $Q$ is {\bf stable} if whenever
$0<\bar\mu a<\infty$ and $\alpha<\beta$
there is a $k\ge 1$ such that
$\bar\mu^{2k}d_k(Q,a,\alpha,\beta)<(\bar\mu a)^{2k}$\cmmnt{;  that is,
$d_k(Q,a,\alpha,\beta)\ne a\otimes\ldots\otimes a$}.

\cmmnt{We have the following relationships between the two concepts of
stability.}

\vleader{48pt}{465P}{Theorem} Let $(X,\Sigma,\mu)$ be a semi-finite measure
space, with measure algebra $(\frak A,\bar\mu)$.

(a) Suppose that $A\subseteq\eusm L^0(\Sigma)$ and that
$Q=\{f^{\ssbullet}:f\in A\}\subseteq L^0(\mu)$, identified with
$L^0=L^0(\frak A)$\cmmnt{ (364Ic\formerly{3{}64Jc})}.
Then $Q$ is stable\cmmnt{ in the sense of 465O} iff every countable
subset of $A$ is stable\cmmnt{ in the sense of 465B}.

(b) If $\mu$ is strictly localizable and $Q\subseteq L^0(\mu)$ is
stable, then there is a stable set $B\subseteq\eusm L^0$ such that
$Q=\{f^{\ssbullet}:f\in B\}$.

\proof{{\bf (a)(i)} Suppose that all countable subsets of $A$ are
stable, and take $a\in\frak A$ such that $0<\bar\mu a<\infty$ and
$\alpha<\beta$ in $\Bbb R$.   For each $k\in\Bbb N$ there is a countable
set $Q_k\subseteq Q$ such that
$d_k(Q_k,a,\alpha,\beta)=d_k(Q,a,\alpha,\beta)$, because
$a^{\otimes 2k}$ has finite measure in $\Tensorhat_{2k}\frak A$.
Now there is a
countable set $A'\subseteq A$ such that  $\{f^{\ssbullet}:f\in
A'\}=\bigcup_{k\in\Bbb N}Q_k$.   Let $E\in\Sigma$ be such that
$E^{\ssbullet}=a$.   As $\mu E=\bar\mu a\in\ooint{0,\infty}$ and $A'$ is
stable, there is a $k\ge 1$ such that
$(\mu^{2k})^*D_k(A',E,\alpha,\beta)<(\mu E)^{2k}$.   Because $A'$ is
countable, $D_k(A',E,\alpha,\beta)$ is measurable.   But 325He tells us
that we have an order-continuous
measure-preserving Boolean homomorphism $\pi$ from the measure algebra
of $\mu^{2k}$ to $\Tensorhat_{2k}\frak A$, such that
$\pi(\prod_{i<2k}F_i)^{\ssbullet}
=F_0^{\ssbullet}\otimes\ldots\otimes F_{2k-1}^{\ssbullet}$ for all
$F_0,\ldots,F_{2k-1}\in\Sigma$;  accordingly

$$\eqalign{\bar\mu^{2k}d_k(Q,a,\alpha,\beta)
&=\bar\mu^{2k}d_k(Q_k,a,\alpha,\beta)
\le\bar\mu^{2k}d_k(\bigcup_{i\in\Bbb N}Q_i,a,\alpha,\beta)\cr
&=\bar\mu^{2k}(\pi D_k(A',E,\alpha,\beta)^{\ssbullet})
=\mu^{2k}D_k(A',E,\alpha,\beta)\cr
&<(\mu E)^{2k}
=\bar\mu^{2k}a^{\otimes 2k}.\cr}$$

\noindent As $a$, $\alpha$ and $\beta$ are arbitrary, $Q$ is stable.

\medskip

\quad{\bf (ii)} Now suppose that $Q$ is stable and that $A'$ is any
countable subset of $A$.   Take $E\in\Sigma$ such that $0<\mu E<\infty$,
and $\alpha<\beta$ in $\Bbb R$.   Set $a=E^{\ssbullet}\in\frak A$.
This time, writing $Q'$ for $\{f^{\ssbullet}:f\in A'\}$, we have

\Centerline{$\pi D_k(A',E,\alpha,\beta)^{\ssbullet}
=d_k(Q',a,\alpha,\beta)
\Bsubseteq d_k(Q,a,\alpha,\beta)$}

\noindent for every $k\ge 1$.   There is some $k$ such that
$d_k(Q,a,\alpha,\beta)\ne a^{\otimes 2k}$, and in this case
$\mu^{2k}D_k(A',E,\alpha,\beta)<(\mu E)^{2k}$;  as $E$, $\alpha$ and
$\beta$ are arbitrary, $A'$ is stable.

\medskip

{\bf (b)(i)} If $\mu X=0$ this is trivial;  suppose that $\mu X>0$.
Replacing $\mu$ by its completion does not change either $L^0(\mu)$ or
the stable subsets of $\Bbb R^X$ (241Xb, 465Ci), and leaves $\mu$
strictly localizable (212Gb), so we may suppose that $\mu$ is complete.
Let $\familyiI{E_i}$ be a decomposition of $X$ into sets of finite
measure.   Amalgamating any negligible $E_i$ into other non-negligible
ones, we may suppose that $\mu E_i>0$ for each $i$.   Writing $\mu_i$
for the subspace measure on $E_i$, we have a consistent lifting $\phi_i$
for $\mu_i$ (346J).   Set $\phi E=\bigcup_{i\in I}\phi_i(E\cap X_i)$
for $E\in\Sigma$;  then $\phi$ is a lifting for $\mu$.   Let $\theta$ be
the corresponding lifting from $\frak A$ to $\Sigma$ (341B) and
$T:L^{\infty}(\frak A)\to\eusm L^{\infty}(\Sigma)$ the associated
linear operator, defined by saying that $T(\chi a)=\chi(\theta a)$ for
every $a\in\frak A$ (363F).   Since $\theta(a)^{\ssbullet}=a$ for every
$a\in\frak A$, $(Tv)^{\ssbullet}=v$ for every $v\in L^{\infty}$.

\medskip

\quad{\bf (ii)} We need to know that if $v\in L^{\infty}$ and
$\alpha<\alpha'$, then
$\{x:(Tv)(x)\le\alpha\}\subseteq\phi(\{x:(Tv)(x)\le\alpha'\})$.   \Prf\
Let $v'\in S(\frak A)$ be such that
$\|v-v'\|_{\infty}\le\bover12(\alpha'-\alpha)$ (363C), and set
$\gamma=\bover12(\alpha+\alpha')$.  Express $v'$ as
$\sum_{i=0}^n\alpha_i\chi a_i$ where $a_0,\ldots,a_n\in\frak A$ are
disjoint.   Then $Tv'=\sum_{i=0}^n\alpha_i\chi(\theta a_i)$.   Now

\Centerline{$\|Tv-Tv'\|_{\infty}\le\|v-v'\|_{\infty}\le\gamma-\alpha
=\alpha'-\gamma$,}

\noindent so

$$\eqalignno{\{x:(Tv)(x)\le\alpha\}
&\subseteq\{x:(Tv')(x)\le\gamma\}\cr
&=\bigcup\{\theta a_i:i\le n,\,\alpha_i\le\gamma\}
=\phi(\bigcup\{\theta a_i:i\le n,\,\alpha_i\le\gamma\})\cr
\displaycause{because $\phi(\theta a)=\theta a$ for every $a\in\frak A$}
&=\phi(\{x:(Tv')(x)\le\gamma\})
\subseteq\phi(\{x:(Tv)(x)\le\alpha'\}),\cr}$$

\noindent as claimed.\ \Qed

Similarly, if $\beta'<\beta$ then
$\{x:(Tv)(x)\ge\beta\}\subseteq\phi(\{x:(Tv)(x)\ge\beta'\})$.

\medskip

\quad{\bf (iii)} For the moment, suppose that
$Q\subseteq L^{\infty}(\frak A)$, which we may identify with
$L^{\infty}(\mu)$
(363I).   Set $B=T[Q]$, so that $Q=\{f^{\ssbullet}:f\in B\}$.   Then $B$
is stable.   \Prf\ Let $E\in\Sigma$ be such that $0<\mu E<\infty$, and
$\alpha<\beta$.   Let $i\in I$ be such that $\mu(E\cap E_i)>0$, and
$\alpha'$, $\beta'\in\Bbb R$ such that $\alpha<\alpha'<\beta'<\beta$.
Setting $a=(E\cap E_i)^{\ssbullet}$, we have $0<\bar\mu a<\infty$, so
there is some $k\in\Bbb N$ such that
$d_k(Q,a,\alpha',\beta')\ne a^{\otimes 2k}$.
Let $\pi$ be the measure-preserving Boolean
homomorphism from the measure algebra of $\mu^{2k}$ to
$\Tensorhat_{2k}\frak A$ described in part (a) of this proof;  as noted
in 325He, the present context is enough to ensure that $\pi$ is an
isomorphism.   So there is a $W\in\dom\mu^{2k}$ such that
$\pi W^{\ssbullet}=d_k(Q,a,\alpha',\beta')$;  since

\Centerline{$d_k(Q,a,\alpha',\beta')\Bsubseteq a^{\otimes 2k}
\Bsubseteq(E_i^{2k})^{\ssbullet}$,}

\noindent we may suppose that $W\subseteq E_i^{2k}$.   If $f\in B$, then

\Centerline{$\pi D_k(\{f\},E,\alpha',\beta')^{\ssbullet}
=d_k(\{f^{\ssbullet}\},a,\alpha',\beta')
\Bsubseteq d_k(Q,a,\alpha',\beta')$,}

\noindent so $D_k(\{f\},E,\alpha',\beta')\setminus W$ is negligible.

At this point, recall that $\phi_i$ was supposed to be a consistent
lifting for $\mu_i$.   So we have a lifting $\phi'$ of $\mu_i^{2k}$ such
that $\phi'(\prod_{j<2k}F_j)=\prod_{j<2k}\phi_iF_j$ for all
$F_0,\ldots,F_{2k-1}\in\Sigma_i$.   In particular, if $f\in B$ and we
set

\Centerline{$F_{2j}=\{x:x\in E\cap E_i,\,f(x)\le\alpha\}$,
\quad$F_{2j+1}=\{x:x\in E\cap E_i,\,f(x)\ge\beta\}$,}

\Centerline{$F'_{2j}=\{x:x\in E\cap E_i,\,f(x)\le\alpha'\}$,
\quad$F'_{2j+1}=\{x:x\in E\cap E_i,\,f(x)\ge\beta'\}$,}

\noindent for $j<k$, we shall have

$$\eqalignno{\prod_{j<2k}F_j
&\subseteq\prod_{j<2k}\phi F'_j\cr
\displaycause{by (ii) above, because $f=Tv$ for some $v$}
&=\prod_{j<2k}\phi_iF'_j
=\phi'(\prod_{j<2k}F'_j)
\subseteq\phi'W\cr}$$

\noindent because $\prod_{j<2k}F'_j=D_k(\{f\},E,\alpha',\beta')$.
As $f$ is arbitrary, $D_k(B,E\cap E_i,\alpha,\beta)\subseteq\phi'W$.
But now

$$\eqalignno{(\mu^{2k})^*D_k(B,E\cap E_i,\alpha,\beta)
&\le\mu^{2k}(\phi'W)
=\mu_i^{2k}(\phi'W)\cr
\displaycause{251Wl}
&=\mu_i^{2k}W
=\mu^{2k}W
=\bar\mu^{2k}d_k(Q,a,\alpha',\beta')\cr
&<\bar\mu^{2k}(a^{\otimes 2k})
=\mu(E\cap E_i)^{2k}.\cr}$$

\noindent As usual, it follows that
$(\mu^{2k})^*D_k(B,E,\alpha,\beta)<(\mu E)^{2k}$;  as $E$, $\alpha$ and
$\beta$ are arbitrary, $B$ is stable, as claimed.\ \Qed

\medskip

\quad{\bf (iv)} Thus the result is true if $Q$ is included in the unit
ball of $L^{\infty}$.   In general, set $h(\alpha)=\tanh\alpha$ for
$\alpha\in\Bbb R$, and consider

\Centerline{$A=\{f:f\in\eusm L^0(\Sigma)$, $f^{\ssbullet}\in Q\}$,
\quad$A'=\{hf:f\in A\}$,
\quad$Q'=\{(hf)^{\ssbullet}:f\in A\}$.}

\noindent By (a), every countable subset of $A$
is stable, so every countable subset of $A'$ is stable (465Ck) and $Q'$ is
stable.   But $Q'$ is included in the unit ball
of $L^{\infty}$, so there is a stable set $B'\subseteq\eusm L^0$ such
that $Q'=\{g^{\ssbullet}:g\in B'\}$.   Setting $B=\{h^{-1}g:g\in B'\}$,
$B$ is stable and $\{f^{\ssbullet}:f\in B\}=Q$.   So we have the general
theorem.
}%end of proof of 465P

\cmmnt{
\leader{465Q}{Remarks} Using 465Pa, we can work through the
first part
of this section to get a list of properties of stable subsets of $L^0$.
For instance,
the convex hull of an order-bounded stable set in $L^0$ is stable, as in
465Na.   It is harder to relate such results as 465M to the idea of
stability in $L^0$, but the argument of 465Pb gives a line to follow:
if $(X,\Sigma,\mu)$ is complete and strictly localizable, there is a
linear operator $T:L^{\infty}(\mu)\to\eusm L^{\infty}(\Sigma)$, defined
from a lifting, such that, for $Q\subseteq L^{\infty}$, $T[Q]$ is stable
iff $Q$ is stable.   So when $\mu$ is a complete probability measure, we
can look at the averages
$\psi_{wk}(v)=\bover1k\sum_{i=0}^{k-1}(Tv)(w(i))$ for $v\in Q$,
$w\in X^{\Bbb N}$ to devise criteria for stability of $Q$ in terms of
the linear functionals $\psi_{wk}$.

Working in $L^1$, however, we can look for results of a different type,
as follows.
}%end of comment

\leader{465R}{Theorem}\cmmnt{ ({\smc Talagrand 84})} % 11-4-1
Let $(\frak A,\bar\mu)$ and $(\frak B,\bar\nu)$ be measure algebras, and
$T:L^1(\frak A,\bar\mu)\to L^1(\frak B,\bar\nu)$ a bounded linear
operator.   If $Q$ is stable and order-bounded in
$L^1(\frak A,\bar\mu)$, then $T[Q]\subseteq L^1(\frak B,\bar\nu)$
is stable.

\proof{{\bf (a)} To begin with (down to (d) below) let us
suppose that

\inset{$(\frak A,\bar\mu)$ and
$(\frak B,\bar\nu)$ are the measure algebras of measure spaces
$(X,\Sigma,\mu)$ and $(Y,\Tau,\nu)$,}

\noindent so that we can identify $L^1_{\bar\mu}$,
$L^1_{\bar\nu}$ with $L^1(\mu)$ and $L^1(\nu)$ (365B),

\inset{$Q$ is countable,}

\noindent so that $Q$ can be expressed as $\{f^{\ssbullet}:f\in A\}$,
where $A\subseteq\eusm L^0(\Sigma)$ is countable and stable (465P),

\inset{$T$ is positive,

$\mu$ and $\nu$ are totally finite,

$T(\chi X^{\ssbullet})=\chi Y^{\ssbullet}$,

$Q\subseteq L^{\infty}(\frak A)$ is $\|\,\|_{\infty}$-bounded,}

\noindent so that we may take $A\subseteq\eusm L^{\infty}$ to be
$\|\,\|_{\infty}$-bounded,

\inset{$\nu Y=1$,

$\mu X=1$,

and that $\|T\|\le 1$.}

\medskip

{\bf (b)} The idea of the argument is that for any $n\ge 1$ we have a
positive linear
operator $U_n:L^1(\mu^n)\to L^1(\nu^n)$ defined as follows.

If $f_0,\ldots,f_{n-1}\in\eusm L^1(\mu)$, set
$(f_0\otimes\ldots\otimes f_{n-1})(w)=\prod_{i=0}^nf_i(w(i))$ whenever
$w\in\prod_{i<n}\dom f_i$.
Now we can define
$u_0\otimes u_1\otimes\ldots\otimes u_{n-1}\in L^1(\mu^n)$, for
$u_0,\ldots,u_{n-1}\in L^1(\mu)$, by saying that
$f_0^{\ssbullet}\otimes\ldots\otimes f_{n-1}^{\ssbullet}
=(f_0\otimes\ldots\otimes f_{n-1})^{\ssbullet}$ for all
$f_0,\ldots,f_{n-1}\in\eusm L^1(\mu)$, as in 253E.

Define the operators $U_n$ inductively.   $U_1=T$.   Given that
$U_n:L^1(\mu^n)\to
L^1(\nu^n)$ is a positive linear operator, then we have a bilinear
operator $\psi:L^1(\mu^n)\times L^1(\mu)\to L^1(\nu^{n+1})$ defined by
saying that $\psi(q,u)=U_nq\otimes Tu$ for $q\in L^1(\mu^n)$, $u\in
L^1(\mu)$, where
$\otimes:L^1(\nu^n)\times L^1(\nu)
\to L^1(\nu^n\times\nu)\cong L^1(\nu^{n+1})$ is the operator of 253E.   By
253F, there is a (unique) bounded linear operator
$U_{n+1}:L^1(\mu^{n+1})\to L^1(\nu^{n+1})$ such that
$U_{n+1}(q\otimes u)=\psi(q,u)$ for all $q\in L^1(\mu^n)$,
$u\in L^1(\mu)$.   To see that
$U_{n+1}$ is positive, use 253Gc.   (Remember that we are
supposing that $T$ is positive.)   Continue.

Now it is easy to check that

\Centerline{$U_n(u_0\otimes\ldots\otimes u_{n-1})
=Tu_0\otimes\ldots\otimes Tu_{n-1}$}

\noindent for all $u_0,\ldots,u_{n-1}\in L^1(\mu)$.   Moreover,
$\|U_{n+1}\|\le\|U_n\|\|T\|$ for every $n$ (see 253F), so $\|U_n\|\le 1$
for every $n$.

For $i<n\in\Bbb N$, we have a natural operator $R_{ni}:L^1(\mu)\to
L^1(\mu^n)$, defined by saying that
$R_{ni}f^{\ssbullet}=(f\pi_{ni})^{\ssbullet}$ for every
$f\in\eusm L^1(\mu)$, where $\pi_{ni}(w)=w(i)$ for $w\in X^n$.
Similarly, we have
an operator $S_{ni}:L^1(\nu)\to L^1(\nu^n)$.   Observe that

\Centerline{$R_{ni}u
=e\otimes\ldots\otimes e\otimes u\otimes e\otimes\ldots\otimes e$}

\noindent where $e=\chi X^{\ssbullet}$ and the $u$ is put in the
position corresponding to the coordinate $i$.   Since
$Te=(\chi Y)^{\ssbullet}=e'$ say,

\Centerline{$U_nR_{ni}u
=e'\otimes\ldots\otimes Tu\otimes\ldots\otimes e'=S_{ni}Tu$}

\noindent for every $u\in L^1(\mu)$.

\medskip

{\bf (c)} Let $B\subseteq\eusm L^{\infty}(\Tau)$ be a countable
$\|\,\|_{\infty}$-bounded set such that $T[Q]=\{g^{\ssbullet}:g\in B\}$.
($T[Q]$ is $\|\,\|_{\infty}$-bounded because $T$ is positive and
$T(\chi X)^{\ssbullet}=(\chi Y)^{\ssbullet}$.)   I seek to show that $B$
is stable by using the criterion 465M(v).   Let $\epsilon>0$.   Then
there is an $m\ge 1$ such that
$\int f_{kl}(w)\mu^{\Bbb N}(dw)\le\epsilon$ for any $k$, $l\ge m$,
writing

\Centerline{$f_{kl}(w)=\sup_{f\in A}|\Bover1k\sum_{i=0}^{k-1}f(w(i))
  -\Bover1l\sum_{i=0}^{l-1}f(w(i))|$}

\noindent for $w\in X^{\Bbb N}$;  note that $f_{kl}$ is measurable
because $A$ is countable.

Take any $k$, $l\ge m$ and consider

\Centerline{$g_{kl}(z)=\sup_{g\in B}|\Bover1k\sum_{i=0}^{k-1}g(z(i))
  -\Bover1l\sum_{i=0}^{l-1}g(z(i))|$}

\noindent for $z\in Y^{\Bbb N}$.   I claim that
$\int g_{kl}d\nu^{\Bbb N}\le\epsilon$.   \Prf\ Set $n=\max(k,l)$.   Then
$\int g_{kl}d\nu^{\Bbb N}=\int\tilde g\,d\nu^n$, where

\Centerline{$\tilde g(z)=\sup_{g\in B}|\Bover1k\sum_{i=0}^{k-1}g(z(i))
  -\Bover1l\sum_{i=0}^{l-1}g(z(i))|$}

\noindent for $z\in Y^n$.   If we look at $\tilde g^{\ssbullet}$ in
$L^1(\nu^n)$, we see that it is

\Centerline{$\sup_{g\in B}|\Bover1k\sum_{i=0}^{k-1}S_{ni}g^{\ssbullet}
  -\Bover1l\sum_{i=0}^{l-1}S_{ni}g^{\ssbullet}|$}

\noindent where $S_{ni}:L^1(\nu)\to L^1(\nu^n)$ is defined in (b) above.
Thus

\Centerline{$\tilde g^{\ssbullet}
=\sup_{v\in T[Q]}|\Bover1k\sum_{i=0}^{k-1}S_{ni}v
  -\Bover1l\sum_{i=0}^{l-1}S_{ni}v|$.}

\noindent Similarly, setting

\Centerline{$\tilde f(w)=\sup_{f\in A}|\Bover1k\sum_{i=0}^{k-1}f(w(i))
  -\Bover1l\sum_{i=0}^{l-1}f(w(i))|$}

\noindent for $w\in X^n$,

\Centerline{$\tilde f^{\ssbullet}
=\sup_{u\in Q}|\Bover1k\sum_{i=0}^{k-1}R_{ni}u
  -\Bover1l\sum_{i=0}^{l-1}R_{ni}u|$.}

\noindent Now consider $U_n\tilde f^{\ssbullet}$.   For any $v\in T[Q]$,
we can express $v$ as $Tu$ where $u\in Q$, so

$$\eqalignno{\bigl|\Bover1k\sum_{i=0}^{k-1}S_{ni}v
  -\Bover1l\sum_{i=0}^{l-1}S_{ni}v\bigr|
&=\bigl|\Bover1k\sum_{i=0}^{k-1}S_{ni}Tu
  -\Bover1l\sum_{i=0}^{l-1}S_{ni}Tu\bigr|\cr
&=\bigl|\Bover1k\sum_{i=0}^{k-1}U_nR_{ni}u
  -\Bover1l\sum_{i=0}^{l-1}U_nR_{ni}u\bigr|\cr
\displaycause{because $U_nR_{ni}=S_{ni}T$, as noted in (b) above}
&=\bigl|U_n(\Bover1k\sum_{i=0}^{k-1}R_{ni}u
  -\Bover1l\sum_{i=0}^{l-1}R_{ni}u)\bigr|\cr
&\le U_n(|\Bover1k\sum_{i=0}^{k-1}R_{ni}u
  -\Bover1l\sum_{i=0}^{l-1}R_{ni}u|)\cr
\displaycause{because $U_n$ is positive}
&\le U_n\tilde f^{\ssbullet}.\cr}$$

\noindent As $v$ is arbitrary, $\tilde g^{\ssbullet}\le U_n\tilde
f^{\ssbullet}$, and

$$\eqalignno{\int g_{kl}d\nu^{\Bbb N}
&=\int\tilde g\,d\nu^n
=\|\tilde g^{\ssbullet}\|_1
\le\|U_n\tilde f^{\ssbullet}\|_1
\le\|\tilde f^{\ssbullet}\|_1\cr
\displaycause{because $\|U_n\|\le 1$}
&\le\epsilon\cr}$$

\noindent because $k$, $l\ge m$.\ \Qed

\medskip

{\bf (d)} As $\epsilon$ is arbitrary, $B$ satisfies the criterion
465M(v), and is stable.   So $T[Q]$ is stable, by 465Pa in the other
direction.

\medskip

{\bf (e)} Now let us seek to unwind the list of special assumptions used
in the argument above.   Suppose we drop the last two, and assume only
that

\inset{$(\frak A,\bar\mu)$ and
$(\frak B,\bar\nu)$ are the measure algebras of measure spaces
$(X,\Sigma,\mu)$ and $(Y,\Tau,\nu)$,

$Q$ is countable,

$T$ is positive,

$\mu$ and $\nu$ are totally finite,

$T(\chi X^{\ssbullet})=\chi Y^{\ssbullet}$,

$Q\subseteq L^{\infty}(\frak A)$ is $\|\,\|_{\infty}$-bounded,

$\nu Y=1$.}

\noindent Then $T[Q]$ is stable.   \Prf\ Define
$\mu_1:\Sigma\to\coint{0,\infty}$ by setting
$\mu_1E=\int T(\chi E^{\ssbullet})$ for every $E\in\Sigma$.   Then
$\mu_1$ is countably
additive because $T$ is (sequentially) order-continuous (355Ka).   If
$\mu E=0$ then $\chi E^{\ssbullet}=0$ in $L^1(\mu)$ and $\mu_1E=0$, so
$\mu_1$ is truly continuous with respect to $\mu$ (232Bb) and has a
Radon-Nikod\'ym derivative (232E).   By 465Cj, $A$ is stable with
respect to $\mu_1$, while $\mu_1X=\nu Y=1$, because
$T(\chi X)^{\ssbullet}=\chi Y^{\ssbullet}$.

Let $(\frak A_1,\bar\mu_1)$ be the measure algebra of $\mu_1$.   If
$E\in\Sigma$ and $\mu_1E=0$, then $T(\chi E^{\ssbullet})=0$.
Accordingly we can define an additive function
$\theta:\frak A_1\to L^1(\nu)$ by setting
$\theta E^{\ssbullet}=T(\chi E^{\ssbullet})$ for
every $E\in\Sigma$.   (Note that the two $^{\ssbullet}$s here must be
interpreted differently.   In the formula $\theta E^{\ssbullet}$, the
equivalence class $E^{\ssbullet}$ is to be taken in $\frak A_1$.   In
the formula $\chi E^{\ssbullet}=(\chi E)^{\ssbullet}$, the equivalence
class is to be taken in $L^0(\mu)$.   In the rest of this proof I will
pass over such points without comment;  I hope the context will always
make it clear how each $^{\ssbullet}$ is to be read.)
Because $T$ is positive, $\theta$ is non-negative, and by the definition
of $\mu_1$ we have $\|\theta a\|_1=\bar\mu_1a$ for every
$a\in\frak A_1$.   So we have a positive linear operator
$T_1:L^1(\frak A_1,\bar\mu_1)\to L^1(\nu)$ defined by setting

\Centerline{$T_1(\chi E^{\ssbullet})=\theta E^{\ssbullet}
=T(\chi E^{\ssbullet})$}

\noindent for every $E\in\Sigma$ (365K).

If $f:X\to\Bbb R$ is simple (that is, a linear combination of
indicator functions of sets in $\Sigma$), then
$T_1f^{\ssbullet}=Tf^{\ssbullet}$.   So this is also true for every
$f\in\eusm L^{\infty}(\Sigma)$;  in particular, it is true for every
$f\in A$, so that $T[Q]=T_1[Q_1]$, where
$Q_1=\{f^{\ssbullet}:f\in A\}\subseteq L^1(\mu_1)$.   But $\mu_1$, $Q_1$
and $T_1$ satisfy all the
conditions of (a), so (b)-(d) tell us that $T_1[Q_1]$ is stable, and
$T[Q]$ is stable, as required.\ \Qed

\medskip

{\bf (f)} The next step is to drop the condition `$\nu Y=1$'.   But this
is elementary, since we are still assuming that $\nu$ is totally finite,
and multiplying $\nu$ by a non-zero scalar doesn't change $L^0(\nu)$ or
the stability of any of its subsets, while the case $\nu Y=0$ is
trivial.   So we conclude that if

\inset{$(\frak A,\bar\mu)$ and
$(\frak B,\bar\nu)$ are the measure algebras of measure spaces
$(X,\Sigma,\mu)$ and $(Y,\Tau,\nu)$,

$Q$ is countable,

$T$ is positive,

$\mu$ and $\nu$ are totally finite,

$T(\chi X^{\ssbullet})=\chi Y^{\ssbullet}$,

$Q\subseteq L^{\infty}(\frak A)$ is $\|\,\|_{\infty}$-bounded,}

\noindent then $T[Q]$ is stable.

\medskip

{\bf (g)} We can now attack what remains.   We find that if

\inset{$(\frak A,\bar\mu)$ and
$(\frak B,\bar\nu)$ are the measure algebras of measure spaces
$(X,\Sigma,\mu)$ and $(Y,\Tau,\nu)$,

$Q$ is countable,

$T$ is positive,}

\noindent then $T[Q]$ is stable.   \Prf\ At this point recall that we
are supposing that $Q$ is order-bounded.   Let $u_0\in L^1(\mu)$ be such
that $|u|\le u_0$ for every $u\in Q$;  let $f_0\in\eusm L^0(\Sigma)^+$
be such that $f_0^{\ssbullet}=u_0$.   Setting
$A'=\{(f\wedge f_0)\vee(-f_0):f\in A\}$, $A'$ is still stable, because
its image in $L^1(\mu)$ is still $Q$, or otherwise.
Set $v_0=Tu_0$, and let $g_0\in\eusm L^0(\Tau)^+$ be such that
$g_0^{\ssbullet}=v_0$.   Because $T$ is positive, $|Tu|\le T|u|\le v_0$
for every $u\in Q$.   So we can represent $T[Q]$ as
$\{g^{\ssbullet}:g\in B\}$, where $B\subseteq\eusm L^0(\Tau)$ is a
countable set and $|g|\le g_0$ for every $g\in B$.   Set
$F_0=\{y:g_0(y)\ne 0\}$.

Define measures $\mu_1$, $\nu_1$ by setting $\mu_1E=\int_Ef_0d\mu$ for
$E\in\Sigma$, $\nu_1F=\int_Fg_0d\nu$ for $F\in\Tau$.   Then both $\mu_1$
and $\nu_1$ are totally finite.   By 465Cj, $A'$ is stable with respect
to $\mu_1$.   Set $A_1=\{\Bover{f}{f_0}:f\in A'\}$, interpreting
$\Bover{f}{f_0}(x)$ as $0$ if $f_0(x)=0$;  then $A_1$ is stable with
respect to $\mu_1$, by 465Ch, and $\|f\|_{\infty}\le 1$ for every
$f\in A_1$.   Take $Q_1=\{f^{\ssbullet}:f\in A_1\}\subseteq L^1(\mu_1)$,
so that $Q_1$ is stable.

We have a norm-preserving positive linear operator
$R:L^1(\mu_1)\to L^1(\mu)$ defined by setting
$Rf^{\ssbullet}=(f\times f_0)^{\ssbullet}$
for every $f\in\eusm L^1(\mu_1)$ (use 235A).   Observe that $R[Q_1]=Q$
and $R(\chi X)^{\ssbullet}=u_0$.   Similarly, we have a norm-preserving
positive linear operator $S:L^1(\nu_1)\to L^1(\nu)$ defined by setting
$Sg^{\ssbullet}=(g\times g_0)^{\ssbullet}$ for $g\in\eusm L^1(\nu_1)$.
The set of values of $S$ is just

\Centerline{$\{g^{\ssbullet}:g\in\eusm L^1(\nu),\,g(y)=0$ whenever
$g_0(y)=0\}$,}

\noindent which is the band in $L^1(\nu)$ generated by $v_0$.   So

\Centerline{$\{u:u\in L^1(\mu_1),\,TR|u|\in S[L^1(\nu_1)]\}$}

\noindent is a band in $L^1(\nu_1)$ containing $\chi X^{\ssbullet}$, and
must be the whole of $L^1(\mu_1)$.   Thus we have a positive linear
operator $T_1=S^{-1}TR:L^1(\mu_1)\to L^1(\nu_1)$, and $T_1(\chi
X)^{\ssbullet}=\chi Y^{\ssbullet}$ in $L^1(\nu_1)$.

By (f), $T_1[Q_1]$ is stable in $L^1(\nu_1)$.   Observe that
$T_1[Q_1]=\{S^{-1}g^{\ssbullet}:g\in B\}=\{g^{\ssbullet}:g\in B_1\}$,
where $B_1=\{\Bover{g}{g_0}:g\in B\}$, interpreting $\Bover{g}{g_0}(y)$
as $0$ if $y\in Y\setminus F_0$.   Consequently $B_1$ and
$B=\{g\times g_0:g\in B_1\}$ are stable with respect to $\nu_1$.   By 465Cj, once
more, $B$ is stable with respect to $\nu_0$, where

\Centerline{$\nu_0F=\int_F\Bover{1}{g_0}d\nu_1=\nu(F\cap F_0)$}

\noindent for any $F\in\Tau$.   But because $g(y)=0$ whenever $g\in B$
and $y\in Y\setminus F_0$,

\Centerline{$(\nu^{2k})^*D_k(B,F,\alpha,\beta)
=(\nu^{2k})^*D_k(B,F\cap F_0,\alpha,\beta)
=(\nu_0^{2k})^*D_k(B,F,\alpha,\beta)$}

\noindent whenever $F\in\Tau$, $\alpha<\beta$ and $k\ge 1$;  so $B$ is
also stable with respect to $\nu$, and $Q=\{g^{\ssbullet}:g\in B\}$ is
stable in $L^1(\nu)$.\ \Qed

\medskip

{\bf (h)} The worst is over.   If we are not told that $T$ is positive,
we know that it is expressible as the difference of positive linear
operators $T_1$ and $T_2$ (371D);  now $T_1[Q]$ and $T_2[Q]$ will be
stable, by the work above, so $T[Q]\subseteq T_1[Q]-T_2[Q]$ is stable,
by 465Nc.   If we are not told that $Q$ is countable, we refer to 465P
to see that we need only check that countable subsets of $T[Q]$ are
stable, and these are images of countable subsets of $Q$.   Finally, the
identification of the abstract measure algebras $(\frak A,\bar\mu_1)$
and $(\frak B,\bar\nu_1)$ with the measure algebras of measure spaces
is Theorem 321J.
}%end of proof of 465R

\leader{*465S}{R-stable sets}\cmmnt{ The theory above has been
developed in the
context of general measure (or probability) spaces and the `ordinary'
product measure of measure spaces.   For $\tau$-additive measures -- in
particular, for Radon measures -- we have an alternative product
measure, as described in \S417.}   If $(X,\frak T,\Sigma,\mu)$ is a
semi-finite $\tau$-additive topological measure space such that $\mu$ is
inner regular with respect to the Borel sets, write $\tilde\mu^I$ for
the $\tau$-additive product measure on $X^I$\cmmnt{, as described in
417C (for
the product of two spaces) and 417E (for the product of any family of
probability spaces);  we can extend the construction of
417C to arbitrary finite products (417D)}.   \cmmnt{Now say that}
$A\subseteq\Bbb R^X$ is
{\bf R-stable} if whenever $0<\mu E<\infty$ and $\alpha<\beta$ there is
a $k\ge 1$ such that
$(\tilde\mu^{2k})^*D_k(A,E,\alpha,\beta)<(\mu E)^{2k}$.
\cmmnt{Because we have a version of Fubini's theorem for the products
of $\tau$-additive topological measures (417H), all the
arguments of this section can be applied to R-stable sets, yielding
criteria for R-stability exactly like those in 465M.}

%417Xf for product of sets of full outer measure

Because the $\tau$-additive product measure extends the c.l.d.\ product
measure, stable sets are always R-stable.   \cmmnt{(We must have

\Centerline{$(\tilde\mu^{2k})^*D_k(A,E,\alpha,\beta)
\le(\mu^{2k})^*D_k(A,E,\alpha,\beta)$}

\noindent for all $k$, $A$, $E$, $\alpha$ and $\beta$.)   For an example
of an R-stable set which is not stable, see 465U.}

\cmmnt{The concept of `R-stability' is used in {\smc Talagrand 84}
in applications to the integration of vector-valued
functions.   I give one result, however, to show how it is relevant to a
question with a natural expression in the language of this chapter.}

\leader{*465T}{Proposition}\cmmnt{ ({\smc Talagrand 84})} % 9-4-2
Let $(X,\frak T,\Sigma,\mu)$ be a semi-finite
$\tau$-additive topological measure space such that $\mu$ is inner
regular with respect to the Borel sets.   If $A\subseteq C(X)$ is such
that every countable subset of $A$ is R-stable, then $A$ is R-stable.

\proof{ For any $\alpha<\beta$ and $k\ge 1$,

$$\eqalign{D_k(A,X,\alpha,\beta)
=\bigcup_{f\in A}\{w:w\in
X^{2k},\,&f(w(2i))<\alpha,\cr
&f(w(2i+1))>\beta\text{ for }i<k\}\cr}$$

\noindent is open.   Suppose that $0<\mu E<\infty$.   Because all the
product measures $\tilde\mu^{2k}$ are $\tau$-additive, we can find a
countable set $A'\subseteq A$ such that

\Centerline{$\tilde\mu^{2k}D_k(A,E,\alpha,\beta)
=\tilde\mu^{2k}D_k(A',E,\alpha,\beta)$}

\noindent for every $k\ge 1$ and all rational $\alpha$, $\beta$.   Now,
if $\alpha<\beta$, there are rational $\alpha'$, $\beta'$ such that
$\alpha<\alpha'<\beta'<\beta$, and a $k\ge 1$ such that
$\tilde\mu^{2k}D_k(A',E,\alpha',\beta')<(\mu E)^{2k}$;  in which case

$$\eqalign{(\tilde\mu^{2k})^*D_k(A,E,\alpha,\beta)
&\le\tilde\mu^{2k}D_k(A,E,\alpha',\beta')
=\tilde\mu^{2k}D_k(A',E,\alpha',\beta')\cr
&\le\tilde\mu^{2k}D_k(A',E,\alpha',\beta')
<(\mu E)^{2k}.\cr}$$

\noindent As $E$, $\alpha$ and $\beta$ are arbitrary, $A$ is R-stable.
}%end of proof of 465T

\leader{*465U}{}\cmmnt{ I come now to the promised example of an
R-stable set which is not stable.   I follow the construction in
{\smc Talagrand 88}, which displays an interesting characteristic related
to 465O-465P above.

\medskip

\noindent}{\bf Example} There is a Radon probability space with an
R-stable set of continuous functions which is not stable.

\proof{{\bf (a)} Let $(X,\Sigma,\mu)$ be an atomless
probability space\cmmnt{ (e.g., the unit interval with Lebesgue
measure)}.   Define $\sequencen{r_n}$ by setting $r_0=1$, $r_1=2$,
$r_{n+1}=2^{nr_n}$ for $n\ge 1$;  then $2^n\le r_n<r_{n+1}$ for every
$n$.   Let $\sequencen{\Sigma_n}$ be an increasing sequence of finite
subalgebras of $\Sigma$, each $\Sigma_n$ having $r_n$ atoms of the same
size;  this
is possible because $r_{n+1}$ is always a multiple of $r_n$.   Write
$\Cal H_n$ for the set of atoms of $\Sigma_n$.   Next, for each
$n\in\Bbb N$, let $\family{H}{\Cal H_n}{G_H}$ be an independent family
in $\Sigma_{n+1}$ of sets of measure $2^{-n}$;  such a family exists
because $r_{n+1}$ is a multiple of $2^{nr_n}$.

Let $\Cal E$ be the family of all sets expressible in the form
$E=\bigcup_{i\in\Bbb N}H_i$ where, for some strictly increasing sequence
$\sequence{i}{n_i}$ in $\Bbb N$, $H_i\in\Cal H_{n_i}$ and
$H_j\subseteq G_{H_i}$ whenever $i<j$ in $\Bbb N$.   Set
$A=\{\chi E:E\in\Cal E\}\subseteq\eusm L^0(\Sigma)$.

\medskip

{\bf (b)} $A$ is stable.   \Prf\ Suppose that $F\in\Sigma$ and that
$\mu F>0$.   Take $n\in\Bbb N$ so large that
$3\cdot 2^{-n}<(\mu F)^2$.   Set
$\Cal H=\{H:H\in\Cal H_n,\,\mu(H\cap F)>0\}$;  enumerate $\Cal H$ as
$\ofamily{i}{m}{H_i}$;  set $F_1=F\cap\bigcup_{i<m}H_i$,
$V=\prod_{j<m}(H_j\cap F_1)\subseteq F_1^m$.   Because $\mu F_1=\mu F$,
$m\ge r_n\mu F\ge 3$.

Consider

\Centerline{$U_k=\bigcup_{H\in\Cal H_k}(H\times H)
\cup(H\times G_H)\cup(G_H\times H)\subseteq X^2$}

\noindent for $k\in\Bbb N$.   Then

\Centerline{$\mu^2U_k
\le r_k(\Bover1{r_k^2}+\Bover1{2^kr_k}+\Bover1{2^kr_k})
\le 3\cdot 2^{-k}$,}

\noindent so

\Centerline{$\mu^2(F_1^2\setminus\bigcup_{k>n}U_k)
\ge(\mu F)^2-3\sum_{k=n+1}^{\infty}2^{-k}>0$.}

\noindent Set

\Centerline{$V=\{w:w\in F_1^{2m},\,w(2i)\in H_i
\text{ for every }i<m,\,
(w(1),w(3))\notin\bigcup_{k>n}U_k\}$.}

\noindent Then

\Centerline{$\mu^{2m}(V)
\ge\prod_{i=0}^{m-1}\mu(F_1\cap H_i)
\cdot\mu^2(F_1^2\setminus\bigcup_{k>n}U_k)\cdot(\mu F)^{m-2}
>0$.}

\noindent\Quer\ Suppose, if possible, that there is a point
$w\in V\cap D_m(A,F,0,1)$.   Then there is an $E\in\Cal E$ such that
$w(2i)\notin E$, $w(2i+1)\in E$ for $i<m$.   Express $E$ as
$\bigcup_{j\in\Bbb N}E_j$
where $E_j\in\Cal H_{n_j}$ for every $j\in\Bbb N$, where
$\sequence{j}{n_j}$ is strictly increasing, and $E_k\subseteq G_{E_j}$
whenever $j<k$.   Because $w(2i)\in F_1\cap H_i\setminus E$, $E_j$
cannot include $H_i$ for any $i<m$, $j\in\Bbb N$;  so
$E_j\cap H_i=\emptyset$ whenever $i<m$ and $n_j\le n$.
Because $w(1)$, $w(3)\in E$, there are $j_0$, $j_1\in\Bbb N$ such that
$w(1)\in E_{j_0}$ and $w(3)\in E_{j_1}$.   Since both $w(1)$ and $w(3)$
belong to $F_1\subseteq\bigcup_{i<m}H_i$, $n_{j_0}$ and $n_{j_1}$ are
both greater than $n$.   But now

--- if $j_0=j_1$, then $(w(1),w(3))\in E_{j_0}^2$,

--- if $j_0<j_1$, then $(w(1),w(3))\in E_{j_0}\times E_{j_1}
\subseteq E_{j_0}\times G_{E_{j_0}}$,

--- if $j_1<j_0$, then $(w(1),w(3))\in E_{j_0}\times E_{j_1}
\subseteq G_{E_{j_1}}\times E_{j_1}$,

\noindent so in any case $(w(1),w(3))\in U_k$ where
$k=\min(n_{j_0},n_{j_1})>n$,
which is impossible.\ \Bang

Thus $D_m(A,F,0,1)$ does not meet $V$, and
$(\mu^{2m})^*D_m(A,F,0,1)<(\mu F)^{2m}$.

Now if $\alpha<\beta$, then
$D_m(A,F,\alpha,\beta)=D_m(A,F,0,1)$ if $0\le\alpha<\beta\le 1$,
$\emptyset$ otherwise;  so in all cases
$(\mu^{2m})^*D_m(A,F,\alpha,\beta)\penalty-100<(\mu F)^{2m}$.
As $F$ also is arbitrary, $A$ is stable, as claimed.\ \Qed

\medskip

{\bf (c)} Now let $(Z,\frak S,\Tau,\nu)$ be the Stone space of the
measure algebra of
$(X,\Sigma,\mu)$ (321K), so that $\nu$ is a Radon measure (411Pe).   For
$E\in\Sigma$, write $E^*$ for the
corresponding open-and-closed set in $Z$, so that $E\mapsto
E^*:\Sigma\to\Tau$ is a measure-preserving Boolean homomorphism.
Set $A^*=\{\chi E^*:E\in\Cal E\}\subseteq C(Z)$.   Write $\nu^m$ for the
c.l.d.\ product measure on $Z^m$ for $m\ge 1$.   We already know that
$\nu^2$ is not a topological measure (419E, 419Xc).

\medskip

{\bf (d)} The point is that $(\nu^{2m})^*D_m(A^*,Z,0,1)=1$ for every
$m\ge 1$.   \Prf\Quer\ Suppose, if possible, otherwise.   Then there is
a set $\tilde W\subseteq Z^{2m}$ such that $\nu^{2m}(\tilde W)>0$ and
$\tilde W\cap D_m(A^*,Z,0,1)\ne\emptyset$.   There
is an $\epsilon>0$ such that

\Centerline{$\tilde V=\{v:v\in Z^m,\,
\nu^m\{u:u\in Z^m,\,u\#v\in\tilde W\}$ is
defined and greater than $m\epsilon\}$}

\noindent has non-zero inner measure for $\nu^m$.   Now there are sets
$\tilde F_{ij}\in\Tau$, for $i\in\Bbb N$ and $j<m$, such that

\Centerline{$Z^m\setminus\tilde V
\subseteq\bigcup_{i\in\Bbb N}\prod_{j<m}\tilde F_{ij}$,
\quad$\sum_{i=0}^{\infty}\prod_{j=0}^m\nu\tilde F_{ij}<1$.}

\noindent Enlarging the $\tilde F_{ij}$ slightly if need be, we may
suppose that they are all open-and-closed (322Rc), therefore expressible
as $F_{ij}^*$ where $F_{ij}\in\Sigma$ for $i\in\Bbb N$, $j<m$.
Set $V=X^m\setminus\bigcup_{i\in\Bbb N}\prod_{j<m}F_{ij}$, so that

\Centerline{$\mu^mV
=\nu^m(Z^m\setminus\bigcup_{i\in\Bbb N}\prod_{j<m}F_{ij}^*)>0$.}

I seek to choose $\sequence{k}{n_k}$ in $\Bbb N$, $\sequence{k}{H_k}$
and $\langle E_{kj}\rangle_{k\in\Bbb N,j<m}$ in $\Sigma$ inductively, in
such a way that

\Centerline{$2^{-n_0}\le\bover12\epsilon$,}

\Centerline{$E_{kj}\subseteq E_{ij}\cap G_{H_i}$ for $i<k$ and $j<m$,}

\Centerline{$\prod_{j<m}E_{kj}\cap\prod_{j<m}F_{kj}=\emptyset$,}

\Centerline{$\mu^m(V\cap G_{H_k}^m\cap\prod_{j<m}E_{kj})>0$,}

\Centerline{$n_i<n_k$ for every $i<k$,}

\Centerline{$H_k\in\Cal H_{n_k}$,}

\Centerline{if $k=i_km+j_k$, where $i_k\in\Bbb N$ and $j_k<m$, then
$\mu(H_k\cap E_{kj_k})>0$,}

\noindent for every $k\in\Bbb N$.   The induction proceeds as follows.
Set $E'_{kj}=X$ if $k=0$, $G_{H_{k-1}}\cap E_{k-1,j}$ otherwise, so that
$\mu(V\cap\prod_{j<m}E'_{kj})>0$.   Because $V\cap\prod_{j<m}F_{kj}$ is
empty, we can find $E_{kj}\subseteq E'_{kj}$, for $j<m$, such that
$\mu^m(V\cap\prod_{j<m}E_{kj})>0$ and
$\prod_{j<m}E_{kj}\cap\prod_{j<m}F_{kj}=\emptyset$.   Set
$\eta=\mu E_{kj_k}$, $\delta=\mu^m(V\cap\prod_{j<m}E_{kj})$, so that
$\eta$ and $\delta$ are both strictly positive.

Now take $n_k$ so large that

\Centerline{(if $k=0$) $2^{-n_k}\le\bover12\epsilon$,
\quad$n_k>n_i$ for $i<k$,
\quad$(1-2^{-mn_k})^{\eta r_{n_k}}<\delta$.}

\noindent (This is possible because
$\lim_{n\to\infty}2^{-mn}r_n=\infty$.)
Set $\Cal H=\{H:H\in\Cal H_{n_k},\,\mu(H\cap E_{kj_k})>0\}$;  then
$\#(\Cal H)\ge\eta r_{n_k}$.   Consider the family
$\family{H}{\Cal H}{G_H^m}$.   These are stochastically independent sets
of measure $2^{-mn_k}$, so their union has measure
$1-(1-2^{-mn_k})^{\#(\Cal H)}>1-\delta$, and there is an $H_k\in\Cal H$
such that $\mu^m(V\cap G_{H_k}^m\cap\prod_{j<m}E_{kj})>0$.   Thus the
induction continues.

Look at the sequence $\sequence{k}{H_k}$ and its union $E$.   We have
$H_k\in\Cal H_{n_k}$ for every $k$;  moreover, if $i<k$, then
$\mu(H_k\cap E_{kj_k})>0$, while $E_{kj_k}\subseteq G_{H_i}$;  since
$H_k$ is an atom of $\Sigma_{n_k}$, while $G_{H_i}\in\Sigma_{n_k}$,
$H_k\subseteq G_{H_i}$.   Thus $E\in\Cal E$.   Next, whenever
$i\le k\in\Bbb N$,

\Centerline{$\prod_{j<m}E_{kj}\cap\prod_{j<m}F_{ij}
\subseteq\prod_{j<m}E_{ij}\cap\prod_{j<m}F_{ij}=\emptyset$,}

\noindent so
$\prod_{j<m}E_{kj}\cap\bigcup_{i\le k}\prod_{j<m}F_{ij}=\emptyset$.   At
the same time, we know that

\Centerline{$E^m\cap\prod_{j<m}E_{kj}
\supseteq\prod_{j<m}H_{km+j}\cap E_{kj}
\supseteq\prod_{j<m}H_{km+j}\cap E_{km+j,j}$}

\noindent has non-zero measure.   So
$\mu^m(E^m\setminus\bigcup_{i\le k}\prod_{j<m}F_{ij})>0$.

Moving back to $Z$, this translates into

\Centerline{$\nu^m((E^*)^m\setminus\bigcup_{i\le k}\prod_{j<m}F^*_{ij})
>0$.}

\noindent But this means that $(E^*)^m$ is not included in
$\bigcup_{i\le k}\prod_{j<m}F^*_{ij}$, for any $k\in\Bbb N$.   Because
$E^*$ is compact and every $F^*_{ij}$ is open,
$(E^*)^m$ is not included in $\bigcup_{i\in\Bbb N}\prod_{j<m}F^*_{ij}$,
and there is some $v\in(E^*)^m\cap\tilde V$.

By the definition of $\tilde V$,

$$\eqalign{\nu^m\{u:u\#v\in\tilde W\}
&>m\epsilon
\ge m\sum_{k=0}^{\infty}2^{-n_0-k}
\ge m\sum_{k=0}^{\infty}2^{-n_k}\cr
&=m\sum_{k=0}^{\infty}\mu H_k
\ge m\mu E
=m\nu E^*.\cr}$$

\noindent So there must be some $u$ such that $u\#v\in\tilde W$ and
$u(j)\notin E^*$ for every $j<m$.   But now, setting $w=u\#v$, we have
$w(2j)\notin E^*$, $w(2j+1)\in E^*$ for $j<m$, and $w\in
D_m(A^*,Z,0,1)\cap\tilde W$;  which is supposed to be impossible.\
\Bang\Qed

\medskip

{\bf (e)} This shows that $A^*$ is not stable.   It is, however,
R-stable.   \Prf\ We have a measure algebra isomorphism between the
measure algebras of $\mu$ and $\nu$ defined by the map $E\mapsto
E^*:\Sigma\to\Tau$.   The corresponding isomorphism between $L^0(\mu)$
and $L^0(\nu)$ takes $\{f^{\ssbullet}:f\in A\}$ to
$\{h^{\ssbullet}:h\in A^*\}$.   By 465Pa and (b) above,
$\{f^{\ssbullet}:f\in A\}$ is stable in $L^0(\mu)$, so
$\{h^{\ssbullet}:h\in A^*\}$ is stable in $L^0(\nu)$, and every
countable subset of $A^*$ is stable.
Since $A^*\subseteq C(X)$, it follows that $A^*$ is stable (465T).\ \Qed
}%end of proof of 465U

\cmmnt{
\leader{*465V}{Remark} This example is clearly related to
419E.   The argument here is significantly deeper, but it does have an
idea in common with that in 419E, besides the obvious point that both
involve the Stone spaces of atomless probability spaces.
Suppose that, in the context of 465U, we take $\Cal E_0$ to be the
family of sets $E$ expressible as the union of a {\it finite} chain
$H_0,\ldots,H_k$ where $H_i\in\Cal H_{n_i}$ for $i\le k$ and
$H_j\subseteq G_{H_i}$ for $i<j\le k$.   Then we find, on repeating the
argument of (b) in the proof of 465U, that the countable set
$A_0^*=\{\chi E^*:E\in\Cal E_0\}$ is stable, so that, setting
$W_m=D_m(A_0^*,Z,0,1)$, $\nu^{2m}W_m$ is small for large $m$.   On the
other hand, setting

\Centerline{$\tilde W_m=\bigcup\{V:V\subseteq Z^m$ is open,
$V\setminus W_m$ is negligible$\}$,}

\noindent we see that $D_m(A^*,Z,0,1)\subseteq\tilde W_m$, so that
$(\nu^{2m})^*\tilde W_m=1$ for all $m\ge 1$.   Of course, writing
$\tilde\nu^{2m}$ for the Radon product measure on $Z^{2m}$, we have
$\tilde\nu^{2m}(\tilde W_m)=\nu^{2m}W_m<1$ for large $m$, just as in
419E.

Both $A\subseteq\eusm L^0(\Sigma)$ and
$A^*\subseteq\eusm L^0(\Tau)$ are relatively pointwise compact.   Note
that while I took $(X,\Sigma,\mu)$ and $(Z,\Tau,\nu)$ to be quite
separate, it is entirely possible for them to be actually the same
space.   In this case it is natural to take every $\Sigma_n$ to consist
of open-and-closed sets, so that every member of $\Cal E$ is open, and
$E^*$ becomes identified with the closure of $E$ for $E\in\Cal E$.
}%end of comment

\exercises{\leader{465X}{Basic exercises (a)}
%\spheader 465Xa
Let $(X,\Sigma,\mu)$ be a semi-finite measure space, and
$\sequencen{f_n}$ a sequence of measurable real-valued functions on $X$
which converges a.e.   Show that $\{f_n:n\in\Bbb N\}$ is stable.
%465B

\spheader 465Xb Let $\Cal C$ be the family of convex sets in $\BbbR^r$.
Show that $\{\chi C:C\in\Cal C\}$ is stable with respect to Lebesgue
measure on $\BbbR^r$, but that if $r\ge 2$ there is a Radon probability
measure $\nu$ on $\BbbR^r$ such that
$\{\chi C:C\in\Cal C$, $C$ is closed$\}$ is not stable with respect to
$\nu$.
%465B

\spheader 465Xc Show that, for any $M\ge 0$, the set of functions
$f:\Bbb R\to\Bbb R$ of variation at most $M$ is stable with
respect to any Radon measure on $\Bbb R$.   \Hint{show that if $\mu$ is
a Radon measure on $\Bbb R$ and $E\in\dom\mu$ has non-zero finite
measure, and $(2k-1)(\beta-\alpha)>M$, then
$(\mu^{2k})^*D_k(A,E,\alpha,\beta)<(\mu E)^{2k}$.}
%465B

\spheader 465Xd Let $(X,\Sigma,\mu)$ be a
semi-finite measure space, and
$A\subseteq\Bbb R^X$.   Show that $A$ is stable iff $\{f^+:f\in A\}$ and
$\{f^-:f\in A\}$ are both stable.
%465C

\spheader 465Xe Let $(X,\Sigma,\mu)$ and $(Y,\Tau,\nu)$ be
$\sigma$-finite measure
spaces, and $\phi:X\to Y$ an \imp\ function.   Show that if
$B\subseteq\BbbR^Y$ is stable with respect to $\nu$, then
$\{g\phi:g\in B\}$ is stable with respect to $\mu$.
%465C

\spheader 465Xf Let $(X,\Sigma,\mu)$ be a semi-finite measure space and
$A\subseteq\Bbb R^X$.   Suppose that $\mu$ is inner regular with
respect to the family
$\{F:F\in\Sigma,\,\{f\times\chi F:f\in A\}$ is stable$\}$.   Show that
$A$ is stable.
%465C

\spheader 465Xg Let $(X,\Sigma,\mu)$ be a totally finite measure space
and $\Tau$ a $\sigma$-subalgebra of $\Sigma$.   Let
$A\subseteq\eusm L^0(\Tau)$ be any set.   (i) Show that if $A$ is
$\mu\restrp\Tau$-stable
then it is $\mu$-stable.   (ii) Give an example to show that $A$ can be
$\mu$-stable and pointwise compact without being
$\mu\restrp\Tau$-stable.   \Hint{take $\mu$ to be Lebesgue measure on
$[0,1]$ and $\Tau$ the countable-cocountable algebra.}
%465C

\spheader 465Xh Let $(X,\Sigma,\mu)$ be a
semi-finite measure space, and
$A\subseteq\Bbb R^X$.   For $g:X\to\coint{0,\infty}$ set
$A_g=\{(f\wedge g)\vee(-g):f\in A\}$.
Show that $A$ is stable iff $A_g$ is stable for every integrable
$g:X\to\coint{0,\infty}$.
%465C

\spheader 465Xi Let $(X,\Sigma,\mu)$ be a
semi-finite measure space.   A set
$A\subseteq\BbbR^X$ is said to have the {\bf Bourgain property} if whenever
$E\in\Sigma$, $\mu E>0$ and $\epsilon>0$, there are
non-negligible measurable sets $F_0,\ldots,F_n\subseteq E$ such that for
every $f\in A$ there is an $i\le n$ such that the oscillation
$\sup_{x,y\in F_i}|f(x)-f(y)|$ of $f$ on $F_i$ is at most $\epsilon$.
Show that in this case $A$ is stable.
%465C  to be moved query

\spheader 465Xj Let $X$ be a topological space, and $\mu$ a
$\tau$-additive effectively locally finite topological measure on $X$.
Show that any equicontinuous subset of $C(X)$ has the Bourgain property, so
is stable.
%465Xi 465C to be moved query

\spheader 465Xk Let $(X,\Sigma,\mu)$ be a locally determined measure
space and
$A\subseteq\Bbb R^X$ a stable set.   Let $\overline{A}$ be the closure
of $A$ for the topology of pointwise convergence.    Show that
$\{f^{\ssbullet}:f\in\overline{A}\}$ is just the closure of
$\{f^{\ssbullet}:f\in A\}\subseteq L^0(\mu)$ for the topology of
convergence in measure.
%465D

\spheader 465Xl Show that there is a disjoint family $\Cal I$ of finite
subsets of $[0,1]$ such that $A=\{\chi I:I\in\Cal I\}$ is not stable,
though $A$ is pointwise compact and the identity map on $A$ is
continuous for the topology of pointwise convergence and the topology of
convergence in measure.
%465F

\spheader 465Xm Let $(X,\Sigma,\mu)$ be a probability space.   Show that
$A\subseteq\Bbb R^X$ is stable iff

\Centerline{$\inf_{m\in\Bbb N}
\bigl((\mu^{2m})^*D_m(A,X,\alpha,\beta)\bigr)^{1/m}=0$}

\noindent whenever $\alpha<\beta$ in $\Bbb R$.
%465J

\spheader 465Xn Let $(X,\Sigma,\mu)$ be a
semi-finite measure space.   Show that a
countable set $A\subseteq\eusm L^0(\Sigma)$ is not stable iff there are
$E\in\Sigma$ and $\alpha<\beta$ such that $0<\mu E<\infty$ and
$\mu^m\tilde D_m(A,E,\alpha,\beta)=(\mu E)^m$ for every $m\ge 1$, where
$\tilde D_m(A,E,\alpha,\beta)$ is the set of those $w\in E^m$ such that
for every $I\subseteq m$ there is an $f\in A$ such that
$f(w(i))\le\alpha$
for $i\in I$, $f(w(i))\ge\beta$ for $i\in m\setminus I$.   \Hint{see
part (iii) of case 2 of the proof of 465L.}
%465L

\spheader 465Xo Let $(X,\Sigma,\mu)$ be a semi-finite measure space, and
$A\subseteq\eusm L^0(\Sigma)$ a set which is compact and metrizable for
the topology of pointwise convergence.   Show that $A$ is stable.
\Hint{otherwise, apply the
ideas of case 2 in the proof of 465L to a countable dense subset of $A$
to obtain a sequence which contradicts the conclusion of 465Xa.}
%465L

\spheader 465Xp Let $(X,\Sigma,\mu)$ be a probability space and
$A\subseteq\eusm L^0(\Sigma)$ a uniformly bounded set of functions.
Show that $A$ is stable iff

\Centerline{$
\lim_{n\to\infty}\overlineint
  \sup_{f\in A,k,l\ge n}
\bigl|\Bover1k\sum_{i=0}^{k-1}f(w(i))
  -\Bover1l\sum_{i=0}^{l-1}f(w(i))\bigr|\mu^{\Bbb N}(dw)=0$.}
%465M

\spheader 465Xq Show that there is a set $A\subseteq\BbbR^{[0,1]}$ such
that $g(x)=\sup_{f\in A}|f(x)|$ is finite for every $x$, and $A$ is
stable with respect to Lebesgue measure, but its convex hull $\Gamma(A)$
is not.   \Hint{take $A=\{g(x)\chi\{x\}:x\in[0,1]\}$, where $g$ is such
that for every $m\ge 1$, every non-negligible compact set
$K\subseteq[0,1]^m$ there is a $w\in K$ such that $g(w(i))=2^{i+1}$ for
every $i<m$.}
%465N

\spheader 465Xr Show that there is a sequence $\sequencen{f_n}$ of
functions from $[0,1]$ to $\Bbb N$ such that $\{f_n:n\in\Bbb N\}$ is stable
for Lebesgue measure on $[0,1]$, but $\{f_m-f_n:m$, $n\in\Bbb N\}$ is not.
%465N

\spheader 465Xs Let $(X,\Sigma,\mu)$ be a strictly localizable measure
space and $Q\subseteq L^0(\mu)$ a set which is stable in the sense of
465R.  Show that the closure of $Q$ (for the topology of convergence in
measure) is stable.   \Hint{465Xk.}
%465R

\spheader 465Xt Let $(X,\Sigma,\mu)$ be a probability space, $\Tau$ a
$\sigma$-subalgebra of $\Sigma$, and $A\subseteq\Bbb R^X$ a countable
stable set of $\Sigma$-measurable functions such that
$\int\sup_{f\in A}|f(x)|\mu(dx)<\infty$.   Show that if for each
$f\in A$ we choose a conditional expectation $g_f$ of $f$ on $\Tau$
(requiring
each $g_f$ to be $\Tau$-measurable and defined everywhere on $X$), then
$\{g_f:f\in A\}$ is stable.
%465R

\spheader 465Xu Give an example of a probability algebra
$(\frak A,\bar\mu)$, a conditional expectation operator
$P:L^1_{\bar\mu}\to L^1_{\bar\mu}$ (365R), and a uniformly integrable
stable set $A\subseteq L^1_{\bar\mu}$ such that $P[A]$ is not stable.
\Hint{start by picking
a sequence $\sequencen{v_n}$ in $P[L^1_{\bar\mu}]$ which is
norm-convergent to $0$ but not stable, and express this as
$\sequencen{Pu_n}$ where $\bar\mu\Bvalue{u_n>0}\le 2^{-n}$ for every
$n$.}
%465R

\leader{465Y}{Further exercises (a)}
%\spheader 465Ya
Find an integrable continuous function
$f:\coint{0,1}^2\to\coint{0,\infty}$ such that, in the notation of
465H, $\limsup_{k\to\infty}\int fd\nu_{wk}^2=\infty$ for almost every
$w\in\coint{0,1}^{\Bbb N}$.   \Hint{arrange
for $f(x,x)$ to grow very fast as $x\uparrow 1$.}
%465H

\spheader 465Yb Show that in 465M we may replace the condition `$A$ is
uniformly bounded' by the condition `$|f|\le f_0$ for every $f\in A$,
where $f_0$ is integrable'.   \Hint{{\smc Talagrand 87}.}
%465M

%\spheader 465Yc Let $(X,\Sigma,\mu)$ be a probability space, and
%$A\subseteq\BbbR^X$ a uniformly bounded set.   Show that $A$ is stable iff
%for every $\epsilon>0$ there is a finite subalgebra $\Tau$ of $\Sigma$, a
%sequence $\sequence{k}{h_k}$ of measurable functions on $X^{\Bbb N}$, and a
%family $\family{f}{A}{g_f}$ of $\Tau$-measurable functions such that

%\Centerline{$h_k(w)\ge\Bover1k\sum_{i=0}^{k-1}|f(w(i))-g_f(w(i))|$
%for every $w\in X^{\Bbb N}$, $k\ge 1$ and $f\in A$,}

%\Centerline{$\limsup_{k\to\infty}h_k(w)\le\epsilon$ for almost every
%$w\in X^{\Bbb N}$.}

%\noindent\Hint{465M(iii)$\Rightarrow$465M(v).}
%465M

\spheader 465Yc Let $(X,\Sigma,\mu)$ be a probability space, and
$A\subseteq\BbbR^X$ a uniformly bounded set.   Show that $A$ is stable iff
for every $\epsilon>0$ there are a stable set $B\subseteq\BbbR^X$, a
sequence $\sequence{k}{h_k}$ of measurable functions on $X^{\Bbb N}$, and a
family $\family{f}{A}{g_f}$ in $B$ such that

\Centerline{$h_k(w)\ge\Bover1k\sum_{i=0}^{k-1}|f(w(i))-g_f(w(i))|$
for every $w\in X^{\Bbb N}$, $k\ge 1$ and $f\in A$,}

\Centerline{$\limsup_{k\to\infty}h_k(w)\le\epsilon$ for almost every
$w\in X^{\Bbb N}$.}

\noindent\Hint{465M(iii)$\Rightarrow$465M(v), with a little help from
465Nc.}
%465N

\spheader 465Yd Let $(X,\Sigma,\mu)$ be a semi-finite measure space,
and $A$, $B\subseteq\BbbR^X$ pointwise bounded stable sets.   Show that
$\{f\times g:f\in A$, $g\in B\}$ is stable.   \Hint{for the uniformly
bounded case use 465M(iii) and 465Yc;  then extend as in 465Nc.}
%465Yc 465N

\spheader 465Ye Let $(X,\Sigma,\mu)$ and $(Y,\Tau,\nu)$ be semi-finite
measure spaces, with c.l.d.\ product $(X\times Y,\Lambda,\lambda)$.
Suppose that $A\subseteq\BbbR^X$ and $B\subseteq\BbbR^Y$ are pointwise
bounded stable sets.   Show that $\{f\otimes g:f\in A$, $g\in B\}$ is
stable with respect to $\lambda$, where $(f\otimes g)(x,y)=f(x)g(y)$.
%465Yd 465N

\spheader 465Yf Let $(X,\Sigma,\mu)$ be a semi-finite measure space,
$A\subseteq\BbbR^X$ a uniformly bounded stable set and
$h:\Bbb R\to\Bbb R$ a continuous
function.   Show that $\{hf:f\in A\}$ is stable.
\Hint{use 465Yd, 465Nc and the Stone-Weierstrass theorem.}
%465Yd 465N

\spheader 465Yg Let $(X,\frak T,\Sigma,\mu)$ be a $\tau$-additive
topological measure space such that $\mu$ is inner regular with respect
to the Borel sets.   Show that a countable R-stable subset of $\Bbb R^X$
is stable.
%465S

\spheader 465Yh Let $(X,\frak T,\Sigma,\mu)$ be a complete
$\tau$-additive topological probability
space such that $\mu$ is inner regular with respect to the Borel sets,
and $A\subseteq[0,1]^X$ an R-stable set.   Suppose that $\epsilon>0$
is such that $\int fd\mu\le\epsilon^2$ for every $f\in A$.   Show that
there are an $n\ge 1$ and a Borel set $W\subseteq X^n$ and a
$\gamma>\tilde\mu^nW$ (writing $\tilde\mu^n$ for the $\tau$-additive
product measure on $X^n$) such that $\int fd\nu\le 3\epsilon$ whenever
$f\in A$, $\nu:\Cal PX\to[0,1]$ is a point-supported
probability measure and $\nu^nW\le\gamma$.
%465L 465S

\spheader 465Yi Set $X=\prod_{n=2}^{\infty}\Bbb Z_n$, where each $\Bbb Z_n$
is the cyclic group of order $n$ with its discrete topology;  let $\mu$ be
the Haar probability measure on $X$.   Let $\action_l$ be the left shift
action of $X$ on $\BbbR^X$, so that
$(a\action_lf)(x)=f(x-a)$ for $a$, $x\in X$ and $f\in\BbbR^X$
(i) Show that for any $n\in\Bbb N$ there is a
compact negligible set $K_n\subseteq X$ such that for every $I\in[X]^n$
there are uncountably many $a\in X$ such that $I\subseteq K_n+a$.   (ii)
Show that there is a negligible set $E\subseteq X$ such that
$\{a\action_l\chi E:a\in X\}$ is dense in $\{0,1\}^X$ for the topology of
pointwise convergence.   (iii) Show that if $f\in C(X)$ then
$\{a\action_lf:a\in X\}$ is stable.   (iv) Show that there is a sequence
$\sequencen{f_n}$ in $C(X)$ such that $\{f_n:n\in\Bbb N\}$ is stable but
$\{a\action_lf_n):a\in X$, $n\in\Bbb N\}$ is not.
(v) Find expressions of these
results when $X$ is replaced by the circle group or by $\Bbb R$.
}%end of exercises

\endnotes{
\Notesheader{465}
The definition in 465B arose naturally when M.Talagrand and I were
studying pointwise compact sets of measurable functions;  we found that
in many cases a set of functions was relatively pointwise compact
because it was stable (465Db).   Only later did it appear that the
concept was connected with Glivenko-Cantelli classes in the theory of
empirical measures, as explained in {\smc Talagrand 87}.

It is not the case that all pointwise compact sets of
measurable functions are stable.   In fact I have already offered
examples in 463Xh
%countable-cocountable
and 464E above.   In both cases it is easy to check from the definition
in 465B that they are not stable, as can also be deduced from 465G.
Another example is in 465Xl.   You will observe however that all these
examples are `pathological' in the sense that either the measure space
is irregular (from some points of view, indeed, any measure space not
isomorphic to Lebesgue measure on the unit interval can be dismissed as
peripheral), or the set of functions is uninteresting.   It is clear
from 465R and 465T, for instance, that we should start with countable
sets.   So it is natural to ask:  if we have a {\it separable} pointwise
compact set of real-valued measurable functions on the unit interval,
must it be stable?   It turns out that this is undecidable in
Zermelo-Fraenkel set theory ({\smc Shelah \& Fremlin 93});  I hope to
return to the question in Volume 5.   (If we ask for `metrizable',
instead of `separable', we get a positive answer;  see 465Xo.)

The curious phrasing of the statement of 465M(iii), with the auxiliary
functions $h_k$, turns on the fact that all the expressions
`$\sup_{f\in A}\ldots$' here give rise to functions which need not be measurable.
Thus the simple pointwise convergence described in (ii) and (iv) is not
at all the same thing as the convergence in (v), which may be thought of
as a kind of $\|\,\|_1$-convergence if we write
$\|g\|_1=\overline{\int}|g|$ for arbitrary real-valued functions $g$.
(Since the sets $A$ here are uniformly bounded, it may equally be
thought of as convergence in measure.)   Similarly, 465M(iii) is saying
much more than just

\Centerline{$\lim_{k\to\infty}\sup_{f\in A}\Bover1k\sum_{i=0}^{k-1}
\bigl|f(w(i))-\Expn(f|\Tau)(w(i))\bigr|=0$ a.e.,}

\noindent though of course for countable sets $A$ these distinctions
disappear.   On the other hand, since the convergence is certainly not
monotonic in $k$, $\|\,\|_1$-convergence does not imply pointwise
convergence.   So we can look for something stronger than either (iv) or
(v) of 465M, as in 465Xp.   But this is still nowhere near the strength
of 465M(iii), in which $|\sum\ldots|$ is replaced by $\sum|\ldots|$.
For further variations, see {\smc Talagrand 87} and {\smc Talagrand 96}.

If we wish to adapt the ideas here to spaces of equivalence classes of
functions rather than spaces of true functions, we find that problems of
measurability evaporate, and that (because the definition of stability
looks only at sets of finite measure) all the relevant suprema can be
interpreted as suprema of countable sets.   Consequently a subset of
$L^0(\mu)$ or $L^0(\frak A)$ is stable if all its countable subsets are
stable (465Pa).   It is remarkable that, for strictly localizable
measures $\mu$, we can lift any stable set in $L^0$ to a stable set in
$\eusm L^0$ (465Pb, 465Q).   By moving to function spaces we get a
language in which to express a new kind of permanence property of stable
sets (465R, 465Xt).   See also {\smc Talagrand 89}.

The definition of `stable set' of functions seems to be utterly
dependent on the underlying measure space.   But 465R tells us that in
fact the property of being an order-bounded stable subset of
$L^1(\bar\mu)$ is invariant under normed space automorphisms.
(`Order-boundedness' is a normed space invariant in $L^1$ spaces by the
Chacon-Krengel theorem, 371D.)    Since stability can be defined in
terms of order-bounded sets (465Xh), we could, for instance, develop a
theory of stable sets in abstract $L$-spaces.

The theory of stable sets is of course bound intimately to the theory of
product measures, and such results as 465J have independent interest as
theorems about sets in product spaces.   So any new theory of product
measures will give rise to a new theory of stable sets.   In particular,
the $\tau$-additive product measures in \S417 lead to R-stable sets
(465S).   It is instructive to work through the details, observing how
the properties of the product are employed.   Primarily, of course, we
need `associative' and `commutative' laws, and Fubini's theorem;  but
some questions of measurability need to be re-examined, as in 465Yh.

I have starred 465U because it involves the notion of R-stability.   In
fact this appears only in the final stage, and the construction, as set
out in part (a) of the proof, is an instructive challenge to any
intuitive concept of what stable sets are like.
}%end of notes

\discrpage

