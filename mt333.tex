\frfilename{mt333.tex}
\versiondate{27.6.08}
\copyrightdate{1996}

\def\chaptername{Maharam's theorem}
\def\sectionname{Closed subalgebras}

\newsection{333}

Proposition 332P tells us, in effect, which totally finite measure
algebras can be
embedded as closed subalgebras of each other.   Similar techniques make
it possible to describe the possible forms of such embeddings.   In this
section I give the fundamental theorems on extension of
measure-preserving homomorphisms from closed subalgebras (333C, 333D);
these rely on the concept of `relative Maharam type' (333A).   I go on
to describe possible canonical forms for structures
$(\frak A,\bar\mu,\frak C)$, where $(\frak A,\bar\mu)$ is a totally
finite measure algebra and
$\frak C$ is a closed subalgebra of $\frak A$ (333K, 333N).   I end the
section with a description of fixed-point subalgebras (333R).

\leader{333A}{Definitions (a)} Let $\frak A$ be a Boolean algebra and
$\frak C$ a subalgebra of $\frak A$.   The {\bf relative Maharam type of
$\frak A$ over $\frak C$}, $\tau_{\frak C}(\frak A)$, is the smallest
cardinal of any set $A\subseteq\frak A$ such that
$A\cup\frak C\,\,\tau$-generates $\frak A$.

\spheader 333Ab In this section, I will regularly use the following
notation:
if $\frak A$ is a Boolean algebra, $\frak C$ is a subalgebra of
$\frak A$, and $a\in\frak A$, then I will write $\frak C_a$ for
$\{c\Bcap a:c\in\frak C\}$.   Observe that $\frak C_a$ is a subalgebra of
the principal ideal $\frak A_a$\prooflet{ (because
$c\mapsto c\Bcap a:\frak C\to\frak A_a$ is a Boolean homomorphism);
it is included in $\frak C$ iff $a\in\frak C$}.

\spheader 333Ac\cmmnt{ Still taking $\frak A$ to be a Boolean algebra and
$\frak C$
to be a subalgebra of $\frak A$,} I will say that an element $a$ of
$\frak A$ is {\bf relatively \Mth\ over $\frak C$} if
$\tau_{\frak C_b}(\frak A_b)=\tau_{\frak C_a}(\frak A_a)$ for every
non-zero $b\Bsubseteq a$.

\spheader 333Ad If $\kappa$ is a cardinal\cmmnt{ which
is either infinite or zero},
I will write $(\frak B_{\kappa},\bar\nu_{\kappa})$ for the measure algebra
of the usual measure $\nu_{\kappa}$ on $\{0,1\}^{\kappa}$.   \cmmnt{I
hope that there will be no confusion between this notation and the use, in
333C-333F, %333C 333D 333E 333F
of the formula $\frak B_b$ for the principal ideal generated by $b$
in an arbitrary Boolean algebra $\frak B$.}

\leader{333B}{}\cmmnt{ Evidently this is a generalization of the
ordinary concept of Maharam type as used in \S\S331-332;  if
$\frak C=\{0,1\}$ then $\tau_{\frak C}(\frak A)=\tau(\frak A)$.   The
first step is naturally to check the results corresponding to 331H.

\medskip

\noindent}{\bf Lemma} Let $\frak A$ be a Boolean algebra and $\frak C$ a
subalgebra of $\frak A$.

(a) If $a\Bsubseteq b$ in $\frak A$, then
$\tau_{\frak C_a}(\frak A_a)\le\tau_{\frak C_b}(\frak A_b)$.
In particular, $\tau_{\frak C_a}(\frak A_a)\le\tau_{\frak C}(\frak A)$
for every $a\in\frak A$.

(b) The set $\{a:a\in\frak A$ is relatively \Mth\ over
$\frak C\}$ is order-dense in $\frak A$.

(c) If $\frak A$ is Dedekind complete and $\frak C$ is order-closed in
$\frak A$, then $\frak C_a$ is order-closed in $\frak A_a$.

(d) If $a\in\frak A$ is relatively \Mth\ over $\frak C$
then either $\frak A_a=\frak C_a$, so that
$\tau_{\frak C_a}(\frak A_a)=0$ and $a$ is a relative atom of $\frak A$
over $\frak C$\cmmnt{ (definition:  331A)}, or
$\tau_{\frak C_a}(\frak A_a)\ge\omega$.

(e) If $\frak D$ is another subalgebra of $\frak A$ and
$\frak D\subseteq\frak C$, then

\Centerline{$\tau(\frak A_a)=\tau_{\{0,a\}}(\frak A_a)
\ge\tau_{\frak D_a}(\frak A_a)
\ge\tau_{\frak C_a}(\frak A_a)
\ge\tau_{\frak A_a}(\frak A_a)=0$}

\noindent for every $a\in\frak A$.

\proof{{\bf (a)} Let $D\subseteq\frak A_b$ be a set of cardinal
$\tau_{\frak C_b}(\frak A_b)$ such that
$D\cup\frak C_b\,\,\tau$-generates $\frak A_b$.   Set
$D'=\{d\Bcap a:d\in D\}$.   Then $D'\cup\frak C_a\,\,\tau$-generates
$\frak A_a$.   \Prf\ Apply 313Mc to the map
$d\mapsto d\Bcap a:\frak A_b\to\frak A_a$, as in 331Hc.\ \QeD\
Consequently

\Centerline{$\tau_{\frak C_a}(\frak A_a)\le\#(D')\le\#(D)
=\tau_{\frak C_b}(\frak A_b)$,}

\noindent as claimed.    Setting $b=1$ we get
$\tau_{\frak C_a}(\frak A_a)\le\tau_{\frak C}(\frak A)$.

\medskip

{\bf (b)} Just as in the proof of 332A, given
$b\in\frak A\setminus\{0\}$, there is an $a\in\frak A_b\setminus\{0\}$
minimising $\tau_{\frak C_a}(\frak A_a)$, and this $a$ must be
relatively \Mth\ over $\frak C$.

\medskip

{\bf (c)} $\frak C_a$ is the image of the Dedekind complete Boolean
algebra $\frak C$ under the
order-continuous Boolean homomorphism $c\mapsto c\Bcap a$, so must be
order-closed (314F(a-i)).

\medskip

{\bf (d)} Suppose that $\tau_{\frak C_a}(\frak A_a)$ is finite.   Let
$D\subseteq\frak A_a$ be a finite set such that $D\cup\frak
C_a\,\,\tau$-generates $\frak A_a$.   Then there is a non-zero
$b\in\frak A_a$ such that $b\Bcap d$ is either $0$ or $b$ for every
$d\in D$.   But this means that $\frak C_b=\{d\Bcap b:d\in D\cup\frak
C_a\}$, which $\tau$-generates $\frak A_b$;  so that $\tau_{\frak
C_b}(\frak A_b)=0$.   Since $a$ is relatively \Mth\ over
$\frak C$, $\tau_{\frak C_a}(\frak A_a)$ must be zero, that is, $\frak
A_a=\frak C_a$.

\medskip

{\bf (e)} The middle inequality is true just because $\frak A_a$ will be
$\tau$-generated by $D\cup\frak C_a$ whenever it is $\tau$-generated by
$D\cup\frak D_a$.   The neighbouring inequalities are special cases of
the middle one, and the outer equalities are elementary.
}%end of proof of 333B

\leader{333C}{Theorem} Let $(\frak A,\bar\mu)$ and $(\frak B,\bar\nu)$
be totally finite measure algebras, and $\frak C$ a closed subalgebra of
$\frak A$.   Let $\phi:\frak C\to\frak B$ be a measure-preserving
Boolean homomorphism.

(a) If\cmmnt{, in the notation of 333A,}
$\tau_{\frak C}(\frak A)\le\tau_{\phi[\frak C]_b}(\frak B_b)$
for every non-zero
$b\in\frak B$, there is a measure-preserving Boolean homomorphism
$\pi:\frak A\to\frak B$ extending $\phi$.

(b) If $\tau_{\frak C_a}(\frak A_a)=\tau_{\phi[\frak C]_b}(\frak B_b)$
for every non-zero $a\in\frak A$, $b\in\frak B$, then there is a measure
algebra isomorphism $\pi:\frak A\to\frak B$ extending $\phi$.

\proof{ In both parts, the idea is to use the technique of the proof of
331I to construct $\pi$ as the last of an increasing family
$\langle\pi_{\xi}\rangle_{\xi\le\kappa}$ of measure-preserving
homomorphisms from closed subalgebras $\frak C_{\xi}$ of $\frak A$,
where $\kappa=\tau_{\frak C}(\frak A)$.   Let
$\langle a_{\xi}\rangle_{\xi<\kappa}$ be a family in $\frak A$ such that
$\frak C\cup\{a_{\xi}:\xi<\kappa\}\,\,\tau$-generates $\frak A$.   Write
$\frak D$ for $\phi[\frak C]$;  remember that $\frak D$ is a closed
subalgebra of $\frak B$ (324L).

\medskip

{\bf (a)(i)} In this case, we can describe the $\frak C_{\xi}$
immediately;   $\frak C_{\xi}$ will be the closed subalgebra of
$\frak A$ generated by $\frak C\cup\{a_{\eta}:\eta<\xi\}$.   The
induction starts with $\frak C_0=\frak C$, $\pi_0=\phi$.

\medskip

\quad{\bf (ii)} For the inductive step to a successor ordinal $\xi+1$,
where $\xi<\kappa$, suppose that $\frak C_{\xi}$ and $\pi_{\xi}$ have
been defined.   Take any non-zero $b\in\frak B$.   We are supposing that
$\tau_{\frak D_b}(\frak B_b)\ge\kappa>\#(\xi)$, so $\frak B_b$ cannot be
$\tau$-generated by

\Centerline{$D=\frak D_b\cup\{b\Bcap\pi_{\xi}a_{\eta}:\eta<\xi\}
=\pi_{\xi}[\frak C]_b\cup\{b\Bcap\pi_{\xi}a_{\eta}:\eta<\xi\}
=\psi[\frak C\cup\{a_{\eta}:\eta<\xi\}]$,}

\noindent writing $\psi c=b\Bcap\pi_{\xi}c$ for $c\in\frak C_{\xi}$.
As $\psi$ is order-continuous, $\psi[\frak C_{\xi}]$ is precisely the
closed subalgebra of $\frak B_b$ generated by $D$ (314H), and is
therefore not the whole of $\frak B_b$.

But this means that
$\frak B_b\ne\{b\Bcap \pi_{\xi}c:c\in\frak C_{\xi}\}$.   As $b$ is
arbitrary, $\pi_{\xi}$ satisfies the conditions
of 331D, and has an extension to a measure-preserving Boolean
homomorphism $\pi_{\xi+1}:\frak C_{\xi+1}\to\frak B$, since
$\frak C_{\xi+1}$ is just the closed subalgebra of $\frak A$ generated
by $\frak C\cup\{a_{\xi}\}$.

\medskip

\quad{\bf (iii)} For the inductive step to a non-zero limit ordinal
$\xi\le\kappa$, we can argue exactly as in part (d) of the proof of
331I;  $\frak C_{\xi}$ will be the metric closure of
$\frak C^*_{\xi}=\bigcup_{\eta<\xi}\frak C_{\eta}$, so we can take
$\pi_{\xi}:\frak C_{\xi}\to\frak B$ to be the unique measure-preserving
homomorphism extending $\pi^*_{\xi}=\bigcup_{\eta<\xi}\pi_{\eta}$.

Thus the induction proceeds, and evidently $\pi=\pi_{\kappa}$ will be a
measure-preserving homomorphism from $\frak A$ to $\frak B$ extending
$\phi$.

\medskip

{\bf (b)} (This is rather closer to the proof of 331I, being indeed a
direct generalization of it.)   Observe that the hypothesis (b) implies
that $1_{\frak A}$ is relatively \Mth\ over $\frak C$;  so
either $\kappa=0$, in which case $\frak A=\frak C$, $\frak B=\phi[\frak C]$
and the result is trivial, or $\kappa\ge\omega$, by 333Bd.   Let us
therefore take it that $\kappa$ is infinite.

We are supposing, among other things,
that $\tau_{\frak D}(\frak B)=\kappa$;  let
$\langle b_{\xi}\rangle_{\xi<\kappa}$ be a family in $\frak B$ such that
$\frak B$ is $\tau$-generated by $\frak D\cup\{b_{\xi}:\xi<\kappa\}$.
This time, as in 331I, we shall have to choose further families
$\langle a'_{\xi}\rangle_{\xi<\kappa}$ and
$\langle b'_{\xi}\rangle_{\xi<\kappa}$, and

\inset{$\frak C_{\xi}$ will be the closed subalgebra of $\frak A$
generated by

\Centerline{$\frak
C\cup\{a_{\eta}:\eta<\xi\}\cup\{a'_{\eta}:\eta<\xi\}$,}
}

\inset{$\frak D_{\xi}$ will be the closed subalgebra of $\frak B$
generated by

\Centerline{$\frak
D\cup\{b_{\eta}:\eta<\xi\}\cup\{b'_{\eta}:\eta<\xi\}$,}
}

\inset{$\pi_{\xi}:\frak C_{\xi}\to\frak D_{\xi}$ will be a
measure-preserving homomorphism.
}

\noindent The induction will start with $\frak C_0=\frak C$, $\frak
D_0=\frak D$ and $\pi_0=\phi$, as in (a).

\medskip

\quad{\bf (i)} For the inductive step to a successor ordinal $\xi+1$,
where $\xi<\kappa$, suppose that $\frak C_{\xi}$, $\frak D_{\xi}$ and
$\pi_{\xi}$ have been defined.

\medskip

\qquad\grheada\ Let $b\in\frak B\setminus\{0\}$.  Because

\Centerline{$\tau_{\frak D_b}(\frak B_b)
=\kappa>\#(\{b_{\eta}:\eta<\xi\}\cup\{b'_{\eta}:\eta<\xi\})$,}

\noindent $\frak B_b$ cannot be $\tau$-generated by
$\frak D_b\cup\{b\Bcap b_{\eta}:\eta<\xi\}
  \cup\{b\Bcap b'_{\eta}:\eta<\xi\}$, and
cannot be equal to $\{b\Bcap d:d\in\frak D_{\xi}\}$.   As $b$ is
arbitrary, there is an extension of $\pi_{\xi}$ to a
measure-preserving homomorphism $\phi_{\xi}$ from $\frak C'_{\xi}$ to
$\frak B$,
where $\frak C'_{\xi}$ is the closed subalgebra of $\frak A$ generated
by $\frak C\cup\{a_{\eta}:\eta\le\xi\}\cup\{a'_{\eta}:\eta<\xi\}$.
Setting $b'_{\xi}=\phi_{\xi}(a_{\xi})$, the image
$\frak D'_{\xi}=\phi_{\xi}[\frak C'_{\xi}]$ will be the closed subalgebra
of $\frak B$ generated by
$\frak D\cup\{b_{\eta}:\eta<\xi\}\cup\{b'_{\eta}:\eta\le \xi\}$.

\medskip

\qquad\grheadb\ Next, as in 331I, we must repeat the argument of
($\alpha$), applying it now to
$\phi_{\xi}^{-1}:\frak D_{\xi}\to\frak A$.   If
$a\in\frak A\setminus\{0\}$,

\Centerline{$\tau_{\frak C_a}(\frak A_a)
=\kappa>\#(\{a_{\eta}:\eta\le\xi\}\cup\{a'_{\eta}:\eta<\xi\})$,}

\noindent so that $\frak A_a$ cannot be
$\{a\Bcap c:c\in\frak C'_{\xi}\}$.   As $a$ is arbitrary,
$\phi_{\xi}^{-1}$ has an extension
to a measure-preserving homomorphism
$\psi_{\xi}:\frak D_{\xi+1}\to\frak C_{\xi+1}$, where $\frak D_{\xi+1}$
is the subalgebra of $\frak B$
generated by $\frak D'_{\xi}\cup\{b_{\xi}\}$, that is, the closed
subalgebra of $\frak B$ generated by
$\frak D\cup\{b_{\eta}:\eta\le\xi\}\cup\{b'_{\eta}:\eta\le\xi\}$, and
$\frak C_{\xi+1}$ is the subalgebra of $\frak A$ generated by
$\frak C'_{\xi}\cup\{a'_{\xi}\}$, setting
$a'_{\xi}=\psi_{\xi}(b_{\xi})$.

We can therefore take
$\pi_{\xi+1}=\psi_{\xi}^{-1}:\frak C_{\xi+1}\to\frak D_{\xi+1}$, as in
331I.

\medskip

\quad{\bf (ii)} The inductive step to a non-zero limit ordinal
$\xi\le\kappa$ is exactly the same as in (a) above or in 331I;
$\frak C_{\xi}$ is the metric closure of
$\frak C^*_{\xi}=\bigcup_{\eta<\xi}\frak C_{\eta}$,
$\frak D_{\xi}$ is the metric closure of $\frak
D^*_{\xi}=\bigcup_{\eta<\xi}\frak D_{\eta}$, and $\pi_{\xi}$ is the
unique measure-preserving homomorphism from $\frak C_{\xi}$ to $\frak
D_{\xi}$ extending every $\pi_{\eta}$ for $\eta<\xi$.

\medskip

\quad{\bf (iii)} The induction stops, as before, with
$\pi=\pi_{\kappa}:\frak C_{\kappa}\to\frak D_{\kappa}$, where $\frak
C_{\kappa}=\frak A$, $\frak D_{\kappa}=\frak B$.
}%end of proof of 333C

\leader{333D}{Corollary} Let $(\frak A,\bar\mu)$ and $(\frak B,\bar\nu)$
be
totally finite measure algebras and $\frak C$ a closed subalgebra of
$\frak A$.   Suppose that

\Centerline{$\tau(\frak C)<\max(\omega,\tau(\frak A))\le\min\{\tau(\frak
B_b):b\in\frak B\setminus\{0\}\}$.}

\noindent Then any measure-preserving Boolean homomorphism $\phi:\frak
C\to\frak B$ can be extended to a measure-preserving Boolean
homomorphism $\pi:\frak A\to\frak B$.

\proof{ Set $\kappa=\min\{\tau(\frak B_b):b\in\frak B\setminus\{0\}\}$.
Then for any non-zero $b\in\frak B$,

\Centerline{$\tau_{\phi[\frak C]_b}(\frak B_b)\ge\kappa$.}

\noindent\Prf\ There is a set $C\subseteq\frak C$, of cardinal
$\tau(\frak C)$, which $\tau$-generates $\frak C$, so that
$C'=\{b\Bcap\phi c:c\in C\}\,\,\tau$-generates $\phi[\frak C]_b$.   Now
there is a set $D\subseteq \frak B_b$, of cardinal $\tau_{\phi[\frak
C]_b}(\frak B_b)$, such that $\phi[\frak C]_b\cup D\,\,\tau$-generates
$\frak B_b$.   In this case $C'\cup D$ must $\tau$-generate $\frak B_b$,
so
$\kappa\le\#(C'\cup D)$.   But $\#(C')\le\#(C)<\kappa$ and $\kappa$ is
infinite, so we must have $\#(D)\ge\kappa$, as claimed.\ \Qed

On the other hand, $\tau_{\frak C}(\frak A)\le\tau(\frak A)\le\kappa$.
So we can apply 333Ca to give the result.
}%end of proof of 333D

\vleader{60pt}{333E}{Theorem} Let $(\frak C,\bar\mu)$ be a totally finite
measure algebra and $\kappa$ an infinite cardinal.
Let $(\frak A,\bar\lambda)$ be the localizable measure algebra free
product of $(\frak C,\bar\mu)$ and
$(\frak B_{\kappa},\bar\nu_{\kappa})$\cmmnt{ (notation:  333Ad)}, and
$\varepsilon:\frak C\to\frak A$ the corresponding homomorphism.   Then for any
non-zero $a\in\frak A$,

\Centerline{$\tau_{\varepsilon[\frak C]_a}(\frak A_a)=\kappa$\dvro{.}{,}}

\cmmnt{\noindent in the notation of 333A above.}

\proof{ Recall from 325Dd that $\varepsilon[\frak C]$ is a closed
subalgebra of $\frak A$.

\medskip

{\bf (a)} Let $\langle e_{\xi}\rangle_{\xi<\kappa}$ be the standard
generating family in $\frak B_{\kappa}$, corresponding to the sets
$\{x:x\in\{0,1\}^{\kappa}$, $x(\xi)=1\}$.
Let $\varepsilon':\frak B_{\kappa}\to\frak A$ be the
canonical map, and set $e'_{\xi}=\varepsilon'e_{\xi}$ for each $\xi$.

We know that $\{e_{\xi}:\xi<\kappa\}\,\,\tau$-generates
$\frak B_{\kappa}$ (see part (a) of the proof of 331K).   Consequently
$\varepsilon[\frak C]\cup\{e'_{\xi}:\xi<\kappa\}\,\,\tau$-generates
$\frak A$.
\Prf\ Let $\frak A_1$ be the closed subalgebra of $\frak A$ generated by
$\varepsilon[\frak C]\cup\{e'_{\xi}:\xi<\kappa\}$.   Because
$\varepsilon':\frak B_{\kappa}\to\frak A$ is
order-continuous (325Da),
$\varepsilon'[\frak B_{\kappa}]\subseteq\frak A_1$
(313Mb).   But this means that $\frak A_1$ includes
$\varepsilon[\frak  C]\cup\varepsilon'[\frak B_{\kappa}]$
and therefore includes the image of
$\frak C\otimes\frak B_{\kappa}$ in $\frak A$;  because this is
topologically dense in $\frak A$ (325Dc), $\frak A_1=\frak A$, as
claimed.\ \Qed

\medskip

{\bf (b)} It follows that

\Centerline{$\tau_{\varepsilon[\frak C]_a}(\frak A_a)
\le\tau_{\varepsilon[\frak C]}(\frak A)\le\kappa$}

\noindent (333Ba).

\medskip

{\bf (c)} We need to know that if $\xi<\kappa$ and $a$ belongs to the
closed subalgebra $\frak E_{\xi}$ of $\frak A$ generated by
$\varepsilon[\frak C]\cup\{e'_{\eta}:\eta\ne\xi\}$, then
$\bar\lambda(a\Bcap e'_{\xi})=\bover12\bar\lambda a$.   \Prf\ Set

\Centerline{$E=\varepsilon[\frak C]\cup\{e'_{\eta}:\eta\ne\xi\}$, \quad
$F=\{a_0\Bcap\ldots\Bcap a_n:a_0,\ldots,a_n\in E\}$.}

\noindent Then every member of $F$ is expressible in the form

\Centerline{$a=\varepsilon c\Bcap\inf_{\eta\in J}e'_{\eta}$,}

\noindent where $c\in\frak C$ and $J\subseteq\kappa\setminus\{\xi\}$ is
finite.   Now

\Centerline{$\bar\lambda a
=\bar\mu c\cdot\bar\nu(\inf_{\eta\in J}e_{\eta})
=2^{-\#(J)}\bar\mu c$,}

\Centerline{$\bar\lambda(e'_{\xi}\Bcap a)
=\bar\mu c\cdot\bar\nu(e_{\xi}\Bcap\inf_{\eta\in J}e_{\eta})
=2^{-\#(J\cup\{\xi\})}\bar\mu c=\Bover12\bar\lambda a$.}

\noindent Now consider the set

\Centerline{$G=\{a:a\in\frak A,\,\bar\lambda(e_{\xi}\Bcap a)
=\Bover12\bar\lambda a\}$.}

\noindent We have $1_{\frak A}\in F\subseteq G$, and $F$ is closed under
$\Bcap$.   Secondly, if $a$, $a'\in G$ and $a\Bsubseteq a'$, then

\Centerline{$\bar\lambda(e_{\xi}\Bcap(a'\Bsetminus a))
=\bar\lambda(e_{\xi}\Bcap a')-\bar\lambda(e_{\xi}\Bcap a)
=\Bover12\bar\lambda a'-\Bover12\bar\lambda a
=\Bover12\bar\lambda(a'\Bsetminus a)$,}

\noindent so $a'\Bsetminus a\in G$.   Also, if $H\subseteq G$ is
non-empty and upwards-directed,

\Centerline{$\bar\lambda(e_{\xi}\Bcap\sup H)
=\bar\lambda(\sup_{a\in H}e_{\xi}\Bcap a)
=\sup_{a\in H}\bar\lambda(e_{\xi}\Bcap a)
=\sup_{a\in H}\Bover12\bar\lambda a
=\Bover12\bar\lambda(\sup H)$,}

\noindent so $\sup H\in G$.   By the Monotone Class Theorem (313Gc), $G$
includes the order-closed
subalgebra of $\frak D$ generated by $F$.   But this is just
$\frak E_{\xi}$.\ \Qed

\medskip

{\bf (d)}  The next step is to see that
$\tau_{\varepsilon[\frak C]_a}(\frak A_a)>0$.   \Prf\ By (a) and 323J,
$\frak A$ is the metric closure of
the subalgebra $\frak A_0$ generated by
$\varepsilon[\frak C]\cup\{e'_{\eta}:\eta<\kappa\}$, so there must be an
$a_0\in\frak A_0$
such that $\bar\lambda(a_0\Bsymmdiff a)\le\bover14\bar\lambda a$.   Now
there is
a finite $J\subseteq\kappa$ such that $a_0$ belongs to the subalgebra
$\frak A_1$ generated by $\varepsilon[\frak C]\cup\{e'_{\eta}:\eta\in J\}$.
Take any $\xi\in\kappa\setminus J$ (this is where I use the hypothesis
that $\kappa$ is infinite).   If $c\in\frak C$, then by (c) we have

$$\eqalignno{\bar\lambda((a\Bcap\varepsilon c)\Bsymmdiff(a\Bcap e'_{\xi}))
&=\bar\lambda(a\Bcap(\varepsilon c\Bsymmdiff e'_{\xi}))
\ge\bar\lambda(a_0\Bcap(\varepsilon c \Bsymmdiff e'_{\xi}))
 -\bar\lambda(a\Bsymmdiff a_0)\cr
&=\bar\lambda(a_0\Bcap e'_{\xi})+\bar\lambda(a_0\Bcap \varepsilon c)
  -2\bar\lambda(a_0\Bcap \varepsilon c\Bcap e'_{\xi})
  -\bar\lambda(a\Bsymmdiff a_0)\cr
&=\Bover12\bar\lambda a_0-\bar\lambda(a\Bsymmdiff a_0)\cr
\noalign{\noindent (because both $a_0$ and $a_0\Bcap\varepsilon c$
belong to $\frak E_{\xi}$)}
&\ge\Bover12\bar\lambda a-\Bover32\bar\lambda(a\Bsymmdiff a_0)
>0.\cr}$$

\noindent Thus $a\Bcap e'_{\xi}$ is not of the form $a\Bcap\varepsilon c$
for any $c\in\frak C$, and $\frak A_a\ne\varepsilon[\frak C]_a$, so that
$\tau_{\varepsilon[\frak C]_a}(\frak A_a)>0$.\ \Qed

\medskip

{\bf (e)} It follows that $\tau_{\varepsilon[\frak C]_a}(\frak A_a)$ is
infinite.   \Prf\ There is a non-zero $d\Bsubseteq a$ which is
relatively \Mth\ over $\varepsilon[\frak C]$.   By (d), applied
to $d$, $\tau_{\varepsilon[\frak C]_d}(\frak A_d)>0$;  but now 333Bd tells us
that $\tau_{\varepsilon[\frak C]_d}(\frak A_d)$ must be infinite, so
$\tau_{\varepsilon[\frak C]_a}(\frak A_a)$ is infinite.\ \Qed

\medskip

{\bf (f)} If $\kappa=\omega$, we can stop here.   If $\kappa>\omega$, we
continue, as follows.   Let $D\subseteq\frak A_a$ be any set of cardinal
less than $\kappa$.   Each $d\in D\cup\{a\}$ belongs to the closed
subalgebra of $\frak A$ generated by
$C=\varepsilon[\frak C]\cup\{e'_{\xi}:\xi<\kappa\}$.
But because $\frak A$ is
ccc, this is just the $\sigma$-subalgebra of $\frak A$ generated by
$C$ (331Ge).   So $d$ belongs to the closed subalgebra of $\frak A$
generated by some countable subset $C_d$ of $C$, by 331Gd.   Now
$J_d=\{\eta:e'_{\eta}\in C_d\}$ is countable.   Set
$J=\bigcup_{d\in D\cup\{a\}}J_d$;  then

\Centerline{$\#(J)\le\max(\omega,\#(D\cup\{a\}))
=\max(\omega,\#(D))<\kappa$,}

\noindent so $J\ne\kappa$, and there is a $\xi\in\kappa\setminus J$.
Accordingly $\varepsilon[\frak C]\cup D\cup\{a\}$ is included in
$\frak E_{\xi}$, as defined in (c) above, and
$\varepsilon[\frak C]_a\cup D\subseteq\frak E_{\xi}$.   As
$\frak A_a\cap\frak E_{\xi}$ is a closed
subalgebra of $\frak A_a$, it includes the closed subalgebra
generated by $\varepsilon[\frak C]_a\cup D$.   But $a\Bcap e'_{\xi}$ surely
does not belong to $\frak E_{\xi}$, since

\Centerline{$\bar\lambda(a\Bcap e'_{\xi}\Bcap e'_{\xi})
=\bar\lambda(a\Bcap e'_{\xi})=\Bover12\bar\lambda a>0$,}

\noindent and $\bar\lambda(a\Bcap e'_{\xi}\Bcap
e'_{\xi})\ne\bover12\bar\lambda(a\Bcap e'_{\xi})$.   Thus
$a\Bcap e'_{\xi}$
cannot belong to the closed subalgebra of $\frak A_a$ generated by
$\varepsilon[\frak C]_a\cup D$, and $\varepsilon[\frak C]_a\cup D$ does not
$\tau$-generate $\frak A_a$.   As $D$ is arbitrary,
$\tau_{\phi[\frak C]_a}(\frak A_a)\ge\kappa$.

This completes the proof.
}%end of proof of 333E

\leader{333F}{Corollary} Let $(\frak A,\bar\mu)$ be a totally finite
measure algebra, $\frak C$ a closed subalgebra of $\frak A$ and $\kappa$
an infinite cardinal.

(a) Suppose that $\kappa\ge\tau_{\frak C}(\frak A)$.   Let
$(\frak C\widehat{\otimes}\frak B_{\kappa},\bar\lambda)$ be the localizable
measure algebra free product of $(\frak C,\bar\mu\restrp\frak C)$ and
$(\frak B_{\kappa},\bar\nu_{\kappa})$, and
$\varepsilon:\frak C\to\frak C\widehat{\otimes}\frak B_{\kappa}$ the
corresponding homomorphism.
Then there is a measure-preserving Boolean homomorphism
$\pi:\frak A\to\frak C\widehat{\otimes}\frak B_{\kappa}$ extending
$\varepsilon$.

(b) Suppose further that $\kappa=\tau_{\frak C_a}(\frak A_a)$
for every non-zero $a\in\frak A$.   Then $\pi$ can be
taken to be an isomorphism.

\proof{ All we have to do is apply 333C with
$\frak B=\frak C\widehat{\otimes}\frak B_{\kappa}$, using 333E to see
that the hypothesis

\Centerline{$\tau_{\varepsilon[\frak C]_b}(\frak B_b)=\kappa$ for every
non-zero $b\in\frak B$}

\noindent is satisfied.
}%end of proof of 333F

\leader{333G}{Corollary} Let $(\frak C,\bar\mu)$ be a
totally finite measure algebra.
Suppose that $\kappa\ge\max(\omega,\tau(\frak C))$ is
a cardinal.   Let
$(\frak C\widehat{\otimes}\frak B_{\kappa},\bar\lambda)$ be the
localizable measure algebra free product of $(\frak C,\bar\mu)$ and
$(\frak B_{\kappa},\bar\nu_{\kappa})$.   Then

(a) $\frak C\widehat{\otimes}\frak B_{\kappa}$ is \Mth,
with Maharam type $\kappa$ if $\frak C\ne\{0\}$;

(b) for every measure-preserving Boolean homomorphism
$\phi:\frak C\to\frak C$ there is a measure-preserving automorphism
$\pi:\frak C\widehat{\otimes}\frak B_{\kappa}
\to\frak C\widehat{\otimes}\frak B_{\kappa}$ such that
$\pi(c\otimes 1)=\phi c\otimes 1$ for every
$c\in\frak C$, writing $c\otimes 1$ for the canonical image in
$\frak C\widehat{\otimes}\frak B_{\kappa}$ of any $c\in\frak C$.

\proof{ Write $\frak A$ for $\frak C\widehat{\otimes}\frak B_{\kappa}$,
as in 333E, and $\frak D$ for $\{c\otimes 1:c\in\frak C\}\subseteq\frak A$.

\medskip

{\bf (a)} If $C\subseteq\frak C$ is a set of cardinal $\tau(\frak C)$
which $\tau$-generates $\frak C$, and $B\subseteq\frak B_{\kappa}$ a set
of cardinal $\kappa$ which $\tau$-generates $\frak B_{\kappa}$ (331K),
then $\{c\otimes b:c\in C,\,b\in B\}$ is a set of cardinal at most
$\max(\omega,\tau(\frak C),\kappa)=\kappa$ which $\tau$-generates
$\frak A$ (because the subalgebra it generates is topologically dense in
$\frak A$, by 325Dc).   So $\tau(\frak A)\le\kappa$.   On the other
hand, if $a\in\frak A$ is non-zero, then
$\tau(\frak A_a)\ge\tau_{\frak D_a}(\frak A_a)\ge\kappa$, by 333E;
so $\frak A$ is \Mth, with Maharam type $\kappa$ unless $\frak C=\{0\}$.

\medskip

{\bf (b)} We have a
measure-preserving automorphism $\phi_1:\frak D\to\frak D$ defined by
setting $\phi_1(c\otimes 1)=\phi c\otimes 1$ for every $c\in\frak C$.
Because $\phi_1[\frak D]\subseteq\frak D$, 333Be and 333E tell us that

\Centerline{$\kappa=\tau(\frak A_a)\ge\tau_{\phi_1[\frak D]_a}(\frak
A_a)\ge\tau_{\frak D_a}(\frak A_a)=\kappa$}

\noindent for every non-zero $a\in\frak A$, so we can use 333Cb, with
$\frak B=\frak A$, to see that $\phi_1$ can be extended to a
measure-preserving automorphism on $\frak A$.
}%end of proof of 333G

\vleader{72pt}{333H}{}\cmmnt{ I turn now to the classification of closed
subalgebras.

\medskip

\noindent}{\bf Theorem} Let $(\frak A,\bar\mu)$ be a localizable measure
algebra and $\frak C$ a closed subalgebra of $\frak A$.   Then there are
$\langle\mu_i\rangle_{i\in I}$, $\langle c_i\rangle_{i\in I}$,
$\langle\kappa_i\rangle_{i\in I}$ such that

\inset{for each $i\in I$, $\mu_i$ is a non-negative completely additive
functional on $\frak C$,

\inset{$c_i=\Bvalue{\mu_i>0}\in\frak C$,}

\inset{$\kappa_i$ is $0$ or an infinite cardinal,}

\inset{$(\frak C_{c_i},\mu_i\restrp\frak C_{c_i})$ is
a totally finite measure algebra, writing $\frak C_{c_i}$ for the
principal ideal of $\frak C$ generated by $c_i$,}}

\inset{$\sum_{i\in I}\mu_ic=\bar\mu c$ for every $c\in\frak C$,}

\inset{there is a measure-preserving isomorphism $\pi$ from $\frak A$ to
the simple product
$\prod_{i\in I}\frak C_{c_i}\widehat{\otimes}\frak B_{\kappa_i}$ of the
localizable measure algebra free products
$\frak C_{c_i}\widehat{\otimes}\frak B_{\kappa_i}$ of
$(\frak C_{c_i},\mu_i\restrp\frak C_{c_i})$ and
$(\frak B_{\kappa_i},\bar\nu_{\kappa_i})$.}

\noindent Moreover, $\pi$ may be taken such that

\inset{for every $c\in\frak C$, $\pi c=\langle (c\Bcap c_i)\otimes
1\rangle_{i\in I}$, writing $c\otimes 1$ for the image in $\frak
C_{c_i}\widehat{\otimes}\frak B_{\kappa_i}$ of $c\in\frak C_{c_i}$.}

\cmmnt{\medskip

\noindent{\bf Remark} Recall that $\Bvalue{\mu_i>0}$ is that element of
$\frak C$ such that $\mu_ic>0$ whenever $c\in\frak C$ and
$0\ne c\Bsubseteq\Bvalue{\mu_i>0}$, $\mu_ic\le 0$ whenever $c\in\frak C$
and $c\Bcap\Bvalue{\mu_i>0}=0$ (326S).
}%end of comment

\proof{{\bf (a)} Let $A$ be the set of those elements of $\frak A$ which
are relatively \Mth\ over $\frak C$ (see 333Ac).      By
333Bb, $A$ is order-dense in $\frak A$ (compare part (a) of the proof of
332B), and consequently $A'=\{a:a\in A,\,\bar\mu a<\infty\}$ is
order-dense in $\frak A$.   So there is a partition of unity
$\langle a_i\rangle_{i\in I}$ in $\frak A$ consisting of members of $A'$
(313K).   For
each $i\in I$, set $\mu_ic=\bar\mu(a_i\Bcap c)$ for every $c\in\frak C$;
then $\mu_i$ is
non-negative, and it is completely additive by 327E.   Because
$\langle a_i\rangle_{i\in I}$ is a partition of unity in
$\frak A$,

\Centerline{$\bar\mu c=\sum_{i\in I}\bar\mu(c\Bcap a_i)=\sum_{i\in
I}\mu_ic$}

\noindent for every $c\in\frak C$.   Next, $(\frak
C_{c_i},\mu_i\restrp\frak C_{c_i})$
is a totally finite measure algebra.   \Prf\ $\frak C_{c_i}$ is a
Dedekind $\sigma$-complete Boolean algebra because $\frak C$ is.
$\mu_i\restrp\frak C_{c_i}$ is a non-negative countably additive
functional because $\mu_i$ is.   If $c\in\frak C_{c_i}$ and $\mu_ic=0$,
then $c=0$ by the choice of $c_i$.\ \Qed\   Note also that

\Centerline{$\bar\mu(a_i\Bsetminus c_i)=\mu_i(1\Bsetminus c_i)=0$,}

\noindent so that $a_i\Bsubseteq c_i$.

\medskip

{\bf (b)} By 333Bd, any finite $\kappa_i$ must actually be zero.   The
next element we need is the fact that, for each $i\in I$,
we have a measure-preserving isomorphism $c\mapsto c\Bcap a_i$ from
$(\frak C_{c_i},\mu_i\restrp\frak C_{c_i})$ to $(\frak
C_{a_i},\bar\mu\restrp\frak C_{a_i})$.   \Prf\ Of course this is a ring
homomorphism.   Because $a_i\Bsubseteq c_i$, it is a surjective Boolean
homomorphism.   It is
measure-preserving by the definition of $\mu_i$, and therefore
injective.\ \Qed

\medskip

{\bf (c)} Still focusing on a particular $i\in I$, let $\frak A_{a_i}$
be the principal ideal of $\frak A$ generated by $a_i$.   Then we have a
measure-preserving isomorphism
$\tilde\pi_i:\frak A_{a_i}
\to\frak C_{a_i}\widehat{\otimes}\frak B_{\kappa_i}$,
extending the canonical homomorphism
$c\mapsto c\otimes 1:
 \frak C_{a_i}\to\frak C_{a_i}\widehat{\otimes}\frak B_{\kappa_i}$.
\Prf\ When $\kappa_i$ is
infinite, this is just 333Fb.   But the only other case is
when $\kappa_i=0$, that is, $\frak C_{a_i}=\frak A_{a_i}$, while
$\frak B_{\kappa_i}=\{0,1\}$ and
$\frak C_{a_i}\widehat{\otimes}\frak B_{\kappa_i}\cong\frak C_{c_i}$.\
\Qed

The isomorphism between $(\frak C_{c_i},\mu_i\restrp\frak C_{c_i})$ and
$(\frak C_{a_i},\bar\mu\restrp\frak C_{a_i})$ induces an isomorphism
between
$\frak C_{c_i}\widehat{\otimes}\frak B_{\kappa_i}$ and
$\frak C_{a_i}\widehat{\otimes}\frak B_{\kappa_i}$.   So we have a
measure-preserving isomorphism
$\pi_i:\frak A_{a_i}\to\frak C_{c_i}\widehat{\otimes}\frak B_{\kappa_i}$
such that $\pi_i(c\Bcap a_i)=c\otimes 1$ for every $c\in\frak C_{c_i}$.

\medskip

{\bf (d)} By 322Ld, we have a measure-preserving isomorphism
$a\mapsto\langle a\Bcap a_i\rangle_{i\in I}:
\frak A\to\prod_{i\in I}\frak A_{a_i}$.
Putting this together with the isomorphisms of (c), we have a
measure-preserving isomorphism $\pi$ from $\frak A$ to
$\prod_{i\in I}\frak C_{c_i}\widehat{\otimes}\frak B_{\kappa_i}$,
setting $\pi a=\langle\pi_i(a\Bcap a_i)\rangle_{i\in I}$ for
$a\in\frak A$.   Observe that, for $c\in\frak C$,

\Centerline{$\pi c=\langle\pi_i(c\Bcap a_i)\rangle_{i\in I}
=\langle(c\Bcap c_i)\otimes 1\rangle_{i\in I}$,}

\noindent as required.
}%end of proof of 333H


\vleader{36pt}{333I}{Remarks\cmmnt{ (a)}}\dvro{ Whenever}{ I hope it is
clear that whenever} $(\frak C,\bar\mu)$ is a Dedekind complete measure
algebra, $\langle\mu_i\rangle_{i\in I}$ is a family of non-negative
completely additive functionals on $\frak C$ such that
$\sum_{i\in I}\mu_i=\bar\mu$, and $\langle\kappa_i\rangle_{i\in I}$ is a
family of
cardinals all infinite or zero, then the construction above can be
applied to give a measure algebra $(\frak A,\bar\lambda)$, the product
of the family
$\langle\frak C_{c_i}\widehat{\otimes}\frak B_{\kappa_i}\rangle_{i\in I}$,
together with an order-continuous
measure-preserving
homomorphism $\pi:\frak C\to\frak A$;  and\cmmnt{ that} the partition
of unity
$\langle a_i\rangle_{i\in I}$ in $\frak A$ corresponding to this
product\cmmnt{ (315E)} has $\mu_ic=\bar\lambda(a_i\Bcap\pi c)$ for
every $c\in\frak C$ and $i\in I$, while each principal ideal
$\frak A_{a_i}$ can be
identified with $\frak C_{c_i}\widehat{\otimes}\frak B_{\kappa_i}$, so
that $a_i$ is relatively \Mth\ over $\pi[\frak C]$.   Thus
any structure $(\frak C,\bar\mu,\langle\mu_i\rangle_{i\in
I},\langle\kappa_i\rangle_{i\in I})$ of the type described here
corresponds to an embedding of $\frak C$ as a closed subalgebra of
a localizable measure algebra.

\cmmnt{\medskip

{\bf (b)} The obvious next step is to seek a complete classification of
objects $(\frak A,\bar\mu,\frak C)$, where $(\frak A,\bar\mu)$ is a
localizable
measure algebra and $\frak C$ is a closed subalgebra, corresponding to
the classification of localizable measure algebras in terms of the
magnitudes of their Maharam-type-$\kappa$ components in 332J.   The
general case seems to be complex.   But I can deal with the special case
in which $(\frak A,\bar\mu)$ is totally finite.   In this case, we have
the following facts.
}

\vleader{48pt}{333J}{Lemma} Let $(\frak A,\bar\mu)$ be a totally finite measure
algebra, and $\frak C$ a closed subalgebra.   Let $A$ be the set of
relative atoms of $\frak A$ over $\frak C$.   Then there is
a unique sequence $\sequencen{\mu_n}$ of additive functionals on
$\frak C$ such that (i) $\mu_{n+1}\le\mu_n$ for every $n$ (ii) there is
a disjoint sequence $\sequencen{a_n}$ in $A$ such that
$\sup_{n\in\Bbb N}a_n=\sup A$ and $\mu_nc=\bar\mu(a_n\Bcap c)$ for every
$n\in\Bbb N$ and $c\in\frak C$.

\cmmnt{\medskip

\noindent{\bf Remark} I hope it is plain from my wording that it is the
${\mu_n}$ which are unique, not the $a_n$.
}

\proof{{\bf (a)} For each $a\in\frak A$ set $\theta_a(c)=\bar\mu(c\Bcap a)$
for $c\in\frak C$.   Then $\theta_a$ is a non-negative completely additive
real-valued functional on $\frak C$ (see 326Od).

The key step is I suppose in (c) below;  I approach by a two-stage
argument.   For each $b\in\frak A$ write $A^{\perp}_b$ for
$\{a:a\in A,\,a\Bcap b=0\}$.

\medskip

{\bf (b)} For every $b\in\frak A$ and non-zero $c\in\frak C$ there are
$a\in A^{\perp}_b$, $c'\in\frak C$ such that $0\ne c'\Bsubseteq c$ and
$\theta_a(d)\ge\theta_e(d)$ whenever $d\in\frak C$, $e\in A^{\perp}_b$ and
$d\Bsubseteq c'$.   \Prf\Quer\ Otherwise, choose $\sequencen{a_n}$ and
$\sequencen{c_n}$ as follows.   Since $0$, $c$ won't serve for $a$,
$c'$, there must be an $a_0\in A^{\perp}_b$ such that $\theta_{a_0}(c)>0$.   Let
$\delta>0$ be such that $\theta_{a_0}(c)>\delta\bar\mu c$ and set
$c_0=c\Bcap\Bvalue{\theta_{a_0}>\delta\bar\mu\restrp\frak C}$;  then
$c_0\in\frak C$ and $0\ne c_0\Bsubseteq c$.   Given that $a_n\in A^{\perp}_b$,
$c_n\in\frak C$ and $0\ne c_n\Bsubseteq c$, then there must be
$a_{n+1}\in A^{\perp}_b$, $d_n\in \frak C$ such that $d_n\Bsubseteq c_n$ and
$\theta_{a_{n+1}}(d_n)>\theta_{a_n}(d_n)$.   Set
$c_{n+1}=d_n\Bcap\Bvalue{\theta_{a_{n+1}}>\theta_{a_n}}$, so that
$c_{n+1}\in\frak C$ and $0\ne c_{n+1}\Bsubseteq c_n$, and continue.

There is some $n\in\Bbb N$ such that $n\delta\ge 1$.   For any $i<n$,
the construction ensures that

\Centerline{$0\ne c_{n+1}\Bsubseteq
c_{i+1}\Bsubseteq\Bvalue{\theta_{a_{i+1}}>\theta_{a_i}}$,}

\noindent so $\theta_{a_i}(c_{n+1})<\theta_{a_{i+1}}(c_{n+1})$;  also
$c_{n+1}\Bsubseteq c_0$ so

\Centerline{$\bar\mu(a_i\Bcap c_{n+1})
=\theta_{a_i}(c_{n+1})\ge\theta_{a_0}(c_{n+1})>\delta\bar\mu c_{n+1}$.}

\noindent But this means that
$\sum_{i=0}^{n-1}\bar\mu(a_i\Bcap c_{n+1})>\bar\mu c_{n+1}$ and there
must be distinct $j$, $k<n$ such that $a_j\Bcap a_k\Bcap c_{n+1}$ is non-zero.
Because $a_j$, $a_k\in A$ there are $d'$, $d''\in\frak C$ such that
$a_j\Bcap a_k=a_j\Bcap d'=a_k\Bcap d''$;  set $d=c_{n+1}\Bcap d'\Bcap
d''$, so that $d\in\frak C$ and

\Centerline{$a_j\Bcap d=a_j\Bcap a_k\Bcap c_{n+1}=a_k\Bcap d$,
\quad$\theta_{a_j}(d)=\bar\mu(a_j\Bcap a_k\Bcap c_{n+1})=\theta_{a_k}(d)$.}

\noindent  But as $0\ne d\Bsubseteq\Bvalue{\theta_{a_{i+1}}>\theta_{a_i}}$ for
every $i<n$, $\theta_{a_0}(d)<\theta_{a_1}(d)<\ldots<\theta_{a_n}(d)$, so this is
impossible.   \Bang\Qed

\medskip

{\bf (c)} Now for a global, rather than local, version of the same idea.
For every $b\in\frak A$ there is an $a\in A^{\perp}_b$ such that and
$\theta_a\ge\theta_e$ whenever $e\in A^{\perp}_b$.   \Prf\ (i) By (b), the set $C$
of those $c\in\frak C$ such that there is an $a\in A^{\perp}_b$ such that
$\theta_a\restrp\frak C_c\ge\theta_e\restrp\frak C_c$ for every $e\in A^{\perp}_b$ is
order-dense in $\frak C$.   Let $\langle c_i\rangle_{i\in I}$ be a
partition of unity in $\frak C$ consisting of members of $C$, and for
each $i\in I$ choose $a_i\in A^{\perp}_b$ such that $\theta_{a_i}\restrp\frak
C_{c_i}\ge\theta_e\restrp\frak C_{c_i}$ for every $e\in A^{\perp}_b$.   Consider
$a=\sup_{i\in I}a_i\Bcap c_i$.   (ii) If $a'\in\frak A$ and
$a'\Bsubseteq a$, then for each $i\in I$ there is a $d_i\in\frak C$ such
that $a_i\Bcap a'=a_i\Bcap d_i$.   Set $d'=\sup_{i\in I}c_i\Bcap d_i$;
then (because $\langle c_i\rangle_{i\in I}$ is disjoint)

\Centerline{$a\Bcap d'=\sup_{i\in I}a_i\Bcap c_i\Bcap d_i=\sup_{i\in
I}a_i\Bcap c_i\Bcap a'=a\Bcap a'=a'$.}

\noindent As $a'$ is arbitrary, this shows that $a\in A$.   (iii) Of
course $a\Bcap b=0$, so $a\in A^{\perp}_b$.   Now take any $e\in A^{\perp}_b$ and
$d\in\frak C$.   Then

\Centerline{$\theta_a(d)=\sum_{i\in I}\theta_{a_i}(c_i\Bcap d)\ge\sum_{i\in
I}\theta_e(c_i\Bcap d)=\theta_e(d)$.}

\noindent So this $a$ has the required property.\  \Qed

\medskip

{\bf (d)} Choose $\sequencen{a_n}$ inductively in $A$ so that, for each
$n$, $a_n\Bcap\sup_{i<n}a_i=0$ and $\theta_{a_n}\ge \theta_e$ whenever
$e\in A$ and $e\Bcap\sup_{i<n}a_i=0$.   Set $\mu_n=\theta_{a_n}$.   Because
$a_{n+1}\Bcap\sup_{i<n}a_i=0$, $\mu_{n+1}\le\mu_n$ for each $n$.   Also
$\sup_{n\in\Bbb N}a_n=\sup A$.   \Prf\  Take any $a\in A$ and set
$e=a\Bsetminus\sup_{n\in\Bbb N}a_n$.   Then $e\in A$ and, for any
$n\in\Bbb N$, $e\Bcap\sup_{i<n}a_i=0$, so $\theta_e\le\theta_{a_n}$ and

\Centerline{$\bar\mu e=\theta_e(1)\le\theta_{a_n}(1)=\bar\mu a_n$.}

\noindent   But as $\sequencen{a_n}$ is disjoint, this means that $e=0$,
that is, $a\Bsubseteq\sup_{n\in\Bbb N}a_n$.    As $a$ is arbitrary,
$\sup A\Bsubseteq\sup_{n\in\Bbb N}a_n$.\   \Qed

\medskip

{\bf (e)} Thus we have a sequence $\sequencen{\mu_n}$ of the required
type, witnessed by $\sequencen{a_n}$.   To see that it is unique,
suppose that $\sequencen{\mu'_n}$, $\sequencen{a'_n}$ are another pair
of sequences with the same properties.    Note first that if $c\in\frak
C$ and $0\ne c\Bsubseteq\Bvalue{\mu'_i>0}$ there is some $k\in\Bbb N$
such that $c\Bcap a'_i\Bcap a_k\ne 0$;  this is because
$\bar\mu(a'_i\Bcap c)=\mu'_i(c)>0$, so that $a'_i\Bcap c\ne 0$, while
$a'_i\Bsubseteq\sup A=\sup_{k\in\Bbb N}a_k$.    \Quer\  Suppose, if
possible, that there is some $n$ such that $\mu_n\ne\mu'_n$;  since the
situation is symmetric, there is no loss of generality in supposing that
$\mu'_n\not\le\mu_n$, that is, that $c=\Bvalue{\mu'_n>\mu_n}\ne 0$.
For any $i\le n$, $\mu'_i\ge\mu'_n$ so $c\Bsubseteq\Bvalue{\mu'_i>0}$.
We may therefore choose $c_0,\ldots, c_{n+1}\in\frak C_c\setminus\{0\}$
and $k(0),\ldots,k(n)\in\Bbb N$ such that $c_0=c$ and, for $i\le n$,

\Centerline{$c_i\Bcap a'_i\Bcap a_{k(i)}\ne 0$}

\noindent (choosing $k(i)$, recalling that $0\ne c_i\Bsubseteq
c\Bsubseteq\Bvalue{\mu'_i>0}$),

\Centerline{$c_{i+1}\in\frak C$,
\quad$c_{i+1}\Bsubseteq c_i$,
\quad$c_{i+1}\Bcap a'_i=c_{i+1}\Bcap a_{k(i)}=c_i\Bcap a'_i\Bcap a_{k(i)}$}

\noindent (choosing $c_{i+1}$, using the fact that $a'_i$ and $a_{k(i)}$
both belong to $A$ -- see the penultimate sentence in part (b) of the
proof).   On reaching $c_{n+1}$, we have $0\ne c_{n+1}\Bsubseteq c$ so
$\mu_n(c_{n+1})<\mu'_n(c_{n+1})$.   On the other hand, for each $i\le
n$,

\Centerline{$c_{n+1}\Bcap a'_i\Bcap a_{k(i)}
=c_{n+1}\Bcap c_{i+1}\Bcap a'_i\Bcap a_{k(i)}
=c_{n+1}\Bcap a'_i=c_{n+1}\Bcap a_{k(i)}$,}

so

\Centerline{$\mu_n(c_{n+1})<\mu'_n(c_{n+1})
\le\mu'_i(c_{n+1})=\bar\mu(c_{n+1}\Bcap a'_i)=\bar\mu(c_{n+1}\Bcap
a_{k(i)})=\mu_{k(i)}(c_{n+1})$,}

\noindent and $k(i)$ must be less than $n$.   There are therefore
distinct $i$, $j\le n$ such that $k(i)=k(j)$.   But in this case

\Centerline{$c_{n+1}\Bcap a'_i=c_{n+1}\Bcap a_{k(i)}=c_{n+1}\Bcap
a_{k(j)}=c_{n+1}\Bcap a'_j\ne 0$}

\noindent because $0\ne c_{n+1}\Bsubseteq\Bvalue{\mu'_j>0}$.   So
$a'_i$, $a'_j$ cannot be disjoint, breaking one of the rules of the
construction.\ \BanG\  Thus $\mu_n=\mu'_n$ for every $n\in\Bbb N$.

This completes the proof.
}%end of proof of 333J

\leader{333K}{Theorem} Let $(\frak A,\bar\mu)$ be a totally finite
measure algebra and $\frak C$ a closed subalgebra of $\frak A$.   Then
there are unique families $\sequencen{\mu_n}$,
$\langle\mu_{\kappa}\rangle_{\kappa\in K}$ such that

\inset{$K$ is a countable set of infinite cardinals,}

\inset{for $ i\in\Bbb N\cup K$, $\mu_i$ is a non-negative countably
additive functional on $\frak C$, and
$\sum_{i\in\Bbb N\cup K}\mu_ic=\bar\mu c$ for every $c\in\frak C$,}

\inset{$\mu_{n+1}\le\mu_n$ for every $n\in\Bbb N$, and
$\mu_{\kappa}\ne 0$ for $\kappa\in K$,}

\inset{setting $e_i=\Bvalue{\mu_i>0}\in\frak C$, and
giving the principal ideal $\frak C_{e_i}$ generated by $e_i$ the
measure $\mu_i\restrp\frak C_{e_i}$ for each $ i\in\Bbb N\cup K$,
we have a measure algebra isomorphism

\Centerline{$\pi:\frak A\to\prod_{n\in\Bbb N}\frak C_{e_n}
  \times\prod_{\kappa\in K}
    \frak C_{e_{\kappa}}\widehat{\otimes}\frak B_{\kappa}$}

\noindent such that

\Centerline{$\pi c
=(\sequencen{c\Bcap e_n},
  \langle(c\Bcap e_{\kappa})\otimes 1\rangle_{\kappa\in K})$}

\noindent for each $c\in C$, writing $c\otimes 1$ for the canonical
image in $C_{e_{\kappa}}\widehat{\otimes}\frak B_{\kappa}$ of
$c\in\frak C_{e_{\kappa}}$.}

\proof{{\bf (a)} I aim to use the construction of 333H, but taking much
more care over the choice of $\langle a_i\rangle_{i\in I}$ in part (a)
of the proof there.   We start by taking $\sequencen{a_n}$ as in 333J,
and setting $\mu_nc=\bar\mu(a_n\Bcap c)$ for every $n\in\Bbb N$,
$c\in\frak C$;  then these $a_n$ will deal with the relative atoms over
$\frak C$.

\medskip

{\bf (b)} The further idea required here concerns the treatment of
infinite $\kappa$.   Let $\langle b_i\rangle_{i\in I}$ be any partition
of unity in $\frak A$ consisting of non-zero members of $\frak A$ which
are relatively \Mth\ over $\frak C$, and
$\langle\kappa_i\rangle_{i\in I}$ the corresponding cardinals, so that
$\kappa_i=0$ iff $b_i$ is a relative atom.   Set
$I_1=\{i:i\in I,\,\kappa_i\ge\omega\}$.   Set $K=\{\kappa_i:i\in I_1\}$,
so that $K$
is a countable set of infinite cardinals, and for $\kappa\in K$ set
$J_{\kappa}=\{i:\kappa_i=\kappa\}$,
$a_{\kappa}=\sup_{i\in J_{\kappa}}b_i$ for $\kappa\in K$.   Now every
$a_{\kappa}$ is relatively \Mth\ over $\frak C$.   \Prf\ (Compare 332H.)
$J_{\kappa}$ must be countable, because $\frak A$ is ccc.   If
$0\ne a\Bsubseteq a_{\kappa}$, there is some $i\in J_{\kappa}$ such that
$a\Bcap b_i\ne 0$;  now

\Centerline{$\tau_{\frak C_a}(\frak A_a)
\ge\tau_{\frak C_{a\Bcap b_i}}(\frak A_{a\Bcap b_i})
=\kappa_i=\kappa$}

\noindent (333Ba).
At the same time, for each $i\in J_{\kappa}$, there is a set
$D_i\subseteq\frak A_{b_i}$ such that $\#(D_i)=\kappa$ and
$\frak C_{b_i}\cup D_i\,\,\tau$-generates $\frak A_{b_i}$.   Set
$D=\bigcup_{i\in J_{\kappa}}D_i\cup\{b_i:i\in J_{\kappa}\}$;  then

\Centerline{$\#(D)\le\max(\omega,\#(J_{\kappa}),\sup_{i\in
K}\#(D_i))=\kappa$.}

\noindent   Let $\frak B$ be the closed subalgebra of
$\frak A_{a_{\kappa}}$ generated by $\frak C_{a_{\kappa}}\cup D$.   Then

\Centerline{$\frak C_{b_i}\cup D_i\subseteq\{b\cap b_i:b\in \frak B\}
=\frak B\cap\frak A_{b_i}$}

\noindent so $\frak B\supseteq\frak A_{b_i}$ for each $i\in J_{\kappa}$,
and $\frak B=\frak A_{a_{\kappa}}$.   Thus $\frak C_{a_{\kappa}}\cup
D\,\,\tau$-generates $\frak A_{a_{\kappa}}$, and

\Centerline{$\tau_{\frak C_{a_{\kappa}}}(\frak A_{a_{\kappa}})
\le\kappa\le\min_{0\ne a\Bsubseteq a_{\kappa}}\tau_{\frak C_a}(\frak
A_a)$.}

\noindent This shows that $a_{\kappa}$ is relatively \Mth\
over $\frak C$, with
$\tau_{\frak C_{a_{\kappa}}}(\frak A_{a_{\kappa}})=\kappa$.\ \Qed

Since evidently $\langle J_{\kappa}\rangle_{\kappa\in K}$ and
$\langle a_{\kappa}\rangle_{\kappa\in K}$ are disjoint, and
$\sup_{\kappa\in K}a_{\kappa}=\sup_{i\in I_1}b_i$,
this process yields a partition
$\langle a_i\rangle_{i\in\Bbb N\cup K}$ of unity in $\frak A$.
Now the arguments of 333H show that we get an isomorphism $\pi$ of the
kind described.

\medskip

{\bf (c)} To see that the families $\sequencen{\mu_n}$,
$\langle\mu_{\kappa}\rangle_{\kappa\in K}$ (and therefore the $e_i$ and
the $(\frak C_{e_i},\mu_i\restrp\frak C_{e_i})$, but not $\pi$) are
uniquely defined, argue as follows.   Let $A$ be the set of
those $a\in\frak A$ which are relatively \Mth\ over $\frak C$.
Take  families
$\sequencen{\tilde\mu_n}$,
$\langle\tilde\mu_{\kappa}\rangle_{\kappa\in\tilde K}$
which correspond to an isomorphism

\Centerline{$\tilde\pi:\frak A\to\frak D
=\prod_{n\in\Bbb N}\frak C_{\tilde e_n}
  \times\prod_{\kappa\in\tilde K}
  \frak C_{\tilde e_{\kappa}}\widehat{\otimes}\frak B_{\kappa}$,}

\noindent writing $\tilde e_i=\Bvalue{\tilde\mu_i>0}$ for
$i\in\Bbb N\cup\tilde K$.   In the simple product
$\prod_{n\in\Bbb N}\frak C_{\tilde e_n}
  \times\prod_{\kappa\in \tilde K}
  \frak C_{\tilde e_{\kappa}}\widehat{\otimes}\frak B_{\kappa}$,
we have a partition of unity
$\langle e_i^*\rangle_{i\in\Bbb N\cup\tilde K}$ corresponding to the
product structure.   Now for $d\Bsubseteq e_i^*$, we have

$$\eqalign{\tau_{\tilde\pi[\frak C]_d}(\frak D_d)
&=0\text{ if } i\in\Bbb N,\cr
&=\kappa\text{ if } i=\kappa\in\tilde K.\cr}$$

\noindent So $\tilde K$ must be

\Centerline{$\{\kappa:\kappa\ge\omega,\,\Exists a\in A,
\,\tau_{\frak C_a}(\frak A_a)=\kappa\}=K$,}

\noindent and for $\kappa\in\tilde K$,

\Centerline{$\tilde\pi^{-1} e_{\kappa}^*
=\sup\{a:a\in A,\,\tau_{\frak C_a}(\frak A_a)=\kappa\}=a_{\kappa}$,}

\noindent so that $\tilde\mu_{\kappa}=\mu_{\kappa}$.   On the other hand,
$\sequencen{\tilde\pi^{-1}e_n^*}$ must be a disjoint sequence with
supremum $\sup A$, and the corresponding functionals $\tilde\mu_n$ are
supposed to form a non-increasing sequence, so must be equal to the
$\mu_n$ by 333J.
}%end of proof of 333K

\leader{333L}{Remark} Thus for the classification of structures
$(\frak A,\bar\mu,\frak C)$, where $(\frak A,\bar\mu)$ is a totally finite
measure algebra and $\frak C$ is a closed subalgebra, it will be enough
to classify objects $(\frak C,\bar\mu,\sequencen{\mu_n},
\langle\mu_{\kappa}\rangle_{\kappa\in K})$, where

\inset{$(\frak C,\bar\mu)$ is a totally finite measure algebra,}

\inset{$\sequencen{\mu_n}$ is a non-increasing sequence of
non-negative countably
additive functionals on $\frak C$,}

\inset{$K$ is a countable set of infinite cardinals\cmmnt{ (possibly
empty)},}

\inset{$\langle\mu_{\kappa}\rangle_{\kappa\in K}$ is a family of
non-zero non-negative countably additive functionals on $\frak C$,}

\inset{$\sum_{n=0}^{\infty}\mu_n+\sum_{\kappa\in K}\mu_{\kappa}=\bar\mu$.}

\cmmnt{\noindent To do this we need the concept of `standard
extension' of a
countably additive functional on a closed subalgebra of a measure
algebra, treated in 327F-327G, together with the following idea.
}%end of comment

\vleader{48pt}{333M}{Lemma} Let $(\frak C,\bar\mu)$ be a
totally finite measure
algebra and $\langle\mu_i\rangle_{i\in I}$ a family of countably
additive functionals on $\frak C$.   For $i\in I$, $\alpha\in\Bbb R$
set $e_{i\alpha}=\Bvalue{\mu_i>\alpha\bar\mu}$\cmmnt{ (326T)}, and
let $\frak C_0$ be the closed subalgebra of $\frak C$
generated by $\{e_{i\alpha}:i\in I,\,\alpha\in\Bbb R\}$.   Write
$\Sigma$ for the
$\sigma$-algebra of subsets of $\BbbR^I$ generated by sets of the form
$E_{i\alpha}=\{x:x(i)>\alpha\}$ as $i$ runs over $I$ and $\alpha$ runs
over $\Bbb R$.   Then

(a) there is a measure $\mu$, with domain $\Sigma$, such that there
is a measure-preserving isomorphism
$\pi:\Sigma/\Cal N_{\mu}\to\frak C_0$ for which
$\pi E_{i\alpha}^{\ssbullet}=e_{i\alpha}$ for every $i\in I$ and
$\alpha\in\Bbb R$, writing $\Cal N_{\mu}$ for $\mu^{-1}[\{0\}]$;

(b) this formula determines both $\mu$ and $\pi$;

(c) for every $E\in\Sigma$ and $i\in I$, we have

\Centerline{$\mu_i\pi E^{\ssbullet}=\int_Ex(i)\mu(dx)$;}

(d) for every $i\in I$, $\mu_i$ is the standard extension of
$\mu_i\restrp\frak C_0$ to $\frak C$;

(e) for every $i\in I$, $\mu_i\ge 0$ iff $x(i)\ge 0$ for
$\mu$-almost every $x$;

(f) for every $i$, $j\in I$, $\mu_i\ge\mu_j$ iff $x(i)\ge x(j)$ for
$\mu$-almost every $x$;

(g) for every $i\in I$, $\mu_i=0$ iff $x(i)=0$ for $\mu$-almost
every $x$.

\proof{{\bf (a)} Express $(\frak C,\bar\mu)$ as the measure algebra of a
measure space $(Y,\Tau,\nu)$;  write $\phi:\Tau\to\frak C$ for the
corresponding homomorphism.   For each $i\in I$ let $f_i:Y\to\Bbb R$ be
a $\Tau$-measurable, $\nu$-integrable function such that
$\int_Hf_i=\mu_i\phi H$ for every $H\in\Tau$.   Define
$\psi:Y\to\BbbR^I$ by setting $\psi(y)=\langle f_i(y)\rangle_{i\in I}$;
then $\psi^{-1}[E_{i\alpha}]\in\Sigma$, and
$e_{i\alpha}=\phi(\psi^{-1}[E_{i\alpha}])$ for
every $i\in I$ and $\alpha\in\Bbb R$.   (See part (a) of the proof of
327F.)   So
$\{E:E\subseteq\BbbR^I,\,\psi^{-1}[E]\in\Tau\}$, which is a
$\sigma$-algebra of subsets of $\BbbR^I$, contains every $E_{i\alpha}$,
and therefore includes $\Sigma$;  that is, $\psi^{-1}[E]\in\Tau$ for
every $E\in\Sigma$.   Accordingly we may define $\mu$ by setting
$\mu E=\nu\psi^{-1}[E]$ for every $E\in\Sigma$, and $\mu$ will be a
measure on $\BbbR^I$ with domain $\Sigma$.   The Boolean homomorphism
$E\mapsto\phi\psi^{-1}[E]:\Sigma\to\frak C$ has kernel $\Cal N_{\mu}$,
so descends to a homomorphism $\pi:\Sigma/\Cal N_{\mu}\to\frak C$,
which is measure-preserving.   To see that
$\pi[\Sigma/\Cal N_{\mu}]=\frak C_0$, observe that because $\Sigma$ is
the $\sigma$-algebra generated by
$\{E_{i\alpha}:i\in I,\,\alpha\in\Bbb R\}$,
$\pi[\Sigma/\Cal N_{\mu}]$ must be the closed subalgebra of
$\frak C$ generated by
$\{\pi E_{i\alpha}^{\ssbullet}:i\in I,\,\alpha\in\Bbb R\}
=\{e_{i\alpha}:i\in I,\,\alpha\in\Bbb R\}$, which is $\frak C_0$.

\medskip

{\bf (b)} Now suppose that $\mu'$, $\pi'$ have the same properties.
Consider

\Centerline{$\Cal A=\{E:E\in\Sigma,\,\pi E^{\ssbullet}=\pi'
E^{\smallcirc}\}$,}

\noindent where I write $E^{\ssbullet}$ for the equivalence class of $E$
in $\Sigma/\Cal N_{\mu}$, and $E^{\smallcirc}$ for the equivalence
class of $E$ in $\Sigma/\Cal N_{\mu'}$.   Then $\Cal A$ is a
$\sigma$-subalgebra of $\Sigma$, because $E\mapsto\pi E^{\ssbullet}$,
$E\mapsto \pi'E^{\smallcirc}$ are both sequentially order-continuous
Boolean homomorphisms, and contains every $E_{i\alpha}$, so must be the
whole
of $\Sigma$.   Consequently

\Centerline{$\mu E=\bar\mu\pi
E^{\ssbullet}=\bar\mu\pi'E^{\smallcirc}=\mu'E$}

\noindent for every $E\in\Sigma$, and $\mu'=\mu$;  it follows at
once that $\pi'=\pi$.   So $\mu$ and $\pi$ are uniquely determined.

\medskip

{\bf (c)} If $E\in\Sigma$ and $i\in I$,

$$\eqalignno{\int_Ex(i)\mu(dx)
&=\int x(i)\chi E(x)\mu(dx)
=\int\psi(y)(i)\chi E(\psi(y))\nu(dy)\cr
\noalign{\noindent (applying 235G\formerly{2{}35I}
to the inverse-measure-preserving function $\psi:Y\to\BbbR^I$)}
&=\int_{\psi^{-1}[E]}f_i(y)\nu(dy)\cr
\noalign{\noindent (by the definition of $\psi$)}
&=\mu_i\phi(\psi^{-1}[E])\cr
\noalign{\noindent (by the choice of $f_i$)}
&=\mu_i\pi E^{\ssbullet}\cr}$$

\noindent by the definition of $\pi$.

\medskip

{\bf (d)} For every $\alpha\in\Bbb R$,
$\Bvalue{\mu_i>\alpha\bar\mu}$ belongs to $\frak C_0$, so must be
equal to $\Bvalue{\mu_i\restrp\frak C_0>\alpha\bar\mu\restrp\frak C_0}$.
Thus $\mu_i$ is the standard extension of $\mu_i\restrp\frak C_0$ (327G).

\medskip

{\bf (e)-(g)} The point is that, because the standard-extension operator
is order-preserving (327F(b-ii)),

$$\eqalignno{\mu_i\ge 0
&\iff\mu_i\restrp\frak C_0\ge 0\cr
&\iff\int_Ex(i)\mu(dx)\ge 0\text{ for every }E\in\Sigma\cr
&\iff x(i)\ge 0\,\,\mu\text{-a.e.},\cr
\mu_i\ge\mu_j
&\iff\mu_i\restrp\frak C_0\ge \mu_j\restrp\frak C_0\cr
&\iff\int_Ex(i)\mu(dx)\ge\int_Ex(j)\mu(dx)\text{ for every
}E\in\Sigma\cr
&\iff x(i)\ge x(j)\,\,\mu\text{-a.e.},\cr
\mu_i= 0
&\iff\mu_i\restrp\frak C_0= 0\cr
&\iff\int_Ex(i)\mu(dx)= 0\text{ for every }E\in\Sigma\cr
&\iff x(i)= 0\,\,\mu\text{-a.e.}.\cr}$$
}%end of proof of 333M


\leader{333N}{A canonical form for closed subalgebras} We now have all
the elements required to describe a canonical form for structures

\Centerline{$(\frak A,\bar\mu,\frak C)$,}

\noindent where $(\frak A,\bar\mu)$ is a totally finite measure algebra
and
$\frak C$ is a closed subalgebra of $\frak A$.   The first step is the
matching of such structures with structures

\Centerline{$(\frak
C,\bar\mu,\sequencen{\mu_n},\langle\mu_{\kappa}\rangle_{\kappa\in K})$,}

\noindent where $(\frak C,\bar\mu)$ is a totally finite measure algebra,
$\sequencen{\mu_n}$ is a non-increasing sequence of
non-negative countably additive
functionals on $\frak C$, $K$ is a countable set of infinite cardinals,
$\langle\mu_{\kappa}\rangle_{\kappa\in K}$ is a family of non-zero
non-negative countably additive functionals on $\frak C$, and
$\sum_{n=0}^{\infty}\mu_n+\sum_{\kappa\in K}\mu_{\kappa}
=\bar\mu$\cmmnt{;  this is
the burden of 333K}.

Next, given any structure of this second kind, we have a corresponding
closed subalgebra $\frak C_0$ of $\frak C$, a measure $\mu$ on $\Bbb
R^I$, where $I=\Bbb N\cup K$, and an isomorphism $\pi$ from the measure
algebra $\frak C_0^*$ of $\mu$ to $\frak C_0$, all uniquely
defined from the family $\langle\mu_i\rangle_{i\in I}$\cmmnt{ by the
process of
333M}.   \cmmnt{For any $E$ belonging to the domain $\Sigma$ of $\mu$,
and
$i\in I$, we have

\Centerline{$\mu_i\pi E^{\ssbullet}=\int_Ex(i)\mu(dx)$}

\noindent (333Mc), so that} $\mu_i\restrp\frak C_0$ is fixed by $\pi$ and
$\mu$.   Moreover, the functionals $\mu_i$ can be recovered from their
restrictions to $\frak C_0$\cmmnt{ by the formulae of 327F
(333Md)}.   Thus
from $(\frak C,\bar\mu,\langle\mu_i\rangle_{i\in I})$ we are led, by a
canonical and reversible process, to the structure

\Centerline{$(\frak C,\bar\mu,\frak C_0,I,\mu,\pi)$.}

But the extension $\frak C$ of $\frak C_0=\pi[\frak C_0^*]$ can
be described, up to isomorphism, by the same process as before;  that
is, it corresponds to a sequence $\sequencen{\theta'_n}$ and a family
$\langle\theta'_{\kappa}\rangle_{\kappa\in L}$ of countably additive
functionals on $\frak C_0$ satisfying the conditions of 333K.   We can
transfer these to $\frak C_0^*$, where they correspond to
families $\sequencen{\theta_n}$,
$\langle\theta_{\kappa}\rangle_{\kappa\in L}$ of absolutely continuous
countably additive functionals defined on $\Sigma$, setting

\Centerline{$\theta_jE=\theta'_j\pi E^{\ssbullet}$}

\noindent for $E\in\Sigma$, $j\in\Bbb N\cup L$.   This process too is
reversible;  every absolutely continuous countably additive functional
$\nu$ on $\Sigma$ corresponds to countably additive functionals on
$\frak C_0^*$ and $\frak C_0$.   \cmmnt{Let me repeat that the
results of 327F mean that the whole structure
$(\frak C,\bar\mu,\langle\mu_i\rangle_{i\in I})$ can be recovered from
$(\frak C_0,\bar\mu\restrp\frak C_0,
\langle\mu_i\restrp\frak C_0\rangle_{i\in I})$ if we can get the
description of $(\frak C,\bar\mu)$ right, and that
the requirements $\mu_i\ge 0$, $\mu_n\ge\mu_{n+1}$, $\mu_{\kappa}\ne 0$,
$\sum_{i\in I}\mu_i=\bar\mu$ imposed in 333K will survive the process
(327F(b-iv)).
}

Putting all this together, a structure $(\frak A,\bar\mu,\frak C)$
leads, in
a canonical and (up to isomorphism) reversible way, to a structure

\Centerline{$(K,\mu,L,\langle\theta_j\rangle_{j\in\Bbb N\cup L})$}

\noindent such that

\inset{$K$ and $L$ are countable sets of infinite cardinals,}

\inset{$\mu$ is a totally finite measure on $\BbbR^I$, where
$I=\Bbb N\cup K$, and its domain $\Sigma$ is precisely the
$\sigma$-algebra of subsets of $\BbbR^I$ defined by the coordinate
functionals,}

\inset{for $\mu$-almost every $x\in\BbbR^I$ we have $x(i)\ge 0$ for
every $i\in I$, $x(n)\ge x(n+1)$ for every $n\in\Bbb N$ and $\sum_{i\in
I}x(i)=1$,}

\inset{for $\kappa\in K$, $\mu\{x:x(\kappa)>0\}>0$,}

\cmmnt{\noindent (these two clauses corresponding to the requirements
$\mu_i\ge 0$, $\mu_n\ge\mu_{n+1}$, $\sum_{i\in I}\mu_i=\bar\mu$,
$\mu_{\kappa}\ne 0$ -- see 333M(e)-(g))}

\inset{for $j\in J=\Bbb N\cup L$, $\theta_j$ is a non-negative
countably additive functional on $\Sigma$,}

\inset{$\theta_n\ge\theta_{n+1}$ for every $n\in\Bbb N$,
$\theta_{\kappa}\ne 0$ for every $\kappa\in L$, $\sum_{j\in
J}\theta_j=\mu$.}

\cmmnt{
\leader{333O}{Remark} I do not envisage quoting the result above very
often.   Indeed I do not claim that its final form adds anything to the
constituent results 333K, 327F and 333M.   I have taken the trouble
to spell it out, however, because it does not seem to me obvious that
the trail is going to end quite as quickly as it does.   We need to use
333K twice, but only twice.   The most important use of the ideas
expressed here, I suppose, is in constructing examples to strengthen our
intuition for the structures $(\frak A,\bar\mu,\frak C)$ under
consideration, and
I hope that you will experiment in this direction.}

\leader{333P}{}\cmmnt{ At the risk of trespassing on the province of
Chapter 38, I turn now to a special type of closed subalgebra, in which
there is a particularly elegant alternative form for a canonical
description.
The first step is an important result concerning automorphisms of
homogeneous probability algebras.

\medskip

\noindent}{\bf Proposition} Let $(\frak B,\bar\nu)$ be a homogeneous
probability algebra.   Then there is a measure-preserving automorphism
$\phi:\frak B\to\frak B$ such that

\Centerline{$\lim_{n\to\infty}\bar\nu(c\Bcap\phi^n(b))
=\bar\nu c\cdot\bar\nu b$}

\noindent for all $b$, $c\in\frak B$.

\proof{{\bf (a)} The case $\frak B=\{0,1\}$ is trivial ($\phi$ is, and
must be, the identity map) so we may take it that
$(\frak B,\bar\nu)=(\frak B_{\kappa},\bar\nu_{\kappa})$ for some infinite
cardinal $\kappa$ is an infinite cardinal.   Because
$\#(\kappa\times\Bbb Z)=\max(\omega,\kappa)=\kappa$, there must be a
permutation $\theta:\kappa\to\kappa$ such that every orbit of $\theta$ in
$\kappa$ is infinite (take $\theta$ to correspond to the bijection
$(\xi,n)\mapsto(\xi,n+1):\kappa\times\Bbb Z\to\kappa\times\Bbb Z$).
This induces a permutation
$\hat\theta:\{0,1\}^{\kappa}\to\{0,1\}^{\kappa}$ through the formula
$\hat\theta(x)=x\theta$ for every $x\in\{0,1\}^{\kappa}$, and of course
$\hat\theta$ is an automorphism of the measure space
$(\{0,1\}^{\kappa},\nu_{\kappa})$.   We therefore have a
corresponding automorphism $\phi$ of $\frak B$, setting
$\phi E^{\ssbullet}=(\hat\theta^{-1}[E])^{\ssbullet}$ for every $E$ in the
domain $\Tau_{\kappa}$ of $\nu_{\kappa}$.

\medskip


{\bf (b)} Let $\Cal E$ be the family of subsets $E$ of
$\{0,1\}^{\kappa}$ which are determined by coordinates in finite sets,
that is, are expressible in the form $E=\{x:x\restr J\in \tilde E\}$ for
some finite set $J\subseteq\kappa$ and some
$\tilde E\subseteq\{0,1\}^J$;  equivalently, expressible as a finite union of
basic cylinder sets $\{x:x\restr J=y\}$.   Then $\Cal E$ is a
subalgebra of $\Tau_{\kappa}$, so $\frak C=\{E^{\ssbullet}:E\in\Cal E\}$ is a
subalgebra of $\frak B$.

\medskip


{\bf (c)} Now if $b$, $c\in\frak C$, there is an $n\in\Bbb N$ such that
$\bar\nu(c\Bcap\phi^m(b))=\bar\nu c\cdot\bar\nu b$ for every $m\ge n$.
\Prf\ Express $b$, $c$ as $E^{\ssbullet}$, $F^{\ssbullet}$ where
$E=\{x:x\restr J\in\tilde E\}$, $F=\{x:x\restr K\in\tilde F\}$ and $J$,
$K$ are finite
subsets of $\kappa$.   For $\xi\in K$, all the $\theta^n(\xi)$ are
distinct, so only finitely many of them can belong to $J$;  as $K$ is
also finite, there is an $n$ such that $\theta^m[J]\cap K=\emptyset$ for
every $m\ge n$.   Fix $m\ge n$.   Then $\phi^m(b)=H^{\ssbullet}$ where

\Centerline{$H=\{x:x\theta^m\in E\}=\{x:x\theta^m\restr J\in\tilde E\}
=\{x:x\restr L\in\tilde H\}$,}

\noindent where $L=\theta^m[J]$ and
$\tilde H=\{z\theta^{-m}:z\in\tilde E\}$.   So
$\bar\nu(c\Bcap\phi^m(b))=\nu(F\cap H)$.   But $L$ and $K$ are disjoint,
because $m\ge n$, so $F$ and $H$ must be independent (cf.\ 272K), and

\Centerline{$\bar\nu(c\Bcap\phi^m(b))=\nu F\cdot\nu H
=\nu F\cdot\nu E=\bar\nu c\cdot\bar\nu b$,}

\noindent as claimed.\ \Qed

\medskip

{\bf (d)} Now recall that for every $E\in\Tau_{\kappa}$ and
$\epsilon>0$ there is an $E'\in\Cal E$ such that
$\nu(E\symmdiff E')\le\epsilon$ (254Fe).   So, given $b$, $c\in\frak B$
and $\epsilon>0$, we can
find $b'$, $c'\in\frak C$ such that $\bar\nu(b\Bsymmdiff b')\le\epsilon$
and $\bar\nu(c\Bsymmdiff c')\le\epsilon$, and in this case

$$\eqalign{\limsup_{n\to\infty}
&|\bar\nu(c\Bcap\phi^n(b))-\bar\nu c\cdot\bar\nu b|\cr
&\le\limsup_{n\to\infty}|
\bar\nu(c\Bcap\phi^n(b))-\bar\nu(c'\Bcap\phi^n(b'))|\cr
&\qquad\qquad\qquad\qquad
  +|\bar\nu(c'\Bcap\phi^n(b'))-\bar\nu c'\cdot\bar\nu b'|
  +|\bar\nu c\cdot\bar\nu b-\bar\nu c'\cdot\bar\nu b'|\cr
&=\limsup_{n\to\infty}|
\bar\nu(c\Bcap\phi^n(b))-\bar\nu(c'\Bcap\phi^n(b'))|
  +|\bar\nu c\cdot\bar\nu b-\bar\nu c'\cdot\bar\nu b'|\cr
&\le\limsup_{n\to\infty}\bar\nu(c\Bsymmdiff c')
  +\bar\nu(\phi^n(b)\Bsymmdiff\phi^n(b'))\cr
&\qquad\qquad\qquad\qquad
  +\bar\nu c|\bar\nu b-\bar\nu b'|+|\bar\nu c-\bar\nu c'|\bar\nu b'\cr
&\le\bar\nu(c\Bsymmdiff c')+\bar\nu(b\Bsymmdiff b')
  +\bar\nu c\cdot\bar\nu(b\Bsymmdiff b')
  +\bar\nu(c\Bsymmdiff c')\cdot\bar\nu b'
\le 4\epsilon.\cr}$$

\noindent As $\epsilon$ is arbitrary,

\Centerline{$\lim_{n\to\infty}\bar\nu(c\Bcap\phi^n(b))
=\bar\nu c\cdot\bar\nu b$,}

\noindent as required.
}%end of proof of 333P

\cmmnt{\medskip

\noindent{\bf Remark} Automorphisms of this type are called {\bf
mixing} (see 372O below).
}

\leader{333Q}{Corollary} Let $(\frak C,\bar\mu_0)$ be a totally finite
measure algebra and $(\frak B,\bar\nu)$ a probability algebra which is
{\it either} homogeneous {\it or} purely atomic with finitely many atoms
all of the same measure.   Let $(\frak A,\bar\mu)$ be the localizable
measure algebra free product of $(\frak C,\bar\mu_0)$ and
$(\frak B,\bar\nu)$. Then
there is a measure-preserving automorphism $\pi:\frak A\to\frak A$ such
that

\Centerline{$\{a:a\in\frak A,\,\pi a=a\}=\{c\otimes 1:c\in\frak C\}$.}

\medskip

\cmmnt{
\noindent{\bf Remark} I am following 315N in using the notation
$c\otimes b$ for the intersection in $\frak A$ of the canonical images
of $c\in\frak C$ and $b\in\frak B$.   By 325D(c-i) I need not distinguish
between the free product $\frak C\otimes\frak B$ and its image in
$\frak A$.
}

\proof{ Set $\gamma=\bar\mu 1=\bar\mu_01$.

\medskip

{\bf (a)} Let me deal with the case of atomic $\frak B$ first.
In this case, if $\frak B$ has $n+1$ atoms $b_0,\ldots,b_n$, let
$\phi:\frak B\to\frak B$ be the measure-preserving homomorphism
cyclically permuting these atoms, so that
$\phi b_0=b_1,\ldots,\phi b_n=b_0$.   Because $\phi$ is an
automorphism of $(\frak B,\bar\nu)$, it
induces an automorphism $\pi$ of $(\frak A,\bar\mu)$;  any member of
$\frak A$ is uniquely expressible as $a=\sup_{i\le n}c_i\otimes b_i$, and
now $\pi a=\sup_{i\le n}c_i\otimes b_{i+1}$, if we set $b_{n+1}=b_0$.   So
$\pi a=a$ iff
$c_i=c_{i+1}$ for $i<n$ and $c_n=c_0$, that is, iff all the $c_i$
are the same and $a=\sup_{i\le n}c\otimes b_i=c\otimes 1$ for some
$c\in\frak C$.

\medskip

{\bf (b)} If $\frak B$ is homogeneous, then take a mixing
measure-preserving
automorphism $\phi:\frak B\to\frak B$ as described in 333P.   As in (a),
this corresponds to an automorphism $\pi$ of $\frak A$, defined by
saying that $\pi(c\otimes b)=c\otimes\phi(b)$ for every $c\in\frak C$,
$b\in\frak A$.   Of course $\pi(c\otimes 1)=c\otimes 1$ for every
$c\in\frak C$.

Now suppose that $a\in\frak A$ and $\pi a=a$;  I need to show that
$a\in\frak C_1=\{c\otimes 1:c\in\frak C\}$.
Take any $\epsilon\in\ocint{0,\bover14}$.
We know that $\frak C\otimes\frak B$ is topologically dense in $\frak A$
(325Dc), so there is an $a'\in\frak C\otimes\frak B$ such that
$\bar\mu(a\Bsymmdiff a')\le\epsilon^2$.   Express $a'$ as
$\sup_{i\in I}c_i\otimes b_i$, where $\langle c_i\rangle_{i\in I}$ is a
finite partition of unity in $\frak C$ (315Oa).   Then

\Centerline{$\pi a'=\sup_{i\in I}c_i\otimes\phi(b_i)$,
\quad$\pi^n(a')=\sup_{i\in I}c_i\otimes\phi^n(b_i)$ for every
$n\in\Bbb N$.}

\noindent So we can get a formula for

$$\eqalign{\lim_{n\to\infty}\bar\mu(a'\Bcap\pi^n(a'))
&=\lim_{n\to\infty}\bar\mu(\sup_{i\in
I}c_i\otimes(b_i\Bcap\phi^n(b_i)))\cr
&=\lim_{n\to\infty}\sum_{i\in
I}\bar\mu_0c_i\cdot\bar\nu(b_i\Bcap\phi^n(b_i))
=\sum_{i\in I}\bar\mu_0c_i\cdot(\bar\nu b_i)^2.\cr}$$

\noindent It follows that

$$\eqalign{\sum_{i\in I}\bar\mu_0c_i\cdot(\bar\nu b_i)^2
&=\lim_{n\to\infty}\bar\mu(a'\Bcap\pi^n(a'))\cr
&\ge\limsup_{n\to\infty}\bar\mu(a\Bcap\pi^n(a))
  -\bar\mu(a\Bsymmdiff a')-\bar\mu(\pi^n(a)\Bsymmdiff\pi^n(a'))\cr
&=\bar\mu a-2\bar\mu(a\Bsymmdiff a')
\ge\bar\mu a'-3\bar\mu(a\Bsymmdiff a')
\ge\sum_{i\in I}\bar\mu_0c_i\cdot\bar\nu b_i-3\epsilon^2,\cr}$$

\noindent that is,

\Centerline{$\sum_{i\in I}
 \bar\mu_0c_i\cdot\bar\nu b_i\cdot(1-\bar\nu b_i)
\le 3\epsilon^2$.}

\noindent But this means that, setting
$J=\{i:i\in I,\,\bar\nu b_i\cdot(1-\bar\nu b_i)\ge\epsilon\}$, we must have
$\sum_{i\in J}\bar\mu_0c_i\le 3\epsilon$.   Set

\Centerline{$K=\{i:i\in I,\,\bar\nu b_i\ge 1-2\epsilon\}$,
\quad$L=\{i:i\in I\setminus K,\,\bar\nu b_i\le 2\epsilon\}$,
\quad$c=\sup_{i\in K}c_i$.}

\noindent   Then $I\setminus(K\cup L)\subseteq J$, so

$$\eqalign{\bar\mu(a'\Bsymmdiff(c\otimes 1))
&=\sum_{i\in I\setminus K}\bar\mu_0c_i\cdot\bar\nu b_i+\sum_{i\in
K}\bar\mu_0c_i\cdot(1-\bar\nu
b_i)\cr
&\le\sum_{i\in J}\bar\mu_0c_i\cdot\bar\nu b_i+\sum_{i\in
L}\bar\mu_0c_i\cdot\bar\nu b_i+\sum_{i\in
K}\bar\mu_0c_i\cdot(1-\bar\nu b_i)\cr
&\le\sum_{i\in J}\bar\mu_0c_i+2\epsilon\sum_{i\in
L}\bar\mu_0c_i+2\epsilon\sum_{i\in K}\bar\mu_0c_i
\le 3\epsilon+2\epsilon\gamma,\cr}$$

\noindent and

\Centerline{$\bar\mu(a\Bsymmdiff(c\otimes 1))
\le\epsilon^2+3\epsilon+2\epsilon\gamma$.}

As $\epsilon$ is arbitrary, $a$ belongs to the topological closure of
$\frak C_1$.   But of course $\frak C_1$ is a closed subalgebra of
$\frak A$ (325Dd), so must actually contain $a$.

As $a$ is arbitrary, $\pi$ has the required property.
}%end of proof of 333Q

\leader{333R}{}\cmmnt{ Now for the promised special type of closed
subalgebra.   It will be convenient to have the following temporary
notation.}   For an integer
$n\ge 1$, I will\cmmnt{ (for this paragraph only)}
write $\frak B_n$ for the power set of
$\{0,\ldots,n\}$ and set $\bar\nu_nb=\bover1{n+1}\#(b)$ for
$b\in\frak B_n$.    \cmmnt{(The natural interpretation of
$(\frak B_n,\bar\nu_n)$ as defined in
333Ad corresponds to $(\frak B_{2^{n+1}-1},\bar\nu_{2^{n+1}-1})$ here,
so we have a match if $n=0$.)}

\medskip

\noindent{\bf Theorem} Let $(\frak A,\bar\mu)$ be a totally finite
measure algebra and $\frak C$ a subset of $\frak A$.
Then the following are equiveridical:

(i) there is some set $G$ of measure-preserving automorphisms of
$\frak A$ such that

\Centerline{$\frak C=\{c:c\in\frak A,\,\pi c=c$ for every $\pi\in G\}$;}

(ii) $\frak C$ is a closed subalgebra of $\frak A$ and there is a
partition of unity $\langle e_i\rangle_{i\in I}$ in $\frak C$, where $I$
is a countable set of cardinals, such that $\frak A$ is
isomorphic to $\prod_{i\in I}\frak C_{e_i}\widehat{\otimes}\frak B_i$,
writing $\frak C_{e_i}$ for the principal ideal of $\frak C$ generated
by $e_i$ and endowed with $\bar\mu\restrp\frak C_{e_i}$, and
$\frak C_{e_i}\widehat{\otimes}\frak B_i$ for the localizable measure
algebra
free product of $\frak C_{e_i}$ and $\frak B_i$ -- the isomorphism being
one which takes any $c\in\frak C$ to
$\langle (c\Bcap e_i)\otimes 1\rangle_{i\in I}$\cmmnt{, as in 333H and
333K};

(iii) there is a single measure-preserving automorphism $\pi$ of
$\frak A$ such that

\Centerline{$\frak C=\{c:c\in\frak A,\,\pi c=c\}$.}

\proof{{\bf (a)(i)$\Rightarrow$(ii)($\pmb{\alpha}$)} $\frak C$ is a
subalgebra because every $\pi\in G$ is a Boolean homomorphism, and it is
order-closed because every $\pi$ is order-continuous (324Kb).   (Or,
if you prefer, $\frak C$ is topologically closed because every $\pi$ is
continuous.)

\medskip

\quad\grheadb\ Because $\frak C$ is a closed subalgebra of $\frak A$,
its embedding can be described in terms of families $\sequencen{\mu_n}$,
$\langle\mu_{\kappa}\rangle_{\kappa\in K}$ as in Theorem 333K.   Set
$I=K\cup\Bbb N$.    Recall that each $\mu_i$ is defined by setting
$\mu_ic=\bar\mu(c\Bcap a_i)$, where $\langle a_i\rangle_{i\in I}$ is a
partition of unity in $\frak A$ (see the proofs of 333H and 333K).
For $\kappa\in K$, $a_{\kappa}$ is the maximal element of $\frak A$
which is relatively \Mth\ over $\frak C$ with relative
Maharam type $\kappa$ (part (b) of the proof of 333K).   Consequently we
must have $\pi a_{\kappa}=a_{\kappa}$ for any measure algebra
automorphism of $(\frak A,\bar\mu)$ which leaves $\frak C$ invariant;
in particular, for every $\pi\in G$.   Thus $a_{\kappa}\in \frak C$ for
every $\kappa\in K$.

\medskip

\quad\grheadc\
Now consider the relatively atomic part of $\frak A$.   The elements
$a_n$, for $n\in\Bbb N$, are not uniquely defined.   However, the
functionals $\mu_n$ and their supports $e'_n=\Bvalue{\mu_n>0}$
are uniquely defined from the structure $(\frak A,\bar\mu,\frak C)$
and therefore invariant under $G$.
Since

$$\eqalign{e'_n
&=1\Bsetminus\sup\{c:c\in\frak C,\,\mu_nc\le 0\}\cr
&=1\Bsetminus\sup\{c:c\in\frak C,\,c\Bcap a_n=0\}
=\inf\{c:c\in\frak C,\,c\Bsupseteq a_n\},\cr}$$

\noindent and
$\sup_{n\in\Bbb N}a_n=1\Bsetminus\sup_{\kappa\in K}a_{\kappa}$ belongs
to $\frak C$, while
$e'_n\supseteq e'_{n+1}$ for every $n$, we must have
$e'_0=\sup_{n\in\Bbb N}a_n$.

Let $G^*$ be the set of all those automorphisms $\pi$ of the measure
algebra $(\frak A,\bar\mu)$ such that $\pi c=c$ for every $c\in\frak C$.
Then of course $G^*$ is a group including $G$.   Now $\sup_{\pi\in
G^*}\pi a_n$ must be invariant under every member of $G^*$, so belongs
to $\frak C$;  it includes $a_n$ and is included in any member of $\frak
C$ including $a_n$, so must be $e'_n$.

\medskip

\quad\grheadd\ I claim now that if $n\in\Bbb N$ then
$e'_n\Bcap\Bvalue{\mu_0>\mu_n}=0$.   \Prf\Quer\   Otherwise, set
$c=\Bvalue{\mu_0>\mu_n}\Bcap e'_n$.   Then $\mu_0c>0$ so
$c\Bcap a_0\ne 0$.   By the last remark in ($\gamma$), there is a
$\pi\in G^*$ such
that $c\Bcap a_0\Bcap\pi a_n\ne 0$.   Now there is a $c'\in\frak C$ such
that $c\Bcap a_0\Bcap\pi a_n=c'\Bcap a_0$, and of course we may suppose
that $c'\Bsubseteq c$.   But this means that

\Centerline{$\pi(c'\Bcap a_n)
=c'\Bcap\pi a_n\Bsupseteq c'\Bcap a_0\Bcap\pi a_n= c'\Bcap a_0$,}

\noindent so that

\Centerline{$\mu_nc'=\bar\mu(c'\Bcap a_n)
=\bar\mu\pi(c'\Bcap a_n)
\ge\bar\mu(c'\Bcap a_0)
=\mu_0c'$,}

\noindent which is impossible, because
$0\ne c'\Bsubseteq\Bvalue{\mu_0>\mu_n}$.\ \Bang\Qed

So $\mu_0c\le\mu_nc$ whenever $c\in\frak C$ and $c\Bsubseteq e'_n$.
Because the $\mu_k$ have been chosen to be a non-increasing sequence, we
must have $\mu_0c=\mu_1c=\ldots=\mu_nc$ for every $c\Bsubseteq e'_n$.

\medskip

\quad\grheade\ Recalling now that
$\sum_{i\in I}\mu_i=\bar\mu\restrp\frak C$, we see that
$\mu_0c\le\bover1{n+1}\bar\mu c$ for every $c\Bsubseteq e'_n$.   It
follows that if $e^*=\inf_{n\in\Bbb N}e'_n$, $\mu_0e^*=0$;
but this must mean that $e^*=0$.   Consequently, setting
$e_n=e'_n\Bsetminus e'_{n+1}$ for $n\in\Bbb N$,
$e_{\kappa}=a_{\kappa}$ for $\kappa\in K$,  we find that
$\langle e_i\rangle_{i\in I}$ is a partition of unity in $\frak C$.

Moreover, for $n\in\Bbb N$ and $c\Bsubseteq e_n$, we must have
$\mu_{n+1}c=0$,

\Centerline{$\bar\mu c=\sum_{i\in I}\mu_ic=\sum_{k=0}^n\mu_kc
=(n+1)\mu_0c$,}

\noindent so that $\mu_kc=\bover1{n+1}\mu c$ for every $k\le n$.   But this
means that we have a measure-preserving homomorphism
$\psi_n:\frak A_{e_n}\to\frak C_{e_n}\widehat{\otimes}\frak B_n$ given
by setting

\Centerline{$\psi_n(a_k\Bcap c)=c\otimes\{k\}$}

\noindent whenever $c\in\frak C_{e_n}$ and $k\le n$;  this is well-defined
because $e_n\Bsubseteq e'_k$, so that $a_k\Bcap c\ne a_k\Bcap c'$ if
$c$, $c'$ are distinct members of $\frak C_{e_n}$, and it is
measure-preserving because

\Centerline{$\bar\mu(a_k\Bcap c)=\mu_kc=\Bover1{n+1}\bar\mu c
=\bar\mu c\cdot\bar\nu_n\{k\}$}

\noindent for all relevant $k$ and $c$.    Because $\frak B_n$ is
finite, $\psi_n$ is surjective.

\medskip

\quad\grheadz\ Just as in 333H, we now see that because $\langle
e_i\rangle_{i\in I}$ is a partition of unity in $\frak A$ as well as in
$\frak C$, we can identify $\frak A$ with $\prod_{i\in I}\frak A_{e_i}$
and therefore with $\prod_{i\in I}\frak C_{e_i}\widehat{\otimes}\frak
B_i$.

\medskip

{\bf (b)(ii)$\Rightarrow$(iii)} Let us work in
$\frak D=\prod_{i\in I}\frak C_{e_i}\widehat{\otimes}\frak B_i$, writing
$\psi:\frak A\to\frak D$ for the given isomorphism.
For each $i\in I$, we have a measure-preserving automorphism $\pi_i$ of
$\frak C_{e_i}\widehat{\otimes}\frak B_i$ with fixed-point subalgebra
$\{c\otimes 1:c\in\frak C_{e_i}\}$ (333Q).   For
$d=\langle d_i\rangle_{i\in I}\in\frak D$, set

\Centerline{$\pi d=\langle \pi_id_i\rangle_{i\in I}$.}

\noindent Then $\pi$ is a measure-preserving automorphism because every
$\pi_i$ is.   If $\pi d=d$, then for every $i\in I$ there must be a
$c_i\Bsubseteq e_i$ such that $d_i=c_i\otimes 1$.   But this means that
$d=\psi c$, where $c=\sup_{i\in I}c_i\in\frak C$.   Thus the fixed-point
subalgebra of $\pi$ is just $\psi[\frak C]$.   Transferring the
structure $(\frak D,\psi[\frak C],\pi)$ back to $\frak A$, we obtain a
measure-preserving automorphism $\psi^{-1}\pi\psi$ of $\frak A$ with
fixed-point subalgebra $\frak C$, as required.

\medskip

{\bf (c)(iii)$\Rightarrow$(i)} is trivial.
}%end of proof of 333R

\exercises{
\leader{333X}{Basic exercises $\pmb{>}$(a)}
%\sqheader 333Xa
Show that, in the proof of 333H,
$c_i=\upr(a_i,\frak C)$ (definition: 313S) for every $i\in I$.
%333H

\spheader 333Xb In Lemma 333J, show that every relative
atom in $\frak A$ over $\frak C$ belongs to the closed subalgebra of
$\frak A$ generated by $\frak C\cup\{a_n:n\in\Bbb N\}$.
%333J

\spheader 333Xc In the context of Lemma 333M, show that
if $I$ is countable we have
a one-to-one correspondence between atoms $c$ of $\frak C_0$ and points
$x$ of non-zero mass in $\BbbR^I$, given by the formula
$\pi\{x\}^{\ssbullet}=c$.
%333M

\spheader 333Xd Let $(\frak A,\bar\mu)$ be totally finite measure
algebra and $G$ a set of measure-preserving Boolean homomorphisms from
$\frak A$ to itself such that $\pi\phi\in G$ for all $\pi$, $\phi\in G$.
(i) Show that $a\Bsubseteq\sup_{\pi\in G}\pi a$ for every $a\in\frak A$.
\Hint{if $\pi c\Bsubseteq c$, where $\pi\in G$ and $c\in\frak A$, then
$\pi c=c$;  apply this to $c=\sup_{\pi\in G}\pi a$.}   (ii) Set
$\frak C=\{c:c\in\frak A,\,\pi c=c$ for every $\pi\in G\}$.   Show that
$\sup_{\pi\in G}\pi a=\upr(a,\frak C)$ for every $a\in\frak A$.
%333R

\leader{333Y}{Further exercises (a)} Show that when $I=\Bbb N$ the
algebra $\Sigma$ of
subsets of $\BbbR^I$, used in 333M, is precisely the Borel
$\sigma$-algebra as described in 271Ya.
%333M

\spheader 333Yb Let $(\frak A,\bar\mu)$ be a totally finite measure
algebra, and $\frak B$, $\frak C$ two closed subalgebras of $\frak A$
with $\frak C\subseteq\frak B$.   Show that
$\tau_{\frak C}(\frak B)\le\tau_{\frak C}(\frak A)$.   \Hint{use 333K
and the ideas of 332T.}
%333K

\spheader 333Yc Let $(\frak A,\bar\mu)$ be a probability algebra.
Show that $\frak A$ is homogeneous iff there is a measure-preserving
automorphism of $\frak A$ which is mixing in the sense of 333P.
%333P

\spheader 333Yd Let $(\frak A,\bar\mu)$ be a totally finite measure
algebra, and $G$ a set of measure-preserving Boolean homomorphisms from
$\frak A$ to itself.   Set $\frak C=\{c:c\in\frak A,\,\pi c=c$ for every
$\pi\in G\}$.   Show that $\frak C$ is a closed subalgebra of $\frak A$
of the type described in 333R.   \Hint{in the language of part (a) of
the proof of 333R, show that every $a_{\kappa}$ still belongs to
$\frak C$.}
%333R
}%end of exercises

\cmmnt{
\Notesheader{333} I have done my best, in the first part of this
section, to follow the lines already laid out in \S\S331-332, using
what should (once you have seen them) be natural generalizations of the
former definitions and arguments.   Thus the Maharam type
$\tau(\frak A)$ of an algebra is just the relative Maharam type
$\tau_{\{0,1\}}(\frak A)$, and $\frak A$ is \Mth\ iff it
is relatively \Mth\ over $\{0,1\}$.   To help you trace
through the correspondence, I list the code numbers:  331Fa$\to$333Aa,
331Fb$\to$333Ac, 331Hc$\to$333Ba, 331Hd$\to$333Bd, 332A$\to$333Bb,
331I$\to$333Cb, 331K$\to$333E,
331L$\to$333Fb, 332B$\to$333H, 332J$\to$333K.   333D overlaps
with 332P.   Throughout, the principle is the same:  everything can be
built up from products and free products.

Theorem 333Ca does not generalize any explicitly stated result, but
overlaps with Proposition 332P.   In the proof of 333E I have used a
new idea;  the same method would of course have worked just as well for
331K, but I thought it worth while to give an example of an
alternative
technique, which displays a different facet of homogeneous algebras, and
a different way in which the algebraic, topological and metric
properties of homogeneous algebras interact.   The argument of
331K-331L
relies (without using the term) on the fact that measure algebras of
Maharam type $\kappa$ have topological density at most
$\max(\kappa,\omega)$ (see 331Ye), while the the argument of 333E uses
the rather more sophisticated concept of stochastic independence.

Corollary 333Fa is cruder than the more complicated results which
follow, but I think that it is invaluable as a first step in forming a
picture of the possible embeddings of a given (totally finite) measure
algebra $\frak C$ in a larger algebra $\frak A$.   If we think of
$\frak C$ as the measure algebra of a measure space $(X,\Sigma,\mu)$,
then we can be sure that $\frak A$ is representable as a closed
subalgebra of
the measure algebra of $X\times\{0,1\}^{\kappa}$ for some $\kappa$, that
is, the measure algebra of $\lambda\restrp\Tau$ where $\lambda$ is the
product measure on $X\times\{0,1\}^{\kappa}$ and $\Tau$ is some
$\sigma$-subalgebra of the domain of $\lambda$;  the embedding of
$\frak C$ in $\frak A$ being defined by the formula
$E^{\ssbullet}\to(E\times\{0,1\}^{\kappa})^{\ssbullet}$ for $E\in\Sigma$
(325A, 325D).   Identifying, in our imaginations, both $X$ and
$\{0,1\}^{\kappa}$ with the unit interval, we can try to picture
everything in the unit square -- and these pictures, although
necessarily inadequate for algebras of uncountable Maharam type, already
give a great deal of scope for invention.

I said above that everything can be constructed from simple products and
free products, judiciously combined;  of course some further ideas must
be mixed with these.   The difference between 332B and 333H, for
instance, is partly in the need for the functionals $\mu_i$ in the
latter, whereas in the former the decomposition involves only principal
ideals with the induced measures.   Because the $\mu_i$ are
completely additive, they all have supports $c_i$ (326Xl) and we get
measure
algebras $(\frak C_{c_i},\mu_i\restrp\frak C_{c_i})$ to use in the
products.   (I note that the $c_i$ can be obtained directly from the
$a_i$, without mentioning the functionals $\mu_i$, by the process of
333Xa.)   The fact that the $c_i$ can overlap means that the `relatively
atomic' part of the larger algebra $\frak A$ needs a much more careful
description than before;  this is the burden of 333J, and also the
principal complication in the proof of 333R.   The `relatively atomless'
part is (comparatively) straightforward, since we can use the same kind
of amalgamation as before (part (c-i) of the proof of 332J, part (b)
of the proof of 333K), simplified because I am no longer seeking to deal
with algebras of infinite magnitude.

Theorem 333K gives a canonical form for superalgebras of a given totally
finite measure algebra $(\frak C,\bar\mu)$, taking the structure
$(\frak C,\bar\mu)$ itself for granted.   I hope it is clear that while
the $\mu_i$ amd $e_i$ and the algebra
$\widehat{\frak A}
=\prod_{n\in\Bbb N}\frak C_{e_n}
  \times\prod_{\kappa\in K}
  \frak C_{e_{\kappa}}\widehat{\otimes}\frak B_{\kappa}$ and the embedding of
$\frak C$ in $\widehat{\frak A}$ are uniquely defined, the rest of the
isomorphism $\pi:\frak A\to\widehat{\frak A}$ generally is not.   Even
when the $a_{\kappa}$ are uniquely defined the isomorphisms between
$\frak A_{a_{\kappa}}$ and
$\frak C_{e_{\kappa}}\widehat{\otimes}\frak B_{\kappa}$ depend on
choosing generating families in the $\frak A_{a_{\kappa}}$;
see the proof of 333Cb.

To understand the possible structures
$(\frak C,\langle\mu_i\rangle_{i\in I})$ of that theorem, we have to go
rather deeper.   The route I have chosen is to pick out the subalgebra
$\frak C_0$ of $\frak C$ determined by $\langle\mu_i\rangle_{i\in I}$
and identify it with the measure algebra of a particular measure on
$\BbbR^I$.   Perhaps I should apologise for not stating explicitly in
the course of 333N that the measure $\mu$ there is a `Borel
measure' (see 333Ya);  but I am afraid of opening a door to an
invasion of ideas which belong in Volume 4.   Besides, if I were going
to do anything more with these measures than observe that they are
uniquely defined by the construction proposed, I would complete them and
call them Radon measures.   In order to validate this approach, I must
show that the $\mu_i$ can be recovered from their restrictions to
$\frak C_0$;  this is 333Md, and is the motive for the discussion of
`standard
extensions' in \S327.   No doubt there are other ways of doing it.
One temptation which I felt it right to resist was the idea of
decomposing $\frak C$ into its homogeneous principal ideals;  this
seemed merely an additional complication.   Of course the subalgebra
$\frak C_0$ has countable Maharam type (being $\tau$-generated by the
elements $e_{iq}$, for $i\in I$ and $q\in\Bbb Q$, of 333M), so that its
decomposition is relatively simple, being just a matter of picking out
the atoms (333Xc).

Another way of looking at the expression in 333N is to observe that
$\frak A$ is obtained by amalgamating two extensions of the core subalgebra
$\frak C_0$;  one defined by
$\langle\mu_i\restrp\frak C_0\rangle_{i\in\Bbb N\cup K}$, and one by
$\langle\theta'_j\rangle_{j\in\Bbb N\cup L}$.   After using the
second family to represent $(\frak C,\bar\mu\restrp\frak C,\frak C_0)$,
we obtain standard extensions $\mu_i$ from which we can represent
$(\frak A,\bar\mu,\frak C)$.   Put this way, it seems that the process
demands that the steps be performed in the given order.
In fact it can be made symmetric;  but for that I think we need the theory
of `relative free products', which I will come to in \S458
of Volume 4.

In 333P I find myself presenting an important fact about homogeneous
measure algebras, rather out of context;  but I hope that it will help
you to believe that I have by no means finished with the insights which
Maharam's theorem provides.   I give it here for the sake of 333R.   For
the moment, I invite you to think of 333R as just a demonstration of the
power of the techniques I have developed in this chapter, and of the
kind of simplification (in the equivalence of conditions (i) and (iii))
which seems to arise repeatedly in the theory of measure algebras.  But
you will see that the first step to understanding any automorphism will
be a description of its fixed-point subalgebra, so 333R will also be
basic to the theory of automorphisms of measure algebras.   I note that
the hypothesis (i) of 333R can in fact be relaxed (333Yd), but this
seems to need an extra idea.
}%end of notes

\discrpage
