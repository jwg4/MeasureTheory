\frfilename{mt1a3.tex}
\versiondate{18.12.03}
\copyrightdate{1994}

\def\chaptername{Appendix}
\def\sectionname{Lim sups and lim infs}

\newsection{1A3}

It occurs to me that not every foundation
course in real analysis has time to deal with the concepts $\limsup$ and
$\liminf$.

\leader{1A3A}{Definition (a)} For a real sequence $\sequencen{a_n}$,
write

\Centerline{$\limsup_{n\to\infty}a_n=\lim_{n\to\infty}\sup_{m\ge n}a_m
=\inf_{n\in\Bbb N}\sup_{m\ge n}a_m$,}

\Centerline{$\liminf_{n\to\infty}a_n=\lim_{n\to\infty}\inf_{m\ge n}a_m
=\sup_{n\in\Bbb N}\inf_{m\ge n}a_m$;}

\noindent  if we
allow the values $\pm \infty$, both for suprema and infima and for
limits\cmmnt{ (see 112Ba)}, these will always be 
defined\cmmnt{, because the sequences

\Centerline{$\sequencen{\sup_{m\ge n}a_m}$,
\quad$\sequencen{\inf_{m\ge n}a_m}$}

\noindent are monotonic}.


\cmmnt{\header{1A3Ab}{\bf (b)} Explicitly:


\Centerline{$\limsup_{n\to\infty}a_n=\infty\,\iff\,\{a_n:n\in\Bbb N\}$
is unbounded above,}

\Centerline{$\limsup_{n\to\infty}a_n=-\infty\,\iff\,\lim_{n\to\infty}a_n
=-\infty$,}

\noindent that is, if and only if for every $a\in\Bbb R$ there is an
$n_0\in\Bbb N$ such that $a_n\le a$ for every $n\ge n_0$;

\Centerline{$\liminf_{n\to\infty}a_n=-\infty\, \iff\,\{a_n:n\in\Bbb N\}$
is unbounded
below,}

\Centerline{$\liminf_{n\to\infty}a_n=\infty\, \iff
\,\lim_{n\to\infty}a_n
=\infty$,}

\noindent that is, if and only if for every $a\in\Bbb R$ there is an
$n_0\in\Bbb N$ such that $a_n\ge a$ for every $n\ge n_0$.
}%end of comment

\header{1A3Ac}{\bf (c)} For \cmmnt{finite $a\in\Bbb R$, we have

\inset{$\limsup_{n\to\infty}a_n=a$ iff (i) for
every $\epsilon>0$ there is an $n_0\in\Bbb N$ such that $a_n\le
a+\epsilon$ for every $n\ge n_0$ (ii) for every $\epsilon>0$,
$n_0\in\Bbb N$ there is an $n\ge n_0$ such that $a_n\ge a-\epsilon$,}

\noindent while

\inset{$\liminf_{n\to\infty}a_n=a$ iff (i) for
every $\epsilon>0$ there is an $n_0\in\Bbb N$ such that $a_n\ge
a-\epsilon$ for every $n\ge n_0$ (ii) for every $\epsilon>0$,
$n_0\in\Bbb N$ there is an $n\ge n_0$ such that $a_n\le a+\epsilon$.}

\noindent Generally, for }%end of comment
$u\in[-\infty,\infty]$, we can say that

\inset{$\limsup_{n\to\infty}a_n=u$ iff (i) for every $v>u$ (if any)
there is an
$n_0\in\Bbb N$ such that $a_n\le v$ for every $n\ge n_0$ (ii) for every
$v<u$, $n_0\in\Bbb N$ there is an $n\ge n_0$ such that $a_n\ge v$,}

\inset{$\liminf_{n\to\infty}a_n=u$ iff (i) for every $v<u$ there is an
$n_0\in\Bbb N$ such that $a_n\ge v$ for every $n\ge n_0$ (ii) for every
$v>u$, $n_0\in\Bbb N$ there is an $n\ge n_0$ such that $a_n\le v$.}


\leader{1A3B}{}\cmmnt{ We have the following basic results.

\medskip

\noindent}{\bf Proposition} For any sequences $\sequencen{a_n}$,
$\sequencen{b_n}$ in $\Bbb R$,

(a) $\liminf_{n\to\infty}a_n\le\limsup_{n\to\infty}a_n$,

(b) $\lim_{n\to\infty}a_n=u\in[-\infty,\infty]$ iff
$\limsup_{n\to\infty}a_n=\liminf_{n\to\infty}a_n=u$,

(c) $\liminf_{n\to\infty}a_n=-\limsup_{n\to\infty}(-a_n)$,

(d) $\limsup_{n\to\infty}(a_n+b_n)
\le\limsup_{n\to\infty}a_n+\limsup_{n\to\infty}b_n$,

(e) $\liminf_{n\to\infty}(a_n+b_n)
\ge\liminf_{n\to\infty}a_n+\liminf_{n\to\infty}b_n$,

(f) $\limsup_{n\to\infty}ca_n=c\limsup_{n\to\infty}a_n$ if $c\ge 0$,

(g) $\liminf_{n\to\infty}ca_n=c\liminf_{n\to\infty}a_n$ if $c\ge 0$,

\noindent with the proviso in (d) and (e) that we must be able to
interpret the right-hand-side of the inequality according to the rules in 
135A, while in (f) and (g) we take $0\cdot\infty=0\cdot(-\infty)=0$.

\proof{{\bf (a)} $\sup_{m\ge n}a_m\ge\inf_{m\ge n}a_m$ for every
$n$, so

\Centerline{$\limsup_{n\to\infty}a_n=\lim_{n\to\infty}\sup_{m\ge
n}a_m\ge\lim_{n\to\infty}\inf_{m\ge n}a_m=\limsup_{n\to\infty}a_n$.}

\medskip

{\bf (b)} Using the last description of $\limsup_{n\to\infty}$ and
$\liminf_{n\to\infty}$ in 1A3Ac, and a corresponding description of
$\lim_{n\to\infty}$, we have

$$\eqalign{\lim_{n\to\infty}&a_n=u\cr
&\iff\text{ for every }v>u\text{ there is an }n_1\in\Bbb N
    \text{ such that }a_n\le v\text{ for every }n\ge n_1\cr
&\qquad\text{ and for every }v<u\text{ there is an }n_2\in\Bbb N
    \text{ such that }a_n\ge v\text{ for every }n\ge n_2\cr
&\iff\text{ for every }v>u\text{ there is an }n_1\in\Bbb N
    \text{ such that }a_n\le v\text{ for every }n\ge n_1\cr
&\qquad\text{ and for every }v<u,\,n_0\in\Bbb N
    \text{ there is an }n\ge n_0\text{ such that }a_n\ge v\cr
&\qquad\text{ and for every }v<u\text{ there is an }n_2\in\Bbb N
    \text{ such that }a_n\ge v\text{ for every }n\ge n_2\cr
&\qquad\text{ and for every }v>u,\,n_0\in\Bbb N
    \text{ there is an }n\ge n_0\text{ such that }a_n\le v\cr
&\iff\limsup_{n\to\infty}a_n=\liminf_{n\to\infty}a_n=u.\cr}$$

\medskip

{\bf (c)} This is just a matter of turning the formulae upside down:

$$\eqalign{\liminf_{n\to\infty}a_n
&=\sup_{n\in\Bbb N}\inf_{\lower 0.4ex\hbox{$\scriptstyle m\ge n$}}a_m
=\sup_{n\in\Bbb N}(-\sup_{m\ge n}(-a_m))\cr
&=-\inf_{\lower 0.4ex\hbox{$\scriptstyle n\in\Bbb N$}}
     \sup_{m\ge n}(-a_m)
=-\limsup_{n\to\infty}(-a_n).\cr}$$

\medskip

{\bf (d)} If $v>\limsup_{n\to\infty}a_n+\limsup_{n\to\infty}b_n$, there
are $v_1$, $v_2$ such that $v_1>\limsup_{n\to\infty}a_n$,
$v_2>\limsup_{n\to\infty}b_n$ and $v_1+v_2=v$.   Now there are $n_1$,
$n_2\in\Bbb N$ such that $\sup_{m\ge n_1}a_n\le v_1$ and 
$\sup_{m\ge n_2}b_n\le v_2$;  so that

$$\eqalign{\sup_{m\ge\max(n_1,n_2)}a_m+b_m
&\le\sup_{m\ge\max(n_1,n_2)}a_m+\sup_{m\ge\max(n_1,n_2)}b_m\cr
&\le\sup_{m\ge n_1}a_m+\sup_{m\ge n_2}b_m
\le v_1+v_2
= v.\cr}$$

\noindent As $v$ is arbitrary,

\Centerline{$\limsup_{n\to\infty}a_n+b_n=\inf_{n\in\Bbb N}\sup_{m\ge
n}a_m+b_m\le\limsup_{n\to\infty}a_n+\limsup_{n\to\infty}b_n$.}

\medskip

{\bf (e)} Putting (c) and (d) together,

$$\eqalign{\liminf_{n\to\infty}a_n+b_n
&=-\limsup_{n\to\infty}(-a_n)+(-b_n)\cr
&\ge-\limsup_{n\to\infty}(-a_n)-\limsup_{n\to\infty}(-b_n)
=\liminf_{n\to\infty}a_n+\liminf_{n\to\infty}b_n.\cr}$$

\medskip

{\bf (f)} Because $c\ge 0$,

$$\eqalign{\limsup_{n\to\infty}ca_n
&=\inf_{n\in\Bbb N}\sup_{m\ge n}ca_m
=\inf_{n\in\Bbb N}c\sup_{m\ge n}a_m\cr
&=c\inf_{n\in\Bbb N}\sup_{m\ge n}a_m
=c\limsup_{n\to\infty}a_n.\cr}$$

\medskip

{\bf (g)} Finally,

\Centerline{$\liminf_{n\to\infty}ca_n
=-\limsup_{n\to\infty}c(-a_n)
=-c\limsup_{n\to\infty}(-a_n)
=c\liminf_{n\to\infty}a_n$.}
}%end of proof of 1A3B

\cmmnt{
\leader{1A3C}{Remark} Of course the familiar results that
$\lim_{n\to\infty}a_n+b_n=\lim_{n\to\infty}a_n+\lim_{n\to\infty}b_n$,
$\lim_{n\to\infty}ca_n=c\lim_{n\to\infty}a_n$ are immediate corollaries
of 1A3B.
}%end of comment

\leader{*1A3D}{Other expressions of the same idea} 
The concepts of $\limsup$ and
$\liminf$ may be applied in any context in which we can consider the
limit of a real-valued function.   For instance, if $f$ is a real-valued
function defined (at least) on a punctured interval of the form
$\{x:0<|c-x|\le\epsilon\}$ where $c\in\Bbb R$ and $\epsilon>0$, then

\Centerline{$\limsup_{t\to c}f(t)
=\lim_{\delta\downarrow 0}\sup_{0<|t-c|\le\delta}f(t)
=\inf_{0<\delta\le\epsilon}\sup_{0<|t-c|\le\delta}f(t)$,}

\Centerline{$\liminf_{t\to c}f(t)
=\lim_{\delta\downarrow 0}\inf_{0<|t-c|\le\delta}f(t)
=\sup_{0<\delta\le\epsilon}\inf_{0<|t-c|\le\delta}f(t)$,}

\noindent allowing $\infty$ and $-\infty$ whenever they seem called for.
Or if $f$ is defined on the half-open interval
$\ocint{c,c+\epsilon}$, we can say

\Centerline{$\limsup_{t\downarrow c}f(t)
=\lim_{\delta\downarrow 0}\sup_{c<t\le c+\delta}f(t)
=\inf_{0<\delta\le\epsilon}\sup_{c<t\le c+\delta}f(t)$,}

\Centerline{$\liminf_{t\downarrow c}f(t)
=\lim_{\delta\downarrow 0}\inf_{c<t\le c+\delta}f(t)
=\sup_{0<\delta\le\epsilon}\inf_{c<t\le c+\delta}f(t)$.}

\noindent Similarly, if $f$ is defined on $\coint{M,\infty}$ for some
$M\in\Bbb R$, we have

\Centerline{$\limsup_{t\to\infty}f(t)
=\lim_{a\to\infty}\sup_{t\ge a}f(t)
=\inf_{a\ge M}\sup_{t\ge a}f(t)$,}

\Centerline{$\liminf_{t\to\infty}f(t)
=\lim_{a\to\infty}\inf_{t\ge a}f(t)
=\sup_{a\ge M}\inf_{t\ge a}f(t)$.}

\cmmnt{\noindent A further extension of the idea is 
examined briefly in 2A3S in Volume 2.}

\frnewpage

