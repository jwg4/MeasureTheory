\frfilename{mt355.tex}
\versiondate{1.12.07}
\copyrightdate{1996}

\def\chaptername{Riesz spaces}
\def\sectionname{Spaces of linear operators}

\newsection{355}

We come now to a discussion of linear operators between Riesz spaces.
Linear operators are central to any kind of functional
analysis, and a feature of the theory of Riesz spaces is the way the
order structure picks out certain classes of operators for special
consideration.   Here I concentrate on positive and order-continuous
operators, with a brief mention of sequential order-continuity.   It
turns out, in fact, that we need to work with operators which are
differences of positive operators or of order-continuous positive
operators.   I define the basic spaces $\eurm L^{\sim}$, $\eurm
L^{\times}$ and $\eurm L^{\sim}_c$ (355A, 355G), with their most
important properties (355B, 355E, 355H-355I)
and some remarks on the special case of Banach lattices (355C, 355K).
At the same time I give an important theorem on extension of operators
(355F) and a corollary (355J).

The most important case is of course that in which the codomain is
$\Bbb R$, so that our operators become real-valued functionals;
I shall come to these in the next section.

\leader{355A}{Definition}  Let $U$ and $V$ be Riesz spaces.   A linear
operator $T:U\to V$ is
{\bf order-bounded} if $T[A]$ is order-bounded in $V$ for every
order-bounded $A\subseteq U$.

I will write $\eurm L^{\sim}(U;V)$ for the set of order-bounded linear
operators from $U$ to $V$.

\vleader{108pt}{355B}{Lemma} If $U$ and $V$ are Riesz spaces,

(a) a linear operator $T:U\to V$ is order-bounded iff
$\{Tu:0\le u\le w\}$ is bounded above in $V$ for every $w\in U^+$;

(b) in particular, any positive linear operator from $U$ to $V$ belongs
to $\eurm L^{\sim}=\eurm L^{\sim}(U;V)$;

(c) $\eurm L^{\sim}$ is a linear space;

(d) if $W$ is another Riesz space and $T:U\to V$ and $S:V\to W$ are
order-bounded linear operators, then $ST:U\to W$ is order-bounded.

\proof{{\bf (a)} This is elementary.   If $T\in\eurm L^{\sim}$ and
$w\in U^+$, $[0,w]$ is
order-bounded, so its image must be order-bounded in $V$, and in
particular bounded above.   On the other hand, if $T$ satisfies the
condition, and $A$ is order-bounded, then $A\subseteq[u_1,u_2]$ for some
$u_1\le u_2$, and

\Centerline{$T[A]\subseteq T[u_1+[0,u_2-u_1]]=Tu_1+T[[0,u_2-u_1]]$}

\noindent is bounded above;  similarly, $T[-A]$ is bounded above, so
$T[A]$ is bounded below;  as $A$ is arbitrary, $T$ is order-bounded.

\medskip

{\bf (b)} If $T$ is positive then $\{Tu:0\le u\le w\}$ is bounded above
by $Tw$ for every $w\ge 0$, so $T\in\eurm L^{\sim}$.

\medskip

{\bf (c)} If $T_1$, $T_2\in\eurm L^{\sim}$, $\alpha\in\Bbb R$ and
$A\subseteq U$ is order-bounded, then there are $v_1$, $v_2\in V$ such
that $T_i[A]\subseteq[-v_i,v_i]$ for both $i$.   Setting
$v=(1+|\alpha|)v_1+v_2$,
$(\alpha T_1+T_2)[A]\subseteq[-v,v]$;  as $A$ is arbitrary, $\alpha
T_1+T_2$ belongs to $\eurm L^{\sim}$;  as $\alpha$, $T_1$, $T_2$ are
arbitrary, and since the zero operator surely belongs to $\eurm
L^{\sim}$, $\eurm L^{\sim}$ is a linear subspace of the space of all
linear operators from $U$ to $V$.

\medskip

{\bf (d)} This is immediate from the definition;  if $A\subseteq U$ is
order-bounded, then $T[A]\subseteq V$ and $(ST)[A]=S[T[A]]\subseteq W$
are order-bounded.
}%end of proof of 355B

\leader{355C}{Theorem} If $U$ and $V$ are Banach lattices then every
order-bounded linear operator\cmmnt{ (in particular, every positive
linear operator)} from $U$ to $V$ is continuous.

\proof{ \Quer\ Suppose, if possible, that $T:U\to V$ is an order-bounded
linear operator which is not continuous.   Then for each $n\in\Bbb N$ we
can find a $u_n\in U$ such that $\|u_n\|\le 2^{-n}$ but $\|Tu_n\|\ge n$.
Now $u=\sup_{n\in\Bbb N}|u_n|$ is defined in $U$ (354C), and there is a
$v\in V$ such that $-v\le Tw\le v$ whenever $-u\le w\le u$;  but this
means that $\|v\|\ge\|Tu_n\|\ge n$ for every $n$, which is impossible.\
\Bang
}%end of proof of 355C

\leader{355D}{Lemma} Let $U$ be a Riesz space and $V$ any linear space
over $\Bbb R$.   Then a function $T:U^+\to V$ extends to a linear
operator from $U$ to $V$ iff

\Centerline{$T(u+u')=Tu+Tu'$,\quad $T(\alpha u)=\alpha Tu$}

\noindent for all $u$, $u'\in U^+$ and every $\alpha>0$, and in this
case the extension is unique.

\proof{ For in this case we can, and must, set

\Centerline{$T_1u=Tu_1-Tu_2$ whenever $u_1$, $u_2\in U^+$ and
$u=u_1-u_2$;}

\noindent it is elementary to check that this defines $T_1u$ uniquely
for every $u\in U$, and that $T_1$ is a linear operator extending $T$.
}%end of proof of 355D

\leader{355E}{Theorem}  Let $U$ be a Riesz space and $V$ a Dedekind
complete Riesz space.

(a) The space $\eurm L^{\sim}$ of
order-bounded linear operators from $U$ to $V$ is a Dedekind complete
Riesz space;  its positive cone is the set of positive linear operators
from $U$ to $V$.   In particular, every order-bounded linear operator
from $U$ to $V$ is expressible as the difference of positive linear
operators.


(b) For $T\in\eurm L^{\sim}$, $T^+$ and $|T|$ are defined in the Riesz
space $\eurm L^{\sim}$ by the formulae

\Centerline{$T^+(w)=\sup\{Tu:0\le u\le w\}$,}

\Centerline{$|T|(w)=\sup\{Tu:|u|\le
w\}=\sup\{\sum_{i=0}^n|Tu_i|:\sum_{i=0}^n|u_i|=w\}$}

\noindent for every $w\in U^+$.

(c) If $S$, $T\in\eurm L^{\sim}$ then

\Centerline{$(S\vee T)(w)=\sup_{0\le u\le w}Su+T(w-u)$,
\quad$(S\wedge T)(w)=\inf_{0\le u\le w}Su+T(w-u)$}

\noindent for every $w\in U^+$.

(d) Suppose that $A\subseteq \eurm L^{\sim}$ is non-empty and
upwards-directed.   Then $A$ is bounded above in $\eurm L^{\sim}$ iff
$\{Tu:T\in A\}$ is bounded above in $V$ for every $u\in U^+$, and in
this case $(\sup A)(u)=\sup_{T\in A}Tu$ for every $u\ge 0$.

(e) Suppose that $A\subseteq(\eurm L^{\sim})^+$ is non-empty and
downwards-directed.   Then $\inf A=0$ in $\eurm L^{\sim}$ iff
$\inf_{T\in A}Tu=0$ in $V$ for every $u\in U^+$.

\proof{{\bf (a)(i)} Suppose that $T\in\eurm L^{\sim}$.   For $w\in U^+$
set $R_T(w)=\sup\{Tu:0\le u\le w\}$;  this is defined because $V$ is
Dedekind complete and $\{Tu:0\le u\le w\}$ is bounded above in $V$.
Then $R_T(w_1+w_2)=R_Tw_1+R_Tw_2$ for all $w_1$, $w_2\in U^+$.   \Prf\
Setting $A_i=[0,w_i]$ for each $i$, $w=w_1+w_2$
and $A=[0,w]$, then of course
$A_1+A_2\subseteq A$;  but also $A\subseteq A_1+A_2$, because if
$u\in A$ then $u=(u\wedge w_1)+(u-w_1)^+$, and
$0\le(u-w_1)^+\le(w-w_1)^+=w_2$, so $u\in A_1+A_2$.   Consequently

$$\eqalign{R_Tw
&=\sup T[A]
=\sup T[A_1+A_2]=\sup(T[A_1]+T[A_2])\cr
&=\sup T[A_1]+\sup T[A_2]
=R_Tw_1+R_Tw_2\cr}$$

\noindent by 351Dc.\ \QeD\   Next, it is easy to see that
$R_T(\alpha w)=\alpha R_Tw$ for $w\in U^+$ and $\alpha>0$,
just because $u\mapsto\alpha u$, $v\mapsto\alpha v$ are isomorphisms of the partially ordered linear spaces $U$ and $V$.   It follows from 355D that we can
extend $R_T$ to a linear operator from $U$ to $V$.

Because $R_Tu\ge T0=0$ for every $u\in U^+$, $R_T$ is a positive linear
operator.   But also $R_Tu\ge Tu$ for every $u\in U^+$, so $R_T-T$ is
also positive, and $T=R_T-(R_T-T)$ is the difference of two positive
linear operators.

\medskip

\quad{\bf (ii)} This shows that every order-bounded operator is a
difference of positive operators.   But of course if $T_1$ and $T_2$ are
positive, then $(T_1-T_2)u\le T_1w$ whenever $0\le u\le w$ in $U$, so
that $T_1-T_2$ is order-bounded, by the criterion in 355Ba.   Thus
$\eurm
L^{\sim}$ is precisely the set of differences of positive operators.

\medskip

\quad{\bf (iii)} Just as in 351F, $\eurm L^{\sim}$ is a partially
ordered linear space if we say that $S\le T$ iff $Su\le Tu$ for every
$u\in U^+$.   Now it is a Riesz space.   \Prf\ Take any
$T\in\eurm L^{\sim}$.   Then $R_T$, as defined in (i), is an upper bound for $\{0,T\}$ in $\eurm L^{\sim}$.   If $S\in\eurm L^{\sim}$ is any other upper bound for
$\{0,T\}$, then for any $w\in U^+$ we must have
$Sw\ge Su\ge Tu$ whenever $u\in[0,w]$, so that $Sw\ge R_Tw$;  as $w$ is
arbitrary, $S\ge R_T$;  as $S$ is arbitrary, $R_T=\sup\{0,T\}$ in
$\eurm L^{\sim}$.   Thus $\sup\{0,T\}$ is defined in $\eurm L^{\sim}$ for every
$T\in\eurm L^{\sim}$;  by 352B, $\eurm L^{\sim}$ is a Riesz space.\ \Qed

(I defer the proof that it is Dedekind complete to (d-ii) below.)

\medskip

{\bf (b)} As remarked in (a-iii), $R_T=T^+$ for each
$T\in\eurm L^{\sim}$;  but this is just the formula given for $T^+$.   Now, if $T\in\eurm L^{\sim}$ and $w\in U^+$,

$$\eqalign{|T|(w)
&=2T^+w-Tw
=2\sup_{u\in[0,w]}Tu-Tw\cr
&=\sup_{u\in[0,w]}T(2u-w)
=\sup_{u\in[-w,w]}Tu,\cr}$$

\noindent which is the first formula offered for $|T|$.   In particular,
if $|u|\le w$ then $Tu$, $-Tu=T(-u)$ are both less than or equal to
$|T|(w)$, so that $|Tu|\le|T|(w)$.   So if $u_0,\ldots,u_n$ are such
that $\sum_{i=0}^n|u_i|=w$, then

\Centerline{$\sum_{i=0}^n|Tu_i|
\le\sum_{i=0}^n|T|(|u_i|)
=|T|(w)$.}

\noindent Thus $B=\{\sum_{i=0}^n|Tu_i|:\sum_{i=0}^n|u_i|=w\}$ is bounded
above by $|T|(w)$.   On the other hand, if $v$ is an upper bound for $B$
and $|u|\le w$, then

\Centerline{$Tu\le|Tu|+|T(w-|u|)|\le v$;}

\noindent as $u$ is arbitrary, $|T|(w)\le v$;  thus $|T|(w)$ is the
least upper bound for $B$.   This completes the proof of part (b) of
the theorem.

\medskip

{\bf (c)} We know that $S\vee T=T+(S-T)^+$ (352D), so that

$$\eqalign{(S\vee T)(w)
&=Tw+(S-T)^+(w)
=Tw+\sup_{0\le u\le w}(S-T)(u)\cr
&=\sup_{0\le u\le w}Tw+(S-T)(u)
=\sup_{0\le u\le w}Su+T(w-u)\cr}$$

\noindent for every $w\in U^+$, by the formula in (b).   Also from 352D
we have $S\wedge T=S+T-T\vee S$, so that

$$\eqalignno{(S\wedge T)(w)
&=Sw+Tw-\sup_{0\le u\le w}Tu+S(w-u)\cr
&=\inf_{0\le u\le w}Sw+Tw-Tu-S(w-u)\cr
\noalign{\noindent (351Db)}
&=\inf_{0\le u\le w}Su+T(w-u)\cr}$$

\noindent for $w\in U^+$.

\medskip

{\bf (d)(i)} Now suppose that $A\subseteq\eurm L^{\sim}$ is non-empty
and upwards-directed and that $\{Tu:T\in A\}$ is bounded above in $V$
for every $u\in U^+$.   In this case, because $V$ is Dedekind complete,
we may set $Ru=\sup_{T\in A}Tu$ for every $u\in U^+$.   Now
$R(u_1+u_2)=Ru_1+Ru_2$ for all $u_1$, $u_2\in U^+$.   \Prf\  Set
$B_i=\{Tu_i:T\in A\}$ for each $i$, $B=\{T(u_1+u_2):T\in A\}$.   Then
$B\subseteq B_1+B_2$, so


\Centerline{$R(u_1+u_2)
=\sup B\le\sup(B_1+B_2)=\sup B_1+\sup B_2=Ru_1+Ru_2$.}

\noindent On the other hand, if $v_i\in B_i$ for both $i$, there are
$T_i\in A$ such that $v_i=T_iu_i$ for each $i$;  because $A$ is
upwards-directed, there is a $T\in A$ such that $T\ge T_i$ for both $i$,
and now

\Centerline{$R(u_1+u_2)\ge T(u_1+u_2)=Tu_1+Tu_2\ge
T_1u_1+T_2u_2=v_1+v_2$.}

\noindent As $v_1$, $v_2$ are arbitrary,

\Centerline{$R(u_1+u_2)\ge\sup(B_1+B_2)=\sup B_1+\sup B_2=Ru_1+Ru_2$.
\Qed}

\noindent It is also easy to see that $R(\alpha u)=\alpha Ru$ for every
$u\in U^+$ and $\alpha>0$.   So, using 355D again, $R$ has an extension
to a linear operator from $U$ to $V$.

If we fix any $T_0\in A$, we have $T_0u\le Ru$ for every $u\in U^+$,
so $R-T_0$ is a positive linear operator, and $R=(R-T_0)+T_0$ belongs to
$\eurm L^{\sim}$.   Again, $Tu\le Ru$ for every $T\in A$ and $u\in U^+$, so
$R$ is an upper bound for $A$ in $\eurm L^{\sim}$;  and, finally, if $S$
is any upper bound for $A$ in $\eurm L^{\sim}$, then
$Su$ is an upper bound for $\{Tu:T\in A\}$, and must be greater than or
equal to $Ru$, for every $u\in U^+$;  so that $R\le S$ and $R=\sup A$ in
$\eurm L^{\sim}$.

\medskip

\quad{\bf (ii)} Consequently
$\eurm L^{\sim}$ is Dedekind complete.   \Prf\ If
$A\subseteq\eurm L^{\sim}$ is
non-empty and bounded above by $S$ say, then $A'=\{T_0\vee
T_1\vee\ldots\vee T_n:T_0,\ldots,T_n\in A\}$ is upwards-directed and
bounded above by $S$, so $\{Tu:T\in A'\}$ is bounded above by $Su$ for
every $u\in U^+$;  by (i) just above, $A'$ has a supremum in $\eurm
L^{\sim}$, which will also be the supremum of $A$.\ \Qed

\medskip

{\bf (e)} Suppose that $A\subseteq(\eurm L^{\sim})^+$ is non-empty and
downwards-directed.   Then $-A=\{-T:T\in A\}$ is non-empty and
upwards-directed, so

$$\eqalign{\inf A=0
&\iff\sup(-A)=0\cr
&\iff\sup_{T\in A}(-Tu)=0\text{ for every }u\in U^+\cr
&\iff\inf_{T\in A}Tu=0\text{ for every }u\in U^+.\cr}$$
}%end of proof of 355E

\leader{355F}{Theorem} Let $U$ and $V$ be Riesz spaces,
$U_0\subseteq U$ a Riesz subspace and $T_0:U_0\to V$ a
positive linear operator such that
$Su=\sup\{T_0w:w\in U_0,\,0\le w\le u\}$ is defined in $V$ for every
$u\in U^+$.   Suppose {\it either} that
$U_0$ is order-dense and that $T_0$ is order-continuous {\it or}
that $U_0$ is solid.

(a) There is a unique positive linear operator $T:U\to V$, extending $T_0$,
which agrees with $S$ on $U^+$.

(b) If $T_0$ is a Riesz homomorphism so is $T$.

(c) If $T_0$ is order-continuous so is $T$.

(d) If $U_0$ is order-dense and $T_0$ is an injective Riesz
homomorphism, then $T$ is injective.

(e) If $U_0$ is order-dense and $T_0$ is order-continuous then $T$ is
the only order-continuous positive linear operator from $U$ to $V$
extending $T_0$.

\proof{{\bf (a)(i)} (The key.) If $u$, $u'\in U^+$ then $S(u+u')=Su+Su'$.
\Prf\ If $w$, $w'\in U_0^+$, $w\le u$ and $w'\le u'$, then
$w+w'\le u+u'$, so

\Centerline{$T_0w+T_0w'=T_0(w+w')\le S(u+u')$;}

\noindent as $w$ and $w'$ are arbitrary, $Su+Su'\le S(u+u')$ (351Dc).   In
the other direction, suppose that $w\in U_0^+$ and $w\le u+u'$.

\medskip

\qquad{\bf case 1} Suppose that $U_0$ is solid.   Then $w\wedge u$ and
$(w-u)^+$ belong to $U_0$, while $w\wedge u\le u$ and
$(w-u)^+\le(u+u'-u)^+=u'$;  so

\Centerline{$T_0w=T_0(w\wedge u+(w-u)^+)=T_0(w\wedge u)+T_0(w-u)^+
\le Su+Su'$;}

\noindent as $w$ is arbitrary, $S(u+u')\le Su+Su'$ and we must have
equality.

\medskip

\qquad{\bf case 2} Suppose that $U_0$ is order-dense and $T_0$ is
order-continuous.   Set $A=\{v:v\in U_0^+$, $v\le w\wedge u\}$ and
$B=\{v:v\in U_0^+$, $v\le(w-u)^+\}$.   Then (taking the suprema in $U$)
$w\wedge u=\sup A$ and $(w-u)^+=\sup B$, because $U_0$ is order-dense;
by 351Dc again, $w=\sup(A+B)$ in $U$ and therefore $w=\sup(A+B)$ in $U_0$.
Also both $A$ and $B$ are upwards-directed, so $A+B$ also is.
Because $T_0$ is order-continuous,

\Centerline{$T_0w=\sup T_0[A+B]=\sup(T_0[A]+T_0[B])\le Su+Su'$.}

\noindent So once again we must have $S(u+u')\le Su+Su'$ and therefore
$S(u+u')=Su+Su'$.\ \Qed

\medskip

\quad{\bf (ii)} Of course $S(\alpha u)=\alpha Su$ whenever $u\in U^+$ and
$\alpha\ge 0$.   By 355D, $S$ has a unique extension to a linear operator
$T:U\to V$.   As $Tu=Su\ge 0$ whenever $u\ge 0$, $T$ is positive.   If
$u\in U_0^+$ then $Su=T_0u$, so $T$ extends $T_0$.

\medskip

{\bf (b)} Suppose that $T_0$ is a Riesz homomorphism.   If $u\wedge u'=0$
in $U$, then $w\wedge w'=0$ and $T_0w\wedge T_0w'=0$ whenever
$w\in U_0\cap[0,u]$ and $w'\in U_0\cap[0,u']$.   By 352Ea,
$Tu\wedge Tu'=Su\wedge Su'=0$ in $V$.   By 352G(iv), $T$ is a Riesz
homomorphism.

\medskip

{\bf (c)} Now suppose that $T_0$ is order-continuous.
Suppose that $B\subseteq U^+$ is non-empty and upwards-directed and has a
supremum $u_0\in U$.   Of course $Tu\le Tu_0$ for every $u\in B$, so $Tu_0$
is an upper bound for $T[B]$.   On the other hand, suppose that
$v$ is an upper bound for $T[B]$.   If $w\in U^+_0$ and
$u\in U^+$, $w\wedge u=\sup\{w':w'\in U_0$, $0\le w'\le w\wedge u\}$.
\Prf\ If $U_0$ is solid, $w\wedge u\in U_0$;  and otherwise $U_0$ is
order-dense.\ \QeD\  So if $w\in U_0$ and $0\le w\le u_0$,

\Centerline{$w=w\wedge\sup B=\sup_{u\in B}w\wedge u
=\sup_{u\in B}\sup(U_0\cap[0,w\wedge u])=\sup C$,}

\noindent where

\Centerline{$C=\{w':w'\in U_0^+$, $w'\le w\wedge u$ for some $u\in B\}$.}

\noindent Since $C$ is upwards-directed,

\Centerline{$T_0w=\sup T_0[C]\le v$.}

\noindent As $w$ is arbitrary, $Tu_0\le v$;  as $v$ is arbitrary,
$Tu_0=\sup T[B]$;  as $B$ is arbitrary, $T$ is order-continuous (351Ga).

\medskip

{\bf (d)} If $U_0$ is order-dense and $T_0$ is an injective
Riesz homomorphism, then for any non-zero $u\in U$ there is a non-zero
$w\in U_0$ such that $|w|\le|u|$;  so that

\Centerline{$|Tu|=T|u|\ge T_0|w|>0$}

\noindent because $T$ is a Riesz homomorphism, by (b).   As $u$ is
arbitrary, $T$ is injective.

\medskip

{\bf (e)} Finally, if $U_0$ is order-dense then any order-continuous
positive linear operator extending $T_0$ must agree with $S$ on $U^+$ and
is therefore equal to $T$.
}%end of proof of 355F

\vleader{48pt}{355G}{Definition} Let $U$ be a Riesz space and $V$ a Dedekind
complete Riesz space.   Then $\eurm L^{\times}(U;V)$ will be the set of
those $T\in\eurm L^{\sim}(U;V)$ expressible as the difference of
order-continuous positive linear operators, and $\eurm L^{\sim}_c(U;V)$
will be the set of those $T\in\eurm L^{\sim}(U;V)$ expressible as the
difference of sequentially order-continuous positive linear operators.

\cmmnt{Because a composition of (sequentially) order-continuous
functions is (sequentially) order-continuous, we shall have}

\Centerline{$ST\in\eurm L^{\times}(U;W)$ whenever
$S\in\eurm L^{\times}(V;W)$, $T\in\eurm L^{\times}(U;V)$,}

\Centerline{$ST\in\eurm L^{\sim}_c(U;W)$ whenever
$S\in\eurm L^{\sim}_c(V;W)$, $T\in\eurm L^{\sim}_c(U;V)$,}

\noindent for all Riesz spaces $U$ and all Dedekind complete Riesz
spaces $V$, $W$.

\leader{355H}{Theorem} Let $U$ be a Riesz space and $V$ a Dedekind
complete Riesz space.   Then

(i) $\eurm L^{\times}=\eurm L^{\times}(U;V)$ is a band in
$\eurm L^{\sim}=\eurm L^{\sim}(U;V)$,
therefore a Dedekind complete Riesz space in its own right;

(ii) a member $T$ of $\eurm L^{\sim}$ belongs to $\eurm L^{\times}$ iff
$|T|$ is order-continuous.

\proof{ There is a fair bit to check, but each individual step is easy
enough.

\medskip

{\bf (a)} Suppose that $S$, $T$ are order-continuous positive linear
operators from $U$ to $V$.   Then $S+T$ is order-continuous.   \Prf\ If
$A\subseteq U$ is non-empty, downwards-directed and has infimum $0$,
then for any $u_1$, $u_2\in A$ there is a $u\in A$ such that $u\le u_1$,
$u\le u_2$, and now
$(S+T)(u)\le Su_1+Tu_2$.   Consequently any lower bound for $(S+T)[A]$
must also be a lower bound for $S[A]+T[A]$.   But since

\Centerline{$\inf(S[A]+T[A])=\inf S[A]+\inf T[A]=0$}

\noindent (351Dc), $\inf (S+T)[A]$ must also be $0$;  as $A$ is
arbitrary, $S+T$ is order-continuous, by 351Ga.\ \Qed

\medskip

{\bf (b)} Consequently $S+T\in\eurm L^{\times}$ for all $S$, $T\in\eurm
L^{\times}$.   Since $-T$ and $\alpha T$ belong to $\eurm L^{\times}$
for every $T\in\eurm L^{\times}$ and $\alpha\ge 0$, we see that
$\eurm L^{\times}$ is a linear subspace of $\eurm L^{\sim}$.

\medskip

{\bf (c)} If $T:U\to V$ is an order-continuous linear operator, $S:U\to
V$ is linear and $0\le S\le T$, then $S$ is order-continuous.   \Prf\ If
$A\subseteq U$ is non-empty, downwards-directed and has infimum $0$,
then any lower bound of $S[A]$ must also be a lower bound of $T[A]$, so
$\inf S[A]=0$;  as $A$ is arbitrary, $S$ is order-continuous.\ \Qed

It follows that $\eurm L^{\times}$ is a solid linear subspace of $\eurm
L^{\sim}$.   \Prf\ If $T\in\eurm L^{\times}$ and $|S|\le|T|$ in $\eurm
L^{\sim}$, express $T$ as $T_1-T_2$ where $T_1$, $T_2$ are
order-continuous positive linear operators.   Then

\Centerline{$S^+$, $S^-\le|S|\le|T|\le T_1+T_2$,}

\noindent so $S^+$ and $S^-$ are order-continuous and $S=S^+-S^-\in\eurm
L^{\times}$.\ \Qed

Accordingly $\eurm L^{\times}$ is a Dedekind complete Riesz space in its
own right (353J(b-i)).

\wheader{355H}{4}{2}{2}{36pt}

{\bf (d)} The argument of (c) also shows that if $T\in\eurm L^{\times}$
then $|T|$ is order-continuous;  so that for $T\in\eurm L^{\sim}$,

\Centerline{$T\in\eurm L^{\times}\iff|T|\in\eurm L^{\times}\iff |T|$ is
order-continuous.}

\medskip

{\bf (e)} If $C\subseteq(\eurm L^{\times})^+$ is non-empty,
upwards-directed and has a supremum $T\in L^{\sim}$, then $T$ is
order-continuous, so belongs to $\eurm L^{\times}$.   \Prf\ Suppose that
$A\subseteq U^+$ is non-empty, upwards-directed and has supremum $w$.
Then

\Centerline{$Tw=\sup_{S\in C}Sw=\sup_{S\in C}\sup_{u\in A}Su
=\sup_{u\in A}Tu$,}

\noindent putting 355Ed and 351G(a-iii) together.   So (using 351Ga
again) $T$ is order-continuous.\ \QeD\  Consequently $\eurm L^{\times}$
is a band in $\eurm L^{\sim}$ (352Ob).

This completes the proof.
}%end of proof of 355H

\leader{355I}{Theorem} Let $U$ be a Riesz space and $V$ a Dedekind
complete Riesz space.   Then $\eurm L^{\sim}_c(U;V)$ is a band in
$\eurm L^{\sim}(U;V)$, and a member $T$ of $\eurm L^{\sim}(U;V)$ belongs to
$\eurm L^{\sim}_c(U;V)$ iff $|T|$ is sequentially order-continuous.

\proof{ Copy the arguments of 355H.}

\leader{355J}{Proposition} Let $U$ be a Riesz space and $V$ a Dedekind
complete Riesz space.   Let $U_0\subseteq U$ be an order-dense Riesz
subspace;  then $T\mapsto T\restr U_0$ is an embedding of
$\eurm L^{\times}(U;V)$ as a solid linear subspace of
$\eurm L^{\times}(U_0;V)$.   In particular, any operator in
$\eurm L^{\times}(U_0;V)$
has at most one extension in $\eurm L^{\times}(U;V)$.

\proof{{\bf (a)} Because the embedding $U_0\embedsinto U$ is
positive and order-continuous (352Nb), $T\restr U_0$ is positive and
order-continuous whenever $T$ is;  so
$T\restr U_0\in\eurm L^{\times}(U_0;V)$ whenever
$T\in\eurm L^{\times}(U;V)$.   Because the map $T\mapsto T\restr U_0$ is
linear, the image $W$ of $\eurm L^{\times}(U;V)$ is a linear subspace of
$\eurm L^{\times}(U_0;V)$.

\medskip

{\bf (b)} If $T\in\eurm L^{\times}(U;V)$ and $T\restr U_0\ge 0$, then
$T\ge 0$.   \Prf\Quer\ Suppose, if possible, that there is a $u\in U^+$
such that $Tu\not\ge 0$.   Because $|T|\in\eurm L^{\times}(U;V)$ is
order-continuous and $A=\{v:v\in U_0,\,v\le u\}$ is an upwards-directed
set with supremum $u$, $\inf\{|T|(u-v):v\in A\}=0$ and there is a $v\in
A$ such that $Tu+|T|(u-v)\not\ge 0$.   But $Tv=Tu+T(v-u)\le Tu+|T|(u-v)$
so $Tv\not\ge 0$ and $T\restr U_0\not\ge 0$.\ \Bang\Qed

This shows that the map $T\mapsto T\restr U_0$ is an order-isomorphism
between $\eurm L^{\times}(U;V)$ and $W$, and in particular is injective.

\medskip

{\bf (c)} Now suppose that $S_0\in W$ and that $|S|\le|S_0|$ in
$\eurm L^{\times}(U_0;V)$.   Then $S\in W$.   \Prf\ Take
$T_0\in\eurm L^{\times}(U;V)$ such that $T_0\restr U_0=S_0$.
Then $S_1=|T_0|\restr U_0$ is a positive member of $W$ such that
$S_0\le S_1$ and $-S_0\le S_1$,
so $S^+\le S_1$.   Consequently, for any $u\in U^+$,

\Centerline{$\sup\{S^+v:v\in U_0,\,0\le v\le u\}
\le\sup\{S_1v:v\in U_0,\,0\le v\le u\}\le|T_0|(u)$}

\noindent is defined in $V$ (recall that we are assuming that $V$ is
Dedekind complete).   But this means that $S^+$ has an extension to an
order-continuous positive linear operator from $U$ to $V$ (355F), and
belongs to $W$.   Similarly, $S^-\in W$, so $S\in W$.\ \Qed

This shows that $W$ is a solid linear subspace of $\eurm
L^{\times}(U_0;V)$, as claimed.
}%end of proof of 355J

\leader{355K}{Proposition} Let $U$ be a Banach lattice with an
order-continuous norm.

(a) If $V$ is any Archimedean Riesz space and $T:U\to V$ is a positive
linear operator, then $T$ is order-continuous.

(b) If $V$ is a Dedekind complete Riesz space then $\eurm
L^{\times}(U;V)=\eurm L^{\sim}(U;V)$.

\proof{{\bf (a)} Suppose that $A\subseteq U^+$ is non-empty and
downwards-directed and has infimum $0$.   Then for each $n\in\Bbb N$
there is a $u_n\in A$ such that $\|u_n\|\le 4^{-n}$.   By 354C,
$u=\sup_{n\in\Bbb N}2^nu_n$ is defined in $U$.   Now $Tu_n\le 2^{-n}Tu$
for every $n$, so any lower bound for $T[A]$ must also be a lower bound
for $\{2^{-n}Tu:n\in\Bbb N\}$ and therefore (because $V$ is Archimedean)
less than or equal to $0$.   Thus $\inf T[A]=0$;  as $A$ is arbitrary,
$T$ is order-continuous.

\medskip

{\bf (b)} This is now immediate from 355Ea and the definition of
$\eurm L^{\times}$.
}%end of proof of 355K

\exercises{\leader{355X}{Basic exercises $\pmb{>}$(a)}
%\spheader 355Xa
Let $U$ and $V$ be arbitrary Riesz spaces.  (i) Show that the set $\eurm
L(U;V)$ of all linear operators from $U$ to $V$ is a partially ordered
linear space if we say that $S\le T$ whenever $Su\le Tu$ for every $u\in
U^+$.   (ii) Show that if $U$ and $V$ are Banach lattices then the set
of positive operators is closed in the normed space $\eurm B(U;V)$ of
bounded linear operators from $U$ to $V$.
%355C

\sqheader 355Xb  If $U$ is a Riesz space and $\|\,\|$, $\|\,\|'$ are two
norms on $U$ both rendering it a Banach lattice, show that they are
equivalent, that is, give rise to the same topology.
%355C

\spheader 355Xc Let $U$ be a Riesz space with a Riesz norm, $V$ an
Archimedean Riesz space with an order unit, and $T:U\to V$ a linear
operator which is continuous for the given norm on $U$ and the
order-unit norm on $V$.   Show that $T$ is order-bounded.
%355C

\spheader 355Xd Let $U$ be a Riesz space, $V$ an Archimedean Riesz
space, and $T:U^+\to V^+$ a map such that $T(u_1+u_2)=Tu_1+Tu_2$ for all
$u_1$, $u_2\in U^+$.   Show that $T$ has an extension to a linear
operator from $U$ to $V$.
%355D

\sqheader 355Xe Show that if $r$, $s\ge 1$ are integers then the Riesz
space $\eurm L^{\sim}(\BbbR^r;\BbbR^s)$ can be identified with the space
of real $s\times r$ matrices, saying that a matrix is positive iff every
coefficient is positive, so that if
$T=\langle\tau_{ij}\rangle_{1\le i\le s,1\le j\le r}$ then $|T|$, taken
in $\eurm L^{\sim}(\BbbR^r;\BbbR^s)$, is
$\langle|\tau_{ij}|\rangle_{1\le i\le s,1\le j\le r}$.   Show that a
positive matrix represents a Riesz
homomorphism iff each row has at most one non-zero coefficient.
%355E

\sqheader 355Xf Let $U$ be a Riesz space and $V$ a Dedekind complete
Riesz space.   Show that if $T_0,\ldots,T_n\in\eurm L^{\sim}(U;V)$ then

\Centerline{$(T_0\vee\ldots\vee T_n)(w)
=\sup\{\sum_{i=0}^nT_iu_i:u_i\ge 0\Forall i\le
n,\,\sum_{i=0}^nu_i=w\}$}

\noindent for every $w\in U^+$.
%355E

\sqheader 355Xg Let $U$ be a Riesz space, $V$ a Dedekind complete Riesz
space, and $A\subseteq\eurm L^{\sim}(U;V)$ a non-empty set.   Show that
$A$ is bounded above in $\eurm L^{\sim}(U;V)$ iff
$C_w=\{\sum_{i=0}^nT_iu_i:T_0,\ldots,T_n\in A,\,u_0,\ldots,u_n\in
U^+,\,\sum_{i=0}^nu_i=w\}$ is bounded above in $V$ for every $w\in U^+$,
and in this case $(\sup A)(w)=\sup C_w$ for every $w\in U^+$.
%355E

\leader{355Y}{Further exercises (a)}
%\spheader 355Ya
Let $U$ and $V$ be Banach lattices.   For
$T\in\eurm L^{\sim}=\eurm L^{\sim}(U;V)$, set

\Centerline{$\|T\|_{\sim}
=\sup_{w\in U^+,\|w\|\le 1}\inf\{\|v\|:|Tu|\le v$ whenever $|u|\le w\}$.}

\noindent Show that
$\|\,\|_{\sim}$ is a norm on $\eurm L^{\sim}$ under which $\eurm
L^{\sim}$ is a Banach space, and that the set of positive linear
operators is closed in $\eurm L^{\sim}$.
%355C

\spheader 355Yb Give an example of a continuous linear operator from
$\ell^2$ to itself which is not order-bounded.
%355C

\spheader 355Yc Let $U$ and $V$ be Riesz spaces and $T:U\to V$ a linear
operator.   (i) Show that for any $w\in U^+$,
$C_w=\{\sum_{i=0}^n|Tu_i|:u_0,\ldots,u_n\in U^+,\,\sum_{i=0}^nu_i=w\}$
is upwards-directed, and has the same upper bounds as $\{Tu:|u|\le w\}$.
\Hint{352Fd.}
(ii) Show that if $\sup C_w$ is defined for every
$w\in U^+$, then $S=T\vee(-T)$ is defined in the partially ordered
linear space $\eurm L^{\sim}(U;V)$ and $Sw=\sup C_w$ for every
$w\in U^+$.
%355E

\spheader 355Yd Let $U$, $V$ and $W$ be Riesz spaces, of which $V$ and
$W$ are Dedekind complete.   (i) Show that for any $S\in\eurm
L^{\times}(V;W)$, the map $T\mapsto ST:L^{\sim}(U;V)\to\eurm
L^{\sim}(U;W)$ belongs to $\eurm L^{\times}(\eurm L^{\sim}(U;V);\eurm
L^{\sim}(U;W))$, and is a Riesz homomorphism if $S$ is.   \Hint{355Yc.}
(ii) Show that for any
$T\in\eurm L^{\sim}(U;V)$, the map $S\mapsto ST:L^{\sim}(V;W)\to\eurm
L^{\sim}(U;W)$ belongs to $\eurm L^{\times}(L^{\sim}(V;W);\eurm
L^{\sim}(U;W))$.
%355E, 355Yc

\spheader 355Ye Let $\nu_{\Bbb N}$ be the usual measure on
$\{0,1\}^{\Bbb N}$ and
$\pmb{c}$ the Banach lattice of convergent sequences.   Find a
linear operator $T:L^2(\nu_{\Bbb N})\to\pmb{c}$ which is norm-continuous,
therefore
order-bounded, such that $0$ and $T$ have no common upper bound in the
partially ordered linear space
of all linear operators from $L^2(\nu_{\Bbb N})$ to $\pmb{c}$.
%355E (solution in mt35bits)

\spheader 355Yf Let $U$ and $V$ be Banach lattices.   Let
$\eurm L^{\text{reg}}$ be the linear space of operators from $U$ to $V$
expressible as the difference of positive operators.   For
$T\in\eurm L^{\text{reg}}$ let $\|T\|_{\text{reg}}$ be

\Centerline{$\inf\{\|T_1+T_2\|:T_1$, $T_2:U\to V$ are positive,
$T=T_1-T_2\}$.}

\noindent Show that $\|\,\|_{\text{reg}}$ is a norm under which
$\eurm L^{\text{reg}}$ is complete.
%355E

\spheader 355Yg Let $U$ and $V$ be Riesz spaces.   For this exercise
only, say that $\eurm L^{\times}(U;V)$ is to be the set
of linear operators $T:U\to V$ such that whenever $A\subseteq U$ is
non-empty,
downwards-directed and has infimum $0$ then $\{v:v\in
V^+,\,\exists\,w\in
A,\,|Tu|\le v$ whenever $|u|\le w\}$ has infimum $0$ in $V$.   (i) Show
that $\eurm L^{\times}(U;V)$ is a linear space.   (ii) Show that if $U$
is Archimedean then $\eurm L^{\times}(U;V)\subseteq\eurm L^{\sim}(U;V)$.
(iii) Show that if $U$ is Archimedean and $V$ is Dedekind complete then
this definition agrees with that of 355G.   (iv) Show that for any Riesz
spaces $U$, $V$ and $W$, $ST\in\eurm L^{\times}(U;W)$ for every
$S\in\eurm L^{\times}(V;W)$ and $T\in\eurm L^{\times}(U;V)$.   (v) Show
that if $U$ and $V$ are Banach lattices, then $\eurm L^{\times}(U;V)$ is
closed in $\eurm L^{\sim}(U;V)$ for the norm $\|\,\|_{\sim}$
of 355Ya.   (vi) Show that if $V$ is Archimedean and $U$ is a Banach
lattice with an order-continuous norm, then $\eurm L^{\times}(U;V)=\eurm
L^{\sim}(U;V)$.
%355G

\spheader 355Yh Let $U$ be a Riesz space and $V$ a Dedekind complete
Riesz space.   Show that the band projection
$P:\eurm L^{\sim}(U;V)\to\eurm L^{\times}(U;V)$ is given by the formula

$$\eqalign{(PT)(w)=\inf\{\sup_{u\in A}Tu:A\subseteq U^+
\text{ is non-empty},&\text{ upwards-directed}\cr
&\text{ and has supremum }w\}\cr}$$

\noindent for every $w\in U^+$, $T\in(\eurm L^{\sim}(U;V))^+$.   (Cf.\
362Bd.)
%355H mt35bits


\spheader 355Yi Show that if $U$ is a Riesz space with the countable sup
property
(241Ye), then $\eurm L^{\sim}_c(U;V)=\eurm L^{\times}(U;V)$ for every
Dedekind complete Riesz space $V$.

\spheader 355Yj Let $U$ and $V$ be Riesz spaces, of which $V$ is
Dedekind complete, and $U_0$ a solid linear subspace of $U$.   Show that
the map $T\mapsto T\restr U_0$ is an order-continuous Riesz homomorphism
from $\eurm L^{\times}(U;V)$ onto a solid linear subspace of
$\eurm L^{\times}(U_0;V)$.
%355J

\spheader 355Yk Let $U$ be a uniformly complete Riesz space (354Yi) and
$V$ a Dedekind complete Riesz space.   Let $U_{\Bbb C}$, $V_{\Bbb C}$ be
their complexifications (354Yl).   Show that the complexification of
$\eurm L^{\sim}(U;V)$ can be identified with the complex linear space of
linear operators $T:U_{\Bbb C}\to V_{\Bbb C}$ such that
$B_T(w)=\{|Tu|:|u|\le w\}$ is bounded above in $V$ for every $w\in U^+$,
and that now $|T|(w)=\sup B_T(w)$ for every
$T\in\eurm L^{\sim}(U;V)_{\Bbb C}$ and $w\in U^+$.
\Hint{if $u$, $v\in U$ and
$|u+iv|=w$, then $u$ and $v$ can be simultaneously approximated for the
order-unit norm $\|\,\|_w$ on the solid linear subspace generated by $w$
by finite sums $\sum_{j=0}^n(\cos\theta_j)w_j$,
$\sum_{j=0}^n(\sin\theta_j)w_j$ where $w_j\in U^+$, $\sum_{j=0}^nw_j=w$.
Consequently $|T(u+iv)|\le|T|(w)$ for every $T\in\eurm L^{\sim}_{\Bbb
C}$.}
%+
}%end of exercises

\cmmnt{
\Notesheader{355} I have had to make some choices in the basic
definitions of this chapter (355A, 355G).   For Dedekind complete
codomains $V$, there is no doubt what $\eurm L^{\sim}(U;V)$ should be,
since the order-bounded operators (in the sense of 355A) are just the
differences of positive operators (355Ea).   (These are sometimes called
`{\bf regular}' operators.)
When $V$ is not Dedekind complete, we have to choose between the two
notions, as not every order-bounded operator need be regular (355Ye).
In my previous book ({\smc Fremlin 74a}) I chose the regular operators;
I have still not encountered any really persuasive reason to settle
definitively on
either class.   In 355G the technical complications in dealing with any
natural equivalent of the larger space (see 355Yg) are such that I have
settled for the narrower class, but explicitly restricting the
definition to the case in which $V$ is Dedekind complete.   In the
applications in this book, the codomains are nearly always Dedekind
complete, so we can pass these questions by.

The elementary extension technique in 355D may recall the definition of
the Lebesgue integral (122L-122M).   In the same way, 351G may remind
you of the theorem that a linear operator between normed spaces is
continuous everywhere if it is continuous anywhere, or of the
corresponding results
about Boolean homomorphisms and additive functionals on Boolean algebras
(313L, 326Ka, 326R).

Of course 355Ea is the central fact about the space
$\eurm L^{\sim}(U;V)$
for Dedekind complete $V$;  because we get a new Riesz space from old
ones, the prospect of indefinite recursion immediately presents itself.
For Banach lattices, $\eurm L^{\sim}(U;V)$ is a linear subspace of the
space $\eurm B(U;V)$ of bounded linear operators (355C);  the question
of when the two are equal will be of great importance to us.   I give
only the vaguest hints on how to show that they can be different (355Yb,
355Ye), but these should be enough to make it plain that equality is the
exception rather than the rule.   It is also very useful that we have
effective formulae to describe the Riesz space operations on
$\eurm L^{\sim}(U;V)$ (355E, 355Xf-355Xg, 355Yc).   You may wish to compare
these with the corresponding formulae for additive functionals on
Boolean algebras in 326Yd and 362B.

If we think of $\eurm L^{\sim}$ as somehow corresponding to the space of
bounded additive functionals on a Boolean algebra, the bands
$\eurm L^{\sim}_c$ and $\eurm L^{\times}$ correspond to the spaces of
countably
additive and completely additive functionals.   In fact (as will appear
in \S362) this correspondence is very close indeed.   For the moment,
all I have sought to establish is that $\eurm L^{\sim}_c$ and
$\eurm L^{\times}$ are indeed bands.   Of course any case in which
$\eurm L^{\sim}(U;V)=\eurm L^{\sim}_c(U;V)$ or
$\eurm L^{\sim}_c(U;V)=\eurm L^{\times}(U;V)$ is of interest
(355Kb, 355Yi).

Between Banach lattices, positive linear operators are continuous
(355C);  it follows at once that the Riesz space structure determines
the topology (355Xb), so that it is not to be wondered at that there are
further connexions between the norm and the spaces $\eurm L^{\sim}$ and
$\eurm L^{\times}$, as in 355K.

355F will be a basic tool in the theory of representations of Riesz
spaces;  if we can represent an order-dense Riesz subspace of $U$ as a
subspace of a Dedekind complete space $V$, we have at least some chance
of expressing $U$ also as a subspace of $V$.   Of course it has other
applications, starting with analysis of the dual spaces.
}%end of comment

\discrpage

