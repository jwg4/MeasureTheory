\frfilename{mt5a.tex} 
\versiondate{21.11.13} 
\copyrightdate{2007} 
      
\def\chaptername{Appendix} 
\def\sectionname{Introduction} 
      
\gdef\topparagraph{} 
\gdef\bottomparagraph{Appendix to Vol.\ 5 {\it intro.}} 
      
\centerline{\bf Appendix to Volume 5} 
      
\medskip 
      
\centerline{\bf Useful facts} 
      
\medskip 
      
For this volume, the most substantial ideas demanded are, 
naturally enough, in
set theory.   Fragments of general set theory are in \S5A1, with cardinal
arithmetic and infinitary combinatorics.
\S5A2 contains results from Shelah's pcf theory,
restricted to those which are actually used in this book.   \S5A3
describes the language I will use when I discuss forcing constructions;
in essence, I follow {\smc Kunen 80}, but with some variations which need
to be signalled.   

As usual, some bits of general topology are needed;   I give these in
\S5A4, starting with a list of cardinal functions to complement the
definitions in \S511.   There is a tiny piece of real analysis in \S5A5.
In \S5A6 are notes on a few undecidable propositions, mostly standard.   

\discrpage 
      
      
