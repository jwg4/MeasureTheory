\wheader{}{10}{4}{4}{100pt}

Introduction to Volume 2 \vtmpb{3.3.03}\pagereference{10}{}

%wheader parameters:  #1 new paragraph name
 % #2#3#4  \vskip parameters if page break not forced (no. of points)
 % #5  spare height demanded if page break not to be forced
\wheader{}{10}{4}{4}{100pt}

 *Chapter 21: Taxonomy of measure spaces

\chapintrosection{17.1.15}{12}{}

\section{211}{Definitions}{20.11.03}{12}{}
{Complete, totally finite, $\sigma$-finite, strictly localizable,
semi-finite, localizable, locally determined measure spaces;  atoms;
elementary relationships;  countable-cocountable measures.}

\section{212}{Complete spaces}{10.9.04}{17}{}
{Measurable and integrable functions on complete spaces;
completion of a measure.}

\section{213}{Semi-finite, locally determined and localizable
spaces}{13.9.13}{23}{}
{Integration on semi-finite spaces;  c.l.d.\ versions;  measurable
envelopes;  characterizing localizability and strict localizability.}

\section{214}{Subspaces}{22.5.09}{33}{}
{Subspace measures on arbitrary subsets;  integration;  direct
sums of measure spaces;  *extending measures to well-ordered families of
sets.}

\section{215}{$\sigma$-finite spaces and the principle of exhaustion}
{13.11.13}{43}{}
{The principle of exhaustion;  characterizations of $\sigma$-finiteness;
the intermediate value theorem for atomless measures.}

\section{*216}{Examples}{25.9.04}{47}{}
{A complete localizable non-locally-determined space;  a complete
locally determined non-localizable space;  a complete locally determined
localizable space which is not strictly localizable.}

\wheader{}{10}{4}{4}{100pt}

Chapter 22:  The Fundamental Theorem of Calculus

\chapintrosection{24.2.14}{53}{}

\section{221}{Vitali's theorem in $\Bbb{R}$}{2.6.03}{53}{}
{Vitali's theorem for intervals in $\Bbb{R}$.}

\section{222}{Differentiating an indefinite integral}{20.11.03/18.10.04}
{56}{}
{Monotonic functions are differentiable a.e., and their
derivatives are integrable;  $\bover{d}{dx}\int_a^xf=f$ a.e.;
*the Denjoy-Young-Saks theorem.}

\section{223}{Lebesgue's density theorems}{9.9.04}{64}{}
{$f(x)=\lim_{h{\downarrow}0}\bover1{2h}\int_{x-h}^{x+h}f$ a.e.\ ($x$);
density points;
$\lim_{h{\downarrow}0}\bover1{2h}\int_{x-h}^{x+h}|f-f(x)|=0$ a.e.\ ($x$);
the Lebesgue set of a function.}

\section{224}{Functions of bounded variation}{29.9.04}{68}{}
{Variation of a function;  differences of monotonic functions;
sums and products, limits, continuity and differentiability for b.v.\
functions;  an inequality for ${\int}f{\times}g$.}

\section{225}{Absolutely continuous functions}{16.8.15}{76}{}
{Absolute continuity of indefinite integrals;  absolutely
continuous functions on $\Bbb{R}$;  integration by parts;  lower
semi-continuous functions;  *direct images of negligible sets;  the
Cantor function.}

\section{*226}{The Lebesgue decomposition of a function of bounded
variation}{6.11.13}{86}{}
{Sums over arbitrary index sets;  saltus functions;  the Lebesgue
decomposition.}

\wheader{}{10}{4}{4}{100pt}

 Chapter 23: The Radon-Nikod\'ym theorem

\chapintrosection{17.11.04}{93}{}

\section{231}{Countably additive functionals}{25.8.15}{93}{}
{Additive and countably additive functionals;  Jordan and Hahn
decompositions.}

\section{232}{The Radon-Nikod\'ym theorem}{5.6.02}{97}{}
{Absolutely and truly continuous additive functionals;  truly
continuous functionals are indefinite integrals;  *the Lebesgue
decomposition of a countably additive functional.}

\section{233}{Conditional expectations}{16.6.02}{106}{}
{$\sigma$-subalgebras;  conditional expectations of integrable
functions;  convex functions;  Jensen's inequality.}

\section{234}{Operations on measures}{11.4.09}{113}{}
{Inverse-measure-preserving functions;  image measures;  sums of measures;
indefinite-integral measures;  ordering of measures.}

\section{235}{Measurable transformations}{30.3.03}{123}{}
{The formula ${\int}g(y)\nu(dy)={\int}J(x)g(\phi(x))\mu(dx)$;
detailed conditions of applicability;  \imp\ functions;  the image
measure catastrophe;  using the Radon-Nikod\'ym theorem.}

\wheader{}{10}{4}{4}{100pt}

 Chapter 24:  Function spaces

\chapintrosection{15.11.13}{134}{}  

\section{241}{$\eusm{L}^0$ and $L^0$}{6.11.03}{134}{}
{The linear, order and multiplicative structure of $L^0$;  Dedekind completeness and localizability;  action
of Borel functions.}

\section{242}{$L^1$}{19.11.03}{142}{}
{The normed lattice $L^1$;  integration as a linear functional;
completeness and Dedekind completeness;  the Radon-Nikod\'ym theorem and
conditional expectations;  convex functions;  dense subspaces.}

\section{243}{$L^{\infty}$}{30.4.04}{152}{}{The normed lattice
$L^{\infty}$;  norm-completeness;  the duality
between $L^1$ and $L^{\infty}$;  localizability, Dedekind completeness
and the identification $L^{\infty}\cong(L^1)^*$.}

\section{244}{$L^p$}{6.3.09}{159}{}
{The normed lattices $L^p$, for $1<p<\infty$;  H\"older's
inequality;  completeness and Dedekind completeness;
$(L^p)^*\cong{L^q}$;  conditional expectations;  *uniform convexity.}

\section{245}{Convergence in measure}{25.3.06}{173}{}
{The topology of (local) convergence in measure on $L^0$;
pointwise convergence;  localizability and Dedekind completeness;
embedding $L^p$ in $L^0$;  $\|\,\|_1$-\vthsp{}convergence and
convergence in
measure;  $\sigma$-finite spaces, metrizability and sequential
convergence.}

\section{246}{Uniform integrability}{17.11.06}{184}{}
{Uniformly integrable sets in $\eusm{L}^1$ and $L^1$;  elementary
properties;  disjoint-sequence characterizations;  $\|\,\|_1$ and
convergence in measure on uniformly integrable sets.}

\section{247}{Weak compactness in $L^1$}{26.8.13}{192}{}
{A subset of $L^1$ is uniformly integrable iff it is relatively weakly
compact.}

\wheader{}{10}{4}{4}{100pt}

 Chapter 25:  Product measures

\chapintrosection{31.5.03}{197}{}

\section{251}{Finite products}{10.11.06}{197}{}
{Primitive and c.l.d.\ products;  basic properties;  Lebesgue
measure on $\Bbb{R}^{r+s}$ as a product measure;  products of direct sums
and subspaces;  c.l.d.\ versions.}

\section{252}{Fubini's theorem}{6.12.07}{213}{}
{When ${\iint}f(x,y)dxdy$ and ${\int}f(x,y)d(x,y)$ are equal;  measures
of ordinate sets;  *the volume of a ball in $\Bbb{R}^r$.}

\section{253}{Tensor products}{18.4.08}{229}{}
{Bilinear operators;  bilinear operators $L^1(\mu){\times}L^1(\nu){\to}W$
and linear operators $L^1(\mu\times\nu){\to}W$;  positive bilinear
operators and the ordering 48 $L^1(\mu\times\nu)$;
conditional expectations;  upper integrals.}

\section{254}{Infinite products}{23.2.16}{239}{}
{Products of arbitrary families of probability spaces;  basic
properties;  \imp\ functions;  usual measure on $\{0,1\}^I$;
$\{0,1\}^{\Bbb{N}}$ isomorphic, as measure space, to $[0,1]$;  subspaces
of full outer measure;  sets determined by coordinates in
a subset of the index set;  generalized associative law for products of
measures;  subproducts as image measures;  factoring functions through
subproducts;  conditional expectations on subalgebras corresponding to
subproducts;  products of localizable spaces;  products of atomless spaces.}

\section{255}{Convolutions of functions}{3.7.08}{257}{}
{Shifts in $\Bbb{R}^2$ as measure space automorphisms;
convolutions of functions on $\Bbb{R}$;
${\int}h\times(f*g)={\int}h(x+y)f(x)g(y)d(x,y)$;  $f*(g*h)=(f*g)*h$;
$\|f*g\|_1\le
\|f\|_1\|g\|_1$;  %this break required by  mtchrefs.for
the groups $\Bbb{R}^r$ and $\ocint{-\pi,\pi}$.}

\section{256}{Radon measures on $\Bbb{R}^r$}{6.8.15}{268}{}
{Definition of Radon measures on $\Bbb{R}^r$;  completions of Borel
measures;  Lusin measurability;  image measures;
products of two Radon measures;  semi-continuous functions.}

\section{257}{Convolutions of measures}{14.8.13}{276}{}
{Convolution of totally finite Radon measures on $\Bbb{R}^r$;
${\int}h\,d(\nu_1*\nu_2)={\iint}h(x+y)\nu_1(dx)\nu_2(dy)$;
$\nu_1*(\nu_2*\nu_3)=(\nu_1*\nu_2)*\nu_3$;  convolutions and
Radon-Nikod\'ym derivatives.}

\wheader{}{10}{4}{4}{100pt}

 Chapter 26:  Change of variable in the integral

\chapintrosection{5.9.03}{279}{}

\section{261}{Vitali's theorem in $\BbbR^r$}{11.12.12}{279}{}
{Vitali's theorem for balls in $\BbbR^r$;  Lebesgue's Density
Theorem;  Lebesgue sets.}

\section{262}{Lipschitz and differentiable functions}{11.8.15}{287}{}
{Lipschitz functions;  elementary properties;  differentiable
functions from $\BbbR^r$ to $\BbbR^s$;  differentiability and partial
derivatives;  approximating a differentiable function by piecewise 
affine functions;  *Rademacher's theorem.}

\section{263}{Differentiable transformations in $\BbbR^r$}{4.4.13}{298}{}
{In the formula ${\int}g(y)dy={\int}J(x)g(\phi(x))dx$, find $J$ when
$\phi$ is (i) linear (ii) differentiable;  detailed conditions of
applicability;  polar coordinates;  the case of non-injective $\phi$;
the one-dimensional case.}

\section{264}{Hausdorff measures}{12.5.03}{310}{}
{$r$-dimensional Hausdorff measure on $\BbbR^s$;  Borel sets are
measurable;  Lipschitz functions;  if $s=r$, we have a multiple of
Lebesgue measure;  *Cantor measure as a Hausdorff measure.}

\section{265}{Surface measures}{3.9.13}{320}{}
{Normalized Hausdorff measure;  action of linear operators and
differentiable functions;  surface measure on a sphere.}

\section{*266}{The Brunn-Minkowski inequality}{28.1.09}{338}{}
{Arithmetic and geometric means;  essential closures;  the
Brunn-Minkowski inequality.}

\wheader{}{10}{4}{4}{100pt}

 Chapter 27:  Probability theory

\chapintrosection{26.8.13}{332}{}

\section{271}{Distributions}{11.12.08}{333}{}
{Terminology;  distributions as Radon measures;  distribution
functions;  densities;  transformations of random variables;  *distribution
functions and convergence in measure.}

\section{272}{Independence}{3.4.09}{339}{}
{Independent families of random variables;  characterizations of
independence;  joint distributions of (finite) independent families, and
product measures;   the zero-one law;  $\Expn(X{\times}Y)$, $\Var(X+Y)$;
distribution of a sum as convolution of distributions;  Etemadi's
inequality;  *Hoeffding's inequality.}

\section{273}{The strong law of large numbers}{2.12.09}{352}{}
{$\bover1{n+1}\sum_{i=0}^nX_i{\to}0$ a.e.\ if the $X_n$ are
independent with zero expectation and either (i)
$\sum_{n=0}^{\infty}\bover1{(n+1)^2}$
\discretionary{}{}{}$\Var(X_n)<\infty$ or (ii)
$\sum_{n=0}^{\infty}\Expn(|X_n|^{1+\delta})<\infty$ for some $\delta>0$
or (iii) the $X_n$ are identically distributed.}

\section{274}{The Central Limit Theorem}{13.4.10}{364}{}
{Normally distributed r.vs;  Lindeberg's
condition for the Central Limit Theorem;
corollaries;  estimating $\int_{\alpha}^{\infty}e^{-x^2/2}dx$.}

\section{275}{Martingales}{3.12.12}{375}{}
{Sequences of $\sigma$-algebras, and martingales adapted to them;
up-crossings;   Doob's Martingale Convergence Theorem;  uniform
integrability, $\|\,\|_1$-\vthsp{}convergence and martingales as
sequences of
conditional expectations;  reverse martingales;  stopping times.}

\section{276}{Martingale difference sequences}{16.4.13}{386}{}
{Martingale difference sequences;  strong law
of large numbers for m.d.ss.;  Koml\'os' theorem.}

\wheader{}{10}{4}{4}{100pt}

 Chapter 28:  Fourier analysis

\chapintrosection{17.1.15}{395}{}

\section{281}{The Stone-Weierstrass theorem}{4.12.12}{395}{}
{Approximating a function on a compact set by members of a given
lattice or algebra of functions;  real and complex cases;  approximation
by polynomials and trigonometric functions;  Weyl's Equidistribution
Theorem in $[0,1]^r$.}

\section{282}{Fourier series}{24.9.09}{406}{}
{Fourier and Fej\'er sums;  Dirichlet and Fej\'er kernels;
Riemann-Lebesgue lemma;  uniform convergence of Fej\'er sums of a
continuous function;  a.e.\ and $\|\,\|_1$-convergence of Fej\'er sums of 
an integrable
function;  $\|\,\|_2$-convergence of Fourier sums of a square-integrable
function;  convergence of Fourier sums of a differentiable or b.v.\
function;  convolutions and Fourier coefficients.}

\section{283}{Fourier transforms I}{31.3.13}{424}{}
{Fourier and inverse Fourier transforms;  elementary properties;
$\int_0^{\infty}\bover1x\sin{x}\,dx
\ifdim\pagewidth>467pt\penalty-100\fi
=\bover12\pi$;  the formula
$\varhatf\varcheck{\phantom{h}}=f$ for differentiable and b.v.\ $f$;
convolutions;  $e^{-x^2/2}$;
${\int}f\times\varhat{g}=\int\varhatf{\times}g$.}

\section{284}{Fourier transforms II}{30.8.13}{439}{}
{Test functions;  $\varhat{h}\varcheck{\phantom{h}}=h$;  tempered
functions;  tempered functions which represent each other's transforms;
convolutions;  square-integrable functions;  Dirac's delta function.}

\section{285}{Characteristic functions}{18.9.14}{455}{}
{The characteristic function of a distribution;  independent
r.vs;  the normal distribution;  the vague topology on the space of
distributions, and sequential convergence of characteristic functions;
Poisson's theorem;  convolutions of distributions.}

\section{*286}{Carleson's theorem}{30.3.16}{470}{}
{The Hardy-Littlewood Maximal Theorem;  the Lacey-Thiele proof of
Carleson's theorem for square-integrable functions on $\Bbb{R}$ and
$\ocint{-\pi,\pi}$.}

\wheader{}{10}{4}{4}{100pt}

 Appendix to Volume 2

\chapintrosection{3.8.15}{502}{}

\section{2A1}{Set theory}{20.1.13}{502}{}
{Ordered sets;  transfinite recursion;  ordinals;  initial
ordinals;  Schr\"oder-\vthsp{}Bernstein theorem;  filters;  Axiom of
Choice;
Zermelo's Well-Ordering Theorem;  Zorn's Lemma;  ultrafilters;  a theorem
in combinatorics.}

\section{2A2}{The topology of Euclidean space}{30.11.09}{508}{}
{Closures;  continuous functions;  compact sets;  open sets in $\Bbb{R}$.}

\section{2A3}{General topology}{25.7.07}{511}{}
{Topologies;  continuous functions;  subspace topologies;
closures and interiors;
Hausdorff topologies;  pseudometrics;  convergence of sequences;
compact spaces;  cluster points of sequences;  convergence of filters;
lim sup and lim inf;  product topologies;  dense subsets.}

\section{2A4}{Normed spaces}{4.3.14}{519}{}
{Normed spaces;  linear subspaces;  Banach spaces;
bounded linear operators;  dual spaces;  extending a linear operator
from a dense subspace;  normed algebras.}

\section{2A5}{Linear topological spaces}{13.11.07}{522}{}
{Linear topological spaces;  topologies defined by functionals;
convex sets;  completeness;  weak topologies.}

\section{2A6}{Factorization of matrices}{10.11.14}{526}{}
{Determinants;  orthonormal families;  $T=PDQ$ where $D$ is
diagonal and $P$, $Q$ are orthogonal.}

\wheader{}{10}{4}{4}{100pt}

% Concordance \pagereference{545}{}

\medskip

References for Volume 2 \vtmpb{17.12.12}\pagereference{528}{}

% \medskip

% Index to Volumes 1 and 2

% \qquad Principal topics and results \pagereference{529}{}

% \qquad General index \pagereference{543}{}

%570 pages
