\frfilename{mt372.tex}
\versiondate{7.12.08/17.7.11}
\copyrightdate{1995}

\def\chaptername{Linear operators between function spaces}
\def\sectionname{The ergodic theorem}

\newsection{372}

I come now to one of the most remarkable topics in measure theory.   I
cannot do it justice in the space I have allowed for it here, but I can
give the basic theorem (372D, 372F) and a variety of the corollaries
through which it is regularly used (372E,
372G-372J), %372G 372H 372J
together with brief
notes on one of its most famous and characteristic applications (to
continued fractions, 372L-372N) %372L 372M 372N
and on `ergodic' and `mixing'
transformations (372O-372S). %372O 372Q 372S
In the first half of the section (down to
372G) I express the arguments in the abstract language of measure
algebras and their associated function spaces, as developed in Chapter
36;  the second half, from 372H onwards,
contains translations of the
results into the language of measure spaces and measurable functions,
the more traditional, and more readily applicable, forms.

\leader{372A}{Lemma} Let $U$ be a reflexive Banach space and $T:U\to U$
a bounded linear operator of norm at most $1$.   Then

\Centerline{$V=\{u+v-Tu:u,\,v\in U,\,Tv=v\}$}

\noindent is dense in $U$.

\proof{ Of course $V$ is a linear subspace of $U$.   \Quer\ Suppose, if
possible, that it is not dense.   Then there is a non-zero $h\in U^*$
such that $h(v)=0$ for every $v\in V$ (3A5Ad).
Take $u\in U$ such that $h(u)\ne 0$.   Set

\Centerline{$u_n=\Bover1{n+1}\sum_{i=0}^nT^iu$}

\noindent for each $n\in\Bbb N$, taking $T^0$ to be the identity
operator;  because

\Centerline{$\|T^iu\|\le\|T^i\|\|u\|\le\|T\|^i\|u\|\le\|u\|$}

\noindent for each $i$, $\|u_n\|\le\|u\|$ for every $n$.   Note also
that $T^{i+1}u-T^iu\in V$ for every $i$, so that
$h(T^{i+1}u-T^iu)=0$;  accordingly $h(T^iu)=h(u)$ for every $i$, and
$h(u_n)=h(u)$ for every $n$.

Let $\Cal F$ be any non-principal ultrafilter on $\Bbb N$.   Because $U$
is reflexive, $v=\lim_{n\to\Cal F}u_n$ is defined in $U$ for the weak
topology on $U$ (3A5Gc).   Now $Tv=v$.   \Prf\ For each $n\in\Bbb
N$,

\Centerline{$Tu_n-u_n
=\Bover1{n+1}\sum_{i=0}^n(T^{i+1}u-T^iu)
=\Bover1{n+1}(T^{n+1}u-u)$}

\noindent has norm at most $\bover2{n+1}\|u\|$.   So
$\sequencen{Tu_n-u_n}\to 0$ for the norm topology $U$ and therefore for
the weak topology, and surely $\lim_{n\to\Cal F}Tu_n-u_n=0$.   On the
other hand (because $T$ is continuous for the weak topology, 2A5If)

\Centerline{$Tv=\lim_{n\to\Cal F}Tu_n=\lim_{n\to\Cal F}(Tu_n-u_n)
+\lim_{n\to\Cal F}u_n=0+v=v$,}

\noindent where all the limits are taken for the weak topology.\ \Qed

But this means that $v\in V$, while

\Centerline{$h(v)=\lim_{n\to\Cal F}h(u_n)=h(u)\ne 0$,}

\noindent contradicting the assumption that $h\in V^{\smallcirc}$.\
\Bang
}%end of proof of 372A

\leader{372B}{Lemma} Let $(\frak A,\bar\mu)$ be a measure algebra, and
$T:L^1\to L^1$ a positive linear operator of norm at most $1$, where
$L^1=L^1(\frak A,\bar\mu)$.   Take any $u\in L^1$ and $m\in\Bbb N$, and
set

\Centerline{$a=\Bvalue{u>0}\Bcup\Bvalue{u+Tu>0}\Bcup\Bvalue{u+Tu+T^2u>0}
\Bcup\ldots\Bcup\Bvalue{u+Tu+\ldots+T^mu>0}$.}

\noindent Then $\int_au\ge 0$.

\proof{ Set $u_0=u$, $u_1=u+Tu,\ldots,u_m=u+Tu+\ldots+T^mu$,
$v=\sup_{i\le m}u_i$, so that $a=\Bvalue{v>0}$.   Consider $u+T(v^+)$.
We have $T(v^+)\ge Tv\ge Tu_i$ for every $i\le m$
(because $T$ is positive), so that $u+T(v^+)\ge u+Tu_i=u_{i+1}$ for
$i<m$,
and $u+T(v^+)\ge\sup_{1\le i\le m}u_i$.   Also $u+T(v^+)\ge u$ because
$T(v^+)\ge 0$, so $u+T(v^+)\ge v$.   Accordingly

\Centerline{$\int_au\ge\int_av-\int_aT(v^+)=\int v^+-\int_aT(v^+)
\ge\|v^+\|_1-\|Tv^+\|_1\ge 0$}

\noindent because $\|T\|\le 1$.
}%end of proof of 372B

\leader{372C}{Maximal Ergodic Theorem}
Let $(\frak A,\bar\mu)$ be a measure algebra, and $T:L^1\to L^1$ a
linear operator, where $L^1=L^1(\frak A,\bar\mu)$, such that
$\|Tu\|_1\le\|u\|_1$ for every $u\in L^1$ and
$\|Tu\|_{\infty}\le\|u\|_{\infty}$ for every
$u\in L^1\cap L^{\infty}(\frak A)$.
Set $A_n=\bover1{n+1}\sum_{i=0}^nT^i$ for each
$n\in\Bbb N$.   Then for any $u\in L^1$, $u^*=\sup_{n\in\Bbb N}A_nu$ is
defined in $L^0(\frak A)$, and
$\alpha\bar\mu\Bvalue{u^*>\alpha}\le\|u\|_1$ for every $\alpha>0$.

\proof{{\bf (a)} To begin with, suppose that $T$ is positive and that
$u\ge 0$ in $L^1$.   Note that if $v\in L^1\cap L^{\infty}$, then
$\|T^iv\|_{\infty}\le\|v\|_{\infty}$ for every $i\in\Bbb N$, so
$\|A_nv\|_{\infty}\le\|v\|_{\infty}$ for every $n$;  in particular,
$A_n(\chi a)\le\chi 1$ for every $n$ and every $a$ of finite measure.

For $m\in\Bbb N$ and $\alpha>0$, set

\Centerline{$a_{m\alpha}=\Bvalue{\sup_{i\le m}A_iu>\alpha}$.}

\noindent Then $\alpha\bar\mu a_{m\alpha}\le\|u\|_1$.   \Prf\ Set
$a=a_{m\alpha}$,
$w=u-\alpha\chi a$.   Of course $\sup_{i\le m}A_iu$ belongs to
$L^1$, so $\bar\mu a$ is finite and $w\in L^1$.   For any $i\le
m$,

\Centerline{$A_iw=A_iu-\alpha A_i(\chi a)\ge A_iu-\alpha\chi 1$,}

\noindent so $\Bvalue{A_iw>0}\Bsupseteq\Bvalue{A_iu>\alpha}$.
Accordingly $a\Bsubseteq b$, where

\Centerline{$b=\sup_{i\le m}\Bvalue{A_iw>0}
=\sup_{i\le m}\Bvalue{w+Tw+\ldots+T^iw>0}$.}

\noindent By 372B, $\int_bw\ge 0$.   But this means that

\Centerline{$\alpha\bar\mu a=\alpha\int_b\chi a
=\int_bu-\int_bw\le\int_bu\le\|u\|_1$,}

\noindent as claimed.\   \Qed

It follows that if we set $c_{\alpha}=\sup_{n\in\Bbb N}a_{n\alpha}$,
$\bar\mu c_{\alpha}\le\alpha^{-1}\|u\|_1$ for every $\alpha>0$ and
$\inf_{\alpha>0}c_{\alpha}=0$.   But this is exactly the criterion in
364L(a-ii) for $u^*=\sup_{n\in\Bbb N}A_nu$ to be defined in $L^0$.   And
$\Bvalue{u^*>\alpha}=c_{\alpha}$, so
$\alpha\bar\mu\Bvalue{u^*>\alpha}\le\|u\|_1$ for every $\alpha>0$, as
required.

\medskip

{\bf (b)} Now consider the case of general $T$, $u$.   In this case $T$
is order-bounded and $\||T|\|\le 1$, where $|T|$ is the modulus of $T$
in $\eurm L^{\sim}(L^1;L^1)=\eurm B(L^1;L^1)$ (371D).   If
$w\in L^1\cap L^{\infty}$, then

\Centerline{$||T|w|\le |T||w|
=\sup_{|w'|\le|w|}|Tw'|\le\|w\|_{\infty}\chi 1$,}

\noindent so $\||T|w\|_{\infty}\le\|w\|_{\infty}$.   Thus $|T|$ also
satisfies the conditions of the theorem.
Setting $B_n=\bover1{n+1}\sum_{i=0}^n|T|^i$, $B_n\ge A_n$ in
$\eurm L^{\sim}(L^1;L^1)$ and $B_n|u|\ge A_nu$ for every $n$.   But by
(a), $v=\sup_{n\in\Bbb N}B_n|u|$ is defined in $L^0$ and
$\alpha\bar\mu\Bvalue{v>\alpha}\le\||u|\|_1=\|u\|_1$ for every
$\alpha>0$.   Consequently $u^*=\sup_{n\in\Bbb N}A_nu$ is defined in
$L^0$ and $u^*\le v$, so that
$\alpha\bar\mu\Bvalue{u^*>\alpha}\le\|u\|_1$ for every $\alpha>0$.
}%end of proof of 372C


\leader{372D}{}\cmmnt{ We are now ready for a very general form of the
Ergodic Theorem.   I express it in terms of the space $M^{1,0}$ from
366F and the class $\Cal T^{(0)}$ of operators from 371F.   If these
formulae are unfamiliar, you may like to glance at the statement of 372F
before looking them up.

\medskip

\noindent}{\bf The Ergodic Theorem:  first form} Let $(\frak A,\bar\mu)$
be a measure algebra, and set $M^{1,0}=M^{1,0}(\frak A,\bar\mu)$,
$\Cal T^{(0)}=\Cal T^{(0)}_{\bar\mu,\bar\mu}
\subseteq\eurm B(M^{1,0};M^{1,0})$\cmmnt{ as in
371F-371G}.   Take any $T\in\Cal T^{(0)}$, and set
$A_n=\bover1{n+1}\sum_{i=0}^nT^i:M^{1,0}\to M^{1,0}$ for every $n$.
Then for any $u\in M^{1,0}$, $\sequencen{A_nu}$ is
order*-convergent\cmmnt{ (definition:  367A)} and
$\|\,\|_{1,\infty}$-convergent to a member $Pu$ of $M^{1,0}$.
The operator $P:M^{1,0}\to M^{1,0}$ is a projection onto the linear
subspace $\{u:u\in M^{1,0},\,Tu=u\}$, and $P\in\Cal T^{(0)}$.

\proof{{\bf (a)}  It will be convenient to start with some elementary
remarks.   First, every $A_n$ belongs to $\Cal T^{(0)}$, by 371Ge and
371Ga.    Next, $\sequencen{A_nu}$ is order-bounded in
$L^0=L^0(\frak A)$ for any $u\in M^{1,0}$;  this is because if $u=v+w$,
where $v\in L^1=L^1(\frak A,\bar\mu)$ and
$w\in L^{\infty}=L^{\infty}(\frak A)$,
then $\sequencen{A_nv}$ and $\sequencen{A_n(-v)}$ are bounded above, by
372C, while $\sequencen{A_nw}$ is norm- and order-bounded in
$L^{\infty}$.   Accordingly I can uninhibitedly speak of
$P^*(u)=\inf_{n\in\Bbb N}\sup_{i\ge n}A_iu$ and
$P_*(u)=\sup_{n\in\Bbb N}\inf_{i\ge n}A_iu$
for any $u\in M^{1,0}$, these both being defined in $L^0$.

\medskip

{\bf (b)} Write $V_1$ for the set of those $u\in M^{1,0}$ such that
$\sequencen{A_nu}$ is order*-convergent in $L^0$;
that is, $P^*(u)=P_*(u)$ (367Be).   It is easy to see that $V_1$ is a
linear subspace of $M^{1,0}$ (use 367Ca and 367Cd).   Also it is closed
for $\|\,\|_{1,\infty}$.

\Prf\ We know that $|T|$, taken in $\eurm L^{\sim}(M^{1,0};M^{1,0})$,
belongs to $\Cal T^{(0)}$ (371Gb);  set
$B_n=\bover1{n+1}\sum_{i=0}^n|T|^i$ for each $i$.

Suppose that $u_0\in\overline{V}_1$.   Then for any $\epsilon>0$ there
is a $u\in V_1$ such that $\|u_0-u\|_{1,\infty}\le\epsilon^2$.   Write
$Pu=P^*(u)=P_*(u)$ for the order*-limit of
$\sequencen{A_nu}$.   Express $u_0-u$ as $v+w$ where $v\in L^1$,
$w\in L^{\infty}$ and $\|v\|_1+\|w\|_{\infty}\le 2\epsilon^2$.

Set $v^*=\sup_{n\in\Bbb N}B_n|v|$.   Then
$\bar\mu\Bvalue{v^*>\epsilon}\le2\epsilon$, by 372C.   Next, if
$w^*=\sup_{n\in\Bbb N}B_n|w|$, we surely have
$w^*\le 2\epsilon^2\chi 1$.   Now

\Centerline{$|A_nu_0-A_nu|=|A_nv+A_nw|\le B_n|v|+B_n|w|\le v^*+w^*$}

\noindent for every $n\in\Bbb N$, that is,

\Centerline{$A_nu-v^*-w^*\le A_nu_0\le A_nu+v^*+w^*$}

\noindent for every $n$.   Because $\sequencen{A_nu}$
order*-converges to $Pu$,

\Centerline{$Pu-v^*-w^*\le P_*(u_0)\le P^*(u_0)\le Pu+v^*+w^*$,}

\noindent and $P^*(u_0)-P_*(u_0)\le 2(v^*+w^*)$.   On the other hand,

\Centerline{$\bar\mu\Bvalue{2(v^*+w^*)>2\epsilon+4\epsilon^2}
\le\bar\mu\Bvalue{v^*>\epsilon}+\bar\mu\Bvalue{w^*>2\epsilon^2}
=\bar\mu\Bvalue{v^*>\epsilon}\le 2\epsilon$}

\noindent (using 364Ea for the first inequality).   So

\Centerline{$\bar\mu\Bvalue{P^*(u_0)-P_*(u_0)>2\epsilon(1+2\epsilon)}\le
2\epsilon$.}

\noindent Since $\epsilon$ is arbitrary, $\sequencen{A_nu_0}$
order*-converges to $P^*(u_0)=P_*(u_0)$, and $u_0\in V_1$.   As $u_0$ is
arbitrary, $V_1$ is closed.\   \Qed

\medskip


{\bf (c)} Similarly, the set $V_2$ of those $u\in M^{1,0}$ for which
$\sequencen{A_nu}$ is norm-convergent is a linear subspace of $M^{1,0}$,
and it also is closed.   \Prf\ This is a standard argument.   If
$u_0\in\overline{V}_2$ and $\epsilon>0$, there is a $u\in V_2$ such that
$\|u_0-u\|_{1,\infty}\le\epsilon$.   There is an $n\in\Bbb N$ such that
$\|A_iu-A_ju\|_{1,\infty}\le\epsilon$ for all $i$, $j\ge n$, and now
$\|A_iu_0-A_ju_0\|_{1,\infty}\le 3\epsilon$ for all $i$, $j\ge n$,
because every $A_i$ has norm at most $1$ in $\eurm B(M^{1,0};M^{1,0})$
(371Gc).   As $\epsilon$ is arbitrary,
$\sequencen{A_iu_0}$ is Cauchy;  because $M^{1,0}$ is complete,
it is convergent, and $u_0\in V_2$.   As $u_0$ is arbitrary, $V_2$ is
closed.\   \Qed

\medskip

{\bf (d)} Now let $V$ be
$\{u+v-Tu:u\in M^{1,0}\cap L^{\infty},\,v\in M^{1,0},\,Tv=v\}$.   Then
$V\subseteq V_1\cap V_2$.   \Prf\ If $u\in M^{1,0}\cap L^{\infty}$, then
for any $n\in\Bbb N$

\Centerline{$A_n(u-Tu)=\bover1{n+1}(u-T^{n+1}u)\to 0$}

\noindent for $\|\,\|_{\infty}$, and therefore is both
order*-convergent and convergent for
$\|\,\|_{1,\infty}$;  so $u-Tu\in V_1\cap V_2$.   On the other hand, if
$Tv=v$, then of course $A_nv=v$ for every $n$, so again
$v\in V_1\cap V_2$.\   \Qed

\medskip

{\bf (e)} Consequently $L^2=L^2(\frak A,\bar\mu)\subseteq V_1\cap V_2$.
\Prf\ $L^2\cap V_1\cap V_2$ is a linear subspace;  but also it is closed
for the norm topology of $L^2$, because the identity map from $L^2$ to
$M^{1,0}$ is continuous (369Oe).   We know also that $T\restr L^2$
is an operator of norm at
most $1$ from $L^2$ to itself (371Gd).   Consequently
$W=\{u+v-Tu:u,\,v\in L^2,\,Tv=v\}$ is dense in $L^2$ (372A).   On the
other hand, given $u\in L^2$ and $\epsilon>0$, there is a $u'\in L^2\cap
L^{\infty}$ such that $\|u-u'\|_2\le\epsilon$ (take
$u'=(u\wedge\gamma\chi 1)\vee(-\gamma\chi 1)$ for any $\gamma$ large
enough), and now $\|(u-Tu)-(u'-Tu')\|_2\le 2\epsilon$.   Thus
$W'=\{u'+v-Tu':u'\in L^2\cap L^{\infty},\,v\in L^2,\,Tv=v\}$ is dense in
$L^2$.   But $W'\subseteq V_1\cap V_2$, by (d) above.   Thus
$L^2\cap V_1\cap V_2$ is dense in $L^2$, and is therefore the whole of
$L^2$.\
\Qed

\medskip

{\bf (f)} $L^2\supseteq S(\frak A^f)$ is dense in $M^{1,0}$, by 369Pc,
so $V_1=V_2=M^{1,0}$.   This shows that $\sequencen{A_nu}$ is
norm-convergent and order*-convergent for every $u\in M^{1,0}$.   By
367Da, the limits are the same.   Write $Pu$
for the common value of the limits.

\medskip

{\bf (g)} Of course we now have

\Centerline{$\|Pu\|_{\infty}\le\sup_{n\in\Bbb N}\|A_nu\|_{\infty}
\le\|u\|_{\infty}$}

\noindent for every $u\in L^{\infty}\cap M^{1,0}$, while

\Centerline{$\|Pu\|_1\le\liminf_{n\to\infty}\|A_nu\|_1\le \|u\|_1$}

\noindent for every $u\in L^1$, by Fatou's Lemma.   So $P\in\Cal
T^{(0)}$.
If $u\in M^{1,0}$ and $Tu=u$, then surely $Pu=u$, because $A_nu=u$
for every $u$.   On the other hand, for any $u\in M^{1,0}$, $TPu=Pu$.
\Prf\   Because $\sequencen{A_nu}$ is norm-convergent to $Pu$,

$$\eqalign{\|TPu-Pu\|_{1,\infty}
&=\lim_{n\to\infty}\|TA_nu-A_nu\|_{1,\infty}\cr
&=\lim_{n\to\infty}\Bover1{n+1}\|T^{n+1}u-u\|_{1,\infty}
=0.  \text{ \Qed}\cr}$$

\noindent Thus, writing $U=\{u:Tu=u\}$, $P[M^{1,0}]=U$ and $Pu=u$ for
every $u\in U$.
}%end of proof of 372D

\leader{372E}{Corollary} Let $(\frak A,\bar\mu)$ be a measure algebra,
and $\pi:\frak A^f\to\frak A^f$ a measure-preserving ring homomorphism,
where $\frak A^f=\{a:\bar\mu a<\infty\}$.   Let $T:M^{1,0}\to M^{1,0}$
be the corresponding Riesz homomorphism, where
$M^{1,0}=M^{1,0}(\frak A,\bar\mu)$\cmmnt{ (366H, in particular part
(a-v))}.   Set
$A_n=\bover1{n+1}\sum_{i=0}^nT^i$ for $n\in\Bbb N$.   Then for every
$u\in M^{1,0}$, $\sequencen{A_nu}$ is order*-convergent and
$\|\,\|_{1,\infty}$-convergent to some $v$ such that $Tv=v$.

\proof{ By 366H(a-iv), $T\in\Cal T^{(0)}$, as defined in 371F.   So the
result follows at once from 372D.
}%end of proof of 372E

\leader{372F}{The Ergodic Theorem:  second form} Let $(\frak A,\bar\mu)$
be a measure algebra, and let $T:L^1\to L^1$, where
$L^1=L^1(\frak A,\bar\mu)$, be a linear operator of norm at most $1$
such that $Tu\in L^{\infty}=L^{\infty}(\frak A)$ and
$\|Tu\|_{\infty}\le\|u\|_{\infty}$
whenever $u\in L^1\cap L^{\infty}$.   Set
$A_n=\bover1{n+1}\sum_{i=0}^nT^i:L^1\to L^1$ for every $n$.   Then for
any $u\in L^1$, $\sequencen{A_nu}$ is order*-convergent to an element
$Pu$ of $L^1$.   The operator $P:L^1\to L^1$ is a projection of norm at
most $1$ onto the linear subspace $\{u:u\in L^1,\,Tu=u\}$.

\proof{ By 371Ga, there is an extension of $T$ to a member $\tilde T$ of
$\Cal T^{(0)}$.   So 372D tells us that $\sequencen{A_nu}$ is
order*-convergent to some $Pu\in L^1$ for every $u\in L^1$, and
$P:L^1\to L^1$ is a projection of norm at most $1$, because $P$ is the
restriction of a projection $\tilde P\in\Cal T^{(0)}$.   Also we still
have $TPu=Pu$ for every $u\in L^1$, and $Pu=u$ whenever $Tu=u$, so the
set of values $P[L^1]$ of $P$ must be exactly $\{u:u\in L^1,\,Tu=u\}$.
}%end of proof of 372F

\cmmnt{\medskip

\noindent{\bf Remark} In 372D and 372F I have used the phrase
`order*-convergent' from \S367 without always being specific about the
partially ordered set in which it is to be interpreted.   But, as
remarked in 367E, the notion is robust enough for the omission to be
immaterial here.   Since both $M^{1,0}$ and $L^1$ are solid linear
subspaces of $L^0$, a sequence in $M^{1,0}$ is order*-convergent to a
member of $M^{1,0}$ (when order*-convergence is interpreted in the
partially ordered set $M^{1,0}$) iff it is order*-convergent to the same
point (when convergence is interpreted in the set $L^0$);  and the same
applies to $L^1$ in place of $M^{1,0}$.
}%end of comment

\leader{372G}{Corollary} Let $(\frak A,\bar\mu)$ be a probability
algebra, and $\pi:\frak A\to\frak A$ a measure-preserving Boolean
homomorphism.   Let $T:L^1\to L^1$ be the corresponding Riesz
homomorphism, where $L^1=L^1(\frak A,\bar\mu)$.   Set
$A_n=\bover1{n+1}\sum_{i=0}^nT^i$ for $n\in\Bbb N$.   Then for every
$u\in L^1$, $\sequencen{A_nu}$ is order*-convergent and
$\|\,\,\|_1$-convergent.   If we set $Pu=\lim_{n\to\infty}A_nu$ for each
$u$, $P$ is the conditional expectation operator corresponding to the
fixed-point subalgebra $\frak C=\{a:\pi a=a\}$ of $\frak A$.

\proof{{\bf (a)} The first part is just a special case of 372E;  the
point is that because $(\frak A,\bar\mu)$ is totally finite,
$L^{\infty}(\frak A)\subseteq L^1$, so $M^{1,0}(\frak A,\bar\mu)=L^1$.
Also (because $\bar\mu 1=1$) $\|u\|_{\infty}\le\|u\|_1$ for every $u\in
L^{\infty}$, so the norm $\|\,\|_{1,\infty}$ is actually equal to
$\|\,\|_1$.

\medskip

{\bf (b)} For the last sentence, recall that $\frak C$ is a closed
subalgebra of $\frak A$ (cf.\ 333R).   By 372D or 372F, $P$ is a
projection operator onto the subspace $\{u:Tu=u\}$.   Now
$\Bvalue{Tu>\alpha}=\pi\Bvalue{u>\alpha}$ (365Oc), so $Tu=u$ iff
$\Bvalue{u>\alpha}\in\frak C$ for every $\alpha\in\Bbb R$, that is, iff
$u$ belongs to the canonical image of
$L^1(\frak C,\bar\mu\restrp\frak C)$ in $L^1$ (365R).   To identify $Pu$
further, observe that if $u\in L^1$ and $a\in\frak C$ then

\Centerline{$\int_aTu=\int_{\pi a}Tu=\int_au$}

\noindent (365Ob).   Consequently $\int_aT^iu=\int_au$ for every
$i\in\Bbb N$, $\int_aA_nu=\int_au$ for every $n\in\Bbb N$, and
$\int_aPu=\int_au$ (because $Pu$ is the limit of $\sequencen{A_nu}$ for
$\|\,\|_1$).   But this is enough to define $Pu$ as the conditional
expectation of $u$ on $\frak C$ (365R).
}%end of proof of 372G

\leader{372H}{}\cmmnt{ The Ergodic Theorem is most often expressed in
terms of transformations of measure spaces.   In the next few
corollaries I will formulate such expressions.   The translation is
straightforward.

\medskip

\noindent}{\bf Corollary\dvAformerly{3{}72I}}
Let $(X,\Sigma,\mu)$ be a measure space and
$\phi:X\to X$ an \imp\ function.   Let $f$ be a real-valued function
which is integrable over $X$.   Then

\Centerline{$g(x)
=\lim_{n\to\infty}\Bover1{n+1}\sum_{i=0}^nf(\phi^i(x))$}

\noindent is defined for almost every $x\in X$, and $g\phi(x)=g(x)$ for
almost every $x$.

\proof{ Let $(\frak A,\bar\mu)$ be the measure algebra of
$(X,\Sigma,\mu)$, and $\pi:\frak A\to\frak A$,
$T:L^0(\frak A)\to L^0(\frak A)$ the homomorphisms corresponding to
$\phi$, as in 364Qd.
Set $u=f^{\ssbullet}$ in $L^1(\frak A,\bar\mu)$.   Then for any
$i\in\Bbb N$, $T^iu=(f\phi^i)^{\ssbullet}$ (364Q(c)-(d)), so setting
$A_n=\bover1{n+1}\sum_{i=0}^nT^i$, $A_nu=g_n^{\ssbullet}$, where
$g_n(x)=\bover1{n+1}\sum_{i=0}^nf(\phi^i(x))$ whenever this is
defined.   Now we know from 372F or 372E that $\sequencen{A_nu}$ is
order*-convergent to some $v$ such that $Tv=v$, so $\sequencen{g_n}$
must be convergent almost everywhere (367F), and taking
$g=\lim_{n\to\infty}g_n$ where this is defined, $g^{\ssbullet}=v$.
Accordingly $(g\phi)^{\ssbullet}=Tv=v=g^{\ssbullet}$ and $g\phi\eae g$,
as claimed.
}%end of proof of 372H

\leader{372I}{}\cmmnt{ The following facts will be
useful in the next version of the theorem, and elsewhere.

\medskip

\noindent}{\bf Lemma} Let $(X,\Sigma,\mu)$ be a measure space with
measure algebra $(\frak A,\bar\mu)$.   Let $\phi:X\to X$ be an \imp\
function and $\pi:\frak A\to\frak A$ the associated
homomorphism\cmmnt{, as in 343A and 364Qd}.
Set $\frak C=\{c:c\in\frak A,\,\pi c=c\}$,
$\Tau=\{E:E\in\Sigma,\,\phi^{-1}[E]\symmdiff E$ is negligible$\}$ and
$\Tau_0=\{E:E\in\Sigma,\,\phi^{-1}[E]=E\}$.   Then $\Tau$ and $\Tau_0$
are $\sigma$-subalgebras of $\Sigma$;  $\Tau_0\subseteq\Tau$,
$\Tau=\{E:E\in\Sigma,\,E^{\ssbullet}\in \frak C\}$, and
$\frak C=\{E^{\ssbullet}:E\in \Tau_0\}$.

\proof{ It is easy to see that $\Tau$ and $\Tau_0$ are
$\sigma$-subalgebras of $\Sigma$ and that
$\Tau_0\subseteq\Tau=\{E:E^{\ssbullet}\in\frak C\}$.   So we have only
to check that if $c\in\frak C$ there is an
$E\in\Tau_0$ such that $E^{\ssbullet}=c$.   \Prf\ Start with any
$F\in\Sigma$ such that $F^{\ssbullet}=c$.
Now $F\symmdiff\phi^{-i}[F]$ is negligible for every $i\in\Bbb N$,
because $(\phi^{-i}[F])^{\ssbullet}=\pi^ic=c$.   So if we set

$$\eqalign{E
&=\bigcup_{n\in\Bbb N}\bigcap_{i\ge n}\phi^{-i}[F]\cr
&=\{x:\text{ there is an }n\in\Bbb N\text{ such that }\phi^i(x)\in F
  \text{ for every }i\ge n\},\cr}$$

\noindent $E^{\ssbullet}=c$.
On the other hand, it is easy to check that $E\in\Tau_0$.\   \Qed
}%end of proof of 372I

\leader{372J}{The Ergodic Theorem:  third form}\dvAformerly{3{}72K}
Let $(X,\Sigma,\mu)$ be a probability space and
$\phi:X\to X$ an \imp\ function.
Let $f$ be a real-valued function which is integrable over $X$.   Then

\Centerline{$g(x)
=\lim_{n\to\infty}\Bover1{n+1}\sum_{i=0}^nf(\phi^i(x))$}

\noindent is defined for almost every $x\in X$;  $g\phi\eae g$, and $g$
is a conditional expectation of $f$ on the $\sigma$-algebra
$\Tau=\{E:E\in\Sigma,\,\phi^{-1}[E]\symmdiff E$ is negligible$\}$.
If {\it either} $f$ is $\Sigma$-measurable and defined everywhere in $X$
{\it or} $\phi[E]$ is negligible for every negligible set $E$, then $g$
is a conditional expectation of $f$ on the $\sigma$-algebra
$\Tau_0=\{E:E\in\Sigma,\,\phi^{-1}[E]=E\}$.

\proof{{\bf (a)} We know by 372H that $g$ is defined almost everywhere
and that $g\phi\eae g$.   In the language of the proof of 372H,
$g^{\ssbullet}=v$ is the conditional expectation of $u=f^{\ssbullet}$ on
the closed subalgebra

\Centerline{$\frak C=\{a:a\in\frak A,\,\pi a=a\}
=\{F^{\ssbullet}:F\in\Tau\}
=\{F^{\ssbullet}:F\in\Tau_0\}$,}

\noindent by 372G and 372I.   So $v$ must be expressible as
$h^{\ssbullet}$ where $h:X\to\Bbb R$ is
$\Tau_0$-measurable and is a conditional expectation of $f$ on $\Tau_0$
(and also on $\Tau$).
Since every set of measure zero belongs to $\Tau$,
$g=h\,\,\mu\restrp\Tau$-a.e., and $g$ also is a conditional expectation
of $f$ on $\Tau$.

\medskip

{\bf (b)} Suppose now that $f$ is defined everywhere and
$\Sigma$-measurable.   Here I come to a technical
obstruction.   The definition of `conditional expectation' in 233D
asks for $g$ to be $\mu\restrp\Tau_0$-integrable, and since
$\mu$-negligible sets do not
need to be $\mu\restrp\Tau_0$-negligible we have some more checking to
do, to confirm that $\{x:x\in\dom g,\,g(x)=h(x)\}$ is
$\mu\restrp\Tau_0$-conegligible as well as $\mu$-conegligible.

\medskip

\quad{\bf (i)} For $n\in\Bbb N$, set
$\Sigma_n=\{\phi^{-n}[E]:E\in\Sigma\}$;  then
$\Sigma_n$ is a $\sigma$-subalgebra of $\Sigma$, including $\Tau_0$.
Set $\Sigma_{\infty}=\bigcap_{n\in\Bbb N}\Sigma_n$, still a
$\sigma$-algebra including $\Tau_0$.   Now any negligible set
$E\in\Sigma_{\infty}$ is $\mu\restrp\Tau_0$-negligible.   \Prf\ For each
$n\in\Bbb N$ choose $F_n\in\Sigma$ such that $E=\phi^{-n}[F_n]$.
Because $\phi$ is \imp, every $F_n$ is negligible, so that

\Centerline{$E^*=\bigcap_{m\in\Bbb N}\bigcup_{n\in\Bbb N,j\ge m}
\phi^{-j}[F_n]$}

\noindent is negligible.   Of course
$E=\bigcap_{m\in\Bbb N}\phi^{-m}[F_m]$ is included in $E^*$.   Now

\Centerline{$\phi^{-1}[E^*]
=\bigcap_{m\in\Bbb N}\bigcup_{n\in\Bbb N,j\ge m}\phi^{-j-1}[F_n]
=\bigcap_{m\ge 1}\bigcup_{n\in\Bbb N,j\ge m}\phi^{-j}[F_n]
=E^*$}

\noindent because

\Centerline{$\bigcup_{n\in\Bbb N,j\ge 1}\phi^{-j}[F_n]
\subseteq\bigcup_{n\in\Bbb N,j\ge 0}\phi^{-j}[F_n]$.}

\noindent So $E^*\in\Tau_0$ and $E$ is included in a negligible member
of $\Tau_0$, which is what we needed to know.\   \Qed

\medskip

\quad{\bf (ii)} We are assuming that $f$ is $\Sigma$-measurable and
defined everywhere, so that
$g_n=\bover1{n+1}\sum_{i=0}^nf\frsmallcirc\phi^i$ is
$\Sigma$-measurable and defined everywhere.   If we set
$g^*=\limsup_{n\to\infty}g_n$, then $g^*:X\to[-\infty,\infty]$ is
$\Sigma_{\infty}$-measurable.   \Prf\ For any $m\in\Bbb N$,
$f\frsmallcirc\phi^i$ is $\Sigma_m$-measurable for every $i\ge m$, since
$\{x:f(\phi^i(x))>\alpha\}=\phi^{-m}[\{x:f(\phi^{i-m}(x))>\alpha\}]$ for
every $\alpha$.   Accordingly

\Centerline{$g^*
=\limsup_{n\to\infty}\Bover1{n+1}\sum_{i=m}^nf\frsmallcirc\phi^i$}

\noindent is $\Sigma_m$-measurable.   As $m$ is arbitrary, $g^*$ is
$\Sigma_{\infty}$-measurable.\   \Qed

Since $h$ is surely $\Sigma_{\infty}$-measurable, and
$h=g^*\,\,\mu$-a.e., (i) tells us that $h=g^*\,\,\mu\restrp\Tau_0$-a.e.
But similarly $h=\liminf_{n\to\infty}g_n\,\,\mu\restrp\Tau_0$-a.e., so
we
must have $h=g\,\,\mu\restrp\Tau_0$-a.e.; and $g$, like $h$, is a
conditional expectation of $f$ on $\Tau_0$.

\medskip

{\bf (c)} Finally, suppose that $\phi[E]$ is negligible for every
negligible set $E$.
Then every $\mu$-negligible set is $\mu\restrp\Tau_0$-negligible.
\Prf\
If $E$ is $\mu$-negligible, then $\phi[E]$,
$\phi^2[E]=\phi[\phi[E]],\ldots$ are all negligible, so
$E^*=\bigcup_{n\in\Bbb N}\phi^n[E]$ is negligible, and there is a
measurable negligible set $F\supseteq E^*$.   Now $F_*=\bigcup_{m\in\Bbb
N}\bigcap_{n\ge m}\phi^{-n}[F]$ is a negligible set in $\Tau_0$
including $E$, so $E$ is $\mu\restrp\Tau_0$-negligible.\ \Qed\
Consequently $g=h\,\,\mu\restrp\Tau_0$-a.e., and in this case also $g$
is a conditional expectation of $f$ on $\Tau_0$.
}%end of proof of 372J

\cmmnt{
\leader{372K}{Remark} Parts (b)-(c) of the proof above are
dominated by the technical question of the exact definition of
`conditional expectation of $f$ on $\Tau_0$', and it is natural to be
impatient with such details.   The kind of example I am concerned about
is the following.   Let $C\subseteq[0,1]$ be the Cantor set (134G), and
$\phi:[0,1]\to[0,1]$ a Borel measurable function such that
$\phi[C]=[0,1]$ and $\phi(x)=x$ for $x\in[0,1]\setminus C$.   (For
instance, we could take $\phi$ agreeing with the Cantor function on $C$
(134H).)   Because $C$ is negligible, $\phi$ is \imp\ for Lebesgue
measure $\mu$, and if $f$ is any Lebesgue integrable function then
$g(x)=\lim_{n\to\infty}\Bover1{n+1}\sum_{i=0}^nf(\phi^i(x))$ is defined
and equal to $f(x)$ for every $x\in\dom f\setminus C$.   But for
$x\in C$ we can, by manipulating $\phi$, arrange for $g(x)$ to be almost
anything;  and if $f$ is undefined on $C$ then $g$ will also be
undefined on $C$.   On the other hand, $C$ is not
$\mu\restrp\Tau_0$-negligible, because the only member of $\Tau_0$
including $C$ is $[0,1]$.   So we cannot be sure of being able to form
$\int g\,d(\mu\restrp\Tau_0)$.

If instead of Lebesgue measure itself we took its restriction
$\mu_{\Cal B}$ to the algebra of Borel subsets of $[0,1]$, then $\phi$
would still
be \imp\ for $\mu_{\Cal B}$, but we should now have to worry about the
possibility that $f\restr C$ was non-measurable, so that $g\restr C$
came out to be non-measurable, even if everywhere defined, and $g$ was
not $\mu_{\Cal B}\restrp\Tau_0$-virtually measurable.

In the statement of 372J I have offered two ways of being sure that the
problem does not arise:  check that $\phi[E]$ is negligible whenever $E$
is negligible (so that all negligible sets are
$\mu\restrp\Tau_0$-negligible), or check that $f$ is defined everywhere
and $\Sigma$-measurable.   Even if these conditions are not immediately
satisfied in a particular application, it may be possible to modify the
problem so that they are.   For instance, completing the measure will
leave $\phi$ \imp\ (234Ba\formerly{2{}35Hc}),
will not change the integrable functions but
will make them all measurable (212F, 212Bc), and may enlarge $\Tau_0$
enough to
make a difference.   If our function $f$ is measurable (because the
measure is complete, or otherwise) we can extend it to a measurable
function defined everywhere (121I) and the corresponding extension of
$g$ will be $\mu\restrp\Tau_0$-integrable.   Alternatively, if the
difficulty seems to lie in the behaviour of $\phi$ rather than in the
behaviour of $f$ (as in the example above), it may help to modify $\phi$
on a negligible set.
}%end of comment

\leader{372L}{Continued fractions }\cmmnt{A particularly delightful
application of the results above is to a question which belongs as much
to number theory as to analysis.   It takes a bit of space to describe,
but I hope you will agree with me that it is well worth knowing in
itself, and that it also illuminates some of the ideas above.

\medskip

}{\bf (a)} Set $X=[0,1]\setminus\Bbb Q$.   For $x\in X$, set
$\phi(x)=\fraction{\bover1x}$, the fractional part of $\bover1x$, and
$k_1(x)=\bover1x-\phi(x)=\lfloor\bover1x\rfloor$, the integer part of
$\bover1x$;  then $\phi(x)\in X$ for each $x\in X$, so we may define
$k_n(x)=k_1(\phi^{n-1}(x))$ for every $n\ge 1$.   The strictly positive
integers $k_1(x)$, $k_2(x)$,
$k_3(x),\ldots$ are the {\bf continued fraction coefficients} of $x$.
\cmmnt{Of course }$k_{n+1}(x)=k_n(\phi(x))$ for every $n\ge 1$.
Now define
$\sequencen{p_n(x)}$, $\sequencen{q_n(x)}$ inductively by setting

\Centerline{$p_0(x)=0$,
\quad $p_1(x)=1$,
\quad$p_{n}(x)=p_{n-2}(x)+k_n(x)p_{n-1}(x)$ for $n\ge 1$,}

\Centerline{$q_0(x)=1$,
\quad$q_1(x)=k_1(x)$,
\quad$q_{n}(x)=q_{n-2}(x)+k_n(x)q_{n-1}(x)$ for $n\ge 1$.}

\noindent The {\bf continued fraction approximations}\cmmnt{ or
{\bf convergents}} to $x$ are the quotients $p_n(x)/q_n(x)$.

\cmmnt{(I do not discuss rational $x$, because for my purposes here
these are merely distracting.   But if we set $k_1(0)=\infty$,
$\phi(0)=0$ then the formulae above produce the conventional values for
$k_n(x)$ for rational $x\in\coint{0,1}$.   As for the $p_n$ and $q_n$,
use the formulae above until you get to $x=p_n(x)/q_n(x)$,
$\phi^n(x)=0$, $k_{n+1}(x)=\infty$, and then  set $p_{m}(x)=p_{n}(x)$,
$q_{m}(x)=q_{n}(x)$ for $m\ge n$.)
}

\cmmnt{\spheader 372Lb The point is that the quotients
$r_n(x)=p_n(x)/q_n(x)$ are, in a strong sense, good rational
approximations to $x$.   (See 372Xl(v).)   We have
$r_n(x)<x<r_{n+1}(x)$
for every even $n$ (372Xl).   If $x=\pi - 3$, then the first few
coefficients are

\Centerline{$k_1=7$,\quad $k_2=15$, \quad$k_3=1$,}
%then  292  1  1  1  2  1  3  1  14  2  1  1  2
%http://functions.wolfram.com/Constant/Pi/10

\Centerline{$r_1=\Bover17$,  \quad$r_2=\Bover{15}{106}$,
\quad$r_3=\Bover{16}{113}$;}

\noindent the first and third of these corresponding to the classical
approximations $\pi\bumpeq\Bover{22}7$, $\pi\bumpeq\Bover{355}{113}$.
Or if we take $x=e-2$, we get

\Centerline{$k_1=1$, \quad$k_2=2$, \quad$k_3=1$, \quad$k_4=1$,
\quad$k_5=4$, \quad$k_6=1$, \quad$k_7=1$,}
%then  6  1  1  8  1  1  10  1  1  12  1  1  ...

\Centerline{$r_1=1$,  \quad$r_2=\Bover23$,
\quad$r_3=\Bover34$, \quad$r_4=\Bover57$, \quad$r_5=\Bover{23}{32}$,
\quad$r_6=\Bover{28}{39}$, \quad$r_7=\Bover{51}{71}$;}

\noindent note that the obvious approximations $\bover{17}{24}$,
$\bover{86}{120}$ derived from the series for $e$ are not in fact as
close as the even terms $\bover57$, $\bover{28}{39}$ above, and involve
larger numbers\footnote{There is a remarkable expression for the continued
fraction expansion of $e$, due essentially to Euler;
$k_{3m-1}=2m$, $k_{3m}=k_{3m+1}=1$ for $m\ge 2$.  See {\smc Cohn 06}.}.
}

\spheader 372Lc\cmmnt{ Now we need a variety of miscellaneous facts
about these coefficients, which I list here.

\medskip

\quad}{\bf (i)} For any $x\in X$, $n\ge 1$ we have

\Centerline{$p_{n-1}(x)q_n(x)-p_n(x)q_{n-1}(x)=(-1)^n$,
\quad$\phi^n(x)
=\Bover{p_n(x)-xq_n(x)}{xq_{n-1}(x)-p_{n-1}(x)}$\dvro{,}{}}

\cmmnt{\noindent (induce on $n$), so}

\Centerline{$x=\Bover{p_n(x)+p_{n-1}(x)\phi^n(x)}
{q_n(x)+q_{n-1}(x)\phi^n(x)}$.}

\medskip

\quad{\bf (ii)}\dvro{ For}{ Another easy induction on $n$ shows that
for} any finite string $\tbf{m}=(m_1,\ldots,m_n)$ of strictly positive
integers the set $D_{\tbf{m}}=\{x:x\in X,\,k_i(x)=m_i$ for $1\le i\le
n\}$ is an interval in $X$ on which $\phi^n$ is monotonic, being
strictly increasing if $n$ is even and strictly decreasing if $n$ is
odd.   \prooflet{(For the inductive step, note just that

\Centerline{$D_{(m_1,\ldots,m_n)}
=[\bover1{m_1+1},\bover1{m_1}]
\cap\phi^{-1}[D_{(m_2,\ldots,m_n)}]$.)}
}

\medskip

\quad{\bf (iii)} We also need to know that\cmmnt{ the intervals
$D_{\tbf{m}}$
of (ii) are small;  specifically, that} if $\tbf{m}=(m_1,\ldots,m_n)$,
the length of $D_{\tbf{m}}$ is at most $2^{-n+1}$.   \prooflet{\Prf\ All
the coefficients $p_i$, $q_i$, for $i\le n$, take  constant values
$p_i^*$, $q_i^*$ on $D_{\tbf{m}}$, since they are determined from the
coefficients $k_i$ which are constant on $D_{\tbf{m}}$ by definition.
Now every $x\in D_{\tbf{m}}$ is of the form
$(p^*_n+tp^*_{n-1})/(q^*_n+tq^*_{n-1})$ for some $t\in X$ (see (i)
above) and therefore lies between
$p^*_{n-1}/q^*_{n-1}$ and $p^*_n/q^*_n$.   But the distance between
these is

\Centerline{$\bigl|\Bover{p^*_nq^*_{n-1}-p^*_{n-1}q^*_n}
{q^*_nq^*_{n-1}}\bigr|
=\Bover1{q^*_nq^*_{n-1}}$,}

\noindent by the first formula in (i).   Next, noting that $q^*_{i}\ge
q^*_{i-1}+q^*_{i-2}$ for each $i\ge 2$, we see that $q^*_iq^*_{i-1}\ge
2q^*_{i-1}q^*_{i-2}$ for $i\ge 2$, and therefore that $q_n^*q_{n-1}^*\ge
2^{n-1}$, so that the length of $D_{\tbf{m}}$ is at most
$2^{-n+1}$.\ \Qed
}%end of prooflet

\leader{372M}{Theorem} Set $X=[0,1]\setminus\Bbb Q$, and define
$\phi:X\to X$ as in 372L.   Then for every Lebesgue integrable function
$f$ on $X$,

\Centerline{$\lim_{n\to\infty}\Bover1{n+1}\sum_{i=0}^nf(\phi^i(x))
=\Bover1{\ln 2}\int_{0}^{1}\Bover{f(t)}{1+t}dt$}

\noindent for almost every $x\in X$.

\proof{{\bf (a)} The integral just written, and the phrase `almost
every', refer of course to Lebesgue measure;  but the first step is to
introduce another measure, so I had better give a name $\mu_L$ to
Lebesgue measure on $X$.   Let $\nu$ be the indefinite-integral measure
on $X$ defined by saying that $\nu E=\bover1{\ln
2}\int_E\bover1{1+x}\mu_L(dx)$ whenever this is defined.   The
coefficient $\bover1{\ln 2}$ is of course chosen to make $\nu X=1$.
Because $\bover1{1+x}>0$ for every $x\in X$, $\dom\nu=\dom\mu_L$ and
$\nu$ has just the same negligible sets as $\mu_L$
(234Lc\formerly{2{}34D});   I can
therefore safely use the terms `measurable set', `almost
everywhere' and `negligible' without declaring which measure I have
in mind each time.

\medskip

{\bf (b)} Now $\phi$ is \imp\ when regarded as a function from $(X,\nu)$
to itself.   \Prf\ For each $k\ge 1$, set
$I_k=\coint{\bover1{k+1},\bover1{k}}$.   On $X\cap I_k$,
$\phi(x)=\bover1x-k$.   Observe that $\phi\restr I_k:X\cap I_k\to X$ is
bijective and differentiable relative to its domain in the sense of
262Fb.   Consider, for any measurable $E\subseteq X$,

$$\eqalign{\int_E\Bover1{(y+k)(y+k+1)}\mu_L(dy)
&=\int_{I_k\cap\phi^{-1}[E]}
\Bover1{(\phi(x)+k)(\phi(x)+k+1)}|\phi'(x)|\mu_L(dx)\cr
&=\int_{I_k\cap\phi^{-1}[E]}
\Bover{x^2}{x+1}\Bover1{x^2}\mu_L(dx)
=\ln 2\cdot\nu(I_k\cap\phi^{-1}[E]),\cr}$$

\noindent using 263D (or more primitive results, of course).   But

\Centerline{$\sum_{k=1}^{\infty}\Bover1{(y+k)(y+k+1)}
=\sum_{k=1}^{\infty}\Bover1{y+k}-\Bover1{y+k+1}=\Bover1{y+1}$}

\noindent for every $y\in[0,1]$, so

$$\nu E=\Bover1{\ln
2}\sum_{k=1}^{\infty}\int_E\Bover1{(y+k)(y+k+1)}\mu_L(dy)
=\sum_{k=1}^{\infty}\nu(I_k\cap\phi^{-1}[E])=\nu\phi^{-1}[E].$$

\noindent As $E$ is arbitrary, $\phi$ is \imp.\   \Qed

\medskip

{\bf (c)} The next thing we need to know is that if $E\subseteq X$ and
$\phi^{-1}[E]=E$ then $E$ is either negligible or conegligible.   \Prf\
I use the sets $D_{\tbf{m}}$ of 372L(c-ii).

\medskip

\quad{\bf (i)} For any string $\tbf{m}=(m_1,\ldots,m_n)$ of strictly
positive integers, we have

\Centerline{$x=\Bover{p^*_n+p^*_{n-1}\phi^n(x)}
{q_n^*+q_{n-1}^*\phi^n(x)}$}

\noindent for every $x\in D_{\tbf{m}}$, where $p_n^*$, etc., are defined
from $\tbf{m}$ as in 372L(c-iii).   Recall also that $\phi^n$ is
strictly monotonic on $D_{\tbf{m}}$.   So for any interval
$I\subseteq[0,1]$ (open, closed or half-open) with endpoints
$\alpha<\beta$, $\phi^{-n}[I]\cap D_{\tbf{m}}$ will be of the form
$X\cap J$, where $J$ is an interval with endpoints
$(p^*_n+p^*_{n-1}\alpha)/(q^*_n+q^*_{n-1}\alpha)$,
$(p^*_n+p^*_{n-1}\beta)/(q^*_n+q^*_{n-1}\beta)$ in some order.   This
means that we can estimate $\mu_L(\phi^{-n}[I]\cap
D_{\tbf{m}})/\mu_LD_{\tbf{m}}$, because it is

$$\bover{
\bigl|\bover{p^*_n+p^*_{n-1}\alpha}{q^*_n+q^*_{n-1}\alpha}
-\bover{p^*_n+p^*_{n-1}\beta}{q^*_n+q^*_{n-1}\beta}\bigr|}
{\bigl|\bover{p^*_n}{q^*_n}
-\bover{p^*_n+p^*_{n-1}}{q_n^*+q^*_{n-1}}\bigr|}
=\Bover{(\beta-\alpha)q^*_n(q_n^*+q^*_{n-1})}
{(q_n^*+q_{n-1}^*\alpha)(q_n^*+q_{n-1}^*\beta)}
\ge\Bover{(\beta-\alpha)q_n^*}{q_n^*+q_{n-1}^*}
\ge\Bover12(\beta-\alpha).$$

\noindent Now look at

\Centerline{$\Cal A=\{E:E\subseteq [0,1]$ is Lebesgue measurable,
 $\mu_L(\phi^{-n}[E]\cap D_{\tbf{m}})
  \ge\Bover12\mu_LE\cdot\mu_LD_{\tbf{m}}\}$.}

\noindent Clearly the union of two disjoint members of $\Cal A$ belongs
to $\Cal A$.   Because $\Cal A$ contains every subinterval of $[0,1]$
it includes the algebra $\Cal E$ of subsets of $[0,1]$ consisting of
finite unions of intervals.   Next, the union of any
non-decreasing sequence in $\Cal A$ belongs to $\Cal A$, and the
intersection of a non-increasing sequence likewise.   But this means
that $\Cal A$ must include the
$\sigma$-algebra generated by $\Cal E$ (136G), that is, the Borel
$\sigma$-algebra.   But also, if $E\in\Cal A$ and $H\subseteq[0,1]$ is
negligible, then

\Centerline{$\mu_L(\phi^{-n}[E\symmdiff H]\cap D_{\tbf{m}})
=\mu_L(\phi^{-n}[E]\cap D_{\tbf{m}})
\ge\Bover12\mu_LE\cdot\mu_LD_{\tbf{m}}
=\Bover12\mu_L(E\symmdiff H)\cdot\mu_LD_{\tbf{m}}$}

\noindent and $E\symmdiff H\in\Cal A$.   And this means that every
Lebesgue measurable subset of $[0,1]$ belongs to $\Cal A$ (134Fb).

\medskip

\quad{\bf (ii)} \Quer\ Now suppose, if possible, that $E$ is a
measurable subset of $X$ and that $\phi^{-1}[E]=E$ and $E$ is neither
negligible nor conegligible in $X$.    Set $\gamma=\bover12\mu_LE>0$.
By Lebesgue's density theorem (223B) there is some $x\in X\setminus E$
such that $\lim_{\delta\downarrow 0}\psi(\delta)=0$, where
$\psi(\delta)=\Bover1{2\delta}\mu_L(E\cap[x-\delta,x+\delta])$ for
$\delta>0$.
Take $n$ so large that $\psi(\delta)<\bover12\gamma$ whenever
$0<\delta\le 2^{-n+1}$, and set $m_i=k_i(x)$ for $i\le n$, so that $x\in
D_{\tbf{m}}$.   Taking the least $\delta$ such that
$D_{\tbf{m}}\subseteq[x-\delta,x+\delta]$, we must have
$\delta\le 2^{-n+1}$, because the length of $D_{\tbf{m}}$ is at most
$2^{-n+1}$ (372L(c-iii)), while $\mu_LD_{\tbf{m}}\ge\delta$, because
$D_{\tbf{m}}$ is an interval.   Accordingly

\Centerline{$\mu_L(E\cap D_{\tbf{m}})\le\mu_L(E\cap[x-\delta,x+\delta])
=2\delta\psi(\delta)<\gamma\delta\le\gamma\mu_LD_{\tbf{m}}$.}

\noindent But we also have

\Centerline{$\mu_L(E\cap D_{\tbf{m}})=\mu_L(\phi^{-n}[E]\cap
D_{\tbf{m}})\ge\gamma\mu_L D_{\tbf{m}}$,}

\noindent by (i).\   \Bang

This proves the result.\   \Qed

\medskip

{\bf (d)} The final fact we need in preparation is that $\phi[E]$ is
negligible for every negligible $E\subseteq X$.   This is because $\phi$
is differentiable relative to its domain (see 263D(ii)).

\medskip

{\bf (e)} Now let $f$ be any $\mu_L$-integrable function.   Because
$\bover1{1+x}\le 1$ for every $x$, $f$ is also $\nu$-integrable
(235K\formerly{2{}35M});   consequently, using (b) above and 372J,

\Centerline{$g(x)
=\lim_{n\to\infty}\Bover1{n+1}\sum_{i=0}^nf(\phi^i(x))$}

\noindent is defined for almost every $x\in X$, and is a conditional
expectation of $f$ (with respect to the measure $\nu$) on the
$\sigma$-algebra $\Tau_0=\{E:E$ is measurable, $\phi^{-1}[E]=E\}$.
But we have just seen that $\Tau_0$ consists only of negligible and
conegligible sets, so $g$ must be essentially constant;  since
$\int g\,d\nu=\int fd\nu$, we must have

$$\lim_{n\to\infty}\Bover1{n+1}\sum_{i=0}^nf(\phi^i(x))
=\int fd\nu
=\Bover1{\ln 2}\int_0^1\Bover{f(t)}{1+t}\mu_L(dt)$$

\noindent for almost every $x$ (using 235K to calculate $\int fd\nu$).
}%end of proof of 372M

\leader{372N}{Corollary} For almost every $x\in [0,1]\setminus\Bbb Q$,

\Centerline{$\lim_{n\to\infty}\Bover1n\#(\{i:1\le i\le n,\,k_i(x)=k\})
=\Bover1{\ln 2}(2\ln(k+1)-\ln k-\ln(k+2))$}

\noindent for every $k\ge 1$, where $k_1(x),\ldots$ are the continued
fraction coefficients of $x$.

\proof{ In 372M, set $f=\chi(X\cap[\bover1{k+1},\bover1k])$.   Then (for
$i\ge 1$) $f(\phi^i(x))=1$ if $k_i(x)=k$ and zero otherwise.   So

$$\eqalign{\lim_{n\to\infty}&\Bover1n\#(\{i:1\le i\le n,\,k_i(x)=k\})\cr
&=\lim_{n\to\infty}\Bover1{n}\sum_{i=1}^nf(\phi^i(x))
=\lim_{n\to\infty}\Bover1{n+1}\sum_{i=0}^nf(\phi^i(x))\cr
&=\Bover1{\ln 2}\int_0^1\Bover{f(t)}{1+t}dt
=\Bover1{\ln 2}\int_{1/k+1}^{1/k}\Bover1{1+t}dt\cr
&=\Bover1{\ln 2}(\ln(1+\Bover1k)-\ln(1+\Bover1{k+1}))
=\Bover1{\ln 2}(2\ln(k+1)-\ln k-\ln(k+2)),\cr}$$

\noindent for almost every $x\in X$.
}%end of proof of 372N

\leader{372O}{Mixing and
ergodic \dvrocolon{transformations}}\cmmnt{ This seems an
appropriate moment for some brief notes on three special types of
measure-preserving homomorphism or \imp\ function.

\medskip

\noindent}{\bf Definitions (a)(i)}\discrversionA{\footnote{Amended
2010.}}{} Let $\frak A$ be a Boolean algebra.
Then a Boolean homomorphism $\pi:\frak A\to\frak A$ is {\bf ergodic} if
whenever $a$, $b\in\frak A\setminus\{0\}$ there are $m$, $n\in\Bbb N$ such
that $\pi^ma\Bcap\pi^nb\ne 0$.

\medskip

\quad{\bf (ii)} Let $(\frak A,\bar\mu)$ be a probability
algebra and $\pi:\frak A\to\frak A$ a measure-preserving Boolean
homomorphism.   Then $\pi$ is {\bf mixing}\cmmnt{ (sometimes called {\bf
strongly mixing})}
if $\lim_{n\to\infty}\bar\mu(\pi^na\Bcap b)=\bar\mu a\cdot\bar\mu b$ for
all $a$, $b\in\frak A$.

\medskip

\quad{\bf (iii)}\dvAnew{2011} Let $(\frak A,\bar\mu)$ be a probability
algebra and $\pi:\frak A\to\frak A$ a measure-preserving Boolean
homomorphism.   Then $\pi$ is {\bf weakly mixing}
if $\lim_{n\to\infty}\Bover1n\sum_{i=0}^{n-1}
|\bar\mu(\pi^na\Bcap b)-\bar\mu a\cdot\bar\mu b|=0$ for
all $a$, $b\in\frak A$.

\medskip

\spheader 372Ob Let $(X,\Sigma,\mu)$ be a probability space and
$\phi:X\to X$ an \imp\ function.

\medskip

\quad{\bf (i)} $\phi$ is {\bf ergodic}\cmmnt{ (also called {\bf
metrically
transitive}, {\bf indecomposable})} if every measurable set $E$ such
that $\phi^{-1}[E]=E$ is either negligible or conegligible.

\medskip

\quad{\bf (ii)} $\phi$ is {\bf mixing} if
$\lim_{n\to\infty}\mu(F\cap\phi^{-n}[E])=\mu E\cdot\mu F$ for all $E$,
$F\in\Sigma$.

\medskip

\quad{\bf (iii)}\dvAnew{2011} $\phi$ is {\bf weakly mixing} if
if $\lim_{n\to\infty}\Bover1n\sum_{i=0}^{n-1}
|\mu(F\cap\phi^{-n}[E])-\mu E\cdot\mu F|=0$ for
all $E$, $F\in\Sigma$.


\leader{372P}{}\dvAnew{2010}\cmmnt{ For the principal applications of the
idea in 372O(a-i),
we have an alternative definition in terms of fixed-point subalgebras.

\medskip

\noindent}{\bf Proposition} Let $\frak A$ be a Boolean algebra and
$\pi:\frak A\to\frak A$ a Boolean homomorphism, with fixed-point subalgebra
$\frak C$.

(a) If $\pi$ is ergodic, then $\frak C=\{0,1\}$.

(b) If $\pi$ is an automorphism, then
$\pi$ is ergodic iff
$\sup_{n\in\Bbb Z}\pi^na=1$ for every $a\in\frak A\setminus\{0\}$.

(c) If $\pi$ is an automorphism and $\frak A$ is Dedekind
$\sigma$-complete, then $\pi$ is ergodic iff $\frak C=\{0,1\}$.

\proof{{\bf (a)} If $c\in\frak C$, then $\pi^mc=c$ is disjoint from
$\pi^n(1\Bsetminus c)=1\Bsetminus c$ for all $m$, $n\in\Bbb N$, so one of
$c$, $1\Bsetminus c$ must be zero.

\medskip

{\bf (b)(i)} If $\pi$ is ergodic and $a\ne 0$ and $b\Bcap\pi^na=0$ for
every $n\in\Bbb Z$, then $\pi^mb\Bcap\pi^na=\pi^m(b\Bcap\pi^{n-m}a)=0$ for
all $m$, $n\in\Bbb N$, so $b=0$.   As $b$ is arbitrary,
$\sup_{n\in\Bbb Z}\pi^na=1$;  as $a$ is arbitrary, $\pi$ satisfies the
condition.

\medskip

\quad{\bf (ii)} If $\pi$ satisfies the condition, and $a$,
$b\in\frak A\setminus\{0\}$, then there is an $m\in\Bbb Z$ such that
$\pi^ma\Bcap b\ne 0$;  setting $n=\max(-m,0)$,
$\pi^{m+n}a\Bcap\pi^nb\ne 0$, while $m+n$ and $n$ both belong to $\Bbb N$.
As $a$ and $b$ are arbitrary, $\pi$ is ergodic.

\medskip

{\bf (c)} If $\pi$ is ergodic then $\frak C=\{0,1\}$, by (a).
If $\frak C=\{0,1\}$ and $a\in\frak A\setminus\{0\}$, consider
$c=\sup_{n\in\Bbb Z}\pi^na$, which is defined because $\frak A$ is Dedekind
$\sigma$-complete.   Being an automorphism, $\pi$ is
order-continuous (313Ld), so
$\pi c=\sup_{n\in\Bbb Z}\pi^{n+1}a=c$ and $c\in\frak C$.
Since $c\Bsupseteq a$ is non-zero, $c=1$.   As $a$ is arbitrary,
$\pi$ is ergodic, by (b).
}%end of proof of 372P

\leader{372Q}{}\cmmnt{ The following facts are equally straightforward.

\medskip

\noindent}{\bf Proposition}\dvArevised{ and expanded 2011}
(a) Let $(\frak A,\bar\mu)$ be a probability
algebra, $\pi:\frak A\to\frak A$ a measure-preserving Boolean
homomorphism, and $T:L^0=L^0(\frak A)\to L^0$ the Riesz homomorphism such
that $T(\chi a)=\chi\pi a$ for every $a\in\frak A$.

\quad(i) If $\pi$ is mixing, it is weakly mixing.

\quad(ii) If $\pi$ is weakly mixing, it is ergodic.

\quad(iii) The following
are equiveridical:  ($\alpha$) $\pi$ is
ergodic;  ($\beta$) the only $u\in L^0$ such that $Tu=u$ are the
multiples of
$\chi 1$;   ($\gamma$) for every $u\in L^1=L^1(\frak A,\bar\mu)$,
$\sequencen{\bover1{n+1}\sum_{i=0}^nT^iu}$ order*-converges to
$(\int u)\chi 1$.

\quad(iv)
The following are equiveridical:  ($\alpha$) $\pi$ is mixing;
($\beta$) $\lim_{n\to\infty}\innerprod{T^nu}{v}=\int u\int v$ for all $u$,
$v\in L^2(\frak A,\bar\mu)$.

\quad(v)
The following are equiveridical:  ($\alpha$) $\pi$ is weakly mixing;
($\beta$)
$\lim_{n\to\infty}\Bover1n\sum_{k=0}^{n-1}
  |\innerprod{T^ku}{v}-\int u\int v|=0$ for all $u$,
$v\in L^2(\frak A,\bar\mu)$.

(b) Let $(X,\Sigma,\mu)$ be a probability space, with measure algebra
$(\frak A,\bar\mu)$.   Let $\phi:X\to X$ be an \imp\ function and
$\pi:\frak A\to\frak A$ the associated homomorphism such that
$\pi E^{\ssbullet}=(\phi^{-1}[E])^{\ssbullet}$ for every $E\in\Sigma$.

\quad (i) The following are equiveridical:  ($\alpha$) $\phi$ is
ergodic;
($\beta$) $\pi$ is ergodic;  ($\gamma$) for every $\mu$-integrable
real-valued function $f$,
$\sequencen{\bover1{n+1}\sum_{i=0}^nf(\phi^i(x))}$ converges to $\int f$
for almost every $x\in X$.

\quad (ii) $\phi$ is mixing iff $\pi$ is, and in this case $\phi$ is
weakly mixing.

\quad (iii) $\phi$ is weakly mixing iff $\pi$ is, and in this case $\phi$
is ergodic.

\proof{{\bf (a)(i)-(ii)} Immediate from the definitions.

\medskip

\quad{\bf (iii)\grheada$\Rightarrow$\grheadb} $Tu=u$ iff
$\pi\Bvalue{u>\alpha}=\Bvalue{u>\alpha}$ for
every $\alpha$;  if $\pi$ is ergodic, this means that
$\Bvalue{u>\alpha}\in\{0,1\}$ for every $\alpha$, by 372Pa,
and $u$ must be of the
form $\gamma\chi 1$, where $\gamma=\inf\{\alpha:\Bvalue{u>\alpha}=0\}$.

\medskip

\qquad{\bf \grheadb$\Rightarrow$\grheadc} If ($\beta$) is true and
$u\in L^1$, then we know from 372G that
$\sequencen{\bover1{n+1}\sum_{i=0}^nT^iu}$ is order*-convergent and
$\|\,\|_1$-convergent to some $v$ such that $Tv=v$;  by ($\beta$), $v$
is of the form $\gamma\chi 1$;  and

\Centerline{$\gamma
=\int v=\lim_{n\to\infty}\Bover1{n+1}\sum_{i=0}^n\int T^iu=\int
u$.}

\qquad\grheadc$\Rightarrow$\grheada\  Assuming ($\gamma$), take any
$a\in\frak A$ such that $\pi a=a$, and consider $u=\chi a$.   Then
$T^iu=\chi a$ for every $i$, so

\Centerline{$\chi a=\lim_{n\to\infty}\Bover1{n+1}\sum_{i=0}^nT^iu
=(\int u)\chi 1=\bar\mu a\cdot\chi 1$,}

\noindent and $a$ must be either $0$ or $1$.   By 372Pc, $\pi$ is ergodic.

\medskip

\quad{\bf (iv)}\grheada$\Rightarrow$\grheadb\ Since $\pi$ is mixing,

$$\eqalign{\lim_{n\to\infty}\innerprod{T^n\chi a}{\chi b}
&=\lim_{n\to\infty}\innerprod{\chi\pi^n a}{\chi b}
=\lim_{n\to\infty}\bar\mu(\pi^na\Bcap b)\cr
&=\bar\mu a\cdot\bar\mu b
=\int\chi a\int\chi b\cr}$$

\noindent for all $a$, $b\in\frak A$.   Because
$(u,v)\mapsto\innerprod{T^nu}{v}$ and $(u,v)\mapsto\int u\int v$ are
both bilinear,

\Centerline{$\lim_{n\to\infty}\innerprod{T^nu}{v}=\int u\int v$}

\noindent for all $u$, $v\in S(\frak A)$.   For general $u$,
$v\in L^2(\frak A,\bar\mu)$, take any $\epsilon>0$.   Then there are $u'$,
$v'\in S(\frak A)$ such that

\Centerline{$(\|u-u'\|_2+\|v-v'\|_2)\max(\|u\|_2,\|v\|_2+\|v-v'\|_2)
\le\epsilon$}

\noindent (366C), so that

$$\eqalignno{|\innerprod{T^nu}{v}-\innerprod{T^nu'}{v'}|
&\le|\innerprod{T^nu}{v-v'}|+|\innerprod{T^nu-T^nu'}{v'}|\cr
&\le\|T^nu\|_2\|v-v'\|_2+\|T^nu-T^nu'\|_2\|v'\|_2\cr
&\le\|u\|_2\|v-v'\|_2+\|u-u'\|_2(\|v\|_2+\|v-v'\|_2)\cr
\displaycause{366H(a-iv)}
&\le\epsilon,\cr
|\int u\int v-\int u'\int v'|
&\le|\int u||\int v-v'|+|\int u-u'||\int v'|\cr
&\le\|u\|_2\|v-v'\|_2+\|u-u'\|_2\|v'\|_2
\le\epsilon\cr}$$

\noindent for every $n$, and

\Centerline{$\limsup_{n\to\infty}|\innerprod{T^nu}{v}-\int u\int v|
\le 2\epsilon+\lim_{n\to\infty}|\innerprod{T^nu'}{v'}-\int u'\int v'|
=2\epsilon$.}

\noindent As $\epsilon$ is arbitrary,
$\lim_{n\to\infty}\innerprod{T^nu}{v}=\int u\int v$, as required.

\medskip

\qquad\grheadb$\Rightarrow$\grheada\ This is elementary, as ($\alpha$) is
just the case $u=\chi a$, $v=\chi b$ of ($\beta$).

\medskip

\quad{\bf (v)} The argument is essentially the same as in (iv);
($\alpha$) is a special case of ($\beta$);  if ($\alpha$) is true, then by
linearity ($\beta$) is true when $u$, $v\in S(\frak A)$, and the functional
$(u,v)\mapsto\limsup_{n\to\infty}\Bover1n\sum_{k=0}^{n-1}
   |\innerprod{T^ku}{v}-\int u\int v|$ is continuous.

\medskip

{\bf (b)(i)}\grheada$\Rightarrow$\grheadb\  If $\pi a=a$
there is an $E$ such that $\phi^{-1}[E]=E$ and $E^{\ssbullet}=a$, by
372I;  now $\bar\mu a=\mu E\in\{0,1\}$, so $a\in\{0,1\}$.   Thus
the fixed-point subalgebra of $\pi$ is $\{0,1\}$;  by
372Pc again, $\pi$ is ergodic.

\medskip

\qquad\grheadb$\Rightarrow$\grheadc\ Set $u=f^{\ssbullet}\in L^1$.   In
the language of (a), $T^iu=(f\phi^i)^{\ssbullet}$ for each $i$, as in the
proof of 372H, so that

\Centerline{$(\Bover1{n+1}\sum_{i=0}^nf\phi^i)^{\ssbullet}
=\Bover1{n+1}\sum_{i=0}^nT^iu$}

\noindent is order*-convergent to $(\int u)\chi 1=(\int f)\chi 1$, and
$\bover1{n+1}\sum_{i=0}^nf\phi^i\to\int f$ a.e.

\medskip

\qquad\grheadc$\Rightarrow$\grheada\ If $\phi^{-1}[E]=E$ then, applying
($\gamma$) to $f=\chi E$, we see that $\chi E\eae\mu E\cdot\chi X$, so
that $E$ is either negligible or conegligible.

\medskip

\quad{\bf (ii)-(iii)} Simply translating the definitions, we see that
$\pi$ is mixing, or weakly mixing, iff $\phi$ is.
So the results here are reformulations of (a-i) and (a-ii).
}%end of proof of 372Q

\cmmnt{\leader{372R}{Remarks}\dvAformerly{3{}72Pc}
{\bf (a)} The reason for introducing
`ergodic' homomorphisms in this section is of course 372G/372J;  if
$\pi$ in 372G, or $\phi$ in 372J, is ergodic, then the limit $Pu$ or $g$
must be (essentially) constant, being a conditional expectation on a
trivial subalgebra.

\spheader 372Rb In the definition 372O(b-i) I should perhaps emphasize that
we look only at {\it measurable} sets $E$.   We certainly expect that
there will generally be many sets $E$ for which $\phi^{-1}[E]=E$, since
any union of orbits of $\phi$ will have this property.

\spheader 372Rc Part (c) of the proof of 372M was devoted to showing
that the function $\phi$ there was ergodic;  see also 372Xm.   For
another ergodic transformation see 372Xr.   For examples of mixing
transformations see 333P, 372Xp, 372Xq, 372Xt, 372Xw and 372Xx.

\spheader 372Rd It seems to be difficult to display explicitly
a weakly mixing transformation which is not mixing.   There is an
example in {\smc Chacon 69}, and I give another in 494F in Volume 4.
In a certain sense, however, `most' measure-preserving automorphisms of the
Lebesgue probability algebra are weakly mixing but not mixing;  I will
return to this in 494E.
}%end of comment

\leader{372S}{}\cmmnt{ There is a useful sufficient condition for a
homomorphism or function to be mixing.

\medskip

\noindent}{\bf Proposition} (a) Let $(\frak A,\bar\mu)$ be a probability
algebra, and $\pi:\frak A\to\frak A$ a measure-preserving Boolean
homomorphism.   If $\bigcap_{n\in\Bbb N}\pi^n[\frak A]=\{0,1\}$, then
$\pi$ is mixing.

(b) Let $(X,\Sigma,\mu)$ be a probability space, and $\phi:X\to X$ an
\imp\ function.   Set

\Centerline{$\Tau=\{E:$ for every $n\in\Bbb N$ there is an $F\in\Sigma$
such that $E=\phi^{-n}[F]\}$.}

\noindent If every member of $\Tau$ is either negligible or
conegligible, $\phi$ is mixing.

\proof{{\bf (a)} Let $T:L^0=L^0(\frak A)\to L^0$ be the Riesz
homomorphism associated with $\pi$.   Take any $a$, $b\in\frak A$ and
any non-principal ultrafilter $\Cal F$ on $\Bbb N$.   Then
$\sequencen{T^n(\chi a)}$ is a bounded sequence in the reflexive space
$L^2_{\bar\mu}=L^2(\frak A,\bar\mu)$, so
$v=\lim_{n\to\Cal F}T^n(\chi a)$ is defined for the weak topology of
$L^2_{\bar\mu}$.   Now for each
$n\in\Bbb N$ set $\frak B_n=\pi^n[\frak A]$.   This is a closed
subalgebra of $\frak A$ (314F(a-i)), and contains $\pi^ia$ for every
$i\ge n$.   So if we identify $L^2(\frak B_n,\bar\mu\restrp\frak B_n)$
with the corresponding subspace of $L^2_{\bar\mu}$ (366I), it
contains $T^i(\chi a)$ for every $i\ge n$;  but also it is norm-closed,
therefore weakly closed (3A5Ee), so contains $v$.   This means that
$\Bvalue{v>\alpha}$ must belong to $\frak B_n$ for every $\alpha$ and
every $n$.   But in this case
$\Bvalue{v>\alpha}\in\bigcap_{n\in\Bbb N}\frak B_n=\{0,1\}$ for every
$\alpha$, and $v$ is of the form $\gamma\chi 1$.   Also

\Centerline{$\gamma=\int v=\lim_{n\to\Cal F}\int T^n(\chi a)
=\bar\mu a$.}

\noindent So

\Centerline{$\lim_{n\to\Cal F}\bar\mu(\pi^n a\Bcap b)
=\lim_{n\to\Cal F}\int T^n(\chi a)\times\chi b
=\int v\times\chi b
=\gamma\bar\mu b
=\bar\mu a\cdot\bar\mu b$.}

\noindent But this is true of every non-principal ultrafilter $\Cal F$
on $\Bbb N$, so we must have
$\lim_{n\to\infty}\bar\mu(\pi^na\Bcap b)=\bar\mu a\cdot\bar\mu b$
(3A3Lc).   As $a$ and $b$ are arbitrary, $\pi$ is mixing.

\medskip

{\bf (b)} Let $(\frak A,\bar\mu)$ be the measure algebra of
$(X,\Sigma,\mu)$, and $\pi:\frak A\to\frak A$ the measure-preserving
homomorphism corresponding to $\phi$.
The point is that if $a\in\bigcap_{n\in\Bbb N}\pi^n[\frak A]$,
there is an $E\in\Tau$ such that $E^{\ssbullet}=a$.   \Prf\ For each
$n\in\Bbb N$ there is an $a_n\in\frak A$ such that $\pi^na_n=a$;  say
$a_n=F_n^{\ssbullet}$ where $F_n\in\Sigma$.   Then
$\phi^{-n}[F_n]^{\ssbullet}=a$.   Set

\Centerline{$E=\bigcup_{m\in\Bbb N}\bigcap_{n\ge m}\phi^{-n}[F_n]$,
\quad $E_k=\bigcup_{m\ge k}\bigcap_{n\ge m}\phi^{-(n-k)}[F_n]$}

\noindent for each $k$;  then $E^{\ssbullet}=a$ and

\Centerline{$\phi^{-k}[E_k]
=\bigcup_{m\ge k}\bigcap_{n\ge m}\phi^{-n}[F_n]
=\bigcup_{m\in\Bbb N}\bigcap_{n\ge m}\phi^{-n}[F_n]=E$}

\noindent for every $k$, so $E\in\Tau$.\  \Qed

So $\bigcap_{n\in\Bbb N}\frak A_n=\{0,1\}$ and $\pi$ and $\phi$ are
mixing.
}%end of proof of 372S

\exercises{\leader{372X}{Basic exercises (a)}
%\spheader 372Xa
Let $U$ be any reflexive Banach space, and $T:U\to U$ an operator of
norm at most $1$.   Set $A_n=\bover1{n+1}\sum_{i=0}^nT^i$ for each
$n\in\Bbb N$.   Show that $Pu=\lim_{n\to\infty}A_nu$ is defined (as a
limit for the norm topology) for every $u\in U$, and that $P:U\to U$ is
a projection onto $\{u:Tu=u\}$.  \Hint{show that $\{u:Pu$ is defined$\}$
is a closed linear subspace of $U$ containing $Tu-u$ for every $u\in
U$.}

(This is a version of the {\bf mean ergodic theorem}.)
%372A

\sqheader 372Xb Let $(\frak A,\bar\mu)$ be a measure algebra, and
$T\in\Cal T^{(0)}_{\bar\mu,\bar\mu}$;  set
$A_n=\bover1{n+1}\sum_{i=0}^nT^i$ for $n\in\Bbb N$.   Take any
$p\in\coint{1,\infty}$ and $u\in L^p=L^p(\frak A,\bar\mu)$.   Show that
$\sequencen{A_nu}$ is order*-convergent and
$\|\,\|_p$-convergent to some
$v\in L^p$.   \Hint{put 372Xa together with 372D.}
%372D

\spheader 372Xc Let $(\frak A,\bar\mu)$ be a probability algebra, and
$\pi:\frak A\to\frak A$ a measure-preserving Boolean homomorphism.   Let
$P:L^1\to L^1$ be the operator defined as in 365P/366Hb, where
$L^1=L^1_{\bar\mu}$, so that $\int_aPu=\int_{\pi a}u$ for
$u\in L^1$ and $a\in\frak A$.   Set
$A_n=\bover1{n+1}\sum_{i=0}^nP^i:L^1\to L^1$
for each $i$.   Show that for any $u\in L^1$, $\sequencen{A_nu}$
is order*-convergent and $\|\,\,\|_1$-convergent to the conditional
expectation of $u$ on the subalgebra $\{a:\pi a=a\}$.
%372G

\spheader 372Xd Show that if $f$ is any Lebesgue integrable function on
$\Bbb R$, and $y\in\Bbb R\setminus\{0\}$, then

\Centerline{$\lim_{n\to\infty}\bover1{n+1}\sum_{k=0}^nf(x+ky)=0$}

\noindent for almost every $x\in\Bbb R$.
%372H

\spheader 372Xe Let $(X,\Sigma,\mu)$ be a measure space and
$\phi:X\to X$ an \imp\ function.   Set
$\Tau=\{E:E\in\Sigma,\,\mu(\phi^{-1}[E]\symmdiff E)=0\}$,
$\Tau_0=\{E:E\in\Sigma,\,\phi^{-1}[E]=E\}$.   (i) Show that
$\Tau=\{E\symmdiff F:E\in\Tau_0,\,F\in\Sigma,\,\mu F=0\}$.   (ii) Show
that a set $A\subseteq X$ is $\mu\restrp\Tau_0$-negligible iff
$\phi^n[A]$ is $\mu$-negligible for every $n\in\Bbb N$.
%372J

\sqheader 372Xf Let $\nu$ be a Radon probability measure on $\Bbb R$
such that $\int|t|\nu(dt)$ is finite (cf.\ 271F).   On
$X=\Bbb R^{\Bbb N}$ let $\lambda$ be the product measure obtained when
each factor is given the measure $\nu$.  Define $\phi:X\to X$ by setting
$\phi(x)(n)=x(n+1)$ for $x\in X$, $n\in\Bbb N$.   (i) Show that $\phi$
is \imp.   \Hint{254G.   See also 372Xw below.}    (iii) Set
$\gamma=\int t\nu(dt)$, the
expectation of the distribution $\nu$.   By considering
$\bover1{n+1}\sum_{i=0}^nf\frsmallcirc\phi^i$, where $f(x)=x(0)$ for
$x\in X$, show that
$\lim_{n\to\infty}\bover1{n+1}\sum_{i=0}^nx(i)=\gamma$ for
$\lambda$-almost every $x\in X$.
%372J

\sqheader 372Xg Use the Ergodic Theorem to prove Kolmogorov's Strong Law
of Large Numbers (273I), as follows.   Given a complete probability
space $(\Omega,\Sigma,\mu)$ and an independent identically distributed
sequence $\sequencen{f_n}$ of measurable functions from $\Omega$ to
$\Bbb R$, set $X=\BbbR^{\Bbb N}$ and
$f(\omega)=\sequencen{f_n(\omega)}$ for $\omega\in\Omega$.   Show that
if we give each copy of $\Bbb R$ the distribution of $f_0$ then $f$ is
\imp\ for $\mu$ and the product measure $\lambda$ on $X$.   Now use
372Xf.
%372Xf, 372J

\sqheader 372Xh Let
$\sequencen{f_n}$ be a sequence of real-valued random variables with finite
expectation such that
$(f_0,f_1,\ldots,f_n)$ has the same joint distribution as
$(f_1,f_2,\ldots,f_{n+1})$ for every $n\in\Bbb N$.   Show that
$\sequencen{\Bover1{n+1}\sum_{i=0}^nf_i}$ converges a.e.   \Hint{Let
$(X,\Sigma,\mu)$ be the underlying probability space.   Reduce to the
case in which every $f_i$ is measurable and defined everywhere in $X$.
Define $\theta:X\to\BbbR^{\Bbb N}$ by setting $\theta(x)(n)=f_n(x)$ for
$x\in X$,
$n\in\Bbb N$.   Let $\lambda$ be the image measure $\mu\theta^{-1}$.
Set $\phi(z)(n)=z(n+1)$ for $z\in\BbbR^{\Bbb N}$ and $n\in\Bbb N$.
Show that $\phi$ is \imp\ for $\lambda$, and apply 372J.}
%372J 372Xg out of order
% strong law for stationary processes

\spheader 372Xi Show that the continued fraction coefficients of
$\Bover1{\sqrt 2}$ are $1$, $2$, $2$, $2,\ldots$.
%372L

\sqheader 372Xj For $x\in X=[0,1]\setminus\Bbb Q$ let
$k_1(x),k_2(x),\ldots$ be its continued-fraction coefficients.
Show that $x\mapsto\sequencen{k_{n+1}(x)-1}$ is a bijection between $X$
and $\BbbN^{\Bbb N}$ which is a homeomorphism if $X$ is given its usual
topology (as a subset of $\Bbb R$) and $\BbbN^{\Bbb N}$ is given its
usual product topology (each copy of $\Bbb N$ being given its discrete
topology).
%372L

\spheader 372Xk Set $x=\bover12(\sqrt5-1)$.
Show that, in the notation of 372L, $k_n(x)=1$ and
$q_n(x)=p_{n-1}(x)$ for every $n\ge 1$ and that $\sequencen{p_n(x)}$ is the
Fibonacci sequence.
%372L

\spheader 372Xl For any irrational $x\in[0,1]$ let
$k_1(x),k_2(x),\ldots$ be its continued-fraction coefficients and
$p_n(x)$, $q_n(x)$ the numerators and denominators of its
continued-fraction approximations, as described in 372L.   Write
$r_n(x)=p_n(x)/q_n(x)$.   (i) Show that $x$ lies between $r_n(x)$ and
$r_{n+1}(x)$ for every $n\in\Bbb N$.   (ii) Show that
$r_{n+1}(x)-r_n(x)=(-1)^n/q_n(x)q_{n+1}(x)$ for every $n\in\Bbb N$.
(iii) Show that $|x-r_n(x)|\le 1/q_n(x)^2k_{n+1}(x)$ for every $n\ge 1$.
(iv) Hence show that for almost every $\gamma\in\Bbb R$, the set
$\{(p,q):p\in\Bbb Z,\,q\ge 1,\,|\gamma-\bover{p}{q}|\le\epsilon/q^2\}$
is infinite for every $\epsilon>0$.  (v) Show that if $n\ge 1$,
$p$, $q\in\Bbb N$ and $0<q\le q_n(x)$, then
$|x-\bover{p}{q}|\ge|x-r_n(x)|$, with equality only when $p=p_n(x)$ and
$q=q_n(x)$.
%372L, 372M

\spheader 372Xm In 372M, let $\Tau_1$ be the family $\{E:$ for every
$n\in\Bbb N$ there is a measurable set $F\subseteq X$ such that
$\phi^{-n}[F]=E\}$.   Show that every member of $\Tau_1$ is either
negligible or conegligible.   \Hint{the argument of part (c) of the
proof of 372M still works.}   Hence show that $\phi$ is mixing for the
measure $\nu$.
%372M, 372S

\spheader 372Xn Let $(\frak A,\bar\mu)$ be an atomless probability
algebra.   Show that the following are equiveridical:  (i) $\frak A$ is
homogeneous; (ii) there is an ergodic measure-preserving Boolean
homomorphism $\pi:\frak A\to\frak A$;  (iii) there is a mixing
measure-preserving automorphism $\pi:\frak A\to\frak A$.   \Hint{333P.}
%372O

\spheader 372Xo\dvArevised{2011}
Let $(\frak A,\bar\mu)$ be a probability algebra, and
$\pi:\frak A\to\frak A$ a measure-preserving Boolean homomorphism.   (i)
Show that if $n\ge 1$ then $\pi$ is mixing iff $\pi^n$ is mixing.   (ii)
Show that if $n\ge 1$ then $\pi$ is weakly mixing iff $\pi^n$ is weakly
mixing.   (iii)
Show that if $n\ge 1$ and $\pi^n$ is ergodic then $\pi$ is ergodic.
(iv) Show that if $\pi$ is an automorphism then it is ergodic, or
mixing, or weakly mixing, iff $\pi^{-1}$ is.
%372O

\sqheader 372Xp Consider the {\bf tent map}
$\phi_{\alpha}(x)=\alpha\min(x,1-x)$ for $x\in[0,1]$,
$\alpha\in[0,2]$.   Show that $\phi_2$ is \imp\ and mixing for Lebesgue
measure on $[0,1]$.   \Hint{show that
$\phi_2^{n+1}(x)=\phi_2(\fraction{2^nx})$ for
$n\ge 1$, and hence that $\mu(I\cap\phi_2^{-n}[J])=\mu I\cdot\mu J$
whenever $I$ is of the form
$[2^{-n}k,2^{-n}(k+1)]$ and $J$ is an interval.}
%372O

\spheader 372Xq Consider the {\bf logistic map}
$\psi_{\beta}(x)=\beta x(1-x)$ for $x\in [0,1]$, $\beta\in[0,4]$.   Show
that $\psi_4$ is \imp\ and mixing for the Radon measure on $[0,1]$ with
density function
$t\mapsto\Bover1{\pi\sqrt{t(1-t)}}$.   \Hint{show that the transformation
$t\mapsto\sin^2\bover{\pi t}2$ matches it with the tent map.}   Show that
for almost every $x$,

\Centerline{$\lim_{n\to\infty}\Bover1{n+1}\#(\{i:i\le
n,\,\psi_4^i(x)\le\alpha\})=\Bover2{\pi}\arcsin\sqrt{\alpha}$}

\noindent for every $\alpha\in[0,1]$.
%372O

\spheader 372Xr Let $\mu$ be Lebesgue measure on
$\coint{0,1}$, and fix an irrational number $\alpha\in\coint{0,1}$.
(i) Set $\phi(x)=x+_1\alpha$ for every
$x\in\coint{0,1}$, where $x+_1\alpha$ is whichever of $x+\alpha$,
$x+\alpha-1$ belongs to $\coint{0,1}$.    Show that $\phi$ is \imp.
(ii) Show that if $I\subseteq\coint{0,1}$ is an interval then
$\lim_{n\to\infty}\bover1{n+1}\sum_{i=0}^n\chi I(\phi^i(x))=\mu I$ for
almost every $x\in\coint{0,1}$.   \Hint{this is Weyl's Equidistribution
Theorem (281N).}   (iii) Show that $\phi$ is ergodic.   \Hint{take the
conditional expectation operator $P$ of 372G, and look at
$P(\chi I^{\ssbullet})$ for intervals $I$.}   (iv) Show that $\phi^n$ is
ergodic for any $n\in\Bbb Z\setminus\{0\}$.   (v) Show that $\phi$ is
not weakly mixing.
%372G, 372Q

\spheader 372Xs Let $p$, $q\in [1,\infty]$ be such that
$\bover1p+\bover1q=1$.
(i) Let $(\frak A,\bar\mu)$ be a probability algebra,
$\pi:\frak A\to\frak A$ a mixing measure-preserving homomorphism, and
$T:L^0(\frak A)\to L^0(\frak A)$ the corresponding homomorphism.
Show that $\lim_{n\to\infty}\int T^nu\times v=\int u\int v$ whenever
$u\in L^p(\frak A,\bar\mu)$ and $v\in L^q(\frak A,\bar\mu)$.
\Hint{start with $u$, $v\in S(\frak A)$.}
(ii) Let $(X,\Sigma,\mu)$ be a probability space and
$\phi:X\to X$ a mixing \imp\ function.   Show that
$\lim_{n\to\infty}\int f(\phi^n(x))g(x)dx=\int f\int g$ whenever
$f\in\eusm L^p(\mu)$ and $g\in\eusm L^q(\mu)$.
%372Q

\spheader 372Xt Give $\coint{0,1}$ Lebesgue measure $\mu$, and let
$k\ge 2$ be an integer.   Define $\phi:\coint{0,1}\to\coint{0,1}$ by
setting
$\phi(x)=\fraction{kx}$, the fractional part of $kx$.   Show that $\phi$
is \imp.    Show that $\phi$ is mixing.   \Hint{if
$I=\coint{k^{-n}i,k^{-n}(i+1)}$, $J=\coint{k^{-n}j,k^{-n}(j+1)}$ then
$\mu(I\cap\phi^{-m}[J])=\mu I\cdot\mu J$ for all $m\ge n$.}
%372Q

\spheader 372Xu Let $(X,\Sigma,\mu)$ be a probability space and
$\phi:X\to X$ an ergodic \imp\ function.   Let $f$ be a
$\mu$-virtually measurable function defined almost everywhere in $X$
such that $\int fd\mu=\infty$.   Show that
$\lim_{n\to\infty}\Bover1{n+1}\sum_{i=0}^nf\phi^i$ is infinite a.e.
\Hint{look at the corresponding limits for $f_m=f\wedge m\chi X$.}
%372Q

\spheader 372Xv For irrational $x\in[0,1]$, write $k_1(x),k_2(x),\ldots$
for the continued-fraction coefficients of $x$.   Show that the limit
$\lim_{n\to\infty}\bover1n\sum_{i=1}^nk_i(x)$ is infinite
for almost every
$x$.   \Hint{take $\phi$, $\nu$ as in 372M, and show that
$\int k_1d\nu=\infty$.}
%372Q, 372Xu

\spheader 372Xw Let $(X,\Sigma,\mu)$ be any probability space, and let
$\lambda$ be the product measure on $X^{\Bbb N}$.   Define
$\phi:X^{\Bbb N}\to X^{\Bbb N}$ by setting $\phi(x)(n)=x(n+1)$.  Show
that $\phi$ is
\imp.   Show that $\phi$ satisfies the conditions of 372S, so is mixing.
%372S

\spheader 372Xx Let $(X,\Sigma,\mu)$ be any probability space, and
$\lambda$ the product measure on $X^{\Bbb Z}$.   Define
$\phi:X^{\Bbb Z}\to X^{\Bbb Z}$ by setting $\phi(x)(n)=x(n+1)$.  Show
that $\phi$ is
\imp.   Show that $\phi$ is mixing.   \Hint{show that if $C$, $C'$ are
basic cylinder sets then $\mu(C\cap\phi^{-n}[C'])=\mu C\cdot\mu C'$ for
all $n$ large enough.}   Show that $\phi$ does not
ordinarily satisfy the conditions of 372S.   (Compare 333P.)
%372S

\spheader 372Xy\dvAnew{2011}(i) Let $\frak A$ be a Boolean algebra,
$\pi:\frak A\to\frak A$ a Boolean homomorphism, and
$\phi:\frak A\to\frak A$ an automorphism.   Show that if
$\pi$ is ergodic then $\phi\pi\phi^{-1}$ is ergodic.
(ii) Let $(\frak A,\bar\mu)$ be a probability algebra,
$\pi:\frak A\to\frak A$ a measure-preserving Boolean homomorphism, and
$\phi:\frak A\to\frak A$ a measure-preserving
Boolean automorphism.   Show that if
$\pi$ is mixing, or weakly mixing, then so is $\phi\pi\phi^{-1}$.
%372O out of order query

\leader{372Y}{Further exercises (a)}
%\spheader 372Ya
In 372D, show that the null space of the limit operator $P$ is precisely
the closure in $M^{1,0}$ of the subspace $\{Tu-u:u\in M^{1,0}\}$.

\spheader 372Yb Let $(\frak A,\bar\mu)$ be a measure algebra,
$T\in\Cal T^{(0)}_{\bar\mu,\bar\mu}$, $p\in\ooint{1,\infty}$ and
$u\in L^p(\frak A,\bar\mu)$.  Set
$u^*=\sup_{n\in\Bbb N}\bover1{n+1}\sum_{i=0}^n|T^iu|$.   (i) Show that
for any $\gamma>0$,

\Centerline{$\bar\mu\Bvalue{u^*>\gamma}
\le\Bover2{\gamma}\int_{\Bvalue{|u|>\gamma/2}}|u|$.}

\noindent\Hint{apply 372C to $(|u|-\bover12\gamma\chi 1)^+$.}   (ii)
Show that $\|u^*\|_p\le 2(\Bover{p}{p-1})^{1/p}\|u\|_p$.   ({\it
Hint\/}: show that
$\int_{\Bvalue{|u|>\alpha}}|u|=\alpha\bar\mu\Bvalue{|u|>\alpha}
+\int_{\alpha}^{\infty}\bar\mu\Bvalue{|u|>\beta}d\beta$;  see 365A.
Use 366Xa to show that

\Centerline{$\|u^*\|_p^p\le 2p\int_0^{\infty}\gamma^{p-2}
\int_{\gamma/2}^{\infty}\bar\mu\Bvalue{|u|>\beta}d\beta d\gamma
+2^p\|u\|_p^p$,}

\noindent and reverse the order of integration.   Compare
275Yd.)   (This is {\bf Wiener's Dominated Ergodic Theorem}.)
% 372D

\spheader 372Yc Let $(\frak A,\bar\mu)$ be a probability algebra and $T$
an operator in $\Cal T^{(0)}_{\bar\mu,\bar\mu}$.   Take
$u\in L^1=L^1(\frak A,\bar\mu)$ such that $h(|u|)\in L^1$, where
$h(t)=t\ln t$ for
$t\ge 1$, $0$ for $t\le 1$, and $\bar h$ is the corresponding function
from $L^0(\frak A)$ to itself.   Set
$u^*=\sup_{n\in\Bbb N}\bover1{n+1}\sum_{i=0}^n|T^iu|$.   Show that
$u^*\in L^1$.   \Hint{use
the method of 372Yb to show that
$\int_2^{\infty}\bar\mu\Bvalue{u^*>\gamma}d\gamma\le 2\int\bar h(u)$.}
%372D, 372Yb

\spheader 372Yd Let $U$ be a Banach space, $(\frak A,\bar\mu)$ a
semi-finite measure algebra and $\sequencen{T_n}$ a sequence of
continuous linear operators from $U$ to $L^0=L^0(\frak A)$ with its
topology of convergence in measure.   Suppose that
$\sup_{n\in\Bbb N}T_nu$ is defined in $L^0$ for every $u\in U$.
Show that $\{u:u\in U$, $\sequencen{T_nu}$ is order*-convergent in $L^0\}$
is a norm-closed linear subspace of $U$.
%372D

\spheader 372Ye In 372G, suppose that $\frak A$ is atomless.   Show that
there is always an $a\in\frak A$ such that $\bar\mu a\le\bover12$ and
$\inf_{i\le n}\pi^ia\ne 0$ for every $n$, so that (except in trivial
cases) $\sequencen{A_n(\chi a)}$ will not be
$\|\,\|_{\infty}$-convergent.
%372G

\spheader 372Yf Let $(X,\Sigma,\mu)$ be a measure space with
measure algebra $(\frak A,\bar\mu)$.   Let $\Phi$ be a family of \imp\
functions from $X$ to itself, and for $\phi\in\Phi$ let
$\pi_{\phi}:\frak A\to\frak A$ be the associated
homomorphism.   Set
$\frak C=\{c:c\in\frak A,\,\pi_{\phi}c=c$ for every $\phi\in\Phi\}$,
$\Tau=\{E:E\in\Sigma,\,\phi^{-1}[E]\symmdiff E$ is negligible for every
$\phi\in\Phi\}$ and
$\Tau_0=\{E:E\in\Sigma,\,\phi^{-1}[E]=E$ for every $\phi\in\Phi\}$.
Show that (i) $\Tau$ and $\Tau_0$
are $\sigma$-subalgebras of $\Sigma$ (ii) $\Tau_0\subseteq\Tau$ (iii)
$\Tau=\{E:E\in\Sigma,\,E^{\ssbullet}\in\frak C\}$ (iv) if $\Phi$ is
countable and $\phi\psi=\psi\phi$ for all $\phi$, $\psi\in\Phi$, then
$\frak C=\{E^{\ssbullet}:E\in\Tau_0\}$.
%372I

\spheader 372Yg  Show that an irrational $x\in\ooint{0,1}$ has an
eventually periodic sequence of continued fraction coefficients iff it
is a solution of a quadratic equation with integral coefficients.
%372L  mt37bits

\spheader 372Yh In the language of 372L-372N and 372Xl, show the
following.   (i) For any $x\in X$, $n\ge 2$,
$q_n(x)q_{n-1}(x)\ge 2^{n-1}$, $p_{n}(x)p_{n+1}(x)\ge 2^{n-1}$, so that
$q_{n+1}(x)p_{n}(x)\ge 2^{n-1}$ and $|1-x/r_n(x)|\le 2^{-n+1}$,
$|\ln x-\ln r_n(x)|\le 2^{-n+2}$.   Also $|x-r_n(x)|\ge
1/q_n(x)q_{n+2}(x)$.   (ii) For any $x\in X$, $n\ge 1$,
$p_{n+1}(x)=q_n(\phi(x))$ and
$q_n(x)\prod_{i=0}^{n-1}r_{n-i}(\phi^{i}(x))=1$.   (iii) For any
$x\in X$, $n\ge 1$, $|\ln q_n(x)+\sum_{i=0}^{n-1}\ln\phi^i(x)|\le 4$.
(iv) For almost every $x\in X$,

\Centerline{$\lim_{n\to\infty}\Bover1{n}\ln q_n(x)
=-\Bover1{\ln 2}\int_0^1\Bover{\ln t}{1+t}dt
=\Bover{\pi^2}{12\ln 2}$.}

\noindent \Hint{225Xi, 282Xo.}  (v) For almost every $x\in X$,
$\lim_{n\to\infty}\bover1n\ln|x-r_n(x)|=-\Bover{\pi^2}{6\ln 2}$.
(vi) For almost every $x\in X$, $11^{-n}\le|x-r_n(x)|\le 10^{-n}$ and
$3^n\le q_n(x)\le 4^n$ for all but finitely many $n$.
%372N, 372Xl

\spheader 372Yi\dvArevised{2011}(i)
Let $(X,\Sigma,\mu)$ and $(Y,\Tau,\nu)$ be
probability spaces, with c.l.d.\ product $(X\times Y,\Lambda,\lambda)$.
Suppose that
$\phi:X\to X$ is a weakly mixing \imp\ function and
$\psi:Y\to Y$ is an ergodic
\imp\ function.   Define $\theta:X\times Y\to X\times Y$ by setting
$\theta(x,y)=(\phi(x),\psi(y))$ for all $x$, $y$.   Show that $\theta$
is an ergodic \imp\ function.
(ii) Let $(\frak A,\bar\mu)$ and $(\frak B,\bar\nu)$ be probability
algebras, with probability algebra free product $(\frak C,\bar\lambda)$.
Suppose that $\phi:\frak A\to\frak A$ is a
weakly mixing measure-preserving
Boolean homomorphism and $\psi:\frak B\to\frak B$ is an ergodic
measure-preserving Boolean
homomorphism.   Let $\theta:\frak C\to\frak C$ be the measure-preserving
Boolean homomorphism such that $\theta(a\otimes b)=\phi a\otimes\psi b$
for all $a\in\frak A$ and $b\in\frak B$ (325Xe).   Show that $\theta$ is
ergodic.
%372O

\spheader 372Yj Let $\langle(X_i,\Sigma_i,\mu_i)\rangle_{i\in I}$ be any
family of probability spaces, with product $(X,\Lambda,\lambda)$.
Suppose that for each $i\in I$ we are given an \imp\ function
$\phi_i:X_i\to X_i$.   (i) Show that there is a corresponding \imp\
function
$\phi:X\to X$ given by setting $\phi(x)(i)=\phi_i(x(i))$ for $x\in X$,
$i\in I$.   (ii) Show that if every $\phi_i$ is mixing so is $\phi$.
(iii)\dvAnew{2011}
Show that if every $\phi_i$ is weakly mixing so is $\phi$.
%372O 372Yi

\spheader 372Yk Give an example of an ergodic measure-preserving
automorphism $\phi:\coint{0,1}\to\coint{0,1}$ such that $\phi^2$ is not
ergodic.   \Hint{set $\phi(x)=\bover12(1+\phi_0(2x))$ for $x<\bover12$,
$x-\bover12$ for $x\ge\bover12$.   See also 388Xe.}
%von Neumann transf
%can we have ergodic $\phi$ such that $\phi^n$ not ergodic for any
% n\ge 2 ?  I suppose so
%372O, 372Xo

\spheader 372Yl Show that there is an ergodic $\phi:[0,1]\to[0,1]$ such
that $(\xi_1,\xi_2)\mapsto(\phi(\xi_1),\phi(\xi_2)):[0,1]^2\to[0,1]^2$
is not ergodic.   \Hint{372Xr.}
%372Yj, 372Xr, 372O, 372Q

\spheader 372Ym Let $M$ be an $r\times r$ matrix with integer
coefficients and non-zero determinant, where $r\ge 1$.   Let
$\phi:\coint{0,1}^r\to\coint{0,1}^r$ be the function such that
$\phi(x)-Mx\in\Bbb Z^r$ for every $x\in\coint{0,1}^r$.   Show that
$\phi$ is \imp\ for Lebesgue measure on $\coint{0,1}^r$.
%372Q  372Xt %mt37bits

\spheader 372Yn(i)\dvArevised{2011}
Let $(\frak A,\bar\mu)$ be a probability algebra,
$\pi:\frak A\to\frak A$ a weakly mixing measure-preserving Boolean
homomorphism, and $T=T_{\pi}:L^1_{\bar\mu}\to L^1_{\bar\mu}$ the
corresponding linear operator (365O).   Show that if
$u\in L^1_{\bar\mu}$ is such that $\{T^nu:n\in\Bbb N\}$ is relatively
compact for the norm topology, then $u=\alpha\chi 1$ for some $\alpha$.
(ii) Let $\mu$ be Lebesgue measure on $\coint{0,1}$, $(\frak A,\bar\mu)$
its measure algebra, $\alpha\in\coint{0,1}$ an irrational number,
$\phi(x)=x+_1\alpha$ for $x\in\coint{0,1}$ (as in 372Xr), and
$T:L^1(\mu)\to L^1(\mu)$ the linear operator defined by setting
$Tg^{\ssbullet}=(g\phi)^{\ssbullet}$ for $g\in\eusm L^1(\mu)$.   Show
that $\{T^n:n\in\Bbb Z\}$ is relatively compact for the strong operator
topology on $\eurm B(L^1(\mu);L^1(\mu))$.
%372Xr 372Q

\spheader 372Yo In 372M, show that for any measurable set
$E\subseteq X$, $\lim_{n\to\infty}\mu_L\phi^{-n}[E]=\nu E$.
\Hint{recall that
$\phi$ is mixing for $\nu$ (372Xm).   Hence show that
$\lim_{n\to\infty}\int_{\phi^{-n}[E]}g\,d\nu=\nu E\cdot\int g\,d\nu$ for
any integrable $g$.   Apply this to a Radon-Nikod\'ym derivative of
$\mu_L$ with respect to $\nu$.}   (I understand that this result is due
to Gauss.)
%372L, 372S, 372Xm

\spheader 372Yp\dvAnew{2010} (i) Show that there are a Boolean algebra
$\frak A$ and an automorphism $\pi:\frak A\to\frak A$ which is not
ergodic, but has fixed-point algebra $\{0,1\}$.
%finite-cofinite algebra of  \Bbb Z\times\{0,1\}
(ii) Show that there are a $\sigma$-finite measure algebra
$(\frak A,\bar\mu)$ and a measure-preserving Boolean homomorphism
$\pi:\frak A\to\frak A$ which is not ergodic, but has fixed-point algebra
$\{0,1\}$.
%372P mt37bits out of order query

\spheader 372Yq\dvAnew{2010}
For a Boolean algebra $\frak A$ and a Boolean homomorphism
$\pi:\frak A\to\frak A$, write $T_{\pi}$ for the corresponding operator
from $L^{\infty}(\frak A)$ to itself, as defined in 363F.
(i) Suppose that $\frak A$ is a Boolean algebra, $\pi:\frak A\to\frak A$ is
a Boolean homomorphism, $u\in L^{\infty}(\frak A)$ and $T_{\pi}u=u$.
Show that if either $\pi$ is ergodic or $\frak A$ is
Dedekind $\sigma$-complete and the fixed-point subalgebra of $\pi$ is
$\{0,1\}$, then $u$ must be a multiple of $\chi 1$.
(ii)\dvAformerly{3{}63Yk}
Find a Boolean algebra $\frak A$, an automorphism
$\pi:\frak A\to\frak A$ with fixed-point algebra $\{0,1\}$,
and a $u\in L^{\infty}(\frak A)$, not a
multiple of $\chi 1$, such that $T_{\pi}u=u$.
%372P mt37bits out of order query

\spheader 372Yr\dvAnew{2011} Set $\Cal F_d
=\{I:I\subseteq\Bbb N$, $\lim_{n\to\infty}\Bover1n\#(I\cap n)=1\}$.
(i) Show that $\Cal F_d$ is a filter on $\Bbb N$.   (ii) Show that for a
bounded sequence $\sequencen{\alpha_n}$ in $\Bbb R$, the following are
equiveridical:  ($\alpha$) $\lim_{n\to\Cal F_d}\alpha_n=0$;   ($\beta$)
$\lim_{n\to\infty}\Bover1{n+1}\sum_{k=0}^n|\alpha_k|=0$;
($\gamma$)
$\lim_{n\to\infty}\Bover1{n+1}\sum_{k=0}^n\alpha^2_k=0$.
($\Cal F_d$ is called the {\bf (asymptotic) density filter}.)
%372O out of order query

\spheader 372Ys\dvAnew{2011} Let $(\frak A,\bar\mu)$ be a probability
algebra, and $\phi:\frak A\to\frak A$ a measure-preserving Boolean
homomorphism.   (i) Show that there are a probability algebra
$(\frak C,\bar\lambda)$, a measure-preserving Boolean homomorphism
$\pi:\frak A\to\frak C$, and a measure-preserving automorphism
$\tilde\phi:\frak C\to\frak C$ such that $\tilde\phi\pi=\pi\phi$ and
$\frak C$ is the closure of
$\bigcup_{n\in\Bbb N}\tilde\phi^{-n}[\pi[\frak A]]$ for the
measure-algebra topology.   \Hint{328J.}   (ii) Show that
$\tilde\phi$ is ergodic, or mixing, or weakly mixing iff $\phi$ is.
%372Q

\leaveitout{\spheader 372Y* Let $\Cal F$ be a filter on $\Bbb N$.   If
$(\frak A,\bar\mu)$ is a probability algebra and $\pi:\frak A\to\frak A$
is a measure-preserving Boolean homomorphism, say that $\pi$ is
{\bf $\Cal F$-mixing} if
$\lim_{n\to\Cal F}\bar\mu(\pi^na\Bcap b)=\bar\mu a\cdot\bar\mu b$
for all $a$, $b\in\frak A$;  if
$(X,\Sigma,\mu)$
is a probability space and $\phi:X\to X$ is an \imp\ function, say that
$\phi$ is
{\bf $\Cal F$-mixing} if $\lim_{n\to\Cal F}\mu(E\cap\phi^{-n}[F])
=\mu E\cdot\mu F$ for all $a$, $b\in\frak A$.    (i) Show that in this
case
no finite set can belong to $\Cal F$.   (ii) Show that $\pi$ is mixing
iff it is $\Cal F$-mixing for every non-principal ultrafilter $\Cal F$
on $\Bbb N$.   (iii) Show that if $\pi$ is $\Cal F$-mixing for any
$\Cal F$, it is ergodic.   (iv) In 372Yj, show that if every $\phi_i$ is
$\Cal F$-mixing, so is $\phi$.   (v) In 372Xr, show that $\phi$ is not
$\Cal F$-mixing for any $\Cal F$.   (vi) Show that if
$\Cal F=\{I:\lim_{n\to\infty}\Bover1n\#(I\cap n)=1\}$, then
$\pi$ is $\Cal F$-mixing iff it is weakly mixing.
%?example of $\Cal F$-mixing, not mixing?
%372Q
}%end of leaveitout
}%end of exercises

\cmmnt{\Notesheader{372} I have chosen an entirely conventional route
to the Ergodic Theorem here, through the Mean Ergodic Theorem (372Xa)
or, rather, the fundamental lemma underlying it (372A), and the Maximal
Ergodic Theorem (372B-372C).   What is not to be found in every
presentation is the generality here.   I speak of arbitrary $T\in\Cal
T^{(0)}$, the operators which are contractions both for $\|\,\|_1$ and
for $\|\,\|_{\infty}$, not requiring $T$ to be positive, let alone
correspond to a measure-preserving homomorphism.   (I do not mention
$\Cal T^{(0)}$ in the statement of 372C, but of course it is present in
spirit.)   The work we
have done up to this point puts this extra generality within easy reach,
but as the rest of the section shows, it is not needed for the principal
examples.   Only in 372Xc do I offer an application not associated in the
usual way with a measure-preserving homomorphism or an \imp\ function.

The Ergodic Theorem is an `almost-everywhere pointwise convergence
theorem', like the strong law(s) of large numbers and the martingale
theorem(s) (\S273, \S275).   Indeed Kolmogorov's form of the strong law
can be derived from the Ergodic Theorem (372Xg).   There are some very
strong family resemblances.   For instance, the Maximal Ergodic Theorem
corresponds to the most basic of all the martingale inequalities (275D).
Consequently we have similar results, obtained by similar methods,
concerning the domination of sequences starting from members of $L^p$
(372Yb, 275Yd), though the inequalities are not identical.   (Compare
also 372Yc with 275Ye.)   There are some tantalising reflections of
these traits in results surrounding Carleson's theorem on the pointwise
convergence of square-integrable Fourier series (see \S286 {\it
notes\/}), but Carleson's theorem seems to be much harder than the
others.   Other forms of the strong law (273D, 273H) do not appear to
fit into quite the same pattern, but I note that here, as with the
Ergodic Theorem, we begin with a study of square-integrable functions
(see part (e) of the proof of 372D).

After 372D, there is a contraction and concentration in the scope of the
results, starting with a simple replacement of $M^{1,0}$ with $L^1$
(372F).   Of course it is almost as easy to prove 372D from 372F as the
other way about;  I give precedence to 372D only because $M^{1,0}$ is
the space naturally associated with the class $\Cal T^{(0)}$ of
operators to which these methods apply.   Following this I turn to the
special family of operators to which the rest of the section is devoted,
those associated with measure-preserving homomorphisms (372E), generally
on probability spaces (372G).   This is the point at which we can begin
to identify the limit as a conditional expectation as well as an
invariant element.

Next comes the translation into the language of measure spaces and \imp\
functions, all perfectly straightforward in view of
372I.   These turn 372E into 372H and 372G into the main part of
372J.

In 372J-372K I find myself writing at some length about a technical
problem.   The root of the difficulty is in the definition of
`conditional expectation'.   Now it is generally accepted that any pure
mathematician has `Humpty Dumpty's privilege':  `When {\it I} use
a word, it means just what I choose it to
mean -- neither more nor less'.   With any privilege come duties and
responsibilities;  here, the duty to be self-consistent, and
the responsibility to try to use terms in ways which will not mystify or
mislead the unprepared reader.   Having written down a definition of
`conditional expectation' in Volume 2, I must either stick to it, or go
back and change it, or very carefully explain exactly what modification
I wish to make here.   I don't wish to suggest that absolute consistency
-- in terminology or anything else -- is supreme among mathematical
virtues.   Surely it is better to give local meanings to words, or
tolerate ambiguities, than to suppress ideas which cannot be formulated
effectively otherwise, and among `ideas' I wish to include the
analogies and resonances which a suitable language can suggest.   But I
do say that it is always best to be conscious of what one is doing -- I
go farther:  one of the things which mathematics is for, is to raise our
consciousness of what our thoughts really are.   So I believe it is
right to pause occasionally over such questions.

In 372L-372N (see also 372Xl, 372Xv, 372Xm, 372Xk, 372Yh, 372Yo) I make an
excursion into number theory.   This is a remarkable example of the
power of advanced measure theory to give striking results in other
branches of mathematics.   Everything here is derived from {\smc
Billingsley 65}, who goes farther than I have space for, and gives
references to more.   Here let me point to 372Xj;  almost accidentally,
the construction offers a useful formula for a homeomorphism
between two of the most important spaces of descriptive set theory,
which will be important to us in Volume 4.

I end the section by introducing three terms, `ergodic', `mixing'
and `weakly mixing'
transformations, not because I wish to use them for any new ideas (apart
from the elementary 372P-372S, %372P 372Q 372R 372S
these must wait for \S\S385-387 %385S 385T 386D 387C 387D et seq
below and \S494 in Volume 4), %494D 494E 494F
but because
it may help if I immediately classify some of the \imp\ functions we
have seen (372Xp-372Xr, %372Xp 372Xq 372Xr
372Xt, 372Xw, 372Xx).
Of course in any application of any ergodic theorem it
is of great importance to be able to identify the limits promised by the
theorem, and the point about an ergodic transformation is just that our
averages converge to constant limits (372Q).   Actually proving that a
given \imp\ function is ergodic is rarely quite trivial (see 372M,
372Xq, 372Xr), though a handful of standard techniques cover a large
number of cases, and it is usually obvious when a map is {\it not}
ergodic, so that if an invariant region does not leap to the eye one has
a good hope of ergodicity.   The extra concept of `weakly mixing'
transformation is hardly relevant to anything in this volume (though see
372Yi-372Yj), but
is associated with a remarkable topological fact about automorphism groups
of probability algebras, to come in 494E.

I ought to remark on the odd shift between the definitions of `ergodic
Boolean homomorphism' and `ergodic \imp\ function' in 372O.   The point is
that the version in 372O(b-i) is the standard formulation in this context,
but that its natural translation into the version `a Boolean homomorphism
from a probability algebra to itself is ergodic if its fixed-point
subalgebra is trivial', although perfectly satisfactory in that context,
allows unwelcome phenomena if applied to general Boolean algebras
(372Yp, 372Yq).   The definition in 372O(a-i) is
rather closer to the essential idea
of ergodicity of a dynamical system, which asks that the system should
always evolve along a path which approximates all possible states.
In practice, however,
we shall nearly always be dealing with automorphisms of
Dedekind $\sigma$-complete algebras, for which we can use the
fixed-point criterion of 372Pc.

I take the opportunity to mention two famous functions from $[0,1]$ to
itself, the `tent' and `logistic' maps (372Xp, 372Xq).   In the
formulae $\phi_{\alpha}$, $\psi_{\beta}$ I include redundant parameters;
this is because the real importance of these functions lies in the way
their behaviour depends, in bewildering complexity, on these parameters.
It is only at the extreme values $\alpha=2$, $\beta=4$ that the methods
of this volume can tell us anything interesting.
}%end of comment

\discrpage

