\frfilename{mt417.tex}
\versiondate{9.3.10}
\copyrightdate{2000}

\def\chaptername{Topologies and measures I}
\def\sectionname{$\tau$-additive product measures}

\newsection{417}

The `ordinary' product measures introduced in Chapter 25 have served us
well for a volume and a half.   But we come now to a fundamental
obstacle.   If we start with two Radon measure spaces, their product
measure, as
defined in \S251, need not be a Radon measure (419E).   Furthermore, the
counterexample is one of the basic compact measure spaces of the theory;
and while it is dramatically non-metrizable, there is no other reason to
set it aside.   Consequently, if we wish (as we surely do) to create
Radon measure spaces as
products of Radon measure spaces, we need a new construction.   This is
the object of the present section.   It turns out that the construction
can be adapted to work well beyond the special context of Radon measure
spaces;  the methods here apply to general effectively locally finite
$\tau$-additive topological measures (for the product of finitely many
factors) and to $\tau$-additive topological probability measures (for
the product of infinitely many factors).

The fundamental theorems are 417C and 417E, listing the essential
properties of what I call `$\tau$-additive product measures', which are
extensions of the c.l.d.\ product measures and product probability
measures of Chapter 25.   They depend on a straightforward lemma on the
extension of
a measure to make every element of a given class of sets negligible
(417A).   It is relatively
easy to prove that the extensions are more or less canonical (417D,
417F).   We still have Fubini's theorem for the new product measures
(417H), and the basic operations from \S254 still apply (417J, 417K,
417M).

It is easy to check that if we start with quasi-Radon measures, then the
$\tau$-additive product measure is again quasi-Radon (417N, 417O).   The
$\tau$-additive product of two Radon measures is Radon (417P), and the
$\tau$-additive product of Radon probability measures with compact
supports is Radon (417Q).

In the last part of the section I look at continuous real-valued
functions and Baire $\sigma$-algebras;  it turns out that for these the
ordinary product measures are adequate (417U, 417V).

\leader{417A}{Lemma}\dvArevised{2010}
Let $(X,\Sigma,\mu)$ be a semi-finite measure
space, and $\Cal A\subseteq\Cal PX$ a family of sets such that
$\mu_*(\bigcup_{n\in\Bbb N}A_n)=0$ for every sequence $\sequencen{A_n}$
in $\Cal A$.   Then there is a measure $\mu'$ on $X$, extending $\mu$,
such that

\inset{(i) $\mu'A$ is defined and zero for every $A\in\Cal A$;

(ii) $\mu'$ is complete if $\mu$ is;
%don't complete automatically on account of ref in 495F

(iii) for
every $F$ in the domain $\Sigma'$ of $\mu'$ there is an $E\in\Sigma$
such that $\mu'(F\symmdiff E)=0$;

(iv) whenever $\Cal K$, $\Cal G$ are families of sets such that

\inset{($\alpha$) $\mu$ is inner regular with respect to $\Cal K$,

($\beta$) $K\cup K'\in\Cal K$ for all $K$, $K'\in\Cal K$,

($\gamma$) $\bigcap_{n\in\Bbb N}K_n\in\Cal K$ for every sequence
$\sequencen{K_n}$ in $\Cal K$,

($\delta$) for every $A\in\Cal A$ there is a $G\in\Cal G$, including $A$,
such that $G\setminus A\in\Sigma$,

($\epsilon$) $K\setminus G\in\Cal K$ whenever $K\in\Cal K$ and
$G\in\Cal G$,}

\noindent then $\mu'$ is inner regular with respect to $\Cal K$.}

\noindent In particular, $\mu$ and $\mu'$ have
isomorphic measure algebras, so that $\mu'$ is localizable if $\mu$ is.

\proof{{\bf (a)} Let $\Cal A^*$ be the collection of subsets of $X$ which
can be
covered by a countable subfamily of $\Cal A$.   Then $\Cal A^*$ is a
$\sigma$-ideal of subsets of $X$ and $\mu_*A=0$ for every
$A\in\Cal A^*$. Set

\Centerline{$\Sigma'=\{E\symmdiff A:E\in\Sigma,\,A\in\Cal A^*\}$.}

\noindent Then $\Sigma'$ is a $\sigma$-algebra of subsets of $X$.
\Prf\ (i) $\emptyset=\emptyset\symmdiff\emptyset\in\Sigma'$.   (ii) If
$E\in\Sigma$, $A\in\Cal A^*$ then
$X\setminus(E\symmdiff A)=(X\setminus E)\symmdiff A\in\Sigma'$.   (iii)
If $\sequencen{E_n}$,
$\sequencen{A_n}$ are sequences in $\Sigma$, $\Cal A^*$ respectively,
then

\Centerline{$E=\bigcup_{n\in\Bbb N}E_n\in\Sigma$,
\quad$A=E\symmdiff\bigcup_{n\in\Bbb N}(E_n\symmdiff A_n)
\subseteq\bigcup_{n\in\Bbb N}A_n\in\Cal A^*$,}

\noindent so $\bigcup_{n\in\Bbb N}(E_n\symmdiff A_n)
=E\symmdiff A\in\Sigma'$.\ \Qed

\medskip

{\bf (b)} If $E$, $E'\in\Sigma$, $A$, $A'\in\Cal A^*$ and
$E\symmdiff A=E'\symmdiff A'$, then
$E\symmdiff E'=A\symmdiff A'\in\Cal A^*$ and
$\mu_*(E\symmdiff E')=0$;  because $\mu$ is semi-finite,
$\mu(E\symmdiff E')=0$ and $\mu E=\mu E'$.   Accordingly we can define
$\mu':\Sigma'\to[0,\infty]$ by setting

\Centerline{$\mu'(E\symmdiff A)=\mu E$ whenever $E\in\Sigma$,
$A\in\Cal A^*$.}

\noindent Evidently $\mu'$ extends $\mu$ and $\mu'A=0$ for every
$A\in\Cal A$.   Also $\mu'$ is a measure.   \Prf\ (i)
$\mu'\emptyset=\mu\emptyset=0$.
(ii) If $\sequencen{F_n}$ is a disjoint sequence in $\Sigma'$, with
union $F$, express each $F_n$ as $E_n\symmdiff A_n$ where
$E_n\in\Sigma$, $A_n\in\Cal A^*$;  set $E=\bigcup_{n\in\Bbb N}E_n$, so
that $F\symmdiff E\in\Cal A^*$ and $\mu'F=\mu E$.    If $m\ne n$, then
$E_m\cap E_n\subseteq A_m\cup A_n$, so $\mu(E_m\cap E_n)=0$;
accordingly

\Centerline{$\mu'F=\mu E=\sum_{n=0}^{\infty}\mu E_n
=\sum_{n=0}^{\infty}\mu'F_n$.  \Qed}

\medskip

{\bf (c)}
A subset of $X$ is $\mu'$-negligible iff it can be included in a set of
the form $E\symmdiff A$ where $\mu E=0$ and $A\in\Cal A^*$, so $\mu'$ is
complete if $\mu$ is.
The embedding $\Sigma\embedsinto\Sigma'$ induces a measure-preserving
homomorphism from the measure algebra of $\mu$ to the measure algebra of
$\mu'$ which is an isomorphism just because every member of $\Sigma'$ is
the symmetric difference of a member of $\Sigma$ and a $\mu'$-negligible
set.

\medskip

{\bf (d)} This deals with (i)-(iii) in the statement of the lemma.   Now
suppose that $\Cal K$ and $\Cal G$ are as in (iv).   Take $F\in\Sigma'$ and
$\gamma<\mu'F$.   Take $E\in\Sigma$ and a sequence $\sequencen{A_n}$ in
$\Cal A$ such that $E\symmdiff F\subseteq\bigcup_{n\in\Bbb N}A_n$.
Then $\mu E=\mu'F>\gamma$ so (again because $\mu$ is semi-finite) there is
a $K\in\Cal K\cap\Sigma$ such that $K\subseteq E$ and
$\gamma<\mu K<\infty$;  set
$\epsilon=\bover12(\mu K-\gamma)>0$.   For each $n\in\Bbb N$, choose
$G_n\in\Cal G$ and $K_n\in\Cal K\cap\Sigma$ such that

\Centerline{$A_n\subseteq G_n$,
\quad$E_n=G_n\setminus A_n$ belongs to $\Sigma$,}

\Centerline{$K_n\subseteq K\cap E_n$,
\quad$\mu(K\cap E_n\setminus K_n)\le 2^{-n}\epsilon$.}

\noindent Set $L=\bigcap_{n\in\Bbb N}(K\setminus G_n)\cup K_n$.
Putting the hypotheses (iv-$\epsilon$), (iv-$\beta$) and (iv-$\gamma$)
together, we see that $L\in\Cal K$;  moreover, since $G_n=E_n\cup A_n$
belongs to $\Sigma'$ for every $n$, $L\in\Sigma'$.   Next,
setting $H=\bigcap_{n\in\Bbb N}(K\setminus E_n)\cup K_n$,
$L=H\setminus\bigcup_{n\in\Bbb N}A_n$,
so $\mu'L=\mu H$ and $L\subseteq F$.   But
$K\setminus H\subseteq\bigcup_{n\in\Bbb N}(K\cap E_n\setminus K_n)$ so

\Centerline{$\mu'L=\mu H\ge\mu K-\sum_{n=0}^{\infty}2^{-n}\epsilon
=\gamma$.}

\noindent As $F$ and $\gamma$ are arbitrary, $\mu'$ is inner regular with
respect to $\Cal K$.
}%end of proof of 417A

\leader{417B}{Lemma} Let $X$ and $Y$ be topological spaces, and $\nu$ a
$\tau$-additive topological measure on $Y$.

(a) If $W\subseteq X\times Y$ is open, then
$x\mapsto\nu W[\{x\}]:X\to[0,\infty]$ is lower semi-continuous.

(b) If $\nu$ is effectively locally finite and $\sigma$-finite and
$W\subseteq X\times Y$ is a Borel set, then $x\mapsto\nu W[\{x\}]$ is
Borel measurable.

(c) If $f:X\times Y\to[0,\infty]$ is a lower semi-continuous function,
then $x\mapsto\int f(x,y)\nu(dy):X\to[0,\infty]$ is lower
semi-continuous.

(d) If $\nu$ is totally finite and $f:X\times Y\to\Bbb R$ is a bounded
continuous function, then
$x\mapsto\int f(x,y)\nu(dy)$ is continuous.

(e) If $\nu$ is totally finite and $W\subseteq X\times Y$ is a Baire
set, then $x\mapsto\nu W[\{x\}]$ is Baire measurable.

\proof{{\bf (a)} If $x\in X$ and $\nu W[\{x\}]>\alpha$, then

\Centerline{$\Cal H=\{H:H\subseteq Y$ is open, there is an open set $G$
containing $x$ such that $G\times H\subseteq W\}$}

\noindent is an upwards-directed family of open sets with union
$W[\{x\}]$, so there is an $H\in\Cal H$ such that $\nu H\ge\alpha$.
Now there is an open set $G$ containing $x$ such that
$G\times H\subseteq W$, so that $\nu W[\{x'\}]\ge\alpha$ for every
$x'\in G$.

\medskip

{\bf (b)(i)} Suppose to begin with that $\nu$ is totally finite.   In
this case, the set

$$\eqalign{\{W:W\subseteq X\times Y,\,x\mapsto\nu W[\{x\}]&\text{ is a
Borel measurable function}\cr
&\mskip50mu\text{ defined everywhere on }X\}\cr}$$

\noindent is a Dynkin class containing every open set, so contains every
Borel set, by the Monotone Class Theorem (136B).

\medskip

\quad{\bf (ii)} For the general case, let $\sequencen{Y_n}$ be a
disjoint sequence of sets of finite measure covering $Y$, and for
$n\in\Bbb N$ let $\nu_n$ be the subspace measure on $Y_n$.   Then
$\nu_n$ is effectively locally finite and $\tau$-additive (414K).   If
$W\subseteq X\times Y$ is a Borel set, then $W_n=W\cap(X\times Y_n)$ is
a relatively Borel set for each $n$, so that $x\mapsto\nu_nW_n[\{x\}]$
is Borel measurable, by (i).   Since
$\nu W[\{x\}]=\sum_{n=0}^{\infty}\nu_nW_n[\{x\}]$ for every $x$,
$x\mapsto\nu W[\{x\}]$ is Borel measurable.

\medskip

{\bf (c)} For $i$, $n\in\Bbb N$ set $W_{ni}=\{(x,y):f(x,y)>2^{-n}i\}$,
so that $W_{ni}\subseteq X\times Y$ is open.   Set
$f_n=2^{-n}\sum_{i=1}^{4^n}\chi W_{ni}$;  then $\sequencen{f_n}$ is a
non-decreasing sequence with supremum $f$.   For $n\in\Bbb N$ and
$x\in X$,
$\int f_n(x,y)\nu(dy)=2^{-n}\sum_{i=1}^{4^n}\nu W_{ni}[\{x\}]$, so
$x\mapsto\int f_n(x,y)\nu(dy)$ is lower semi-continuous, by (a) and
4A2B(d-iii).   By 414Ba, $\int f(x,y)\nu(dy)$ is the supremum
$\sup_{n\in\Bbb N}\int f_n(x,y)\nu(dy)$ for every
$x$, so $x\mapsto\int f(x,y)\nu(dy)$ is lower semi-continuous
(4A2B(d-v)).

\medskip

{\bf (d)} Applying (c) to $f+\|f\|_{\infty}\chi(X\times Y)$, we see that
$x\mapsto\int f(x,y)\nu(dy)$ is lower semi-continuous.   Similarly,
$x\mapsto-\int f(x,y)\nu(dy)$ is lower semi-continuous, so
$x\mapsto\int f(x,y)\nu(dy)$ is continuous (4A2B(d-vi)).

\medskip

{\bf (e)} Suppose first that $W$ is a cozero set;  let
$f:X\times Y\to[0,1]$ be a continuous function such that
$W=\{(x,y):f(x,y)>0\}$.   For $n\in\Bbb N$ set
$f_n=nf\wedge\chi(X\times Y)$.   Then $\sequencen{f_n}$ is a
non-decreasing sequence of continuous functions with supremum $\chi W$.
By (d), all the functions $x\mapsto\int f_n(x,y)\nu(dy)$ are continuous,
so their limit $x\mapsto\nu W[\{x\}]$ is Baire measurable.

Now

$$\eqalign{\{W:W\subseteq X\times Y,\,x\mapsto\nu W[\{x\}]
&\text{ is a Baire measurable function}\cr
&\mskip50mu\text{ defined everywhere on }X\}\cr}$$

\noindent is a Dynkin class containing every cozero set, so contains
every Baire set, by the Monotone Class Theorem again.
}%end of proof of 417B

\vleader{72pt}{417C}{Theorem}\cmmnt{ ({\smc Ressel 77})} Let
$(X,\frak T,\Sigma,\mu)$ and
$(Y,\frak S,\Tau,\nu)$ be effectively locally finite $\tau$-additive
topological
measure spaces, with c.l.d.\ product $(X\times Y,\Lambda,\lambda)$.
Then $\lambda$ has an extension to a $\tau$-additive topological measure
$\tilde\lambda$ on $X\times Y$.   Moreover, we can arrange that:

(i) $\tilde\lambda$ is complete, locally determined and effectively
locally finite, therefore strictly localizable;

(ii) if $Q$ belongs to the domain $\tilde\Lambda$ of $\tilde\lambda$,
there is a $Q_1\in\Lambda$ such that $\tilde\lambda(Q\symmdiff Q_1)=0$;
that is to say, the embedding $\Lambda\embedsinto\tilde\Lambda$ induces an
isomorphism between the measure algebras of $\lambda$ and
$\tilde\lambda$;

(iii) if $Q\in\tilde\Lambda$, then

\Centerline{$\tilde\lambda Q=\sup\{\tilde\lambda(Q\cap(G\times H)):
  G\in\frak T,\,\mu G<\infty,\,H\in\frak S,\,\nu H<\infty\}$;}

(iv) if $W\subseteq X\times Y$ is open, then 

\quad($\alpha$) there is an open set
$W_0\in\Lambda$ such that $W_0\subseteq W$ and
$\lambda W_0=\tilde\lambda W$, 

\quad($\beta$) $\tilde\lambda W=\lambda_*W
=\int\nu W[\{x\}]\mu(dx)=\int\mu W^{-1}[\{y\}]\nu(dy)$;

\noindent in particular, $\tilde\lambda(G\times H)=\mu G\cdot\nu H$ 
whenever $G\in\frak T$ and $H\in\frak S$;

(v) the support of $\lambda$ is the product of the supports of $\mu$ and
$\nu$;

(vi) if $\mu$ and $\nu$ are both inner regular with respect to the Borel
sets, so is $\tilde\lambda$;

(vii) if $\mu$ and $\nu$ are both inner regular with respect to the
closed sets, so is $\tilde\lambda$;

(viii) if $\mu$ and $\nu$ are both tight\cmmnt{ (that is, inner
regular with respect to the closed compact sets)}, so is
$\tilde\lambda$.

\proof{ Write

\Centerline{$\Sigma^f=\{E:E\in\Sigma,\,\mu E<\infty\}$,
\quad$\Tau^f=\{F:F\in\Tau,\,\nu F<\infty\}$,}

\Centerline{$\frak T^f=\frak T\cap\Sigma^f$,
\quad$\frak S^f=\frak S\cap\Tau^f$.}

\medskip

{\bf (a)} Let $\Cal U$ be
$\{G\times H:G\in\frak T^f,\,H\in\frak S^f\}$.   Because
$\frak T\subseteq\Sigma$ and $\frak S\subseteq\Tau$,
$\Cal U\subseteq\Lambda$.   $\Cal U$ need not be a base for the topology
of $X\times Y$, unless $\mu$ and $\nu$ are locally finite, but if an
open subset of $X\times Y$ is included in a member of $\Cal U$ it is the
union of the members of $\Cal U$ it includes.   Moreover, if
$Q\in\Lambda$, then $\lambda Q=\sup_{U\in\Cal U}\lambda(Q\cap U)$.
\Prf\ By 412R, $\lambda$ is inner regular with respect to
$\bigcup_{U\in\Cal U}\Cal PU$.\ \Qed

Write $\Cal U_s$ for the set of finite
unions of members of $\Cal U$, and $\frak V$ for the set of non-empty
upwards-directed families $\Cal V\subseteq\Cal U_s$ such that
$\sup_{V\in\Cal V}\lambda V<\infty$.
For each $\Cal V\in\frak V$, fix on a countable $\Cal V'\subseteq\Cal V$
such that $\sup_{V\in\Cal V'}\lambda V=\sup_{V\in\Cal V}\lambda V$;
because $\Cal V$ is upwards-directed, we may
suppose that $\Cal V'=\{V_n:n\in\Bbb N\}$ for some non-decreasing
sequence $\sequencen{V_n}$ in $\Cal V$.   Set
$A(\Cal V)=\bigcup\Cal V\setminus\bigcup\Cal V'$.

\medskip

{\bf (b)(i)} For $V\in\Cal U_s$, set $f_V(x)=\nu V[\{x\}]$ for every
$x\in X$.   This is always defined because $V$ is open;  moreover, $f_V$
is lower semi-continuous, by 417Ba.   Because $V$ is a
finite union of products of sets of finite measure,
$\int f_Vd\mu=\lambda V$.

\medskip

\quad{\bf (ii)} The key to the proof is the following fact:  for any
$\Cal V\in\frak V$, almost
every vertical section of $A(\Cal V)$ is negligible.   \Prf\
$\family{V}{\Cal V}{f_V}$ is a
non-empty upwards-directed set of lower semi-continuous functions.
Set

\Centerline{$g(x)=\nu(\bigcup_{V\in\Cal V}V[\{x\}])$,
\quad$h(x)=\nu(\bigcup_{V\in\Cal V'}V[\{x\}])$}

\noindent for every $x\in X$.   Because $\Cal V$ is upwards-directed and
$\nu$ is $\tau$-additive,

\Centerline{$g(x)=\sup_{V\in\Cal V}\nu V[\{x\}]
=\sup_{V\in\Cal V}f_V(x)$}

\noindent in $[0,\infty]$ for each $x$, so, by 414Ba again,

\Centerline{$\int g\,d\mu=\sup_{V\in\Cal V}\int f_Vd\mu
=\sup_{V\in\Cal V}\lambda V=\sup_{V\in\Cal V'}\lambda V
=\int h\,d\mu$.}

\noindent Since $h\le g$ and $\sup_{V\in\Cal V}\lambda V$ is finite,
$g(x)=h(x)<\infty$ for $\mu$-almost every $x$.   But for any such $x$,
we must have

\Centerline{$\nu(\bigcup\Cal V)[\{x\}]
=\nu(\bigcup\Cal V')[\{x\}]<\infty$,}

\noindent so that

\Centerline{$A(\Cal V)[\{x\}]
=(\bigcup\Cal V)[\{x\}]\setminus(\bigcup\Cal V')[\{x\}]$}

\noindent is negligible.\ \Qed

\medskip

{\bf (c)} \Quer\ Suppose, if possible, that there is a sequence
$\sequencen{\Cal V_n}$ in $\frak V$ such that
$\lambda_*(\bigcup_{n\in\Bbb N}A(\Cal V_n))>0$.    Take $W\in\Lambda$
such that
$W\subseteq\bigcup_{n\in\Bbb N}A(\Cal V_n)$ and $\lambda W>0$.   Because
almost every vertical section of every $A(\Cal V_n)$ is negligible,
almost every vertical section of $W$ is negligible.   But this
contradicts Fubini's theorem (252F).\ \Bang

\medskip

{\bf (d)} Let $\lambda'$ be an extension of $\lambda$ as described in 417A,
$\tilde\lambda$
the c.l.d.\ version of $\lambda'$ (213E), and $\Lambda'$, $\tilde\Lambda$
the domains of $\lambda'$, $\tilde\lambda$ respectively.

\medskip

\quad{\bf (i)} If $W\in\Lambda$, then
$W\in\Lambda'\subseteq\tilde\Lambda$.
Also, because $\lambda$ is semi-finite,

$$\eqalign{\lambda'W
&=\lambda W
=\sup\{\lambda W':W'\subseteq W,\,W\in\Lambda,\,\lambda W'<\infty\}\cr
&\le\sup\{\lambda'W':W'\subseteq W,\,W\in\Lambda',\,\lambda'W'<\infty\}
=\tilde\lambda W
\le\lambda'W.\cr}$$

\noindent Thus $\lambda W=\tilde\lambda W$;  as $W$ is arbitrary,
$\tilde\lambda$ extends $\lambda$.

\medskip

\quad{\bf (ii)} It will be useful if we go directly to one of the
targets:  if $\tilde Q\in\tilde\Lambda$ and
$\gamma<\tilde\lambda\tilde Q$, there is
a $U\in\Cal U$ such that $\tilde\lambda(\tilde Q\cap U)\ge\gamma$.   \Prf\
There is a $Q'\in\Lambda'$ such that $Q'\subseteq\tilde Q$ and
$\gamma<\lambda'Q'<\infty$.   By 417A(iii), there is a $Q\in\Lambda$
such that $\lambda'(Q\symmdiff Q')=0$, so that

\Centerline{$\lambda Q=\lambda'Q=\lambda'Q'>\gamma$.}

\noindent There is a $U\in\Cal U$ such that
$\lambda(Q\cap U)\ge\gamma$, by (a).   Now

\Centerline{$\tilde\lambda(\tilde Q\cap U)
\ge\lambda'(Q'\cap U)=\lambda'(Q\cap U)
=\lambda(Q\cap U)\ge\gamma$.  \Qed}

\medskip

\quad{\bf (iii)} $\tilde\lambda$ is a topological measure.   \Prf\ Let
$W\subseteq X\times Y$ be an open set.   Suppose that
$\tilde Q\in\tilde\Lambda$
and $\tilde\lambda\tilde Q>0$.   By (ii), there is a $U\in\Cal U$ such that
$\tilde\lambda(\tilde Q\cap U)>0$.
Let $\Cal V$ be $\{V:V\in\Cal U_s,\,V\subseteq W\cap U\}$.   Then
$\Cal V\in\frak V$, so $\tilde\lambda A(\Cal V)=\lambda'A(\Cal V)=0$, by
417A(i); since $\bigcup\Cal V'\in\Lambda$,

\Centerline{$W\cap U=\bigcup\Cal V\in\Lambda'\subseteq\tilde\Lambda$.}

\noindent But this means that $\tilde Q\cap U\cap W$ and
$\tilde Q\cap U\setminus W=(\tilde Q\cap U)\setminus(W\cap U)$ belong to
$\tilde\Lambda$ and

\Centerline{$\tilde\lambda_*(\tilde Q\cap W)
   +\tilde\lambda_*(\tilde Q\setminus W)
\ge\tilde\lambda(\tilde Q\cap U\cap W)
  +\tilde\lambda(\tilde Q\cap U\setminus W)
=\tilde\lambda(\tilde Q\cap U)>0$.}

\noindent Because $\tilde\lambda$ is complete and locally determined,
and $\tilde Q$ is arbitrary, this is enough to ensure that
$W\in\tilde\Lambda$ (413F(vii)).\ \Qed

\medskip

\quad{\bf (iv)} $\tilde\lambda$ is $\tau$-additive.   \Prf\Quer\
Suppose, if possible, otherwise;  that there is a non-empty
upwards-directed family $\Cal W$ of open sets in $X\times Y$ such that
$\tilde\lambda W^*>\gamma=\sup_{W\in\Cal W}\tilde\lambda W$, where
$W^*=\bigcup\Cal W$.
In this case, we can find a $Q'\in\Lambda'$ such that
$Q'\subseteq W^*$ and $\lambda'Q'>\gamma$, a $Q\in\Lambda$ such that
$\lambda'(Q'\symmdiff Q)=0$, and a $U\in\Cal U$ such that
$\lambda(Q\cap U)>\gamma$ (using (a) again).   Let
$\Cal V\in\frak V$ be the set of those
$V\in\Cal U_s$ such that $V\subseteq W\cap U$ for some $W\in\Cal W$.
Then $\bigcup\Cal V=W^*\cap U$, so

$$\eqalignno{\gamma
&<\lambda(Q\cap U)
=\lambda'(Q\cap U)
=\lambda'(Q'\cap U)\cr
&\le\tilde\lambda(W^*\cap U)
=\tilde\lambda(\bigcup\Cal V)
=\tilde\lambda(\bigcup\Cal V')\cr
\noalign{\noindent (because $\tilde\lambda A(\Cal V)=0$)}
&=\sup_{V\in\Cal V'}\tilde\lambda V\cr
\noalign{\noindent (because $\Cal V'$ is countable and
upwards-directed)}
&\le\sup_{W\in\Cal W}\tilde\lambda W
\le\gamma,\cr}$$

\noindent which is absurd.\ \Bang\Qed

\medskip

{\bf (e)} Now for the supplementary properties (i)-(viii), in order.

\medskip

\quad{\bf (i)} $\tilde\lambda$ was constructed to be complete and
locally determined.   If $\tilde\lambda \tilde Q>0$, then by (d-ii)
there is a $U\in\Cal U$ such that $\tilde\lambda(\tilde Q\cap U)>0$;
since $U$ is open
and $\tilde\lambda U=\lambda U$ is finite, this shows that
$\tilde\lambda$ is effectively locally finite.   By 414J, it is strictly
localizable.

\medskip

\quad{\bf (ii)} The point is that $\lambda$ also is (strictly)
localizable.  \Prf\ Let $\tilde\mu$, $\tilde\nu$ be the c.l.d.\ versions
of $\mu$ and $\nu$.   These are
$\tau$-additive topological measures (because $\mu$ and $\nu$ are),
complete and locally determined (by construction), and are still
effectively locally finite (cf.\ 412Ha), so are strictly localizable
(414J again).
Now $\lambda$ is the c.l.d.\ product of $\tilde\mu$ and $\tilde\nu$
(251T), therefore strictly localizable (251O).\ \Qed

As remarked in 417A, it follows that $\lambda'$ is localizable, so that
every member $\tilde Q$ of $\tilde\Lambda$ differs by a
$\tilde\lambda$-negligible
set from a member $Q'$ of $\Lambda'$ (213Hb).   Now there is a
$Q\in\Lambda$ such that $\lambda'(Q\symmdiff Q')=0$, in which case
$\tilde\lambda(Q\symmdiff\tilde Q)=0$.

\medskip

\quad{\bf (iii)} This is just (d-ii) above.

\medskip

\quad{\bf (iv)} Let $W\subseteq X\times Y$ be an open set.   Set
$\Cal V=\{V:V\in\Cal U_s,\,V\subseteq W\}$.   Then $\Cal V$ is
upwards-directed and has union $W\cap(X^*\times Y^*)$, where $X^*$ and
$Y^*$ are the unions of the open sets of finite measure in $X$ and $Y$
respectively.   Because $\mu$ and $\nu$ are effectively locally finite,
$X^*$ and $Y^*$ are both conegligible, and $X^*\times Y^*$ is
$\lambda$-conegligible, therefore $\tilde\lambda$-conegligible.
Now, as before,

\Centerline{$\tilde\lambda W
=\tilde\lambda(W\cap(X^*\times Y^*))
=\sup_{V\in\Cal V}\tilde\lambda V
=\sup_{V\in\Cal V}\lambda V
\le\lambda_*W
\le\tilde\lambda W$.}

\noindent If we take a countable set $\Cal V_0\subseteq\Cal V$ such that
$\sup_{V\in\Cal V_0}\lambda V=\sup_{V\in\Cal V}\lambda V$, and set
$W_0=\bigcup\Cal V_0$, then $W_0$ is an open set, belonging to $\lambda$
and included in $W$, and $\lambda W_0=\tilde\lambda W$.

Next, defining the functions $f_V$ as in part (a) of this
proof, we have $\int f_Vd\mu=\lambda V$ for every $V\in\Cal V$;  and
setting $g(x)=\nu W[\{x\}]$, we have $g(x)=\sup_{V\in\Cal V}f_V(x)$ for
every $x\in X^*$.   Once more, 414Ba tells us that

\Centerline{$\int\nu W[\{x\}]\mu(dx)=\int g\,d\mu
=\sup_{V\in\Cal V}\int f_Vd\mu
=\sup_{V\in\Cal V}\lambda V=\tilde\lambda W$.}

\noindent Similarly,

\Centerline{$\int\mu W^{-1}[\{y\}]\nu(dy)=\tilde\lambda W$.}

\noindent (The point here is that while the arguments of part (b) of
this proof give different roles to $\mu$ and $\nu$, the asserted properties
of the extension in part (d), and the following deductions, are
symmetric between the two factors.)

Now, given $G\in\frak T$ and $H\in\frak S$, set $W=G\times H$;  then

\Centerline{$\tilde\lambda(G\times H)=\int\mu W^{-1}[\{y\}]\nu(dy)
=\int_H\mu G\,\nu(dy)=\mu G\cdot\nu H$.}

\medskip

\quad{\bf (v)} Let $Z$ and $Z'$ be the supports of $\mu$, $\nu$
respectively.   (By 411Nd these are defined.)   Then $Z\times Z'$ is a
closed subset of $X\times Y$.
Because $X\setminus Z$ and
$Y\setminus Z'$ are negligible, $Z\times Z'$ is conegligible for both
$\lambda$ and $\tilde\lambda$.
If $W\subseteq X\times Y$ is an open set and
$(x,y)\in W\cap(Z\times Z')$, there are open sets $G\subseteq X$,
$H\subseteq Y$ such that $(x,y)\in G\times H\subseteq W$.   Now

\Centerline{$\tilde\lambda(W\cap(Z\times Z'))
\ge\lambda((G\times H)\cap(Z\times Z'))
=\mu(G\cap Z)\cdot\nu(H\cap Z')>0$.}

\noindent This shows that $Z\times Z'$ is self-supporting, so is the
support of $\lambda$.

\medskip

\quad{\bf (vi), (vii), (viii)} These are all consequences of 417A(iv).
In each case, take $\Cal G$ to be the set of open subsets of $X\times Y$.
Take $\Cal K$ to be either the family of Borel subsets of $X\times Y$
(for (vi)) or the family
of closed subsets of $X\times Y$ (for (vii)) or the family of closed
compact subsets of $X\times Y$ (for (viii)).   Then ($\beta$), ($\gamma$)
and ($\epsilon$) of 417A(iv) are all satisfied.
As for 417A(iv-$\delta$), given any $A\in\Cal A$, let $\Cal V\in\frak V$
be such that $A=A(\Cal V)$, and set $G=\bigcup\Cal V\in\Cal G$;  then
$A\subseteq G$ and $G\setminus A=\bigcup\Cal V'$ belongs to $\Lambda$, as
required.    Finally,
the hypotheses of each part are just what we need in
order to be sure that $\lambda$ is inner regular with respect to
$\Cal K$ (412Sd, 412Sa, 412Sb), as required in 417A(iv-$\alpha$), so we
can conclude that $\lambda'$ is inner regular with respect to $\Cal K$.
By 412Ha, its c.l.d.\ version
$\tilde\lambda$ also is inner regular with respect to $\Cal K$.
}%end of proof of 417C

\leader{417D}{Multiple \dvrocolon{products}}\cmmnt{ Just as with the
c.l.d.\ product measure (see 251W), we can apply the construction of
417C repeatedly to obtain measures on the products of finite families of
$\tau$-additive measure spaces.

\medskip

\noindent}{\bf Proposition} (a) Let
$\familyiI{(X_i,\frak T_i,\Sigma_i,\mu_i)}$ be a finite family of
effectively locally finite $\tau$-additive topological measure spaces.
Then there is a unique complete locally determined effectively locally
finite $\tau$-additive topological measure $\tilde\lambda$ on
$X=\prod_{i\in I}X_i$, inner regular with respect to the Borel sets,
such that
$\tilde\lambda(\prod_{i\in I}G_i)=\prod_{i\in I}\mu_iG_i$ whenever
$G_i\in\frak T_i$ for every $i\in I$.

(b) If now $\family{k}{K}{I_k}$ is a partition of $I$, and
$\tilde\lambda_k$ is the product measure defined by the construction of
(a) on $Z_k=\prod_{i\in I_k}X_i$ for each $k\in K$, then the natural
bijection between $X$ and $\prod_{k\in K}Z_k$ identifies $\tilde\lambda$
with the product of the $\tilde\lambda_k$ defined by the construction of
(a).

\proof{{\bf (a)(i)} Suppose first that every $\mu_i$ is inner regular
with respect to the Borel sets.   Then a direct induction on $\#(I)$,
using 417C for the inductive step, tells us that there is a measure
$\tilde\lambda$ with the required properties.   Note that 417C(vi)
ensures that (in
the present context) all our product measures will be inner regular with
respect to the Borel sets.

\medskip

\quad{\bf (ii)} For the general case, apply (a) to
$\mu_i\restr\Cal B(X_i)$, where $\Cal B(X_i)$ is the Borel
$\sigma$-algebra of $X_i$ for each $i$.

\medskip

\quad{\bf (iii)} To see that
$\tilde\lambda$ is unique, suppose that $\lambda'$ is another measure
with the same properties.   Let $\Cal U$ be the set
$\{\prod_{i\in I}G_i:G_i\in\frak T_i$ for every $i\in I\}$, and
$\Cal U_s$ the set of finite unions
of members of $\Cal U$.   Then $\tilde\lambda$ and $\lambda'$ agree on
$\Cal U_s$.   \Prf\ Suppose that
$U=\bigcup_{j\le n}\prod_{i\in I}G_{ji}$,
where $G_{ji}\in\frak T_i$ for $j\le n$, $i\in I$.
If there is any $j$ such that
$\prod_{i\in I}\mu_iG_{ji}=\infty$, then
$\tilde\lambda U=\lambda'U=\infty$.
Otherwise, set
$L=\{j:j\le n$, $\prod_{i\in I}\mu_iG_{ji}>0\}$ and
$G_i^*=\bigcup_{j\in L}G_{ji}$ for $i\in I$.   Then

\Centerline{$\tilde\lambda(U\setminus\prod_{i\in I}G_i^*)
=0=\lambda'(U\setminus\prod_{i\in I}G_i^*)$.}

\noindent On the other hand, $\mu_iG_{ji}$ must be finite whenever
$j\in L$ and $i\in I$, so $\mu_iG_i^*$ is finite for every $i$.
Consider $\Cal I=\{\prod_{i\in I}G_i:
G_i\in\frak T_i$, $G_i\subseteq G_i^*$ for every $i\in I\}$.
Then $V\cap V'\in\Cal I$ for all
$V$, $V'\in\Cal I$, and $\tilde\lambda$, $\lambda'$ agree on $\Cal I$.
It follows from the Monotone Class Theorem (136C), or otherwise, that
$\tilde\lambda$ and
$\lambda'$ agree on the algebra of subsets of $\prod_{i\in I}G_i^*$
generated by $\Cal I$. In particular,
$\tilde\lambda(U\cap\prod_{i\in I}G_i^*)
=\lambda'(U\cap\prod_{i\in I}G_i^*)$,
so that $\tilde\lambda U=\lambda'U$.\ \Qed

But now, because $\tilde\lambda$ and $\lambda'$ are $\tau$-additive,

\Centerline{$\tilde\lambda W
=\sup\{\tilde\lambda U:U\in\Cal U_s,\,U\subseteq W\}
=\sup\{\lambda'U:U\in\Cal U_s,\,U\subseteq W\}
=\lambda'W$}

\noindent for every open set $W\subseteq X$.   Writing $\Cal B=\Cal B(X)$
for the Borel $\sigma$-algebra of $X$, $\tilde\lambda\restr\Cal B$ and
$\lambda'\restr\Cal B$ are effectively locally finite Borel measures
which agree on the open sets, so must be equal, by 414L.   Since both
$\tilde\lambda$ and $\lambda'$ are complete locally determined measures
defined on $\Cal B$ and inner regular with respect to $\Cal B$, they
also are equal, by 412L.

\medskip

{\bf (b)} Let $\lambda'$ be the measure on $X$ corresponding to the
$\tau$-additive product of the $\tilde\lambda_k$ on $\prod_{k\in K}Z_k$.
Then $\lambda'$ is an effectively locally finite complete locally
determined $\tau$-additive topological measure inner regular with
respect to the zero sets, and if $G_i\in\frak T_i$ for every $i\in I$
then

\Centerline{$\lambda'(\prod_{i\in I}G_i)
=\prod_{k\in K}\tilde\lambda_k(\prod_{i\in I_k}G_i)
=\prod_{k\in K}\prod_{i\in I_k}\mu_iG_i
=\prod_{i\in I}\mu_iG_i$,}

\noindent so $\lambda'=\tilde\lambda$.
}%end of proof of 417D

\vleader{60pt}{417E}{Theorem} Let
$\langle(X_i,\frak T_i,\Sigma_i,\mu_i)\rangle_{i\in I}$
be a family of $\tau$-additive topological probability spaces, with
product probability space $(X,\Lambda,\lambda)$.   Then $\lambda$ is
$\tau$-additive, and has an extension to a $\tau$-additive topological
measure $\tilde\lambda$ on $X$.   Moreover, we can arrange that:

(i) $\tilde\lambda$ is complete;

(ii) if $\tilde Q$ is measured by $\tilde\lambda$, there is a
$Q\in\Lambda$ such that $\tilde\lambda(\tilde Q\symmdiff Q)=0$;  that is to
say, the embedding $\Lambda\embedsinto\tilde\Lambda$ induces an
isomorphism between the measure algebras of $\lambda$ and
$\tilde\lambda$;

(iii) $\tilde\lambda W=\lambda_*W$ for every open set $W\subseteq X$,
and $\tilde\lambda F=\lambda^*F$ for every closed set $F\subseteq X$;

(iv) the support of $\lambda$ is the product of the supports of the
$\mu_i$;

(v) if $\lambda$ is inner regular with respect to the Borel sets, so is
$\tilde\lambda$;

(vi) if $\lambda$ is inner regular with respect to the closed sets, so
is $\tilde\lambda$;

(vii) if $\lambda$ is tight, so is $\tilde\lambda$.

\proof{ The strategy of the proof is the same as in 417C, subject to
some obviously necessary modifications.   The key step, showing that
every union $\bigcup_{n\in\Bbb N}A(\Cal V_n)$ has zero inner measure, is
harder, but we do save a little work because we no longer have to worry
about sets of infinite measure.

\medskip

{\bf (a)} I begin by setting up some machinery.   Let $\Cal C$ be
the family of subsets of $X$ expressible in the form
$\prod_{i\in I}E_i$, where $E_i\in\Sigma_i$ for every $i$ and
$\{i:E_i\ne X_i\}$ is finite.   Let $\Cal U\subseteq\Cal C$ be the
standard basis for
the topology $\frak T$ of $X$, consisting of sets expressible as
$\prod_{i\in I}G_i$ where $G_i\in\frak T_i$ for every $i\in I$ and
$\{i:G_i\ne X_i\}$ is finite.   Write $\Cal U_s$ for the set of finite
unions of members of $\Cal U$, and $\frak V$ for the set of non-empty
upwards-directed families in $\Cal U_s$.   Note that every member of
$\Cal U_s$ is determined by coordinates in some finite subset of $I$
(definition:  254M).

If $J\subseteq I$, write $\lambda_J$ for the product measure on
$\prod_{i\in J}X_i$; we
shall need $\lambda_{\emptyset}$, which is the unique probability
measure on the single-point set
$\{\emptyset\}=\prod_{i\in\emptyset}X_i$.   For
$J\subseteq I$, $v\in\prod_{i\in J}X_i$ and $W\subseteq X$ set

\Centerline{$f_W(v)=\lambda_{I\setminus J}\{w:(v,w)\in W\}$}

\noindent if this is defined, identifying $\prod_{i\in
J}X_i\times\prod_{i\in I\setminus J}X_i$ with $X$.

\medskip

{\bf (b)} We need two easy facts.

\medskip

\quad{\bf (i)} $f_W(v)=\int f_W(v^{\smallfrown}\fraction{t})\mu_j(dt)$
whenever $W\in\Tensorhat_{i\in I}\Sigma_i$, $J\subseteq I$,
$v\in\prod_{i\in J}X_i$ and $j\in I\setminus J$, writing
$v^{\smallfrown}\fraction{t}$ for the member $v\cup\{(j,t)\}$
of $\prod_{i\in J\cup\{j\}}X_i$
extending $v$ and taking the value $t$ at the coordinate $j$.   \Prf\
Let $\Cal A$ be the family of sets $W$ satisfying the property.   Then
$\Cal A$ is a Dynkin class including $\Cal C$, so includes the
$\sigma$-algebra generated by $\Cal C$,
which is $\Tensorhat_{i\in I}\Sigma_i$.\ \Qed

\medskip

\quad{\bf (ii)} If $J\subseteq I$, $v\in\prod_{i\in J}X_i$,
$j\in I\setminus J$ and $V\in\Cal U_s$, and we set
$g(t)=f_V(v^{\smallfrown}\fraction{t})$ for
$t\in X_j$, then $g$ is lower semi-continuous.   \Prf\
We can express $V$ as $\bigcup_{n\le m}\prod_{i\in I}G_{ni}$, where
$G_{ni}\subseteq X_i$ is open for every $n\le m$, $i\in I$.   Now if
$t\in X_j$, we shall have

\Centerline{$\{w:(v^{\smallfrown}\fraction{t},w)\in V\}
\subseteq\{w:(v^{\smallfrown}\fraction{t'},w)\in V\}$}

\noindent whenever

\Centerline{$t'\in H=X_j\cap\bigcap\{G_{nj}:n\le m,\,t\in G_{nj}\}$.}

\noindent So $g(t')\ge g(t)$ for every $t'\in H$, which is an open
neighbourhood of $t$.   As $t$ is arbitrary, $g$ is lower
semi-continuous.\ \Qed

\medskip

{\bf (c)} For each $\Cal V\in\frak V$, fix, for the remainder of this
proof, a countable $\Cal V'\subseteq\Cal V$ such that
$\sup_{V\in\Cal V'}\lambda V=\sup_{V\in\Cal V}\lambda V$;  because
$\Cal V$ is
upwards-directed, we may suppose that $\Cal V'=\{V_n:n\in\Bbb N\}$ for
some non-decreasing sequence $\sequencen{V_n}$ in $\Cal V$.   Set
$A(\Cal V)=\bigcup\Cal V\setminus\bigcup\Cal V'$.

\Quer\ Suppose, if possible, that there is a sequence
$\sequencen{\Cal V_n}$ in $\frak V$ such that
$\lambda_*(\bigcup_{n\in\Bbb N}A(\Cal V_n))>0$.

\medskip

\quad{\bf (i)} We have
$\lambda^*(X\setminus\bigcup_{n\in\Bbb N}A(\Cal V_n))<1$;  let
$\sequencen{C_n}$ be a sequence in $\Cal C$ such that

\Centerline{$X\setminus\bigcup_{n\in\Bbb N}A(\Cal V_n)
\subseteq\bigcup_{n\in\Bbb N}C_n$,
\quad$\sum_{n=0}^{\infty}\lambda C_n=\gamma_0<1$}

\noindent (see 254A-254C).   For each $n\in\Bbb N$, express $\Cal V'_n$
as $\{V_{nr}:r\in\Bbb N\}$ where $\sequence{r}{V_{nr}}$ is
non-decreasing,
and set $W_n=\bigcup\Cal V'_n=\bigcup_{r\in\Bbb N}V_{nr}$.   Let
$J\subseteq I$ be a
countable set such that every $C_n$ and every $V_{nr}$ is dependent on
coordinates in $J$.   Express $J$ as $\bigcup_{k\in\Bbb N}J_k$ where
$J_0=\emptyset$ and, for each $k$, $J_{k+1}$ is equal either to $J_k$ or
to $J_k$ with one point added.   (As in the proof of 254Fa, I am using a
formulation which will apply equally to finite and infinite $I$, though
of course the case of finite $I$ is elementary once we have 417C.)

\medskip

\quad{\bf (ii)} For each $n\in\Bbb N$, set

\Centerline{$W'_n
=\bigcup_{k\in\Bbb N}\{x:x\in X,\,f_{W_n}(x\restr J_k)=1\}$.}

\noindent Then $\lambda(W'_n\setminus W_n)=0$.   \Prf\ For any
$k\in\Bbb N$, if we think of $\lambda$ as the product of $\lambda_{J_k}$
and $\lambda_{I\setminus J_k}$ and of $f_{W_n}$ as a measurable function
on $\prod_{i\in J_k}X_i$, we see that $\{x:f_{W_n}(x\restr J_k)=1\}$ is
of the form $F_k\times\prod_{i\in I\setminus J_k}X_i$, where
$F_k\subseteq\prod_{i\in J_k}X_i$ is measurable;   and

\Centerline{$\lambda((F_k\times\prod_{i\in I\setminus J_k}X_i)
  \setminus W_n)
=\int_{F_k}(1-f_{W_n}(v))\lambda_{J_k}(dv)=0$.}

\noindent Summing over $k$, we see that $W'_n\setminus W_n$ is
negligible.\ \Qed

Observe that every $W'_n$, like $W_n$, is determined by coordinates in
$J$.   So $\bigcup_{n\in\Bbb N}W'_n\setminus W_n$ is of the form
$E\times\prod_{i\in I\setminus J}X_i$ where $\lambda_JE=0$ (254Ob).
There is therefore a sequence $\sequencen{D_n}$ of measurable cylinders
in $\prod_{i\in J}X_i$ such that $E\subseteq\bigcup_{n\in\Bbb N}D_n$ and
$\sum_{n=0}^{\infty}\lambda_JD_n<1-\gamma_0$.   Set
$C'_n=\{x:x\in X,\,x\restr J\in D_n\}\in\Cal C$ for each $n$.  Then
$\bigcup_{n\in\Bbb N}W'_n\setminus W_n
\subseteq\bigcup_{n\in\Bbb N}C'_n$, so

\Centerline{$(X\setminus\bigcup_{n\in\Bbb N}A(\Cal V_n))
  \cup\bigcup_{n\in\Bbb N}W'_n\setminus W_n
\subseteq\bigcup_{n\in\Bbb N}C_n\cup\bigcup_{n\in\Bbb N}C'_n$,}

\Centerline{$\gamma
=\sum_{n=0}^{\infty}\lambda C_n+\sum_{n=0}^{\infty}\lambda C'_n<1$,}

\noindent while each $C_n$ and each $C'_n$ is determined by coordinates
in a finite subset of $J$.

\medskip

\quad{\bf (iii)} For $k\in\Bbb N$, let $P_k$ be the set of those
$v\in\prod_{i\in J_k}X_i$ such that

\Centerline{$\sum_{n=0}^{\infty}f_{C_n}(v)+f_{C'_n}(v)\le\gamma$,
\quad$f_V(v)\le f_{W_n}(v)$ whenever $n\in\Bbb N$ and $V\in\Cal V_n$.}

\noindent Our hypothesis is that

\Centerline{$\sum_{n=0}^{\infty}f_{C_n}(\emptyset)+f_{C'_n}(\emptyset)
=\sum_{n=0}^{\infty}\lambda C_n+\lambda C'_n\le\gamma$,}

\noindent and the $\Cal V'_n$ were chosen such that

\Centerline{$f_V(\emptyset)=\lambda V\le\lambda W_n=f_{W_n}(\emptyset)$}

\noindent for every $n\in\Bbb N$, $V\in\Cal V_n$;  that is,
$\emptyset\in P_0$.

\medskip

\quad{\bf (iv)} Now if $k\in\Bbb N$ and $v\in P_k$ there is a
$v'\in P_{k+1}$
extending $v$.   \Prf\ If $J_{k+1}=J_k$ we can take $v'=v$.   Otherwise,
$J_{k+1}=J_k\cup\{j\}$ for some $j\in I\setminus J_k$.   Now

$$\eqalignno{\gamma
&\ge\sum_{n=0}^{\infty}f_{C_n}(v)+f_{C'_n}(v)
=\sum_{n=0}^{\infty}
\int f_{C_n}(v^{\smallfrown}\fraction{t})
  +f_{C'_n}(v^{\smallfrown}\fraction{t})\,\mu_j(dt)\cr
\displaycause{(b-i) above}
&=\int\sum_{n=0}^{\infty}f_{C_n}(v^{\smallfrown}\fraction{t})
  +f_{C'_n}(v^{\smallfrown}\fraction{t})\,\mu_j(dt),\cr}$$

\noindent so

\Centerline{$H
=\{t:t\in X_j,\,\sum_{n=0}^{\infty}f_{C_n}(v^{\smallfrown}\fraction{t})
  +f_{C'_n}(v^{\smallfrown}\fraction{t})\mu_j(dt)\le\gamma\}$}

\noindent has positive measure.

Next, for $V\in\Cal U_s$, set $g_V(t)=f_V(v^{\smallfrown}\fraction{t})$ for each
$t\in X_j$.   Then $g_V$ is lower semi-continuous, by
(b-ii) above.   For each $n\in\Bbb N$, $\{g_V:V\in\Cal V_n\}$ is an
upwards-directed family of lower semi-continuous functions, so its
supremum $g^*_n$ is lower semi-continuous, and because $\mu_j$ is
$\tau$-additive,

\Centerline{$\int g^*_nd\mu_j=\sup_{V\in\Cal V_n}\int g_Vd\mu_j
=\sup_{V\in\Cal V_n}f_V(v)\le f_{W_n}(v)
=\int f_{W_n}(v^{\smallfrown}\fraction{t})\mu_j(dt)$}

\noindent (using 414B and (b-i) again).   But also, because
$\sequence{r}{V_{nr}}$ is non-decreasing and has union $W_n$,

\Centerline{$f_{W_n}(v^{\smallfrown}\fraction{t})
=\sup_{r\in\Bbb N}f_{V_{nr}}(v^{\smallfrown}\fraction{t})
\le g^*_n(t)$}

\noindent for every $t\in X_j$.   So we must have

\Centerline{$f_{W_n}(v^{\smallfrown}\fraction{t})=g^*_n(t)$ a.e.($t$).}

\noindent And this is true for every $n\in\Bbb N$.

There is therefore a $t\in H$ such that

\Centerline{$f_{W_n}(v^{\smallfrown}\fraction{t})
=g^*_n(v^{\smallfrown}\fraction{t})$ for
every $n\in\Bbb N$.}

\noindent Fix on such a $t$ and set
$v'=v^{\smallfrown}\fraction{t}\in\prod_{i\in J_{k+1}}X_i$;
then $v'\in P_{k+1}$, as required.\ \Qed

\medskip

\quad{\bf (v)} We can therefore choose a sequence $\sequence{k}{v_k}$
such that $v_k\in P_k$ and $v_{k+1}$ extends $v_k$ for each $k$.
Choose $x\in X$ such that $x(i)=v_k(i)$ whenever $k\in\Bbb N$ and
$i\in J_k$, and $x(i)$ belongs to the support of $\mu_i$ whenever
$i\in I\setminus J$.    (Once again, 411Nd tells us that
every $\mu_i$ has a support.)

We need to know that if $k$, $n\in\Bbb N$ and $V\in\Cal V_n$ then
$f_{V\setminus W_n}(v_k)=0$.   \Prf\ For any $r\in\Bbb N$ there is a
$V'\in\Cal V_n$ such that $V\cup V_{nr}\subseteq V'$, so

\Centerline{$f_{V\cup V_{nr}}(v_k)\le f_{V'}(v_k)\le f_{W_n}(v_k)$,}

\noindent and

\Centerline{$f_{V\setminus W_n}(v_k)
\le f_{V\setminus V_{nr}}(v_k)
=f_{V\cup V_{nr}}(v_k)-f_{V_{nr}}(v_k)
\le f_{W_n}(v_k)-f_{V_{nr}}(v_k)
\to 0$}

\noindent as $r\to\infty$.\ \Qed

\medskip

\quad{\bf (vi)} If $n\in\Bbb N$, then $x\notin C_n\cup C'_n$.   \Prf\
$C_n$ and $C'_n$ are determined by coordinates in a finite subset of
$J$, so must be
determined by coordinates in $J_k$ for some $k\in\Bbb N$.   Now
$f_{C_n}(v_k)+f_{C'_n}(v_k)\le\gamma<1$, so $\{y:y\restr J_k=v_k\}$
cannot be included in $C_n\cup C'_n$, and must be disjoint from it;
accordingly $x\notin C_n\cup C'_n$.\ \Qed

\medskip

\quad{\bf (vii)} Because

\Centerline{$(X\setminus\bigcup_{n\in\Bbb N}A(\Cal V_n))
  \cup\bigcup_{n\in\Bbb N}W'_n\setminus W_n
\subseteq\bigcup_{n\in\Bbb N}C_n\cup\bigcup_{n\in\Bbb N}C'_n$,}

\noindent there is some $n\in\Bbb N$ such that

\Centerline{$x\in A(\Cal V_n)\setminus(W'_n\setminus W_n)
\subseteq(\bigcup\Cal V_n)\setminus W'_n$,}

\noindent that is, there is some $V\in\Cal V_n$ such that
$x\in V\setminus W'_n$.   Let $U\in\Cal U$ be such that
$x\in U\subseteq V$.   Express
$U$ as $U'\cap U''$ where $U'\in\Cal U$ is determined by coordinates in
a finite subset of $J$ and $U''\in\Cal U$ is determined by coordinates
in a finite subset of $I\setminus J$.   Let $k\in\Bbb N$ be such that
$U'$ is determined by coordinates in $J_k$.   Then

\Centerline{$f_{U\setminus W_n}(v_k)
\le f_{V\setminus W_n}(v_k)=0$}

\noindent by (v) above.   Now

\Centerline{$\{w:w\in\prod_{i\in I\setminus J_k}X_i,\,
  (v_k,w)\in U\setminus W_n\}
=\{w:(v_k,w)\in U''\setminus W_n\}$}

\noindent (because $(v_k,w)=(x\restr J_k,w)\in U'$ for every $w$), while

\Centerline{$\{w:(v_k,w)\in U''\}$,
\quad$\{w:(v_k,w)\in W_n\}$}

\noindent are stochastically independent because the former is determined by
coordinates in $I\setminus J$, while the latter is determined by coordinates in $J\setminus J_k$.   So we must have

$$\eqalign{0
&=f_{U\setminus W_n}(v_k)
=\lambda_{I\setminus J_k}\{w:(v_k,w)\in U\setminus W_n\}\cr
&=\lambda_{I\setminus J_k}\{w:(v_k,w)\in U''\setminus W_n\}\cr
&=\lambda_{I\setminus J_k}\{w:(v_k,w)\in U''\}
(1-\lambda_{I\setminus J_k}\{w:(v_k,w)\in W_n\}).\cr}$$

At this point, recall that $x(i)$ belongs to the support of $\mu_i$ for
every $i\in I\setminus J$, while $x\in U''$.   So if $U''=\{y:y(i)\in
H_i$
for $i\in K\}$, where $K\subseteq I\setminus J$ is finite and
$H_i\subseteq
X_i$ is open for every $i$, we must have $\mu_iH_i>0$ for every $i$, and

\Centerline{$\lambda_{I\setminus J_k}\{w:(v_k,w)\in U''\}
=\prod_{i\in K}\mu_iH_i>0$.}

\noindent On the other hand, we are also supposing that $x\notin W'_n$,
so

\Centerline{$\lambda_{I\setminus J_k}\{w:(v_k,w)\in W_n\}
=f_{W_n}(v_k)=f_{W_n}(x\restr J_k)<1$.}

\noindent But this means that we have expressed $0$ as the product of
two non-zero numbers, which is absurd.\ \Bang

\medskip

{\bf (d)} Thus $\lambda_*(\bigcup_{n\in\Bbb N}A(\Cal V_n))=0$ for every
sequence $\sequencen{\Cal V_n}$ in $\frak V$.   Accordingly there is an
extension of $\lambda$ to a measure $\tilde\lambda$ on $X$ as in 417A.

Now $\tilde\lambda$ is a topological measure.   \Prf\ If $W\subseteq X$
is open, then $\Cal V=\{V:V\in\Cal U_s,\,V\subseteq W\}$ belongs to
$\frak V$, and $\bigcup\Cal V=W$.   Since $\bigcup\Cal V'\in\Lambda$
(because $\Cal V'$ is countable),

\Centerline{$W=\bigcup\Cal V'\cup A(\Cal V)$}

\noindent is measured by $\tilde\lambda$.\ \Qed

Also, $\tilde\lambda$ is $\tau$-additive.   \Prf\ Let $\Cal W$ be a
non-empty upwards-directed family of open subsets of $X$ with union
$W^*$.   Set

\Centerline{$\Cal V
=\{V:V\in\Cal U_s,\,\,\exists\,W\in\Cal W,\,V\subseteq W\}$.}

\noindent Then $\Cal V\in\frak V$ and $\bigcup\Cal V=W^*$, so
$\tilde\lambda A(\Cal V)=0$ and

\Centerline{$\tilde\lambda W^*=\tilde\lambda(\bigcup\Cal V')
=\sup_{V\in\Cal V'}\tilde\lambda V
\le\sup_{W\in\Cal W}\tilde\lambda W\le\tilde\lambda W^*$}

\noindent (using the fact that $\Cal V'$ is upwards-directed).   As
$\Cal W$ is arbitrary, $\tilde\lambda$ is $\tau$-additive.\ \Qed

Of course it follows at once that $\lambda$ also is $\tau$-additive.

\medskip

{\bf (e)} Now for the supplementary properties (i)-(vi) listed in the
theorem.

\medskip

\quad{\bf (i)} Because $\lambda$ is complete, so is
$\tilde\lambda$, by 417A(ii).

\medskip

\quad{\bf (ii)} As always, the construction ensures that every member of
$\tilde\Lambda$ differs by a $\tilde\lambda$-negligible set from some
member of $\Lambda$.

\medskip

\quad{\bf (iii)} Let $W\subseteq X$ be an open set.   Set
$\Cal V=\{V:V\in\Cal U_s,\,V\subseteq W\}$.   Then

\Centerline{$\tilde\lambda W=\sup_{V\in\Cal V}\tilde\lambda V
=\sup_{V\in\Cal V}\lambda V\le\lambda_*W\le\tilde\lambda W$}

\noindent just because $\tilde\lambda$ is a $\tau$-additive extension of
$\lambda$.   Now if $F\subseteq X$ is closed,

\Centerline{$\tilde\lambda F
=1-\tilde\lambda(X\setminus F)
=1-\lambda_*(X\setminus F)
=\lambda^*F$.}

\medskip

\quad{\bf (iv)} For each $i\in I$ write $Z_i$ for the support of
$\mu_i$, and set $Z=\prod_{i\in I}Z_i$.   This is closed because every
$Z_i$ is.
Its complement is covered by the negligible open sets
$\{x:x\in X,\,x(i)\in X_i\setminus Z_i\}$ as $i$ runs over $I$;  as
$\tilde\lambda$ is $\tau$-additive, the union of the negligible open sets
is negligible, and $Z$ is conegligible.
If $W\subseteq X$ is open and $x\in Z\cap W$, let $U\in\Cal U$ be such
that $x\in U\subseteq W$.
Express $U$ as $\prod_{i\in I}G_i$ where $G_i\in\frak T_i$ for every
$i\in I$ and $J=\{i:G_i\ne X_i\}$ is finite.   Then
$x(i)\in G_i\cap Z_i$, so $\mu_iG_i>0$, for every $i$.   Accordingly

\Centerline{$\tilde\lambda(W\cap Z)
=\tilde\lambda W\ge\lambda U
=\prod_{i\in J}\mu_iG_i>0$.}

\noindent Thus $Z$ is self-supporting and is the support of $\lambda$.

\medskip

\quad{\bf (v), (vi), (vii)} As in the proof of 417C, apply 417A(iv) with
$\Cal G$ the family of open subsets of $X$ and $\Cal K$ either the Borel
$\sigma$-algebra of $X$, or the family of closed subsets of $X$, or the
family of closed compact subsets of $X$,
this time using 412U to confirm that
$\lambda$ is inner regular with respect to $\Cal K$.
}%end of proof of 417E

\leader{417F}{Corollary} Let
$\langle(X_i,\frak T_i,\Sigma_i,\mu_i)\rangle_{i\in I}$
be a family of $\tau$-additive topological probability spaces such that
$\mu_i$ is inner regular with respect to the Borel sets for each $i$.
Then there is a unique complete $\tau$-additive topological measure
$\tilde\lambda$ on $X=\prod_{i\in I}X_i$ which extends the ordinary
product measure and is inner regular with respect to the Borel sets.

\proof{ By 417E(v) we have a measure $\tilde\lambda$ with the right
properties.   If $\lambda'$ is any other complete $\tau$-additive
topological measure,
extending $\lambda$ and inner regular with respect to the family
$\Cal B$ of Borel sets, then $\lambda'W=\tilde\lambda W$ for every open
set $W\subseteq X$.   \Prf\ By the argument of (e-iii) of the proof of
417E, $\lambda'W=\lambda_*W=\tilde\lambda W$.\ \QeD\
By 414L, applied to the Borel measures $\lambda'\restr\Cal B$ and
$\tilde\lambda\restr\Cal B$, $\lambda'W=\tilde\lambda W$ for every Borel
set $W$.   Now $\lambda'$ and $\tilde\lambda$ are supposed to be
complete topological probability measures inner regular with respect to
$\Cal B$, so they must be identical, by 412L or otherwise.
}%end of proof of 417F

\leader{417G}{Notation} In the context of 417D or 417F, I will call
$\tilde\lambda$ the {\bf $\tau$-additive product measure} on
$\prod_{i\in I}X_i$.

\cmmnt{Note that the uniqueness assertions in 417D and 417F mean that
for the products of finitely many probability spaces we do not need to
distinguish
between the two constructions.   The latter also shows that we can
relate 415E to the new method:  if every $\frak T_i$ is separable and
metrizable and every $\mu_i$ is strictly positive, then the `ordinary'
product measure $\lambda$ is a complete topological measure.   Since it
is also inner regular with respect to the Borel sets (412Uc), and
$\tau$-additive (because we now know that it has an extension to a
$\tau$-additive measure) it must be
exactly the $\tau$-additive product measure as described here.}

\leader{417H}{Fubini's theorem for $\tau$-additive product measures} Let
$(X,\frak T,\Sigma,\mu)$ and
$(Y,\frak S,\Tau,\nu)$ be two complete locally determined effectively
locally finite $\tau$-additive topological measure spaces such that both
$\mu$ and $\nu$ are inner regular with respect to the Borel sets.   Let
$\tilde\lambda$ be the $\tau$-additive product measure on $X\times Y$,
and $\tilde\Lambda$ its domain.

(a) Let $f$ be a $[-\infty,\infty]$-valued function such
that $\int f\,d\tilde\lambda$ is defined in $[-\infty,\infty]$ and
$(X\times Y)\setminus\{(x,y):(x,y)\in\dom f,\,f(x,y)=0\}$
can be covered by a set of
the form $X\times\bigcup_{n\in\Bbb N}Y_n$ where $\nu Y_n<\infty$ for
every $n\in\Bbb N$.   Then the repeated integral
$\iint f(x,y)\nu(dy)\mu(dx)$ is defined and equal to
$\int fd\tilde\lambda$.

(b) Let $f:X\times Y\to[0,\infty]$ be lower semi-continuous.   Then

\Centerline{$\biggeriint f(x,y)\nu(dy)\mu(dx)
=\iint f(x,y)\mu(dx)\nu(dy)=\int fd\tilde\lambda$}

\noindent in $[0,\infty]$.

(c) Let $f$ be a $\tilde\Lambda$-measurable real-valued function defined
$\tilde\lambda$-a.e.\ on $X\times Y$.   If either
$\iint|f(x,y)|\nu(dy)\mu(dx)$ or $\iint|f(x,y)|\mu(dx)\nu(dy)$ is
defined and finite, then $f$ is $\tilde\lambda$-integrable.

\proof{{\bf (a)} I use 252B.

\quad{\bf (i)} Write $\Cal W$ for the set of those $W\in\tilde\Lambda$
such that $\int\nu W[\{x\}]\mu(dx)$ is defined in $[0,\infty]$ and equal
to $\tilde\lambda W$.   Then open sets belong to $\Cal W$, by 417C(iv).
Next, any Borel subset of an open set of finite measure
belongs to $\Cal W$.   \Prf\ If $W_0$ is an open set of finite measure,
then $\{W:W\subseteq X\times Y,\,W\cap W_0\in\Cal W\}$ is a Dynkin class
containing every open set, so contains all Borel subsets of
$X\times Y$.\ \Qed

Now suppose that $W\subseteq X\times Y$ is $\tilde\lambda$-negligible
and included in $X\times\bigcup_{n\in\Bbb N}Y_n$, where $\nu Y_n<\infty$
for every $n$.   Then $W\in\Cal W$.   \Prf\ Set
$A=\{x:x\in X,\,\nu^*W[\{x\}]>0\}$.   For each $n$, let $H_n\subseteq Y$
be an open set of finite measure such
that $\nu(Y_n\setminus H_n)\le 2^{-n}$;  we may arrange that
$H_{n+1}\supseteq H_n$ for each $n$.  Set $H=\bigcup_{n\in\Bbb N}H_n$,
so that $W[\{x\}]\setminus H\subseteq\bigcup_{n\in\Bbb N}Y_n\setminus H$
is negligible for every $x\in X$.

Fix an open set $G\subseteq X$ of finite measure and $n\in\Bbb N$ for
the moment.   Because
$\tilde\lambda$ is inner regular with respect to the Borel sets, there
is a Borel set $V\subseteq(G\times H_n)\setminus W$ such that
$\tilde\lambda V=\tilde\lambda((G\times H_n)\setminus W)$, that is,
$\tilde\lambda V'=0$,
where $V'=(G\times H_n)\setminus V\supseteq (G\times H_n)\cap W$.
We know that $V'\in\Cal W$, so

\Centerline{$\int\nu V'[\{x\}]dx=\tilde\lambda V'=0$,}

\noindent and $\nu V'[\{x\}]=0$ for almost every $x\in X$;  but this
means that $H_n\cap W[\{x\}]$ is negligible for almost every $x\in G$.

At this point, recall that $n$ was arbitrary, so
$H\cap W[\{x\}]$ and $W[\{x\}]$ are negligible for almost every
$x\in G$, that is, $A\cap G$ is negligible.   This is true for every
open set $G\subseteq X$ of finite measure.
Because $\mu$ is inner regular with respect to subsets of open sets of
finite measure, and is complete and locally determined, $A$ is
negligible (412Jb).   But this means that $\int\nu W[\{x\}]\mu(dx)$ is
defined and equal to zero, so that $W\in\Cal W$.\ \Qed

\medskip

\quad{\bf (ii)} Now suppose that $\int fd\tilde\lambda$ is defined in
$[-\infty,\infty]$ and that there is a sequence $\sequencen{Y_n}$ of
sets of finite measure in $Y$ such that $f(x,y)$ is defined and zero
whenever $x\in X$ and $y\in Y\setminus\bigcup_{n\in\Bbb N}Y_n$.   Set
$Z=\bigcup_{n\in\Bbb N}Y_n$.   Write $\lambda$ for the c.l.d.\ product
measure on $X\times Y$ and $\Lambda$ for its domain.  Then there is a
$\Lambda$-measurable function $g:X\times Y\to[-\infty,\infty]$ which is
equal $\tilde\lambda$-almost everywhere to $f$.  \Prf\ For $q\in\Bbb Q$
set $W_q=\{(x,y):(x,y)\in\dom f,\,f(x,y)\ge q\}\in\tilde\Lambda$, and
choose $V_q\in\Lambda$ such that $\tilde\lambda(W_q\symmdiff V_q)=0$
(417C(ii));  set $g(x,y)=\sup\{q:q\in\Bbb Q,\,(x,y)\in V_q\}$ for
$x\in X$, $y\in Y$, interpreting $\sup\emptyset$ as $-\infty$.\ \QeD\
Adjusting $g$ if necessary, we may suppose that it is zero on
$X\times(Y\setminus Z)$.   Set

\Centerline{$A=(X\times Y)\setminus\{(x,y):f(x,y)=g(x,y)\}$,}

\noindent so that $A$ is $\tilde\lambda$-negligible and included in
$X\times Z$.   By (i), $\nu A[\{x\}]=0$, that is, $y\mapsto f(x,y)$ and
$y\mapsto g(x,y)$ are equal $\nu$-a.e., for $\mu$-almost every $x$.
Write $\lambda_{X\times Z}$ for the subspace measure induced by
$\lambda$ on $X\times Z$;   note that this is the c.l.d.\ product of
$\mu$ with the subspace measure $\nu_Z$ on $Z$, by 251Q(ii-$\alpha$).

Now we have

$$\eqalignno{\int fd\tilde\lambda
&=\int g\,d\tilde\lambda
=\int g\,d\lambda\cr
\displaycause{by 235Gb, because the identity map from
$(X\times Y,\tilde\lambda)$ to $(X\times Y,\lambda)$ is \imp}
&=\int_{X\times Z}g\,d\lambda
=\int_{X\times Z}g\,d\lambda_{X\times Z}
=\iint_Z g(x,y)\nu_Z(dy)\mu(dx)\cr
\displaycause{by 252B, because $\nu_Z$ is $\sigma$-finite}
&=\iint g(x,y)\nu(dy)\mu(dx)\cr
\displaycause{because $g(x,y)=0$ if $y\in Y\setminus Z$}
&=\iint f(x,y)\nu(dy)\mu(dx).\cr}$$

\woddheader{417H}{4}{2}{2}{24pt}

{\bf (b)} If $f$ is non-negative and lower semi-continuous, set

\Centerline{$W_{ni}=\{(x,y):f(x,y)>2^{-n}i\}$}

\noindent for $n$, $i\in\Bbb N$, and

\Centerline{$f_n=2^{-n}\sum_{i=1}^{4^n}\chi W_{ni}$}

\noindent for $n\in\Bbb N$.   Applying 417C(iv) we see that

\Centerline{$\int f_nd\tilde\lambda=\biggeriint f_n(x,y)dydx
=\iint f_n(x,y)dxdy$}

\noindent in $[0,\infty]$ for every $n$;  taking the limit as
$n\to\infty$,

\Centerline{$\int fd\tilde\lambda=\biggeriint f(x,y)dydx
=\iint f(x,y)dxdy$,}

\noindent because $\sequencen{f_n}$ is a non-decreasing sequence with
limit $f$.

\medskip

{\bf (c)} \Quer\ Suppose, if possible, that $\gamma=\iint|f(x,y)|dydx$
is finite, but that $f$ is not integrable.   Because $\tilde\lambda$ is
semi-finite, there must be a non-negative $\tilde\lambda$-simple
function $g$ such that $g\leae|f|$ and
$\int g\,d\tilde\lambda>\gamma$ (213B).   For each $n\in\Bbb N$, there
are open sets $G_n\subseteq X$, $H_n\subseteq Y$ of finite measure such
that $\tilde\lambda(\{(x,y):g(x,y)\ge 2^{-n}\}\setminus(G_n\times H_n))
\le 2^{-n}$, by 417C(iii);  now $g\times\chi(G_n\times H_n)\to g$ a.e.,
so there is some $n$ such that
$\int_{G_n\times H_n}g\,d\tilde\lambda>\gamma$.   In this case, setting
$g'(x,y)=\min(g(x,y),|f(x,y)|)$ for $(x,y)\in(G_n\times H_n)\cap\dom f$,
$0$ otherwise, we have $g=g'$ a.e.\ on $G_n\times H_n$, so that
$\int g'd\tilde\lambda>\gamma$.   But we can apply (a) to $g'$ to see
that

\Centerline{$\gamma<\int g'd\tilde\lambda
=\biggeriint g'(x,y)dydx\le\iint|f(x,y)|dydx\le\gamma$,}

\noindent which is absurd.\ \Bang

So if $\iint|f(x,y)|dydx$ is finite, $f$ must be
$\tilde\lambda$-integrable.   Of course the same arguments, reversing
the roles of $X$ and $Y$, show that $f$ is $\tilde\lambda$-integrable
if $\iint|f(x,y)|dxdy$ is defined and finite.
}%end of proof of 417H

\leader{417I}{}\cmmnt{ The constructions here have most of the
properties one would hope for.   I give several in the exercises
(417Xd-417Xf, %417Xd, 417Xe, 417Xf,
417Xj).   One fact which is particularly
useful, and also has a trap in it, is the following.

\medskip
\wheader{417I}{0}{0}{0}{48pt}

\noindent}{\bf Proposition} Let $(X,\frak T,\Sigma,\mu)$ and
$(Y,\frak S,\Tau,\nu)$ be effectively locally finite $\tau$-additive
topological
measure spaces in which the measures are inner regular with respect to
the Borel sets, and $\tilde\lambda$ the $\tau$-additive product measure
on $X\times Y$.   Suppose that $A\subseteq X$ and $B\subseteq Y$, and
write $\mu_A$, $\nu_B$ for the corresponding subspace measures;  assume
that both $\mu_A$ and $\nu_B$ are semi-finite.   Then these are also
effectively locally finite,
$\tau$-additive and inner regular with respect to the Borel sets, and
the subspace measure $\tilde\lambda_{A\times B}$ induced by
$\tilde\lambda$ on $A\times B$ is just the $\tau$-additive product
measure of $\mu_A$ and $\nu_B$.

\proof{{\bf (a)} To check that $\mu_A$ and $\nu_B$ are effectively
locally finite, $\tau$-additive and inner regular with respect to the
Borel sets, see 414K and 412P.   Of course $\tilde\lambda_{A\times B}$
inherits the same properties from $\tilde\lambda$, and is
in addition complete
and strictly localizable, because $\tilde\lambda$ is (417C(i), 214Ia).

\medskip

{\bf (b)} Now if $C\in\dom\mu_A$ and $D\in\dom\nu_B$, then
$\tilde\lambda^*(C\times D)=\mu_AC\cdot\nu_BD$.   \Prf\ ($\alpha$) There
are $E\in\Sigma$, $F\in\Tau$ such that $C\subseteq E$, $D\subseteq F$,
$\mu E=\mu^*C$ and $\nu F=\nu^*D$;  in which case

$$\eqalignno{\tilde\lambda^*(C\times D)
&\le\tilde\lambda(E\times F)
=\lambda(E\times F)
=\mu E\cdot\nu F\cr
\noalign{\noindent (251J)}
&=\mu^*C\cdot\nu^*D=\mu_AC\cdot\nu_BD.\cr}$$

\noindent ($\beta$) If $\gamma<\mu_AC\cdot\nu_BD$ then, because $\mu_A$
and $\nu_B$ are semi-finite, there are $C'\subseteq C$, $D'\subseteq D$
such that both have finite outer measure and
$\mu^*C'\cdot\nu^*D'\ge\gamma$.   In
this case, take $E'\in\Sigma$, $F'\in\Tau$ such that $C'\subseteq E'$,
$D'\subseteq F'$ and both $E'$ and $F'$ have finite measure.   Now if
$W\in\dom\tilde\lambda$ and $C\times D\subseteq W$, we have
$C'\times D'\subseteq W\cap(E\times F)$, so that
$\nu(W\cap(E\times F))[\{x\}]\ge\nu^*D'$ for every $x\in C'$, and

\Centerline{$\tilde\lambda W
\ge\int_E\nu(W\cap(E\times F))[\{x\}]\mu(dx)
\ge\mu^*C'\cdot\nu^*D'
\ge\gamma$,}

\noindent by 417Ha.   As $W$ is arbitrary,
$\tilde\lambda^*(C\times D)\ge\gamma$;  as $\gamma$ is arbitrary,
$\tilde\lambda^*(C\times D)\ge\mu_AC\cdot\nu_BD$.\ \Qed

\medskip

{\bf (c)} In particular, if $U\subseteq A$ and $V\subseteq B$ are
relatively open,

\Centerline{$\tilde\lambda_{A\times B}(U\times V)
=\tilde\lambda^*(U\times V)
=\mu_AU\cdot\nu_BV$.}

\noindent But now 417D tells us that $\tilde\lambda_{A\times B}$ must be
exactly the $\tau$-additive product measure of $\mu_A$ and $\nu_B$.
}%end of proof of 417I

\leader{417J}{}\cmmnt{ In order to use 417H effectively in the theory
of infinite products, we need an `associative law' corresponding to
254N.

\medskip

\noindent}{\bf Theorem} Let
$\langle(X_i,\frak T_i,\Sigma_i,\mu_i)\rangle_{i\in I}$ be a family of
$\tau$-additive topological probability
spaces such that every $\mu_i$ is inner regular with respect to the
Borel sets, and $\langle K_j\rangle_{j\in J}$ a partition of $I$.   For
each $j\in J$ let $\tilde\lambda_j$ be the $\tau$-additive product
measure on
$Z_j=\prod_{i\in K_j}X_i$, and write $\tilde\lambda$ for the
$\tau$-additive product measure on $X=\prod_{i\in I}X_i$.   Then the
natural bijection

\Centerline{$x\mapsto\phi(x)
=\langle x\restr K_j\rangle_{j\in J}:X\to\prod_{j\in J}Z_j$}

\noindent identifies $\tilde\lambda$ with the $\tau$-additive product of
the family $\langle\tilde\lambda_j\rangle_{j\in J}$.

In particular, if $K\subseteq I$ is any set, then $\tilde\lambda$ can be
identified with the $\tau$-additive product of the $\tau$-additive
product measures on $\prod_{i\in K}X_i$ and
$\prod_{i\in I\setminus K}X_i$.

\proof{ We have a lot of measures to keep track of;  I hope that the
following notation will not be too opaque.   Write $\lambda$ for the
ordinary product measure on $X$, and for $j\in J$ write $\lambda_j$,
$\tilde\lambda_j$ for
the ordinary and $\tau$-additive product measures on $Z_j$.   Write
$\theta$ for the
{\it ordinary} product measure on $Z=\prod_{j\in J}Z_j$ of the
{\it $\tau$-additive} product measures $\tilde\lambda_j$, and
$\tilde\theta$
for the $\tau$-additive product of the $\tilde\lambda_j$.   Write
$\tilde\lambda^{\#}$ for the measure on $X$ corresponding to
$\tilde\theta$
on $Z$.   (If you like, $\tilde\lambda^{\#}$ is the image measure
$\tilde\theta(\phi^{-1})^{-1}$ defined from $\tilde\theta$ and the
function $\phi^{-1}:Z\to X$.)   Then $\tilde\lambda^{\#}$, like
$\tilde\theta$, is a
complete $\tau$-additive topological measure, inner regular with respect
to the Borel sets, because $\phi:X\to Z$ is a homeomorphism.
If $C\subseteq X$ is a measurable cylinder, it is of the form
$\prod_{i\in I}E_i$ where $E_i\in\Sigma_i$ for each $i$ and
$\{i:i\in I,\,E_i\ne\Sigma_i\}$ is finite.   So $\phi[C]$ is of the form
$\prod_{j\in J}C_j$, where $C_j=\prod_{i\in K_j}E_i$, and

$$\eqalign{\tilde\lambda^{\#}C
&=\tilde\theta(\prod_{j\in J}C_j)
=\theta(\prod_{j\in J}C_j)
=\prod_{j\in J}\tilde\lambda_j C_j\cr
&=\prod_{j\in J}\lambda_jC_j
=\prod_{j\in J}\prod_{i\in K_j}\mu_iE_i
=\prod_{i\in I}\mu_iE_i
=\lambda C.\cr}$$

\noindent But this means, applying 254G to the identity map from
$(X,\tilde\lambda^{\#})$ to $(X,\lambda)$, that $\tilde\lambda^{\#}$
extends $\lambda$.   So it is a complete $\tau$-additive topological
measure,
inner regular with respect to the Borel sets, extending the ordinary
product measure, and by the uniqueness declared in 417F, must be
identical to the $\tau$-additive product measure on $X$, as claimed.
}%end of proof of 417J

\leader{417K}{Proposition} Let
$\familyiI{(X_i,\frak T_i,\Sigma_i,\mu_i)}$
be a family of $\tau$-additive topological probability spaces such that
every $\mu_i$ is inner regular with respect to the Borel sets, and 
$(X,\tilde\Lambda,\tilde\lambda)$ their $\tau$-additive product.
For $J\subseteq I$ let $\tilde\lambda_J$ be the
$\tau$-additive product measure on $X_J=\prod_{i\in J}X_i$, and
$\tilde\Lambda_J$ its domain;  let $\pi_J:X\to X_J$ be the canonical
map.   Then $\tilde\lambda_J$ is the image measure
$\tilde\lambda\pi_J^{-1}$.
In particular, if $W\in\tilde\Lambda$ is determined by coordinates in
$J\subseteq I$, then $\pi_J[W]\in\tilde\Lambda_J$ and
$\tilde\lambda_J\pi_J[W]=\tilde\lambda W$.

\proof{ Because $\tilde\lambda$ is an extension of the ordinary product
measure $\lambda$ on $X$, $\tilde\lambda\pi_J^{-1}$ is an extension of
$\lambda\pi_J^{-1}$, which is the ordinary product measure on $X_J$
(254Oa).   Because $\tilde\lambda$ is a $\tau$-additive topological
measure and $\pi_J$ is continuous, $\tilde\lambda\pi_J^{-1}$ is a
$\tau$-additive topological
measure;  because $\tilde\lambda$ is a complete probability measure, so
is $\tilde\lambda\pi_J^{-1}$.   Finally, $\tilde\lambda\pi_J^{-1}$ is
inner regular with respect to the Borel sets.   \Prf\ Recall that we may
identify $\tilde\lambda$ with the $\tau$-additive product of
$\tilde\lambda_J$ and
$\tilde\lambda_{I\setminus J}$ (417J).   If
$V\in\dom\tilde\lambda\pi_J^{-1}$, that is,
$\pi_J^{-1}[V]\in\tilde\Lambda$,
we can think of $\pi_J^{-1}[V]\subseteq X$ as $V\times X_{I\setminus
J}\subseteq X_I\times X_J$.   In this case, we must have

\Centerline{$\tilde\lambda\pi_J^{-1}[V]
=\int\tilde\lambda_JV\,d\lambda_{I\setminus J}$,}

\noindent by Fubini's theorem for $\tau$-additive products (417Ha);
that is, $\tilde\lambda_JV$ must be defined and equal to
$\tilde\lambda\pi_J^{-1}[V]$.   Now if
$\gamma<\tilde\lambda\pi_J^{-1}[V]$,
there must be a Borel set $V'\subseteq V$ such that
$\tilde\lambda_JV'\ge\gamma$.   In this case, because $\pi_J$ is
continuous, $\pi_J^{-1}[V']$ also is Borel, and
$\tilde\lambda\pi_J^{-1}[V']$ is defined.   As with $V$, this measure
must be $\tilde\lambda_JV'\ge\gamma$.
Since $V$ and $\gamma$ are arbitrary, $\tilde\lambda\pi_J^{-1}$ is inner
regular with respect to the Borel sets, as claimed.\ \Qed

By the uniqueness assertion in 417F, $\tilde\lambda\pi_J^{-1}$ must be
$\tilde\lambda_J$ exactly.

If now $W\in\tilde\Lambda$ is determined by coordinates in $J$, then

\Centerline{$\tilde\lambda_J\pi_J[W]=\tilde\lambda\pi_J^{-1}[\pi_J[W]]
=\tilde\lambda W$.}
}%end of proof of 417K

\vleader{48pt}{417L}{Corollary} Let
$\familyiI{(X_i,\frak T_i,\Sigma_i,\mu_i)}$
be a family of $\tau$-additive topological probability spaces such that
every $\mu_i$ is inner regular with respect to the Borel sets, and
$(X,\tilde\Lambda,\tilde\lambda)$ their
$\tau$-additive product.   Let $\family{j}{J}{K_j}$ be a disjoint family
of subsets of $I$, and for $j\in J$ write $\tilde\Lambda_j$ for the
$\sigma$-algebra of members of $\Lambda$ determined by coordinates in
$K_j$.   Then $\family{j}{J}{\tilde\Lambda_j}$ is a stochastically
independent family of $\sigma$-algebras\cmmnt{ (definition: 272Ab)}.

\proof{ It is enough to consider the case in which $J$ is finite
(272Bb), no $K_j$ is empty (since if $K_j=\emptyset$ then
$\tilde\Lambda_j=\{\emptyset,X\}$) and $\bigcup_{j\in J}K_j=I$ (adding
an extra term if necessary).   In this case, if $W_j\in\tilde\Lambda_j$
for each $j$, then the identification between $X$ and
$\prod_{j\in J}\prod_{i\in K_j}X_i$, as described in 417J, matches
$\bigcap_{j\in J}W_j$ with $\prod_{j\in J}\pi_{K_j}[W_j]$, writing
$\pi_{K_j}(x)$ for $x\restr K_j$.   Now if $\tilde\lambda_j$ is the
$\tau$-additive product measure on $Z_j=\prod_{i\in K_j}X_i$, we have
$\tilde\lambda_j\pi_{K_j}[W_j]=\tilde\lambda W_j$, by 417K.   Since
$\tilde\lambda$ can be identified with the $\tau$-additive product of
$\family{j}{J}{\tilde\lambda_j}$ (417J),

\Centerline{$\tilde\lambda(\bigcap_{j\in J}W_j)
=\prod_{j\in J}\tilde\lambda_j\pi_{K_j}[W_j]
=\prod_{j\in J}\tilde\lambda W_j$.}

As $\family{j}{J}{W_j}$ is arbitrary, $\family{j}{J}{\tilde\Lambda_j}$
is independent.
}%end of proof of 417L

\vleader{60pt}{417M}{Proposition} Let
$\familyiI{(X_i,\frak T_i,\Sigma_i,\mu_i)}$
be a family of $\tau$-additive topological probability spaces such that
every $\mu_i$ is inner regular with respect to the Borel sets and
strictly positive.   For $J\subseteq I$ let $\pi_J$ be the canonical map
from $X$ onto $X_J=\prod_{i\in J}X_i$;  write $\lambda_J$,
$\tilde\lambda_J$
for the ordinary and $\tau$-additive product measures on $X_J$, and
$\Lambda_J$, $\tilde\Lambda_J$ for their domains.   Set
$\tilde\lambda=\tilde\lambda_I$, $\tilde\Lambda=\tilde\Lambda_I$,
$\lambda=\lambda_I$, $\Lambda=\Lambda_I$.

(a) Let $F\subseteq X$ be a closed self-supporting set, and $J$ the
smallest subset of $I$ such that $F$ is determined by coordinates
in $J$\cmmnt{ (4A2B(g-ii))}.   Then

\quad(i) if $W\in\tilde\Lambda$ is such that $W\symmdiff F$ is
$\tilde\lambda$-negligible and determined by coordinates in
$K\subseteq I$, then $K\supseteq J$;

\quad(ii) $J$ is countable;

\quad(iii) there is a $W\in\Lambda$, determined by coordinates in $J$,
such that $W\symmdiff F$ is $\tilde\lambda$-negligible.

(b) $\tilde\lambda$ is inner regular with respect to the family of sets
of the form $\bigcap_{n\in\Bbb N}V_n$ where each $V_n\in\tilde\Lambda$
is determined by finitely many coordinates.

(c) If $W\in\tilde\Lambda$, there are a countable $J\subseteq I$ and
sets $W'$, $W''\in\tilde\Lambda$, determined by coordinates in $J$, such
that $W'\subseteq W\subseteq W''$ and
$\tilde\lambda(W''\setminus W')=0$.   Consequently
$\tilde\lambda\pi_J^{-1}[\pi_J[W]]=\tilde\lambda W$.

\proof{{\bf (a)(i)} \Quer\ Suppose, if possible, otherwise.   Then $F$
is not determined by coordinates in $K$, so there are $x\in F$,
$y\in X\setminus F$ such that $x\restr K=y\restr K$.
Let $U$ be an open set containing $y$, disjoint from $F$, and of
the form $\prod_{i\in I}G_i$, where $G_i\in\frak T_i$ for every $i$ and
$L=\{i:G_i\ne X_i\}$ is finite.   Set

\Centerline{$U'=\{z:z\in X,\,z(i)\in G_i$ for every $i\in L\cap K\}$,}

\Centerline{$U''=\{z:z(i)\in G_i$ for every $i\in L\setminus K\}$.}

\noindent Then $U'\cap W$ is determined by coordinates in $K$, while
$U''$ is determined by coordinates in $I\setminus K$, so

$$\eqalignno{0
&=\tilde\lambda(F\cap U)
=\tilde\lambda(W\cap U)=\tilde\lambda(W\cap U'\cap U'')
=\tilde\lambda(W\cap U')\cdot\tilde\lambda U''\cr
\displaycause{by 417L}
&=\tilde\lambda(F\cap U')\cdot\tilde\lambda U''
=\tilde\lambda(F\cap U')\cdot\prod_{i\in L\setminus K}\mu_iG_i.\cr}$$

\noindent But $y\in U'$, and $x\restr K=y\restr K$, so $x\in F\cap U'$;
as $F$ is self-supporting, $\tilde\lambda(F\cap U')>0$.   Because every
$\mu_i$ is strictly positive, and no $G_i$ is empty,
$\prod_{i\in L\setminus K}\mu_iG_i>0$;  and this is impossible.\ \Bang

\medskip

\quad{\bf (ii)} By 417E(ii), there is a $W_0\in\Lambda$ such that
$\tilde\lambda(F\symmdiff W_0)=0$.   By 254Oc there is a
$W_1\in\Lambda$, determined by coordinates in a countable subset $K$ of
$I$, such that $\lambda(W_0\symmdiff W_1)=0$.   Now
$\tilde\lambda(F\symmdiff W_1)=0$, so (i) tells us that $J\subseteq K$
is countable.

\medskip

\quad{\bf (iii)} By 417K, $\pi_J[F]\in\tilde\Lambda_J$.   By 417E(ii),
there is a $V\in\Lambda_J$ such that $V\symmdiff\pi_J[F]$ is
$\tilde\lambda_J$-negligible.   Set $W=\pi_J^{-1}[V]$.   Then
$W\in\Lambda$, $W$ is determined by coordinates in $J$, and
$W\symmdiff F=\pi_J^{-1}[V\symmdiff\pi_J[F]]$ is
$\tilde\lambda$-negligible.

\medskip

{\bf (b)(i)} Write $\Cal V$ for the set of those members of
$\tilde\Lambda$
which are determined by finitely many coordinates, and $\Cal V_{\delta}$
for the set of intersections of sequences in $\Cal V$.   Because
$\Cal V$ is closed under finite unions, so is $\Cal V_{\delta}$;
$\Cal V_{\delta}$
is surely closed under countable intersections, and $\emptyset$, $X$
belong to $\Cal V_{\delta}$.

\medskip

\quad{\bf (ii)} We need to know that every self-supporting closed set
$F\subseteq X$ belongs to $\Cal V_{\delta}$.   \Prf\ By (a), $F$ is
determined by a countable set $J$ of coordinates.   Express $J$ as the
union of a non-decreasing sequence $\sequencen{J_n}$ of finite sets.
Then $F_n=\pi_{J_n}^{-1}[\overline{\pi_{J_n}[F]}]\in\Cal V$ for each
$n$, and $F=\bigcap_{n\in\Bbb N}F_n\in\Cal V_{\delta}$.\ \Qed

\medskip

\quad{\bf (iii)} Let $\Cal A$ be the family of subsets of $X$ which are
either open or closed.   Then if $A\in\Cal A$, $V\in\tilde\Lambda$ and
$\tilde\lambda(A\cap V)>0$, there is a $K\in\Cal V_{\delta}\cap\Cal A$
such that $K\subseteq A$ and $\tilde\lambda(K\cap V)>0$.   \Prf\
($\alpha$) If $A$ is open, set

\Centerline{$\Cal U=\{U:U\in\Cal V$ is open, $U\subseteq A\}$.}

\noindent Because $\Cal V$ includes a base for the topology of $X$,
$\bigcup\Cal U=A$;  because $\tilde\lambda$ is $\tau$-additive and
$\Cal V$ is closed under finite unions, there is a $U\in\Cal U$ such
that $U\subseteq A$ and
$\tilde\lambda U>\tilde\lambda A-\tilde\lambda(A\cap V)$, so that
$\tilde\lambda(U\cap V)>0$.   ($\beta$) If $A$ is closed, then it
includes a
self-supporting closed set $V$ of the same measure (414F), which belongs
to $\Cal V_{\delta}$, by (ii) just above.\ \Qed

\medskip

\quad{\bf (iv)} By 412C, the restriction $\tilde\lambda\restr\Cal B$ of
$\tilde\lambda$ to the Borel $\sigma$-algebra of $X$ is inner regular
with respect to $\Cal V_{\delta}$.   But $\tilde\lambda$
is just the completion of $\tilde\lambda\restr\Cal B$, so it also is
inner regular with respect to $\Cal V_{\delta}$ (412Ha).

\medskip

{\bf (c)} By (b), we have sequences $\sequencen{V_n}$,
$\sequencen{V'_n}$ in $\Cal V_{\delta}$ such that $V_n\subseteq W$,
$V'_n\subseteq X\setminus W$,
$\tilde\lambda V_n\ge\tilde\lambda W-2^{-n}$ and
$\tilde\lambda V'_n\ge\tilde\lambda(X\setminus W)-2^{-n}$ for every
$n\in\Bbb N$.
Each $V_n$, $V'_n$ is determined by a countable set of coordinates, so
there is a single countable set $J\subseteq I$ such that every $V_n$ and
every $V'_n$ is determined by coordinates in $J$.   Set
$W'=\bigcup_{n\in\Bbb N}V_n$, $W''=X\setminus\bigcup_{n\in\Bbb N}V'_n$;
then $W'$, $W''$ are both determined by coordinates in $J$,
$W'\subseteq W\subseteq W''$ and
$\tilde\lambda(W''\setminus W')=0$, as required.
}%end of proof of 417M

\leader{417N}{Theorem} Let $(X,\frak T,\Sigma,\mu)$ and
$(Y,\frak S,\Tau,\nu)$ be two quasi-Radon measure spaces.   Then the
$\tau$-additive product measure $\tilde\lambda$ on $X\times Y$ is a
quasi-Radon
measure, the unique quasi-Radon measure on $X\times Y$ such that
$\tilde\lambda(E\times F)=\mu E\cdot\nu F$ for every $E\in\Sigma$ and
$F\in\Tau$.

\proof{ $\tilde\lambda$ is a complete, locally
determined, effectively locally finite, $\tau$-additive topological
measure, inner regular with respect to the closed sets (417C(vii)).
But this says just that it
is a quasi-Radon measure.   By 417D, it is the unique quasi-Radon
measure with the right values on measurable rectangles.
}%end of proof of 417N

\leader{417O}{Theorem} Let
$\langle (X_i,\frak T_i,\Sigma_i,\mu_i)\rangle_{i\in I}$ be a family of
quasi-Radon
probability spaces.   Then the $\tau$-additive product measure
$\tilde\lambda$ on $X=\prod_{i\in I}X_i$ is a quasi-Radon measure, the
unique quasi-Radon measure on $X$ extending the ordinary product
measure.

\proof{ By 417E(vi), $\tilde\lambda$ is inner regular with respect to
the closed sets, so is a quasi-Radon measure, which is unique by 417F.
}%end of proof of 417O

\leader{417P}{Theorem} Let $(X,\frak T,\Sigma,\mu)$ and
$(Y,\frak S,\Tau,\nu)$ be Radon measure spaces.   Then the
$\tau$-additive product
measure $\tilde\lambda$ on $X\times Y$ is a Radon measure, the unique
Radon measure on $X\times Y$ such that
$\tilde\lambda(E\times F)=\mu E\cdot\nu F$ whenever $E\in\Sigma$ and
$F\in\Tau$.

\proof{ Of course $X\times Y$ is Hausdorff, and $\tilde\lambda$ is
locally finite (because $\tilde\lambda(G\times H)=\mu G\cdot\nu H$ is
finite whenever $\mu G$ and $\nu H$ are finite).   By 417C(viii),
$\tilde\lambda$ is tight, so is a Radon
measure.   As in 417N, it is uniquely defined by its values on
measurable rectangles.
}%end of proof of 417P

\leader{417Q}{Theorem} Let
$\langle(X_i,\frak T_i,\Sigma_i,\mu_i)\rangle_{i\in I}$ be a family of
Radon probability
spaces, and $\tilde\lambda$ the $\tau$-additive product measure on
$X=\prod_{i\in I}X_i$.   For each $i\in I$, let
$Z_i\subseteq X_i$ be the support of $\mu_i$.   Suppose that
$J=\{i:i\in I,\,Z_i$ is not compact$\}$ is countable.   Then
$\tilde\lambda$ is a
Radon measure, the unique Radon measure on $X$ extending the ordinary
product measure.

\proof{ Of course $X$, being a product of Hausdorff spaces, is
Hausdorff, and $\tilde\lambda$, being totally finite, is locally finite.
Now, given $\epsilon\in\ocint{0,1}$, let
$\langle\epsilon_j\rangle_{j\in J}$ be a
family of strictly positive numbers such that
$\sum_{j\in J}\epsilon_j\le\epsilon$,
and for $j\in J$ choose a compact set $K_j\subseteq X_j$ such that
$\mu_jK_j\ge 1-\epsilon_j$;  for $i\in I\setminus J$, set $K_i=Z_i$, so
that $K_i$ is compact and $\mu_iK_i=1$.   Consider
$K=\prod_{i\in I}K_i$.
Then, using 417E(iii) and 254Lb for the two equalities,

\Centerline{$\tilde\lambda K=\lambda^*K
=\prod_{i\in I}\mu_iK_i\ge\prod_{j\in J}1-\epsilon_j
\ge 1-\epsilon$,}

\noindent where $\lambda$ is the ordinary product measure on $X$.   As
$\epsilon$ is arbitrary, $\tilde\lambda$ satisfies the condition (iv) of
416C, and is a Radon measure.   As in 417F, it is the unique Radon
measure on $X$ extending $\lambda$.
}%end of proof of 417Q

\leader{417R}{Notation} I will use the phrase {\bf quasi-Radon product
measure} for a $\tau$-additive product measure which is in fact a
quasi-Radon measure;   similarly, a {\bf Radon product measure}
is a $\tau$-additive product measure which is a Radon measure.

\leader{417S}{}\cmmnt{ Later I will give an example in which a
$\tau$-additive product measure is different from the corresponding
c.l.d.\ product measure (419E).   In 415E-415F, 415Ye and 416U I
have described cases in which c.l.d.\ measures are $\tau$-additive
product measures.   It remains very unclear when to expect this to
happen.   I can however give a couple of results which show that
sometimes, at least, we can be sure that the two measures coincide.

\medskip

\noindent}{\bf Proposition} (a) Let $(X,\frak T,\Sigma,\mu)$ and
$(Y,\frak S,\Tau,\nu)$ be effectively locally finite $\tau$-additive
topological measure spaces such that both $\mu$ and $\nu$ are inner
regular with respect
to the Borel sets, and $\lambda$ the c.l.d.\ product measure on
$X\times Y$. If every open subset of $X\times Y$ is measured by
$\lambda$, then $\lambda$ is the $\tau$-additive product measure on
$X\times Y$.

(b) Let $\langle(X_i,\frak T_i,\Sigma_i,\mu_i)\rangle_{i\in I}$
be a family of $\tau$-additive topological probability spaces such that
every $\mu_i$ is inner regular with respect to the Borel sets, and
$\lambda$ the ordinary product measure on $X=\prod_{i\in I}X_i$.   If
every open subset of $X$ is measured by $\lambda$, then $\lambda$
is the $\tau$-additive product measure on $X$.

(c) In (b), let $\lambda_J$ be the ordinary product measure on
$X_J=\prod_{i\in J}X_i$ for each $J\subseteq I$, and $\tilde\lambda_J$
the $\tau$-additive product measure.   If $\lambda_J=\tilde\lambda_J$
for every finite $J\subseteq I$, and every $\mu_i$ is strictly positive,
then $\lambda=\tilde\lambda_I$ is the $\tau$-additive product measure on
$X$.

\proof{{\bf (a), (b)} In both cases, $\lambda$ is a complete locally
determined effectively locally finite $\tau$-additive measure which is
inner regular with respect to the Borel sets (assembling facts from
251I, 254F, 412S, 412U, 417C and 417E).   The extra hypothesis added
here is that $\lambda$ is a topological measure, so itself satisfies the
conditions of 417D or 417F, and is the $\tau$-additive product measure.

\medskip

{\bf (c)(i)} The first step is to note that $\lambda_J=\tilde\lambda_J$
for every countable $J\subseteq I$.   \Prf\ Express $J$ as
$\bigcup_{n\in\Bbb
N}J_n$ where $\sequencen{J_n}$ is a non-decreasing sequence of finite
sets.   If $F\subseteq X_J$ is closed, then it is
$\bigcap_{n\in\Bbb N}\pi_n^{-1}[\overline{\pi_n[F]}]$, where
$\pi_n:X_J\to X_{J_n}$ is the canonical map for each $n$.   But every
$\overline{\pi_n[F]}$ is a closed subset of $X_{J_n}$, therefore
measured by $\lambda_{J_n}$;  because $\pi_n$ is \imp\ (417K),
$\pi_n^{-1}[\overline{\pi_n[F]}]\in\dom\lambda_J$ for each $n$, and
$F\in\dom\lambda_J$.   Thus every closed set, therefore every open set
is measured by $\lambda_J$, and $\lambda_J$ is a topological measure;
by (b), $\lambda_J=\tilde\lambda_J$.\ \Qed

\medskip

\quad{\bf (ii)} Suppose that $W\subseteq X$ is open.   By 417M, there
are $W'$, $W''$ measured by $\tilde\lambda$ such that
$W'\subseteq W\subseteq W''$, both $W''$ and $W'$ are determined by
coordinates in a countable set, and
$\tilde\lambda_I(W''\setminus W')=0$.   Let
$J\subseteq I$ be a countable set such that $W'$ and $W''$ are
determined by coordinates in
$J$.   Then $\lambda_J=\tilde\lambda_J$ measures $\pi_J[W']$, by 417K,
so $\lambda$ measures $W'=\pi_J^{-1}[\pi_J[W]]$, by 254Oa.   Similarly,
$\lambda$ measures $W''$.   Now
$\lambda(W''\setminus W')=\tilde\lambda_I(W''\setminus W')=0$, so
$\lambda$ measures $W$.   As $W$ is arbitrary, $\lambda$ is a
topological measure and must be the $\tau$-additive product measure, by
(a).
}%end of proof of 417S

\leader{417T}{Proposition} Let $(X,\frak T,\Sigma,\mu)$ and
$(Y,\frak S,\Tau,\nu)$ be effectively locally finite $\tau$-additive
topological
measure spaces such that both $\mu$ and $\nu$ are inner regular with
respect to the Borel sets, and $\lambda$ the c.l.d.\ product measure on
$X\times Y$.   If $X$ has a conegligible subset with a countable
network\cmmnt{ (e.g., if $X$ is metrizable and $\mu$ is $\sigma$-finite)}, then
$\lambda$ is the $\tau$-additive product measure on $X\times Y$.

\proof{{\bf (a)} Suppose to begin with that $\mu$ and $\nu$ are totally
finite, and that $X$ itself
has a countable network;  let $\sequencen{A_n}$ run
over a network for $X$.   Let $\hat\mu$ be the completion of $\mu$ and
$\hat\Sigma$ its domain.   Let $\tilde\lambda$ be the $\tau$-additive
product measure on $X\times Y$.   (We are going to need Fubini's theorem
both for $\lambda$ and for $\tilde\lambda$.   I will use a sprinkling of
references to \S\S251-252 to indicate which parts of the
argument below depend on the properties of $\lambda$.)

Let $W\subseteq X\times Y$ be an open set.   For each $n\in\Bbb N$, set

\Centerline{$H_n=\bigcup\{H:H\in\frak S,\,A_n\times H\subseteq W\}$,}

\noindent so that $H_n$ is open.   Then
$W=\bigcup_{n\in\Bbb N}A_n\times H_n$.   \Prf\ Of course
$A_n\times H_n\subseteq W$ for every $n\in\Bbb N$.   If $(x,y)\in W$,
there are open sets $G\subseteq X$, $H\subseteq Y$ such that
$(x,y)\in G\times H\subseteq W$;  now there is an $n\in\Bbb N$ such that
$x\in A_n\subseteq G$, so that $H\subseteq H_n$ and
$(x,y)\in A_n\times H_n$.\ \Qed

By 417C(iv), there is an open set $W_0$ in the domain $\Lambda$ of
$\lambda$ such that
$W_0\subseteq W$ and $\tilde\lambda(W\setminus W_0)=0$.   By 417Ha,
applied to $\chi(W\setminus W_0)$,
$A=\{x:\nu(W[\{x\}]\setminus W_0[\{x\}])>0\}$ is $\mu$-negligible.   For
each $n\in\Bbb N$, $x\in X$ set $f_n(x)=\nu(H_n\cap W_0[\{x\}])$;  then
252B tells us that $\int f_nd\mu$ is defined and equal to
$\lambda(W_0\cap(X\times H_n))$.   In particular, $f_n$ is
$\hat\Sigma$-measurable.   Set $E_n=\{x:f_n(x)=\nu H_n\}\in\hat\Sigma$.
If $x\in A_n$, then $H_n\subseteq W[\{x\}]$, so
$A_n\setminus E_n\subseteq A$.

Now, by 252B again,

$$\eqalign{\lambda((E_n\times H_n)\setminus W_0)
&=\int_{E_n}\nu(H_n\setminus W_0[\{x\}])\mu(dx)\cr
&=\int_{E_n}\nu H_n-\nu(H_n\cap W_0[\{x\}])\mu(dx)
=0.\cr}$$

\noindent So if we set $W_1=\bigcup_{n\in\Bbb N}E_n\times H_n$,
$W_1\setminus W\subseteq W_1\setminus W_0$ is $\lambda$-negligible.
On the other hand,

\Centerline{$W\setminus W_1
\subseteq\bigcup_{n\in\Bbb N}(A_n\setminus E_n)\times H_n
\subseteq A\times Y$}

\noindent is also $\lambda$-negligible.   Because $\lambda$ is complete,
$W\in\Lambda$.   As $W$ is arbitrary, $\lambda$ is a topological measure
and is equal to $\tilde\lambda$, by 417Sa.

\medskip

{\bf (b)} Now consider the general case.   Let $Z$ be a conegligible
subset of $X$ with a countable network;  since any subset of a space
with a countable network again has a countable network (4A2Na), we may
suppose that $Z\in\Sigma$.   Again let $W$ be an open set in
$X\times Y$.   This time, take arbitrary $E\in\Sigma$, $F\in\Tau$ of
finite measure, and consider the subspace measures $\mu_{E\cap Z}$ and
$\nu_F$.   These are still effectively locally finite and
$\tau$-additive (414K), and are now totally finite.   Also $E\cap Z$
has a countable network.   So (a) tells us that the relatively open set
$W\cap((E\cap Z)\times F)$ is measured by the c.l.d.\ product of
$\mu_{E\cap Z}$ and $\nu_F$, which is the subspace measure on
$(E\cap Z)\times F$ induced by $\lambda$ (251Q).   Since $\lambda$
surely measures $E\times F$, it measures
$W\cap(Z\times Y)\cap(E\times F)$.   As $E$ and $F$ are arbitrary,
$\lambda$ measures $W\cap(Z\times Y)$ (251H).   But
$\lambda((X\setminus Z)\times Y)=\mu(X\times Z)\cdot\nu Y=0$ (251Ia), so
$\lambda$ also measures $W$.   As $W$ is arbitrary, $\lambda$ is the
$\tau$-additive product measure.
}%end of proof of 417T

\leader{417U}{Proposition} Let
$\familyiI{(X_i,\frak T_i,\Sigma_i,\mu_i)}$
be a family of $\tau$-additive topological probability spaces.   Let
$\lambda$ be the ordinary product probability measure on
$X=\prod_{i\in I}X_i$ and $\Lambda$ its domain.
Then every continuous function $f:X\to\Bbb R$ is
$\Lambda$-measurable, so $\Lambda$ includes the Baire $\sigma$-algebra
of $X$.

\proof{{\bf (a)} Let
$\tilde\lambda$ be a $\tau$-additive topological measure extending
$\lambda$ (417E), and $\tilde\Lambda$ its domain;  then $f$ is
$\tilde\Lambda$-measurable, just because $\tilde\lambda$ is a
topological measure.   For $\alpha\in\Bbb R$, set

\Centerline{$G_{\alpha}=\{x:x\in X,\,f(x)<\alpha\}$,
\quad$H_{\alpha}=\{x:x\in X,\,f(x)>\alpha\}$,}

\Centerline{$F_{\alpha}=\{x:x\in X,\,f(x)=\alpha\}$.}

\noindent Then $\family{\alpha}{\Bbb R}{F_{\alpha}}$ is disjoint, so
$A=\{\alpha:\alpha\in\Bbb R,\,\tilde\lambda F_{\alpha}>0\}$ is
countable, and $A'=\Bbb R\setminus A$ is dense in $\Bbb R$;  let
$Q\subseteq A'$ be a countable dense set.

For each $q\in Q$, let $V_q\subseteq G_q$, $W_q\subseteq H_q$ be such
that

\Centerline{$\lambda V_q=\lambda_*G_q=\tilde\lambda G_q$,
\quad$\lambda W_q=\lambda_*H_q=\tilde\lambda H_q$}

\noindent (413Ea, 417E(iii)).   Then

\Centerline{$\lambda^*(G_q\setminus V_q)
\le\lambda(X\setminus(V_q\cup W_q))
=1-\lambda V_q-\lambda W_q
=\tilde\lambda(X\setminus(G_q\cup H_q))
=0$.}

\noindent Because $\lambda$ is complete, $G_q\setminus V_q$ and $G_q$
belong to $\Lambda$.   But now, if $\alpha\in\Bbb R$,

\Centerline{$\{x:f(x)<\alpha\}=\bigcup_{q\in Q,q<\alpha}G_q\in\Lambda$,}

\noindent so $f$ is $\Lambda$-measurable.

\medskip

\quad{\bf (b)} It follows that every zero set belongs to $\Lambda$, so
that $\Lambda$ must include the Baire $\sigma$-algebra of $X$.
}%end of proof of 417U

\leader{417V}{Proposition} Let $(X,\frak T,\Sigma,\mu)$ and
$(Y,\frak S,\Tau,\nu)$ be effectively locally finite $\tau$-additive
topological
measure spaces, and $(X\times Y,\Lambda,\lambda)$ their c.l.d.\ product.
Then every continuous function $f:X\times Y\to\Bbb R$ is
$\Lambda$-measurable, and the Baire
$\sigma$-algebra of $X\times Y$ is included in $\Lambda$.

\proof{ Let $Z\subseteq X\times Y$ be a zero set.   If $E\in\Sigma$,
$F\in\Tau$ are sets of finite measure, then $Z\cap(E\times F)$ is a zero
set for the relative topology of $E\times F$.   Now the subspace
measures $\mu_E$ and $\nu_F$ are $\tau$-additive topological measures
(414K), so $Z\cap(E\times F)$ is measured by the c.l.d.\ product
$\mu_E\times\nu_F$ of $\mu_E$ and $\nu_F$.   \Prf\ If either $\mu E$ or
$\nu F$ is zero, this is trivial.   Otherwise, they have scalar
multiples $\mu'_E$, $\nuprime_F$ which are probability measures, and of
course are still $\tau$-additive topological measures.   By 417U,
$Z\cap(E\times F)$ is measured by
$\mu'_E\times\nuprime_F$.   Since $\mu_E\times\nu_F$ is just a
scalar multiple of $\mu'_E\times\nuprime_F$, $Z\cap(E\times F)$ is
measured
by $\mu_E\times\nu_F$.\ \QeD\   But $\mu_E\times\nu_F$ is the
subspace measure $\lambda_{E\times F}$ (251Q), so
$Z\cap(E\times F)\in\Lambda$.
As $E$ and $F$ are arbitrary, $Z\in\Lambda$ (251H).

Thus every zero set belongs to $\Lambda$;  accordingly $\Lambda$ must
include the Baire $\sigma$-algebra, and every continuous function must
be $\Lambda$-measurable.
}%end of proof of 417V

\exercises{
\leader{417X}{Basic exercises (a)}
%\spheader 417Xa
Let $(X,\Sigma,\mu)$ be a semi-finite measure space and
$\Cal A$ a family of subsets of $X$.   Show that the following are
equiveridical:  (i) there is a measure $\mu'$ on $X$, extending $\mu$,
such that $\mu'A=0$ for every $A\in\Cal A$;
(ii) $\mu_*(\bigcup_{n\in\Bbb N}A_n)=0$ for every sequence
$\sequencen{A_n}$ in $\Cal A$.
%417A

\spheader 417Xb Let $(X,\Sigma,\mu)$ and $(Y,\Tau,\nu)$ be measure
spaces with topologies with respect to which $\mu$ and $\nu$ are locally
finite.   Show that the c.l.d.\ product measure on $X\times Y$ is
locally finite.
%417- %

\sqheader 417Xc Let $(X,\frak T,\Sigma,\mu)$ and $(Y,\frak S,\Tau,\nu)$
be topological measure spaces such that $\mu$ and $\nu$ are both
effectively locally finite $\tau$-additive Borel measures.   Show that
there is a unique effectively locally finite $\tau$-additive Borel
measure $\lambda'$ on $X\times Y$ such that
$\lambda'(G\times H)=\mu G\cdot\nu H$ for all open sets $G\subseteq X$,
$H\subseteq Y$.
%417D

\sqheader 417Xd Let $\familyiI{(X_i,\frak T_i,\Sigma_i,\mu_i)}$ be a
family of topological probability spaces in which every $\mu_i$ is a
$\tau$-additive Borel measure.   Show that there is a unique
$\tau$-additive Borel measure $\lambda'$ on $X=\prod_{i\in I}X_i$ such
that $\lambda'(\prod_{i\in I}F_i)=\prod_{i\in I}\mu_iF_i$ whenever
$F_i\subseteq X_i$ is closed for every $i\in I$.
%417F

\spheader 417Xe Let $(X,\frak T,\Sigma,\mu)$ and $(Y,\frak S,\Tau,\nu)$
be effectively locally finite $\tau$-additive topological measure spaces
in which the measures are inner regular with respect to the Borel sets,
and $\tilde\lambda$ the $\tau$-additive product measure on $X\times Y$.
Let $\familyiI{X_i}$, $\family{j}{J}{Y_j}$ be decompositions for $\mu$,
$\nu$ respectively (definition: 211E).   Show that
$\langle X_i\times Y_j\rangle_{i\in I,j\in J}$ is a decomposition for
$\tilde\lambda$.   (Cf.\ 251O.)
%417G

\sqheader 417Xf Let
$\langle(X_i,\frak T_i,\Sigma_i,\mu_i)\rangle_{i\in I}$ be a family of
$\tau$-additive topological probability spaces such
that $\mu_i$ is inner regular with respect to the Borel sets for every
$i$, and $\tilde\lambda$ the
$\tau$-additive product measure on $X=\prod_{i\in I}X_i$.   Take
$A_i\subseteq X_i$ for each $i\in I$.   (i) Show that if $\mu_i^*A_i=1$
for every $i$, then the subspace measure induced by $\tilde\lambda$ on
$A=\prod_{i\in I}A_i$ is
just the $\tau$-additive product $\tilde\lambda^{\#}$ of the subspace
measures on the $A_i$.   \Hint{show that if we set
$\lambda'W=\tilde\lambda^{\#}(W\cap A)$ for Borel sets $W\subseteq X$,
then $\lambda'$ satisfies the conditions of 417Xd.}  (ii) Show that in
any case $\tilde\lambda^*A=\prod_{i\in I}\mu^*_iA_i$.   (Cf.\ 254L.)
%417G

\spheader 417Xg Let $\familyiI{(X_i,\frak T_i,\Sigma_i,\mu_i)}$
and $\familyiI{(Y_i,\frak S_i,\Tau_i,\nu_i)}$ be two families of
$\tau$-additive topological probability spaces in which every $\mu_i$
and every $\nu_i$ is inner regular with respect to the Borel sets.   Let
$\tilde\lambda$, $\tilde\lambda'$ be the $\tau$-additive product
measures on $X=\prod_{i\in I}X_i$ and $Y=\prod_{i\in I}Y_i$
respectively.   Suppose that for each $i\in I$ we are given a continuous
\imp\ function $\phi_i:X_i\to Y_i$.   Show that the function
$\phi:X\to Y$ defined by setting $\phi(x)(i)=\phi_i(x(i))$ for $x\in X$,
$i\in I$ is \imp.
%417G   almost cts won't do

\spheader 417Xh Let $(X,\frak T,\Sigma,\mu)$ and
$(Y,\frak S,\Tau,\nu)$ be two complete locally determined effectively
locally
finite $\tau$-additive topological measure spaces such that both $\mu$
and $\nu$ are inner regular with respect to the Borel sets.   Let
$\tilde\lambda$ be
the $\tau$-additive product measure on $X\times Y$, and $\tilde\Lambda$
its domain.   Suppose that $\nu$ is $\sigma$-finite.   Show that for any
$W\in\tilde\Lambda$, $W[\{x\}]\in\Tau$ for almost every $x\in X$,
and $x\mapsto\nu W[\{x\}]$ is measurable.
%417H

\sqheader 417Xi Let $(X,\frak T,\Sigma,\mu)$ be $[0,1]$ with its usual
topology and Lebesgue measure, and let $(Y,\frak S,\Tau,\nu)$ be $[0,1]$
with its discrete topology and counting measure.   (i) Show that both
are Radon measure spaces.   (ii) Show that the c.l.d.\ product measure
on $X\times Y$ is a Radon measure.   \Hint{252Kc, or use 417T and 417P.}
(iii) Show that 417Ha can fail if we omit the hypothesis on
$\{(x,y):f(x,y)\ne 0\}$.
%417H

\spheader 417Xj Let $(X,\frak T,\Sigma,\mu)$ and $(Y,\frak S,\Tau,\nu)$
be two effectively locally finite $\tau$-additive topological measure
spaces.   Let $\lambda$ be the c.l.d.\ product measure and
$\tilde\lambda$ the $\tau$-additive product measure on $X\times
Y$.   Show that $\lambda^*(A\times B)=\tilde\lambda^*(A\times B)$ for
all sets $A\subseteq X$, $B\subseteq Y$.   \Hint{start with $A$, $B$ of
finite outer measure, so that 417I applies.}
%417I  omissible

\spheader 417Xk Let $\langle(X_i,\frak T_i,\Sigma_i,\mu_i)\rangle_{i\in
I}$ be a family of $\tau$-additive topological probability spaces with
strictly positive measures, all inner regular with respect to the Borel
sets, and $(X,\frak T,\tilde\Lambda,\tilde\lambda)$ their
$\tau$-additive product.   For $J\subseteq I$ let $\tilde\lambda_J$ be
the $\tau$-additive product measure on $X_J=\prod_{i\in J}X_i$, and
$\tilde\Lambda_J$ its domain.   (i) Show that if $f$ is a real-valued
$\tilde\Lambda$-measurable function defined $\tilde\lambda$-almost
everywhere on $X$, we can find a countable set $J\subseteq I$ and a
$\tilde\Lambda_J$-measurable function $g$, defined
$\tilde\lambda_J$-almost
everywhere on $X_J$, such that $f$ extends $g\pi_J$.   (ii) In (i), show
that $\int fd\tilde\lambda=\int g\,d\tilde\lambda_J$ if either is
defined in $[-\infty,\infty]$.
%417M

\spheader 417Xl Let
$\langle(X_i,\frak T_i,\Sigma_i,\mu_i)\rangle_{i\in I}$ be a family of
$\tau$-additive topological probability spaces such
that every $\mu_i$ is inner regular with respect to the Borel sets, and
$(X,\frak T,\tilde\Lambda,\tilde\lambda)$ their $\tau$-additive product.
Show that for any $W\in\tilde\Lambda$ there is a smallest set
$J\subseteq I$ for which there is a $W'\in\tilde\Lambda$, determined by
coordinates in $J$, with $\tilde\lambda(W\symmdiff W')=0$.  \Hint{254R.}
%417M

\spheader 417Xm What needs to be added to 417M and 415Xk
% A q-R measure on a product of separable metr spaces is
% completion regular iff closed self-supp sets are determined by
% coordinates in countable sets
to complete a proof of 415E?
%417M

\spheader 417Xn Let $(X,\frak T,\Sigma,\mu)$ be an atomless
$\tau$-additive topological probability space such that $\mu$ is inner
regular with respect to the Borel sets, and $I$ a set of cardinal at
most that of the support of $\mu$.   Show that the set of injective
functions from $I$ to $X$ has full outer measure for the $\tau$-additive
product measure on $X^I$.
%417M

\sqheader 417Xo Let $(X,\frak T,\Sigma,\mu)$ and $(Y,\frak S,\Tau,\nu)$
be Radon measure spaces.   Show that the Radon product measure on
$X\times Y$ is the unique Radon measure $\tilde\lambda$ such that
$\tilde\lambda(K\times L)=\mu K\cdot\nu L$ for all compact sets
$K\subseteq X$, $L\subseteq Y$.
%417P

\sqheader 417Xp Let $I$ be an uncountable set, and $\lambda$,
$\tilde\lambda$ be the ordinary and $\tau$-additive product measures on
$X=\{0,1\}^I$ when each factor is given its usual topology and the Dirac
measure concentrated at $1$.   Show that $\tilde\lambda$ properly extends
$\lambda$, and that the support of
$\tilde\lambda$ is not determined by any countable
set of coordinates.   Find a $\tilde\lambda$-negligible open set
$W\subseteq X$ such that its
projection onto $\{0,1\}^J$ is conegligible for every proper subset
$J$ of $I$.
%417Q

\spheader 417Xq Let $\langle(X_i,\frak T_i,\Sigma_i,\mu_i)\rangle_{i\in
I}$ be a family of Radon probability spaces, and $\tilde\lambda$ the
quasi-Radon product measure on $X=\prod_{i\in I}X_i$.   For each $i\in
I$, let $Z_i\subseteq X_i$ be the support of $\mu_i$.   Show that
$\tilde\lambda$ is a Radon measure iff
$\{i:i\in I,\,Z_i$ is not compact$\}$ is countable.
In particular, show that the ordinary product measure on
$\coint{0,1}^I$, where $I$ is uncountable and
each copy of $\coint{0,1}$ is given Lebesgue measure, is a quasi-Radon
measure, but not a Radon measure.
%417Q

\spheader 417Xr Let $\sequencen{(X_n,\frak T_n,\Sigma_n,\mu_n)}$ be a
sequence of Radon probability spaces.   Show that the Radon product
measure on $X=\prod_{n\in\Bbb N}X_n$ is the unique Radon measure
$\tilde\lambda$ on $X$ such that
$\tilde\lambda(\prod_{n\in\Bbb N}K_n)=\prod_{n=0}^{\infty}\mu_nK_n$
whenever $K_n\subseteq X_n$ is compact for every $n$.
%417Q

\spheader 417Xs Let $\familyiI{(X_i,\frak T_i,\Sigma_i,\mu_i)}$ be a
family of $\tau$-additive topological probability spaces such that every
$\mu_i$ is inner regular with respect to the Borel sets, and $\lambda$,
$\tilde\lambda$ the ordinary and $\tau$-additive product measures on
$X=\prod_{i\in I}X_i$.   Show that if $A\subseteq X$ has
$\tilde\lambda$-negligible boundary, then $A$ is measured by $\lambda$.
%417U

\spheader 417Xt Let us say that a topological space $X$ is
{\bf chargeable} if there is an additive functional
$\nu:\Cal PX\to\coint{0,\infty}$ such that $\nu G>0$ for every non-empty
open set $G\subseteq X$.   (i) Show that if there is a
$\sigma$-finite measure $\mu$ on $X$ such that $\mu_*G>0$ for every
non-empty open set $G$, then $X$ is chargeable.   \Hint{215B(vii),
391G.}  (ii) Show that any separable space is chargeable.   (iii) Show
that $X$ is chargeable iff its regular open algebra is
chargeable in the sense of 391Bb.
\Hint{see the proof of 314P.}   (iv)
Show that any open subspace of a chargeable space is chargeable.   (v)
Show that if $Y\subseteq X$ is dense, then $X$ is chargeable iff $Y$ is
chargeable.   (vi) Show that if $X$ is expressible as the union of
countably many chargeable subspaces, then it is chargeable.   (vii)
Show that any product of chargeable spaces is chargeable.  (Cf.\
391Xb(iii).)
(viii) Show that if $\familyiI{X_i}$ is a family of chargeable spaces
with product $X$, then all regular open subsets of $X$ and all Baire
subsets of $X$ are determined by coordinates in countable sets.
\Hint{4A2Eb, 4A3Mb.}
(ix) Show that a continuous image of a chargeable space is chargeable.
(x) Show that a compact Hausdorff space is chargeable
iff it carries a strictly positive Radon measure.   \Hint{416K.}
%417U

\spheader 417Xu Let $(X,\frak T,\Sigma,\mu)$ and $(Y,\frak S,\Tau,\nu)$
be quasi-Radon measure spaces such that $\mu X\cdot\nu Y>0$.   Show that
the quasi-Radon product measure on $X\times Y$ is completion regular iff
it is equal to the c.l.d.\ product measure and $\mu$ and $\nu$ are both
completion regular.   \Hint{412Sc;  if $\mu E$, $\nu F$ are finite
and $Z\subseteq E\times F$ is a zero set of positive measure, use
Fubini's theorem to show that $Z$ has sections of positive measure.}
%417V

\spheader 417Xv Let $\familyiI{(X_i,\frak T_i,\Sigma_i,\mu_i)}$ be a
family of quasi-Radon probability spaces.   Show that the
quasi-Radon product measure on $\prod_{i\in I}X_i$ is completion regular
iff it is equal to the ordinary product measure and every $\mu_i$ is
completion regular.
%417Xu, %417V

\spheader 417Xw Let $\familyiI{(X_i,\frak T_i,\Sigma_i,\mu_i)}$ be a
family of $\tau$-additive topological probability spaces such that every
$\mu_i$ is inner regular with respect to the Borel sets,
and $\lambda$ the $\tau$-additive product measure on
$X=\prod_{i\in I}X_i$;  write $\Lambda$ for its domain.
(i) Show that if $W\in\Lambda$, $\lambda W>0$ and
$\epsilon>0$ then there are a finite $J\subseteq I$ and a $W'\in\Lambda$
such that $\lambda W'\ge 1-\epsilon$ and for every $x\in W'$ there is a
$y\in W$ such that $x\restr I\setminus J=y\restr I\setminus J$.   (Cf.\
254Sb.)
(ii) Show that if $A\subseteq X$ is determined by coordinates in
$I\setminus\{i\}$ for every $i\in I$ then $\lambda^*A\in\{0,1\}$.
(Cf.\ 254Sa.)
%+

\leader{417Y}{Further exercises (a)}
%\spheader 417Ya
(i) Show that if, in 417A, $\mu$ is strictly localizable, then it has a
strictly localizable extension $\mu'$ with the properties (i)-(iv) there.
(ii) Give an example to show that the construction offered in 417A may not
immediately achieve this result.
%417A %mt41bits

\spheader 417Yb Let $(X,\frak T,\Sigma,\mu)$ and $(Y,\frak S,\Tau,\nu)$
be effectively locally finite $\tau$-additive topological
measure spaces such that $\mu$ and
$\nu$ are both inner regular with respect to the Borel sets.

\quad(i) Fix open sets $G\subseteq X$, $H\subseteq Y$ of finite measure.
Let $\Cal W_{GH}$ be the set of those $W\subseteq X\times Y$ such that
$\theta_{GH}(W)=\int_G\hat\nu(W[\{x\}]\cap H)dx$ is defined, where
$\hat\nu$ is the completion of $\nu$.   ($\alpha$) Show
that every open set belongs to $\Cal W_{GH}$.   ($\beta$) Show that
$\theta_{GH}$ is countably additive in the sense that
$\theta_{GH}(\bigcup_{n\in\Bbb N}W_n)
=\sum_{n=0}^{\infty}\theta_{GH}(W_n)$
for every disjoint sequence $\sequencen{W_n}$ in $\Cal W_{GH}$, and
$\tau$-additive in the sense that
$\theta_{GH}(\bigcup\Cal V)=\sup_{V\in\Cal V}\theta_{GH}(V)$ for every
non-empty upwards-directed family $\Cal V$
of open sets in $X\times Y$.   ($\gamma$) Show that every Borel set
belongs to
$\Cal W_{GH}$.   \Hint{Monotone Class Theorem.}   ($\delta$) Writing
$\Cal B$ for the Borel $\sigma$-algebra of $X\times Y$, show that
$\theta_{GH}\restr\Cal B$ is a
$\tau$-additive Borel measure;  let $\lambda_{GH}$ be its completion.
($\epsilon$) Show that $\lambda_{GH}=\theta_{GH}\restr\Lambda_{GH}$,
where $\Lambda_{GH}=\dom\lambda_{GH}$.   ($\zeta$) Show that
$\lambda_{GH}(E\times F)$ is defined and equal to $\mu E\cdot\nu F$
whenever $E\in\Sigma$, $F\in\Tau$,
$E\subseteq G$ and $F\subseteq H$.   \Hint{start with open $E$ and $F$,
move to Borel $E$ and $F$ with the Monotone Class Theorem.}   ($\eta$)
Writing
$\lambda$ for the c.l.d.\ product measure on $X\times Y$, show that
$\lambda_{GH}(W)$ is defined and equal to $\lambda(W\cap(G\times H))$
whenever $W\in\dom\lambda$.

\quad(ii) Now take $\tilde\Lambda$ to be
$\bigcap\{\Lambda_{GH}:G\in\frak T,\,
H\in\frak S,\,\mu G<\infty,\,\nu H<\infty\}$ and
$\tilde\lambda W=\sup_{G,H}\lambda_{GH}(W)$ for $W\in\tilde\Lambda$.
Show that $\tilde\lambda$ is an extension of $\lambda$ to a complete
locally determined effectively locally finite $\tau$-additive
topological measure on
$X\times Y$ which is inner regular with respect to the Borel sets, so is
the $\tau$-additive product measure as defined in 417G.
%417H

\spheader 417Yc Let $(X,\Sigma,\mu)$ and $(Y,\Tau,\nu)$ be complete
measure spaces with topologies $\frak T$, $\frak S$.   Suppose that
$\mu$ and $\nu$
are effectively locally finite and $\tau$-additive and moreover that
their domains include bases for the two topologies.   Show that the
c.l.d.\ product measure on $X\times Y$ has the same properties.
\Hint{start by
assuming that $\mu X$ and $\nu Y$ are both finite.   If $\Cal V$ is an
upwards-directed family of measurable open sets with measurable open
union $W$, look at $g_V(x)=\nu V[\{x\}]$ for $V\in\Cal V$.}
%417Yb, 417H

\spheader 417Yd Let $\langle(X_i,\frak T_i,\Sigma_i,\mu_i)\rangle_{i\in
I}$ be a family of $\tau$-additive topological probability spaces such
that every $\mu_i$ is inner regular with respect to the Borel sets, and
$(X,\frak T,\tilde\Lambda,\tilde\lambda)$ their $\tau$-additive product.
(i) Show that the following are equiveridical:  ($\alpha$) $\mu_i$ is
strictly positive
for all but countably many $i\in I$;  ($\beta$) whenever
$W\in\tilde\Lambda$
there are a countable $J\subseteq I$ and $W_1$, $W_2\in\tilde\Lambda$,
determined by coordinates in $J$, such that $W_1\subseteq W\subseteq
W_2$ and $\tilde\lambda(W_2\setminus W_1)=0$.   (ii) Show that when
these are false, $\tilde\lambda$ cannot be equal to the ordinary product
measure on $X$.
%417M

\spheader 417Ye Let $(X,\Sigma,\mu)$ and $(Y,\Tau,\nu)$ be measure
spaces with Hausdorff topologies $\frak T$, $\frak S$ such that both $\mu$ and
$\nu$ are inner regular with respect to the families of sequentially
compact sets in
each space.   Show that the c.l.d.\ product measure $\lambda$ on
$X\times Y$ is also inner regular with respect to the sequentially
compact sets, so has an extension to a topological measure which is
inner regular with respect to the sequentially compact sets.
\Hint{412R, 416Yc.}
%417P

\spheader 417Yf Let $\langle(X_i,\Sigma_i,\mu_i)\rangle_{i\in I}$ be a
family of probability spaces with topologies $\frak T_i$ such that every
$\mu_i$ is inner regular with respect to the family of closed countably
compact sets in $X_i$ and every $X_i$ is compact.   Show that the
ordinary product measure $\lambda$ on $X=\prod_{i\in I}X_i$ is also
inner regular with respect to the closed countably compact sets, so has
an extension to a topological measure $\tilde\lambda$ which is inner
regular with respect to the closed countably compact sets in $X$.   Show
that this can be done in such a way that for every
$W\in\dom\tilde\lambda$ there is a $V\in\dom\lambda$ such that
$\tilde\lambda(W\symmdiff V)=0$.
\Hint{412T, 416Yb.}
%417Q

\spheader 417Yg Let $\langle(X_n,\Sigma_n,\mu_n)\rangle_{n\in\Bbb N}$ be
a sequence of probability spaces with
Hausdorff topologies $\frak T_n$ such that
every $\mu_n$ is inner regular with respect to the family of
sequentially compact
sets in $X_n$.   Show that the ordinary product measure $\lambda$ on
$X=\prod_{n\in\Bbb N}X_n$ is also inner regular with respect to the
sequentially compact sets, so has an extension to a topological measure
$\tilde\lambda$ which is inner regular with respect to the sequentially
compact sets in $X$.   Show that this can be done in such a way that for
every $W\in\dom\tilde\lambda$ there is a $V\in\dom\lambda$ such that
$\tilde\lambda(W\symmdiff V)=0$.
%417Yf, 417Q

\spheader 417Yh Let $\familyiI{(X_i,\frak T_i,\Sigma_i,\mu_i)}$ be a
family of quasi-Radon probability spaces, and $\lambda$,
$\tilde\lambda$ the ordinary and quasi-Radon product measures on
$X=\prod_{i\in I}X_i$.   Suppose that all but {\it one} of the
$\frak T_i$ have countable networks and all but {\it countably} many of
the $\mu_i$ are strictly positive.   Show that $\lambda=\tilde\lambda$.
%417T

\spheader 417Yi Let us say that a quasi-Radon measure space
$(X,\frak T,\Sigma,\mu)$ has the {\bf simple product property} if the
c.l.d.\ product measure on $X\times Y$ is equal to the quasi-Radon
product measure for every quasi-Radon measure space
$(Y,\frak S,\Tau,\nu)$.
(i) Show that if $(X,\frak T)$ has a countable network then
$(X,\frak T,\Sigma,\mu)$ has the simple product property.   (ii) Show
that if a quasi-Radon measure space has the simple product property so
do all its subspaces.  (iii) Show that the quasi-Radon product of two
quasi-Radon measure spaces with the simple product property has the
simple product property.   (iv) Show that the quasi-Radon product of any
family of quasi-Radon probability spaces with the simple product
property has the simple product property.   (v) Show that the real line
with the right-facing Sorgenfrey topology (415Xc) and Lebesgue measure
has the simple product property.
%417T
}%end of exercises

\endnotes{\Notesheader{417} The general problem of determining just when
a measure can be extended to a
measure with given properties is one which will recur throughout this
volume.   I have more than once mentioned the Banach-Ulam problem;  if
you like, this is the question of whether there can ever be an extension
of the countable-cocountable measure on a set $X$ to a measure defined
on the whole algebra $\Cal PX$.   This particular question appears to be
undecidable from the ordinary axioms of set theory;  but for many sets
(for instance, if $X=\omega_1$) it is known that the answer is `no'.
(See 419G and 438C.)   This being so, we have to take each
manifestation of the general question on its own merits.   In 417C and
417E the challenge is to take a product measure $\lambda$ defined in
terms of the factor measures alone, disregarding their topological
properties, and extend it
to a topological measure, preferably $\tau$-additive.   Of course there
are important cases in which $\lambda$ is itself already a topological
measure;   for instance, we know that the c.l.d.\ product of Lebesgue
measure on $\Bbb R$ with itself is Lebesgue measure on $\BbbR^2$
(251N), and other examples are in 415E, 415Ye, 416U, 417S-417T,
417Yh and 453I.   But in general
not every open set in the product belongs to the domain of $\lambda$,
even when we have the product of two Radon measures on compact Hausdorff
spaces (419E).

Once we have resolved to grasp the nettle, however, there is a natural
strategy for the proof.   It is easy to see that if $\lambda$, in 417C
or 417E, is to have an extension to a $\tau$-additive topological
measure $\tilde\lambda$, then we must have $\tilde\lambda A(\Cal V)=0$
for every $\Cal V$ belonging to the class $\frak V$.   Now 417A descibes
a sufficient (and obviously necessary) condition for there to be an
extension of $\lambda$ with this property.   So all we have to do is
check.   The check is not perfectly straightforward;  in 417E it uses
all the resources of the original proof that there is a product measure
on an arbitrary product of probability spaces (which I suppose is to be
expected), with 414B (of course) to apply
the hypothesis that the factor measures are $\tau$-additive, and a
couple of extra wrinkles (the $W'_n$ and $C'_n$ of part (c-ii) of the
proof of 417E, and the use of supports in part (c-vii)).

It is worth noting that (both for finite and for infinite products) the
measure algebras of $\lambda$ and $\tilde\lambda$ are identical
(417C(ii), 417E(ii)), so
there is no new work to do in identifying the measure algebra of
$\tilde\lambda$ and the associated function spaces.

An obstacle we face in 417C-417E is the fact that {\it not} every
$\tau$-additive measure $\mu$ has an extension to a $\tau$-additive
topological measure, even when $\mu$ is totally finite and its domain
includes a base for the topology.   (I give an example in 419J.)
Consequently it is not enough, in 417C or 417E, to show that the
ordinary product measure $\lambda$ is $\tau$-additive.   But perhaps I
should remark that if $\lambda$ is inner regular with respect to the
closed sets, this obstacle evaporates (415L).   Accordingly, for the
principal applications (to quasi-Radon and Radon product measures, and
in particular whenever the topological spaces involved are regular) we
have rather easier proofs available, based on the constructions of
\S415.   For completely regular spaces, there is yet another approach,
because the product measures can be described in terms of the integrals
of continuous functions (415I), which by 417U and 417V can be
calculated from the ordinary product measures.   Of course
the proof that $\lambda$ itself is $\tau$-additive is by no means
trivial, especially in the case of infinite products, corresponding to
417E;  but for finite products there are relatively direct arguments,
applying indeed to slightly more general situations (417Yc).   If we
have measures which are inner regular with respect to countably compact
classes of sets, then there may be other ways of approaching the
extension, using theorems from \S413
(see 417Ye-417Yg), %417Ye 417Yf 417Yg
and for compact Radon measure spaces, $\lambda$ becomes tight (412Sb,
412V), so its $\tau$-additivity is elementary.

As always, it is important to recognise which constructions are in some
sense canonical.   The arguments of 417C and 417E allow for the
possibility that the factor measures are defined on $\sigma$-algebras
going well beyond the Borel sets.   For all the principal applications,
however, the measures
will be c.l.d.\ versions of Borel measures, and in particular will be
inner regular with respect to the Borel sets.   In such a context it is
natural to ask for product measures with the same property, and in this
case we can identify a canonical $\tau$-additive topological product
measure, as in 417D and 417F.   (If you prefer to restrict your measures
to Borel $\sigma$-algebras, you again get canonical product Borel
measures (417Xc-417Xd).)   Having done so, we can reasonably expect
`commutative' and `associative' and
`distributive' laws, as in 417Db, 417J and 417Xe.   Subspaces mostly
behave themselves (417I, 417Xf).

Of course extending the product measure means that we get new integrable
functions on the product, so that Fubini's theorem has to be
renegotiated.   Happily, it remains valid, at least in the contexts in
which it was effective before (417Ha);   we still need, in effect, one
of the measures to be $\sigma$-finite.   The theorem still fails for
arbitrary integrable functions on products of
Radon measure spaces, and the same example works as before (417Xi).
In fact this means that we have an
alternative route to the construction of the $\tau$-additive product of
two measures (417Yb).   But note that on this route `commutativity', the
identification of the product measure on $X\times Y$ with that on
$Y\times X$, becomes something which can no longer be taken for granted,
because if we {\it define} $\tilde\lambda W$ to be $\int\nu W[\{x\}]dx$
we have to worry about when, and why, this will be equal to
$\int\mu W^{-1}[\{y\}]dy$.

A version of Tonelli's theorem follows from Fubini's theorem, as before
(417Hc).   We also have results corresponding to most of the theorems
of \S254.   But note that there are two traps.   In the theorem that a
measurable set can be described in terms of a projection onto a
countable subproduct (254O, 417M) we need to suppose that the factor
measures are strictly positive, and in the theorem that a product of
Radon measures is a Radon measure (417Q) we need to suppose that the
factor measures have compact supports.   The basic examples to note in
this context are 417Xp and 417Xq.

It is not well understood when we can expect c.l.d.\ product measures to
be topological measures, even in the case of compact Radon probability
spaces.   Example 419E remains a rather special case, but of course much
more effort has gone into seeking positive results.   Note that the
ordinary product measures of this section are always
effectively locally finite and $\tau$-additive (417C, 417E), so that
they will be equal to the $\tau$-additive products iff they measure
every open set (417S).   Regarding infinite products, the
$\tau$-additive product
measure can fail to be the ordinary product measure in just two ways:
if one of the {\it finite} product measures is not a topological
measure, or if uncountably many of the factor measures are not strictly
positive (417Sc, 417Xp, 417Yd).   So it is finite products which need
to be studied.

Whenever we have a subset $F$ of an infinite product
$X=\prod_{i\in I}X_i$, it is important to know when $F$ is determined by
coordinates in a proper subset of $I$;  in measure theory, we are
particularly interested in sets determined by coordinates in countable
subsets of $I$ (254Mb).   It may happen that there is a {\it smallest}
set $J$ such that $F$ is determined by coordinates in $J$;  for
instance, when we have a topological product and $F$ is closed (4A2Bg).
When we have a product of probability spaces, we sometimes wish to
identify sets $J$ such that $F$ is `essentially' determined by
coordinates in $J$, in the sense that there is an $F'$, determined
by coordinates in $J$, such that $F\symmdiff F'$ is negligible.   In
this context, again, there is a smallest such set (254Rd), which can be
identified in terms of the probability algebra free product of the
measure algebras (325Mb).   In 417Ma the two ideas come together:  under
the conditions there, we get the same smallest $J$ by either route.

In 417Ma, we have a product of strictly positive $\tau$-additive
topological probability measures.   If we keep the `strictly positive'
but abandon everything else, we still have very striking results just
because the product topology is ccc, so that we can apply 4A2Eb.   An
abstract expression of this idea is in 417Xt.
}%end of notes

\discrpage

