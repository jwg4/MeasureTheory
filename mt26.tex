\frfilename{mt26.tex}
\versiondate{5.9.03/24.7.04}
\copyrightdate{2004}

\def\diam{\mathop{\text{diam}}}
\def\dist{\mathop{\text{dist}}}

\def\chaptername{Change of Variable in the Integral}
\newchapter{26}
\def\chaptername{Change of variable in the integral}

I suppose most courses on basic calculus still devote a
substantial amount of time to practice in the techniques of integrating
standard functions.   Surely the most powerful single technique is that
of substitution:  replacing $\int g(y)dy$ by $\int g(\phi(x))\phi'(x)dx$
for an appropriate function $\phi$.   At this level one usually
concentrates on the skills of guessing at appropriate $\phi$ and getting
the formulae right.   I will not address such questions here, except for
rare special cases;  in this book I am concerned rather with validating
the process.   For functions of one variable, it can usually be
justified by an appeal to the Fundamental Theorem of Calculus, 
and for any particular case I would normally
go first to \S225 in the hope that the results there would
cover it.   But for functions of two or
more variables some much deeper ideas are necessary.

I have already treated the general problem of
integration-by-substitution in abstract measure spaces in \S235.
There I described conditions under which $\int g(y)dy=\int
g(\phi(x))J(x)dx$ for an appropriate function $J$.   The context there
gave very little scope for suggestions as to how to compute $J$;  at
best, it could be presented as a Radon-Nikod\'ym derivative (235M).   In
this chapter I give a form of the fundamental theorem for the case of
Lebesgue measure, in which $\phi$ is a more or less differentiable
function between Euclidean spaces, and $J$ is a `Jacobian', the
modulus of the determinant of the derivative of $\phi$ (263D).   This
necessarily depends on a serious investigation of the relationship
between Lebesgue measure and geometry.
The first step is to establish a form of Vitali's theorem for
$r$-dimensional space, together with $r$-dimensional density theorems;
I do this in \S261, following closely the scheme of \S\S221 and 223
above.   We need to know quite a lot about differentiable functions
between Euclidean spaces, and it turns out that the theory is
intertwined with that of `Lipschitz' functions;  I treat these in
\S262.

In the next two sections of the chapter, I turn to a separate problem
for which some of the same techniques turn out to be appropriate:  the
description of surface measure on (smooth) surfaces in Euclidean space,
like the surface of a cone or sphere.   I suppose there is no difficulty
in forming a robust intuition as to what is meant by the `area' of
such a surface and of suitably simple regions within it, and there is a
very strong presumption that there ought to be an expression for this
intuition in terms of measure theory as presented in this book;  but the
details are not I think straightforward.   The first point to note is
that for any calculation of the area of a region $G$ in a surface $S$,
one would always turn at once to a parametrization of the region, that
is, a bijection $\phi:D\to G$ from some subset $D$ of Euclidean space.
But obviously one needs to be sure that the result of the calculation is
independent of the parametrization chosen, and while it would be
possible to base the theory on results showing such independence
directly, that does not seem to me to be a true reflection of the
underlying intuition, which is that the area of simple surfaces, at
least, is something intrinsic to their geometry.   I therefore see no
acceptable alternative to a theory of `$r$-dimensional measure' which
can be described in purely geometric terms.   This is the burden of
\S264, in which I give the definition and most fundamental properties
of Hausdorff $r$-dimensional measure in Euclidean spaces.   With this
established, we find that the techniques of \S\S261-263 are sufficient
to relate it to calculations through parametrizations, which is what I
do in \S265.

The chapter ends with a brief account of the Brunn-Minkowski inequality 
(266C), which is an essential tool for the geometric measure theory of 
convex sets.

\discrpage

