\frfilename{mt394.tex}
\versiondate{13.6.11}
\copyrightdate{2007}

\def\chaptername{Measurable algebras}
\def\sectionname{Talagrand's example}

\def\sbcc{$\sigma$-bounded-cc}
\def\sfcc{$\sigma$-finite-cc}

\newsection{394\dvAnew{2007}}

I rewrite the construction in {\smc Talagrand 08} of an
exhaustive submeasure which is not uniformly exhaustive.

\leader{394A}{}\cmmnt{ I begin with two elementary combinatorial facts.

\medskip

\noindent}{\bf Lemma}  %\Tal{\S2.3}
Suppose that $\Cal K$ is a non-empty finite family of subsets
of $\Bbb N$ and $r\in\Bbb N$ is such that $\#(K)\ge r\#(\Cal K)$ for every
$K\in\Cal K$.   Then we have an enumeration $\ofamily{i}{s}{K_i}$ of 
$\Cal K$ and a
non-decreasing family $\langle n_i\rangle_{i\le s}$ such that
$\#(K_i\cap n_{i+1}\setminus n_i)=r$ for every $i<s$.
 %T(9.1)2.3

\proof{ Set $s=\#(\Cal K)$.
Choose $n_i$, $K_i$ inductively, as follows.   Start
with $n_0=0$.   Given $j<s$,
$n_j\in\Bbb N$ and $\ofamily{i}{j}{K_i}$ such that
$\#(K\setminus n_j)\ge r(s-j)$ for every
$K\in \Cal K_j=\Cal K\setminus\{K_i:i<j\}$,
set

\Centerline{$n_{j+1}=\min\{n:\#(K\cap n\setminus n_j)\ge r$ for some
$K\in \Cal K_j\}$}

\noindent and choose $K_j\in \Cal K_j$ such that
$\#(K_j\cap n_{j+1}\setminus n_j)\ge r$.   Observe that
$\#(K\cap n_{j+1}\setminus n_j)\le r$ for every $K\in \Cal K_j$, so that
$\#(K_j\cap n_{j+1}\setminus n_j)$ must in fact be equal to $r$ and
$\#(K\setminus n_{j+1})\ge r(s-j-1)$ for every $K\in \Cal K_j$;  
thus the induction proceeds.
}%end of proof of 394A

\leader{394B}{Lemma}  %\Tal{3.17}
Suppose that $\ofamily{i}{s}{K_i}$ is a family of finite subsets of 
$\Bbb N$, all of size at least $r\ge 2$,
such that $\max K_i<\min K_{i+1}$ for $i\le s-2$.   Let
$\Cal J$ be a finite subset of
$([\Bbb N]^{<\omega}\setminus\{\emptyset\})\times\coint{0,\infty}$,
and set $\gamma=\sum_{(I,w)\in\Cal J}w$.   Then we can find 
$\ofamily{i}{s}{u_i}$ and $\ofamily{i}{s}{v_i}$ such that
$u_i$, $v_i$ are successive members of $K_i$ for each $i<s$ and, setting
$W=\bigcup_{i<s}v_i\setminus u_i$,

$$\sum_{\Atop{(J,w)\in\Cal J}{\#(J\cap W)\ge\bover12\#(J)}}w
\,\le\,\bover{2\gamma}{r-1}.$$

\proof{ Set $K=\bigcup_{i<s}K_i$ and for
$n\in K\setminus\{\max K\}$ let $n^+$ be the next member of $K$ above $n$.
For $z\in Z=\prod_{i<s}(K_i\setminus\{\max K_i\})$ set

\Centerline{$W(z)
=\bigcup_{i<s}z(i)^+\setminus z(i)$,}

\Centerline{$S(z)
=\sum_{(J,w)\in\Cal J,\#(J\cap W(z))\ge\bover12\#(J)}w$.}

\noindent Consider the uniform probability measure $\lambda$ on
$Z$.   For any set $J\subseteq\Bbb N$,

$$\eqalignno{\int\#(J\cap W(z))\lambda(dz)
&=\sum_{i=0}^{s-1}\int\#(J\cap z(i)^+\setminus z(i))\lambda(dz)\cr
&=\sum_{i=0}^{s-1}\sum_{n\in K_i\setminus\{\max K_i\}}
  \Bover1{\#(K_i)-1}\#(J\cap n^+\setminus n)\cr
&=\sum_{i=0}^{s-1}\Bover{\#(J\cap\max K_i\setminus\min K_i)}{\#(K_i)-1}
\le\Bover{\#(J)}{r-1}.\cr}$$

\noindent So, writing
$V_J=\{z:\#(J\cap W(z))\ge\bover12\#(J)\}$, $\lambda V_J\le\Bover2{r-1}$
for any non-empty $J$.   Now

\Centerline{$\int S(z)\lambda(dz)
=\sum_{(J,w)\in\Cal J}w\lambda V_J
\le\Bover2{r-1}\sum_{(J,w)\in\Cal J}w
=\Bover{2\gamma}{r-1}$,}

\noindent so there must be a $z\in Z$ such that
$S(z)\le\Bover{2\gamma}{r-1}$,
and we can take $u_i=z(i)$, $v_i=z(i)^+$ for each $i$.
}%end of proof of 394B

\leader{394C}{Definitions} \cmmnt{We are now ready to begin work.
The construction is complex and demands a large volume of special notation.

\medskip

}{\bf (a)}  %\Tal{2.1}
I shall work throughout with $X=\prod_{n\in\Bbb N}T_n$ where
$\sequencen{T_n}$ is a sequence of non-empty finite sets
and $\sup_{n\in\Bbb N}\#(T_n)$ is infinite.
 %394Ib
$X$ may be
regarded as a compact Hausdorff space with the product of the discrete
topologies on the $T_n$.   For each $n\in\Bbb N$,
$\frak B_n$ will be the algebra of subsets of $X$ determined by
coordinates less than $n$  %T(9.1)4/27
and $\Cal A_n$ the set of its atoms\cmmnt{, that is, 
the family of sets of the
form $\{x:z\subseteq x\in X\}$ for some $z\in\prod_{i<n}T_i$}.
 %T(9.1)4/31
$\frak B\cmmnt{\mskip5mu=\bigcup_{n\in\Bbb N}\frak B_n}$ will be the 
algebra of open-and-closed subsets of $X$.  %T(9.1)4/28
For $n\in\Bbb N$ and $t\in T_n$, $Y_{nt}$ will be
$\{x:x\in X$, $x(n)=t\}$.

\spheader 394Cb We shall need
a sequence $\sequence{k}{\alpha_k}$ in $\Bbb R$
and a sequence $\sequence{k}{N_k}$ in $\Bbb N$.   \cmmnt{It is easy 
enough to give
appropriate formulae but perhaps the ideas will be clearer if instead} I
declare the properties they must have.

\medskip

\quad{\bf (i)}  %\Tal{3.3}
$\alpha_k>0$ and
$(2^{k+4})^{\alpha_k}\le 2$
 %394Fc
for every $k\in\Bbb N$, $\sequence{k}{\alpha_k}$ is non-increasing,
 %394E
and $\sum_{k=0}^{\infty}\alpha_k\le\bover12$.
 %394Dg
 %\alpha_k\le 1 394K
 %4^{\alpha_k}\le 2 394K

\medskip

\quad{\bf (ii)}  %\Tal{4.2, 5.2}
$N_k\in\Bbb N$ and
 %394M
 %394E 
$2^{-k}(2^{-2k-12}N_k)^{\alpha_k}\ge 2^4$
 %394Gb
for every $k\in\Bbb N$.

\spheader 394Cc %\Tal{3.1}
\cmmnt{Now we come to some of the key ideas.}   For a set
$\Cal I\subseteq\Cal PX\times\Cal P\Bbb N\times\coint{0,\infty}$,
define\cmmnt{ its `spread'} $\spread\Cal I$ to be 
$\bigcup_{(E,I,w)\in\Cal I}E$
and\cmmnt{ its `weight'} $\weight\Cal I$ to be $\sum_{(E,I,w)\in\Cal I}w$.

\spheader 394Cd %\Tal{3.2}
For any family
$\Cal E\subseteq\Cal PX\times\Cal P\Bbb N\times\coint{0,\infty}$ define
$\phi_{\Cal E}:\frak B\to[0,\infty]$ by setting

\Centerline{$\phi_{\Cal E}E
=\inf\{\weight\Cal I:\Cal I\subseteq\Cal E$ is finite,
$E\subseteq\spread\Cal I\}$,}

\noindent counting $\inf\emptyset$ as $\infty$.   \cmmnt{So
$\phi_{\emptyset}\emptyset=0$ and $\phi_{\emptyset}E=\infty$ for
$E\in\frak B\setminus\{\emptyset\}$.}

\spheader 394Ce %\Tal{2.8}
For $D\subseteq X$ and $I\subseteq\Bbb N$ set

\Centerline{$\theta_I(D)
=\{y:y\in X,\,y\restr I=x\restr I\text{ for some }x\in D\}$.}

\spheader 394Cf{\bf (i)} %\Tal{\S3.4}
If $m<n$ in $\Bbb N$, $\phi:\frak B\to[0,\infty]$ is
a function and $E\in\frak B$, then $E$ is
{\bf $\phi$-thin between $m$ and $n$} if
$\phi(X\setminus\theta_{n\setminus m}(A\cap E))\ge 1$ for
every $A\in\Cal A_m$.
 %T(9.1)11/10

\medskip

\quad{\bf (ii)} If $I\subseteq\Bbb N$,
$\phi:\frak B\to[0,\infty]$ is a function and $E\in\frak B$, then
$E$ is {\bf $\phi$-thin along $I$} if it is $\phi$-thin between $m$ and $n$
whenever $m$, $n\in I$ and $m<n$.
 %T(9.1)11/17

\spheader 394Cg %\Tal{\S4.0}
For $k\le p\in\Bbb N$ define $\Cal C_{kp}$
and $\nu_{kp}$ by downwards induction on $k$\cmmnt{, as follows}.   
\cmmnt{Start with }$\Cal C_{pp}=\emptyset$ for every $p$.
\cmmnt{Given $\Cal C_{kp}$, set }$\nu_{kp}=\phi_{\Cal C_{kp}}$.
Given that $k<p$ and
$\Cal C_{k+1,p}$
and $\nu_{k+1,p}=\phi_{\Cal C_{k+1,p}}$ have been defined, set

$$\eqalign{\Cal E_{kp}
&=\{(E,I,w):E\in\frak B,\,I\subseteq\Bbb N,\,
1\le\#(I)\le N_k,\cr
&\mskip150mu
w\ge 2^{-k}\bigl(\Bover{N_k}{\#(I)}\bigr)^{\alpha_k},\,
E\text{ is }\nu_{k+1,p}\text{-thin along }I\},\cr
\Cal C_{kp}&=\Cal E_{kp}\cup\Cal C_{k+1,p}\cr}$$

\noindent and continue.

\spheader 394Ch %\Tal{\S5.0}
Define $\sequence{k}{c_k}$ by setting
$c_0=8$, $c_{k+1}=2^{2\alpha_k}c_k$ for every $k$.


\leader{394D}{Very elementary facts} \cmmnt{In the hope of aiding
digestion of
the definitions here, of which 394Cf and 394Cg are likely to be wholly
obscure to anyone who has not worked through this proof before, I run
over some obvious facts which will be used below.

\medskip

}{\bf (a)}  %\Tal{3.2}
$\phi_{\Cal E}:\frak B\to[0,\infty]$ is a submeasure
for any $\Cal E\subseteq\Cal PX\times\Cal P\Bbb N\times\coint{0,\infty}$.
\cmmnt{(Subadditivity and monotonicity are written into the definition.)}

\spheader 394Db
If $I$, $J\subseteq\Bbb N$ then $\theta_I\theta_J=\theta_{I\cap J}$.
 %394E
If $I\subseteq J\subseteq\Bbb N$ then 
$\theta_I(D)=\theta_I\theta_J(D)\supseteq\theta_J(D)$ for all 
$D\subseteq X$.
If $I\subseteq\Bbb N$ then
$\theta_I(D\cap\theta_I(E))=\theta_I(E\cap\theta_I(D))$ for all $D$,
$E\subseteq X$.
 %394E
For any $I\subseteq\Bbb N$ and any family $\Cal D$ of subsets of $X$,
$\theta_I(\bigcup\Cal D)=\bigcup_{D\in\Cal D}\theta_I(D)$.
 %394Fc

For $n\in\Bbb N$ and $D\subseteq X$, $D\in\frak B_n$ iff $\theta_n(D)=D$.
If $E\in\frak B$ and $I\subseteq\Bbb N$, $\theta_I(E)\in\frak B$.
 %394L
If $m\le n$ in $\Bbb N$, $A\in\Cal A_m$ and $A_1\in\Cal A_n$, then
$A\cap\theta_{n\setminus m}(A_1)\in\Cal A_n$.
 %394Gd
If $m\in\Bbb N$ and $A\in\Cal A_m$ then
$E\mapsto\theta_{\Bbb N\setminus m}(A\cap E):\frak B\to\frak B$ is a
Boolean homomorphism.
 %394L

\spheader 394Dc If $m<n$, $\phi:\frak B\to[0,\infty]$ is a
non-decreasing function,
$E\in\frak B$ is $\phi$-thin between $m$ and $n$
and $E'\in\frak B$ is included in $E$,
then $E'$ is $\phi$-thin between $m$ and $n'$ for
every $n'\ge n$.
 %394Gf

\spheader 394Dd\cmmnt{ All the classes} $\Cal E_{kp}$,
$\Cal C_{kp}$ are closed under increases in the scalar variable and
decreases in the first variable, that is,

----- if $k<p$, $(E,I,w)\in\Cal E_{kp}$, $E'\in\frak B$, $E'\subseteq E$
and $w'\ge w$ then $(E',I,w')\in\Cal E_{kp}$,

----- if $k\le p$, $(E,I,w)\in\Cal C_{kp}$, $E'\in\frak B$, $E'\subseteq E$
and $w'\ge w$ then $(E',I,w')\in\Cal C_{kp}$.

\spheader 394De If $k\le p$\cmmnt{ in $\Bbb N$},
$\Cal C_{kp}=\bigcup_{k\le l<p}\Cal E_{lp}$.

\spheader 394Df If $k<p$\cmmnt{ in $\Bbb N$}, 
$\nu_{kp}\le\nu_{k+1,p}$\cmmnt{,
because $\Cal C_{kp}\supseteq\Cal C_{k+1,p}$}.

\spheader 394Dg $8\le c_k\le 16$ for every
$k\in\Bbb N$\prooflet{, because
$\sum_{k=0}^{\infty}2\alpha_k\le 1$}.

\spheader 394Dh If $k<p$\cmmnt{ in $\Bbb N$}, 
then\cmmnt{ $(X,\{0\},2^{-k}N_k^{\alpha_k})\in\Cal E_{kp}$ 
so} $\nu_{kp}X\le 2^{-k}N_k^{\alpha_k}$ and $\nu_{kp}$ is totally finite.

\leader{394E}{Lemma}  %\Tal{\S5.2}
Suppose that $k\le p$, $m<n$,
$A\in\Cal A_m$, $(E,I,w)\in\Cal C_{kp}$ and $I'=I\cap n\setminus m$
is non-empty.   If $E'=\theta_{n\setminus m}(E\cap A)$ and
$w'\ge\bigl(\Bover{\#(I)}{\#(I')}\bigr)^{\alpha_k}w$,
then $(E',I',w')\in\Cal C_{kp}$.
 %T(9.1)5.2

\proof{ There is an $l$ such that $k\le l<p$ and
$(E,I,w)\in\Cal E_{lp}$.   Now
$E'$ is $\nu_{l+1,p}$-thin along $I'$.   \Prf\
Suppose that $i$, $j\in I'$ and $i<j$, so that $m\le i<j<n$.
Take any $A_1\in\Cal A_i$, and set $A_2=A\cap\theta_{n\setminus m}(A_1)$,
so that $A_2$ also belongs to $\Cal A_i$.   Then, using the list in 394Db,

$$\eqalignno{\theta_{j\setminus i}(E'\cap A_1)
&=\theta_{j\setminus i}(A_1\cap\theta_{n\setminus m}(E\cap A))\cr
&=\theta_{j\setminus i}(\theta_{n\setminus m}
 (A_1\cap\theta_{n\setminus m}(E\cap A)))\cr
&=\theta_{j\setminus i}(\theta_{n\setminus m}
 (E\cap A\cap\theta_{n\setminus m}(A_1)))\cr
&=\theta_{j\setminus i}(\theta_{n\setminus m}(E\cap A_2))
=\theta_{j\setminus i}(E\cap A_2).\cr}$$

\noindent So

\Centerline{$\nu_{l+1,p}(X\setminus\theta_{j\setminus i}(E'\cap A_1))
=\nu_{l+1,p}(X\setminus\theta_{j\setminus i}(E\cap A_2))
\ge 1$}

\noindent because $E$ is $\nu_{l+1,p}$-thin between $i$ and $j$.
As $i$, $j$ and $A_1$ are arbitrary, $E'$ is
$\nu_{l+1,p}$-thin along $I'$.\ \Qed

Of course $\#(I')\le\#(I)\le N_l$.   Finally, because
$\alpha_l\le\alpha_k$, we have

\Centerline{$w'\ge\bigl(\Bover{\#(I)}{\#(I')}\bigr)^{\alpha_k}w
\ge\bigl(\Bover{\#(I)}{\#(I')}\bigr)^{\alpha_l}
   \cdot 2^{-l}\bigl(\Bover{N_l}{\#(I)})^{\alpha_l}
=2^{-l}\bigl(\Bover{N_l}{\#(I')}\bigr)^{\alpha_l}$,}

\noindent and $(E',I',w')\in\Cal E_{lp}\subseteq\Cal C_{kp}$.
}  %end of proof of 394E

\leader{394F}{Corollary} (a)
Suppose that $n\in\Bbb N$ and $k\le p$ and that
$\Cal I\subseteq\Cal C_{kp}$
is a finite set such that $\#(I\cap n)\ge\bover14\#(I)$
whenever $(E,I,w)\in\Cal I$.
Then $\nu_{kp}(\theta_n(\spread\Cal I))\le 2\weight\Cal I$.

(b) Suppose that $m\in\Bbb N$, $k\le p$ and $A\in\Cal A_m$.   Let
$\Cal I$ be a finite subset of $\Cal C_{kp}$ such that
$\#(I\setminus m)\ge\bover14\#(I)$ whenever $(E,I,w)\in\Cal I$.   Then
$\nu_{kp}(\theta_{\Bbb N\setminus m}(A\cap\spread\Cal I))
\le 2\weight\Cal I$.

(c) Suppose that $m<n$ in $\Bbb N$, $k\le p$ and $A\in\Cal A_m$.   Let
$\Cal I$ be a finite subset of $\Cal C_{kp}$ such that
$\#(I\cap n\setminus m)\ge 2^{-k-4}\#(I)$ whenever $(E,I,w)\in\Cal I$.
Then $\nu_{kp}(\theta_{n\setminus m}(A\cap\spread\Cal I))
\le 2\weight\Cal I$.

\proof{{\bf (a)} For each $(E,I,w)\in\Cal I$ set
$E'=\theta_n(E)\in\Cal B_n$, $I'=I\cap n$ and

\Centerline{$w'
=\bigl(\Bover{\#(I)}{\#(I')}\bigr)^{\alpha_k}w\le 4^{\alpha_k}w
\le 2w$.}

\noindent By 394E, with $m=0$ and $A=X$, $(E',I',w')\in\Cal C_{kp}$.
Set $\Cal J=\{(E',I',w'):(E,I,w)\in\Cal I\}$ and $B=\spread\Cal J$.
Then

\Centerline{$B=\bigcup_{(E,I,w)\in\Cal I}\theta_n(E)
=\theta_n(\spread\Cal I)$}

\noindent and

\Centerline{$\nu_{kp}B\le\weight\Cal J
=\sum_{(E,I,w)\in\Cal I}w'
\le 2\weight\Cal I$,}

\noindent as required.

\medskip

{\bf (b)} This time, take $n>m$ so large that $I\subseteq n$ whenever
$(E,I,w)\in\Cal I$.   For $(E,I,w)\in\Cal I$, set

\Centerline{$E'=\theta_{n\setminus m}(A\cap E)$,
\quad$I'=I\setminus m=I\cap n\setminus m$,
\quad
$w'=\bigl(\Bover{\#(I)}{\#(I')}\bigr)^{\alpha_k}w\le 2w$.}

\noindent Then 394E tells us that $(E',I',w')\in\Cal C_{kp}$.
Setting $\Cal J=\{(E',I',w'):(E,I,w)\in\Cal I\}$,

\Centerline{$\theta_{\Bbb N\setminus m}(A\cap\spread\Cal I)
=\bigcup_{(E,I,w)\in\Cal I}\theta_{\Bbb N\setminus m}(A\cap E)
\subseteq\bigcup_{(E,I,w)\in\Cal I}E'
=\spread\Cal J$,}

\noindent so

\Centerline{$\nu_{kp}(\theta_{N\setminus m}(A\cap\spread\Cal I))
\le\weight\Cal J
\le 2\weight\Cal I$.}

\medskip

{\bf (c)} For $(E,I,w)\in\Cal I$ set

\Centerline{$E'=\theta_{n\setminus m}(A\cap E)$,
\quad$I'=I\cap n\setminus m$,
\quad$w'=\bigl(\Bover{\#(I)}{\#(I')}\bigr)^{\alpha_k}w
\le(2^{k+4})^{\alpha_k}w\le 2w$.}

\noindent Then $(E',I',w')\in\Cal C_{kp}$.
Setting $\Cal J=\{(E',I',w'):(E,I,w)\in\Cal I\}$,

\Centerline{$\theta_{n\setminus m}(A\cap\spread\Cal I)
=\bigcup_{(E,I,w)\in\Cal I}\theta_{n\setminus m}(A\cap E)
=\bigcup_{(E,I,w)\in\Cal I}E'=\spread\Cal J$,}

\noindent so

\Centerline{$\nu_{kp}(\theta_{n\setminus m}(A\cap\spread\Cal I))
\le\weight\Cal J
\le 2\weight\Cal I$.}
}  %end of proof of 394F

\leader{394G}{}\cmmnt{ We are at the centre of the argument.

\medskip

\noindent}{\bf Lemma}  %\Tal{\S5.1}
Suppose that $r\in\Bbb N$ and $t\in T_r$.    Then
$\nu_{kp}Y_{rt}\ge c_k$ whenever $k\le p$ in $\Bbb N$.
 %T(9.1)5.1

\proof{ Induce on $p-k$.   

\medskip

{\bf (a)} If $k=p$ then

\Centerline{$\Cal C_{pp}=\emptyset$,
\quad$\nu_{pp}Y_{rt}=\infty$.}

\noindent For the downwards step to $k<p$,
given that $\nu_{k+1,p}Y_{rt}\ge c_{k+1}$, take a
finite set $\Cal I\subseteq\Cal C_{kp}$ such that $\weight\Cal I<c_k$.
The rest of the proof is devoted to showing that
$Y_{rt}\not\subseteq\spread\Cal I$.

\medskip

{\bf (b)} It will help to get a trivial case out of the way.   If
$\Cal I\subseteq\Cal C_{k+1,p}$, then we have

\Centerline{$\weight\Cal I<c_k\le c_{k+1}\le\nu_{k+1,p}Y_{rt}$,}

\noindent by the inductive hypothesis,
so certainly $Y_{rt}\not\subseteq\spread\Cal I$.   Accordingly we may
suppose henceforth that $\Cal I\not\subseteq\Cal C_{k+1,p}$.

A second elementary point is that $\#(I)\ge 2^{2k+12}$ whenever 
$(E,I,w)\in\Cal I$.   \Prf\ We have an $l$ such that $k\le l<p$ and
$(E,I,w)\in\Cal E_{kp}$, so

\Centerline{$2^{-l}\bigl(\Bover{N_l}{\#(I)}\bigr)^{\alpha_l}
\le w\le c_k\le 2^4$}

\noindent and 
$\#(I)\ge 2^{2l+12}\ge 2^{2k+12}$, by the choice of $N_l$.\ \Qed

\medskip

{\bf (c)} Express $\Cal I$ as
$\Cal J\cup\Cal K$ where $\Cal J\subseteq\Cal C_{k+1,p}$ and
$\Cal K\subseteq\Cal E_{kp}$.
Set $s=\#(\Cal K)>0$.   For $(E,I,w)\in\Cal K$ we have
$w\ge 2^{-k}$, so $s\le 2^kc_k\le 2^{k+4}$ (394Dg).
Consequently $\#(I)\ge 2^{2k+12}\ge 2^{k+8}s$ whenever $(E,I,w)\in\Cal K$.
By 394A, we can find $m_0<m_1<\ldots<m_s$
and an enumeration $\ofamily{i}{s}{(E_i,K_i,w_i)}$ of $\Cal K$ such
that $\#(K_i\cap m_{i+1}\setminus m_i)=2^{k+8}$ for $i<s$.

For each $i<s$, set $u'_i=\sup(K_i\cap(r+1))$ and
$K'_i=(K_i\setminus\{u'_i\})\cap m_{i+1}\setminus m_i$, so that
$\#(K'_i)\ge 2^{k+8}-1$.
By 394B we can find successive members $u_i$, $v_i$ of $K'_i$,
for $i<s$, such that, setting $W=\bigcup_{i<s}v_i\setminus u_i$,

$$S
=\sum_{\Atop{(E,I,w)\in\Cal J}{\#(I\cap W)\ge\bover12\#(I)}}w
\le\Bover{2}{2^{k+8}-2}\sum_{(E,I,w)\in\Cal J}w
\le\Bover{2}{2^{k+7}}c_k
\le 2^{-k-2}.$$

\noindent Also $r\notin W$.  \Prf\ If $i<s$ and $u_i\le r$, then 
$u_i\le u'_i\le r$;  but $u_i\ne u'_i$, so $u_i<u'_i$ and
$v_i\le u'_i\le r$.\ \Qed

\medskip

{\bf (d)} Set

\Centerline{$\Cal J_1=\{(E,I,w):(E,I,w)\in\Cal J$,
$\#(I\cap W)\ge\Bover12\#(I)\}$,
\quad$\Cal J_2=\Cal J\setminus\Cal J_1$;}

\noindent then

\Centerline{$\weight\Cal J_1=S\le 2^{-k-2}\le\Bover14$.}

\noindent For $i<s$ set

\Centerline{$\Cal J_{1i}=\{(E,I,w):(E,I,w)\in\Cal J_1$,
$\#(I\cap v_i\setminus u_i)\ge 2^{-k-5}\#(I)\}$.}

\noindent Since $s\le 2^{k+4}$, $\Cal J_1=\bigcup_{i<s}\Cal J_{1i}$.

\medskip

{\bf (e)} Suppose that $i<s$ and $A\in\Cal A_{u_i}$.
Then there is an $A'\in\Cal A_{v_i}$ such that
$A'\subseteq A\setminus(E_i\cup\spread\Cal J_{1i})$.   \Prf\   Set
$C=\theta_{v_i\setminus u_i}(A\cap\spread\Cal J_{1i})\in\frak B_{v_i}$.
By 394Fc, applied to $\Cal C_{k+1,p}$,

\Centerline{$\nu_{k+1,p}C\le 2\weight\Cal J_{1i}\le 2\weight\Cal J_1<1$.}

\noindent As $(E_i,K_i,w_i)\in\Cal E_{kp}$,
$E_i$ is $\nu_{k+1,p}$-thin between $u_i$ and $v_i$,
$\nu_{k+1,p}(X\setminus\theta_{v_i\setminus u_i}(A\cap E_i))\ge 1$ and
$C$ does not include $X\setminus\theta_{v_i\setminus u_i}(A\cap E_i)$.
Since these sets both belong to $\frak B_{v_i}$
there is an $A_1\in\Cal A_{v_i}$ disjoint from both $C$ and
$\theta_{v_i\setminus u_i}(A\cap E_i)$, that is, disjoint from
$\theta_{v_i\setminus u_i}(A\cap(E_i\cup\spread\Cal J_{1i}))$.
Now $A'=A\cap\theta_{v_i\setminus u_i}(A_1)$
belongs to $\Cal A_{v_i}$, is included in $A$
and is disjoint from $E_i\cup\spread\Cal J_{1i}$.\ \Qed

\medskip

{\bf (f)} We can therefore find a function $\Gamma:X\to X$ such that
$\Gamma[X]$ is disjoint from $\spread(\Cal K\cup\Cal J_1)$, while
$\Gamma(x)\restr m$ is determined by $x\restr m$ for every $m\in\Bbb N$.
\Prf\ By (e) just above, we have for each $i<s$
a function $q_i:\Cal A_{u_i}\to\Cal A_{v_i}$
such that $q_i(A)\subseteq A\setminus(E_i\cup\spread\Cal J_{1i})$ for every
$A\in\Cal A_{u_i}$.   We can re-interpret $q_i$ as a function
$h_i:\prod_{n<u_i}T_n\to\prod_{n<v_i}T_n$ defined by saying that if
$A=\{x:x\restr u_i=z\}$ then $q_i(A)=\{x:x\restr v_i=h_i(z)\}$;  note that
$z=h_i(z)\restr u_i$ for every $z\in\prod_{n<u_i}T_n$.
Now, for $x\in X$, define $\Gamma(x)(n)$ inductively by saying that

$$\eqalign{\Gamma(x)(n)
&=x(n)\text{ if }n\in\Bbb N\setminus W,\cr
&=h_i(\Gamma(x)\restr u_i)(n)\text{ if }i<s\text{ and }u_i\le n<v_i.\cr}$$

\noindent Of course this ensures that $\Gamma(x)\restr m$ is determined by
$x\restr m$ for every $m$.   If $i<s$, $x\in X$, and $A\in\Cal A_{u_i}$ is
such that
$\Gamma(x)\in A$, then $\Gamma(x)\in q_i(A)$, which is disjoint from
$E_i\cup\spread\Cal J_{1i}$.   Thus $\Gamma[X]$ is disjoint from
$\bigcup_{i<s}E_i\cup\spread\Cal J_{1i}=\spread(\Cal K\cup\Cal J_1)$.\ \Qed

\medskip

{\bf (g)} Take $(E,I,w)\in\Cal J_2$ and consider
$\nu_{k+1,p}(\Gamma^{-1}[E])$.   

\medskip

\quad{\bf (i)} There is an $l$ such that $k<l<p$
and $(E,I,w)\in\Cal E_{lp}$.
Now if $m$, $n\in I$ are such that $m<n$ and
$n\setminus m$ is disjoint from $W$, $\Gamma^{-1}[E]$ is
$\nu_{l+1,p}$-thin between $m$ and $n$.   \Prf\ Take any $A\in\Cal A_m$.
Because $\Gamma(x)\restr m$ is determined by $x\restr m$,
we can find an $A'\in\Cal A_m$ such that $\Gamma[A]\subseteq A'$.
In this case,

\Centerline{$A\cap\Gamma^{-1}[E]
\subseteq\Gamma^{-1}[\Gamma[A]\cap E]
\subseteq\Gamma^{-1}[A'\cap E]
\subseteq\theta_{n\setminus m}(A'\cap E)$}

\noindent because $\Gamma(x)(i)=x(i)$ whenever $x\in X$ and
$i\in n\setminus m$.   So
$\theta_{n\setminus m}(A\cap\Gamma^{-1}[E])
\subseteq\theta_{n\setminus m}(A'\cap E)$ and

\Centerline{$\nu_{l+1,p}
  (X\setminus\theta_{n\setminus m}(A\cap\Gamma^{-1}[E]))
\ge\nu_{l+1,p}(X\setminus\theta_{n\setminus m}(A'\cap E))
\ge 1$}

\noindent because $E$ is $\nu_{l+1,p}$-thin between $m$ and $n$.\ \Qed

\medskip

\quad{\bf (ii)} As noted in (b), $\#(I)\ge 2^{2k+12}\ge 4s$.
For each $i<s$ such that $\min I\le u_i$,
let $u^-_i$ be the largest element of
$I$ which is less than or equal to $u_i$.
Set $I'=I\setminus(W\cup\{u^-_i:i<s$, $\min I\le u_i\})$.   Then

\Centerline{$\#(I')\ge\Bover{\#(I)}2-s\ge\Bover{\#(I)}4$.}

\noindent Now $\Gamma^{-1}[E]$ is $\nu_{l+1,p}$-thin along $I'$.   \Prf\
Suppose
that $m$, $n\in I'$ and $m<n$.   Let $m^+$ be the least element of $I$ such
that $m<m^+$.   Then $m^+\le n$.   \Quer\ If
$W\cap m^+\setminus m\ne\emptyset$,
there is an $i<s$ such that $m^+\setminus m$ meets $v_i\setminus u_i$, that
is, $m<v_i$ and $u_i<m^+$.
Since $m\in I'\subseteq\Bbb N\setminus W$, $m\le u_i$ and $u^-_i$ is
defined;   now $m\ne u^-_i$ so $m<u^-_i\in I$ and $m^+\le u^-_i\le u_i$.\
\BanG\ Thus
$W\cap m^+\setminus m$ is empty and (i) tells us that $\Gamma^{-1}[E]$ is
$\nu_{l+1,p}$-thin between $m$ and $m^+$, therefore
$\nu_{l+1,p}$-thin between $m$ and $n$ (394Dc).   As $m$
and $n$ are arbitrary, $\Gamma^{-1}[E]$ is $\nu_{l+1,p}$-thin
along $I'$.\ \Qed

\medskip

\quad{\bf (iii)} If we now set $w'=4^{\alpha_l}w$, we see that

\Centerline{$1\le\#(I')\le\#(I)\le N_l$,
\quad$w'\ge 2^{-l}4^{\alpha_l}\bigl(\Bover{N_l}{\#(I)}\bigr)^{\alpha_l}
\ge 2^{-l}\bigl(\Bover{N_l}{\#(I')}\bigr)^{\alpha_l}$,}

\noindent so

\Centerline{$(\Gamma^{-1}[E],I',w')\in\Cal E_{lp}\subseteq\Cal C_{k+1,p}$}

\noindent and

\Centerline{$\nu_{k+1,p}(\Gamma^{-1}[E])\le w'
\le 4^{\alpha_l}w\le 4^{\alpha_k}w$.}

\medskip

{\bf (h)} We are nearly done.   Applying (g) to each member of $\Cal J_2$,

\Centerline{$\nu_{k+1,p}(\Gamma^{-1}[\spread\Cal J_2])
\le 4^{\alpha_k}\weight\Cal J_2
\le 4^{\alpha_k}\weight\Cal I
<4^{\alpha_k}c_k=c_{k+1}\le\nu_{k+1,p}Y_{rt}$}

\noindent by the inductive hypothesis in its full strength.
So there is a
$y\in Y_{rt}\setminus\Gamma^{-1}[\spread\Cal J_2]$.
With (f), this means that $\Gamma(y)$ does not belong to

\Centerline{$\spread(\Cal K\cup\Cal J_1)\cup\spread(\Cal J_2)
=\spread\Cal I$.}

\noindent On the other hand, $\Gamma(y)\in Y_{rt}$ because $r\notin W$.
As $\Cal I$ was arbitrary, $\nu_{kp}Y_{rt}$ must be at least
$c_k$, which is what we need to know to proceed with the induction.
}  %end of proof of 394G

\leader{394H}{Definitions}  %\Tal{4.1}
\cmmnt{ I present the last two
definitions required.}
Fix on a non-principal ultrafilter $\Cal F$ on $\Bbb N$.
For $k\in\Bbb N$, set

\Centerline{$\nu_kE=\lim_{p\to\Cal F}\nu_{kp}E\in[0,\infty]$}

\noindent for every $E\in\frak B$;  \cmmnt{finally, }write $\nu$ for
$\nu_0$.

\leader{394I}{Proposition} (a) %\Tal{\S6.2}
For every $k\in\Bbb N$,
$\nu_k$ is a totally finite submeasure and $\nu_kX\ge 8$.

(b) %\Tal{\S4.1}
$\nu$ is not uniformly exhaustive.

\proof{{\bf (a)} It follows directly from the definition in
392A that $\nu_k$, being a limit of
submeasures, is a submeasure.   By 394Dh,
$\nu_kX\le 2^{-k}N_k^{\alpha_k}$ is finite.   By 394G and 394Dg,

\Centerline{$\nu_kX=\lim_{p\to\Cal F}\nu_{kp}X\ge c_k\ge 8$.}

\medskip

{\bf (b)} For any $n\in\Bbb N$ and $t\in T_n$, 

\Centerline{$\nu Y_{nt}=\lim_{p\to\Cal F}\nu_{0p}Y_{nt}\ge 8$}

\noindent by 394G.    As $\sup_{n\in\Bbb N}\#(T_n)$ is infinite,
and $\family{t}{T_n}{Y_{nt}}$ is disjoint for every $n$, $\nu$ is not
uniformly exhaustive.
}  %end of proof of 394I

\leader{394J}{Lemma} Suppose that $k\in\Bbb N$, $E\in\frak B$,
$I\in[\Bbb N]^{<\omega}$ and
$E$ is $\bover12\nu_k$-thin along $I$.   Then

\Centerline{$\{p:p\ge k$, $E$ is $\nu_{kp}$-thin along $I\}\in\Cal F$.}

\noindent If $k\ge 1$ and $\#(I)=N_{k-1}$, then
$\nu_{k-1}E\le 2^{-k+1}$.

\proof{ If $m$, $n\in I$, $m<n$ and $A\in\Cal A_m$, then
$\nu_k(X\setminus\theta_{n\setminus m}(A\cap E))\ge 2$.   So

\Centerline{$U_{An}
=\{p:p\ge k$, $\nu_{kp}(X\setminus\theta_{n\setminus m}(A\cap E))\ge 1\}$}

\noindent belongs to
$\Cal F$.   Setting $U=\bigcap_{m<n\text{ in }I,A\in\Cal A_m}U_{An}$,
$U\in\Cal F$ and $E$ is $\nu_{kp}$-thin along $I$ for every $p\in U$.

If $k\ge 1$ and $\#(I)=N_{k-1}$, then 
$(E,I,2^{-k+1})\in\Cal E_{k-1,p}$ for every
$p\in U$, so $\nu_{k-1,p}E\le 2^{-k+1}$ for every $p\in U$ and
$\nu_{k-1}E\le 2^{-k+1}$.
}%end of proof of 394J

\vleader{72pt}{394K}{Lemma}  %\Tal{\S6.5}
Let $m$, $k\in\Bbb N$ and
let $\sequence{i}{E_i}$ be a sequence in $\frak B$ such that

\inset{every $E_i$ is determined by coordinates in $\Bbb N\setminus m$,

$\nu_k(\bigcup_{i\le n}E_i)<2$ for every $n\in\Bbb N$.}

\noindent Then for every $\eta>0$ there is a $C\in\frak B$, determined by
coordinates in $\Bbb N\setminus m$, such that $\nu_kC\le 4$ and
$\nu_k(E_i\setminus C)\le\eta$ for each $i$.
 %T(9.1)6.5

\proof{{\bf (a)} For each $n>m$, set

\Centerline{$\tilde E_n=\bigcup\{E_i:i\le n$, $E_i\in\frak B_n\}$,}

\noindent so that $\tilde E_n$ is determined by coordinates in
$n\setminus m$ and $\nu_k\tilde E_n<2$.   Set

\Centerline{$U_n=\{p:p\ge k$, $\nu_{kp}\tilde E_n<2\}
\in\Cal F$.}

\noindent For $p\in U_n$ we can find a finite
$\Cal I_{np}\subseteq\Cal C_{kp}$ such that
$\tilde E_n\subseteq\spread\Cal I_{np}$ and
$\weight\Cal I_{np}\le 2$.   For $r>m$ set

$$\eqalign{\Cal I_{npr}
&=\{(E,I,w):(E,I,w)\in\Cal I_{np},\cr
&\mskip140mu\#(I\cap(r-1)\setminus m)<\Bover12\#(I)
  \le\#(I\cap r\setminus m)\},\cr}$$

\noindent and set

\Centerline{$\Cal I'_{np}=\{(E,I,w):(E,I,w)\in\Cal I_{np}$,
$\#(I\cap m)\ge\Bover14\#(I)\}$.}

\noindent Set $B_{np}=\theta_m(\spread\Cal I'_{np})$;  then

\Centerline{$\nu_{kp}B_{np}\le 2\weight\Cal I'_{np}\le 4$,}

\noindent by 394Fa.   Since $\nu_{kp}X\ge c_k\ge 8$
(394G, 394Dg again), $B_{np}\ne X$ and there is an
$A_{np}\in\Cal A_m$ disjoint from $\spread\Cal I'_{np}$.   Next, for
$m<r\le n$ and $p\in U_n$ set

\Centerline{$\Cal J_{npr}
=\{(\theta_{r\setminus m}(A_{np}\cap E),I\cap r\setminus m,2w):
(E,I,w)\in\Cal I_{npr}\}$,
\quad$F_{npr}=\spread\Cal J_{npr}$.}

\noindent By 394E, $\Cal J_{npr}\subseteq\Cal C_{kp}$, so
$\nu_{kp}F_{npr}\le\weight\Cal J_{npr}\le 2\weight\Cal I_{npr}$.   
Note that $F_{npr}$ is determined by coordinates in
$r\setminus m$ and includes $A_{np}\cap\spread\Cal I_{npr}$.
Now if $m<j\le n$ and $p\in U_n$,
$\tilde E_j\subseteq\bigcup_{m<r\le j}F_{npr}$.   \Prf\Quer\
Otherwise, since both sets are determined by coordinates in $j\setminus m$,
and since $A_{np}\in\Cal A_m$, there is an $A\in\Cal A_j$ with

\Centerline{$A\subseteq A_{np}\cap\tilde E_j
   \setminus\bigcup_{m<r\le j}F_{npr}
\subseteq A_{np}\cap\tilde E_j
   \setminus\bigcup_{m<r\le j}\spread\Cal I_{npr}$.}

\noindent Since $A$ is also disjoint from $\spread\Cal I'_{np}$ and
$A\subseteq\tilde E_j\subseteq\spread\Cal I_{np}$,
$A\subseteq\spread\Cal I$, where

$$\eqalign{\Cal I
&=\Cal I_{np}
  \setminus(\Cal I'_{np}\cup\bigcup_{m<r\le j}\Cal I_{npr})\cr
&\subseteq\{(E,I,w):(E,I,w)\in\Cal I_{np},\,
  \#(I\setminus j)\ge\Bover14\#(I)\}.\cr}$$

\noindent Since $\Cal I\subseteq\Cal C_{kp}$,

$$\eqalignno{8
&\le\nu_{kp}X
=\nu_{kp}(\theta_{\Bbb N\setminus j}(A))
=\nu_{kp}(\theta_{\Bbb N\setminus j}(A\cap\spread\Cal I))
\le 2\weight\Cal I\cr
\displaycause{394Fb}
&\le 4.  \text{ \Bang\Qed}\cr}$$

 %T26/5

\medskip

{\bf (b)} For $r>m$ we can find $F_r\in\frak B$ such that

\inset{$\sum_{r=m+1}^{\infty}\nu_kF_r\le 4$,

$\tilde E_j\subseteq\bigcup_{m<r\le j}F_r$ for every $j>m$,

$F_r$ is determined by coordinates in $r\setminus m$.}

\noindent\Prf\ If $n\ge r>m$, then, because $\frak B_r$ is finite, 
there is a set $F_{nr}\in\frak B_r$ such that
$\{p:p\in U_n,\,F_{npr}=F_{nr}\}$ belongs to $\Cal F$.
Next, if $r>m$ there is an $F_r\in\frak B_r$ such that
$\{n:n\ge r$, $F_{nr}=F_r\}$ belongs to $\Cal F$.   Now

$$\eqalignno{\nu_kF_r
&=\lim_{n\to\Cal F}\nu_kF_{nr}\cr
\displaycause{because 
$\{n:\nu_kF_{nr}=\nu_kF_r\}\supseteq\{n:F_{nr}=F_r\}\in\Cal F$}
&=\lim_{n\to\Cal F}\lim_{p\to\Cal F}\nu_{kp}F_{nr}
=\lim_{n\to\Cal F}\lim_{p\to\Cal F}\nu_{kp}F_{npr}
\le 2\lim_{n\to\Cal F}\lim_{p\to\Cal F}\weight\Cal I_{npr}.\cr}$$

\noindent So, for $s>m$,

$$\eqalign{\sum_{r=m+1}^s\nu_kF_r
&\le 2\sum_{r=m+1}^s\lim_{n\to\Cal F}\lim_{p\to\Cal F}
   \weight\Cal I_{npr}\cr
&=2\lim_{n\to\Cal F}\lim_{p\to\Cal F}
   \sum_{r=m+1}^s\weight\Cal I_{npr}
\le 2\lim_{n\to\Cal F}\lim_{p\to\Cal F}
   \weight\Cal I_{np}
\le 4.\cr}$$

\noindent As $s$ is arbitrary, $\sum_{r=m+1}^{\infty}\nu_kF_r\le 4$.

If $n\ge j>m$, then we saw in (a) that
$\tilde E_j\subseteq\bigcup_{m<r\le j}F_{npr}$ for every
$p\in U_n$.   Since there are many $p$ such that $F_{nr}=F_{npr}$ whenever
$m<r\le j$, $\tilde E_j\subseteq\bigcup_{m<r\le j}F_{nr}$.   Now, given
$j>m$, there are many $n$ such that $F_{nr}=F_r$ whenever $m<r\le j$, so
$\tilde E_j\subseteq\bigcup_{m<r\le j}F_r$.

Finally, take any $r>m$.   Since $F_{npr}$ is determined by coordinates in
$r\setminus m$ whenever $n\ge r$ and $p\in U_n$, $F_{nr}$ is determined by
coordinates in $r\setminus m$ whenever $n\ge r$, and $F_r$ also is
determined by coordinates in $r\setminus m$.\ \Qed

\medskip

{\bf (c)} Let $r_0\ge m$ be such that
$\sum_{r=r_0+1}^{\infty}\nu_kF_r\le\eta$.   Set
$C=\bigcup_{m<r\le r_0}F_r$.   Then
$C$ is determined by coordinates in $\Bbb N\setminus m$ and

\Centerline{$\nu_kC\le\sum_{r=m+1}^{r_0}\nu_kF_r\le 4$.}

\noindent For any $i\in\Bbb N$, there is some $j>r_0$ such that
$E_i\subseteq\tilde E_j$, in which case

\Centerline{$E_i\setminus C\subseteq\bigcup_{r_0<r\le j}F_r$}

\noindent and

\Centerline{$\nu_k(E_i\setminus C)
\le\sum_{r=r_0+1}^j\nu_kF_r\le\eta$,}

\noindent as required.
}  %end of proof of 394K

\leader{394L}{Lemma}  %\Tal{\S6.6}
Suppose that $k\in\Bbb N$, $\epsilon>0$, $m\in\Bbb N$,
$B\in\frak B_m$ and that
$\sequence{i}{E_i}$ is a disjoint sequence in $\frak B$.   Then there
are $n>m$ and $B'\in\frak B_n$ such that $B'\subseteq B$, $B'$ is
$\bover12\nu_k$-thin between $m$ and $n$ and
$\limsup_{i\to\infty}\nu_k(E_i\cap B\setminus B')\le\epsilon$.

\proof{ Set $\eta=\Bover{\epsilon}{\#(\Cal A_m)}$.    For
those $A\in\Cal A_m$ included in $B$ define $C'_A\subseteq A$ as follows.

\medskip

\quad{\bf case 1} If there is some $r$ such that
$\nu_k(\theta_{\Bbb N\setminus m}(A\cap\bigcup_{i\le r}E_i))\ge 2$, set
$C'_A=A\setminus\bigcup_{i\le r}E_i$, so that
$\bover12\nu_k(\theta_{\Bbb N\setminus m}(A\setminus C'_A))\ge 1$ and
$E_i\cap A\setminus C'_A=\emptyset$ for $i>r$.

\medskip

\quad{\bf case 2} If
$\nu_k(\theta_{\Bbb N\setminus m}(A\cap\bigcup_{i\le r}E_i))<2$
for every $r$, then by 394K, applied to the sequence
$\sequence{i}{\theta_{\Bbb N\setminus m}(A\cap E_i)}$,
 we can find a $C\in\frak B$,
determined by coordinates in $\Bbb N\setminus m$,
such that $\nu_kC\le 4$ and
$\nu_k(\theta_{\Bbb N\setminus m}(A\cap E_i)\setminus C)\le\eta$
for every $i$.
Set $C'_A=C\cap A$.   Because $C$ is determined by coordinates in
$\Bbb N\setminus m$ and $A\in\Cal A_m$,
$\nu_k(\theta_{\Bbb N\setminus m}(C'_A))=\nu_kC\le 4$.
Also $E_i\cap A\setminus C'_A
\subseteq\theta_{\Bbb N\setminus m}(A\cap E_i)\setminus C$
so $\nu_k(E_i\cap A\setminus C'_A)\le\eta$ for every $i$.

\medskip

Set

\Centerline{$B'=\bigcup\{C'_A:A\in\Cal A_m$, $A\subseteq B\}$.}

\noindent Then $B'\in\frak B$, $B'\subseteq B$ and

$$\eqalign{\limsup_{i\to\infty}\nu_k(E_i\cap B\setminus B')
&\le\sum_{A\in\Cal A_m,A\subseteq B}\limsup_{i\to\infty}
  \nu_k(E_i\cap A\setminus C'_A)\cr
&\le\sum_{A\in\Cal A_m,A\subseteq B}\eta
\le\epsilon.\cr}$$

Let $n>m$ be such that $C'_A\in\frak B_n$ whenever $A\in\Cal A_m$ and
$A\subseteq B$.   Then $B'$ is $\bover12\nu_k$-thin between $m$ and $n$.
\Prf\ Take any $A\in\Cal A_m$ and set
$\tilde C=\theta_{n\setminus m}(A\cap B')$.
If $A\not\subseteq B$ then $A\cap B'$ and $\tilde C$
are empty and $\nu_k(X\setminus\tilde C)\ge 8$ (394Ia).
Otherwise, $A\cap B'=C'_A\in\frak B_n$ so
$\tilde C=\theta_{\Bbb N\setminus m}(C'_A)$ is disjoint from
$\theta_{\Bbb N\setminus m}(A\setminus C'_A)$ (see the last remark in
394Db).   If $C'_A$ was chosen as in case 1 above,

\Centerline{$\nu_k(X\setminus\tilde C)
\ge\nu_k(\theta_{\Bbb N\setminus m}(A\setminus C'_A))\ge 2$.}

\noindent
If $C'_A$ was chosen as in case 2,

\Centerline{$\nu_k(X\setminus\tilde C)
=\nu_k(X\setminus\theta_{\Bbb N\setminus m}(C'_A))
\ge\nu_kX-\nu_k(\theta_{\Bbb N\setminus m}(C'_A))
\ge 8-4$.}

\noindent So in all three cases we
have $\bover12\nu_k(X\setminus\tilde C)\ge 1$, as required.\ \Qed

Thus we have an appropriate $B'$.
}  %end of proof of 394L

\leader{394M}{Theorem}  %\Tal{\S6.6}
$\nu$ is exhaustive.

\proof{ Let $\sequence{i}{E_i}$ be a disjoint sequence in $\frak B$.
Take any $k\in\Bbb N$ and $\epsilon>0$, and choose $\sequence{j}{B_j}$ and
$\sequence{j}{n_j}$ inductively, as
follows.    $B_0=X$ and $n_0=0$.   Given that $B_j\in\frak B_{n_j}$,
take $n_{j+1}>n_j$ and $B_{j+1}\in\frak B_{n_{j+1}}$ such that
$B_{j+1}\subseteq B_j$, $B_{j+1}$ is
$\bover12\nu_{k+1}$-thin between $n_j$ and $n_{j+1}$, and
$\limsup_{i\to\infty}
  \nu_{k+1}(E_i\cap B_j\setminus B_{j+1})\le\epsilon$
(394L).   Continue.   Note that
$\limsup_{i\to\infty}\nu_{k+1}(E_i\setminus B_j)\le j\epsilon$
for every $j$.

Set $I=\{n_j:j<N_k\}$ and $B=B_{N_k-1}$.    Then $B$ is
$\bover12\nu_{k+1}$-thin along $I$ (use 394Dc).   By 394J, 
$\nu_kB\le 2^{-k}$.

Of course $\nu\le\nu_k\le\nu_{k+1}$ (394Df).   So

\Centerline{$\limsup_{i\to\infty}\nu E_i
\le\nu_kB+\limsup_{i\to\infty}\nu_{k+1}(E_i\setminus B)
\le 2^{-k}+(N_k-1)\epsilon$.}

\noindent As $k$, $\epsilon$ and $\sequence{i}{E_i}$ are arbitrary,
$\nu$ is exhaustive.
}%end of proof of 394M

\leader{394N}{Remarks (a)} 
Note that the whole construction is invariant under
the action of the group $\prod_{n\in\Bbb N}G_n$ where $G_n$ is the group of
all permutations of $T_n$ for each $n$.   In particular, if we
give each $T_n$ a group structure and
$X$ the product group structure, then $\nu$ is translation-invariant.

\spheader 394Nb\cmmnt{ It follows that} $\nu$ is strictly positive.
\prooflet{\Prf\ For each $n\in\Bbb N$, $\nu$ is constant on $\Cal A_n$, so
$\nu E\ge\nu X/\#(\Cal A_n)>0$ for every non-empty $E\in\frak B_n$.\ \QeD}   

\spheader 394Nc We can\cmmnt{ therefore} form the metric completion
$\widehat{\frak B}$ of $\frak B$\cmmnt{, as in 392H}, 
and $\widehat{\frak B}$ will
be a Maharam algebra, with a strictly positive Maharam submeasure $\hat\nu$
continuously extending $\nu$\cmmnt{ (393H)}.   
\cmmnt{Now} $\widehat{\frak B}$ is not
measurable.   \prooflet{\Prf\Quer\ Otherwise, let $\bar\mu$ be such that
$(\widehat{\frak B},\bar\mu)$ is a probability algebra.   Then $\bar\mu$
and $\hat\nu$ are strictly positive Maharam submeasures on 
$\widehat{\frak B}$, so $\hat\nu$ is absolutely continuous with respect to
$\bar\mu$ (393F).   Let $n\ge 1$ be such that $\hat\nu b<8$ whenever
$\bar\mu b\le 1/\#(T_n)$.   Then there must be a $t\in T_n$ such that
$\bar\mu Y_{nt}\le 1/\#(T_n)$;  
but $\hat\nu Y_{nt}=\nu Y_{nt}\ge 8$
(see the proof of 394Ib).\ \Bang\Qed}

\cmmnt{In fact, $\widehat{\frak B}$ is nowhere measurable (394Ya).}

\leader{*394O}{Control measures}\dvAformerly{3{}93O}\cmmnt{ One 
of the original
reasons for studying Maharam submeasures was their connexion with the
following notion.}
Let $\frak A$ be a Dedekind
$\sigma$-complete Boolean algebra and $U$ a Hausdorff linear topological
space.   \cmmnt{ (The idea is intended to apply, in particular, when
$\frak A$ is a $\sigma$-algebra of subsets of a set.)}   A function
$\theta:\frak A\to U$ is a {\bf vector measure} if
$\sum_{n=0}^{\infty}\theta a_n=\lim_{n\to\infty}\sum_{i=0}^n\theta a_i$
is defined in $U$ and equal to $\theta(\sup_{n\in\Bbb N}a_n)$ for every
disjoint sequence $\sequencen{a_n}$ in $\frak A$.
In this case, a non-negative countably additive
functional $\mu:\frak A\to\coint{0,\infty}$ is a {\bf control measure}
for $\theta$ if $\theta a=0$ whenever $\mu a=0$.

%what about an absolute-continuity definition?

\leader{*394P}{Example} There are a metrizable linear topological space
$U$ and a vector measure $\theta:\Sigma\to U$, where $\Sigma$ is a
$\sigma$-algebra of sets, such that $\theta$ has no control measure.

\proof{ As in 394Nc, let $\widehat{\frak B}$ be the metric completion of 
$\frak B$, and $\hat\nu$ the
continuous extension of $\nu$ to $\widehat{\frak B}$.   Give
$L^0=L^0(\widehat{\frak B})$ the topology defined from $\hat\nu$ as in 393K, so that
$L^0$ is a metrizable linear topological space.   By 314M, we can
identify $\widehat{\frak B}$ with a quotient algebra $\Sigma/\Cal N$ where $\Sigma$
is a $\sigma$-algebra of subsets of a set $\Omega$
and $\Cal N$ is a $\sigma$-ideal in $\Sigma$.
Set $\theta E=\chi E^{\ssbullet}\in L^0$ for $E\in\Sigma$.   Then $\theta$
is a vector measure.
\Prf\ If $\sequencen{E_n}$ is a disjoint sequence in $\Sigma$
with union $E$, set $F_n=\bigcup_{i\le n}E_i$, so that
$\chi F_n^{\ssbullet}=\sum_{i=0}^n\chi E_i^{\ssbullet}$ for each $n$.
We have $\hat\nu(E^{\ssbullet}\Bsetminus F_n^{\ssbullet})\to 0$, so that

\Centerline{$\tau(\theta E-\theta F_n)
=\tau(\chi E^{\ssbullet}-\chi F_n^{\ssbullet}))
=\min(1,\hat\nu(E^{\ssbullet}\Bsetminus F_n^{\ssbullet}))\to 0$,}

\noindent where $\tau$ is the functional of the proof of 393K, and
$\theta E=\sum_{i=0}^{\infty}\theta E_i$ in $L^0$.\ \Qed

If $\mu$ is a totally finite measure with domain $\Sigma$, set

\Centerline{$\lambda a=\inf\{\mu E:E\in\Sigma$, $E^{\ssbullet}=a\}$}

\noindent for every $a\in\widehat{\frak B}$.   Note that the infimum is always
attained.   \Prf\ If $\sequencen{E_n}$ is a sequence in $\Sigma$ such that
$E_n^{\ssbullet}=a$ for every $n\in\Bbb N$ and
$\lambda a=\lim_{n\to\infty}\mu E_n$, set $E=\bigcap_{n\in\Bbb N}E_n$;
then $E^{\ssbullet}=a$ and $\mu E=\lambda a$.\ \QeD\  Next, $\lambda$ is
countably additive.   \Prf\ If $\sequencen{a_n}$ is a disjoint sequence in
$\widehat{\frak B}$ with supremum $a$,
take $E_n\in\Sigma$ such that $E_n^{\ssbullet}=a_n$ and
$\mu E_n=\lambda a_n$ for each $n$, and $E\in\Sigma$ such that
$E^{\ssbullet}=a$ and $\mu E=\lambda a$.   Set
$F_n=E\cap E_n\setminus\bigcup_{i<n}E_i$ for each $n$, and
$F=\bigcup_{n\in\Bbb N}F_n$.   Then $F_n^{\ssbullet}=a_n$ and
$F_n\subseteq E_n$, so $\mu F_n=\lambda a_n$ for each $n$;  similarly,
$F^{\ssbullet}=a$ and $F\subseteq E$, so $\mu F=\lambda a$.   Also
$\sequencen{F_n}$ is disjoint and has union $F$.   Accordingly

\Centerline{$\lambda a=\mu F=\sum_{n=0}^{\infty}\mu F_n
=\sum_{n=0}^{\infty}\lambda a_n$.  \Qed}

Since $\widehat{\frak B}$ is not a measurable algebra, $\lambda$ cannot be strictly
positive, and there is a non-zero $a\in\widehat{\frak B}$ such that $\lambda a=0$.
Let $E\in\Sigma$ be such that $E^{\ssbullet}=a$ and $\mu E=0$;  then
$\theta E=\chi a\ne 0$.   So $\mu$ is not a control measure for $\theta$.
}  %end of proof of 394P

\leader{*394Q}{}\cmmnt{ This is not a book about vector measures, but
having gone so far I ought to note
that the generality of the phrase `metrizable linear topological space' in
394P is essential.   If we look only at normed spaces the situation is very
different.

\medskip

\noindent}{\bf Theorem} Let $\frak A$ be a Dedekind $\sigma$-complete
Boolean algebra, $U$ a normed space and $\theta:\frak A\to U$ a vector
measure.   Then $\theta$ has a control measure.

\proof{{\bf (a)} Since $U$ can certainly be embedded in a Banach space
$\hat U$ (3A5Jb), and as $\theta$ will still be a vector measure when
regarded as a map from $\frak A$ to $\hat U$, we may assume from the
beginning that $U$ itself is complete.

\medskip

{\bf (b)} $\theta$ is bounded (that is, $\sup_{a\in\frak A}\|\theta a\|$
is finite).   \Prf\Quer\ (Cf.\ 326M.)
Suppose, if possible, otherwise.   Choose
$\sequencen{a_n}$ inductively, as follows.   $a_0=1$.   Given that
$\sup_{a\Bsubseteq a_n}\|\theta a\|=\infty$, choose $b\Bsubseteq a_n$
such that $\|\theta b\|\ge\|\theta a_n\|+1$.   Then
$\|\theta(a_n\Bsetminus b)\|\ge 1$.   Also

\Centerline{$\sup_{a\Bsubseteq a_n}\|\theta a\|
\le\sup_{a\Bsubseteq a_n}\|\theta(a\Bcap b)\|+\|\theta(a\Bsetminus
b)\|$,}

\noindent so at least one of $\sup_{a\Bsubseteq b}\|\theta a\|$,
$\sup_{a\Bsubseteq a_n\Bsetminus b}\|\theta a\|$ must be infinite.   We
may therefore take $a_{n+1}$ to be either $b$ or $a_n\Bsetminus b$ and
such that $\sup_{a\Bsubseteq a_{n+1}}\|\theta a\|=\infty$.   Observe
that in either case we shall have $\|\theta(a_n\Bsetminus a_{n+1})\|\ge
1$.   Continue.

At the end of the induction we shall have a disjoint sequence
$\sequencen{a_n\Bsetminus a_{n+1}}$ such that
$\|\theta(a_n\Bsetminus a_{n+1})\|\ge 1$ for every $n$, so that
$\sum_{n=0}^{\infty}\theta(a_n\Bsetminus a_{n+1})$ cannot be defined in
$U$;  which is impossible.\ \Bang\Qed

\medskip

{\bf (c)} Accordingly we have a bounded linear operator
$T:L^{\infty}\to U$, where $L^{\infty}=L^{\infty}(\frak A)$, such that
$T\chi=\theta$ (363Ea).

Now the key to the proof is the following fact:  if $\sequencen{u_n}$ is
a disjoint order-bounded sequence in $(L^{\infty})^+$,
$\sequencen{Tu_n}\to 0$ in $U$.   \Prf\ Let $\gamma$ be such that
$u_n\le\gamma\chi 1$ for every $n$.   Let $\epsilon>0$, and let $k$ be
the integer part of $\gamma/\epsilon$.   For $n\in\Bbb N$, $i\le k$ set
$a_{ni}=\Bvalue{u_n>\epsilon(i+1)}$;   then $\sequencen{a_{ni}}$ is
disjoint for each $i$, and if we set $v_n=\epsilon\sum_{i=0}^k\chi
a_{ni}$, we get $v_n\le u_n\le v_n+\epsilon\chi 1$, so
$\|u_n-v_n\|_{\infty}\le\epsilon$.

Because $\sequencen{a_{ni}}$ is disjoint, 
$\sum_{n=0}^{\infty}\theta a_{ni}$ is defined in $U$, 
and $\sequencen{\theta a_{ni}}\to 0$, for each $i\le k$.   Consequently

\Centerline{$Tv_n=\epsilon\sum_{i=0}^k\theta a_{ni}\to 0$}

\noindent as $n\to\infty$.  But

\Centerline{$\|Tu_n-Tv_n\|\le\|T\|\|u_n-v_n\|_{\infty}\le\epsilon\|T\|$}

\noindent for each $n$, so
$\limsup_{n\to\infty}\|Tu_n\|\le\epsilon\|T\|$.   As $\epsilon$ is
arbitrary, $\lim_{n\to\infty}\|Tu_n\|=0$.\ \Qed

\medskip

{\bf (d)} Consider the adjoint operator $T':U^*\to (L^{\infty})^*$.
Recall that $L^{\infty}$ is an $M$-space (363Ba) so that its dual is an
$L$-space (356N).   Write

\Centerline{$A=\{T'g:g\in U^*,\,\|g\|\le 1\}\subseteq(L^{\infty})^*
=(L^{\infty})^{\sim}$.}

\noindent If $u\in L^{\infty}$, then

\Centerline{$\sup_{f\in A}|f(u)|=\sup_{\|g\|\le 1}|(T^*g)(u)|
=\sup_{\|g\|\le 1}|g(Tu)|=\|Tu\|$.}

\noindent Now $A$ is uniformly integrable.   \Prf\ I use the criterion
of 356O.   Of course $\|f\|\le\|T'\|$ for every $f\in A$, so $A$ is
norm-bounded.   If $\sequencen{u_n}$ is an order-bounded disjoint
sequence in $(L^{\infty})^+$, then

\Centerline{$\sup_{f\in A}|f(u_n)|=\|Tu_n\|\to 0$}

\noindent as $n\to\infty$.   So $A$ is uniformly integrable.\ \Qed

\medskip

{\bf (e)} Next, $A\subseteq(L^{\infty})^{\sim}_c$.   \Prf\ If $f\in A$,
it is of the form $T'g$ for some $g\in U^*$, that is,

\Centerline{$f(\chi a)=(T'g)(\chi a)=gT(\chi a)=g(\theta a)$}

\noindent for every $a\in\frak A$.   If now $\sequencen{a_n}$ is a
disjoint sequence in $\frak A$ with supremum $a$,

\Centerline{$f(\chi a)
=g(\theta(\sup_{n\in\Bbb N}a_n))
=g(\sum_{n=0}^{\infty}\theta a_n)
=\sum_{n=0}^{\infty}g(\theta a_n)
=\sum_{n=0}^{\infty}f(\chi a_n)$.}

\noindent So $f\chi$ is countably additive.   By 363K,
$f\in(L^{\infty})^{\sim}_c$.\ \Qed

\medskip

{\bf (f)} Because $A$ is uniformly integrable, there is for each
$m\in\Bbb N$ an $f_m\ge 0$ in $(L^{\infty})^*$ such that
$\|(|f|-f_m)^+\|\le 2^{-m}$ for every $f\in A$;  moreover, we can
suppose that $f_m$ is of the form $\sup_{i\le k_m}|f_{mi}|$ where every
$f_{mi}$ belongs to $A$ (354R(b-iii)), so that
$f_m\in(L^{\infty})^{\sim}_c$ and $\mu_m=f_m\chi$ is countably additive.
Set

\Centerline{$\mu=\sum_{m=0}^{\infty}\Bover1{2^m(1+\mu_m1)}\mu_m$;}

\noindent  then $\mu:\frak A\to\coint{0,\infty}$ is a non-negative
countably additive functional.

Now $\mu$ is a control measure for $\theta$.   \Prf\ If $\mu a=0$, then
$\mu_ma=0$, that is, $f_m(\chi a)=0$, for every $m\in\Bbb N$.   But this
means that if $g\in U^*$ and $\|g\|\le 1$,

\Centerline{$|g(\theta a)|=|(T'g)(\chi a)|
\le f_m(\chi a)+\|(|T'g|-f_m)^+\|\le 2^{-m}$}

\noindent for every $m$, by the choice of $f_m$;  so that $g(\theta
a)=0$.   As $g$ is arbitrary, $\theta a=0$;  as $a$ is arbitrary, $\mu$
is a control measure for $\theta$.\ \Qed
}  %end of proof of 394Q

\exercises{
\leader{394X}{Basic exercises (a)}
%\spheader 394Xa 
Show that the metric completion
$\widehat{\frak B}$ of $\frak B$, as defined in
394N, has many involutions (definition:  382O).
%394N

\leader{394Y}{Further exercises (a)}%
%\spheader 394Ya
(i) 
Show that if $r\in\Bbb N$, $k\le p$ and $E\in\frak B_{r+1}$ are such that 
$\nu_{kp}E<c_k$, then $\nu_{kp}(\theta_r(E))\le\Bover{32}{c_k}\,\nu_{kp}E$.
%mt39bits 
(ii) Show that if $E\in\Cal B_r$ then
$\nu(E\cap Y_{rt})\ge\min(8,\bover14\nu E)$ for every $t\in T_r$.
(iii) Let $\widehat{\frak B}$ be the metric completion of $\frak B$ and
$\hat\nu$ the continuous extension of $\nu$ to $\widehat{\frak B}$.
Show that for every $a\in\frak B$ and
$n\in\Bbb N$ there is a disjoint family
$\langle c_i\rangle_{i\le n}$ such that $c_i\Bsubseteq a$ and
$\hat\nu c_i\ge\min(7,\bover15\hat\nu a)$ for every $i\le n$.
(iv) Show that the
only countably additive real-valued functional on $\widehat{\frak B}$ is 
the zero functional.
(v) Show that $\widehat{\frak B}$ is nowhere measurable.
(vi) Show that if $\nu'$ is a uniformly exhaustive submeasure on
$\frak B$ which is absolutely continuous with respect to $\nu$, then
$\nu'=0$.  
%394G 394N 
}  %end of exercises

\leader{394Z}{Problems (a)}
Does $\widehat{\frak B}$ have an order-closed subalgebra isomorphic to
the measure algebra of Lebesgue measure?   In particular, if we take
$\frak C\subseteq\frak B$ to be the algebra of sets generated by sets of
the form $\{x:x\in X$, $x(n)=0\}$ for $n\in\Bbb N$, is $\nu\restr\frak C$
uniformly exhaustive?

\spheader 394Zb Suppose that instead of taking large sets $T_n$, we simply
set $T_n=\{0,1\}$ for every $n$, but otherwise used the same construction.
Should we then find that $\nu$ was uniformly exhaustive?
\cmmnt{(This might be relevant to (a) above.)}

\spheader 394Zc Is the Boolean algebra $\widehat{\frak B}$ homogeneous?

\endnotes{
\Notesheader{394} `Maharam's problem', or the `control measure problem',
was for fifty years one of the most vexing questions in abstract measure
theory.   To begin with, there were reasonable hopes that there was a
positive answer -- in the language of this book, that every Maharam algebra
was a measurable algebra.   If this had been the case, there would have
been consequences all over the theories of topological Boolean algebras,
topological Riesz spaces and vector measures.   In the 1970s, it began to
seem too much to ask for.   In 1983 the Kalton-Roberts theorem gave new
life to the conjecture for a moment, but {\smc Roberts 93} demonstrated a
major obstacle, which Talagrand (building on some further ideas of I.Farah)
eventually developed into the construction above.   The ideas which for a
generation were collected together by their association with the control
measure problem no longer have this as a unifying principle, and (as after
any successful revolution) are now more naturally grouped in other ways.
In the 2004 edition of this volume, maintained in
{\tt http://www.essex.ac.uk/maths/people/fremlin/mt3.2004}, you can find
a list of formulations of the control measure problem as it
then appeared to be.   There is a relic of this era in 394P.

Now that we know for sure that there are non-measurable Maharam algebras,
it becomes possible to ask questions about their structure.  
Frustratingly, practically none of these questions has yet been answered
even for the examples constructed by Talagrand's method.   (I should of
course note that while I speak of `the' submeasure $\nu$ and `the' Maharam
algebra $\widehat{\frak B}$, they depend on the sequences
$\sequencen{\#(T_n)}$,
$\sequence{k}{\alpha_k}$ and $\sequence{k}{N_k}$ and the filter 
$\Cal F$, and there is every reason to suppose that $\frak c$
non-isomorphic examples can be constructed by the formulae set out above,
without considering elementary variations.)   I will return briefly to
such questions in Volumes 4 and 5, as I come to further properties of
measure algebras which can be interpreted in Maharam algebras.
}  %end of notes

\discrpage

