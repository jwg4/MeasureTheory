\frfilename{mt512.tex}
\versiondate{27.11.13}
\copyrightdate{2002}

\def\chaptername{Cardinal functions}
\def\sectionname{Galois-Tukey connections}

\def\Vojtas{Vojt\'a\v{s} 93}

\newsection{512}

One of the most powerful methods of relating the cardinals associated
with two partially ordered sets $P$ and $Q$ is to identify a `Tukey
function' from one to the other (513D).   It turns out that the idea can
be usefully generalized to other relational structures through the
concept of `Galois-Tukey connection' (512A).   In this section I give
the elementary theory of these connections and their effect on simple
cardinal functions.

\leader{512A}{Definitions (a)} A {\bf supported relation} is a triple
$(A,R,B)$ where $A$ and $B$ are sets and $R$ is a subset of $A\times B$.

\cmmnt{It will be convenient, and I think not dangerous, to abuse
notation by writing $(A,\in,B)$ or $(A,\subseteq,B)$ to mean $(A,R,B)$
where $R$ is $\{(a,b):a\in A,\,b\in B,\,a\in b\}$ or
$\{(a,b):a\in A,\,b\in B,\,a\subseteq b\}$.}

\spheader 512Ab If $(A,R,B)$ is a supported relation, its {\bf dual} is
the supported relation $(A,R,B)^{\perp}=(B,S,A)$ where

\Centerline{$S=(B\times A)\setminus R^{-1}
=\{(b,a):a\in A,\,b\in B,\,(a,b)\notin R\}$.}

\spheader 512Ac If $(A,R,B)$ and $(C,S,D)$ are supported relations, a
{\bf Galois-Tukey connection} from $(A,R,B)$ to $(C,S,D)$ is a pair
$(\phi,\psi)$ such that $\phi:A\to C$ and $\psi:D\to B$ are functions
and $(a,\psi(d))\in R$ whenever $(\phi(a),d)\in S$.

\leaveitout{alternatively: 
$S\frsmallcirc\phi\subseteq\psi^{-1}\frsmallcirc R$}

\spheader 512Ad \cmmnt{({\smc Vojt\'a\v{s} 93})} If $(A,R,B)$ and
$(C,S,D)$ are supported relations, I write $(A,R,B)\prGT(C,S,D)$ if
there is a Galois-Tukey connection from $(A,R,B)$ to $(C,S,D)$, and
$(A,R,B)\equivGT(C,S,D)$ if $(A,R,B)\prGT(C,S,D)$ and
$(C,S,D)\prGT(A,R,B)$.

\leader{512B}{Definitions (a)} If $(A,R,B)$ is a
supported relation, its {\bf covering number} 
$\cov(A,R,B)$\cmmnt{ (sometimes called {\bf norm} $\|(A,R,B)\|$)} is 
the least cardinal of any set $C\subseteq B$ such
that $A\subseteq R^{-1}[C]$;  or $\infty$ if $A\not\subseteq R^{-1}[B]$.
Its {\bf additivity} is $\add(A,R,B)=\cov(A,R,B)^{\perp}$, that is, the
smallest cardinal of any subset $C\subseteq A$ such that
$C\not\subseteq R^{-1}[\{b\}]$ for any $b\in B$;  or $\infty$ if there
is no such $C$.

\cmmnt{Note that $\add(A,R,B)=0$ iff $B=\emptyset$, and that
$\add(A,R,B)=1$ iff $B\ne\emptyset$ and $\cov(A,R,B)=\infty$.}

\spheader 512Bb If $(A,R,B)$ is a supported relation, its {\bf
saturation} $\sat(A,R,B)$ is the least cardinal $\kappa$ such that
whenever $\ofamily{\xi}{\kappa}{a_{\xi}}$ is a family in $A$ then there
are distinct $\xi$, $\eta<\kappa$ and a $b\in B$ such that $(a_{\xi},b)$
and $(a_{\eta},b)$ both belong to $R$;  if there is no such
$\kappa$\cmmnt{ (that is, if $\cov(A,R,B)=\infty$)} I write
$\sat(A,R,B)=\infty$.

\spheader 512Bc If $(A,R,B)$ is a supported relation and $\kappa$ is a
cardinal, say that a subset $A'$ of $A$ is
{\bf $\hbox{$<$}\kappa$-linked} if for
every $I\in[A']^{<\kappa}$ there is a $b\in B$ such that
$I\subseteq R^{-1}[\{b\}]$, and {\bf $\kappa$-linked} if it is
$\hbox{$<$}\kappa^+$-linked\cmmnt{, that is, for every
$I\in[A']^{\le\kappa}$
there is a $b\in B$ such that $I\subseteq R^{-1}[\{b\}]$}.   Now the
{\bf $\hbox{$<$}\kappa$-linking number} $\link_{<\kappa}(A,R,B)$ of
$(A,R,B)$ is the
least cardinal of any cover of $A$ by $\hbox{$<$}\kappa$-linked sets, if
there is such a cover, and otherwise is $\infty$;  and the
{\bf $\kappa$-linking number} $\link_{\kappa}(A,R,B)$ of $(A,R,B)$ is
$\link_{<\kappa^+}(A,R,B)$\cmmnt{, that is, the least cardinal of any
cover of $A$ by $\kappa$-linked sets}.

If $\kappa\le\lambda$, then\cmmnt{ every $\hbox{$<$}\lambda$-linked
set is $\hbox{$<$}\kappa$-linked,
so} $\link_{<\kappa}(A,R,B)\le\link_{<\lambda}(A,R,B)$.
\cmmnt{Note also that} $\link_{\kappa}(A,R,B)$ is equal to
$\cov(A,R,B)$ for every $\kappa\ge\#(A)$, so that
$\link_{<\theta}(A,R,B)\le\cov(A,R,B)$ for every $\theta$.

\leader{512C}{}\cmmnt{ There are two things which should be done at
once:  to plainly state enough of the elementary theory to show at least
that the definitions here lead to a coherent structure;  and to give
examples.   I begin with the theory, which really is elementary.

\medskip

}\noindent{\bf Theorem} Let $(A,R,B)$, $(C,S,D)$ and $(E,T,F)$ be
supported relations.

(a) $(A,R,B)^{\perp\perp}=(A,R,B)$.

(b) If $(\phi,\psi)$ is a Galois-Tukey connection from $(A,R,B)$ to
$(C,S,D)$ and $(\phi',\psi')$ is a Galois-Tukey connection from
$(C,S,D)$ to $(E,T,F)$, then $(\phi'\phi,\psi\psi')$ is a Galois-Tukey
connection from $(A,R,B)$ to $(E,T,F)$.

(c) If $(\phi,\psi)$ is a Galois-Tukey connection from $(A,R,B)$ to
$(C,S,D)$, then $(\psi,\phi)$ is a Galois-Tukey connection from
$(C,S,D)^{\perp}$ to $(A,R,B)^{\perp}$.

(d) $(A,R,B)\prGT(A,R,B)$.

(e) If $(A,R,B)\prGT(C,S,D)$ and $(C,S,D)\prGT(E,T,F)$ then
$(A,R,B)\prGT(E,T,F)$.

(f) $\equivGT$ is an equivalence relation on the class of supported
relations.

(g) If $(A,R,B)\prGT(C,S,D)$ then $(C,S,D)^{\perp}\prGT(A,R,B)^{\perp}$.
So if $(A,R,B)\equivGT(C,S,D)$ then
$(A,R,B)^{\perp}\equivGT(C,S,D)^{\perp}$.

\proof{(a)-(c) are immediate from the definitions.   (d) is trivial
because the identity functions from $A$ and $B$ to themselves form a
Galois-Tukey connection from $(A,R,B)$ to itself.   (e) follows from
(b), and (g) from (c).   (f) is immediate from (d) and (e) and the
symmetry of the definition of $\equivGT$.
}%end of proof of 512C

\leader{512D}{Theorem} Let $(A,R,B)$ and $(C,S,D)$ be supported
relations such that $(A,R,B)\prGT(C,S,D)$.   Then

(a) $\cov(A,R,B)\le\cov(C,S,D)$;

(b) $\add(C,S,D)\le\add(A,R,B)$;
%formerly 512Da

(c) $\sat(A,R,B)\le\sat(C,S,D)$;

(d) $\link_{<\kappa}(A,R,B)\le\link_{<\kappa}(C,S,D)$
for every cardinal $\kappa$.
%formerly 512Dc

\proof{ Let $(\phi,\psi)$ be a Galois-Tukey connection from $(A,R,B)$ to
$(C,S,D)$.   If $D_0\subseteq D$
is such that $C=S^{-1}[D_0]$, then $A=R^{-1}[\psi[D_0]]$;  this shows
that $\cov(A,R,B)\le\cov(C,S,D)$.
If $\kappa=\sat(C,S,D)$ and $\ofamily{\xi}{\kappa}{a_{\xi}}$ is any
family in $A$, then there are a $d\in D$ and distinct $\xi$,
$\eta<\kappa$ such that $(\phi(a_{\xi}),d)\in S$ and
$(\phi(a_{\eta}),d)\in S$, in which case $(a_{\xi},\psi(d))$ and
$(a_{\eta},\psi(d))$ both belong to $R$;  so $\sat(A,R,B)\le\kappa$.
If $\Cal C$ is a cover of $C$ by
$\hbox{$<$}\kappa$-linked sets, then $\{\phi^{-1}[C']:C'\in\Cal C\}$ is
a cover of $A$ by $\hbox{$<$}\kappa$-linked sets;  this shows that
$\link_{<\kappa}(A,R,B)\le\link_{<\kappa}(C,S,D)$.

Finally,
$(C,S,D)^{\perp}\prGT(A,R,B)^{\perp}$, by 512Cc, so

\Centerline{$\add(C,S,D)=\cov(C,S,D)^{\perp}\le\cov(A,R,B)^{\perp}
=\add(A,R,B)$.}
}%end of proof of 512D

\leader{512E}{Examples} \cmmnt{ Of course `supported relations' appear
everywhere in mathematics.   They are important to us here because
covering numbers, saturation and linking numbers, as defined above,
correspond to
important cardinal functions as defined in \S511, and because surprising
Galois-Tukey connections exist, as we shall see in Chapter 52.   The
simplest examples are the following.

\medskip

}{\bf (a)} Let $(P,\le)$ be a pre-ordered set.   Then $(P,\le,P)$
and $(P,\ge,P)$ are supported relations, with duals $(P,\not\ge,P)$ and
$(P,\not\le,P)$.   $\cov(P,\le,P)=\cf P$, $\cov(P,\ge,P)=\ci P$,
$\add(P,\le,P)=\add P$ and $\sat(P,\le,P)=\sat^{\uparrow}(P)$.
\cmmnt{For any cardinal $\kappa$, a subset of $P$ is
upwards-$\hbox{$<$}\kappa$-linked
in the sense of 511Bf iff it is $\hbox{$<$}\kappa$-linked in $(P,\le,P)$
in the sense of
512Bc.   So} $\link_{<\kappa}^{\uparrow}(P)=\link_{<\kappa}(P,\le,P)$.
In particular, $\duparrow(P)=\link_{<\omega}(P,\le,P)$ (511Bg).

\spheader 512Eb\dvArevised{2011}
Let $(X,\frak T)$ be a topological space.   Then

\Centerline{$\pi(X)=\cov(\frak T\setminus\{\emptyset\},\supseteq,
\frak T\setminus\{\emptyset\})$,}

\Centerline{$d(X)=\cov(\frak T\setminus\{\emptyset\},\ni,X)
=\add(X,\notin,\frak T\setminus\{\emptyset\})$,}

\Centerline{$\sat(X)
=\sat(\frak T\setminus\{\emptyset\},\supseteq,
  \frak T\setminus\{\emptyset\})
=\sat(\frak T\setminus\{\emptyset\},\ni,X)$,}

\Centerline{$n(X)=\cov(X,\in,\CalNwd(X))
=\cov(X,\CalNwd(X))$}

\noindent where $\CalNwd(X)$ is the ideal of nowhere dense subsets of $X$.
Note that if $\Cal M(X)$ is the ideal of meager subsets of $X$, then
$\cov(X,\Cal M(X))=n(X)$ unless $n(X)=\omega$, in which case
$\cov(X,\Cal M(X))=1$.

\spheader 512Ec Let $\frak A$ be a Boolean algebra.   Write
$\frak A^+$ for $\frak A\setminus\{0\}$ and $\frak A^-$ for
$\frak A\setminus\{1\}$.   Then

\Centerline{$\pi(\frak A)=\cov(\frak A^+,\Bsupseteqshort,\frak A^+)
=\cov(\frak A^-,\Bsubseteqshort,\frak A^-)$,}

\Centerline{$\sat(\frak A)=\sat(\frak A^+,\Bsupseteqshort,\frak A^+)
=\sat(\frak A^-,\Bsubseteqshort,\frak A^-)$,}

\Centerline{$d(\frak A)
=\link_{<\omega}(\frak A^+,\Bsupseteqshort,\frak A^+)
=\link_{<\omega}(\frak A^-,\Bsubseteqshort,\frak A^-)$,}

\Centerline{$\link(\frak A)=\link_2(\frak A^+,\Bsupseteqshort,\frak A^+)
=\link_2(\frak A^-,\Bsubseteqshort,\frak A^-)$}

\noindent and generally

\Centerline{$\link_{<\kappa}(\frak A)
=\link_{<\kappa}(\frak A^+,\Bsupseteqshort,\frak A^+)
=\link_{<\kappa}(\frak A^-,\Bsubseteqshort,\frak A^-)$,}

\Centerline{$\link_{\kappa}(\frak A)
=\link_{\kappa}(\frak A^+,\Bsupseteqshort,\frak A^+)
=\link_{\kappa}(\frak A^-,\Bsubseteqshort,\frak A^-)$}

\noindent for every cardinal $\kappa$.

\spheader 512Ed Let $X$ be a set and $\Cal I$ an ideal of subsets of
$X$.   Then the dual of $(X,\in,\Cal I)$ is $(\Cal I,\not\ni,X)$;
$\cov(X,\in,\Cal I)=\cov\Cal I$ and $\add(X,\in,\Cal I)=\non\Cal I$.

\spheader 512Ee For a Boolean algebra $\frak A$, write
$\Pou(\frak A)$ for the set of partitions of unity in $\frak A$.
For $C$, $D\in\Pou(\frak A)$, say that $C\sqsubseteq^* D$
if every element of $D$ meets only finitely many members of $C$.
\cmmnt{Then} $\sqsubseteq^*$ is a pre-order on $\Pou(\frak A)$.
\cmmnt{Translating the definition 511Df into this language, we see that}
$\wdistr(\frak A)=\add\Pou(\frak A)$.

\leader{512F}{}\cmmnt{ I now turn to some constructions
involving supported relations and Galois-Tukey connections which will be
useful later.

\medskip

\noindent}{\bf Dominating sets} For any supported relation $(A,R,B)$ and
any cardinal $\kappa$, we can form a corresponding supported relation
$(A,R^{\strprime},[B]^{<\kappa})$, where

\Centerline{$R^{\strprime}=\{(a,I):I\in[B]^{<\kappa}$, 
$a\in R^{-1}[I]\}$.}

%call $R^{\strprime}$ the `right extension' of $R$, or something?
%a name would be useful \query

\noindent\cmmnt{The most important cases to us will be $\kappa=\omega$
and $\kappa=\omega_1$.   When $\kappa$ is a successor cardinal
I will normally write $(A,R^{\strprime},[B]^{\le\lambda})$ rather than
$(A,R^{\strprime},[B]^{<\lambda^+})$.   Purists may wish to revise the
definition of $R^{\strprime}$ so that it is no longer a proper class.}

\leader{512G}{Proposition} Let $(A,R,B)$ and $(C,S,D)$ be supported
relations and $\kappa$, $\lambda$ cardinals.

(a) $(A,R,B)$ is isomorphic to $(A,R^{\strprime},[B]^1)$.

(b) If $(A,R,B)\prGT(C,S,D)$ and $\lambda\le\kappa$ then
$(A,R^{\strprime},[B]^{<\kappa})\prGT(C,S^{\strprime},[D]^{<\lambda})$.

(c) In particular, $(A,R^{\strprime},[B]^{<\kappa})\prGT(A,R,B)$ if
$\kappa\ge 2$.

(d) If $\cf\kappa\ge\lambda$ and
$(A,R^{\strprime},[B]^{<\kappa})\prGT(C,S,D)$ then
$(A,R^{\strprime},[B]^{<\kappa})\prGT(C,S^{\strprime},[D]^{<\lambda})$.

(e)(i) If $\cov(A,R,B)=\infty$ then
$\add(A,R^{\strprime},[B]^{<\kappa})\le 1$.

\quad(ii) If $\cov(A,R,B)<\infty$ then
$\add(A,R^{\strprime},[B]^{<\kappa})\ge\kappa$.

(f) $\cov(A,R,B)
\le\max(\omega,\kappa,\cov(A,R^{\strprime},[B]^{\le\kappa}))$;  if
$\kappa\ge 1$ and $\cov(A,R,B)>\max(\kappa,\omega)$ then
$\cov(A,R,B)\penalty-100=\cov(A,R^{\strprime},[B]^{\le\kappa})$.

\proof{{\bf (a)} is trivial.

\medskip

{\bf (b)} If $(\phi,\psi)$ is a Galois-Tukey connection from $(A,R,B)$
to $(C,S,D)$, then $(\phi,\psi')$ is a Galois-Tukey connection from
$(A,R^{\strprime},[B]^{<\kappa})$ to $(C,S^{\strprime},[D]^{<\lambda})$,
where $\psi'(J)=\psi[J]$ for every $J\in[D]^{<\lambda}$.

\medskip

{\bf (c)} Setting $\phi(a)=a$ for $a\in A$ and $\psi(b)=\{b\}$ for
$b\in B$, $(\psi,\phi)$ is a Galois-Tukey connection from
$(A,R^{\strprime},[B]^{<\kappa})$ to $(A,R,B)$.

\medskip

{\bf (d)} Let $(\phi,\psi)$ be a Galois-Tukey connection from
$(A,R^{\strprime},[B]^{<\kappa})$
to $(C,S,D)$.   Set $\psi'(I)=\bigcup_{d\in I}\psi(d)$ for
$I\in[D]^{<\lambda}$;  then $(\phi,\psi')$ is a
Galois-Tukey connection from $(A,R^{\strprime},[B]^{<\kappa})$ to
$(C,S^{\strprime},[D]^{<\lambda})$.

\medskip

{\bf (e)(i)} There is an $a\in A\setminus R^{-1}[B]$;  now
$(a,I)\notin R^{\strprime}$ for any $I\in[B]^{<\kappa}$, so
$\add(A,R^{\strprime},[B]^{<\kappa})\le 1$.

\medskip

\quad{\bf (ii)} For every $a\in A$ there is a $b_a\in B$ such that
$(a,b_a)\in R$.   If $A'\subseteq A$ and $\#(A')<\kappa$, then
$I=\{b_a:a\in A'\}$ belongs to $[B]^{<\kappa}$, and
$(a,I)\in R^{\strprime}$ for every $a\in A'$;  as $A'$ is arbitrary,
$\add(A,R^{\strprime},[B]^{<\kappa})\ge\kappa$.

\medskip

{\bf (f)} If $\lambda=\cov(A,R^{\strprime},[B]^{\le\kappa})$ is not
$\infty$, let
$\Cal D\subseteq[B]^{\le\kappa}$ be a set of size $\lambda$ such that
$A=(R^{\strprime})^{-1}[\Cal D]$, and set $D=\bigcup\Cal D$;  then
$A\subseteq R^{-1}[D]$, so
$\cov(A,R,B)\le\#(D)\le\max(\omega,\kappa,\lambda)$.

If $\kappa\ge 1$, then
$\cov(A,R^{\strprime},[B]^{\le\kappa})\le\cov(A,R,B)$, by (c) and 512Da,
so if the latter is greater than $\max(\kappa,\omega)$ they are equal1111.
}%end of proof of 512G

\leader{512H}{Simple products (a)} If $\familyiI{(A_i,R_i,B_i)}$ is any
family of supported relations, its
{\bf simple product} is $(\prod_{i\in I}A_i,T,\prod_{i\in I}B_i)$ where
$T=\{(a,b):(a(i),b(i))\in R_i$ for every $i\in I\}$.

\spheader 512Hb Let $\familyiI{(A_i,R_i,B_i)}$ and
$\familyiI{(C_i,S_i,D_i)}$ be two families of supported relations, with
simple products $(A,R,B)$ and $(C,S,D)$.   If
$(A_i,R_i,B_i)\prGT(C_i,S_i,D_i)$ for every $i$, then
$(A,R,B)\prGT(C,S,D)$.   \prooflet{\Prf\ For each $i$, let
$(\phi_i,\psi_i)$ be a Galois-Tukey connection from $(A_i,R_i,B_i)$ to
$(C_i,S_i,D_i)$.   Define $\phi:A\to C$ and $\psi:D\to B$ by setting
$\phi(\familyiI{a_i})=\familyiI{\phi_i(a_i)}$,
$\psi(\familyiI{d_i})=\familyiI{\psi_i(d_i)}$ for $\familyiI{a_i}\in A$,
$\familyiI{d_i}\in D$;  then

$$\eqalign{(\phi(\familyiI{a_i}),\familyiI{d_i})\in S
&\Longrightarrow (\phi_i(a_i),d_i)\in S_i\text{ for every }i\in I\cr
&\Longrightarrow (a_i,\psi_i(d_i))\in R_i\text{ for every }i\in I\cr
&\Longrightarrow (\familyiI{a_i},\psi(\familyiI{d_i}))\in R.\cr}$$

\noindent So $(\phi,\psi)$ is a Galois-Tukey connection and
$(A,R,B)\prGT(C,S,D)$.\ \Qed}%end of prooflet

\spheader 512Hc Let $\familyiI{(A_i,R_i,B_i)}$ be a family of supported
relations with simple product $(A,R,B)$.   Suppose that no $A_i$ is
empty.   Then $\add(A,R,B)=\min_{i\in I}\add(A_i,R_i,B_i)$, interpreting
$\min\emptyset$ as $\infty$ if $I=\emptyset$.   \prooflet{\Prf\ Set
$\kappa=\add(A,R,B)$ and
$\kappa'=\min_{i\in I}\add(A_i,R_i,B_i)$.   If $I=\emptyset$ then
$A=B=\{\emptyset\}$ and $R=\{(\emptyset,\emptyset)\}$ so
$\add(A,R,B)=\infty$.    Otherwise, if $C\subseteq A$ and
$\#(C)<\kappa'$, then, for each $i$,
$\#(\{c(i):c\in C\})<\add(A_i,R_i,B_i)$, so there is a $b_i\in B_i$ such
that $(c(i),b_i)\in R_i$ for every $c\in C$;  now
$(c,\familyiI{b_i})\in R$ for every $c\in C$;  as $C$ is arbitrary,
$\kappa\ge\kappa'$.   In the other direction, if $i\in I$ and
$C'\in[A_i]^{<\kappa}$, then (because no $A_j$ is empty) there is a
$C\in[A]^{<\kappa}$ such that $C'=\{c(i):c\in C\}$.   Now there is a
$b\in B$ such that $(c,b)\in R$ for every $c\in C$, so that
$(c',b(i))\in R_i$ for every $c'\in C'$.  As $i$ and $C'$ are arbitrary,
$\kappa'\le\kappa$.\ \Qed}

\spheader 512Hd Suppose that $(A,R,B)$ and $(C,S,D)$ are supported
relations with simple product $(A\times C,T,B\times D)$.   Let $\kappa$
be an infinite cardinal and define $(A,R^{\strprime},[B]^{<\kappa})$,
$(C,S^{\strprime},[D]^{<\kappa})$ and
$(A\times C,T^{\strprime},[B\times D]^{<\kappa})$ as in 512F.   Then

\Centerline{$(A,R^{\strprime},[B]^{<\kappa})
  \times(C,S^{\strprime},[D]^{<\kappa})
\equivGT(A\times C,T^{\strprime},[B\times D]^{<\kappa})$.}

\prooflet{\noindent\Prf\ Express
$(A,R^{\strprime},[B]^{<\kappa})\times(C,S^{\strprime},[D]^{<\kappa})$
as $(A\times C,\tilde T,[B]^{<\kappa}\times[D]^{<\kappa})$.

\medskip

\quad{\bf (i)} Set
$\phi(a,c)=(a,c)$ for all $a\in A$, $c\in C$, and for
$I\in[B\times D]^{<\kappa}$ set

\Centerline{$\psi(I)
=(\pi_1[I],\pi_2[I])\in[B]^{<\kappa}\times[D]^{<\kappa}$,}

\noindent where $\pi_1(b,d)=b$ and $\pi_2(b,d)=d$ for $b\in B$,
$d\in D$.   If $a\in A$, $c\in C$ and $I\in[B\times D]^{<\kappa}$ are
such that $(\phi(a,c),I)\in T^{\strprime}$, then there must be a
$(b,d)\in I$ such that $((a,c),(b,d))\in T$, that is, $(a,b)\in R$ and
$(c,d)\in S$;  now $b\in\pi_1[I]$ and $d\in\pi_2[I]$, so
$(a,\pi_1[I])\in R^{\strprime}$ and $(c,\pi_2[I])\in S^{\strprime}$ and

\Centerline{$((a,c),\psi(I))
=((a,c),(\pi_1[I],\pi_2[I]))\in\tilde T$.}

\noindent As $a$, $c$ and $I$ are arbitrary, $(\phi,\psi)$ is a
Galois-Tukey connection and

\Centerline{$(A,R^{\strprime},[B]^{<\kappa})
  \times(C,S^{\strprime},[D]^{<\kappa})
\prGT(A\times C,T^{\strprime},[B\times D]^{<\kappa})$.}

\medskip

\quad{\bf (ii)} In the other direction, given
$(J,K)\in[B]^{<\kappa}\times[D]^{<\kappa}$ set
$\psi'(J,K)=J\times K\in[B\times D]^{<\kappa}$.   (This is where we need
to suppose that $\kappa$ is infinite.)   If now
$(\phi(a,c),(J,K))\in\tilde T$, that is,
$(a,J)\in R^{\strprime}$ and $(c,K)\in S^{\strprime}$, there are
$b\in J$ and $d\in K$ such that $(a,b)\in R$ and $(c,d)\in S$, so that
$((a,c),(b,d))\in T$ and $((a,c),\psi'(J,K))\in T^{\strprime}$.
As $a$, $c$, $J$ and $K$ are arbitrary, $(\phi,\psi')$ is a
Galois-Tukey connection and

\Centerline{$(A\times C,T^{\strprime},[B\times D]^{<\kappa})
\prGT(A,R^{\strprime},[B]^{<\kappa})
\times(C,S^{\strprime},[D]^{<\kappa})$.\ \Qed}
}%end of prooflet

\spheader 512He If $\familyiI{(P_i,\le_i)}$ is a family of
pre-ordered sets, with product $(P,\le)$\cmmnt{ (511A)},
then $(P,\le\nobreak,P)$ is just
\penalty-100$\prod_{i\in I}(P_i,\le_i\nobreak,P_i)$ in the sense here.

\vleader{72pt}{512I}{Sequential compositions}
Let $(A,R,B)$ and $(C,S,D)$ be supported relations.   Their {\bf
sequential composition} $(A,R,B)\ltimes(C,S,D)$ is
$(A\times C^B,T,B\times D)$, where

\Centerline{$T
=\{((a,f),(b,d)):(a,b)\in R,\,f\in C^B,\,(f(b),d)\in S\}$.}

\noindent Their {\bf dual sequential composition}
$(A,R,B)\rtimes(C,S,D)$ is $(A\times C,\tilde T,B\times D^A)$ where

$$\eqalign{\tilde T
&=\{((a,c),(b,g)):a\in A,\,b\in B,\,c\in C,\,g\in D^A\cr
&\mskip150mu
  \text{and either }(a,b)\in R\text{ or }(c,g(a))\in S\}.\cr}$$

\leader{512J}{Proposition} Let $(A,R,B)$ and $(C,S,D)$ be supported
relations.

(a)
$(A,R,B)\rtimes(C,S,D)=((A,R,B)^{\perp}\ltimes(C,S,D)^{\perp})^{\perp}$.

(b) $\cov((A,R,B)\ltimes(C,S,D))$ is the cardinal product
$\cov(A,R,B)\cdot\cov(C,S,D)$ unless
$B=C=\emptyset\ne A$, if we use the interpretations

\Centerline{$0\cdot\infty=\infty\cdot 0=0$,
\quad$\kappa\cdot\infty=\infty\cdot\kappa=\infty\cdot\infty=\infty$ for
every cardinal $\kappa\ge 1$.}

(c) $\add((A,R,B)\ltimes(C,S,D))=\min(\add(A,R,B),\add(C,S,D))$ unless
$A\times C=\emptyset\ne B\times D$.

\proof{{\bf (a)} is just a matter of disentangling the
definitions.

\medskip

{\bf (b)} Define $T\subseteq (A\times C^B)\times(B\times D)$ as in 512I.

\medskip

\quad{\bf (i)}
Suppose first that neither $A$ nor $C$ is empty, that
$A\subseteq R^{-1}[B]$ and that $C\subseteq S^{-1}[D]$.   If
$B_0\subseteq B$ and $D_0\subseteq D$ are such that
$A\subseteq R^{-1}[B_0]$ and $C\subseteq S^{-1}[D_0]$, then for any
$a\in A$ and $f\in C^B$ there are $b\in B_0$ and $d\in D_0$ such that
$(a,b)\in R$ and $(f(b),d)\in S$, so that
$(a,f)\in T^{-1}[B_0\times D_0]$.   So
$\cov((A,R,B)\ltimes(C,S,D))\le\cov(A,R,B)\cdot\cov(C,S,D)$.

On the other hand, if $H\subseteq B\times D$ is such that
$A\times C^B\subseteq T^{-1}[H]$, set
$B_0=\{b:C\subseteq S^{-1}[H[\{b\}]]\}$.   Then
$\#(H[\{b\}])\ge\cov(C,S,D)$ for $b\in B_0$.   Also
$A\subseteq R^{-1}[B_0]$.   \Prf\Quer\ Otherwise, take
$a\in A\setminus R^{-1}[B_0]$.   For $b\in B\setminus B_0$, choose
$f(b)\in C\setminus S^{-1}[H[\{b\}]]$;  for $b\in B_0$,
take $f(b)$ to be any member of $C$.   There is supposed to be a
member
$(b,d)$ of $H$ such that $((a,f),(b,d))\in T$, that is, $(a,b)\in R$ and
$(f(b),d)\in S$.   But now $b\notin B_0$, by the choice of $a$, and
$(f(b),d)\notin S$, by the choice of $f$;  so we have a
contradiction.\
\Bang\Qed

So $\#(B_0)\ge\cov(A,R,B)$ and $\#(H)\ge\cov(A,R,B)\cdot\cov(C,S,D)$;
as $H$ is arbitrary,
$\cov((A,R,B)\ltimes(C,S,D))\ge\cov(A,R,B)\cdot\cov(C,S,D)$.

\medskip

\quad{\bf (ii)}
If $A=\emptyset$ then $A\times C^B=\emptyset$ so $\cov(A,R,B)$ and
$\cov((A,R,B)\ltimes(C,S,D))$ are both zero.   If $C=\emptyset$ and
$B\ne\emptyset$ then $\cov(C,S,D)=\cov((A,R,B)\ltimes(C,S,D))=0$.
If $A$ and $C$ are non-empty and $A\not\subseteq R^{-1}[B]$, then
$A\times C^B\not\subseteq T^{-1}[B\times D]$, so
$\cov(A,R,B)=\cov((A,R,B)\ltimes\cov(C,S,D))=\infty$, while
$\cov(C,S,D)\ge 1$.   If $A$ and $C$ are non-empty and
$A\subseteq R^{-1}[B]$ and $C\not\subseteq S^{-1}[D]$, then
$B\ne\emptyset$;  if we take $c\in C\setminus S^{-1}[D]$ and any member
$a$ of $A$, and set $f(b)=c$ for every $b\in B$, then
$(a,f)\notin T^{-1}[B\times D]$, so
$\cov((A,R,B)\ltimes(C,S,D))=\cov(C,S,D)=\infty$, while
$\cov(A,R,B)\ge 1$.   So with the single exception of
$B=C=\emptyset\ne A$ (in which case the empty function belongs to $C^B$,
so that $\cov((A,R,B)\ltimes(C,S,D))=\infty$, while $\cov(C,S,D)=0$) we
have $\cov((A,R,B)\ltimes(C,S,D))=\cov(A,R,B)\cdot\cov(C,S,D)$.

\medskip

{\bf (c)} Assume throughout that either $A\times C\ne\emptyset$ (so that
$A\times C^B\ne\emptyset$) or that $B\times D=\emptyset$.

\medskip

\quad{\bf (i)} $\add((A,R,B)\ltimes(C,S,D))\le\add(A,R,B)$.   \Prf\ If
$\add(A,R,B)=\infty$ the result is trivial.   If $B\times D=\emptyset$
then $\add((A,R,B)\ltimes(C,S,D))=0\le\add(A,R,B)$.   Otherwise, our
hypothesis ensures that $C$ is not empty;  take
$A_0\subseteq A$ such that $\#(A_0)=\add(A,R,B)$ and
$A_0\not\subseteq R^{-1}[\{b\}]$ for any $b\in B$, take any $b_0\in B$
and any $f_0\in C^B$;  then there is no $(b,d)\in B\times D$ such that
$((a,f_0(b_0)),(b,d))\in T$ for every $a\in A_0$, so
$\add((A,R,B)\ltimes(C,S,D))\le\#(A_0)=\add(A,R,B)$.\ \Qed

\medskip

\quad{\bf (ii)} $\add((A,R,B)\ltimes(C,S,D))\le\add(C,S,D)$.   \Prf\
Again, if $\add(C,S,D)=\infty$ or $B\times D=\emptyset$ the result is
immediate.   Otherwise, $A\ne\emptyset$.   Take
$C_0\subseteq C$ such that $\#(C_0)=\add(C,S,D)$ and there is no
$d\in D$ such that $C_0\subseteq S^{-1}[\{d\}]$, for
$c\in C_0$ set $f_c(b)=c$ for every $b\in B$, and fix any $a_0\in A$;
then there is no $(b,d)\in B\times D$ such that
$((a_0,f_c),(b,d))\in T$ for every $c\in C_0$, so
$\add((A,R,B)\ltimes(C,S,D))\le\#(C_0)=\add(C,S,D)$.\ \Qed

\medskip

\quad{\bf (iii)}
$\add((A,R,B)\ltimes(C,S,D))\ge\min(\add(A,R,B),\add(C,S,D))$.
\Prf\ If
$H\subseteq A\times C^B$ and $\#(H)$ is less than
$\min(\add(A,R,B),\add(C,S,D))$, set
$A_0=\{a:(a,f)\in H\}$ and $F=\{f:(a,f)\in H\}$.
Then there are a $b\in B$ such that $(a,b)\in R$ for any $a\in A_0$,
and a $d\in D$ such that $(f(b),d)\in S$ for any $f\in F$, so that
$((a,f),(b,d))\in T$ for any $(a,f)\in H$.   As $H$ is arbitrary,
$\add((A,R,B)\ltimes(C,S,D))\ge\min(\add(A,R,B),\add(C,S,D))$.\ \Qed
}%end of proof of 512J

\leader{512K}{}\cmmnt{ The following fact will be used in \S526.
%526F

\medskip

\noindent}{\bf Lemma} Suppose that $(A,R,B)$ and $(C,S,D)$ are supported
relations, and $P$ is a partially ordered set.   Suppose that
$\family{p}{P}{A_p}$ is a family of subsets of $A$ such that

\Centerline{$(A_p,R,B)\prGT(C,S,D)$ for every $p\in P$,}

\Centerline{$A_p\subseteq A_q$ whenever $p\le q$ in $P$,
\quad$\bigcup_{p\in P}A_p=A$.}

\noindent Then $(A,R,B)\prGT(P,\le,P)\ltimes(C,S,D)$.

\proof{ If $C=\emptyset$ the result is trivial, since every $A_p$ is
empty and $B$ can be empty only if $D$ is.   So we may suppose that
$C\ne\emptyset$.   For each $p\in P$, let $(\phi_p,\psi_p)$ be a
Galois-Tukey
connection from $(A_p,R,B)$ to $(C,S,D)$.   For $a\in A$, let
$r(a)\in P$ be such that $a\in A_{r(a)}$, and set $f_a(p)=\phi_p(a)$
whenever $p\in P$ and $a\in A_p$;  for other $p\in P$ take $f_a(p)$ to
be any member of $C$.   Set $\phi(a)=(r(a),f_a)$ for $a\in A$.   For
$q\in P$, $d\in D$ set $\psi(q,d)=\psi_q(d)\in B$.   Now $(\phi,\psi)$
is a Galois-Tukey connection from $(A,R,B)$ to
$(P,\le,P)\ltimes(C,S,D)$.
\Prf\ Suppose that $a\in A$ and $(q,d)\in P\times D$ are such that
$r(a)\le q$ and $(f_a(q),d)\in S$.   Then $a\in A_{r(a)}\subseteq A_q$
so $f_a(q)=\phi_q(a)$.   Because $(\phi_q,\psi_q)$ is a Galois-Tukey
connection, $(a,\psi(q,d))=(a,\psi_q(d))\in R$.\ \Qed

So we have the result.
}%end of proof of 512K

\exercises{\leader{512X}{Basic exercises (a)}%
%\spheader 512Xa
(i) Suppose that $A\subseteq A'$, $B'\subseteq B$ and
that $R$ is any relation.   Show that $(A,R,B)\prGT(A',R,B')$.
(ii) Show that $(\emptyset,\emptyset,\{\emptyset\})\prGT(A,R,B)
\prGT(\{\emptyset\},\emptyset,\emptyset)$ for every supported relation
$(A,R,B)$.
%512A

\spheader 512Xb Let $(A,R,B)$ be any supported relation.   Show that
$\sat(A,R,B)\le(\link_2(A,R,B))^+$.
%512B

\spheader 512Xc\dvAformerly{5{}14Xa}
Let $(X,\frak T)$ be a topological space and
$(Y,\frak T_Y)$ an open subspace.    Show that
$(\frak T_Y\setminus\{\emptyset\},\supseteq,
  \frak T_Y\setminus\{\emptyset\})
\prGT(\frak T\setminus\{\emptyset\},\supseteq,
  \frak T\setminus\{\emptyset\})$.
%5A4B 512E

\spheader 512Xd Let $(X,\frak T)$ and $(Y,\frak S)$ be topological spaces.
(i) Show that if $Y$ is a continuous image of $X$,
$(\frak S\setminus\{\emptyset\},\penalty-50\ni\nobreak,Y)\penalty-100
   \prGT(\frak T\setminus\{\emptyset\},\ni,X)$.
(ii) Show that if $X$ and $Y$ are compact and Hausdorff and
there is an irreducible continuous surjection from $X$ onto $Y$, then
$(\frak T\setminus\{\emptyset\},\ni,X)\equivGT
(\frak S\setminus\{\emptyset\},\ni,Y)$ and
$(\frak T\setminus\{\emptyset\},\supseteq,\frak T\setminus\{\emptyset\})
\equivGT
(\frak S\setminus\{\emptyset\},\supseteq,\frak S\setminus\{\emptyset\})$,
so $d(X)=d(Y)$ and $\pi(X)=\pi(Y)$.
%512E 512Xc

\spheader 512Xe Let $\familyiI{(A_i,R_i,B_i)}$ be a family of supported
relations with simple product $(A,R,B)$.   Show that $(A,R,B)^{\perp}$
can be naturally identified with the simple product of
$\familyiI{(A_i,R_i,B_i)^{\perp}}$.
%512H

\spheader 512Xf Let $(A,R,B)$ be a supported relation and $\kappa>0$ a
cardinal.   Show that
$(A,R^{\strprime},[B]^{\le\kappa})\prGT(A,R,B)^{\kappa}$,
where $(A,R,B)^{\kappa}$ is the simple product of $\kappa$ copies of
$(A,R,B)$ and $R^{\strprime}=\{(a,J):a\in R^{-1}[J]\}$ as
usual.
%512H

\spheader 512Xg Let $(A,R,B)$ and $(C,S,D)$ be supported relations, and
$(A\times C,T,B\times D)$ their simple product.   (i) Show that if
$C\ne\emptyset$, then
$(A,R,B)\prGT(A\times C,T,B\times D)$.   (ii) Show that
$(A\times C,T,B\times D)\prGT(A,R,B)\ltimes(C,S,D)$.
(iii) Show that (using the conventions of 512Jb)
$\cov(A\times C,T,B\times D)=\cov(A,R,B)\cdot\cov(C,S,D)$.
%512I

\spheader 512Xh Let $(A_0,R_0,B_0)$, $(A_1,R_1,B_1)$, $(C_0,S_0,D_0)$
and $(C_1,S_1,D_1)$ be supported relations such
that \ifdim\pagewidth>467pt\break\fi
$(A_0,R_0,B_0)\prGT(A_1,R_1,B_1)$ and
$(C_0,S_0,D_0)\prGT(C_1,S_1,D_1)$.   Show that

\Centerline{$(A_0,R_0,B_0)\ltimes(C_0,S_0,D_0)
\prGT(A_1,R_1,B_1)\ltimes(C_1,S_1,D_1)$,}

\Centerline{$(A_0,R_0,B_0)\rtimes(C_0,S_0,D_0)
\prGT(A_1,R_1,B_1)\rtimes(C_1,S_1,D_1)$.}
%512J
}%end of exercises

\endnotes{
\Notesheader{512} Much of this section is cluttered by
the repeated names $(A,R,B)$ of `supported relations'.   In fact these
could probably be dispensed with.   While I am reluctant to alter the
general convention I use in this book, that a `relation' is neither more
nor less than a class of ordered pairs, it is clear that in all
significant cases our supported relation $(A,R,B)$ will be such that
$A=\{a:(a,b)\in R\}$ and $B=\{b:(a,b)\in R\}$, so that $A$ and $B$ can
be recovered from the set $R$.   But this would make impossible the very
useful convention that `$(X,\in,\Cal A)$' is to be interpreted as
`$(X,\{(x,A):A\in\Cal A,\,x\in X\cap A\},\Cal A)$', and since nearly
every mathematical argument in this context demands names for the
domains and codomains of the relations, it seems easier to write these
in each time.

An important feature of the theory here is that while it is very common
for our relations to be reasonably well-behaved by some
criterion (for instance, we may have Polish spaces $A$ and
$B$ and a coanalytic set $R\subseteq A\times B$), the functions in a
Galois-Tukey connection are not required to have any properties beyond
those declared in the definition.   Of course the most important
Galois-Tukey
connections are those which are `natural' in some sense, and are
constructed in a way which does not involve totally unscrupulous use
of the axiom of choice.
I will return to this question in the next section.
}%end of notes

\discrpage

