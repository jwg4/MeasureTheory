\frfilename{mt381.tex}
\versiondate{19.7.06}
\copyrightdate{2003}

\def\End{\mathop{\text{End}}}
\def\cycleii#1#2#3{\cycle{#1\,_{#2}\,#3}}

\def\chaptername{Automorphisms}
\def\sectionname{Automorphisms of Boolean algebras}

\newsection{381}

I begin the chapter with a preparatory section of definitions (381B) and
mostly elementary facts.   A fundamental method of constructing
automorphisms is in 381C-381D.   The idea of `support' of an
endomorphism is explored in  381E-381G,   %381E 381G 381F
a first look at `periodic' and `aperiodic' parts is in 381H, and basic
facts about `full subgroups' are in 381I-381J.   We start to go deeper
with the notion of `recurrence', treated in
381L-381P.  %381L 381M 381N 381P
I describe how these phenomena appear when we represent an endomorphism
as a map on the Stone space of an algebra (381Q).   I end by introducing
a `cycle notation' for certain automorphisms.

\leader{381A}{The group $\Aut\frak A$} For any Boolean algebra
$\frak A$, I write $\Aut\frak A$ for the set of automorphisms of
$\frak A$, that is, the set of bijective Boolean homomorphisms
$\pi:\frak A\to\frak A$.   This is a group\cmmnt{, being a subgroup of
the group
of all permutations of $\frak A$ (use 312G)}.   Note that
every member
of $\Aut\frak A$ is order-continuous\cmmnt{;  this is because it must
be an isomorphism of the order structure of $\frak A$ (313Ld)}.

%definitions
\leader{381B}{}\cmmnt{ The primary aim of this chapter is to study
automorphisms of probability algebras.   In the context of the present
section, this means that for a first reading you can take it that all
algebras are Dedekind complete.   The methods can however be used in
many other contexts, at the price of complicating some of the statements
of the lemmas.   It is also interesting, and occasionally important, to
apply some of the ideas to general Boolean homomorphisms.   In the
following definitions I try to set out a language to make this possible.

\medskip

\noindent}{\bf Definitions (a)}
%\spheader 381Ba
If $\frak A$ is a Boolean algebra and
$\pi:\frak A\to\frak A$ is a Boolean homomorphism, \cmmnt{I say that}
$a\in\frak A$ {\bf supports} $\pi$ if $\pi d=d$ for every
$d\Bsubseteq 1\Bsetminus a$.

\spheader 381Bb Let $\frak A$ be a Boolean algebra and
$\pi:\frak A\to\frak A$ a Boolean homomorphism.   If
$\min\{a:a\in\frak A$ supports $\pi\}$ is
defined in $\frak A$, I will call it the {\bf support} $\supp\pi$ of
$\pi$.

\spheader 381Bc If $\frak A$ is a Boolean algebra, an automorphism
$\pi:\frak A\to\frak A$ is
{\bf periodic}, with {\bf period} $n\ge 1$, if $\frak A\ne\{0\}$,
$\pi^n$ is the identity operator and $1$ is the support of $\pi^i$
whenever $1\le i<n$.

\spheader 381Bd If $\frak A$ is a Boolean algebra, a Boolean
homomorphism $\pi:\frak A\to\frak A$
is {\bf aperiodic} if the support of $\pi^n$ is $1$ for every $n\ge 1$.
I remark immediately that if $\pi$ is aperiodic, so is $\pi^n$ for every
$n\ge 1$\cmmnt{ (see 381Xc)}.   \cmmnt{Note
that if $\frak A=\{0\}$ then the trivial automorphism of $\frak A$ is
counted as aperiodic.}

\spheader 381Be If $\frak A$ is a Boolean
algebra, a subgroup $G$ of $\Aut\frak A$ is {\bf full} if whenever
$\langle a_i\rangle_{i\in I}$ is a partition of
unity in $\frak A$, $\langle\pi_i\rangle_{i\in I}$ is a family in
$G$, and $\pi\in\Aut\frak A$ is such that $\pi d=\pi_id$ whenever
$i\in I$ and $d\Bsubseteq a_i$, then $\pi\in G$.

\spheader 381Bf If $\frak A$ is a Boolean
algebra, a subgroup $G$ of $\Aut\frak A$ is {\bf countably full} if
whenever $\langle a_i\rangle_{i\in I}$ is a countable partition of
unity in $\frak A$, $\langle\pi_i\rangle_{i\in I}$ is a family in
$G$, and $\pi\in\Aut\frak A$ is such that $\pi d=\pi_id$ whenever
$i\in I$ and $d\Bsubseteq a_i$, then $\pi\in G$.

\spheader 381Bg If $\frak A$ is a Boolean algebra, $a\in\frak A$ and
$\pi:\frak A\to\frak A$ is a Boolean homomorphism, I say that $\pi$ is
{\bf recurrent on $a$} if for every non-zero $b\Bsubseteq a$ there is a
$k\ge 1$ such that $a\Bcap\pi^kb\ne 0$.
If $\pi\in\Aut\frak A$ and $\pi$ and $\pi^{-1}$ are both recurrent on
$a$, I say that $\pi$ is {\bf doubly recurrent} on $a$.

%piecing automorphisms
\leader{381C}{}\cmmnt{ Before setting out to explore the concepts just
listed, I give a fundamental result on piecing automorphisms together
from fragments.

\medskip

\noindent}{\bf Lemma} Let $\frak A$ be a Boolean algebra, and
$\langle a_i\rangle_{i\in I}$, $\langle b_i\rangle_{i\in I}$ two
partitions of unity in $\frak A$.   Assume

\quad{\it either} that $I$ is finite

\quad{\it or} that $I$ is countable and $\frak A$ is Dedekind
$\sigma$-complete

\quad{\it or} that $\frak A$ is Dedekind complete.

\noindent Suppose that for each $i\in I$ we have an isomorphism
$\pi_i:\frak A_{a_i}\to\frak A_{b_i}$ between the corresponding
principal ideals.   Then there is a unique $\pi\in\Aut\frak A$ such that
$\pi d=\pi_id$ whenever $i\in I$ and $d\Bsubseteq a_i$.

\proof{ By 315F, we may identify $\frak A$ with each of the products
$\prod_{i\in I}\frak A_{a_i}$, $\prod_{i\in I}\frak A_{b_i}$;   now
$\pi$ corresponds to the isomorphism between the two products induced by
the $\pi_i$.
}%end of proof of 381C

\leader{381D}{Corollary} Let $\frak A$ be a homogeneous Boolean
algebra,
and $A$, $B$ two partitions of unity in $\frak A$, neither containing
$0$.   Let $\theta:A\to B$ be a bijection.   Suppose

\quad{\it either} that $A$, $B$ are finite

\quad{\it or} that $A$, $B$ are countable and $\frak A$ is Dedekind
$\sigma$-complete

\quad{\it or} that $\frak A$ is Dedekind complete.

\noindent Then there is an automorphism of $\frak A$ extending $\theta$.

\proof{ For every $a\in A$, the principal ideals $\frak A_a$,
$\frak A_{\theta a}$ are isomorphic to the whole algebra $\frak A$, and
therefore to each other;  let $\pi_a:\frak A_a\to\frak A_{\theta a}$ be
an isomorphism.   Now apply 381C.
}%end of proof of 381D

%supports
\leader{381E}{Lemma} Let $\frak A$ be a Boolean algebra, and $\pi$,
$\phi$, $\psi:\frak A\to\frak A$ Boolean homomorphisms of which $\pi$ is
injective.

(a) If $a\in\frak A$ supports $\phi$ then $\phi a=a$ and
$\phi d\Bsubseteq a$ for every $d\Bsubseteq a$.

(b) If $a\in\frak A$ supports both $\phi$ and $\psi$ then it supports
$\phi\psi$.

(c) Let $A$ be the set of elements of $\frak A$
supporting $\phi$.   Then $A$ is non-empty and closed under $\Bcap$;
also $b\in A$ whenever $b\Bsupseteq a\in A$.   If $\phi$ is
order-continuous, then $\inf B\in A$ whenever $B\subseteq A$ has an
infimum in $\frak A$.

(d) If $a\in\frak A$ supports $\pi\phi$, then $\phi a$ supports
$\pi\phi$.

(e) If $\pi$ commutes with $\phi$, and $a\in\frak A$ is such that
$\pi a$ supports $\phi$, then $a$ supports $\phi$.

(f) If $\phi$ is supported by $a$ and $\psi$ is supported by $b$, where
$a\Bcap b=0$, then $\phi\psi=\psi\phi$.

(g) For any $n\ge 1$ and $a\in\frak A$, $a$ supports $\pi^n$ iff $\pi a$
supports $\pi^n$.   \cmmnt{Consequently} $\pi(\supp\pi^n)=\supp\pi^n$
if $\pi^n$ has a support.

(h) If $\pi\in\Aut\frak A$ and $a\in\frak A$ supports $\pi$, then $a$
supports $\pi^{-1}$.

(i) If $\pi\in\Aut\frak A$ and $a\in\frak A$, then $a$ supports $\pi$
iff $d\Bsubseteq a$ whenever $d\Bcap\pi d=0$ iff $d\Bcap\pi d\ne 0$
whenever $0\ne d\Bsubseteq 1\Bsetminus a$.

(j) If $\pi\in\Aut\frak A$ and $a\in\frak A$ supports $\phi$, then
$\pi a$ supports $\pi\phi\pi^{-1}$.

(k) If $a\in\frak A$ supports $\phi$, and $\pi_1$, $\pi_2\in\Aut\frak A$
agree on $\frak A_a$, then $\pi_1\phi\pi_1^{-1}=\pi_2\phi\pi_2^{-1}$.

\proof{{\bf (a)} $\phi(1\Bsetminus a)=1\Bsetminus a$, so $\phi a=a$, and
if $d\Bsubseteq a$ then $\phi d\Bsubseteq\phi a=a$.

\medskip

{\bf (b)} If $d\Bcap a=0$ then $\phi d=d=\psi d$ so $\phi\psi d=d$.

\medskip

{\bf (c)} Of course $1\in A$, because $\phi 0=0$;  and it is also
obvious that if $b\Bsupseteq a\in A$ then $b\in A$.   If $a$, $b\in A$
and $d\Bcap a\Bcap b=0$, then
$\phi d=\phi(d\Bsetminus a)\Bcup\phi(d\Bsetminus b)=d$.
If $\phi$ is order-continuous, $B\subseteq A$ is
non-empty and $c=\inf B$ is defined in $\frak A$, then for any
$d\Bsubseteq 1\Bsetminus c$ we have

\Centerline{$d=d\Bsetminus c=\sup_{b\in B}d\Bsetminus b$,}

\noindent and

\Centerline{$\phi d=\sup_{b\in B}\phi(d\Bsetminus b)
=\sup_{b\in B}d\Bsetminus b=d$.}

\noindent So in this case $c$ supports $\phi$.


\medskip

{\bf (d)} If $d\Bcap\phi a=0$ then
$\pi d\Bcap a=\pi d\Bcap\pi\phi a=0$, so $\pi\phi\pi d=\pi d$ and
(because $\pi$ is injective) $\phi\pi d=d$.

\medskip

{\bf (e)} If $d\Bcap a=0$ then $\pi d\Bcap\pi a=0$, so
$\pi\phi d=\phi\pi d=\pi d$ and $\phi d=d$.

\medskip

{\bf (f)} If $d\Bsubseteq a$ then $\phi d\Bsubseteq a$ and $\psi d=d$ so
$\psi\phi d=\phi d=\phi\psi d$;  if $d\Bsubseteq b$ then
$\psi\phi d=\phi\psi d=\psi d$;  and if
$d\Bsubseteq 1\Bsetminus(a\Bcup b)$ then $\psi\phi d=\phi\psi d=d$.

\medskip

{\bf (g)} Because $\pi$ is injective, so is $\pi^{n-1}$.   So if $a$
supports $\pi^n=\pi^{n-1}\pi$, so does $\pi a$, by (d).   On the other
hand, $\pi$ commutes with $\pi^n$, so if $\pi a$ supports $\pi^n$ so
does $a$, by (e).

If $c=\supp\pi^n$ then $\pi c$ supports $\pi^n$ so $c\Bsubseteq\pi c$.
Consequently $\pi^ic\Bsubseteq\pi^{i+1}c$ for every $i\in\Bbb N$ and
$c\Bsubseteq\pi c\Bsubseteq\pi^nc$.   But as $\pi^nc=c$, by (a), $\pi c=c$.


\medskip

{\bf (h)} If $d\Bcap a=0$ then $\pi d=d$ so $d=\pi^{-1}d$.

\medskip

{\bf (i)}\grheada\ If $a$ supports $\pi$ and $d\Bcap\pi d=0$, then

\Centerline{$d\Bcap(d\Bsetminus a)
=d\Bcap\pi(d\Bsetminus a)\Bsubseteq d\Bcap\pi d=0$,}

\noindent so $d\Bsubseteq a$.

\medskip

\quad\grheadb\ If $d\Bsubseteq a$ whenever $d\Bcap\pi d=0$, and
$0\ne d'\Bsubseteq 1\Bsetminus a$, then of course $d'\Bcap\pi d'\ne 0$.

\medskip

\quad\grheadc\ If $a$ does not support $\pi$, there is a
$c\Bsubseteq 1\Bsetminus a$ such that $\pi c\ne c$.   So one of
$c\Bsetminus\pi c$, $\pi c\Bsetminus c$ is non-zero.   If
$c\Bsetminus\pi c\ne 0$, take this for $d$;  then
$d\Bsubseteq 1\Bsetminus a$ and
$\pi d\Bcap d\Bsubseteq\pi c\Bsetminus\pi c=0$.   Otherwise, take
$d=\pi^{-1}(\pi c\Bsetminus c)$;  then
$0\ne d\Bsubseteq c\Bsubseteq 1\Bsetminus a$, while

\Centerline{$d\Bcap\pi d
=(c\Bsetminus\pi^{-1}c)\Bcap(\pi c\Bsetminus c)=0$.}

\medskip

{\bf (j)} If $d\Bcap\pi a=0$ then $\pi^{-1}d\Bcap a=0$ so
$\phi\pi^{-1}d=\pi^{-1}d$ and $\pi\phi\pi^{-1}d=d$.

\medskip

{\bf (k)}  For $d\Bsubseteq a$, $\pi_2^{-1}\pi_1 d=\pi_2^{-1}\pi_2 d=d$,
so $\pi_2^{-1}\pi_1$ is supported by $1\Bsetminus a$.   By (f),
$\phi\pi_2^{-1}\pi_1=\pi_2^{-1}\pi_1\phi$, so

\Centerline{$\pi_1\phi\pi_1^{-1}=\pi_2\pi_2^{-1}\pi_1\phi\pi_1^{-1}
=\pi_2\phi\pi_2^{-1}\pi_1\pi_1^{-1}=\pi_2\phi\pi_2^{-1}$.}
}%end of proof of 381E

\leader{381F}{Corollary} If $\frak A$ is a Dedekind complete Boolean
algebra, then every order-continuous Boolean homomorphism
$\phi:\frak A\to\frak A$ has a support.

\proof{ By 381Ec, $\inf\{a:a\in\frak A$ supports $\phi\}$ is the support
of $\phi$.
}%end of proof of 381F

\leader{381G}{Corollary} Let $\frak A$ be a Boolean algebra, and suppose
that $\pi\in\Aut\frak A$ has a support $e$.

(a) $\pi e=e$.

(b) $e=\sup\{d:d\in\frak A$, $d\Bcap\pi d=0\}$.

(c) $e$ is the support of $\pi^{-1}$.

(d) For any $\phi\in\Aut\frak A$, $\phi e$ is the support of
$\phi\pi\phi^{-1}$.

\proof{{\bf (a)} 381Ea.

\medskip

{\bf (b)} 381Ei.

\medskip

{\bf (c)} 381Eh.

\medskip

{\bf (d)} By 381Ej, $\phi e$ supports
$\phi\pi\phi^{-1}$.   At the same time, if $a\in\frak A$ supports
$\phi\pi\phi^{-1}$, then $\phi^{-1}a$ supports $\pi$, so
$e\Bsubseteq\phi^{-1}a$ and $a\Bsupseteq\phi e$.   Thus $\phi e$ is the
smallest element of $\frak A$ supporting $\phi\pi\phi^{-1}$ and is the
support of $\phi\pi\phi^{-1}$.
}%end of proof of 381G

%periodic
\leader{381H}{Proposition} Let $\frak A$ be a Dedekind
$\sigma$-complete Boolean
algebra and $\pi:\frak A\to\frak A$ an injective Boolean homomorphism
such that $\pi^n$ has a support for every $n\in\Bbb N$.   Then there is
a partition of unity $\langle c_i\rangle_{1\le i\le\omega}$ in $\frak A$
such that $\pi c_i\Bsubseteq c_i$ for every $i$,
$\pi\restrp\frak A_{c_n}$ is
periodic with period $n$ whenever $n\in\Bbb N\setminus\{0\}$
and $c_n\ne 0$, and
$\pi\restrp\frak A_{c_{\omega}}$ is aperiodic.

\proof{ Set

$$\eqalign{c_1&=1\Bsetminus\supp\pi,\cr
c_n&=\inf_{i<n}\supp\pi^i\Bsetminus\supp\pi^n\text{ for }n\ge 2,\cr
c_{\omega}&=\inf_{n\in\Bbb N}\supp\pi^n.\cr}$$

\noindent Then $\langle c_i\rangle_{1\le i\le\omega}$ is a partition of
unity.   By 381Eg, $\pi c_n=c_n$ for every $n$, so
$\pi c_{\omega}\Bsubseteq c_{\omega}$.   If $d\Bsubseteq c_n$, where
$1\le n\in\Bbb N$, then $d\Bcap\supp\pi^n=0$ so $\pi^nd=d$.   If
$1\le i<j\le\omega$ and
$0\ne a\Bsubseteq c_j$, then $a\Bsubseteq\supp\pi^i$ so there is a
$d\Bsubseteq a$ such that
$(\pi\restrp\frak A_{c_n})^id=\pi^id\ne d$;  thus if
$n\in\Bbb N\setminus\{0\}$ (and
$c_n\ne 0$) $\pi\restrp\frak A_{c_n}$ is periodic with period $n$, while
$\pi\restrp\frak A_{c_{\omega}}$ is aperiodic.
}%end of proof of 381H

\cmmnt{\medskip

\noindent{\bf Remark} The hypothesis `every $\pi^n$ has a support' will
be satisfied if $\frak A$ is Dedekind complete and $\pi$ is
order-continuous (381F).   For other sufficient conditions see
382E.
}%end of comment

%(countably) full subgroups
\leader{381I}{Full and countably full subgroups} If $\frak A$ is a
Boolean algebra, \cmmnt{it is obvious that} the intersection of any
family of (countably) full subgroups of $\Aut\frak A$ is again
(countably) full.   We may therefore speak of the (countably) full
subgroup of $\frak A$ generated by an element of $\Aut\frak A$.

\medskip

\noindent{\bf Proposition} Let $\frak A$ be a Boolean algebra.

(a)\dvAformerly{3{}81Xe} Let $G$ be a subgroup
of $\Aut\frak A$.   Let $H$ be the set of those $\pi\in\Aut\frak A$ such
that for every non-zero $a\in\frak A$ there are a non-zero
$b\Bsubseteq a$ and a $\phi\in G$ such that $\pi c=\phi c$ for every
$c\Bsubseteq b$.
Then $H$ is a full subgroup of $\Aut\frak A$, the smallest full
subgroup of $\frak A$ including $G$.

(b) Suppose that $\frak A$ is Dedekind $\sigma$-complete and $\pi$,
$\phi\in\Aut\frak A$.   Then the following are equiveridical:

\inset{(i) $\phi$ belongs to the countably full subgroup of
$\Aut\frak A$ generated by $\pi$;

(ii) there is a partition of unity $\langle a_n\rangle_{n\in\Bbb Z}$
in $\frak A$ such that $\phi c=\pi^nc$ whenever $n\in\Bbb Z$ and
$c\Bsubseteq a_n$.}

(c) Suppose that $\frak A$ is Dedekind complete, and $\pi$,
$\phi\in\Aut\frak A$.  Then the following are equiveridical:

\inset{(i) $\phi$ belongs to the full subgroup of
$\Aut\frak A$ generated by $\pi$;

(ii) for every non-zero $a\in\frak A$ there are a non-zero
$b\Bsubseteq a$ and an $n\in\Bbb Z$ such that $\phi c=\pi^nc$
for every $c\Bsubseteq b$;

(iii) $\phi$ belongs to the countably full subgroup of $\Aut\frak A$
generated by $\pi$;

(iv) $\inf_{n\in\Bbb Z}\supp(\pi^n\phi)=0$.}

\proof{{\bf (a)(i)} $\pi_2\pi_1\in H$ for all $\pi_1$, $\pi_2\in H$.
\Prf\ Let $a\in\frak A$ be non-zero;  then there are a non-zero
$b\Bsubseteq a$ and a $\phi_1\in G$ such that $\pi_1$ and $\phi_1$ agree on
the principal ideal $\frak A_b$.   Next, there are a non-zero
$c\Bsubseteq\pi_1b$ and a $\phi_2\in G$ such that $\pi_2$ and $\phi_2$
agree on $\frak A_c$.   Set $d=\pi_1^{-1}c$;  then
$d\in\frak A_a\setminus\{0\}$, and $\phi_2\phi_1$ is a member of $G$
agreeing with $\pi_2\pi_1$ on $\frak A_d$.   As $a$ is arbitrary,
$\pi_2\pi_1\in H$.\ \Qed

\medskip

\quad{\bf (ii)} $\pi^{-1}\in H$ for every $\pi\in H$.  \Prf\ If
$a\in\frak A\setminus\{0\}$, there are a non-zero $b\Bsubseteq\pi^{-1}a$
and a $\phi\in G$ such that $\pi$ and $\phi$ agree on $\frak A_b$;  now
$0\ne\pi b\Bsubseteq a$ and $\pi^{-1}$ and $\phi^{-1}$ agree on
$\frak A_{\pi b}$.   As $a$ is arbitrary, $\pi^{-1}\in H$.\ \QeD\
Of course $H\supseteq G$, so $H$ is a subgroup of $\Aut\frak A$.

\medskip

\quad{\bf (iii)} Suppose now that $\familyiI{a_i}$ is a partition of unity
in $\frak A$, $\familyiI{\pi_i}$ is a family in $H$, and
$\pi\in\Aut\frak A$ is such that $\pi$ agrees with $\pi_i$ on
$\frak A_{a_i}$ for every $i\in I$.   Then $\pi\in H$.   \Prf\
If $a\in\frak A\setminus\{0\}$, there is an $i\in I$ such that
$b=a\Bcap a_i$ is non-zero;  now $\pi$ agrees with $\pi_i$ on $b$.\ \QeD\
So $H$ is a full subgroup of $\Aut\frak A$.

\medskip

\quad{\bf (iv)} If $H'$ is any full subgroup of $\Aut\frak A$ including
$G$, then $H'\supseteq H$.   \Prf\ If $\pi\in H$, then
$B=\{b:$ there is a $\phi\in G$ agreeing with $\pi$ on $\frak A_b\}$ is an
order-dense subset of $\frak A$, so there is a partition $\familyiI{a_i}$
of unity in $\frak A$ such that $a_i\in B$ for every $i$.   For each
$i\in I$, let $\pi_i\in G$ be such that $\pi$ and $\pi_i$ agree on
$\frak A_{a_i}$;  then $\familyiI{(a_i,\pi_i)}$ witnesses that $\pi\in H'$.
As $\pi$ is arbitrary, $H\subseteq H'$.\ \Qed

\medskip

{\bf (b)} (ii)$\Rightarrow$(i) is trivial.   In the other
direction, let $G$ be the family of all those automorphisms $\psi$ of
$\frak A$ such that there is a partition of unity
$\langle a_n\rangle_{n\in\Bbb Z}$
in $\frak A$ such that $\psi c=\pi^nc$ whenever $n\in\Bbb Z$ and
$c\Bsubseteq a_n$.   Then $G$ is a countably full subgroup of
$\Aut\frak A$ containing $\pi$.

\Prf\ Of course $\pi\in G$ (set $a_1=1$, $a_n=0$ for $n\ne 1$).

Take $\psi_1$, $\psi_2\in G$.   Let
$\langle a_n\rangle_{n\in\Bbb Z}$,
$\langle a'_n\rangle_{n\in\Bbb Z}$ be partitions of unity
in $\frak A$ such that $\psi_1c=\pi^nc$ whenever $n\in\Bbb Z$ and
$c\Bsubseteq a_n$, while $\psi_2c=\pi^nc$ whenever $n\in\Bbb Z$ and
$c\Bsubseteq a'_n$.   Then
$\langle a'_n\Bcap\psi_2^{-1}a_m\rangle_{m,n\in\Bbb Z}$ is a partition
of unity.   If $c\Bsubseteq a'_n\Bcap\psi_2^{-1}a_m$, then
$\psi_2c=\pi^nc\Bsubseteq a_m$, so $\psi_1\psi_2c=\pi^{m+n}c$.   So if
we set $b_n=\sup_{i\in\Bbb Z}a'_i\Bcap\psi_2^{-1}a_{n-i}$ for each
$n\in\Bbb Z$, $\family{n}{\Bbb Z}{b_n}$ is a partition of unity in
$\frak A$ witnessing that $\psi_1\psi_2\in G$.
At the same time, $\family{n}{\Bbb Z}{\psi_1a_{-n}}$ is a partition of
unity witnessing that $\psi_1^{-1}\in G$.   As $\psi_1$ and $\psi_2$ are
arbitrary, $G$ is a subgroup of $\Aut\frak A$.

Now suppose that $\langle a_i\rangle_{i\in I}$ is a countable partition
of unity in
$\frak A$ and that $\psi\in\Aut\frak A$ is such that for every $i\in I$
there is a $\psi_i\in G$ such that $\psi c=\psi_ic$ for every
$c\Bsubseteq a_i$.   For each $i\in I$ let $\family{n}{\Bbb Z}{a_{in}}$
be a partition of unity such that $\psi_ic=\pi^nc$ whenever
$c\Bsubseteq a_{in}$.   Then
$\langle a_i\Bcap a_{in}\rangle_{i\in I,n\in\Bbb Z}$ is a partition of
unity such that $\psi c=\pi^nc$ whenever $c\Bsubseteq c_i\Bcap a_{in}$.
So setting $b_n=\sup_{i\in I}a_i\Bcap a_{in}$ for each $n\in\Bbb Z$,
$\family{n}{\Bbb Z}{b_n}$ is a partition of unity witnessing that
$\psi\in G$.   As $\psi$ is arbitrary, $G$ is countably full.\ \Qed

Accordingly $G$ must include (indeed, must coincide with) the countably
full subgroup generated by $\pi$, and (i)$\Rightarrow$(ii).

\medskip

{\bf (c)(i)$\Rightarrow$(ii)} is a special case of (a).

\medskip

\quad{\bf (ii)$\Rightarrow$(iii)} For $n\in\Bbb Z$, let $B_n$ be the set
of those $b\in\frak A$ such that $\phi c=\pi^nc$ for every
$c\Bsubseteq b$.   Set $b_n=\sup B_n$ for each $n$;  then if
$c\Bsubseteq b_n$,

\Centerline{$\phi c=\phi(\sup_{b\in B_n}b\Bcap c)
=\sup_{b\in B_n}\phi(b\Bcap c)=\sup_{b\in B_n}\pi^n(b\Bcap c)
=\pi^n c$.}

\noindent Set

$$\eqalign{a_n&=b_n\Bsetminus\sup_{0\le i<n}b_i
\text{ if }n\in\Bbb N,\cr
&=b_n\Bsetminus\sup_{i>n}b_i
\text{ if }n\in\Bbb Z\setminus\Bbb N;\cr}$$

\noindent then $\langle a_n\rangle_{n\in\Bbb Z}$ is disjoint,

\Centerline{$\sup_{n\in\Bbb Z}a_n
=\sup_{n\in\Bbb Z}b_n=\sup(\bigcup_{n\in\Bbb Z}B_n)=1$,}

\noindent and $\phi c=\pi^nc$ for every $c\Bsubseteq a_n$, $n\in\Bbb Z$.
Thus $\phi$ satisfies condition (ii) of (a) and belongs to the countably
full subgroup generated by $\pi$.

\medskip

{\bf (iii)$\Rightarrow$(i)} is trivial.

\medskip

{\bf (ii)$\Leftrightarrow$(iv)} The point is that, for $n\in\Bbb Z$ and
$b\in\frak A$,

$$\eqalign{\phi c=\pi^nc\text{ for every }c\Bsubseteq b
&\iff\pi^{-n}\phi c=c\text{ for every }c\Bsubseteq b\cr
&\iff b\Bcap\supp(\pi^{-n}\phi)=0.\cr}$$

\noindent So we have

$$\eqalign{\text{(ii)}
&\iff\Forall a\in\frak A\setminus\{0\}
\Exists n\in\Bbb Z,\,b\text{ such that }0\ne b\Bsubseteq a
\text{ and }b\Bcap\supp(\pi^{-n}\phi)=0\cr
&\iff\Forall a\in\frak A\setminus\{0\}
\Exists n\in\Bbb Z,\,a\Bsetminus\supp(\pi^{-n}\phi)\ne 0\cr
&\iff\inf_{n\in\Bbb Z}\supp(\pi^{-n}\phi)=0,\cr}$$

\noindent as required.
}%end of proof of 381I

\leader{381J}{Lemma} Let $\frak A$ be a Boolean algebra, and
$\pi\in\Aut\frak A$.   Suppose that $\phi$ belongs to the full subgroup
of $\Aut\frak A$ generated by $\pi$.

(a) If $c\in\frak A$ is such that
$\pi c=c$, then $\phi c=c$.

(b) If $a\in\frak A$ supports $\pi$ then it supports $\phi$.

\proof{{\bf (a)} Let $G$ be the set of all $\psi\in\Aut\frak A$ such
that $\psi c=c$.   Then $G$ is a subgroup of $\Aut\frak A$ containing
$\pi$.   Also $G$ is full.   \Prf\ If $\familyiI{a_i}$ is a partition of
unity in $\frak A$, $\familyiI{\psi_i}$ is a family in $G$, and
$\psi\in\Aut\frak A$ is such that $\psi d=\psi_id$ whenever
$d\Bsubseteq a_i$, then

\Centerline{$\psi c=\sup_{i\in I}\psi(c\Bcap a_i)
=\sup_{i\in I}\psi_i(c\Bcap a_i)=\sup_{i\in I}\psi_ic\Bcap\psi_ia_i
=\sup_{i\in I}c\Bcap\psi_ia_i=c$.}

\noindent So $\psi\in G$;  as $\psi$ is arbitrary, $G$ is full.\ \QeD\
So $\phi\in G$ and $\phi c=c$, as claimed.

\medskip

{\bf (b)} If $c\Bcap a=0$ then $\pi c=c$ so $\phi c=c$.
}%end of proof of 381J

%Boolean endomorphisms
\leader{381K}{Lemma} Let $\frak A$ be a Dedekind $\sigma$-complete
Boolean algebra and
$\pi:\frak A\to\frak A$ a sequentially order-continuous Boolean
homomorphism.

(a) If $a\in\frak A$ and $a^*=\inf_{k\in\Bbb N}\sup_{i\ge k}\pi^ia$,
then $\pi a^*=a^*$.

(b) If $a\in\frak A$ is such that
$a\Bsubseteq\sup_{i\ge 1}\pi^ia$, then
$\sup_{i\ge k}\pi^ia=\sup_{i\in\Bbb N}\pi^ia$ for every $k\in\Bbb N$.

\proof{{\bf (a)} Because $\pi$ is sequentially order-continuous,

$$\eqalignno{\pi a^*
&=\inf_{k\in\Bbb N}\sup_{i\ge k}\pi^{i+1}a\cr
\displaycause{313Lc}
&=\inf_{k\in\Bbb N}\sup_{i\ge k+1}\pi^ia
=\inf_{k\ge 1}\sup_{i\ge k}\pi^ia
=\inf_{k\in\Bbb N}\sup_{i\ge k}\pi^ia
=a^*.\cr}$$

\medskip

{\bf (b)} Induce on $k$.   For $k=0$ the result is just the hypothesis.
For the inductive step to $k+1$, because $\pi$ is
sequentially order-continuous, so is $\pi^k$ (313Ic), so

$$\eqalignno{\sup_{i\ge k+1}\pi^ia
&=\sup_{i\ge 1}\pi^k\pi^ia
=\pi^k(\sup_{i\ge 1}\pi^ia)\cr
&=\pi^k(\sup_{i\in\Bbb N}\pi^ia)
=\sup_{i\ge k}\pi^ia
=\sup_{i\in\Bbb N}\pi^ia,\cr}$$

\noindent and the induction continues.
}%end of proof of 381K

%recurrence
\vleader{72pt}{381L}{Lemma} Let $\frak A$ be a Dedekind $\sigma$-complete
Boolean algebra and $\pi\in\Aut\frak A$.
Then for any $a\in\frak A$, the following are equiveridical:

\inset{(i) $\pi$ is recurrent on $a$;

(ii) $a\Bsubseteq\sup_{n\ge 1}\pi^{-n}a$;

(iii) there is some $k\ge 1$ such that
$a\Bsubseteq\sup_{n\ge k}\pi^{-n}a$;

(iv) $a\Bsubseteq\sup_{n\ge k}\pi^{-n}a$ for every $k\in\Bbb N$.}

\proof{{\bf (i)$\Rightarrow$(ii)} If (i) is true, set
$b=a\Bsetminus\sup_{n\ge 1}\pi^{-n}a$.    Then $a\Bcap\pi^nb=0$ for
every $n\ge 1$, so $b=0$, that is, $a\Bsubseteq\sup_{n\ge 1}\pi^{-n}a$.

\medskip

\quad{\bf (ii)$\Rightarrow$(i)} If (ii) is true and
$0\ne b\Bsubseteq a$, then there is some $n\ge 1$ such that
$b\Bcap\pi^{-n}a\ne 0$, that is, $\pi^nb\Bcap a\ne 0$;  as $b$ is
arbitrary, $\pi$ is recurrent on $a$.

\medskip

\quad{\bf (iv)$\Rightarrow$(ii)$\Leftrightarrow$(iii)} are trivial.

\medskip

\quad{\bf (ii)$\Rightarrow$(iv)} Apply 381Kb to $\pi^{-1}$.
}%end of proof of 381L

\leader{381M}{}\cmmnt{ It is with the idea of `recurrence' that we
start to get genuine surprises.   The first fundamental construction is
that of `induced automorphism' in the following sense.

\medskip

\noindent}{\bf Proposition} Let $\frak A$ be a Dedekind
$\sigma$-complete Boolean algebra and $a\in\frak A$.   Suppose that
$\pi\in\Aut\frak A$ is doubly recurrent on $a$.
Then we have a Boolean automorphism
$\pi_a:\frak A_a\to\frak A_a$ defined by saying that $\pi_ad=\pi^nd$
whenever $n\ge 1$ and
$d\Bsubseteq a\Bcap\pi^{-n}a\Bsetminus\sup_{1\le i<n}\pi^{-i}a$;  I will
call $\pi_a$ the {\bf induced automorphism} on $\frak A_a$.

\proof{ For $n\ge 1$ set

\Centerline{$d_n=a\Bcap\pi^{-n}a\Bsetminus\sup_{1\le i<n}\pi^{-i}a$.}

\noindent If $1\le m<n$ then

\Centerline{$d_n\Bsubseteq\pi^{-n}a\Bsetminus\pi^{-m}a$,
\quad$d_m\Bsubseteq\pi^{-m}a$}

\noindent so $d_m\Bcap d_n=0$.   Also

\Centerline{$d_m\Bsubseteq a$,
\quad$\pi^{n-m}d_n\Bcap a=\pi^{n-m}(d_n\Bcap\pi^{-(n-m)}a)=0$}

\noindent so

\Centerline{$\pi^nd_n\Bcap\pi^md_m=\pi^m(\pi^{n-m}d_n\Bcap d_m)=0$.}

\noindent Finally,
$\sup_{n\ge 1}d_n=a\Bcap\sup_{n\ge 1}\pi^{-n}a=a$, because $\pi$ is
recurrent on $a$ (using (a)).

It follows that $\langle d_n\rangle_{n\ge 1}$ is a partition of unity
in $\frak A_a$.   Since $\langle\pi^nd_n\rangle_{n\ge 1}$ also is a
disjoint family in $\frak A_a$, and

$$\eqalign{\sup_{n\ge 1}\pi^nd_n
&=\sup_{n\ge 1}(\pi^na\Bcap a\Bsetminus\sup_{1\le i<n}\pi^{n-i}a)\cr
&=a\Bcap\sup_{n\ge 1}(\pi^na\Bsetminus\sup_{1\le i<n}\pi^ia)
=a\Bcap\sup_{n\ge 1}\pi^na
=a,\cr}$$

\noindent (because $\pi^{-1}$ is recurrent on $a$),
$\langle\pi^nd_n\rangle_{n\ge 1}$ is another partition of
unity.   So we have an automorphism
$\pi_a:\frak A_a\to\frak A_a$ defined by setting $\pi_ad=\pi^nd$ if
$d\Bsubseteq d_n$ (381C).
}%end of proof of 381M

\leader{381N}{Lemma} Let $\frak A$ be a Dedekind
$\sigma$-complete Boolean algebra and $a\in\frak A$.   Suppose that
$\pi\in\Aut\frak A$ is doubly recurrent on $a$.    Let
$\pi_a\in\Aut\frak A_a$ be the induced automorphism.

(a) $\pi^{-1}$ is doubly recurrent on $a$, and the induced automorphism
$(\pi^{-1})_a$ is $(\pi_a)^{-1}$.

(b) For every $n\in\Bbb N$ there is a partition of unity
$\langle b_i\rangle_{i\ge n}$ in $\frak A_a$ such that $\pi_a^nb=\pi^ib$
whenever $i\ge n$ and $b\Bsubseteq b_i$.

(c) If $n\ge 1$ and $0\ne b\Bsubseteq a\Bcap\pi^{-n}a$, there are a
non-zero $b'\Bsubseteq b$ and a $j$ such that $1\le j\le n$ and
$\pi^nd=\pi_a^jd$ for every $d\Bsubseteq b'$.

(d) Suppose that $m\ge 1$ is such that $a\Bcap\pi^ia=0$ for $1\le i<m$.
Then for any $n\ge 1$ we have a disjoint family
$\langle b_{ni}\rangle_{1\le i\le\lfloor n/m\rfloor}$, with supremum
$a\Bcap\pi^{-n}a$, such that $\pi^nd=\pi_a^id$ whenever
$1\le i\le\lfloor\bover{n}{m}\rfloor$ and $d\Bsubseteq b_{ni}$.

(e) Suppose that $b\Bsubseteq a$.   Then $\pi$ is doubly recurrent on
$b$ iff $\pi_a$ is doubly recurrent on $b$, and in this case
$\pi_b=(\pi_a)_b$, where $(\pi_a)_b$ is the automorphism of $\frak A_b$
induced by $\pi_a$.

(f) Suppose that $c\in\frak A$ is such that $\pi c=c$.   Then $\pi$ is
doubly recurrent on $a\Bcap c$, and
$\pi_{a\Bcap c}=\pi_a\restrp\frak A_{a\Bcap c}$;  in particular,
$\pi_a(a\Bcap c)=a\Bcap c$.

(g) If $\pi$ is aperiodic, so is $\pi_a$.

(h) Suppose that $a\Bcap\pi a=0$, and that $b\Bsubseteq a$ is such that
$b\Bcap\pi_ab=0$.   Then $b$, $\pi b$ and $\pi^2b$ are all disjoint.

(i) There is an automorphism $\tilde\pi_a\in\Aut\frak A$ defined by
setting $\tilde\pi_ad=\pi_ad$ for $d\Bsubseteq a$, $\tilde\pi_ad=d$ for
$d\Bsubseteq 1\Bsetminus a$, and $\tilde\pi_a$ belongs to the countably
full subgroup of $\Aut\frak A$ generated by $\pi$.

\proof{ Set $d_n=a\Bcap\pi^{-n}a\Bsetminus\sup_{1\le i<n}\pi^{-i}a$ for
$n\ge 1$, so that $\langle d_n\rangle_{n\ge 1}$ and
$\langle\pi^nd_n\rangle_{n\ge 1}$ are partitions of unity in
$\frak A_a$, and $\pi_ab=\pi^nb$ for $b\Bsubseteq d_n$.

\medskip

{\bf (a)} By the symmetry in the definition of `doubly recurrent',
$\pi^{-1}$ is doubly recurrent on $a$ iff $\pi$ is.   In this case,

\Centerline{$\pi^nd_n=\pi^na\Bcap a\Bsetminus\sup_{1\le i<n}\pi^{n-i}a
=a\Bcap\pi^na\Bcap a\Bsetminus\sup_{1\le i<n}\pi^ia$}

\noindent so $(\pi^{-1})_ab=\pi^{-n}b=(\pi_a)^{-1}b$ for every
$b\Bsubseteq\pi^nd_n$;  as $\sequencen{\pi_nd_n}$ is a partition of
unity in $\frak A_a$, $(\pi^{-1})_a=(\pi_a)^{-1}$.

\medskip

{\bf (b)} Induce on $n$.   For $n=0$ we can take $b_0=a$ and $b_i=0$ for
$i>0$.   For the inductive step to $n+1$, let
$\langle b_i\rangle_{i\ge n}$ be a partition of unity in $\frak A_a$
such that $\pi_a^nb=\pi^ib$ for $b\Bsubseteq b_i$.   Then
$\langle\pi_a^{-1}b_i\rangle_{i\ge n}$ and
$\langle d_k\Bcap\pi_a^{-1}b_i\rangle_{k\ge 1,i\ge n}$ are partitions of
unity in $\frak A_a$.   If $b\Bsubseteq d_k\Bcap\pi_a^{-1}b_i$, then
$\pi_ab=\pi^kb\Bsubseteq b_i$, so $\pi_a^{n+1}b=\pi^{k+i}b$.   This
means that if we set
$b'_j=\sup_{k\ge 1,i\ge n,k+i=j}d_k\Bcap\pi_a^{-1}b_i$ for $j\ge n+1$,
$\langle b'_j\rangle_{j\ge n+1}$ will be a partition of unity in
$\frak A_a$, and $\pi_a^{n+1}b=\pi^jb$ whenever $b\Bsubseteq b_j$.   So
the induction continues.

\medskip

{\bf (c)} Induce on $n$.   If $b\Bcap\pi^{-i}a=0$ for
$1\le i<n$ then we can take $b'=b$ and $j=1$.   Otherwise, take the
first $i\ge 1$ such that $b_1=b\Bcap\pi^{-i}a\ne 0$.
Then $\pi_ad=\pi^id$ for every $d\Bsubseteq b_1$.   Also
$\pi^{n-i}\pi^ib_1\Bsubseteq a$, so, by the inductive hypothesis, there
are a non-zero $c\Bsubseteq\pi^ib_1$ and a $j$ such that $1\le j\le n-i$
and $\pi^{n-i}d=\pi_a^jd$ for every $d\Bsubseteq c$.   Setting
$b'=\pi^{-i}c\Bsubseteq b_1$, we have $0\ne b'\Bsubseteq b$ and

\Centerline{$\pi^nd=\pi^{n-i}\pi^id=\pi_a^j\pi_ad=\pi_a^{j+1}d$}

\noindent whenever $d\Bsubseteq b'$.   So the induction continues.

\medskip

{\bf (d)} Again induce on $n$.    If $1\le n<m$ then $a\Bcap\pi^{-n}a=0$
and the result is trivial.   If $n=m$, then $a\Bcap\pi^{-n}a=d_n$ and
$\pi_ad=\pi^nd$ for every $d\Bsubseteq d_n$, so we can set $b_{n1}=d_n$.
For the inductive step to $n>m$, we have

$$\eqalign{a\Bcap\pi^{-n}a
&=d_n\Bcup\sup_{m\le k<n}(d_k\Bcap\pi^{-n}a)
=d_n\Bcup\sup_{m\le k<n}(d_k\Bcap\pi^{-k}(a\Bcap\pi^{-(n-k)}a))\cr
&=d_n\Bcup\sup_{\Atop{m\le k\le n-m}{1\le j\le\lfloor (n-k)/m\rfloor}}
  (d_k\Bcap\pi^{-k}b_{n-k,j})\cr}$$

\noindent by the inductive hypothesis, while
$\langle d_k\Bcap\pi^{-k}b_{n-k,j}
\rangle_{m\le k\le n-m,1\le j\le\lfloor(n-k)/m\rfloor}$ is disjoint.
Now if $m\le k\le n-m$ and $1\le j\le\lfloor\bover{n-k}m\rfloor$ and
$d\Bsubseteq d_k\Bcap\pi^{-k}b_{n-k,j}$, we have
$\pi_ad=\pi^kd\Bsubseteq b_{n-k,j}$, so
$\pi^nd=\pi^{n-k}\pi_ad=\pi_a^{j+1}d$;  while if $d\Bsubseteq d_n$ then
$\pi^nd=\pi_ad$.   So we can set

\Centerline{$b_{n1}=d_n$,
\quad$b_{ni}=\sup_{m\le k\le n-m}d_k\Bcap b_{n-k,i-1}$}

\noindent for $2\le i\le\lfloor\bover{n}{m}\rfloor$, and the induction
will continue.


\medskip

{\bf (e)} Applying (b) and (d) to $\pi$ and $\pi^{-1}$, and using 381L
and (a), we see that $\pi$ is doubly recurrent on $b$ iff $\pi_a$ is
doubly recurrent on $b$.

In this case, set $D=\{d:d\in\frak A_b,\,\pi_bd=(\pi_a)_bd\}$.   Then
$D$ is order-dense in $\frak A_b$.   \Prf\ Take any non-zero
$c\in\frak A_b$.
Since $b\Bsubseteq\sup_{n\ge 1}\pi^{-n}b$, there is an $n\ge 1$
such that $c'=c\Bcap\pi^{-n}b\Bsetminus\sup_{1\le i<n}\pi^{-i}b$ is
non-zero.   Next, there is a non-zero $d\Bsubseteq c'$ such that for
every $m\le n$ either $d\Bsubseteq\pi^{-m}a$ or $d\Bcap\pi^{-m}a=0$.
Enumerate $\{m:m\le n,\,d\Bsubseteq\pi^{-m}a\}$ in ascending order as
$(m_0,\ldots,m_k)$ (note that as $c'\Bsubseteq a\Bcap\pi^{-n}a$, we
must have $m_0=0$ and $m_k=n$).   Set $d_i=\pi^{m_i}d$ for $i\le k$, so
that

\Centerline{$d_0=d$,
\quad$\pi^{m_{i+1}-m_i}d_i=d_{i+1}\Bsubseteq a$,}

\noindent while

\Centerline{$\pi^jd_i=\pi^{m_i+j}d\Bsubseteq 1\Bsetminus a$}

\noindent for $1\le j<m_{i+1}-m_i$;  that is, $d_{i+1}=\pi_ad_i$ for
$i<k$.   Thus

\Centerline{$\pi_a^kd=\pi^{m_k}d=\pi^nd\Bsubseteq b$,}

\noindent while

\Centerline{$\pi_a^id=d_i=\pi^{m_i}d
\Bsubseteq\pi^{m_i}c'\Bsubseteq 1\Bsetminus b$}

\noindent for every $i<k$, and

\Centerline{$(\pi_a)_bd=\pi_a^kd=\pi^nd=\pi_bd$,}

\noindent so that $d\in D$.   As $c$ is arbitrary, $D$ is
order-dense.\ \Qed

Because $\pi_b$ and $(\pi_a)_b$ are both order-continuous Boolean
homomorphisms on $\frak A_b$, and every member of $\frak A_b$ is a
supremum of some subset of $D$ (313K), $\pi_b=(\pi_a)_b$, as required.

\wheader{381N}{4}{2}{2}{24pt}

{\bf (f)} We have

\Centerline{$a\Bcap c
\Bsubseteq\sup_{n\ge 1}\pi^{-n}a\Bcap c
=\sup_{n\ge 1}\pi^{-n}a\Bcap\pi^{-n}c
=\sup_{n\ge 1}\pi^{-n}(a\Bcap c)$,}

\noindent so $\pi$ is recurrent on $a\Bcap c$;  similarly, $\pi^{-1}$ is
recurrent on $a\Bcap c$.   If $n\ge 1$ and

\Centerline{$d
\Bsubseteq a\Bcap c\Bcap\pi^{-n}(a\Bcap c)
  \Bsetminus\sup_{1\le i<n}\pi^{-i}(a\Bcap c)
=c\Bcap a\Bcap\pi^{-n}a\Bsetminus\sup_{1\le i<n}\pi^{-i}a$,}

\noindent then $\pi_{a\Bcap c}d=\pi^nd=\pi_ad$.   So $\pi_a$ extends
$\pi_{a\Bcap c}$, as claimed.

\medskip

{\bf (g)} If $0\ne b\Bsubseteq a$, and $n\ge 1$, then (b) tells us that
there are a non-zero $c\Bsubseteq b$ and an $i\ge n$ such that
$\pi_a^nd=\pi^id$ for every $d\Bsubseteq c$.   Now we are supposing that
$\supp\pi^i=1$, so there is a $d\Bsubseteq c$ such that $\pi^id\ne d$,
that is, $\pi_a^nd\ne d$.   As $b$ is arbitrary, $\supp\pi_a^n=a$;  as
$n$ is arbitrary, $\pi_a$ is aperiodic.

\medskip

{\bf (h)} Of course $\pi b\Bsubseteq\pi a$ is disjoint from
$b\Bsubseteq a$;  it follows that $\pi b\Bcap\pi^2b=\pi(b\Bcap\pi b)=0$.
If $c=b\Bcap\pi^{-2}b$, then
$c\Bsubseteq a\Bcap\pi^{-2}a\Bsetminus\pi^{-1}a$, so

\Centerline{$\pi^2b\Bcap b=\pi^2c
=\pi_ac\Bsubseteq\pi_ab$}

\noindent is disjoint from $b$ and must be $0$.   So $b$, $\pi b$ and
$\pi^2b$ are all disjoint.

\medskip

{\bf (i)} By 381C, the formula defines an automorphism $\tilde\pi_a$.
Setting $d_0=1\Bsetminus a$, $\sequencen{d_n}$ is a partition of unity
in $\frak A$ and $\tilde\pi_ad=\pi^nd$ for $d\Bsubseteq d_n$, so
$\tilde\pi_a$ belongs to the countably full subgroup of $\Aut\frak A$
generated by $\pi$.
}%end of proof of 381N

\leader{381O}{Lemma} Let $\frak A$ be a Boolean algebra and
$\pi:\frak A\to\frak A$ a Boolean homomorphism.   Then the following are
equiveridical:

(i) $\pi$ is recurrent on every $a\in\frak A$;

(ii) for every non-zero $a\in\frak A$ there is a $k\ge 1$ such that
$a\Bcap\pi^ka\ne 0$;

(iii) $a=\sup_{k\ge 1}a\Bcap\pi^ka$ for every $a\in\frak A$.

\proof{{\bf (i)$\Rightarrow$(ii)} If (i) is true, and
$a\in\frak A\setminus\{0\}$, then taking $b=a$ in the definition 381Bg
we see that there is a $k\ge 1$ such that $a\Bcap\pi^ka\ne 0$.

\medskip

{\bf (ii)$\Rightarrow$(iii)} Suppose (ii) is true.   \Quer\ If
$a\in\frak A$ is not the supremum of $\{a\Bcap\pi^ka:k\ge 1\}$, let
$b\Bsubseteq a$ be non-zero and disjoint from $\pi^ka$ for every
$k\ge 1$.   Then $b\Bcap\pi^kb=0$ for every $k\ge 1$, which is
impossible.\ \Bang

\medskip

{\bf (iii)$\Rightarrow$(i)} Suppose (iii) is true.   If
$0\ne b\Bsubseteq a$ then $b=\sup_{k\ge 1}b\Bcap\pi^kb$, so there is
certainly some $k\ge 1$ such that $b\Bcap\pi^kb\ne 0$, in which case
$a\Bcap\pi^kb\ne 0$.   As $b$ is arbitrary, $\pi$ is recurrent on $a$;
as $a$ is arbitrary, (i) is true.
}%end of proof of 381O

\cmmnt{\medskip

\noindent{\bf Remark} The condition `recurrent on every $a\in\frak A$'
looks, and is, very restrictive;  but it is satisfied by the
homomorphisms we care about most (386A).
}%end of comment

\leader{381P}{Proposition} Let $\frak A$ be a Boolean algebra and
$\pi:\frak A\to\frak A$ a Boolean homomorphism which is recurrent on
every $a\in\frak A$.   Then $\pi$ is aperiodic iff $\frak A$ is
relatively atomless\cmmnt{ (definition:  331A)} over the fixed-point
algebra $\frak C=\{c:c\in\frak A$, $\pi c=c\}$.   In particular, if
$\pi$ is ergodic, it is aperiodic iff $\frak A$ is atomless.

\proof{ It is elementary to check that $\frak C$ is a subalgebra of
$\frak A$.

\medskip

{\bf (a)} Suppose that $\pi$ is not aperiodic.   Then there is a least
$n\ge 1$ such that $1$ is not the support of $\pi^n$;  that is, there is
a non-zero $a\in\frak A$ such that $\pi^nd=d$ for every
$d\Bsubseteq a$.   Now if $0\ne b\Bsubseteq a$ and $1\le i<n$ there is a
non-zero $b'\Bsubseteq b$ such that $b'\Bcap\pi^ib'=0$.   \Prf\ We are
supposing that the support of $\pi^i$ is $1$, so there is a
$d\Bsubseteq b$ such that $d\ne\pi^id$.   If $d\Bsetminus\pi^id\ne 0$,
take $b'=d\Bsetminus\pi^id$.   Otherwise, try
$b'=d\Bsetminus\pi^{n-i}d$;  then

\Centerline{$\pi^ib'=\pi^id\Bsetminus\pi^nd=\pi^id\Bsetminus d\ne 0$,}

\noindent so $b'\ne 0$, while
$b'\Bcap\pi^ib'\Bsubseteq d\Bsetminus\pi^nd=0$.\ \QeD\

We can therefore find a non-zero $b\Bsubseteq a$ such that
$b\Bcap\pi^ib=0$ whenever $1\le i<n$.   Now $b$ is a relative atom of
$\frak A$ over $\frak C$.   \Prf\ If $d\Bsubseteq b$, set
$c=\sup_{0\le i<n}\pi^id$.   Then $\pi c=\sup_{1\le i\le n}\pi^id=c$,
so $c\in\frak C$, while $b\Bcap\pi^id=0$ for $1\le i<n$, so
$d=b\Bcap c$.\ \QeD\  Thus $b$ witnesses that $\frak A$ is
not relatively atomless over $\frak C$.

\medskip

{\bf (b)(i)} Note that if $a\in\frak A$ and $a\Bsubseteq\pi a$ then
$a=\pi a$.   \Prf\Quer\ Otherwise, set $b=\pi a\Bsetminus a$.   Then
$\pi^nb=\pi^{n+1}a\Bsetminus\pi^na$ for every $n$;  also
$a\Bsubseteq\pi a\Bsubseteq\pi^2a\Bsubseteq\ldots$,
so $\sequencen{\pi^nb}$ is
disjoint.   But in this case $\pi$ cannot be recurrent on $b$.\
\Bang\Qed

\medskip

\quad{\bf (ii)} Suppose that $\frak A$ is not relatively atomless over
$\frak C$.   Then there is a relative atom $a\in\frak A$;  as $\pi$ is
recurrent on $a$, there is a first $n\ge 1$ such that
$a\Bcap\pi^na\ne 0$.   Then $\pi^nb=b$ for every
$b\Bsubseteq a\Bcap\pi^na$.   \Prf\ Because $a$ is a relative atom over
$\frak C$, there is a $c\in\frak C$ such that $b=a\Bcap c$.
Now $\pi^nb=\pi^na\Bcap c\Bsupseteq b$.   Set
$b_1=\sup_{0\le i<n}\pi^ib$;  then
$\pi b_1=\sup_{1\le i\le n}\pi^ib\Bsupseteq b_1$.   So $b_1=\pi b_1$, by
(i), and $\pi^nb\Bsubseteq\sup_{i<n}\pi^ib$.   Next,

\Centerline{$\pi^nb\Bcap\pi^ib=\pi^i(\pi^{n-i}b\Bcap b)
\Bsubseteq\pi^i(\pi^{n-i}a\Bcap a)=0$}

\noindent for $0<i<n$, so $\pi^nb\Bsubseteq b$ and $\pi^nb=b$.\ \QeD\
Thus $a\Bcap\pi^na$ witnesses that $\pi$ is not aperiodic.

\medskip

{\bf (c)} Finally, if $\pi$ is ergodic, then $\frak C=\{0,1\}$ (372Pa),
so that `relatively atomless over $\frak C$' becomes `atomless'.
}%end of proof of 381P

\leader{381Q}{}\cmmnt{ As far as possible I will express the ideas of
this chapter in `pure' Boolean algebra terms, without shifting to
measure spaces or Stone spaces.   However there is a crucial argument in
\S382 for which the Stone representation is an invaluable aid, and
anyone studying the subject has to be able to use it.

\medskip

\noindent}{\bf Proposition} Let $\frak A$ be a
Boolean algebra and $Z$ its Stone space.
For $a\in\frak A$ let $\widehat{a}$ be the corresponding open-and-closed
subset of $Z$\cmmnt{;  recall that $\widehat{a}$ can be identified
with the Stone space of $\frak A_a$ (312T)}.   For a Boolean
homomorphism $\pi:\frak A\to\frak A$ let $f_{\pi}:Z\to Z$ be the
continuous function such that
$\widehat{\pi a}=f_{\pi}^{-1}[\widehat{a}]$ for every
$a\in\frak A$\cmmnt{ (312Q)}.

(a) If $a$, $b\in\frak A$ and $\phi:\frak A_a\to\frak A_b$ is a Boolean
homomorphism represented by a continuous function
$g:\widehat{b}\to\widehat{a}$, then
$\pi\in\Aut\frak A$ agrees with $\phi$ on $\frak A_a$ iff $f_{\pi}$
agrees with $g$ on $\widehat{b}$.

(b) If $\pi:\frak A\to\frak A$ is a Boolean homomorphism, then
$a\in\frak A$ supports $\pi$ iff
$\widehat{a}\supseteq\{z:f_{\pi}(z)\ne z\}$.
So $a$ is the support of $\pi$ iff
$\widehat{a}=\overline{\{z:f_{\pi}(z)\ne z\}}$.

(c) Suppose that $\frak A$ is Dedekind complete and $\pi$,
$\phi\in\Aut\frak A$.   Let $G$ be the full subgroup of $\Aut\frak A$
generated by $\pi$.   Then

$$\eqalign{\phi\in G
&\iff\bigcup_{n\in\Bbb Z}\interior\{x:f_{\phi}(z)=f_{\pi}^n(z)\}
  \text{ is dense in }Z\cr
&\iff\{z:f_{\phi}(z)\in\{f_{\pi}^n(z):n\in\Bbb Z\}\}
\text{ is comeager in }Z.\cr}$$

(d) A Boolean homomorphism $\pi:\frak A\to\frak A$ is recurrent on
$a\in\frak A$ iff
$\widehat{a}\subseteq\overline{\bigcup_{n\ge 1}f_{\pi}^n[\widehat{a}]}$.

(e) Suppose that $\frak A$ is Dedekind $\sigma$-complete,
$\pi\in\Aut\frak A$ is recurrent on $a\in\frak A$, and
that $\pi_a\in\Aut\frak A_a$ is the induced automorphism\cmmnt{ (381M)}.
Let $f_{\pi_a}$ be the corresponding
autohomeomorphism of $\widehat{a}$.   For $k\ge 1$, set
$G_k=\{z:z\in\widehat{a}$, $f^k(z)\in\widehat{a}$,
$f^i(z)\notin\widehat{a}$ for $1\le i<k\}$.   Then
$\bigcup_{k\ge 1}G_k=\widehat{a}\cap\bigcup_{k\ge 1}f^{-k}[\widehat{a}]$
is a dense open subset of $\widehat{a}$ and $f_{\pi_a}(z)=f_{\pi}^k(z)$
whenever $k\ge 1$ and $z\in G_k$.

\proof{ Recall that $f_{\pi\phi}=f_{\phi}f_{\pi}$ for all Boolean
homomorphisms $\pi$, $\phi:\frak A\to\frak A$ (312R).

\medskip

{\bf (a)} The point is that $\{\widehat{d}:d\Bsubseteq a\}$ is a base
for the Hausdorff topology of of $\widehat{a}$.   So if
$g\ne f_{\pi}\restr\widehat{b}$, there are a $z\in\widehat{b}$ such that
$f_{\pi}(z)\ne g(z)$ and a $d\Bsubseteq a$ such that
$g(z)\in\widehat{d}$ and $f_{\pi}(z)\notin\widehat{d}$.   In this case,

\Centerline{$z\in g^{-1}[\widehat{d}]\setminus f_{\pi}^{-1}[\widehat{d}]
=\widehat{\phi d}\setminus\widehat{\pi d}$,}

\noindent and $\phi\ne\pi\restrp\frak A_a$.   On the other hand, if
$g=f_{\pi}\restr\widehat{b}$, then

\Centerline{$\widehat{\pi d}=f_{\pi}^{-1}[\widehat{d}]
=g^{-1}[\widehat{d}]=\widehat{\phi d}$}

\noindent for every $d\Bsubseteq a$, and $\phi=\pi\restrp\frak A_a$.

\medskip

{\bf (b)}

$$\eqalign{a\in\frak A\text{ supports }\pi
&\iff\pi\text{ agrees with the identity on }1\Bsetminus a\cr
&\iff f_{\pi}(z)=z\text{ for every }z\in\widehat{\pi(1\Bsetminus a)}
=Z\setminus\widehat{a}\cr
&\iff\widehat{a}\supseteq\{z:f_{\pi}(z)\ne z\}\cr
&\iff\widehat{a}\supseteq\overline{\{z:f_{\pi}(z)\ne z\}}.\cr}$$

\noindent So the smallest such $a$, if there is one, must have
$\widehat{a}=\overline{\{z:f_{\pi}(z)\ne z\}}$.

\medskip

{\bf (c)} If $\phi\in G$, let $\family{n}{\Bbb Z}{a_n}$ be a partition
of unity in $\frak A$ such that $\phi b=\pi^nb$ whenever $n\in\Bbb Z$
and $b\Bsubseteq a_n$ (381I).   Then $g(z)=f_{\pi}^n(z)$ whenever
$z\in\widehat{\phi a_n}$ ((a) above).   As $\sup_{n\in\Bbb Z}\phi a_n=1$
in $\frak A$,

\Centerline{$\bigcup_{n\in\Bbb Z}\interior\{x:f_{\phi}(z)=f_{\pi}^n(z)\}
\supseteq\bigcup_{n\in\Bbb Z}\widehat{\phi a_n}$}

\noindent is dense (313Ca).

If $\bigcup_{n\in\Bbb Z}\interior\{x:f_{\phi}(z)=f_{\pi}^n(z)\}$ is
dense, it is a dense open subset of
$\{z:f_{\phi}(z)\in\{f_{\pi}^n(z):n\in\Bbb Z\}$, so the latter is
comeager.

If $\{z:f_{\phi}(z)\in\{f_{\pi}^n(z):n\in\Bbb Z\}\}$ is comeager, set
$F_n=\{z:f_{\phi}(z)=f_{\pi}^n(z)\}$ for each $n$.   Then
$F_n\setminus\interior F_n$ is nowhere dense for each $n$, and
$Z\setminus\bigcup_{n\in\Bbb Z}F_n$ is meager, so
$\bigcup_{n\in\Bbb Z}\interior F_n$ is comeager, therefore dense (by
Baire's theorem, 3A3G).   If $a\in\frak A$ is non-zero, there are an
$n\in\Bbb Z$ such that $\widehat{\phi a}\Bcap\interior F_n\ne\emptyset$
and a $b\in\frak A$ such that
$\emptyset\ne\widehat{b}\Bsubseteq\widehat{\phi a}\Bcap F_n$, in which
case $0\ne\phi^{-1}b\Bsubseteq a$ and $\phi c=\pi^nc$ for every
$c\Bsubseteq b$.   By 381I(c-ii), $\phi\in G$.   So the cycle is
complete.

\medskip

{\bf (d)}

$$\eqalign{\pi\text{ is recurrent on }a
&\iff\text{whenever }0\ne b\Bsubseteq a\text{ there is a }k\ge 1\cr
&\mskip200mu
  \text{ such that }a\Bcap\pi^kb\ne 0\cr
&\iff\text{whenever }0\ne b\Bsubseteq a\text{ there is a }k\ge 1\cr
&\mskip200mu
  \text{ such that }
  \widehat{a}\cap(f_{\pi}^k)^{-1}[\widehat{b}]\ne\emptyset\cr
&\iff\text{whenever }0\ne b\Bsubseteq a\text{ there is a }k\ge 1\cr
&\mskip200mu
 \text{ such that }f_{\pi}^k[\widehat{a}]\cap\widehat{b}\ne\emptyset\cr
&\iff\widehat{a}\Bcap\bigcup_{k\ge 1}f_{\pi}^k[\widehat{a}]
  \text{ is dense in }\widehat{a}\cr
&\iff\widehat{a}
  \Bsubseteq\overline{\bigcup_{k\ge 1}f_{\pi}^k[\widehat{a}]}.\cr}$$

\medskip

{\bf (e)}
Set $d_k=a\Bcap\pi^{-k}a\Bsetminus\sup_{1\le i<k}\pi^{-i}a$, so that
$\pi^kd_k=a\Bcap\pi^ka\Bsetminus\sup_{1\le i<k}\pi^ia$.   Since $\pi^k$
and $\pi_a$ agree on $\frak A_{d_k}$, (a) tells us that $f_{\pi}^k$ and
$f_{\pi_a}$ agree on

\Centerline{$\widehat{\pi^kd_k}=\widehat{\pi_ad_k}
=f_{\pi_a}^{-1}[\widehat{d_k}]=G_k$.}

\noindent Because
$\sup_{k\ge 1}\pi^kd_k=a$, $\bigcup_{k\ge 1}G_k$ is dense in
$\widehat{a}$.
}%end of proof of 381Q

%3{8}1G
\leader{381R}{Cyclic \dvrocolon{automorphisms}}\cmmnt{ I end the
section by describing a notation which is often useful.

\medskip

\noindent}{\bf Definition} Let $\frak A$ be a Boolean algebra.

\medskip

{\bf (a)} Suppose that $a$, $b$ are disjoint members of $\frak A$ and
that $\pi\in\Aut\frak A$ is such that $\pi a=b$.   I will write
$\cycleii{a}{\pi}{b}$ for the member $\psi$ of $\Aut\frak A$ defined by
setting

$$\eqalign{\psi d&=\pi d\text{ if }d\Bsubseteq a,\cr
&=\pi^{-1}d\text{ if }d\Bsubseteq b,\cr
&=d\text{ if }d\Bsubseteq 1\Bsetminus(a\Bcup b).\cr}$$

\noindent Observe that in this case (if $a\ne 0$) $\psi$ is an
involution, that is, has order $2$ in the group $\Aut\frak A$;
I will call such a $\psi$ an {\bf exchanging involution}, and say that
it {\bf exchanges} $a$ with $b$.

\medskip

{\bf (b)} More generally, if $a_1,\ldots,a_n$ are disjoint elements of
$\frak A$ and $\pi_i\in\Aut\frak A$ are such that $\pi_ia_i=a_{i+1}$ for
each $i<n$, then I will write

\Centerline{$\cycle{a_1\,_{\pi_1}\,a_2\,_{\pi_2}\,
\ldots\,_{\pi_{n-1}}\,a_n}$}

\noindent for that $\psi\in\Aut\frak A$ such that

$$\eqalign{\psi d&=\pi_id\text{ if }1\le i<n,\,d\Bsubseteq a_i,\cr
&=\pi_1^{-1}\pi_2^{-1}\ldots\pi_{n-1}^{-1}d
  \text{ if }d\Bsubseteq a_n,\cr
&=d\text{ if }d\Bsubseteq 1\Bsetminus\sup_{i\le n}a_i.\cr}$$

\medskip

{\bf (c)} It will occasionally be convenient to use the same notation
when each $\pi_i$ is a Boolean isomorphism between the principal ideals
$\frak A_{a_i}$ and $\frak A_{a_{i+1}}$, rather than an automorphism of
the whole algebra $\frak A$.

\leaveitout{\spheader 381Rd \cmmnt{A variant of this notation gives us
a quick way of describing some of the automorphisms constructed by the
method in 381C.}
Suppose that $\langle a_i\rangle_{i\in I}$,
$\langle b_i\rangle_{i\in I}$ are two disjoint families in $\frak A$
such that $a_i\Bcap b_j=0$ for all $i$, $j\in I$.   Assume
{\it either} that $I$ is finite
{\it or} that $I$ is countable and $\frak A$ is Dedekind
$\sigma$-complete
{\it or} that $\frak A$ is Dedekind complete.
Suppose that for each $i\in I$ we have an automorphism
$\pi_i\in\Aut\frak A$.   Then I write
$\prod_{i\in I}\cycleii{a_i}{\pi_i}{b_i}$ for the automorphism
$\pi:\frak A\to\frak A$ such that

$$\eqalign{\pi d
&=\pi_id\text{ whenever }i\in I
  \text{ and }d\Bsubseteq a_i\cr
&=\pi_i^{-1}d\text{ whenever }i\in I
  \text{ and }d\Bsubseteq b_i\cr
&=d\text{ whenever }
  d\Bsubseteq 1\Bsetminus\sup_{i\in I}(a_i\Bcup b_i).\cr}$$

\noindent Of course such a $\pi$ will be an exchanging involution,
exchanging $\sup_{i\in I}a_i$ with $\sup_{i\in I}b_i$, except in the
trivial case in which every $a_i$, $b_i$ is zero.
}%end of leaveitout

\cmmnt{\medskip

\noindent{\bf Remark} The point of this notation is that we can expect
to use the standard techniques for manipulating cycles that are (I
suppose) familiar to you from elementary group theory;  the principal
change
is that we have to keep track of the subscripted automorphisms $_{\pi}$.
The following results are typical.}

%3{8}1H
\vleader{48pt}{381S}{Lemma} Let $\frak A$ be a Boolean algebra.

(a) If $\psi=\cycleii{a}{\pi}{b}$ is an exchanging involution in
$\Aut\frak A$, then

\Centerline{$\psi=\cycleii{a}{\psi}{b}
=\cycleii{b}{\psi}{a}=\cycleii{b}{\pi^{-1}}{a}$}

\noindent has support $a\Bcup b$.

(b) If $\pi=\cycleii{a}{\pi}{b}$ is an exchanging involution in
$\Aut\frak A$,   then for any $\phi\in\Aut\frak A$,

\Centerline{$\phi\pi\phi^{-1}
=\cycleii{\phi a}{\phi\pi\phi^{-1}}{\phi b}$}

\noindent is another exchanging involution.

(c) If $\pi=\cycleii{a}{\pi}{b}$ and $\phi=\cycleii{c}{\phi}{d}$ are
exchanging involutions, and $a$, $b$, $c$, $d$ are all disjoint, then
$\pi$ and $\phi$ commute, and $\psi=\pi\phi=\phi\pi$ is another
exchanging involution, being $\cycleii{a\Bcup c}{\psi}{b\Bcup d}$.

(d)\dvAnew{2010} If $G$ is a countably full subgroup of $\Aut\frak A$,
$a_1,\ldots,a_n\in\frak A$ are disjoint, and $\pi_1,\ldots,\pi_{n-1}\in G$,
then

\Centerline{$\cycle{a_1\,_{\pi_1}\,a_2\,_{\pi_2}\,
\ldots\,_{\pi_{n-1}}\,a_n}\in G$.}
%\Centerline needed for Lulu version

\proof{{\bf (a)} Check the action of $\psi$ on the principal ideals
$\frak A_a$, $\frak A_b$, $\frak A_{1\Bsetminus(a\Bcup b)}$.

\medskip

{\bf (b)} $\phi a\Bcap\phi b=\phi(a\Bcap b)=0$ and

\Centerline{$\phi\pi\phi^{-1}\phi a=\phi\pi a=\phi b$,}

\noindent so $\psi=\cycleii{\phi a}{\phi\pi\phi^{-1}}{\phi b}$ is
well-defined.   Now check the action of $\psi$ on the principal ideals
$\frak A_{\phi a}$, $\frak A_{\phi b}$, $\frak A_{1\Bsetminus\phi(a\Bcup
b)}$.

\medskip

{\bf (c)} Check the action
of $\psi$ on each of the principal ideals $\frak A_a,\ldots,\frak A_e$,
where $e=1\Bsetminus(a\Bcup b\Bcup c\Bcup d)$.

\medskip

{\bf (d)} Immediate from the definitions in 381Rb and 381Be.
}%end of proof of 381S

%3{8}1I
\cmmnt{
\leader{381T}{Remark} I must emphasize that while, after a little
practice, calculations of this kind become easy and safe, they are
absolutely dependent on all the cycles present involving only members of
one list of disjoint elements of $\frak A$.   If, for instance, $a$,
$b$, $c$ are disjoint, then

\Centerline{$\cycleii{a}{\pi}{b}\cycleii{b}{\phi}{c}
=\cycle{a\,_{\pi}\,b\,_{\phi}\,c}$.}

\noindent But if $a\Bcap c\ne 0$ then there is no expression for the
product in this language.    Secondly, of course, we must be scrupulous
in checking, at every use of the notation
$\cycle{a_1\,_{\pi_1}\,\ldots\,a_n}$, that $a_1,\ldots,a_n$ are disjoint
and that $\pi_ia_i=a_{i+1}$ for $i<n$.   Thirdly, a significant problem
can arise if the automorphisms involved don't match.   Consider for
instance the product

\Centerline{$\psi=\cycleii{a}{\pi}{b}\cycleii{a}{\phi}{b}$.}

\noindent Then we have $\psi d=\pi^{-1}\phi d$ if $d\Bsubseteq a$,
$\pi\phi^{-1}d$ if $d\Bsubseteq b$;  $\psi$ is not necessarily
expressible as a product of `disjoint' cycles.   Clearly there are
indefinitely complex variations possible on this theme.   A possible
formal expression of a sufficient condition to avoid these difficulties
is the following.   Restrict yourself to calculations involving a fixed
list $a_1,\ldots,a_n$ of disjoint elements of $\frak A$ for which you
can describe a family of isomorphisms $\phi_{ij}:\frak A_{a_i}\to\frak
A_{a_j}$ such that $\phi_{ii}$ is always the identity on $\frak
A_{a_i}$, $\phi_{jk}\phi_{ij}=\phi_{ik}$ for all $i$, $j$, $k$, and
whenever $a_i\,_{\pi}\,a_j$ appears in a cycle of the calculation, then
$\pi$ agrees with $\phi_{ij}$ on $\frak A_{a_i}$.   Of course this would
be intolerably unwieldy if it were really necessary to exhibit all the
$\phi_{ij}$ every time.   I believe however that it is usually easy
enough to form a mental picture of the actions of the isomorphisms
involved sufficiently clear to offer confidence that such $\phi_{ij}$
are indeed present;  and in cases of doubt, then {\it after} performing
the formal operations it is always straightforward to check that the
calculations are valid, by looking at the actions of the automorphisms
on each relevant principal ideal.
}%end of comment

\exercises{\leader{381X}{Basic exercises (a)}
%\spheader 381Xa %3{8}1Xa
Let $X$ be a set and $\Sigma$ an algebra of
subsets of $X$ containing all singleton sets.   Show that $\Aut\Sigma$
can be identified with the group of permutations $f:X\to X$ such that
$f[E]$ and $f^{-1}[E]$ belong to $\Sigma$ for every $E\in\Sigma$.
%381A

\spheader 381Xb Let $\frak A$ and $\frak B$ be Boolean algebras, and
$\langle a_i\rangle_{i\in I}$, $\langle b_i\rangle_{i\in I}$ partitions
of unity in $\frak A$, $\frak B$ respectively.   Assume
{\it either} that $I$ is finite {\it or} that $I$ is countable and
$\frak B$ is Dedekind $\sigma$-complete {\it or} that $\frak B$ is
Dedekind complete.   Suppose that for each $i\in I$ we have a Boolean
homomorphism $\pi_i:\frak A_{a_i}\to\frak B_{b_i}$.   (i) Show that
there is a Boolean homomorphism $\pi:\frak A\to\frak B$ extending every
$\pi_i$.   (ii) Show that $\pi$ is injective iff every $\pi_i$ is.
(iii) Show that if {\it either} $I$ is finite {\it or} $I$ is countable
and $\frak A$ is Dedekind $\sigma$-complete {\it or} $\frak A$ is
Dedekind complete, then
$\pi$ is surjective iff every $\pi_i$ is.    (iv) Show that
$\pi$ is order-continuous, or sequentially order-continuous,
iff every $\pi_i$ is.
%381C

\spheader 381Xc Let $\frak A$ be a Boolean algebra.   Show that if
$\pi\in\Aut\frak A$ and $k\in\Bbb Z\setminus\{0\}$, then $\pi$ is
aperiodic iff $\pi^k$ is.
%381H

\spheader 381Xd %3{8}5Xb
In 381H, show that the family
$\langle c_i\rangle_{1\le i\le\omega}$ is uniquely determined.
%381H

\sqheader 381Xe Let $(X,\Sigma,\mu)$ be a countably separated measure
space (definition:  343D), $\frak A$ its measure algebra, $f:X\to X$
an \imp\ function and $\pi:\frak A\to\frak A$ the induced homomorphism
(343A).   (i) Show that the support of $\pi$ is
$\{x:x\in X$, $f(x)\ne x\}^{\ssbullet}$.    (ii) Show that $\pi$ is
periodic, with period $n\ge 1$, iff $\mu X>0$,
$f^n(x)=x$ for almost every $x$ and
$\{x:f^i(x)=x\}$ is negligible for $1\le i<n$.
%381H

\spheader 381Xf %3{8}7
Let $(X,\Sigma,\mu)$ be a localizable measure space,
with measure algebra $(\frak A,\bar\mu)$.   Suppose that $\pi$ and
$\phi$ are automorphisms of $\frak A$, and that $\pi$ is represented
by a measure space automorphism $f:X\to X$.   Show that the
following are equiveridical:  (i) $\phi$ belongs to the full subgroup of
$\Aut\frak A$ generated by $\pi$;  (ii) there is a function $g:X\to X$,
representing $\phi$, such that $g(x)\in\{f^n(x):n\in\Bbb Z\}$ for every
$x\in X$.   \Hint{for (ii)$\Rightarrow$(i), consider measurable
envelopes of sets $F\cap g[A_n]$, where $A_n=\{x:g(x)=f^n(x)\}$ and
$\mu F<\infty$.}
%381I mt38bits

\spheader 381Xg %3{8}5Xa
Let $\frak A$ be a Boolean algebra,
not $\{0\}$, and $\pi:\frak A\to\frak A$ an automorphism with
fixed-point subalgebra $\frak C$.   Show that
$\pi$ is periodic, with period $n\ge 1$, iff $\pi\restrp\frak A_c$ has
order $n$ in the group $\Aut\frak A_c$ whenever
$c\in\frak C\setminus\{0\}$.
Show that $\pi$ is aperiodic iff $\pi\restrp\frak A_c$
has infinite order in the group $\Aut\frak A_c$ whenever
$c\in\frak C\setminus\{0\}$.
%381L

\spheader 381Xh\dvAnew{2009}
Let $\frak A$ be a Dedekind complete Boolean algebra,
$G$ a subgroup of $\Aut\frak A$ and $\phi\in\Aut\frak A$.   Show that
$\phi$ belongs to the full subgroup of $\Aut\frak A$ generated by $G$ iff
$\inf_{\pi\in G}\supp(\pi\phi)=0$.
%381I

\spheader 381Xi Let $\frak A$ be a Boolean algebra.   Let us say that a
subgroup $G$ of $\Aut\frak A$ is {\bf finitely full} if whenever
$\langle a_i\rangle_{i\in I}$ is a finite partition of
unity in $\frak A$, $\langle\pi_i\rangle_{i\in I}$ is a family in
$G$, and $\pi\in\Aut\frak A$ is such that $\pi a=\pi_ia_i$ whenever
$i\in I$ and $a\Bsubseteq a_i$, then $\pi\in G$.   Show that if $\pi$,
$\phi\in\Aut\frak A$ then $\phi$ belongs to the finitely full subgroup
of $\Aut\frak A$ generated by $\pi$ iff there are an $n\in\Bbb N$ and a
partition of unity $\langle a_i\rangle_{-n\le i\le n}$ in $\frak A$ such
that $\phi d=\pi^id$ whenever $|i|\le n$ and $d\Bsubseteq a_i$.
%381I

\spheader 381Xj Let $\frak A$ be a Boolean algebra and
$\pi:\frak A\to\frak A$ a Boolean homomorphism which is recurrent on
$a\in\frak A$.   Show that for any non-zero $b\Bsubseteq a$ and any
$n\in\Bbb N$ there is a $k\ge n$ such that $a\Bcap\pi^kb\ne 0$.
%381L

\spheader 381Xk Let $\frak A$ be a Boolean algebra,
$\pi:\frak A\to\frak A$ a Boolean homomorphism, and $a\in\frak A$.
Show that the following are equiveridical:  (i) $\pi$ is recurrent on
every $b\Bsubseteq a$;
(ii) for every non-zero $b\Bsubseteq a$ there is an $n\ge 1$ such that
$b\Bcap\pi^nb\ne 0$;  (iii) $b=\sup_{n\ge 1}b\Bcap\pi^nb$ for every
$b\Bsubseteq a$.
%381L

\sqheader 381Xl Let $(X,\Sigma,\mu)$ be a measure space, $\frak A$ its
measure algebra, $f:X\to X$ a measure space automorphism, and
$\pi$ the corresponding automorphism of $\frak A$.   (i) Show that if
$E\in\Sigma$ then $\pi$ is doubly recurrent on $a=E^{\ssbullet}$ iff
$E\setminus\bigcup_{n\ge 1}f^{-n}[E]$ and
$E\setminus\bigcup_{n\ge 1}f^n[E]$ are negligible.   (ii) Show that in
this case there is a measurable $F\subseteq E$ such that $E\setminus F$
is negligible and $\{n:n\in\Bbb Z$, $f^n(x)\in F\}$ is unbounded above
and below in $\Bbb Z$ for every $x\in F$.   (iii) For $x\in F$ let
$k(x)=\min\{n:n\ge 1$, $f^n(x)\in F\}$.   Show that
$x\mapsto f^{k(x)}(x):F\to F$ represents the induced automorphism
$\pi_a$ on the principal ideal $\frak A_a$.
%381M

\spheader 381Xm For a Boolean algebra $\frak A$, a Boolean homomorphism
$\pi:\frak A\to\frak A$ is {\bf nowhere aperiodic} if
$\inf\{a:a\in\frak A$, $a$ supports $\pi^n$ for some $n\ge 1\}=0$.
Show that if $\frak A$ is Dedekind $\sigma$-complete and
$\pi\in\Aut\frak A$ is nowhere aperiodic and doubly recurrent on
$a\in\frak A$, then the induced automorphism $\pi_a$ is nowhere
aperiodic.
%381N

\spheader 381Xn Let $\frak A$ be a Dedekind $\sigma$-complete Boolean
algebra, $\pi\in\Aut\frak A$ an automorphism and $\frak C$ the
fixed-point subalgebra of $\pi$.   Suppose that $\pi$ is doubly
recurrent on $a\in\frak A$ and that $\pi_a$ is the induced automorphism
on $\frak A_a$.   Show that the fixed-point subalgebra of $\pi_a$ is
$\{c\Bcap a:c\in\frak C\}$, so that if $\pi$ is ergodic, so is
$\pi_a$.
%381N

\spheader 381Xo Let $\frak A$ be a Boolean algebra with Stone space $Z$,
and $\pi:\frak A\to\frak A$ a Boolean homomorphism corresponding to
$f:Z\to Z$.   (i) Show that $\pi$ is periodic, with period $n\ge 1$, iff
$Z\ne\emptyset$,
$f^n(z)=z$ for every $z\in Z$ and $\{z:f^i(z)=z\}$ is nowhere dense
whenever $1\le i<n$.   (ii) Show that $\pi$ is aperiodic iff
$\{z:f^n(z)=z$, $f^n(w)\ne z$ for every $w\ne z\}$ is nowhere dense for
every $n\ge 1$.
%381Q

\spheader 381Xp Let $\frak A$ be a Dedekind $\sigma$-complete Boolean
algebra, $G$ a subgroup of $\Aut\frak A$ and
$G^*$ the countably full subgroup
of $\Aut\frak A$ generated by $G$.   Suppose that every member of $G$ has a
support.   Show that every member of $G^*$ has a support.
%381I out of order query

\leader{381Y}{Further exercises (a)}
%\spheader 381Ya %3{8}5Ya
(i) Give an example to show that the word `injective' in
the statement of 381H is essential.   (ii) Give an example to show that, in
381H, we can have $\pi c_{\omega}\ne c_{\omega}$.
%381H

\spheader 381Yb %3{8}7Ya
Let $\frak A$ be a Dedekind complete Boolean algebra and
$G$ a semigroup of order-continuous Boolean homomorphisms from $\frak A$
to itself.   Let us say that $G$ is {\bf full} if whenever
$\phi:\frak A\to\frak A$ is an order-continuous Boolean homomorphism,
and there is a
partition of unity $\langle a_i\rangle_{i\in I}$ in $\frak A$ such that
for every $i\in I$ there is a $\pi_i\in G$ such that $\phi a=\pi_i a$
for every $a\Bsubseteq a_i$, then $\phi\in G$.   Show that if $\phi$ and
$\pi$ are order-continuous Boolean homomorphisms from $\frak A$ to
itself, then the following are equiveridical:  (i) $\phi$ belongs to the
full semigroup generated by $\pi$;  (ii) for every non-zero
$a\in\frak A$ there are a non-zero $b\Bsubseteq a$ and an $n\in\Bbb N$
such that
$\phi d=\pi^nd$ for every $d\Bsubseteq b$;  (iii) there is a partition
of unity $\langle a_n\rangle_{n\in\Bbb N}$ in $\frak A$ such that
$\phi a=\pi^na$ whenever $n\in\Bbb N$ and $a\Bsubseteq a_n$.
%381I

\spheader 381Yc Give an example of a Dedekind $\sigma$-complete Boolean
algebra $\Aut\frak A$ and an automorphism $\pi$ of $\frak A$ such
that the countably full subgroup generated by $\pi$ is not full.
% \frak A=\Bbb Z\times\omega_1, countable-cocountable algebra
%381I

\spheader 381Yd Let $\frak A$ be a Dedekind complete Boolean algebra,
and let $G$ be the countably full subgroup of $\Aut\frak A$ generated by
a subset $A$ of $\Aut\frak A$.   Show that if {\it either} $A$ is
countable {\it or} $\frak A$ is ccc, then $G$ is full.
%381I
}%end of exercises

\endnotes{
\Notesheader{381} There are no long individual proofs in this section,
and in so far as there is any delicacy in the arguments it is as often
as not because (as in 381E) I am taking facts which are easy to prove
for automorphisms of Dedekind complete algebras and separating out the
parts which happen to be true in greater generality.   However the parts
are numerous enough for the sum to be not entirely predictable.   The
most important ideas are surely in 381M-381N.

In 381Q I give indications, including the minimum necessary for an
application in the next section, of how to express the concepts here in
terms of continuous functions on Stone spaces.   When we come, in \S383
and onwards, to look specifically at measure algebras, many of our
homomorphisms will be derived from \imp\ functions, and the results will
be more effective if we can display them in terms of functions on
measure spaces.   Some appropriate translations are in
381Xe-381Xl.  %381Xe 381Xf 381Xl
But these I will avoid in the proofs of the main theorems because not
all automorphisms of measure algebras can be represented by
automorphisms of the measure spaces we start from (343Jc).   Of course
Lebesgue measure is different, in ways explored in \S344, and classical
ergodic theory has not needed to make a clear distinction here.   One of
my purposes in this volume is to set out a framework in which
transformations of measure {\it spaces} take their proper place as an
inspiration for the theory rather than a foundation.
}%end of notes

\discrpage

