\frfilename{mt36.tex}
\versiondate{13.8.98/2.2.02}
\copyrightdate{1996}

\def\chaptername{Function Spaces}
\def\sectionname{Introduction}

\newchapter{36}

Chapter 24 of Volume 2 was devoted to the elementary theory of the \lq
function spaces' $L^0$, $L^1$, $L^2$ and $L^{\infty}$ associated with a
given measure space.
In this chapter I return to these spaces to show how they can be related
to the more abstract themes of the present volume.   In particular, I
develop constructions to demonstrate, as clearly as I can, the way in
which all the function spaces associated with a measure space in fact
depend only on its measure algebra;  and how many of their features can
(in my view) best be understood in terms of constructions involving
measure algebras.

The chapter is very long, not because there are many essentially new ideas, but because the intuitions I seek to develop depend,
for their logical foundations, on technically complex arguments.   This
is perhaps best exemplified by \S364.   If two measure spaces
$(X,\Sigma,\mu)$ and $(Y,\Tau,\nu)$ have isomorphic measure algebras
$(\frak A,\bar\mu)$, $(\frak B,\bar\nu)$ then the spaces $L^0(\mu)$,
$L^0(\nu)$ are isomorphic as topological $f$-algebras;  and more:  for
any isomorphism between $(\frak A,\bar\mu)$ and $(\frak B,\bar\nu)$
there is a unique corresponding isomorphism between the $L^0$ spaces.
The intuition involved is in a way very simple.   If $f$, $g$ are
measurable real-valued functions on $X$ and $Y$ respectively, then
$f^{\ssbullet}\in L^0(\mu)$ will correspond to $g^{\ssbullet}\in
L^0(\nu)$ if and only if
$\Bvalue{f^{\ssbullet}>\alpha}=\{x:f(x)>\alpha\}^{\ssbullet}\in\frak A$
corresponds to
$\Bvalue{g^{\ssbullet}>\alpha}=\{y:g(y)>\alpha\}^{\ssbullet}\in\frak B$
for every $\alpha$.
But the check that this formula is consistent, and defines an
isomorphism of the required kind, involves a good deal of detailed work.
It turns out, in fact, that the {\it measures} $\mu$ and $\nu$ do not
enter this part of the argument at all, except through their ideals of
negligible sets (used in the construction of $\frak A$ and $\frak B$).
This is already evident, if you look for it, in the theory of
$L^0(\mu)$;  in \S241, as written out, you will find that the measure of an individual set is not once mentioned, except in the exercises.
Consequently there is an invitation to develop the theory with algebras
$\frak A$ which are not necessarily measure algebras.   Here is another
reason for the length of the chapter;  substantial parts of the work are
being done in greater generality than the corresponding sections of
Chapter 24, necessitating a degree of repetition.   Of course this is
not \lq measure theory' in the strict sense;  but for thirty years now
measure theory has been coloured by the existence of these
generalizations, and I think it is useful to understand which parts
of the theory apply only to measure algebras, and which can be extended
to other $\sigma$-complete Boolean algebras, like the algebraic theory
of $L^0$, or even to all Boolean algebras, like the theory of
$L^{\infty}$.

Here, then, are two of the objectives of this chapter:  first, to
express the ideas of Chapter 24 in ways making explicit their
independence of particular measure spaces, by setting up constructions
based exclusively on the measure algebras involved;  second, to set out
some natural generalizations to other algebras.   But to justify the
effort needed I ought to point to some mathematically significant idea
which demands these constructions for its expression, and here I mention
the categorical nature of the constructions.   Between Boolean algebras
we have a variety of natural and important classes of \lq morphism';
for instance, the Boolean homomorphisms and the
order-continuous Boolean homomorphisms;  while between measure algebras
we have in addition the
measure-preserving Boolean homomorphisms.   Now it turns out that if we
construct the $L^p$ spaces in the natural ways then morphisms between
the underlying algebras give rise to morphisms between their $L^p$
spaces.   For instance, any Boolean homomorphism from $\frak A$ to
$\frak B$ produces a multiplicative norm-contractive Riesz homomorphism
from $L^{\infty}(\frak A)$ to $L^{\infty}(\frak B)$;  if $\frak A$ and
$\frak B$ are Dedekind $\sigma$-complete, then any sequentially
order-continuous Boolean homomorphism from $\frak A$ to $\frak B$
produces a sequentially order-continuous multiplicative Riesz
homomorphism from $L^0(\frak A)$ to $L^0(\frak B)$;  and if $(\frak
A,\bar\mu)$ and $(\frak B,\bar\nu)$ are measure algebras, then any
measure-preserving Boolean homomorphism from $\frak A$ to $\frak B$
produces norm-preserving Riesz homomorphisms from $L^p(\frak A,\bar\mu)$
to $L^p(\frak B,\bar\nu)$ for every $p\in[1,\infty]$.   All of
these are \lq functors', that is, a composition of homomorphisms between
algebras gives rise to a composition of the corresponding operators
between their function spaces, and are \lq covariant', that is, a
homomorphism from $\frak A$ to $\frak B$ leads to an operator from
$L^p(\frak A)$ to $L^p(\frak B)$.   But the same constructions lead us
to a functor which is \lq contravariant':  starting from an
order-continuous Boolean homomorphism from a semi-finite measure algebra
$(\frak A,\bar\mu)$ to a measure algebra $(\frak B,\bar\nu)$, we have an
operator from $L^1(\frak B,\bar\nu)$ to $L^1(\frak A,\bar\mu)$.   This
last is in fact a kind of conditional expectation operator.   In my view
it is not possible to make sense of the theory of measure-preserving
transformations without at least an intuitive grasp of these ideas.

Another theme is the characterization of each construction in terms of
universal mapping theorems:  for instance, each $L^p$ space, for $1\le
p\le\infty$, can be characterized as Banach lattice in terms of
factorizations of functions of an appropriate class from the underlying
algebra to Banach lattices.

Now let me try to sketch a route-map for the journey ahead.   I begin
with two sections on the space $S(\frak A)$;  this construction applies
to any Boolean algebra (indeed, any Boolean ring), and corresponds to
the space of `simple functions' on a measure space.   Just because it
is especially close to the algebra (or ring) $\frak A$, there is a
particularly large number of universal mapping theorems corresponding to
different aspects of its structure (\S361).   In \S362 I seek to relate
ideas on additive functionals on Boolean algebras from Chapter 23 and
\S\S326-327 to the theory of Riesz space duals in \S356.   I then
turn to the systematic discussion of the function spaces of Chapter 24:
$L^{\infty}$ (\S363), $L^0$ (\S364), $L^1$ (\S365) and other $L^p$
(\S366), followed by an account of convergence in measure (\S367).
While all these sections are dominated by the objectives sketched in the
paragraphs above, I do include a few major theorems not covered by the
ideas of Volume 2, such as the Kelley-Nachbin characterization of the
Banach spaces $L^{\infty}(\frak A)$ for Dedekind complete $\frak A$
(363R).   In the last two sections of the chapter I turn to
the use of $L^0$ spaces in the representation of Archimedean Riesz
spaces (\S368) and of Banach lattices separated by their
order-continuous duals (\S369).

\discrpage

