\frfilename{mt354.tex}
\versiondate{18.8.08}
\copyrightdate{2000}

\def\chaptername{Riesz Spaces}
\def\sectionname{Banach lattices}

\newsection{354}

The next step is a brief discussion of norms on Riesz spaces.   I start
with the essential definitions (354A, 354D) with the principal
properties of
general Riesz norms (354B-354C) and order-continuous norms (354E).   I
then describe two of the most important classes of Banach lattice:
$M$-spaces (354F-354L) %354F 354G 354H 354I 354J 354K 354L
and $L$-spaces (354M-354R), %354M 354N 354O 354P 354Q 354R
with their elementary
properties.   For $M$-spaces I give the basic representation theorem
(354K-354L), and for $L$-spaces I give a note on uniform integrability
(354P-354R).  %354P 354Q 354R

\leader{354A}{Definitions (a)} If $U$ is a Riesz space, a {\bf Riesz
norm} or {\bf lattice norm}
%Deville Godefroy & Zizler
on $U$ is a norm $\|\,\|$ such that $\|u\|\le\|v\|$ whenever
$|u|\le|v|$\cmmnt{;  that is, a norm such that $\||u|\|=\|u\|$ for
every $u$ and $\|u\|\le\|v\|$ whenever $0\le u\le v$}.

\spheader 354Ab A {\bf Banach lattice} is a Riesz space with a Riesz
norm under which it is complete.

\cmmnt{\medskip

\noindent{\bf Remark} We have already seen many examples of Banach
lattices;  I list some in 354Xa below.
}

\leader{354B}{Lemma} Let $U$ be a Riesz space with a Riesz norm
$\|\,\|$.

(a) $U$ is Archimedean.

(b) The maps $u\mapsto|u|$ and $u\mapsto u^+$ are uniformly continuous.

(c) For any $u\in U$, the sets $\{v:v\le u\}$ and $\{v:v\ge u\}$ are
closed;  in particular, the positive cone of $U$ is closed.

(d) Any band in $U$ is closed.

(e) If $V$ is a norm-dense Riesz subspace of $U$, then $V^+=\{v:v\in
V,\,v\ge 0\}$ is norm-dense in the positive cone $U^+$ of $U$.

\proof{{\bf (a)} If $u$, $v\in U$ are such that $nu\le v$ for every
$n\in\Bbb N$, then $nu^+\le v^+$ so $n\|u^+\|\le\|v^+\|$ for every $n$,
and $\|u^+\|=0$, that is, $u^+=0$ and $u\le 0$.   As $u$, $v$ are
arbitrary, $U$ is Archimedean.

\medskip

{\bf (b)} For any $u$, $v\in U$, $||u|-|v||\le|u-v|$ (352D), so
$\||u|-|v|\|\le\|u-v\|$;  thus $u\mapsto|u|$ is uniformly continuous.
Consequently $u\mapsto \bover12(u+|u|)=u^+$ is uniformly continuous.

\medskip

{\bf (c)} Now $\{v:v\le u\}=\{v:(v-u)^+=0\}$ is closed because the
function $v\mapsto (v-u)^+$ is continuous and $\{0\}$ is closed.
Similarly $\{v:v\ge u\}=\{v:(u-v)^+=0\}$ is closed.

\medskip

{\bf (d)} If $V\subseteq U$ is a band, then $V=V^{\perp\perp}$ (353Bb),
that is, $V=\{v:|v|\wedge|w|=0$ for every $w\in V^{\perp}\}$.   Because
the function $v\mapsto |v|\wedge|w|=\bover12(|v|+|w|-||v|-|w||)$ is
continuous, all the sets $\{v:|v|\wedge|w|=0\}$ are closed, and so is
their intersection $V$.

\medskip

{\bf (e)} Observe that $V^+=\{v^+:v\in V\}$ and $U^+=\{u^+:u\in U\}$;
recall that $u\mapsto u^+$ is continuous, and apply 3A3Eb.
}%end of proof of 354B

\leader{354C}{Lemma} If $U$ is a Banach lattice and $\sequencen{u_n}$ is
a sequence in $U$ such that $\sum_{n=0}^{\infty}\|u_n\|<\infty$, then
$\sup_{n\in\Bbb N}u_n$ is defined in $U$, with
$\|\sup_{n\in\Bbb N}u_n\|\le\sum_{n=0}^{\infty}\|u_n\|$.

\proof{ Set $v_n=\sup_{i\le n}u_i$ for each $n$.   Then

\Centerline{$0\le v_{n+1}-v_n\le(u_{n+1}-u_n)^+\le|u_{n+1}-u_n|$}

\noindent for each $n\in\Bbb N$, so

\Centerline{$\sum_{n=0}^{\infty}\|v_{n+1}-v_n\|
\le\sum_{n=0}^{\infty}\|u_{n+1}-u_n\|
\le\sum_{n=0}^{\infty}\|u_{n+1}\|+\|u_n\|$}

\noindent is finite, and $\sequencen{v_n}$ is Cauchy.   Let $u$ be its
limit;  because $\sequencen{v_n}$ is non-decreasing, and the sets
$\{v:v\ge v_n\}$ are all closed, $u\ge v_n$ for each $n\in\Bbb N$.   On
the other hand, if $v\ge v_n$ for every $n$, then

\Centerline{$(u-v)^+=\lim_{n\to\infty}(v_n-v)^+=0$,}

\noindent and $u\le v$.   So

\Centerline{$u=\sup_{n\in\Bbb N}v_n=\sup_{n\in\Bbb N}u_n$}

\noindent is the required supremum.

To estimate its norm, observe that $|v_n|\le\sum_{i=0}^n|u_i|$ for each
$n$ (induce on $n$, using the last item in 352D for the inductive step),
so that

\Centerline{$\|u\|=\lim_{n\to\infty}\|v_n\|
\le\sum_{i=0}^{\infty}\||u_i|\|=\sum_{i=0}^{\infty}\|u_i\|$.}
}%end of proof of 354C

\leader{354D}{}\cmmnt{ I come now to the basic properties according to
which we classify Riesz norms.

\medskip

\noindent}{\bf Definitions (a)} A {\bf Fatou norm} on a Riesz space $U$
is a Riesz norm on $U$ such that whenever $A\subseteq U^+$ is non-empty
and upwards-directed and has a least upper bound in $U$, then
$\|\sup A\|=\sup_{u\in A}\|u\|$.   \cmmnt{(Observe that, once we know
that $\|\,\|$ is a Riesz norm, we can be sure that $\|u\|\le\|\sup A\|$
for every $u\in A$, so that all we shall need to check is that
$\|\sup A\|\le\sup_{u\in A}\|u\|$.)}

\spheader 354Db A Riesz norm on a Riesz space $U$ has the {\bf Levi
property} if every upwards-directed norm-bounded set is bounded above.

\spheader 354Dc A Riesz norm on a Riesz space $U$ is
{\bf order-continuous} if $\inf_{u\in A}\|u\|=0$ whenever $A\subseteq U$
is a non-empty downwards-directed set with infimum $0$.

\leader{354E}{Proposition} Let $U$ be a Riesz space with an
order-continuous Riesz norm $\|\,\|$.

(a) If $A\subseteq U$ is non-empty and upwards-directed and has a
supremum, then $\sup A\in\overline{A}$.

(b) $\|\,\|$ is Fatou.

(c) If $A\subseteq U$ is non-empty and upwards-directed and bounded
above, then for every $\epsilon>0$ there is a $u\in A$ such that
$\|(v-u)^+\|\le\epsilon$ for every $v\in A$;
%new 2008
that is, the filter $\Cal F(A\closeuparrow)$ on $U$ generated by
$\{\{v:v\in A$, $u\le v\}:u\in A\}$ is a Cauchy filter.

(d) Any non-decreasing order-bounded sequence in $U$ is Cauchy.

(e) If $U$ is a Banach lattice it is Dedekind complete.

(f) Every order-dense Riesz subspace of $U$ is norm-dense.

%new 2008
(g) Every norm-closed solid linear subspace of $U$ is a band.

\proof{{\bf (a)} Suppose that $A\subseteq U$ is non-empty and
upwards-directed and has a least upper bound $u_0$.    Then
$B=\{u_0-u:u\in A\}$ is downwards-directed and has infimum $0$.   So
$\inf_{u\in A}\|u_0-u\|=0$, and $u_0\in\overline{A}$.

\medskip

{\bf (b)} If, in (a), $A\subseteq U^+$, then we must have

\Centerline{$\|u_0\|\le\inf_{u\in A}\|u\|+\|u-u_0\|
\le\sup_{u\in A}\|u\|$.}

\noindent As $A$ is arbitary, $\|\,\|$ is a Fatou norm.

\medskip

{\bf (c)} Let $B$ be the set of upper bounds for $A$.   Then $B$ is
downwards-directed;  because $A$ is upwards-directed,
$B-A=\{v-u:v\in B,\,u\in A\}$ is downwards-directed.  By 353F,
$\inf(B-A)=0$.   So
there are $w\in B$, $u\in A$ such that $\|w-u\|\le\epsilon$.   Now if
$v\in A$,

\Centerline{$(v-u)^+=(v\vee u)-u\le w-u$,}

\noindent so $\|(v-u)^+\|\le\epsilon$.

In terms of the filter $\Cal F(A\closeuparrow)$, this tells us
that if $v_0$, $v_1$ belong to $F_u=\{v:v\in A$, $v\ge u\}$ then
$|v_0-v_1|\le w-u$ so $\|v_0-v_1\|\le\epsilon$ and the diameter of $F_u$ is
at most $\epsilon$.   As $\epsilon$ is arbitrary,
$\Cal F(A\closeuparrow)$ is a Cauchy filter.

\medskip


{\bf (d)} If $\sequencen{u_n}$ is a non-decreasing order-bounded
sequence, and $\epsilon>0$, then, applying (c)
to $\{u_n:n\in\Bbb N\}$, we find
that there is an $m\in\Bbb N$ such that $\|u_m-u_n\|\le\epsilon$
whenever $m\ge n$.

\medskip

{\bf (e)} Now suppose that $U$ is a Banach lattice.   Let $A\subseteq U$
be any non-empty set with an upper bound.   Set
$B=\{u_0\vee\ldots\vee u_n:u_0,\ldots,u_n\in A\}$, so that $B$ is
upwards-directed and has the same upper bounds as $A$.
Let $\Cal F(B\closeuparrow)$ be the filter on $U$ generated by
$\{B\cap\coint{v,\infty}:v\in B\}$.   By (c), this is a Cauchy filter with
a limit $u^*$ say.   For every $u\in A$, $\coint{u,\infty}$ is a closed set
belonging to $\Cal F(B\closeuparrow)$, so contains $u^*$;  thus $u^*$ is an
upper bound for $A$.   If $w$ is any upper bound for $A$,
$\ocint{-\infty,w}$ is a closed set belonging to $\Cal F(B\closeuparrow)$,
so contains $u^*$;  thus $u^*=\sup A$ and $A$ has a supremum.

\medskip

{\bf (f)} If $V$ is an order-dense Riesz subspace of $U$ and $u\in U^+$,
set $A=\{v:v\in V,\,v\le u\}$.   Then $A$ is upwards-directed and
has supremum $u$, so $u\in\overline{A}\subseteq\overline{V}$, by (a).
Thus $U^+\subseteq\overline{V}$;  it follows at once that
$U=U^+-U^+\subseteq\overline{V}$.

\medskip

{\bf (g)} If $V$ is a norm-closed solid linear subspace of $U$,
and $A\subseteq V^+$ is a non-empty, upwards-directed subset of $V$ with a
supremum in $U$, then $\sup A\in\overline{A}\subseteq V$, by (a);  by
352Ob, $V$ is a band.
}%end of proof of 354E

\vleader{72pt}{354F}{Lemma} If $U$ is an Archimedean Riesz space with an
order unit $e$\cmmnt{ (definition:  353L)}, there is a Riesz norm
$\|\,\|_e$ defined on $U$ by the formula

\Centerline{$\|u\|_e=\min\{\alpha:\alpha\ge 0$, $|u|\le\alpha e\}$}

\noindent for every $u\in U$.

\proof{ This is a routine verification.   Because $e$ is an order-unit,
$\{\alpha:\alpha\ge 0$, $|u|\le\alpha e\}$ is always non-empty,
so always has an
infimum $\alpha_0$ say;  now $|u|-\alpha_0e\le\delta e$ for every
$\delta>0$, so (because $U$ is Archimedean) $|u|-\alpha_0e\le 0$ and
$|u|\le\alpha_0e$, so that the minimum is attained.   In particular,
$\|u\|_e=0$ iff $u=0$.   The subadditivity and homogeneity of $\|\,\|_e$
are immediate from the facts that $|u+v|\le|u|+|v|$, $|\alpha
u|=|\alpha||u|$.
}%end of proof

\leader{354G}{Definitions (a)} If $U$ is an Archimedean Riesz space and
$e$ an order unit in $U$, the norm $\|\,\|_e$ as defined in 354F is the
{\bf order-unit norm} on $U$ associated with $e$.

\spheader 354Gb An {\bf $M$-space} is a Banach lattice in which the norm
is an order-unit norm.

\spheader 354Gc If $U$ is an $M$-space, its {\bf standard order unit} is
the order unit $e$ such that $\|\,\|_e$ is the norm of $U$.
\cmmnt{(To see that $e$ is uniquely defined, observe that it is
$\sup\{u:u\in U,\,\|u\|\le 1\}$.)}

\leader{354H}{Examples (a)} For any set $X$, $\ell^{\infty}(X)$ is an
$M$-space with standard order unit $\chi X$.   \cmmnt{(As remarked in
243Xl, the completeness of $\ell^{\infty}(X)$ can be regarded as the
special case of 243E in which $X$ is given counting measure.)}

\spheader 354Hb For any topological space $X$, the space $C_b(X)$ of
bounded continuous real-valued functions on $X$ is an $M$-space with
standard order
unit $\chi X$.   \cmmnt{(It is a Riesz subspace of $\ell^{\infty}(X)$
containing the order unit of $\ell^{\infty}(X)$, therefore in its own
right an Archimedean Riesz space with order unit.   To see that it is
complete, it is enough to observe that it is closed in
$\ell^{\infty}(X)$ because a uniform limit of continuous functions is
continuous (3A3Nb).)}

\spheader 354Hc For any measure space $(X,\Sigma,\mu)$, the space
$L^{\infty}(\mu)$ is an $M$-space with standard order unit $\chi
X^{\ssbullet}$.

\leader{354I}{Lemma} Let $U$ be an Archimedean Riesz space with order
unit $e$, and $V$ a subset of $U$ which is dense for the
order-unit norm $\|\,\|_e$.   Then for any $u\in U$ there are sequences
$\sequencen{v_n}$, $\sequencen{w_n}$ in $V$ such that
$v_n\le v_{n+1}\le u\le w_{n+1}\le w_n$ and $\|w_n-v_n\|_e\le 2^{-n}$
for every $n$;  so that $u=\sup_{n\in\Bbb N}v_n=\inf_{n\in\Bbb N}w_n$ in
$U$.

If $V$ is a Riesz subspace of $U$, and $u\ge 0$, we may suppose that
$v_n\ge 0$ for every $n$.   Consequently $V$ is order-dense in $U$.

\proof{ For each $n\in\Bbb N$,
take $v_n$, $w_n\in V$ such that

\Centerline{$\|u-\Bover3{2^{n+3}}e-v_n\|_e\le\Bover1{2^{n+3}}$,
\quad$\|u+\Bover3{2^{n+3}}e-w_n\|_e\le\Bover1{2^{n+3}}$.}

\noindent Then

\Centerline{$u-\Bover1{2^{n+1}}e\le v_n\le u-\Bover1{2^{n+2}}e\le u
\le u+\Bover1{2^{n+2}}e\le w_n\le u+\Bover1{2^{n+1}}e$.}

\noindent Accordingly $\sequencen{v_n}$ is non-decreasing,
$\sequencen{w_n}$ is non-increasing and $\|w_n-v_n\|_e\le 2^{-n}$ for
every $n$.   Because $U$ is Archimedean,
$\sup_{n\in\Bbb N}v_n=\inf_{n\in\Bbb N}w_n=u$.

If $V$ is a Riesz subspace of $U$, then replacing $v_n$ by $v_n^+$ if
necessary we may suppose that every $v_n$ is non-negative;  and $V$ is
order-dense by the definition in 352Na.
}%end of proof of 354I

\leader{354J}{Proposition} Let $U$ be an Archimedean Riesz space with an
order unit $e$.   Then $\|\,\|_e$ is Fatou and has the Levi property.

\proof{ This is elementary.   If $A\subseteq U^+$ is non-empty,
upwards-directed and norm-bounded, then it is bounded above by
$\alpha e$, where $\alpha=\sup_{u\in A}\|u\|_e$.   This is all that is
called for in the Levi property.   If
moreover $\sup A$ is defined, then $\sup A\le\alpha e$ so
$\|\sup A\|\le\alpha$, as required in the Fatou property.
}%end of proof of 354J

\leader{354K}{Theorem} Let $U$ be an Archimedean Riesz space with order
unit $e$.   Then it can be embedded as an
order-dense and norm-dense Riesz subspace of $C(X)$, where $X$ is a
compact Hausdorff space, in such a way that $e$ corresponds to $\chi X$
and $\|\,\|_e$ corresponds to $\|\,\|_{\infty}$;  moreover, this
embedding is essentially unique.

\proof{ This is nearly word-for-word a repetition of 353M.   The only
addition is the mention of the norms.   Let $X$ and $T:U\to C(X)$ be
as in 353M.   Then, for any $u\in U$, $|u|\le\|u\|_ee$, so that

\Centerline{$|Tu|=T|u|\le\|u\|_eTe=\|u\|_e\chi X$,}

\noindent and $\|Tu\|_{\infty}\le\|u\|_e$.   On the other hand, if
$0<\delta<\|u\|_e$ then $u_1=(|u|-\delta e)^+>0$, so that
$Tu_1=(|Tu|-\delta\chi X)^+>0$ and $\|Tu\|_{\infty}\ge\delta$;  as
$\delta$ is arbitrary, $\|Tu\|_{\infty}\ge\|u\|_e$.
}%end of proof of 354K

\leader{354L}{Corollary} Any $M$-space $U$ is isomorphic,
as Banach lattice,
to $C(X)$ for some compact Hausdorff $X$, and the isomorphism is
essentially unique.
$X$ can be identified with the set of Riesz homomorphisms $x:U\to\Bbb R$
such that $x(e)=1$, where $e$ is the standard order unit of $U$,
with the topology induced by the product topology on $\BbbR^U$.

\proof{ By 354K,
there are a compact Hausdorff space $X$ and an embedding of $U$ as a
norm-dense Riesz subspace of $C(X)$ matching $\|\,\|_e$ to
$\|\,\|_{\infty}$.   Since $U$ is complete under $\|\,\|_e$, its image
is closed in $C(X)$
(3A4Ff), and must be the whole of $C(X)$.   The expression is
unique just in so far as the expression of 353M/354K is unique.
In particular, we may, if we wish, take $X$ to be the set of normalized
Riesz homomorphisms from $U$ to $\Bbb R$, as in the proof of 353M.
}%end of proof of 354L

\cmmnt{\medskip

\noindent{\bf Remark} The set of uniferent Riesz
homomorphisms from $U$ to $\Bbb R$ is sometimes called the {\bf
spectrum} of $U$.
}%end of comment

\leader{354M}{}\cmmnt{ I come now to a second fundamental class of
Banach lattices, in a strong sense `dual' to the class of $M$-spaces,
as will appear in \S356.

\medskip

\noindent}{\bf Definition} An {\bf $L$-space} is a Banach lattice $U$
such that $\|u+v\|=\|u\|+\|v\|$ whenever $u$, $v\in U^+$.

\medskip

\noindent{\bf Example} If $(X,\Sigma,\mu)$ is any measure space, then
$L^1(\mu)$, with its norm $\|\,\|_1$, is an
$L$-space\cmmnt{ (242D, 242F)}.   \cmmnt{In
particular, taking $\mu$ to be counting measure on $\Bbb N$,} $\ell^1$
is an $L$-space\cmmnt{ (242Xa)}.

\leader{354N}{Theorem} If $U$ is an $L$-space, then its norm is
order-continuous and has the Levi property.

\proof{{\bf (a)} Both of these are consequences of the following fact:
if $A\subseteq U$ is norm-bounded and non-empty and upwards-directed,
then $\sup A$ is defined in $U$ and belongs to the norm-closure of $A$
in $U$.   \Prf\ Fix $u_0\in A$;  set $B=\{u-u_0:u\in A$, $u\ge u_0\}$.
Then $B\subseteq U^+$ is norm-bounded, non-empty and upwards-directed.
Set $\gamma=\sup_{u\in B}\|u\|$.
Consider the filter $\Cal F(B\closeuparrow)$
on $U$ generated by sets of the form $\{v:v\in B$, $v\ge u\}$ for $u\in B$.
If $\epsilon>0$ there is a $u\in B$ such that $\|u\|\ge\gamma-\epsilon$;
now if $v$, $v'\in B\cap\coint{u,\infty}$, there is a $w\in B$ such that
$v\vee v'\le w$, so that

\Centerline{$\|v-v'\|\le\|w-u\|=\|w\|-\|u\|\le\epsilon$.}

\noindent As $\epsilon$ is arbitrary, $\Cal F(B\closeuparrow)$ is
Cauchy and has
a limit $u^*$ say.   If $u\in B$, $\coint{u,\infty}$ is a closed set
belonging to $\Cal F(B\closeuparrow)$, so contains $u^*$;
thus $u^*$ is an
upper bound for $B$.   If $w$ is an upper bound for $B$, then
$\ocint{-\infty,w}$ is a closed set belonging to $\Cal F(B\closeuparrow)$, so
contains $u^*$;  thus $u^*$ is the least upper bound of $B$.   And
$B\in\Cal F(B\closeuparrow)$, so $u^*\in\overline{B}$.

Because $u\mapsto u_0$ is an order-preserving homeomorphism,

\Centerline{$u^*+u_0=\sup\{u:u_0\le u\in A\}=\sup A$}

\noindent and $u^*+u_0\in\overline{A}$, as required.\ \Qed

\medskip

{\bf (b)} This shows immediately that the norm has the Levi property.
But also it must be order-continuous.   \Prf\ If $A\subseteq U$ is
non-empty and downwards-directed and has infimum $0$, take any
$u_0\in A$ and consider $B=\{u_0-u:u\in A,\,u\le u_0\}$.   Then $B$ is
upwards-directed and has supremum $u_0$, so $u_0\in\overline{B}$ and

\Centerline{$\inf_{u\in A}\|u\|\le\inf_{v\in B}\|u_0-v\|=0$.  \Qed}
}%end of proof of 354N

\leader{354O}{Proposition} If $U$ is an $L$-space and $V$ is a
norm-closed Riesz subspace of $U$, then $V$ is an $L$-space in its own
right.   In particular, any band in $U$ is an $L$-space.

\proof{ For any Riesz subspace $V$ of $U$, we surely have
$\|u+v\|=\|u|+\|v\|$ whenever $u$, $v\in V^+$;  so if $V$ is
norm-closed, therefore a Banach lattice, it must be an $L$-space.   But
in any Banach lattice, a band is norm-closed (354Bd), so a band in an
$L$-space is again an $L$-space.
}%end of proof of 354O

\leader{354P}{Uniform integrability
in \dvrocolon{$L$-spaces}}\cmmnt{ Some of the ideas of \S246 can be
readily expressed in this abstract context.

\medskip

\noindent}{\bf Definition} Let $U$ be an $L$-space.   A set
$A\subseteq U$ is {\bf uniformly integrable} if for every $\epsilon>0$
there is a
$w\in U^+$ such that $\|(|u|-w)^+\|\le\epsilon$ for every $u\in A$.

\leader{354Q}{}\cmmnt{ Since I have already used the phrase
`uniformly integrable' based on a different formula, I had better check
instantly that the two definitions are consistent.

\medskip

\noindent}{\bf Proposition} If $(X,\Sigma,\mu)$ is any measure space,
then a subset of $L^1=L^1(\mu)$ is uniformly integrable in the sense of
354P iff it is uniformly integrable in the sense of 246A.

\proof{{\bf (a)} If $A\subseteq L^1$ is uniformly integrable in the
sense of 246A, then for any $\epsilon>0$ there are $M\ge 0$,
$E\in\Sigma$ such that $\mu E<\infty$ and
$\int(|u|-M\chi E^{\ssbullet})^+\le\epsilon$ for every $u\in A$;  now $w=M\chi E^{\ssbullet}$ belongs to $(L^1)^+$ and
$\|(|u|-w)^+\|\le\epsilon$ for
every $u\in A$.   As $\epsilon$ is arbitrary, $A$ is uniformly
integrable in the sense of 354P.

\medskip

{\bf (b)} Now suppose that $A$ is uniformly integrable in the sense of
354P.   Let $\epsilon>0$.   Then there is a $w\in (L^1)^+$ such that
$\|(|u|-w)^+\|\le\bover12\epsilon$ for every $u\in A$.   There is a
simple function $h:X\to\Bbb R$ such that
$\|w-h^{\ssbullet}\|\le\bover12\epsilon$ (242Mb);  now take
$E=\{x:h(x)\ne 0\}$, $M=\sup_{x\in X}|h(x)|$ (I pass over the trivial
case $X=\emptyset$), so that $h\le M\chi E$ and

\Centerline{$(|u|-M\chi E^{\ssbullet})^+
\le(|u|-w)^++(w-M\chi E^{\ssbullet})^+
\le(|u|-w)^++(w-h^{\ssbullet})^+$,}

\Centerline{$\int(|u|-M\chi E^{\ssbullet})^+
\le\|(|u|-w)^+\|+\|w-h^{\ssbullet}\|\le\epsilon$}

\noindent for every $u\in A$.   As $\epsilon$ is arbitrary, $A$ is
uniformly integrable in the sense of 354P.
}%end of proof of 354Q

\leader{354R}{}\cmmnt{ I give abstract versions of the easiest results
from \S246.

\medskip

\noindent}{\bf Theorem} Let $U$ be an $L$-space.

(a) If $A\subseteq U$ is uniformly integrable, then

\quad (i) $A$ is norm-bounded;

\quad (ii) every subset of $A$ is uniformly integrable;

\quad (iii) for any $\alpha\in\Bbb R$, $\alpha A$ is uniformly
integrable;

\quad (iv) there is a uniformly integrable, solid, convex, norm-closed
set $C\supseteq A$;

\quad (v) for any other uniformly integrable set $B\subseteq U$, $A\cup
B$ and $A+B$ are uniformly integrable.

(b) For any set $A\subseteq U$, the following are equiveridical:

\quad (i) $A$ is uniformly integrable;

\quad (ii) $\lim_{n\to\infty}(|u_n|-\sup_{i<n}|u_i|)^+=0$ for every
sequence $\sequencen{u_n}$ in $A$;

\quad (iii) {\it either} $A$ is empty {\it or} for every $\epsilon>0$
there are $u_0,\ldots,u_n\in A$ such that
\ifdim\pagewidth=390pt\penalty-1000\fi
$\|(|u|-\sup_{i\le n}|u_i|)^+\|\le\epsilon$ for every $u\in A$;

\quad (iv) $A$ is norm-bounded and any disjoint sequence in the solid
hull of $A$ is norm-convergent to $0$.

(c) If $V\subseteq U$ is a closed Riesz subspace then a subset of $V$ is
uniformly integrable when regarded as a subset of $V$ iff it is
uniformly integrable when regarded as a subset of $U$.

\proof{{\bf (a)(i)} There must be a $w\in U^+$ such that
$\int(|u|-w)^+\le 1$ for every $u\in A$;  now

\Centerline{$|u|\le|u|-w+|w|\le(|u|-w)^++|w|$,
\quad$\|u\|\le\|(|u|-w)^+\|
+\|w\|\le 1+\|w\|$}

\noindent for every $u\in A$, so $A$ is norm-bounded.

\medskip

\quad{\bf (ii)} This is immediate from the definition.

\medskip

\quad{\bf (iii)} Given $\epsilon>0$, we can find $w\in U^+$ such
that $|\alpha|\|(|u|-w)^+\|\le\epsilon$ for every
$u\in A$;  now $\|(|v|-|\alpha|w)^+\|\le\epsilon$ for every
$v\in\alpha A$.

\medskip

\quad{\bf (iv)} If $A$ is empty, take $C=A$.   Otherwise, try

\Centerline{$C=\{v:v\in U,\,\|(|v|-w)^+\|
  \le\sup_{u\in A}\|(|u|-w)^+\|$ for every $w\in U^+\}$.}

\noindent Evidently $A\subseteq C$, and $C$ satisfies the definition
354M because $A$ does.   The functionals

\Centerline{$v\mapsto\|(|v|-w)^+\|:U\to\Bbb R$}

\noindent are all continuous for $\|\,\|$ (because the operators
$v\mapsto|v|$, $v\mapsto v-w$, $v\mapsto v^+$, $v\mapsto\|v\|$ are
continuous), so $C$ is closed.   If $|v'|\le|v|$ and $v\in C$, then

\Centerline{$\|(|v'|-w)^+\|\le\|(|v|-w)^+\|
\le\sup_{u\in A}\|(|u|-w)^+\|$}

\noindent for every $w$, and $v'\in C$.   If $v=\alpha v_1+\beta v_2$
where  $v_1$, $v_2\in C$, $\alpha\in[0,1]$ and $\beta=1-\alpha$, then
$|v|\le\alpha|v_1|+\beta|v_2|$, so

\Centerline{$|v|-w\le(\alpha|v_1|-\alpha w)+(\beta|v_2|-\beta w)
\le(\alpha|v_1|-\alpha w)^++(\beta|v_2|-\beta w)^+$}

\noindent and

\Centerline{$(|v|-w)^+\le\alpha(|v_1|-w)^++\beta(|v_2|-w)^+$}

\noindent for every $w$;  accordingly

$$\eqalign{\|(|v|-w)^+\|
&\le\alpha\|(|v_1|-w)^+\|+\beta\|(|v_2|-w)^+\|\cr
&\le(\alpha+\beta)\sup_{u\in A}\|(|u|-w)^+\|
=\sup_{u\in A}\|(|u|-w)^+\|\cr}$$

\noindent for every $w$, and $v\in C$.

Thus $C$ has all the required properties.

\medskip

\quad{\bf (v)} I show first that $A\cup B$ is uniformly integrable.
\Prf\ Given $\epsilon>0$, let $w_1$, $w_2\in U^+$ be such that

\Centerline{$\|(|u|-w_1)^+\|\le\epsilon$ for every $u\in A$,
\quad$\|(|u|-w_2)^+\|\le\epsilon$ for every $u\in B$.}

\noindent Set $w=w_1\vee w_2$;  then
$\|(|u|-w)^+\|\le\epsilon$ for every $u\in A\cup B$.   As $\epsilon$ is
arbitrary, $A\cup B$ is uniformly integrable.\ \Qed

Now (iv) tells us that there is a convex uniformly integrable set $C$
including $A\cup B$, and in this case $A+B\subseteq 2C$, so $A+B$ is
also uniformly integrable, using (ii) and (iii).

\medskip

{\bf (b)(i)$\Rightarrow$(ii)\&(iv)} Suppose that $A$ is uniformly
integrable and that $\sequencen{u_n}$ is any sequence in the solid hull
of $A$.   Set $v_n=\sup_{i\le n}|u_i|$ for $n\in\Bbb N$ and

\Centerline{$v'_0=v_0=|u_0|$,
\quad$v'_n=v_n-v_{n-1}=(|u_n|-\sup_{i<n}|u_i|)^+$}

\noindent for $n\ge 1$.   Given $\epsilon>0$, there is a $w\in U^+$ such
that $\|(|u|-w)^+\|\le\epsilon$ for every $u\in A$, and therefore for
every $u$ in the solid hull of $A$.   Of course
$\sup_{n\in\Bbb N}\|v_n\wedge w\|\le\|w\|$ is finite, so there is an
$n\in\Bbb N$ such
that $\|v_i\wedge w\|\le\epsilon+\|v_n\wedge w\|$ for every
$i\in\Bbb N$.   But now, for $m>n$,

$$\eqalign{v'_m\le(|u_m|-v_n)^+
&\le(|u_m|-|u_m|\wedge w)^++((|u_m|\wedge w)-v_n)^+\cr
&\le(|u_m|-w)^++(v_m\wedge w)-(v_n\wedge w),\cr}$$

\noindent so that

$$\eqalign{\|v'_m\|
&\le\|(|u_m|-w)^+\|+\|(v_m\wedge w)-(v_n\wedge w)\|\cr
&=\|(|u_m|-w)^+\|+\|v_m\wedge w\|-\|v_n\wedge w\|
\le 2\epsilon,\cr}$$

\noindent using the $L$-space property of the norm for the equality in
the middle.   As $\epsilon$ is arbitrary, $\lim_{n\to\infty}v'_n=0$.
As $\sequencen{u_n}$ is arbitrary, condition (ii) is satisfied;  but so
is condition (iv), because we know from (a-i) that $A$ is norm-bounded,
and if $\sequencen{u_n}$ is disjoint then $v'_n=|u_n|$ for every $n$, so
that in this case $\lim_{n\to\infty}u_n=0$.

\medskip

\quad{\bf (ii)$\Rightarrow$(iii)$\Rightarrow$(i)} are elementary.

\medskip

\quad{\bf not-(i)$\Rightarrow$not-(iv)} Now suppose that $A$
is not uniformly integrable.   If it is not norm-bounded, we can stop.
Otherwise, there is some $\epsilon>0$ such that
$\sup_{u\in A}\|(|u|-w)^+\|>\epsilon$ for every $w\in U^+$.
Consequently we shall be able to choose inductively a sequence
$\sequencen{u_n}$ in $A$ such that
$\|(|u_n|-2^n\sup_{i<n}|u_i|)^+\|>\epsilon$ for every $n\ge 1$.
Because $A$ is norm-bounded,
$\sum_{i=0}^{\infty}2^{-i}\|u_i\|$ is finite, and we can set

\Centerline{$v_n
=(|u_n|-2^n\sup_{i<n}|u_i|-\sum_{i=n+1}^{\infty}2^{-i}|u_i|)^+$}

\noindent for each $n$.   (The sum $\sum_{i=n+1}^{\infty}2^{-i}|u_i|$ is
defined because $\langle\sum_{i=n+1}^m2^{-i}|u_i|\rangle_{m\ge n+1}$ is
a Cauchy sequence.)   We have $v_m\le|u_m|$,

$$\eqalign{v_m\wedge v_n
&\le(|u_m|-2^{-n}|u_n|)^+\wedge(|u_n|-2^n|u_m|)^+\cr
&\le(2^n|u_m|-|u_n|)^+\wedge(|u_n|-2^n|u_m|)^+=0\cr}$$

\noindent whenever $m<n$, so $\sequencen{v_n}$ is a disjoint sequence in
the solid hull of $A$;  while

\Centerline{$\|v_n\|
\ge\|(|u_n|-2^n\sup_{i<n}|u_i|)^+\|-\sum_{i=n+1}^{\infty}2^{-i}\|u_i\|
\ge\epsilon-2^{-n}\sup_{u\in A}\|u\|\to\epsilon$}

\noindent as $n\to\infty$, so condition (iv) is not satisfied.

\medskip

{\bf (c)} Now this follows at once, because conditions (b-ii) and (b-iv)
are satisfied in $V$ iff they are satsified in $U$.
}%end of proof of 354R

\exercises{\leader{354X}{Basic exercises $\pmb{>}$(a)}
%\spheader 354Xa
Work through the proofs that the following are all Banach lattices.  (i)
$\BbbR^r$ with ($\alpha$) $\|x\|_1=\sum_{i=1}^r|\xi_i|$ ($\beta$)
$\|x\|_2=\sqrt{\sum_{i=1}^r|\xi_i|^2}$ ($\gamma$)
$\|x\|_{\infty}=\max_{i\le r}|\xi_i|$, where $x=(\xi_1,\ldots,\xi_r)$.
(ii) $\ell^p(X)$, for any set $X$ and any $p\in[1,\infty]$ (242Xa,
243Xl, 244Xn).   (iii) $L^p(\mu)$, for any measure space
$(X,\Sigma,\mu)$ and any $p\in[1,\infty]$ (242F, 243E, 244G).   (iv)
$\pmb{c}_0$, the space of sequences convergent to $0$, with the norm
$\|\,\|_{\infty}$ inherited from $\ell^{\infty}$.
%354A

\spheader 354Xb Let $\langle U_i\rangle_{i\in I}$ be any family of
Banach lattices.   Write $U$ for their Riesz space product (352K), and
in $U$ set

\Centerline{$\|u\|_1=\sum_{i\in I}\|u(i)\|$,
\quad$V_1=\{u:\|u\|_1<\infty\}$,}

\Centerline{$\|u\|_{\infty}=\sup_{i\in I}\|u(i)\|$ (counting
$\sup\emptyset$ as $0$),
\quad$V_{\infty}=\{u:\|u\|_{\infty}<\infty\}$.}

\noindent Show that $V_1$, $V_{\infty}$ are solid linear subspaces of
$U$ and are Banach lattices under their norms $\|\,\|_1$,
$\|\,\|_{\infty}$.
%354A

\spheader 354Xc Let $U$ be a Riesz space with a Riesz norm.   Show that
the maps $(u,v)\mapsto u\wedge v$, $(u,v)\mapsto u\vee v:U\times U\to U$
are uniformly continuous.
%354B

\sqheader 354Xd Let $U$ be a Riesz space with a Riesz norm.   (i) Show
that any order-bounded set in $U$ is norm-bounded.   (ii) Show that in
$\BbbR^r$, with any of the standard Riesz norms (354Xa(i)),
norm-bounded sets are order-bounded.   (iii) Show that in
$\ell^1(\Bbb N)$ there is a
sequence converging to $0$ (for the norm) which is not order-bounded.
(iv) Show that in $\pmb{c}_0$ any sequence converging to $0$ is
order-bounded, but there is a norm-bounded set which is not
order-bounded.
%354B

\spheader 354Xe Let $U$ be a Riesz space with a Riesz norm.   Show that
it is a Banach lattice iff non-decreasing Cauchy sequences are
convergent.   \Hint{if $\|u_{n+1}-u_n\|\le 2^{-n}$ for every $n$, show
that $\sequencen{\sup_{i\le n}u_i}$ is Cauchy, and that
$\sequencen{u_n}$ converges to $\inf_{n\in\Bbb N}\sup_{m\ge n}u_m$.}
%354C

\spheader 354Xf Let $U$ be a Riesz space with a Riesz norm.   Show that
$U$ is a Banach lattice iff every non-decreasing Cauchy sequence
$\sequencen{u_n}$ in $U^+$ has a least upper bound $u$ with
$\|u\|=\lim_{n\to\infty}\|u_n\|$.
%354C, 354Xe

\spheader 354Xg Let $U$ be a Banach lattice.   Suppose that
$B\subseteq U$ is solid and $\sup_{n\in\Bbb N}u_n\in B$ whenever $\sequencen{u_n}$
is a non-decreasing sequence in $B$ with a supremum in $U$.   Show that
$B$ is closed.   \Hint{show first that $u\in B$ whenever there is a
sequence $\sequencen{u_n}$ in $B\cap U^+$ such that
$\|u-u_n\|\le 2^{-n}$ for every $n$;  do this by considering
$v_m=\inf_{n\ge m}u_n$.}
%354C

\spheader 354Xh Let $U$ be any Riesz space with a Riesz norm.   Show
that the Banach space completion of $U$ (3A5Jb) has a unique
partial ordering under which it is a Banach lattice.

\sqheader 354Xi Show that $\pmb{c}_0$ is a Banach lattice with an
order-continuous norm which does not have the Levi property.
%354D

\sqheader 354Xj Show that $\ell^{\infty}$, with $\|\,\|_{\infty}$, is a
Banach lattice with a Fatou norm which has the Levi property but is not
order-continuous.
%354D

\spheader 354Xk Let $U$ be a Riesz space with a Fatou norm.   Show that
if $V\subseteq U$ is a regularly embedded Riesz subspace then the induced norm on $V$ is a Fatou norm.
%354D

\spheader 354Xl Let $U$ be a Riesz space and $\|\,\|$ a Riesz norm on
$U$ which is order-continuous in the sense of 354Dc.   Show that its
restriction to $U^+$ is order-continuous in the sense of 313H.
%354D

\spheader 354Xm Let $U$ be a Riesz space with an order-continuous norm.
Show that if $V\subseteq U$ is a regularly embedded Riesz subspace then
the induced norm on $V$ is order-continuous.
%354D

\spheader 354Xn Let $U$ be a Dedekind $\sigma$-complete Riesz space with
a Fatou norm which has the Levi property.   Show that it is a Banach
lattice.   \Hint{354Xf.}
%354D

\spheader 354Xo Let $\langle U_i\rangle_{i\in I}$ be any family of
Banach lattices and let $V_1$, $V_{\infty}$ be the subspaces of
$U=\prod_{i\in I}U_i$ as described in 354Xb.   (i) Show that $V_1$,
$V_{\infty}$ have norms which are Fatou, or have the Levi property,
iff every $U_i$ has.   (ii) Show that the norm of $V_1$ is order-continuous
iff the norm of every $U_i$ is.    (iii) Show that $V_{\infty}$ is an
$M$-space iff every $U_i$ is.   (iv) Show that $V_1$ is an $L$-space iff
every $U_i$ is.
%354M

\spheader 354Xp Let $U$ be a Banach lattice with an order-continuous
norm.   (i) Show that a sublattice of
$U$ is norm-closed iff it is order-closed in the sense of 313Da.   (ii) Show
that a norm-closed Riesz subspace of $U$
is itself a Banach lattice with an order-continuous norm.
%354E

\sqheader 354Xq Let $U$ be an $M$-space and $V$ a norm-closed Riesz
subspace of $U$ containing the standard order unit of $U$.   (i) Show
that $V$, with the induced norm, is an $M$-space.   (ii) Deduce that the
space $\pmb{c}$ of convergent sequences is an $M$-space if given the
norm $\|\,\|_{\infty}$ inherited from $\ell^{\infty}$.
%354G

\spheader 354Xr Show that a Banach lattice $U$ is an $M$-space iff
($\alpha$) its norm is a Fatou norm with the Levi property ($\beta$)
$\|u\vee v\|=\max(\|u\|,\|v\|)$ for all $u$, $v\in U^+$.
%354J

\sqheader 354Xs Describe a topological space $X$ such that the space
$\pmb{c}$ of convergent sequences (354Xq) can be identified with $C(X)$.
%354K, 354Xq

\spheader 354Xt Let $D\subseteq\Bbb R$ be any non-empty set, and $V$ the
space of functions $f:D\to\Bbb R$ of bounded variation (\S224).   For
$f\in V$ set $\|f\|=\sup\{|f(t_0)|+\sum_{i=1}^n|f(t_i)-f(t_{i-1})|:
t_0\le t_1\le\ldots\le t_n$ in $D\}$ (224Yb).   Let $C$ be the set of
bounded non-decreasing functions from $D$ to $\coint{0,\infty}$.
Show that $C$ is the
positive cone of $V$ for a Riesz space ordering under which $V$ is an
$L$-space.
%354M

\leader{354Y}{Further exercises (a)}
%\spheader 354Ya
Let $U$ be a Riesz space with a Riesz norm, and $V$ a norm-dense Riesz
subspace of $U$.   Suppose that the induced norm on $V$ is Fatou, when
regarded as a norm on the Riesz space $V$.   Show (i) that $V$ is
order-dense in $U$ (ii) that the norm of $U$ is Fatou.   \Hint{for (i),
show
that if $u\in U^+$, $v_n\in V^+$ and $\|u-v_n\|\le 2^{-n-2}\|u\|$ for
every $n$, then $\|v_0-\inf_{i\le n}v_i\|\le\bover12\|u\|$ for every
$n$, so that $0$ cannot be $\inf_{n\in\Bbb N}v_n$ in $V$.}
%354D

\spheader 354Yb
Let $U$ be a Riesz space with a Riesz norm.   Show that the following
are equiveridical:  (i) $\lim_{n\to\infty}u_n=0$ whenever $\sequencen{u_n}$
is a disjoint order-bounded sequence in $U^+$ (ii)
$\lim_{n\to\infty}u_{n+1}-u_n=0$ for every order-bounded non-decreasing
sequence $\sequencen{u_n}$ in $U$ (iii) whenever $A\subseteq U^+$ is a
non-empty downwards-directed set in $U^+$ with infimum $0$,
$\inf_{u\in A}\sup_{v\in A,v\le u}\|u-v\|=0$.   \Hint{for
(i)$\Rightarrow$(ii), show by
induction that $\lim_{n\to\infty}u_n=0$ whenever $\sequencen{u_n}$ is an
order-bounded sequence such that, for some fixed $k\ge 1$,
$\inf_{i\in K}u_i=0$ for every $K\subseteq \Bbb N$ of size $k$;
now show that if $\sequencen{u_n}$ is non-decreasing and $0\le u_n\le u$
for every $n$, then $\inf_{i\in K}(u_{i+1}-u_i-\bover1ku)^+=0$ whenever
$K\subseteq\Bbb N$ and $\#(K)=k\ge 1$.   For (iii)$\Rightarrow$(i), set
$A=\{u:\exists\,n,\,u\ge u_i\Forall i\ge n\}$.   See {\smc Fremlin
74a}, 24H.}
%354E

\spheader 354Yc Show that any Riesz space with an order-continuous norm
has the countable sup property (definition:  241Ye).
%354E, 354Yb

\spheader 354Yd Let $U$ be a Banach lattice.   Show that the following
are equiveridical:  (i) the norm on $U$ is order-continuous;  (ii) $U$
satisfies the conditions of 354Yb;   (iii) every order-bounded monotonic
sequence in $U$ is Cauchy.
%354E, 354Yb

\spheader 354Ye Let $U$ be a Riesz space with a Fatou norm.   Show that
the norm on $U$ is order-continuous iff it satisfies the conditions of
354Yb.
%354E, 354Yb

\spheader 354Yf For $f\in C([0,1])$, set $\|f\|_1=\int|f(x)|dx$.   Show
that $\|\,\|_1$ is a Riesz norm on $C([0,1])$ satisfying the conditions
of 354Yb, but is not order-continuous.
%354E, 354Yb

\spheader 354Yg Let $U$ be a Riesz space with a Riesz norm $\|\,\|$.
Show that $(U,\|\,\|)$ satisfies the conditions of 354Yb iff the norm of
its completion is order-continuous.
%354E, 354Yb

\spheader 354Yh Let $U$ be a Riesz space with a Riesz norm, and
$V\subseteq U$ a norm-dense Riesz subspace such that the induced norm on
$V$ is order-continuous.   Show that the norm of $U$ is
order-continuous.   \Hint{use 354Ya.}
%354E, 354Yb

\spheader 354Yi Let $U$ be an Archimedean Riesz space.   For any
$e\in U^+$, let $U_e$ be the solid linear subspace of $U$ generated by
$e$, so that $e$ is an order unit in $U_e$, and let $\|\,\|_e$ be the
corresponding order-unit norm on $U_e$.   We say that $U$ is {\bf
uniformly complete} if $U_e$ is complete under $\|\,\|_e$ for every
$e\in U^+$.   (i) Show that any Banach lattice is uniformly complete.
(ii) Show that any Dedekind $\sigma$-complete Riesz space is uniformly
complete (cf.\ 354Xn).   (iii) Show that if $U$ is a uniformly complete
Riesz space with a Riesz norm which has the Levi property, then $U$ is a
Banach lattice.   (iv) Show that if $U$ is a Banach lattice then a set
$A\subseteq U$ is closed, for the norm topology, iff $A\cap U_e$ is
$\|\,\|_e$-closed for every $e\in U^+$.   (v) Let $V$ be a solid linear
subspace of $U$.   Show that the quotient Riesz space $U/V$ is
Archimedean iff $V\cap U_e$ is $\|\,\|_e$-closed for every $e\in U^+$.
(vi) Show that if $U$ is uniformly complete and $V\subseteq U$ is a
solid linear subspace such that $U/V$ is Archimedean, then $U/V$ is
uniformly complete.   (vii) Show that $U$ is Dedekind $\sigma$-complete
iff it is uniformly complete and has the principal projection property
(353Xb).   \Hint{for (vii), use 353Yc.}
%354H, 353Yc

\spheader 354Yj Let $U$ be an Archimedean Riesz space with an order unit,
endowed with its order-unit norm.   Let $Z$ be the unit ball
of $U^*$.   Show that for a linear functional $f:U\to\Bbb R$ the
following are equiveridical:  (i) $f$ is an {\bf extreme point} of $Z$,
that is, $f\in Z$ and $Z\setminus\{f\}$ is convex (ii) $|f(e)|=1$ and
one of $f$, $-f$ is a Riesz homomorphism.
%353M

\spheader 354Yk Let $U$ be a Banach lattice such that
$\|u+v\|=\|u\|+\|v\|$ whenever $u\wedge v=0$.   Show that $U$ is an
$L$-space.   \Hint{by 354Yd, the norm is order-continuous, so $U$ is
Dedekind complete.   If $u$, $v\ge 0$, set $e=u+v$, and represent $U_e$
as $C(X)$ where $X$ is extremally disconnected (353Yb);  now approximate
$u$ and $v$ by functions taking only finitely many values to show that
$\|u+v\|=\|u\|+\|v\|$.}
%354M

\spheader 354Yl Let $U$ be a uniformly complete Archimedean Riesz space
(354Yi).   Set $U_{\Bbb C}=U\times U$ with the complex linear structure
defined
by identifying $(u,v)\in U\times U$ as $u+iv\in U_{\Bbb C}$, so that
$u=\Real(u+iv)$, $v=\Imag(u+iv)$ and
$(\alpha+i\beta)(u+iv)=(\alpha u-\beta v)+i(\alpha v+\beta u)$.   (i)
Show that for $w\in U_{\Bbb C}$ we can define $|w|\in U$ by setting
$|w|=\sup_{|\zeta|=1}\Real(\zeta w)$.   (ii) Show that if $U$ is a
uniformly complete Riesz subspace of $\BbbR^X$ for some set $X$, then we
can identify $U_{\Bbb C}$ with the linear subspace of $\Bbb C^X$
generated by $U$.   (iii) Show that
$|w+w'|\le|w|+|w'|$, $|\gamma w|=|\gamma||w|$ for all $w\in U_{\Bbb C}$,
$\gamma\in\Bbb C$.   (iv) Show that if $w\in U_{\Bbb C}$ and
$|w|\le u_1+u_2$, where $u_1$, $u_2\in U^+$, then $w$ is expressible as
$w_1+w_2$ where $|w_j|\le u_j$ for both $j$.   \Hint{set $e=u_1+u_2$ and
represent $U_e$ as $C(X)$.}   (v) Show that if $U_0$ is a solid linear
subspace of $U$, then, for $w\in U_{\Bbb C}$, $|w|\in U_0$ iff $\Real
w$, $\Imag w$ both belong to $U_0$.   (vi) Show that if $U$ has a Riesz
norm then we have a norm on $U_{\Bbb C}$ defined by setting
$\|w\|=\||w|\|$, and that if $U$ is a Banach lattice then $U_{\Bbb C}$
is a (complex) Banach space.   (vii) Show
that if $U=L^p(\mu)$, where $(X,\Sigma,\mu)$ is a measure space and
$p\in[1,\infty]$, then $U_{\Bbb C}$ can be identified with
$L^p_{\Bbb C}(\mu)$ as
defined in 242P, 243K and 244P.   (We may call $U_{\Bbb C}$ the {\bf
complexification} of the Riesz space $U$.)

\spheader 354Ym Let $(X,\Sigma,\mu)$ be a measure space and $V$ a Banach
lattice.   Write $\eusm L^1_V$ for the space of Bochner integrable
functions from conegligible subsets of $X$ to $V$, and $L^1_V$ for the
corresponding set of equivalence classes (253Yf).   (i) Show that
$L^1_V$ is a Banach lattice under the ordering defined by saying that
$f^{\ssbullet}\le g^{\ssbullet}$ iff $f(x)\le g(x)$ in $V$ for
$\mu$-almost every $x\in X$.  (ii) Show that when $V=L^1(\nu)$, for some
other measure space $(Y,\Tau,\nu)$, then this ordering of $L^1_V$ agrees
with the ordering of $L^1(\lambda)$ where $\lambda$ is the (c.l.d.)
product measure on $X\times Y$ and we identify $L^1_V$ with
$L^1(\lambda)$, as in 253Yi.   (iii) Show that if $V$ has an
order-continuous norm, so has $L^1_V$.   \Hint{354Yd.}   (iv) Show
that if $\mu$ is Lebesgue measure on $[0,1]$ and $V=\ell^{\infty}$, then
$L^1_V$ is not Dedekind $\sigma$-complete.
}%end of exercises

\endnotes{
\Notesheader{354} Apart from some of the exercises, the material of this
section is pretty strictly confined to ideas which will be useful later
in this volume.   The basic Banach lattices of measure theory are the
$L^p$ spaces of Chapter 24;  these all have Fatou norms with the Levi
property (244Yf-244Yg), and for $p<\infty$ their norms are
order-continuous (244Ye).   In Chapter 36 I will return to these spaces
in a more abstract context.   Here I am mostly concerned to establish a
vocabulary in which their various properties, and the relationships
between these properties, can be expressed.

In normed Riesz spaces we have a very rich mixture of structures, and
must take particular care over the concepts of `boundedness',
`convergence' and `density', which have more than one possible
interpretation.
In particular, we must scrupulously distinguish between
`order-bounded' and `norm-bounded' sets.   I have not yet formally
introduced any of the various concepts of order-convergence (see \S367),
but I think that even so it is best to get into the habit of reminding
oneself, when a convergent sequence appears, that it is convergent for
the norm topology, rather than in any sense related directly to the
order structure.

I should perhaps warn you that for the study of
$M$-spaces 354L is not as helpful as it may look.   The trouble is that
apart from a few special cases (as in 354Xs) the topological space used
in the representation is actually more complicated and mysterious than
the $M$-space it is representing.

After the introduction of $M$-spaces, this section becomes a natural
place for `uniformly complete' spaces (354Yi).   For the moment I
leave these in the exercises.   But I mention them now because they
offer a straightforward route towards a theory of `complex Riesz
spaces' (354Yl).   In large parts of functional analysis it is natural,
and in some parts it is necessary, to work with normed spaces over
$\Bbb C$ rather than over $\Bbb R$, and for $L^2$ spaces in particular
it is useful to have a proper grasp of the complex case.   And while the
insights offered by the theory of Riesz spaces are not especially
important in such areas, I think we should always seek connexions
between different topics.   So it is worth remembering that uniformly
complete Riesz spaces have complexifications.

I shall have a great deal more to say about $L$-spaces when I come to
spaces of additive functionals (\S362) and to $L^1$ spaces again (\S365)
and to linear operators on them (\S371);  and before that, there will be
something in the next section on their duals, and on $L$-spaces which
are themselves dual spaces.   For the moment I just give some easy
results, direct translations of the corresponding facts in \S246, which
have natural expressions in the language of this section, holding deeper
ideas over.   In particular, the characterization of uniformly
integrable sets as relatively weakly compact sets (247C) is valid in
general $L$-spaces (356Q).

For an extensive treatment of Banach lattices, going very much deeper
than I have space for in this volume, see {\smc Lindenstrauss \&
Tzafriri 79}.   For a careful exposition of a great deal of useful
information, see {\smc Schaefer 74}.
}%end of comment

\discrpage

