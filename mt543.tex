\frfilename{mt543.tex}
\versiondate{11.11.13}
\copyrightdate{2004}

\def\chaptername{Real-valued-measurable cardinals}
\def\sectionname{The Gitik-Shelah theorem}

\newsection{543}

I come now to the leading case at the centre of the work of the last two
sections.   If our $\omega_1$-saturated $\sigma$-ideal of sets is the
null ideal of a measure with domain $\Cal PX$, it has some even more
striking properties than those already discussed.   I will go farther into
these later in the chapter.   But I will begin with what is known about one
of the first questions I expect a reader of this book to ask:   if
$(X,\Cal PX,\mu)$ is a probability space, what can, or must, its measure
algebra be?   There can, of course, be a purely atomic part;  the
interesting question relates to the atomless part, if any, always
remembering that we need a special act of faith to believe that there can
be an atomless case.   Here we find that the Maharam type of an atomless
probability defined on a power set must be greater than its additivity
(543F), which must itself be `large' (541L).

\leader{543A}{Definitions (a)} A {\bf\rvm\ cardinal} is an uncountable
cardinal $\kappa$ such that there is a $\kappa$-additive probability
measure $\nu$ on $\kappa$, defined on every subset of $\kappa$, for which
all singletons are negligible.   In this context I will call $\nu$ a
{\bf witnessing probability}.

\spheader 543Ab If $\kappa$ is a regular uncountable cardinal, a
probability measure $\nu$ on $\kappa$ with domain $\Cal P\kappa$ is
{\bf normal} if its null ideal $\Cal N(\nu)$ is normal.   In this
case,\cmmnt{ $\nu$ must be $\kappa$-additive (541H, 521Ad)
and zero on singletons,
so $\kappa$ is \rvm, and} I will say that $\nu$ is a {\bf normal
witnessing probability}.

\spheader 543Ac An {\bf\am\ cardinal} is a \rvm\ cardinal with an
atomless witnessing probability.

\leader{543B}{}\cmmnt{ Collecting ideas which have already
appeared, some of them more than once, we have the following.

\medskip

\noindent}{\bf Proposition} (a) Let $(X,\Cal PX,\mu)$ be a totally finite
measure space in which singletons are negligible and $\mu X>0$.
Then $\kappa=\add\mu$ is \rvm,
and there are a non-negligible $Y\subseteq X$ and a function
$g:Y\to\kappa$ such that the normalized image measure
$B\mapsto\Bover1{\mu Y}\mu g^{-1}[B]$ is a normal witnessing probability
on $\kappa$.

(b) Every \rvm\ cardinal is \qm\cmmnt{ (definition: 542A)} and has a
normal witnessing
probability;  in particular, every \rvm\ cardinal is uncountable and
regular.

(c) If $\kappa\le\frak c$ is a \rvm\ cardinal, then $\kappa$ is \am, and
every witnessing probability on $\kappa$ is atomless.

(d) If $\kappa>\frak c$ is a \rvm\ cardinal, then $\kappa$ is \2vm, and
every witnessing probability on $\kappa$ is purely atomic.

(e) A cardinal $\lambda$ is measure-free\cmmnt{ (definition: 438A)}
iff there is no \rvm\ cardinal $\kappa\le\lambda$;  $\frak c$ is
measure-free iff there is no \am\ cardinal.

(f) Again suppose that $(X,\Cal PX,\mu)$ is a totally finite
measure space.

\quad(i) If $\mu$ is purely atomic, $\add\mu$ is either $\infty$ or
a \2vm\ cardinal.

\quad(ii) If $\mu$ is not purely atomic, $\add\mu$ is \am.

\proof{{\bf (a)} By 521Ad again,
$\kappa$ is the additivity of the null ideal
$\Cal N(\mu)$ of $\mu$;  because $\mu$ is
$\sigma$-finite, $\Cal N(\mu)$ is $\omega_1$-saturated;  and of course
$\kappa\ge\omega_1$.   By 541J, there are $Y\subseteq X$ and
$g:Y\to\kappa$ such that
$\Cal I=\{B:B\subseteq\kappa$, $\mu g^{-1}[B]=0\}$
is a normal ideal on $\kappa$.   In particular, $\kappa\notin\Cal I$ and
$Y$ is non-negligible.   Set $\nu B=\Bover1{\mu Y}\mu g^{-1}[B]$ for
$B\subseteq\kappa$;  then $\nu$ is a probability measure with domain
$\Cal P\kappa$.   Its null ideal $\Cal N(\nu)=\Cal I$ is normal, so it is
a normal measure and witnesses that $\kappa$ is \rvm.

\medskip

{\bf (b)} If $\kappa$ is \rvm, then (a) tells us that any witnessing
probability on $\kappa$ can be used to define a normal witnessing
probability $\nu$ say.   Since $\kappa$ is the additivity of a
$\sigma$-ideal, it must be uncountable and regular (513C(a-i));  also
$\Cal N(\nu)$ is an $\omega_1$-saturated normal ideal, so $\kappa$ is
\qm.

\medskip

{\bf (c)-(d)} Apply 541P.   If $\kappa$ is \rvm, it is a regular
uncountable cardinal, and if $\nu$ is a witnessing probability on $\kappa$
then $\Cal N(\nu)$ is a proper $\omega_1$-saturated $\kappa$-additive
ideal of subsets of $\kappa$.    Taking $\frak A$ to be the measure
algebra of $\nu$, then 541P tells us that either $\frak A$ is atomless
and $\kappa\le\frak c$, or $\frak A$ is purely atomic and $\kappa$ is
\2vm, in which case $\kappa$ is surely greater than $\frak c$ (541N).
Turning this round, if $\kappa\le\frak c$ then $\frak A$ and $\nu$ must
be atomless and $\kappa$ is \am, while if $\kappa>\frak c$ then
$\frak A$ and $\nu$ are purely atomic and $\kappa$ is \2vm.

\medskip

{\bf (e)} If $\kappa\le\lambda$ is \rvm, then any witnessing probability
on $\kappa$ extends to a probability measure with domain $\Cal P\lambda$
which is zero on singletons, so $\lambda$ is not measure-free.   If
$\lambda$ is not measure-free, let $\mu$ be a probability measure with
domain $\Cal P\lambda$ which is zero on singletons;  then (a) tells us
that $\add\mu$ is \rvm, and $\add\mu\le\lambda$ because
$\lambda=\bigcup\Cal N(\mu)$.

If there is an \am\ cardinal $\kappa$, then $\kappa$ is \rvm\ and
there is an atomless witnessing probability on $\kappa$, and
$\kappa\le\frak c$, by (d).   So in this case $\frak c$ is not
measure-free.   On the other hand, if $\frak c$ is not measure-free, there
is a \rvm\ cardinal $\kappa\le\frak c$, which is \am, by (c).

\medskip

{\bf (f)(i)} If $\mu$ is purely atomic, and $\add\mu$ is not $\infty$,
set $\kappa=\add\mu$ and let
$\ofamily{\xi}{\kappa}{A_{\xi}}$ be a family of negligible sets in $X$ with
non-negligible union $A$.   Let $E\subseteq A$ be an atom for $\mu$.
Repeating the construction of (a),
but starting from the subspace $(E,\Cal PE,\mu_E)$,
we see that the normalized image
measure constructed on $\kappa=\add\mu_E$ can take only the two values $0$
and $1$, so that its null
ideal is $2$-saturated and witnesses that $\kappa$ is \2vm.

\medskip

\quad{\bf (ii)} If $\mu$ is not purely atomic, $E$ be the atomless part of
$X$, so that $\mu_E$ is atomless and $\mu_{X\setminus E}$ is purely atomic.
Singletons in $E$ must be
negligible, so (a) tells us that $\add\mu_E$ is a \rvm\ cardinal;
also there is an \imp\
function from $E$ to $[0,\mu E]$ (343Cc), so $E$ can be covered by
$\frak c$ negligible sets and
$\add\mu_E\le\frak c$ is \am, by (c) here.   Now (i) tells us that
$\frak c\le\add\mu_{X\setminus E}$, so
$\add\mu=\min(\add\mu_E,\add\mu_{X\setminus E})=\add\mu_E$ is \am.
}%end of proof of 543B

\cmmnt{\medskip

\noindent{\bf Remark} 543Bc-543Bd are {\bf Ulam's dichotomy}.
}

\vleader{72pt}{543C}{Theorem}\cmmnt{ (see {\smc Kunen n70})}
Suppose that $(Y,\Cal PY,\nu)$ is a
$\sigma$-finite measure space and that $(X,\frak T,\Sigma,\mu)$ is a
$\sigma$-finite quasi-Radon measure space with $w(X)<\add\nu$.   Let
$f:X\times Y\to[0,\infty]$ be any function.   Then

\Centerline{$\overlineint\bigl(\int f(x,y)\nu(dy)\bigr)\mu(dx)
\le\int\bigl(\overline{\intop}f(x,y)\mu(dx)\bigr)\nu(dy)$.}

\proof{ Because $\mu$ is $\sigma$-finite and effectively locally finite,
there is a sequence of open sets of finite measure with conegligible
union in $X$.   Since none of the integrals are changed by deleting a
negligible subset of $X$, and the weight of any subset of $X$ is at most
the weight of $X$, we may suppose that this conegligible union is
$X$ itself, so that $\mu$ is outer regular with respect to the open sets
(412Wb).   Set $\lambda=w(X)<\add\nu$;  let
$\langle G_{\xi}\rangle_{\xi<\lambda}$ enumerate a base for the topology
of $X$.   Fix $\epsilon>0$.   Because $\nu$ is $\sigma$-finite, we have
a function $y\mapsto\epsilon_y:Y\to\ooint{0,\infty}$ such that
$\int\epsilon_y\nu(dy)\le\epsilon$.   For each $y\in Y$, let
$h_y:X\to[0,\infty]$ be a lower semi-continuous function such that
$f(x,y)\le h_y(x)$ for every $x\in X$ and

\Centerline{$\int h_y(x)\mu(dx)
\le\epsilon_y+\overline{\intop}f(x,y)\mu(dx)$}

\noindent (412Wa).   For $I\subseteq\lambda$, $x\in X$ and $y\in Y$, set

\Centerline{$f_I(x,y)
=\sup(\{0\}\cup\{s:\enskip\Exists\xi\in I$, $x\in G_{\xi}$,
  $h_y(x')\ge s\Forall x'\in G_{\xi}\})$.}

\noindent Then $f_I$ is expressible as
$\sup_{\xi\in I,s\in\Bbb Q^+}s\chi(G_{\xi}\times B_{\xi s})$,
writing $\Bbb Q^+$ for the set of non-negative rational numbers and
$B_{\xi s}$ for $\{y:h_y\ge s\chi G_{\xi}\}$.
So $f_I$ is $(\Sigma\tensorhat\Cal PY)$-measurable for all countable
$I$, and for such $I$ we shall have

\Centerline{$\iint f_I(x,y)\mu(dx)\nu(dy)
=\iint f_I(x,y)\nu(dy)\mu(dx)$,}

\noindent by Fubini's theorem (252C).   Next, for any $I\subseteq\lambda$,
$x\mapsto f_I(x,y)$ is lower semi-continuous for each $y$, and

\Centerline{$\sup_{I\in[\lambda]^{<\omega}}f_I(x,y)=h_y(x)$}

\noindent for all $x\in X$ and $y\in Y$, because each $h_y$
is lower semi-continuous.   So

\Centerline{$\sup_{I\in[\lambda]^{<\omega}}\int f_I(x,y)\mu(dx)
=\int h_y(x)\mu(dx)$}

\noindent for each $y\in Y$ (414Ba).   Because
$\lambda<\add\nu$, it follows that

\Centerline{$\sup_{I\in[\lambda]^{<\omega}}
  \iint f_I(x,y)\mu(dx)\nu(dy)
=\iint h_y(x)\mu(dx)\nu(dy)$}

\noindent (521B(d-i)).  On the other hand, if we write

\Centerline{$g_I(x)=\int f_I(x,y)\nu(dy)$}

\noindent for $x\in X$ and finite $I\subseteq\lambda$, then
$g_I$ also is lower semi-continuous.   \Prf\ If $x\in X$, set
$J=\{\xi:\xi\in I$, $x\in G_{\xi}\}$ and
$H=X\cap\bigcap_{\xi\in J}G_{\xi}$;  then $f_I(x,y)\le f_I(x',y)$ whenever
$x'\in H$ and $y\in Y$, so $g_I(x)\le g_I(x')$ for $x'\in H$, while
$x\in\interior H$.\ \QeD\   So $g=\sup_{I\in[\lambda]^{<\omega}}g_I$
is lower semi-continuous, and $\int g(x)\mu(dx)
=\sup_{I\in[\lambda]^{<\omega}}\int g_I(x)\mu(dx)$.   Also

\Centerline{$g(x)=\sup_{I\in[\lambda]^{<\omega}}\int f_I(x,y)\nu(dy)
=\int h_y(x)\nu(dy)\ge\int f(x,y)\nu(dy)$}

\noindent for every $x\in X$.   So we have

$$\eqalign{\overline{\int}\mskip-10mu\int f(x,y)\nu(dy)\mu(dx)
&\le\int g(x)\mu(dx)
=\sup_{I\in[\lambda]^{<\omega}}\int g_I(x)\mu(dx)\cr
&=\sup_{I\in[\lambda]^{<\omega}}\iint f_I(x,y)\nu(dy)\mu(dx)\cr
&=\sup_{I\in[\lambda]^{<\omega}}\iint f_I(x,y)\mu(dx)\nu(dy)\cr
&=\iint h_y(x)\mu(dx)\nu(dy)
\le\epsilon+\int\overline{\int}f(x,y)\mu(dx)\nu(dy).\cr}$$

\noindent As $\epsilon$ is arbitrary, we have the result.
}%end of proof of 543C

\cmmnt{\medskip

\noindent{\bf Remark} Compare 537O.}

\vleader{72pt}{543D}{Corollary} Let $\kappa$ be a \rvm\ cardinal, with
witnessing probability $\nu$, and $(X,\frak{T},\Sigma,\mu)$
a totally finite quasi-Radon measure space with $w(X)<\kappa$.

(a) If $C\subseteq X\times\kappa$ then

\Centerline{$\overlineint\nu C[\{x\}]\mu(dx)
\le\int\mu^*C^{-1}[\{\xi\}]\nu(d\xi)$.}

(b) If $A\subseteq X$ and $\#(A)\le\kappa$, then there is
a $B\subseteq A$ such that $\#(B)<\kappa$ and $\mu^*B=\mu^*A$.

(c) If $\langle C_{\xi}\rangle_{\xi<\kappa}$ is a
family in $\Cal PX\setminus\Cal N(\mu)$ such that
$\#(\bigcup_{\xi<\kappa}C_{\xi})<\kappa$, then there are distinct $\xi$,
$\eta<\kappa$ such that $\mu^*(C_{\xi}\cap C_{\eta})>0$.

(d) If we have a family $\langle h_{\xi}\rangle_{\xi<\kappa}$ of
functions such that $\dom h_{\xi}$ is a non-negligible subset of
$X$ for each $\xi$ and $\#(\bigcup_{\xi<\kappa}h_{\xi})<\kappa$\cmmnt{
(identifying each $h_{\xi}$ with its graph)},
then there are distinct $\xi$, $\eta<\kappa$ such that

\Centerline{$\mu^*\{x:x\in\text{dom}(h_{\xi})\cap\text{dom}(h_{\eta}),
  \,h_{\xi}(x)=h_{\eta}(x)\}>0$.}

\proof{{\bf (a)} Apply 543C to $\chi C:X\times\kappa\to\Bbb R$.

\medskip

{\bf (b)}\Quer\ Suppose, if possible, otherwise.
Then surely $\#(A)=\kappa$;  let $f:\kappa\to A$ be a bijection.   Set

\Centerline{$C=\{(f(\eta),\xi):\eta\le\xi<\kappa\}
\subseteq X\times\kappa$.}

\noindent If $x\in A$,

\Centerline{$\nu C[\{x\}]=\nu\{\xi:f^{-1}(x)\le\xi<\kappa\}=1$,}

\noindent so $\overline{\int}\nu C[\{x\}]\mu(dx)=\mu^*A$.
If $\xi<\kappa$,

\Centerline{$\mu^*C^{-1}[\{\xi\}]=\mu^*\{f(\eta):\eta\le\xi\}<\mu^*A$,}

\noindent so $\int\mu^*C^{-1}[\{\xi\}]\nu(d\xi)<\mu^*A$.
But this contradicts (a).\ \Bang

\medskip

{\bf (c)} Let $\tilde\nu$ be the probability on $\kappa\times\kappa$
defined by writing

\Centerline{$\tilde\nu A=\int\nu A[\{\xi\}]\,\nu(d\xi)$ for every
$A\subseteq\kappa\times\kappa$.}

\noindent Then $\tilde\nu$ is $\kappa$-additive, by 521B(d-ii).   Set

$$\eqalign{C
&=\{(x,(\xi,\eta)):\xi,\,\eta\text{ are distinct members of }\kappa,\,
  x\in C_{\xi}\cap C_{\eta}\}\cr
&\subseteq X\times(\kappa\times\kappa).\cr}$$

\noindent Set

\Centerline{$E=\{x:x\in X$, $\nu\{\xi:x\in C_{\xi}\}=0\}$.}

\noindent Because $\#(\bigcup_{\xi<\kappa}C_{\xi})<\kappa$,

\Centerline{$\{\xi:E\cap C_{\xi}\ne\emptyset\}
=\bigcup\{\{\xi:x\in C_{\xi}\}:x\in E\cap\bigcup_{\eta<\kappa}C_{\eta}\}$}

\noindent is $\nu$-negligible, and
there is a $\xi<\kappa$ with $C_{\xi}\cap E=\emptyset$;
thus $\mu^*(X\setminus E)>0$.   Now if $x\in X\setminus E$ then

\Centerline{$\tilde\nu\{(\xi,\eta):(x,(\xi,\eta))\in C\}
=(\nu\{\xi:x\in C_{\xi}\})^2>0$.}

\noindent So we have

$$\eqalign{0
&<\overline{\int}\tilde\nu\{(\xi,\eta):(x,(\xi,\eta))\in C\}\mu(dx)\cr
&\le\int\mu^*\{x:(x,(\xi,\eta))\in C\}\tilde\nu(d(\xi,\eta))\cr}$$

\noindent by 543C, and there must be distinct $\xi$, $\eta<\kappa$
such that $\mu^*\{x:(x,(\xi,\eta))\in C\}>0$, as required.

\medskip

{\bf (d)} Set $Y=\bigcup_{\xi<\kappa}h_{\xi}[X]$.
Give $X\times Y$ the measure $\tilde\mu$ and topology
$\frak T'$ defined as follows.   The domain of $\tilde\mu$ is to
be the family $\tilde\Sigma$ of subsets $H$ of
$X\times Y$ for which there are $E$, $E'\in\Sigma$ with
$\mu(E'\setminus E)=0$ and
$E\times Y\subseteq H\subseteq E'\times Y$;
and for such $H$, $\tilde\mu H$
is to be $\mu E=\mu E'$.   The topology $\frak{T}'$ is
to be just the family $\{G\times Y:G\in\frak{T}\}$.
It is easy to check that
$(X\times Y,\frak{T}',\tilde\Sigma,\tilde\mu)$
is a totally finite quasi-Radon measure space of weight less than
$\kappa$, and that $\tilde\mu^*h_{\xi}=\mu^*(\dom h_{\xi})>0$
for each $\xi<\kappa$.   So (c) gives the result.
}%end of proof of 543D

\leader{543E}{The Gitik-Shelah theorem}\cmmnt{ ({\smc Gitik \& Shelah
89}, {\smc Gitik \& Shelah 93})} Let $\kappa$ be an
\am\ cardinal, with witnessing probability $\nu$.   Then the Maharam
type of $\nu$ is at least $\min(\kappa^{(+\omega)},2^{\kappa})$.

\proof{{\bf (a)} To begin with (down to the end of (g) below) let us
suppose that $\nu$ is \Mth, with Maharam type $\lambda$;
of course $\lambda$ is infinite, because $\nu$ is atomless.   Let
$(\frak A,\bar\nu)$ be the measure algebra of $\nu$,
$\nu_{\lambda}$ the usual measure of $\{0,1\}^{\lambda}$ and
$\frak B_{\lambda}$ the measure algebra of
$\nu_{\lambda}$;  then there is a
measure-preserving isomorphism $\phi:\frak B_{\lambda}\to\frak A$.
Because $\nu_{\lambda}$ is a compact measure (342Jd), there is a
function $f:\kappa\to\{0,1\}^{\lambda}$ such that
$\phi(E^{\ssbullet})=f^{-1}[E]^{\ssbullet}$ whenever $\nu_{\lambda}$
measures $E$ (343B).

\medskip

{\bf (b)}\Quer\ Suppose, if possible, that
$\lambda<\min(\kappa^{(+\omega)},2^{\kappa})$.

Set $\zeta=\max(\lambda^+,\kappa^+)$.   Then we have an infinite cardinal
$\delta<\kappa$, a stationary set $S\subseteq\zeta$, and a family
$\family{\alpha}{S}{g_{\alpha}}$ of functions from $\kappa$ to
$2^{\delta}$ such that $g_{\alpha}[\kappa]\subseteq\alpha$ for every
$\alpha\in S$ and $\#(g_{\alpha}\cap g_{\beta})<\kappa$ for distinct
$\alpha$, $\beta\in S$.   Moreover,

----- if $\lambda<\Tr(\kappa)$ (definition:  5A1Lb), then
$g_{\alpha}[\kappa]\subseteq\kappa$ for every $\alpha\in S$;

----- if $\lambda\ge\Tr(\kappa)$, then
$g_{\alpha}\restr\gamma=g_{\beta}\restr\gamma$ whenever $\gamma<\kappa$
is a limit ordinal and $\alpha$, $\beta\in S$ are such that
$g_{\alpha}(\gamma)=g_{\beta}(\gamma)$.

\medskip

\Prf\ {\bf case 1} If $\lambda<\Tr(\kappa)$, then $\zeta\le\Tr(\kappa)$
(5A1Ma);  as $\zeta$ is a successor cardinal, there is a family
$\ofamily{\alpha}{\zeta}{g_{\alpha}}$ of functions from $\kappa$ to
$\kappa$ such that $\#(g_{\alpha}\cap g_{\beta})<\kappa$ for all
distinct $\alpha$, $\beta<\zeta$.   Set $S=\zeta\setminus\kappa$, so
that $S$ is a stationary set in $\zeta$ and
$g_{\alpha}[\kappa]\subseteq\alpha$ for every $\alpha\in S$.
We know that $\kappa\le\frak c$;  set $\delta=\omega$, so that
$\delta<\kappa\le 2^{\delta}$ and $g_{\alpha}$ is a function from
$\kappa$ to $2^{\delta}$ for every $\alpha$.

\medskip

\quad{\bf case 2} Suppose that $\lambda\ge\Tr(\kappa)$.   Then

\Centerline{$\kappa<\Tr(\kappa)<\lambda^+=\zeta
\le\min(2^{\kappa},\kappa^{(+\omega)})$,}

\noindent so $\sup_{\delta<\kappa}2^{\delta}\ge\zeta$, by 5A1Mb;
and as this supremum is attained (542E),
there is a cardinal $\delta<\kappa$ such that
$2^{\delta}\ge\zeta$.   Because
$\kappa<\lambda<\kappa^{(+\omega)}$,
$\lambda$ is regular, and of course $\lambda>\omega_1$.
So 5A1O gives us the functions we need.\ \Qed

\medskip

{\bf (c)} Fix an injective function $h:2^{\delta}\to\{0,1\}^{\delta}$.
For $\alpha\in S$, $\iota<\delta$ set

\Centerline{$U_{\alpha\iota}
=\{\xi:\xi<\kappa$, $(hg_{\alpha}(\xi))(\iota)=1\}$,}

\noindent and choose a Baire set
$H_{\alpha\iota}\subseteq\{0,1\}^{\lambda}$ such that
$\phi^{-1}(U^{\ssbullet}_{\alpha\iota})=H^{\ssbullet}_{\alpha\iota}$ in
$\frak B_{\lambda}$.   Define
$\tilde g_{\alpha}:\{0,1\}^{\lambda}\to\{0,1\}^{\delta}$ by setting

$$\eqalign{(\tilde g_{\alpha}(x))(\iota)
&=1\text{ if }x\in H_{\alpha\iota},\cr
&=0\text{ otherwise}.\cr}$$

\noindent Then

$$\eqalign{\{\xi:
  \xi<\kappa,\,\tilde g_{\alpha}f(\xi)\ne hg_{\alpha}(\xi)\}
&=\bigcup_{\iota<\delta}\{\xi:
  (\tilde g_{\alpha}f(\xi))(\iota)\ne(hg_{\alpha}(\xi))(\iota)\}\cr
&=\bigcup_{\iota<\delta}
  U_{\alpha\iota}\symmdiff f^{-1}[H_{\alpha\iota}]
\in\Cal N(\nu)\cr}$$

\noindent because $\delta<\kappa=\add\Cal N(\nu)$.
Set $V_{\alpha}=\{\xi:\tilde g_{\alpha}f(\xi)=hg_{\alpha}(\xi)\}$, so
that $\nu V_{\alpha}=1$, for each $\alpha\in S$.

\medskip

{\bf (d)} Because every $H_{\alpha\iota}$ is a Baire
set, there is for each $\alpha\in S$ a set $I_{\alpha}\subseteq\lambda$
such that $\#(I_{\alpha})\le\delta$ and $H_{\alpha\iota}$ is determined
by coordinates in $I_{\alpha}$ for every $\iota<\delta$, that is,
$\tilde g_{\alpha}(x)=\tilde g_{\alpha}(y)$ whenever $x$,
$y\in\{0,1\}^{\lambda}$ and $x\restr I_{\alpha}=y\restr I_{\alpha}$.
Because $\lambda<\cf\zeta$, there is an $M\subseteq\lambda$ such that

\Centerline{$S_1=\{\alpha:\alpha\in S$, $I_{\alpha}\subseteq M\}$}

\noindent is stationary in $\zeta$ and $\cf(\#(M))\le\delta$ (5A1J);
because $\lambda<\kappa^{(+\omega)}$ and $\cf(\kappa)=\kappa>\delta$,
$\#(M)<\kappa$.   Set $\pi_M(z)=z\restr M$ for $z\in\{0,1\}^{\lambda}$,
and $f_M=\pi_Mf$, so that $f_M:\kappa\to\{0,1\}^M$ is \imp\ for
$\nu$ and the usual measure $\nu_M$ of $\{0,1\}^M$.
For $w\in\{0,1\}^M$ define $\psi(w)\in\{0,1\}^{\lambda}$ by setting

$$\eqalign{\psi(w)(\xi)
&=w(\xi)\text{ if }\xi\in M,\cr
&=0\text{ otherwise}.\cr}$$

\noindent If we set

\Centerline{$g^*_{\alpha}
=\tilde g_{\alpha}\psi:\{0,1\}^M\to\{0,1\}^{\delta}$,}

\noindent then $g^*_{\alpha}$ is Baire measurable in each coordinate, while
$g^*_{\alpha}\pi_M=\tilde g_{\alpha}$ for $\alpha\in S_1$.

\medskip

{\bf (e)} For each $\alpha\in S_1$, there is a $\theta_{\alpha}<\kappa$
such that $\mu^*_M(f_M[V_{\alpha}\cap\theta_{\alpha}])=1$.
\Prf\ Apply 543Db to $f_M[V_{\alpha}]\subseteq\{0,1\}^M$.   There
must be a set $B\subseteq f_M[V_{\alpha}]$ such that $\#(B)<\kappa$ and
$\mu_M^*B=\mu_M^*(f_M[V_{\alpha}])$;  because $\kappa$ is regular,
there is a $\theta_{\alpha}<\kappa$ such that
$B\subseteq f_M[V_{\alpha}\cap\theta_{\alpha}]$.   On the other hand,
because $f_M$ is \imp,
$\mu_M^*(f_M[V_{\alpha}])\ge\nu V_{\alpha}=1$.\ \Qed

Evidently we may take it that every $\theta_{\alpha}$ is a non-zero
limit ordinal.

\medskip

{\bf (f)} Because $\zeta=\cf\zeta>\kappa$, there is a $\theta<\kappa$
such that

\Centerline{$S_2=\{\alpha:\alpha\in S_1$, $\theta_{\alpha}=\theta\}$}

\noindent is stationary in $\zeta$.   Now there is a
$Y\in[2^{\delta}]^{<\kappa}$ such that
$S_3=\{\alpha:\alpha\in S_2$, $g_{\alpha}[\theta]\subseteq Y\}$ is
stationary in $\zeta$.

\medskip

\Prf\ {\bf case 1} If $\lambda<\Tr(\kappa)$, then $g_{\alpha}[\theta]$
is a subset of $\kappa$, and is therefore bounded above in $\kappa$, for
each $\alpha$.   Let $\theta'<\kappa$ be such that

\Centerline{$S_3
=\{\alpha:\alpha\in S_2$, $g_{\alpha}[\theta]\subseteq\theta'\}$}

\noindent is stationary in $\zeta$, and take $Y=\theta'$.

\medskip

\quad{\bf case 2} If $\lambda\ge\Tr(\kappa)$, then
$g_{\alpha}(\theta)<\alpha$ for $\alpha\in S_2$;
by the Pressing-Down Lemma there is a $\theta'<\zeta$ such that

\Centerline{$S'_2
=\{\alpha:\alpha\in S_2$, $g_{\alpha}(\theta)=\theta'\}$}

\noindent is stationary in $\zeta$.   Then
$g_{\alpha}\restr\theta=g_{\beta}\restr\theta$ for all $\alpha$,
$\beta\in S'_2$;  take $Y$
to be the common value of $g_{\alpha}[\theta]$ for
$\alpha\in S'_2$.\ \Qed

\medskip

{\bf (g)} For each $\alpha\in S_3$, set

\Centerline{$Q_{\alpha}=f_M[V_{\alpha}\cap\theta]
=f_M[V_{\alpha}\cap\theta_{\alpha}]$,}

\noindent so that $\mu^*_MQ_{\alpha}=1$.  If $y\in Q_{\alpha}$, take
$\xi\in V_{\alpha}\cap\theta$ such that $f_M(\xi)=y$;  then

\Centerline{$g_{\alpha}^*(y)=g_{\alpha}^*\pi_Mf(\xi)
=\tilde g_{\alpha}f(\xi)=hg_{\alpha}(\xi)\in h[Y]$.}

\noindent Thus
$g_{\alpha}^*\restr Q_{\alpha}\subseteq f_M[\theta]\times h[Y]$ for
every $\alpha\in S_3$, and we can apply
543Dd to $X=\{0,1\}^M$, $\mu=\mu_M$ and the family
$\langle g^*_{\alpha}\restr Q_{\alpha}\rangle_{\alpha\in S'}$,
where $S'\subseteq S_3$ is a set with cardinal $\kappa$,
to see that there are distinct $\alpha$, $\beta\in S_3$ such that
$\mu^*_M\{y:y\in Q_{\alpha}\cap Q_{\beta}$,
$g^*_{\alpha}(y)=g^*_{\beta}(y)\}>0$.
Now, however, consider

\Centerline{$E=\{y:y\in\{0,1\}^M$, $g^*_{\alpha}(y)=g^*_{\beta}(y)\}$.}

\noindent Then $E=\bigcap_{\iota<\delta}E_{\iota}$, where

\Centerline{$E_{\iota}
=\{y:y\in\{0,1\}^M$, $g^*_{\alpha}(y)(\iota)=g^*_{\beta}(y)(\iota)\}$}

\noindent is a Baire subset of $\{0,1\}^M$ for each $\iota<\delta$.
Because $\delta<\kappa$,

$$\eqalign{\nu f_M^{-1}[E]
&=\nu(\bigcap_{\iota<\delta}f_M^{-1}[E_{\iota}])
=\inf_{I\in[\delta]^{<\omega}}
  \nu(\bigcap_{\iota\in I}f_M^{-1}[E_{\iota}])\cr
&=\inf_{I\in[\delta]^{<\omega}}\mu_M(\bigcap_{\iota\in I}E_{\iota})
\ge\mu_M^*E
>0.}$$

\noindent Consequently

$$\eqalign{0
&<\nu f^{-1}_M[E]
=\nu\{\xi:g^*_{\alpha}\pi_M f(\xi)=g^*_{\beta}\pi_M f(\xi)\}\cr
&=\nu\{\xi:\tilde g_{\alpha}f(\xi)=\tilde g_{\beta}f(\xi)\}
=\nu\{\xi:\xi\in V_{\alpha}\cap V_{\beta},
  \,\tilde g_{\alpha}f(\xi)=\tilde g_{\beta}f(\xi)\}\cr
&=\nu\{\xi:hg_{\alpha}(\xi)=hg_{\beta}(\xi)\}
=\nu\{\xi:g_{\alpha}(\xi)=g_{\beta}(\xi)\}\cr}$$

\noindent (because $h$ is injective).   But this is
absurd, because in (b) above we chose
$g_{\alpha}$, $g_{\beta}$ in such a way that
$\{\xi:g_{\alpha}(\xi)=g_{\beta}(\xi)\}$ would be bounded in $\kappa$.\
\BanG\

\medskip

{\bf (h)} Thus the result is true for \Mth\ witnessing probabilities on
$\kappa$.   In general, if $\nu$ is any witnessing probability on
$\kappa$, there is a non-negligible $A\subseteq\kappa$ such that the
subspace measure on $A$ is \Mth;  setting
$\nuprime C=\Bover1{\nu A}\nu(A\cap C)$ for $C\subseteq\kappa$, we
obtain a
\Mth\ witnessing probability $\nuprime$.   Now the Maharam type of $\nu$
is at least as great as the Maharam type of $\nuprime$, so is at least
$\min(2^{\kappa},\kappa^{(+\omega)})$, as required.
}%end of proof of 543E

\leader{543F}{Theorem} Let $(X,\Cal PX,\mu)$ be an atomless
semi-finite measure space.   Write $\kappa=\add\mu$.   Then the
Maharam type
of $(X,\Cal PX,\mu)$ is at least $\min(\kappa^{(+\omega)},2^{\kappa})$,
and in particular is greater than $\kappa$.

\proof{ Let $\ofamily{\xi}{\kappa}{E_{\xi}}$ be a family in
$\Cal N(\mu)$ such that
$E=\bigcup_{\xi<\kappa}E_{\xi}\notin\Cal N(\mu)$.   Let $F\subseteq E$
be a set of non-zero finite measure.
Set $f(x)=\min\{\xi:x\in E_{\xi}\}$ for $x\in F$.   Let $\mu_F$ be the
subspace measure on $F$ and $\mu'=(\mu F)^{-1}\mu_F$ the corresponding
probability measure;  of course $\dom\mu'=\Cal PF$ and
$\mu'(F\cap E_{\xi})=0$ for every $\xi<\kappa$.   Note also that

\Centerline{$\add\mu'=\add\mu_F\ge\add\mu\ge\kappa$}

\noindent (521Fc).   Let $\nu$ be the image measure $\mu'f^{-1}$, so
that $\dom\nu=\Cal P\kappa$ and $\nu$ is $\kappa$-additive (521Hb).
Also $\nu\{\xi\}\le\mu'E_{\xi}=0$ for every $\xi$, so $\nu$ witnesses
that $\kappa$ is \rvm.   Next, $\mu_F$ is atomless (214Ka), so $\mu'$
also is.   There is therefore a function $g:F\to[0,1]$ which is \imp\
for $\mu'$ and Lebesgue measure (343Cb), and $F$ can be covered by
$\frak c$ negligible sets;  accordingly $\add\mu'\le\frak c$ so
$\kappa\le\frak c$ and $\nu$ must be atomless (543Bc).

Let $(\frak A,\bar\mu)$, $(\frak A',\bar\mu')$ and $(\frak B,\bar\nu)$
be the measure algebras of $\mu$, $\mu'$ and $\nu$ respectively.   Then
$\frak A'$ is isomorphic to a principal ideal of $\frak A$ (322I), so
$\tau(\frak A)\ge\tau(\frak A')$ (514Ed).   Next, $f:F\to\kappa$ induces
a measure-preserving Boolean homomorphism from $\frak B$ to $\frak A'$,
so that $\tau(\frak A')\ge\tau(\frak B)$ (332Tb).   Now 543E tells us
that

\Centerline{$\min(\kappa^{(+\omega)},2^{\kappa})
\le\tau(\frak B)\le\tau(\frak A)$,}

\noindent as required.
}%end of proof of 543F

\leader{543G}{Corollary} Let $(X,\Cal PX,\nu)$ be an
atomless probability space, and $\kappa=\add\nu$.   Let
$(Z,\Sigma,\mu)$ be a compact probability space with Maharam type
$\lambda\le\min(2^{\kappa},\kappa^{(+\omega)})$ (e.g.,
$Z=\{0,1\}^{\lambda}$ with its usual measure).   Then there is an \imp\
function $f:X\to Z$.

\proof{ Let $(\frak A,\bar\mu)$ and $\frak B,\bar\nu)$ be the measure
algebras of $\mu$, $\nu$ respectively.   By 543F, the Maharam type of
the subspace measure $\nu_C$ is at least $\lambda$ whenever
$C\subseteq X$ and $\nu C>0$;  that is, every non-zero principal ideal
of $\frak B$ has Maharam type at least $\lambda$.   So there is
a measure-preserving Boolean homomorphism from $\frak A$ to $\frak B$
(332P).   Because $\mu$ is compact, this is represented by an \imp\
function from $X$ to $Z$ (343B).
}%end of proof of 543G

\leader{543H}{Corollary} If $\kappa$ is an \am\ cardinal, and
$(Z,\mu)$ is a compact probability space with Maharam type at most
$\min(2^{\kappa},\kappa^{(+\omega)})$, then there is an extension of
$\mu$ to a $\kappa$-additive measure defined on $\Cal PZ$.

\proof{ Let $\nu$ be a witnessing probability on $\kappa$;  by 543G,
there is an \imp\ function $f:X\to Z$;  now the image measure
$\nu f^{-1}$ extends $\mu$ to $\Cal PZ$.
}%end of proof of 543H

\leader{543I}{Corollary} If $\kappa$ is an \am\ cardinal, with
witnessing probability $\nu$, and $2^{\kappa}\le\kappa^{(+\omega)}$,
then $(\kappa,\Cal P\kappa,\nu)$ is \Mth, with
Maharam type $2^{\kappa}$.

\proof{ If $C\in\Cal P\kappa\setminus\Cal N(\nu)$, then
the Maharam type of the subspace measure on $C$ is at least
$2^{\kappa}$, by
543F;  but also it cannot be greater than $\#(\Cal PC)=2^{\kappa}$.
}%end of proof of 543I

\leader{543J}{Proposition} Let $\kappa$ be an \am\ cardinal, $\nu$ a
witnessing probability on $\kappa$, and $\frak A$ the measure algebra of
$\nu$.   Then

(a) there is a $\gamma<\kappa$ such that $2^{\gamma}=2^{\delta}$ for
every cardinal $\delta$ such that $\gamma\le\delta<\kappa$;

(b) the cardinal power $\tau(\frak A)^{\gamma}$ is $2^{\kappa}$;

(c) if $\frak c<\kappa^{(+\omega_1)}$, then
$\#(\frak A)=\tau(\frak A)^{\omega}=2^{\kappa}$.

%(d) if $\kappa=\frak c$ then $\tau(\frak A)=2^{\frakc}$.

\proof{ Use 542F-542G.   Because $\kappa$ is \qm\ and
$\kappa\le\frak c$, 542Fa tells us that there is a $\gamma$ as in (a);
and now 542Fb and 515M tell us that

\Centerline{$2^{\kappa}=\tau(\frak A)^{\gamma}
=(\tau(\frak A)^{\omega})^{\gamma}=\#(\frak A)^{\gamma}$.}

\noindent If $\frak c<\kappa^{(+\omega_1)}$, then 542G tells us
that $\#(\frak A)=2^{\kappa}$.
}%end of proof of 543J

\leader{543K}{Proposition} Let $\kappa$ be an \am\ cardinal.
If there is a witnessing
probability on $\kappa$ with Maharam type $\lambda$, then there is a
\Mth\
normal witnessing probability $\nu$ on $\kappa$ with Maharam type
at most $\lambda$.

\proof{ Repeat the proof of 543Ba, with $X=\kappa$ and $\mu$ a witnessing
probability on $\kappa$ with Maharam type $\lambda$.   Taking a
non-negligible $Y\subseteq\kappa$ and $g:Y\to\kappa$ such that
$\nu=\Bover1{\mu Y}\mu_Yg^{-1}$ is normal, then $g$ induces an
embedding of the measure algebra of $\nu$ into a principal ideal of the
measure algebra of $\mu$, so the Maharam type of $\nu$ is at most
$\lambda$.   There is now
an $E\in\Cal P\kappa\setminus\Cal N(\nu)$ such that
the subspace measure $\nu_E$ is \Mth, and
setting $\nuprime A=\nu(A\cap E)/\nu E$ for $A\subseteq\kappa$ we
obtain a \Mth\ probability measure $\nuprime$ with Maharam
type less than or equal to $\lambda$.   Now $\nuprime$ is again normal.
\Prf\ Let $\ofamily{\xi}{\kappa}{I_{\xi}}$ be any family in
$\Cal N(\nuprime)$, and set
$I=\{\xi:\xi<\kappa$, $\xi\in\bigcup_{\eta<\xi}I_{\eta}\}$.   Then
$I_{\xi}\cap E\in\Cal N(\nu)$ for every $\xi$, so
$I\cap E=\{\xi:\xi<\kappa$, $\xi\in\bigcup_{\eta<\xi}I_{\eta}\cap E\}$
is $\nu$-negligible and $I$ is $\nuprime$-negligible.\ \QeD\   So we
have an appropriate normal witnessing probability.
}%end of proof of 543K

\leader{543L}{Proposition} Suppose that $\nu$ is a \Mth\
witnessing probability on an \am\ cardinal
$\kappa$ with Maharam type $\lambda$.   Then there is a
\Mth\ witnessing probability $\nuprime$ on
$\kappa$ with Maharam type at least
$\Tr_{\Cal N(\nu)}(\kappa;\lambda)$.

\proof{ Let $\nu_1$ be the $\kappa$-additive probability on
$\kappa\times\kappa$ given by

\Centerline{$\nu_1 C=\int\nu C[\{\xi\}]\nu(d\xi)$ for every
$C\subseteq\kappa\times\kappa$.}

\noindent Set $\theta=\Tr_{\Cal N(\nu)}(\kappa;\lambda)$.
By 541F there is a family $F\subseteq\lambda^{\kappa}$
such that $\#(F)=\theta$ and $\{\xi:f(\xi)=g(\xi)\}\in\Cal N(\nu)$
for all distinct $f$, $g\in F$.   Let
$\langle E_{\xi}\rangle_{\xi<\lambda}$
be a $\nu$-stochastically independent family of subsets of $\kappa$
of $\nu$-measure $\bover12$.   For each $f\in F$ set

\Centerline{$C_f=\{(\xi,\eta):\xi<\kappa$, $\eta\in E_{f(\xi)}\}$.}

\noindent Then for any non-empty finite subset $I$ of
$F$, $\nu(\bigcap_{f\in I}E_{f(\xi)})=2^{-\#(I)}$ for
$\nu$-almost every $\xi$, so that

\Centerline{$\nu_1(\bigcap_{f\in I}C_f)=2^{-\#(I)}$.}

\noindent Thus $\langle C_f\rangle_{f\in F}$ is
stochastically independent
for $\nu_1$, and the Maharam type of the subspace measure
$(\nu_1)_C$ is at least
$\#(F)=\theta$ whenever $\nu_1C>0$.   Once again, take
$\nu_2C=\nu_1(C\cap D)/\nu_1 D$ for some $D$ for which
$(\nu_1)_D$ is \Mth, to obtain a \Mth\
$\kappa$-additive probability $\nu_2$ with Maharam type at least $\theta$.
Finally, of course, $\nu_2$ can be copied onto a
probability $\nuprime$ on $\kappa$, as asked for.
}%end of proof of 543L


\exercises{\leader{543X}{Basic exercises (a)}
%\spheader 543Xa
Let $\kappa$ be an \am\ cardinal.
Show that the following are equiveridical:  (i) every witnessing
probability $\nu$ on $\kappa$ is \Mth\;  (ii) any two
witnessing probabilities on $\kappa$ have the same
Maharam type.
%543K

\spheader 543Xb Let $\kappa$ be an \am\ cardinal.
Show that the following are equiveridical:  (i) every normal witnessing
probability $\nu$ on $\kappa$ is \Mth\;  (ii) any two
normal witnessing probabilities on $\kappa$ have the same
Maharam type.
%543K

\spheader 543Xc Suppose that $\frak c$ is \am.   Show that there is a
\Mth\ normal witnessing probability on $\frak c$ with
Maharam type $2^{\frakc}$.   \Hint{542Ga, 5A1Mc.}
%543L

\leader{543Y}{Further exercises (a)}
%\spheader 543Ya
Let $\nu$ be a witnessing probability on an
atomlessly-measurable cardinal $\kappa$ with Maharam type $\lambda$.
Let $F$ be the set of all functions $f\subseteq\kappa\times\lambda$ such
that $\dom f\notin\Cal N(\nu)$, and let $\theta$ be

$$\eqalign{\sup\{\#(F_0):F_0&\subseteq F,\,\{\xi:\xi\in\dom f\cap\dom g,
\,f(\xi)=g(\xi)\}\in\Cal N(\nu)\cr
&\mskip250mu\text{ for all distinct }f,\,g\in F_0\}.\cr}$$

\noindent Show that there is a witnessing probability $\nuprime$ on
$\kappa$ with Maharam type at least $\theta$.
%543L mt54bits
}%end of exercises

\leader{543Z}{Problems} Let $\kappa$ be an \am\ cardinal.

\spheader 543Za Must every witnessing
probability $\nu$ on $\kappa$ be \Mth?   \cmmnt{(See 555E.)}
%543I 543J

\spheader 543Zb Must every normal witnessing probability $\nu$ on $\kappa$
be \Mth?

\leaveitout{more questions:  if $\frak c$ is \am, must any witnessing
probability have Maharam type $2^{\frakc}$?

If $\kappa$ has a witnessing probability with Maharam type $\lambda$,
does it have a normal witnessing probability with the same type?
}%end of leaveitout

\endnotes{
\Notesheader{543} The results of 543I-543J leave a tantalizingly narrow
gap;  it seems possible that the Maharam type of a witnessing probability
on an \am\ cardinal $\kappa$ is determined by $\kappa$ (543Xa, 543Za).
If so,
there is at least a chance that there is a proof depending on no ideas
more difficult than those above.   To find a counter-example, however,
we may need not only to make some strong assumptions about the
potential existence of appropriate large cardinals, but also to find a new
method of constructing models with \am\ cardinals.   Possibly we get
a different question if we look at normal witnessing
probabilities (543Zb).   A positive answer to either part of 543Z
would have implications for transversal numbers (543L, 543Ya).

}%end of notes

\discrpage



	
