\frfilename{mt326.tex}
\versiondate{21.5.11}
\copyrightdate{2001}

\def\chaptername{Measure algebras}
\def\sectionname{Additive functionals on Boolean algebras}

\newsection{326}

I devote two sections to the general theory of additive functionals on
measure algebras.   As many readers will rightly be in a hurry to get on
to the next two chapters, I remark that the only significant result
needed for \S\S331-332 is the Hahn decomposition of a countably
additive functional (326M), and that this is no more than a translation
into the language of measure algebras of a theorem already given in
Chapter 23.   The concept of `standard extension' of a
countably additive functional from a subalgebra (327F-327G) will be used
for a theorem in \S333, and as preparation for Chapter 36.

I begin with notes on the space of additive functionals on an arbitrary
Boolean algebra (326A-326D), corresponding to 231A-231B, but adding a
more general form of the Jordan decomposition of a bounded additive
functional into positive and negative parts (326D).
The next four paragraphs are starred, because they will not be needed in
this volume;  326E is essential if you want to look at additive functionals
on free products, 326F is a basic classification criterion, and 326H is an
important extension of a fundamental fact about atomless measures noted in
215D, but all can be passed over on first reading.
The next
subsection (326I-326M) deals with countably additive functionals,
corresponding to 231C-231F.   In 326N-326T I develop a new idea, that of
`completely additive' functional, which does not match anything in
the previous treatment.

\leader{326A}{Additive functionals:  Definition}  Let $\frak A$ be a
Boolean algebra.   A functional $\nu:\frak A\to\Bbb R$ is {\bf
finitely additive}, or just {\bf additive}, if
$\nu(a\Bcup b)=\nu a+\nu b$ whenever $a$, $b\in\frak A$ and
$a\Bcap b=0$.

\cmmnt{
A non-negative additive functional is sometimes called a {\bf finitely
additive measure} or {\bf charge}.}

\leader{326B}{Elementary facts} Let $\frak A$ be a Boolean algebra and
$\nu:\frak A\to\Bbb R$ a finitely additive functional.   \cmmnt{The
following
will I hope be obvious.}

\header{326Ba}{\bf (a)} $\nu 0=0$\cmmnt{ (because $\nu 0=\nu 0+\nu
0$)}.

\header{326Bb}{\bf (b)} If $c\in\frak A$, then $a\mapsto\nu(a\Bcap c)$
is additive\cmmnt{ (because $(a\Bcap c)\Bcup(b\Bcap c)=(a\Bcup
b)\Bcap c$)}.

\header{326Bc}{\bf (c)} $\alpha\nu$ is an additive functional
for any
$\alpha\in\Bbb R$.   If $\nuprime$ is another finitely additive functional
on $\frak
A$, then $\nu+\nuprime$ is  additive.

\header{326Bd}{\bf (d)} If $\langle\nu_i\rangle_{i\in I}$ is any family
of finitely additive functionals such that
$\nuprime a=\sum_{i\in I}\nu_i a$ is defined in $\Bbb R$ for every
$a\in \frak A$, then $\nuprime$ is additive.

\header{326Be}{\bf (e)} If $\frak B$ is another Boolean algebra and
$\pi:\frak B\to\frak A$ is a Boolean homomorphism, then $\nu\pi:\frak
B\to\Bbb R$ is additive.   In particular, if $\frak B$ is a subalgebra
of $\frak A$, then
$\nu\restrp\frak B:\frak B\to\Bbb R$ is  additive.

\header{326Bf}{\bf (f)} $\nu$ is non-negative iff it is
order-preserving -- that is,

\Centerline{$\nu a\ge 0\text{ for every }a\in\frak A
\iff \nu b\le\nu c\text{ whenever }b\Bsubseteq c$\dvro{.}{}}

\prooflet{\noindent (because $\nu c=\nu b+\nu(c\Bsetminus b)$ if
$b\Bsubseteq c$).}

\leader{326C}{The space of additive functionals} Let $\frak A$ be any
Boolean algebra.   From 326Bc we see that the set $M$ of all finitely
additive real-valued functionals on $\frak A$ is a linear
space\cmmnt{ (a
linear subspace of $\BbbR^{\frak A}$)}.   We give it the ordering
induced by that of $\BbbR^{\frak A}$\cmmnt{, so that $\nu\le\nuprime$ iff
$\nu a\le\nuprime a$ for every $a\in\frak A$}.   This renders it a partially
ordered linear space\cmmnt{ (because $\BbbR^{\frak A}$ is)}.

\leader{326D}{The Jordan decomposition (I):  Proposition} Let $\frak A$
be a Boolean algebra, and $\nu$ a finitely additive real-valued
functional on $\frak A$.   Then the following are equiveridical:

(i) $\nu$ is bounded;

(ii) $\sup_{n\in\Bbb N}|\nu a_n|<\infty$ for every disjoint sequence
$\sequencen{a_n}$ in $\frak A$;

(iii) $\lim_{n\to\infty}|\nu a_n|=0$ for every disjoint sequence
$\sequencen{a_n}$ in $\frak A$;

(iv) $\sum_{n=0}^{\infty}|\nu a_n|<\infty$ for every disjoint sequence
$\sequencen{a_n}$ in $\frak A$;

(v) $\nu$ is expressible as the difference of two non-negative additive
functionals.

\proof{{\bf (a)(i)$\Rightarrow$(v)} Assume that $\nu$ is bounded.   For
each
$a\in\frak A$, set

\Centerline{$\nu^+a=\sup\{\nu b:b\Bsubseteq a\}$.}

\noindent Because $\nu$ is bounded, $\nu^+$ is real-valued.   Now
$\nu^+$ is  additive.   \Prf\ If $a$, $b\in\frak A$ and $a\Bcap b=0$,
then

$$\eqalignno{\nu^+(a\Bcup b)
&=\sup_{c\Bsubseteq a\Bcup b}\nu c
=\sup_{d\Bsubseteq a,e\Bsubseteq b}\nu(d\Bcup e)
=\sup_{d\Bsubseteq a,e\Bsubseteq b}\nu d+\nu e\cr
\noalign{\noindent (because $d\Bcap e\Bsubseteq a\Bcap b=0$ whenever
$d\Bsubseteq a$, $e\Bsubseteq b$)}
&=\sup_{d\Bsubseteq a}\nu d+\sup_{e\Bsubseteq b}\nu e
=\nu^+a+\nu^+b.\text{ \Qed}\cr}$$

\noindent Consequently $\nu^-=\nu^+-\nu$ also is additive (326Bc).

Since

\Centerline{$0=\nu 0\le\nu^+a$,\quad $\nu a\le\nu^+a$}

\noindent for every $a\in\frak A$, $\nu^+\ge 0$ and $\nu^-\ge 0$.   Thus
$\nu=\nu^+-\nu^-$ is the difference of two non-negative  additive
functionals.

\medskip

{\bf (b)(v)$\Rightarrow$(iv)} If $\nu$ is expressible as
$\nu_1-\nu_2$, where $\nu_1$ and $\nu_2$ are non-negative  additive
functionals, and $\sequencen{a_n}$ is disjoint, then

\Centerline{$\sum_{i=0}^n\nu_ja_i=\nu_j(\sup_{i\le n}a_i)\le\nu_j1$}

\noindent for every $n$, both $j$, so that

\Centerline{$\sum_{i=0}^{\infty}|\nu a_i|
\le\sum_{i=0}^{\infty}\nu_1a_i+\sum_{i=0}^{\infty}\nu_2a_i
\le\nu_11+\nu_21<\infty$.}

\medskip

{\bf (c)(iv)$\Rightarrow$(iii)$\Rightarrow$(ii)} are trivial.

\medskip

{\bf (d) not-(i)$\Rightarrow$not-(ii)} Suppose that $\nu$ is unbounded.
Choose sequences $\sequencen{a_n}$, $\sequencen{b_n}$ inductively, as
follows.   $b_0=1$.   Given that $\sup_{a\Bsubseteq b_n}|\nu a|=\infty$,
choose $c_n\Bsubseteq b_n$ such that $|\nu c_n|\ge|\nu b_n|+n$;  then
$|\nu c_n|\ge n$ and

\Centerline{$|\nu(b_n\Bsetminus c_n)|=|\nu b_n-\nu c_n|\ge|\nu c_n|-|\nu
b_n|\ge n$.}

\noindent We have

$$\eqalign{\infty
&=\sup_{a\Bsubseteq b_n}|\nu a|
=\sup_{a\Bsubseteq b_n}|\nu (a\Bcap c_n)+\nu(a\Bsetminus c_n)|\cr
&\le\sup_{a\Bsubseteq b_n}|\nu (a\Bcap c_n)|+|\nu(a\Bsetminus c_n)|
\le\sup_{a\Bsubseteq b_n\cap c_n}|\nu a|+
   \sup_{a\Bsubseteq b_n\Bsetminus c_n}|\nu a|,\cr}$$

\noindent so at least one of
$\sup_{a\Bsubseteq b_n\cap c_n}|\nu a|$, $\sup_{a\Bsubseteq
b_n\Bsetminus
c_n}|\nu a|$  must be infinite;  take $b_{n+1}$ to be one of $c_n$,
$b_n\Bsetminus c_n$ such that $\sup_{a\Bsubseteq b_{n+1}}|\nu
a|=\infty$, and set $a_n=b_n\Bsetminus b_{n+1}$, so that $|\nu a_n|\ge
n$.   Continue.

On completing the induction, we have a disjoint sequence
$\sequencen{a_n}$ such that $|\nu a_n|\ge n$ for every $n$, so that (ii)
is false.
}%end of proof of 326D

\cmmnt{\medskip

\noindent{\bf Remark} I hope that this reminds you of the decomposition
of a function of bounded variation as the difference of monotonic
functions (224D).
}

\leader{*326E}{Additive functionals on free
\dvrocolon{products}}\dvAformerly{3{}26Q}\cmmnt{ In Volume 4, when we
return to the
construction of measures on product spaces, the following fundamental
fact will be useful.

\medskip

\noindent}{\bf Theorem} Let $\familyiI{\frak A_i}$ be a non-empty family
of Boolean algebras, with free product $\frak A$;  write
$\varepsilon_i:\frak A_i\to\frak A$ for the canonical maps, and

\Centerline{$C=\{\inf_{j\in J}\varepsilon_j(a_j):J\subseteq I$ is
finite, $a_j\in\frak A_j$ for every $j\in J\}$.}

\noindent Suppose that $\theta:C\to\Bbb R$ is such that

\Centerline{$\theta c
=\theta(c\Bcap\varepsilon_i(a))
  +\theta(c\Bcap\varepsilon_i(1\Bsetminus a))$}

\noindent whenever $c\in C$, $i\in I$ and $a\in\frak A_i$.   Then there
is a unique finitely additive functional $\nu:\frak A\to\Bbb R$ extending
$\theta$.

\proof{{\bf (a)} It will help if I note at once that $\theta 0=0$.
\Prf

\Centerline{$\theta
0=\theta(0\Bcap\varepsilon_i(0))+\theta(0\Bcap\varepsilon_i(1))
=2\theta 0$}

\noindent for any $i\in I$.\ \Qed

\medskip

{\bf (b)} The key is of course the following fact:  if
$\langle c_r\rangle_{r\le m}$ and $\langle d_s\rangle_{s\le n}$ are two
disjoint families in $C$ with the same supremum in $\frak A$, then
$\sum_{r=0}^m\theta c_r=\sum_{s=0}^n\theta d_s$.   \Prf\ Let
$J\subseteq I$ be a finite set and $\frak B_i\subseteq\frak A_i$ a
finite subalgebra, for each $i\in J$, such that every $c_r$ and every
$d_s$ belongs to the subalgebra $\frak A_0$ of $\frak A$ generated by
$\{\varepsilon_j(b):j\in J,\,b\in\frak B_j\}$.   Next, if $j\in J$ and
$b\in\frak B_j$, then

\Centerline{$\sum_{r=0}^m\theta c_r
=\sum_{r=0}^m\theta(c_r\Bcap\varepsilon_j(b))
+\sum_{r=0}^m\theta(c_r\Bsetminus\varepsilon_j(b))$.}

\noindent We can therefore find a disjoint family
$\langle c'_r\rangle_{r\le m'}$ in $C\cap\frak A_0$ such that

\Centerline{$\sup_{r\le m'}c'_r=\sup_{r\le m}c_r$,
\quad$\sum_{r=0}^{m'}\theta c'_r=\sum_{r=0}^m\theta c_r$,}

\noindent and whenever $r\le m'$, $j\in J$ and $b\in\frak B_j$ then
either $c'_r\Bsubseteq\varepsilon_j(b)$ or $c'_r\Bcap\varepsilon_j(b)=0$;
that is, every $c'_r$ is either $0$ or of the form
$\inf_{j\in J}\varepsilon_j(b_j)$ where $b_j$ is an atom of $\frak B_j$
for every $j$.   Similarly, we can find $\langle d'_s\rangle_{s\le n'}$
such that

\Centerline{$\sup_{s\le n'}d'_s=\sup_{s\le n}d_s$,
\quad$\sum_{s=0}^{n'}\theta d'_s=\sum_{s=0}^n\theta d_s$,}

\noindent and whenever $s\le n'$ and $j\in J$ then
$d'_s$ is either $0$ or of the form
$\inf_{j\in J}\varepsilon_j(b_j)$ where $b_j$ is an atom of $\frak B_j$
for every $j$.   But we now have $\sup_{r\le m'}c'_r=\sup_{s\le n'}d'_s$
while for any $r\le m'$, $s\le n'$ either $c'_r=d'_s$ or
$c'_r\Bcap d'_s=0$.   It follows that the non-zero terms in the finite
sequence $\langle c'_r\rangle_{r\le m'}$ are just a rearrangement of the
non-zero terms in $\langle d'_s\rangle_{s\le n'}$, so that

\Centerline{$\sum_{r=0}^m\theta c_r
=\sum_{r=0}^{m'}\theta c'_r=\sum_{s=0}^{n'}\theta d'_s
=\sum_{s=0}^n\theta d_s$,}

\noindent as required.\ \Qed

\medskip

{\bf (c)} By 315Kb, this means that we have a functional
$\nu:\frak A\to\Bbb R$ such that
$\nu(\sup_{r\le m}c_r)=\sum_{r=0}^m\theta c_r$ whenever
$\langle c_r\rangle_{r\le m}$ is a disjoint family in $C$.   It is now
elementary to check that $\nu$ is additive, and it is clearly the only
additive functional on $\frak A$ extending $\theta$.
}%end of proof of 326E

\leader{*326F}{}\cmmnt{ I give a couple of pages to an interesting
property of
additive functionals on Dedekind $\sigma$-complete Boolean
algebras.   I do not think it will be used in this book, and it really
belongs to the theory of vector measures, which is hardly considered here,
but the ideas are important, and the following definition has other
uses.

\medskip

\noindent}{\bf Definition}\dvAformerly{3{}26Ya}
Let $\frak A$ be a Boolean algebra, and $\nu$ a
finitely additive functional on $\frak A$.
\cmmnt{I will say that} $\nu$ is
{\bf properly atomless} if for every $\epsilon>0$ there is a finite
partition
$\familyiI{a_i}$ of unity in $\frak A$ such that $|\nu a|\le\epsilon$
whenever $i\in I$ and $a\Bsubseteq a_i$.

\leader{*326G}{Lemma}\dvAnew{2011} Let $\frak A$ be a Boolean algebra.

(a)(i) If $\nu$, $\nu':\frak A\to\Bbb R$ are properly atomless
finitely additive functionals and $\alpha\in\Bbb R$, then $\alpha\nu$
and $\nu+\nu'$ are properly atomless additive functionals.

\quad(ii) If $\nu:\frak A\to\Bbb R$ is a properly atomless
finitely additive
functional, then $\nu$ is bounded and $\nu$ can be expressed as the
difference of two non-negative properly atomless additive
functionals.

(b) Suppose that $\frak A$ is Dedekind $\sigma$-complete and that
$\familyiI{\nu_i}$ is a family of non-negative additive functionals on
$\frak A$ such that for every $a\in\frak A$ there are an
$\alpha\in[\bover13,\bover23]$ and an $a'\Bsubseteq a$ such that
$\nu_ia'=\alpha\nu_ia$ for every $i\in I$.
Then for any $a\in\frak A$ there is a
non-decreasing family $\family{t}{[0,1]}{a_t}$ in $\frak A$ such that
$a_0=0$, $a_1=a$ and $\nu_ia_t=t\nu_ia$ for every $t\in[0,1]$ and $i\in I$.

(c) Suppose that $\frak A$ is Dedekind $\sigma$-complete and that
$\nu_0,\ldots,\nu_n:\frak A\to\coint{0,\infty}$ are
properly atomless additive functionals such that $\nu_ia\le\nu_0a$ for
every $i\le n$ and $a\in\frak A$.   Then for any $a\in\frak A$ there is a
non-decreasing family $\family{t}{[0,1]}{a_t}$ in $\frak A$ such that
$a_0=0$, $a_1=a$ and $\nu_ia_t=t\nu_ia$ for every $t\in[0,1]$ and $i\le n$.

\proof{{\bf (a)(i)} Let $\epsilon>0$.   Then there are finite partitions
$\familyiI{a_i}$, $\family{j}{J}{b_j}$ of unity in $\frak A$ such that
$|\nu a|\le\Bover{\epsilon}{2+|\alpha}$ whenever $i\in I$ and
$a\Bsubseteq a_i$, while $|\nu'a|\le\Bover{\epsilon}2$ whenever $j\in J$
and $a\Bsubseteq b_j$.   Now $|(\alpha\nu)(a)|\le\epsilon$ whenever
$i\in I$ and $a\Bsubseteq a_i$.   Moreover,
$\family{(i,j)}{I\times J}{a_i\Bcap b_j}$ is a finite partition of unity in
$\frak A$, and $|(\nu+\nu')(a)|\le\epsilon$ whenever $i\in I$, $j\in J$ and
$a\Bsubseteq a_i\Bcap b_j$.

\medskip

\quad{\bf (ii)}\grheada\ There is a finite partition
$\family{j}{J}{c_j}$ of unity in $\frak A$ such that $|\nu a|\le 1$
whenever $i\in J$ and $a\Bsubseteq c_j$;  now
$|\nu a|\le\sum_{j\in J}|\nu(a\Bcap c_j)|\le\#(J)$ for every $a\in\frak A$,
so $\nu$ is bounded.

\medskip

\qquad\grheadb\ Define $\nu^+$ as in part (a) of the proof of 326D, so that
$\nu^+:\frak A\to\coint{0,\infty}$ is additive.    Now $\nu^+$ is properly
atomless.   \Prf\ Given $\epsilon>0$, there is a finite partition
$\familyiI{a_i}$ of unity in $\frak A$ such that $|\nu a|\le\epsilon$
whenever $i\in I$ and $a\Bsubseteq a_i$;  in which case
$\nu^+a=\sup_{b\Bsubseteq a}\nu b\le\epsilon$ whenever $i\in I$ and
$a\Bsubseteq a_i$.\ \QeD\   As in 326D, $\nu^-=\nu^+-\nu$ is
non-negative, and by
(i) just above (or otherwise)
it is properly atomless, so $\nu=\nu^+-\nu^-$ is the
difference of non-negative properly atomless functionals.

\medskip

{\bf (b)} If $\nu_ia=0$ for every $i\in I$, we can take $a_t=0$ for
$0\le t<1$ and $a_1=a$.   So suppose that $k\in I$ is such that
$\nu_ka>0$.   For $i\in I$, set $\gamma_i=\Bover{\nu_ia}{\nu_ka}$.   Choose
$\sequencen{D_n}$ inductively, as follows.   $D_0=\{0,a\}$.
Given that $D_n$ is a finite totally ordered subset of
$\{b:b\Bsubseteq a\}$ containing $0$ and $a$
and $\nu_id=\gamma_i\nu_kd$ for every $d\in D_n$ and $i\in I$,
then for each
$d\in D_n\setminus\{a\}$ let $d'$ be the next member of $D_n$ strictly
including $d$, and take $b_d\Bsubseteq d'\Bsetminus d$,
$\alpha_d\in[\bover13,\bover23]$ such that
$\nu_ib_d=\alpha_d\nu_i(d'\Bsetminus d)$ for every $i\in I$.
Then

\Centerline{$\nu_i(d\Bcup b_d)
=(1-\alpha_d)\nu_id+\alpha_d\nu_id'
=\gamma_i((1-\alpha_d)\nu_kd+\alpha_d\nu_kd')
=\gamma_i\nu_k(d\Bcup b_d)$}

\noindent for every $i$.   Set $D_{n+1}=D_n\cup\{d\Bcup b_d:d\in D_n\}$;
observe that $D_{n+1}$ is still totally ordered, and continue.
At the end of the induction,
it is easy to see that $\nu_k(d'\Bsetminus d)\le(\bover23)^n\nu_ia$
whenever $n\in\Bbb N$ and $d\Bsubset d'$ are successive members
of $D_n$.

Set $D=\bigcup_{n\in\Bbb N}D_n$.   Then $D$ is a
countable totally ordered set with
least element $0$ and greatest element $a$, and $\{\nu_kd:d\in D\}$ is
dense in $[0,\nu_ka]$.   For $t\in\ocint{0,1}$, set
$a_t=\sup\{d:d\in D$, $\nu_kd\le t\nu_ka\}$;  this is where we need to know
that $\frak A$ is Dedekind $\sigma$-complete.   Set $a_0=0$.
Then $\langle a_t\rangle_{t\in[0,1]}$ is a non-decreasing family with
$a_0=0$ and $a_1=a$.   If $0<t<1$, $i\in I$ and $\epsilon>0$, there are
$d$, $d'\in D$ such that

\Centerline{$t\nu_ka-\epsilon\le\nu_k d\le t\nu_ka<\nu_kd'
\le t\nu_ka+\epsilon$,}

\Centerline{$t\nu_ia-\gamma_i\epsilon
\le\nu_i d\le t\nu_ia<\nu_id'
\le t\nu_ia+\gamma_i\epsilon$;}

\noindent in this case $d\Bsubseteq a_t\Bsubseteq d'$, so

\Centerline{$t\nu_ia-\gamma_i\epsilon\le\nu_ia_t
\le t\nu_ia+\gamma_i\epsilon$;}

\noindent as $\epsilon$ is arbitrary, $\nu_ia_t=t\nu_ia$.   Thus we have a
suitable family $\langle a_t\rangle_{t\ge 0}$.

\medskip

{\bf (c)} Induce on $n$.

\medskip

\quad{\bf (i)} The induction starts with a single non-negative
properly atomless functional $\nu_0$.    Now for any $a\in\frak A$ there is
an $a'\Bsubseteq a$ such that $\bover13\nu_0a\le\nu_0a'\le\bover23\nu_0a$.
\Prf\ This is trivial if $\nu_0a=0$.   Otherwise, let $C$ be a
finite partition of unity in $\frak A$ such that $\nu_0c\le\bover13\nu_0a$
for every $c\in C$.   Enumerate $C$ as $\langle c_i\rangle_{i<m}$ and for
$i\le m$ set $b_i=a\Bcap\sup_{j<i}c_j$.   Then $b_0=0$, $b_m=a$ and
$\nu_0b_{i+1}-\nu_0b_i\le\nu_0c_i\le\bover13\nu_0a$ for each $i$.
So there
must be an $i\le m$ such that $\bover13\nu_0a\le\nu_0b_i\le\bover23\nu_0a$,
and we can set $a'=b_i$.\ \Qed

Now (b), with $I=\{0\}$, gives the result.

\medskip

\quad{\bf (ii)} For the inductive step to $n\ge 1$, I show first that if
$a\in\frak A$ there is an $a'\Bsubseteq a$ such that
$\nu_ia'=\bover12\nu_ia$ for every $i\le n$.   \Prf\ By the inductive
hypothesis, we have a non-decreasing family $\family{t}{[0,1]}{a_t}$ such
that $a_0=0$, $a_1=a$ and $\nu_ia_t=t\nu_ia$ whenever $t\in[0,1]$ and
$i<n$.   Now observe that for $0\le s\le t\le 1$,

\Centerline{$|\nu_na_t-\nu_na_s|=\nu_n(a_t\Bsetminus a_s)
\le\nu_0(a_t\Bsetminus a_s)=(t-s)\nu_0a$.}

\noindent So the functions $t\mapsto\nu_na_t:[0,1]\to\coint{0,\infty}$ and
$f:[0,\bover12]\to\coint{0,\infty}$ are continuous, where
$f(t)=\nu_na_{t+\bover12}-\nu_na_t$ for $0\le t\le\bover12$.   However,
$f(0)+f(\bover12)=\nu_na$, so $\bover12\nu_na$ lies between $f(0)$ and
$f(\bover12)$ and there is a $t\in[0,\bover12]$ such that
$f(t)=\bover12\nu_na$.   Set $a'=a_{t+\bover12}\Bsetminus a_t$;  then
$\nu_ia'=\bover12\nu_ia$ for every $i\le n$, as required.\ \Qed

Once again (b), with $I=\{0,\ldots,n\}$, shows that for any $a\in\frak A$
we have a non-decreasing family $\family{t}{[0,1]}{a_t}$ such that
$a_0=0$, $a_1=1$ and $\nu_ia_t=t\nu_ia$ whenever $t\in[0,1]$ and $i\le n$.
}%end of proof of 326G

\leader{*326H}{Liapounoff's convexity
theorem}\dvAformerly{3{}26Ye}\cmmnt{ ({\smc Liapounoff
40})}
Let $\frak A$ be a Dedekind $\sigma$-complete Boolean
algebra, and $r\ge 1$ an integer.   Suppose that $\nu:\frak A\to\BbbR^r$
is additive in the sense that $\nu(a\Bcup b)=\nu a+\nu b$ whenever
$a\Bcap b=0$\cmmnt{ (see 361B)}, and
properly atomless in the sense that for every $\epsilon>0$ there is a
finite partition $\langle a_j\rangle_{j\in J}$ of unity in $\frak A$ such
that $\|\nu a\|\le\epsilon$ whenever $j\in J$ and $a\Bsubseteq a_j$.
Then $\{\nu a:a\in\frak A\}$ is a convex set in
$\BbbR^r$.\cmmnt{(\footnote{I
learnt this version of the theorem from K.P.S.Bhaskara Rao.})}

\proof{ For $1\le i\le r$, let $\nu_i$ be the $i$th component of $\nu$, so
that $\nu a=\langle\nu_ia\rangle_{1\le i\le r}$ for each $a\in\frak A$.
Then every $\nu_i$ is additive.   Moroever, it is properly atomless.
\Prf\ Given $\epsilon>0$, there is a finite partition $\family{j}{J}{a_j}$
of unity in $\frak A$ such that $|\nu_ia|\le\|\nu a\|\le\epsilon$ whenever
$j\in J$ and $a\Bsubseteq a_j$.\ \QeD\   So we can express $\nu_i$ as
$\nu_i^+-\nu_i^-1$ where $\nu_i^+$ and $\nu_i^{-1}$ are non-negative
properly atomless non-negative functionals (326G(a-ii)).
Set $\tilde\nu a=\sum_{i=1}^r\nu_i^+a+\nu_i^-a$ for $a\in\frak A$.   Then
$\tilde\nu$ is again properly atomless (326G(a-i)).

Suppose that $x$, $y\in\nu[\frak A]$ and $\alpha\in[0,1]$.   Let $a$,
$b\in\frak A$ be such that $\nu a=x$ and $\nu b=y$.   By 326Gc, applied to
$\tilde\nu,\nu_1^+,\nu_1^-,\ldots,\nu_r^+,\nu_r^-$, there is an
$c\Bsubseteq a\Bsetminus b$ such that

\Centerline{$\nu_i^+c=\alpha\nu_i^+(a\Bsetminus b)$,
\quad$\nu_i^-c=\alpha\nu_i^-(a\Bsetminus b)$,}

\noindent for every $i\le r$, so that $\nu_ic=\alpha\nu_i(a\Bsetminus b)$
for every $i\le r$.   Similarly, there is a $d\Bsubseteq b\Bsetminus a$
such that $\nu d=(1-\alpha)\nu(b\Bsetminus d)$.   Now
$e=c\Bcup(a\Bcap b)\Bcup d$,

$$\eqalign{\alpha x+(1-\alpha)y
&=\alpha\nu a+(1-\alpha)\nu b\cr
&=\alpha\nu(a\Bsetminus b)+\alpha\nu(a\Bcap b)+(1-\alpha)\nu(a\Bcap b)
   +(1-\alpha)\nu(b\Bsetminus a)\cr
&=\nu c+\nu(a\Bcap b)+\nu d
=\nu(c\Bcup(a\Bcap b)\Bcup d)
\in\nu[\frak A].\cr}$$

\noindent As $x$, $y$ and $\alpha$ are arbitrary, $\nu[\frak A]$ is convex.
}%end of proof of 326H

\leader{326I}{Countably additive functionals:
Definition}\dvAformerly{3{}26E}
Let $\frak A$ be a Boolean algebra.   A functional
$\nu:\frak A\to\Bbb R$ is {\bf countably additive} or
{\bf $\sigma$-additive} if $\sum_{n=0}^{\infty}\nu a_n$ is defined and
equal to $\nu(\sup_{n\in\Bbb N}a_n)$ whenever
$\sequencen{a_n}$ is a disjoint sequence in $\frak A$ and
$\sup_{n\in\Bbb N}a_n$ is defined in $\frak A$.

\cmmnt{A warning is perhaps in order.   It can happen that $\frak A$
is presented to us as a subalgebra of a larger algebra $\frak B$;  for
instance, $\frak A$ might be an algebra of sets, a subalgebra of some
$\sigma$-algebra $\Sigma\subseteq\Cal PX$.   In this case, there may be
sequences in $\frak A$ which
have a supremum in $\frak A$ which is not a supremum in $\frak B$
(indeed, this will happen just when the embedding is not sequentially
order-continuous).   So we can have a countably additive functional
$\nu:\frak B\to\Bbb R$ such that $\nu\restrp\frak A$ is not countably
additive in the sense used here.   A similar phenomenon will arise when
we come to the Daniell integral in Volume 4 (\S436).
}

\leader{326J}{Elementary facts} Let $\frak A$ be a Boolean algebra
and $\nu:\frak A\to\Bbb R$ a countably additive functional.

\header{326Ja}{\bf (a)} $\nu$ is finitely additive.   \prooflet{(Setting
$a_n=0$ for every $n$, we see from the definition in 326I that
$\nu 0=0$.   Now, given $a\Bcap b=0$, set $a_0=a$, $a_1=b$, $a_n=0$ for
$n\ge 2$ to see that $\nu(a\Bcup b)=\nu a+\nu b$.)}

\header{326Jb}{\bf (b)} If $\sequencen{a_n}$ is a non-decreasing
sequence in $\frak A$ with a supremum $a\in\frak A$, then

\Centerline{$\nu a\cmmnt{\mskip5mu =\nu
a_0+\sumop_{n=0}^{\infty}\nu(a_{n+1}\Bsetminus a_n)
}=\lim_{n\to\infty}\nu a_n$.}

\header{326Jc}{\bf (c)} If $\sequencen{a_n}$ is a non-increasing
sequence in $\frak A$ with an infimum $a\in\frak A$,
then\cmmnt{ $\sequencen{a_0\Bsetminus a_n}$ is a non-decreasing sequence
with supremum $a_0\Bsetminus a$, so}

\Centerline{$\nu a\cmmnt{\mskip5mu =\nu a_0-\nu(a_0\Bsetminus a)
=\nu a_0-\lim_{n\to\infty}\nu(a_0\Bsetminus a_n)
}=\lim_{n\to\infty}\nu a_n$.}

\header{326Jd}{\bf (d)} If $c\in\frak A$, then $a\mapsto\nu(a\Bcap c)$
is countably additive.   \prooflet{(For
$\sup_{n\in\Bbb N}a_n\Bcap c=c\Bcap\sup_{n\in\Bbb N}a_n$ whenever the
right-hand-side is defined, by 313Ba.)}

\header{326Je}{\bf (e)} $\alpha\nu$ is a countably additive functional
for any
$\alpha\in\Bbb R$.   If $\nuprime$ is another countably additive functional
on $\frak A$, then $\nu+\nuprime$ is countably additive.

\header{326Jf}{\bf (f)} If $\frak B$ is another Boolean algebra and
$\pi:\frak B\to\frak A$ is a sequentially order-continuous Boolean
homomorphism, then $\nu\pi$ is a countably additive functional on $\frak
B$.   \prooflet{(For if $\sequencen{b_n}$ is a disjoint sequence in
$\frak B$ with supremum $b$, then $\sequencen{\pi b_n}$ is a disjoint
sequence with supremum $\pi b$.)}

\header{326Jg}{\bf (g)} If $\frak A$ is Dedekind $\sigma$-complete and
$\frak B$ is a $\sigma$-subalgebra of $\frak A$, then
$\nu\restrp\frak B:\frak B\to\Bbb R$ is countably additive.
\prooflet{(For the identity map from $\frak B$ to $\frak A$ is
sequentially order-continuous, by 314Gb.)}


\leader{326K}{Corollary}\dvAformerly{3{}26G}
Let $\frak A$ be a Boolean algebra and $\nu$ a
finitely additive real-valued functional on $\frak A$.

(a) $\nu$ is countably additive iff $\lim_{n\to\infty}\nu a_n=0$
whenever $\sequencen{a_n}$ is a non-increasing sequence in $\frak A$
with infimum $0$ in $\frak A$.

(b) If $\nuprime$ is an additive functional on $\frak A$ and
$|\nuprime a|\le\nu a$ for every $a\in\frak A$, and $\nu$ is countably
additive, then $\nuprime$ is countably additive.

(c) If $\nu$ is non-negative, then $\nu$ is countably additive iff it is
sequentially order-continuous.

\proof{{\bf (a)(i)} If $\nu$ is countably additive and $\sequencen{a_n}$
is a non-increasing sequence in $\frak A$ with infimum $0$, then
$\lim_{n\to\infty}\nu a_n\penalty-100=0$ by 326Jc.   
{\bf (ii)} If $\nu$ satisfies
the condition, and $\sequencen{a_n}$ is a disjoint sequence in $\frak A$
with supremum $a$, set $b_n=a\Bsetminus\sup_{i\le n}a_i$ for each
$n\in\Bbb N$;  then $\sequencen{b_n}$ is non-increasing and has infimum
$0$, so

\Centerline{$\nu a-\sum_{i=0}^n\nu a_i
=\nu a-\nu(\sup_{i\le n}a_i)
=\nu b_n\to 0$}

\noindent as $n\to\infty$, and $\nu a=\sum_{n=0}^{\infty}\nu a_n$;  thus
$\nu$ is countably additive.

\medskip

{\bf (b)} If $\sequencen{a_n}$ is a disjoint sequence in $\frak A$ with
supremum $a$, set $b_n=\sup_{i\le n}a_i$ for each $n$;  then $\nu
a=\lim_{n\to\infty}\nu b_n$, so

\Centerline{$\lim_{n\to\infty}|\nuprime a-\nuprime b_n|
=\lim_{n\to\infty}|\nuprime(a\Bsetminus b_n)|
\le\lim_{n\to\infty}\nu(a\Bsetminus b_n)
=0$,}

\noindent and

\Centerline{$\sum_{n=0}^{\infty}\nuprime a_n=\lim_{n\to\infty}\nuprime b_n
=\nuprime a$.}

\medskip

{\bf (c)}  If $\nu$ is countably additive, then it is sequentially
order-continuous by 326Jb-326Jc.   If $\nu$ is sequentially
order-continuous, then of course it satisfies the condition of (a), so
is countably additive.
}%end of proof of 326K

\leader{326L}{The Jordan decomposition (II):  Proposition} Let $\frak A$
be a Boolean algebra and $\nu$ a bounded countably additive
real-valued functional on $\frak A$.   Then $\nu$ is expressible as the
difference of two non-negative countably additive functionals.

\proof{ Consider the functional $\nu^+a=\sup_{b\Bsubseteq a}\nu b$
defined in the proof of 326D.   If $\sequencen{a_n}$ is a disjoint
sequence in $\frak A$ with supremum $a$, and $b\Bsubseteq a$, then

\Centerline{$\nu b=\sum_{n=0}^{\infty}\nu(b\Bcap
a_n)\le\sum_{n=0}^{\infty}\nu^+a_n$.}

\noindent As $b$ is arbitrary, $\nu^+a\le\sum_{n=0}^{\infty}\nu^+a_n$.
But of course

\Centerline{$\nu^+a\ge\nu^+(\sup_{i\le n}a_i)=\sum_{i=0}^n\nu^+a_i$}

\noindent for every $n\in\Bbb N$, so
$\nu^+a=\sum_{n=0}^{\infty}\nu^+a_n$.   As $\sequencen{a_n}$ is
arbitrary, $\nu^+$ is countably additive.

Now $\nu^-=\nu^+-\nu$ also is countably additive, and $\nu=\nu^+-\nu^-$
is the difference of non-negative countably additive functionals.
}%end of proof of 326L

\leader{326M}{The Hahn decomposition:  Theorem}\dvAformerly{3{}26I}
Let $\frak A$ be a
Dedekind $\sigma$-complete Boolean algebra and $\nu:\frak A\to\Bbb R$ a
countably additive functional.   Then $\nu$ is bounded and there is a
$c\in\frak A$ such that $\nu a\ge 0$ whenever $a\Bsubseteq c$, while
$\nu a\le 0$ whenever $a\Bcap c=0$.

\ifwithproofs
\medskip

\noindent{\bf first proof} By 314M, there are a set $X$ and a
$\sigma$-algebra $\Sigma$ of subsets of $X$ and a sequentially
order-continuous Boolean homomorphism $\pi$ from $\Sigma$ onto
$\frak A$.   Set $\nu_1=\nu\pi:\Sigma\to\Bbb R$.   Then $\nu_1$ is
countably additive (326Jf).   So $\nu_1$ is bounded and there is a set
$H\in\Sigma$
such that $\nu_1F\ge 0$ whenever $F\in\Sigma$ and $F\subseteq H$ and
$\nu_1F\le 0$ whenever $F\in\Sigma$ and $F\cap H=\emptyset$ (231Eb).
Set $c=\pi H\in\frak A$.   If $a\Bsubseteq c$, then there is an
$F\in\Sigma$ such that $\pi F=a$;  now $\pi(F\cap H)=a\Bcap c=a$, so
$\nu a=\nu_1(F\cap H)\ge 0$.   If $a\Bcap c=0$, then there is an
$F\in\Sigma$
such that $\pi F=a$;  now $\pi(F\setminus H)=a\Bsetminus c=a$, so
$\nu a=\nu_1(F\setminus H)\le 0$.

\medskip

\noindent{\bf second proof (a)} Note first that $\nu$ is bounded.
\Prf\ If $\sequencen{a_n}$ is a disjoint sequence in $\frak A$, then
$\sum_{n=0}^{\infty}\nu a_n$ must exist and be equal to
$\nu(\sup_{n\in\Bbb N}a_n)$;  in particular,
$\lim_{n\to\infty}\nu a_n=0$.   By 326D, $\nu$ is bounded.
\Qed

\medskip

{\bf (b)(i)} We know that $\gamma=\sup\{\nu a:a\in\frak A\}<\infty$.
Choose
a sequence $\sequencen{a_n}$ in $\frak A$ such that
$\nu a_n\ge\gamma-2^{-n}$ for every $n\in\Bbb N$.   For $m\le n\in\Bbb
N$, set $b_{mn}=\inf_{m\le i\le n}a_i$.   Then
$\nu b_{mn}\ge\gamma-2\cdot 2^{-m}+2^{-n}$ for every $n\ge m$.   \Prf\
Induce
on $n$.   For $n=m$, this is due to the choice of $a_m=b_{mm}$.   For
the inductive step, we have $b_{m,n+1}=b_{mn}\Bcap a_{n+1}$, while
surely
$\gamma\ge\nu(a_{n+1}\Bcup b_{mn})$, so

$$\eqalignno{\gamma+\nu b_{m,n+1}
&\ge\nu(a_{n+1}\Bcup b_{mn})+\nu(a_{n+1}\Bcap b_{mn})\cr
&=\nu a_{n+1}+\nu b_{mn}
\ge\gamma-2^{-n-1}+\gamma-2\cdot 2^{-m}+2^{-n}\cr
\noalign{\noindent (by the choice of $a_{n+1}$ and the inductive
hypothesis)}
&=2\gamma-2\cdot 2^{-m}+2^{-n-1}.\cr}$$

\noindent Subtracting $\gamma$ from both sides, $\nu b_{m,n+1}\ge\gamma
-2\cdot 2^{-m}+2^{-n-1}$ and the induction proceeds.\ \Qed

\medskip

\quad{\bf (ii)} Set

\Centerline{$b_m=\inf_{n\ge m}b_{mn}=\inf_{n\ge m}a_n$.}

\noindent Then

\Centerline{$\nu b_m=\lim_{n\to\infty}\nu b_{mn}
\ge\gamma-2\cdot 2^{-m}$,}

\noindent by 326Jc.
Next, $\sequencen{b_n}$ is non-decreasing, so setting $c=\sup_{n\in\Bbb
N}b_n$ we have

\Centerline{$\nu c=\lim_{n\to\infty}\nu b_n\ge\gamma$;}

\noindent  since $\nu c$
is surely less than or equal to $\gamma$, $\nu c=\gamma$.

If $b\in\frak A$ and $b\Bsubseteq c$, then

\Centerline{$\nu c-\nu b=\nu(c\Bsetminus
b)\le\gamma=\nu c$,}

\noindent so $\nu b\ge 0$.   If $b\in\frak A$ and $b\Bcap
c=0$ then

\Centerline{$\nu c+\nu b=\nu(c\Bcup b)\le\gamma=\nu c$}

\noindent so $\nu b\le
0$.   This completes the proof.
\else\fi %end of proof of 326M

\leader{326N}{Completely additive functionals:  Definition} Let $\frak
A$ be a Boolean algebra.   A functional $\nu:\frak A\to\Bbb R$ is {\bf
completely additive} or {\bf $\tau$-additive} if it is finitely additive
and $\inf_{a\in A}|\nu a|=0$ whenever $A$ is a non-empty
downwards-directed set in $\frak A$ with infimum $0$.

\leader{326O}{Basic facts}\dvAformerly{3{}26K}
Let $\frak A$ be a Boolean algebra and $\nu$
a completely additive real-valued functional on $\frak A$.

\header{326Oa}{\bf (a)} $\nu$ is countably additive.   \prooflet{\Prf\
If $\sequencen{a_n}$ is a non-increasing sequence in $\frak A$ with
infimum $0$, then for any infinite $I\subseteq\Bbb N$ the set
$\{a_i:i\in I\}$ is downwards-directed and has infimum $0$, so
$\inf_{i\in I}|\nu a_i|=0$;  which means that $\lim_{n\to\infty}\nu a_n$
must be zero.   By 326Ka, $\nu$ is countably additive.\ \Qed}

\header{326Ob}{\bf (b)} Let $A$ be a non-empty downwards-directed set in
$\frak A$ with infimum $0$.   Then for every $\epsilon>0$ there is an
$a\in A$ such that $|\nu b|\le\epsilon$ whenever $b\Bsubseteq a$.
\prooflet{\Prf\Quer\ Suppose, if possible, otherwise.   Set

\Centerline{$B=\{b:|\nu b|\ge\epsilon,\,\exists\,a\in A,\,b\Bsupseteq
a\}$.}

\noindent If $a\in A$ there is a $b'\Bsubseteq a$ such that $|\nu
b'|>\epsilon$.   Now $\{a'\Bsetminus b':a'\in A,\,a'\Bsubseteq a\}$ is
downwards-directed and has infimum $0$, so there is an $a'\in A$ such
that $a'\Bsubseteq a$ and $|\nu(a'\Bsetminus b')|\le|\nu b'|-\epsilon$.
Set $b=b'\Bcup a'$;  then $a'\Bsubseteq b$ and

\Centerline{$|\nu b|=|\nu b'+\nu(a'\Bsetminus b')|
\ge|\nu b'|-|\nu(a'\Bsetminus b')|\ge\epsilon$,}

\noindent so $b\in B$.   But also $b\Bsubseteq a$.   Thus every member
of $A$ includes some member of $B$.   Since every member of $B$ includes
a member of $A$, $B$ is downwards-directed and has infimum $0$;  but
this is impossible, since $\inf_{b\in B}|\nu b|\ge\epsilon$.\ \Bang\Qed}

\header{326Oc}{\bf (c)} If $\nu$ is non-negative, it is
order-continuous.   \prooflet{\Prf\ (i) If $A$ is a non-empty
upwards-directed set with supremum $a_0$, then $\{a_0\Bsetminus a:a\in
A\}$ is a non-empty downwards-directed set with infimum $0$, so

\Centerline{$\sup_{a\in A}\nu a
=\nu a_0-\inf_{a\in A}\nu(a_0\Bsetminus a)
=\nu a_0$.}

\noindent (ii) If $A$ is a non-empty
downwards-directed set with infimum $a_0$, then $\{a\Bsetminus a_0:a\in
A\}$ is a non-empty downwards-directed set with infimum $0$, so

\Centerline{$\inf_{a\in A}\nu a
=\nu a_0+\inf_{a\in A}\nu(a\Bsetminus a_0)
=\nu a_0$. \Qed}
}

\header{326Od}{\bf (d)} If $c\in\frak A$, then $a\mapsto\nu(a\Bcap c)$
is completely additive.   \prooflet{\Prf\ If $A$ is a non-empty
downwards-directed set with infimum $0$, so is $\{a\Bcap c:a\in A\}$, and
$\inf_{a\in A}|\nu(a\Bcap c)|=0$.\ \Qed}

\header{326Oe}{\bf (e)} $\alpha\nu$ is a completely additive functional
for any
$\alpha\in\Bbb R$.   If $\nuprime$ is another completely additive functional
on $\frak A$, then $\nu+\nuprime$ is completely additive.   \prooflet{\Prf\
We know from 326Bc that $\nu+\nuprime$ is additive.   Let $A$ be a
non-empty
downwards-directed set with infimum $0$.   For any $\epsilon>0$, (b)
tells us that there are $a$, $a'\in A$ such that $|\nu b|\le\epsilon$
whenever $b\Bsubseteq a$ and $|\nuprime b|\le\epsilon$ whenever $b\Bsubseteq
a'$.   But now, because $A$ is downwards-directed, there is a $b\in A$
such that $b\Bsubseteq a\Bcap a'$, which means that $|\nu
b+\nuprime b|\le|\nu b|+|\nuprime b|$ is at most $2\epsilon$.   As $\epsilon$ is
arbitrary, $\inf_{a\in A}|(\nu+\nuprime)(a)|=0$, and $\nu+\nuprime$ is
completely additive.\ \Qed}

\header{326Of}{\bf (f)} If $\frak B$ is another Boolean algebra and
$\pi:\frak B\to\frak A$ is an order-continuous Boolean homomorphism,
then $\nu\pi$ is a completely additive functional on $\frak B$.
\prooflet{\Prf\ By
326Be, $\nu\pi$ is additive.   If $B\subseteq\frak B$ is a non-empty
downwards-directed set with infimum $0$ in $\frak B$, then $\pi[B]$ is a
non-empty downwards-directed set with infimum $0$ in $\frak A$, because
$\pi$ is order-continuous, so $\inf_{b\in B}|\nu\pi b|=0$.\ \Qed}
In particular, if $\frak B$ is a regularly embedded subalgebra of
$\frak A$, then $\nu\restrp\frak B$ is completely additive.

\header{326Og}{\bf (g)} If $\nuprime$ is another additive functional on
$\frak A$ and $|\nuprime a|\le\nu a$ for every $a\in\frak A$, then $\nuprime$ is
completely additive.   \prooflet{\Prf\ If $A\subseteq\frak A$ is
non-empty and downwards-directed and $\inf A=0$, then
$\inf_{a\in A}|\nuprime a|\le\inf_{a\in A}\nu a=0$.\ \Qed}

\leader{326P}{}\cmmnt{ I squeeze a useful fact in here.

\medskip

\noindent}{\bf Proposition} If $\frak A$ is a ccc Boolean algebra, a
functional $\nu:\frak A\to\Bbb R$ is countably additive iff it is
completely additive.

\proof{ If $\nu$ is completely additive it is countably additive, by
326Oa.   If $\nu$ is countably additive and $A$ is a
non-empty downwards-directed set in $\frak A$ with infimum $0$, then
there is a (non-empty) countable subset $B$ of $A$ also with infimum $0$
(316E).   Let $\sequencen{b_n}$ be a sequence running over $B$, and
choose $\sequencen{a_n}$ in $A$ such that $a_0=b_0$, $a_{n+1}\Bsubseteq
a_n\Bcap b_n$ for every $n\in\Bbb N$.   Then $\sequencen{a_n}$ is a
non-increasing sequence with infimum $0$, so
$\lim_{n\to\infty}\nu a_n=0$
(326Jc) and $\inf_{a\in A}|\nu a|=0$.   As $A$ is arbitrary, $\nu$ is
completely additive.
}%end of proof of 326P

\leader{326Q}{The Jordan decomposition (III):  Proposition} Let
$\frak A$ be a Boolean algebra and $\nu$ a completely additive
real-valued functional on $\frak A$.   Then $\nu$ is bounded and
expressible as the difference
of two non-negative completely additive functionals.

\proof{{\bf (a)} I must first check that $\nu$ is bounded.   \Prf\ Let
$\sequencen{a_n}$ be a disjoint sequence in $\frak A$.   Set

\Centerline{$A=\{a:a\in\frak A,$ there is an $n\in\Bbb N$ such that
$a_i\Bsubseteq a$ for every $i\ge n\}$.}

\noindent Then $A$ is closed under $\Bcap$, and if $b$ is any lower
bound for $A$ then $b\Bsubseteq 1\Bsetminus a_n\in A$, so
$b\Bcap a_n=0$,
for every $n\in\Bbb N$;  but this means that $1\Bsetminus b\in A$, so
that $b\Bsubseteq 1\Bsetminus b$ and $b=0$.   Thus $\inf A=0$.   By
326Ob, there is an $a\in A$ such that $|\nu b|\le 1$ whenever
$b\Bsubseteq a$.   By the
definition of $A$, there must be an $n\in\Bbb N$ such that
$|\nu a_i|\le 1$ for every $i\ge n$.   But this means that
$\sup_{n\in\Bbb N}|\nu a_n|$ is finite.   As $\sequencen{a_n}$ is
arbitrary, $\nu$ is bounded, by 326D(ii).\ \Qed

\medskip

{\bf (b)} As in 326D and 326L, set $\nu^+a=\sup_{b\Bsubseteq a}\nu b$
for every $a\in\frak A$.   Then $\nu^+$ is completely additive.   \Prf\
We know that $\nu^+$ is additive.   If $A$ is a non-empty
downwards-directed subset of $\frak A$ with infimum $0$, then for every
$\epsilon>0$ there is an $a\in A$ such that $|\nu b|\le\epsilon$
whenever $b\Bsubseteq a$;  in particular, $\nu^+a\le\epsilon$.   As
$\epsilon$ is arbitrary, $\inf_{a\in A}\nu^+a=0$;  as $A$ is arbitrary,
$\nu^+$ is completely additive.\ \Qed

Consequently $\nu^-=\nu^+-\nu$ is completely additive (326Oe) and
$\nu=\nu^+-\nu^-$ is the difference of non-negative completely additive
functionals.
}%end of proof of 326Q

\leader{326R}{}\cmmnt{ I give an alternative definition of
`completely additive' which you may feel clarifies the concept.

\medskip

\noindent}{\bf Proposition} Let $\frak A$ be a Boolean algebra, and
$\nu:\frak A\to\Bbb R$ a function.   Then the following are
equiveridical:

(i) $\nu$ is completely additive;

(ii) $\nu 1=\sum_{i\in I}\nu a_i$ whenever $\langle a_i\rangle_{i\in I}$
is a partition of unity in $\frak A$;

(iii) $\nu a=\sum_{i\in I}\nu a_i$ whenever
$\langle a_i\rangle_{i\in I}$ is a disjoint family in $\frak A$ with
supremum $a$.

\proof{ (For notes on sums $\sum_{i\in I}$, see 226A.)

\medskip

{\bf (a)(i)$\Rightarrow$(ii)} If $\nu$ is completely additive and
$\langle a_i\rangle_{i\in I}$ is a partition of unity in $A$, then
(inducing on
$\#(J)$) $\nu(\sup_{i\in J}a_i)=\sum_{i\in J}\nu a_i$ for every finite
$J\subseteq I$.   Consider

\Centerline{$A=\{1\Bsetminus\sup_{i\in J}a_i:J\subseteq I$ is
finite$\}$.}

\noindent Then $A$ is non-empty and downwards-directed and has infimum
$0$, so for every
$\epsilon>0$ there is an $a\in A$ such that $|\nu b|\le\epsilon$
whenever $b\Bsubseteq a$ (326Ob again).
Express $a$ as $1\Bsetminus\sup_{i\in J}a_i$
where $J\subseteq I$ is finite.   If now $K$ is another finite subset of
$I$ including $J$,

\Centerline{$|\nu 1-\sum_{i\in K}a_i|
=|\nu(1\Bsetminus\sup_{i\in K}a_i)|\le\epsilon$.}

\noindent As remarked in 226Ad, this means that
$\nu 1=\sum_{i\in I}\nu a_i$, as claimed.

\medskip

{\bf (b)(ii)$\Rightarrow$(iii)} Suppose that $\nu$ satisfies the
condition (ii), and that $\langle a_i\rangle_{i\in I}$ is a disjoint
family with supremum $a$.   Take any $j\notin I$, set $J=I\cup\{j\}$ and
$a_j=1\Bsetminus a$;  then $\langle a_i\rangle_{i\in J}$,
$(a,1\Bsetminus a)$ are both partitions of unity, so

\Centerline{$\nu(1\Bsetminus a)+\nu a=\nu 1=\sum_{i\in J}\nu a_i
=\nu(1\Bsetminus a)+\sum_{i\in I}\nu a_i$,}

\noindent and $\nu a=\sum_{i\in I}\nu a_i$.

\medskip

{\bf (c)(iii)$\Rightarrow$(i)} Suppose that $\nu$ satisfies (iii).
Then $\nu$ is additive.

\medskip

\quad\grheada\ $\nu$ is bounded.   \Prf\ Let $\sequencen{a_n}$ be a
disjoint sequence in $\frak A$.   Applying Zorn's Lemma to the set $\Cal
C$ of all disjoint families $C\subseteq\frak A$ including $\{a_n:n\in
\Bbb N\}$, we find a partition of unity $C\supseteq\{a_n:n\in\Bbb N\}$.
Now $\sum_{c\in C}\nu c$ is defined in $\Bbb R$, so $\sup_{n\in\Bbb
N}|\nu a_n|\le\sup_{c\in C}|\nu c|$ is finite.   By 326D, $\nu$ is
bounded.\ \Qed

\medskip

\quad\grheadb\ Define $\nu^+$ from $\nu$ as in 326D.  Then $\nu^+$
satisfies the same condition as $\nu$.   \Prf\ Let $\langle
a_i\rangle_{i\in I}$ be a disjoint family in $\frak A$ with supremum
$a$.   Then
for any $b\subseteq a$, we have $b=\sup_{i\in I}b\Bcap a_i$, so

\Centerline{$\nu b=\sum_{i\in I}\nu(b\Bcap a_i)
\le\sum_{i\in I}\nu^+a_i$.}

\noindent Thus $\nu^+a\le\sum_{i\in I}\nu^+a_i$.   But of course

$$\eqalign{\sum_{i\in I}\nu^+a_i
&=\sup\{\sum_{i\in J}\nu^+a_i:J\subseteq I\text{ is finite}\}\cr
&=\sup\{\nu^+(\sup_{i\in J}a_i):J\subseteq I\text{ is finite}\}
\le\nu^+a,\cr}$$

\noindent so $\nu^+a=\sum_{i\in I}\nu^+a_i$.\ \Qed

\medskip

\quad\grheadc\ It follows that $\nu^+$ is completely additive.   \Prf\
If $A$ is a non-empty downwards-directed set with infimum $0$, then
$B=\{b:\,\Exists a\in A,\,b\Bcap a=0\}$ is order-dense in $\frak A$, so
there is a
partition of unity $\langle b_i\rangle_{i\in I}$ lying in $B$ (313K).
Now if $J\subseteq I$ is finite, there is an $a\in A$ such that
$a\Bcap\sup_{i\in J}b_i=0$ (because $A$ is downwards-directed), and

\Centerline{$\nu^+a+\sum_{i\in J}\nu^+b_i\le\nu^+1$.}

\noindent Since $\nu^+1=\sup_{J\subseteq I\text{ is finite}}\sum_{i\in
J}\nu^+b_i$, $\inf_{a\in A}\nu^+a=0$.   As $A$ is arbitrary, $\nu^+$ is
completely additive.\ \Qed

\medskip

\quad\grheadd\ Now consider $\nu^-=\nu^+-\nu$.   Of course

\Centerline{$\nu^-a
=\nu^+a-\nu a
=\sum_{i\in I}\nu^+a_i-\sum_{i\in I}\nu a_i
=\sum_{i\in I}\nu^-a_i$}

\noindent whenever $\langle a_i\rangle_{i\in I}$ is a disjoint family in
$\frak A$ with supremum $a$.   Because $\nu^-$ is non-negative, the
argument of ($\gamma$)
shows that $\nu^-=(\nu^-)^+$ is completely additive.   So
$\nu=\nu^+-\nu^-$ is completely additive, as required.
}%end of proof of 326R

\leader{326S}{}\cmmnt{ For completely additive functionals, we have a
useful refinement of the Hahn decomposition.   I give it in a form
adapted to the applications I have in mind.

\medskip

\noindent}{\bf Proposition}\dvAformerly{3{}26O}
Let $\frak A$ be a Dedekind
$\sigma$-complete Boolean algebra and $\nu:\frak A\to\Bbb R$ a
completely additive functional.   Then there is a unique element of
$\frak A$, which I will denote $\Bvalue{\nu>0}$,\cmmnt{ `the region
where $\nu>0$',} such that $\nu a>0$ whenever
$0\ne a\Bsubseteq\Bvalue{\nu>0}$, while $\nu a\le 0$ whenever
$a\Bcap\Bvalue{\nu>0}=0$.

\proof{ Set

\Centerline{$C_1=\{c:c\in\frak A\setminus\{0\},\,\nu a>0$ whenever
$0\ne a\Bsubseteq c\}$,}

\Centerline{$C_2=\{c:c\in\frak A,\,\nu a\le 0$ whenever $a\Bsubseteq c\}$.}

\noindent Then $C_1\cup C_2$ is order-dense in $\frak A$.   \Prf\ There
is a $c_0\in\frak A$ such that $\nu a\ge 0$ for every $a\Bsubseteq c_0$ and
$\nu a\le 0$ whenever $a\Bcap c_0=0$ (326M).   Given
$b\in\frak A\setminus\{0\}$, then $b\Bsetminus c_0\in C_2$, so if $b\Bsetminus c_0\ne 0$ we can stop.   Otherwise, $b\Bsubseteq c_0$.   If $b\in C_1$
we can stop.   Otherwise, there is a non-zero $c\Bsubseteq b$ such that
$\nu c\le 0$;  but in this case $\nu a\ge 0$ and $\nu(c\Bsetminus a)\ge 0$
so $\nu a=0$ for every $a\Bsubseteq c$, and $c\in C_2$.\ \Qed

There is therefore a partition of unity $D\subseteq C_1\cup C_2$.   Now
$D\cap C_1$ is countable.   \Prf\ If $d\in D\cap C_1$, $\nu d>0$.   Also

\Centerline{$\#(\{d:d\in D,\,\nu d\ge 2^{-n}\})\le 2^n\sup_{a\in\frak
A}\nu a$}

\noindent is finite for each $n$, so $D\cap C_1$ is the union of a
sequence of finite sets, and is countable.\ \Qed

Accordingly $D\cap C_1$ has a supremum $e$.   If $0\ne a\Bsubseteq e$
then

\Centerline{$\nu a=\sum_{c\in D}\nu(a\Bcap c)=\sum_{c\in D\cap
C_1}\nu(a\Bcap c)\ge 0$}

\noindent by 326R.   Also there must be some $c\in D\cap C_1$ such that
$a\Bcap c\ne 0$, in which case $\nu(a\Bcap c)>0$, so that $\nu a>0$.
If $a\Bcap e=0$, then

\Centerline{$\nu a=\sum_{c\in D}\nu(a\Bcap c)=\sum_{c\in D\cap
C_2}\nu(a\Bcap c)\le 0$.}

Thus $e$ has the properties demanded of $\Bvalue{\nu>0}$.   To see that
$e$ is unique, we need observe only that if $e'$ has the same properties
then $\nu(e\Bsetminus e')\le 0$ (because $(e\Bsetminus e')\Bcap e'=0$),
so $e\Bsetminus e'=0$ (because $e\Bsetminus e'\Bsubseteq e$).
Similarly, $e'\Bsetminus e=0$ and $e=e'$.   Thus we may properly denote
$e$ by the formula $\Bvalue{\nu>0}$.
}%end of proof of 326S

\leader{326T}{Corollary} Let $\frak A$ be a Dedekind $\sigma$-complete
Boolean algebra and $\mu$, $\nu$ two completely additive functionals on
$\frak A$.   Then there is a unique element of $\frak A$, which I will
denote $\Bvalue{\mu>\nu}$,\cmmnt{ `the region where $\mu>\nu$',} such
that

\Centerline{$\mu a>\nu a$ whenever $0\ne a\Bsubseteq\Bvalue{\mu>\nu}$,}

\Centerline{$\mu a\le\nu a$ whenever $a\Bcap\Bvalue{\mu>\nu}=0$.}

\proof{ Apply 326S to the functional $\mu-\nu$, and set
$\Bvalue{\mu>\nu}=\Bvalue{\mu-\nu>0}$.
}%end of proof of 326T

\exercises{
\leader{326X}{Basic exercises (a)}
%\spheader 326Xa
Let $\frak A$ be a Boolean algebra
and $\nu:\frak A\to\Bbb R$ a finitely additive functional.  Show that
(i) $\nu(a\Bcup b)=\nu a+\nu b-\nu(a\Bcap b)$ (ii)
$\nu(a\Bcup b\Bcup c)=\nu a+\nu
b+\nu c-\nu(a\Bcap b)-\nu(a\Bcap c)-\nu(b\Bcap c)+\nu(a\Bcap b\Bcap c)$
for all $a$, $b$, $c\in\frak A$.
Generalize
these results to longer sequences in $\frak A$.
%326B

\spheader 326Xb\dvAformerly{3{}26Ya}
Let $\frak A$ be a Boolean algebra.
(i) Show that a finitely additive functional $\nu$ is properly atomless iff
there is a properly atomless additive functional $\nu'$ such
that $|\nu a|\le \nu'a$ for every $a\in\frak A$.   (ii) Show that
a non-negative finitely additive functional $\nu$ on $\frak A$
is properly atomless iff whenever $\nuprime$ is a
non-zero finitely additive functional such that
$0\le\nuprime a\le\nu a$ for every $a\in\frak A$ there
is an $a\in\frak A$ such that $\nuprime a$ and $\nuprime(1\Bsetminus a)$
are both non-zero.
%326F

\spheader 326Xc\dvAnew{2011}
(i) Suppose that $\frak A$ is a Dedekind $\sigma$-complete
Boolean algebra
and $\nu:\frak A\to\Bbb R$ is countably additive.
Show that $\Cal I=\{a:\nu b=0$ for every $b\Bsubseteq a\}$ is an ideal of
$\frak A$.   Show that the following
are equiveridical:  ($\alpha$) $\nu$ is
properly atomless;  ($\beta$) whenever $\nu a\ne 0$ there is a
$b\Bsubseteq a$ such that $\nu b\notin\{0,\nu a\}$;  ($\gamma$) the
quotient algebra $\frak A/\Cal I$ is atomless.
(ii) Find an atomless Dedekind complete Boolean algebra $\frak A$ and a
finitely additive
$\nu:\frak A\to[0,1]$ such that $\nu a>0$ for every non-zero $a\in\frak A$
but $\nu$ is not properly atomless.
%326J

\spheader 326Xd Let $\frak A$ be a Boolean algebra and
$\nu:\frak A\to \Bbb R$ a finitely additive functional.
Show that the following are equiveridical:
(i) $\nu$ is countably additive;
(ii) $\lim_{n\to\infty}\nu a_n=\nu a$ whenever $\sequencen{a_n}$
is a non-decreasing sequence in $\frak A$ with supremum $a$.
%326K

\spheader 326Xe Let $\frak A$ be a Dedekind
$\sigma$-complete Boolean algebra and $\nu:\frak A\to \Bbb R$ a
finitely additive functional.
Show that the following are equiveridical:
(i) $\nu$ is countably additive;
(ii) $\lim_{n\to\infty}\nu a_n=0$ whenever $\sequencen{a_n}$ is a
sequence in $\frak A$ and $\inf_{n\in\Bbb N}\sup_{m\ge
n}a_m=0$;
(iii) $\lim_{n\to\infty}\nu a_n=\nu a$ whenever $\sequencen{a_n}$
is a sequence in $\frak A$ and
$a=\inf_{n\in\Bbb N}\sup_{m\ge n}a_m=\sup_{n\in\Bbb N}\inf_{m\ge n}a_m$.
\noindent\Hint{for (i)$\Rightarrow$(iii), consider
non-negative $\nu$ first.}
%326K

\spheader 326Xf Let $X$ be an uncountable set, and $J$ an
infinite subset of $X$.   Let $\frak A$ be the
finite-cofinite algebra of $X$ (316Yl), and for $a\in A$ set
$\nu a=\#(a\cap J)$ if $a$ is finite,
$-\#(J\setminus a)$ if $a$ is cofinite.
Show that $\nu$ is countably additive and unbounded.
%326M

\sqheader 326Xg Let $\frak A$ be the algebra of subsets of $[0,1]$
generated by the family of (closed) intervals.   Show that there is a
unique additive functional $\nu:\frak A\to\Bbb R$ such that
$\nu[\alpha,\beta]=\beta-\alpha$ whenever $0\le \alpha\le\beta\le 1$.
Show that $\nu$ is  countably additive but not completely additive.
%326N

\spheader 326Xh(i) Let $(X,\Sigma,\mu)$ be any atomless probability
space.   Show that $\mu:\Sigma\to\Bbb R$ is a countably additive
functional which is not completely additive.   (ii) Let $X$ be any
uncountable set and $\mu$ the countable-cocountable measure on $X$
(211R).   Show that $\mu$ is countably additive but not completely
additive.
%326N

\spheader 326Xi\dvAnew{2011}
Let $\frak A$ be an atomless Boolean algebra.   Show that
every completely additive functional on $\frak A$ is properly atomless.
%326O

\spheader 326Xj Let $\frak A$ be a Boolean algebra and
$\nu:\frak A\to\Bbb R$ a function.   (i) Show that $\nu$ is finitely
additive iff $\sum_{i\in I}\nu a_i=\nu 1$ for every finite partition of
unity $\langle a_i\rangle_{i\in I}$.   (ii) Show that $\nu$ is countably
additive iff $\sum_{i\in I}\nu a_i=\nu 1$ for every countable partition
of unity $\langle a_i\rangle_{i\in I}$.
%326R

\spheader 326Xk Show that 326S can fail if $\nu$ is only countably
additive, rather than completely additive.   \Hint{326Xh.}
%326S

\spheader 326Xl Let $\frak A$ be a Boolean algebra and $\nu$ a finitely
additive real-valued functional on $\frak A$.   Let us say that
$a\in\frak A$ is
a {\bf support} of $\nu$ if ($\alpha$) $\nu b=0$ whenever $b\Bcap a=0$
($\beta$) for every non-zero $b\Bsubseteq a$ there is a $c\Bsubseteq b$
such that $\nu c\ne 0$.   (i) Check that $\nu$ can have at most one
support.   (ii) Show that if $a$ is a support for $\nu$ and $\nu$ is
bounded, then the principal ideal $\frak A_a$ generated by $a$ is ccc.
(iii) Show that if $\frak A$ is Dedekind $\sigma$-complete and $\nu$ is
countably additive, then $\nu$ is completely additive iff it has a
support, and that in the language of 326S this is
$\Bvalue{\nu>0}\Bcup\Bvalue{-\nu>0}$.   (iv) Taking $J=X$ in 326Xf, show
that $X$ is the support of the functional $\nu$ there.
%326S
}

\exercises{
\leader{326Y}{Further exercises (a)}
%\spheader 326Ya
Show that there is a finitely additive functional
$\nu:\Cal P\Bbb N\to\Bbb R$ such that $\nu\{n\}=1$ for every
$n\in\Bbb N$, so that $\nu$ is not bounded.
({\it Hint\/}:  Use Zorn's Lemma to
construct a maximal linearly independent subset of $\ell^{\infty}$
including $\{\chi\{n\}:n\in\Bbb N\}$, and hence to construct a linear
map $f:\ell^{\infty}\to\Bbb R$ such that $f(\chi\{n\})=1$ for every
$n$.)
%326D Hamel basis would be useful?

\spheader 326Yb Let $\frak A$ be any infinite Boolean algebra.
Show that there is an unbounded finitely additive functional
$\nu:\frak A\to\Bbb R$.   ({\it Hint\/}: let $\sequencen{t_n}$ be a
sequence of distinct points in the Stone space of $\frak A$, and set
$\nu a=\nuprime\{n:t_n\in\widehat a\}$ for a suitable $\nuprime$.)
%326D, 326Ya

\spheader 326Yc Let $\frak A$ be a Boolean algebra, and give
$\Bbb R^{\frak A}$ its product topology.   Show that the space of
finitely additive functionals on $\frak A$ is a closed subset of
$\BbbR^{\frak A}$, but that the space of bounded finitely additive
functionals is closed only when $\frak A$ is finite.
%326D

\spheader 326Yd Let $\frak A$ be a Boolean algebra, and $M$ the
linear space of all bounded finitely additive real-valued functionals on
$\frak A$.   For $\nu$, $\nuprime\in M$ say that $\nu\le\nuprime$ if
$\nu a\le\nuprime a$ for every $a\in\frak A$.   Show that

\quad (i) $\nu^+$, as defined in the proof of 326D, is just
$\sup\{0,\nu\}$ in $M$;

\quad(ii) $M$ is a Dedekind complete Riesz space (241E-241F, 353G);

\quad(iii) for $\nu$, $\nuprime\in M$, $|\nu|=\nu\vee(-\nu)$,
$\nu\vee\nuprime$ and
$\nu\wedge\nuprime$ are given by the formulae

\Centerline{$|\nu|(a)=\sup_{b\Bsubseteq a}\nu b-\nu(a\Bsetminus b)$,
\quad$(\nu\vee\nuprime)(a)=\sup_{b\Bsubseteq a}\nu b+\nuprime(a\Bsetminus b)$,}

\Centerline{$(\nu\wedge\nuprime)(a)
=\inf_{b\Bsubseteq a}\nu b+\nuprime(a\Bsetminus b)$;}

\quad(iv) for any non-empty $A\subseteq M$, $A$ is bounded above in $M$
iff

\Centerline{$\sup\{\sum_{i=0}^n\nu_ia_i:\nu_i\in A$ for each $i\le n$,
$\langle a_i\rangle_{i\le n}$ is disjoint$\}$}

\noindent is finite, and then $\sup A$ is defined by the formula

\Centerline{$(\sup A)(a)=\sup\{\sum_{i=0}^n\nu_ia_i:\nu_i\in A$ for each
$i\le n$, $\langle a_i\rangle_{i\le n}$ is disjoint,
$\sup_{i\le n}a_i=a\}$}

\noindent for every $a\in\frak A$;

\quad(v) setting $\|\nu\|=|\nu|(1)$, $\|\,\|$ is an order-continuous
norm (definition:  354Dc) on $M$ under which $M$ is a Banach lattice.
%326D

\spheader 326Ye Let $\frak A$ be a Boolean algebra.   A functional
$\nu:\frak A\to\Bbb C$ is {\bf finitely additive} if its real and
imaginary parts are.
Show that the space of bounded finitely additive functionals from
$\frak A$ to $\Bbb C$ is a Banach space under the total variation norm
$\|\nu\|=\sup\{\sum_{i=0}^n|\nu a_i|:\langle a_i\rangle_{i\le n}$
is a partition of unity in $\frak A\}$.
%326D, 326Yd

\spheader 326Yf Let $\frak A$ and $\frak B$ be Boolean algebras and
$\mu$, $\nu$ finitely additive functionals on $\frak A$, $\frak B$
respectively.   Show that there is a unique finitely
additive functional $\lambda$ on the free product
$\frak A\otimes\frak B\to\Bbb R$ such that
$\lambda(a\otimes b)=\mu a\cdot\nu b$ for all $a\in\frak A$,
$b\in\frak B$.
%326E

\spheader 326Yg Let $\langle\frak A_i\rangle_{i\in I}$ be a family of
Boolean algebras, with free product
$(\bigotimes_{i\in I}\frak A_i,\langle\varepsilon_i\rangle_{i\in I})$,
and for each $i\in I$ let
$\nu_i$ be a finitely additive functional on $\frak A_i$ such that
$\nu_i1=1$.   Show that there is a unique finitely additive functional
$\nu:\bigotimes_{i\in I}\frak A_i\to\Bbb R$ such that
$\nu(\inf_{i\in J}\varepsilon_i(a_i))=\prod_{i\in J}\nu_ia_i$ whenever
$J\subseteq I$ is
non-empty and finite and $a_i\in\frak A_i$ for each $i\in J$.
%326E

\spheader 326Yh\dvAformerly{3{}26Ya}
Let $\frak A$ be a Dedekind $\sigma$-complete Boolean
algebra and $\nu:\frak A\to\coint{0,\infty}$ a countably additive
functional.   Show that $\nu$ is
properly atomless iff whenever $a\in\frak A$ and
$\nu a\ne 0$ there is a $b\Bsubseteq a$ such that $0<\nu b<\nu a$.
%326F out of order query

\spheader 326Yi\dvAnew{2011}
Let $\frak A$ be a Dedekind $\sigma$-complete Boolean
algebra and $\nu:\frak A\to\Bbb R$ a countably additive functional.   Show
that $\nu[\frak A]$ is a compact subset of $\Bbb R$.
%taking  a_n  such that  \nu a_n\to\alpha , subsequence deciding each atom,
%reduce to atomless case in which we have a convex set and need only look
%at the endpts
%326H 326M

\spheader 326Yj\dvAnew{2011}
Let $\frak G$ be the regular open algebra of $\Bbb R$
(314P).   Find a properly atomless finitely additive
$\nu:\frak G\to\Bbb R$ such that $\nu[\frak G]$.
%take \nu^+ , \nu^-  corresponding to mutually singular measures with
% support \Bbb R
%326Yi 326M

\spheader 326Yk\dvAnew{2011} ({\smc Halmos 48})
Let $\frak A$ be a Dedekind $\sigma$-complete Boolean
algebra and $r\ge 1$ an integer.   (i) Let $C\subseteq\BbbR^r$ be a
non-empty bounded convex set, and for $z\in\BbbR^r$ set
$H_z=\{x:\varinnerprod{x}{z}=\sup_{y\in C}\varinnerprod{y}{z}\}$.
Suppose that $H_z\cap\overline{C}\subseteq C$ for every
$z\in\BbbR^r\setminus\{0\}$.   Show that $C$ is closed.
(ii) Suppose that $\nu:\frak A\to\BbbR^r$ is
countably additive in the sense that all its coordinates are countably
additive functionals.   Show that $\nu[\frak A]$ is compact.
% Dinculeanu 67?  Aliprantis & Border 13.33 atomless case only
%326Yi 326H 326M

\spheader 326Yl Let $\frak A$ be a Boolean algebra, and give it
the topology $\frak T_{\sigma}$ for which the closed sets are the
sequentially order-closed sets.   Show that a finitely additive
functional $\nu:\frak A\to\Bbb R$ is countably additive iff it is
continuous for $\frak T_{\sigma}$.
%326K

\spheader 326Ym Let $\frak A$ be a Boolean algebra, and
$M_{\sigma}$ the set of all bounded countably additive real-valued
functionals on $\frak A$.   Show that $M_{\sigma}$ is a closed
and order-closed linear subspace of the normed space $M$ of all additive
functionals on $\frak A$ (326Yd), and that $|\nu|\in M_{\sigma}$
whenever $\nu\in M_{\sigma}$.
%326L

\spheader 326Yn Let $\frak A$ be a Boolean algebra and $\nu$ a
non-negative finitely additive functional on $\frak A$.   Set

\Centerline{$\nu_{\sigma}a
=\inf\{\sup_{n\in\Bbb N}\nu a_n:\sequencen{a_n}$ is a non-decreasing
sequence with supremum $a\}$}

\noindent for every $a\in\frak A$.   Show that $\nu_{\sigma}$ is
countably additive, and is $\sup\{\nuprime:\nuprime\le\nu$ is countably
additive$\}$.
%326L

\spheader 326Yo\dvAformerly{3{}26Yo} Let $\frak A$ be a Dedekind
$\sigma$-complete Boolean algebra and $\sequencen{\nu_n}$ a sequence of
countably additive real-valued functionals on $\frak A$ such that
$\nu a=\lim_{n\to\infty}\nu_na$ is defined in $\Bbb R$ for every
$a\in\frak A$.   Show that $\nu$ is countably additive.   ({\it Hint\/}:
use arguments from part (a) of the proof of 247C to see that
$\lim_{n\to\infty}\sup_{k\in\Bbb N}|\nu_ka_n|=0$ for every disjoint
sequence $\sequencen{a_n}$ in $\frak A$, and therefore that
$\lim_{n\to\infty}\sup_{k\in\Bbb N}|\nu_ka_n|=0$ whenever
$\sequencen{a_n}$ is a non-increasing sequence with infimum $0$.)
%326M

\spheader 326Yp Let $\frak A$ be a Boolean algebra, and
$M_{\tau}$ the set of all completely additive real-valued functionals on
$\frak A$.   Show that $M_{\tau}$ is a closed and order-closed linear
subspace of the normed space $M$ of all additive functionals, and that
$|\nu|\in M_{\tau}$ whenever $\nu\in M_{\tau}$.
%326O

\spheader 326Yq Let $\frak A$ be a Boolean algebra and $\nu$ a
non-negative finitely additive functional on $\frak A$.   Set

\Centerline{$\nu_{\tau}b=\inf\{\sup_{a\in A}\nu a:A$ is a non-empty
upwards-directed  set with supremum $b\}$}

\noindent for every $b\in\frak A$.   Show that $\nu_{\tau}$ is
completely additive, and is $\sup\{\nuprime:\nuprime\le\nu$ is
completely additive$\}$.
%326O

\spheader 326Yr Let $\frak A$ be a Boolean algebra, and give it
the topology $\frak T$ for which the closed sets are the order-closed
sets (313Xb).   Show that a finitely additive functional
$\nu:\frak A\to\Bbb R$ is completely additive iff it is continuous for
$\frak T$.
%326O

\spheader 326Ys Let $X$ be a set, $\Sigma$ any $\sigma$-algebra
of subsets of $X$, and $\nu:\Sigma\to\Bbb R$ a functional.   Show that
$\nu$ is completely additive iff there are sequences $\sequencen{x_n}$,
$\sequencen{\alpha_n}$ such that $\sum_{n=0}^{\infty}|\alpha_n|<\infty$
and $\nu E=\sum_{n=0}^{\infty}\alpha_n\chi E(x_n)$ for every
$E\in\Sigma$.
%326O

}%end of exercises

\cmmnt{
\Notesheader{326}
I have not mentioned the phrase `measure algebra' anywhere in this
section, and in principle this material could have been part of Chapter
31;  but
countably additive functionals are kissing cousins of measures, and most
of the ideas here surely belong to `measure theory' rather than to
`Boolean algebra', in so far as such divisions are meaningful at all.
I have given as much as possible of the theory in a
general form because the simplifications which are possible when we look
only at measure algebras are seriously confusing if they are allowed too
much prominence.   In particular, it is important to understand that the
principal properties of
completely additive functionals do not depend on Dedekind completeness
of the algebra, provided we take care over the definitions.
Similarly, the definition of `countably additive' functional for
algebras which are not Dedekind $\sigma$-complete needs a moment's
attention to the phrase `and $\sup_{n\in\Bbb N}a_n$ is defined in
$\frak A$'.    It can happen that a functional is countably additive
mostly because there are too few such sequences (326Xf).

The formulations I have chosen as principal definitions (326A, 326I,
326N) are those which I find closest to my own intuitions of the
concepts, but you may feel that 326K(i),
326Xe(iii) and 326R, or 326Yl and 326Yr, provide useful alternative
patterns.   The point is that countable additivity corresponds to
sequential order-continuity
(326Jb, 326Jc, 326Jf), while  complete additivity corresponds to
order-continuity (326Oc, 326Of);  the difficulty is that we must
consider functionals which are not order-preserving, so that the simple
definitions in 313H cannot be applied directly.   It is fair to say that
all the additive functionals $\nu$ we need to understand are bounded,
and therefore may be studied in terms of their positive and negative
parts $\nu^+$, $\nu^-$, which are order-preserving (326Bf);  but many of
the most important applications of these ideas depend precisely on using
facts about $\nu$ to deduce facts about $\nu^+$ and $\nu^-$.

It is in 326D that we seem to start getting more out of the theory than
we have put in.   The ideas here have vast ramifications.   What it
amounts to is that we can discover much more than we might expect by
looking at disjoint sequences.   To begin with, the conditions here lead
directly to 326M and 326Q:  every completely additive functional is
bounded, and every countably additive functional on a Dedekind
$\sigma$-complete Boolean algebra is bounded.   (But note 326Ya-326Yb.)

I have expressed 326H in terms of an additive function from a Boolean
algebra to a finite-dimensional space (it is already non-trivial in the
two-dimensional case, which would correspond to an additive complex-valued
functional, as in 326Ye).
It is usually regarded as a theorem about countably additive
functions, or `vector measures' (see 394O below), but rather remarkably we
do not in fact need countable additivity.   Of course it can also be
regarded as a kind of ham-sandwich theorem for measures;  we can
simultaneously bisect an element of a Dedekind $\sigma$-complete Boolean
algebra with respect to finitely many additive functionals.   If you like,
the dimensionality requirement of the ordinary ham-sandwich theorems of
topology is met by the requirement of atomlessness here.   A companion
result, also due to Liapounoff,
which requires countable additivity but allows atoms, is in 362Yx.

Naturally enough, the theory of countably additive functionals on
general Boolean algebras corresponds closely to the special case of
countably additive functionals on $\sigma$-algebras of sets, already
treated in \S\S231-232 for the sake of the Radon-Nikod\'ym theorem.
This should make 326I-326M very straightforward.
When we come to completely additive functionals, however, there is room
for many surprises.   The natural map from a $\sigma$-algebra of
measurable sets to the corresponding measure algebra is sequentially
order-continuous but rarely order-continuous, so that there can be
completely additive functionals on the measure algebra which do not
correspond to completely additive functionals on the $\sigma$-algebra.
Indeed there are very few completely additive functionals on
$\sigma$-algebras of sets (326Ys).   Of course these surprises can arise
only when there is a difference between completely additive and
countably additive functionals, that is, when the algebra involved is
not ccc (326P).   But I think that neither 326Q nor 326R is obvious.

I find myself generally using the phrase `countably additive' in
preference to `completely additive' in the context of ccc algebras,
where there is no difference between them.   This is an attempt at
user-friendliness;  the phrase `countably additive' is the commoner one
in ordinary use.   But I must say that my personal inclination is to the
other side.   The reason why so many theorems apply to countably
additive functionals in these contexts is just that they are
completely additive.

I have given two proofs of 326M.   I certainly assume
that if you have got this far you are acquainted with the
Radon-Nikod\'ym theorem and the associated basic facts about countably
additive functionals on $\sigma$-algebras of sets;  so that the `first
proof' should be easy and natural.   On the other hand, there are purist
objections on two fronts.   First, it relies on the Stone
representation, which involves a much stronger form of the axiom of
choice than is actually necessary.   Second, the classical Hahn
decomposition in 231E is evidently a special case of 326M, and if we
need both (as we certainly do) then one expects the ideas to stand out
more clearly if they are applied directly to the general case.   In fact
the two versions of the argument are so nearly identical that (as you
will observe, if you have Volume 2 to hand) they can share nearly every
word.   You can take the `second proof', therefore, as a worked example
in the translation of ideas from the context of $\sigma$-algebras of
sets to the context of Dedekind $\sigma$-complete Boolean algebras.
What makes it possible is the fact that the only limit operations
referred to involve countable families.

Arguments not involving limit operations can generally, of course, be
applied to all Boolean algebras;  I have lifted some exercises
(326Yd, 326Yn) from \S231 %231Yh, 231Yf
to give you some practice in such generalizations.

Almost any non-trivial measure provides an example of a countably
additive functional on a Dedekind $\sigma$-complete algebra which is not
completely additive (326Xh).   The question of whether such a functional
can exist on a Dedekind complete algebra is the `Banach-Ulam
problem', to which I will return in 363S.

In this section I have looked only at questions which can be adequately
treated in terms of the underlying algebras $\frak A$, without using any
auxiliary structure.   To go much farther we shall need to study the
`function spaces' $S(\frak A)$ and $L^{\infty}(\frak A)$ of Chapter 36.
In particular, the ideas of 326Ya, 326Yd-326Ye and
326Ym-326Yq %326Ym 326Yn 326Yo 326Yp 326Yq
will make better sense when redeveloped in \S362.
}%end of notes

\discrpage

