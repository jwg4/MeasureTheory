\frfilename{mt526.tex}
\versiondate{24.1.14}
\copyrightdate{2009}

\def\chaptername{Cardinal functions of measure theory}
\def\sectionname{Asymptotic density zero}

\newsection{526}

In \S491, I devoted some paragraphs to the ideal $\Cal Z$ of subsets
of $\Bbb N$ with asymptotic density zero, as part of an investigation
into equidistributed sequences
in topological measure spaces.   Here I return to $\Cal Z$ to examine
its place in the Tukey ordering of partially ordered sets.   We find
that it lies strictly between
$\NN$ and $\ell^1$ (526B, 526J, 526L)
but in some sense is closer to $\ell^1$
(526Ga).   On the way, I mention the ideal $\CalNwd$
of nowhere dense subsets of $\NN$
(526H-526L) %526H 526I 526J 526K 526L
and ideals of sets with negligible closures
(526I-526M). %526I 526J 526K 526L 526M

\leader{526A}{Proposition} For $I\subseteq\Bbb N$, set
$\nu I=\sup_{n\ge 1}\bover1n\#(I\cap n)$.

(a) $\nu$ is a strictly positive submeasure\cmmnt{ (definition:
392A)} on $\Cal P\Bbb N$.   We have a
metric $\rho$ on $\Cal P\Bbb N$
defined by setting $\rho(I,J)=\nu(I\symmdiff J)$ for all $I$,
$J\subseteq\Bbb N$, under which the Boolean operations $\cup$, $\cap$,
$\symmdiff$ and $\setminus$ and upper asymptotic density
$d^*:\Cal P\Bbb N\to[0,1]$ are uniformly continuous and
$\Cal P\Bbb N$ is complete.

(b) $\Cal Z$ is a separable closed subset of $\Cal P\Bbb N$.

(c) If $\Cal I\subseteq\Cal Z$ is such that $\sum_{I\in\Cal I}\nu I$ is
finite, then $\bigcup\Cal I\in\Cal Z$.

(d) With the subspace topology, $(\Cal Z,\subseteq)$
is a metrizably compactly based directed set\cmmnt{ (definition:  513K)}.

\proof{{\bf (a)} It is elementary to check that $\nu$ is a strictly
positive submeasure.   By 392H,
$\rho$ is a metric under which the Boolean operations are
uniformly continuous.   Since

\Centerline{$|d^*(I)-d^*(J)|\le d^*(I\symmdiff J)\le\nu(I\symmdiff J)$}

\noindent for all $I$, $J\subseteq\Bbb N$, $d^*$ is uniformly
continuous.
Let $\sequence{j}{I_j}$ be a sequence in $\Cal P\Bbb N$ such that
$\rho(I_j,I_{j+1})\le 2^{-j}$ for every $j\in\Bbb N$.   Set
$I=\bigcap_{m\in\Bbb N}\bigcup_{j\ge m}I_j$.   For any $m\in\Bbb N$ and
$n\ge 1$,

\Centerline{$\Bover1n\#(n\cap(I_m\symmdiff I))
\le\Bover1n\sum_{j=m}^{\infty}\#(n\cap(I_j\symmdiff I_{j+1}))
\le\sum_{j=m}^{\infty}\rho(I_j,I_{j+1})\le 2^{-m+1}$,}

\noindent so $\rho(I_m,I)\le 2^{-m+1}$.   Thus $\sequence{j}{I_j}$ is
convergent to $I$;  as $\sequence{j}{I_j}$ is arbitrary, $\Cal P\Bbb N$
is complete (cf.\ 2A4E).

\medskip

{\bf (b)} $\Cal Z$ is a closed subset of $\Cal P\Bbb N$.   \Prf\ If $I$
belongs to the closure $\overline{\Cal Z}$ of $\Cal Z$, and
$\epsilon>0$, let $J\in\Cal Z$ be such that
$\rho(I,J)\le\bover12\epsilon$, and let $m\ge 1$ be such that
$\bover1n\#(J\cap n)\le\bover12\epsilon$ for every $n\ge m$;  then
$\bover1n\#(I\cap n)\le\epsilon$ for every $n\ge m$.   As $\epsilon$ is
arbitrary, $I\in\Cal Z$;  as $I$ is arbitrary, $\Cal Z$ is closed.\ \Qed

$\Cal Z$ is separable because $[\Bbb N]^{<\omega}$ is a
countable dense set.   (If $I\in\Cal Z$ and $n\in\Bbb N$,
$\rho(a,a\cap n)\le\sup_{m>n}\bover1m\#(m\cap I)$.)

\medskip

{\bf (c)} Let $\epsilon>0$.   Then there is a finite
$\Cal I_0\subseteq\Cal I$ such that
$\sum_{I\in\Cal I\setminus\Cal I_0}\nu I\le\epsilon$.   Set
$J=\bigcup\Cal I$,
$J_0=\bigcup\Cal I_0$;  then $J_0\in\Cal Z$ so there is an
$n_0\in\Bbb N$ such
that $\#(J_0\cap n)\le n\epsilon$ for every $n\ge n_0$.
If $n\ge n_0$, then

\Centerline{$\#(J\cap n)
\le\#(J_0\cap n)+\sum_{I\in\Cal I\setminus\Cal I_0}\#(I\cap n)
\le n\epsilon+\sum_{I\in\Cal I\setminus\Cal I_0}n\nu I
\le 2n\epsilon$.}

\noindent As $\epsilon$ is arbitrary, $J\in\Cal Z$.

\medskip

{\bf (d)} $\Cal Z$ is closed under $\cup$, so is a directed set under
$\subseteq$, and $\cup:\Cal Z\times\Cal Z\to\Cal Z$ is continuous.   If
$a\in\Cal Z$, then on $\{b:b\subseteq a\}$ the topology $\frak T_{\rho}$
induced by $\rho$ agrees with the usual compact Hausdorff topology
$\frak S$ of $\Cal P\Bbb N\cong\{0,1\}^{\Bbb N}$.   \Prf\ If $n\in\Bbb N$
and $\rho(b,c)<\bover1{n+1}$, then $b\cap n=c\cap n$;  so $\frak T_{\rho}$
is finer than $\frak S$ on $\Cal P\Bbb N$.   If $\epsilon>0$, there is an
$m\in\Bbb N$ such that $\#(a\cap n)\le n\epsilon$ whenever $n\ge m$;  now
$\rho(b,c)\le\epsilon$ whenever $b$, $c\subseteq a$ and $b\cap m=c\cap m$.
So $\frak S$ is finer than $\frak T_{\rho}$ on $\{b:b\subseteq a\}$.\ \QeD\
Since $\{b:b\subseteq a\}$ is $\frak S$-compact, it is
$\frak T_{\rho}$-compact.

Now suppose that $\sequencen{a_n}$ is a sequence in $\Cal Z$ with
$\frak T_{\rho}$-limit $a$.   Then it has a subsequence
$\sequence{k}{a_{n_k}}$ such that $\rho(a,a_{n_k})\le 2^{-k}$ for every
$k$.   Set $b=\bigcup_{k\in\Bbb N}a_{n_k}$.   Then, given $\epsilon>0$,
let $r$, $m\in\Bbb N$ be such that $2^{-r}\le\epsilon$ and
$\#(n\cap(a\cup\bigcup_{k\le r}a_{n_k}))\le n\epsilon$ for every $n\ge m$;
then

$$\eqalign{\#(n\cap b)
&\le\#(n\cap(a\cup\bigcup_{k\le r}a_{n_k}))
  +\sum_{k=r+1}^{\infty}\#(n\cap a_{n_k}\setminus a)\cr
&\le n\epsilon+\sum_{k=r+1}^{\infty}2^{-k}n
\le 2n\epsilon\cr}$$

\noindent for every $n\ge m$.   So $b\in\Cal Z$ and
$\{a_{n_k}:k\in\Bbb N\}$ is bounded above in $\Cal Z$.
}%end of proof of 526A

\leader{526B}{Proposition}\cmmnt{ ({\smc Fremlin 91})}
$\BbbN^{\Bbb N}\prT\Cal Z\prT\ell^1$.

\proof{{\bf (a)} For $\alpha\in\BbbN^{\Bbb N}$, set

\Centerline{$\phi(\alpha)=\{2^ni:n\in\Bbb N$, $i\le\alpha(n)\}$.}

\noindent Then $\phi(\alpha)\in\Cal Z$, because if $k\in\Bbb N$ then

\Centerline{$\#(m\cap\phi(\alpha))
  \le\sum_{n=0}^k\alpha(n)+\lceil 2^{-k}m\rceil$}

\noindent for every $m$.   Also $\phi:\BbbN^{\Bbb N}\to\Cal Z$
is a Tukey function, because if $\phi(\alpha)\subseteq a\in\Cal Z$ then
$\alpha(n)\le\min\{i:2^ni\notin a\}$ for every $n\in\Bbb N$.   So
$\NN\prT\Cal Z$.

\medskip

{\bf (b)} Give $\Cal Z$ the metric $\rho$ of 526A.
Then $\Cal Z$ is complete and separable and
the lattice operation $\cup$ is uniformly continuous (526Aa).   By 524C,
$(\Cal Z,\subseteq^{\strprime},[\Cal Z]^{<\omega})
\prGT(\ell^1(\omega),\le,\ell^1(\omega))$.   Since $\Cal Z$
is upwards-directed, $(\Cal Z,\subseteq,\penalty-100\Cal Z)
\equivGT(\Cal Z,\subseteq^{\strprime},[\Cal Z]^{<\omega})$ (513Id) and
$(\Cal Z,\subseteq,\Cal Z)\prGT(\ell^1,\le,\ell^1)$, that is,
$\Cal Z\prT\ell^1$.
}%end of proof of 526B

\leader{526C}{}\cmmnt{ The next three lemmas are steps on the way to
Theorem 526F.   I give them in much more generality than is required by
that theorem because a couple of them will be useful later,
and I think they are interesting in themselves.   But if
you are reading this primarily for the sake of 526F, you might save time by
looking ahead to the proof there and working backwards, extracting
arguments adequate for the special case of 526E which is actually required.

\medskip

\noindent}{\bf Lemma} Let $\sequencen{(\frak A_n,\bar\mu_n)}$ be a
sequence of purely atomic probability algebras, and
$\frak A=\prod_{n\in\Bbb N}\frak A_n$ the simple product algebra.   Then
there is an order-continuous Boolean homomorphism
$\pi:\frak A\to\Cal P\Bbb N$ such that
$\limsup_{n\to\infty}\bar\mu_na(n)$ is the upper asymptotic density
$d^*(\pi a)$ for every $a\in\frak A$;
consequently, $\lim_{n\to\infty}\bar\mu_na(n)$ is the asymptotic density
$d(\pi a)$ of $\pi a$ if either is defined.

\proof{{\bf (a)} For each $n\in\Bbb N$, let $C_n$ be the set of atoms of
$\frak A_n$, and choose rational numbers $\alpha_n(c)$ such that
$\alpha_n(c)\le\bar\mu_nc$ for each $c\in C_n$, $\sum_{c\in
C_n}\alpha_n(c)>1-2^{-n}$, and $\{c:c\in C_n$, $\alpha_n(c)>0\}$ is
finite.   Express $\alpha_n(c)$ as
$r_n(c)/s_n$ for each $c\in C_n$, where $r_n(c)\in\Bbb N$ and
$s_n\in\Bbb N\setminus\{0\}$;
let $\family{c}{C_n}{I_n(c)}$ be a disjoint family of subsets of
$\Bbb N$ with $\#(I_n(c))=r_n(c)$ for each $c$, and set
$J_n=\bigcup_{c\in C_n}I_n(c)$;
let $\pi_n:\frak A_n\to\Cal PJ_n$ be the Boolean
homomorphism such that $\pi_n(c)=I_n(c)$ for each
$c\in C_n$.   Then

\Centerline{$(1-2^{-n})s_n<s_n\sum_{c\in C_n}\alpha_n(c)=\sum_{c\in
C_n}r_n(c)=\#(J_n)\le s_n$.}

\noindent Note that $\#(J_n)>0$.   Also

$$\eqalign{\#(J_n)(\bar\mu_nd-2^{-n})
&\le\#(J_n)\sum_{c\in C_n,c\subseteq d}\alpha_n(c)\cr
&\le s_n\sum_{c\in C_n,c\subseteq d}\alpha_n(c)
=\sum_{c\in C_n,c\subseteq d}r_n(c)
=\#(\pi_nd)\cr}$$

\noindent and

\Centerline{$(1-2^{-n})\#(\pi_nd)=(1-2^{-n})s_n\sum_{c\in
C_n,c\Bsubseteq d}\alpha_n(c)\le\#(J_n)\cdot\bar\mu_nd$}

\noindent for every $d\in\frak A_n$.   So, for $a\in\frak A$,

\Centerline{$\limsup_{n\to\infty}\bar\mu_na(n)
=\limsup_{n\to\infty}\Bover{\#(\pi_na(n))}{\#(J_n)}$.}

\medskip

{\bf (b)} Let $\sequencen{m_n}$ be a sequence in $\Bbb N$ such that

\Centerline{$m_n\#(J_n)\ge 2^n\max(\#(J_{n+1}),\sum_{i<n}m_i\#(J_i))$}

\noindent for every $n$.   Set $n_k=n$ if
$\sum_{i<n}m_i\le k<\sum_{i\le n}m_i$ where $n\in\Bbb N$;  then
$\lim_{k\to\infty}n_k=\infty$.   Set $l_k=\sum_{i<k}\#(J_{n_i})$, so
that $l_{k+1}-l_k=\#(J_{n_k})$ for each $k$, and let
$\phi_k:\Cal PJ_{n_k}\to\Cal P(l_{k+1}\setminus l_k)$ be a Boolean
isomorphism;  set

\Centerline{$\pi a=\bigcup_{k\in\Bbb N}\phi_k\pi_{n_k}a(n_k)$}

\noindent for $a\in\frak A$, so that $\pi:\frak A\to\Cal P\Bbb N$ is an
order-continuous Boolean isomorphism.

\medskip

{\bf (c)} Let $a\in\frak A$, and set

\Centerline{$\gamma=\limsup_{n\to\infty}\bar\mu_na(n)
=\limsup_{n\to\infty}\Bover{\#(\pi_na(n))}{\#(J_n)}$,}

\Centerline{$\gamma'=\limsup_{l\to\infty}\Bover1l\#(l\cap\pi a)$.}

\noindent Then $\gamma\le\gamma'$.   \Prf\ Setting
$l'_n=\sum_{i<n}m_i\#(J_i)$, we have $\#(l'_{n+1}\cap\pi a)\ge
m_n\#(\pi_na(n))$, while
$l'_{n+1}=l'_n+m_n\#(J_n)\le(1+2^{-n})m_n\#(J_n)$ for each $n$;  but
this means that

$$\gamma'\ge\limsup_{n\to\infty}\bover1{l'_{n+1}}\#(l'_{n+1}\cap\pi a)
\ge\limsup_{n\to\infty}\bover{m_n\#(\pi_na(n))}{(1+2^{-n})m_n\#(J_n)}
=\gamma. \text{ \Qed}$$

\noindent Also $\gamma'\le\gamma$.   \Prf\ Let $\epsilon>0$.   Let
$n^*\ge 1$ be such that $2^{-n^*}\le\epsilon$ and
$\#(\pi_na(n))\le(\gamma+\epsilon)\#(J_n)$ for every $n\ge n^*$.
Suppose that $l\ge l'_{n^*+1}$.   Then $l$ is of the form
$l'_{n+1}+j\#(J_{n+1})+i$ where $n\ge n^*$, $j<m_{n+1}$ and
$i<\#(J_{n+1})$.   Now $l'_{n+1}=l'_n+m_n\#(J_n)$, so

$$\eqalign{\#(l'_{n+1}\cap\pi a)
&\le l'_n+m_n\#(\pi_na(n))
\le l'_n+m_n\#(J_n)(\gamma+\epsilon)\cr
&\le m_n\#(J_n)(\gamma+\epsilon+2^{-n})
\le m_n\#(J_n)(\gamma+2\epsilon)\cr}$$

\noindent by the choice of $m_n$.   Accordingly

$$\eqalignno{\#(l\cap\pi a)
&\le m_n\#(J_n)(\gamma+2\epsilon)+(j+1)\#(\pi_{n+1}a(n+1))\cr
&\le m_n\#(J_n)(\gamma+2\epsilon)+j(\gamma+\epsilon)\#(J_{n+1})
  +\#(J_{n+1})\cr
&\le (\gamma+2\epsilon)l+\#(J_{n+1})
\le (\gamma+2\epsilon)l+2^{-n}m_n\#(J_n)\cr
\displaycause{by the choice of $m_n$}
&\le (\gamma+3\epsilon)l.\cr}$$

\noindent As this is true for any $l\ge l'_{n^*+1}$,
$\gamma'\le\gamma+3\epsilon$;  as $\epsilon$ is arbitrary,
$\gamma'\le\gamma$.\ \Qed

\medskip

{\bf (d)} Thus

\Centerline{$\limsup_{n\to\infty}\bar\mu_na(n)
=\limsup_{n\to\infty}\Bover1n\#(n\cap\pi a)$}

\noindent for every $a\in\frak A$.   But as $\pi$ is a Boolean
homomorphism, it follows at once that

\Centerline{$\liminf_{n\to\infty}\bar\mu_na(n)
=\liminf_{n\to\infty}\Bover1n\#(n\cap\pi a)$}

\noindent for every $a$, so that the limits are equal if either is
defined.
}%end of proof of 526C

\leader{526D}{Lemma} Let $(\frak A,\bar\mu)$ be a semi-finite measure
algebra, and $\kappa\ge\max(\omega,c(\frak A),\tau(\frak A))$ a cardinal.
Let $(\frak B_{\kappa},\bar\nu_{\kappa})$ be the measure algebra of the
usual measure on $\{0,1\}^{\kappa}$, and $\gamma>0$.   Then there is a
function $\theta:\frak A\to\frak B_{\kappa}$ such that

(i) $\theta(\sup A)=\sup\theta[A]$ for every non-empty $A\subseteq\frak A$ such
that $\sup A$ is defined in $\frak A$;

(ii) $\bar\nu_{\kappa}\theta(a)=1-e^{-\gamma\bar\mu a}$ for every
$a\in\frak A$, interpreting $e^{-\infty}$ as $0$;

(iii) if $\familyiI{a_i}$ is a disjoint family in $\frak A$, and
$\frak C_i$ is the closed subalgebra of $\frak B_{\kappa}$ generated by
$\{\theta(a):a\Bsubseteq a_i\}$ for each $i$, then $\familyiI{\frak C_i}$ is
stochastically independent.

\proof{{\bf (a)} By 495J, we have exactly this result for some
probability algebra $(\frak B,\bar\lambda)$ in place of
$(\frak B_{\kappa},\bar\nu_{\kappa})$.
Set $\frak A^f=\{a:a\in\frak A$, $\bar\mu a<\infty\}$, and
give $\frak A^f$ its measure metric $\rho$ (323Ad).   Then
$\theta\restrp\frak A^f$ is uniformly continuous for $\rho$ and the measure
metric $\sigma$ of $\frak B$.   \Prf\ If $\epsilon>0$, there is a
$\delta>0$ such that $|e^{-\gamma s}-e^{-\gamma t}|\le\bover12\epsilon$
whenever $s$, $t\in\coint{0,\infty}$ and $|s-t|\le\delta$.   Now if $a$,
$a'\in\frak A$ and $\bar\mu(a\Bsymmdiff a')\le\delta$, set
$b=a\Bcap a'$;  then $\theta(b)\Bsubseteq\theta(a)$ and
$\bar\mu a-\bar\mu b\le\delta$, so

\Centerline{$\sigma(\theta(a),\theta(b))
=\bar\lambda(\theta(a)\Bsetminus\theta(b))
=\bar\lambda\theta(a)-\bar\lambda\theta(b)
=e^{-\gamma\bar\mu b}-e^{-\gamma\bar\mu a}\le\Bover12\epsilon$.}

\noindent Similarly, $\sigma(\theta(a'),\theta(b))\le\bover12\epsilon$ so
$\sigma(\theta(a),\theta(a'))\le\epsilon$.
As $\epsilon$ is arbitrary, this gives the result.\ \Qed

\medskip

{\bf (b)} By 521Eb, there is a set $B\subseteq\frak A^f$, of
cardinal at most $\kappa$, which is dense for $\rho$.   Accordingly
$\theta[B]$ is dense in $f[\frak A^f]$ for $\sigma$.
Taking $\frak D$ to be
the closed subalgebra of $\frak B$ generated by $\theta[B]$,
$\tau(\frak D)\le\kappa$ and $\theta[\frak A^f]\subseteq\frak D$.
But if $a\in\frak A\setminus\frak A^f$ then $\theta(a)=1$, so
$\theta[\frak A]\subseteq\frak D$.   Now there is a measure-preserving
Boolean homomorphism $\phi:\frak D\to\frak B_{\kappa}$ (332N), and
$\phi\theta:\frak A\to\frak B_{\kappa}$ has the properties we need.
}%end of proof of 526D

\vleader{48pt}{526E}{Lemma} Let $\sequencen{(\frak A_n,\bar\mu_n)}$ be a
sequence of finite probability algebras and $\sequencen{\gamma_n}$ a
sequence in $\ooint{0,\infty}$.   Write $P$ for the set

\Centerline{$\{p:p\in\prod_{n\in\Bbb N}\frak A_n$,
$\lim_{n\to\infty}\gamma_n\bar\mu_np(n)=0\}$,}

\noindent with the ordering inherited from the product partial order on
$\prod_{n\in\Bbb N}\frak A_n$.   Then $P\prT\Cal Z$.

\proof{{\bf (a)} By 526D, we can find for each $n$ a probability
algebra $(\frak B_n,\bar\nu_n)$ and a function
$\theta_n:\frak A_n\to\frak B_n$ such that, for all $a$, $a'\in\frak A_n$,

\Centerline{$\theta_n(a\Bcup a')=\theta_n(a)\Bcup\theta_n(a')$,}

\Centerline{$\bar\nu_n\theta_n(a)=1-\exp(-\gamma_n\bar\mu a)$.}

\noindent We may suppose that $\frak B_n$ is generated by
$\theta_n[\frak A_n]$, so is itself finite.   Set
$\frak A=\prod_{n\in\Bbb N}\frak A_n$,
$\frak B=\prod_{n\in\Bbb N}\frak B_n$,
$\theta(p)=\sequencen{\theta_n(p(n))}$ for
$p\in\frak A$;  then $\theta(\sup A)=\sup\theta[A]$ for
any non-empty subset $A$ of $\frak A$.   Set

\Centerline{$Q=\{q:q\in\prod_{n\in\Bbb N}\frak B_n$,
$\lim_{n\to\infty}\bar\nu_nq(n)=0\}$.}

\noindent Then $\theta\restr P$ is a Tukey function from $P$ to $Q$.
\Prf\ $P=f^{-1}[Q]$, because $\lim_{n\to\infty}\gamma_n\xi_n=0$ iff
$\lim_{n\to\infty}1-e^{-\gamma_n\xi_n}=0$.
So $\theta\restr P$ is a function from $P$ to $Q$.   If $q\in Q$,
$A=\{p:p\in\frak A$, $\theta(p)\Bsubseteq q\}$ has a supremum
$p_0\in\frak A$;  now $\theta(p_0)=\sup\theta[A]\Bsubseteq q$, so
$\theta(p_0)\in Q$
and $p_0\in P$ is an upper bound for $A$ in $P$.\ \Qed

\medskip

{\bf (b)} By 526C, we have an order-continuous Boolean homomorphism
$\pi:\frak B\to\Cal P\Bbb N$ such that $\pi(q)\in\Cal Z$
iff $q\in Q$.   Now $\pi\restr Q$ is a Tukey function from $Q$ to
$\Cal Z$.   \Prf\ If $d\in\Cal Z$, set
$B=\{q:q\in\frak B$, $\pi(q)\subseteq d\}$.   Because $\pi$ is an
order-continuous Boolean homomorphism, $B$ contains its supremum, and
$B$ is bounded above in $Q$.\ \Qed

\medskip

{\bf (c)} Thus $\pi\theta\restr P:P\to\Cal Z$ is a Tukey function and
$P\prT\Cal Z$.
}%end of proof of 526E

\leader{526F}{Theorem}
$(\ell^1,\le,\ell^1)\prGT(\NN,\le,\NN)\ltimes(\Cal Z,\subseteq,\Cal Z)$.

\proof{{\bf (a)} Let $Q\subseteq\NN$ be the set of strictly increasing
sequences $\alpha$ such that $\alpha(0)>0$.   For $\alpha\in Q$, set

$$\eqalign{P_{\alpha}
&=\{x:x\in\ell^1,\,\|x\|_{\infty}\le \alpha(0),\,
  \lim_{n\to\infty}2^n\sum_{i=\alpha(n)}^{\infty}x(i)^+=0\}\cr
&=\{x:x\in\ell^{\infty},\,\|x\|_{\infty}\le \alpha(0),\,
  \lim_{n\to\infty}2^n\sum_{i=\alpha(n)}^{\alpha(n+1)-1}x(i)^+=0\}\cr}$$

\noindent because

$$2^n\sum_{i=\alpha(n)}^{\infty}x(i)^+
=\sum_{m=n}^{\infty}2^{n-m}2^m
\sum_{i=\alpha(m)}^{\alpha(m+1)-1}x(i)^+
\le 2\sup_{m\ge n}2^m\sum_{i=\alpha(m)}^{\alpha(m+1)-1}x(i)^+$$

\noindent for every $n$ and $x$.

The point is that $P_{\alpha}\prT\Cal Z$.   \Prf\ For each $n\in\Bbb N$
set $k_n=2^{2n}(\alpha(n+1)-\alpha(n))$,

\Centerline{$V_n=(\alpha(n+1)\setminus \alpha(n))\times
k_n\alpha(0)\subseteq\Bbb N\times\Bbb N$,
\quad$\frak A_n=\Cal PV_n$,}

\noindent and let $\bar\mu_n$ be the uniform probability measure on
$\frak A_n$, so that $\bar\mu_nd=\#(d)/\#(V_n)$ for $d\subseteq V_n$.
For $n\in\Bbb N$ and $x\in\ell^{\infty}$ set

\Centerline{$f_n(x)=\{(i,j):\alpha(n)\le i<\alpha(n+1)$,
$j<k_n\min(\alpha(0),x(i))\}\subseteq V_n$.}

\noindent Then

\Centerline{$|\#(f_n(x))
  -k_n\sum_{\alpha(n)\le i<\alpha(n+1)}\min(\alpha(0),x(i)^+)|
\le\alpha(n+1)-\alpha(n)$,}

\noindent so if $\|x\|_{\infty}\le \alpha(0)$ then

$$\eqalign{\bigl|2^n\alpha(0)(\alpha(n+1)-\alpha(n))\bar\mu f_n(x)
  -2^n\sum_{i=\alpha(n)}^{\alpha(n+1)-1}x(i)^+\bigr|
&\le 2^n(\alpha(n+1)-\alpha(n))/k_n\cr
&=2^{-n}.\cr}$$

\noindent Accordingly

\Centerline{$P_{\alpha}=\{x:x\in\ell^{\infty}$,
$\|x\|_{\infty}\le\alpha(0)$,
$\lim_{n\to\infty}2^n(\alpha(n+1)-\alpha(n))\bar\mu_nf_n(x)
=0\}$.}

Let $\frak A=\prod_{n\in\Bbb N}\frak A_n$ be the simple product of the
Boolean algebras $\frak A_n$, and $I$ the ideal

\Centerline{$\{a:a\in\frak A$,
$\lim_{n\to\infty}
 2^n(\alpha(n+1)-\alpha(n))\bar\mu_na(n)=0\}$}

\noindent of $\frak A$.
For $x\in\ell^{\infty}$, set $f(x)=\sequencen{f_n(x)}$.
Observe that $f:\ell^{\infty}\to\frak A$ is supremum-preserving in the
sense that $f(\sup A)=\sup f[A]$ for any non-empty bounded subset $A$ of
$\ell^{\infty}$.

The last formula for $P_{\alpha}$ shows that $f(x)\in I$ for every
$x\in P_{\alpha}$.   But if $a\in I$,
$A=\{x:x\in P_{\alpha}$, $f(x)\Bsubseteq a\}$ is upwards-directed and
has a supremum $x_0\in\ell^{\infty}$, with $\|x_0\|_{\infty}\le
\alpha(0)$.   Now $f(x_0)=\sup_{x\in A}f(x)\Bsubseteq a$, so $x_0\in
P_{\alpha}$ and is an
upper bound for $A$ in $P_{\alpha}$.   Thus $f\restr P_{\alpha}$ is a
Tukey function from $P_{\alpha}$ to $I$, and $P_{\alpha}\prT I$.
By 526E, $I\prT\Cal Z$, so $P_{\alpha}\prT\Cal Z$.\ \Qed

Thus $(P_{\alpha},\le,P_{\alpha})\prGT(\Cal Z,\subseteq,\Cal Z)$;  it
follows at once that
$(P_{\alpha},\le,\ell^1)\prGT(\Cal Z,\subseteq,\Cal Z)$.

\medskip

{\bf (b)} Now, for $\alpha\in\NN$, take $\tilde P_\alpha=P_{\beta}$ where
$\beta(n)=1+n+\max_{i\le n}\alpha(i)$ for $n\in\Bbb N$.   Then
$\tilde P_{\alpha}\subseteq\tilde P_{\alpha'}$ whenever
$\alpha\le\alpha'$ in
$\NN$, $\bigcup_{\alpha\in\NN}\tilde P_\alpha=\ell^1$ and
$(\tilde P_\alpha,\le,\ell^1)\prGT(\Cal Z,\subseteq,\Cal Z)$ for every
$\alpha$;  so
$(\ell^1,\le,\ell^1)\prGT(\NN,\le,\NN)\ltimes(\Cal Z,\subseteq,\Cal Z)$,
by 512K.
}%end of proof of 526F

\leader{526G}{Corollary} Let $\Cal N$ be the ideal of Lebesgue
negligible subsets of $\Bbb R$.

(a) $\add_{\omega}\Cal Z=\add\Cal N=\add_{\omega}\ell^1$ and
$\cf\Cal Z=\cf\Cal N=\cf\ell^1$.

(b) If $\Cal A\subseteq\Cal Z$ and $\#(\Cal A)<\add\Cal N$, there is a
$J\in\Cal Z$ such that $I\setminus J$ is finite for every $I\in\Cal A$.

%note:  this works for all analytic P-ideals, see {\smc Farah 00},
%5.6.3.

\proof{{\bf (a)(i)} Putting 526B and 513Ie together, we see that

\Centerline{$\add_{\omega}\BbbN^{\Bbb N}\ge\add_{\omega}\Cal Z
\ge\add_{\omega}\ell^1$,}

\noindent that is,

\Centerline{$\frak b\ge\add_{\omega}\Cal Z\ge\add\Cal N$}

\noindent (522A, 524I).   Next, we can deduce from 526F that
$\add_{\omega}\ell^1
\ge\min(\add_{\omega}\BbbN^{\Bbb N},\add_{\omega}\Cal Z)$.   \Prf\ Let
$(\phi,\psi)$ be a Galois-Tukey connection from
$(\ell^1,\le,\ell^1)$ to

\Centerline{$(\NN,\le,\NN)\ltimes(\Cal Z,\subseteq,\Cal Z)
=(\NN\times\Cal Z^{\NN},T,\NN\times\Cal Z)$,}

\noindent where

\Centerline{$T=\{((p,f),(q,a)):p\le q$ in $\NN$, $f(q)\subseteq a\in\Cal
Z\}$.}

\noindent We can interpret $\phi$ as a pair $(\phi_1,\phi_2)$ where
$\phi_1$ is a function from $\ell^1$ to $\NN$ and $\phi_2$ is a function
from $\ell^1\times\NN$ to $\Cal Z$, and saying that $(\phi,\psi)$ is a
Galois-Tukey connection means just that

\Centerline{if $\phi_1(x)\le q$ and $\phi_2(x,q)\subseteq a$ then
$x\le\psi(q,a)$.}

Now suppose that $A\subseteq\ell^1$ and
$\#(A)<\min(\add_{\omega}\NN,\add_{\omega}\Cal Z)$.
Then there is a sequence
$\sequencen{q_n}$ in $\NN$ such that for every $x\in A$ there is an
$n\in\Bbb N$ such that $\phi_1(x)\le q_n$.   Next, for each $n\in\Bbb N$
there is a sequence
$\sequence{m}{a_{nm}}$ in $\Cal Z$ such that for every $x\in A$ there is
an $m\in\Bbb N$ such that $\phi_2(x,q_n)\subseteq a_{nm}$.   In this
case, $B=\{\psi(q_n,a_{nm}):m$, $n\in\Bbb N\}$ is a countable subset of
$\Cal Z$, and for every $x\in A$ there are $m$, $n\in\Bbb N$ such that
$\phi_1(x)\le q_n$ and $\phi_2(x,q_n)\subseteq a_{nm}$, so that
$x\le\psi(q_n,a_{nm})\in B$.   As $A$ is arbitrary,
$\add_{\omega}\ell^1\ge\min(\add_{\omega}\NN,\add_{\omega}\Cal Z)$.\
\Qed

Thus we have
$\add\Cal N\ge\min(\frak b,\add_{\omega}\Cal Z)=\add_{\omega}\Cal Z$,
and $\add_{\omega}\Cal Z=\add\Cal N$.   And we know from 524I, with
$\kappa=\omega$ there, that $\add\Cal N=\add_{\omega}\ell^1$


\medskip

\quad{\bf (ii)} On the other hand, 524I, 526F, 512Da and 512Jb tell us
that

$$\eqalign{\cf\Cal N
&=\cf\ell^1
=\cov(\ell^1,\subseteq,\ell^1)
\le\cov((\NN,\le,\NN)\ltimes(\Cal Z,\subseteq,\Cal Z))\cr
&=\max(\cov(\NN,\le,\NN),\cov(\Cal Z,\subseteq,\Cal Z))
=\max(\frak d,\cf\Cal Z).\cr}$$

\noindent But from 526B we see that $\frak d\le\cf\Cal Z\le\cf\ell^1$,
so $\cf\Cal Z=\cf\Cal N$, while 524I tells us that $\cf\Cal N=\cf\ell^1$.

\medskip

{\bf (b)} By (a), there is a countable set $\Cal D\subseteq\Cal Z$ such
that every member of $\Cal A$ is included in a member of $\Cal D$.   By
491Ae, there is a $J\in\Cal Z$ such that
$I\setminus J$ is finite for every $I\in\Cal D$;  this $J$ serves.
}%end of proof of 526G

\leader{526H}{}\cmmnt{ I turn now to ideals of nowhere dense sets.

\medskip

\noindent}{\bf Proposition}
Let $\CalNwd$ be the ideal of nowhere dense subsets
of $\NN$ and $\Cal M$ the ideal of meager subsets of $\NN$.

(a) $\CalNwd$ is isomorphic, as partially ordered set, to
$(\CalNwd)^{\Bbb N}$.

(b) $(\CalNwd,\subseteq^{\strprime},[\CalNwd]^{\le\omega})
\equivGT(\Cal M,\subseteq,\Cal M)$.

(c) $\CalNwd\prT\ell^1$.

(d) Let $X$ be a set and $\Cal V$ a countable family of subsets of $X$.
Set

\Centerline{$\Cal D=\{D:D\subseteq X$, for every $V\in\Cal V$ there is a
$V'\in\Cal V$ such that $V'\subseteq V\setminus D\}$.}

\noindent Then $\Cal D\prT\CalNwd$.

(e) If $X$ is any non-empty Polish space without isolated points, and
$\CalNwd(X)$ is the ideal of nowhere dense subsets of $X$, then
$\CalNwd\equivT\CalNwd(X)$.

(f) If $X$ is a compact metrizable space and
$\Cal C_{\text{nwd}}$ is the family of closed nowhere dense
subsets of $X$ with the Fell\cmmnt{ (or Vietoris)}
topology\cmmnt{ (4A2T)},
then $(\Cal C_{\text{nwd}},\subseteq)$ is a metrizably compactly based
directed set.

\cmmnt{\medskip

\noindent{\bf Remark} Recall that if $R$ is any relation then
$R^{\strprime}$ is the relation
$\{(x,B):(x,y)\in R$ for some $y\in B\}$;  see 512F-512G.
}%end of comment

\proof{ Enumerate $S=\bigcup_{n\in\Bbb N}\BbbN^n$
as $\sequencen{\sigma_n}$.
For $\sigma\in S$ write
$I_{\sigma}=\{\alpha:\sigma\subseteq\alpha\in\NN\}$.

\medskip

{\bf (a)} Define $\phi:\CalNwd\to\CalNwd^{\Bbb N}$ by setting
$\phi(F)(n)=\{\alpha:\fraction{n}^{\smallfrown}\alpha\in F\}$, where
$\fraction{n}^{\smallfrown}\alpha=(n,\alpha(0),\alpha(1),\ldots)$
for $n\in\Bbb N$ and $\alpha\in\NN$.   Then $\phi$ is an isomorphism
between $\CalNwd$ and $\CalNwd^{\Bbb N}$.

\medskip

{\bf (b)(i)} Choose $\phi:\Cal M\to\CalNwd^{\Bbb N}$ such that
$M\subseteq\bigcup_{n\in\Bbb N}\phi(M)(n)$ for every $M\in\Cal M$.
Then $\phi$ is a Tukey function so
$\Cal M\prT\CalNwd^{\Bbb N}\cong\CalNwd$, that is,
$(\Cal M,\subseteq,\Cal M)\prGT(\CalNwd,\subseteq,\CalNwd)$.   By 513Id
and 512Gb,

\Centerline{$(\Cal M,\subseteq,\Cal M)
\equivGT(\Cal M,\subseteq^{\strprime},[\Cal M]^{\le\omega})
\prGT(\CalNwd,\subseteq^{\strprime},[\CalNwd]^{\le\omega})$.}

\woddheader{526H}{4}{2}{2}{42pt}

\quad{\bf (ii)} For $n\in\Bbb N$ and $\tau\in\BbbN^n$, define
$g_{\tau}:\NN\to\NN$ by saying that
$g_{\tau}(\alpha)=\tau^{\smallfrown}\alpha$, that is,

$$\eqalign{g_{\tau}(\alpha)(i)
&=\tau(i)\text{ if }i<n,\cr
&=\alpha(i-n)\text{ otherwise}.\cr}$$

\noindent Note that $g_{\tau}$ is a homeomorphism between
$\NN$ and $I_{\tau}$, so that $g_{\tau}[F]$ and $g_{\tau}^{-1}[F]$ are
nowhere dense whenever $F$ is.

Now for any $F\in\CalNwd$ we can find a $\phi(F)\in\CalNwd$ such that
$F\subseteq\phi(F)$ and for every $\sigma\in S$ either
$I_{\sigma}\cap\phi(F)=\emptyset$
or there is a $\tau\in S$, extending $\sigma$, such that
$g_{\tau}[F]\subseteq\phi(F)$.   \Prf\ Choose $\sequencen{\tau_n}$,
$\sequencen{\upsilon_n}$ inductively,
as follows.  Given that $I_{\upsilon_i}\cap(F\cup
g_{\tau_j}[F])=\emptyset$ for all $i$, $j<n$, set
$E=I_{\sigma_n}\cap(F\cup\bigcup_{j<n}g_{\tau_j}[F])$.   If
$E=\emptyset$ set $\upsilon_n=\sigma_n$ and $\tau_n=\emptyset$, so that
$g_{\tau_n}[F]=F$.
If $E$ is not empty, it is still nowhere dense, so we can find
$\upsilon_n\supseteq\sigma_n$ such that
$I_{\upsilon_n}\cap E=\emptyset$.   In this case,
$\bigcup_{i\le n}I_{\upsilon_i}$ is a closed set not including
$I_{\sigma_n}$, so we can find a $\tau_n\supseteq\sigma_n$ such that
$I_{\tau_n}\cap\bigcup_{i\le n}I_{\upsilon_i}=\emptyset$,
and $I_{\upsilon_i}\cap g_{\tau_n}[F]=\emptyset$ for $i\le n$.
Thus in both cases we shall have
$\bigcup_{i\le n}I_{\upsilon_i}\cap(F\cup\bigcup_{j\le n}g_{\tau_j}[F])
=\emptyset$,
and the induction proceeds.

Set $\phi(F)=\overline{F\cup\bigcup_{j\in\Bbb N}g_{\tau_j}[F]}$.
Because $\upsilon_i\supseteq\sigma_i$ and $\phi(F)\cap I_{\upsilon_i}$ is
empty for every $i\in\Bbb N$, $\phi(F)\in\CalNwd$.
If $\sigma\in S$ is such that $\phi(F)$ meets $I_{\sigma}$, there is
an $n\in\Bbb N$ such that $\sigma=\sigma_n$;  now we cannot have
$\upsilon_n=\sigma_n$ so we must have
$\tau_n\supseteq\sigma_n$ and $g_{\tau_n}[F]\subseteq\phi(F)$.   Thus we
have found an suitable set $\phi(F)$.\ \Qed

For each $M\in\Cal M$ let $\Cal E_M$ be a non-empty countable family of
closed nowhere dense sets covering $M$, and set
$\psi(M)=\{g_{\tau}^{-1}[E]:E\in\Cal E_M$, $\tau\in S\}$.
Then $(\phi,\psi)$ is a Galois-Tukey connection from
$(\CalNwd,\subseteq^{\strprime},[\CalNwd]^{\le\omega})$ to
$(\Cal M,\subseteq,\Cal M)$.   \Prf\ Suppose that $F\in\CalNwd$ and
$M\in\Cal M$ are such that $\phi(F)\subseteq M$.
If $F=\emptyset$ then certainly there is an $F'\in\psi(M)$ covering $F$.
Otherwise, $\phi(F)$ is a non-empty closed set included in the union of
the countable set $\Cal E_M$ of closed sets.   By Baire's theorem, there
must be a $\sigma\in S$ and an $E\in\Cal E_M$ such that
$\emptyset\ne\phi(F)\cap I_{\sigma}\subseteq E$.
In this case, there is a $\tau\supseteq\sigma$ such that
$g_{\tau}[F]\subseteq\phi(F)$, so that
$g_{\tau}[F]\subseteq E$ and $F\subseteq g_{\tau}^{-1}[E]\in\psi(M)$ and
$F\subseteq^{\strprime}\psi(M)$.   As $F$ and $M$ are arbitrary,
$(\phi,\psi)$ is a Galois-Tukey connection.\ \Qed

\medskip

\quad{\bf (iii)} Thus we have

\Centerline{$(\Cal M,\subseteq,\Cal M)
\prGT(\CalNwd,\subseteq^{\strprime},[\CalNwd]^{\le\omega})
\prGT(\Cal M,\subseteq,\Cal M)$}

\noindent and $(\Cal M,\subseteq,\Cal M)
\equivGT(\CalNwd,\subseteq^{\strprime},[\CalNwd]^{\le\omega})$.

\medskip

{\bf (c)} We can use the idea of 522O.   Let $\sequencen{U_n}$ enumerate
a base for the topology of $\NN$ not containing $\emptyset$.
By 522N, there is for each
$n\in\Bbb N$ a countable family $\Cal V_n$
of open subsets of $U_n$ such that $\bigcap\Cal V\ne\emptyset$ for every
$\Cal V\in[\Cal V_n]^{\le 2^n}$ and every dense open subset of $U_n$
includes some member of $\Cal V_n$.   Enumerate $\Cal V_n$
as $\sequence{m}{U_{nm}}$.

For each $F\in\CalNwd$ let $f_F:\Bbb N\to\Bbb N$ be such that
$F\cap U_{n,f_F(n)}=\emptyset$ for every $n\in\Bbb N$, and for $n$,
$i\in\Bbb N$ set

$$\eqalign{\phi(F)(2^n(2i+1)-1)
&=2^{-n}\text{ if }f_F(n)=i,\cr
&=0\text{ otherwise}.\cr}$$

\noindent Then
$\sum_{i=0}^{\infty}\phi(F)(i)=2$ for each $F$, so
we have a function $\phi:\CalNwd\to\ell^1$.

Suppose that $x\in\ell^1$.   Set
$\Cal A=\{F:F\in\CalNwd$, $\phi(F)\le x\}$ and $E=\bigcup\Cal A$.
The set

\Centerline{$K=\{n:\#(\{i:x(2^n(2i+1)-1)\ge 2^{-n}\})\ge 2^n\}$}

\noindent is finite;   set $k=\sup(\{0\}\cup K)$.   If $n>k$,
then $\#(\{f_F(n):F\in\Cal A\})<2^n$, so
$\bigcap_{F\in\Cal A}U_{n,f_F(n)}$ is a non-empty open subset
of $U_n$ disjoint from $\bigcup_{F\in\Cal A}F=E$.   Thus
$\{n:U_n\subseteq\overline{E}\}\subseteq\{0,\ldots,k\}$ is finite,
and therefore in fact is empty, that is, $E\in\CalNwd$.

As $x$ is arbitrary, $\phi:\CalNwd\to\ell^1$ is a Tukey function, and
witnesses that $\CalNwd\prT\ell^1$.

\medskip

{\bf (d)} If $\Cal V=\emptyset$ then $\Cal D=\Cal PX$ has a greatest
element and the result is trivial (any function from $\Cal D$ to
$\CalNwd$ will be a Tukey function).   Otherwise,
choose a function $h:S\to\Cal V\cup\{X\}$ such that $h(\emptyset)=X$
and $\sequence{i}{h(\sigma^{\smallfrown}\fraction{i})}$ runs over
$\{V:h(\sigma)\supseteq V\in\Cal V\}$ for every $\sigma\in S$.
Note that $h(\tau)\subseteq h(\sigma)$ whenever $\tau\supseteq\sigma$,
and that $\{h(\tau):\sigma\subseteq\tau\in\BbbN^n\}=\{V:V\in\Cal V$,
$V\subseteq h(\sigma)\}$ whenever $m\in\Bbb N$, $\sigma\in\BbbN^m$ and
$n>m$.
For each $D\in\Cal D$ we can choose a sequence $\sequencen{\tau_{Dn}}$
in $S$ such that $\tau_{Dn}\supseteq\sigma_n$ and $D\cap h(\tau_{Dn})$
is empty and $\#(\tau_{Dn})\ge n$ for
every $n\in\Bbb N$.
Set $\phi(D)=\NN\setminus\bigcup_{n\in\Bbb N}I_{\tau_{Dn}}$, so that
$\phi(D)\in\CalNwd$.

Take any $F\in\CalNwd$, and set $D_0=\bigcup\{D:D\in\Cal D$,
$\phi(D)\subseteq F\}$.   Then $D_0\in\Cal D$.   \Prf\ Let $V\in\Cal V$.
Let $\upsilon\in\BbbN^1$ be such that
$h(\upsilon)=V$.   Take $\tau\supseteq\upsilon$ such that $F\cap
I_{\tau}=\emptyset$.   \Quer\ If $D_0\cap h(\tau)\ne\emptyset$, then
there is a $D\in\Cal D$ such that $\phi(D)\subseteq F$
and $D\cap h(\tau)\ne\emptyset$.   $I_{\tau}\cap\phi(D)$ is empty, that
is, $I_{\tau}\subseteq\bigcup_{n\in\Bbb N}I_{\tau_{Dn}}$;  because
$\#(\tau_{Dn})\ge n$ for every $n$, this can happen only because there
is some $n\in\Bbb N$ such that $\tau_{Dn}\subseteq\tau$.   But this
means that $D\cap h(\tau)\subseteq D\cap h(\tau_{Dn})=\emptyset$, which
is impossible.\ \BanG\  Thus $D_0\cap h(\tau)$ is
empty, and $h(\tau)$ is a member of $\Cal V$ included in
$V\setminus D_0$.   As $V$ is arbitrary, $D_0\in\Cal V$.\ \Qed

As $F$ is arbitrary, $\phi$ is a Tukey function and $\Cal D\prT\CalNwd$,
as claimed.

\medskip

{\bf (e)(i)} Taking $\Cal V$ to be a countable base for the topology of $X$
not containing $\emptyset$, we have

\Centerline{$\CalNwd(X)=\{F:F\subseteq X$, for every $V\in\Cal V$ there is a
$V'\in\Cal V$ such that $V'\subseteq V\setminus F\}$,}

\noindent so (d) tells us that $\CalNwd(X)\prT\CalNwd$.

\medskip

\quad{\bf (ii)} $X$ has a dense subset $Y$ which is
homeomorphic to $\NN$ (5A4Ie).   Let $\CalNwd(Y)$ be the family of nowhere
dense subsets of $Y$.   For $F\in\CalNwd(Y)$ let $\phi(F)$ be its closure in
$X$.   Then $\phi$ is a Tukey function from $\CalNwd(Y)$ to $\CalNwd(X)$, so
$\CalNwd\cong\CalNwd(Y)\prT\CalNwd(X)$.

\medskip

{\bf (f)} By 4A2Tg,
the Fell topology on the family $\Cal C$ of all closed subsets of $X$
is compact and metrizable.   $E\cup F\in\Cal C_{\text{nwd}}$ for all $E$,
$F\in\Cal C_{\text{nwd}}$, and
$\cup:\Cal C_{\text{nwd}}\times\Cal C_{\text{nwd}}\to\Cal C_{\text{nwd}}$
is continuous (4A2T(b-ii)).   If $F\in\Cal C_{\text{nwd}}$, the set
$\{E:E\in\Cal C_{\text{nwd}}$, $E\subseteq F\}
=\{E:E\in\Cal C$, $E\cup F=F\}$ is closed in $\Cal C$, therefore compact.
Now suppose that $\sequence{k}{E_k}$ is a sequence in $\Cal C_{\text{nwd}}$
converging to $E\in\Cal C_{\text{nwd}}$.   If $X=\emptyset$ then of course
$\{E_k:k\in\Bbb N\}$ is bounded above in $\Cal C_{\text{nwd}}$.
Otherwise, let $\sequencen{U_n}$ run over a base for the topology of $X$
not containing $\emptyset$, and choose $\sequencen{k_n}$, $\sequencen{V_n}$
inductively, as follows.   Given $k_i\in\Bbb N$ for $i<n$, let
$V_n\subseteq U_n$ be a non-empty open set such that
$\overline{V}_n\cap(E\cup\bigcup_{i<n}E_{k_i})=\emptyset$;  given that
$E\cap\overline{V}_i=\emptyset$ for $i\le n$, choose $k_n\ge n$ such that
$E_{k_n}\cap\bigcup_{i\le n}\overline{V}_i$ is empty.   (This is possible
because $\bigcup_{i\le n}\overline{V}_i$ is compact, so the family of
sets disjoint from it is open in the Fell topology.)   Continue.   At
the end of the induction, $G=\bigcup_{n\in\Bbb N}V_n$ is a dense open set
disjoint from $\bigcup_{n\in\Bbb N}E_{k_n}$, so $X\setminus G$ is an upper
bound for $\{E_{k_n}:n\in\Bbb N\}$ in $\Cal C_{\text{nwd}}$.   Thus all the
conditions of 513K are satisfied, and $\Cal C_{\text{nwd}}$ is metrizably
compactly based.
}%end of proof of 526H

\leader{526I}{}\cmmnt{ A related type of ideal is the following.   I
express the result in more general form because it has some measure
theory in it.

\medskip

\noindent}{\bf Proposition}\cmmnt{ ({\smc Fremlin 91})} Let $X$ be a
second-countable topological space and $\mu$ a $\sigma$-finite
topological measure on $X$.
Let $\Cal E$ be the ideal of subsets of $X$ with negligible closures.
Then, writing $\CalNwd$ for the ideal of nowhere dense subsets of $\NN$,
$\Cal E\prT\CalNwd$ and $\Cal E\prT\Cal Z$.

\proof{{\bf (a)} If $\mu X=0$ then $\Cal E$ has a greatest element and
the result is trivial.
Otherwise, there is a probability measure on $X$ with the same
measurable sets and the same
negligible sets as $\mu$ (215B(vii));  so we may suppose that $\mu$
itself is a probability measure.   Let $\Cal U$ be a countable base for
the topology of $X$,
containing $X$ and closed under finite unions.

\medskip

{\bf (b)} For $k\in\Bbb N$ let $\Cal V_k$ be the countable set
$\{V:V\in\Cal U$, $\mu V>1-2^{-k}\}$.   Set

\Centerline{$\Cal E_k=\{E:E\subseteq X$, for every $V\in\Cal V_k$ there
is a $U\in\Cal V_k$ such that $U\subseteq V\setminus E\}$.}

\noindent Then $\Cal E=\bigcap_{k\in\Bbb N}\Cal E_k$.   \Prf\ Because
$X\in\Cal V_k$, $\mu\overline{E}\le 2^{-k}$ for every $E\in\Cal E_k$, so
$\bigcap_{k\in\Bbb N}\Cal E_k\subseteq\Cal E$.
On the other hand, if $E\in\Cal E$ and $k\in\Bbb N$ and $V\in\Cal V_k$,
then $\mu(V\setminus\overline{E})>1-2^{-k}$ and
$\Cal U'=\{U:U\in\Cal U$, $U\subseteq V\setminus E\}$ has union
$V\setminus\overline{E}$.   As $\Cal U'$ is countable, there is a finite
$\Cal U'_1\subseteq\Cal U'$ such that $U=\bigcup\Cal U'_1$ has measure
greater than $1-2^{-k}$, so that
$U\in\Cal V_k$ and $U\subseteq V\setminus E$.   As $V$ is arbitrary,
$E\in\Cal V_k$;  as $E$ and $k$ are arbitrary,
$\Cal E\subseteq\bigcap_{k\in\Bbb N}\Cal E_k$.\ \QeD

This means that the map $E\mapsto(E,E,E,\ldots)$ is a Tukey function
from $\Cal E$ to $\prod_{k\in\Bbb N}\Cal E_k$, so that
$\Cal E\prT\prod_{k\in\Bbb N}\Cal E_k$.   At the same time,
$\Cal E_k\prT\CalNwd$ for every $k$, by 526Hd.   So
$\Cal E\prT\CalNwd^{\Bbb N}\cong\CalNwd$ (513Eg, 526Ha).

\medskip

{\bf (c)} Let $\frak A$ be the countable subalgebra of $\Cal PX$
generated by $\Cal U$.   Then there is a Boolean homomorphism
$\pi:\frak A\to\Cal P\Bbb N$ such that $d(\pi E)$ is defined
and equal to $\mu E$ for every $E\in\frak A$.   \Prf\ This is easy to
prove directly (see 491Xu), but we can also argue as follows.
Let $\sequencen{\frak A_n}$ be a non-decreasing
sequence of finite subalgebras
of $\frak A$ with union $\frak A$.   By 526C, we have a Boolean
homomorphism $\pi':\prod_{n\in\Bbb N}\frak A_n\to\Cal P\Bbb N$ such that
$d(\pi'\sequencen{E_n})=\lim_{n\to\infty}\mu E_n$ whenever
$E_n\in\frak A_n$ for every $n$ and the limit on the right is defined.
For each $n\in\Bbb N$ let $\pi_n:\frak A\to\frak A_n$ be a Boolean
homomorphism extending the identity
homomorphism on $\frak A_n$ (314K, or otherwise);  set
$\pi E=\pi'\sequencen{\pi_nE}$ for $E\in\frak A$;  this works.\ \Qed

Let $\sequencen{V_n}$ be a sequence running over the closed sets
belonging to $\frak A$.
Let $\sequencen{k_n}$ be a strictly increasing sequence in $\Bbb
N\setminus\{0\}$ such that $\Bover1{k_n}\#(k_n\cap\pi E)\ge\mu E-2^{-n}$
whenever $E$ belongs to the subalgebra $\frak B_n$ of
$\frak A$ generated by $\{V_i:i\le n\}$.   Define
$\phi:\Cal E\to\Cal P\Bbb N$ by setting

\Centerline{$\phi(E)
=\bigcap\{k_i\cup\pi V_i:i\in\Bbb N$, $E\subseteq V_i\}$.}

\noindent Then $\phi$ is a Tukey function from $\Cal E$ to $\Cal Z$.

\medskip

\Prf\ {\bf (i)} If $E\in\Cal E$ and $\epsilon>0$ there is a $U\in\Cal U$
such that $U\subseteq X\setminus E$ and $\mu U\ge 1-\epsilon$.
Let $i\in\Bbb N$ be such that $X\setminus U=V_i$;  then
$\phi(E)\subseteq k_i\cup\pi V_i$, so

\Centerline{$d^*(\phi(E))\le d^*(\pi V_i)=\mu V_i\le\epsilon$.}

\noindent As $\epsilon$ is arbitrary, $\phi(E)\in\Cal Z$.   Thus $\phi$
is a function from $\Cal E$ to $\Cal Z$.

\medskip

\quad{\bf (ii)} Take any $\Cal A\subseteq\Cal E$, and set
$F=\overline{\bigcup\Cal A}$, $a=\bigcup_{E\in\Cal E}\phi(E)$.
If $n\in\Bbb N$ and $i\in k_n\setminus a$, then $i\notin\phi(E)$ for
every $E\in\Cal A$, so for every $E\in\Cal A$ there is a $j<n$ such
that $E\subseteq V_j$ and $i\notin\pi V_j$.
Set $F_{ni}=\bigcup\{V_j:j<n$, $i\notin\pi V_j\}$, so that $i\notin\pi
F_{ni}$, while $\bigcup\Cal A\subseteq F_{ni}$ and $F\subseteq F_{ni}$.
Set $F_n=\Bbb N\cap\bigcap_{i\in k_n\setminus a}F_{ni}$,
so that $F\subseteq F_n$
and no member of $k_n\setminus a$ belongs fo $\pi F_n$, that is,
$k_n\cap\pi F_n\subseteq a$.   Note that
$F_n\in\frak B_n$.   So we have

\Centerline{$\Bover1{k_n}\#(k_n\cap a)\ge\Bover1{k_n}\#(k_n\cap\pi
F_n)\ge\mu F_n-2^{-n}\ge\mu F-2^{-n}$.}

\noindent This means that $d^*(a)\ge\mu F$.   So if $\{\phi(E):E\in\Cal
A\}$ is bounded above in $\Cal Z$, $\Cal A$ must be bounded above in
$\Cal E$;  that is, $\phi$ is a
Tukey function.\ \Qed

Thus $\Cal E\prT\Cal Z$ also.
}%end of proof of 526I

\leader{526J}{Proposition} Let $\Cal E_{\text{Leb}}$ be the ideal of
subsets of $\Bbb R$ whose closures are Lebesgue negligible.
Then $\NN\prT\Cal E_{\text{Leb}}$ but $\Cal E_{\text{Leb}}\not\prT\NN$;
consequently $\Cal Z\not\prT\NN$, $\CalNwd\not\prT\NN$ and
$\ell^1\not\prT\NN$.

\proof{{\bf (a)} Enumerate $\Bbb Q\cap[0,1]$ as $\sequence{i}{q_i}$.
Define $\phi:\NN\to\Cal E_{\text{Leb}}$ by setting
$\phi(f)(n)=\{n+q_i:n\in\Bbb N$, $i\le f(n)\}$.   Then it is
easy to see that $\phi$ is a Tukey function, because if $F\subseteq\NN$
and $\{f(n):f\in F\}$ is unbounded, then $\bigcup_{f\in F}\phi(f)$ is
dense in $[n,n+1]$ so
does not belong to $\Cal E_{\text{Leb}}$.

\medskip

{\bf (b)} Let $\psi:\Cal E_{\text{Leb}}\to\NN$ be any function.   Let
$\mu$ be Lebesgue measure on $\Bbb R$, and choose $\sequencen{f(n)}$
inductively in $\Bbb N$ such that
$\mu^*\{t:t\in[0,1]$, $\psi(\{t\})(i)\le f(i)$ for every $i\le
n\}>\bover12$ for every $n$.   Set

\Centerline{$A_n=\{t:t\in[0,1]$, $\psi(\{t\})(i)\le f(i)$ for every
$i\le n\}$,
\quad$F=\bigcap_{n\in\Bbb N}\overline{A_n}$}

\noindent so that $\mu F\ge\bover12$.   Let $\sequencen{U_n}$ enumerate
the set of open intervals of $\Bbb R$, meeting $F$,
with rational endpoints,
and for each $n\in\Bbb N$ choose $t_n\in A_n\cap U_n$.   Then
$\psi(\{t_n\})(i)\le f(i)$ whenever $n\ge i$, so
$\{\psi(\{t_n\}):n\in\Bbb N\}$ is bounded above in $\NN$;
but $\overline{\{t_n:n\in\Bbb N\}}$ includes $F$, so
$\{\{t_n\}:n\in\Bbb N\}$ is not bounded above in $\Cal E_{\text{Leb}}$.
Thus $\psi$ cannot be a Tukey function.

\medskip

{\bf (c)} Accordingly $\Cal E_{\text{Leb}}\not\prT\NN$;  since
$\Cal E_{\text{Leb}}\prT\Cal Z\prT\ell^1$ and
$\Cal E_{\text{Leb}}\prT\CalNwd$ (526I, 526B), $\Cal Z\not\prT\NN$,
$\CalNwd\not\prT\NN$ and $\ell^1\not\prT\NN$.
}%end of proof of 526J

\leader{526K}{Proposition} Let $\CalNwd$ be the ideal of nowhere dense
subsets of $\NN$.   Then
$\Cal Z\not\prT\CalNwd$, so $\Cal Z\not\prT\Cal E_{\text{Leb}}$
and $\ell^1\not\prT\CalNwd$.

\proof{ Let $\phi:\Cal Z\to\CalNwd$ be any function.
Let $\sequencen{U_n}$ enumerate a base for the topology of $\NN$ which
contains $\emptyset$ and is closed under finite
unions.   For each $n\in\Bbb N$, set

\Centerline{$a_n=\{i:i\in\Bbb N$, $\phi(a)\cap U_n\ne\emptyset$ whenever
$i\in a\in\Cal Z\}$.}

Set

\Centerline{$a=\{\min(a_n\setminus n^2):n\in\Bbb N$, $a_n\not\subseteq
n^2\}$}

\noindent (interpreting $n^2$ in the formula above as a member of
$\Bbb N$ rather than as a subset of $\BbbN^2$).
Then $a\in\Cal Z$ and $a\cap a_n\ne\emptyset$ whenever $a_n$ is
infinite.   Set
$K=\{n:n\in\Bbb N$, $\phi(a)\cap U_n=\emptyset\}$, so that $K$ is infinite
and $\bigcup_{n\in K}U_n=\NN\setminus\overline{\phi(a)}$ is dense, while
$a_n$ is finite for every $n\in K$
(since otherwise there is an $i\in a\cap a_n$, and $\phi(a)\cap U_n$
will not be empty).   For $n\in\Bbb N$, $\bigcup_{m\in K\cap n}U_m$
belongs to $\Cal U$;  let
$r(n)\in\Bbb N$ be such that $U_{r(n)}=\bigcup_{m\in K\cap n}U_m$.
Then $r(n)\in K$ for every $n$, so $a_{r(n)}$ is always finite.
Take a strictly increasing sequence $\sequencen{k_n}$ in $K$ such that
$a_{r(n)}\subseteq k_n$ for every $n$.   For $i<k_0$, set
$b_i=\{i\}$;  for
$k_n\le i<k_{n+1}$, choose $b_i\in\Cal Z$ such that $i\in b_i$ and
$\phi(b_i)\cap U_{r(n)}$ is empty (such exists because
$i\notin a_{r(n)}$).

Examine $E=\bigcup_{i\in\Bbb N}\phi(b_i)\subseteq\NN$.   If $m\in K$
then $U_m\subseteq U_{r(n)}$ for every $n>m$ so
$U_m\cap\phi(b_i)=\emptyset$ for every $i\ge k_{m+1}$ and
$U_m\cap E=\bigcup_{i<k_{m+1}}U_m\cap\phi(b_i)$ is nowhere dense.   As
$\bigcup_{m\in K}U_m$ is a dense open set, $E$ is nowhere dense.   On
the other hand,
$\bigcup_{i\in\Bbb N}b_i=\Bbb N$.   So $\{b:b\in\Cal Z$,
$\phi(b)\subseteq E\}$ is not bounded above in $\Cal Z$, and $\phi$
cannot be a Tukey function.   As $\phi$ is arbitrary,
$\Cal Z\not\prT\CalNwd$.

Because $\Cal E_{\text{Leb}}\prT\CalNwd$ (526I) and $\Cal Z\prT\ell^1$
(526B), it follows that $\Cal Z\not\prT\Cal E_{\text{Leb}}$
and $\ell^1\not\prT\CalNwd$.
}%end of proof of 526K

\leader{526L}{Proposition}\cmmnt{ ({\smc M\'atrai p09})}
$\CalNwd\not\prT\Cal Z$, so $\CalNwd\not\prT\Cal E_{\text{Leb}}$
and $\ell^1\not\prT\Cal Z$.
%M\'atrai 29.6.09

\proof{{\bf (a)(i)} Fix a non-empty
zero-dimensional compact metrizable space $X$ without isolated
points, and write $\CalNwd(X)$ for the ideal of nowhere dense subsets of
$X$;  the bulk of the argument here will be a proof that
$\CalNwd(X)\not\prT\Cal Z$.
Let $\Cal V$ be the family of non-empty open-and-closed subsets of $X$.
For $V\in\Cal V$ write
$\CalNwd(V)=\CalNwd(X)\cap\Cal PV$ for the family of nowhere
dense subsets of $V$.
As in 526A, set
$\nu I=\sup_{n\ge 1}\Bover1n\#(I\cap n)$ for $I\subseteq\Bbb N$.
Take any function $f:\CalNwd(X)\to\Cal Z$.

\medskip

\quad{\bf (ii)} Let $Q$ be the set of pairs $\sigma=(m_{\sigma},I_{\sigma})$
where $I_{\sigma}\subseteq m_{\sigma}\in\Bbb N$;  for $\sigma$, $\tau\in Q$,
say that $\sigma\le\tau$ if either $\sigma=\tau$ or
$2m_{\sigma}\le m_{\tau}$ and $I_{\sigma}=m_{\sigma}\cap I_{\tau}$.   Then
$(Q,\le)$ is a partially ordered set.   For
$\sigma\in Q$ and $\epsilon>0$, let $D(\sigma,\epsilon)$ be the set
of those $E\subseteq X$ for which there is an
$F\in\CalNwd(X)$, including $E$, such that
$f(F)\cap m_{\sigma}\subseteq I_{\sigma}$ and
$\nu(f(F)\setminus I_{\sigma})\le\epsilon$.

\medskip

\quad{\bf (iii)} If $\sigma\in Q$, $\epsilon>0$ and $k\ge 2m_{\sigma}$,
then

\Centerline{$D(\sigma,\epsilon)
\subseteq\bigcup\{D(\tau,\epsilon):\sigma\le\tau\in Q$,
$m_{\tau}=k$, $\nu(I_{\tau}\setminus I_{\sigma})\le\epsilon\}$.}

\noindent\Prf\ If $E\in D(\sigma,\epsilon)$, let $F\in\CalNwd(X)$ be such
that $E\subseteq F$, $f(F)\cap m_{\sigma}\subseteq I_{\sigma}$ and
$\nu(f(F)\setminus I_{\sigma})\le\epsilon$.   Set
$\tau=(k,I_{\sigma}\cup(k\cap f(F))$;  then $\sigma\le\tau$ and $F$
witnesses that $E\in D(\tau,\epsilon)$, while
$\nu(I_{\tau}\setminus I_{\sigma})\le\epsilon$.\ \Qed

\medskip

\quad{\bf (iv)} If $\sigma\in Q$ and $\epsilon$, $\delta>0$, then

\Centerline{$D(\sigma,\epsilon)
  \subseteq\bigcup\{D(\tau,\delta):\sigma\le\tau\in Q$,
$\nu(I_{\tau}\setminus I_{\sigma})\le\epsilon\}$.}

\noindent\Prf\ If $E\in D(\sigma,\epsilon)$, let
$F\in\CalNwd(X)$ be
such that $E\subseteq F$, $F\cap m_{\sigma}\subseteq I_{\sigma}$ and
$\nu(f(F)\setminus I_{\sigma})\le\epsilon$.   As
$f(F)\in\Cal Z$, there is a $k\ge 2m_{\sigma}$ such that
$\nu(f(F)\setminus k)\le\delta$.   Set
$\tau=(k,I_{\sigma}\cup(k\cap f(F))$;  then
$F$ witnesses that $E\in D(\tau,\delta)$, while
$\nu(I_{\tau}\setminus I_{\sigma})\le\epsilon$.\ \Qed

\medskip

\quad{\bf (v)} Suppose that $n\ge 1$ and that
$\langle\sigma_j\rangle_{j\le n}$, $\langle\tau_j\rangle_{j\le n}$ are
finite sequences in $Q$ such
that $m_{\tau_j}\le m_{\sigma_j}$ for $j\le n$ and
$\sigma_j\le\tau_{j+1}$ for $j<n$.    Then
$\nu(\bigcup_{j<n}I_{\tau_{j+1}}\setminus I_{\sigma_j})
\le 3\max_{j<n}\nu(I_{\tau_{j+1}}\setminus I_{\sigma_j})$.
\Prf\ Note first that we certainly have
$m_{\sigma_j}\le m_{\tau_{j+1}}\le m_{\sigma_{j+1}}$ for every $j<n$.
Set $K=\bigcup_{j<n}I_{\tau_{j+1}}\setminus I_{\sigma_j}$ and
$\epsilon=\max_{j<n}\nu(I_{\tau_{j+1}}\setminus I_{\sigma_j})$.
If $m\in\Bbb N$, set $J=\{j:j<n$, $\sigma_j\ne\tau_{j+1}$,
$m_{\sigma_j}\le m\}$,
$J'=\{j:j\in J$, $m_{\tau_{j+1}}\le m\}$.   Then

$$\eqalignno{\#(m\cap K)
&\le\sum_{j\in J}\#(m\cap I_{\tau_{j+1}}\setminus I_{\sigma_j})\cr
\displaycause{because if $j<n$ and $m\le m_{\sigma_j}$, then
$m\cap I_{\tau_{j+1}}\setminus I_{\sigma_j}=\emptyset$}
&\le\epsilon m+\sum_{j\in J'}\#(I_{\tau_{j+1}}\setminus I_{\sigma_j})\cr
\displaycause{because $\#(J\setminus J')\le 1$}
&\le\epsilon(m+\sum_{j\in J'}m_{\tau_{j+1}})
\le\epsilon(m+2m)\cr
\displaycause{because
$2m_{\tau_{j+1}}\le 2m_{\sigma_{j+1}}
\le 2m_{\sigma_{j'}}\le m_{\tau_{j'+1}}\le m$
whenever $j$, $j'$ are successive members of
$J'$}
&=3\epsilon m.\cr}$$

\noindent As $m$ is arbitrary, $\nu K\le 3\epsilon$.\ \Qed

\medskip

{\bf (b)(i)} Suppose that $V\in\Cal V$ and that
$\Cal C_0,\ldots,\Cal C_n\subseteq\CalNwd(X)$ are such that
every nowhere dense subset of $V$ is included in some member of
$\bigcup_{i\le n}\Cal C_i$.   Then there is an $i\le n$ such that every
nowhere dense subset of $V$ is included in some member of $\Cal C_i$.
\Prf\Quer\ Otherwise, for each $i\le n$ we can find a nowhere dense
subset $E_i$ of $V$ not included in any member of $\Cal C_i$;  now
$E=\bigcup_{i\le n}E_i$ is a nowhere dense subset of $V$ not included in
any member of $\bigcup_{i\le n}\Cal C_i$.\  \Bang\Qed

\medskip

\quad{\bf (ii)} Suppose that $V\in\Cal V$ and that $\sequencen{\Cal C_n}$
is a sequence of subsets of $\CalNwd(X)$ such that
every nowhere dense subset of $V$ is included in some member of
$\bigcup_{n\in\Bbb N}\Cal C_n$.   Then for any $E\in\CalNwd(V)$ there are
a $U\in\Cal V$ and an $n\in\Bbb N$ such that $E\subseteq U$ and every
nowhere dense subset of $U$ is included in some member of $\Cal C_n$.
\Prf\ As $V\ne\emptyset$ we can suppose that $E\ne\emptyset$.
Let $\sequencen{U_n}$ be a non-increasing sequence in $\Cal V$ such
that $U_0=V$ and $\bigcap_{n\in\Bbb N}U_n=\overline{E}$.   \Quer\ If, for
every $n\in\Bbb N$, there is an $E_n\in\CalNwd(U_n)$ not included in any
member of $\Cal C_n$, consider $F=\bigcup_{n\in\Bbb N}E_n$;  then
$F\in\CalNwd(V)$ but $F$ is not included in any member of any $\Cal C_n$.\
\Bang\  So some $U_n$ will serve.\ \Qed

\medskip

{\bf (c)} (The key.)  Suppose that $V\in\Cal V$, $\sigma\in Q$
and $\epsilon>0$ are such that $\CalNwd(V)\subseteq D(\sigma,\epsilon)$.
Let $\sequencen{\epsilon_n}$ be a sequence in $\ooint{0,\infty}$.
Then there are an
$n\in\Bbb N$, $U_0,\ldots,U_n\in\Cal V$ and $\tau\in Q$ such
that

\Centerline{$\sigma\le\tau$,
\quad$\nu(I_{\tau}\setminus I_{\sigma})\le 8\epsilon$,}

\Centerline{$V\subseteq\bigcup_{j\le n}U_j$,
\quad$\CalNwd(U_j)\subseteq D(\tau,\epsilon_j)$ for every $j\le n$.}

\noindent\Prf\ It is enough to consider the case in which
$\sum_{n=0}^{\infty}\epsilon_n\le\epsilon$.
Let $\sequencen{x_n}$ run over a dense subset of $V$.
Choose $\sequencen{\sigma_n}$, $\sequencen{k_n}$,
$\langle U_n\rangle_{n\ge 1}$ and $\langle\tau_n\rangle_{n\ge 1}$
inductively, as follows.   Start with
$\sigma_0=\sigma$, $k_0=m_{\sigma_0}$.
Given that $\CalNwd(V)\subseteq D(\sigma_n,\epsilon)$,
we know from (a-iv) that

\Centerline{$\CalNwd(V)\subseteq D(\sigma_n,\epsilon)
\subseteq\bigcup\{D(\tau,\epsilon_{n+1}):\sigma_n\le\tau\in Q$,
   $\nu(I_{\tau}\setminus I_{\sigma_n})\le\epsilon\}$,}

\noindent so by (b-ii) we can find a $U_{n+1}\in\Cal V$ and a
$\tau_{n+1}\ge\sigma_n$ such that $x_n\in U_{n+1}$,
$\nu(I_{\tau_{n+1}}\setminus I_{\sigma_n})\le\epsilon$ and
$\CalNwd(U_{n+1})\subseteq D(\tau_{n+1},\epsilon_{n+1})$.
Next, taking $k_{n+1}=\max(m_{\tau_{n+1}},2m_{\sigma_n})$,
(a-iii) tells us that

\Centerline{$\CalNwd(V)\subseteq D(\sigma_n,\epsilon)
\subseteq\bigcup\{D(\tau,\epsilon):\sigma_n\le\tau\in Q$,
  $m_{\tau}=k_{n+1}$, $\nu(I_{\tau}\setminus I_{\sigma_n})\le\epsilon\}$,}

\noindent so from (b-i) we see that there is a $\sigma_{n+1}\in Q$ such
that $\CalNwd(V)\subseteq D(\sigma_{n+1},\epsilon)$,
$m_{\sigma_{n+1}}=k_{n+1}$, $\sigma_n\le\sigma_{n+1}$ and
$\nu(I_{\sigma_{n+1}}\setminus I_{\sigma_n})\le\epsilon$.   Continue.

At the end of the induction, set $E=V\setminus\bigcup_{n\in\Bbb N}U_{n+1}$.
Because $\{x_n:n\in\Bbb N\}$ is dense in $V$, so is
$\bigcup_{n\in\Bbb N}U_{n+1}$, and $E\in\CalNwd(V)$.   By (a-iv) and
(b-ii) again, there are a $U_0\in\Cal V$ and a $\tau_0\ge\sigma$ such
that $E\subseteq U_0$, $\CalNwd(U_0)\subseteq D(\tau_0,\epsilon_0)$ and
$\nu(I_{\tau_0}\setminus I_{\sigma})\le\epsilon$.   Now
$V\subseteq\bigcup_{n\in\Bbb N}U_n$;  since $V$ is compact, there is an
$n\in\Bbb N$ such that $V\subseteq\bigcup_{j\le n}U_j$.

I have still to define $\tau$.   Set $k=2\max(k_n,m_{\tau_0})$.   For each
$j\le n$, (a-iii) and (b-i), as before, show us that there is an
$\upsilon_j\in Q$ such that $\tau_j\le\upsilon_j$,
$m_{\upsilon_j}=k$,
$\nu(I_{\upsilon_j}\setminus I_{\tau_j})\le\epsilon_j$ and
$\CalNwd(U_j)\subseteq D(\upsilon_j,\epsilon_j)$.
Try setting $\tau=(k,\bigcup_{j\le n}I_{\upsilon_j})$.
Then surely $\CalNwd(U_j)\subseteq D(\tau,\epsilon_j)$ for each $j$.
To estimate $\nu(I_{\tau}\setminus I_{\sigma})$, set
$K=\bigcup_{j<n}I_{\tau_{j+1}}\setminus I_{\sigma_j}$,
$K'=\bigcup_{j<n}I_{\sigma_{j+1}}\setminus I_{\sigma_j}$.   By (a-v),
$\nu K$ and $\nu K'$ are both at most $3\epsilon$.   Now

$$\eqalign{I_{\tau}\setminus I_{\sigma}
&\subseteq\bigcup_{j\le n}(I_{\upsilon_j}\setminus I_{\tau_j})
   \cup(I_{\tau_0}\setminus I_{\sigma})\cr
&\mskip100mu
   \cup\bigcup_{j<n}(I_{\tau_{j+1}}\setminus I_{\sigma_j})
   \cup\bigcup_{j\le n}(I_{\sigma_j}\setminus I_{\sigma})\cr
&=\bigcup_{j\le n}(I_{\upsilon_j}\setminus I_{\tau_j})
   \cup(I_{\tau_0}\setminus I_{\sigma})
   \cup K\cup K',\cr}$$

\noindent and

$$\eqalign{\nu(I_{\tau}\setminus I_{\sigma})
&\le\sum_{j=0}^n\nu(I_{\upsilon_j}\setminus I_{\tau_j})
  +\nu(I_{\tau_0}\setminus I_{\sigma})
  +\nu K+\nu K'\cr
&\le 7\epsilon+\sum_{j=0}^n\epsilon_j
\le 8\epsilon,\cr}$$

\noindent as required.\ \Qed

\medskip

{\bf (d)} Now we can find $T\subseteq S=\bigcup_{n\in\Bbb N}\BbbN^n$,
$\family{t}{T}{\delta_t}$,
$\family{t}{T}{\sigma_t}$, $\family{t}{T}{\tau_t}$ and
$\family{t}{T}{V_t}$ such that

\inset{$T$ is a tree (that is, $t\restr k\in T$ whenever $t\in T$ and
$k\in\Bbb N$),

$\delta_t>0$ for every $t\in T$, $\sum_{t\in T}\delta_t<\infty$,

$\sigma_t\in Q$, $\tau_t\in Q$, $\sigma_t\le\tau_t$
for every $t\in T$,

$\sigma_t=\tau_{t\restr n}$ whenever $n\in\Bbb N$ and
$t\in T\cap\BbbN^{n+1}$,

$\nu(I_{\tau_t}\setminus I_{\sigma_t})\le\delta_t$ for every
$t\in T$,

$V_t\in\Cal V$, $\CalNwd(V_t)\subseteq D(\sigma_t,\delta_t)$ for every
$t\in T$,

$\bigcup\{V_t:t\in T\cap\BbbN^n\}=X$ for every $n\in\Bbb N$.}

\noindent\Prf\ Begin by choosing strictly positive $\delta_t$, for
$t\in S$, such that $\delta_{\emptyset}=1$
and $\sum_{t\in S}\delta_t$ is finite.
Now choose $T_n\subseteq\BbbN^n$ and
$\family{t}{T_n}{\sigma_t}$, $\family{t}{T_n}{V_t}$
inductively, as follows.
Start with $T_0=\{\emptyset\}$,
$\sigma_{\emptyset}=(0,\emptyset)$ and $V_{\emptyset}=X$.
Then

\Centerline{$\CalNwd(V_{\emptyset})=\CalNwd(X)=D((0,\emptyset),1)
=D(\sigma_{\emptyset},\delta_{\emptyset})$,}

\noindent so the process starts.
Given that $T_n$, $\family{t}{T_n}{\sigma_t}$ and
$\family{t}{T_n}{V_t}$
have been defined, then for each $t\in T_n$ use (c) to find $n_t\in\Bbb N$,
$\langle V_{t^{\smallfrown}\fraction{i}}\rangle_{i\le n_t}
\in\Cal V^{n_t+1}$
and $\tau_t\in Q$ such that $\sigma_t\le\tau_t$,
$\nu(I_{\tau_t}\setminus I_{\sigma_t})\le\delta_t$,
$V_t\subseteq\bigcup_{i\le n_t}V_{t^{\smallfrown}\fraction{i}}$
and
$\CalNwd(V_{t^{\smallfrown}\fraction{i}})
\subseteq D({\tau_t},\delta_{t^{\smallfrown}\fraction{i}})$ for every
$i\le n_t$.   Set
$T_{n+1}=\{t^{\smallfrown}\fraction{i}:t\in T_n$, $i\le n_t\}$ and
$\sigma_t=\tau_{t\restr n}$ for every $t\in T_{n+1}$, and continue.
At the end of the construction, set $T=\bigcup_{n\in\Bbb N}T_n$.\ \Qed

\medskip

{\bf (e)} Let $\sequencen{y_n}$ run over a dense subset of $X$.   For
$n\in\Bbb N$, take $t_n\in T\cap\BbbN^n$ such that
$y_n\in V_{t_n}$.   Since
$\CalNwd(V_{t_n})\subseteq D(\sigma_{t_n},\delta_{t_n})$, we can
choose an $F_n\in\CalNwd(X)$, containing $y_n$, such that
$\nu(f(F_n)\setminus I_{\sigma_{t_n}})\le\delta_{t_n}$.
Now $\{f(F_n):n\in\Bbb N\}$ is bounded above in $\Cal Z$.   \Prf\ Set
$K=\bigcup_{t\in T}I_{\sigma_t}$.   As $I_{\sigma_{\emptyset}}=\emptyset$,
\Centerline{$K
=\bigcup_{n\in\Bbb N}\bigcup_{t\in T\cap\BbbN^{n+1}}
  I_{\sigma_t}\setminus I_{\sigma_{t\restr n}}
=\bigcup_{t\in T}I_{\tau_t}\setminus I_{\sigma_t}$;}

\noindent as $\sum_{t\in T}\nu(I_{\tau_t}\setminus I_{\sigma_t})$ is
finite, $K\in\Cal Z$ (526Ac).   Next,

\Centerline{$\bigcup_{n\in\Bbb N}f(F_n)\setminus K
\subseteq\bigcup_{n\in\Bbb N}f(F_n)\setminus I_{\sigma_{t_n}}$;}

\noindent as

\Centerline{$\sum_{n=0}^{\infty}\nu(f(F_n)\setminus I_{\sigma_{t_n}})
\le\sum_{n=0}^{\infty}\delta_{t_n}$}

\noindent is finite, $\bigcup_{n\in\Bbb N}f(F_n)\setminus K\in\Cal Z$,
so $\bigcup_{n\in\Bbb N}f(F_n)$ also belongs to $\Cal Z$, and is an upper
bound for $\{f(F_n):n\in\Bbb N\}$.\ \Qed

\medskip

{\bf (f)}
On the other hand, $\{F_n:n\in\Bbb N\}$ is certainly not bounded above in
$\CalNwd(X)$, since $\bigcup_{n\in\Bbb N}F_n$ includes the dense set
$\{y_n:n\in\Bbb N\}$.   So $f$ cannot be a Tukey function.
Since $f$ is arbitrary, $\CalNwd(X)\not\prT\Cal Z$.

\medskip

{\bf (g)} Since $\CalNwd(X)\equivT\CalNwd$ (526He), it follows that
$\CalNwd\not\prT\Cal Z$.
Since $\NN\prT\Cal E_{\text{Leb}}\penalty-100\prT\Cal Z$ (526I, 526J)
and $\CalNwd\prT\ell^1$ (526Hc),
we see that
$\CalNwd\not\prT\Cal E_{\text{Leb}}$ and
$\ell^1\not\prT\Cal Z$.
}%end of proof of 526L

\cmmnt{\medskip

\noindent{\bf Remark} A somewhat stronger result is in
{\smc Solecki \& Todor\v{c}evi\'c 10}.}

\leader{526M}{}\cmmnt{ Having introduced ideals of sets with
negligible closures, I add a simple result which will be useful later.

\medskip

\noindent}{\bf Proposition} Let $X$ be a second-countable topological
space and $\mu$ a $\sigma$-finite topological measure on $X$.
Let $\Cal E$ be the ideal of subsets of $X$ with negligible closures,
$\Cal N(\mu)$ the null ideal of $\mu$, and $\Cal M$ the ideal of
meager subsets of $\NN$.   Then

\Centerline{$(\Cal E,\subseteq,\Cal N(\mu))\prGT(\Cal M,\not\ni,\NN)$;}

\noindent consequently $\add(\Cal E,\subseteq,\Cal N(\mu))\ge\frakmctbl$.

\proof{{\bf (a)} Suppose first that $\mu$ is a probability measure.
Let $\Cal U$ be a countable base for the topology of $X$, containing
$\emptyset$ and closed under finite unions.   For each $n\in\Bbb N$,
let $\sequence{i}{U_{ni}}$ run over
$\{U:U\in\Cal U$, $\mu U\ge 1-2^{-n}\}$.   For $f\in\NN$, set

\Centerline{$\psi(f)
=\bigcap_{n\in\Bbb N}\bigcup_{i\ge n}X\setminus U_{i,f(i)}
\in\Cal N(\mu)$.}

\noindent For $E\in\Cal E$, set

\Centerline{$\phi(E)=\{f:f\in\NN$, $E\not\subseteq\psi(f)\}$.}

\noindent Then $\phi(E)\in\Cal M$.   \Prf\ Since
$X\setminus\overline{E}$ is a conegligible open set, we can find for
each $i\in\Bbb N$ a $g(i)\in\Bbb N$ such that
$E\cap U_{i,g(i)}=\emptyset$.   Now

\Centerline{$M
=\bigcup_{n\in\Bbb N}\bigcap_{i\ge n}\{f:f\in\NN$, $f(i)\ne g(i)\}$}

\noindent belongs to $\Cal M$.   If $f\in\phi(E)$, there is an
$x\in E\setminus\psi(f)$, so that $x\in\bigcap_{i\ge n}U_{i,f(i)}$ for
some $n$;   now if $i\ge n$, we have
$x\in U_{i,f(i)}\setminus U_{i,g(i)}$, so $f(i)\ne g(i)$;  thus
$f\in M$.   Accordingly $\phi(E)\subseteq M\in\Cal M$.\ \Qed

Now $(\phi,\psi)$ is a Galois-Tukey connection from
$(\Cal E,\subseteq,\Cal N(\mu))$ to $(\Cal M,\not\ni,\NN)$, and
$(\Cal E,\subseteq,\penalty-100\Cal N(\mu))\prGT(\Cal M,\not\ni,\NN)$.

\medskip

{\bf (b)} If $\mu X=0$, then of course
$(\Cal E,\subseteq,\Cal N(\mu))\prGT(\Cal M,\not\ni,\NN)$ (take
$\phi(E)=\emptyset$ for every $E\in\Cal E$, $\psi(f)=X$ for every
$f\in\NN$).   Otherwise, there is a probability measure $\nu$ on $X$
with the same domain and the same null ideal as $\mu$, so (a) tells us
that $(\Cal E,\subseteq,\Cal N(\mu))\prGT(\Cal M,\not\ni,\NN)$.

\medskip

{\bf (c)} Accordingly

\Centerline{$\add(\Cal E,\subseteq,\Cal N(\mu))
\ge\add(\Cal M,\not\ni,\NN)=\cov\Cal M$}

\noindent (512Db).    But, writing $\Cal M(\Bbb R)$ for the ideal of
meager subsets of $\Bbb R$, $\cov\Cal M=\cov\Cal M(\Bbb R)=\frakmctbl$,
by 522Wb and 522Sa.
}%end of proof of 526M

\cmmnt{\medskip

\noindent{\bf Remark} If $X=\Bbb R$ and $\mu$ is Lebesgue measure, then
$\add(\Cal E,\subseteq,\Cal N(\mu))=\frakmctbl$ and
$\cov(\Cal E,\subseteq,\Cal N(\mu))=\non\Cal M$;   see
{\smc Bartoszy\'nski \& Shelah 92} or
{\smc Bartoszy\'nski \& Judah 95}, 2.6.14.
}%end of comment

\exercises{\leader{526X}{Basic exercises (a)}
%\spheader 526Xa
Let $\nu:\Cal P\Bbb N\to[0,1]$ be the submeasure
described in 526A.   Show that
$d^*(I)=\lim_{n\to\infty}\nu(I\setminus n)$ for every
$I\subseteq\Bbb N$.
%526A

\spheader 526Xb
For $I$, $J\subseteq\Bbb N$ say that $I\subseteq^*J$ if $I\setminus J$
is finite.
Show that $(\Cal Z,\subseteq^*,\Cal Z)
\equivGT(\Cal
Z,\discretionary{}{}{}\subseteq^{\strprime},\discretionary{}{}{}[\Cal
Z]^{\le\omega})$.
%526B

\spheader 526Xc Let $\CalNwd$ be the ideal of nowhere dense subsets of
$\NN$ and $\Cal M$ the ideal of meager subsets of $\NN$.   Show that
$\add_{\omega}\CalNwd=\add\Cal M$,
$\non\CalNwd=\non\Cal M$, $\cov\CalNwd=\frakmctbl$ and
$\cf\CalNwd=\cf\Cal M$.
%526H

\spheader 526Xd Let $X$ be a topological space with a countable
$\pi$-base, and $\CalNwd(X)$ the ideal of nowhere dense subsets of $X$.
Show that $\CalNwd(X)\prT\CalNwd$,
where $\CalNwd$ is the ideal of nowhere dense subsets of $\NN$.
%526H

\spheader 526Xe In 526Hf, show that
$\Cal C_{\text{nwd}}$ is a G$_{\delta}$ subset of the family $\Cal C$
of all closed subsets of $X$ with its Fell topology, so is a Polish
space in the subspace topology.
%526H

\spheader 526Xf Let $\Cal C_{\text{Leb}}$ be the family of closed
Lebesgue negligible subsets of $[0,1]$.   Show that $\Cal C_{\text{Leb}}$
with its Fell topology
is a Polish space and a metrizably compactly based directed set.
%526I

\spheader 526Xg Let $\Cal E_{\text{Leb}}$ be the ideal of subsets of
$\Bbb R$ with
negligible closures.   (i) Show that it is Tukey equivalent to the
partially ordered
set $\Cal C_{\text{Leb}}$ of 526Xf.   (ii) Show that it is isomorphic to
$\Cal E_{\text{Leb}}^{\Bbb N}$.   (iii) Show that if we write
$\Cal E_{\sigma}$ for the
$\sigma$-ideal of subsets of $\Bbb R$ generated by
$\Cal E_{\text{Leb}}$, then
$(\Cal E_{\text{Leb}},\subseteq^{\strprime},
[\Cal E_{\text{Leb}}]^{\le\omega})
\equivGT(\Cal E_{\sigma},\subseteq,\Cal E_{\sigma})$.
(iv) Show that $\add_{\omega}\Cal E_{\text{Leb}}=\add\Cal E_{\sigma}$
and $\cf\Cal E_{\text{Leb}}=\cf\Cal E_{\sigma}$.

\leader{526Y}{Further exercises (a)}
%\spheader 526Ya
Let $X$ be a locally compact separable
metrizable space.   Let $\Cal C_{\text{nwd}}$ be the
family of closed nowhere dense sets in $X$ with its Fell topology.   Show
that $\Cal C_{\text{nwd}}$ is a metrizably compactly based directed set.
%526H

\spheader 526Yb Let $\frak Z$ be the asymptotic density
algebra $\Cal P\Bbb N/\Cal Z$
and define $\bar d^*:\frak Z\to[0,1]$ by setting
$\bar d^*(I^{\ssbullet})=d^*(I)$ for every $I\subseteq\Bbb N$, as in
491I.   Show that if $A\subseteq\frak Z$ is non-empty, downwards-directed
and has infimum $0$, and $\#(A)<\frak p$, then
$\inf_{a\in A}\bar d^*(a)=0$.   (Compare 491Id.)
%n99527

\spheader 526Yc Show that $\wdistr(\frak Z)=\omega_1$.
%n99527, 3C

\spheader 526Yd Show that
$\frak m(\frak Z)\ge\frak m_{\sigma\text{-linked}}$.
%n99527, 3D

\spheader 526Ye Show that $\FN(\Cal P\Bbb N)\le\FN(\frak Z)
\le\max(\FN^*(\Cal P\Bbb N),(\cf\Cal N)^+)$.
%n99527, 3E

\spheader 526Yf Show that $\tau(\frak Z)\ge\frak p$.
%n99527, 3A
}%end of exercises

\endnotes{
\Notesheader{526} The `positive' results of this section are
straightforward enough, except perhaps for 526F.   As elsewhere in this
chapter, I am attempting to
describe a framework which will accommodate the many arguments which
have been found effective in discussing the cardinal functions of these
partially ordered sets.
I note that in this section I use the symbol $\Cal M$ to represent the
ideal of meager subsets of $\NN$, rather than the ideal of meager
subsets of $\Bbb R$, as
elsewhere in the chapter.   If you miss this point, however, none of the
formulae here are dangerous, because the two ideals are Tukey
equivalent, and indeed isomorphic (522Wb).

When we come to `negative' results, we have problems of a new kind.
The special character of Tukey functions is that they need not be of any
particular type.   They are not asked to be order-preserving, and
even if we have partially ordered sets with natural Polish
topologies (as in 526A, 526Xe and 526Xf, for instance),
Tukey functions between them are not
required to be Borel functions.   This means that
in order to show that there is {\it no} Tukey function between a given
pair of partially ordered sets, we have had to consider arbitrary
functions, or seek to calculate suitable
invariants which we know to be related to the Tukey ordering, like
precaliber triples (516C), and show that they are incompatible with the
existence of a Tukey function.   For a discussion of a class of invariants
giving very sharp distinctions, see {\smc M\'atrai p09}, \S3.

Putting 526B and 526H-526L %526Hc, 526I, 526J, 526K, 526L
together, we find that we have a complete description of the Tukey ordering
on the set $\{\NN,\Cal E_{\text{Leb}},\CalNwd,\Cal Z,\ell^1\}$, given by
the diagram

\def\tmphrule{\hskip0.4em\raise
2.5pt\hbox{\leaders\hrule\hskip1.7em\hfil}\hskip0.4em}
\def\tmpstrut{\vrule height10.5pt depth5.5pt width0pt}

$$\vbox{\offinterlineskip
\halign{\hfil#\hfil&\hfil#\hfil&\hfil#\hfil&\hfil#\hfil
  &\hfil#\hfil&\hfil#\hfil&\hfil#\hfil
  &\hfil#\hfil&\hfil#\hfil&\hfil#\hfil&\hfil#\hfil\cr
&\tmpstrut&\tmpstrut&$\CalNwd$&\tmphrule&$\ell^1$\cr
&\tmpstrut&\tmpstrut&\vrule&&\vrule\cr
&\tmpstrut$\NN$&\tmphrule&$\Cal E_{\text{Leb}}$&\tmphrule
  &$\Cal Z$\cr
}}$$

\noindent if we interpret this in the same way as for Cicho\'n's diagram
(522B).   Moreover, this is exact, in that no two of the five are
Tukey equivalent, and $\Cal Z$ and $\CalNwd$ are Tukey incomparable.
Note that all five of these partially ordered sets
are either themselves metrizably compactly
based directed sets (526A, 513Xj, 513Xl) or are Tukey equivalent to
metrizably compactly based directed sets (526He-526Hf, 526Xf-526Xg).

In 526Yb-526Yf %526Yb 526Yc 526Yd 526Ye 526Yf
I list miscellaneous facts about the asymptotic density algebra.   A
remarkable description of its Dedekind completion is in 556S below.
}%end of notes

\discrpage

