\frfilename{mt434.tex}
\versiondate{1.9.04}
\copyrightdate{2000}

\def\Yprod#1{\prod_{i\in#1}Y_i}
\hyphenation{Pfef-fer}

\def\chaptername{Topologies and measures II}
\def\sectionname{Borel measures}

\newsection{434}

What one might call the fundamental question of topological measure
theory is the following.

\Centerline{\it What kinds of measures can arise on what kinds of
topological space?}

\noindent Of course this question has inexhaustible ramifications,
corresponding to all imaginable properties of measures and topologies
and connexions between them.   The challenge I face here is that of
identifying particular ideas as being more important than others, and
the chief difficulty lies in the bewildering variety of topological
properties which have been studied, any of which may have implications
for the measure theory of the spaces involved.   In this section and the
next I give a sample of what is known, necessarily biased and
incomplete.   I try however to include the results which are most often
applied and enough others for the proofs to contain, between them, most
of the non-trivial arguments which have been found effective in this
area.

In 434A I set out a crude classification of Borel measures on
topological spaces.   For compact Hausdorff spaces, at least, the first
question is whether they carry Borel measures which are not, in effect,
Radon measures;  this leads us to the definition of `Radon' space (434C)
which is also of interest in the context of general Hausdorff spaces.
I give a brief account of the properties of Radon spaces (434F, 434Nd).
I look also at two special topics:
`quasi-dyadic' spaces (434O-434Q) and a
construction of Borel product measures by integration of sections
(434R).

In the study of Radon spaces we find ourselves looking at `universally
measurable' subsets of topological spaces (434D-434E).   These are
interesting in themselves, and also interact with constructions
from earlier parts of this treatise (434S-434T).
Three further classes of topological space, defined in terms of the
types of topological measure which they carry, are the
`Borel-measure-compact', `Borel-measure-complete' and `pre-Radon'
spaces;  I discuss them briefly in 434G-434J.   They provide useful
methods for deciding whether Hausdorff spaces are Radon (434K).

\cmmnt{
\vleader{72pt}{434A}{Types of Borel measures} In \S411 I introduced
the following properties which a Borel measure may or may not have:

\qquad(i) inner regularity with respect to closed sets;

\qquad(ii) inner regularity with respect to zero sets;

\qquad(iii) tightness\cmmnt{ (that is, inner regularity with respect
to closed compact sets)};

\qquad(iv) $\tau$-additivity.

\noindent These are of course interrelated.   (ii)$\Rightarrow$(i) just
because zero sets are closed, and (iii)$\Rightarrow$(iv) by 411E;
in a Hausdorff space, (iii)$\Rightarrow$(i);  and for an effectively
locally finite measure on a regular topological space,
(iv)$\Rightarrow$(i) (414Mb).

On a regular Hausdorff space, therefore, we can divide totally finite
Borel measures into four classes:

\inset{(A) measures which are not inner regular with respect to the
closed sets;}

\inset{(B) measures which are inner regular with respect to the closed
sets, but not $\tau$-additive nor tight;}

\inset{(C) measures which are $\tau$-additive and inner regular with
respect to the closed sets, but not inner regular with respect to the
compact sets;}

\inset{(D) tight measures;}

\noindent and each of the classes (B)-(D) can be further subdivided into
those which are completion regular (B$_1$, C$_1$, D$_1$) and those which
are not (B$_0$, C$_0$, D$_0$).   Examples may be found in 434Xf (type
A), 411Q and 439K (type B$_0$), 439J (type B$_1$), 415Xc and 434Xa (type
C$_1$) and 434Xb (type D$_0$), while Lebesgue measure itself is of type
D$_1$, and any direct sum of spaces of types D$_0$ and C$_1$ will have
type C$_0$.   (The space in 439J depends for its construction on
supposing that there is a cardinal which is not measure-free.   It
seems that
no convincing example of a space of class B$_1$, that is, a completion
regular, non-$\tau$-additive Borel probability measure on a completely
regular Hausdorff space, is known which does not depend on some special
axiom beyond ordinary ZFC.   For one of the obstacles to finding such a
space, see 434Q.)

Note that a totally finite Borel measure $\mu$ on a regular Hausdorff
space can be extended to a quasi-Radon measure iff $\mu$ is of class C
or D (415M), and that in this case the quasi-Radon measure must be
just the completion $\hat\mu$ of $\mu$.   $\hat\mu$ will be of the same
type, on the classification here, as $\mu$;  in particular, $\hat\mu$
will be a Radon measure iff $\mu$ is of class D (416F).
}%end of comment

\cmmnt{
\leader{434B}{Compact, analytic and K-analytic spaces} For any
class of topological spaces, we can enquire which of the seven types of
measure described above can be realized by measures on spaces of that
class.   The enquiry is limited only by our enterprise and diligence in
seeking out new classes of topological space.   For the spaces studied
in \S\S432-433, however, we have something worth repeating here.   On a
K-analytic Hausdorff space, a semi-finite Borel measure which is inner
regular with respect to the closed sets is tight (432B, 432D);
consequently classes B and C of 434A
cannot appear, and we are left with only the types A, D$_0$ and D$_1$,
all of which appear on compact Hausdorff spaces (434Xb, 434Xf).   On an
analytic Hausdorff space
we have further simplifications:  every semi-finite Borel measure is
tight (433Ca), and (if $X$ is
regular) every closed set is a zero set (423Db).   Thus on an analytic
regular Hausdorff space only type D$_1$, of the seven types in 434A, can
appear.   (If the topology is not regular, we may also get measures of
type D$_0$;  see 434Ya.)
}%end of comment

\leader{434C}{Radon spaces:  Definition}\cmmnt{ For K-analytic
Hausdorff spaces, therefore, we have a large gap between the `bad'
measures of class A and the `good' measures of class D;  furthermore, we
have an important class of spaces in which type A cannot appear.   It is
natural to enquire further into the spaces in which every (totally
finite) Borel measure is of class D, and (given that no exact
description can be found) we are led, as usual, to a definition.}   A
Hausdorff space $X$ is {\bf Radon} if every totally finite Borel measure
on $X$ is tight.

\leader{434D}{Universally measurable sets}\cmmnt{ Before going farther
with the study of Radon spaces it will be useful to spend a couple of
paragraphs on the following concept.}   Let $X$ be a topological space.

\spheader 434Da I will say that a subset $E$
of $X$ is {\bf universally measurable} (in $X$) if it is measured by the
completion of every totally finite Borel measure on $X$\cmmnt{;  that
is, for every totally finite Borel measure $\mu$ on $X$ there is a Borel
set $F\subseteq X$ such that $E\symmdiff F$ is $\mu$-negligible}.

\spheader 434Db
A subset of $X$ is universally measurable iff it is measured by every
complete locally determined topological measure on $X$.
\prooflet{\Prf\ (i) Suppose that $A\subseteq X$ is universally
measurable and that $\mu$ is a complete locally determined topological
measure on $X$.   Let $F\subseteq X$ be such that $\mu F$ is defined and
finite.   Then we have a totally finite Borel measure $\nu$ on $X$
defined by setting $\nu E=\mu(F\cap E)$ for every Borel set $E\subseteq
X$.   Now there are Borel sets $E$, $B\subseteq X$ such that $A\symmdiff
E\subseteq B$ and $\nu B=0$.   In this case,
$(A\cap F)\symmdiff(E\cap F)\subseteq B\cap F$ and $\mu(B\cap F)=0$, so
that (because $\mu$ is complete) $A\cap F$ is measured by $\mu$.
Because $F$ is arbitrary and $\mu$ is locally determined, $A$ is
measured by $\mu$.   (ii) Suppose that $A\subseteq X$ is measured by
every complete locally determined topological measure on $X$.   Then, in
particular, it is measured by the completion of any totally finite Borel
measure, so is universally measurable.\ \Qed}

\spheader 434Dc The family
$\Sigma_{\text{um}}$ of universally measurable subsets of $X$ is a
$\sigma$-algebra closed under Souslin's operation and including the
Borel $\sigma$-algebra.   \prooflet{(For it is the intersection of the
domains of a family of complete totally finite measures, and all these
are $\sigma$-algebras including the Borel $\sigma$-algebra and closed
under Souslin's operation, by 431A.)}   In particular, Souslin-F sets
are universally measurable, so (if $X$ is Hausdorff) K-analytic and
analytic sets are\cmmnt{ (422Ha, 423C)}.

\spheader 434Dd Note that a function $f:X\to\Bbb R$ is
$\Sigma_{\text{um}}$-measurable iff it is $\mu$-virtually measurable for
every totally finite Borel measure $\mu$ on $X$\cmmnt{ (122Q, 212Fa)}.
Generally, if $Y$ is another topological space, I will say that 
$f:X\to Y$ is {\bf universally measurable} if 
$f^{-1}[H]\in\Sigma_{\text{um}}$
for every open set $H\subseteq Y$\cmmnt{;  that is, if $f$ is
$(\Sigma_{\text{um}},\Cal B(Y))$-measurable, where $\Cal B(Y)$ is the
Borel $\sigma$-algebra of $Y$}.

\spheader 434De In fact, if $f:X\to Y$ is universally measurable, then
it is $(\Sigma_{\text{um}},\Sigma^{(Y)}_{\text{um}})$-measurable, where
$\Sigma^{(Y)}_{\text{um}}$ is the algebra of universally measurable
subsets of $Y$.   \prooflet{\Prf\ Take $F\in\Sigma^{(Y)}_{\text{um}}$
and a totally finite Borel measure $\mu$ on $X$.   If $\hat\mu$ is the
completion of $\mu$, then the image measure $\nu=\hat\mu f^{-1}$ is a
complete totally finite topological measure on $Y$, so measures $F$,
and $f^{-1}[F]\in\dom\hat\mu$.   As $\mu$ is arbitrary,
$f^{-1}[F]\in\Sigma_{\text{um}}$;  as $F$ is arbitrary, $f$
is $(\Sigma_{\text{um}},\Sigma^{(Y)}_{\text{um}})$-measurable.\ \Qed}

\spheader 434Df It follows that if $Z$ is a third topological space
and $f:X\to Y$, $g:Y\to Z$ are universally measurable, then $gf:X\to Z$
is universally measurable.

\leader{434E}{Universally Radon-measurable sets}\cmmnt{ A companion
idea is the following.} Let $X$ be a Hausdorff space.

\spheader 434Ea I will say that a
subset $E$ of $X$ is {\bf universally Radon-measurable} if it is
measured by every Radon measure on $X$.

\spheader 434Eb The family
$\Sigma_{\text{uRm}}$ of universally Radon-measurable subsets of $X$ is
a $\sigma$-algebra closed under Souslin's operation and including the
algebra\cmmnt{ $\Sigma_{\text{um}}$}
of universally measurable subsets of $X$\cmmnt{ (and, {\it a fortiori},
including the Borel $\sigma$-algebra)}.   \prooflet{(Use 434Db and the
idea of 434Dc.)}

\spheader 434Ec If $Y$ is another topological space, I will say that a function
$f:X\to Y$ is {\bf universally
Radon-measurable} if $f^{-1}[H]\in\Sigma_{\text{uRm}}$ for every open set
$H\subseteq Y$.   A function $f:X\to\Bbb R$ is
$\Sigma_{\text{uRm}}$-measurable iff it is $\mu$-virtually measurable
for every totally finite tight Borel measure $\mu$ on $X$.
\cmmnt{(Compare 434Dd.)}

\leader{434F}{Elementary properties of Radon spaces:  Proposition}
Let $X$ be a Hausdorff space.

(a) The following are equiveridical:

\quad(i) $X$ is a Radon space;

\quad(ii) every semi-finite Borel measure on $X$ is tight;

\quad(iii) if $\mu$ is a locally finite Borel measure on $X$, its
c.l.d.\ version $\tilde\mu$ is a Radon measure;

\quad(iv) whenever $\mu$ is a totally finite Borel measure on $X$, and
$G\subseteq X$ is an open set with $\mu G>0$, then there is a compact
set $K\subseteq G$ such that $\mu K>0$;

\quad(v) whenever $\mu$ is a non-zero totally finite Borel measure on
$X$, there is a Radon subspace $Y$ of $X$ such that $\mu^*Y>0$.

(b) If $Y\subseteq X$ is a subspace which is a Radon space in its
induced topology, then $Y$ is universally measurable in $X$.

(c) If $X$ is a Radon space and $Y\subseteq X$,
then $Y$ is Radon iff it is universally measurable in $X$ iff it is
universally Radon-measurable in $X$.
In particular, all Borel subsets and all Souslin-F subsets of $X$ are
Radon spaces.

(d) The family of Radon subspaces of $X$ is closed under Souslin's 
operation and set difference.

\proof{{\bf (a)(i)$\Rightarrow$(ii)} Let $\mu$ be a semi-finite Borel
measure on $X$, $E\subseteq X$ a Borel set and $\gamma<\mu E$.   Because
$\mu$ is semi-finite, there is a Borel set $H$ of finite measure such
that $\mu(E\cap H)>\gamma$.   Set $\nu F=\mu(F\cap H)$ for every Borel
set $F\subseteq X$;  then $\nu$ is a totally finite Borel measure on
$X$, and $\nu E>\gamma$.   Because $X$ is a Radon space, there is a
compact set $K\subseteq E$ such that $\nu K\ge\gamma$, and now $\mu
K\ge\gamma$.   As $\mu$, $E$ and $\gamma$ are arbitrary, (ii) is true.

\medskip

\quad{\bf (ii)$\Rightarrow$(i)} and {\bf (i)$\Rightarrow$(v)} are
trivial.

\medskip

\quad{\bf (v)$\Rightarrow$(iv)} Assume (v), and let $\mu$ be a totally
finite Radon measure on $X$ and $G$ a non-negligible open set.   Set
$\nu E=\mu(E\cap G)$ for every Borel set $E\subseteq X$.   Then $\nu$ is
a non-zero totally finite Borel measure on $X$, so there is a Radon
subspace $Y$ of $X$ such that $\nu^*Y>0$.   The subspace measure $\nu_Y$
on $Y$ is a Borel measure on $Y$, so is tight.   Since
$\nu_Y(Y\setminus G)=\nu(X\setminus G)=0$, $\nu_Y(Y\cap G)>0$ and there
is a compact set $K\subseteq Y\cap G$ such that $\nu_YK>0$.   Now $\mu
K>0$.   As $\mu$ and $G$ are arbitrary, (iv) is true.

\medskip

\quad{\bf not-(i)$\Rightarrow$not-(iv)} If $X$ is not Radon, there is a
totally finite Borel measure $\mu$ on $X$ which is not tight.   By
416F(iii), there is an open set $G\subseteq X$ such that

\Centerline{$\mu G>\sup_{K\subseteq G\text{ is compact}}\mu K=\gamma$}

\noindent say.   Let $\Cal K$ be the family of compact subsets of $G$.
By 215B(v), there is a non-decreasing sequence $\sequencen{K_n}$ in
$\Cal K$ such that $\mu(K\setminus F)=0$ for every $K\in\Cal K$, where
$F=\bigcup_{n\in\Bbb N}K_n$.   Observe that

\Centerline{$\mu F=\lim_{n\to\infty}\mu K_n\le\gamma<\mu G$.}

\noindent Now set
$\nu E=\mu(E\cap G\setminus F)$ for every Borel set $E\subseteq X$.
Then $\nu$ is a
Borel measure on $X$, and $\nu G>0$.   If $K\subseteq G$ is compact,
then $\nu K=\mu(K\setminus F)=0$.   So $\nu$ and $G$ witness that (iii)
is false.

\medskip

\quad{\bf (i)$\Rightarrow$(iii)} The point is that $\tilde\mu$ is tight.
\Prf\ If
$\tilde\mu E>\gamma$, then, because $\tilde\mu$ is semi-finite, there is
a set $E'\subseteq E$ such that $\gamma<\tilde\mu E'<\infty$;  now there
is a Borel set $H\subseteq E'$ such that $\mu H=\tilde\mu E'$ (213Fc).
Setting $\nu F=\mu(H\cap F)$ for every Borel set $F$, $\nu$ is a totally
finite Borel measure on $X$ and $\nu H>\gamma$, so there is a compact
set $K\subseteq H$ such that $\nu K\ge\gamma$.   Since
$\mu K<\infty$, $\tilde\mu K=\mu K\ge\gamma$ (213Fa), while
$K\subseteq E$.   As $E$ and $\gamma$ are arbitrary, $\tilde\mu$ is tight.\ \Qed

On the other hand, every point of $X$ belongs to an open set of finite
measure for $\mu$, which is still of finite measure for $\tilde\mu$
(213Fa again).   So $\tilde\mu$ is locally finite;  since it is surely
complete and locally determined, it is a Radon measure.

\medskip

\quad{\bf (iii)$\Rightarrow$(i)} Assume (iii), and let $\mu$ be a totally
finite Borel measure on $X$.   Then its c.l.d.\ version $\tilde\mu$ is
tight.   But $\tilde\mu$
extends $\mu$ (213Hc), so $\mu$ also is tight.   As $\mu$ is arbitrary,
$X$ is a Radon space.

\medskip

{\bf (b)} Let $\mu$ be a totally finite Borel measure on $X$, and
$\hat\mu$ its completion;  let $\epsilon>0$.   Let $\mu_Y$ be the
subspace measure on $Y$, so that $\mu_Y$ is a totally finite Borel
measure on $Y$, and is tight.
There is a compact set $K\subseteq Y$ such that
$\nu K\ge\mu_YY-\epsilon$.   But this means that

\Centerline{$\mu^*Y=\mu_YY\le\mu_YK+\epsilon
=\mu^*(K\cap Y)+\epsilon=\mu K+\epsilon\le\mu_*Y+\epsilon$.}

\noindent As $\epsilon$ is arbitrary, $\mu_*Y=\mu^*Y$, and $Y$ is
measured by $\hat\mu$ (413Ef);  as $\mu$ is
arbitrary, $Y$ is universally measurable.

\medskip

{\bf (c)} (i) If $Y$ is Radon, it is universally measurable, by
(b).   (ii) If $Y$ is universally measurable, it is universally
Radon-measurable, by 434Eb.  (iii) Suppose that $Y$ is universally
Radon-measurable, and that $\nu$
is a totally finite Borel measure on $Y$.   For Borel sets
$E\subseteq X$, set $\mu E=\nu(E\cap Y)$.   Then $\mu$ is a totally
finite Borel
measure on $X$, so its c.l.d.\ version $\tilde\mu$ is a Radon measure on
$X$, by (a-iii).
We are supposing that $Y$ is universally Radon-measurable, so, in
particular, it must be measured by $\tilde\mu$.   We have

$$\eqalignno{\tilde\mu(X\setminus Y)
&=\sup_{K\subseteq X\setminus Y\text{ is compact}}\tilde\mu K
=\sup_{K\subseteq X\setminus Y\text{ is compact}}\mu K\cr
\displaycause{213Ha, because $\mu$ is totally finite}
&=\sup_{K\subseteq X\setminus Y\text{ is compact}}\nu(K\cap Y)
=0,\cr}$$

\noindent and $Y$ is $\tilde\mu$-conegligible.

Now suppose that $E\subseteq Y$ is a (relatively) Borel subset of $Y$.
Then $E$ is of the form $F\cap Y$ where $F$ is a Borel subset of $X$, so
that

$$\eqalign{\nu E
&=\mu F
=\tilde\mu F
=\tilde\mu(Y\cap F)
=\tilde\mu E\cr
&=\sup_{K\subseteq E\text{ is compact}}\mu K
=\sup_{K\subseteq E\text{ is compact}}\nu K.\cr}$$

\noindent As $E$ is arbitrary, $\nu$ is tight;  as $\nu$ is arbitrary,
$Y$ is a Radon space.

By 434Dc, it follows that all Borel subsets and all Souslin-F subsets of
$X$ are Radon spaces.

\medskip

{\bf (d)} The first step is to note that if $\sequencen{E_n}$ is a
sequence of Radon subspaces of $X$ with union $E$, then $E$ is Radon;
this is immediate from (a-v) above.

Now let $\family{\sigma}{S}{E_{\sigma}}$ be a Souslin
scheme, consisting of Radon subsets of $X$, with kernel $A$.   We know
that $E=\bigcup_{\sigma\in S}E_{\sigma}$ is a Radon space.   Every
$E_{\sigma}$ is universally measurable in $E$, by (b), so $A$ also is
(434Dc), and must be Radon, by (c).   Thus the family of Radon subspaces
of $X$ is closed under Souslin's operation.

If $E$ and $F$ are Radon subsets of $X$, then $E\cup F$ is Radon, and,
just as above, $F$ is universally measurable in $E\cup F$.
But this means that $E\setminus F=(E\cup F)\setminus F$ is universally
measurable in $E\cup F$, so that $E\setminus F$ is Radon.
}%end of proof of 434F

\leader{434G}{}\cmmnt{ Just as we can address the question `when can
we be sure that every Borel measure is of class D?' in terms of the
definition of `Radon' space (434C), we can form other classes of
topological space by declaring that the Borel measures they support must
be of certain kinds.   Three definitions which lead to interesting
patterns of ideas are the following.

\medskip

\noindent}{\bf Definitions (a)} A topological space $X$ is {\bf
Borel-measure-compact}\cmmnt{ ({\smc Gardner \& Pfeffer 84})}
if every totally finite Borel measure on
$X$ which is inner regular with respect to the closed sets is
$\tau$-additive\cmmnt{, that is, $X$ carries no measure of class B in
the classification of 434A}.

\spheader 434Gb A topological space $X$ is 
{\bf Borel-measure-complete}\cmmnt{ ({\smc Gardner \& Pfeffer 84})} if 
every totally finite Borel measure on
$X$ is $\tau$-additive.   \cmmnt{(If $X$ is regular and Hausdorff,
this amounts to saying that $X$ carries no measures of classes A or B in
the classification of 434A.)}

\spheader 434Gc A Hausdorff space $X$ is {\bf pre-Radon}\cmmnt{ (also
called
`{\bf hypo-radonian}', `{\bf semi-radonian}')} if every $\tau$-additive
totally finite Borel measure on $X$ is tight.   \cmmnt{(If $X$ is
regular, this amounts to saying that
$X$ carries no measure of class C in the classification of 434A.)}

\leader{434H}{Proposition} Let $X$ be a topological space and $\Cal B$
its Borel $\sigma$-algebra.

(a) The following are equiveridical:

\quad(i) $X$ is Borel-measure-compact;

\quad(ii) every semi-finite Borel measure on $X$ which is inner regular
with respect to the closed sets is $\tau$-additive;

\quad(iii) every effectively locally
finite Borel measure on $X$ which is inner regular with respect to the
closed sets has an extension to a quasi-Radon measure;

\quad(iv) every totally finite Borel measure on $X$ which is inner
regular with respect to the closed sets has a support;

\quad(v) if
$\mu$ is a non-zero totally finite Borel measure on $X$, inner regular
with respect to the closed sets, and $\Cal G$ is an open cover of $X$,
then there is some $G\in\Cal G$ such that $\mu G>0$.

(b) If $X$ is Lindel\"of\cmmnt{ (in particular, if $X$ is a
K-analytic Hausdorff space)}, it is Borel-measure-compact.

(c) If $X$ is Borel-measure-compact and $A\subseteq X$ is a Souslin-F
set, then $A$ is Borel-measure-compact in its subspace topology.
In particular, any Baire subset of $X$ is Borel-measure-compact.

\proof{{\bf (a)(i)$\Rightarrow$(ii)} Assume (i), and let $\mu$ be a
semi-finite Borel measure on $X$ which is inner regular with respect to
the closed sets.   Let $\Cal G$ be an
upwards-directed family of open sets with union $G^*$, and
$\gamma<\mu G^*$.   Because $\mu$ is semi-finite, there is an
$H\in\Cal B$ such that $\mu H<\infty$ and $\mu(H\cap G^*)\ge\gamma$.
Set $\nu E=\mu(E\cap H)$ for every $E\in\Cal B$;  then $\nu$ is a
totally finite Borel measure on $X$.   For any $E\in\Cal B$,

\Centerline{$\nu E=\mu(E\cap H)
=\sup\{\mu F:F\subseteq E\cap H$ is closed$\}
\le\sup\{\nu F:F\subseteq E$ is closed$\}$,}

\noindent so $\nu$ is inner regular with respect to the closed sets, and
must be $\tau$-additive.   Now

\Centerline{$\gamma
\le\nu G^*
=\sup_{G\in\Cal G}\nu G
\le\sup_{G\in\Cal G}\mu G$.}

\noindent As $\gamma$ and $\Cal G$ is arbitrary, $\mu$ is
$\tau$-additive.

\medskip

\quad{\bf (ii)$\Rightarrow$(iii)} Assume (ii), and let $\mu$ be an
effectively locally finite Borel measure on $X$ which is inner regular
with respect to the closed sets.   Then it is semi-finite (411Gd),
therefore $\tau$-additive.   By 415L, it has an extension to a
quasi-Radon measure on $X$.

\medskip

\quad{\bf (iii)$\Rightarrow$(i)} If (iii) is true and $\mu$ is a totally
finite Borel measure on $X$ which is inner regular with respect to the
closed sets, then $\mu$ has an extension to a
quasi-Radon measure, which is $\tau$-additive, so $\mu$ also is
$\tau$-additive (411C).

\medskip

\quad{\bf (i)$\Rightarrow$(iv)} Use 411Nd.

\medskip

\quad{\bf (iv)$\Rightarrow$(v)} Suppose that (iv) is true, that $\mu$ is
a non-zero totally finite Borel measure on $X$ which is inner regular
with respect to the closed sets, and that $\Cal G$ is an open cover of
$X$.   If $F$ is the support of $\mu$, then $\mu F>0$ so
$F\ne\emptyset$;  there must be some $G\in\Cal G$ meeting $F$, and now
$\mu G>0$.

\medskip

\quad{\bf not-(i)$\Rightarrow$not-(v)} Suppose that there is a totally
finite Borel measure $\mu$ on $X$, inner regular with respect to the
closed sets, which is not $\tau$-additive.   Let $\Cal G$ be an
upwards-directed family of open sets such that $\mu G^*>\gamma$, where
$G^*=\bigcup\Cal G$ and $\gamma=\sup_{G\in\Cal G}\mu G$.   Let
$\sequencen{G_n}$ be a non-decreasing sequence in $\Cal G$ such that
$\mu(G\setminus G^*_0)$ for every $G\in\Cal G$, where
$G^*_0=\bigcup_{n\in\Bbb N}G_n$ (215B(v)).   Then $\mu G^*_0\le\gamma$,
so there is a closed set $F\subseteq G^*\setminus G^*_0$ such that
$\mu F>0$.

Let $\nu$ be the Borel measure on $X$ defined by setting
$\nu E=\mu(E\cap F)$ for every $E\in\Cal B$.   As in the argument for
(i)$\Rightarrow$(ii), $\nu$ is inner regular with respect to the closed
sets.   Consider $\Cal H=\Cal G\cup\{X\setminus F\}$;  this is an open
cover of $X$.   If $G\in\Cal G$ then $\nu G\le\mu(G\setminus G^*_0)=0$,
so $\nu H=0$ for every $H\in\Cal H$;  thus $\nu$ and $\Cal H$ witness
that (v) is false.

\medskip

{\bf (b)} Use (a-v) and 422Gg.

\medskip

{\bf (c)} Let $\mu$ be a Borel measure on $A$ which is inner regular
with respect to the closed sets, that is to say, the relatively closed
sets in $A$.   Let $\nu$ be the corresponding Borel measure on $X$,
defined by setting $\nu E=\mu(A\cap E)$ for every $E\in\Cal B$.   Let
$\hat\nu$ be the completion of $\nu$.   Putting 431D and 421M together,
we see that $\hat\nu A=\sup\{\hat\nu F:F\subseteq A$ is closed in $X\}$,
that is, $\nu X=\sup\{\mu F:F\subseteq A$ is closed in $X\}$.   But this
means that if $E\in\Cal B$ and $\gamma<\nu E$, there is a closed set $F$
in $X$ such that $F\subseteq A$ and $\mu(E\cap F)>\gamma$;  now
there is a relatively closed set $F'\subseteq A$ such that
$F'\subseteq E\cap F$ and $\mu F'\ge\gamma$, and as $F'$ must be
relatively closed in $F$ it is closed in $X$, while $\nu F'\ge\gamma$.
Since $E$ and $\gamma$ are arbitrary, $\nu$ is inner regular with
respect to the closed sets, and will be $\tau$-additive.

Now suppose that $\Cal G$ is an upwards-directed family of relatively
open subsets of $A$.   Set
$\Cal H=\{H:H\subseteq X$ is open, $H\cap A\in\Cal G\}$.   Then $\Cal H$
is upwards-directed, so

\Centerline{$\mu(\bigcup\Cal G)
=\nu(\bigcup\Cal H)
=\sup_{H\in\Cal H}\nu H
=\sup_{G\in\Cal G}\mu G$.}

\noindent As $\mu$ and $\Cal G$ are arbitrary, $A$ is
Borel-measure-compact.

By 421L, it follows that any Baire subset of $X$ is
Borel-measure-compact.
}%end of proof of 434H

\leader{434I}{Proposition} Let $X$ be a topological space.

(a) The following are equiveridical:

\quad(i) $X$ is Borel-measure-complete;

\quad(ii) every semi-finite Borel measure on $X$ is $\tau$-additive;

\quad(iii) every totally finite Borel measure on $X$ has
a support;

\quad(iv) whenever $\mu$ is a totally finite Borel measure on $X$ there
is a base $\Cal U$ for the topology of $X$ such that
$\mu(\bigcup\{U:U\in\Cal U,\,\mu U=0\})=0$.

(b) If $X$ is regular, it is Borel-measure-complete
iff every effectively locally finite Borel measure on $X$ has an
extension to a quasi-Radon measure.

(c) If $X$ is Borel-measure-complete, it is Borel-measure-compact.

(d) If $X$ is Borel-measure-complete, so is every subspace of $X$.

(e) If $X$ is hereditarily Lindel\"of\cmmnt{ (for instance, if $X$ is
separable and metrizable, see 4A2P(a-iii))}, it is
Borel-measure-complete\cmmnt{, therefore Borel-measure-compact}.

\proof{{\bf (a)(i)$\Rightarrow$(ii)} Use the argument of
(i)$\Rightarrow$(ii) of 434Ha;  this case is simpler, because we do not
need to check that the auxiliary measure $\nu$ is inner regular.

\medskip

\quad{\bf (ii)$\Rightarrow$(i)} is trivial.

\medskip

\quad{\bf (i)$\Rightarrow$(iv)} If $X$ is Borel-measure-complete and
$\mu$ is a totally finite Borel measure on $X$, take $\Cal U$ to be the
family of all open subsets of $X$.   This is surely a base for the
topology, and setting $\Cal U_0=\{U:U\in\Cal U,\,\mu U=0\}$, $\Cal U_0$
is upwards-directed so
$\mu(\bigcup\Cal U_0)=\sup_{U\in\Cal U_0}\mu U=0$, as required.

\medskip

\quad{\bf (iv)$\Rightarrow$(iii)} Assume (iv), and  let $\mu$ be a
totally finite Borel measure on $X$.   Take a base $\Cal U$ as in (iv),
so that $\mu(\bigcup\Cal U_0)=0$, where $\Cal U_0$ is the family of
negligible members of $\Cal U$.   Set $F=X\setminus\bigcup\Cal U_0$, so
that $F$ is a conegligible closed set.   If $G\subseteq X$ is an open
set meeting $F$, there is a member $U$ of $\Cal U$ such that $U\subseteq
G$ and $U\cap F\ne\emptyset$;  now $U\notin\Cal U$ so

\Centerline{$\mu(G\cap F)=\mu G\ge\mu U>0$.}

\noindent As $G$ is arbitrary, $F$ is self-supporting and is the support
of $\mu$.

\medskip

\quad{\bf (iii)$\Rightarrow$(i)} Assume (iii), and let $\mu$ be a
totally finite Borel measure on $X$.   Let $\Cal G$ be an
upwards-directed family of open sets with union $G^*$.   Set
$\gamma=\sup_{G\in\Cal G}\mu G$.   Let $\sequencen{G_n}$ be a
non-decreasing sequence in $\Cal G$ such that
$\mu(G\setminus G^*_0)$ for every $G\in\Cal G$, where
$G^*_0=\bigcup_{n\in\Bbb N}G_n$ (215B(v)).   Then $\mu G^*_0\le\gamma$.
Let $\nu$ be the Borel measure on $X$ defined by setting
$\mu E=\mu(E\cap G^*\setminus G_0^*)$ for every $E\in\Cal B$.   Then
$\nu$ has a support $F$ say.   Now $\nu G=0$ for every $G\in\Cal G$, so
$F\cap G=\emptyset$ for every $G\in\Cal G$, and $F\cap G^*=\emptyset$;
but this means that

\Centerline{$\mu(G^*\setminus G_0^*)=\nu X=\nu F
=\mu(F\cap G^*\setminus G_0^*)=0$.}

\noindent Accordingly $\mu G^*=\gamma$.   As $\mu$ and $\Cal G$ are
arbitrary, $X$ is Borel-measure-complete.

\medskip

{\bf (b)} If $X$ is
Borel-measure-complete and $\mu$ is an effectively locally finite Borel
measure on $X$, then $\mu$ is $\tau$-additive, by (a-ii), so extends to
a quasi-Radon measure on $X$, by 415Cb.   If effectively locally finite
Borel measures on $X$ extend to quasi-Radon measures, then any totally
finite Borel measure is $\tau$-additive, by 411C, and $X$ is
Borel-measure-complete.

\medskip

{\bf (c)} Immediate from the definitions.

\medskip

{\bf (d)} If $Y\subseteq X$ and $\mu$ is a totally finite Borel measure
on $Y$, let $\nu$ be the Borel measure on $X$ defined by setting
$\nu E=\mu(E\cap Y)$ for every Borel set
$E\subseteq X$.   Then $\nu$ is $\tau$-additive.
So if $\Cal G$ is an upwards-directed family of relatively open subsets
of $Y$, and we set $\Cal H=\{H:H\subseteq X$ is open, $H\cap Y\in\Cal
G\}$, we shall get

\Centerline{$\mu(\bigcup\Cal G)=\nu(\bigcup\Cal H)
=\sup_{H\in\Cal H}\nu H=\sup_{G\in\Cal G}\mu G$.}

\noindent As $\mu$ and $\Cal G$ are arbitrary, $Y$ is
Borel-measure-complete.

\medskip

{\bf (e)} If $\mu$ is a totally finite Borel measure on $X$ and $\Cal G$
is a non-empty upwards-directed family of open subsets of $X$ with union
$G^*$, then there is a sequence $\sequencen{G_n}$ in $\Cal G$ with union
$G^*$, by 4A2H(c-i).   Because $\Cal G$ is upwards-directed, there is a
non-decreasing sequence $\sequencen{G'_n}$ in $\Cal G$ such that
$G'_n\supseteq G_n$ for every $n\in\Bbb N$, so that

\Centerline{$\mu G^*=\lim_{n\to\infty}\mu G'_n
\le\sup_{G\in\Cal G}\mu G$.}

\noindent As $\mu$ and $\Cal G$ are arbitrary, $X$ is
Borel-measure-complete.
}%end of proof of 434I

\vleader{72pt}{434J}{Proposition} Let $X$ be a Hausdorff space.

(a) The following are equiveridical:

\quad(i) $X$ is pre-Radon;

\quad(ii) every effectively locally finite $\tau$-additive Borel measure
on $X$ is tight;

\quad(iii) whenever $\mu$ is a non-zero totally finite
$\tau$-additive Borel measure on $X$, there is a compact set
$K\subseteq X$ such that $\mu K>0$;

\quad(iv) whenever $\mu$ is a totally finite $\tau$-additive Borel
measure on $X$, $\mu X=
\ifdim\pagewidth=390pt\break\fi
\sup_{K\subseteq X\text{ is compact}}\mu K$;

\quad(v) whenever $\mu$ is a locally finite effectively locally finite
$\tau$-additive Borel measure on $X$, the c.l.d.\ version of $\mu$ is a
Radon measure on $X$.

(b) If $X$ is pre-Radon, then every locally finite
quasi-Radon measure on $X$ is a Radon measure.

(c) If $X$ is regular and every totally finite quasi-Radon measure on
$X$ is a Radon measure, then $X$ is pre-Radon.

(d) If $X$ is pre-Radon, then any universally
Radon-measurable subspace\cmmnt{ (in particular, any Borel subset or
Souslin-F subset)} of $X$ is pre-Radon.

(e) If $A\subseteq X$ is pre-Radon in its subspace topology, it is
universally Radon-measurable in $X$.

(f) If $X$ is K-analytic\cmmnt{ (for instance, if it is compact)}, it
is pre-Radon.

(g) If $X$ is completely regular and \v{C}ech-complete\cmmnt{ (for
instance, if it is locally compact (4A2Gk), or metrizable and complete
under a metric inducing its topology (4A2Md))}, it is pre-Radon.

(h) If $X=\prod_{i\in I}X_i$ where $\familyiI{X_i}$ is a countable
family of pre-Radon Hausdorff spaces, then $X$ is pre-Radon.

(i) If every point of $X$ belongs to a pre-Radon open subset of $X$,
then $X$ is pre-Radon.

\proof{{\bf (a)(i)$\Rightarrow$(ii)} Suppose that $X$ is pre-Radon, that
$\mu$ is an effectively locally finite $\tau$-additive Borel measure on
$X$, that $E\subseteq X$ is Borel, and that
$\gamma<\mu E$.   Because $\mu$ is semi-finite, there is a Borel set
$H\subseteq X$ of finite measure such that $\mu(H\cap E)>\gamma$.   Set
$\nu F=\mu(F\cap H)$ for every Borel set $F\subseteq X$;  then $\nu$ is
a totally finite Borel measure on $X$, and is $\tau$-additive by 414Ea.
Now $\nu E>\gamma$, so there is a compact set $K\subseteq E$ such that
$\gamma\le\nu K\le\mu K$.   As $E$ is arbitrary, $\mu$ is tight.

\medskip

\quad{\bf (ii)$\Rightarrow$(iii)} is trivial.

\medskip

\quad{\bf (iii)$\Rightarrow$(iv)} Assume (iii), and let $\mu$ be a
totally finite $\tau$-additive Borel measure on $X$.   Let $\Cal K$ be
the family of compact subsets of $X$ and set
$\alpha=\sup_{K\in\Cal K}\mu K$.
\Quer\ Suppose, if possible, that $\mu X>\alpha$.   Let
$\sequencen{K_n}$ be a sequence in $\Cal K$ such that
$\sup_{n\in\Bbb N}\mu K_n=\alpha$, and set $L=\bigcup_{n\in\Bbb N}K_n$;
then

\Centerline{$\mu L=\lim_{n\to\infty}\mu(\bigcup_{i\le n}K_i)=\alpha$.}

\noindent Set $\nu E=\mu(E\setminus L)$ for every Borel set
$E\subseteq X$.   Then $\nu$ is a non-zero totally finite Borel measure
on $X$, and is $\tau$-additive, by 414Ea again.   So there is a
$K\in\Cal K$ such that $\nu K>0$.   But now there is an $n\in\Bbb N$
such that
$\nu K+\mu K_n>\alpha$, and in this case $K\cup K_n\in\Cal K$ and

\Centerline{$\mu(K\cup K_n)=\mu(K\setminus K_n)+\mu K_n
\ge\nu K+\mu K_n>\alpha$,}

\noindent which is impossible.\ \BanG\
So $\mu X=\alpha$, as required.

\medskip

\quad{\bf (iv)$\Rightarrow$(i)} Assume (iv), and let $\mu$ be a totally
finite $\tau$-additive Borel measure on $X$.   Suppose that
$E\subseteq X$ is Borel and that $\gamma<\mu E$.   By (iii), there is a
compact set $K\subseteq X$ such that $\mu K>\mu X-\mu E+\gamma$, so that
$\mu(E\cap K)>\gamma$.   Consider the subspace measure $\mu_K$ on $K$.
By 414K, this is $\tau$-additive, so inner regular with respect to the
closed subsets of $K$ (414Mb).   There is therefore a relatively
closed subset $F$ of $K$ such that $F\subseteq K\cap E$ and
$\mu_KF\ge\gamma$;  but now $F$ is a compact subset of $E$ and $\mu
F\ge\gamma$.   As $E$ and $\gamma$ are arbitrary, $\mu$ is tight.   As
$\mu$ is arbitrary, $X$ is
pre-Radon.

\medskip

\quad{\bf (ii)$\Rightarrow$(v)} Assume (ii), and let $\mu$ be a locally
finite effectively locally finite $\tau$-additive Borel measure on $X$.
Then $\mu$ is tight, so by
416F(ii) its c.l.d.\ version is a Radon measure.

\medskip

\quad{\bf (v)$\Rightarrow$(iv)} Assume (v), and let $\mu$ be a totally
finite $\tau$-additive Borel measure on $X$.   Then the c.l.d.\ version
$\tilde\mu$ of $\mu$ is a Radon measure;  but $\tilde\mu$ extends $\mu$
(213Hc), so

\Centerline{$\sup_{K\subseteq X\text{ is compact}}\mu K
=\sup_{K\subseteq X\text{ is compact}}\tilde\mu K
=\tilde\mu X=\mu X$.}

\medskip

{\bf (b)} Let $\mu$ be a locally finite quasi-Radon measure on $X$.   By
(a-ii), $\mu$ is tight;  by 416C, $\mu$ is a Radon measure.

\medskip

{\bf (c)} Let $\mu$ be a totally finite $\tau$-additive Borel measure on
$X$.   Because $X$ is regular, the c.l.d.\ version $\tilde\mu$ or $\mu$ is a
quasi-Radon measure (415Cb), therefore a Radon measure;  but $\tilde\mu$
extends $\mu$
(213Hc again), so $\mu$, like $\tilde\mu$, must be tight.   As $\mu$ is
arbitrary, $X$ is pre-Radon.

\medskip

{\bf (d)} Let $A$ be a universally Radon-measurable subset of $X$, and $\mu$
a totally finite $\tau$-additive Borel measure on
$A$.   Set
$\nu E=\mu(E\cap A)$ for every Borel set $E\subseteq X$;  then $\nu$ is
a totally finite $\tau$-additive Borel measure on $X$.   So its c.l.d.\
version (that is, its completion $\hat\nu$, by 213Ha) is a Radon measure
on $X$, by (a-v).   Now $\hat\nu$ measures $A$, so

\Centerline{$\mu A
=\nu^*A
=\hat\nu A
=\sup\{\hat\nu K:K\subseteq A$ is compact$\}
=\sup\{\mu K:K\subseteq A$ is compact$\}$.}

\noindent By (a-iv), $A$ is pre-Radon.

\medskip

{\bf (e)} Let $\mu$ be a totally finite Radon measure on $X$.   Then the
subspace measure $\mu_A$ is $\tau$-additive (414K), so its restriction
$\nu$ to the Borel $\sigma$-algebra of $A$ is still $\tau$-additive.
Because $A$ is pre-Radon,

$$\eqalign{\mu^*A
=\mu_AA
&=\nu A
=\sup\{\nu K:K\subseteq A\text{ is compact}\}\cr
&=\sup\{\mu K:K\subseteq A\text{ is compact}\}
=\mu_*A,\cr}$$

\noindent and $\mu$ measures $A$ (413Ef).   As $\mu$ is arbitrary, $A$
is universally Radon-measurable.

\medskip

{\bf (f)} Put 432B and (a-iv) together.

\medskip

{\bf (g)} If we identify $X$ with a G$_{\delta}$ set in a compact
Hausdorff space $Z$, then $Z$ is pre-Radon, by (f), so $X$ is pre-Radon,
by (d).

\medskip

{\bf (h)} Let $\mu$ be a totally finite $\tau$-additive Borel measure on
$X$, and $\epsilon>0$.   Let $\familyiI{\epsilon_i}$ be a family of
strictly positive real numbers such that
$\sum_{i\in I}\epsilon_i\le\epsilon$ (4A1P).   For each $i\in I$ and Borel
set $F\subseteq X_i$, set $\mu_iF=\mu\pi_i^{-1}[F]$, where
$\pi_i(x)=x(i)$ for $x\in X$;  because $\pi_i:X\to X_i$ is continuous,
$\mu_i$ is a totally finite $\tau$-additive Borel measure on $X_i$.
Because $X_i$ is pre-Radon, we can find a compact set $K_i\subseteq X_i$
such that $\mu_i(X_i\setminus K_i)\le\epsilon_i$, by (a-iv).   Now
$K=\prod_{i\in I}K_i$ is compact (3A3J), and
$X\setminus K\subseteq\bigcup_{i\in I}\pi_i^{-1}[X_i\setminus K_i]$, so

\Centerline{$\mu(X\setminus K)\le\sum_{i\in I}\mu_i(X_i\setminus K_i)
\le\sum_{i\in I}\epsilon_i\le\epsilon$.}

As $\epsilon$ and $\mu$ are arbitrary, $X$ satisfies the condition of
(a-iv), and is pre-Radon.

\medskip

{\bf (i)} Let $\Cal G$ be a cover of $X$ by pre-Radon open sets.   Let
$\mu$ be a non-zero totally finite $\tau$-additive Borel measure on $X$.
Then $\mu X=\sup\{\mu(\bigcup\Cal G_0):\Cal G_0\subseteq\Cal G$ is
finite$\}$, so there is some $G\in\Cal G$ such that $\mu G>0$.   Now the
subspace measure $\mu_G$ is a non-zero totally finite $\tau$-additive
Borel measure on $G$, so there is a compact set $K\subseteq G$ such that
$\mu_GK>0$, in which case $\mu K>0$.   As $\mu$ is arbitrary, $X$ is
pre-Radon, by (a-iii).
}%end of proof of 434J

\leader{434K}{}\cmmnt{ I return to criteria for deciding whether
Hausdorff spaces are Radon.

\wheader{434K}{6}{2}{2}{72pt}

\noindent}{\bf Proposition} (a) A Hausdorff space is Radon iff it is
Borel-measure-complete and pre-Radon.

(b) An analytic Hausdorff space is Radon.
\cmmnt{In particular, any compact metrizable space is Radon and any
Polish space is Radon.}

(c) $\omega_1$ and $\omega_1+1$, with their order topologies, are not
Radon.

(d) For a set $I$, $[0,1]^I$ is Radon iff $\{0,1\}^I$ is Radon iff $I$
is countable.

(e) A hereditarily Lindel\"of K-analytic Hausdorff space is Radon;  in
particular, the split interval\cmmnt{ (343J, 419L)} is Radon.

\proof{{\bf (a)} Put the definitions 434C, 434Gb and 434Gc together,
recalling that a tight measure is necessarily $\tau$-additive (411E).

\medskip

{\bf (b)} 433Cb.

\medskip

{\bf (c)} Dieudonn\'e's measure (411Q) is a Borel measure on $\omega_1$
which is not tight,
so $\omega_1$ is certainly not a Radon space;  as it is an open set in
$\omega_1+1$, and the subspace topology inherited from $\omega_1+1$ is
the order topology of $\omega_1$ (4A2S(a-iii)), $\omega_1+1$ cannot be
Radon (434Fc).

\medskip

{\bf (d)} If $I$ is countable, then $\{0,1\}^I$ and $[0,1]^I$ are
compact metrizable spaces, so are Radon.   If $I$ is uncountable, then
$\omega_1+1$, with its order topology, is homeomorphic to a closed
subset of $\{0,1\}^I$.   \Prf\ Set $\kappa=\#(I)$.   For
$\xi\le\omega_1$, $\eta<\kappa$ set $x_{\xi}(\eta)=1$ if $\eta<\xi$, $0$
if $\xi\le\eta$.   The map $\xi\mapsto
x_{\xi}:\omega_1+1\to\{0,1\}^{\kappa}$ is injective
because $\kappa\ge\omega_1$, and is continuous because all the sets
$\{\xi:x_{\xi}(\eta)=0\}=(\omega_1+1)\cap(\eta+1)$ are open-and-closed
in $\omega_1+1$.   Since $\omega_1+1$ is compact in its order topology
(4A2S(a-i)), it is homeomorphic to its image in
$\{0,1\}^{\kappa}\cong\{0,1\}^I$.\ \Qed

By 434Fc, $\{0,1\}^I$ cannot
be a Radon space.   Since $\{0,1\}^I$ is a closed subset of $[0,1]^I$,
$[0,1]^I$ also is not a Radon space.

\medskip

{\bf (e)} Suppose that $X$ is a hereditarily Lindel\"of K-analytic
Hausdorff space.   Then it is Borel-measure-complete by 434Ie and
pre-Radon by 434Jf, so by (a) here it is Radon.

Since the split interval is compact and Hausdorff and hereditarily
Lindel\"of (419La), it is a Radon space.
}%end of proof of 434K

\leader{434L}{}\cmmnt{ It is worth noting an elementary special
property of metric spaces.

\medskip

\noindent}{\bf Proposition} If $(X,\rho)$ is a metric space, then any
quasi-Radon measure on $X$ is inner regular with respect to the
totally bounded subsets of $X$.

\proof{ Let $\mu$ be a quasi-Radon measure on $X$ and $\Sigma$
its domain.   Suppose that $E\in\Sigma$ and $\gamma<\mu E$.   Then there
is an open set $G$ of finite measure such that $\mu(E\cap G)>\gamma$;
set $\delta=\mu(E\cap G)-\gamma$.   For $n\in\Bbb N$, $I\subseteq X$ set
$H(n,I)=\bigcup_{x\in I}\{y:\rho(y,x)<2^{-n}\}$.   Then
$\{H(n,I):I\in[X]^{<\omega}\}$ is an
upwards-directed family of open sets covering $X$.   Because $\mu$ is
$\tau$-additive, there is a finite set $I_n\subseteq X$ such that
$\mu(G\setminus H(n,I_n))\le 2^{-n-1}\delta$.   Consider
$F=\bigcap_{n\in\Bbb N}H(n,I_n)$.   This is
totally bounded and $\mu(G\setminus F)\le\delta$, so $E\cap F$ is
totally bounded and $\mu(E\cap F)\ge\gamma$.   As $E$ and $\gamma$ are
arbitrary,
$\mu$ is inner regular with respect to the totally bounded sets.
}%end of proof of 434L

\leader{434M}{}\cmmnt{ I turn next to a couple of ideas depending on
countable compactness.

\medskip

\noindent}{\bf Lemma} Let $X$ be a countably compact topological space
and $\Cal E$ a non-empty family of closed subsets of $X$ with the finite
intersection
property.   Then there is a Borel probability measure $\mu$ on $X$,
inner regular
with respect to the closed sets, such that $\mu F=1$ for every
$F\in\Cal E$.

\proof{{\bf (a)} By Zorn's lemma, $\Cal E$ is included in a maximal
family $\Cal E^*$ of
closed subsets of $X$ with the finite intersection property.

\quad(i) If $F\subseteq X$ is closed and $F\cap F_0\cap\ldots\cap
F_n\ne\emptyset$ for every $F_0,\ldots,F_n\in\Cal E^*$, then $\Cal
E^*\cup\{F\}$ has the finite intersection property, so $F\in\Cal E^*$.

\quad(ii) If $F$, $F'\in\Cal E^*$, then $F\cap F'\cap F_0\cap\ldots\cap
F_n\ne\emptyset$ for all $F_0,\ldots,F_n\in\Cal E^*$, so $F\cap
F'\in\Cal E^*$.

\quad(iii) If $F\subseteq X$ is closed and $F\cap F'\in\Cal E^*$ for every
$F'\in\Cal E^*$, then $F\cap
F_0\cap\ldots\cap F_n\in\Cal E^*$ for every $F_0,\ldots,F_n\in\Cal E^*$
(because $F_0\cap\ldots\cap F_n\in\Cal E^*$, by (ii)), so
$F\in\Cal E^*$.

\quad(iv) If $\sequencen{F_n}$ is a sequence in $\Cal E^*$, 
with intersection
$F$, and $F'\in\Cal E^*$, then $F'\cap\bigcap_{i\le n}F_i\ne\emptyset$
for every $n\in\Bbb N$.   Because $X$ is countably compact,
$F'\cap F\ne\emptyset$ (4A2G(f-ii)).   As $F'$ is arbitrary,
$F\in\Cal E^*$, by (iii).
Thus $\Cal E^*$ is closed under countable intersections.

\medskip

{\bf (b)} Set

\Centerline{$\Sigma=\{E:E\subseteq X$, there is an $F\in\Cal E^*$ such
that either $F\subseteq E$ or $F\cap E=\emptyset\}$,}

\noindent and define $\hat\mu:\Sigma\to\{0,1\}$ by saying that
$\hat\mu E=1$ if there is some $F\in\Cal E^*$ such that $F\subseteq E$,
$0$ otherwise.   Then $\hat\mu$ is a probability measure on $X$.
\Prf\

\quad(i) $\emptyset\in\Sigma$ because $\Cal E^*\supseteq\Cal E$ is not empty.

\quad(ii) $X\setminus E\in\Sigma$ whenever $E\in\Sigma$ because the
definition of $\Sigma$ is symmetric between $E$ and $X\setminus E$.

\quad(iii) If $\sequencen{E_n}$ is any sequence in $\Sigma$ with union $E$,
then either there are $n\in\Bbb N$ and $F\in\Cal E^*$ such that 
$F\subseteq E_n\subseteq E$ and $E\in\Sigma$, or for every $n\in\Bbb N$ there is an
$F_n\in\Cal E^*$ such that $F_n\cap E_n=\emptyset$.   In this case
$F=\bigcap_{n\in\Bbb N}F_n\in\Cal E^*$, by (a-iv), and $E\cap
F=\emptyset$, so again $E\in\Sigma$.   Thus $\Sigma$ is a
$\sigma$-algebra of subsets of $X$.

\quad(iv) $\hat\mu\emptyset=0$ because $\emptyset$ cannot belong to $\Cal
E^*$.

\quad(v) If $\sequencen{E_n}$ is any disjoint sequence in $\Sigma$ with union
$E$, then either there is some $n$ such that $\hat\mu E_n=1$, in which
case $\hat\mu E_i=0$ for every $i\ne n$ (because any two members of
$\Cal E^*$ must meet) and $\hat\mu E=1=\sum_{i=0}^{\infty}\hat\mu E_i$,
or $\hat\mu E_i=0$ for every $i$, in which case, just as in (iii),
$\hat\mu E=0=\sum_{i=0}^{\infty}\hat\mu E_i$.   Thus $\hat\mu$ is a
measure.

\quad(vi) Because $\Cal E^*\ne\emptyset$, $\hat\mu X=1$.   Thus $\hat\mu$ is
a probability measure.\ \Qed

\medskip

{\bf (c)} If $F\subseteq X$ is a closed set, then either $F$ itself
belongs to $\Cal E^*$, so $F\in\Sigma$, or there is some $F'\in\Cal E^*$
such that $F\cap F'=\emptyset$, in which case again $F\in\Sigma$.   So
$\Sigma$ contains every closed set, therefore every Borel set, and
$\hat\mu$ is a topological measure.   By construction, $\hat\mu$ is
inner regular with respect to $\Cal E^*$ and therefore with respect to
the closed sets.   Finally, if $F\in\Cal E$ then $F\in\Cal E^*$, so
$\hat\mu F=1$.   We may therefore take $\mu$ to be the restriction of
$\hat\mu$ to
the Borel $\sigma$-algebra of $X$, and $\mu$ will be a Borel measure on
$X$,
inner regular with respect to the closed sets, such that $\mu E=1$ for
every $E\in\Cal E$.
}%end of proof of 434M

\leader{434N}{Proposition} (a) Let $X$ be a Borel-measure-compact
topological space.   Then closed countably compact subsets of $X$ are
compact.

(b) Let $X$ be a Borel-measure-complete topological space.
Then countably compact subsets of $X$ are compact.

(c) Let $X$ be a Hausdorff Borel-measure-complete topological space.
Then compact subsets of $X$ are countably tight.

(d) In particular, any Radon compact Hausdorff space is countably tight.

\proof{{\bf (a)} Let $C$ be a closed countably compact subset of $X$.
Let $\Cal F$ be an ultrafilter on $C$.   Let $\Cal E$ be the
family of closed subsets of $C$ belonging to $\Cal F$.
Then $\Cal E$ has the finite intersection property, so by 434M there is
a Borel probability measure $\mu$ on $C$, inner regular with respect to
the closed sets, such that $\mu E=1$ for every $E\in\Cal E$.
Let $\nu$ be the Borel measure on $X$ defined by setting
$\nu H=\mu(C\cap H)$ for every Borel set $H\subseteq X$.   Then $\nu$ is
also inner regular with respect to the closed sets (of either $C$ or
$X$);  because $X$ is Borel-measure-compact, $\nu$ has a support $F$
(434H(a-iv)).   Since
$\nu F=\nu X=1$, $F\cap C\ne\emptyset$;  take $x\in F\cap C$.   If $G$
is any open
set (in $X$) containing $x$, then $\mu(C\setminus G)=\nu(X\setminus G)<1$, so
$C\setminus G\notin\Cal F$ and $C\cap G\in\Cal F$.   As $G$ is
arbitrary, $\Cal F\to x$;  as $\Cal F$ is arbitrary, $C$ is compact.

\medskip

{\bf (b)} Repeat the argument of (a).   Let $C$ be a countably compact
subset of $X$ and $\Cal F$ an ultrafilter on $C$.   Let $\Cal E$ be
the family of relatively closed subsets of $C$ belonging to $\Cal F$.
Then there is
a Borel probability measure $\mu$ on $C$ such that $\mu E=1$ for every
$E\in\Cal E$.
Let $\nu$ be the Borel measure on $X$ defined by setting
$\nu H=\mu(C\cap H)$ for every Borel set $H\subseteq X$.   Because $X$
is Borel-measure-complete, $\nu$ has a support $F$ (434I(a-iii)).
Since $\nu F=\nu X=1$, $F\cap C\ne\emptyset$;  take $x\in F\cap C$.   If
$G$ is any open
set containing $x$, then $\nu(X\setminus G)<1$, so
$C\setminus G\notin\Cal F$ and $C\cap G\in\Cal F$.   As $G$ is
arbitrary, $\Cal F\to x$;  as $\Cal F$ is arbitrary, $C$ is compact.

\medskip

{\bf (c)} Again let $C$ be a (countably) compact subset of $X$.   Take
$A\subseteq C$, and set
$C_0=\bigcup\{\overline{B}:B\in[A]^{\le\omega}\}$.   Then $C_0$ is
countably compact.   \Prf\ If $\sequencen{y_n}$ is any sequence in
$C_0$, it has a cluster point $y\in C$.   For
each $n\in\Bbb N$ there is a countable set $B_n\subseteq A$ such that
$y_n\in\overline{B_n}$.   Now $B=\bigcup_{n\in\Bbb N}B_n$ is a countable
subset of $A$, and $y\in\overline{B}\subseteq C_0$, so $y$ is a cluster
point of $\sequencen{y_n}$ in $C_0$.   As $\sequencen{y_n}$ is
arbitrary, $C_0$ is countably compact.\ \Qed

By (b), $C_0$ is compact, therefore closed, and must include
$\overline{A}$.   Thus every point of $\overline{A}$ is in the closure
of some countable subset of $A$.   As $A$ is arbitrary, $C$ is countably
tight.

\medskip

{\bf (d)} Finally, a compact Radon Hausdorff space is
Borel-measure-complete (434Ka) and countably compact, therefore
countably tight.
}%end of proof of 434N

\leader{434O}{Quasi-dyadic \dvrocolon{spaces}}\cmmnt{ I wish now to
present a result in an entirely different direction.   Measures of type
B$_1$ in the classification of 434A (completion regular, but not
$\tau$-additive) seem to be hard to come by.   The next theorem shows
that on a substantial class of spaces they cannot appear.   First, we
need a definition.

\medskip

\noindent}{\bf Definition} A topological space $X$ is {\bf quasi-dyadic}
if it is expressible as a continuous image of a product of separable
metrizable spaces.

\cmmnt{I give some elementary results to indicate what kind of spaces
we have here.}

\leader{434P}{Proposition} (a) A continuous image of a quasi-dyadic
space is quasi-dyadic.

(b) Any product of quasi-dyadic spaces is quasi-dyadic.

(c) A space with a countable network is quasi-dyadic.

(d) A Baire subset of a quasi-dyadic space is quasi-dyadic.

(e) If $X$ is any topological space, a countable union of quasi-dyadic
subspaces of $X$ is quasi-dyadic.

\proof{{\bf (a)} Immediate from the definition.

\medskip

{\bf (b)} Again immediate;  if $X_i$ is a continuous image of
$\prod_{j\in J_i}Y_{ij}$, where $Y_{ij}$ is a separable metrizable space for
every $i\in I$ and $j\in J_i$, then $\prod_{i\in I}X_i$ is a continuous
image of $\prod_{i\in I,j\in J_i}Y_{ij}$.

\medskip

{\bf (c)} Let $\Cal E$ be a countable network for the
topology of $X$.   On $X$ let $\sim$ be the equivalence relation in
which $x\sim y$ if they
belong to just the same members of $\Cal E$;  let $Y$ be the space
$X/\backstep2\sim$ of equivalence classes, and $\phi:X\to Y$ the
canonical map.   $Y$ has a separable metrizable topology with base
$\{\phi[E]:E\in\Cal
E\}\cup\{\phi[X\setminus E]:E\in\Cal E\}$.   Let $I$ be any set such
that $\#(\{0,1\}^I)\ge\#(X)$, and for each $y\in Y$ let
$f_y:\{0,1\}^I\to y$ be a surjection.   Then we have a
continuous surjection $f:Y\times\{0,1\}^I\to X$ given by saying that
$f(y,z)=f_y(z)$ for $y\in Y$ and $z\in\{0,1\}^I$.

\medskip

{\bf (d)} Let $\familyiI{Y_i}$ be a
family of separable metrizable spaces with product $Y$ and $f:Y\to X$ a
continuous surjection.   If $W\subseteq Y$ is a Baire set, it is
determined by coordinates in a countable subset of $I$ (4A3Nb),
so can be regarded as $W'\times\prod_{i\in I\setminus J}Y_i$, where
$J\subseteq I$ is countable and $W'\subseteq\prod_{i\in J}Y_i$;  as
$\prod_{i\in J}Y_i$ and $W'$ are separable metrizable spaces
(4A2Pa), $W$ can be
thought of as a product of separable metrizable spaces.   Now the set
$\{E:E\subseteq X,\,f^{-1}[E]$ is a Baire set in $Y\}$ is a
$\sigma$-algebra containing every zero set in $X$, so contains every
Baire set.   Thus every Baire subset of $X$ is a continuous image of a
Baire subset of $Y$, and is therefore quasi-dyadic.

\medskip

{\bf (e)} If $E_n\subseteq X$ is quasi-dyadic for each $n\in\Bbb N$,
then $Z=\Bbb N\times\prod_{n\in\Bbb N}E_n$ is quasi-dyadic, and
$f:Z\to\bigcup_{n\in\Bbb N}E_n$ is a continuous surjection, where
$f(n,\langle x_i\rangle_{{i\in\Bbb N}})=x_n$.   So
$\bigcup_{n\in\Bbb N}E_n$ is quasi-dyadic.
}%end of proof of 434P

\leader{434Q}{Theorem}\cmmnt{ ({\smc Fremlin \& Grekas 95})} A
semi-finite completion regular topological measure on a quasi-dyadic
space is $\tau$-additive.

\proof{\Quer\ Suppose, if possible, otherwise.

\medskip

{\bf (a)} The first step is the standard reduction to the case in which
$\mu X=1$ and $X$ is covered by open sets of zero measure.   In detail:
suppose that $X$ is a quasi-dyadic space and $\mu_0$ is a semi-finite
completion regular topological measure on $X$ which is not
$\tau$-additive.   Let $\Cal G$ be an upwards-directed family of open sets in $X$ such that $\mu_0(\bigcup\Cal G)$ is strictly greater than $\sup_{G\in\Cal G}\mu_0G=\gamma$ say.   Let $\sequencen{G_n}$ be a 
non-decreasing
sequence in $\Cal G$ such that $\lim_{n\to\infty}\mu_0G_n=\gamma$, and
set $H_0=\bigcup_{n\in\Bbb N}G_n$, so that $\mu_0H_0=\gamma$;  take a
closed set $Z\subseteq\bigcup\Cal G$ such that $\gamma<\mu_0Z<\infty$.
Set $\mu_1E=\mu_0(E\cap Z\setminus H_0)$ for every Borel set
$E\subseteq X$.   Then $\mu_1$ is a non-zero totally finite Borel
measure on $X$, and is
completion regular.   \Prf\ If $E\subseteq X$ is a Borel set and
$\epsilon>0$, there is
a zero set $F\subseteq E\cap Z\setminus H_0$ such that
$\mu_0F\ge\mu_0(E\cap X\setminus H_0)-\epsilon$, and now
$\mu_1F\ge\mu_1E-\epsilon$.\ \QeD\
Note that $\mu_1(X\setminus Z)=\mu_1G=0$ for every $G\in\Cal G$.

For Borel sets $E\subseteq X$,
set $\mu E=\mu_1E/\mu_1X$;  then $\mu$ is a completion regular Borel
probability measure on $X$, and $\Cal G\cup\{X\setminus Z\}$ is a cover
of $X$ by open negligible sets.

\medskip

{\bf (b)} Now let $\langle Y_i\rangle_{i\in I}$ be a family
of separable metrizable spaces such that there is a continuous surjection
$f:Y\to X$, where $Y=\Yprod{I}$.   For each $i\in I$ let
$\Cal B_i$ be a countable base for the topology of $Y_i$.   For
$J\subseteq I$ let $\Cal C(J)$ be the family of all non-empty open cylinders in
$Y$ expressible in the form

\Centerline{$\{s:s(i)\in B_i\Forall i\in K\}$,}

\noindent where $K$ is a finite subset of $J$ and
$B_i\in\Cal B_i$ for each $i\in K$;  thus $\Cal C(I)$ is a base for the
topology of $Y$.   Set $\Cal C_0(J)=\{U:U\in\Cal C(J),\,\mu^*f[U]=0\}$
for each $J\subseteq I$.   Note that (because every $\Cal B_i$ is
countable) $\Cal C(J)$ and $\Cal C_0(J)$ are countable for every
countable subset $J$ of $I$.   It is easy to see that
$\Cal C(J)\cap\Cal C(K)=\Cal C(J\cap K)$ for all $J$, $K\subseteq I$,
because if $U\in\Cal C(I)$ it belongs to $\Cal C(J)$ iff
its projection onto $X_i$ is the whole of $X_i$ for every $i\notin J$.

For each negligible set $E\subseteq X$, let
$\langle F_n(E)\rangle_{n\in\Bbb N}$ be a family of zero sets, subsets
of $X\setminus E$, such that $\sup_{n\in\Bbb N}\mu F_n(E)=1$.   Then each
$f^{-1}[F_n(E)]$ is a zero set in $Y$, so there is a countable set
$M(E)\subseteq I$ such that all the sets $f^{-1}[F_n(E)]$ are determined
by coordinates in $M(E)$ (4A3Nc).   Let $\Cal J$ be the family of
countable subsets $J$ of
$I$ such that $M(f[U])\subseteq J$ for every $U\in\Cal C_0(J)$;  then
$\Cal J$ is cofinal with $[I]^{\le\omega}$, that is, every countable
subset of $I$ is included in some member of $\Cal J$.   \Prf\ If we
start from any countable subset $J_0$ of $I$ and set

\Centerline{$J_{n+1}=J_n\cup\bigcup\{M(f[U]):U\in\Cal C_0(J_n)\}$}

\noindent for each $n\in\Bbb N$, then every $J_n$ is countable, and
$\bigcup_{n\in\Bbb N}J_n\in\Cal
J$, because $\sequencen{J_n}$ is non-decreasing, so $\Cal
C_0(\bigcup_{n\in\Bbb N}J_n)=\bigcup_{n\in\Bbb N}\Cal C_0(J_n)$.\ \Qed

\medskip

{\bf (c)} For each $J\in\Cal J$, set

\Centerline{$Q_J=\bigcap\{\bigcup_{n\in\Bbb N}F_n(f[U]):U\in\Cal
C_0(J)\}$.}

\noindent Then $\mu Q_J=1$ and $f^{-1}[Q_J]$ is determined by
coordinates
in $J$, while $f^{-1}[Q_J]\cap U=\emptyset$ whenever $U\in\Cal C_0(J)$.

If $G\subseteq X$ is an open set, then $G\cap Q_J=\emptyset$
whenever $J\in\Cal J$ and there is a negligible Baire set $Q\supseteq G$
such that $f^{-1}[Q]$ is determined by coordinates in $J$.   \Prf\  Set
$H=\pi_J^{-1}[\pi_J[f^{-1}[G]]]$, where $\pi_J:Y\to\Yprod{J}$ is the
canonical map;  then $H$ is a union of members of $\Cal C(J)$, because
$f^{-1}[G]$ is open in $Y$ and $\pi_J[f^{-1}[G]]$ is open in
$\Yprod{J}$.   Also, because $f^{-1}[Q]$ is determined by coordinates in
$J$, $H\subseteq f^{-1}[Q]$, so $f[H]\subseteq Q$ and $\mu^*f[H]=0$;
thus all the members of $\Cal C(J)$ included in $H$ actually belong to
$\Cal C_0(J)$, and $H\cap f^{-1}[Q_J]=\emptyset$.
But this means that $f^{-1}[G]\cap f^{-1}[Q_J]=\emptyset$ and (because
$f$ is a surjection) $G\cap Q_J=\emptyset$, as
claimed.\ \QeD\   In particular, if $G$ is a negligible open set in $X$,
then $G\cap
Q_J=\emptyset$ whenever $J\in\Cal J$ and $J\supseteq M(G)$.

\medskip

{\bf (d)} If $J\in\Cal J$, there are $s$, $s'\in f^{-1}[Q_J]$ such that
$s\restr J=s'\restr J$ and $f(s)$, $f(s')$ can be separated by open sets
in $X$.   \Prf\ Start from any $x\in Q_J$ and take a negligible
open set $G$ containing $x$ (recall that our hypothesis is that $X$ is
covered by negligible open sets).   For each $n\in\Bbb N$ let
$h_n:X\to\Bbb R$ be a continuous function such that
$F_n(G)=h_n^{-1}[\{0\}]$.   We know that $G\cap Q_J\ne\emptyset$, while
$G\subseteq X\setminus(\bigcup_{n\in\Bbb N}F_n(G)\cap Q_J)$, which is a
negligible Baire set;  by (c), $f^{-1}[X\setminus(\bigcup_{n\in\Bbb
N}F_n(G)\cap Q_J)]$ is not determined by coordinates in $J$, and there
must be some $n$ such that $f^{-1}[F_n(G)\cap Q_J]$ is not determined by
coordinates in $J$.   Accordingly
there must be $s$, $s'\in Y$ such that $s\restr J=s'\restr J$, $s\in
f^{-1}[F_n(G)\cap Q_J]$ and $s'\notin f^{-1}[F_n(G)\cap Q_J]$.   Now
$s\in f^{-1}[Q_J]$, which is determined by coordinates in $J$;  since
$s\restr J=s'\restr J$, $s'\in f^{-1}[Q_J]$ and
$s'\notin f^{-1}[F_n(G)]$.
Accordingly $h_n(f(s))=0\ne h_n(f(s'))$ and $f(s)$, $f(s')$ can be
separated by open sets.\ \Qed

\medskip

{\bf (e)} We are now ready to embark on the central construction of the
argument.   We may choose inductively, for ordinals $\xi<\omega_1$, sets
$J_{\xi}\in\Cal J$, negligible open sets $G_{\xi}$, 
$G'_{\xi}\subseteq X$, points $s_{\xi}$, $s'_{\xi}\in Y$ and sets 
$U_{\xi}$, $V_{\xi}$, $V'_{\xi}\in\Cal C(I)$ such that

\inset{$J_{\eta}\subseteq J_{\xi}$, $U_{\eta}$, $V_{\eta}$, $V'_{\eta}$
all belong to $\Cal C(J_{\xi})$ and $G_{\eta}\cap Q_{J_{\xi}}=\emptyset$
whenever $\eta<\xi<\omega_1$ (using the results of (b) and (c) to choose
$J_{\xi}$);}

\inset{$s_{\xi}\restr J_{\xi}=s'_{\xi}\restr J_{\xi}$, $s_{\xi}\in
f^{-1}[Q_{J_{\xi}}]$ and $f(s_{\xi})$ and $f(s'_{\xi})$ can be separated
by open sets in $X$ (using (d) to choose $s_{\xi}$, $s'_{\xi}$);}

\inset{$G_{\xi}$, $G'_{\xi}$ are disjoint negligible open sets
containing
$f(s_{\xi})$, $f(s'_{\xi})$ respectively (choosing $G_{\xi}$,
$G'_{\xi}$);}

\inset{$U_{\xi}\in\Cal C(J_{\xi})$, $V_{\xi}$, $V'_{\xi}\in\Cal
C(I\setminus J_{\xi})$, $s_{\xi}\in U_{\xi}\cap V_{\xi}\subseteq
f^{-1}[G_{\xi}]$, $s'_{\xi}\in U_{\xi}\cap V'_{\xi}
\subseteq f^{-1}[G'_{\xi}]$ (choosing
$U_{\xi}$, $V_{\xi}$, $V'_{\xi}$, using the fact that $s_{\xi}\restr
J_{\xi}=s'_{\xi}\restr J_{\xi}$).}

\noindent On completing this construction, take for each $\xi<\omega_1$
a finite set $K_{\xi}\subseteq J_{\xi+1}$ such that $U_{\xi}$, $V_{\xi}$
and $V'_{\xi}$ all belong to $\Cal C(K_{\xi})$.   By the $\Delta$-system
Lemma (4A1Db), there is an uncountable $A\subseteq\omega_1$ such
that $\langle K_{\xi}\rangle_{\xi\in A}$ is a $\Delta$-system with root
$K$ say.    For $\xi\in A$, express $U_{\xi}$ as 
$\tilde U_{\xi}\cap U'_{\xi}$ where $\tilde U_{\xi}\in\Cal C(K)$ and 
$U'_{\xi}\in\Cal C(K_{\xi}\setminus K)$.   Then there are only countably many
possibilities for $\tilde U_{\xi}$, so there is an uncountable
$B\subseteq A$ such that $\tilde U_{\xi}$ is constant for $\xi\in B$;
write $\tilde U$ for the constant value.   Let $C\subseteq B$ be an
uncountable set, not containing $\min A$, such that $K_{\xi}\setminus K$
does not meet $J_{\eta}$ whenever $\xi$, $\eta\in C$ and $\eta<\xi$
(4A1Eb).   Let $D\subseteq C$ be such that $D$ and $C\setminus D$
are both uncountable.

Note that $K\subseteq K_{\eta}\subseteq J_{\xi}$ whenever $\eta$,
$\xi\in A$ and $\eta<\xi$, so that $K\subseteq J_{\xi}$ for every
$\xi\in C$.   Consequently $U'_{\xi}$, $V_{\xi}$ and $V'_{\xi}$ all
belong to $\Cal C(K_{\xi}\setminus K)$ for every $\xi\in C$.

\medskip

{\bf (f)} Consider the open set

\Centerline{$G=\bigcup_{\xi\in D}G_{\xi}\subseteq X$.}

\noindent At this point the argument divides.

\medskip

\quad{\bf case 1} Suppose $\mu^*(G\cap f[\tilde U])>0$.   Then there is
a Baire set $Q\subseteq G$ such that $\mu^*(Q\cap f[\tilde U])>0$.   Let
$J\subseteq I$ be a countable set such that $f^{-1}[Q]$ is determined by
coordinates in $J$.
Let $\gamma\in C\setminus D$ be so large that $K_{\xi}\setminus K$ does
not meet $J$ for any $\xi\in A$ with $\xi\ge\gamma$.   Then 
$Q\cap Q_{J_{\gamma}}\cap f[\tilde U]$ is not empty;  take
$s\in \tilde U\cap f^{-1}[Q\cap Q_{J_{\gamma}}]$.   Because the
$K_{\xi}\setminus K$ are disjoint from each other and from 
$J\cup J_{\gamma}$ for $\xi>\gamma$, we may modify $s$ to form $s'$
such that
$s'\restr J\cup J_{\gamma}=s\restr J\cup J_{\gamma}$ and 
$s'\in U'_{\xi}\cap V'_{\xi}$ for every $\xi>\gamma$;  now
$s'\in\tilde U$ (because $K\subseteq J_{\gamma}$), so 
$s'\in\tilde U\cap U'_{\xi}\cap V'_{\xi}\subseteq f^{-1}[G'_{\xi}]$ and 
$f(s')\notin G_{\xi}$ whenever $\xi>\gamma$.   On the other hand, if $\xi\in D$ and 
$\xi<\gamma$,
$G_{\xi}\cap Q_{J_{\gamma}}=\emptyset$, while $s'\in f^{-1}[Q_{J_{\gamma}}]$ (because
$f^{-1}[Q_{J_{\gamma}}]$  is determined by coordinates in $J_{\gamma}$),
so again $f(s')\notin G_{\xi}$.

Thus $f(s')\notin G$.   But $s'\restr J=s\restr J$ so 
$f(s)\in Q\subseteq G$;  which is impossible.

This contradiction disposes of the possibility that 
$\mu^*(G\cap f[\tilde U])>0$.

\medskip

\quad{\bf case 2} Suppose that $\mu^*(G\cap f[\tilde U])=0$.   In this
case there is a negligible Baire set $Q\supseteq G\cap f[\tilde U]$.
Let $J\subseteq I$ be a countable set such that $f^{-1}[Q]$ is
determined
by coordinates in $J$.   Let $\gamma<\omega_1$ be such that 
$J\cap J_{\gamma}=J\cap\bigcup_{\xi<\omega_1}J_{\xi}$ and 
$J\cap K_{\xi}\setminus K=\emptyset$ for every $\xi\ge\gamma$.   Take $\xi\in D$ 
such that $\xi\ge\gamma$.   Then

\Centerline{$\tilde U\cap U'_{\xi}\cap V_{\xi}
\subseteq f^{-1}[G_{\xi}]\cap\tilde U
\subseteq f^{-1}[G\cap f[\tilde U]]
\subseteq f^{-1}[Q]$,}

\noindent so $\tilde U\subseteq f^{-1}[Q]$, because $U'_{\xi}\cap V_{\xi}$ is a 
non-empty member of $\Cal C(I\setminus J)$.   But this
means that $\mu^*f[\tilde U]=0$ and $\mu^*f[U_{\xi}]=0$.   On the other
hand, we have $s_{\xi}\in U_{\xi}\cap f^{-1}[Q_{J_{\xi}}]$, so
$U_{\xi}\notin\Cal C_0(J_{\xi})$ and $\mu^*f[U_{\xi}]>0$.\ \Bang

Thus this route also is blocked and we must abandon the original
hypothesis that there is a quasi-dyadic space with a semi-finite
completion regular topological measure which is not $\tau$-additive.
}%end of proof of 434Q

\leader{434R}{}\cmmnt{ There is a useful construction of Borel product
measures which can be fitted in here.

\medskip

\noindent}{\bf Proposition} Let $X$ and $Y$ be topological spaces with
Borel measures $\mu$ and $\nu$;  write $\Cal B(X)$, $\Cal B(Y)$ for the
Borel $\sigma$-algebras of $X$ and $Y$ respectively.   If {\it either}
$X$ is first-countable {\it or} $\nu$ is $\tau$-additive and effectively
locally finite, there is a Borel measure $\lambda_B$ on $X\times Y$
defined by the formula

\Centerline{$\lambda_BW
=\sup_{F\in\Cal B(Y),\nu F<\infty}\int\nu(W[\{x\}]\cap F)\mu(dx)$}

\noindent for every Borel set $W\subseteq X\times Y$.   Moreover

\quad(i) if $\mu$ is semi-finite, then $\lambda_B$ agrees with the
c.l.d.\ product measure $\lambda$ on $\Cal B(X)\tensorhat\Cal B(Y)$, and
the c.l.d.\ version $\tilde\lambda_B$ of $\lambda_B$ extends $\lambda$;

\quad(ii) if $\nu$ is $\sigma$-finite, then
$\lambda_BW=\int\nu W[\{x\}]\mu(dx)$ for every Borel set
$W\subseteq X\times Y$;

\quad(iii) if both $\mu$ and $\nu$ are $\tau$-additive and effectively
locally finite, then $\lambda_B$ is just the restriction of the
$\tau$-additive product measure $\tilde\lambda$\cmmnt{ (417D, 417G)}
to the Borel $\sigma$-algebra of $X\times Y$;  in particular,
$\lambda_B$ is $\tau$-additive.

%when is $\lambda_B$ inner regular wrt closed sets?

\proof{{\bf (a)} The point is that $x\mapsto\nu(W[\{x\}]\cap F)$ is
lower semi-continuous whenever $W\subseteq X\times Y$ is open and
$\nu F<\infty$.   \Prf\ Of course $W[\{x\}]$ is always open, so $\nu$
always measures $W[\{x\}]\cap F$.   Take any $\alpha\in\Bbb R$ and set
$G=\{x:x\in X,\,\nu(W[\{x\}]\cap F)>\alpha\}$;  let $x_0\in G$.

\medskip

\quad\grheada\ Suppose that $X$ is first-countable.   Let
$\sequencen{U_n}$ be a
non-increasing sequence running over a base of open neighbourhoods of
$x_0$.   For each $n\in\Bbb N$, set

\Centerline{$V_n=\bigcup\{V:V\subseteq Y$ is open,
$U_n\times V\subseteq W\}$.}

\noindent Then $\sequencen{V_n}$ is a non-decreasing sequence with union
$W[\{x\}]$, so there is an $n\in\Bbb N$ such that
$\nu(V_n\cap F)>\alpha$.
Now $V_n\subseteq W[\{x\}]$ for every $x\in U_n$, so $U_n\subseteq G$.

\medskip

\quad\grheadb\ Suppose that $\nu$ is $\tau$-additive and effectively
locally finite.   Set

\Centerline{$\Cal V=\{V:V\subseteq Y$ is open, $U\times V\subseteq W$
for some open set $U$ containing $x_0\}$.}

\noindent Then $\Cal V$ is an upwards-directed family of open sets with
union $W[\{x_0\}]$, so there is a $V\in\Cal V$ such that
$\nu(V\cap F)>\alpha$ (414Ea).   Let $U$ be an open set containing $x_0$
such that $U\times V\subseteq W$;  then $V\subseteq W[\{x\}]$ for every
$x\in U$, so $U\subseteq G$.

\medskip

\quad\grheadc\ Thus in either case we have an open set containing $x_0$
and included in $G$.   As $x_0$ is arbitrary, $G$ is open;  as $\alpha$
is arbitrary, $x\mapsto\nu(W[\{x\}]\cap F)$ is lower semi-continuous.\
\Qed

\medskip

{\bf (b)} It follows that $x\mapsto\nu(W[\{x\}]\cap F)$ is Borel
measurable whenever $W\subseteq X\times Y$ is a Borel set and
$\nu F<\infty$.
\Prf\ Let $\Cal W$ be the family of sets $W\subseteq X\times Y$ such
that $W[\{x\}]$ is
a Borel set for every $x\in X$ and $x\mapsto\nu(W[\{x\}]\cap F)$ is
Borel measurable.   Then every open subset of $X\times Y$ belongs to
$\Cal W$ (by (a) above), $W\setminus W'\in\Cal W$ whenever $W$,
$W'\in\Cal W$ and
$W'\subseteq W$, and $\bigcup_{n\in\Bbb N}W_n\in\Cal W$ whenever
$\sequencen{W_n}$ is a non-decreasing sequence in $\Cal W$.   By the
Monotone Class Theorem (136B), $\Cal W$ includes the $\sigma$-algebra
generated by the open sets, that is, the Borel $\sigma$-algebra of
$X\times Y$.\
\Qed

\medskip

{\bf (c)} It is now easy to check that
$W\mapsto\int\nu(W[\{x\}]\cap F)\mu(dx)$ is a Borel measure on
$X\times Y$ whenever $\nu F<\infty$, and therefore that $\lambda_B$, as
defined here, is a Borel measure.

\wheader{434R}{4}{2}{2}{108pt}

{\bf (d)} Now suppose that $\mu$ is semi-finite, and that
$W\in\Cal B(X)\tensorhat\Cal B(Y)$.   Then

$$\eqalignno{\lambda W
&=\sup_{\mu E<\infty,\nu F<\infty}\lambda(W\cap(E\times F))\cr
\displaycause{by the definition of `c.l.d.\ product measure', 251F}
&=\sup_{\mu E<\infty,\nu F<\infty}\int_E\nu(W[\{x\}]\cap F)\mu(dx)\cr
\displaycause{by Fubini's theorem, 252C, applied to the product of the
subspace measures $\mu_E$ and $\nu_F$}
&=\sup_{\nu F<\infty}\int\nu(W[\{x\}]\cap F)\mu(dx)\cr
\displaycause{by 213B, because $\mu$ is semi-finite}
&=\lambda_BW.\cr}$$

\medskip

{\bf (e)} If, on the other hand, $\nu$ is $\sigma$-finite, let
$\sequencen{F_n}$ be a non-decreasing sequence of sets of finite measure
covering $Y$;  then

$$\eqalign{\lambda_BW
&=\sup_{\nu F<\infty}\int\nu(W[\{x\}]\cap F)\mu(dx)
\ge\sup_{n\in\Bbb N}\int\nu(W[\{x\}]\cap F_n)\mu(dx)\cr
&=\int\sup_{n\in\Bbb N}\nu(W[\{x\}]\cap F_n)\mu(dx)
=\int\nu W[\{x\}]\mu(dx)
\ge\lambda_BW\cr}$$

\noindent for any Borel set $W\subseteq X$.

\medskip

{\bf (f)} If both $\mu$ and $\nu$ are $\tau$-additive and effectively
locally finite, so that we have a $\tau$-additive product measure
$\tilde\lambda$, then Fubini's theorem for such measures (417H) tells us
that $\lambda_BW=\tilde\lambda W$ at least when $W\subseteq X\times Y$
is a Borel set and $\tilde\lambda W$ is finite.   If $W$ is any Borel
subset of $X\times Y$, then, as in (d),

$$\eqalign{\lambda_BW
&=\sup_{\mu E<\infty,\nu F<\infty}\int_E\nu(W[\{x\}]\cap F)\mu(dx)\cr
&=\sup_{\mu E<\infty,\nu F<\infty}\tilde\lambda(W\cap(E\times F))
=\tilde\lambda W\cr}$$

\noindent by 417C(iii).
}%end of proof of 434R

\cmmnt{\medskip

\noindent{\bf Remark} The case in which $X$ is first-countable is due to
{\smc Johnson 82}.}

\vleader{60pt}{*434S}{}\cmmnt{ The concept of `universally measurable'
set
enables us to extend a number of ideas from earlier sections.   First,
recall a problem from the very beginning of measure theory on the real
line:  the composition of Lebesgue measurable functions need not be
Lebesgue measurable (134Ib), while the composition of a Borel measurable
function with a Lebesgue measurable function is measurable (121Eg).   In
fact we can replace `Borel measurable' by `universally measurable', as
follows.

\medskip

\noindent}{\bf Proposition} Let $(X,\Sigma,\mu)$ be a complete locally
determined measure space, $Y$ and $Z$ topological spaces, $f:X\to Y$ a
measurable function and $g:Y\to Z$ a universally measurable function.
Then $gf:X\to Z$ is measurable.   In particular, $f^{-1}[F]\in\Sigma$
for every universally measurable set $F\subseteq Y$.

\proof{ Let $H\subseteq Z$ be an open set and $E\in\Sigma$ a set of
finite measure.   Let $\mu_E$ be the subspace measure on $E$.   Then the
image measure $\nu=\mu_E(f\restr E)^{-1}$ is a complete totally finite
topological measure on $Y$, so its domain contains $g^{-1}[H]$, and

\Centerline{$E\cap(gf)^{-1}[H]
=(f\restr E)^{-1}[g^{-1}[H]]
\in\dom\mu_E\subseteq\Sigma$.}

\noindent As $E$ is arbitrary and $\mu$ is locally determined,
$(gf)^{-1}[H]\in\Sigma$;  as $H$ is arbitrary, $gf$ is measurable.

Applying this to $g=\chi F$, we see that $f^{-1}[F]\in\Sigma$ for every
universally measurable $F\subseteq Y$.
}%end of proof of 434S

\leader{*434T}{}\cmmnt{ The next remark concerns the concept
$\Bvalue{u\in E}$ of \S364.

\medskip

\noindent}{\bf Proposition} Let $(\frak A,\bar\mu)$ be a localizable
measure algebra.   Write $\Sigma_{\text{um}}$ for the algebra of
universally measurable subsets of $\Bbb R$.

(a) For any $u\in L^0=L^0(\frak A)$, we have a sequentially
order-continuous Boolean homomorphism
$E\mapsto\Bvalue{u\in E}:\Sigma_{\text{um}}\to\frak A$ defined by saying
that

$$\eqalign{\Bvalue{u\in E}
&=\sup\{\Bvalue{u\in F}:F\subseteq E\text{ is Borel}\}
=\sup\{\Bvalue{u\in K}:K\subseteq E\text{ is compact}\}\cr
&=\inf\{\Bvalue{u\in F}:F\supseteq E\text{ is Borel}\}
=\inf\{\Bvalue{u\in G}:G\supseteq E\text{ is open}\}\cr}$$

\noindent for every $E\in\Sigma_{\text{um}}$.

(b) For any $u\in L^0$ and universally measurable function
$h:\Bbb R\to\Bbb R$ we have a corresponding element $\bar h(u)$ of $L^0$
defined by the formula

\Centerline{$\Bvalue{\bar h(u)\in E}=\Bvalue{u\in h^{-1}[E]}$
for every $E\in\Sigma_{\text{um}}$, $u\in L^0$.}

\proof{ We can regard $(\frak A,\bar\mu)$ as the measure algebra of a
complete strictly localizable measure space $(X,\Sigma,\mu)$ (322O), in
which case $L^0$ can be identified with $L^0(\mu)$ 
(364Ic\formerly{3{}64Jc}).   Write
$\Cal B$ for the Borel $\sigma$-algebra of $\Bbb R$.

\medskip

{\bf (a)} Let $f:X\to\Bbb R$ be a $\Sigma$-measurable
function representing $u$.   Then $f^{-1}[E]\in\Sigma$ for every
$E\in\Sigma_{\text{um}}$, by 434S.   Setting
$\phi E=(f^{-1}[E])^{\ssbullet}$, $\phi:\Sigma_{\text{um}}\to\frak A$ is
a sequentially order-continuous Boolean homomorphism.

Now

\Centerline{$\phi E=\sup\{\Bvalue{u\in K}:
  K\subseteq E\text{ is compact}\}$}

\noindent for every $E\in\Sigma_{\text{um}}$.   \Prf\ If $H\in\Sigma$
and $\mu H<\infty$, then (writing $\mu_H$ for the subspace measure on
$H$) the image measure $\mu_H(f\restr H)^{-1}$ is a complete topological
measure, and its restriction $\nu$ to the Borel $\sigma$-algebra 
$\Cal B$ of
$\Bbb R$ is a totally finite Borel measure.   Now $E$ is measured by the
completion $\hat\nu$ of $\nu$, which is a Radon measure (256C), so for any
$\epsilon>0$ there are a compact $K\subseteq E$ and a Borel $F\supseteq E$
such that $\nu F=\hat\nu E\le\nu K+\epsilon$.   In this
case,

\Centerline{$\Bvalue{u\in K}=(f^{-1}[K])^{\ssbullet}
\Bsubseteq\phi E\Bsubseteq(f^{-1}[F])^{\ssbullet}=\Bvalue{u\in F}$,}

\noindent using the formula of 364Ib, while

$$\eqalign{\bar\mu(H^{\ssbullet}\Bcap\phi E)
&\le\bar\mu(H^{\ssbullet}\Bcap\Bvalue{u\in F})
=\mu(H\cap f^{-1}[F])\cr
&=\nu F
\le\nu K+\epsilon
=\bar\mu(H^{\ssbullet}\Bcap\Bvalue{u\in K})+\epsilon.\cr}$$

\noindent As $\epsilon$ is arbitrary, 

\Centerline{$H^{\ssbullet}\Bcap\phi E
=\sup\{H^{\ssbullet}\Bcap\Bvalue{u\in K}:
  K\subseteq E\text{ is compact}\}$;}
  
\noindent as $H$ is arbitrary, and $(\frak A,\bar\mu)$ is semi-finite,
$\phi E=\sup\{\Bvalue{u\in K}:K\subseteq E\text{ is compact}\}$.
\QeD

Applying this to $\Bbb R\setminus E$, we see that
$\phi E=\inf\{\Bvalue{u\in G}:G\supseteq E\text{ is open}\}$.
Of course it follows at once that

\Centerline{$\Bvalue{u\in E}
=\sup\{\Bvalue{u\in F}:F\subseteq E\text{ is Borel}\}
=\inf\{\Bvalue{u\in F}:F\supseteq E\text{ is Borel}\}$.}

\noindent We can therefore identify the sequentially
order-continuous Boolean homomorphism $\phi$ with
$E\mapsto\Bvalue{u\in E}$, as described.

\medskip

{\bf (b)} Once again identifying $u$ with $f^{\ssbullet}$ where
$f$ is $\Sigma$-measurable, we see that $hf$ is $\Sigma$-measurable (by
434S), so we have a corresponding element $(hf)^{\ssbullet}$ of $L^0$.
If $E\in\Sigma_{\text{um}}$, then

\Centerline{$\Bvalue{(hf)^{\ssbullet}\in E}
=((hf)^{-1}[E])^{\ssbullet}
=(f^{-1}[h^{-1}[E]])^{\ssbullet}
=\Bvalue{u\in h^{-1}[E]}$,}

\noindent using 434De to check that $h^{-1}[E]\in\Sigma_{\text{um}}$, so
that we can identify $(hf)^{\ssbullet}$ with $\bar h(u)$, as described.
}%end of proof of 434T

\exercises{
\leader{434X}{Basic exercises $\pmb{>}$(a)}
%\sqheader 434Xa
Let $A\subseteq[0,1]$ be any non-measurable set.   Show
that the subspace measure on $A$ is completion regular and
$\tau$-additive but not tight.
%434A

\sqheader 434Xb Let $X$ be any Hausdorff space with a point $x$ such
that $\{x\}$ is not a G$_{\delta}$ set;  for instance, $X=\omega_1+1$
and $x=\omega_1$, or $X=\{0,1\}^I$ for any uncountable set $I$ and $x$
any point of $X$.   Show that setting $\mu E=\chi E(x)$ we get a tight
Borel measure on $X$ which is not completion regular.
%434A

\sqheader 434Xc Let $X$ be a topological space.
(i) Show that if $A\subseteq X$ is universally measurable in $X$, then
$A\cap Y$ is universally measurable in $Y$ for any set $Y\subseteq X$.
(ii) Show that if $Y\subseteq X$ is universally measurable in $X$, and
$A\subseteq Y$ is universally measurable in $Y$, then $A$ is universally
measurable in $X$.   (iii) Suppose that $X$ is the product of a
countable family $\familyiI{X_i}$ of
topological spaces, and $E_i\subseteq X_i$ is a universally measurable
set for each $i\in I$.   Show that $\prod_{i\in I}E_i$ is universally
measurable in $X$.
%434D

\spheader 434Xd Let $X$ be an analytic 
Hausdorff space.   (i) Suppose that $Y$ is a
topological space and $W$ is a Borel subset of $X\times Y$.   Show that
$W[X]$ is a universally measurable subset of $Y$.   \Hint{423N.}  
(ii)\dvAnew{2010} Let $A$ be a subset of $X$.   Show that the following are
equiveridical:  ($\alpha$) $A$ is universally measurable in $X$;  ($\beta$)
$f^{-1}[A]$ is Lebesgue measurable for every Borel measurable function
$f:[0,1]\to X$;  ($\gamma$) $f^{-1}[A]$ is measured by the usual measure on
$\{0,1\}^{\Bbb N}$ for every continuous function $f:\{0,1\}^{\Bbb N}\to X$.
%434D

\spheader 434Xe Let $X$ be a Hausdorff space.
(i) Show that, for $A\subseteq X$, the following are equiveridical:
($\alpha$) $A$ is universally Radon-measurable in $X$;  ($\beta$) $A$ is
measured by every atomless Radon probability measure on $X$;  ($\gamma$)
$A\cap K$ is universally Radon-measurable in $K$ for every compact
$K\subseteq X$.
(ii) Show that if
$A\subseteq X$ is universally Radon-measurable in $X$, then $A\cap Y$ is
universally Radon-measurable in $Y$ for any set $Y\subseteq X$.
(iii) Show that if $Y\subseteq X$ is universally Radon-measurable in
$X$, and $A\subseteq Y$ is universally Radon-measurable in $Y$, then $A$
is universally Radon-measurable in $X$.
(iv) Show that if $\Cal G$ is an open cover of $X$, and $A\subseteq X$
is such that $A\cap G$ is universally Radon-measurable (in $G$ or in
$X$) for every $G\in\Cal G$, then $A$ is universally Radon-measurable in
$X$.
(v) Show that if $Y$ is another Hausdorff space, and
$\Sigma^{(X)}_{\text{uRm}}$, $\Sigma^{(Y)}_{\text{uRm}}$ are the
algebras of universally
Radon-measurable subsets of $X$, $Y$ respectively, then every continuous
function from $X$ to $Y$ is
$(\Sigma^{(X)}_{\text{uRm}},\Sigma^{(Y)}_{\text{uRm}})$-measurable.
(vi) Suppose that $X$ is the product of a countable family
$\familyiI{X_i}$ of
topological spaces, and $E_i\subseteq X_i$ is a universally
Radon-measurable set for each $i\in I$.   Show that $\prod_{i\in I}E_i$
is universally Radon-measurable in $X$.
%434E

\sqheader 434Xf
(i) Let $\mu_0$ be Dieudonn\'e's measure on $\omega_1$.
Give $\omega_1+1=\omega_1\cup\{\omega_1\}$ its compact Hausdorff order
topology, and define a Borel measure $\mu$ on $\omega_1+1$ by setting
$\mu E=\mu_0(E\cap\omega_1)$ for every Borel set $E\subseteq\omega_1+1$.
Show that $\mu$ is a complete probability measure and is neither
$\tau$-additive nor inner regular with respect to the closed sets.
(ii) Show that the universally measurable subsets of $\omega_1+1$ are
just its Borel sets.
\Hint{4A3J, 411Q.}   (iii) Show that every totally finite
$\tau$-additive topological measure on $\omega_1+1$ has a countable
support.   (iv) Show that every subset of $\omega_1+1$ is universally
Radon-measurable.
%434E

\spheader 434Xg(i) Show that there is a set $X\subseteq[0,1]$ such that
$K\cap X$ and $K\setminus X$ are both of cardinal $\frak c$ for every
uncountable compact set $K\subseteq[0,1]$.   \Hint{4A3Fa, 423K.}
(ii) Show that if we give $X$ its subspace topology, then every subset
of $X$ is universally Radon-measurable, but not every subset is
universally measurable.   \Hint{every compact subset of $X$ is
countable, so every Radon measure on $X$ is purely atomic, but $X$ has
full outer Lebesgue measure in $[0,1]$.}
%434E

\spheader 434Xh Show that a Hausdorff space $X$ is Radon iff ($\alpha$)
every compact subset of $X$ is Radon ($\beta$) for every non-zero
totally finite Borel measure $\mu$ on $X$ there is a compact subset $K$
of $X$ such that $\mu K>0$.   \Hint{434F(a-v).}
%434F

\sqheader 434Xi(i) Let $X$ and $Y$ be K-analytic Hausdorff spaces and
$f:X\to Y$ a continuous surjection.   Suppose that $F\subseteq Y$ and
that $f^{-1}[F]$ is universally Radon-measurable in $X$.   Show that $F$
is universally Radon-measurable in $Y$.   \Hint{432G.}
(ii) Let $X$ and $Y$ be analytic Hausdorff spaces and $f:X\to Y$ a Borel
measurable surjection.
Suppose that $F\subseteq Y$ and that $f^{-1}[F]$ is universally
Radon-measurable in $X$.   Show that $F$ is universally Radon-measurable
in $Y$.   \Hint{433D.}
%434F

\spheader 434Xj Show that if $X$ is a perfectly normal space then it is
Borel-measure-compact iff it is Borel-measure-complete.
%434I

\spheader 434Xk Let $X$ be a Radon Hausdorff space.   (i) Show that $X\times Y$ is
Borel-measure-compact whenever $Y$ is Borel-measure-compact.   (ii) Show that 
$X\times Y$ is Borel-measure-complete whenever $Y$ is Borel-measure-complete.   
%434I

\spheader 434Xl Show that if we give $\omega_1+1$ its order topology, it
is Borel-measure-compact but not Borel-measure-complete or pre-Radon,
and its open subset $\omega_1$ is not Borel-measure-compact.
%434J

\spheader 434Xm Show that $\Bbb R$, with the right-facing Sorgenfrey
topology, is Borel-measure-complete and Borel-measure-compact, but not
Radon or pre-Radon.
%434J

\spheader 434Xn Let $X$ be a topological space.   (i) Show that the
family of Borel-measure-complete subsets of $X$ is closed under
Souslin's operation.
(ii) Show that the union of a sequence of Borel-measure-compact subsets
of $X$ is Borel-measure-compact.
(iii) Show that if $X$ is Hausdorff then the family of pre-Radon subsets
of $X$ is closed under Souslin's operation.   \Hint{in (i) and (iii),
start by showing that the family under consideration is closed under
countable unions.}
%434J

\spheader 434Xo Show that $\ooint{0,1}^{\omega_1}$ is not pre-Radon.
%434J

\spheader 434Xp Let $X$ be a separable metrizable space.   Show that the following 
are equiveridical:  (i) $X$ is a Radon space;  (ii) $X$ is a pre-Radon space;   
(iii) there is a metric on $X$, defining the topology of $X$, such that $X$ is 
universally Radon-measurable in its completion;  (iv) whenever $Y$ is a separable 
metrizable space and $X'$ is a subset of $Y$ such that there is a Borel isomorphism 
between $X$ and $X'$, then $X'$ is universally measurable in $Y$;  (v) $X$ is a Radon 
space under any separable metrizable topology giving rise to the same Borel sets as 
the original topology.
%434J

\sqheader 434Xq Show that a K-analytic Hausdorff space is Radon iff all
its compact subsets are Radon.   \Hint{432B, 434Xh.}
%434K

\spheader 434Xr Suppose that $X$ is a K-analytic Hausdorff space such
that every Radon measure on $X$ is completion regular.   Show that $X$
is a Radon space.
%434K 434Xq

\spheader 434Xs Let $X$ and $Y$ be topological spaces, and suppose that
$Y$ has a countable network.   (i) Show that if $X$ is
Borel-measure-complete, then $X\times Y$ is Borel-measure-complete.
(ii) Show that if $X$ and $Y$ are Radon Hausdorff spaces, then
$X\times Y$ is Radon.
%434K

\spheader 434Xt Let $\sequencen{X_n}$ be a sequence of topological
spaces;  write $X=\prod_{n\in\Bbb N}X_n$ and $Z_n=\prod_{i\le n}X_i$ for
each $n$.   (i) Show that if every $Z_n$ is Borel-measure-complete, so
is $X$.  (ii) Show that if every $Z_n$ is Hausdorff and pre-Radon, so is
$X$.   (iii) Show that if every $Z_n$ is Hausdorff and Radon, so is $X$.
%434K

\spheader 434Xu(i) Let $\sequencen{X_n}$ be a sequence of Radon
Hausdorff
spaces such that $\prod_{i\le n}K_i$ is Radon whenever $n\in\Bbb N$ and
$K_i\subseteq X_i$ is compact for every $i\le n$.   Show that
$X=\prod_{n\in\Bbb N}X_n$ is Radon.
(ii) Let $\sequencen{X_n}$ be a sequence of Radon Hausdorff
spaces with countable networks.   Show that $\prod_{n\in\Bbb N}X_n$ is
Radon.
%434K

\spheader 434Xv Show that if, in 434R, $\nu$ is $\sigma$-finite, then
$\int g\,d\lambda_B=\iint g(x,y)\nu(dy)\mu(dx)$ for every
$\lambda_B$-integrable function $g:X\times Y\to\Bbb R$.
%434R

\spheader 434Xw Show that the product measure construction of 434R is
`associative' and `distributive' in the sense that (under appropriate
hypotheses) the product measures on $(X\times Y)\times Z$ and
$X\times(Y\times Z)$ agree, and those on
$\bigcup_{i,j\in\Bbb N}(X_i\times Y_j)$ and
$(\bigcup_{i\in\Bbb N}X_i)\times(\bigcup_{j\in\Bbb N}Y_j)$ agree.
%434R

\sqheader 434Xx Show that the product measure construction of 434R is
not `commutative';  indeed, taking $\mu=\nu$ to be Dieudonn\'e's measure
on $\omega_1$, show that the Borel measures $\lambda_1$, $\lambda_2$ on
$\omega_1^2$ defined by setting

\Centerline{$\lambda_1 W=\biggerint\nu W[\{\xi\}]\mu(d\xi)$,
\quad$\lambda_2 W=\biggerint\mu W^{-1}[\{\eta\}]\nu(d\eta)$}

\noindent are different.
%434R

\spheader 434Xy Read through \S271, looking for ways to apply the
concept `$\Pr(\pmb{X}\in E)$' for random variables $\pmb{X}$ and
universally measurable sets $E$.
%434T

\spheader 434Xz Let $\Sigma_{\text{um}}$ 
be the algebra of universally measurable 
subsets of $\Bbb R$, and $\mu$ the restriction of Lebesgue measure to
$\Sigma_{\text{um}}$.   
Show that $\mu$ is translation-invariant, but has no
translation-invariant lifting.   \Hint{345F.}
%434D query out of order

\leader{434Y}{Further exercises (a)}
%\spheader 434Ya
Set $X=\Bbb N\setminus\{0,1\}$.   For $m$, $p\in X$ set
$U_{mp}=m+p\Bbb N$;  show that $\{U_{mp}:m$, $p\in X$ are coprime$\}$ is
a base for a connected Hausdorff topology on $X$.
\Hint{$pq\in\overline{U}_{mp}$ for every $q\ge 1$.   See 
{\smc Steen \& Seebach 78}, ex.\ 60.}   Show that $X$ is a second-countable analytic
Hausdorff space and carries a Radon measure which is not completion
regular.
%434B

\spheader 434Yb Let $(X,\Sigma,\mu)$ be a semi-finite measure space and
$\frak T$ a topology on $X$ such that $\mu$ is inner regular with
respect to the closed sets.   Suppose that $Y$ is a topological space
with a countable network consisting of universally measurable sets, and
that $f:X\to Y$ is measurable.   Show that $f$ is almost continuous.
%4{}18J 4{}18Yg 434D

\spheader 434Yc Let $X$ be a Hausdorff space.   Let $\Sigma$ be the
family of those subsets $E$ of $X$ such that $f^{-1}[E]$ has the Baire
property in $Z$ whenever $Z$ is a compact Hausdorff space and $f:Z\to X$
is continuous.   Show that $\Sigma$ is a $\sigma$-algebra of subsets of
$X$ closed under Souslin's operation.   Show that every member of
$\Sigma$ is universally Radon-measurable.
%434E

\spheader 434Yd Let $X$ be a completely regular Hausdorff space.   Show
that the following are equiveridical:  ($\alpha$) $X$ is Radon;
($\beta$) $X$ is a universally measurable subset of its
Stone-\v{C}ech compactification;  ($\gamma$) whenever $Y$ is a Hausdorff
space and $X'$ is a subspace of $Y$ which is homeomorphic to $X$, then
$X'$ is universally measurable in $Y$.
%434F

\spheader 434Ye Let $X$ be a completely regular Hausdorff space.   Show
that the following are equiveridical:  ($\alpha$) $X$ is pre-Radon;
($\beta$) $X$ is a universally Radon-measurable subset of its
Stone-\v{C}ech compactification;  ($\gamma$) whenever $Y$ is a Hausdorff
space and $X'$ is a subspace of $Y$ which is homeomorphic to $X$, then
$X'$ is universally Radon-measurable in $Y$.
%434J

\spheader 434Yf Set $X=\omega_1+1$, with its order topology, and let
$\Sigma$ be the $\sigma$-algebra of subsets of $X$ generated by the
countable sets and the set $\Omega$ of limit ordinals in $X$.   Show
that there is a unique probability measure $\mu$ on $X$ with domain
$\Sigma$ such that $\mu\xi=\mu\Omega=0$ for every $\xi<\omega_1$.   Show
that $\mu$ is inner regular with respect to the Borel sets, is defined
on a base for the topology of the compact Hausdorff space $X$, but has
no extension to a topological measure on $X$.
%434Xl 434J

\spheader 434Yg Let $X$ be a metrizable space without isolated points,
and $\mu$ a $\sigma$-finite Borel measure on $X$.   Show
that there is a conegligible meager set.   \Hint{there is a dense set
$D\subseteq X$ such that $\{\{d\}:d\in D\}$ is
$\sigma$-metrically-discrete.}
%434L

\spheader 434Yh Give an example of a Hausdorff uniform space $(X,\Cal W)$ 
with a quasi-Radon probability measure which is not inner regular with
respect to the totally bounded sets.
%434L

\spheader 434Yi Show that $\beta\Bbb N$ is not countably tight,
therefore not Borel-measure-complete.
%434N

\spheader 434Yj Let $X$ be a quasi-dyadic space.   Suppose that
$\langle(G_{\xi},H_{\xi})\rangle_{\xi<\omega_1}$ is a family of pairs of
disjoint non-empty open sets.   Show that there is an uncountable
$A\subseteq\omega_1$ such that $\bigcap_{\xi\in
B}G_{\xi}\cap\bigcap_{\xi\in A\setminus B}H_{\xi}$ is non-empty for
every $B\subseteq A$.   \Hint{start by supposing that $X$ is itself a
product of separable metrizable spaces, and that every $G_{\xi}$, $H_{\xi}$
is an open cylinder set;  use the $\Delta$-system Lemma.}
%434P

\spheader 434Yk(i) Show that the split interval is not quasi-dyadic.
(ii) Show that $\Bbb R$, with the Sorgenfrey right-facing topology, is
not quasi-dyadic.   (iii) Show that $\omega_1$ and $\omega_1+1$, with
their order topologies, are not quasi-dyadic.   \Hint{434Yj.}
%434P, 434Yj, 4{}23X?

\spheader 434Yl Show that a perfectly normal quasi-dyadic space is
Borel-measure-compact.
%434Q

\spheader 434Ym Let $\sequencen{X_n}$ be a sequence of first-countable
spaces, and $\mu_n$ a Borel probability measure on $X_n$ for each $n$.   
For $n\in\Bbb N$ set $Z_n=\prod_{i<n}X_i$, and let $\lambda_n$ be the
product Borel measure on $Z_n$ constructed by repeatedly using the
method of 434R (cf.\ 434Xw).   Show that there is a unique Borel measure
$\lambda$ on $Z=\prod_{n\in\Bbb N}X_n$ such that all the canonical maps
from $Z$ to $Z_n$ are \imp.   Show that, for any $n$, $\lambda$ can be
identified with the product of $\lambda_n$ and a suitable product
measure on $\prod_{i\ge n}X_i$.
%434R

\spheader 434Yn ({\smc Aldaz 97}) A topological space $X$ is {\bf
countably metacompact} if whenever $\Cal G$ is a countable open cover of
$X$ then there is a point-finite open cover $\Cal H$ of $X$ refining
$G$.
(i) Show that $X$ is countably metacompact iff whenever
$\sequencen{F_n}$ is a
non-increasing sequence of closed sets with empty intersection in $X$
then there is a sequence $\sequencen{G_n}$ of open sets, with empty
intersection, such that $F_n\subseteq G_n$ for every $n$.
(ii) Let $X$
be any topological space and $\nu:\Cal PX\to[0,1]$ a finitely additive
functional such that $\nu X=1$.   Show that there is a finitely additive
$\nuprime:\Cal PX\to[0,1]$ such that
$\nu F\le\nuprime F=\inf\{\nuprime G:G\supseteq F$ is open$\}$ for every
closed $F\subseteq X$.   \Hint{413Q.}
(iii) Show that if $X$ is countably metacompact and $\mu$ is any Borel
probability measure on $X$, there is a Borel probability measure $\mu'$
on $X$, inner regular with respect to the closed sets, such that
$\mu F\le\mu' F$ for every closed set $F\subseteq X$;  so that $\mu$ and
$\mu'$ agree on the Baire $\sigma$-algebra of $X$.
%+

\spheader 434Yo Let $X$ be a totally ordered set with its order
topology.   Show that any $\tau$-additive Borel probability measure on
$X$ has countable Maharam type.
\Hint{$\{\ocint{-\infty,x}^{\ssbullet}:x\in X\}$ generates the measure
algebra.}
%+

\spheader 434Yp ({\smc Oxtoby 70}) Let $\mu$ be an atomless strictly 
positive 
Radon probability measure on $\BbbN^{\Bbb N}$.   (i) Show that if 
$\sequencen{\alpha_n}$ is any sequence in $[0,1]$ such that 
$\sum_{n=0}^{\infty}\alpha_n=1$, then there is a partition $\sequencen{U_n}$ of 
$\BbbN^{\Bbb N}$ into open sets such that 
$\mu U_n=\alpha_n$ for every $n$.   (ii) Show that if $\nu$ is any other 
atomless strictly positive Radon probability measure on 
$\BbbN^{\Bbb N}$, there is a homeomorphism 
$f:\BbbN^{\Bbb N}\to\Bbb N^{\Bbb N}$ such that $\nu=\mu f^{-1}$.
%434+

\spheader 434Yq If $X$ is a topological space and $\rho$ is a metric on
$X$, $X$ is {\bf$\sigma$-fragmented} by $\rho$ if for every $\epsilon>0$
there is a countable cover $\Cal A$ of $X$ such that whenever 
$\emptyset\ne B\subseteq A\in\Cal A$ there is a non-empty relatively open
subset of $B$ of $\rho$-diameter at most $\epsilon$.   Now suppose that $X$
is a Hausdorff space which is $\sigma$-fragmented by a metric $\rho$ such
that (i) $X$ is complete under $\rho$ (ii) the topology generated by $\rho$
is finer than the given topology on $X$.   Show that $X$ is a pre-Radon
space.

\spheader 434Yr Let $X$ be a Hausdorff space and $\mu$ an atomless strictly
localizable tight Borel measure on $X$.  
Show that $\mu$ is $\sigma$-finite.
\Hint{{\smc Fremlin n05}.}

\spheader 434Ys\dvAnew{2010}
If $X$ is a topological space, a set $A\subseteq X$ is 
{\bf universally capacitable} if 
$c(A)=\sup\{c(K):K\subseteq A$ is compact$\}$ for every Choquet capacity 
$c$ on $X$.   (i) Show that if $X$ is a Hausdorff space and $\pi_1$,
$\pi_2:X\times X\to X$ are the coordinate maps, then we have a 
Choquet capacity $c$
on $X\times X$ defined by saying that $c(A)=0$ if $A\subseteq X\times X$
and there is a Borel set $E\subseteq X$ including $\pi_1[A]$ and disjoint
from $\pi_2[A]$, and $c(A)=1$ for other $A\subseteq X\times X$.
(ii) Show that there is a universally measurable subset of $\Bbb R$ which
is not universally capacitable.   \Hint{423L.}
%434D out of order query

\spheader 434Yt\dvAnew{2010}
Let $X$ be a Hausdorff space such that there is a countable family 
$\Cal A$ of universally Radon-measurable subsets of $X$ which separates the
points of $X$ in the sense that whenever $I\in[X]^2$ there is an 
$A\in\Cal A$ such that $\#(I\cap A)=1$.   Show that two
Radon probability measures on $X$ which agree on $\Cal A$ are identical.
%416Yh 434E out of order query
}%end of exercises

\leader{434Z}{Problems (a)} Must every Radon compact Hausdorff space be
sequentially compact?

\spheader 434Zb Must a Hausdorff continuous image of a Radon compact
Hausdorff space be Radon?

\endnotes{\Notesheader{434} I said that the fundamental question of
topological measure theory is `which measures can appear on which
topological spaces'?   In this section I have concentrated on Borel
measures, classified according to the scheme laid out in \S411.   (Of
course there are other kinds of classification.   One of the most
interesting is the Maharam classification of Chapter 33:  we can ask
what measure algebras can appear from topological measures on a given
topological space.   I will return to this idea in \S531 of Volume 5;
for the moment I pass it by, with only 434Yo to give a taste.)
We can ask
this question from either of two directions.   The obvious approach is
to ask, for a given class of topological spaces, which types of measure
can appear.   But having discovered that (for instance) there are
several types of topological space on which all (totally finite) Borel
measures are tight, we can use
this as a definition of a class of topological spaces, and ask the
ordinary questions about this class.   Thus we have `Radon',
`Borel-measure-complete', `Borel-measure-compact' and `pre-Radon' spaces
(434C, 434G).   I have given precedence to the first partly
to honour the influence of {\smc Schwartz 73} and partly because a
compact Hausdorff space is always Borel-measure-compact and pre-Radon
(434Hb, 434Jf) and is Borel-measure-complete iff it is Radon (434Ka).
In effect, `Borel-measure-complete' means `Borel measures are
quasi-Radon'
(434Ib), `pre-Radon' means `quasi-Radon measures are Radon' (434Jb), and
`Radon' means `Borel measures are Radon' (434F(a-iii)).   These slogans
have to be interpreted with care;  but it is true that a Hausdorff space
is Radon iff it is both Borel-measure-complete and pre-Radon (434Ka).

The concept of `Radon' space is in fact one of the important
contributions of measure
theory to general topology, offering a variety of challenging
questions.   One which has attracted some attention is the problem of
determining when products of Radon spaces are Radon.   Uncountable
products hardly ever are (434K);  for countable products it is enough
to understand products of finitely many compact spaces (434Xu);  but the
product of two compact spaces already seems to lead us into undecidable
questions (438Xq, {\smc Wage 80}).   Two more very natural questions are
in 434Z.   One of the obstacles to the investigation is the rather small
number of Radon compact Hausdorff spaces which are known.   I should
remark that if the continuum hypothesis (for instance) is true, then
every compact Hausdorff space in which countably compact sets are closed
is sequentially compact ({\smc Ismail \& Nyikos 80}, or
{\smc Fremlin 84}, 24Nc), so that in this case we have a quick answer to
434Za.

You will recognise the construction of 434M as a universal version of
Dieudonn\'e's measure (411Q).   `Tightness' is of great
interest for other reasons ({\smc Engelking 89}), and here is
very helpful in giving quick proofs that spaces are not Radon (434Yi).

A large proportion of the definitions in general topology can be
regarded as different abstractions from the concept of metrizability.
Countable tightness is an obvious example;  so is `first-countability'
(434R).   In quite a different direction we have `metacompactness'
(438J, 434Yn).   The construction of the product measure in 434R is an
obvious idea, as soon as you have seen Fubini's theorem, but it is not
obvious just when it will work.

`Quasi-dyadic' spaces are a relatively recent invention;  I introduce
them here only as a vehicle for the argument of 434Q.   Of course a
dyadic space is
quasi-dyadic;   for basic facts on dyadic spaces, see 4A2D, 4A5T and
{\smc Engelking 89}, \S3.12 and 4.5.9-4.5.11.
}%end of notes

\discrpage

