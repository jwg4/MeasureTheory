\frfilename{mt567.tex}
\versiondate{31.10.14}
\copyrightdate{2006}

\def\Game{\mathop{\text{Game}}\nolimits}
\def\StrI{\mathop{\text{Str}}\nolimits_{\text{I}}}
\def\StrII{\mathop{\text{Str}}\nolimits_{\text{II}}}
\def\WO{\mathop{\text{WO}}}

\def\chaptername{Choice and determinacy}
\def\sectionname{Determinacy}

\newsection{567}

So far, this chapter has been looking at set theories which are weaker than
the standard theory ZFC, and checking which of the principal results of
measure theory can still be proved.   I now turn to an axiom which directly
contradicts the axiom of choice, and leads to a very different world.
This is AD, the `axiom of determinacy', defined in terms of strategies for
infinite games (567A-567C).  %567A 567B 567C
The first step is to confirm that we automatically have a weak version of
countable choice which is enough to make Lebesgue measure well-behaved
(567D-567E).   Next, in separable metrizable spaces all
subsets are universally measurable and have the Baire property (567G).
Consequently (at least when we can use AC($\omega$)) linear operators
between Banach spaces are bounded (567H), additive functionals on
$\sigma$-complete Boolean algebras are countably additive (567J), and
many $L$-spaces are reflexive (567K).   In a different direction, we
find that $\omega_1$ is \2vm\ (567L) and that there are many surjections
from $\Bbb R$ onto ordinals (567M).

At the end of the section I include two celebrated results in ZFC
(567N, 567O) which depend on some of the same ideas.

\leader{567A}{Infinite games} \cmmnt{ I return to
an idea introduced in \S451. %451V

\medskip

}{\bf (a)} Let $X$ be a non-empty set and $A$ a subset of
$X^{\Bbb N}$.   In the corresponding infinite game $\Game(X,A)$, players I
and II choose members of $X$ alternately, so that I chooses
$x(0),x(2),\ldots$ and II chooses $x(1),x(3),\ldots$;  a {\bf play} of the
game is an element of $X^{\Bbb N}$;  player I wins the play $x$ if
$x\in A$, otherwise II wins.   A {\bf strategy} for I is a function
$\sigma:\bigcup_{n\in\Bbb N}X^n\to X$;  a play $x\in X^{\Bbb N}$ is {\bf
consistent} with $\sigma$ if
$x(2n)=\sigma(\ofamily{i}{n}{x(2i+1)})$ for every $n$\cmmnt{, that is,
if I uses the function $\sigma$ to decide his move from the previous
moves by his opponent};
$\sigma$ is a {\bf winning strategy} if every play consistent with $\sigma$
belongs to $A$\cmmnt{, that is, if I wins whenever he follows the
strategy $\sigma$}.   Similarly, a strategy for II is a function
$\tau:\bigcup_{n\ge 1}X^n\to X$;  a play $x$ is consistent with $\tau$ if
$x(2n+1)=\tau(\langle x(2i)\rangle_{i\le n})$ for every $n$;  and $\tau$
is a winning strategy for II if
$x\notin A$ whenever $x\in X^{\Bbb N}$ and $x$ is consistent with $\tau$.

\spheader 567Ab A set $A\subseteq X^{\Bbb N}$ is {\bf determined} if
either I or II has a winning strategy in $\Game(X,A)$.
\cmmnt{Note that we need to know the set $X$ as well as the set $A$ to
specify the game in question.}

\spheader 567Ac It will sometimes be convenient to describe games with
`rules', so that the players are required to choose points in subsets
of $X$ (determined by the moves so far) at each move.   Such a description
can be regarded as specifying $A$ in the form $(A'\cup G)\setminus H$,
where $G$ is the set of plays in which II is the first to break a rule, $H$
is the set of plays in which I is the first to break a rule, and $A'$ is
the set of plays in which both obey the rules and I wins.

\spheader 567Ad Not infrequently the `rules' will specify different sets
for the moves of the two players, so that I always chooses a point in
$X_1$ and II always chooses a point in $X_2$;
setting $X=X_1\cup X_2$ we can reduce this
to the formalization above.

\leader{567B}{Theorem} Let $X$ be a non-empty well-orderable set.
Give $X$ its discrete topology and $X^{\Bbb N}$ the product topology.
If $F\subseteq X^{\Bbb N}$ is closed then $\Game(X,F)$ is determined.

\proof{{\bf (a)} Fix a well-ordering $\preccurlyeq$ of $X$.
Define $\family{\xi}{\On}{W_{\xi}}$ by setting

\Centerline{$W_0
=\{w:w\in\bigcup_{n\in\Bbb N}X^{2n+1}$, $w\not\subseteq x$ for any
$x\in F\}$,}

$$\eqalign{W_{\xi}
&=\{w:w\in\bigcup_{n\in\Bbb N}X^{2n+1}
  \text{, there is some }t\in X\text{ such that}\cr
&\mskip200mu w^{\smallfrown}\fraction{t}^{\smallfrown}\fraction{u}
   \in\bigcup_{\eta<\xi}W_{\eta}
\text{ for  every }u\in X\}\cr}$$

\noindent if $\xi>0$.  (As in 562A, I write $\fraction{t}\in X^1$ for the
one-term sequence with value $t$, and $^{\smallfrown}$ for concatenation of
sequences.)   If $w\in W_0$ then of course
$w^{\smallfrown}\fraction{t}^{\smallfrown}\fraction{u}\in W_0$ for
all $t$, $u\in X$;  so $W_0\subseteq W_1$, and of course
that $W_{\xi}\subseteq W_{\xi'}$ whenever $1\le\xi\le\xi'$ in $\On$.
There is therefore an ordinal $\zeta$ such that $W_{\zeta+1}=W_{\zeta}$;
write $W$ for $W_{\zeta}$.

For $w\in W$, let $r(w)\le\zeta$ be the least ordinal such that
$w\in W_{r(w)}$.   If $r(w)>0$ then there is some $t\in X$ such that
such that
$w^{\smallfrown}\fraction{t}^{\smallfrown}\fraction{u}
\in\bigcup_{\eta<r(w)}W_{\eta}$, that
is, $r(w^{\smallfrown}\fraction{t}^{\smallfrown}\fraction{u})<r(w)$, for
every $u\in X$.

Let
$V$ be the set of those $v\in\bigcup_{n\in\Bbb N}X^{2n}$ such that there is
a $u\in X$ such that $v^{\smallfrown}\fraction{u}\notin W$.
Observe that if $w\in\bigcup_{n\in\Bbb N}X^{2n+1}\setminus W$ then
$w\notin W_{\zeta+1}$ so $w^{\smallfrown}\fraction{t}\in V$ for every
$t\in X$.

\medskip

{\bf (b)} Suppose that $\emptyset\in V$.   Define
$\sigma:\bigcup_{n\in\Bbb N}X^n\to X$ inductively by saying that

\inset{$\sigma(\emptyset)$ is the $\preccurlyeq$-least member
$t$ of $X$ such
that the one-element sequence $\emptyset^{\smallfrown}\fraction{t}$
does not belong to $W$,

if $v\in X^{n+1}$ and
$w=(\sigma(v\restr 0),v(0),\sigma(v\restr 1),v(1),\ldots,
\sigma(v\restr n),v(n))\in V$,
take $\sigma(v)$ to be the $\preccurlyeq$-least member $t$ of $X$ such that
$w^{\smallfrown}\fraction{t}\notin W$,

for other $v\in X^{n+1}$ take $\sigma(v)$ to be the $\preccurlyeq$-least
member of $X$.}

\noindent Then $\sigma$ is a winning strategy for I.   \Prf\ If
$x$ is a play consistent with $\sigma$, then an induction on $n$ shows that
$x\restr 2n\in V$ and $x\restr 2n+1\notin W$ for every $n$.   In
particular, $x\restr 2n+1\notin W_0$, that is, there is a member of $F$
extending $x\restr 2n+1$, for every $n$.   As $F$ is closed, $x\in F$ and I
wins the play $x$.\ \Qed

\medskip

{\bf (c)} Suppose that $\emptyset\notin V$, that is, $w\in W$ for every
$w\in X^1$.   Define
$\tau:\bigcup_{n\ge 1}X^n\to X$ inductively by saying

\inset{if $v\in X^n$ and
$w=(v(0),\tau(v\restr 1),v(1),\tau(v\restr 2),\ldots,v(n-1))$ belongs to
$W\setminus W_0$,
then $\tau(v)$ is the $\preccurlyeq$-least $t\in X$ such that
$r(w^{\smallfrown}\fraction{t}^{\smallfrown}\fraction{u})<r(w)$
for every $u\in X$,

for other $v\in X^n$, $\tau(v)$ is the $\preccurlyeq$-least member of $X$.}

\noindent Then $\tau$ is a winning strategy for II.   \Prf\ Let $x$ be a
play consistent with $\tau$.   Then an induction on $n$ tells us that

\Centerline{$x\restr 2n+1\in W$,
\quad if $x\restr 2n+1\notin W_0$ then $r(x\restr 2n+3)<r(x\restr 2n+1)$}

\noindent for every $n\in\Bbb N$.  Since $\sequencen{r(x\restr 2n+1)}$
cannot be strictly decreasing, there is some $n\in\Bbb N$ such that
$x\restr 2n+1\in W_0$ and $x\notin F$.   Thus II wins the play $x$.\ \Qed

\medskip

{\bf (d)} Putting (b) and (c) together we see that $F$ is determined.
}%end of proof of 567B

\vleader{36pt}{567C}{The axiom of determinacy (a)}
The standard `axiom of determinacy' is the statement

\Centerline{(AD)\quad Every subset of $\BbbN^{\Bbb N}$ is determined.}

\noindent Evidently it will follow that every subset of $X^{\Bbb N}$ is
determined for any countable set $X$.  \cmmnt{(If $X\subseteq\Bbb N$,
a game on $X$ can be regarded as a game on $\Bbb N$ in which there is a
rule that the players must always choose points in $X$.   See also 567Xc.)}

\spheader 567Cb At the same time, it will be useful to consider a weak form
of the axiom of countable choice:   for any set $X$, write
AC($X;\omega$) for the statement

\Centerline{$\prod_{n\in\Bbb N}A_n\ne\emptyset$ whenever $\sequencen{A_n}$
is a sequence of non-empty subsets of $X$.}

\leader{567D}{Theorem}\cmmnt{ ({\smc Mycielski 64})}
AD implies AC($\Bbb R;\omega$).

\proof{ Since we know that $\Bbb R$ is equipollent with $\NN$, we can look
at AC($\NN;\omega$).
Let $\sequencen{A_n}$ be a sequence of non-empty subsets of $\NN$.   Set

\Centerline{$A
=\{x:x\in\BbbN^{\Bbb N}$, $\sequencen{x(2n+1)}\notin A_{x(0)}\}$.}

\noindent Then I has no winning strategy in $\Game(\Bbb N,A)$,
because if $\sigma$
is a strategy for I in $\Game(\Bbb N,A)$ set $k=\sigma(\emptyset)$;
there is a
point $y\in A_k$, and II need only play $x(2n+1)=y(n)$ for each $n$.

So II has a winning strategy $\tau$ say.   Define
$g:\Bbb N\to\BbbN^{\Bbb N}$ by saying that
$g(n)(i)=\tau(e_{ni})$ for $n$, $i\in\Bbb N$,
where $e_{ni}\in\BbbN^{i+1}$, $e_{ni}(0)=n$,
$e_{ni}(j)=0$ for $1\le j\le i$.   If now $n\in\Bbb N$,
I plays $(n,0,0,\ldots)$ and
II follows the strategy $\tau$, the resulting play
$(n,g(n)(0),0,g(n)(1),0,\ldots)$ must not belong to $A$ so $g(n)\in A_n$.
}%end of proof of 567D

\leader{567E}{Consequences of AC($\Bbb R;\omega$)} Suppose that
AC($\Bbb R;\omega$) is true.

\spheader 567Ea If a set $X$ is the image of a subset $Y$ of $\Bbb R$
under a function $f$, then AC($X;\omega$) is true.   \prooflet{\Prf\ If
$\sequencen{A_n}$ is a sequence of non-empty subsets of $X$, then
there is an $x\in\prod_{n\in\Bbb N}f^{-1}[A_n]$, and
$\sequencen{f(x(n))}\in\prod_{n\in\Bbb N}A_n$.\ \Qed}

\spheader 567Eb In particular, taking
$S^*=\bigcup_{n\ge 1}\BbbN^n$\cmmnt{ as
in \S562}, AC($\Cal PS^*;\omega$) is true.   It follows that (in any
second-countable space $X$) every sequence of
codable Borel sets is codable and the family of codable Borel sets is a
$\sigma$-algebra, coinciding with the Borel $\sigma$-algebra $\Cal B(X)$
on its
ordinary definition.   Moreover, since $\Cal B(X)$ is an image of
$\Cal PS^*$, we have AC($\Cal B(X);\omega$)\cmmnt{,
countable choice for collections
of Borel sets}.  Similarly, the family of codable Borel functions
becomes the ordinary family of Borel-measurable functions, and we have
countable choice for sets of Borel real-valued functions on $X$.

\spheader 567Ec Consequently the results of \S562-565 give us
large parts of the elementary theory of Borel measures on
second-countable spaces.   At the same time, if $X$ is second-countable,
the union of a sequence of meager subsets of $X$ is
meager\cmmnt{ (because we have
countable choice for sequences of nowhere dense closed sets)}, so the
Baire-property algebra of $X$ is a $\sigma$-algebra.

\spheader 567Ed We also find that the supremum of a sequence
of countable
ordinals is again countable.   \prooflet{\Prf\ Let $\sequencen{\xi_n}$
be a sequence in $\omega_1$.   Using
AC($\Bbb R;\omega)$), we can choose for each $n\in\Bbb N$ a subset
$\preccurlyeq_n$ of $\Bbb N\times\Bbb N$ which is a well-ordering
of $\Bbb N$ with order type $\max(\omega,\xi_n)$.   Now we have a
well-ordering $\preccurlyeq$ of $\Bbb N^2$ defined by saying that
$(i,j)\preccurlyeq(i',j')$ if $i<i'$ or $i=i'$ and $j\preccurlyeq_ij'$.
In this case, the order type $\xi$ of $\preccurlyeq$ will be greater than
or equal to
every $\xi_n$, so that $\sup_{n\in\Bbb N}\xi_n\le\xi$ is
countable.\ \Qed}

\leader{567F}{Lemma}\cmmnt{ (see {\smc Mycielski \& \'Swierczkowski 64})}
[AC($\Bbb R;\omega$)] Suppose that
$A\subseteq\{0,1\}^{\Bbb N}$ is a continuous image of a subset $B$ of
$\{0,1\}^{\Bbb N}$ such that $(h^{-1}[B]\cap F)\cup H\subseteq\NN$
is determined
whenever $h:\BbbN^{\Bbb N}\to\{0,1\}^{\Bbb N}$ is continuous,
$F\subseteq\BbbN^{\Bbb N}$ is closed and $H\subseteq\BbbN^{\Bbb N}$ is
open.

(a) $A$ is universally measurable.

(b) $A$ has the Baire property in $\{0,1\}^{\Bbb N}$.

\proof{ Fix a continuous surjection $f:B\to A$.   Let $\Cal E$ be the
countable algebra
of subsets of $\{0,1\}^{\Bbb N}$ determined by coordinates in finite
sets, that is to say, the algebra of open-and-closed subsets of
$\{0,1\}^{\Bbb N}$ (311Xh).

\medskip

{\bf (a)(i)} Let $\mu$ be a Borel probability measure on
$\{0,1\}^{\Bbb N}$ and $\hat\mu$ its completion.
If $Z\subseteq\{0,1\}^{\Bbb N}$ is closed and not
negligible, then at least one of $Z\cap A$, $Z\setminus A$
has non-zero inner measure.

\Prf\ Let $\sequencen{E_n}$ enumerate
$\Cal E$.   Set $\epsilon_n=2^{-2n-2}\mu Z$ for $n\in\Bbb N$.
In $(\{0,1\}\times\Cal E)^{\Bbb N}$ consider the
game in which the players choose $(k_0,K_0),(k_1,K_1),\ldots$ such that
$K_0=Z$ and for each $n\in\Bbb N$

\Centerline{$k_n\in\{0,1\}$,
\quad$K_n\in\Cal E$,
\quad$\mu K_{2n+1}\le\epsilon_n$.}

\noindent I wins if $y=\sequencen{k_{2n}}$ belongs to $B$ and
$f(y)\notin\bigcup_{n\in\Bbb N}K_{2n+1}$.   Observe that when $y\in B$,
$f(y)\in\bigcup_{n\in\Bbb N}K_{2n+1}$ iff there is an $m\in\Bbb N$
such that $f(w)\in\bigcup_{i<m}K_{2i+1}$ whenever $w\in B$ and
$w\restr m=y\restr m$;  so  I
wins iff $y\in B$ and at every stage $((k_0,K_0),\ldots,(k_{2m},K_{2m}))$
there is a $w\in B$ such that $w(i)=k_{2i}$ for $i<m$ and
$f(w)\notin\bigcup_{i<m}K_{2i+1}$.   So the payoff set $D$
of plays $\sequencen{(k_n,K_n)}$ won by I is of the form
$(h^{-1}[B]\cap F)\cup H$ where
$h:(\{0,1\}\times\Cal E)^{\Bbb N}\to\{0,1\}^{\Bbb N}$ is
continuous, $F\subseteq (\{0,1\}\times\Cal E)^{\Bbb N}$ is closed and
$H\subseteq(\{0,1\}\times\Cal E)^{\Bbb N}$ is open.
(Here $H$ is
the set of plays which are won because II is the first to break a rule.)
Consequently $D$ is determined.

\medskip

\qquad{\bf case 1} Suppose that I has a winning strategy $\sigma$.
For each play $\sequencen{(k_n,K_n)}$ consistent with $\sigma$,
$f(\sequencen{k_{2n}})$ is defined and belongs to $A$.
Since the set of plays consistent with $\sigma$ is a closed subset of
$(\{0,1\}\times\Cal E)^{\Bbb N}$, the set $C$ of points obtainable in this
way is an analytic subset of $Z$, therefore measured by $\hat\mu$ (563I).
\Quer\ If $\hat\mu C=0$, then there is an open set $G\supseteq C$ such that
$\mu G<\epsilon_0$ (563Fd).   In this case, II can play in such a way that

\Centerline{$K_{2n+1}\subseteq G$,
\quad$\mu(G\setminus\bigcup_{i\le n}K_{2i+1})<\epsilon_{n+1}$,}

\Centerline{if
$E_n\subseteq G$ then $E_n\subseteq\bigcup_{i\le n}K_{2i+1}$}

\noindent for every $n$.   But now, taking I's responses under $\sigma$,
we have a play of $\Game(\{0,1\}\times\Cal E,D)$
in which $\bigcup_{n\in\Bbb N}K_{2n+1}=G$
includes $C$, so contains $f(\sequencen{k_n})$, and is won by II;  which is
supposed to be impossible.\ \Bang

So in this case $\mu_*A\ge\mu C>0$.

\medskip

\qquad{\bf case 2} Suppose that II has a winning strategy $\tau$.
For each $n\in\Bbb N$ and $u\in\{0,1\}^n$, let $L(u)$ be the second
component of
$\tau(\ofamily{i}{n}{(u(i),\emptyset)})$;  set
$G=\bigcup_{n\in\Bbb N}\bigcup_{u\in\{0,1\}^n}L(u)$, so that
$\mu G\le\sum_{n=0}^{\infty}2^n\epsilon_n<\mu Z$.   If we take any
$y\in B$, then we have a play $\sequencen{(k_n,K_n)}$
of $\Game(\{0,1\}\times\Cal E,D)$,
consistent with $\tau$, in which $k_{2n}=y(n)$ and $K_{2n}=\emptyset$ for
each $n$.   Since II wins this play, $f(y)$ must belong to

\Centerline{$\bigcup_{n\in\Bbb N}K_{2n+1}
=\bigcup_{n\in\Bbb N}L(y\restr n)\subseteq G$.}

\noindent As $y$ is arbitrary, $A\subseteq G$
and $\mu_*(Z\setminus A)\ge\mu(Z\setminus G)>0$.\ \Qed

\medskip

\quad{\bf (ii)} Write $\Cal K$ for the family of compact sets
$K\subseteq\{0,1\}^{\Bbb N}$ such that $A\cap K$ is Borel.
If $E\subseteq\{0,1\}^{\Bbb N}$ and $\mu E>0$, there is a
$K\in\Cal K$ such that $K\subseteq E$ and $\mu K>0$.   \Prf\ There is a
closed $Z\subseteq E$ such that $\mu Z>0$ (563Fd again).   By (i),
at least one of $\mu_*(Z\cap A)$,
$\mu_*(Z\setminus A)$ is non-zero, and there is a compact
set $K$ of non-zero measure which is included in one of $Z\cap A$,
$Z\setminus A$.   But now $K\in\Cal K$.\ \Qed

Now (because we have countable choice for subsets of $\Cal K$)
there is a
sequence $\sequencen{K_n}$ in $\Cal K$ such that
$\sup_{n\in\Bbb N}\mu K_n=\sup_{K\in\Cal K}\mu K$;  setting
$E=\{0,1\}^{\Bbb N}\setminus\bigcup_{n\in\Bbb N}K_n$, $E$ must be
negligible, while $A\setminus E$ is a Borel set;  so $A$ is measured by
$\hat\mu$.   As $\mu$ is arbitrary, $A$ is universally measurable.

\medskip

{\bf (b)(i)} If $V\in\Cal E\setminus\{\emptyset\}$ then
either $V\cap A$ is meager or there is a
$V'\in\Cal E\setminus\{\emptyset\}$ such that $V'\subseteq V$
and $V'\setminus A$ is meager.   \Prf\ Set
$\Cal U=\{E:E\in\Cal E\setminus\{\emptyset\}$, $E\subseteq V\}$
and let $\preccurlyeq$ be a
well-ordering of $\Cal U$ (in order type $\omega$, if you like).
Consider the game on $\{0,1\}\times\Cal U$
in which the players choose $(k_0,U_0),(k_1,U_1),\ldots$ such that,
for each $n\in\Bbb N$,

\Centerline{$k_n\in\{0,1\}$,
\quad$U_n\in\Cal U$,
\quad$U_{n+1}\subseteq U_n$.}

\noindent I wins if $y=\sequencen{k_{2n}}$ belongs to $B$ and
$f(y)\in\bigcap_{n\in\Bbb N}U_n$.   Because the $U_n$ are all
open-and-closed,
this game is determined for the same reasons as the game of (a).

\medskip

\qquad{\bf case 1} Suppose that I has a winning strategy $\sigma$;
say that $\sigma_2(w)\in\Cal U$ is the second component of $\sigma(w)$ for each
$w\in\bigcup_{n\in\Bbb N}(\{0,1\}\times\Cal U)^n$.   For each $n\in\Bbb N$
let $\Cal U_n$ be the set of those $U\in\Cal U$ such that
$v\restr n=v'\restr n$ for all $v$, $v'\in U$.   Let
$(k',V')=\sigma(\emptyset)$ be
I's first move when following $\sigma$.   Let $Q$ be the set of
positions in the game consistent with $\sigma$ and with II to move,
that is, finite sequences

\Centerline{$q=\langle(k_i,U_i)\rangle_{i\le 2n}
\in(\{0,1\}\times\Cal U)^{2n+1}$}

\noindent such that
$(k_{2m},U_{2m})=\sigma(\langle(k_{2i+1},U_{2i+1})\rangle_{i<m})$
for every $m\le n$ and $\langle U_i\rangle_{i\le 2n}$ is non-increasing.
For such a $q$, set $V_q=U_{2n}$ and

\Centerline{$W_q
=\bigcup\{\sigma_2({\langle(k_{2i+1},U_{2i+1})\rangle_{i<n}}
   ^{\smallfrown}\fraction{(k,U)}):
k\in\{0,1\}$, $U\in\Cal U_n$, $U\subseteq V_q\}$.}

\noindent Then $W_q$ is an open subset of $V_q$;  but also it is dense in
$V_q$, because if $W\subseteq V_q$ is open and not empty there is a
$U\in\Cal U_n$ included in $W$ and
$\sigma_2({\langle(k_{2i+1},U_{2i+1})\rangle_{i<n}}^{\smallfrown}
  \fraction{(k,U)})$
is a non-empty subset of $U$.    $Q$ is countable, so
$E=\bigcap_{q\in Q}W_q\cup(\{0,1\}^{\Bbb N}\setminus\overline{V}_q)$ is
comeager in $\{0,1\}^{\Bbb N}$.

\Quer\ If $V'\setminus A$ is not meager, there is an
$x\in E\cap V'\setminus A$.
Define $\sequencen{(k_n,U_n)}$ inductively,
as follows.   $(k_0,U_0)=(k',V')$.   Given that
$q=\langle(k_i,U_i)\rangle_{i\le 2n}$ belongs to $Q$ and $x\in V_q$,
then $x\in W_q$ so there are $k\in\{0,1\}$, $U\in\Cal U_n$ such that $x
\in\sigma_2({\langle(k_{2i+1},U_{2i+1})\rangle_{i<n}}^{\smallfrown}
  \fraction{(k,U)})$;
take the
lexicographically first such pair $(k,U)$ for $(k_{2n+1},U_{2n+1})$,
and set
$(k_{2n+2},U_{2n+2})=\sigma(\langle(k_{2i+1},U_{2i+1})\rangle_{i\le n})$.
Then $q'=\langle(k_i,U_i)\rangle_{i\le 2n+2}$ belongs to $Q$
and $V_{q'}=U_{2n+2}=\sigma_2(\langle(k_{2i+1},U_{2i+1})\rangle_{i\le n})$
contains $x$, so the induction can continue.

At the end of this induction, $\sequencen{(k_n,U_n)}$ will be a play of the
game consistent with $\sigma$ in which the only point of
$\bigcap_{n\in\Bbb N}U_n$ is $x$ and does not belong to $A$.
So either $y=\sequencen{k_{2n}}$ does not belong to $B$ or
$f(y)\notin\bigcap_{n\in\Bbb N}U_n$;  in either case, II wins the play;
which is supposed to be impossible.\ \Bang

So in this case $V'\setminus A$ is meager.

\medskip

\qquad{\bf case 2} Suppose that II has a winning strategy $\tau$;
say that $\tau_2(w)$ is the second component of $\tau(w)$ for each
$w\in\bigcup_{n\ge 1}(\{0,1\}\times\Cal U)^n$.
Let $Q$ be the set of objects

\Centerline{$q=(\ofamily{i}{2n}{(k_i,U_i)},k)$}

\noindent such that $\ofamily{i}{2n}{(k_i,U_i)}$ is a finite sequence in
$\{0,1\}\times\Cal U$ consistent with $\tau$ (allowing the empty string
when $n=0$) and $k\in\{0,1\}$.   For such a $q$, set $V_q=U_{2n-1}$ (if
$n>0$) or $V_q=V$ (if $n=0$);  set

\Centerline{$W_q
=\bigcup\{\tau_2({\ofamily{i}{n}{(k_{2i},U_{2i})}}^{\smallfrown}
  \fraction{(k,U)}):
U\in\Cal U$, $U\subseteq V_q\}$,}

\noindent so that $W_q$ is a dense subset of $V_q$.   $Q$ is countable, so
$E=\bigcap_{q\in Q}W_q\cup(\{0,1\}^{\Bbb N}\setminus\overline{V}_q)$ is
comeager.

\Quer\ If there is an $x$ in $A\cap V\cap E$, let
$y\in B$ be such that $f(y)=x$, and define
$\sequencen{(k_n,U_n)}$ as follows.   Given that
$q=(\ofamily{i}{2n}{(k_i,U_i)},y(n))$ belongs to $Q$ and $x\in V_q$,
then $x\in W_q$ so there is a $U\in\Cal U$ such that
$x\in\tau_2({\ofamily{i}{n}{(k_{2i},U_{2i})}}^{\smallfrown}
  \penalty-100\fraction{(y(n),U)})$;
take the
$\preccurlyeq$-first such $U$ for $U_{2n}$, set $k_{2n}=y(n)$ and
$(k_{2n+1},U_{2n+1})=\tau(\langle(k_{2i},U_{2i})\rangle_{i\le n})$,
so that $q'=(\langle(k_i,U_i)\rangle_{i\le 2n+1},\penalty-100y(n+1))$
belongs to $Q$ and
$V_{q'}=U_{2n+1}=\tau_2(\langle(k_{2i},U_{2i})\rangle_{i\le n})$
contains $x$.

At the end of this induction, $\sequencen{(k_n,U_n)}$ will be a play of the
game consistent with $\tau$ in which
$f(\sequencen{k_{2n}})=x\in\bigcap_{n\in\Bbb N}U_n$, so that I wins, which
is supposed to be impossible.\ \Bang

Thus in this case $A\cap V$ must be meager.\ \Qed

\medskip

\quad{\bf (ii)} Now let $G$ be the union of those $V\in\Cal E$
such that $V\setminus A$ is meager;  then
$G\setminus A$ is meager.   (This is where we need AC($\Bbb R;\omega$).)
If $V\in\Cal E$ and $V\subseteq\{0,1\}^{\Bbb N}\setminus G$,
then $V'\setminus A$ is non-meager for every whenever
$V\in\Cal E\setminus\{\emptyset\}$ and $V'\subseteq V$,
so $V\cap A$ is meager;  accordingly $G'\cap A$ is meager, where
$G'=\{0,1\}^{\Bbb N}\setminus\overline{G}$.   But this means that
$G\symmdiff A
\subseteq(G\setminus A)\cup(G'\cap A)\cup(\overline{G}\setminus G)$ is
meager and $A$ has the Baire property.
}%end of proof of 567F

\leader{567G}{Theorem} [AD] In any Hausdorff
second-countable space, every subset is
universally measurable and has the Baire property.

\proof{
Let $X$ be a Hausdorff second-countable space, $\sequencen{U_n}$ a sequence
running over a base for the topology of $X$, and $A\subseteq X$.

\medskip

{\bf (a)} Define
$g:X\to\{0,1\}^{\Bbb N}$ by setting $g(x)=\sequencen{\chi U_n(x)}$ for
$x\in X$;  then $g$ is injective and Borel measurable.
If $\mu$ is a Borel probability measure on $X$, we have a Borel
probability measure $\nu=\mu g^{-1}\restr\Cal B(\{0,1\}^{\Bbb N})$
on $\{0,1\}^{\Bbb N}$.   By 567Fa, $g[A]$ is measured by the completion
$\hat\nu$ of $\nu$;   let $F$, $H\subseteq\{0,1\}^{\Bbb N}$ be Borel sets
such that $\nu H=0$ and $g[A]\symmdiff F\subseteq H$;  then
$A\symmdiff g^{-1}[F]\subseteq g^{-1}[H]$ is $\mu$-negligible, so $A$ is
measured by $\hat\mu$.   As $\mu$ is arbitrary, $A$ is universally
measurable.

\medskip

{\bf (b)} Set
$G=\bigcup\{U_n:n\in\Bbb N$, $U_n\cap A$ has the Baire property$\}$,
so that $G\cap A$ has the Baire property.   (Remember that as we have a
bijection between $X$ and a subset of $\Bbb R$, we have
countable choice for subsets of $X$, so that the ideal of meager subsets
of $X$
is a $\sigma$-ideal and the Baire-property algebra is a $\sigma$-algebra.)
Set
$V=X\setminus(\bigcup_{n\in\Bbb N}\partial U_n\cup\overline{G})$;  then
$G\cup V$ is comeager in $X$, and $A\setminus V$ has the Baire
property.   If $V$ is empty, we can stop.   Otherwise, let $\Cal V$
be the countable algebra of subsets of $V$ generated by
$\{V\cap U_n:n\in\Bbb N\}$.   Since $A\cap U$ does not have the Baire
property (in $X$) for any non-empty relatively open subset $U$ of $V$,
$V$ has no isolated points and $\Cal V$ is atomless.   So $\Cal V$ is
isomorphic to the algebra of open-and-closed subsets of $\{0,1\}^{\Bbb N}$
(316M) and there is a Boolean-independent sequence
$\sequencen{V_n}$ in $\Cal V$ generating $\Cal V$.   Define
$h:V\to\{0,1\}^{\Bbb N}$ by setting $h(x)=\sequencen{\chi V_n(x)}$ for
$x\in V$.   Then $h[V]$ is dense in $\{0,1\}^{\Bbb N}$ and $h^{-1}[H]$ is
dense in $V$ for every dense open set $H\subseteq\{0,1\}^{\Bbb N}$;
consequently $h^{-1}[M]$ is meager in $V$ and in $X$ whenever
$M\subseteq\{0,1\}^{\Bbb N}$ is either nowhere dense or meager.
By 567Fb, $h[A]$ has the Baire property in $\{0,1\}^{\Bbb N}$;  express it
as $H\symmdiff M$ where $H$ is open and $M$ is meager;  then
$A\cap V=h^{-1}[h[A]]=h^{-1}[H]\symmdiff h^{-1}[M]$ has the
Baire property in $X$, so $A$ has the Baire property in $X$, as required.
}%end of proof of 567G

\leader{567H}{Theorem} (a) [AD] Let $X$ be a Polish group and $Y$ a
topological group which is either separable or Lindel\"of.
Then every group homomorphism from $X$ to $Y$ is continuous.

(b) [AD+AC($\omega$)] Let $X$ be an abelian
topological group which is complete under a metric defining its topology,
and $Y$ a topological group which is either separable or Lindel\"of.
Then every group homomorphism from $X$ to $Y$ is continuous.

(c) [AD+AC($\omega$)]
Let $X$ be a complete metrizable linear topological space,
$Y$ a linear topological space and $T:X\to Y$ a linear operator.
Then $T$ is continuous.   In particular, every linear operator between
Banach spaces is a bounded operator.

\proof{{\bf (a)(i)} Let $f:X\to Y$ be a homomorphism, and
$V$ a neighbourhood of the identity in $Y$.
Let $W$ be an open neighbourhood of the identity in $Y$
such that $W^{-1}W\subseteq V$.   Then there is countable family $\Cal H$
of left translates of $W$ which covers $Y$.   \Prf\ If $Y$ is
separable, let $D$ be a countable dense subset of $Y$, and set
$\Cal H=\{yW:y\in D\}$.   If $Y$ is Lindel\"of, we have only to note that
$\{yW:y\in Y\}$ is an open cover of $Y$, so has a countable subcover.\ \Qed

\medskip

\quad{\bf (ii)} Since $X$ is a Baire space (561Ea),
and the ideal of meager subsets of $X$ is a $\sigma$-ideal (see
part (b) of the proof of 567G), and $\{f^{-1}[H]:H\in\Cal H\}$ is a
countable cover of $X$, there is an $H\in\Cal H$ such that
$E=f^{-1}[H]$ is non-meager.
Now $E^{-1}E$ is a neighbourhood of the
identity in $X$.   \Prf\ By 567G, $E$ has the Baire property;  let
$G$ be a non-empty open set in $X$ such that $G\setminus E$ is meager.
Set $U=\{x:Gx\cap G\ne\emptyset\}$;  then $U$ is a neighbourhood of the
identity in $X$.   If $x\in U$, then

\Centerline{$Gx\cap G
\subseteq(Ex\cap E)\cup(Gx\setminus Ex)\cup(G\setminus E)
=(Ex\cap E)\cup(G\setminus E)x\cup(G\setminus E)$.}

\noindent Since $Gx\cap G$ is non-meager, while $G\setminus E$ and
$(G\setminus E)x$ are meager, $Ex\cap E\ne\emptyset$ and $x\in E^{-1}E$.
Thus $E^{-1}E\supseteq U$ is a neighbourhood of the identity.\ \Qed

\medskip

\quad{\bf (iii)} Let $y\in Y$ be such that $H=yW$.
If $x$, $z\in E$, $y^{-1}f(x)$ and $y^{-1}f(z)$ both belong to
$W$, so

\Centerline{$f(x^{-1}z)=f(x)^{-1}f(z)\in W^{-1}yy^{-1}W=W^{-1}W
\subseteq V$.}

\noindent Thus $f^{-1}[V]\supseteq E^{-1}E$ is a neighbourhood of the
identity in $X$.   As $V$ is arbitrary, $f$ is continuous at the identity,
therefore continuous.

\medskip

{\bf (b)} \Quer\ Otherwise, there is a neighbourhood $V$ of the identity
$e_Y$ of $Y$ such that $f^{-1}[V]$ is not a neighbourhood of the identity
$e_X$ of $X$.   Let $\rho$ be a metric on $X$, defining its topology, under
which $X$ is complete.   Then for each $n\in\Bbb N$ we can choose an
$x_n\in X$ such that $\rho(x_n,e_X)\le 2^{-n}$ and $f(x_n)\notin V$.
(This is where we need AC($\omega$).)   For finite $J\subseteq\Bbb N$
set $u_J=\prod_{n\in J}x_n$, starting from $u_{\emptyset}=e_X$.
We can define an infinite $I\subseteq\Bbb N$ inductively by saying that

\Centerline{$I=\{n:$ whenever $J\subseteq I\cap n$
then $\rho(u_J,u_Jx_n)\le 2^{-\#(I\cap n)}\}$.}

\noindent This will ensure
that $v_K=\lim_{n\to\infty}u_{K\cap n}$ is defined for every
$K\subseteq I$.   Note that $v_{K\cup\{m\}}=v_Kx_m$ whenever $m\in I$ and
$K\subseteq I\setminus\{m\}$
(this is where we need to know that $X$ is abelian).

Give $\Cal PI$ its usual topology.
Let $W$ be a neighbourhood of $e_Y$ such that $W^{-1}W\subseteq V$.
By the argument of (a) above, applied to the map
$K\mapsto f(v_K):\Cal PI\to Y$,
there is a $y\in Y$ such that $E=\{K:K\subseteq I$, $f(v_K)\in yW\}$
is non-meager in $\Cal PI$.   Looking at the topological group
$(\Cal PI,\symmdiff)$, we see that there is a neighbourhood $U$ of
$\emptyset$ in $\Cal PI$ included in $\{K\symmdiff L:K$, $L\in E\}$.
Taking any sufficiently large $n\in I$, we have $\{n\}\in U$, so there must
be a $K\in E$ such that $n\notin K$ and $K\cup\{n\}\in E$.
In this case $f(v_K)\in yW$, $f(v_{K\cup\{n\}})\in yW$ and

\Centerline{$f(x_n)=f(v_K^{-1}v_{K\cup\{n\}})
=f(v_K)^{-1}f(v_{K\cup\{n\}})\in W^{-1}W\subseteq V$,}

\noindent which is impossible.\ \Bang

\medskip

{\bf (c)} \Quer\ Otherwise, there is a neighbourhood $V$ of $0$ in $Y$ such
that $T^{-1}[V]$ is not a neighbourhood of $0$ in $X$;  we can suppose that
$\alpha y\in V$ whenever $y\in V$ and $|\alpha|\le 1$.
Let $\rho$ be a metric on $X$, defining its topology, under
which $X$ is complete.   Let $W$ be a neighbourhood of $0$ in $Y$ such that
$W-W\in V$.   Then for each $n\in\Bbb N$ we can choose an
$x_n\in X\setminus nT^{-1}[V]$ such that
$\rho(x_n,0)\le 2^{-n}$.   Define $I\in[\Bbb N]^{\omega}$ and
$\langle v_K\rangle_{K\subseteq I}$ as in (b), but using additive notation
rather than multiplicative.   This time we are not
supposing that $Y$ is separable.   However, there must be an $m\in\Bbb N$
such that $E=\{K:T(v_K)\in mW\}$ is non-meager.   As before, we can find
$n\in I\setminus m$ and $K\in E$ such that
$n\notin L$ and $K\cup\{n\}\in E$.   So the calculation gives

\Centerline{$Tx_n=Tv_{K\cup\{n\}}-Tv_K\in mW-mW
\subseteq mV\subseteq nV$,}

\noindent again contrary to the choice of $x_n$.\ \Bang
}%end of proof of 567H

\leader{567I}{Proposition} [AC($\Bbb R;\omega$)] Let $\widehat{\Cal B}$ be
the Baire-property algebra of $\Cal P\Bbb N$.   Then every
$\widehat{\Cal B}$-measurable real-valued additive functional on
$\Cal P\Bbb N$ is of the form $a\mapsto\sum_{n\in a}\gamma_n$ for some
$\sequencen{\gamma_n}\in\ell^1$.

\proof{ As noted in 567Ec, $\widehat{\Cal B}$ is a $\sigma$-algebra of
subsets of $\Cal P\Bbb N$.

\medskip

{\bf (a)(i)} If $G\subseteq\Cal P\Bbb N$ is a dense open set and
$m\in\Bbb N$, there are an $m'>m$ and an $L\subseteq m'\setminus m$ such
that $\{a:a\subseteq\Bbb N$, $a\cap m'\setminus m=L\}\subseteq G$.   \Prf\
The set $H=\{b:b\subseteq\Bbb N\setminus m$, $I\cup b\in G$ for every
$I\subseteq m\}$ is a dense open subset of $\Cal P(\Bbb N\setminus m)$, so
there are an $m'>m$ and an $L\subseteq m'\setminus m$ such that
$H\supseteq\{b:b\subseteq\Bbb N\setminus m$, $b\cap m'\setminus m=L\}$;
this pair $m'$, $L$ works.\ \Qed

\medskip

\quad{\bf (ii)} If $G\subseteq\Cal P\Bbb N$ is comeager, there are a
strictly increasing sequence $\sequencen{m_n}$ in $\Bbb N$ and sets
$L_n\subseteq m_{n+1}\setminus m_n$, for $n\in\Bbb N$, such that

\Centerline{$G
\supseteq\{a:a\subseteq\Bbb N$, $a\cap m_{n+1}\setminus m_n=L_n$ for
infinitely many $n\}$.}

\noindent\Prf\ Let $\sequencen{G_n}$ be a non-increasing sequence of dense
open sets such that $G\supseteq\bigcap_{n\in\Bbb N}G_n$, and choose
$\sequencen{m_n}$, $\sequencen{L_n}$ inductively such that $m_n<m_{n+1}$,
$L_n\subseteq m_{n+1}\setminus m_n$ and
$\{a:a\subseteq\Bbb N$, $a\cap m_{n+1}\setminus m_n=L_n\}\subseteq G_n$ for
every $n$.\ \Qed

\medskip

\quad{\bf (iii)} If $G\subseteq\Cal P\Bbb N$ is comeager, and
$a\subseteq\Bbb N$, then there are $b_0$, $b'_0$, $b_1$, $b'_1\in G$ such
that

\Centerline{$b_0\subseteq b'_0$,
\quad$b_1\subseteq b'_1$,
\quad$(b'_0\setminus b_0)\cap(b'_1\setminus b_1)=\emptyset$,
\quad$(b'_0\setminus b_0)\cup(b'_1\setminus b_1)=a$.}

\noindent\Prf\ Let $\sequencen{m_n}$ and $\sequencen{L_n}$ be as in (ii).
Set

\Centerline{$b_0=\bigcup_{n\in\Bbb N}L_{2n}$,
\quad
$b'_0=b_0\cup(a\cap m_0)
\cup\bigcup_{n\in\Bbb N}a\cap m_{2n+2}\setminus m_{2n+1}$,}

\Centerline{$b_1=\bigcup_{n\in\Bbb N}L_{2n+1}$,
\quad$b'_1=b_1\cup\bigcup_{n\in\Bbb N}a\cap m_{2n+1}\setminus m_{2n}$.\
\Qed}

\noindent So if $\nu:\Cal P\Bbb N\to\Bbb R$ is additive,
$\sup_{a\subseteq\Bbb N}|\nu a|\le 4\sup_{b\in G}|\nu b|$.

\medskip

\quad{\bf (iv)} If $G\subseteq\Cal P\Bbb N$ is comeager, there is a
disjoint sequence $\sequencen{a_n}$ in $G$.   \Prf\ Take
$\sequencen{m_n}$ and $\sequencen{L_n}$ as in (ii), and set
$a_n=\bigcup_{i\in\Bbb N}L_{2^n(2i+1)}$ for each $n$.\ \Qed

\medskip

{\bf (b)} If $\nu:\Cal P\Bbb N\to\Bbb R$ is additive and
$\widehat{\Cal B}$-measurable,
it is bounded.   \Prf\ Let $M\in\Bbb N$ be such that
$E=\{a:|\nu a|\le M\}$ is non-meager.   Then there are an $m\in\Bbb N$ and
$J\subseteq m$ such that $V_{mJ}\setminus E$ is meager, where
$V_{mJ}=\{a:a\cap m=J\}$.   For $K\subseteq m$, $a\subseteq\Bbb N$
set $\phi_K(a)=a\symmdiff K$;  then $\phi_K$ is an autohomeomorphism of
$\Cal P\Bbb N$, so $\phi_K[V_{mJ}\setminus E]$ is meager.   Let $G$ be the
comeager set
$\Cal P\Bbb N\setminus\bigcup_{K\subseteq m}\phi_K[V_{mJ}\setminus E]$.
Set $\delta=\sum_{i<m}|\nu\{i\}|$;  then
$|\nu\phi_K(a)-\nu a|\le\delta$ whenever $K\subseteq m$
and $a\subseteq\Bbb N$.   If $b\in G$, set $K=(b\cap m)\symmdiff J$;  then
$\phi_K(b)\in V_{mJ}\setminus(V_{mJ}\setminus E)\subseteq E$, so
$|\nu b|\le M+\delta$.   So (a-iii) tells us that $|\nu a|\le 4(M+\delta)$
for every $a\subseteq\Bbb N$, and $\nu$ is bounded.\ \Qed

\medskip

{\bf (c)} If $\nu:\Cal P\Bbb N\to\Bbb R$ is
additive and $\widehat{\Cal B}$-measurable
and $\nu\{n\}=0$ for every $n\in\Bbb N$, then $E=\{a:\nu a\ge\epsilon\}$ is
meager for every $\epsilon>0$.
\Prf\Quer\ Otherwise, let $m\in\Bbb N$ and $J\subseteq m$ be such that
$V_{mJ}\setminus E$ is meager.   Let $G$ be the comeager set
$\Cal P\Bbb N\setminus\bigcup_{K\subseteq m}\phi_K[V_{mJ}\setminus E]$, as
in (b).   This time, $\nu a=\nu\phi_K(a)$ whenever $K\subseteq m$ and
$a\subseteq\Bbb N$, so $\nu a\ge\epsilon$ for every $a\in G$.   But
(a-iv) tells us that there is a disjoint sequence
$\sequencen{a_n}$ in $G$, and now
$\sup_{n\in\Bbb N}\nu(\bigcup_{i\le n}a_i)=\infty$, contradicting (b).\
\Bang\Qed

\medskip

{\bf (d)} If $\nu:\Cal P\Bbb N\to\Bbb R$ is additive and
$\widehat{\Cal B}$-measurable
and $\nu\{n\}=0$ for every $n\in\Bbb N$, then $\nu=0$.   \Prf\ By (c),
applied to $\nu$ and $-\nu$, $G=\{a:\nu a=0\}$ is comeager.   By (a-iii),
$\nu$ must be identically zero.\ \Qed

\medskip

{\bf (e)} Now suppose that $\nu$ is any additive
$\widehat{\Cal B}$-measurable functional.   Set $\gamma_n=\nu\{n\}$ for
each $n$.   By (b), $\sequencen{\gamma_n}\in\ell^1$.   Setting
$\nu'a=\nu a-\sum_{n\in a}\gamma_n$ for $a\subseteq\Bbb N$, $\nu'$ is still
additive and $\widehat{\Cal B}$-measurable, and $\nu'\{n\}=0$ for every
$n$, so (d) tells us that $\nu'=0$ and
$\nu a=\sum_{n\in a}\gamma_n$ for every $a$, as required.
}%end of proof of 567I

\leader{567J}{Proposition} [AD] A finitely additive functional on a
Dedekind $\sigma$-complete Boolean algebra is countably additive.

\proof{ Let $\frak A$ be a Dedekind $\sigma$-complete Boolean algebra,
$\nu$ a finitely additive functional on $\frak A$ and $\sequencen{a_n}$ a
disjoint sequence in $\frak A$ with supremum $a$.   Set
$\lambda c=\nu(\sup_{n\in c}a_n)$ for $c\subseteq\Bbb N$.
Then $\lambda$ is an
additive functional on $\Cal P\Bbb N$.   By 567G, it is
$\widehat{\Cal B}(\Cal P\Bbb N)$-measurable;  by 567I,

\Centerline{$\nu a=\lambda\Bbb N=\sum_{n=0}^{\infty}\lambda\{n\}
=\sum_{n=0}^{\infty}\nu a_n$.}
}%end of proof of 567J

\leader{567K}{Theorem} [AD+AC($\omega$)] If $U$ is an $L$-space with a weak
order unit, it is reflexive.

\proof{ By 561Hb, $U$ is isomorphic to $L^1(\frak A,\bar\mu)$ for
some totally finite measure algebra $(\frak A,\bar\mu)$;  now $U^*$ can be
identified with $L^{\infty}(\frak A)$.   Next, $L^{\infty}(\frak A)^*$ can
be identified with the space of bounded finitely additive functionals on
$\frak A$, as in 363K;  by 567J, these are all countably additive.
Because we have countable choice, $\frak A$ is ccc (566M),
so countably additive functionals are completely additive and correspond
to members of $L^1$, as in 365Ea.   Thus the canonical embedding of $U$ in
$U^{**}$ is surjective.
}%end of proof of 567K

\leader{567L}{Theorem}\cmmnt{ (R.M.Solovay)} [AD] $\omega_1$ is \2vm.

\medskip

\noindent{\bf Remark}\cmmnt{ The definition in 541M speaks of
`regular uncountable
cardinals'.}   In the present context I will use the formulation
`an initial ordinal $\kappa$ is
\2vm\ if there is a proper $\kappa$-additive
$2$-saturated ideal $\Cal I$
of $\Cal P\kappa$ containing singletons', where here
`$\kappa$-additive' means that $\bigcup_{\eta<\xi}J_{\eta}\in\Cal I$
whenever $\xi<\kappa$ and $\ofamily{\eta}{\xi}{J_{\eta}}$ is a family in
$\Cal I$.

\proof{{\bf (a)} Let $\StrI$ be the set of strategies for player I in games
of the form $\Gamma(\Bbb N,.)$, that is, $\StrI$ is the set of functions
from $\bigcup_{n\in\Bbb N}\BbbN^n$ to $\Bbb N$;  for $\sigma\in\StrI$ and
$x\in\NN$, let $\sigma*x\in\NN$ be the play in which I follows the strategy
$\sigma$ and II plays the sequence  $x$, that is,

\Centerline{$(\sigma*x)(2n)=\sigma(x\restr n)$,
\quad$(\sigma*x)(2n+1)=x(n)$}

\noindent  for $n\in\Bbb N$.
Similarly, let $\StrII$ be the set of functions from
$\bigcup_{n\ge 1}\BbbN^n$ to $\Bbb N$
and for $\tau\in\StrII$, $x\in\NN$, $n\in\Bbb N$ set

\Centerline{$(\tau*x)(2n)=x(n)$,
\quad$(\tau*x)(2n+1)=\tau(x\restr(n+1))$.}

We can find bijections $g:\NN\to\StrI\cup\StrII$ and
$h:\NN\to\WO(\Bbb N)$, where $\WO(\Bbb N)\subseteq\Cal P(\BbbN^2)$ is the
set of well-orderings of $\Bbb N$.   \Prf\ Since
$S=\bigcup_{n\in\Bbb N}\BbbN^n$ and $S^*=\bigcup_{n\ge 1}\BbbN^n$ are
countably infinite, $\StrI=\BbbN^{S}$ and $\StrII=\BbbN^{S^*}$ are
equipollent with $\NN$.    As $\Cal P(\BbbN^2)\sim\Cal P\Bbb N\sim\NN$,
there is an injection from $\WO(\Bbb N)$ to $\NN$.   In the reverse
direction, there are an injection from $\NN$ to the set $F$ of permutations
of $\Bbb N$, and an injection from $F$ to $\WO(\Bbb N)$;  so
the Schroeder-Bernstein theorem tells us that $\WO(\Bbb N)\sim\NN$.\ \Qed

Define $f:\WO(\Bbb N)\to\omega_1$ by
saying that $f(\preccurlyeq)=\otp(\Bbb N,\preccurlyeq)$ for
$\preccurlyeq\mskip5mu\in\WO(\Bbb N)$.

\medskip

{\bf (b)} For $x\in\NN$ let $L_x\subseteq\NN$ be the smallest set such that

\inset{$x\in L_x$,

whenever $y$, $z\in L_x$ then $g(y)*z\in L_x$,

whenever $y\in L_x$ then $\sequencen{y(2n)}$ and
$\sequencen{y(2^k(2n+1))}$ belong to $L_x$ for every $k\in\Bbb N$.}

\noindent Observe that $L_x$ is countable and that $L_y\subseteq L_x$
whenever $y\in L_x$.   For $x\in\NN$, set $C_x=\{y:y\in\NN$, $x\in L_y\}$.

\medskip

{\bf (c)} For any sequence $\sequencen{x_n}$ in $\NN$ there is an
$x\in\NN$ such that $C_x\subseteq\bigcap_{n\in\Bbb N}C_{x_n}$.   \Prf\ Set
$x(0)=0$ and $x(2^k(2n+1))=x_k(n)$ for $k$, $n\in\Bbb N$.
Then $x_k\in L_x$
for every $k$.   So if $y\in C_x$ and $n\in\Bbb N$, we have
$x_n\in L_x\subseteq L_y$ and $y\in C_{x_n}$.\ \Qed

Let $\Cal F$ be the filter on $\NN$ generated by $\{C_x:x\in\NN\}$;  then
(because AC($\Bbb R;\omega$) is true)
$\Cal F$ is closed under countable intersections.

\medskip

{\bf (d)} Suppose that $A\subseteq\NN$ is such that whenever $x\in A$ and
$L_y=L_x$ then $y\in A$.

\medskip

\quad{\bf (i)} If I has a winning strategy in $\Game(\Bbb N,A)$ then
$A\in\Cal F$.   \Prf\ Let $\sigma\in\StrI$ be a winning strategy for I, and
consider $x=g^{-1}(\sigma)\in\NN$.   Suppose that $y\in\NN$ and
$x\in L_y$, and consider $z=\sigma*y\in A$.   As $z=g(x)*y$ belongs to
$L_y$, $L_z\subseteq L_y$;  on the other hand,
$y(n)=z(2n+1)$ for every $n$, so $y\in L_z$ and $L_y\subseteq L_z$.
So $L_y=L_z$ and $y\in A$.
As $y$ is arbitrary, $C_x\subseteq A$ and $A\in\Cal F$.\ \Qed

\medskip

\quad{\bf (ii)} If II has a winning strategy in $\Game(\Bbb N,A)$ then
$\NN\setminus A\in\Cal F$.   \Prf\ Let $\tau\in\StrII$ be a winning
strategy for II, and
consider $x=g^{-1}(\tau)\in\NN$.   Suppose that $y\in\NN$ and
$x\in L_y$, and consider $z=\tau*y\in\NN\setminus A$.   As before,
$L_z\subseteq L_y$;  this time,
$y(n)=z(2n)$ for every $n$ so $y\in L_z$ and $L_y\subseteq L_z$.
So $y\notin A$.   As $y$ is arbitrary, $C_x\subseteq\NN\setminus A$ and
$\NN\setminus A\in\Cal F$.\ \Qed

\medskip

{\bf (e)} For $x\in\NN$ set $\phi(x)=\sup_{y\in L_x}f(h(y))$;  because
$L_x$ is countable, $\phi(x)<\omega_1$ (567Ed).   Let $\Cal G$ be the image
filter $\phi[[\Cal F]]$.   Because $\Cal F$ is closed under countable
intersections, so is $\Cal G$.   If $B\subseteq\omega_1$ then $\phi^{-1}[B]$
satisfies the condition of (d), so that one of $\phi^{-1}[B]$,
$\NN\setminus\phi^{-1}[B]$ belongs to $\Cal F$ and one of $B$,
$\omega_1\setminus B$ belongs to $\Cal G$;  as $B$ is arbitrary, $\Cal G$
is an ultrafilter.

\medskip

{\bf (f)} Finally, $\Cal G$ does not contain any singletons.   \Prf\ If
$\xi<\omega_1$, there is an $x\in\NN$ such that $f(h(x))=\xi+1$.   Now
$C_x\in\Cal F$ so $\phi[C_x]\in\Cal G$.   If $y\in C_x$ then $x\in L_y$ so
$\xi+1\le\phi(y)$;  accordingly $\xi\notin\phi[C_x]$ and
$\{\xi\}\notin\Cal G$.\ \QeD\   So $\Cal G$ (or, if you like, the ideal
$\{\omega_1\setminus B:B\in\Cal G\}$) witnesses that $\omega_1$ is \2vm.
}%end of proof of 567L

\leader{567M}{Theorem}\cmmnt{ ({\smc Moschovakis 70})}
[AD] Let $\alpha$ be an
ordinal such that there is a surjection from $\Cal P\Bbb N$ onto $\alpha$.
Then there is a surjection from $\Cal P\Bbb N$ onto $\Cal P\alpha$.

\proof{ The formulae will run slightly more smoothly if we work with
surjections
from $\NN$ rather than from $\Cal P\Bbb N$;  of course this makes no
difference to the result.

\medskip

{\bf (a)}  We may suppose that $\alpha$ is uncountable.
Let $f:\NN\to\alpha$ be a surjection.   I seek to define
inductively a family $\langle g_{\xi}\rangle_{\xi\le\alpha}$ such that
$g_{\xi}$ is a surjection from $\NN$ onto $\Cal P\xi$ for every
$\xi\le\alpha$.
As in the proof of 567L, let
$\StrI$ be the set of functions from $\bigcup_{n\in\Bbb N}\BbbN^n$ to
$\Bbb N$, and
$\StrII$ the set of functions from $\bigcup_{n\ge 1}\BbbN^n$ to
$\Bbb N$;  fix a surjection $h:\NN\to\StrI\cup\StrII$.
For $\sigma\in\StrI$, $\tau\in\StrII$ and $x\in\NN$ let $\sigma*x$,
$\tau*x$ be the plays in games on $\Bbb N$ as described in the proof of
567L.

\medskip

{\bf (b)} Start by setting $g_n(x)=n\cap x[\Bbb N]$ for $x\in\NN$ and
$n\in\Bbb N$.

\medskip

{\bf (c)} For the inductive step to a non-zero limit
ordinal $\xi\le\alpha$, given $\ofamily{\eta}{\xi}{g_{\eta}}$, then for
$x\in\NN$ set

\Centerline{$\eta_x=f(\sequencen{x(4n)})$,
\quad$\zeta_x=f(\sequencen{x(4n+1)})$,}

$$\eqalign{E_x
&=g_{\eta_x}(\sequencen{x(4n+2)})\text{ if }\eta_x<\xi,\cr
&=\emptyset\text{ otherwise},\cr
F_x
&=g_{\zeta_x}(\sequencen{x(4n+3)})\text{ if }\zeta_x<\xi,\cr
&=\emptyset\text{ otherwise}.\cr}$$

\noindent Next, for $D\subseteq\xi$, set

$$\eqalign{A_D
&=\{x:x\in\NN,\,\eta_x<\xi,\,E_x=D\cap\eta_x\cr
&\mskip70mu
\text{ and either }\zeta_x\le\eta_x\text{ or }\zeta_x\ge\xi
\text{ or }F_x\ne D\cap\zeta_x\}.\cr}$$

\noindent (The idea is that the players are competing to see who can
capture the largest initial segment of $D$ with the pair $(\eta_x,E_x)$
determined by I's moves or the pair $(\zeta_x,F_x)$ determined by
II's moves;  for definiteness, if neither correctly defines an
initial segment, then II wins, while if they seize the same
segment $(\eta_x,E_x)=(\zeta_x,F_x)$, then I wins.)   Finally, define
$g:\StrI\cup\StrII\to\Cal P\xi$ by setting

\Centerline{$g(\sigma)=\bigcup\{D:\sigma$ is a winning strategy for I in
$\Game(\Bbb N,A_D)\}$ if $\sigma\in\StrI$,}

\Centerline{$g(\tau)=\bigcup\{D:\tau$ is a winning strategy for II in
$\Game(\Bbb N,A_D)\}$ if $\tau\in\StrII$.}

We find that $g$ is a surjection onto $\Cal P\xi$.   \Prf\ Take any
$D\subseteq\xi$.

\medskip

\quad{\bf case 1}
Suppose that I has a winning strategy $\sigma$ in $\Game(\Bbb N,A_D)$.
Then $D\subseteq g(\sigma)$.   \Quer\ If $D\ne g(\sigma)$, there is a $D'$,
distinct from $D$, such that $\sigma$ is a winning strategy for I in
$\Game(\Bbb N,D')$.   Let $\zeta<\xi$ be such that
$D\cap\zeta\ne D'\cap\zeta$.   Then there is
a $z\in\NN$ such that $f(\sequencen{z(2n)})=\zeta$ and
$g_{\zeta}(\sequencen{z(2n+1)})=D\cap\zeta$.   In this case, taking
$x=\sigma*z$, we have $x(4n+1)=z(2n)$ and $x(4n+3)=z(2n+1)$ for every $n$,
so $\zeta_x=\zeta$ and $F_x=D\cap\zeta$.   Since $x\in A_D$,
we have $\eta_x<\xi$, $E_x=D\cap\eta_x$ and $\zeta\le\eta_x$.
But also $x\in A_{D'}$, so $E_x=D'\cap\eta_x$ and $D\cap\zeta=D'\cap\zeta$,
contrary to the choice of $\zeta$.\ \BanG\   Thus $D=g(\sigma)$.

\medskip

\quad{\bf case 2}
Suppose that II has a winning strategy $\tau$ in $\Game(\Bbb N,A_D)$.
Then $D\subseteq g(\tau)$.   \Quer\ If $D\ne g(\tau)$, there is a $D'$,
distinct from $D$, such that $\tau$ is a winning strategy for II in
$\Game(\Bbb N,D')$.   Let $\zeta<\xi$ be such that
$D\cap\zeta\ne D'\cap\zeta$.   Again, there is
a $z\in\NN$ such that $f(\sequencen{z(2n)})=\zeta$ and
$g_{\zeta}(\sequencen{z(2n+1)})=D\cap\zeta$.   This time, taking
$x=\tau*z$, we have $x(4n)=z(2n)$ and $x(4n+2)=z(2n+1)$ for every $n$,
so $\eta_x=\zeta$ and $E_x=D\cap\eta_x$.   Since $x\notin A_D$, we must
have $\eta_x<\zeta_x<\xi$ and $F_x=D\cap\zeta_x$;  since also
$x\notin A_{D'}$, $F_x=D'\cap\zeta_x$;  so that
$D\cap\zeta=F_x\cap\zeta=D'\cap\zeta$, which is impossible.\ \BanG\
Thus $D=g(\tau)$.

\medskip

Thus in either case $D\in g[\StrI\cup\StrII]$.   As $D$ is arbitrary,
$g[\StrI\cup\StrII]=\Cal P\xi$.\ \Qed\

Setting $g_{\xi}=gh$, the induction proceeds.

\medskip

{\bf (d)} For the inductive step to $\xi+1$ where $\omega\le\xi<\alpha$,
set

\Centerline{$h_{\xi}(0)=\xi$,
$h_{\xi}(n)=n-1$ for $n\in\omega\setminus\{0\}$,
$h_{\xi}(\eta)=\eta$ if $\omega\le\eta<\xi$,}

\Centerline{$g_{\xi+1}(x)=h_{\xi}[g_{\xi}(x)]$ for $x\in\NN$.}

\medskip

{\bf (e)} At the end of the induction, $g_{\alpha}$
is the required surjection onto $\Cal P\alpha$.
%Kanamori 28.15
}%end of proof of 567M

\leader{567N}{Theorem}\cmmnt{ ({\smc Martin 70})} [AC]
Assume that there is a \2vm\
cardinal.   Then every coanalytic subset of $\BbbN^{\Bbb N}$ is determined.

\proof{ Let $A\subseteq\NN$ be a coanalytic set.

\medskip

{\bf (a)} Set $S^*=\bigcup_{n\ge 1}\BbbN^n$.   For $\upsilon$,
$\upsilon'\in S^*$ say that $\upsilon\preccurlyeq\upsilon'$ if either
$\upsilon$ extends $\upsilon'$ or there is an
$n<\min(\#(\upsilon),\#(\upsilon'))$ such that
$\upsilon\restr n=\upsilon'\restr n$ and $\upsilon(n)<\upsilon'(n)$.   Then
$\preccurlyeq$ is a total order, and its restriction to $\BbbN^n$ is the
lexicographic well-ordering for each $n\ge 1$.

For $w\in\bigcup_{n\in\Bbb N}\BbbN^n$, set
$I_w=\{x:w\subseteq x\in\NN\}$.   Fix an enumeration
$\sequence{i}{\upsilon_i}$ of $S^*$ such that
$\#(\upsilon_i)\le i+1$ for every $i\in\Bbb N$.

\medskip

{\bf (b)} $A'=\NN\setminus A$ is Souslin-F (423Eb);  express it as

\Centerline{$A'=\bigcup_{y\in\NN}\bigcap_{n\ge 1}F_{y\restr n}$}

\noindent where $F_{\upsilon}$ is closed for every $\upsilon\in S^*$.
Replacing $F_{\upsilon}$ by
$\bigcap_{1\le i\le\#(\upsilon)}F_{\upsilon\restr i}$ if necessary, we may
suppose that $F_{\upsilon}\subseteq F_{\upsilon'}$ whenever
$\upsilon\supseteq\upsilon'$, as in 421Cf\Latereditions.

For $x\in\NN$, set

\Centerline{$T_x=\{\upsilon:\upsilon\in S^*$,
$I_{x\restr\#(\upsilon)}\cap F_{\upsilon}\ne 0\}$,}

\noindent and define a relation $\preccurlyeq_x$ on $S^*$ by saying that

$$\eqalign{\upsilon\preccurlyeq_x\upsilon'\iff
&\text{ either }\upsilon,\,\upsilon'\in T_x
   \text{ and }\upsilon\preccurlyeq\upsilon'\cr
&\text{ or }\upsilon\in T_x\text{ and }\upsilon'\notin T_x\cr
&\text{ or }\upsilon,\,\upsilon'\notin T_x\text{ and }i\le j
  \text{ where }\upsilon=\upsilon_i,\,\upsilon'=\upsilon_j.\cr}$$

\noindent Then $\preccurlyeq_x$ is a total ordering, since it copies the
total ordering $\preccurlyeq$ on $T_x$ and the
well-ordering induced by the
enumeration $\sequence{i}{\upsilon_i}$ on $S^*\setminus T_x$, and puts one
below the other.

Note that if $n\in\Bbb N$, $x$, $y\in\NN$ are such that
$x\restr n=y\restr n$, and $i<n$,
then $x\restr\#(\upsilon_i)=y\restr\#(\upsilon_i)$,
so $\upsilon_i\in T_x$ iff $\upsilon_i\in T_y$.   Consequently, for $i$,
$j<n$, $\upsilon_i\preccurlyeq_x\upsilon_j$ iff
$\upsilon_i\preccurlyeq_y\upsilon_j$.   It follows that for every
$w\in\BbbN^n$ we have a total ordering $\preccurlyeq'_w$ of $n$
defined by saying that $i\preccurlyeq'_wj$ iff
$\upsilon_i\preccurlyeq_x\upsilon_j$ whenever $x\in I_w$.

\medskip

{\bf (c)} If $x\in\NN$ and $\preccurlyeq_x$ is not a well-ordering, then
$x\notin A$.   \Prf\ Let $D\subseteq S^*$ be a non-empty set with no
$\preccurlyeq_x$-least member.   Then $D\cap T_x$ is an initial segment of
$D$.   Since $S^*\setminus T_x$ is certainly
well-ordered by $\preccurlyeq_x$, $D\cap T_x\ne\emptyset$.
Define $\sequencen{D_n}$, $\sequencen{y(n)}$ as follows.   $D_0=D\cap T_x$.
Given that $D_n$ is a non-empty
initial segment of $D$ and that $\upsilon\supseteq y\restr n$ for every
$\upsilon\in D_n$, then $y\restr n$ cannot be the least member of $D$, so
$D_n\ne\{y\restr n\}$;  set
$y(n)=\min\{\upsilon(n):\upsilon\in D_n\setminus\{y\restr n\}\}$,

\Centerline{$D_{n+1}
=\{\upsilon:\upsilon\in D_n\setminus\{y\restr n\}$, $\upsilon(n)=y(n)\}$.}

\noindent Because
$\preccurlyeq_x$ agrees with $\preccurlyeq$ on $T_x$, $D_{n+1}$ is a
non-empty initial segment of $D$, and the induction continues.

If $m$, $n\in\Bbb N$, then there is an $\upsilon\in T_x$ such that
$\upsilon\supseteq y\restr\max(m,n)$, and

\Centerline{$I_{x\restr m}\cap F_{y\restr n}
\supseteq I_{x\restr\#(\upsilon)}\cap F_{\upsilon}\ne\emptyset$.}

\noindent As $m$ is arbitrary and
$F_{y\restr n}$ is closed, $x\in F_{y\restr n}$;  as
$n$ is arbitrary, $x\in A'$ and $x\notin A$.\ \Qed

\medskip

{\bf (d)} Let $\kappa$ be a \2vm\ cardinal, and give $\Bbb N\times\kappa$
its discrete topology.
In $(\Bbb N\times\kappa)^{\Bbb N}$ consider the set
$F$ of sequences $\sequencen{(x(n),\xi(n))}$ such that

\Centerline{whenever $i$, $j\in\Bbb N$,
$\upsilon_i\subset\upsilon_j$ and $x\in F_{\upsilon_j}$, then
$\xi(2j)<\xi(2i)$.}

\noindent Then $F$ is closed
for the product topology;  by 567B, $F$ is determined.

\medskip

{\bf (e)} Suppose I has a winning strategy $\sigma$ in the game
$\Game(\Bbb N\times\kappa,F)$.
Then I has a winning strategy in $\Game(\Bbb N,A)$.   \Prf\ For
$\ofamily{i}{n}{k_i}\in\Bbb N^n$ take
$\sigma'(\ofamily{i}{n}{k_i})$ to be the first component of
$\sigma(\ofamily{i}{n}{(k_i,0)})$.   If $x$ is any play of
$\Game(\Bbb N,A)$ consistent with $\sigma'$, then for each $n$ set
$\xi(2n+1)=0$ and let $\xi(2n)$ be the second component of
$\sigma(\ofamily{i}{n}{(x(2i+1),0)})$.   Then
$\sequencen{(x(n),\xi(n))}$ is a play of $\Game(\Bbb N\times\kappa,F)$
consistent with $\sigma$, so is won by I.   \Quer\ If $x\notin A$,
let $y\in\NN$ be such that $x\in F_{y\restr n}$ for every $n\in\Bbb N$.
Set $I=\{i:i\in\Bbb N$, $y\supseteq\upsilon_i\}$;  then $I$ is infinite,
and there is an infinite $J\subseteq I$ such that
$\upsilon_i\subset\upsilon_j$ whenever $i$, $j\in J$ and $i<j$, while
$x\in F_{\upsilon_j}$ for every $j\in J$.
But now we see that $\xi(2j)<\xi(2i)$ whenever $i<j$ in
$J$, which is impossible.\ \Bang

Thus $x\in A$;  as $x$ was arbitrary, $\sigma'$ is a winning strategy for I
in $\Game(\Bbb N,A)$.\ \Qed

\medskip

{\bf (f)} Suppose II has a winning strategy $\tau$ in
$\Game(\Bbb N\times\kappa,F)$.   Then II has a winning strategy in
$\Game(\Bbb N,A)$.   \Prf\ Fix a normal $\kappa$-additive ultrafilter
$\Cal F$
on $\kappa$ (541Ma).   For $w=(k_0,\ldots,k_{2n})\in\Bbb N^{2n+1}$
consider the
function $f_w:[\kappa]^{n+1}\to\Bbb N$ defined by saying that
$f_w(J)$ is to be the first component of
$\tau(\langle(k_{2i},\xi_i)\rangle_{i\le n})$
where $(\xi_0,\ldots,\xi_n)$ is
that enumeration of $J$ such that, for $i$, $j\le n$, $\xi_i\le\xi_j$ iff
$i\preccurlyeq'_wj$.   Then for each $m\in\Bbb N$ there is a
$C_{wm}\in\Cal F$ such that either $f_w(J)\le m$
for every $J\in[C_{wm}]^{n+1}$
or $f_w(J)>m$ for every $J\in[C_{wm}]^{n+1}$ (4A1L).  Setting
$C=\bigcap_{m,n\in\Bbb N}\bigcap_{w\in\BbbN^{2n+1}}C_{ w m}$,
$C\in\Cal F$ and every $f_w$ is constant on $[C]^{n+1}$.
Let $\rho(w)$ be the constant value of $f_w\restr[C]^{n+1}$.

Define $\tau':\bigcup_{n\ge 1}\BbbN^n\to\Bbb N$ inductively, saying that
$\tau'(k_0,\ldots,k_n)=\rho(w)$
whenever $ w(2i)=k_i$ for $i\le n$ and
$ w(2i+1)=\tau'(k_0,\ldots,k_i)$ for $i<n$.
Suppose that $x$ is a play of $\Game(\Bbb N,A)$ consistent with
$\tau'$.   \Quer\ If $x\in A$, then
$\preccurlyeq_x$ is a well-ordering, by (c).   The order type of
$(S^*,\preccurlyeq_x)$ is countable, so is surely less than
$\otp(C)=\kappa$, and we have a function $\theta:S^*\to C$ such that
$\theta(\upsilon)\le\theta(\upsilon')$ iff
$\upsilon\preccurlyeq_x\upsilon'$.

Define $\sequencen{\xi(n)}$ by saying that

$$\eqalign{\xi(n)&=\theta(\upsilon_j)\text{ if }n=2j
  \text{ is even },\cr
&=\text{ the second component of }
  \tau((x(0),\xi(0)),(x(2),\xi(2)),\ldots,(x(2j),\xi(2j)))\cr
&\mskip200mu\text{ if }n=2j+1\text{ is odd}.\cr}$$

\noindent For $i$, $j\le n\in\Bbb N$, setting
$ w=x\restr 2n+1$,
we have $\xi(2i)$, $\xi(2j)\in C$ and

\Centerline{$\xi(2i)\le\xi(2j)
\iff\theta(\upsilon_i)\le\theta(\upsilon_j)
\iff\upsilon_i\preccurlyeq_x\upsilon_j
\iff i\preccurlyeq'_wj$.}

\noindent So

\Centerline{$x(2n+1)=\rho(w)=f_w(\{\xi(2i):i\le n\})$}

\noindent is the first component of
$\tau((x(0),\xi(0)),\ldots,\penalty-100(x(2n),\xi(2n)))$;  thus
$\sequencen{(x(n),\xi(n))}$ is a play of $\Game(\Bbb N\times\kappa,F)$
consistent with $\tau$, and is won by II.   There must therefore be
$i$, $j\in\Bbb N$ such that $\upsilon_i\subset\upsilon_j$,
$x\in F_{\upsilon_j}$ and and
$\xi(2i)\le\xi(2j)$.   Now $\upsilon_j\in T_x$ and
$\upsilon_i\preccurlyeq_x\upsilon_j$, so
$\upsilon_i\preccurlyeq\upsilon_j$;  which is impossible.\ \Bang

So $x\notin A$;  as $x$ is arbitrary, $\tau'$ is a winning strategy for
II in $\Game(\Bbb N,A)$.\ \Qed

\medskip

{\bf (g)} Putting (d), (e) and (f) together, we see that $A$ is determined.
}%end of proof of 567N

\leader{567O}{Corollary}\dvArevised{2010} [AC]
If there is a \2vm\ cardinal, then every 
PCA\cmmnt{ ($=\pmb{\Sigma}^1_2$)} subset of any Polish space is 
universally measurable.

\proof{{\bf (a)} Let $A\subseteq\{0,1\}^{\Bbb N}$ be PCA.   Then
there is a coanalytic subset $B$ of $\NN\times\{0,1\}^{\Bbb N}$
such that $A$ is the projection of $B$.   Of course this means that there
is a coanalytic subset $B'$ of $\{0,1\}^{\Bbb N}$ such that
$A$ is a continuous image of
$B'$, since $\NN\times\{0,1\}^{\Bbb N}$ is homeomorphic to a G$_{\delta}$
subset of $\{0,1\}^{\Bbb N}$, and any homeomorphism must carry $B$ to a
coanalytic subset of $\{0,1\}^{\Bbb N}$, by 423Rc.
Now $(h^{-1}[B']\cap F)\cup H$ is coanalytic whenever
$h:\NN\to\{0,1\}^{\Bbb N}$ is continuous, $F\subseteq\NN$ is closed and
$H\subseteq\NN$ is open;  by 567N, $(\Bbb N,(h^{-1}[B']\cap F)\cup H)$ is
always determined;  by 567F, $A$ is measured by the usual measure $\nu$ on
$\{0,1\}^{\Bbb N}$.

\medskip

{\bf (b)} Now suppose that $X$ is a Polish space,
$A\subseteq X$ is a PCA set and $\mu$ is a
Borel probability measure on $X$ with completion $\hat\mu$.
Then there is a Borel measurable function $f:\{0,1\}^{\Bbb N}\to X$
such that $\hat\mu$ is the image measure $\nu f^{-1}$.
\Prf\ Let $(\frak A,\bar\mu)$ and $(\frak B,\bar\nu)$ be the measure
algebras of $\mu$, $\nu$ respectively.   Then $\frak A$ has Maharam type at
most $w(X)=\omega$ (531Aa), so there is a measure-preserving Boolean
homomorphism $\pi:\frak A\to\frak B$ (332N).   Now $\hat\mu$ is
a Radon measure (433Cb), so there is a function
$f_0:\{0,1\}^{\Bbb N}\to X$ such that $f_0^{-1}[E]$ is measured by
$\nu$, and $\nu f_0^{-1}[E]=\hat\mu E$, whenever $E$ is measured by
$\hat\mu$
(416Wb).   In this case, $f_0$ is almost continuous (433E) and there is a
sequence $\sequencen{K_n}$ of compact subsets of $\{0,1\}^{\Bbb N}$ such
that $\lim_{n\to\infty}\nu K_n=1$ and $f_0\restr K_n$ is continuous for
every $n$.   Fix any $x_0\in X$ and set $f(z)=f_0(z)$ for
$z\in\bigcup_{n\in\Bbb N}K_n$, $x$ for
other $z\in\{0,1\}^{\Bbb N}$;  then $f$ is Borel measurable and equal
$\nu$-a.e.\ to $f_0$, so $f$ also is \imp\ for $\nu$ and $\hat\mu$.
Finally, because $f$ likewise is almost continuous, the image measure
$\nu f^{-1}$ on $X$ is a Radon measure (418I), and must be exactly
$\hat\mu$ (416Eb).\ \Qed

Since $f^{-1}[A]$ is PCA (423Rd), $\nu$ measures $f^{-1}[A]$ and $\hat\mu$
measures $A$.   As $\mu$ is arbitrary, $A$ is universally measurable.
}%end of proof of 567O

\exercises{\leader{567X}{Basic exercises (a)}
%\spheader 567Xa
Let $X$ be a non-empty well-orderable set, with its discrete topology,
and $G\subseteq X^{\Bbb N}$ an open set.   Show that $G$ is determined.
%567B

\spheader 567Xb [AC($\Bbb R;\omega$)] Let $A\subseteq\NN$ be such that
$\{x:\fraction{n}^{\smallfrown}x\in A\}$ is determined for every
$n\in\Bbb N$.   Show that $\NN\setminus A$ is determined.
%567B

\spheader 567Xc Show that AD is true iff every subset of $\{0,1\}^{\Bbb N}$
is determined.   \Hint{For $x\in\{0,1\}^{\Bbb N}$ set
$I_x=\{n:x(2n)=1\}$, $J_x=\{n:x(2n+1)=1\}$;  set
$C_{\text{I}}=\{x:\sup I_x>\sup J_x\}$,
$D=\{x:I_x$ and $J_x$ are both infinite$\}$.   Define $f:D\to\NN$ by
setting $f(x)(0)=\min I_x$, $f(x)(2n+1)=\min\{k:f(x)(2n)+k\in J_x\}$,
$f(x)(2n+2)=\min\{k:f(x)(2n+1)+k+1\in I_x\}$.   Show that if
$A\subseteq\NN$ and $C_{\text{I}}\cup f^{-1}[A]$ is determined, then
$A$ is determined.}
%567C

\spheader 567Xd [AC($\Bbb R;\omega$)] (i) Show that the intersection of a
sequence of closed cofinal subsets of $\omega_1$ is cofinal.   (ii) Show
that we have a unique topological probability measure on $\omega_1$ which
is zero on singletons and inner regular with respect to the closed sets.
%567E

\spheader 567Xe [AD] Show that if $f:[0,1]^2\to\Bbb R$ is a bounded
function, then $\iint f(x,y)dxdy$ and $\iint f(x,y)dydx$ are defined and
equal, where the integrations are with respect to Lebesgue measure on
$[0,1]$.
%567G

\spheader 567Xf [AD] Let $\mu$ be a Radon measure on a Polish space $X$,
and $\Cal E$ a well-ordered family of subsets of $X$.   Show that
$\mu(\bigcup\Cal E)=\sup_{E\in\Cal E}\mu E$.   \Hint{567Xe.}
%567Xe 567G

\spheader 567Xg [AD+AC($\omega$)]
Show that there are no interesting Sierpi\'nski sets, in the
sense that every atomless probability space has an uncountable negligible
subset.
%567G

\spheader 567Xh [AD] Show that every semi-finite measure space is
perfect.
%567G

\spheader 567Xi [AD] Show that if $X$ is a separable Banach space and $Y$
is a normed space then every linear operator from $X$ to $Y$ is bounded.
%567H

\spheader 567Xj\dvAnew{2011}
[DC] Let $I$ be a set, and $\widehat{\Cal B}$ the Baire-property algebra of
$\Cal PI$ with its usual topology.   Show that every
$\widehat{\Cal B}$-measurable real-valued
finitely additive functional on $\Cal PI$ is completely additive.
\Hint{remember to prove that $\Cal PI$ is a Baire space.}
%567I out of order query

\spheader 567Xk [AD] (i) Show that there is no non-principal ultrafilter on
$\Bbb N$.   (ii) Show that $\{0,1\}^{\Bbb R}$ is not compact.
%567J

\spheader 567Xl [AD] Show that there is no linear
lifting for Lebesgue measure on $\Bbb R$.   \Hint{567J.}
%567J

\spheader 567Xm [AD] (i) Show that $\ell^1(\Bbb R)$ is not reflexive.
(ii) Show that $\ell^1(\omega_1)$ is not reflexive.
%567G 567L

\spheader 567Xn [AD] (i) Show that there is no injective function from
$\omega_1$ to $\Bbb R$.   (ii) Show that there is no family
$\ofamily{\xi}{\omega_1}{f_{\xi}}$ such that $f_{\xi}$ is an injective
function from $\xi$ to $\Bbb N$ for every $\xi<\omega_1$.   (iii) Show that
there is no function $f:\omega_1\times\Bbb N\to\omega_1$ such that
$\{f(\xi,n):n\in\Bbb N\}$ is a cofinal subset
of $\xi$ for every non-zero limit ordinal $\xi<\omega_1$.   \Hint{567L.}
%567L

\spheader 567Xo (i) Show that there is a set $A\subseteq\omega_1^{\Bbb N}$
such that $\Game(\omega_1,A)$ is not determined.   \Hint{Set II the task of
enumerating $x(0)$;  see 567D and 567Xn.}
(ii) Show that there is a set $A\subseteq(\Cal P\Bbb R)^{\Bbb N}$
such that $\Game(\Cal P\Bbb R,A)$ is not determined.
%567Xn 567L Mycielski 64? Kanamori 03, 27.12

\spheader 567Xp [AD] Show that there is a surjective function from
$\Bbb R$ to $\Cal B(\Bbb R)$, but no injective function from
$\Cal B(\Bbb R)$ to $\Bbb R$.   \Hint{567E, 561Xd.}
%567Xn, 567L

\spheader 567Xq [AC] Show that if there is a \2vm\ cardinal and
$A\subseteq\NN$ is analytic then $A$ is determined.
%567N

\spheader 567Xr [AC] Suppose that there is a \2vm\ cardinal.   Show
that every PCA subset of $\Bbb R$ has the Baire property.
%567O

\leader{567Y}{Further exercises (a)}
%\spheader 567Ya
Let $X$ be a non-empty
set and $A\subseteq X^{\Bbb N}$.   A
{\bf quasi-strategy for I} in $\Game(X,A)$ is a function
$\sigma:\bigcup_{n\in\Bbb N}X^n\to\Cal PX\setminus\{\emptyset\}$;  it
is a winning quasi-strategy if $x\in A$ whenever $x\in X^{\Bbb N}$ and
$x(2n)\in\sigma(\ofamily{i}{n}{x(2i+1)})$ for every $n$.   Similarly, a
winning quasi-strategy for II is a function
$\tau:\bigcup_{n\ge 1}X^n\to\Cal PX\setminus\{\emptyset\}$ such that
$x\notin A$ whenever $x\in X^{\Bbb N}$ and
$x(2n+1)\in\tau(\ofamily{i}{n}{x(2i)})$ for every $n$.
(i) Show that if $X$ is any non-empty discrete space
and $F\subseteq X^{\Bbb N}$ is closed then at least one player has a
winning quasi-strategy in $\Game(X,F)$.
(ii) Show that DC is true iff there is no game $\Game(X,A)$ such that
both players have winning quasi-strategies.
%567B mt56bits

\spheader 567Yb [AD] Show that every uncountable subset of $\Bbb R$ has a
non-empty perfect subset.
({\it Hint\/}:  Let $A\subseteq\{0,1\}^{\Bbb N}$.
Enumerate $\bigcup_{n\in\Bbb N}\{0,1\}^n$ as
$\sequence{j}{\upsilon_j}$.   For $x\in\NN$ set

\Centerline{$f(x)
={\upsilon_{x(0)}}^{\smallfrown}\fraction{\min(1,x(1))}^{\smallfrown}
  {\upsilon_{x(2)}}^{\smallfrown}\fraction{\min(1,x(3))}^{\smallfrown}
  \ldots$.}

\noindent Consider $\Game(\Bbb N,f^{-1}[A])$.)
%567G Kanamori 03, 27.5

\spheader 567Yc\dvAnew{2010} [AD]
Let $X$ be an analytic Hausdorff space, and
$c:\Cal PX\to[0,\infty]$ a submodular Choquet capacity.   Show that
$c(A)=\sup\{c(K):K\subseteq A$ is compact$\}$  for every
$A\subseteq X$.  (Cf.\ 479Yj.)
%567G out of order query
%mt56bits

\spheader 567Yd\dvAnew{2011}
[AC($\Bbb R;\omega$)] Let $\frak A$ be a Dedekind $\sigma$-complete Boolean
algebra, and $\nu:\frak A\to\Bbb R$ an additive functional which is Borel
measurable for the order-sequential topology on $\frak A$.   Show that
$\nu$ is countably additive.
%567J out of order query

\spheader 567Ye Let $\Theta$ be the least ordinal such that there
is no surjection from $\Cal P\Bbb N$ onto $\Theta$.
(i) [AC($\omega$)] Show that $\cf\Theta>\omega$.
(ii) [AD] Show that $\Theta=\omega_{\Theta}$.
%use method of 567M to build a family
%$\ofamily{\xi}{\Theta}{f_{\xi}}$ such that
%$f_{\xi}:\Cal P\Bbb N\to\omega_{\xi}$ is surjective for every
%$\xi<\theta$.
%any chance of  \cf\Theta>\omega  with  AD  bu not  AC(\omega) ? query
%567M

\spheader 567Yf
[AC] Suppose that there is a \2vm\ cardinal.   Show that
every uncountable PCA subset of $\Bbb R$ has a non-empty perfect subset.
%567O 567Yb Kanamori 27.14
}%end of exercises

\endnotes{
\Notesheader{567} The consequences of the axiom of determinacy are so
striking that the question of its consistency is particularly pressing.
In fact W.H.Woodin has determined
its consistency strength, in terms of large cardinals
({\smc Kanamori 03}, 32.16, or {\smc Jech 03}, 33.27), and
this is less than that of the existence of a supercompact cardinal;  so it
seems safe enough.

%\query what about AD+DC?

In ZFC, 567B is most naturally thought of as a basic special case of
Martin's theorem that
every Borel subset of $X^{\Bbb N}$, for any discrete space $X$,
is determined ({\smc Martin 75}, or {\smc Kechris 95}, 20.5).
The idea of the proof is that if II has no winning strategy, then all I has
to do is to avoid positions from which II can win.   But for a proof in ZF
we need more than this.   It would not be enough to show that for every
first move by I, there is a winning strategy for II from the resulting
position;  we should
need to show that these can be pieced together as a single
function $\tau:\bigcup_{n\ge 1}X^n\to X$.   Turning this round, AD must
imply a weak form of the axiom of choice (567D;  see also 567Xo).
In the particular case
of 567B, we have a basic set $W_0$ of winning positions for II with a
trivial family $\family{w}{W_0}{\tau_w}$ of strategies.   (Starting
from a position in
$W_0$, II can simply play the $\preccurlyeq$-least point of $X$ to get
a position from which I cannot avoid $W_0$.)
From these we can work backwards to construct a
family $\family{w}{W}{\tau_w}$ of
strategies, where $W=\bigcup_{\xi\in\On}W_{\xi}$;  so that if
$\fraction{u}\in W$ for every $u\in X$, we can assemble these into a
winning strategy for II in $\Game(X,F)$.

The central result of the section is I suppose 567G.   From the point of
view of a real analyst like myself, as opposed to a logician or set
theorist, this is the door into a different world, explored in
567H-567K, %567H 567J 567K
567Xe-567Xm %567Xe 567Xf 567Xg 567Xi 567Xl 567Xk 567Xm
and 567Yb.
In 567I we have a result which is already interesting in ZFC.   Recall that
in ZFC there are non-trivial
additive functionals on $\Cal P\Bbb N$ which are measurable
in the sense of \S464 (464Jb);
none of them can be Baire-property-measurable.

I have not talked about `automatic continuity' in this book.   If you have
seen anything of this subject you will recognise the three parts of
567H as versions of standard results on homomorphisms which are
measurable in some sense.   I do not know whether the hypothesis `abelian'
is necessary in 567Hb.
If you like, 567J can also be thought of as an
automatic-continuity result.

You will see that 567H-567J %567H 567I 567J
depend on 567Gb rather than on 567Ga;  that is, on category rather than on
measure.   It is not clear how much can be proved if we assume, as an
axiom, that every subset of $\Bbb R$ is Lebesgue measurable (together with
AC($\omega$) at least, of course), rather than that every subset of
$\Bbb R$ has the Baire property.

In 567L far more is true, at least with AD+DC;  $\omega_2$, as
well as $\omega_1$,
is \2vm, and the filter on $\omega_1$ generated by the closed
cofinal sets is an ultrafilter ({\smc Kanamori 03}, \S28, or
{\smc Jech 03}, Theorem 33.12).
I am not sure what we should think of as a `\rvm\ cardinal' in this
context.   In the language of 566Xl, AD implies that $\Bbb R$ is not
measure-free, and Lebesgue measure is $\kappa$-additive for every initial
ordinal $\kappa$ (567Xf).
For further combinatorial consequences of AD, see {\smc Kanamori 03}.
Note that AD implies CH in the form `every uncountable subset of $\Bbb R$
is equipollent with $\Bbb R$' (567Yb).   But the relationship of $\Bbb R$
with $\omega_1$ is quite different.   ZF is enough to build a surjection
from $\Bbb R$ onto $\omega_1$.   AD implies that there is no injection
from $\omega_1$ into $\Bbb R$ (567Xn) but that
there are surjections from $\Bbb R$ onto much larger initial ordinals
(567M, 567Ye).

In 567N-567O I return to the world of ZFC;  they are in this section
because they depend on 567B and 567F.   Once again, much more is known
about determinacy compatible with AC, and
may be found in {\smc Kanamori 03} or {\smc Jech 03}.

}%end of notes

\discrpage
