\frfilename{mt21.tex} 
\versiondate{17.1.15} 
\copyrightdate{1994} 
 
\def\chaptername{Taxonomy of measure spaces} 
 
\gdef\topparagraph{} 
   \gdef\bottomparagraph{Chap.\ 21 {\it intro.}} 
   \gdef\newparagraph{Chap.\ 21 {\it intro.}} 
   \gdef\sectionname{Introduction} 
   \gdef\headlinesectionname{Introduction} 
   \centerline{\bf *Chapter 21} 
   \medskip 
   \centerline{\bf \chaptername} 
   \medskip 
 
I begin this volume with a `starred chapter'.   The point is that I do 
not really recommend this chapter for beginners.   It deals with a 
variety of technical questions which are of great importance for the 
later development of the subject, but are likely to be both abstract and 
obscure for anyone who has not encountered the problems these techniques 
are designed to solve.   On the other hand, if (as is customary) this 
work is omitted, and the ideas are introduced only when urgently needed, 
the student is likely to finish with very vague ideas on which theorems 
can be expected to apply to which types of measure space, and with no 
vocabulary in which to express those ideas.   I therefore 
take a few pages to introduce the terminology and concepts which can be 
used to distinguish `good' measure spaces from others, with a few of 
the basic relationships.   The only paragraphs which are immediately 
relevant to the theory set out in Volume 1 are those on `complete',  
`$\sigma$-finite' and 
`semi-finite' measure spaces (211A, 211D, 211F, 211Lc, \S212,  
213A-213B, 215B), and on Lebesgue measure (211M).   For the rest, I think  
that a newcomer 
to the subject can very well pass over this chapter for the time being, and return to it 
for particular items when the text of later chapters refers to it. 
On the other hand, it can also be used as an introduction to the flavour  
of the `purest' kind of measure theory, the study of measure spaces for  
their own sake, with a systematic discussion of a few of the elementary  
constructions. 
 
\discrpage 
 
