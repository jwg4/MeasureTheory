\frfilename{mt466.tex}
\versiondate{2.8.13}
\copyrightdate{2000}

\def\chaptername{Pointwise compact sets of measurable functions}
\def\sectionname{Measures on linear topological spaces}

\newsection{466}

In this section I collect a number of results on the special properties
of topological measures on linear topological spaces.   The most
important is surely Phillips' theorem (466A-466B):  on any Banach space,
the weak and norm topologies give rise to the same totally finite Radon
measures.   This is not because the weak and norm topologies have the
same Borel $\sigma$-algebras, though this does happen
in interesting cases (466C-466E, \S467).   When the Borel
$\sigma$-algebras are different, we can still
ask whether the Borel measures
are `essentially' the same, that is, whether every (totally finite)
Borel measure for the weak topology extends to a Borel measure for the
norm topology.   A construction due to M.Talagrand (466H, 466Ia) gives a
negative answer to the general question.

Just as in $\BbbR^r$, a totally finite quasi-Radon measure on a locally
convex linear topological space is determined by its characteristic
function (466K).   I end the section with a note on measurability
conditions sufficient to ensure that a linear operator between Banach
spaces is continuous (466L-466M), and with brief remarks on Gaussian
measures (466N-466O).

\leader{466A}{Theorem} Let $(X,\frak T)$ be a metrizable locally convex
linear topological space and $\mu$ a
$\sigma$-finite measure on $X$ which is quasi-Radon for the weak
topology $\frak T_s(X,X^*)$.   Then the support of $\mu$ is
separable, so $\mu$ is quasi-Radon for the original topology $\frak T$.
If $X$ is complete and $\mu$ is locally finite with respect to
$\frak T$, then $\mu$ is Radon for $\frak T$.

\proof{{\bf (a)} Let $\sequencen{V_n}$ be a sequence running over a base
of neighbourhoods of $0$ in $X$, and for $n\in\Bbb N$ set
$V_n^{\smallcirc}=\{f:f\in X^*$, $f(x)\le 1$ for every $x\in V_n\}$.
Then each $V_n^{\smallcirc}$ is a convex
$\frak T_s$-compact subset of $X^*$ (4A4Bf), and
$X^*=\bigcup_{n\in\Bbb N}V_n^{\smallcirc}$.   Let $Z\subseteq X$ be the
support of $\mu$, and for $n\in\Bbb N$ set
$K_n=\{f\restr Z:f\in V_n^{\smallcirc}\}$.   Since the map
$f\mapsto f\restr Z:X^*\to C(Z)$ is linear and continuous for
$\frak T_s$ and the topology $\frak T_p$ of pointwise convergence on
$Z$, each $K_n$ is convex and $\frak T_p$-compact.   Next, the subspace
measure $\mu_Z$ on $Z$ is $\sigma$-finite and strictly positive, so by
463G each $K_n$ is $\frak T_p$-metrizable.

Let $\Cal U$ be the base for $\frak T_p$ consisting of sets of the form

\Centerline{$U(I,h,\epsilon)
=\{f:f\in C(Z),\,|f(x)-h(x)|<\epsilon$ for every $x\in I\}$,}

\noindent where $I\subseteq Z$ is finite, $h\in\BbbR^I$ and
$\epsilon>0$.
For each $n\in\Bbb N$, write $\Cal V_n=\{K_n\cap U:U\in\Cal U\}$, so
that $\Cal V_n$ is a base for the subspace topology of $K_n$
(4A2B(a-vi)).   Since this is compact and metrizable, therefore
second-countable, there is a countable base $\Cal V'_n\subseteq\Cal V_n$
(4A2Ob).   Now there is a countable set $\Cal U'\subseteq\Cal U$
such that $\Cal V'_n\subseteq\{K_n\cap U:U\in\Cal U'\}$ for every
$n\in\Bbb N$, and a countable set $D\subseteq Z$ such that

\Centerline{$\Cal U'\subseteq\{U(I,h,\epsilon):I\subseteq D$ is finite,
$h\in\BbbR^I$, $\epsilon>0\}$.}

Let $D'\subseteq X$ be the linear span of $D$ over the rationals.   Then
$D'$ is again countable, and its $\frak T$-closure $Y$ is a linear
subspace of $X$ (4A4Bg).   \Quer\ If $Z\not\subseteq Y$, then
$\mu(X\setminus Y)>0$.   But, writing

$$\eqalign{Y^{\smallcirc}
&=\{f:f\in X^*,\,f(x)\le 1\text{ for every }x\in Y\} \cr
&=\{f:f\in X^*,\,f(x)=0\text{ for every }x\in Y\},\cr}$$

\noindent $X\setminus Y$ must be
$\bigcup_{f\in Y^{\smallcirc}}\{x:|f(x)|>0\}$, by 4A4Eb.
Because $\mu$ is $\tau$-additive, there must be an
$f\in Y^{\smallcirc}$ such that $\mu\{x:|f(x)|>0\}>0$.   Let
$n\in\Bbb N$ be such that $f\in V_n^{\smallcirc}$.   Since $f(x)=0$ for
every $x\in D$,
$f\restr Z$ belongs to the same members of $\Cal V'_n$ as the zero
function;  since $\Cal V'_n$ is a base for the Hausdorff subspace
topology of $K_n$, $f\restr Z$ actually is the zero function, and
$\{x:|f(x)|>0\}\subseteq X\setminus Z$, so $\mu Z<\mu X$, contrary to
the choice of $Z$.\ \Bang

Thus $Z\subseteq Y$.   Because $Y=\overline{D'}$ is $\frak T$-separable,
and $\frak T$ is metrizable, $Z$ also is $\frak T$-separable
(4A2P(a-iv)).

\medskip

{\bf (b)} The subspace topology $\frak T_Y$ induced on $Y$ by $\frak T$
is a separable metrizable locally convex linear space topology, so the
Borel $\sigma$-algebras of $\frak T_Y$ and the associated weak topology
$\frak T_s(Y,Y^*)$ are equal (4A3V).   But $\frak T_s(Y,Y^*)$ is
just the subspace topology induced on $Y$ by $\frak T_s$ (4A4Ea),
Accordingly the subspace measure $\mu_Y$ on $Y$ is
$\frak T_s(Y,Y^*)$-quasi-Radon (415B).   Since every $\frak T_Y$-open
set is $\frak T_s(Y,Y^*)$-Borel, $\mu_Y$ is a topological measure for
$\frak T_Y$.   Since $\frak T_Y$ is finer than $\frak T_s(Y,Y^*)$,
$\mu_Y$ is effectively locally finite for $\frak T_Y$ and inner regular
with respect to the $\frak T_Y$-closed sets, and therefore is a
quasi-Radon measure for $\frak T_Y$ (415D(i)).

By 415J, there is a measure $\tilde\mu$ on $X$, quasi-Radon for
$\frak T$, such that $\tilde\mu E=\mu_Y(E\cap Y)$ whenever $\tilde\mu$
measures $E$.   But as $Y$ is $\frak T$-closed and $\mu$-conegligible,
and $\mu$ is complete, we have $\mu=\tilde\mu$, and $\mu$ is quasi-Radon
for $\frak T$.

\medskip

{\bf (c)} If $X$ is complete and $\mu$ is locally finite with respect to
$\frak T$, then $(X,\frak T)$ is a pre-Radon space (434Jg), so $\mu$ is
a Radon measure for $\frak T$ (434Jb).
}%end of proof of 466A

\leader{466B}{Corollary}\cmmnt{ (Compare 462I.)}  If $X$ is a Banach
space and $\mu$ is a totally
finite measure on $X$ which is quasi-Radon for the weak topology of $X$,
it is a Radon measure for both the norm topology and the weak topology.

\proof{ By 466A, $\mu$ is a Radon measure for the norm topology;  by
418I, or otherwise, it is a Radon measure for the weak topology.
}%end of proof of 466B

\cmmnt{\medskip

\noindent{\bf Remark} Thus Banach spaces, with their weak topologies,
are pre-Radon.
}%end of comment

\leader{466C}{Definition} A normed space $X$ has a {\bf Kadec norm}
(also called {\bf Kadec-Klee norm}) if
the norm and weak topologies coincide on the sphere $\{x:\|x\|=1\}$.
Of course they will then also coincide on any sphere
$\{x:\|x-y\|=\alpha\}$.

\cmmnt{\medskip

\noindent{\bf Example} For any set $I$ and
any $p\in\coint{1,\infty}$, the
Banach space $\ell^p(I)$ has a Kadec norm.   \prooflet{\Prf\
Set $S=\{x:\|x\|_p=1\}$.   If $x\in S$ and $\epsilon>0$, take
$\eta\in\ocint{0,1}$ such that
$2\eta+(2p\eta)^{1/p}\le\epsilon$.   Let
$J\subseteq I$ be a finite set such that
$\sum_{i\in I\setminus J}|x(i)|^p\le\eta^p$.   Set
$H=\{y:y\in\ell^p(I),\,\sum_{i\in J}|y(i)-x(i)|^p<\eta^p\}$;
then $H$ is open for the weak topology of $\ell^p(I)$.
If $y\in H\cap S$, then, writing $x_J$ for
$x\times\chi J$, etc.,

\Centerline{$\|x_J\|_p\ge 1-\|x_{I\setminus J}\|_p
\ge 1-\eta$,}

\Centerline{$\|y_J\|_p
\ge\|x_J\|_p-\eta\ge 1-2\eta$,
\quad$\|y_J\|_p^p\ge 1-2p\eta$,}


$$\eqalign{\|y-x\|_p
&\le\|y_J-x_J\|_p+\|x_{I\setminus J}\|_p+\|y_{I\setminus J}\|_p\cr
&\le\eta+\eta+(1-\|y_J\|_p^p)^{1/p}
\le 2\eta+(2p\eta)^{1/p}
\le\epsilon.\cr}$$

\noindent Thus $\{y:y\in S$, $\|y-x\|_p\le\epsilon\}$ is a
neighbourhood of $x$ for the subspace weak topology on $S$;   as $x$ and
$\epsilon$ are arbitrary, the weak and norm topologies agree on $S$.\ \Qed}%end of prooflet

For further examples, see 467B {\it et seq}.}%end of comment

\leader{466D}{Proposition}\cmmnt{ ({\smc Hansell 01})} Let $X$ be a
normed space with a Kadec norm.   Then there is a network for the norm
topology on $X$ expressible in the form $\bigcup_{n\in\Bbb N}\Cal V_n$,
where for each $n\in\Bbb N\,\,\Cal V_n$ is an isolated family for
the weak topology and
$\bigcup\Cal V_n$ is the difference of two closed sets for the weak
topology.

\proof{ Let $\Cal U$ be a
$\sigma$-disjoint base for the norm topology of $X$ (4A2L(g-ii));
express it as
$\bigcup_{n\in\Bbb N}\Cal U_n$ where every $\Cal U_n$ is disjoint.   For
rational numbers $q$, $q'$ with $0<q<q'$, set
$S_{qq'}=\{x:q<\|x\|\le q'\}$, and for $A\subseteq X$ write $W(A,q,q')$
for the interior of $A\cap S_{qq'}$ taken in the subspace weak topology
of $S_{qq'}$.   Set
$\Cal V_{nqq'}=\{W(U,q,q'):U\in\Cal U_n\}$, so that $\Cal V_{nqq'}$ is a
disjoint family of relatively-weakly-open subsets of $S_{qq'}$, and is
an isolated family for the weak topology.   Now
$\bigcup_{n\in\Bbb N,q,q'\in\Bbb Q,0<q<q'}\Cal V_{nqq'}$ is a network
for the norm topology on $X\setminus\{0\}$.   \Prf\ If
$x\in X\setminus\{0\}$ and $\epsilon>0$, then take
$n\in\Bbb N$, $U\in\Cal U_n$ such that
$x\in U\subseteq\{y:\|y-x\|\le\epsilon\}$.  Let $\delta>0$ be such that
$\{y:\|y-x\|\le\delta\}\subseteq U$.   Next, because $\|\,\|$ is a Kadec
norm, there is a weak neighbourhood $V$ of $0$ such that
$\|y-x\|\le\bover12\delta$ whenever $y\in x-V$ and $\|y\|=\|x\|$.   Let
$V'$ be an open weak neighbourhood of $0$ such that $V'+V'\subseteq V$.
Let $\eta\in\ooint{0,1}$ be such that $\eta\|x\|\le\bover12\delta$ and
$y\in V'$ whenever $\|y\|\le\eta\|x\|$.
If $y\in x-V'$ and $(1-\eta)\|x\|\le\|y\|\le(1+\eta)\|x\|$, then

\Centerline{$\|y-\Bover{\|x\|}{\|y\|}y\|
=|1-\Bover{\|x\|}{\|y\|}|\,\|y\|
=|\Bover{\|y\|}{\|x\|}-1|\,\|x\|\le\eta\|x\|\le\Bover12\delta$,}

\Centerline{$x-\Bover{\|x\|}{\|y\|}y
=(x-y)+(y-\Bover{\|x\|}{\|y\|}y)\in V'+V'\subseteq V$,}

\noindent so $\|x-\Bover{\|x\|}{\|y\|}y\|\le\bover12\delta$ and
$\|x-y\|\le\delta$
and $y\in U$.   This means that if we take $q$, $q'\in\Bbb Q$ such that
$(1-\eta)\|x\|\le q\le\|x\|\le q'\le(1+\eta)\|x\|$, then
$(x-V')\cap S_{qq'}\subseteq U$ and $x\in W(U,q,q')\in\Cal V_{nqq'}$.
Since of course $W(U,q,q')\subseteq U\subseteq\{y:\|y-x\|\le\epsilon\}$,
and $x$ and $\epsilon$ are arbitrary, we have the result.\ \Qed

To get a $\sigma$-isolated family for the weak topology which is
a network for the norm topology on the whole of $X$, we just have to add
the singleton set $\{0\}$.   To see that the union of each of our
isolated families is the difference of two weakly open sets,
observe that $\bigcup\Cal V_{nqq'}$
is a relatively weakly open subset of $S_{qq'}$, which is the difference
of the weakly open sets $\{x:\|x\|>q\}$ and $\{x:\|x\|>q'\}$.
}%end of proof of 466D

\leader{466E}{Corollary} Let $X$ be a normed space with a Kadec norm.

(a) The norm and weak topologies give rise to the same Borel
$\sigma$-algebras.

(b) The weak topology has a $\sigma$-isolated network, so is
hereditarily weakly $\theta$-refinable.

\proof{{\bf (a)} Write $\Cal B_{\|\,\|}$, $\Cal B_{\frak T_s}$ for the
Borel $\sigma$-algebras for the weak and norm topologies.   Of course
$\Cal B_{\frak T_s}\subseteq\Cal B_{\|\,\|}$.   Let
$\sequencen{\Cal V_n}$ be a sequence covering a network for the norm
topology as in 466D.   Because
$\Cal V_n$ is (for the weak topology) isolated and its union
belongs to $\Cal B_{\frak T_s}$, $\bigcup\Cal W\in\Cal B_{\frak T_s}$
for every $n\in\Bbb N$ and $\Cal W\subseteq\Cal V_n$.   But this means that
$\bigcup\Cal W\in\Cal B_{\frak T_s}$ for every
$\Cal W\subseteq\Cal V=\bigcup_{n\in\Bbb N}\Cal V_n$;  and as
$\Cal V$ is a network for the norm topology, every
norm-open set belongs to $\Cal B_{\frak T_s}$, and
$\Cal B_{\|\,\|}\subseteq\Cal B_{\frak T_s}$.   Thus the two Borel
$\sigma$-algebras are equal.

\medskip

{\bf (b)} Of course $\Cal V$ is also a network for the weak topology, so
the weak topology has a $\sigma$-isolated network;  by
438Ld, it is hereditarily weakly $\theta$-refinable.
}%end of proof of 466E

\vleader{48pt}{466F}{Proposition} 
Let $X$ be a Banach space with a Kadec norm.
Then the following are equiveridical:

(i) $X$ is a Radon space in its norm topology;

(ii) $X$ is a Radon space in its weak topology;

(iii) the weight of $X$ (for the norm topology) is
measure-free\cmmnt{ in the sense of \S438}.

\proof{{\bf (a)(i)$\Leftrightarrow$(ii)} By 466Ea, the norm and weak topologies
give rise to the same algebra $\Cal B$ of Borel sets.   If $X$ is a
Radon space in its norm topology, then any totally finite measure with
domain $\Cal B$ is inner regular with respect to the norm-compact sets,
therefore inner regular with respect to the weakly compact sets, and $X$
is Radon in its weak topology.   If $X$ is a Radon space in its weak
topology, then any totally finite measure $\mu$ with domain $\Cal B$ has
a completion $\hat\mu$ which is a Radon measure for the weak topology,
therefore also for the norm topology, by 466B;  as $\mu$ is arbitrary,
$X$ is a Radon space for the norm topology.

\medskip

{\bf (b)(i)$\Leftrightarrow$(iii)} is a special case of 438H.
}%end of proof of 466F

\leader{466G}{Definition} A partially ordered set $X$ has the
{\bf $\sigma$-interpolation property}\cmmnt{ or
{\bf countable separation
property}} if whenever $A$, $B$ are non-empty countable subsets of $X$ and
$x\le y$ for every $x\in A$, $y\in B$, then there is a $z\in X$ such
that $x\le z\le y$ for every $x\in A$ and $y\in B$.   \cmmnt{A Dedekind
$\sigma$-complete partially ordered set (314Ab) always has the
$\sigma$-interpolation property.}

\leader{466H}{Proposition}\cmmnt{ ({\smc Jayne \& Rogers 95})} Let $X$
be a Riesz space with a Riesz norm, given its
weak topology $\frak T_s=\frak T_s(X,X^*)$.   Suppose that ($\alpha$)
$X$ has the $\sigma$-interpolation property ($\beta$) there is a
strictly increasing
family $\langle p_{\xi}\rangle_{\xi<\omega_1}$ in $X$.   Then there is a
$\frak T_s$-Borel probability measure $\mu$ on $X$ such that

(i) $\mu$ is not inner regular with respect to the $\frak T_s$-closed
sets;

(ii) $\mu$ is not $\tau$-additive for the topology $\frak T_s$;

(iii) $\mu$ has no extension to a norm-Borel measure on $X$.

\noindent Accordingly $(X,\frak T_s)$ is not a Radon space (indeed, is
not Borel-measure-complete).

\proof{{\bf (a)} Let $K$ be the set

\Centerline{$\{f:f\in X^*,\,f\ge 0,\,\|f\|\le
1\}=\{f:\|f\|\le 1\}\cap\bigcap_{x\in X^+}\{f:f(x)\ge 0\}$,}

\noindent so that $K$ is a weak*-closed subset of the unit ball of
$X^*$ and is weak*-compact.   Because $X^*$ is a solid linear subspace
of the order-bounded dual $X^{\sim}$ of $X$ (356Da), every member of
$X^*$ is the difference of two non-negative members of $X^*$, and $K$
spans $X^*$.   We shall need to know that if $x<y$ in $X$, there is an
$f\in K$ such that $f(x)<f(y)$;  set $f=\Bover1{\|g\|}|g|$ where
$g\in X^*$ is such that $g(x)\ne g(y)$.   (Recall that the norm of $X^*$
is a Riesz norm, as also noted in 356Da.)

Set

\Centerline{$A=\bigcup_{\xi<\omega_1}\{x:x\in X,\,x\le p_{\xi}\}$;}

\noindent then every sequence in $A$ has an upper bound in $A$, but $A$
has no greatest member.   It follows that if $B\subseteq K$ is countable
there is an $x\in A$ such that $f(x)=\sup_{y\in A}f(y)$ for every
$f\in B$, so that $f(x)=f(y)$ whenever $x\le y\in A$ and $f\in B$.

Let $\Cal I$ be the family of those sets $E\subseteq X$ such
that $A\cap E$ is bounded above in $A$.   Then $\Cal I$ is a $\sigma$-ideal
of subsets of $X$.
\Prf\ Of course $\emptyset\in\Cal I$.   If $\sequencen{E_n}$ is a sequence
in $\Cal I$ and $E\subseteq\bigcup_{n\in\Bbb N}E_n$,
there is for each $n\in\Bbb N$ an $x_n\in A$ which is an upper bound for
$E_n\cap A$.   Let $x\in A$ be an upper bound for $\{x_n:n\in\Bbb N\}$;
then $x$ is an upper bound for
$E\cap A$ in $A$.   So $E\in\Cal I$.\ \Qed

\medskip

{\bf (b)} For $G\in\frak T_s$ and $k\in\Bbb N$, let $W(G,k)$ be the set of
those $x\in X$ for which there are $f_0,\ldots,f_k\in K$
such that $\{y:y\in X$, $|f_i(y)-f_i(x)|\le 2^{-k}$ for every $i\le k\}$
is included in $G$.   Because $K$ spans $X^*$,
$G=\bigcup_{k\in\Bbb N}W(G,k)$.   So if $G\in\frak T_s\setminus\Cal I$
there is a $k\in\Bbb N$ such that $W(G,k)\notin\Cal I$.

\medskip

{\bf (c)} (The key.)  If $\sequencen{G_n}$ is any sequence in
$\frak T_s\setminus\Cal I$, then
$\bigcap_{n\in\Bbb N}G_n\notin\Cal I$.   \Prf\ Start from any $z^*\in A$.
For each $n\in\Bbb N$, take $k_n\in\Bbb N$ such that
$W(G_n,k_n)\notin\Cal I$.   For each $z\in A$ and $n\in\Bbb N$, choose
$w_{zn}\in A\cap W(G_n,k_n)$ such that
$w_{zn}\ge z$;  now choose a family
$\langle f_{nzi}\rangle_{z\in A,i\le k_n}$ in $K$ such that

\Centerline{$\{y:|f_{nzi}(y-w_{zn})|\le 2^{-k_n}$ for every $i\le k_n\}
\subseteq G_n$.}

\noindent Let $\Cal F$ be any ultrafilter on $A$ containing
$\{x:x\in A$, $x\ge z\}$ for every $z\in A$, and write
$f_{ni}=\lim_{z\to\Cal F}f_{nzi}$ for $n\in\Bbb N$ and $i\le k_n$, the
limit being taken for the weak* topology on $K$.   Let $z_1^*>z^*$ be
such that $z_1^*\in A$ and $f_{ni}(x-z_1^*)=0$ whenever $x\in A$,
$x\ge z_1^*$, $n\in\Bbb N$ and $i\le k_n$.

Choose sequences $\sequencen{x_n}$, $\sequencen{y_n}$ and
$\sequencen{z_n}$ in $A$ inductively, as follows.   Set $y_0=z_1^*$.
Given that $y_n\ge z_1^*$, the set

\Centerline{$C_n=\{z:z\in A,\,z\ge y_n,\,
|(f_{nzi}-f_{ni})(y_n-z_1^*)|\le 2^{-k_n}$ for every $i\le k_n\}$}

\noindent belongs to $\Cal F$, so is not empty;  take $z_n\in C_n$.
Because $y_n\ge z_1^*$, we have $f_{ni}(y_n-z_1^*)=0$ and
$|f_{nz_ni}(y_n-z_1^*)|\le 2^{-k_n}$, for every $i\le k_n$.
Set $x_n=w_{z_nn}$, so that $x_n\in A$, $x_n\ge z_n$ and
$\{y:|f_{nz_ni}(y-x_n)|\le 2^{-k_n}$ for every $i\le k_n\}$ is included in
$G_n$.   Now let $y_{n+1}\in A$ be such that
$y_{n+1}\ge x_n$ and $f_{nz_ni}(y-y_{n+1})=0$ whenever $y\in A$ and
$y\ge y_{n+1}$.   Of course

\Centerline{$y_{n+1}\ge x_n\ge z_n\ge y_n\ge z_1^*$.}

\noindent Continue.

At the end of the induction, let $z_2^*$ be an upper bound for
$\{y_n:n\in\Bbb N\}$ in $A$.   For $n\in\Bbb N$, set

\Centerline{$u_n=z_1^*+\sum_{j=0}^nx_j-y_j$,
\quad$v_n=u_n+z_2^*-y_{n+1}$.}

\noindent Then $\sequencen{u_n}$ is non-decreasing and
$u_n\le v_n\le z_2^*$ for every $n\in\Bbb N$;  moreover,

\Centerline{$v_n-v_{n+1}=y_{n+1}-x_{n+1}-y_{n+1}+y_{n+2}\ge 0$}

\noindent for every $n$, so $\sequencen{v_n}$ is non-increasing, and
$u_m\le v_n$ for all $m$, $n\in\Bbb N$.   Because $X$ has the
$\sigma$-interpolation property, there is an $x\in X$ such that
$u_n\le x\le v_n$ for every $n\in\Bbb N$.   Since
$z_1^*\le x\le z_2^*$, $x\in A$ and $x>z^*$.

Take any $n\in\Bbb N$.   Then

\Centerline{$0\le f_{nz_ni}(x-u_n)\le f_{nz_ni}(v_n-u_n)
=f_{nz_ni}(z_2^*-y_{n+1})=0$}

\noindent for every $i\le k_n$.   On the other hand,

\Centerline{$x_n-u_n=(y_n-z_1^*)-\sum_{j=0}^{n-1}(x_j-y_j)$}

\noindent lies between $0$ and $y_n-z_1^*$, so

\Centerline{$0\le f_{nz_ni}(x_n-u_n)\le 2^{-k_n}$}

\noindent for every $i\le k_n$, and $|f_{nz_ni}(x-x_n)|\le 2^{-k_n}$ for
every $i$.   Thus $x\in G_n$.   As $n$ is arbitrary,
$x\in\bigcap_{n\in\Bbb N}G_n$.

As $x>z^*$, this shows that $z^*$ is not an upper bound of
$\bigcap_{n\in\Bbb N}G_n\cap A$.   As $z^*$ is arbitrary,
$\bigcap_{n\in\Bbb N}G_n\notin\Cal I$.\ \Qed

\medskip

{\bf (d)} Set

\Centerline{$\Cal K_0=\{G\setminus H:G,\,H\in\frak T_s$, $G\notin\Cal I$,
$H\in\Cal I\}$.}

\noindent Then $\bigcap_{n\in\Bbb N}E_n\ne\emptyset$ for any sequence
$\sequencen{E_n}$ in $\Cal K_0$.   \Prf\ Express each $E_n$ as
$G_n\setminus H_n$ where $G_n$, $H_n$ are $\frak T_s$-open,
$G_n\notin\Cal I$ and $H_n\in\Cal I$.   Then
$\bigcup_{n\in\Bbb N}H_n\in\Cal I$,
as noted in (a), while $\bigcap_{n\in\Bbb N}G_n\notin\Cal I$, by (c);  so
$\bigcap_{n\in\Bbb N}E_n
=\bigcap_{n\in\Bbb N}G_n\setminus\bigcup_{n\in\Bbb N}H_n$ is
non-empty.\ \Qed

It follows that $\Cal K=\Cal K_0\cup\{\emptyset\}$ is a countably
compact class in the sense of 413L.   Moreover, $E\cap E'\in\Cal K$ for
all $E$, $E'\in\Cal K$ (using (a) and (c) again), so if we define
$\phi_0:\Cal K\to\{0,1\}$ by
writing $\phi_0(E)=1$ for $E\in\Cal K_0$ and $\phi_0(\emptyset)=0$, then
$\Cal K$ and $\phi_0$ will satisfy all the conditions of 413M.   There
is therefore a measure $\hat\mu$ on $X$ extending $\phi_0$ and inner
regular with respect to $\Cal K_{\delta}$, the family of sets
expressible as intersections of sequences in $\Cal K$.   The domain of
$\hat\mu$ must include every member of $\Cal K$;  but if $G\in\frak T_s$
then either $G$ or $X\setminus G$ belongs to $\Cal K_0$, so is
measured by $\hat\mu$, and $\hat\mu$ is a topological measure.

We need to observe that, because $\phi_0$ takes only the values $0$ and
$1$, $\hat\mu E\le 1$ for every $E\in\Cal K_{\delta}$, and $\hat\mu X\le
1$;  since $\phi_0X=1$, $\hat\mu X=1$ and $\hat\mu$ is a probability
measure.

\medskip

{\bf (e)} We may therefore take $\mu$ to be the restriction of $\hat\mu$
to the algebra $\Cal B$ of $\frak T_s$-Borel sets,
and $\mu$ is a $\frak T_s$-Borel probability
measure.   Now $\mu$ is not inner regular with respect to the
$\frak T_s$-closed
sets.   \Prf\ For each $\xi<\omega_1$, $p_{\xi}<p_{\xi+1}<p_{\xi+2}$, so
there are $g_{\xi}$, $h_{\xi}\in K$ such that
$g_{\xi}(p_{\xi})<g_{\xi}(p_{\xi+1})$ and
$h_{\xi}(p_{\xi+1})<h_{\xi}(p_{\xi+2})$.   Let $D\subseteq\omega_1$ be
any set such that $D$ and $\omega_1\setminus D$ are both uncountable,
and set

\Centerline{$G=\bigcup_{\xi\in D}\{x:g_{\xi}(p_{\xi})<g_{\xi}(x)$,
$h_{\xi}(x)<h_{\xi}(p_{\xi+2})\}$.}

\noindent Then $G\in\frak T_s$, and
$p_{\xi+1}\in G$ for every $\xi\in D$,
so $G\notin\Cal I$ and $\mu G=\phi_0G=1$.   On the other hand, if
$\eta\in\omega_1\setminus D$, then for every $\xi\in D$ either
$\xi<\eta$ and $h_{\xi}(p_{\xi+2})\le h_{\xi}(p_{\eta+1})$, or
$\xi>\eta$ and $g_{\xi}(p_{\eta+1})\le g_{\xi}(p_{\xi})$;  thus
$p_{\eta+1}\notin G$ for any $\eta\in\omega_1\setminus D$, and
$X\setminus G\notin\Cal I$.   But this means that if $F\subseteq G$ is
closed then $X\setminus F\in\Cal K_0$ and $\mu F=0$.   Thus
$\mu G>\sup_{F\subseteq G\text{ is closed}}\mu F$.\ \Qed

\medskip

{\bf (f)} Because $\frak T_s$ is regular, $\mu$ cannot
be $\tau$-additive, by 414Mb.   It follows at once that $(X,\frak T_s)$
is not Borel-measure-complete, and in particular is not a Radon space.
To see that $\mu$ has no extension to a
norm-Borel measure, we need to look again at the set $A$.
For each $\xi<\omega_1$, set $F_{\xi}=\{x:x\le p_{\xi}\}$.   Then every
$F_{\xi}$ is norm-closed (354Bc) and $\ofamily{\xi}{\omega_1}{F_{\xi}}$
is an increasing family with union $A$.   Consequently, $A$ is
norm-closed (4A2Ld, 4A2Ka).   At the same time, every $F_{\xi}$ is
convex (cf.\ 351Ce), so $A$ is also convex.   It follows that $A$, like
every $F_{\xi}$, is
$\frak T_s$-closed (3A5Ee).   So $\mu$ measures $A$ and every $F_{\xi}$.
Because $X\setminus F_{\xi}$ is a $\frak T_s$-open set not belonging to
$\Cal I$, $\mu F_{\xi}=0$, for every $\xi<\omega_1$;  because
$X\setminus A$ is a $\frak T_s$-open set belonging to $\Cal I$,
$\mu A=1$.

But $\omega_1$ is a measure-free cardinal (438Cd), so 438I tells us
that $\lambda A=\sup_{\xi<\omega_1}\lambda F_{\xi}$ for any
semi-finite norm-Borel measure $\lambda$ on $X$.   Thus $\mu$ has no
extension to a norm-Borel measure, and the proof is complete.
}%end of proof of 466H

\leader{466I}{Examples} The following spaces satisfy the hypotheses of
466H.

\spheader 466Ia \cmmnt{ ({\smc Talagrand 78a}, or {\smc Talagrand 84},
16-1-2)} $X=\ell^{\infty}(I)$ or
$\{x:x\in\ell^{\infty}(I),\,\{i:x(i)\ne 0\}$ is countable$\}$, where $I$
is uncountable.   \prooflet{\Prf\ $X$
has the $\sigma$-interpolation property because it is Dedekind complete,
and if $\ofamily{\xi}{\omega_1}{i_{\xi}}$ is any family of distinct
elements of $I$, we can set $p_{\xi}(i_{\eta})=1$ for $\eta\le\xi$,
$p_{\xi}(i)=0$ for all other $i\in I$ to obtain a strictly increasing
family $\ofamily{\xi}{\omega_1}{p_{\xi}}$ in $X$.\ \Qed}

\spheader 466Ib \cmmnt{ ({\smc de Maria \& Rodriguez-Salinas 91})}
$X=\ell^{\infty}/\pmb{c}_0$\cmmnt{, where $\pmb{c}_0$ is the space of
real sequences converging to $0$}.

\prooflet{\Prf\ {\bf (i)} To see that $X$ has the
$\sigma$-interpolation property, let $A$, $B\subseteq X$ be non-empty
countable sets such that $u\le v$ for all $u\in A$, $v\in B$.   Let
$\sequencen{x_n}$, $\sequencen{y_n}$ be sequences in $\ell^{\infty}$
such that $A=\{x_n^{\ssbullet}:n\in\Bbb N\}$ and
$B=\{y_n^{\ssbullet}:n\in\Bbb N\}$.   Set $\tilde x_n=\sup_{i\le n}x_i$,
$\tilde y_n=\inf_{i\le n}y_i$ for $n\in\Bbb N$;  then
$\tilde x_n^{\ssbullet}\le\tilde y_n^{\ssbullet}$, so
$(\tilde x_n-\tilde y_n)^+\in\pmb{c}_0$.   Set

\Centerline{$k_n
=\max(\{n\}\cup\{i:\tilde x_n(i)\ge\tilde y_n(i)+2^{-n}\})$}

\noindent for $n\in\Bbb N$, and define $x\in\ell^{\infty}$ by writing

$$\eqalign{x(i)&=0\text{ if }i\le k_0,\cr
&=\tilde x_n(i)\text{ if }k_n<i\le k_{n+1}.\cr}$$

\noindent Then it is easy to check that $u\le x^{\ssbullet}\le v$ for
every $u\in A$, $v\in B$;  as $A$ and $B$ are arbitrary, $X$ has the
$\sigma$-interpolation property.

\medskip

\quad{\bf (ii)} To see that $X$ has the other property, recall that
there is a family $\ofamily{\xi}{\omega_1}{I_{\xi}}$ of infinite subsets
of $\Bbb N$ such that $I_{\xi}\setminus I_{\eta}$ is finite if
$\eta\le\xi$, infinite if $\xi<\eta$ (4A1Fa).   Setting
$p_{\xi}=\chi(\Bbb N\setminus I_{\xi})^{\ssbullet}$, we have a strictly
increasing family $\ofamily{\xi}{\omega_1}{p_{\xi}}$ in $X$.\ \Qed
}%end of prooflet

\leader{466J}{Theorem} Let $X$ be a linear topological space and
$\Sigma$ its cylindrical $\sigma$-algebra.   If $\mu$ and $\nu$ are
probability measures with domain $\Sigma$ such that
$\int e^{if(x)}\mu(dx)=\int e^{if(x)}\nu(dx)$ for every
$f\in X^*$, then $\mu=\nu$.

\proof{ Define $T:X\to\BbbR^{X^*}$ by setting $(Tx)(f)=f(x)$ for
$f\in X^*$, $x\in X$.   Then $T$ is linear and continuous for the weak
topology of $X$.   So if $F\subseteq\BbbR^{X^*}$ is a Baire set for the
product topology of $\BbbR^{X^*}$, $T^{-1}[F]$ is a Baire set for the
weak topology of $X$ (4A3Kc) and belongs to $\Sigma$ (4A3U).   We
therefore have Baire measures $\mu'$, $\nuprime$ on $\BbbR^{X^*}$ defined by
saying that $\mu'F=\mu T^{-1}[F]$ and $\nuprime F=\nu T^{-1}[F]$ for every
Baire set $F\subseteq\BbbR^{X^*}$.

If $h:\BbbR^{X^*}\to\Bbb R$ is a continuous linear functional, it can be
expressed in the form $h(z)=\sum_{i=0}^n\alpha_iz(f_i)$ where
$f_0,\ldots,f_n\in X^*$ and $\alpha_0,\ldots,\alpha_n\in\Bbb R$.   So

\Centerline{$h(Tx)=\sum_{i=0}^n\alpha_i(Tx)(f_i)
=\sum_{i=0}^n\alpha_if_i(x)=f(x)$}

\noindent for every $x\in X$, where $f=\sum_{i=0}^n\alpha_if_i$.   This
means that

\Centerline{$\int e^{ih(z)}\mu'(dz)
=\int e^{ih(Tx)}\mu(dx)
=\int e^{if(x)}\mu(dx)
=\int e^{if(x)}\nu(dx)
=\int e^{ih(z)}\nuprime(dz)$.}

\noindent As $h$ is arbitrary, $\mu'=\nuprime$ (454Pa).

Now let $\Sigma'$ be the family of subsets of $X$ of the form
$T^{-1}[F]$ where $F\subseteq\BbbR^{X^*}$ is a Baire set.   This is a
$\sigma$-algebra and contains all sets of the form $\{x:f(x)\ge\alpha\}$
where $f\in X^*$ and $\alpha\in\Bbb R$.   So every member of $X^*$ is
$\Sigma'$-measurable and $\Sigma'$ must include the cylindrical
$\sigma$-algebra of $X$.   Since $\mu$ and $\nu$ agree on $\Sigma'$ they
must be identical.
}%end of proof of 466J

\leader{466K}{Proposition} If $X$ is a locally convex linear topological
space and $\mu$, $\nu$ are quasi-Radon probability measures on $X$
such that $\int e^{if(x)}\mu(dx)=\int e^{if(x)}\nu(dx)$ for every
$f\in X^*$, then $\mu=\nu$.

\proof{ Write $\frak T$ for the given topology on $X$ and
$\frak T_s=\frak T_s(X,X^*)$ for the weak topology.   By 466J, $\mu$
and $\nu$ must agree on the cylindrical
$\sigma$-algebra $\Sigma$ of $X$.   Since $\Sigma$ includes a base for
$\frak T_s$, every weakly open set $G$ is the union of an
upwards-directed family of open sets belonging to $\Sigma$;  as $\mu$
and $\nu$ are $\tau$-additive, $\mu G=\nu G$.   Consequently $\mu$ and
$\nu$ agree on $\frak T_s$-closed sets, and therefore on
$\frak T$-closed convex sets (4A4Ed).   Write $\Cal H$ for the family of
$\frak T$-open sets which are expressible as the union of a
non-decreasing sequence of $\frak T$-closed convex sets;  then $\mu$ and
$\nu$ agree on $\Cal H$.   If $\tau$ is a $\frak T$-continuous seminorm
on $X$, $x_0\in X$ and $\alpha>0$, then
$\{x:\tau(x-x_0)<\alpha\}\in\Cal H$;  and sets of this kind constitute a
base for $\frak T$ (4A4Cb).   Also the intersection of two members of
$\Cal H$ belongs to $\Cal H$.   By 415H(v), $\mu=\nu$.
}%end of proof of 466K

\cmmnt{\medskip

\noindent{\bf Remark} This generalizes 285M and 454Xk, which are the
special cases $X=\BbbR^r$ (for finite $r$) and $X=\BbbR^I$;  see also
445Xq.
}%end of comment

\leader{466L}{Proposition} Suppose that $X$ and $Y$ are Banach spaces
and that $T:X\to Y$ is a linear operator
such that $gT:X\to\Bbb R$ is universally Radon-measurable\cmmnt{, in
the sense of 434Ec,} for every $g\in Y^*$.   Then $T$ is continuous.

\proof{\Quer\ Suppose, if possible, otherwise.   Then there is a
$g\in Y^*$ such that $gT$ is
not continuous (4A4Ib).   For each $n\in\Bbb N$, take $x_n\in X$ such
that $\|x_n\|=2^{-n}$ and $g(Tx_n)>2n$.
Define $h:\{0,1\}^{\Bbb N}\to X$ by setting
$h(t)=\sum_{n=0}^{\infty}t(n)x_n$ (4A4Ie).   Then $h$ is continuous,
because $\|h(t)-h(t')\|\le\sum_{n=0}^{\infty}2^{-n}|t(n)-t'(n)|$ for all
$t$, $t'\in\{0,1\}^{\Bbb N}$.   Let $\nu$ be the usual measure on
$\{0,1\}^{\Bbb N}$;  then the image measure $\mu=\nu h^{-1}$ is a Radon
measure on $X$ (418I), so $gT$ must be
$\dom\mu$-measurable, and $\phi=gTh$ is $\dom\nu$-measurable.   In this
case there is an $m\in\Bbb N$ such that
$E=\{t:|\phi(t)|\le m\}$ has measure greater than $\bover12$.   But as
$g(Tx_m)>2m$, we see that if $t\in E$ then $t'\notin E$, where $t'$
differs from $t$ at the $m$th coordinate only, so that
$|\phi(t)-\phi(t')|=g(Tx_m)$.   Since the map $t\mapsto t'$ is an
automorphism of the measure space $(\{0,1\}^{\Bbb N},\nu)$,
$\nu E\le\bover12$, which is impossible.\ \Bang
}%end of proof of 466L

\leader{466M}{Corollary} If $X$ is a Banach space, $Y$ is a separable
Banach space, and $T:X\to Y$ is a
linear operator such that the graph of $T$ is a Souslin-F set in
$X\times Y$, then $T$ is continuous.

\proof{ It will be enough to show that $T\restr Z$ is continuous for every
separable closed linear subspace $Z$ of $X$ (because then it must be
sequentially continuous, and we can use 4A2Ld).   Write
$\Gamma\subseteq X\times Y$ for the graph of $T$.   If $H\subseteq Y$
is open, then $\Gamma\cap(Z\times H)$ is a Souslin-F set in the
Polish space $Z\times Y$, so is analytic (423Eb), and its projection
$(T\restr Z)^{-1}[H]$ also is analytic (423Ba), therefore
universally measurable (434Dc).   Thus $T\restr Z:Z\to Y$ is a
universally measurable function, and $gT\restr Z$ must be universally
measurable for any $g\in Y^*$ (434Df).   By 466L, $T\restr Z$ is
continuous;  as $Z$ is arbitrary, $T$ is continuous.
}%end of proof of 466M.

%even if $T$ has closed graph, we do need both $X$ and $Y$ to be
%complete

\leader{466N}{Gaussian \dvrocolon{measures}}\dvAnew{2010}\cmmnt{ Some of
the ideas of \S456 can be adapted to the present context, as follows.

\medskip

\noindent}{\bf Definition} If $X$ is a linear topological space,
\cmmnt{I will say that}
a probability measure $\mu$ on $X$ is a {\bf centered Gaussian measure} if
its domain includes the cylindrical $\sigma$-algebra of $X$ and
every continuous linear functional on $X$
is either zero almost everywhere or a normal random
variable with zero expectation.   \cmmnt{(Thus
a `centered Gaussian distribution' on $\BbbR^I$, as defined in 456A,
is a distribution in the sense of 454K which is a centered Gaussian
measure when $\BbbR^I$ is thought of as a linear topological space.)}

\cmmnt{Warning!  many authors reserve the phrase `Gaussian measure' for
strictly positive measures.}

\leader{466O}{Proposition}\dvAnew{2010} Let $X$ be a separable Banach
space, and $\mu$ a
probability measure on $X$.   Suppose that there is a linear subspace $W$
of $X^*$, separating the points of $X$, such that every element of $W$ is
$\dom\mu$-measurable and either zero a.e.\ or a normal random
variable with zero expectation.
Then $\mu$ is a centered Gaussian measure with respect to
the norm topology of $X$.

\proof{{\bf (a)} As $W$ separates the points of $X$,
$X\setminus\{0\}=\bigcup_{f\in W}\{x:f(x)\ne 0\}$.
Because $X$ is Polish, therefore hereditarily Lindel\"of, there is a
countable set $I\subseteq W$ still separating the points of $X$.
Let $W_0$ be the linear subspace of $X^*$ generated by $I$.

Define $T:X\to\BbbR^I$ by setting $(Tx)(f)=f(x)$ for
$f\in I$.   Then $T$ is an injective linear operator, and is continuous for
$\frak T_s(X,W_0)$ and the usual topology of $\BbbR^I$.
Let $\lambda$ be the distribution of the family $\family{f}{I}{f}$;
$T$ is \imp\ for $\hat\mu$ and $\lambda$, where $\hat\mu$ is
the completion  of $\mu$ (454J(iv)).
If $g:\BbbR^I\to\Bbb R$ is a continuous linear functional, then
$gT\in W_0$ (use 4A4Be);  now
the distribution of $g$, with respect to the probability measure
$\lambda$, is
just the distribution of $gT$ with respect to $\hat\mu$ and $\mu$,
and is therefore
either normal or the Dirac measure concentrated at $0$.   So $\lambda$ is a
centered Gaussian distribution in the sense of 456Ab.   Because $I$ is
countable, $\lambda$ is a Radon measure (454J(iii)).

\medskip

{\bf (b)} If $\epsilon>0$, there is a norm-compact $K\subseteq X$ such that
$\hat\mu K$ is defined and is at least $1-\epsilon$.
\Prf\ As $X$ and $\BbbR^I$ are analytic (423B), there is a Radon
measure $\mu'$ on $X$ such that $\lambda=\mu'T^{-1}$ (432G).
Of course $\mu'X=\lambda\BbbR^I=1$.
There is a compact set $K\subseteq X$ such that
$\mu'K\ge 1-\epsilon$;  now $T[K]$ is compact and $K=T^{-1}[T[K]]$, so

\Centerline{$\hat\mu K=\mu T^{-1}[T[K]]=\lambda T[K]=\mu'K\ge 1-\epsilon$.
\Qed}

\medskip

{\bf (c)} Now suppose that $g\in X^*$.   Then there is a sequence
$\sequencen{g_n}$ in $W_0$ such that $\sequencen{g_n}\to g\,\,\mu$-a.e.
\Prf\ For each $n\in\Bbb N$, there is a compact set $K_n\subseteq X$ such
that $\hat\mu K_n\ge 1-2^{-n-1}$;  we can suppose that
$K_{n+1}\supseteq K_n$ for each $n$.
$W_0$ is dense in $X^*$ for the weak*-topology $\frak T_s(X^*,X)$ (4A4Eh);
being convex, it is dense for the Mackey topology $\frak T_k(X^*,X)$
(4A4F), and there is a $g_n\in W_0$ such that
$\sup_{x\in K_n}|g_n(x)-g(x)|\le 2^{-n}$.   Now
$g(x)=\lim_{n\to\infty}g_n(x)$ for every $x$ in the $\mu$-conegligible set
$\bigcup_{n\in\Bbb N}K_n$.\ \Qed

Set $\sigma_n=\sqrt{\Var(g_n)}$ for each $n$.   Then
$\{\sigma_n:n\in\Bbb N\}$ is bounded.   \Prf\
Set $M=\sup_{x\in K_0}|g(x)|$.   If $n\in\Bbb N$ and $\sigma_n\ne 0$,
$|g_n(x)|\le M+1$ for every $x\in K_0$, and

$$\eqalign{\Bover12
&\le\mu K_0
\le\Pr(|g_n|\le M+1)
=\Bover1{\sigma_n\sqrt{2\pi}}\int_{-M-1}^{M+1}e^{-t^2/2\sigma_n^2}dt
\le\Bover{2(M+1)}{\sigma_n\sqrt{2\pi}},\cr}$$

\noindent so $\sigma_n\le\Bover{4(M+1)}{\sqrt{2\pi}}$.\ \Qed

We therefore have a strictly increasing
sequence $\sequence{k}{n_k}$ such that
$\sigma=\lim_{k\to\infty}\sigma_{n_k}$
is defined in $\coint{0,\infty}$.   For each $k$, let $\nu_k$ be the
distribution of $g_{n_k}$ and $\phi_k$ its characteristic function;
let $\nu$, $\phi$ be the distribution and characteristic function of
$g$.   Since $\sequence{k}{g_{n_k}}\to g$ a.e.,

\Centerline{$\int h\,d\nu
=\int hg\,d\mu=\lim_{k\to\infty}\int hg_{n_k}d\mu
=\lim_{k\to\infty}\int h\,d\nu_k$}

\noindent for every bounded continuous function $h:\Bbb R\to\Bbb R$,
and $\phi(t)=\lim_{k\to\infty}\phi_k(t)$ for every $t\in\Bbb R$, by
285L.   But, for each $k$,

\Centerline{$\phi_k(t)
=\exp(-\Bover12\sigma_{n_k}^2t^2)$}

\noindent by 285E if $\sigma_{n_k}>0$ and by direct calculation if
$\sigma_{n_k}=0$, as then $g_{n_k}=0$ almost everywhere.

Accordingly $\phi(t)=\exp(-\Bover12\sigma^2t^2)$ for every $t$.   But this
means that $\nu$ is either the Dirac measure concentrated at $0$ (if
$\sigma=0$) or a normal distribution with zero expectation (if $\sigma>0$).

\medskip

{\bf (d)} As $g$ is arbitrary, $\mu$ is a centered Gaussian measure.
}%end of proof of 466O


\exercises{\leader{466X}{Basic exercises (a)}
%\spheader 466Xa
Let $(X,\frak T)$ be a metrizable locally convex linear topological
space and $\mu$ a totally finite measure on $X$ which is quasi-Radon for
the topology $\frak T$.   Show that $\mu$ is quasi-Radon for the weak
topology $\frak T_s(X,X^*)$.
%466A

\sqheader 466Xb Let $\sequencen{e_n}$ be the usual orthonormal basis of
$\ell^2$.   Give $\ell^2$ the Radon probability measure $\nu$ such that
$\nu\{e_n\}=2^{-n-1}$ for every $n$.   Let $I$ be an uncountable set,
and set $X=(\ell^2)^I$ with the product linear structure and the product
topology $\frak T$, each copy of $\ell^2$ being given its norm topology.
(i) Let $\lambda$ be the $\tau$-additive product of copies of
$\nu$ (417G).   Show that $\lambda$ is quasi-Radon for $\frak T$ but is
not inner regular with respect to the $\frak T_s$-closed sets.
(ii) Write $\frak T_s$ for the weak topology of $X$.
Let $\lambda_s$ be the $\tau$-additive product measure of copies of
$\nu$ when each copy of $\ell^2$ is given its weak topology instead of
its norm topology.   Show that $\lambda_s$ is quasi-Radon for $\frak T_s$
but does not measure every $\frak T$-Borel set.   \Hint{setting
$E=\{e_n:n\in\Bbb N\}$,
$\lambda(E^I)=1$ and $E^I$ is relatively $\frak T_s$-compact.}
%466A, 466Xa

\sqheader 466Xc Let $X$ be a metrizable locally convex linear
topological space and $\mu$ a $\tau$-additive totally finite measure on
$X$ with domain the cylindrical $\sigma$-algebra of $X$.   Show that
$\mu$ has an extension to a quasi-Radon measure on $X$.   \Hint{4A3U,
415N.}
%466A

\spheader 466Xd Let $X$ be a metrizable locally convex linear
topological space which is Lindel\"of in its weak topology, and $\Sigma$
the cylindrical $\sigma$-algebra of $X$.   Show that any totally finite
measure with domain $\Sigma$ has an extension to a quasi-Radon measure
on $X$.
%466A

\sqheader 466Xe Let $X$ be a separable Banach space.   Show that it is a
Radon space when given its weak topology.
%466A 466Xd
%is there a ZFC example of weight \omega_1?

\spheader 466Xf Let $K$ be a compact metrizable space, and $C(K)$ the
Banach space of continuous real-valued functions on $K$.   Show that the
$\sigma$-algebra of subsets of $C(K)$ generated by the functionals
$x\mapsto x(t):C(K)\to\Bbb R$, for $t\in K$, is just the cylindrical
$\sigma$-algebra of $C(K)$.   \Hint{4A2Pe.}   Examine the connexions
between this and 454Sa, 462Z and 466Xd.
%466E

\spheader 466Xg Let $K$ be a scattered compact Hausdorff space.   Show that
the weak topology and the topology of pointwise convergence on $C(K)$ have
the same Borel $\sigma$-algebras.
%466E 466Xf

\spheader 466Xh Re-write part (d) of the proof of 466H to avoid any
appeal to results from \S413.
%466H

\spheader 466Xi Let $X$ be a locally convex linear topological space and
$\mu$, $\nu$ two totally finite quasi-Radon measures on $X$.   Show that
if $\mu$ and $\nu$ give the same measure to every half-space
$\{x:f(x)\ge\alpha\}$, where $f\in X^*$ and $\alpha\in\Bbb R$, then
$\mu=\nu$.
%466K

\spheader 466Xj Let $X$ be a Hilbert space and $\mu$, $\nu$ two totally
finite Radon measures on $X$.   Show that if $\mu$ and $\nu$ give the
same measure to every ball $B(x,\delta)$, where $x\in X$ and
$\delta\ge 0$, then $\mu=\nu$.   \Hint{every open half-space is the
union of a non-decreasing sequence of balls.}
%466Xi, 466K

\spheader 466Xk Let $f:\Bbb R\to\Bbb R$ be a function such that $f(1)=1$
and $f(x+y)=f(x)+f(y)$ for all $x$, $y\in\Bbb R$.    Show that the
following are equiveridical:  (i) $f(x)=x$ for every $x\in\Bbb R$;  (ii)
$f$ is continuous at some point;  (iii) $f$ is bounded on some non-empty
open set;  (iv) $f$ is bounded on some Lebesgue measurable set of
non-zero measure;  (v) $f$ is Lebesgue measurable;  (vi) $f$ is Borel
measurable;  *(vii) $f$ is bounded on some non-meager G$_{\delta}$
set;  *(viii) $f$ is $\widehat{\Cal B}$-measurable, where
$\widehat{\Cal B}$ is the Baire-property algebra of $\Bbb R$.
\Hint{443Db.}
%466L

\spheader 466Xl Set $X=\{x:x\in\BbbR^{\Bbb N},\,\{n:x(n)\ne 0\}$ is
finite$\}$, and give $X$ any norm.   Show that any linear operator from
$X$ to any normed space is universally measurable.
%466L

\spheader 466Xm\dvAnew{2010}
Let $X$ be a linear topological space and $\mu$ a
centered Gaussian measure on $X$.
(i) Let $Y$ be another linear topological space and $T:X\to Y$ a
continuous linear operator.
Show that the image measure $\mu T^{-1}$ is a centered Gaussian measure on
$Y$.
(ii) Show that $X^*\subseteq\eusm L^2(\mu)$.
(iii) Let us say that the {\bf covariance matrix} of
$\mu$ is the family $\langle\sigma_{fg}\rangle_{f,g\in X^*}$, where
$\sigma_{fg}=\int f\times g\,d\mu$ for $f$, $g\in X^*$.   Suppose that
$\nu$ is
another centered Gaussian measure on $X$ with the same covariance matrix.
Show that $\mu$ and $\nu$ agree on the cylindrical $\sigma$-algebra of $X$.
%466N

\spheader 466Xn\dvAnew{2010}
Let $\familyiI{X_i}$ be a family of linear topological
spaces with product $X$.   Suppose that for each $i$ we have a centered
Gaussian measure $\mu_i$ on $X_i$.   Show that the product probability
measure $\prod_{i\in I}\mu_i$ is a centered Gaussian measure on $X$.
%466Xm 466N

\spheader 466Xo\dvAnew{2010}
Let $X$ be a linear topological space.   Show that the convolution of
two quasi-Radon centered Gaussian measures on $X$ is a centered Gaussian
measure.
%466Xn 466N

\spheader 466Xp\dvAnew{2010}
Let $X$ be a separable Banach space, and $\mu$ a
complete measure on $X$.   Show that the following are
equiveridical:  (i) $\mu$ is a centered Gaussian measure on $X$;
(ii) $\mu$ extends a centered Gaussian Radon measure on $X$;
(iii) there are a set $I$, an injective continuous linear operator
$T:X\to\BbbR^I$ and a centered Gaussian distribution $\lambda$ on
$\BbbR^I$ such that $T$ is \imp\ for $\mu$ and $\lambda$;
(iv) whenever $I$ is a set and $T:X\to\BbbR^I$ is a continuous linear
operator there is a centered Gaussian distribution $\lambda$ on
$\BbbR^I$ such that $T$ is \imp\ for $\mu$ and $\lambda$.
%466O

\spheader 466Xq Let $X$ be a Banach space, and $\mu$ a Radon measure on
$X$.   Show that, with respect to $\mu$, the unit ball of $X^*$ is a
stable set of functions in the sense of \S465.   \Hint{465Xj.}
%466Z out of order query

\sqheader 466Xr Let $I$ be an infinite set.
Show that Talagrand's measure, interpreted as a measure
on $\ell^{\infty}(I)$ (464R), is not $\tau$-additive for the weak
topology.
%466Z out of order query

\leader{466Y}{Further exercises (a)}
%\spheader 466Ya
Give an example of a Hausdorff locally convex linear topological space
$(X,\frak T)$ with a probability measure $\mu$ on $X$ which is a Radon
measure for the weak topology $\frak T_s(X,X^*)$ but not for the
topology $\frak T$.   \Hint{take $C=C([0,1])$ and $X=C^*$ with the
Mackey topology for the dual pair $(X,C)$, that is, the topology of
uniform convergence on weakly compact subsets of $C$.}
%466A

\spheader 466Yb\dvAnew{2011}
Let $X$ be a normed space and $\frak T$ a linear space topology on $X$
such that the unit ball of $X$ is $\frak T$-closed
and the topology on the unit sphere $S$ induced by $\frak T$ is finer
than the norm topology on $S$.
(i) Show that every norm-Borel subset of $X$
is $\frak T$-Borel.   (ii) Show that if $\frak T$ is coarser than the norm
topology, then it has a $\sigma$-isolated network.
%466E 466D

\spheader 466Yc\dvAnew{2011}
(i) Let $X$ be a Banach space.   Set $S=\bigcup_{n\in\Bbb N}\{0,1\}^n$ and
suppose that $\family{\sigma}{S}{K_{\sigma}}$ is a family of
non-empty weakly compact
convex subsets of $X$ such that $K_{\sigma}\subseteq K_{\tau}$ whenever
$\sigma$, $\tau\in S$ and $\sigma$ extends $\tau$.   ($\alpha$) Show that
there is a weakly Radon
probability measure on $X$ giving measure at least $2^{-n}$ to $K_{\sigma}$
whenever $n\in\Bbb N$ and $\sigma\in\{0,1\}^n$.   ($\beta$) Show that
there are a $\sigma\in S$ and $x\in K_{\sigma^{\smallfrown}\fraction{0}}$,
$y\in K_{\sigma^{\smallfrown}\fraction{1}}$ such that $\|x-y\|\le 1$.
(ii) Let $X$ be a locally convex Hausdorff linear topological space.
and $\family{\sigma}{S}{A_{\sigma}}$ a family of
non-empty relatively weakly
compact subsets of $X$ such that $A_{\sigma}\subseteq A_{\tau}$ whenever
$\sigma$, $\tau\in S$ and $\sigma$ extends $\tau$.   For $\sigma\in S$, set
$C_{\sigma}=A_{\sigma^{\smallfrown}\fraction{1}}
  -A_{\sigma^{\smallfrown}\fraction{0}}$.   Show that
$0\in\overline{\bigcup_{\sigma\in S}C_{\sigma}}$.
%466A 432Yb

\spheader 466Yd Find Banach spaces $X$ and $Y$ and a linear operator
from $X$ to $Y$ which is not continuous but whose graph is an
F$_{\sigma}$ set in $X\times Y$.
%466M  %mt46bits

\spheader 466Ye\dvAnew{2010}
Let $X$ be a complete Hausdorff locally convex linear topological
space and $\mu$ a Radon probability measure on $X$.   Suppose that there is
a linear subspace $W$ of $X^*$, separating the points of $X$, such that
every member of $W$ is either zero a.e.\ or a normal random variable with
zero expectation.   Show that $\mu$ is a centered Gaussian measure.
%466O mt46bits


\discrversionA{\inset{\spheader 466Y?\dvAnew{2009}
Let $I^{\|}$ be the split interval with its usual compact
Hausdorff topology.   Show that the weak topology and the topology of
pointwise convergence on $C(I^{\|})$ have the same Borel $\sigma$-algebras.
({\smc Mohammed \& Saadi p09}\#.)
%466Xg 466E Mohammed \& Saadi p09#
\leaveitout{there is a countable family $\Cal V$ of $\frak T_p$-closed
sets such
that $\{U\cap V:U$ is $\frak T_p$-open, $V\in\Cal V\}$ is a
$\frak T_s$-network.
}}}{}
}%end of exercises

\allowmorestretch{468}{
\leader{466Z}{Problems (a)} Does every probability measure defined on
the $\frak T_s(\ell^{\infty},(\ell^{\infty})^*)$-Borel sets of
$\ell^{\infty}\cmmnt{\mskip5mu =\ell^{\infty}(\Bbb N)}$ extend to a
measure defined on the $\|\,\|_{\infty}$-Borel sets?
}%end of allowmorestretch

\cmmnt{It is by no means obvious that the Borel sets of $\ell^{\infty}$ are
different for the weak and norm topologies;  for a proof see
{\smc Talagrand 78b}.}

\spheader 466Zb Assume that $\frak c$ is measure-free.   Does it
follow that $\ell^{\infty}$, with its weak topology, is a Radon space?

\cmmnt{Note that a positive answer to 464Z (with $I=\Bbb N$) would
settle this, since Talagrand's measure, when interpreted as a measure
on $\ell^{\infty}$, cannot agree on the weakly Borel sets with any
Radon measure on $\ell^{\infty}$ (466Xq, 466Xr).}

\endnotes{
\Notesheader{466}
I have given a proof of 466A using the machinery of \S463;  when the
measure $\mu$ is known to be a Radon measure for the weak topology,
rather than just a
quasi-Radon measure or a $\tau$-additive measure on the cylindrical
algebra (466Xc), the theorem is older than this, and for an instructive
alternative approach see {\smc Talagrand 84}, 12-1-4.   Another proof is
in {\smc Jayne \& Rogers 95}.

On any Banach space we have at least three important $\sigma$-algebras:
the norm-Borel $\sigma$-algebra (generated by the norm-open sets), the
weak-Borel
algebra (generated by the weakly open sets) and the cylindrical algebra
(generated by the continuous linear functionals).   (Note
that the Baire
$\sigma$-algebras corresponding to the norm and weak topologies are the
norm-Borel algebra (4A3Kb) and the cylindrical algebra (4A3U).)   If our
Banach space is naturally represented as a subspace of some $\BbbR^I$
(e.g., because it is a space of continuous functions), then we have in
addition the $\sigma$-algebra generated by the functionals
$x\mapsto x(i)$ for $i\in I$, and the Borel algebra for the topology of
pointwise convergence.
We correspondingly have natural questions concerning when these algebras
coincide, as in 466E and 4A3V and 466Xf, and when a measure on one of
the algebras leads to a measure on another, as in 466A-466B and 466Xc.

The question of which Banach spaces are Radon spaces in their norm
topologies is, if not exactly `solved', at least reducible to a
classical problem in set theory by the results in \S438.   It seems much
harder to decide which
non-separable Banach spaces are Radon spaces in their weak topologies.
We have a simple positive result for spaces with Kadec norms (466F), and
after some labour a negative result for a couple of standard examples
(466I), but no effective general criterion is known.
Even the case of $\ell^{\infty}$ seems still to be open in
`ordinary' set theories (466Zb).   $\ell^{\infty}$ is of particular
importance in this context because the dual of any separable Banach
space is isometrically isomorphic to a linear subspace of
$\ell^{\infty}$ (4A4Id).   So a positive answer to either question in
466Z would have very interesting consequences -- and would be
correspondingly surprising.

466L and 466M belong to a large family of results of the general form:
if, between spaces with both topological and algebraic structures, we
have a homomorphism (for the algebraic structures) which is not
continuous, then it is wildly irregular.   I hope to return to some of
these ideas in Volume 5.   For the moment I just give a version of the
classical result that an additive function $f:\Bbb R\to\Bbb R$ which is
Lebesgue measurable must be continuous (466Xk).   The definition of
`universally measurable' function which I gave in \S434 has a number of
paradoxical aspects.   I have already remarked that in some contexts we
might prefer to use the notion of `universally
Radon-measurable' function;  this is also appropriate for 466L.   But
when our space $X$, for any reason, has few Borel
measures, as in 466Xl, there are correspondingly many universally
measurable functions defined on $X$.   Of course 466M can also be
thought of as a generalization of the closed graph theorem;  but note
that, unlike the closed graph theorem, it needs a separable codomain
(466Yd).

The point of 466O is that the most familiar separable Banach spaces are
presented with continuous linear embeddings into $\BbbR^{\Bbb N}$, and of
course any separable Banach space $X$ has such a presentation.   We can now
describe the centered Gaussian Radon measures on $X$ in terms of
centered Gaussian distributions on $\BbbR^{\Bbb N}$, as in 466Xp.   But
perhaps the most important centered Gaussian measure is Wiener measure
(477Yj), which is not in fact on a Banach space.

A curious geometric question concerning measures on metric spaces is the
following.   If two totally finite Radon measures on a metric space
agree on balls, must they be identical?   It is known that (even for
compact spaces) the answer, in general, is `no' ({\smc Davies 71});
in Hilbert spaces the answer is `yes' (466Xj);   and in fact the same is
true in any normed space ({\smc Preiss \& Ti\v{s}er 91}).


}%end of notes

\discrpage

