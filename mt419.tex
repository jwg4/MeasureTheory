\frfilename{mt419.tex}
\versiondate{2.12.05}
\copyrightdate{2002}

\def\chaptername{Topologies and measures I}
\def\sectionname{Examples}

\newsection{419}

In \S216, I went much of the way to describing examples of spaces with
all the possible combinations of the properties considered in Chapter
21.   When we come to topological measure spaces, the number of
properties involved makes it unreasonable to seek any such comprehensive
list.   I therefore content myself with seven examples
%419A 419C 419D 419E 419J 419K 419L
to indicate some of the boundaries of the theory developed here.

The first example (419A) is supposed to show that the hypothesis
`effectively locally finite' which appears in so many of the theorems of
this chapter cannot as a rule be replaced by `locally finite'.   The
next two (419C-419D) address technical questions concerning the
definition of `Radon measure', and show how small variations in the
definition can lead to very different kinds of measure space.   The
fourth example (419E) shows that the $\tau$-additive product measures of
\S417 are indeed new constructions.   419J is there to
show that extension theorems of the types proved in \S415 and \S417
cannot be taken for granted.   The classic example 419K
exhibits one of the obstacles to generalizations of Prokhorov's theorem
(418M, 418Q).   Finally, I return to the split interval (419L) to
describe its standard topology and its relation to the measure
introduced in 343J.

\vleader{60pt}{419A}{Example} There is a locally compact Hausdorff space
$X$ with a complete, $\sigma$-finite, locally finite, $\tau$-additive
topological measure $\mu$, inner regular with respect to the closed
sets, which has a closed subset $Y$, of measure $1$, such that the
subspace measure $\mu_Y$
on $Y$ is not $\tau$-additive.   In particular, $\mu$ is not effectively
locally finite.

\proof{{\bf (a)} Let $Q$ be a countably infinite set, not containing any
ordinal.   Fix an enumeration $\sequencen{q_n}$ of $Q$, and for
$A\subseteq Q$ set $\nu A=\sum\{\bover{1}{n+1}:q_n\in A\}$.   Let $\Cal I$
be the ideal $\{A:A\subseteq Q,\,\nu A<\infty\}$.
For any sets $I$, $J$, say that $I\subseteq^*J$ if $I\setminus J$ is
finite;  then $\subseteq^*$ is a reflexive transitive relation.

Let $\kappa$ be the smallest cardinal of any family
$\Cal K\subseteq\Cal I$ for which there is no $I\in\Cal I$ such that
$K\subseteq^*I$ for
every $K\in\Cal K$.   Then $\kappa$ is uncountable.   \Prf\ (i) Of
course $\kappa$ is infinite.   (ii) If $\sequencen{I_n}$ is any sequence
in $\Cal I$, then for each $n\in\Bbb N$ we can find a finite
$I'_n\subseteq I_n$
such that $\nu(I_n\setminus I'_n)\le 2^{-n}$;  setting
$I=\bigcup_{n\in\Bbb N}I_n\setminus I'_n$, we have $\nu I\le 2<\infty$,
while $I_n\subseteq^* I$ for every $n$.   Thus $\kappa>\omega$.\ \Qed

\medskip

{\bf (b)} There is a family $\langle I_{\xi}\rangle_{\xi<\kappa}$ in
$\Cal I$ such that (i) $I_{\eta}\subseteq^*I_{\xi}$ whenever
$\eta\le\xi<\kappa$ (ii) there is no $I\in\Cal I$ such that
$I_{\xi}\subseteq^*I$ for every
$\xi<\kappa$.   \Prf\ Take a family $\langle K_{\xi}\rangle_{\xi<\kappa}$
in $\Cal I$ such that there is no $I\in\Cal I$ such that
$K_{\xi}\subseteq^*I$ for every $\xi<\kappa$.   Choose
$\langle I_{\xi}\rangle_{\xi<\kappa}$ in $\Cal I$ inductively in such a
way that

\Centerline{$K_{\xi}\subseteq^*I_{\xi}$,
\quad$I_{\eta}\subseteq^*I_{\xi}$ for every $\eta<\xi$.}

\noindent (This can be done because
$\{K_{\xi}\}\cup\{I_{\eta}:\eta<\xi\}$
will always be a subset of $\Cal I$ of cardinal less than $\kappa$.)
If now $I_{\xi}\subseteq^*I$ for every $\xi<\kappa$, then
$K_{\xi}\subseteq^*I$
for every $\xi<\kappa$, so $I\notin\Cal I$.\ \Qed

Hence, or otherwise, we see that $\kappa$ is regular.   \Prf\ If
$A\subseteq\kappa$ and $\#(A)<\kappa$, then there is an $I\in\Cal I$
such that $I_{\zeta}\subseteq^*I$ for every $\zeta\in A$; now there must
be a $\xi<\kappa$ such that $I_{\xi}\not\subseteq^*I$, in which case
$\zeta<\xi$ for every $\zeta\in A$, and $A$ is not cofinal with
$\kappa$.\ \Qed

\medskip

{\bf (c)} Set $X=Q\cup\kappa$.   (This is where it is helpful to have
arranged at the start that no ordinal belongs to $Q$, so that
$Q\cap\kappa=\emptyset$.)   Let $\frak T$ be the family of sets
$G\subseteq X$ such that

\inset{$G\cap\kappa$ is open for the order topology of $\kappa$,

for every $\xi\in G\cap\kappa\setminus\{0\}$ there is an
$\eta<\xi$ such that $I_{\xi}\setminus I_{\eta}\subseteq^*G$,

if $0\in G$ then $I_0\subseteq^*G$.}

\medskip

\quad{\bf (i)} This is a Hausdorff topology on $X$.   \Prf\ ($\alpha$)
It is easy to check that $X\in\frak T$, $\emptyset\in\frak T$ and
$\bigcup\Cal G\in\frak T$ for every $\Cal G\subseteq\frak T$.
($\beta$) Suppose that $G$, $H\in\frak T$.   Then $(G\cap H)\cap\kappa
=(G\cap\kappa)\cap(H\cap\kappa)$ is open for the order topology of
$\kappa$.   If $\xi\in G\cap H\cap\kappa\setminus\{0\}$ there are
$\eta$, $\zeta<\xi$ such that $I_{\xi}\setminus I_{\eta}\subseteq^*G$
and $I_{\xi}\setminus I_{\zeta}\subseteq^*H$, and now

\Centerline{$\alpha=\max(\eta,\zeta)<\xi$,
\quad$I_{\eta}\cup I_{\zeta}\subseteq^*I_{\alpha}$,}

\noindent so

\Centerline{$I_{\xi}\setminus I_{\alpha}\subseteq^*(I_{\xi}\setminus
I_{\eta})\cap(I_{\xi}\setminus I_{\zeta})\subseteq^*G\cap H$.}

\noindent Finally, if $0\in G\cap H$ then $I_0\subseteq^*G\cap H$.   So
$G\cap H\in\frak T$.   Thus $\frak T$ is a topology on $X$.   ($\gamma$)
For any $\xi<\kappa$, the set $E_{\xi}=(\xi+1)\cup I_{\xi}$ is
open-and-closed for $\frak T$;  for any $q\in Q$, $\{q\}$ is
open-and-closed.   Since these sets separate the points of $X$,
$\frak  T$ is Hausdorff.\ \Qed

\medskip

\quad{\bf (ii)} The sets $E_{\xi}$ of the last paragraph are all compact
for $\frak T$.   \Prf\ Let $\Cal F$ be an ultrafilter on $X$ containing
$E_{\xi}$.   ($\alpha$) If a finite set $K$ belongs to $\Cal F$, then
$\Cal F$ must contain $\{x\}$ for some $x\in K$, and converges to $x$.
So suppose henceforth that $\Cal F$ contains no finite set.   ($\beta$)
If $E_0\in\Cal F$, then for any open set $G$ containing $0$,
$E_0\setminus G$ is finite, so does not belong to $\Cal F$, and
$G\in\Cal F$;  as $G$ is
arbitrary, $\Cal F\to 0$.   ($\gamma$) If $E_0\notin\Cal F$, let
$\eta\le\xi$ be the least ordinal such that $E_{\eta}\in\Cal F$.   If
$G$ is an open set containing $\eta$, there are $\zeta'$, $\zeta''<\eta$
such that $I_{\eta}\setminus I_{\zeta'}\subseteq^*G$ and
$\ocint{\zeta'',\eta}\subseteq G$;  so that
$E_{\eta}\setminus E_{\zeta}\subseteq^*G$,
where $\zeta=\max(\zeta',\zeta'')<\eta$.   Now
$E_{\eta}\in\Cal F$, $E_{\zeta}\notin\Cal F$ and $(E_{\eta}\setminus
E_{\zeta})\setminus G\notin\Cal F$, so that $G\in\Cal F$.   As $G$ is
arbitrary, $\Cal F\to\eta$.   ($\delta$) As $\Cal F$ is arbitrary,
$E_{\xi}$ is compact.\ \Qed

\medskip

\quad{\bf (iii)} It follows that $\frak T$ is locally compact.   \Prf\
For $q\in Q$, $\{q\}$ is a compact open set containing $q$;  for
$\xi<\kappa$, $E_{\xi}$ is a compact open set containing $\xi$.\ \Qed

\medskip

\quad{\bf (iv)} The definition of $\frak T$ makes it clear that
$Q\in\frak T$, that is, that $\kappa$ is a closed subset of $X$.   We
need also to check that the subspace topology $\frak T_{\kappa}$ on
$\kappa$ induced by $\frak T$ is just the order topology of $\kappa$.
\Prf\ ($\alpha$) By the definition of $\frak T$, $G\cap\kappa$ is open
for the order topology of $\kappa$ for every $G\in\frak T$.    ($\beta$)
For any $\xi<\kappa$, $E_{\xi}$ is open-and-closed for $\frak T$ so
$\xi+1=E_{\xi}\cap\kappa$ is
open-and-closed for $\frak T_{\kappa}$.   But this means that all sets
of the forms $\coint{0,\xi}=\bigcup_{\eta<\xi}\eta+1$ and
$\ooint{\xi,\kappa}=\kappa\setminus(\xi+1)$ belong to $\frak
T_{\kappa}$;  as these generate the order topology, every open set for
the order topology belongs to $\frak T_{\kappa}$, and the two topologies
are equal.\ \Qed

\medskip

{\bf (d)} Now let $\Cal F$ be the filter on $X$ generated by the cofinal
closed
sets in $\kappa$.   Because the intersection of any sequence of closed
cofinal sets in $\kappa$ is another (4A1Bd), the intersection of any
sequence in $\Cal F$ belongs to $\Cal F$.   So

\Centerline{$\Sigma=\Cal F\cup\{X\setminus F:F\in\Cal F\}$}

\noindent is a $\sigma$-algebra of subsets of $X$, and we have a measure
$\mu_1:\Sigma\to\{0,1\}$ defined by saying that $\mu_1F=1$,
$\mu_1(X\setminus F)=0$ if $F\in\Cal F$.

\medskip

{\bf (e)} Set $\mu E=\nu(E\cap Q)+\mu_1E$ for $E\in\Sigma$.   Then $\mu$
is a measure.   Let us work through the properties called for.

\medskip

\quad{\bf (i)} If $\mu E=0$ and $A\subseteq E$, then $X\setminus
A\supseteq X\setminus E\in\Cal F$, so $A\in\Sigma$.   Thus $\mu$ is
complete.

\medskip

\quad{\bf (ii)} $\mu\kappa=1$ and $\mu\{q\}$ is finite for every $q\in
Q$, so $\mu$ is $\sigma$-finite.

\medskip

\quad{\bf (iii)} If $G\subseteq X$ is open, then $\kappa\setminus G$ is
closed, in the order topology of $\kappa$;  if it is cofinal with
$\kappa$, it belongs to $\Cal F$;  otherwise, $\kappa\cap G\in\Cal F$.
Thus in either case $G\in\Sigma$, and $\mu$ is a topological measure.

\medskip

\quad{\bf (iv)} The next thing to note is that $\mu G=\nu(G\cap Q)$ for
every open set $G\subseteq X$.   \Prf\ If $G\notin\Cal F$ this is
trivial.   If $G\in\Cal F$, then $\kappa\setminus G$ cannot be cofinal
with $\kappa$, so there is a $\xi<\kappa$ such that
$\kappa\setminus\xi\subseteq G$.
\Quer\ If $G\cap Q\in\Cal I$, then $(G\cap Q)\cup I_{\xi}\in\Cal I$.
There must be a least $\eta<\kappa$ such that
$I_{\eta}\not\subseteq^*(G\cap Q)\cup I_{\xi}$;  of course $\eta>\xi$,
so $\eta\in G$.   There is some $\zeta<\eta$ such that
$I_{\eta}\setminus I_{\zeta}\subseteq^*G$;  but as
$I_{\zeta}\subseteq^*G\cup I_{\xi}$, by the choice of $\eta$, we must
also have $I_{\eta}\subseteq^*G\cup I_{\xi}$,
which is impossible.\ \BanG\   Thus $G\cap Q\notin\Cal I$ and $\mu
G=\nu(G\cap Q)=\infty$.\ \Qed

\medskip

\quad{\bf (v)} It follows that $\mu$ is $\tau$-additive.   \Prf\ Suppose
that $\Cal G\subseteq\frak T$ is a non-empty upwards-directed set with
union $H$.   Then

\Centerline{$\mu H=\nu(H\cap Q)=\sup_{G\in\Cal G}\nu(G\cap Q)
=\sup_{G\in\Cal G}\mu G$}

\noindent because $\nu$ is $\tau$-additive (indeed, is a Radon measure)
with respect to the discrete topology on $Q$.\ \Qed

\medskip

\quad{\bf (vi)} $\mu$ is inner regular with respect to the closed sets.
\Prf\ Take $E\in\Sigma$ and $\gamma<\mu E$.   Then
$\gamma-\mu_1E<\nu(E\cap Q)$, so there is a finite
$I\subseteq E\cap Q$ such that $\nu I>\gamma-\mu_1E$.   If
$\mu_1E=0$, then $I\subseteq E$ is already a closed set with
$\mu I>\gamma$.   Otherwise, $E\cap\kappa\in\Cal F$, so there is a
cofinal closed
set $F\subseteq\kappa$ such that $F\subseteq E$;  now $F$ is closed in
$X$ (because $\kappa$ is closed in $X$ and the subspace topology on
$\kappa$ is the order topology), so $I\cup F$ is closed, and
$\mu(I\cup F)>\gamma$.   As $E$ and $\gamma$ are arbitrary,
$\mu$ is inner regular with respect to the closed sets.\ \Qed

\medskip

\quad{\bf (vii)} $\mu$ is locally finite.   \Prf\ For any $\xi<\kappa$,
$E_{\xi}$ is an open set containing $\xi$, and $\mu E_{\xi}=\nu I_{\xi}$
is finite.   For any $q\in Q$, $\{q\}$ is an open set containing $q$,
and $\mu\{q\}=\nu\{q\}$ is finite. \Qed

\medskip

\quad{\bf (viii)} Now consider $Y=\kappa$.   This is surely a closed
set, and $\mu\kappa=1$.   I noted in (c-iv) above that the subspace
topology $\frak T_{\kappa}$ is just the order topology of $\kappa$.
But this means that $\{\xi:\xi<\kappa\}$ is an upwards-directed family
of negligible relatively open sets with union $\kappa$, so that the
subspace measure $\mu_{\kappa}=\mu_1$ is not $\tau$-additive.

\medskip

\quad{\bf (ix)} It follows from 414K that $\mu$ cannot be effectively
locally finite;  but it is also obvious from the work above that
$\kappa$ is a measurable set, of non-zero measure, such that
$\mu(\kappa\cap G)=0$
whenever $G$ is an open set of finite measure.
}%end of proof of 419A

%for a similar example where $\mu$ is not localizable, see
%mt51x


\leader{419B}{Lemma} For any non-empty set $I$, there is a dense
G$_{\delta}$ set in $[0,1]^I$ which is negligible for the usual measure
on $[0,1]^I$.

\proof{ Fix on some $i_0\in I$, and set $\pi(x)=x(i_0)$ for each
$x\in[0,1]^I$, so that $\pi$ is continuous and \imp\ for the usual
topologies and measures on $[0,1]^I$ and $[0,1]$.   For each $n\in\Bbb N$
let $G_n\supseteq[0,1]\cap\Bbb Q$ be an open subset of $[0,1]$ with
measure at most $2^{-n}$, so that $\pi^{-1}[G_n]$ is an open set of
measure at most $2^{-n}$, and $E=\bigcap_{n\in\Bbb N}\pi^{-1}[G_n]$ is a
G$_{\delta}$ set of measure $0$.   If $H\subseteq[0,1]^I$ is any
non-empty open set, its
image $\pi[H]$ is open in $[0,1]$, so contains some rational number, and
meets $\bigcap_{n\in\Bbb N}G_n$;  but this means that
$H\cap E\ne\emptyset$, so $E$ is dense.
}%end of proof of 419B

\leader{419C}{Example}\cmmnt{ ({\smc Fremlin 75b})} There is a
completion regular Radon measure space $(X,\frak T,\Sigma,\mu)$ such
that

(i) there is an $E\in\Sigma$ such that $\mu(F\symmdiff E)>0$ for
every Borel set $F\subseteq X$, that is, not every element of the
measure algebra of $\mu$ can be represented by a Borel set;

(ii) $\mu$ is not outer regular with respect to the Borel sets;

(iii) writing $\nu$ for the restriction of $\mu$ to the Borel
$\sigma$-algebra of $X$, $\nu$ is a locally finite, effectively locally
finite, tight\cmmnt{ (that is, inner regular with respect to the
compact sets)} $\tau$-additive completion regular
topological measure, and there is a set
$Y\subseteq X$ such that the subspace measure $\nu_Y$ is not semi-finite.

\proof{{\bf (a)} For each $\xi<\omega_1$ set
$X_{\xi}=[0,1]^{\omega_1\setminus\xi}$, and take $\mu_{\xi}$ to be the
usual measure on $X_{\xi}$;  write $\Sigma_{\xi}$ for its domain.
Note that $\mu_{\xi}$ is a completion regular Radon measure for the
usual topology $\frak T_{\xi}$ of $X_{\xi}$ (416U).   Set
$X=\bigcup_{\xi<\omega_1}X_{\xi}$, and let $\mu$ be the direct sum
measure on $X$ (214L), that is, write

\Centerline{$\Sigma=\{E:E\subseteq X,\,E\cap X_{\xi}\in\Sigma_{\xi}$ for
every $\xi<\omega_1\}$,}

\Centerline{$\mu E=\sum_{\xi<\omega_1}\mu_{\xi}(E\cap X_{\xi})$ for
every $E\in\Sigma$.}

\noindent Then $\mu$ is a complete locally determined (in fact, strictly
localizable) measure on $X$.   Write $\Sigma$ for its domain.

\medskip

{\bf (b)} For each $\eta<\omega_1$ let
$\langle\beta_{\xi\eta}\rangle_{\xi\le\eta}$ be a summable family of
strictly positive real numbers with $\beta_{\eta\eta}=1$ (4A1P).
Define $g_{\eta}:X\to\Bbb R$ by setting

$$\eqalign{g_{\eta}(x)&=\Bover1{\beta_{\xi\eta}}x(\eta)
  \text{ if }x\in X_{\xi}\text{ where }\xi\le\eta,\cr
&=0\text{ if }x\in X_{\xi}\text{ where }\xi>\eta.\cr}$$

\noindent Now define $f:X\to\omega_1\times\BbbR^{\omega_1}$ by setting

\Centerline{$f(x)=(\xi,\langle g_{\eta}(x)\rangle_{\eta<\omega_1})$}

\noindent if $x\in X_{\xi}$.   Note that $f$ is injective.   Let
$\frak T$ be the topology on $X$ defined by $f$, that is, the family
$\{f^{-1}[W]:W\subseteq\omega_1\times\BbbR^{\omega_1}$ is open$\}$,
where $\omega_1$ and $\BbbR^{\omega_1}$ are given their usual
topologies (4A2S, 3A3K), and their product is given its product
topology.   Because $f$ is injective, $\frak T$ can be identified with
the subspace topology on $f[X]$;  it is Hausdorff and completely
regular.

\medskip

{\bf (c)} For $\xi$, $\eta<\omega_1$, $g_{\eta}\restr X_{\xi}$ is
continuous for the compact topology $\frak T_{\xi}$.   Consequently
$f\restr X_{\xi}$ is continuous, and the subspace topology on $X_{\xi}$
induced by $\frak T$ must be $\frak T_{\xi}$ exactly.   It follows that
$\mu$ is a Radon measure for $\frak T$.   \Prf\ (i) We know already that
$\mu$ is complete and locally determined.   (ii) If $G\in\frak T$ then
$G\cap X_{\xi}\in\frak T_{\xi}\subseteq\Sigma_{\xi}$ for every
$\xi<\omega_1$, so $G\in\Sigma$;  thus $\mu$ is a topological measure.
(iii) If $E\in\Sigma$ and $\mu E>0$, there is a $\xi<\omega_1$ such that
$\mu_{\xi}(E\cap X_{\xi})>0$.   Because $\mu_{\xi}$ is a Radon measure,
there is a $\frak T_{\xi}$-compact set $F\subseteq E\cap X_{\xi}$ such
that $\mu_{\xi}F>0$.   Now $F$ is $\frak T$-compact and $\mu F>0$.   As
$E$ is arbitrary, $\mu$ is tight (using 412B).   (iv) If $x\in X$, take
that $\xi<\omega_1$ such that $x\in X_{\xi}$, and consider

\Centerline{$G
=f^{-1}[(\xi+1)\times\{w:w\in\Bbb R^{\omega_1},\,w(\xi)<2\}]$.}

\noindent Because $\xi+1$ is open in $\omega_1$,
$G\in\frak T$.   Because $g_{\xi}(x)=x(\xi)\le 1$, $x\in G$.   Now for
$\zeta\le\xi$,

$$\eqalign{\mu_{\zeta}(G\cap X_{\zeta})
&=\mu_{\zeta}\{x:x\in X_{\zeta},\,g_{\xi}(x)<2\}\cr
&=\mu_{\zeta}\{x:x\in X_{\zeta},\,\beta_{\zeta\xi}^{-1}x(\xi)<2\}\cr
&=\mu_{\zeta}\{x:x\in X_{\zeta},\,x(\xi)<2\beta_{\zeta\xi}\}
\le 2\beta_{\zeta\xi},\cr}$$

\noindent so

\Centerline{$\mu G=\sum_{\zeta\le\xi}\mu_{\zeta}(G\cap X_{\zeta})
\le 2\sum_{\zeta\le\xi}\beta_{\zeta\xi}<\infty$.}

\noindent As $x$ is arbitrary, $\mu$ is locally finite, therefore a
Radon measure.\ \Qed

We also find that $\mu$ is completion regular.   \Prf\ If $E\subseteq X$
and $\mu E>0$, then there is a $\xi<\omega_1$ such that
$\mu(E\cap X_{\xi})>0$.   Because $\mu_{\xi}$ is completion regular,
there is a set
$F\subseteq E\cap X_{\xi}$, a zero set for $\frak T_{\xi}$, such that
$\mu F>0$.   Now $X_{\xi}$ is a G$_{\delta}$ set in $X$ (being the
intersection of the open sets $\bigcup_{\eta<\zeta<\xi+1}X_{\zeta}$ for
$\eta<\xi$, unless $\xi=0$, in which case $X_{\xi}$ is actually open), so
$F$ is a G$_{\delta}$ set in $X$ (4A2C(a-iv));  being a compact
G$_{\delta}$ set in
a completely regular space, it is a zero set (4A2F(h-v)).

Thus every set of positive measure includes a zero set of positive
measure.   So $\mu$ is inner regular with respect to the zero sets
(412B).\ \Qed

\medskip

{\bf (d)} The key to the example is the following fact:  if
$G\subseteq X$ is open, then {\it either} there is a cofinal closed set
$V\subseteq\omega_1$ such that $G\cap X_{\xi}=\emptyset$ for every
$\xi\in V$ {\it or} $\{\xi:\mu(G\cap X_{\xi})\ne 1\}$ is countable.
\Prf\ Suppose that $A=\{\xi:G\cap X_{\xi}\ne\emptyset\}$ meets every
cofinal closed set, that is, is stationary (4A1C).   Then
$B=A\cap\Omega$ is stationary, where $\Omega$ is the set of non-zero
countable limit ordinals (4A1Bb, 4A1Cb).   Let
$H\subseteq\omega_1\times\BbbR^{\omega_1}$ be
an open set such that $G=f^{-1}[H]$.

For each $\xi\in B$ choose $x_{\xi}\in G\cap X_{\xi}$.   Then
$f(x_{\xi})\in H$, so there must be a $\zeta_{\xi}<\xi$, a finite set
$I_{\xi}\subseteq\omega_1$, and a $\delta_{\xi}>0$ such that $z\in H$
whenever $z=(\gamma,\ofamily{\eta}{\omega_1}{t_{\eta}})
\in\omega_1\times\BbbR^{\omega_1}$,
$\zeta_{\xi}<\gamma\le\xi$ and
$|t_{\eta}-g_{\eta}(x)|<\delta_{\xi}$ for every $\eta\in I_{\xi}$.
Because $\xi$ is a non-zero limit ordinal,
$\zeta'_{\xi}=\sup(\{\zeta_{\xi}\}\cup(I_{\xi}\cap\xi))<\xi$.

By the Pressing-Down Lemma (4A1Cc), there is a $\zeta<\omega_1$
such that $C=\{\xi:\xi\in B,\,\zeta'_{\xi}=\zeta\}$ is uncountable.
\Quer\ Suppose, if possible, that $\zeta<\eta<\omega_1$ and
$\mu(G\cap X_{\eta})<1$.   Then there is a measurable subset $F$ of
$X_{\eta}\setminus G$, determined by coordinates in a countable set
$J\subseteq\omega_1\setminus\eta$, such that $\mu F=\mu_{\eta}F>0$
(254Ff).   Let $\xi\in C$ be
such that $\eta<\xi$ and $J\subseteq\xi$, and take any $y\in F$.   If we
define $y'\in X_{\eta}$ by setting

$$\eqalign{y'(\gamma)&=y(\gamma)
  \text{ for }\gamma\in\xi\setminus\eta\cr
&=x_{\xi}(\gamma)\text{ for }\gamma\in \omega_1\setminus\xi,\cr}$$

\noindent then $y'\in F$.   But also
$\zeta_{\xi}\le\zeta_{\xi}'=\zeta<\eta<\xi$ and
$\xi\setminus\eta\subseteq\xi\setminus\zeta'_{\xi}$ is disjoint from
$I_{\xi}$, so $g_{\gamma}(y')=g_{\gamma}(x_{\xi})$ for every
$\gamma\in I_{\xi}$, since both are zero if $\gamma<\eta$ and otherwise
$y'(\gamma)=x_{\xi}(\gamma)$.   By the choice of $\zeta_{\xi}$ and
$I_{\xi}$ we must have $f(y')\in H$ and $y'\in F\cap G$;  which is
impossible.\ \Bang

Thus $\mu(G\cap X_{\eta})=1$ for every $\eta>\zeta$, as required by the
second alternative.\ \Qed

\medskip

{\bf (e)} For each $\xi<\omega_1$, let $\Cal I_{\xi}$ be the family of
negligible meager subsets of $X_{\xi}$.   Then $\Cal I_{\xi}$ is a
$\sigma$-ideal;  note that it contains every closed negligible set,
because $\mu_{\xi}$ is strictly positive.   Set

\Centerline{$\Tau_{\xi}=\Cal I_{\xi}\cup\{X_{\xi}\setminus F:F\in\Cal
I_{\xi}\}$,}

\noindent so that $\Tau_{\xi}$ is a $\sigma$-algebra of subsets of
$X_{\xi}$, containing every conegligible open set, and
$\mu_{\xi}F\in\{0,1\}$ for every $F\in\Tau_{\xi}$.   Set

\Centerline{$\Tau
=\{E:E\in\Sigma,\,\{\xi:E\cap X_{\xi}\notin\Tau_{\xi}\}$
is non-stationary$\}$.}

\noindent Then $\Tau$ is a $\sigma$-subalgebra of $\Sigma$ (because the
non-stationary sets form a $\sigma$-ideal of subsets of $\omega_1$,
4A1Cb), and contains every open set, by (d);  so includes the Borel
$\sigma$-algebra $\Cal B$ of $X$.

If we set

\Centerline{$E_{\xi}
=\{x:x\in X_{\xi},\,x(\xi)\le\bover12\}$ for each $\xi<\omega_1$,
\quad$E=\bigcup_{\xi<\omega_1}E_{\xi}$,}

\noindent then $E\in\Sigma$.   But if $F\subseteq X$ is a Borel set,
$F\in\Tau$ so $\mu(E\symmdiff F)=\infty$.   This proves the property (i)
claimed for the example.

\medskip

{\bf (f)} Next, for each $\xi<\omega_1$, take a negligible dense
G$_{\delta}$ set $E'_{\xi}\subseteq X_{\xi}$ (419B).  Set
$Y=\bigcup_{\xi<\omega_1}E'_{\xi}$, so that $\mu Y=0$.   If
$F\supseteq Y$ is a Borel set, then
$F\cap X_{\xi}\supseteq E_{\xi}\notin\Cal I_{\xi}$ for every
$\xi<\omega_1$, while $F\in\Tau$, so
$\{\xi:\mu_{\xi}(F\cap X_{\xi})=0\}$ is non-stationary and
$\mu F=\infty$.   Thus $\mu$ is not
outer regular with respect to the Borel sets.   Taking
$\nu=\mu\restr\Cal B$, the subspace measure $\nu_Y$ is not semi-finite.
\Prf\ We have just seen that $\nu_YY=\nu^*Y$ is infinite.   If
$F\in\Cal B$ and $\nu F<\infty$, then
$A=\{\xi:\mu_{\xi}(F\cap X_{\xi})>0\}$ is countable, so
$F_0=\bigcup_{\xi\in A}E'_{\xi}$ and
$F_1=F\setminus\bigcup_{\xi\in A}X_{\xi}$ are negligible Borel sets;
since $F\cap Y\subseteq F_0\cup F_1$, $\nu_Y(F\cap Y)=0$.   But this
means that $\nu_Y$ takes no values in $\ooint{0,\infty}$ and is not
semi-finite.\ \Qed
}%end of proof of 419C

\cmmnt{\medskip

\noindent{\bf Remark} $X$ here is not locally compact.   But as it is
Hausdorff and completely regular, it can be embedded as a subspace of a
locally compact Radon measure space $(X',\frak T',\Sigma',\mu')$ (416T).
Now $\mu'$ still has the properties (i)-(iii).
}%end of comment

\leader{419D}{Example}\cmmnt{ ({\smc Fremlin 75b})} There is a
complete locally determined $\tau$-additive completion regular
topological measure space $(X,\frak T,\Sigma,\mu)$ in which $\mu$ is
tight and compact sets have finite measure, but $\mu$ is not
localizable.

\proof{{\bf (a)} Let $I$ be a set of cardinal greater than $\frak c$.
Set $X=[0,1]^I$.   For $i\in I$, $t\in[0,1]$ set
$X_{it}=\{x:x\in X,\,x(i)=t\}$.   Give $X_{it}$ its natural topology
$\frak T_{it}$ and
measure $\mu_{it}$, with domain $\Sigma_{it}$, defined from the
expression of $X_{it}$ as $[0,1]^{I\setminus\{i\}}\times\{t\}$, each
factor $[0,1]$ being given its usual topology and Lebesgue measure, and
the singleton factor $\{t\}$ being given its unique (discrete) topology
and (atomic) probability measure.   By 416U, $\mu_{it}$ is a completion
regular Radon measure.   Set

\Centerline{$\frak T=\{G:G\subseteq X,\,G\cap X_{it}\in\frak T_{it}$ for
all $i\in I$, $t\in[0,1]\}$,}

\Centerline{$\Sigma=\{E:E\subseteq X,\,E\cap X_{it}\in\Sigma_{it}$ for
all $i\in I$, $t\in[0,1]\}$,}

\Centerline{$\mu E=\sum_{i\in I,t\in[0,1]}\mu_{it}(E\cap X_{it})$ for
every $E\in\Sigma$.}

\noindent (Compare 216D.)   Then it is easy to check that $\frak T$ is a
topology.   $\frak T$ is Hausdorff because it is
finer\cmmnt{(= larger)} than the usual
topology $\frak S$ on $X$;  because each $\frak T_{it}$ is the subspace
topology induced by $\frak S$, it is also the subspace topology induced
by $\frak T$.   Next, the definition of $\mu$ makes it a locally
determined measure;  it is a tight complete topological measure because
every $\mu_{it}$ is.

\medskip

{\bf (b)} If $K\subseteq X$ is compact, $\mu K<\infty$.   \Prf\Quer\
Otherwise, $M=\{(i,t):i\in I,\,t\in[0,1],\,\mu_{it}(K\cap X_{it})>0\}$
must be infinite.   Take any sequence $\sequencen{(i_n,t_n)}$ of
distinct elements of $M$.   Choose a sequence $\sequencen{x_n}$ in $K$
inductively, as follows.   Given $\langle x_m\rangle_{m<n}$, then set
$C_{ni}=\{x_m(i):m<n\}$ for each $i\in I$, and

\Centerline{$A_n
=\{x:x\in X_{i_nt_n},\,x(i)\notin C_{ni}$
  for $i\in I\setminus\{i_n\}\}$;}

\noindent then $\mu_{i_nt_n}^*A_n=1$ (254Lb).   Since
$\mu_{i_nt_n}(X_{it}\cap X_{ju})=0$ whenever $(i,t)\ne(j,u)$, there must
be some $x_n\in K\cap A_n\setminus\bigcup_{m\ne n}X_{i_m,t_m}$.
Continue.

This construction ensures that if $i\in I$ and $m<n$, either $i\ne
i_n$ so $x_n(i)\notin C_{ni}$ and $x_n(i)\ne x_m(i)$, or $i=i_n\ne i_m$
and $x_m\notin X_{i_nt_n}$ so $x_n(i)=t_n\ne x_m(i)$, or $i=i_m=i_n$ and
$x_n(i)=t_n\ne t_m=x_m(i)$.   But this means that $\{x_n:n\in\Bbb N\}$
is an infinite set meeting each $X_{it}$ in at most one point, and is
closed for $\frak T$;  so $\sequencen{x_n}$ has no cluster point for
$\frak T$, which is impossible.\ \Bang\Qed

\medskip

{\bf (c)} $\mu$ is not localizable.   \Prf\ Fix on any $k\in I$ and
consider $\Cal E=\{X_{kt}:t\in[0,1]\}$.   \Quer\ If $E\in\Sigma$ is an
essential supremum for $\Cal E$, then $E\cap X_{kt}$ must be
$\mu_{kt}$-conegligible for every $t\in [0,1]$.   We can therefore find
a countable set $J_t\subseteq I$ and a $\mu_{kt}$-conegligible set
$F_t\subseteq E\cap X_{kt}$, determined by coordinates in $J_t$.   At
this point recall that $\#(I)>\frak c$, so there is some $j\in
I\setminus(\{k\}\cup\bigcup_{t\in[0,1]}J_t)$.   Since $X_{j0}\cap
X_{kt}$ is negligible for every $t\in[0,1]$, $X_{j0}\cap E$ must be
negligible, and $\int_0^1\nu H_tdt=0$, where
\Centerline{$H_t=\{y:y\in[0,1]^{I\setminus\{j,k\}},\,(y,0,t)\in E\}$}

\noindent and $\nu$ is the usual measure on $[0,1]^{I\setminus\{j,k\}}$,
identifying $X$ with $[0,1]^{I\setminus\{j,k\}}\times[0,1]\times[0,1]$.
But because $F_t$ is determined by coordinates in $I\setminus\{j\}$, we
can identify it with $F'_t\times[0,1]\times\{t\}$ where $F'_t$ is a
$\nu$-conegligible subset of $[0,1]^{I\setminus\{j,k\}}$, and
$F'_t\subseteq H_t$, so $\nu H_t=1$ for every $t$, which is absurd.\
\Bang

Thus $\Cal E$ has no essential supremum in $\Sigma$, and $\mu$ cannot be
localizable.\ \Qed

\medskip

{\bf (d)} I have still to check that $\mu$ is completion regular.
\Prf\ If $E\in\Sigma$ and $\mu E>0$, there are $i\in I$, $t\in[0,1]$
such that $\mu_{it}(E\cap X_{it})>0$, and an $F\subseteq E\cap X_{it}$,
a zero set for the subspace topology of $X_{it}$, such that
$\mu_{it}F>0$.   But now
observe that $X_{it}$ is a zero set in $X$ for the usual topology
$\frak S$, so that $F$ is a zero set for $\frak S$ (4A2G(c-i))
and therefore
for the finer topology $\frak T$.   By 412B, this is enough to show that
$\mu$ is inner regular with respect to the zero sets.\ \Qed
}%end of example 419D

\cmmnt{\medskip

\noindent{\bf Remark} It may be worth noting that the topology $\frak T$
here is not regular.   See {\smc Fremlin 75b}, p.\ 106.
}%end of comment

\leader{419E}{Example}\cmmnt{ ({\smc Fremlin 76})} Let
$(Z,\frak S,\Tau,\nu)$ be the Stone space of the
measure algebra of Lebesgue measure on $[0,1]$, so that $\nu$ is a
strictly positive completion regular Radon probability
measure\cmmnt{ (411P)}.   Then the c.l.d.\ product
measure\cmmnt{ $\lambda$} on $Z\times Z$ is not a topological measure,
so is not equal to the $\tau$-additive product measure $\tilde\lambda$,
and $\tilde\lambda$ is not completion regular.

\proof{ Consider the sets $W$, $\tilde W$ described in 346K.   We have
$W\in\Lambda=\dom\lambda$ and $\tilde W=\bigcup\Cal V$, where

\Centerline{$\Cal V=\{G\times H:G,\,H\subseteq Z$ are
open-and-closed, $(G\times H)\setminus W$ is negligible$\}$.}

\noindent $\tilde W$ is a union of open sets, therefore must be open in
$Z^2$.   And $\lambda_*\tilde W\le\lambda W$.   \Prf\Quer\ Otherwise,
there is a $V\in\Lambda$ such that $V\subseteq\tilde W$ and
$\lambda V>\lambda W$.   Now $\lambda$ is tight, by 412Sb, so there is a
compact set $K\subseteq V$ such that $K\in\Lambda$ and
$\lambda K>\lambda W$.   There must be $U_0,\ldots,U_n\in\Cal V$ such
that $K\subseteq\bigcup_{i\le n}U_i$.   But $\lambda(U_i\setminus W)=0$
for every $i$, so $\lambda(K\setminus W)=0$ and
$\lambda K\le\lambda W$.\ \Bang\Qed

However, the construction of 346K arranged that $\lambda^*\tilde W$
should be $1$ and $\lambda W$ strictly less than $1$.   So
$\lambda_*\tilde W<\lambda^*\tilde W$ and $\tilde W\notin\Lambda$.
Accordingly $\lambda$ is not a topological measure and cannot be equal
to the Radon measure $\tilde\lambda$ of 417P.

We know that $\lambda$ is inner regular with respect to the zero sets
(412Sc) and is defined on every zero set (417V), while $\tilde\lambda$
properly extends $\lambda$.   But this means that $\tilde\lambda$ cannot
be inner regular with respect to the zero sets, by 412L, that is, cannot
be completion regular.
}%end of proof of 419E

\leader{419F}{Theorem}\cmmnt{ ({\smc Rao 69})}
%also {\smc Kunen 68}
$\Cal P(\omega_1\times\omega_1)
=\Cal P\omega_1\tensorhat\Cal P\omega_1$\cmmnt{, the $\sigma$-algebra
of subsets of
$\omega_1$ generated by $\{E\times F:E,\,F\subseteq\omega_1\}$}.

\proof{{\bf (a)} Because $\omega_1\le\frak c$, there is an injection
$h:\omega_1\to\{0,1\}^{\Bbb N}$;  set $E_i=\{\xi:h(\xi)(i)=1\}$ for each
$i\in\Bbb N$.

\medskip

{\bf (b)} Suppose that $A\subseteq\omega_1$ has countable vertical
sections.   Then $A\in\Cal P\omega_1\tensorhat\Cal P\omega_1$.   \Prf\
Set $B=A^{-1}[\omega_1]$ and for $\xi\in B$ choose a surjection
$f_{\xi}:\Bbb N\to A[\{\xi\}]$.   Set $g_n(\xi)=f_{\xi}(n)$ for $\xi\in B$
and $n\in\Bbb N$, and $A_n=\{(\xi,f_{\xi}(n)):\xi\in B\}$ for
$n\in\Bbb N$.   Then

$$\eqalign{A_n
&=\{(\xi,\eta):\xi\in B,\,\eta=g_n(\xi)\}\cr
&=\{(\xi,\eta):\xi\in B,\,\eta<\omega_1,\,h(g_n(\xi))=h(\eta)\}\cr
&=\bigcap_{i\in\Bbb N}
\bigl(\{(\xi,\eta):\xi\in g_n^{-1}[E_i],\,\eta\in E_i\}
  \cup\{(\xi,\eta):\xi\in B\setminus g_n^{-1}[E_i],
  \,\eta\in\omega_1\setminus E_i\}\bigr)\cr
&\in\Cal P\omega_1\tensorhat\Cal P\omega_1.\cr}$$

\noindent So

\Centerline{$A=\bigcup_{n\in\Bbb N}A_n\in\Cal P\omega_1\tensorhat\Cal
P\omega_1$.   \Qed}

\medskip

{\bf (c)} Similarly, if a subset of $\omega_1\times\omega_1$ has
countable horizontal sections, it belongs to $\Cal
P\omega_1\tensorhat\Cal
P\omega_1$.   But for any $A\subseteq\omega_1\times\omega_1$, $A=A'\cup
A''$ where

$$\eqalign{A'
&=\{(\xi,\eta):(\xi,\eta)\in A,\,\eta\le\xi\}
  \text{ has countable vertical sections},\cr
A''
&=\{(\xi,\eta):(\xi,\eta)\in A,\,\xi\le\eta\}
  \text{ has countable horizontal sections},\cr}$$

\noindent so both $A'$ and $A''$ belong to $\Cal P\omega_1\tensorhat\Cal
P\omega_1$ and $A$ also does.
}%end of proof of 419F

\leader{419G}{Corollary}\cmmnt{ ({\smc Ulam 30})} Let $Y$ be a set of
cardinal at most $\omega_1$ and $\mu$ a semi-finite measure with
domain $\Cal PY$.   Then $\mu$ is point-supported;  in particular,
if $\mu$ is $\sigma$-finite
there is a countable conegligible set $A\subseteq Y$.

\proof{ \Quer\ Suppose, if possible, otherwise.

\medskip

{\bf (a)} We can suppose that $Y$ is either countable or actually equal
to $\omega_1$.   Let $\mu_0$ be the point-supported part of $\mu$, that
is, $\mu_0A=\sum_{y\in A}\mu\{y\}$ for every
$A\subseteq Y$;  then $\mu_0$ is a point-supported measure (112Bd), so
is not equal to $\mu$.   Let $A\subseteq Y$ be
such that $\mu_0A\ne\mu A$.   Then $\mu_0A<\mu A$;  because $\mu$ is
semi-finite, there is a set $B\subseteq A$ such that $\mu_0A<\mu B<\infty$.
Set $\nu C=\mu(B\cap C)-\mu_0(B\cap C)$ for $C\subseteq Y$;  then
$\nu$ is a non-zero totally finite measure with domain $\Cal PY$,
and is zero on singletons.

\medskip

{\bf (b)} As $\nu C=0$ for every countable $C\subseteq Y$,
$Y$ is uncountable and $Y=\omega_1$.
Let $\lambda=\nu\times\nu$ be the
product measure on $\omega_1\times\omega_1$.   By 419F, the domain of
$\lambda$ is the whole of $\Cal P(\omega_1\times\omega_1)$;  in
particular, it contains
the set $V=\{(\xi,\eta):\xi\le\eta<\omega_1\}$.   Now by Fubini's
theorem

\Centerline{$\lambda V=\biggerint\nu V[\{\xi\}]\nu(d\xi)
=\int\nu(\omega_1\setminus\xi)\nu(d\xi)
=(\nu\omega_1)^2>0$,}

\noindent and also

\Centerline{$\lambda V=\biggerint\nu V^{-1}[\{\eta\}]\nu(d\eta)
=\int\nu(\eta+1)\nu(d\eta)
=0$.  \Bang}
}%end of proof of 419G

\cmmnt{\medskip

\noindent{\bf Remark} I ought to remark that this result, though not
419F, is valid for many other cardinals besides $\omega_1$;  see, in
particular, 438C below.   There will be more on this topic in Chapter 54 of
Volume 5.
}%end of comment

\leader{419H}{}\cmmnt{ For the next two examples
it will be helpful to know
some basic facts about Lebesgue measure which seemed a little advanced
for Volume 1 and for which I have not found a suitable place since.

\medskip

\noindent}{\bf Lemma} If $(X,\frak T,\Sigma,\mu)$ is an atomless
Radon measure space and $E\in\Sigma$ has non-zero measure, then
$\#(E)\ge\frak c$.

\proof{ The subspace measure on $E$ is a Radon measure (416Rb) therefore
compact (416Wa) and perfect (342L), and is not purely atomic;
by 344H, there is in fact a
negligible subset of $E$ of cardinal $\frak c$.
}%end of proof of 419H

\leader{419I}{}\cmmnt{ The next result is a strengthening of 134D.

\medskip

\noindent}{\bf Lemma} Let $\mu$ be Lebesgue measure on $\Bbb R$, and $H$
any measurable subset of $\Bbb R$.   Then there is a disjoint family
$\langle A_{\alpha}\rangle_{\alpha<\frak c}$ of subsets of $H$ such that
$H$ is a measurable envelope of every $A_{\alpha}$;  in particular,
$\mu_*A_{\alpha}=0$ and $\mu^*A_{\alpha}=\mu H$ for every
$\alpha<\frak c$.

\proof{ If $\mu H=0$, we can take every $A_{\alpha}$ to be empty;  so
suppose that $\mu H>0$.   Let $\Cal E$ be the family of closed subsets
of $H$ of non-zero measure.   By 4A3Fa, $\#(\Cal E)\le\frak c$;
enumerate $\Cal E\times\frak c$ as
$\langle(F_{\xi},\alpha_{\xi})\rangle_{\xi<\frak c}$
(3A1Ca).   Choose $\langle x_{\xi}\rangle_{\xi<\frak c}$ inductively, as
follows.   Given $\langle x_{\eta}\rangle_{\eta<\xi}$, where
$\xi<\frak c$, $F_{\xi}$ has cardinal (at least) $\frak c$, by 419H, so
cannot be included in $\{x_{\eta}:\eta<\xi\}$;  take any
$x_{\xi}\in F_{\xi}\setminus\{x_{\eta}:\eta<\xi\}$, and continue.

At the end of the induction, set

\Centerline{$A_{\alpha}=\{x_{\xi}:\xi<\frak c,\,\alpha_{\xi}=\alpha\}$.}

\noindent Then the $A_{\alpha}$ are disjoint just because the $x_{\xi}$
are distinct.

\Quer\ Suppose, if possible, that $H$ is not a measurable envelope of
$A_{\alpha}$ for
some $\alpha$.   Then $\mu_*(H\setminus A_{\alpha})>0$ (413Ei), so there
is a non-negligible measurable set $E\subseteq H\setminus A_{\alpha}$.
Now there is an $F\in\Cal E$ such that
$F\subseteq E$.   Let $\xi<\frak c$ be such that $F=F_{\xi}$ and
$\alpha=\alpha_{\xi}$;  then $x_{\xi}\in A_{\alpha}\cap F$, which is
impossible.\ \Bang

Thus $H$ is always a measurable envelope of $A_{\alpha}$.   It follows
from the definition of `measurable envelope' that
$\mu^*A_{\alpha}=\mu H$.   But also, if $\alpha<\frak c$,
$\mu_*A_{\alpha}\le\mu_*(H\setminus A_{\alpha+1})$, which is $0$, as we
have just seen.   So we have a suitable family.
}%end of proof of 419I

\leader{419J}{Example}\dvAformerly{4{}19H}
There is a complete probability space
$(X,\Sigma,\mu)$ with a Hausdorff topology $\frak T$
on $X$ such that $\mu$ is
$\tau$-additive and inner regular with respect to the Borel sets,
$\frak T$ is generated by $\frak T\cap\Sigma$, but $\mu$ has no
extension to a topological measure.

\proof{{\bf (a)} Set $Y=\omega_1+1=\omega_1\cup\{\omega_1\}$.
Let $\Tau$ be the $\sigma$-algebra of subsets of $Y$ generated by
$\{\{\xi\}:\xi<\omega_1\}$.   Let $\nu$ be the
probability measure with domain $\Tau$ defined by the formula

\Centerline{$\nu F=\Bover12\#(F\cap\{0,\omega_1\})$ for every
$F\in\Tau$.}

Set

\Centerline{$\frak S=\{\emptyset,Y\}\cup\{H:0\in H\subseteq\omega_1\}$.}

\noindent This is a topology on $Y$, and every subset of $Y$ is a Borel
set for $\frak S$;  so $\nu$ is surely inner regular with respect to the
Borel sets.

Note that

\Centerline{$\{\{0,\alpha\}:\alpha<\omega_1\}\cup\{Y\}$}

\noindent is a base for $\frak S$ included in $\Tau$.

\medskip

{\bf (b)} Set $Z=Y^{\Bbb N}\times[0,1]$.   Let $\lambda$ be the
product probability measure on $Z$
when each copy of $Y$ is given the measure
$\nu$ and $[0,1]$ is given Lebesgue measure $\mu_L$;  let
$\frak U$ be the product topology when each copy of $Y$ is given the
topology $\frak S$ and $[0,1]$ its usual topology.
Let $\Lambda$ be the domain of $\lambda$ and $\Lambda_0$ the
$\sigma$-algebra generated by sets of the form
$\{(x,t):x(i)\in F$, $t\in[0,1]$, $t<q\}$ where $i\in\Bbb N$,
$F\in\Tau$ and
$q\in\Bbb Q$;  then $\nu$ is inner regular with respect to $\Lambda_0$
(see 254Ff).
Note that $\frak U\cap\Lambda_0$ is a base for $\frak U$ because
$\frak S\cap\Tau$ is a base for $\frak S$, and $\lambda$ is inner regular
with respect to the $\frak U$-Borel sets (412Uc, or otherwise).

Define $\phi:\{0,1\}^{\Bbb N}\to Y^{\Bbb N}$ by setting

$$\eqalign{\phi(u,t)(n)&=0\text{ if }u(n)=0,\cr
&=\omega_1\text{ if }u(n)=1,\cr}$$

\noindent and set $\psi(u,t)=(\phi(u),t)$ for $u\in\{0,1\}^{\Bbb N}$,
$t\in[0,1]$.
Then $\psi$ is continuous, just because $\frak U$ is a
product topology.
Let $\nu_{\omega}\times\mu_L$ be the product measure on
$\{0,1\}^{\Bbb N}\times[0,1]$;
then $\psi$ is \imp\ for $\nu_{\omega}\times\mu_L$ and $\lambda$, by 254H.

If $V\in\Lambda_0$ there is an $\alpha<\omega_1$ such that

\inset{\noindent
if $x$, $y\in Y^{\Bbb N}$, $t\in[0,1]$ and $y(i)=x(i)$ whenever
$\min(x(i),y(i))<\alpha$, then $(x,t)\in V$ iff $(y,t)\in V$.}

\noindent \Prf\ Let $\Cal W$ be the family of sets
$V\subseteq Z$ with this property.   Then $\Cal W$ is a $\sigma$-algebra
of subsets of $Z$ containing every measurable cylinder,
so includes $\Lambda_0$.\ \QeD\

\medskip

{\bf (c)} $\#(\Lambda_0)\le\frak c$.   \Prf\ Set

\Centerline{$A_{i\xi}
=\{(x,t):x\in Y^{\Bbb N}$, $t\in[0,1]$, $x(i)=\xi\}$,
\quad $A'_q
=\{(x,t):x\in Y^{\Bbb N}$, $t\in[0,1]$, $t\le q\}$;}

\noindent for $i\in\Bbb N$,
$\xi<\omega_1$ and $q\in\Bbb Q$, and

\Centerline{$\Cal A=\{A_{i\xi}:i\in\Bbb N$, $\xi<\omega_1\}
\cup\{A'_q:q\in\Bbb Q\}$.}

\noindent Then $\Lambda_0$ is the
$\sigma$-algebra of subsets of $Z$ generated by
$\Cal A$, and $\#(\Cal A)=\omega_1\le\frak c$,
so $\#(\Lambda_0)\le\frak c$ (4A1O).\ \Qed

\medskip

{\bf (d)} There is a family $\ofamily{\xi}{\frak c}{z_{\xi}}$ in
$Z$ such that

\inset{($\alpha$)
whenever $W\in\Lambda$ and $\lambda W>0$ there is a $\xi<\frak c$
such that $z_{\xi}\in W$ and $\lambda(H\cap W)>0$ whenever $H$ is a
measurable open subset of $Z$ containing $z_{\xi}$,

($\beta$)
setting $z_{\xi}=(x_{\xi},t_{\xi})$ for each $\xi$, there is for every
$\xi<\frak c$ a $j\in\Bbb N$ such that $0<x_{\xi}(j)<\omega_1$,

($\gamma$) $t_{\xi}\ne t_{\eta}$ if $\eta<\xi<\frak c$.}

\noindent\Prf\ By 4A3Fa, the set of closed subsets of
$\{0,1\}^{\Bbb N}\times[0,1]$ has
cardinal at most $\frak c$, so there is a family
$\ofamily{\xi}{\frak c}{(K_{\xi},V_{\xi})}$ running over all pairs $(K,V)$
such that $V\in\Lambda_0$ and
$K\subseteq\psi^{-1}[V]$ is a non-negligible compact set.
Choose $\ofamily{\xi}{\frak c}{(x_{\xi},t_{\xi})}$ inductively, as follows.
Given $\ofamily{\eta}{\xi}{t_{\eta}}$ where $\xi<\frak c$, then

\Centerline{$\{t:t\in[0,1]$,
$\nu_{\omega}K_{\xi}^{-1}[\{t\}]>0\}$}

\noindent is a non-negligible measurable subset of $[0,1]$,
so has cardinal $\frak c$ (419H);  let $t_{\xi}$ be a point of this set
distinct from every $t_{\eta}$ for $\eta<\xi$.   Now Lemma 345E tells us
that there are points $u$, $u'\in K_{\xi}^{-1}[\{t_{\xi}\}]$
which differ at exactly one coordinate $j\in\Bbb N$;  suppose that
$u(j)=1$ and $u'(j)=0$.

Let $\alpha<\omega_1$ be such that
if $x$, $y\in Y^{\Bbb N}$, $t\in[0,1]$ and $y(i)=x(i)$ whenever
$\min(x(i),y(i))<\alpha$, then $(x,t)\in V_{\xi}$ iff $(y,t)\in V_{\xi}$.
Define $x_{\xi}\in Y^{\Bbb N}$ by setting $x_{\xi}(j)=\alpha$ and
$x_{\xi}(i)=\phi(u)(i)$ for $i\ne j$.   Then
$z_{\xi}=(x_{\xi},t_{\xi})$ belongs to $V_{\xi}$.   If $H\subseteq Z$ is
any open set containing $z_{\xi}$, we have a sequence
$\sequence{i}{H_i}$ in $\frak S$ such that
$x_{\xi}\in\prod_{i\in\Bbb N}H_i$ and
$\prod_{i\in\Bbb N}H_i\times\{t_{\xi}\}\subseteq H$;  now $H_j\ne\emptyset$
so $0\in H_j$ and $\phi(u')\in\prod_{i\in\Bbb N}H_i$, so that
$(u',t_{\xi})\in K_{\xi}\cap\psi^{-1}[H]$.
Continue.

The construction ensures that ($\beta$) and ($\gamma$) are satisfied.
Now, if $\lambda W>0$, let $V\in\Lambda_0$ be such that $V\subseteq W$ and
$\lambda V>0$.   In this case,
$(\nu_{\omega}\times\mu_L)(\psi^{-1}[V])>0$;  let
$K\subseteq\psi^{-1}[V]$ be a self-supporting non-negligible compact set.
Let $\xi<\frak c$ be such that $(K,V)=(K_{\xi},V_{\xi})$.   Then
$z_{\xi}\in V_{\xi}=V\subseteq W$.   If $H$ is a measurable open subset of
$Z$ containing $z_{\xi}$, then $K\cap\psi^{-1}[H]$ is not empty;  as $\psi$
is continuous and \imp\ and $K$ is self-supporting,

\Centerline{$0<(\nu_{\omega}\times\mu_L)(K\cap\psi^{-1}[H])
\le(\nu_{\omega}\times\mu_L)\psi^{-1}[V\cap H]
=\lambda(V\cap H)\le\lambda(W\cap H)$.}

\noindent So ($\alpha$) is satisfied.\ \Qed

\medskip

{\bf (e)} Set $X=\{z_{\xi}:\xi<\frak c\}$ and let $\mu$ be the subspace
measure on $X$ induced by $\lambda$;  let $\frak T$ be the subspace
topology on $X$.

\medskip

\quad{\bf (i)} $\lambda$ is complete, so $\mu$ also is.   Next,
$\mu X=\lambda^*X=1$.   \Prf\Quer\ Otherwise, there is a $W\in\Lambda$ such
that $\lambda W>0$ and $X\cap W=\emptyset$.   But we know that there is now
some $\xi<\frak c$ such that $z_{\xi}\in W$.\ \Bang\Qed

\medskip

\quad{\bf (ii)} $\frak T$ is Hausdorff because the projection from $X$ to
$[0,1]$ is injective and continuous.   $\frak T$ is generated by
$\frak T\cap\Sigma$ because $\frak U$ is generated by $\frak U\cap\Lambda$.
$\mu$ is inner regular with respect to the Borel sets because $\lambda$ is
(412Pb).

\medskip

\quad{\bf (iii)} $\mu$ is $\tau$-additive.   \Prf\Quer\ Suppose, if
possible, otherwise.   Then there is an upwards-directed family $\Cal G$
of measurable open subsets of $X$ such that $G^*=\bigcup\Cal G$ is
measurable and $\mu G^*>\sup_{G\in\Cal G}\mu G$.   Let $\Cal H$ be the
family of sets $H\in\Lambda\cap\frak U$ such that $H\cap X$ is
included in some member of $\Cal G$;  because $\frak U$ is generated by
$\frak U\cap\Lambda$, $G^*=W\cap X$, where $W=\bigcup\Cal H$.
At the same time, there is a $V\in\Lambda$ such that $G^*=X\cap V$.

Because
$\Cal G$ is upwards-directed, so is $\Cal H$.   Because $X$ has full outer
measure,

\Centerline{$\sup_{H\in\Cal H}\lambda H=\sup_{H\in\Cal H}\mu(X\cap H)
\le\sup_{G\in\Cal G}\mu G<\mu G^*=\lambda V$.}

\noindent Let $\sequencen{H_n}$ be a non-decreasing sequence in $\Cal H$
such that $\sup_{n\in\Bbb N}\lambda H_n=\sup_{H\in\Cal H}\lambda H$, and
set $W_0=\bigcup_{n\in\Bbb N}H_n$;  then $\lambda W_0<\lambda V$ and
$\lambda(H\setminus W_0)=0$ for every $H\in\Cal H$.
However, $\lambda(V\setminus W_0)>0$, so
there is a $z\in X\cap V\setminus W_0$ such that
$\lambda(H\cap V\setminus W_0)>0$ for every measurable open set $H$
containing $z$.   As $z\in X\cap V=X\cap W$,
there must be an $H\in\Cal H$ containing $z$, so this is
impossible.\ \Bang\Qed

\medskip

\quad{\bf (iv)} \Quer\ Suppose, if possible, that there is a
topological measure $\tilde\mu$ on $X$ agreeing with $\mu$
on every open set in the domain of $\mu$.   For each $i\in\Bbb N$, set
$\pi_i(x)=x(i)$ for $(x,t)\in X$.   Every subset of $Y$ is a Borel set for
$\frak S$;  because $\pi_i$ is continuous for $\frak T$ and $\frak S$,
the image measure
$\tilde\mu\pi_i^{-1}$ has domain $\Cal PY$.   Now $\#(Y)=\omega_1$,
so there must be a countable conegligible set (419G), and there must be
some $\alpha_i<\omega_1$ such that
$\tilde\mu\pi_i^{-1}(\omega_1\setminus\alpha_i)=0$.   On the other hand,

\Centerline{$\tilde\mu\pi_i^{-1}(\alpha_i\setminus\{0\})
=\mu\pi_i^{-1}(\alpha_i\setminus\{0\})
=\lambda\{(x,t):0<x(i)<\alpha_i\}
=\nu(\alpha_i\setminus\{0\})=0$,}

\noindent so $\tilde\mu\pi_i^{-1}(\omega_1\setminus\{0\})=0$.

But (d-$\beta$) ensures that

\Centerline{$X=\bigcup_{i\in\Bbb N}\pi_i^{-1}(\omega_1\setminus\{0\})$,}

\noindent so this is impossible.\ \Bang

Thus we have the required example.
}%end of proof of 419J

\cmmnt{\medskip

\noindent{\bf Remark} I note that the topology of $X$ is not regular.
Of course the phenomenon here cannot arise with regular spaces, by
415M.}

\leader{419K}{Example} ({\smc Blackwell 56}) There are sequences
$\sequencen{X_n}$, $\sequencen{\frak T_n}$ and $\sequencen{\nu_n}$ such
that (i) for each $n$,
$(X_n,\frak T_n)$ is a separable metrizable space and $\nu_n$ is a
quasi-Radon probability measure on $Z_n=\prod_{i\le n}X_i$ (ii) for
$m\le n$ the canonical map $\pi_{mn}:Z_n\to Z_m$ is \imp\ (iii) there is
no probability measure on $Z=\prod_{i\in\Bbb N}X_i$ such that all the
canonical maps from $Z$ to $Z_n$ are \imp.

\proof{ Let $\sequencen{A_n}$ be a disjoint sequence of subsets of
$[0,1]$ such that $\mu_*([0,1]\setminus A_n)=0$, that is, $\mu^*A_n=1$
for every $n$, where $\mu$ is Lebesgue measure (using 419I).   Set
$X_n=\bigcup_{i\ge n}A_i$, so that $\sequencen{X_n}$ is a non-increasing
sequence of sets of outer measure $1$ with empty intersection.   For
each $n\ge 1$, we have a map $f_n:X_n\to Z_n$ defined by setting
$f_n(x)(i)=x$ for every $i\le n$, $x\in X_n$.   Let $\nu_n$ be the image
measure $\mu_{X_n}f_n^{-1}$, where $\mu_{X_n}$ is the subspace measure
on $X_n$ induced by $\mu$.   Note that $f_n$ is a homeomorphism between
$X_n$ and the diagonal $\Delta_n=\{z:z\in Z_n,\,z(i)=z(j)$ for all $i$,
$j\le n\}$, which is a closed subset of $Z_n$;  so that $\nu_n$, like
$\mu_{X_n}$, is a quasi-Radon probability measure.

If $m\le n$, then $\pi_{mn}$ is \imp, where $\pi_{mn}(z)(i)=z(i)$ for
$z\in Z_n$ and $i\le m$.   \Prf\ If $W\subseteq Z_m$ is measured by
$\nu_m$, then $f_m^{-1}[W]$ is measured by $\mu_{X_m}$, so is of the
form $X_m\cap E$ where $E$ is Lebesgue measurable.   But in this case
$f_n^{-1}[\pi_{mn}^{-1}[W]]=X_n\cap E$, so that

\Centerline{$\nu_n(\pi_{mn}^{-1}[W])
=\mu_{X_n}(f_n^{-1}[\pi_{mn}^{-1}[W]])
=\mu^*(X_n\cap E)
=\mu E
=\mu^*(X_m\cap E)
=\nu_mW$.   \Qed}

\Quer\ But suppose, if possible, that there is a probability measure
$\nu$ on $Z=\prod_{i\in\Bbb N}X_i$ such that $\pi_n:Z\to Z_n$ is \imp\
for every $n$, where $\pi_n(z)(i)=z(i)$ for $z\in Z$ and $i\le n$.   Then

\Centerline{$\nu\pi_n^{-1}[\Delta_n]=\nu_n\Delta_n
=\mu_{X_n}f_n^{-1}[\Delta_n]=1$}

\noindent for each $n$, so

\Centerline{$1=\nu(\bigcap_{n\in\Bbb N}\pi_n^{-1}[\Delta_n])
=\nu\{z:z\in Z,\,z(i)=z(j)$ for all $i$, $j\in\Bbb N\}
=\nu\emptyset$,}

\noindent because $\bigcap_{n\in\Bbb N}X_n=\emptyset$;  which is
impossible.\ \Bang
}%end of proof of 419K

\leader{419L}{The split interval again (a)}\cmmnt{ For the sake of an
example in \S343, I have already introduced the `split interval' or
`double arrow space'.   As this construction gives us a topological
measure space
of great interest, I repeat it here.}   Let $I^{\|}$ be the set
$\{a^+:a\in[0,1]\}\cup\{a^-:a\in[0,1]\}$.   Order it by saying that

\Centerline{$a^+\le b^+\iff a^-\le b^+\iff a^-\le b^-\iff a\le b$,
\quad$a^+\le b^-\iff a<b$.}

\noindent Then\cmmnt{ it is easy to check that} $I^{\|}$ is a totally
ordered space, and\cmmnt{ that it} is Dedekind complete.
\prooflet{(If $A\subseteq[0,1]$ is a non-empty set, then
$\sup_{a\in A}a^-=(\sup A)^-$,
while $\sup_{a\in A}a^+$ is either $(\sup A)^+$ or $(\sup A)^-$,
depending on whether
$\sup A$ belongs to $A$ or not.)}   Its greatest element is $1^+$ and
its least
element is $0^-$.  Consequently the order topology on $I^{\|}$ is a
compact Hausdorff topology\cmmnt{ (4A2Ri,
{\smc Alexandroff \& Urysohn 1929})}.   \cmmnt{Note
that} $Q=\{q^+:q\in[0,1]\cap\Bbb Q\}\cup\{q^-:q\in[0,1]\cap\Bbb Q\}$ is
dense\prooflet{, because it meets
every non-trivial interval in $I^{\|}$}.   \cmmnt{By 4A2E(a-ii) and
4A2Rn,} $I^{\|}$ is ccc and hereditarily Lindel\"of.

\spheader 419Lb If we define $h:I^{\|}\to[0,1]$ by writing
$h(a^+)=h(a^-)=a$
for every $a\in[0,1]$, then $h$ is continuous\prooflet{, because
$\{x:h(x)<a\}=\{x:x<a^-\}$, $\{x:h(x)>a\}=\{x:x>a^+\}$ for every
$a\in[0,1]$}.
Now\cmmnt{ we can describe the Borel sets of $I^{\|}$, as follows:}  a
set $E\subseteq I^{\|}$ is
Borel iff there is a Borel set $F\subseteq[0,1]$ such that
$E\symmdiff h^{-1}[F]$ is countable.   \prooflet{\Prf\ Write $\Sigma_0$
for the family
of subsets $E$ of $I^{\|}$ such that $E\symmdiff h^{-1}[F]$ is countable
for some Borel set $F\subseteq [0,1]$.   It is easy to check that
$\Sigma_0$ is a $\sigma$-algebra of subsets of $I^{\|}$.
(If $E\symmdiff h^{-1}[F]$ is countable, so is
$(I^{\|}\setminus E)\symmdiff h^{-1}[[0,1]\setminus F]$;  if
$E_n\symmdiff h^{-1}[F_n]$ is countable for every $n$, so is
$(\bigcup_{n\in\Bbb N}E_n)\symmdiff h^{-1}[\bigcup_{n\in\Bbb N}F_n]$.)
Because the topology of $I^{\|}$ is Hausdorff, every singleton set is
closed, so every countable set is Borel.   Also $h^{-1}[F]$ is Borel for
every Borel set $F\subseteq[0,1]$, because $h$ is continuous (4A3Cd).
So if $E\symmdiff h^{-1}[F]$ is countable for some Borel set
$F\subseteq[0,1]$,
$E=h^{-1}[F]\symmdiff(E\symmdiff h^{-1}[F])$ is a Borel set in $I^{\|}$.
Thus $\Sigma_0$ is included in the Borel $\sigma$-algebra $\Cal B$ of
$I^{\|}$.   On the other
hand, if $J\subseteq I^{\|}$ is an interval, $h[J]$ also is an interval,
therefore a Borel set, and $h^{-1}[h[J]]\setminus J$ can contain at most
two points, so $J\in\Sigma_0$.   If $G\subseteq I^{\|}$ is open, it is
expressible
as $\bigcup_{i\in I}J_i$, where $\langle J_i\rangle_{i\in I}$ is a
disjoint family of non-empty open intervals (4A2Rj).   As $X$ is ccc,
$I$ must
be countable.   Thus $G$ is expressed as a countable union of members of
$\Sigma_0$ and belongs to $\Sigma_0$.   But this means that the Borel
$\sigma$-algebra $\Cal B$ must
be included in $\Sigma_0$, by the definition of `Borel algebra'.   So
$\Cal B=\Sigma_0$, as claimed.\ \Qed}%end of prooflet

\spheader 419Lc In 343J I described the standard measure $\mu$ on
$I^{\|}$;  its domain is the set $\Sigma
=\{h^{-1}[F]\symmdiff M:F\in\Sigma_L,\,M\subseteq I^{\|},\,\mu_Lh[M]=0\}$,
where $\Sigma_L$ is the set of Lebesgue measurable
subsets of $[0,1]$ and $\mu_L$ is Lebesgue measure,
and $\mu E=\mu_Lh[E]$ for $E\in\Sigma$.   $h$ is \imp\ for $\mu$ and
$\mu_L$.

\cmmnt{The new fact I wish to mention is:}  $\mu$ is a completion
regular Radon measure.
\prooflet{\Prf\ I noted in 343Ja that it is a complete probability
measure;  {\it a fortiori}, it is locally determined and locally finite.
If $G\subseteq I^{\|}$ is open, then we can express it as
$h^{-1}[F]\symmdiff C$ for some Borel set $F\subseteq [0,1]$ and countable
$C\subseteq I^{\|}$ ((b)
above), so it belongs to $\Sigma$;  thus $\mu$ is a topological measure.
If $E\in\Sigma$ and $\mu E>\gamma$, then
$F=[0,1]\setminus h[I^{\|}\setminus E]$
is Lebesgue measurable, and $\mu E=\mu_LF$.   So there is a compact set
$L\subseteq F$ such that $\mu_LL\ge\gamma$.   But now
$K=h^{-1}[L]\subseteq E$ is closed, therefore compact, and
$\mu K\ge\gamma$.   Moreover, $L$
is a zero set, being a closed set in a metrizable space (4A2Lc), so $K$
is a zero set (4A2C(b-iv)).   As $E$ and $\gamma$ are arbitrary, $\mu$
is
inner regular with respect to the compact zero sets, and is a completion
regular Radon measure.\ \Qed}%end of prooflet

\exercises{
\leader{419X}{Basic exercises (a)}
%\spheader 419Xa
Show that the topological space $X$ of 419A is zero-\vthsp dimensional.
%419A

\spheader 419Xb Give an example of a compact Radon probability space in
which every dense G$_{\delta}$ set is conegligible.   \Hint{411P.}
%419B

\spheader 419Xc In 419E, show that we can start from any atomless
probability measure in place of Lebesgue measure on $[0,1]$.
%419E

\sqheader 419Xd(i) Show that if $E\subseteq\BbbR^2$ is Lebesgue
measurable, with non-zero measure, then it cannot be covered by fewer
than $\frak c$ lines.   \Hint{if $H=\{t:\mu_1E[\{t\}]>0\}$, where
$\mu_1$ is Lebesgue measure on $\Bbb R$, then $\mu_1H>0$, so
$\#(H)=\frak c$.   So if we have a family $\Cal L$ of lines, with
$\#(\Cal L)<\frak c$, there must be a $t\in H$ such that
$L_t=\{t\}\times\Bbb R$ does not belong to $\Cal L$.   Now
$\#(L_t\cap E)=\frak c$ and each member of $\Cal L$ meets $L_t\cap E$ in
at most one point.}   (ii) Show that there is a subset $A$ of $\BbbR^2$,
of full outer measure, which meets every vertical line and every
horizontal line in exactly one point.   \Hint{enumerate $\Bbb R$ as
$\ofamily{\xi}{\frakc}{t_{\xi}}$ and the closed sets of non-zero measure
as $\ofamily{\xi}{\frakc}{F_{\xi}}$.   Choose
$\ofamily{\xi}{\frakc}{I_{\xi}}$ such that every $I_{\xi}$ is finite,
no two points of $I_{\xi}\cup\bigcup_{\eta<\xi}I_{\eta}$ lie on any
horizontal or vertical line, the lines $\{t_{\xi}\}\times\Bbb R$ and
$\Bbb R\times\{t_{\xi}\}$ both meet
$I_{\xi}\cup\bigcup_{\eta<\xi}I_{\eta}$, and $I_{\xi}$ meets $F_{\xi}$.}
(iii) Show that there is a subset $B$ of $\BbbR^2$, of full outer
measure, such that every straight line meets $B$ in exactly two points.
\Hint{enumerate the straight lines in $\BbbR^2$ as
$\ofamily{\xi}{\frakc}{L_{\xi}}$.   Choose
$\ofamily{\xi}{\frakc}{J_{\xi}}$ such that every $J_{\xi}$ is finite,
no three points of $J_{\xi}\cup\bigcup_{\eta<\xi}J_{\eta}$ lie on any
line, $L_{\xi}\cap(J_{\xi}\cup\bigcup_{\eta<\xi}J_{\eta})$ has just two
points and $J_{\xi}\cap F_{\xi}\ne\emptyset$.}
%419H

\spheader 419Xe Show that there is a subset $A$ of the Cantor set $C$
(134G) such that $A+A$ is not Lebesgue measurable.
\Hint{enumerate the closed
non-negligible subsets of $C+C=[0,2]$ as $\ofamily{\xi}{\frak c}{F_{\xi}}$.
Choose $x_{\xi}\in C$, $y_{\xi}\in C$, $z_{\xi}\in F_{\xi}$ so that
$z_{\xi}\notin A_{\xi}+A_{\xi}$ and $x_{\xi}+y_{\xi}\in F_{\xi}$ and
$A_{\xi+1}+A_{\xi+1}$ does not meet $\{z_{\eta}:\eta\le\xi\}$, where
$A_{\xi}=\{x_{\eta}:\eta<\xi\}\cup\{y_{\eta}:\eta<\xi\}$.}
%419H

\spheader 419Xf Let $\BbbR^{\|}$ be the {\bf split line}, that is, the
set $\{a^+:a\in\Bbb R\}\cup\{a^-:a\in\Bbb R\}$, ordered by the rules in
419L. Show that $\BbbR^{\|}$ is a Dedekind complete totally ordered
set, so that
its order topology $\frak T$ is locally compact.   Write $\mu_L$ for
Lebesgue measure
on $\Bbb R$ and $\Sigma_L$ for its domain.   Set $h(a^+)=h(a^-)=a$ for
$a\in\Bbb R$, $\Sigma
=\{E:E\subseteq X,\,h[E]\in\Sigma_L,\,\mu_L(h[E]\cap h[X\setminus E])=0\}$,
$\mu E=\mu_Lh[E]$ for $E\in\Sigma$.   Show that
$\mu$ is a completion regular Radon measure on $\BbbR^{\|}$ and that
$h$ is continuous and \imp\ for $\mu$ and
$\mu_L$.   Show that the set $\{a^+:a\in\Bbb R\}$, with the induced
topology and measure, is isomorphic, as quasi-Radon measure space, to
the right-facing Sorgenfrey line (415Xc) with Lebesgue measure.   Show
that $\BbbR^{\|}$ and the Sorgenfrey line are hereditarily Lindel\"of.
%419L

\spheader 419Xg Let $\mu$ be Lebesgue measure on $[0,1]$ and $\Sigma$
its domain.   Let $I^{\|}$ be the split interval.   (i) Show that the
functions $x\mapsto x^+:[0,1]\to I^{\|}$ and
$x\mapsto x^-:[0,1]\to I^{\|}$ are measurable.
\Hint{419Lb.}   (ii) Show that the function
$x\mapsto(x^+,x^-):[0,1]\to (I^{\|})^2$ is not measurable.  \Hint{the
subspace topology on $\{(x^+,x^-):x\in[0,1]\}$ is discrete.}
%419L

\sqheader 419Xh(i) Again writing $I^{\|}$ for the split interval, show
that the function which exchanges $x^+$ and $x^-$ for every $x\in[0,1]$
is a Borel automorphism and an automorphism for the usual Radon measure
$\nu$ on $I^{\|}$, but is not almost continuous.   (ii) Show that
if we set $f(x)=x^+$ for $x\in[0,1]$, then $f$ is \imp\ for Lebesgue
measure $\mu_L$ on $[0,1]$, but the image measure $\mu_Lf^{-1}$ is
not $\nu$ (nor, indeed, a Radon measure).
%419L

\spheader 419Xi Show that the split
interval $I^{\|}$ is perfectly normal, but that
$I^{\|}\times I^{\|}$ is not perfectly normal.
%419L

\leader{419Y}{Further exercises (a)}
%\spheader 419Ya
In the example of 419E, show that there is a Borel set $V\subseteq Z^2$
such that $\tilde\lambda V=0$ and $\lambda^*V=1$.
%419E

\spheader 419Yb
Show that if $\Cal A\subseteq\Cal P\omega_1$ is any family with
$\#(\Cal A)\le\omega_1$, there is a countably generated $\sigma$-algebra
$\Sigma$ of subsets of $\omega_1$ such that $\Cal A\subseteq\Sigma$.
%419F

\spheader 419Yc Show that the split interval
with its usual topology and measure has the simple product
property (417Yi).
%419L

}%end of exercises

\endnotes{\Notesheader{419}
The construction of the locally compact space $X$ in 419A from the
family $\langle I_{\xi}\rangle_{\xi<\kappa}$ is a standard device which
has been used many times.   The relation $\subseteq^*$ also appears in
many contexts.   In effect, part of the argument is taking place in the
quotient algebra $\frak A=\Cal PQ/[Q]^{<\omega}$, since $I\subseteq^*J$
iff $I^{\ssbullet}\Bsubseteq J^{\ssbullet}$ in $\frak A$;  setting
$\Cal I^{\#}=\{I^{\ssbullet}:I\in\Cal I\}$, the cardinal $\kappa$ is
$\min\{\#(A):A\subseteq \Cal I^{\#}$ has no upper bound in
$\Cal I^{\#}\}$, the `additivity' of the partially ordered set
$\Cal I^{\#}$.
Additivities of partially ordered sets will be one of the important
concerns of Volume 5.   I remark that we do not need to know whether
(for instance) $\kappa=\omega_1$ or $\kappa=\frak c$.   This is an early
taste of the kind of manoeuvre which has become a staple of
set-theoretic analysis.   It happens that the cardinal $\kappa$ here is
one of the most important cardinals of set-theoretic measure theory;  it
is `the additivity of Lebesgue measure' (529Xe\formerly{5{}29Xc}), 
and under that name will appear repeatedly in Volume 5.

Observe that the measure $\mu$ of 419A only just fails to be a
quasi-Radon measure;  it is locally finite instead of being effectively
locally finite.   And it would be a Radon measure if it were inner
regular with respect to the compact sets, rather than just with respect
to the closed sets.

419C and 419D are relevant to the question:  have I given the `right'
definition of Radon measure space?   419C is perhaps more important.
Here we have a Radon measure space (on my definition) for which the
associated Borel measure is not localizable.   (If $\frak A$ is the
measure algebra of the measure $\mu$, and $\frak B$ the measure algebra of
$\mu\restr\Cal B$ where $\Cal B$ is the Borel $\sigma$-algebra of $X$,
then the
embedding $\Cal B\embedsinto\Sigma$ induces an embedding of $\frak B$ in
$\frak A$ which represents $\frak B$ as an order-dense subalgebra of
$\frak A$, just because $\mu$ is inner regular with respect to $\Cal B$.
Property (i) of 419C shows that $\frak B\ne\frak A$, so $\frak B$ cannot
be Dedekind complete in itself, by 314Ib.)   Since (I believe)
localizable versions of measure spaces should almost always be
preferred, I take this
as strong support for my prejudice in favour of insisting that `Radon'
measure spaces should be locally determined as well as complete.
Property (ii) of 419C is not I think of real significance, but is
further evidence, to be added to 415Xh, that outer regularity is like an
exoskeleton:  it may inhibit growth above a certain size.

In 419D I explore the consequences of omitting the condition `locally
finite' from the definition of Radon measure.   Even if we insist
instead that compact sets should have finite measure, we are in danger
of getting a non-localizable measure.   Of course this particular space
is pathological in terms of most of the criteria of this chapter -- for
instance, every non-empty open set has infinite measure, and the
topology is not regular.

Perhaps the most important example in the section is 419E.   The
analysis of $\tau$-additive product measures in \S417 was long and
difficult, and if these were actually equal to the familiar product
measures in all important cases the structure of the theory would be
very different.   But we find
that for one of the standard compact Radon probability spaces of the
theory, the c.l.d.\ product measure on its square is not a Radon
measure, and something has to be done about it.

\wheader{419 Notes}{0}{0}{0}{72pt}

I present 419J here to indicate one of the obstacles to any
simplification of the arguments in 417C and 417E.   It is not
significant in itself, but it offers a
welcome excuse to describe some fundamental facts about $\omega_1$
(419F-419G).   Similarly, 419K asks for some
elementary facts about Lebesgue measure (419H-419I) which seem to have
got left out.   This example really is important in itself, as it
touches on the general problem of representing stochastic processes, to
which I will return in Chapter 45.
}%end of notes

\discrpage  %mt41conc to come

