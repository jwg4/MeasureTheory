\frfilename{mt363.tex}
\versiondate{4.3.08}
\copyrightdate{2000}

\def\chaptername{Function spaces}
\def\sectionname{$L^{\infty}$}

\newsection{363}

In this section I set out to describe an abstract construction for
$L^{\infty}$ spaces on arbitrary Boolean algebras,
corresponding to the $L^{\infty}(\mu)$ spaces of \S243.   I begin with
the definition of $L^{\infty}(\frak A)$ (363A) and elementary facts
concerning its own structure and the embedding
$S(\frak A)\embedsinto L^{\infty}(\frak A)$ (363B-363D).   I give the basic universal mapping
theorems which define the Banach lattice structure of $L^{\infty}$
(363E) and a description of the action of Boolean homomorphisms on
$L^{\infty}$ spaces (363F-363G) before discussing the representation of
$L^{\infty}(\Sigma)$ and $L^{\infty}(\Sigma/\Cal I)$ for
$\sigma$-algebras $\Sigma$
and ideals $\Cal I$ of sets (363H).   This leads at once to the
identification of
$L^{\infty}(\mu)$, as defined in Volume 2, with $L^{\infty}(\frak A)$,
where $\frak A$ is the measure algebra of $\mu$ (363I).   Like
$S(\frak A)$, $L^{\infty}(\frak A)$ determines the algebra $\frak A$
(363J).   I
briefly discuss the dual spaces of $L^{\infty}$;  they correspond
exactly to the duals of $S$ described in \S362 (363K).   Linear
functionals on $L^{\infty}$ can for some purposes be treated as
`integrals' (363L).

In the second half of the section I present some of the theory of
Dedekind complete and
$\sigma$-complete algebras.   First, $L^{\infty}(\frak A)$ is Dedekind
($\sigma$-\nobreak)complete iff $\frak A$ is (363M).   The spaces
$L^{\infty}(\frak A)$, for Dedekind $\sigma$-complete $\frak A$, are
precisely the Dedekind $\sigma$-complete Riesz spaces with order unit
(363N-363P).   The spaces $L^{\infty}(\frak A)$, for Dedekind
complete $\frak A$, are precisely the normed spaces which may be put in
place of $\Bbb R$ in the Hahn-Banach theorem (363R).   Finally, I
mention some equivalent forms of the Banach-Ulam problem (363S).

\leader{363A}{Definition} Let $\frak A$ be a Boolean algebra, with Stone
space $Z$.   I will write $L^{\infty}(\frak A)$ for the space
$C(Z)=C_b(Z)$ of
continuous real-valued functions from $Z$ to $\Bbb R$, endowed with the
linear structure, order structure, norm and multiplication of
$C(Z)=C_b(Z)$.   \cmmnt{ (Recall that because $Z$ is compact (311I),
$\{u(z):z\in Z\}$
is bounded for every $u\in L^{\infty}(\frak A)=C(Z)$ (2A3N(b-iii)), that
is, $C(Z)=C_b(Z)$.   Of
course if $\frak A=\{0\}$, so that $Z=\emptyset$, then $C(Z)$ has just
one member, the empty function.)}

\leader{363B}{Theorem} Let $\frak A$ be any Boolean algebra;  write
$L^{\infty}$ for $L^{\infty}(\frak A)$.

(a) $L^{\infty}$ is an $M$-space;  its standard order unit is the
constant function taking the value $1$ at each point;  in particular,
$L^{\infty}$ is a Banach lattice with a Fatou norm and the Levi
property.

(b)  $L^{\infty}$ is a commutative Banach algebra and an $f$-algebra.

(c) If $u\in L^{\infty}$ then $u\ge 0$ iff there is a $v\in L^{\infty}$
such that $u=v\times v$.

\proof{{\bf (a)} See 354Hb and 354J.

\medskip

{\bf (b)-(c)} are obvious from the definitions of Banach algebra (2A4J)
and $f$-algebra (352W) and the ordering of $L^{\infty}=C(Z)$.
}%end of proof of 363B

\leader{363C}{Proposition} Let $\frak A$ be any Boolean algebra.   Then
$S(\frak A)$ is a norm-dense, order-dense Riesz subspace of
$L^{\infty}(\frak A)$, closed under multiplication.

\proof{ Let $Z$ be the Stone space of $\frak A$.   Using the definition
of $S=S(\frak A)$ set out in 361D, it is obvious that $S$ is a linear
subspace of $L^{\infty}=L^{\infty}(\frak A)=C(Z)$ closed under
multiplication.   Because $S$, like $L^{\infty}$, is a Riesz subspace of
$\Bbb R^Z$ (361Ee), $S$ is a Riesz subspace of $L^{\infty}$.   By the
Stone-Weierstrass theorem (in either of the forms given in 281A and
281E), $S$ is
norm-dense in $L^{\infty}$.   Consequently it is order-dense (354I).
}%end of proof of 363C

\leader{363D}{Proposition} Let $\frak A$ be a Boolean algebra.   If we
regard $\chi a\in S(\frak A)$\cmmnt{ (361D)} as a member of
$L^{\infty}(\frak A)$ for each
$a\in\frak A$, then $\chi:\frak A\to L^{\infty}(\frak A)$ is additive,
order-preserving, order-continuous and a lattice homomorphism.

\proof{ Because the embedding
$S=S(\frak A)\embedsinto L^{\infty}(\frak A)=L^{\infty}$ is a Riesz
homomorphism, $\chi:\frak A\to L^{\infty}$ is
additive and a lattice homomorphism (361F-361G).   Because $S$ is
order-dense in $L^{\infty}$ (363C), the embedding
$S\embedsinto L^{\infty}$ is order-continuous (352Nb), so
$\chi:\frak A\to L^{\infty}$ is order-continuous (361Gb).
}%end of proof of 363D

\leader{363E}{Theorem} Let $\frak A$ be a Boolean algebra, and $U$ a
Banach space.   Let $\nu:\frak A\to U$ be a bounded additive function.

(a) There is a unique bounded linear operator
$T:L^{\infty}(\frak A)\to U$ such that $T\chi=\nu$;  in this case
$\|T\|=\sup_{a,b\in\frak A}\|\nu a-\nu b\|$.

(b) If $U$ is a Banach lattice then $T$ is positive iff $\nu$ is
non-negative;  and in this case $T$ is order-continuous iff $\nu$ is
order-continuous, and sequentially order-continuous iff $\nu$ is
sequentially order-continuous.

(c) If $U$ is a Banach lattice then $T$ is a Riesz homomorphism iff
$\nu$ is a lattice homomorphism iff $\nu a\wedge\nu b=0$ whenever
$a\Bcap b=0$.

\proof{ Write $S=S(\frak A)$, $L^{\infty}=L^{\infty}(\frak A)$.

\medskip

{\bf (a)} By 361I there is a unique bounded linear operator $T_0:S\to U$
such that $T_0\chi=\nu$, and $\|T_0\|=\sup\{\|\nu a-\nu b\|:a$,
$b\in\frak A\}$.   But because $U$ is a Banach space and $S$ is dense in
$L^{\infty}$, $T_0$ has a unique extension to a bounded linear operator
$T:L^{\infty}\to U$ with the same norm (2A4I).

\medskip

{\bf (b)(i)} If $T$ is positive then $T_0$ is positive so $\nu$ is
non-negative, by 361Ga.

\medskip

\quad{\bf (ii)} If $\nu$ is non-negative then $T_0$ is positive, by
361Ga in the other direction.   But
if $u\in L^{\infty+}$ and $\epsilon>0$, then by 354I there is a
$v\in S^+$ such that
$\|u-v\|_{\infty}\le\epsilon$;  now $\|Tu-Tv\|\le\epsilon\|T\|$.   But
$Tv=T_0v$ belongs to the
positive cone $U^+$ of $U$.   As $\epsilon$ is arbitrary, $Tu$ belongs
to the closure of $U^+$, which is $U^+$ (354Bc).   As $u$ is
arbitrary, $T$ is positive.

\medskip

\quad{\bf (iii)} Now suppose that $\nu$ is order-continuous as well as
non-negative, and that $A\subseteq L^{\infty}$ is a non-empty
downwards-directed set with infimum $0$.   Set

\Centerline{$B=\{v:v\in S$, there is some $u\in A$ such that
$v\ge u\}$.}

\noindent Then $B$ is downwards-directed (indeed, $v_1\wedge v_2\in B$
for every $v_1$, $v_2\in B$), and $u=\inf\{v:v\in B,\,u\le v\}$ for
every $u\in A$ (354I again), so $B$ has the same lower bounds as $A$ and
$\inf B=0$ in $L^{\infty}$ and in $S$.   But we know from 361Gb that
$T_0$ is order-continuous, while any lower bound for $\{Tu:u\in A\}$ in
$U$ must also be a lower bound for $\{Tv:v\in B\}=\{T_0v:v\in B\}$, so
$\inf_{u\in A}Tu=\inf_{v\in B}T_0v=0$ in $U$.   As $A$ is arbitrary, $T$
is order-continuous (351Ga).

\medskip

\quad{\bf (iv)} Suppose next that $\nu$ is only sequentially
order-continuous, and that $\sequencen{u_n}$ is a non-increasing
sequence in $L^{\infty}$ with infimum $0$.   For each $n$, $k$ choose
$w_{nk}\in S$ such that $u_n\le w_{nk}$ and
$\|w_{nk}-u_n\|_{\infty}\le 2^{-k}$ (354I once more), and set
$w'_n=\inf_{j,k\le n}w_{jk}$ for each
$n$.   Then $\sequencen{w'_n}$ is a non-increasing sequence in $S$, and
any lower bound of $\{w'_n:n\in\Bbb N\}$ is also a lower bound of
$\{u_n:n\in\Bbb N\}$, so $0=\inf_{n\in\Bbb N}w'_n$ in $S$ and
$L^{\infty}$.   Since $T_0:S\to U$ is sequentially order-continuous
(361Gb),

\Centerline{$\inf_{n\in\Bbb N}Tu_n\le\inf_{n\in\Bbb N}Tw'_n
=\inf_{n\in\Bbb N}T_0w'_n=0$}

\noindent in $U$.   As $\sequencen{u_n}$ is arbitrary, $T$ is
sequentially order-continuous.

\medskip

\quad{\bf (v)} On the other hand, if $T$ is order-continuous or
sequentially order-continuous, so is $\nu=T\chi$, because $\chi$ is
order-continuous (363D).

\medskip

{\bf (c)} We know that $T_0:S\to U$ is a Riesz homomorphism iff $\nu$ is
a lattice homomorphism iff $\nu a\wedge\nu b=0$ whenever $a\Bcap b=0$,
by 361Gc.   But $T_0$ is a Riesz homomorphism iff $T$ is.   \Prf\ If $T$
is a Riesz homomorphism so is $T_0$, because the embedding
$S\embedsinto L^{\infty}$ is a Riesz homomorphism.   On the other hand, if
$T_0$ is a
Riesz homomorphism, then the functions $u\mapsto u^+\mapsto T(u^+)$,
$u\mapsto Tu\mapsto (Tu)^+$ are continuous (by 354Bb) and agree on $S$,
so agree on
$L^{\infty}$, and $T$ is a Riesz homomorphism, by 352G.\ \Qed
}%end of proof of 363E

\leader{363F}{Theorem} Let $\frak A$ and $\frak B$ be Boolean algebras,
and $\pi:\frak A\to\frak B$ a Boolean homomorphism.

(a) There is an associated multiplicative Riesz homomorphism
$T_{\pi}:L^{\infty}(\frak A)\to L^{\infty}(\frak B)$, of norm at most
$1$, defined by saying that $T_{\pi}(\chi a)=\chi(\pi a)$ for every
$a\in\frak A$.

(b) For any $u\in L^{\infty}(\frak A)$, there is a
$u'\in L^{\infty}(\frak A)$ such that $T_{\pi}u=T_{\pi}u'$ and
$\|u'\|_{\infty}=\|T_{\pi}u\|_{\infty}\le\|u\|_{\infty}$.

(c)(i) The kernel of $T_{\pi}$ is the norm-closed linear subspace of
$L^{\infty}(\frak A)$ generated by $\{\chi a:a\in\frak A,\,\pi a=0\}$.

\quad(ii) The set of values of $T_{\pi}$ is the norm-closed linear subspace
of $L^{\infty}(\frak B)$ generated by $\{\chi(\pi a):a\in\frak A\}$.

(d) $T_{\pi}$ is surjective iff $\pi$ is surjective, and in this case
$\|v\|_{\infty}=\min\{\|u\|_{\infty}:T_{\pi}u=v\}$ for every
$v\in L^{\infty}(\frak B)$.

(e) $T_{\pi}$ is injective iff $\pi$ is injective, and in this case
$\|T_{\pi}u\|_{\infty}=\|u\|_{\infty}$ for every
$u\in L^{\infty}(\frak A)$.

(f) $T_{\pi}$ is order-continuous, or sequentially order-continuous, iff
$\pi$ is.

(g) If $\frak C$ is another Boolean algebra and $\theta:\frak B\to\frak
C$ is another Boolean homomorphism, then $T_{\theta\pi}
=T_{\theta}T_{\pi}:L^{\infty}(\frak A)\to L^{\infty}(\frak C)$.

\proof{ Let $Z$ and $W$ be the Stone spaces of $\frak A$ and $\frak B$.
By 312Q there is a continuous function $\phi:W\to Z$ such that
$\widehat{\pi a}=\phi^{-1}[\widehat a]$ for every $a\in\frak A$, where
$\widehat{a}$ is the open-and-closed subset of $Z$ corresponding to
$a\in\frak A$.   Write $T$ for $T_{\pi}$.

\medskip

{\bf (a)} For $u\in L^{\infty}(\frak A)=C(Z)$, set
$Tu=u\phi:W\to\Bbb R$.   Then $Tu\in C(W)=L^{\infty}(\frak B)$.   It is
obvious, or at any rate very easy to check, that
$T:L^{\infty}(\frak A)\to L^{\infty}(\frak B)$ is linear,
multiplicative, a Riesz homomorphism and of norm $1$
unless $\frak B=\{0\}$, $W=\emptyset$.    If $a\in\frak A$, then

\Centerline{$T(\chi a)=(\chi a)\phi=(\chi\widehat a)\phi
=\chi(\phi^{-1}[\widehat a])=\chi(\pi a)$,}

\noindent identifying $\chi a\in L^{\infty}(\frak A)$ with the
indicator function $\chi\widehat a:Z\to\{0,1\}$ of the set
$\widehat a$.   Of course $T_{\pi}=T$ is the only continuous linear
operator with these properties, by 363Ea.

\medskip

{\bf (b)} Set $\alpha=\|Tu\|_{\infty}$,
$u'(z)=\med(-\alpha,u(z),\alpha)$ for $z\in Z$;  that is,
$u'=\med(-\alpha e,u,\alpha e)$ in $L^{\infty}(\frak A)$, where
$e$ is the standard order unit of $L^{\infty}(\frak A)$.   Then $Te$ is
the standard order unit of $L^{\infty}(\frak B)$, so

\Centerline{$Tu'=\med(-\alpha Te,Tu,\alpha Te)=Tu$}

\noindent (because $T$ is a lattice homomorphism, see 3A1Ic), while

\Centerline{$\|u'\|_{\infty}\le\alpha=\|Tu\|_{\infty}=\|Tu'\|_{\infty}
\le\|u'\|_{\infty}\le\|u\|_{\infty}$.}

\medskip

{\bf (c)(i)} Let $U$ be the closed linear subspace of
$L^{\infty}(\frak A)$ generated by $\{\chi a:\pi a=0\}$, and $U_0$ the
kernel of $T$.
Because $T$ is continuous and linear, $U_0$ is a closed linear subspace,
and $T(\chi a)=\chi 0=0$ whenever $\pi a=0$;  so $U\subseteq U_0$.   Now
take any $u\in U_0$ and $\epsilon>0$.   Then $T(u^+)=(Tu)^+=0$, so
$u^+\in U_0$.   By 354I there is a $u'\in S(\frak A)$ such that
$0\le u'\le u^+$ and $\|u^+-u'\|_{\infty}\le\epsilon$.   Now
$0\le Tu'\le Tu^+=0$, so $Tu'=0$.   Express $u'$ as
$\sum_{i=0}^n\alpha_i\chi a_i$ where
$\alpha_i\ge 0$ for each $i$.   For each $i$,
$\alpha_i\chi(\pi a_i)=T(\alpha_i\chi a_i)=0$, so $\pi a_i=0$ or
$\alpha_i=0$;  in either
case $\alpha_i\chi a_i\in U$.   Consequently $u'\in U$.   As $\epsilon$
is arbitrary and $U$ is closed, $u^+\in U$.   Similarly,
$u^-=(-u)^+\in U$ and $u=u^+-u^-\in U$.   As $u$ is arbitrary,
$U_0\subseteq U$ and $U_0=U$.

\medskip

\quad{\bf (ii)} Let $V$ be the closed linear subspace of
$L^{\infty}(\frak B)$ generated by $\{\chi(\pi a):a\in\frak A\}$, and
$V_0=T[L^{\infty}(\frak A)]$.   Then $T[S(\frak A)]\subseteq V$, so

\Centerline{$V_0
=T[\overline{S(\frak A)}]
\subseteq\overline{T[S(\frak A)]}
\subseteq\overline{V}
=V$.}

\noindent On the other hand, $V_0$ is a closed linear subspace in
$L^{\infty}(\frak B)$.   \Prf\ It is a linear
subspace because $T$ is a linear operator.   To see that it is closed,
take any $v\in\overline{V}_0$.   Then there is a sequence
$\sequencen{v_n}$ in $V_0$ such that
$\|v-v_n\|_{\infty}\le 2^{-n}$ for every $n\in\Bbb N$.
Choose $u_n\in L^{\infty}(\frak A)$ such that $Tu_0=v_0$, while
$Tu_n=v_n-v_{n-1}$ and $\|u_n\|_{\infty}=\|v_n-v_{n-1}\|_{\infty}$ for
$n\ge 1$ (using (b) above).   Then

\Centerline{$\sum_{n=1}^{\infty}\|u_n\|_{\infty}
\le\sum_{n=1}^{\infty}\|v-v_n\|_{\infty}+\|v-v_{n-1}\|_{\infty}$}

\noindent is finite, so $u=\lim_{n\to\infty}\sum_{i=0}^nu_i$ is defined
in the Banach space $L^{\infty}(\frak A)$, and

\Centerline{$Tu=\lim_{n\to\infty}\sum_{i=0}^nTu_i
=\lim_{n\to\infty}v_n=v$.}

\noindent As $v$ is arbitrary, $V_0$ is closed.\ \QeD\   Since
$\chi(\pi a)=T(\chi a)\in V_0$ for every $a\in\frak A$, $V\subseteq V_0$
and $V=V_0$, as required.

\medskip

{\bf (d)}  If $\pi$ is surjective, then $T$ is surjective, by (c-ii).
If $T$ is surjective and $b\in\frak B$, then there is a
$u\in L^{\infty}(\frak A)$ such that $Tu=\chi b$.   Now there is a
$u'\in S(\frak A)$ such that $\|u-u'\|_{\infty}\le\bover13$, so that
$\|Tu'-\chi b\|_{\infty}\le\bover13$.   Taking $a\in\frak A$ such that
$\{z:u'(z)\ge\bover12\}=\widehat a$, we must have $\pi a=b$, since

\Centerline{$\widehat b=\{w:(Tu')(w)\ge\bover12\}
=\phi^{-1}[\widehat a]=\widehat{\pi a}$.}

\noindent As $b$ is arbitrary, $\pi$ is surjective.

Now (b) tells us that in this case
$\|v\|_{\infty}=\min\{\|u\|_{\infty}:Tu=v\}$ for every
$v\in L^{\infty}(\frak B)$.

\medskip

{\bf (e)} By (c-i), $T$ is injective iff $\pi$ is injective.   In
this case, for any $u\in L^{\infty}(\frak A)$,

$$\eqalignno{\|Tu\|_{\infty}
&=\|T|u|\|_{\infty}\cr
\noalign{\noindent (because $T$ is a Riesz homomorphism)}
&\ge\sup\{\|Tu'\|_{\infty}:u'\in S(\frak A),\,u'\le|u|\}\cr
&=\sup\{\|u'\|_{\infty}:u'\in S(\frak A),\,u'\le|u|\}\cr
\noalign{\noindent (by 361Jd)}
&=\|u\|_{\infty}\cr
\noalign{\noindent (by 354I)}
&\ge\|Tu\|_{\infty},\cr}$$

\noindent and $\|Tu\|_{\infty}=\|u\|_{\infty}$.

\medskip

{\bf (f)} If $T$ is (sequentially) order-continuous then $\pi=T\chi$ is
(sequentially) order-continuous, by 363D.   If $\pi$ is (sequentially)
order-continuous then $\chi\pi:\frak A\to L^{\infty}(\frak B)$ is
(sequentially) order-continuous, so $T$ is (sequentially)
order-continuous, by 363Eb.

\medskip

{\bf (g)} This is elementary, in view of the uniqueness of
$T_{\theta\pi}$.
}%end of proof of 363F

\leader{363G}{Corollary}  Let $\frak A$ be a Boolean algebra.

(a) If $\frak C$ is a subalgebra of $\frak A$, then
$L^{\infty}(\frak C)$ can be identified, as Banach lattice and as Banach
algebra, with the
closed linear subspace of $L^{\infty}(\frak A)$ generated by
$\{\chi c:c\in\frak C\}$.

(b) If $\Cal I$ is an ideal of $\frak A$, then
$L^{\infty}(\frak A/\Cal I)$ can be identified, as Banach lattice and as
Banach algebra, with the
quotient space $L^{\infty}(\frak A)/V$, where $V$ is the closed linear
subspace of $L^{\infty}(\frak A)$ generated by $\{\chi a:a\in\Cal I\}$.

\proof{ Apply 363Fc-363Fd
to the identity map from $\frak C$ to $\frak A$ and
the canonical map from $\frak A$ onto $\frak A/\Cal I$.
}%end of proof of 363G

\leader{363H}{Representations of
\dvrocolon{$L^{\infty}(\frak A)$}}\cmmnt{ Much of
the importance of the concept of $L^{\infty}(\frak A)$ arises from the
way it is naturally represented in the contexts in which the most
familiar Boolean algebras appear.

\medskip

\noindent}{\bf Proposition} Let $X$ be a set and $\Sigma$ an
algebra of subsets of $X$.

(a) Write $S(\Sigma)$ for the linear subspace of $\ell^{\infty}(X)$
generated by the indicator functions of members of $\Sigma$,
and $\eusm L^{\infty}$ for its $\|\,\|_{\infty}$-closure 
in $\ell^{\infty}(X)$.

\quad(i) $L^{\infty}(\Sigma)$ can be identified, as Banach lattice and 
Banach algebra, with $\eusm L^{\infty}$;  if $E\in\Sigma$, then
$\chi E$, defined in $L^{\infty}(\Sigma)$ as in 361D, can be identified
with the indicator function of $E$ regarded as a subset of $X$.

\quad(ii) A bounded function $f:X\to\Bbb R$ belongs to $\eusm L^{\infty}$
iff whenever $\alpha<\beta$ in $\Bbb R$ there is an $E\in\Sigma$ such that
$\{x:f(x)>\beta\}\subseteq E\subseteq\{x:f(x)>\alpha\}$.

\quad(iii) In particular, $L^{\infty}(\Cal PX)$
can be identified with $\ell^{\infty}(X)$.

(b) Now suppose that $\Sigma$ is a $\sigma$-algebra of subsets of $X$.

\quad(i) $\eusm L^{\infty}$ is just the set of bounded $\Sigma$-measurable
real-valued functions on $X$.   

\quad(ii) If $\frak A$ is a Dedekind $\sigma$-complete Boolean algebra and
$\pi:\Sigma\to\frak A$ is a surjective sequentially order-continuous
Boolean homomorphism with kernel $\Cal I$, then $L^{\infty}(\frak A)$
can be identified, as Banach lattice and Banach algebra, with
$\eusm L^{\infty}/\eusm W$, where
$\eusm W=\{f:f\in\eusm L^{\infty},\,\{x:f(x)\ne 0\}\in\Cal I\}$
is a solid linear subspace and closed ideal
of $\eusm L^{\infty}$.   For $f\in\eusm L^{\infty}$,

\Centerline{$\|f^{\ssbullet}\|_{\infty}
=\min\{\alpha:\alpha\ge 0,\,\{x:|f(x)|>\alpha\}\in\Cal I\}$.}

\quad(iii) In particular, if $\Cal I$ is any $\sigma$-ideal of $\Sigma$ and
$E\mapsto E^{\ssbullet}$ is the canonical homomorphism from $\Sigma$
onto $\frak A=\Sigma/\Cal I$, then we have an identification of
$L^{\infty}(\frak A)$ with a quotient of $\eusm L^{\infty}$, and for any
$E\in\Sigma$ we can identify $\chi(E^{\ssbullet})\in L^{\infty}(\frak
A)$ with the equivalence class $(\chi E)^{\ssbullet}\in\eusm
L^{\infty}/\eusm W$ of the indicator function $\chi E$.

\proof{{\bf (a)(i)} By 361L, $S(\Sigma)$, as described here, can be
identified with $S(\Sigma)$ as defined in 361D.   Because 
the normed space $\ell^{\infty}(X)$ is complete,
$\eusm L^{\infty}$ can be identified with the 
normed space completion of $S(\Sigma)$ for
$\|\,\|_{\infty}$;  but 363C shows that the same is true of 
$L^{\infty}(\Sigma)$.   Thus we have a canonical Banach space isomorphism
between $\eusm L^{\infty}$ and $L^{\infty}(\Sigma)$.   Because
multiplication and the lattice operations are $\|\,\|_{\infty}$-continuous,
both in $\eusm L^{\infty}$ and in $L^{\infty}(\Sigma)$, this isomorphism is
multiplicative and order-preserving, that is, identifies
$\eusm L^{\infty}$ with $L^{\infty}(\Sigma)$ as Banach algebra and Banach
lattice.   In the language of 363E, $\eusm L^{\infty}$ is the image of
$L^{\infty}(\Sigma)$ in $\ell^{\infty}(X)$ 
under the operator associated with
the additive function $E\mapsto\chi E:\Sigma\to\ell^{\infty}(X)$.

\medskip

\quad{\bf (ii)}\grheada\ If $f\in\eusm L^{\infty}$ and $\alpha<\beta$ in
$\Bbb R$, let $g\in S(\Sigma)$ be such that 
$\|f-g\|_{\infty}\le\bover12(\beta-\alpha)$.   Set 
$E=\{x:g(x)>\bover12(\alpha+\beta)\}$;  by 361G or otherwise,
$E\in\Sigma$, and 
$\{x:f(x)>\beta\}\subseteq E\subseteq\{x:f(x)>\alpha\}$.

\medskip

\qquad\grheadb\ If $f$ satisfies the condition, take any $\epsilon>0$.
Let $n\in\Bbb N$ be such that $\|f\|_{\infty}<n\epsilon$.   For
$-n\le i\le n$, let $E_i\in\Sigma$ be such that
$\{x:f(x)>(i+1)\epsilon\}\subseteq E_i\subseteq\{x:f(x)>i\epsilon\}$.
Set $g(x)=\epsilon\sum_{i=-n}^n\chi E_i-\epsilon n$ for $x\in X$;  then
$g\in S(\Sigma)$ and $\|f-g\|_{\infty}\le\epsilon$.   As $\epsilon$ is
arbitrary, $f\in\eusm L^{\infty}$.

\medskip

\quad{\bf (iii)} Now (ii) shows that if $\Sigma=\Cal PX$ we shall have
$\eusm L^{\infty}=\ell^{\infty}(X)$ and $L^{\infty}(\Cal PX)$ becomes
identified with $\ell^{\infty}(X)$.

\medskip

{\bf (b)(i)} If $\Sigma$ is a $\sigma$-algebra and $f:X\to\Bbb R$ is 
bounded then

$$\eqalign{f\text{ is }\Sigma\text{-measurable}
&\iff\{x:f(x)>\alpha\}\in\Sigma\text{ for every }\alpha\in\Bbb R\cr 
&\iff\text{ whenever }\alpha\in\Bbb R,\,n\in\Bbb N
  \text{ there is an }E\in\Sigma\cr
&\mskip50mu  \text{ such that }
  \{x:f(x)>\alpha+2^{-n}\}\subseteq E\subseteq\{x:f(x)>\alpha\}\cr
&\iff\text{ whenever }\beta>\alpha\text{ there is an }E\in\Sigma\cr
&\mskip50mu  \text{ such that }\{x:f(x)>\beta\}\subseteq
  E\subseteq\{x:f(x)>\alpha\}\cr
&\iff f\in\eusm L^{\infty}\cr}$$

\noindent by (a-ii) above.

\medskip

\quad{\bf (ii)}\grheada\ 
By 363F, we have a multiplicative Riesz homomorphism
$T=T_{\pi}$ from $L^{\infty}(\Sigma)$ to $L^{\infty}(\frak A)$ which is
surjective (363Fd) and has kernel the closed linear subspace $W$ of
$L^{\infty}(\Sigma)$ generated by $\{\chi E:E\in\Cal I\}$.   Now under
the identification described in (a), $W$ corresponds to $\eusm W$.   \Prf\
$\eusm W$ is a linear subspace of $\eusm L^{\infty}$ because

\Centerline{$\{x:(f+g)(x)\ne 0\}
\subseteq\{x:f(x)\ne 0\}\cup\{x:g(x)\ne 0\}\in\Cal I$,}

\Centerline{$\{x:(\alpha f)(x)\ne 0\}\subseteq\{x:f(x)\ne 0\}\in\Cal I$}

\noindent whenever $f$, $g\in\eusm W$ and $\alpha\in\Bbb R$.   
If $\sequencen{f_n}$ is a sequence in $\eusm W$ converging to
$f\in\eusm L^{\infty}$, then

\Centerline{$\{x:f(x)\ne 0\}
\subseteq\bigcup_{n\in\Bbb N}\{x:f_n(x)\ne 0\}\in\Cal I$,}

\noindent so $f\in\eusm W$.   Thus $\eusm W$ is a closed linear subspace
of $\eusm L^{\infty}$.   If $E\in\Cal I$, then $\chi E$,
taken in $S(\Sigma)$ or $L^{\infty}(\Sigma)$, corresponds to the
function $\chi E:X\to\{0,1\}$, which belongs to $\eusm W$;  so that $W$
must correspond to the closed linear span in $\eusm L^{\infty}$ of such
indicator functions, which is a subspace of $\eusm W$.   
On the other hand, if $f\in\eusm W$ and $\epsilon>0$, set

\Centerline{$E_n=\{x:n\epsilon<f(x)\le (n+1)\epsilon\}$,
\quad$E'_n=\{x:-(n+1)\epsilon\le f(x)<-n\epsilon\}$}

\noindent for $n\in\Bbb N$;  all these belong to $\Cal I$, so
$g=\epsilon\sum_{n=0}^{\infty}(\chi E_n-\chi E'_n)\in\eusm W$ 
corresponds to a
member of $W$, while $\|f-g\|_{\infty}\le\epsilon$.   As $W$ is closed,
$f$ also must correspond to some member of $W$.   As $f$ is arbitrary,
$W$ and $\eusm W$ match exactly.\ \Qed

\medskip

\qquad\grheadb\ Because $T$ is a multiplicative Riesz homomorphism,
$L^{\infty}(\frak A)\cong L^{\infty}(\Sigma)/W$
is matched canonically, in its linear, order
and multiplicative structures, with $\eusm L^{\infty}/\eusm W$.   We
know also that

\Centerline{$\|v\|_{\infty}
=\min\{\|u\|_{\infty}:u\in L^{\infty}(\Sigma),\,Tu=v\}$}

\noindent for every $v\in L^{\infty}(\frak A)$ (363Fd), that is, that
the norm of $L^{\infty}(\frak A)$ corresponds to the quotient norm on
$L^{\infty}(\Sigma)/W$.

As for
the given formula for the norm, take any $f\in\eusm L^{\infty}$.   There
is a $g\in \eusm L^{\infty}$ such that
$Tf=Tg$ and $\|Tf\|_{\infty}
=\|g\|_{\infty}$.   (Here I am treating $T$ as an operator from
$\eusm L^{\infty}$ onto $L^{\infty}(\frak A)$.)   In this case

\Centerline{$\{x:|f(x)|>\|Tf\|_{\infty}\}
\subseteq\{x:f(x)\ne g(x)\}\in\Cal I$.}

\noindent On the other hand, if $\alpha\ge 0$ and
$\{x:|f(x)|>\alpha\}\in\Cal I$, and we
set $h=\med(-\alpha\chi X,f,\alpha\chi X)$, then
$Th=Tf$, so $\|Tf\|_{\infty}\le\|h\|_{\infty}\le\alpha$.

\medskip

\quad{\bf (iii)} Put (a-i) and (ii) just above together.
}%end of proof of 363H

\leader{363I}{Corollary} Let $(X,\Sigma,\mu)$ be a measure space, with
measure algebra $\frak A$.   Then $L^{\infty}(\mu)$ can be identified,
as Banach lattice and Banach algebra, with $L^{\infty}(\frak A)$;  the
identification matches $(\chi E)^{\ssbullet}\in L^{\infty}(\mu)$ with
$\chi(E^{\ssbullet})\in L^{\infty}(\frak A)$, for every $E\in\Sigma$.

\cmmnt{\medskip

\noindent{\bf Remark} The space I called $\eusm L^{\infty}(\mu)$ in
Chapter 24 is not strictly speaking the space
$\eusm L^{\infty}\cong L^{\infty}(\Sigma)$ of 363H;  I took
$\eusm L^{\infty}(\mu)\subseteq\eusm L^0(\mu)$ to be the set of essentially
bounded, virtually measurable functions defined almost everywhere in
$X$, and in general this is larger.   But, as remarked in the notes to
\S243, $L^{\infty}(\mu)$ can equally well be regarded as a quotient of
what I there called $\eusm L^{\infty}_{\Sigma}$, which is the
$\eusm L^{\infty}$ above, because every function in
$\eusm L^{\infty}(\mu)$ is equal almost everywhere to some member of
$\eusm L^{\infty}_{\Sigma}$.
}%end of comment

\leader{363J}{Recovering the algebra $\frak A$:  Proposition}  Let
$\frak A$ be a Boolean algebra.   For $a\in\frak A$ write $V_a$ for the
solid linear subspace of $L^{\infty}(\frak A)$ generated by $\chi a$.
Then $a\mapsto V_a$ is a Boolean isomorphism between $\frak A$ and the
algebra of projection bands in $L^{\infty}(\frak A)$.

\proof{ The proof is nearly identical to that of 361K.   If
$a\in\frak A$, $u\in V_a$ and $v\in V_{1\Bsetminus a}$, then
$|u|\wedge|v|=0$ because $\chi a\wedge\chi(1\Bsetminus a)=0$;  and if
$w\in L^{\infty}(\frak A)$ then

\Centerline{$w=(w\times\chi a)+(w\times\chi(1\Bsetminus a))
\in V_a+V_{1\Bsetminus a}$}

\noindent because $|w\times\chi a|\le\|w\|_{\infty}\chi a$ and
$|w\times\chi(1\Bsetminus a)|\le\|w\|_{\infty}\chi(1\Bsetminus a)$.   So
$V_a$ and $V_{1\Bsetminus a}$ are complementary projection bands in
$L^{\infty}=L^{\infty}(\frak A)$.   Next, if $U\subseteq L^{\infty}$ is
a projection band, then $\chi 1$ is expressible as $u+v$ where $u\in U$,
$v\in U^{\perp}$;  thinking of $L^{\infty}$ as the space of continuous
real-valued functions on the Stone space $Z$ of $\frak A$, $u$ and $v$
must be the indicator functions of complementary subsets $E$, $F$
of $Z$, which must be open-and-closed, so that $E=\widehat a$,
$F=\widehat{1\Bsetminus a}$.   In this case $V_a\subseteq U$ and
$V_{1\Bsetminus a}\subseteq U^{\perp}$, so $U$ must be $V_a$ precisely.
Thus $a\mapsto V_a$ is surjective.   Finally, just as in 361K,
$a\Bsubseteq b\iff V_a\subseteq V_b$, so we have a Boolean isomorphism.
}%end of proof of 363J

\leader{363K}{Dual spaces of $L^{\infty}$\dvro{: }{}}\cmmnt{ The
questions treated in \S362 yield nothing new in the present context.
I spell out the details.

\medskip

\noindent}{\bf Proposition} Let $\frak A$ be a Boolean algebra.   Let
$M$, $M_{\sigma}$ and $M_{\tau}$ be the $L$-spaces of bounded finitely
additive functionals, bounded countably additive functionals and
completely additive functionals on $\frak A$.   Then the embedding
$S(\frak A)\embedsinto L^{\infty}(\frak A)$ induces Riesz space isomorphisms
between $S(\frak A)^{\sim}\cong M$ and
$L^{\infty}(\frak A)^{\sim}=L^{\infty}(\frak A)^*$,
$S(\frak A)^{\sim}_c\cong M_{\sigma}$
and $L^{\infty}(\frak A)^{\sim}_c$, and
$S(\frak A)^{\times}\cong M_{\tau}$ and $L^{\infty}(\frak A)^{\times}$.

\proof{ Write $S=S(\frak A)$, $L^{\infty}=L^{\infty}(\frak A)$.

\medskip

{\bf (a)}
For the identifications $S^{\sim}\cong M$, $S^{\sim}_c\cong M_{\sigma}$
and $S^{\times}\cong M_{\tau}$ see 362A.

\medskip

{\bf (b)} $L^{\infty*}=L^{\infty\sim}$ either because $L^{\infty}$ is a
Banach lattice (356Dc) or because $L^{\infty}$ has an order-unit
norm, so that a linear functional on $L^{\infty}$ is order-bounded iff
it is bounded on the unit ball.

\medskip

{\bf (c)}
If $f$ is a positive linear functional on $L^{\infty}$, then $f\restr S$
is a positive linear functional.   Because $S$ is order-dense in
$L^{\infty}$ (363C), the embedding is order-continuous (352Nb);
so if $f$ is (sequentially) order-continuous, so is $f\restr S$.
Accordingly the restriction operator $f\mapsto f\restr S$ gives maps
from $L^{\infty\sim}$ to $S^{\sim}$, $(L^{\infty})^{\sim}_c$ to
$S^{\sim}_c$ and $L^{\infty\times}$ to $S^{\times}$.   If
$f\in L^{\infty\sim}$ and $f\restr S\ge 0$, then $f(u^+)\ge 0$ for every
$u\in S$ and therefore for every $u\in L^{\infty}$, and $f\ge 0$;  so
all these restriction maps are injective positive linear operators.

\medskip

{\bf (d)} I need to show that they are surjective.

\medskip

\quad{\bf (i)} If $g\in S^{\sim}$, then $g$ is bounded on the unit ball
$\{u:u\in S,\,\|u\|_{\infty}\le 1\}$, so has an extension to a
continuous linear $f:L^{\infty}\to\Bbb R$ (2A4I);  thus
$S^{\sim}=\{f\restr S:f\in L^{\infty\sim}\}$.   This means that
$f\mapsto f\restr S$ is actually a Riesz space isomorphism between
$L^{\infty\sim}$ and $S^{\sim}$.   In particular,
$|f|\restr S=|f\restr S|$ for any $f\in L^{\infty\sim}$.

\medskip

\quad{\bf (ii)} If $f:L^{\infty}\to \Bbb R$ is a positive linear
operator and $f\restr S\in S^{\sim}_c$, let $\sequencen{u_n}$ be a
non-increasing sequence in $L^{\infty}$ with infimum $0$.   For each
$n$, $k\in\Bbb N$ there is a $v_{nk}\in S$ such that
$u_n\le v_{nk}\le u_n+2^{-k}e$, where $e$ is the standard order unit of
$L^{\infty}$ (354I, as usual);  set
$w_n=\inf_{i,k\le n}v_{ik}$;  then $\sequencen{w_n}$ is a non-increasing
sequence in $S$ with infimum $0$, so

\Centerline{$0\le\inf_{n\in\Bbb N}f(u_n)\le\inf_{n\in\Bbb N}f(w_n)=0$.}

\noindent As $\sequencen{u_n}$ is arbitrary,
$f\in(L^{\infty})^{\sim}_c$.   Consequently, for general
$f\in L^{\infty\sim}$,

\Centerline{$f\in (L^{\infty})^{\sim}_c \iff |f|\in(L^{\infty})^{\sim}_c
\iff |f\restr S|\in S^{\sim}_c\iff f\restr S\in S^{\sim}_c$,}

\noindent and the map $f\mapsto f\restr S:(L^{\infty})^{\sim}_c\to
S^{\sim}_c$ is a Riesz space isomorphism.

\medskip

\quad{\bf (iii)} Similarly, if $f\in L^{\infty\sim}$ is non-negative and
$f\restr S\in S^{\times}$, then whenever $A\subseteq L^{\infty}$ is
non-empty, downwards-directed and has infimum $0$,
$B=\{w:w\in S,\,\exists\,u\in A,\,w\ge u\}$ has infimum $0$, so
$\inf_{u\in A}f(u)\le\inf_{w\in B}f(w)\le 0$ and $f\in L^{\infty\times}$.
As in (ii), it follows that $f\mapsto f\restr S$ is a surjection from
$L^{\infty\times}$ onto $S^{\times}$.
}%end of proof of 363K

\leader{*363L}{Integration with respect to a finitely additive
functional (a)} If $\frak A$ is a Boolean algebra and
$\nu:\frak A\to\Bbb R$ is a bounded additive functional,
then\cmmnt{ by 363K} we
have a corresponding functional $f_{\nu}\in L^{\infty}(\frak A)^*$
defined by saying that $f_{\nu}(\chi a)=\nu a$ for every $a\in\frak A$.
There are contexts in which it is convenient\cmmnt{, and even
helpful,} to use
the formula $\dashint u\,d\nu$ in place of $f_{\nu}(u)$ for
$u\in L^{\infty}=L^{\infty}(\frak A)$.   \cmmnt{When doing so, we must
of course remember
that we may have lost some of the standard properties of `integration'.
But enough of our intuitions (including, for instance, the idea of
stochastic independence)
%exercise?
remain valid to make the formula a guide to interesting ideas.}
% have seen  \dashint  called the "Bartle integration operator"

\spheader 363Lb Let $M$ be the $L$-space of bounded finitely additive
functionals on $\frak A$\cmmnt{ (362B)}.   Then we have a function
$(u,\nu)\mapsto\dashint u\,d\nu:L^{\infty}\times M\to\Bbb R$.
Now this map
is bilinear.   \prooflet{\Prf\ For $\mu$, $\nu\in M$, $u$,
$v\in L^{\infty}$ and $\alpha\in\Bbb R$,

\Centerline{$\dashint u+v\,d\nu=\dashint u\,d\nu+\dashint v\,d\nu$,
\quad$\dashint\alpha u\,d\nu=\alpha\dashint u\,d\nu$}

\noindent just because $f_{\nu}$ is linear.   On the other side, we have

\Centerline{$(f_{\mu}+f_{\nu})(\chi a)
=f_{\mu}(\chi a)+f_{\nu}(\chi a)=\mu a+\nu a
=(\mu+\nu)(a)=f_{\mu+\nu}(\chi a)$}

\noindent for every $a\in\frak A$, so that $f_{\mu}+f_{\nu}$ and
$f_{\mu+\nu}$ must agree on $S(\frak A)$ and therefore on $L^{\infty}$.
But this means that
$\dashint u\,d(\mu+\nu)=\dashint u\,d\mu+\dashint u\,d\nu$.   Similarly,
$\dashint u\,d(\alpha\mu)=\alpha\dashint u\,d\mu$.\ \Qed}

\spheader 363Lc If $\nu$ is non-negative, we have
$\dashint u\,d\nu\ge 0$ whenever $u\ge 0$\cmmnt{, as in part (c) of the
proof of 363K}.   \cmmnt{Consequently, for any $\nu\in M$ and
$u\in L^{\infty}$,

$$\eqalign{|\dashint u\,d\nu|
&=|\dashint u^+\,d\nu^+-\dashint u^-d\nu^+
  -\dashint u^+d\nu^-+\dashint u^-d\nu^-|\cr
&\le\dashint u^+\,d\nu^++\dashint u^-d\nu^++\dashint u^+d\nu^-+\dashint u^-d\nu^-\cr
&=\dashint|u|d|\nu|
\le\dashint\|u\|_{\infty}\chi 1\,d|\nu|
=\|u\|_{\infty}|\nu|(1)
=\|u\|_{\infty}\|\nu\|.\cr}$$

\noindent So} $(u,\nu)\mapsto\dashint u\,d\nu$ has norm\cmmnt{ (as defined
in 253Ab)} at most $1$.   If $\frak A\ne 0$, the norm is exactly $1$.
\cmmnt{(For this we need to know that there is a $\nu\in M^+$ such
that $\nu 1=1$.   Take any $z$ in the Stone space of $\frak A$ and set
$\nu a=1$ if $z\in\widehat{a}$, $0$ otherwise.)}

\cmmnt{\spheader 363Ld We do not have any result corresponding to
B.Levi's theorem in this language, because (even if
$\nu$ is non-negative and countably additive) there is no reason to
suppose that $\sup_{n\in\Bbb N}u_n$ is defined in $L^{\infty}$ just
because $\sup_{n\in\Bbb N}\dashint u_nd\nu$ is finite.   But if $\nu$ is
countably additive and $\frak A$ is Dedekind $\sigma$-complete, we have
something corresponding to
Lebesgue's Dominated Convergence Theorem (363Yg).}

\spheader 363Le \cmmnt{One formula which we can imitate in the
present context is that of 252O, where the ordinary integral is
represented in the form

\Centerline{$\dashint fd\mu=\int_0^{\infty}\mu\{x:f(x)\ge t\}dt$}

\noindent for non-negative $f$.
In the context of general Boolean algebras, we cannot directly
represent the set $\Bvalue{f\ge t}=\{x:f(x)\ge t\}$ (though in the next
section I will show that in Dedekind $\sigma$-complete Boolean algebras
there is an effective expression of this idea, and I will use it in the
principal definition of \S365).   But what we can say is the following.}
If $\frak A$ is any Boolean algebra, and
$\nu:\frak A\to\coint{0,\infty}$ is a non-negative additive functional,
and $u\in L^{\infty}(\frak A)^+$, then

\Centerline{$\dashint u\,d\nu
=\int_0^{\infty}\sup\{\nu a:t\chi a\le u\}dt$\dvro{.}{,}}

\noindent\cmmnt{where the right-hand integral is taken with respect
to Lebesgue measure.}   \prooflet{\Prf\ (i) For $t\ge 0$ set
$h(t)=\sup\{\nu a:t\chi a\le u\}$.   Then $h$ is non-increasing and zero
for $t>\|u\|_{\infty}$, so $\int_0^{\infty}h(t)dt$ is defined in
$\Bbb R$.   If we set $h_n(t)=h(2^{-n}(k+1))$ whenever $k$, $n\in\Bbb N$
and $2^{-n}k\le t<2^{-n}(k+1)$, then $\sequencen{h_n(t)}$ is a
non-decreasing sequence which converges to $h(t)$ whenever $h$ is
continuous at $t$, which is almost everywhere (222A, or otherwise);  so
$\int_0^{\infty}h(t)dt=\lim_{n\to\infty}\int_0^{\infty}h_n(t)dt$.
Next, given $n\in\Bbb N$ and $\epsilon>0$, we can choose for each
$k\le k^*=\lfloor 2^n\|u\|_{\infty}\rfloor$ an $a_k$ such that
$2^{-n}(k+1)\chi a_k\le u$ and $\nu a_k\ge h(2^{-n}(k+1))-\epsilon$.
In this case $\sum_{k=0}^{k^*}2^{-n}\chi a_k\le u$, so

$$\eqalign{\int_0^{\infty}h_n(t)dt
&=2^{-n}\sum_{k=0}^{k^*}h(2^{-n}(k+1))
\le\|u\|_{\infty}\epsilon+2^{-n}\sum_{k=0}^{k^*}\nu a_k\cr
&=\|u\|_{\infty}\epsilon+\dashint\sum_{k=0}^{k^*}2^{-n}\chi a_kd\nu
\le\|u\|_{\infty}\epsilon+\dashint u\,d\nu.\cr}$$

\noindent As $n$ and $\epsilon$ are arbitrary,
$\int_0^{\infty}h(t)dt\le\dashint u\,d\nu$.   (ii) In the other direction,
there is for any $\epsilon>0$ a $v\in S(\frak A)$ such that
$v\le u\le v+\epsilon\chi 1$.   If we express $v$ as
$\sum_{j=0}^m\gamma_j\chi c_j$ where $c_0\Bsupseteq\ldots\Bsupseteq c_m$
and $\gamma_j\ge 0$ for every $j$ (361Ec), then we shall have
$h(t)\ge\nu c_k$ whenever $t\le\sum_{j=0}^k\gamma_j$, so

\Centerline{$\int_0^{\infty}h(t)dt\ge\sum_{k=0}^m\gamma_k\nu c_k
=\dashint v\,d\nu\ge\dashint u\,d\nu-\epsilon\nu 1$.}

\noindent As $\epsilon$ is arbitrary,
$\int_0^{\infty}h(t)dt\ge\dashint u\,d\nu$ and the two `integrals' are
equal.\ \Qed}

\cmmnt{\spheader 363Lf The formula $\dashint\,d\nu$ is especially natural
when $\frak A$ is an algebra of sets, so that $L^{\infty}$ can be
directly interpreted as a space of functions (363Ha);  better still,
when $\frak A$ is actually a $\sigma$-algebra of subsets of a set $X$,
$L^{\infty}$ can be identified with the space of bounded
$\frak A$-measurable functions on $X$, as in 363Hb.   So in such
contexts I may write $\dashint g\,d\nu$ or even $\dashint g(x)\nu(dx)$
when $g:X\to\Bbb R$ is bounded and
$\frak A$-measurable, and $\nu:\frak A\to\Bbb R$ is a bounded additive
functional.   But I will try to take care to signal any such deviation
from the normal principle that the symbol $\int$ refers to the
sequentially order-continuous integral defined in \S122 with the minor
modifications introduced in \S\S133 and 135.}

\vleader{108pt}{363M}{}\cmmnt{ Now I come to a fundamental fact underlying a
number of theorems in both this volume and the last.

\medskip

\noindent}{\bf Theorem} Let $\frak A$ be a Boolean algebra.

(a) $\frak A$ is Dedekind $\sigma$-complete iff $L^{\infty}(\frak A)$ is
Dedekind $\sigma$-complete.

(b) $\frak A$ is Dedekind complete iff $L^{\infty}(\frak A)$ is Dedekind
complete.

\proof{{\bf (a)(i)} Suppose that $\frak A$ is Dedekind
$\sigma$-complete.   By 314M, we may identify $\frak A$ with a quotient
$\Sigma/\Cal M$, where $\Cal M$ is the ideal of meager subsets of the
Stone space $Z$ of $\frak A$, and
$\Sigma=\{E\symmdiff A:E\in\Cal E,\,A\in\Cal M\}$, writing
$\Cal E=\{\widehat a:a\in\frak A\}$ for the
algebra of open-and-closed subsets of $Z$.   By 363Hb,
$L^{\infty}=L^{\infty}(\frak A)$ can be identified with
$\eusm L^{\infty}/\eusm V$, where $\eusm L^{\infty}$ is the space of
bounded $\Sigma$-measurable functions from $Z$ to $\Bbb R$, and
$\eusm V$ is the space of functions zero except on a member of $\Cal M$.

Now suppose that $\sequencen{u_n}$ is a sequence in $L^{\infty}$ with an
upper bound $u\in L^{\infty}$.   Express $u_n$, $u$ as
$f_n^{\ssbullet}$, $f^{\ssbullet}$ where $f_n$, $f\in\eusm L^{\infty}$.
Set $g(z)=\sup_{n\in\Bbb N}\min(f_n(z),f(z))$ for every $z\in Z$;  then
$g\in\eusm L^{\infty}$ (121F), so we have a corresponding member
$v=g^{\ssbullet}$ of $L^{\infty}$.   For each $n\in\Bbb N$, $u\ge u_n$
so $(f_n-f)^+\in\eusm V$,

\Centerline{$\{z:f_n(z)>g(z)\}\subseteq\{z:f_n(z)>f(z)\}\in\Cal M$}

\noindent and $v\ge u_n$.   If $w\in L^{\infty}$ and $w\ge u_n$ for
every $n$, then express $w$ as $h^{\ssbullet}$ where
$h\in\eusm L^{\infty}$;  we have $(f_n-h)^+\in\eusm V$ for every $n$, so

\Centerline{$\{z:g(z)>h(z)\}
\subseteq\bigcup_{n\in\Bbb N}\{z:f_n(z)>h(z)\}\in\Cal M$}

\noindent because $\Cal M$ is a $\sigma$-ideal, and $(g-h)^+\in\eusm V$,
so $w\ge v$.   Thus $v=\sup_{n\in\Bbb N}u_n$ in $L^{\infty}$.   As
$\sequencen{u_n}$ is arbitrary, $L^{\infty}$ is Dedekind
$\sigma$-complete (using 353G).

\medskip

\quad{\bf (ii)} Now suppose that $L^{\infty}$ is Dedekind
$\sigma$-complete, and that $A$ is a countable non-empty set in
$\frak A$.   In this case $\{\chi a:a\in A\}$ has a least upper bound
$u$ in $L^{\infty}$.   Take $v\in S(\frak A)$ such that $0\le v\le u$
and $\|u-v\|_{\infty}\le\bover13$;  set $b=\Bvalue{v>\bover13}$, as
defined in 361Eg.   If $a\in A$, then
$\|(\chi a-v)^+\|_{\infty}\le\|u-v\|_{\infty}\le\bover13$, so
$\bover23\chi a\le v$ and
$a\Bsubseteq b$.   If $c\in\frak A$ is any upper bound for $A$, then
$v\le u\le\chi c$ so $b\Bsubseteq c$.   Thus $b=\sup A$ in $\frak A$.
As $A$ is arbitrary, $\frak A$ is Dedekind $\sigma$-complete.

\medskip

{\bf (b)(i)} For the second half of this theorem I use an argument which
depends on joining the representation described in (a-i) above with the
original definition of $L^{\infty}$ in 363A.   The point is that
$C(Z)\subseteq\eusm L^{\infty}$, and for any
$f\in C(Z)=L^{\infty}(\frak A)$ its equivalence class $f^{\ssbullet}$ in
$\eusm L^{\infty}/\eusm V$
corresponds to $f$ itself.   \Prf\ Perhaps it will help to give a name
$T$ to the canonical isomorphism from $\eusm L^{\infty}/\eusm V$ to
$L^{\infty}$.   Then $V=\{f:Tf^{\ssbullet}=f\}$ is a
closed linear subspace of $C(Z)$, because $f\mapsto f^{\ssbullet}$ and
$T$ are continuous linear operators.   But if $a\in\frak A$, then
$(\widehat a)^{\ssbullet}$, the equivalence class of
$\widehat a\in\Sigma$ in $\Sigma/\Cal M$, corresponds to $a$ (see the
proof of 314M), so
$(\chi\widehat a)^{\ssbullet}\in\eusm L^{\infty}/\eusm V$
corresponds to $\chi a$;  that is,
$T(\chi\widehat a)^{\ssbullet}=\chi\widehat a$, if we identify
$\chi a\in L^{\infty}$ with $\chi\widehat a:Z\to\{0,1\}$.   So $V$
contains $\chi\widehat a$ for every $a\in\frak A$;  because $V$ is a
linear subspace, $S(\frak A)\subseteq V$;  because $V$ is closed,
$L^{\infty}\subseteq V$.\ \Qed

For a general $f\in\eusm L^{\infty}$, $g=Tf^{\ssbullet}$ must be the
unique member of $C(Z)$ such that $g^{\ssbullet}=f^{\ssbullet}$, that
is, such that $\{z:g(z)\ne f(z)\}$ is meager.

\medskip

\quad{\bf (ii)} Suppose now that $\frak A$ is actually Dedekind
complete.   In this case $Z$ is extremally disconnected (314S).
Consequently every open set belongs to $\Sigma$.   \Prf\ If $G$ is open,
then $\overline{G}$ is open-and-closed;  but $A=\overline{G}\setminus G$
is a closed set with empty interior, so is meager, and
$G=\overline{G}\symmdiff A\in\Sigma$.\ \Qed

Let $A\subseteq L^{\infty}=C(Z)$ be any
non-empty set with an upper bound in $C(Z)$.    For each $z\in Z$ set
$g(z)=\sup_{u\in A}u(z)$.   Then

\Centerline{$G_{\alpha}=\{z:g(z)>\alpha\}
=\bigcup_{u\in A}\{z:u(z)>\alpha\}$}

\noindent is open for every $\alpha\in\Bbb R$ (that is, $g$ is lower
semi-continuous).   Thus $G_{\alpha}\in \Sigma$ for every $\alpha$, so
$g\in\eusm L^{\infty}$, and $v=Tg^{\ssbullet}$ is defined in $C(Z)$.
For any $u\in A$, $g\ge u$ in $\eusm L^{\infty}$, so

\Centerline{$v=Tg^{\ssbullet}\ge Tu^{\ssbullet}=u$}

\noindent in $L^{\infty}$;  thus $v$ is an upper bound for $A$ in
$L^{\infty}$.   On the other hand, if $w$ is any upper bound for $A$ in
$L^{\infty}=C(Z)$, then surely $w(z)\ge u(z)$ for every $z\in Z$ and
$u\in A$, so $w\ge g$ and

\Centerline{$w=Tw^{\ssbullet}\ge Tg^{\ssbullet}=v$.}

\noindent This means that $v$ is the least upper bound of $A$.   As $A$
is arbitrary, $L^{\infty}$ is Dedekind complete.

\medskip

\quad{\bf (iii)} Finally, if $L^{\infty}$ is Dedekind complete, then the
argument of (a-ii), applied to arbitrary non-empty subsets $A$ of
$\frak A$, shows that $\frak A$ also is Dedekind complete.
}%end of proof of 363M

\leader{363N}{}\cmmnt{ Much of the importance of $L^{\infty}$ spaces
in the theory of Riesz spaces arises from the next result.

\medskip

\noindent}{\bf Proposition} Let $U$ be a Dedekind $\sigma$-complete
Riesz space with an order unit.   Then $U$ is isomorphic, as Riesz
space, to $L^{\infty}(\frak A)$, where $\frak A$ is the algebra of
projection bands in $U$.

\proof{{\bf (a)} By 353M, $U$ is isomorphic to a norm-dense Riesz
subspace of $C(X)$ for some compact Hausdorff space $X$;  for the rest
of this argument, therefore, we may suppose that $U$ actually is such a
subspace.

\medskip

{\bf (b)} $U=C(X)$.   \Prf\ If $g\in C(X)$ then by 354I there are
sequences $\sequencen{f_n}$, $\sequencen{f'_n}$ in $U$ such that
$f_n\le g\le g_n$ and $\|g_n-f_n\|_{\infty}\le 2^{-n}$ for every $n$.
Now $\{f_n:n\in\Bbb N\}$ has a least upper bound $f$ in $U$;  since we
must have $f_n\le f\le g_n$ for every $n$, $f=g$ and $g\in U$.\ \Qed

\medskip

{\bf (c)} Next, $X$ is zero-dimensional.   \Prf\ Suppose that
$G\subseteq X$ is open and $x\in G$.   Then there is an open set $G_1$
such that $x\in G_1\subseteq\overline{G}_1\subseteq G$ (3A3Bb).
There is an $f\in C(X)$ such that $0\le f\le\chi G_1$ and $f(x)>0$
(also by 3A3Bb);  write $H$ for $\{y:f(y)>0\}$.   Set
$g=\sup_{n\in\Bbb N}(nf\wedge\chi X)$, the supremum being taken in
$U=C(X)$.   For each
$y\in H$, we must have $g(y)\ge\min(1,nf(y))$ for every $n$, so that
$g(y)=1$.   On the other hand, if $y\in X\setminus\overline{H}$, there
is an $h\in C(X)$ such that $h(y)>0$ and $0\le
h\le\chi(X\setminus\overline{H})$;  now $h\wedge f=0$ so $h\wedge g=0$
and $g(y)=0$.   Thus $\chi H\le g\le\chi\overline{H}$.   The set
$\{y:g(y)\in\{0,1\}\}$ is closed and includes
$H\cup(X\setminus\overline{H})$ so must be the whole of $X$;  thus
$G_2=\{y:g(y)>\bover12\}=\{y:g(y)\ge\bover12\}$ is open-and-closed, and
we have

\Centerline{$x\in H\subseteq
G_2\subseteq\overline{H}\subseteq\overline{G}_1\subseteq G$.}

\noindent As $x$, $G$ are arbitrary, the set of open-and-closed subsets
of $X$ is a base for the topology of $X$, and $X$ is
zero-dimensional.\ \Qed

\medskip

{\bf (d)} We can therefore identify $X$ with the Stone space of its
algebra $\Cal E$ of open-and-closed sets (311J).   But in this case 363A
immediately identifies $U=C(X)$ with $L^{\infty}(\Cal E)$.   By 363J,
$\Cal E$ is isomorphic to $\frak A$, so $U\cong L^{\infty}(\frak A)$.
}%end of proof of 363N

\cmmnt{\medskip

\noindent{\bf Remark} Note that in part (c) of the argument above, we
have to take care over the interpretation of `sup'.   In the
space of all real-valued functions on $X$, the supremum of
$\{nf\wedge\chi X:n\in\Bbb N\}$ is just $\chi H$.   But $g$ is supposed
to be the least {\it continuous} function greater than or equal to
$nf\wedge\chi X$ for every $n$, and is therefore likely to be strictly
greater than $\chi H$, even though sandwiched between $\chi H$ and
$\chi\overline{H}$.
}%end of comment

\leader{363O}{Corollary} Let $U$ be a Dedekind $\sigma$-complete
$M$-space.   Then $U$ is isomorphic, as Banach lattice, to
$L^{\infty}(\frak A)$, where $\frak A$ is the algebra of projection
bands of $U$.

\proof{ This is merely the special case of 363N in which $U$ is known
from the start to be complete under an order-unit norm.}

\leader{363P}{Corollary} Let $U$ be any Dedekind $\sigma$-complete Riesz
space and $e\in U^+$.   Then the solid linear subspace $U_e$ of $U$
generated by $e$ is isomorphic, as Riesz space, to $L^{\infty}(\frak A)$
for some Dedekind $\sigma$-complete Boolean algebra $\frak A$;  and if
$U$ is Dedekind complete, so is $\frak A$.

\proof{ Because $U$ is Dedekind
$\sigma$-complete, so is $U_e$ (353J(a-i)).   Apply 363N to $U_e$ to see
that $U_e\cong L^{\infty}(\frak A)$
for some $\frak A$.    Because $U_e$ is Dedekind $\sigma$-complete, so
is $\frak A$, by
363Ma;  while if $U$ is Dedekind complete, so are $U_e$ and $\frak A$,
by 353J(b-i) and 363Mb.
}%end of proof of 363P

\leader{363Q}{}\cmmnt{ The next theorem will be a
striking characterization of
the Dedekind complete $L^{\infty}$ spaces as normed spaces.   As a
warming-up exercise I give a much simpler result concerning their nature
as Banach lattices.

\medskip

\noindent}{\bf Proposition} Let $\frak A$ be a Dedekind complete Boolean
algebra.   Then for any Banach lattice $U$, a linear operator $T:U\to
L^{\infty}=L^{\infty}(\frak A)$ is continuous iff it is order-bounded,
and in this case $\|T\|=\||T|\|$, where the modulus $|T|$ is taken in
$\eurm L^{\sim}(U;L^{\infty})$.

\proof{ It is generally true that order-bounded operators between Banach
lattices are continuous (355C).   If $T:U\to L^{\infty}$ is continuous,
then for any $w\in U^+$

\Centerline{$|u|\le w\Longrightarrow\|u\|\le\|w\|
\Longrightarrow \|Tu\|_{\infty}\le\|T\|\|w\|
\Longrightarrow |Tu|\le\|T\|\|w\|e e$,}

\noindent where $e$ is the standard order unit of $L^{\infty}$.
So $T$ is order-bounded.
As $L^{\infty}$ is Dedekind complete (363Mb), $|T|$ is defined in
$\eurm L^{\sim}(U;L^{\infty})$ (355Ea).   For any $w\in U$,

\Centerline{$|T||w|=\sup\{|Tu|:|u|\le|w|\}\le\|T\|\|w\|e$,}

\noindent so $\||T|(w)\|\le\|T\|\|w\|$;  accordingly $\||T|\|\le\|T\|$.
On the other hand, of course,

\Centerline{$|Tw|\le|T||w|\le\||T|\|\|w\|e$}

\noindent for every $w\in U$, so $\|T\|\le\||T|\|$ and the two norms are
equal.
}%end of proof of 363Q

\cmmnt{\medskip

\noindent{\bf Remark} Of course what is happening here is that the
spaces $L^{\infty}(\frak A)$, for Dedekind complete $\frak A$, are just
the Dedekind complete $M$-spaces;  this is an elementary consequence of
363N and 363M.}%end of comment

\leader{363R}{}\cmmnt{ Now for something much deeper.

\medskip

\noindent}{\bf Theorem} Let $U$ be a normed space over $\Bbb R$.   Then
the following are equiveridical:

(i)  there is a Dedekind complete Boolean algebra $\frak A$ such that
$U$ is isomorphic, as normed space, to $L^{\infty}(\frak A)$;

(ii) whenever $V$ is a normed space, $V_0$ a linear subspace of $V$, and
$T_0:V_0\to U$ is a bounded linear operator, there is an extension of
$T_0$ to a bounded linear operator $T:V\to U$ with $\|T\|=\|T_0\|$.

\proof{ For the purposes of the argument below, let us say that a normed
space $U$ satisfying the condition (ii) has the `Hahn-Banach
property'.

\medskip

{\bf Part A: (i)$\Rightarrow$(ii)} I have to show that $L^{\infty}(\frak
A)$ has the Hahn-Banach property for every Dedekind complete Boolean
algebra $\frak A$.   Let $V$ be a normed space, $V_0$ a linear subspace
of $V$, and $T_0:V_0\to L^{\infty}=L^{\infty}(\frak A)$ a bounded linear
operator.   Set $\gamma=\|T_0\|$.

Let $\frak P$ be the set of all functions $T$ such that $\dom T$ is a
linear subspace of $V$ including $V_0$ and $T:\dom T\to U$ is a bounded
linear operator extending $T_0$ and with norm at most $\gamma$.   Order
$\frak P$ by saying that $T_1\le T_2$ if $T_2$ extends $T_1$.   Then any
non-empty totally ordered subset $\frak Q$ of $\frak P$ has an upper
bound in $\frak P$.   \Prf\ Set $\dom T=\bigcup\{\dom T_1:T_1\in\frak
Q\}$, $Tv=T_1v$ whenever $T_1\in\frak Q$ and $v\in\dom T_1$;  it is
elementary to check that $T\in\frak P$, so that $T$ is an upper bound
for $\frak Q$ in $\frak P$.\ \Qed

By Zorn's Lemma, $\frak P$ has a maximal element $\tilde T$.   Now
$\dom\tilde T=V$.   \Prf\Quer\ Suppose, if possible, otherwise.   Write
$\tilde V=\dom \tilde T$ and take any $\tilde v\in V\setminus \tilde V$;
let $V_1$ be the linear span of $\tilde V\cup\{\tilde v\}$, that is,
$\{v+\alpha\tilde v:v\in \tilde V,\,\alpha\in\Bbb R\}$.

If $v_1$, $v_2\in \tilde V$ then, writing $e$ for the standard order
unit of $L^{\infty}$,

$$\eqalign{\tilde Tv_1+\tilde Tv_2
&=\tilde T(v_1+v_2)
\le\|\tilde T(v_1+v_2)\|_{\infty}e\cr
&\le\gamma\|v_1+v_2\|e
\le\gamma\|v_1-\tilde v\|e+\gamma\|v_2+\tilde v\|e,\cr}$$

\noindent so

\Centerline{$\tilde Tv_1-\gamma\|v_1-\tilde v\|e\le\gamma\|v_2+\tilde
v\|e-\tilde Tv_2$.}

\noindent Because $L^{\infty}$ is Dedekind complete (363Mb),

\Centerline{$\tilde u
=\sup_{v_1\in \tilde V}\tilde Tv_1-\gamma\|v_1-\tilde v\|e$}

\noindent is defined in $L^{\infty}$ and $\tilde u\le\gamma\|v_2+\tilde
v\|e-Tv_2$
for every $v_2\in \tilde V$.   Putting these together, we have

\Centerline{$\tilde Tv+\tilde u\le\gamma\|v+\tilde v\|e$,
\quad $\tilde Tv-\tilde u\le\gamma\|v-\tilde v\|e$}

\noindent for all $v\in \tilde V$.   Consequently, if $v\in \tilde V$,
then for $\alpha>0$

\Centerline{$\tilde Tv+\alpha \tilde u
=\alpha(\tilde T(\bover1{\alpha}v)+\tilde u)
\le\alpha\gamma\|\bover1{\alpha}v+\tilde v\|e
=\gamma\|v+\alpha \tilde v\|e$,}

\noindent while for $\alpha<0$

\Centerline{$\tilde Tv+\alpha \tilde u
=|\alpha|(\tilde T(-\bover1{\alpha}v)-\tilde u)
\le|\alpha|\gamma\|-\bover1{\alpha}v-\tilde v\|e
=\gamma\|v+\alpha \tilde v\|e$,}

\noindent and of course

\Centerline{$\tilde Tv\le\|\tilde Tv\|_{\infty}e\le\gamma\|v\|e$.}

\noindent So we have

\Centerline{$\tilde Tv+\alpha \tilde u\le\gamma\|v+\alpha \tilde v\|e$}

\noindent for every $v\in \tilde V$, $\alpha\in\Bbb R$.

Define $T_1:V_1\to L^{\infty}$ by setting $T_1(v+\alpha \tilde v)
=\tilde Tv+\alpha
\tilde u$ for every $v\in \tilde V$, $\alpha\in\Bbb R$.   (This is
well-defined because $\tilde v\notin\tilde V$, so any member of $V_1$ is
uniquely expressible as $v+\alpha\tilde v$ where $v\in\tilde V$ and
$\alpha\in\Bbb R$.)    Then $T_1$ is a linear
operator, extending $T_0$, from a linear subspace of $V$ to
$L^{\infty}$.
But from the calculations above we know that $T_1v\le\gamma\|v\|e$ for
every $v\in V_1$;  since we also have

\Centerline{$T_1v=-T_1(-v)\ge-\gamma\|-v\|e=-\gamma\|v\|e$,}

\noindent $\|T_1v\|_{\infty}\le\gamma\|v\|$ for every $v\in V_1$, and
$T_1\in\frak P$.   But now $T_1$ is a member of $\frak P$ properly
extending $\tilde T$, which is supposed to be impossible.\ \Bang\Qed

Accordingly $\tilde T:V\to L^{\infty}$ is an extension of $T_0$ to the
whole of $V$, with the same norm as $T_0$.   As $V$ and $T_0$ are
arbitrary, $L^{\infty}$ has the Hahn-Banach property.

\medskip

{\bf Part B: (ii)$\Rightarrow$(i)} Now let $U$ be a normed space with
the Hahn-Banach property.   If $U=\{0\}$ then of course it is isomorphic
to $L^{\infty}(\frak A)$, where $\frak A=\{0\}$, so henceforth I will
take it for granted that $U\ne\{0\}$.

\medskip

{\bf (a)} Let $Z$ be the unit ball of the dual $U^*$ of $U$, with the
weak* topology.   Then $Z$ is a compact Hausdorff space (3A5F).
For $u\in U$ set $Z_u=\{z:z\in Z,\,|z(u)|=\|u\|\}$;  then $Z_u$ is a
closed subset of $Z$ (because $f\mapsto f(u)$ is continuous), and is
non-empty, by the Hahn-Banach theorem (3A5Ab, or Part A above!)
Now let $\frak P$ be the set of those closed sets $X\subseteq Z$ such
that $X\cap Z_u\ne\emptyset$ for every $u\in U$.   If
$\frak Q\subseteq\frak P$ is non-empty and totally ordered, then
$\bigcap\frak Q\in\frak P$, because for any $u\in U$

\Centerline{$\{X\cap Z_u:X\in\frak Q\}$}

\noindent is a downwards-directed family of non-empty compact sets, so
must have non-empty intersection.   By Zorn's Lemma, upside down,
$\frak P$ has a minimal element $X$;  with its relative topology, $X$ is a
compact Hausdorff space.

\medskip

{\bf (b)} We have a linear operator $R:U\to C(X)$ given by setting
$(Ru)(x)=x(u)$ for every $u\in U$, $x\in X$;  because $X\subseteq Z$,
$\|Ru\|_{\infty}\le\|u\|$, and because $X\in\frak P$,
$\|Ru\|_{\infty}=\|u\|$, for every $u\in U$.   Moreover, if
$G\subseteq X$ is a non-empty open set (in the relative topology of $X$)
then
$X\setminus G$ cannot belong to $\frak P$, because $X$ is minimal, so
there is a (non-zero) $u\in U$ such that $|x(u)|<\|u\|$ for every
$x\in X\setminus G$.   Replacing $u$ by $\|u\|^{-1}u$ if need be, we may
suppose that $\|u\|=1$.

What this means is that $W=R[U]$ is a linear subspace of $C(X)$ which is
isomorphic, as normed space, to $U$, and has the property that whenever
$G\subseteq X$ is a non-empty relatively open set there is an $f\in W$
such that $\|f\|_{\infty}=1$ and $|f(x)|<1$ for every $x\in X\setminus G$.
Observe that, because $X\setminus G$ is
compact, there is now some $\alpha<1$ such that $|f(u)|\le\alpha$ for
every $f\in X\setminus G$.

Because $W$ is isomorphic to $U$, it has the Hahn-Banach property.

\medskip

{\bf (c)} Now consider $V=\ell^{\infty}(X)$, $V_0=W$,
$T_0:V_0\to W$ the identity map.   Because $W$ has the Hahn-Banach
property, there is a linear operator $T:\ell^{\infty}(X)\to W$,
extending $T_0$, and of norm $\|T_0\|=1$.

\medskip

{\bf (d)} If $h\in\ell^{\infty}(X)$ and $x_0\in
X\setminus\overline{\{x:h(x)\ne 0\}}$, then $(Th)(x_0)=0$.   \Prf\Quer\
Otherwise, set
$G=\{y:y\in X\setminus\overline{\{x:h(x)\ne 0\}},\,(Th)(y)\ne 0\}$.
This is a non-empty open set in $X$, so there
are $f\in W$, $\alpha<1$ such that $\|f\|_{\infty}=1$
and $|f(x)|\le\alpha$ for every $x\in X\setminus G$.

Because $\|f\|_{\infty}=1$, there must be some $x_1\in X$ such that
$|f(x_1)|=1$,
and of course $x_1\in G$, so that $(Th)(x_1)\ne 0$.   But let
$\delta>0$ be such that $\delta\|h\|_{\infty}\le 1-\alpha$.   Then,
because $h(x)=0$ for $x\in G$,
$|f(x)|+|\delta h(x)|\le 1$
for every $x\in X$, and $\|f+\delta h\|_{\infty}$,
$\|f-\delta h\|_{\infty}$ are both less than or equal to $1$.   As
$Tf=f$ and $\|T\|=1$,
this means that

\Centerline{$\|f+\delta Th\|_{\infty}\le 1$,
\quad $\|f-\delta Th\|_{\infty}\le 1$;}

\noindent consequently

\Centerline{$|f(x_1)|+\delta|(Th)(x_1)|
=\max(|(f+\delta Th)(x_1)|,|(f-\delta Th)(x_1)|)\le 1$.}

\noindent But $|f(x_1)|=1$ and $\delta(Th)(x_1)\ne 0$, so this is
impossible.\ \Bang\Qed

\medskip

{\bf (e)} It follows that $Th=h$ for every $h\in C(X)$.   \Prf\Quer\
Suppose, if possible, otherwise.   Then there is a $\delta>0$ such that
$G=\{x:|(Th)(x)-h(x)|>\delta\}$ is not empty.   Let $f\in W$ be such
that $\|f\|=1$ but $|f(x)|<1$ for every $x\in X\setminus G$.   Then
there is an $x_0\in X$ such that $|f(x_0)|=1$;  of course $x_0$ must
belong to $G$.   Set $f_1=\Bover{h(x_0)}{f(x_0)}f$, so that $f_1\in W$
and $f_1(x_0)=h(x_0)$.   Set

\Centerline{$h_1(x)=\med(h(x)-\delta,f_1(x),h(x)+\delta)$}

\noindent for $x\in X$.   Then $h_1\in C(X)$.   Setting

\Centerline{$H=\{x:|h(x)-h(x_0)|+|f_1(x)-f_1(x_0)|<\delta\}$,}

\noindent $H$ is an open set containing $x_0$ and

\Centerline{$|f_1(x)-h(x)|\le|f_1(x_0)-h(x_0)|+\delta=\delta$,
\quad $h_1(x)=f_1(x)$}

\noindent for
every $x\in H$.   Consequently
$x_0\notin\overline{\{x:(f_1-h_1)(x)\ne 0\}}$, and $T(f_1-h_1)(x_0)=0$,
by (d).   But this means that

\Centerline{$(Th_1)(x_0)=(Tf_1)(x_0)=f_1(x_0)=h(x_0)$,}

\noindent so that

\Centerline{$|h(x_0)-(Th)(x_0)|=|T(h_1-h)(x_0)|
\le\|T(h_1-h)\|_{\infty}\le\|h_1-h\|_{\infty}\le\delta$,}

\noindent which is impossible, because $x_0\in G$.\ \Bang\Qed

\medskip

{\bf (f)} This tells us at once that $W=C(X)$.   But (d) also tells us
that $X$ is extremally disconnected.   \Prf\  Let $G\subseteq X$ be any
open set.   Then $\chi X=\chi G+\chi(X\setminus G)$, so

\Centerline{$\chi X=T(\chi X)=h_1+h_2$,}

\noindent where $h_1=T(\chi G)$, $h_2=T(\chi(X\setminus G))$.   Now
from (d) we see that $h_1$
must be zero on $X\setminus\overline{G}$ while $h_2$ must be zero on
$G$.   Thus we have $h_1(x)=1$ for $x\in G$;  as $h_1$ is continuous,
$h_1(x)=1$ for $x\in\overline{G}$, and $h_1=\chi\overline G$.   Of
course it follows that $\overline{G}$ is open.   As $G$ is arbitrary,
$X$ is extremally disconnected.\ \Qed

\medskip

{\bf (g)} Being
also compact and Hausdorff, therefore regular (3A3Bb), $X$ is
zero-dimensional (3A3Bd).   We may therefore identify $X$
with the Stone space of its regular open algebra $\RO(X)$ (314S),
and $W=C(X)$ with $L^{\infty}(\RO(X))$.
Thus $R:U\to C(X)$ is a Banach space isomorphism between $U$ and
$C(X)\cong L^{\infty}(\RO(X))$;  so $U$ is of the type declared.
}%end of proof of 363R

\leader{363S}{The Banach-Ulam \dvrocolon{problem}}\cmmnt{ At a couple
of points already (232Hc, the notes to \S326) I have remarked on a
problem which was early recognised as a fundamental question in abstract
measure theory.   I now set out some formulations of the problem which
arise naturally from the work done so far.   I will do this by writing
down a list of equiveridical statements;  the `Banach-Ulam problem'
asks whether they are true.

I should remark that this is not generally counted as an `open' problem.
It is in fact believed by most of us that these statements are
independent of the usual axioms of Zermelo-Fraenkel set theory,
including the axiom of choice and even the continuum hypothesis.   As
such, this problem belongs to Volume 5 rather than anywhere earlier, but
its manifestations will become steadily more obtrusive as we continue
through this volume and the next, and I think it will be helpful to
begin collecting them now.
The ideas needed to show that the statements here imply each other are
already accessible;  in particular, they involve no set theory beyond
Zorn's Lemma.   These implications constitute the following theorem,
derived from {\smc Luxemburg 67a}.

\medskip

\noindent}{\bf Theorem} The following statements are equiveridical.

(i) There are a set $X$ and a probability measure $\nu$, with domain
$\Cal PX$, such that $\nu\{x\}=0$ for every $x\in X$.

(ii) There are a localizable measure space $(X,\Sigma,\mu)$ and an
absolutely continuous countably additive functional
$\nu:\Sigma\to\Bbb R$ which is not truly continuous, so has no
Radon-Nikod\'ym derivative\cmmnt{ (definitions:  232Ab, 232Hf)}.

(iii) There are a Dedekind complete Boolean algebra $\frak A$ and a
countably additive functional $\nu:\frak A\to\Bbb R$ which is not
completely additive.

(iv) There is a Dedekind complete Riesz space $U$ such that
$U^{\sim}_c\ne U^{\times}$.

\proof{{\bf (a)(i)$\Rightarrow$(ii)} Let $X$ be a set with a probability
measure $\nu$, defined on $\Cal PX$, such that $\nu\{x\}=0$ for every
$x\in X$.   Let $\mu$ be counting measure on $X$.   Then $(X,\Cal
PX,\mu)$ is strictly localizable, and $\nu:\Cal PX\to\Bbb R$ is
countably additive;  also $\nu E=0$ whenever $\mu E$ is finite, so $\nu$
is absolutely continuous with respect to $\mu$.   But if $\mu E<\infty$
then $E$ is finite and $\nu(X\setminus E)=1$, so $\nu$ is not truly
continuous, and has no Radon-Nikod\'ym derivative (232D).

\medskip

{\bf (b)(ii)$\Rightarrow$(iii)} Let $(X,\Sigma,\mu)$ be a localizable
measure space and $\nu:\Sigma\to\Bbb R$ an absolutely continuous
countably additive functional which is not truly continuous.   Let
$(\frak A,\bar\mu)$ be the measure algebra of $\mu$;  then we have an
absolutely continuous countably additive functional
$\bar\nu:\frak A\to\Bbb R$ defined by setting
$\bar\nu E^{\ssbullet}=\nu E$ for every
$E\in\Sigma$ (327C).   Since $\nu$ is not truly continuous, $\bar\nu$ is
not completely additive (327Ce).   Also $\frak A$ is Dedekind complete,
because $\mu$ is localizable, so $\frak A$ and $\bar\nu$ witness (iii).

\medskip

{\bf (c)(iii)$\Rightarrow$(i)} Let $\frak A$ be a Dedekind complete
Boolean algebra and $\nu:\frak A\to\Bbb R$ a countably additive
functional which is not completely additive.   Because $\nu$ is bounded
(326M), therefore expressible as the difference of non-negative
countably additive functionals (326L), there must be a non-negative
countably additive functional $\nuprime$ on $\frak A$ which is not
completely additive.

By 326R, there is a partition of unity $\familyiI{a_i}$ in $\frak A$
such that $\sum_{i\in I}\nuprime a_i<\nuprime1$.   Set
$K=\{i:i\in I,\,\nuprime a_i>0\}$;  then $K$ must be countable, so

\Centerline{$\nuprime(\sup_{i\in I\setminus K}a_i)
=\nuprime1-\nuprime(\sup_{i\in K}a_i)
=\nuprime1-\sum_{i\in K}\nuprime a_i
>0$.}

\noindent For $J\subseteq I$ set
$\mu J=\nuprime(\sup_{i\in J\setminus K}a_i)$;  the supremum is always
defined because $\frak A$ is Dedekind complete.   Because $\nuprime$ is
countably additive and non-negative, so is $\mu$;  because $\nuprime a_i=0$
for $i\in J\setminus K$, $\mu\{i\}=0$ for every $i\in I$.   Multiplying
$\mu$ by a suitable scalar, if need be, $(I,\Cal PI,\mu)$ witnesses that
(i) is true.

\medskip

{\bf (d)(iii)$\Rightarrow$(iv)} If $\frak A$ is a Dedekind complete
Boolean algebra with a countably additive functional which is not
completely additive, then $U=L^{\infty}(\frak A)$ is a Dedekind complete
Riesz space (363Mb) and $U^{\sim}_c\ne U^{\times}$, by 363K (recalling,
as in (c) above, that the functional must be bounded).

\medskip

{\bf (e)(iv)$\Rightarrow$(iii)} Let $U$ be a Dedekind complete Riesz
space such that $U^{\times}\ne U^{\sim}_c$.   Take
$f\in U^{\sim}_c\setminus U^{\times}$;  replacing $f$ by $|f|$ if need be,
we may suppose that $f\ge 0$ is sequentially order-continuous but not
order-continuous (355H, 355I).   Let $A$ be a non-empty
downwards-directed set in $U$, with infimum $0$, such that
$\inf_{u\in A}f(u)>0$ (351Ga).   Take $e\in A$, and consider the solid
linear subspace $U_e$ of $U$ generated by $e$;  write $g$ for the
restriction of $f$ to $U_e$.   Because the embedding of $U_e$ in $U$ is
order-continuous, $g\in (U_e)^{\sim}_c$;  because $A\cap U_e$ is
downwards-directed and has infimum $0$, and

\Centerline{$\inf_{u\in A\cap U_e}g(u)=\inf_{u\in A}f(u)>0$,}

\noindent $g\notin U_e^{\times}$.   But $U_e$ is a Riesz space with
order unit $e$, and is Dedekind complete because $U$ is;  so it
can be identified with $L^{\infty}(\frak A)$ for some Boolean algebra
$\frak A$ (363N), and $\frak A$ is Dedekind complete, by 363M.

Accordingly we have a Dedekind complete Boolean algebra $\frak A$ such
that $L^{\infty}(\frak A)^{\sim}_c\ne L^{\infty}(\frak A)^{\times}$.
By 363K, there is a (bounded) countably additive functional on $\frak A$
which is not completely additive, and (iii) is true.
}%end of proof of 363S

\exercises{\leader{363X}{Basic exercises (a)}
%\spheader 363Xa
Let $\frak A$ be a Boolean algebra and $U$ a Banach algebra.   Let
$\nu:\frak A\to U$ be a bounded additive function and
$T:L^{\infty}(\frak A)\to U$ the corresponding bounded linear operator.
Show that $T$ is multiplicative iff $\nu(a\Bcap b)=\nu a\times\nu b$ for
all $a$, $b\in\frak A$.
%363E

\sqheader 363Xb Let $\frak A$, $\frak B$ be Boolean algebras and
$T:L^{\infty}(\frak A)\to L^{\infty}(\frak A)$ a linear operator.
Show that the following are equiveridical:  (i) there is a Boolean
homomorphism $\pi:\frak A\to\frak A$ such that $T=T_{\pi}$ (ii)
$T(u\times v)=Tu\times Tv$ for all $u$, $v\in L^{\infty}(\frak A)$ (iii)
$T$ is a Riesz homomorphism and $Te_{\frak A}=e_{\frak B}$, where
$e_{\frak A}$ is the standard order unit of $L^{\infty}(\frak A)$.
%363F

\spheader 363Xc Let $\frak A$, $\frak B$ be Boolean algebras and
$T:L^{\infty}(\frak A)\to L^{\infty}(\frak B)$ a Riesz homomorphism.
Show that there are a Boolean homomorphism $\pi:\frak A\to\frak B$ and a
$v\ge 0$ in $L^{\infty}(\frak B)$ such that $Tu=v\times T_{\pi}u$ for
every $u\in L^{\infty}(\frak A)$, where $T_{\pi}$ is the operator
associated with $\pi$ (363F).
%363F, 363Xb


\spheader 363Xd Let $\frak A$ be a Boolean algebra and $\frak C$ a
subalgebra of $\frak A$.   Show that $L^{\infty}(\frak C)$, regarded as
a subspace of $L^{\infty}(\frak A)$ (363Ga), is order-dense in
$L^{\infty}(\frak A)$ iff $\frak C$ is order-dense in $\frak A$.
%363G

\sqheader 363Xe Let $(X,\Sigma,\mu)$ be a measure space with measure
algebra $\frak A$, and $\eusm L^{\infty}$ the space of bounded
$\Sigma$-measurable real-valued functions on $X$.  A {\bf linear
lifting} of $\mu$ is a positive linear operator
$T:L^{\infty}(\frak A)\to\eusm L^{\infty}$ such that
$T(\chi 1_{\frak A})=\chi X$ and
$(Tu)^{\ssbullet}=u$ for every $u\in L^{\infty}(\frak A)$, writing
$f\mapsto f^{\ssbullet}$ for the canonical map from $\eusm L^{\infty}$
to $L^{\infty}(\frak A)$ (363H-363I).   (i) Show that if
$\theta:\frak A\to\Sigma$ is a lifting in the sense of 341A then
$T_{\theta}$, as
defined in 363F, is a linear lifting.   (ii) Show that if
$T:L^{\infty}(\frak A)\to\eusm L^{\infty}$ is a linear lifting, then
there is a corresponding lower density
$\underline{\theta}:\frak A\to\Sigma$ defined by setting
$\underline{\theta}a=\{x:T(\chi a)(x)=1\}$ for each $a\in\frak A$.
(iii) Show that
$\underline{\theta}$, as defined in (ii), is a lifting iff $T$ is a
Riesz homomorphism iff $T$ is multiplicative.
%363I

\spheader 363Xf Let $U$ be any commutative ring with
multiplicative identity $1$.   Show that the set $A$ of
{\bf idempotents} in $U$ (that is, elements $a\in U$ such that $a^2=a$)
is a Boolean algebra with identity $1$, writing $a\Bcap b=ab$,
$1\Bsetminus a=1-a$ for $a$, $b\in A$.
%363J

\spheader 363Xg Let $\frak A$ be a Boolean algebra.   Show that
$\frak A$ is isomorphic to the Boolean algebras of multiplicative
idempotents of $S(\frak A)$ and $L^{\infty}(\frak A)$.
%363J, 363Xf

\spheader 363Xh Let $\frak A$ be a Dedekind $\sigma$-complete Boolean
algebra.  (i) Show that for any $u\in L^{\infty}(\frak A)$,
$\alpha\in\Bbb R$ there are elements $\Bvalue{u\ge\alpha}$,
$\Bvalue{u>\alpha}\in\frak A$, where $\Bvalue{u\ge\alpha}$ is the
largest $a\in\frak A$ such that $u\times\chi a\ge\alpha\chi a$, 
and $\Bvalue{u>\alpha}=\sup_{\beta>\alpha}\Bvalue{u\ge\beta}$.
(ii) Show that in the context of 363Hb, if $u$ corresponds to
$f^{\ssbullet}$ for $f\in\eusm L^{\infty}$, then
$\Bvalue{u\ge\alpha}=\{x:f(x)\ge\alpha\}^{\ssbullet}$,
$\Bvalue{u>\alpha}=\{x:f(x)>\alpha\}^{\ssbullet}$.   (iii) Show that if
$A\subseteq L^{\infty}$ is non-empty and $v\in L^{\infty}$, then $v=\sup
A$ iff $\Bvalue{v>\alpha}=\sup_{u\in A}\Bvalue{u>\alpha}$ for every
$\alpha\in\Bbb R$;  in particular, $v=u$ iff
$\Bvalue{v>\alpha}=\Bvalue{u>\alpha}$ for every $\alpha$. (iv) Show that
a function $\phi:\Bbb R\to\frak A$ is of the form
$\phi(\alpha)=\Bvalue{u>\alpha}$ iff ($\alpha$)
$\phi(\alpha)=\sup_{\beta>\alpha}\phi(\beta)$ for every $\alpha\in\Bbb
R$ ($\beta$) there is an $M$ such that $\phi(M)=0$, $\phi(-M)=1$.   (v)
Put (iii) and (iv) together to give a proof that $L^{\infty}$ is
Dedekind $\sigma$-complete if $\frak A$ is.
%363M

\spheader 363Xi Let $\frak A$ be a Dedekind $\sigma$-complete Boolean
algebra and $U\subseteq L^{\infty}(\frak A)$ a (sequentially)
order-closed Riesz subspace containing $\chi 1$.   Show that $U$ can be
identified with $L^{\infty}(\frak B)$ for some (sequentially)
order-closed subalgebra $\frak B\subseteq\frak A$.
\Hint{set $\frak B=\{b:\chi b\in U\}$ and use 363N.}
%363N

\leader{363Y}{Further exercises (a)}
%\spheader 363Ya
Let $\frak A$ be a Boolean algebra.
Given the linear structure, ordering, multiplication and norm of
$S(\frak A)$ as described in \S361, show that a norm completion of
$S(\frak A)$ will serve for $L^{\infty}(\frak A)$ in the sense that all
the results of 363B-363Q can be proved with no use of the axiom of
choice except an occasional appeal to countably many choices in
sequential forms of the theorems.
%363A

\spheader 363Yb Let $\frak A$ be a Boolean algebra.   Show that
$\frak A$ is ccc iff $L^{\infty}(\frak A)$ has the countable sup
property (241Ye, 353Ye).
%363C

\spheader 363Yc Let $X$ be an extremally disconnected topological space,
and $\RO(X)$ its regular open algebra.   Show that there is a natural
isomorphism between $L^{\infty}(\RO(X))$ and $C_b(X)$.
%363E

\spheader 363Yd Let $\frak A$ be a Boolean algebra.   (i) If
$u\in L^{\infty}=L^{\infty}(\frak A)$, show that $|u|=e$, the standard
order unit
of $L^{\infty}$, iff $\max(\|u+v\|_{\infty},\|u-v\|_{\infty})>1$
whenever $v\in L^{\infty}\setminus\{0\}$.   (ii) Show that if $u$,
$v\in L^{\infty}$ then $|u|\wedge|v|=0$ iff $\|\alpha u+v+w\|_{\infty}
\le\max(\|\alpha u+w\|_{\infty},\|v+w\|_{\infty})$ whenever
$\alpha=\pm 1$ and $w\in L^{\infty}$.
(iii) Show that if $T:L^{\infty}\to L^{\infty}$ is a normed space
automorphism then there are a Boolean automorphism
$\pi:\frak A\to\frak A$ and
a $w\in L^{\infty}$ such that $|w|=e$ and $Tu=w\times T_{\pi}u$ for
every $u\in L^{\infty}$.
%363Xb, 363F

\spheader 363Ye Let $X$ be a set, $\Sigma$ an algebra of subsets of $X$,
and $\Cal I$ an ideal in $\Sigma$, and
$\eusm L^{\infty}$ the set of bounded functions $f:X\to\Bbb R$ such that
whenever $\alpha<\beta$ in $\Bbb R$ there is an $E\in\Sigma$ such that
$\{x:f(x)\le\alpha\}\subseteq E\subseteq\{x:f(x)\le\beta\}$, as in 363H.   
(i) Show
that $\eusm L^{\infty}=\{g\phi:g\in C(Z)\}$, where $Z$ is the Stone
space of $\Sigma$ and $\phi:X\to Z$ is a function (to be described).
(ii) Show
that $L^{\infty}(\Sigma/\Cal I)$ can be identified, as Banach lattice
and Banach algebra, with $\eusm L^{\infty}/\eusm V$, where $\eusm V$ is
the set of those functions $f\in\eusm L^{\infty}$ such that for every
$\epsilon>0$ there is a member of $\Cal I$ including
$\{x:|f(x)|\ge\epsilon\}$.
%363H

\spheader 363Yf\dvAformerly{3{}63Yg} 
Let $(X,\Sigma,\mu)$ be a complete probability space
with measure algebra $\frak A$.   Let $\sequencen{\frak B_n}$ be a
non-decreasing sequence of closed subalgebras of $\frak A$ such that
$\frak A$ is the closed subalgebra of itself generated by
$\bigcup_{n\in\Bbb N}\frak B_n$, and set
$\Sigma_n=\{F:F^{\ssbullet}\in\frak B_n\}$ for each $n$.   Let
$P_n:L^1(\mu)\to L^1(\mu\restr\Sigma_n)$ be the conditional expectation
operator for each $n$, so that $P_n\restr L^{\infty}(\mu)$ is a positive
linear operator from $L^{\infty}(\mu)\cong L^{\infty}(\frak A)$ to
$L^{\infty}(\mu\restr\Sigma_n)\cong L^{\infty}(\frak B_n)$.   Suppose
that we are given for each $n$ a lifting $\theta_n:\frak B_n\to\Sigma_n$
and that $\theta_{n+1}b=\theta_nb$ whenever $n\in\Bbb N$ and
$b\in\frak B_n$.
Let $T_n:L^{\infty}(\frak B_n)\to\eusm L^{\infty}$ be the corresponding
linear liftings (363Xe), and $\Cal F$ any non-principal ultrafilter
on $\Bbb N$.   (i) Show that for any $u\in L^{\infty}(\frak A)$,
$\sequencen{T_nP_nu}$ converges almost everywhere.   (ii) For
$u\in L^{\infty}(\frak A)$ set $(Tu)(x)=\lim_{n\to\Cal F}(T_nP_nu)(x)$
for $x\in X$, $u\in L^{\infty}(\frak A)$.   Show that $T$ is a linear
lifting for $\mu$.   (iii) Use 363Xe(ii) and 341J to show that there is a
lifting $\theta$ of $\mu$ extending every $\theta_n$.   (iv) Use this as
the countable-cofinality inductive step in a proof of the Lifting
Theorem (using partial liftings rather than partial lower densities, as
suggested in 341Li).
%363I, 363Xe

\spheader 363Yg Let $\frak A$ be a Boolean algebra and
$\nu:\frak A\to\Bbb R$ a
bounded countably additive functional.   Suppose that
$\sequencen{u_n}$ is an order-bounded sequence in
$L^{\infty}(\frak A)$ such that $\inf_{n\in\Bbb N}\sup_{m\ge n}u_m$
and $\sup_{n\in\Bbb N}\inf_{m\ge n}u_m$ are defined in
$L^{\infty}(\frak A)$ and equal to $u$ say.   Show
that $\int u\,d\nu=\lim_{n\to\infty}\int u_nd\nu$.
%363L

\spheader 363Yh Let $\Sigma$ be the family of those sets
$E\subseteq[0,1]$ such that $\mu(\interior E)=\mu\overline{E}$, where
$\mu$ is Lebesgue measure.   (i) Show that $\Sigma$ is an algebra of
subsets of $[0,1]$ and that every member of $\Sigma$ is Lebesgue
measurable.   (ii) Show that if we identify $L^{\infty}(\Sigma)$ with a
set of real-valued functions on $[0,1]$, as in 363H, then we get just
the space of Riemann integrable functions.   (iii) Show that if we write
$\nu$ for $\mu\restr\Sigma$, then $\dashint d\nu$, as defined in 363L,
is just the Riemann integral.
%363L

\spheader 363Yi\dvAformerly{3{}63Yj}
Let $X$ be a compact Hausdorff
space.   Let us say that a linear subspace $U$ of $C(X)$ is
{\bf $\ell^{\infty}$-{\vthsp}complemented} in $C(X)$ if there is a linear
subspace $V$ such that $C(X)=U\oplus V$ and
$\|u+v\|_{\infty}=\max(\|u\|_{\infty},\|v\|_{\infty})$ for all $u\in U$,
$v\in\ V$.   Show that there is a one-to-one correspondence between such
subspaces $U$ and open-and-closed subsets $E$ of $X$, given by setting
$U=\{u:u\in C(X),\,u(x)=0\Forall x\in X\setminus E\}$.   Hence show
that if $\frak A$ is any Boolean algebra, there is a canonical
isomorphism between $\frak A$ and the partially ordered set of
$\ell^{\infty}$-complemented subspaces of $L^{\infty}(\frak A)$.
%363R mt36bits
}%end of exercises

\endnotes{
\Notesheader{363} As with $S(\frak A)$, I have chosen a definition of
$L^{\infty}(\frak A)$ in terms of the Stone space of $\frak A$;  but as
with $S(\frak A)$, this is optional (363Ya).   By and large the basic
properties of $L^{\infty}$ are derived very naturally from those of $S$.
The spaces $L^{\infty}(\frak A)$, for general Boolean algebras
$\frak A$, are not in fact particularly important;  they have too few
properties not shared by all the spaces $C(X)$ for compact Hausdorff
$X$.   The point at which it becomes helpful to interpret $C(X)$ as
$L^{\infty}(\frak A)$ is when $C(X)$ is Dedekind $\sigma$-complete.
The spaces $X$ for which this is true are difficult to picture,
and alternative representations of $L^{\infty}$ along the lines of
363H-363I can be easier on the imagination.

For Dedekind $\sigma$-complete $\frak A$, there is an alternative
description of members of $L^{\infty}(\frak A)$ in terms of objects
`$\Bvalue{u>\alpha}$' (363Xh);  I will return to this idea in the next
section.   For the moment I remark only that it gives an alternative
approach to 363M not necessarily depending on the representation of
$L^{\infty}$ as a quotient $\eusm L^{\infty}/\eusm V$ nor on an analysis
of a Stone space.   I used a version of such an argument in the proof of
363M which I gave in {\smc Fremlin 74a}, 43D.

I spend so much time on 363M not only because Dedekind completeness is
one of the basic properties of any lattice, but because it offers an
abstract expression of one of the central results of Chapter 24.   In
243H I showed that $L^{\infty}(\mu)$ is always Dedekind
$\sigma$-complete, and that it is Dedekind complete if $\mu$ is
localizable.   We can now relate this to the results of 321H and 322Be:
the measure algebra of any measure is Dedekind $\sigma$-complete,
and the measure algebra of a localizable measure is Dedekind complete.

The ideas of the proof of 363M can of course be rearranged in various
ways.   One uses 353Yb:  for completely regular spaces $X$, $C(X)$
is Dedekind complete iff $X$ is extremally disconnected;  while for
compact Hausdorff spaces, $X$ is extremally disconnected iff it is the
Stone space of a Dedekind complete algebra.   With the right
modification of the concept `extremally disconnected' (314Yf), the
same approach works for Dedekind $\sigma$-completeness.

363R is the `Nachbin-Kelley theorem';  it is
commonly phrased `a normed space $U$ has the
Hahn-Banach extension property iff it is isomorphic, as normed space, to
$C(X)$ for some compact extremally disconnected Hausdorff space $X$',
but the expression in terms of $L^{\infty}$ spaces seems natural in the
present context.   The implication in one direction (Part A of the
proof) calls for nothing but a check through one of the standard proofs
of the Hahn-Banach theorem to make sure that the argument applies in the
generalized form.   Part B of the proof has ideas in it;  I have tried
to set it out in a way suggesting that if you can remember the
construction of the set $X$ the rest is just a matter of a little
ingenuity.

One way of trying to understand the multiple structures of $L^{\infty}$
spaces is by looking at the corresponding automorphisms.   We observe,
for instance, that an operator $T$ from $L^{\infty}(\frak A)$ to itself
is a Banach algebra automorphism iff it is a Banach lattice automorphism
preserving the standard order unit iff it corresponds to an automorphism
of the algebra $\frak A$ (363Xb).   Of course there are Banach space
automorphisms of $L^{\infty}$ which do not respect the order or
multiplicative structure;  but they have to be closely related to
algebra isomorphisms (363Yd).

I devote a couple of exercises (363Xe, 363Yf) to indications of how the
ideas here are relevant to the Lifting Theorem.   If you found the
formulae of the proof of 341G obscure it may help to work through the
parallel argument.

A lecture by W.A.J.Luxemburg on the equivalence between (i) and (iv) in
363S was one of the turning points in my mathematical apprenticeship.
I introduce it here, even though the real importance of the Banach-Ulam
problem lies in the metamathematical ideas it has nourished, because
these formulations provide a focus for questions which arise naturally
in this volume and which otherwise might prove distracting.   The next
group of significant ideas in this context will appear in \S438.
}%end of comment

\discrpage

