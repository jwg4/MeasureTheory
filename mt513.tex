\frfilename{mt513.tex}
\versiondate{23.2.14}
\copyrightdate{2004}

\def\chaptername{Cardinal functions}
\def\sectionname{Partially ordered sets}

\newsection{513}

In \S\S511-512 I have given long
lists of definitions.   It is time I filled in details of
the most elementary relationships between the various concepts
introduced.   Here I treat some of those which can be expressed in the
language of partially ordered sets.   I begin with notes on cofinality
and saturation, with the Erd\H{o}s-Tarski theorem (513B).
In this context, Galois-Tukey connections take on particularly direct
forms (513D-513E);  for directed sets, we have an alternative definition
of Tukey equivalence (513F).   The majority of the cardinal functions
defined so far on partially ordered sets are determined by their cofinal
structure (513G, 513If\cmmnt{;  see also 516Ga below}).

In the last third of the section
(513K-513O), %513K 513L 513M 513N 513O
I discuss Tukey functions between directed sets with a special kind of
topological structure, which I call `metrizably compactly based';  the
point is
that for Polish metrizably compactly based directed sets, if there is
any Tukey function between them, there must be one
which is measurable in an appropriate sense (513O).

\leader{513A}{}\cmmnt{ It will help to have an elementary lemma on
maximal antichains.

\medskip

\noindent}{\bf Lemma} Let $P$ be a partially ordered set.

(a) If $Q\subseteq P$ is cofinal and $A\subseteq Q$ is an up-antichain,
there is a maximal up-antichain $A'$ in $P$ such that
$A\subseteq A'\subseteq Q$.   In particular, $Q$ includes a maximal
up-antichain.

(b) If $A\subseteq P$ is a maximal up-antichain,
$Q=\bigcup_{q\in A}\coint{q,\infty}$ is cofinal with $P$.

\proof{{\bf (a)} Let $A'\subseteq Q$ be maximal subject to being an
up-antichain in $P$ including $A$.   Then for any $p\in P\setminus A'$,
there are a $q\in Q$ such that $p\le q$, and an $r\in A'$ such that

\Centerline{$\emptyset\ne\coint{r,\infty}\cap\coint{q,\infty}
\subseteq\coint{r,\infty}\cap\coint{p,\infty}$,}

\noindent so $A'\cup\{p\}$ is not an up-antichain.   But this means that
$A'$ is a maximal up-antichain in $P$.

Starting from $A=\emptyset$, we see that $Q$ includes a maximal
up-antichain.

\medskip

{\bf (b)} If $p\in P$, then either $p\in A\subseteq Q$, or $A\cup\{p\}$
is not an up-antichain, so that there is some $q\in A$ such that

\Centerline{$\emptyset\ne\coint{p,\infty}\cap\coint{q,\infty}
\subseteq Q\cap\coint{p,\infty}$.}
}%end of proof of 513A

\leader{513B}{Theorem} Let $P$ be a partially ordered set.

(a) $\bu P\le\cf P\le\#(P)$.

(b) $\sat^{\uparrow}(P)$ is either finite or a regular uncountable
cardinal.

(c) $c^{\uparrow}(P)$ is the predecessor of $\sat^{\uparrow}(P)$ if
$\sat^{\uparrow}(P)$ is a successor cardinal, and otherwise is equal to
$\sat^{\uparrow}(P)$.

\proof{{\bf (a)} To see that $\cf P\le\#(P)$ all we have to note is that
$P$ is a cofinal subset of itself.   To see that $\bu P\le\cf P$, set
$\kappa=\cf P$ and let $\ofamily{\xi}{\kappa}{p_{\xi}}$ enumerate a
cofinal subset of $P$.   Set

\Centerline{$A
=\{\xi:\xi<\kappa,\,p_{\xi}\not\le p_{\eta}$ for any $\eta<\xi\}$,
\quad$Q=\{p_{\xi}:\xi\in A\}$.}

\noindent If $p\in P$ there is a least $\xi<\kappa$ such that
$p\le p_{\xi}$, and now $\xi\in A$;  so $Q$ is cofinal with $P$.   If
$p\in P$, there is some $\xi\in A$ such that $p\le p_{\xi}$, and now
$\{q:q\in Q$, $q\le p$, $p\not\le q\}\subseteq\{p_{\eta}:\eta<\xi\}$
has cardinal less than $\kappa$, so that $Q$ witnesses that
$\bu P\le\kappa$.

\medskip

{\bf (b)(i)} Set $\kappa=\sat^{\uparrow}(P)$.
For $p\in P$, set $\theta(p)=\sat^{\uparrow}(\coint{p,\infty})$.   Note
that if $p\in P$ and $B\subseteq P$ is any up-antichain, then
$B_p=\{q:q\in B$, $\coint{q,\infty}\cap\coint{p,\infty}\ne\emptyset\}$
has less than $\theta(p)$ members.   \Prf\ For $q\in B_p$, choose
$q'\in\coint{q,\infty}\cap\coint{p,\infty}$.   Because $B$ is an
up-antichain, $\{q':q\in B_p\}$ is an up-antichain and $q\mapsto q'$ is
injective;  so $\#(B_p)=\#(\{q':q\in B_p\})<\theta(p)$.\ \Qed

If $p\le q$ in $P$, any up-antichain in $\coint{q,\infty}$ is also an
up-antichain in $\coint{p,\infty}$, so $\theta(q)\le\theta(p)$.   It
follows that $Q=\{p:p\in P,\,\theta(q)=\theta(p)$ for every $q\ge p\}$
is cofinal with $P$.   Let $A\subseteq Q$ be a maximal up-antichain
(513Aa);  then $\#(A)<\kappa$.

\medskip

\quad{\bf (ii)} \Quer\ Suppose, if possible, that $\kappa=\omega$.

\medskip

\qquad{\bf case 1} Suppose there is a $p\in A$ such that
$\theta(p)=\omega$.   Then we can choose $\sequencen{p_n}$,
$\sequencen{q_n}$ inductively, as follows.   $p_0=p$.   Given that
$p_n\in Q$ and $\theta(p_n)=\omega$, there must be $p_{n+1}$,
$q_n\in\coint{p_n,\infty}$ such that
$\coint{p_{n+1},\infty}\cap\coint{q_n,\infty}=\emptyset$;  now
$p_{n+1}\in Q$ and $\theta(p_{n+1})=\omega$.   Continue.   At the end of
the induction, $\{q_n:n\in\Bbb N\}$ is an infinite up-antichain in $P$,
which is impossible.

\medskip

\qquad{\bf case 2} Suppose that $\theta(p)<\omega$ for every $p\in A$.
Then $n=\sum_{p\in A}\theta(p)$ is finite.   Let $B\subseteq P$ be any
up-antichain.   For each $p\in A$, set
$B_p=\{q:q\in B,\,\coint{q,\infty}\cap\coint{p,\infty}\ne\emptyset\}$;
as noted in (i), $\#(B_p)<\theta(p)$ for every $p\in A$, so
$\#(\bigcup_{p\in A}B_p)\le n$.   But
$B=\bigcup_{p\in A}B_p$, because $A$ is a maximal up-antichain, so
$\#(B)\le n$.   As $B$ is arbitrary,
$\sat^{\uparrow}(P)\le n+1<\omega$.\ \Bang

Thus $\kappa\ne\omega$.

\medskip

\quad{\bf (iii)} \Quer\ Suppose, if possible, that $\kappa$
is a singular infinite cardinal.   Set $\lambda=\cf\kappa$ and let
$\ofamily{\xi}{\lambda}{\kappa_{\xi}}$ be a strictly increasing family
of cardinals with supremum $\kappa$.

\medskip

\qquad{\bf case 1} Suppose there is a $p\in Q$ such that
$\theta(p)=\kappa$.   Then, because $\lambda<\kappa$, there is an
up-antichain $B\subseteq\coint{p,\infty}$ with cardinal $\lambda$;
enumerate $B$ as $\ofamily{\xi}{\lambda}{p_{\xi}}$.   For each
$\xi<\lambda$, $\theta(p_{\xi})>\kappa_{\xi}$, so there is an
up-antichain $C_{\xi}\subseteq\coint{p_{\xi},\infty}$ with cardinal
$\kappa_{\xi}$.   Now $C=\bigcup_{\xi<\lambda}C_{\xi}$ is an
up-antichain in $P$ with cardinal $\kappa$, which is supposed to be
impossible.

\medskip

\qquad{\bf case 2} Suppose that $\theta(p)<\kappa$ for every $p\in Q$.

\medskip

\qquad\quad{\bf case 2a} Suppose that $\sup_{p\in A}\theta(p)<\kappa$.
Then there is an up-antichain $C\subseteq P$ such that
$\#(C)$ is greater than
$\max(\omega,\#(A),\sup_{p\in A}\theta(p))$.   For each $p\in A$ set
$C_p=\{q:q\in C,\,\coint{q,\infty}\cap\coint{p,\infty}\ne\emptyset\}$,
so that $\#(C_p)<\theta(p)$.   It follows that
$C\ne\bigcup_{p\in A}C_p$.   But if
$q\in C\setminus\bigcup_{p\in A}C_p$, there is a $q'\in Q$ such that
$q'\ge q$, and now $A\cup\{q'\}$ is an up-antichain in $Q$ strictly
including  $A$, which is impossible.

\medskip

\qquad\quad{\bf case 2b} Suppose that $\sup_{p\in A}\theta(p)=\kappa$.
Then we can choose inductively a family
$\ofamily{\xi}{\lambda}{p_{\xi}}$ in $A$ such that
$\theta(p_{\xi})>\max(\kappa_{\xi},\sup_{\eta<\xi}\theta(p_{\eta}))$ for
each $\xi$;  the point being that when we come to choose $p_{\xi}$,
$\ofamily{\eta}{\xi}{\theta(p_{\eta})}$ is a family of fewer than
$\cf\kappa$ cardinals less than $\kappa$, so has supremum less than
$\kappa$.   Now all the $p_{\xi}$ must be distinct.   For each $\xi$,
let $B_{\xi}\subseteq\coint{p_{\xi},\infty}$ be an up-antichain of size
$\kappa_{\xi}$;  then $\bigcup_{\xi<\lambda}B_{\xi}$ is an up-antichain
in $P$ of size $\kappa$, which is impossible.\ \Bang

Thus $\kappa$ cannot be a singular infinite cardinal.

\medskip

{\bf (c)} All we need to know is that
$c^{\uparrow}(P)=\sup\{\kappa:\kappa<\sat^{\uparrow}(P)\}$.
}%end of proof of 513B

\cmmnt{\medskip

\noindent{\bf Remark} (b) is sometimes called the {\bf Erd\H{o}s-Tarski
theorem}.}

\leader{513C}{Cofinalities of cardinal
\dvrocolon{functions}}\cmmnt{ We can say a little about the possible
cofinalities of the cardinals which have appeared so far.

\medskip

\noindent}{\bf Proposition} (a) Let $P$ be a partially ordered set with
no greatest member.

\quad(i) If $\add P$ is greater than $2$\cmmnt{ (that is, $P$ is
upwards-directed)}, it is a regular infinite cardinal, and there is a
family $\ofamily{\xi}{\add P}{p_{\xi}}$ in $P$ such that
$p_{\eta}<p_{\xi}$ whenever $\eta<\xi<\add P$, but
$\{p_{\xi}:\xi<\add P\}$ has no upper bound in $P$.

\quad(ii) If $\cf P$ is infinite, its cofinality is at least $\add P$.

(b) Let $\Cal I$ be an ideal of subsets of a set $X$ such that
$\bigcup\Cal I=X\notin\Cal I$.

\quad(i) $\cf(\add\Cal I)=\add\Cal I\le\cf(\cf\Cal I)$.

\quad(ii) $\cf(\non\Cal I)\ge\add\Cal I$.

\quad(iii) If $\cov\Cal I=\cf\Cal I$ then $\cf(\cf\Cal I)\ge\non\Cal I$.

\proof{{\bf (a)(i)} By 511Hd, $\add P\ge\omega$;  by 511He,
$\add P<\infty$, so $\add P$ is an infinite cardinal.   Let
$\ofamily{\xi}{\add P}{q_{\xi}}$ be a family in $P$ with no upper bound
in $P$.   Choose $\ofamily{\xi}{\add P}{p_{\xi}}$ inductively, as follows.   Given $p_{\xi}$, where $\xi<\add P$, there is a
$p'_{\xi}\in P$ such that $p'_{\xi}\not\le p_{\xi}$;  let $p_{\xi+1}$ be an upper bound of $\{p_{\xi},p'_{\xi},q_{\xi}\}$.   
For a limit ordinal
$\xi<\add P$, let $p_{\xi}$ be an upper bound of $\{p_{\eta}:\eta<\xi\}$.   This will ensure that $p_{\eta}<p_{\xi}$ whenever $\xi<\eta<\add P$ and that $\{p_{\xi}:\xi<\add P\}$ has no upper bound, since such a bound would have to be a bound for
$\{q_{\xi}:\xi<\add P\}$.

\Quer\ If $\add P$ is singular, express it as
$\sup_{\xi<\lambda}\kappa_{\xi}$, where $\lambda<\add P$ and
$\kappa_{\xi}<\add P$ for each $\xi<\lambda$.   Then for
each $\xi<\lambda$, $\{p_{\eta}:\eta<\kappa_{\xi}\}$ has an upper bound
$r_{\xi}$ in $P$;  but now $\{r_{\xi}:\xi<\lambda\}$ has an upper bound
in $P$, which is also an upper bound of $\{p_{\eta}:\eta<\add P\}$.\
\BanG\   Thus $\add P$ is regular.

\medskip

\quad{\bf (ii)} \Quer\ If $\cf(\cf P)<\add P$, express $\cf P$ as
$\sup_{\xi<\lambda}\kappa_{\xi}$ where $\lambda<\add P$ and
$\kappa_{\xi}<\cf P$ for each $\xi<\lambda$.   Let
$\ofamily{\eta}{\cf P}{p_{\eta}}$ enumerate a cofinal subset of $P$.
Then $\{p_{\eta}:\eta<\kappa_{\xi}\}$ is never cofinal with $P$, so
there is for each $\xi<\lambda$ a $q_{\xi}\in P$ such that
$q_{\xi}\not\le p_{\eta}$ for every $\eta<\kappa_{\xi}$.   But now there
is a $q\in P$ which is an upper bound for $\{q_{\xi}:\xi<\lambda\}$, and
$q\not\le p_{\eta}$ for any $\eta<\cf P$.\ \Bang

\medskip

{\bf (b)(i)} Because $\bigcup\Cal I=X\notin\Cal I$, $\cf\Cal I$ and
$\add\Cal I$ are both infinite, so this is just a special case of (a).

\medskip

\quad{\bf (ii)} \Quer\ If $\cf(\non\Cal I)<\add\Cal I$, express
$\non\Cal I$ as $\sup_{\xi<\lambda}\kappa_{\xi}$ where
$\lambda<\add\Cal I$ and
$\kappa_{\xi}<\non\Cal I$ for each $\xi<\lambda$.    Let
$\ofamily{\eta}{\non\Cal I}{x_{\eta}}$ enumerate a subset of $X$ not
belonging to $\Cal I$.   Then $I_{\xi}=\{x_{\eta}:\eta<\kappa_{\xi}\}$
belongs to $\Cal I$ for each $\xi<\lambda$;  but this means that
$\bigcup_{\xi<\lambda}I_{\xi}=\{x_{\eta}:\eta<\non\Cal I\}$ belongs to
$\Cal I$.\ \Bang

\medskip

\quad{\bf (iii)} Set $\cf(\cf\Cal I)=\kappa$.   Let
$\ofamily{\xi}{\kappa}{\Cal A_{\xi}}$ be a family of subsets of
$\Cal I$, all with cardinal less than $\cf\Cal I=\cov\Cal I$, such that
$\bigcup_{\xi<\kappa}\Cal A_{\xi}$ is cofinal with $\Cal I$.   Because
$\#(\Cal A_{\xi})<\cov\Cal I$, there is an
$x_{\xi}\in X\setminus\bigcup\Cal A_{\xi}$ for each $\xi<\kappa$.   Now
$\{x_{\xi}:\xi<\kappa\}$  is not included in any member of
$\Cal A_{\xi}$ for any $\xi$, so cannot belong to $\Cal I$ and witnesses
that $\non\Cal I\le\kappa$.
}%end of proof of 513C

\leader{513D}{}\cmmnt{ Galois-Tukey connections between partial orders
have some distinctive features which make a special language
appropriate.

\medskip

\noindent}{\bf Definition} Let $P$ and $Q$ be pre-ordered sets.
A function $\phi:P\to Q$ is a {\bf Tukey function} if $\phi^{-1}[B]$ is
bounded above in $P$ whenever $B\subseteq Q$ is bounded above in $Q$.
A function $\psi:Q\to P$ is a {\bf dual Tukey function} (also called
`cofinal function', `convergent function') if $\psi[B]$ is
cofinal with $P$ whenever $B\subseteq Q$ is cofinal with $Q$.

If $P$ and $Q$ are pre-ordered sets, I
will write `$P\prT Q$' if $(P,\le,P)\prGT(Q,\le,Q)$, and `$P\equivT Q$'
if $(P,\le,P)\equivGT(Q,\le,Q)$;  in the latter case I say that $P$ and
$Q$ are {\bf Tukey equivalent}.   It follows immediately from 512C that
$\prT$ is reflexive and transitive, and of course $P\equivT Q$ iff
$P\prT Q$ and $Q\prT P$.

\leader{513E}{Theorem} Let $P$ and $Q$ be pre-ordered sets.

(a) If $(\phi,\psi)$ is a Galois-Tukey connection from $(P,\le,P)$ to
$(Q,\le,Q)$ then $\phi:P\to Q$ is a Tukey function and $\psi:Q\to P$ is
a dual Tukey function.

(b)(i) A function $\phi:P\to Q$ is a Tukey function iff there is a
function $\psi:Q\to P$ such that $(\phi,\psi)$ is a Galois-Tukey
connection from $(P,\le,P)$ to $(Q,\le,Q)$.

\quad(ii) A function $\psi:Q\to P$ is a dual Tukey function iff there is
a function $\phi:P\to Q$ such that $(\phi,\psi)$ is a Galois-Tukey
connection from $(P,\le,P)$ to $(Q,\le,Q)$.

\quad(iii) If $\psi:Q\to P$ is order-preserving and $\psi[Q]$ is cofinal
with $P$, then $\psi$ is a dual Tukey function.

(c) The following are equiveridical, that is, if one is true so are the
others:

\quad(i) $P\prT Q$;

\quad(ii) there is a Tukey function $\phi:P\to Q$;

\quad(iii) there is a dual Tukey function $\psi:Q\to P$.

\wheader{513E}{0}{0}{0}{48pt}
(d)(i) Let $f:P\to Q$ be such that $f[P]$ is cofinal with $Q$ and, for $p$,
$p'\in P$, $f(p)\le f(p')$ iff $p\le p'$.   Then $P\equivT Q$.

\quad(ii) Suppose that $A\subseteq P$ is cofinal with $P$.   Then
$A\equivT P$

\quad(iii) For $p$, $q\in P$ say that $p\equiv q$ if $p\le q$ and
$q\le p$;  let $\tilde P$ be the partially ordered set of equivalence
classes in $P$ under the equivalence relation $\equiv$\cmmnt{ (511A,
511Ha)}.   Then $P\equivT\tilde P$.

\wheader{513E}{0}{0}{0}{36pt}
(e) Suppose now that $P\prT Q$.   Then

\quad (i) $\cf P\le\cf Q$;

\quad (ii) $\add P\ge\add Q$;

\quad (iii) $\sat^{\uparrow}(P)\le\sat^{\uparrow}(Q)$,
$c^{\uparrow}(P)\le c^{\uparrow}(Q)$;

\quad (iv)
$\link_{<\kappa}^{\uparrow}(P)\le\link_{<\kappa}^{\uparrow}(Q)$ for any
cardinal $\kappa$;

\quad (v) $\link^{\uparrow}(P)\le\link^{\uparrow}(Q)$,
$\duparrow(P)\le\duparrow(Q)$.

(f) If $P$ and $Q$ are Tukey equivalent, then $\cf P=\cf Q$ and
$\add P=\add Q$.

(g) If $\familyiI{P_i}$ and $\familyiI{Q_i}$ are families of pre-ordered
ordered sets such that $P_i\prT Q_i$ for every $i$, then
$\prod_{i\in I}P_i\prT\prod_{i\in I}Q_i$.

(h) If $0<\kappa<\add P$ then
$P\equivT P^{\kappa}$.   In particular, if $P$ is upwards-directed then
$P\equivT P\times P$.

\proof{{\bf (a)} To say that $(\phi,\psi)$ is a Galois-Tukey connection
from $(P,\le,P)$ to $(Q,\le,Q)$ means just that $p\le\psi(q)$ whenever
$\phi(p)\le q$.   Now if $B\subseteq Q$ has an upper bound $q$,
$\psi(q)$ is an upper bound for $\phi^{-1}[B]$;  as $B$ is arbitrary,
$\phi$ is a Tukey function.   Similarly, if
$B\subseteq Q$ is cofinal, then for any $p\in P$ there is a $q\in B$
such that $\phi(p)\le q$ and $p\le\psi(q)$, so $\psi[B]$ is cofinal with
$P$.   As $B$ is arbitrary, $\psi$ is a dual Tukey function.

\medskip

{\bf (b)(i)} If $\phi:P\to Q$ is a Tukey function, then for each
$q\in Q$ set $A_q=\{p:p\in P,\,\phi(p)\le q\}$.
$A_q$ must be bounded above in $P$;  take $\psi(q)$ to be any upper
bound for $A_q$ in $P$.   Then we see that $p\le\psi(q)$ whenever
$\phi(p)\le q$, so that $(\phi,\psi)$ is a Galois-Tukey connection from
$(P,\le,P)$ to $(Q,\le,Q)$.   Together with (a), this proves (i).

\medskip

\quad{\bf (ii)} If $\psi:Q\to P$ is a dual Tukey function, then for each
$p\in P$ set $B_p=\{q:q\in Q,\,\psi(q)\not\ge p\}$.   Then $\psi[B_p]$
is not cofinal with $P$, so $B_p$ cannot be cofinal with $Q$, and there
must be a $\phi(p)\in P$ such that $\phi(p)\not\le q$ for any
$q\in B_p$.   Turning
this round, if $\phi(p)\le q$ then $q\notin B_p$ and $p\le\psi(q)$;  so
$(\phi,\psi)$ is a Galois-Tukey connection from $(P,\le,P)$ to
$(Q,\le,Q)$.   Together with (a), this proves (ii).

\medskip

\quad{\bf (iii)} Because $\psi[Q]$ is cofinal with $P$, we have a
function $\phi:P\to Q$ such that $p\le\psi\phi(p)$ for every $p\in P$.
Now, for any $p\in P$ and $q\in Q$,

\Centerline{$\phi(p)\le q\Longrightarrow p\le\psi\phi(p)\le\psi(q)$,}

\noindent so $(\phi,\psi)$ is a Galois-Tukey connection and $\psi$ is a
dual Tukey function.

\medskip

{\bf (c)} This follows immediately from (a) and (b).

\medskip

{\bf (d)(i)} $f$ is a Tukey function.   \Prf\ If $A\subseteq P$ and $f[A]$
is bounded above in $Q$, let $q$ be an upper bound for $f[A]$.   Because
$f[P]$ is cofinal with $Q$, there is a $p_0\in P$ such that $q\le f(p_0)$.
If now $p\in A$, we have $f(p)\le q\le f(p_0)$ so $p\le p_0$;  thus $A$ is
bounded above in $P$.   As $A$ is arbitrary, $f$ is a Tukey function.\
\QeD\   So $P\prT Q$.

$f$ is a dual Tukey function.   \Prf\ If $A\subseteq P$ is cofinal with
$P$, and $q\in Q$, there are a $p\in P$ such that $q\le f(p)$, and a
$p'\in A$ such that $p\le p'$;  in which case

\Centerline{$q\le f(p)\le f(p')\in f[A]$.}

\noindent Thus $f[A]$ is cofinal with $Q$;  as $A$ is arbitrary, $f$ is a
dual Tukey function.\ \QeD\  So $Q\prT P$ and $P\equivT Q$.

\medskip

\quad{\bf (ii)} Apply (i) to the identity map from $A$ to $P$.

\medskip

\quad{\bf (iii)} Apply (i) to the canonical map from $P$ to $\tilde P$.

\medskip

{\bf (e)} This is just a restatement of the results in 512D, using the
identifications listed in 512Ea.   The only omission concerns
cellularities, for which I have not set out a formal definition in the
context of supported relations;  but if $A\subseteq P$ is an
up-antichain and $\phi:P\to Q$ is a Tukey function, then
$\{\phi(a),\phi(a')\}$ can have no upper bound in $Q$ for any distinct
$a$, $a'\in A$, so $\phi[A]$ is an up-antichain in $Q$ with the same
cardinality as $A$, and $\#(A)\le c^{\uparrow}(Q)$.   As $A$ is
arbitrary, $c^{\uparrow}(P)\le c^{\uparrow}(Q)$ and
$\sat^{\uparrow}(P)\le\sat^{\uparrow}(Q)$.

\medskip

{\bf (f)} follows at once from (e).

\medskip

{\bf (g)} This is a special case of 512H.

\medskip

{\bf (h)} Let $Q\subseteq P^{\kappa}$ be the set of constant functions.
Because $\kappa\ge 1$, $Q$ is isomorphic to $P$;  because
$\kappa<\add P$, $Q$ is cofinal with $P^{\kappa}$;  so
$P\cong Q\equivGT P^{\kappa}$.
}%end of proof of 513E

\leader{513F}{Theorem}\cmmnt{ ({\smc Tukey 40})} Suppose that $P$ and
$Q$ are upwards-directed partially ordered sets.   Then $P$ and $Q$ are
Tukey equivalent iff there is a partially ordered set $R$
such that $P$ and $Q$ are both isomorphic, as partially ordered sets, to
cofinal subsets of $R$.

\proof{{\bf (a)} Suppose that $P$ and $Q$ are Tukey equivalent.   Then
there are Tukey functions $\phi:P\to Q$ and $\psi:Q\to P$.   Set
$S=(P\times\{0\})\cup(Q\times\{1\})$, with a relation $\le$ defined by
saying that

\inset{$(p,0)\le(q,1)$ iff ($\alpha$) there is a $p'\ge p$ in $P$ such
that $\phi(p')\le q$ in $Q$ ($\beta$) $q'\le q$ in $Q$ whenever
$q'\in Q$ and $\psi(q')\le p$ in $P$,

$(q,1)\le(p,0)$ iff ($\alpha$) there is a $q'\ge q$ in $Q$ such that
$\psi(q')\le p$ in $P$ ($\beta$) $p'\le p$ in $P$ whenever $p'\in P$ and
$\phi(p')\le q$ in $Q$,

$(p',0)\le(p,0)$ iff $p'\le p$ in $P$,

$(q',1)\le (q,1)$ iff $q'\le q$ in $Q$.}

\noindent Of course $\le$ is reflexive.   To see that it is transitive,
observe that if $(p,0)\le(q,1)\le(\tilde p,0)$ then there is a
$p'\ge p$ such that $\phi(p')\le q$, and now $p'\le\tilde p$, so
$(p,0)\le(\tilde p,0)$.   Similarly, $(q,1)\le(\tilde q,1)$ whenever
$(q,1)\le(p,0)\le(\tilde q,1)$.   The other cases to check are equally
easy.   It is {\it not} necessarily the case that $\le$ is
antisymmetric, since it is possible to have $(p,0)\le(q,1)\le(p,0)$;
but we have an equivalence relation $\cong$ on $S$ defined by saying
that $s\cong t$ if $s\le t$ and $t\le s$, and a natural partial order on
the set $R$ of equivalence classes defined by saying that
$s^{\ssbullet}\le t^{\ssbullet}$ iff $s\le t$.

The map $p\mapsto(p,0)^{\ssbullet}:P\to R$ is an order-isomorphism
between $P$ and its image $\tilde P\subseteq R$.   Now for any $q\in Q$
there is a $p\in P$ such that $(q,1)\le (p,0)$.   \Prf\ Since $\phi$ is
a Tukey function, $A=\{p':\phi(p')\le q\}$ must be bounded above in $P$;
let $p_0$ be an upper bound for $A$.   Now because $P$ is
upwards-directed,
there is a $p\in P$ such that $p_0\le p$ and $\psi(q)\le p$, and in this
case $(q,1)\le (p,0)$.\ \QeD\   This is what we need to see that
$\tilde P$ is cofinal with $R$.   Similarly, $Q$ is order-isomorphic to
its canonical image in $R$, and this too is cofinal with $R$.
So both $P$ and $Q$ are isomorphic to cofinal subsets of $R$.

\medskip

{\bf (b)} Conversely, if $P$ and $Q$ are both isomorphic to cofinal
subsets of a partially ordered set $R$, then $P$, $R$ and $Q$ are all
Tukey equivalent, by 513Ed.
}%end of proof of 513F

\leader{513G}{}\cmmnt{ We shall repeatedly want to use some elementary
facts about cofinal subsets.

\medskip

\noindent}{\bf Proposition} Let $P$ be a pre-ordered set and $Q$ a
cofinal subset of $P$.   Then

(a) $\add Q=\add P$;

(b) $\cf Q=\cf P$;

(c) $\sat^{\uparrow}(Q)=\sat^{\uparrow}(P)$,
$c^{\uparrow}(Q)=c^{\uparrow}(P)$;

(d) $\link_{<\kappa}^{\uparrow}(Q)=\link_{<\kappa}^{\uparrow}(P)$ for
any cardinal $\kappa$;  in particular,
$\link^{\uparrow}(Q)=\link^{\uparrow}(P)$ and
$\duparrow(Q)=\duparrow(P)$;

(e) $\bu Q=\bu P$.

\proof{ All except (e) are consequences of 513Ed and 513Ee.
As for bursting numbers, every cofinal subset of $Q$ is also cofinal
with $P$, so $\bu P\le\bu Q$.   For the reverse inequality, let $Q_1$ be
a cofinal subset of $P$ such that
$\#(\{q:q\in Q_1$, $q\le p$, $p\not\le q\})<\bu P$
for every $p\in p$.   Let $\phi:Q_1\to Q$ be any function such that
$\phi(q)\ge q$ for every $q\in Q_1$, so that $\phi[Q_1]$ is cofinal with
$Q$.   If $q\in Q$, then

\Centerline{$\{q':q'\in\phi[Q_1]$, $q'\le q$, $q\not\le q'\}
\subseteq\{\phi(q''):q''\in Q_1$, $q''\le q$, $q\not\le q''\}$}

\noindent has cardinal less than $\bu P$, and $\phi[Q_1]$ witnesses
that $\bu Q\le\bu P$.
}%end of proof of 513G

\leader{513H}{Definition} Let $P$ be a partially ordered set.   Its
{\bf $\sigma$-additivity} $\add_{\omega}P$ is the smallest cardinal of
any subset $A$ of $P$ such
that $A\not\subseteq\bigcup_{q\in D}\ocint{-\infty,q}$ for any countable
set $D\subseteq P$.   If there is no such set\cmmnt{, that is, if
$\cf P\le\omega$}, I write $\add_{\omega}P=\infty$.

\leader{513I}{Proposition} Let $P$ be a partially ordered set.   As in
512F, write $p\le^{\strprime}A$, for $p\in P$ and $A\subseteq P$, if
there is a $q\in A$ such that $p\le q$.

(a) $\add_{\omega}P=\add(P,\le^{\strprime},[P]^{\le\omega})$.

(b) $\max(\omega_1,\add P)\le\add_{\omega}(P)$.

(c) If $\add_{\omega}P$ is an infinite cardinal, it is regular.

(d) If $2\le\kappa\le\add P$,
then $(P,\le^{\strprime},[P]^{<\kappa})\equivGT(P,\le,P)$.   So if
$\add P>\omega$, $\add_{\omega}(P)=\add P$.

(e) If $Q$ is another partially ordered set and
$(P,\le^{\strprime},[P]^{\le\omega})
\prGT(Q,\le^{\strprime},[Q]^{\le\omega})$\cmmnt{ (in particular, if
$P\prT Q$)} then $\add_{\omega}P\ge\add_{\omega}Q$.

(f) If $Q\subseteq P$ is cofinal with $P$, then
$\add_{\omega}Q=\add_{\omega}P$.

(g) If $\kappa\le\cf P$ then
$\add(P,\le^{\strprime},[P]^{<\kappa})\le\cf P$.
So if $\cf P>\omega$ then $\add_{\omega}P\le\cf P$.

(h) If $\cf(\cf P)>\omega$ then $\cf(\cf P)\ge\add_{\omega}P$.

\proof{{\bf (a)} All we have to do is to disentangle the definitions in
512Ba, 512F and 513H.

\medskip

{\bf (b)} is immediate from the definition of $\add_{\omega}$.

\medskip

{\bf (c)} \Quer\ Suppose, if possible, that $\add_{\omega}P=\kappa$
where $\kappa>\max(\omega,\cf\kappa)$.   Express $\kappa$ as
$\sup_{\xi<\lambda}\kappa_{\xi}$ where $\kappa_{\xi}<\kappa$ for every
$\xi<\lambda=\cf\kappa$.   Let $A\subseteq P$ be a set with cardinal
$\kappa$ such that $A\not\subseteq\bigcup_{q\in D}\ocint{-\infty,q}$ for
any countable set $D\subseteq P$.   Express $A$ as
$\bigcup_{\xi<\lambda}A_{\xi}$ where $\#(A_{\xi})=\kappa_{\xi}$ for each
$\xi<\lambda$.   For each $\xi<\lambda$, there is a countable set
$D_{\xi}\subseteq P$ such that
$A_{\xi}\subseteq\bigcup_{q\in D_{\xi}}\ocint{-\infty,q}$.   Set
$B=\bigcup_{\xi<\lambda}D_{\xi}$;  then $\#(B)\le\lambda<\kappa$, so
there is a countable set $D\subseteq P$ such that
$B\subseteq\bigcup_{q\in D}\ocint{-\infty,q}$.   But now
$A\subseteq\bigcup_{q\in D}\ocint{-\infty,q}$.\ \Bang

\medskip

{\bf (d)} By 512Gc, $(P,\le^{\strprime},[P]^{<\kappa})\prGT(P,\le,P)$.
In the other direction, because $\kappa\le\add P$, we have a function
$\psi:[P]^{<\kappa}\to P$ such that $I\subseteq\ocint{-\infty,\psi(I)}$
for every $I\in[P]^{<\kappa}$;  so if we set $\phi(p)=p$ for $p\in P$,
$(\phi,\psi)$ will be a Galois-Tukey connection from $(P,\le,P)$ to
$(P,\le^{\strprime},[P]^{<\kappa})$, and
$(P,\le,P)\prGT(P,\le^{\strprime},[P]^{<\kappa})$.

Now if $\add P>\omega$,

\Centerline{$\add_{\omega}P=\add(P,\le^{\strprime},[P]^{\le\omega})
=\add(P,\le,P)=\add P$.}

\medskip

{\bf (e)} Use (a) with 512Db and 512Gb.

\medskip

{\bf (f)} Use (e) and 513Ed.

\medskip

{\bf (g)} Let $Q\subseteq P$ be a cofinal subset of $P$ with cardinal
$\cf P$.   If $A\subseteq P$ is such that every member of $Q$ is
dominated by a member of $A$, then $A$ also is cofinal, so
$\#(A)\ge\kappa$;  thus $Q$ witnesses that
$\add(P,\le^{\strprime},[P]^{<\kappa})\le\cf P$.   Putting
$\kappa=\omega_1$
we see that if $\cf P>\omega$ then $\add_{\omega}P\le\cf P$.

\medskip

{\bf (h)} \Quer\ If
$\omega<\cf(\cf P)=\lambda<\add_{\omega}P$ let
$Q\subseteq P$ be a cofinal set of size $\cf P$ and express $Q$ as
$\bigcup_{\xi<\lambda}Q_{\xi}$ where $\#(Q_{\xi})<\cf P$ and
$Q_{\xi}\subseteq Q_{\eta}$ whenever $\xi\le\eta<\lambda$.   For each
$\xi<\lambda$, $Q_{\xi}$ cannot be cofinal with $P$, so there is a
$p_{\xi}\in P$ such that $p_{\xi}\not\le q$ for any $q\in Q_{\xi}$.
Now $A=\{p_{\xi}:\xi<\lambda\}$ has cardinal less than $\add_{\omega}P$,
so there is a countable set $D\subseteq P$ such that
$A\subseteq\bigcup_{r\in D}\ocint{-\infty,r}$.   For each $r\in D$ there
is a $q_r\in Q$ such that $r\le q_r$;  let $\xi_r<\lambda$ be such that
$q_r\in Q_{\xi_r}$.   Because $\lambda$ is uncountable and regular
(being the cofinality of a cardinal), $\zeta=\sup_{r\in D}\xi_r$ is less
than $\lambda$, and $q_r\in Q_{\zeta}$ for every $r\in D$.   But now
there is an $r\in D$ such that $p_{\zeta}\le r\le q_r\in Q_{\zeta}$,
contrary to the choice of $p_{\zeta}$.\ \Bang
}%end of proof of 513I

\cmmnt{\medskip

\noindent{\bf Remark} The point of (b) and (d) here is that there are
significant cases in which $\add P<\omega_1<\add_{\omega}P$.
}%end of comment

\leader{*513J}{Cofinalities
of \dvrocolon{products}}\dvAformerly{5{}11I}\cmmnt{ It is easy to find
the additivity of a product of partially ordered sets (511Hg).
Calculating the cofinality of a
product of partially ordered sets is surprisingly difficult, and there
are some extraordinary results in this area.   (See
{\smc Burke \& Magidor 90};  there is a taster in 542J below.)
Here I will give just one special fact which will be useful.

\medskip

\noindent}{\bf Proposition} Suppose that the generalized continuum
hypothesis is true.    Let $\familyiI{P_i}$ be a family of
non-empty partially ordered sets with product $P$.  Set

\Centerline{$\kappa=\#(\{i:i\in I$, $\cf P_i>1\})$,
\quad$\lambda=\sup_{i\in I}\cf P_i$.}

\noindent Then

(i) if $\kappa$ and $\lambda$ are both finite,
$\cf P$ is the cardinal product $\prod_{i\in I}\cf P_i$;

(ii) if $\lambda>\kappa$ and there is some $\gamma<\lambda$ such that
$\cf\lambda>\#(\{i:i\in I$, $\cf P_i>\gamma\})$, then
$\cf P=\lambda$;

(iii) otherwise, $\cf P=\max(\kappa^+,\lambda^+)$.

% all except (a) work for supported relations

\proof{{\bf (a)} For each $i\in I$, let $Q_i\subseteq P_i$ be a cofinal
set of size $\cf P_i$.   Then $Q=\prod_{i\in I}Q_i$ is cofinal with
$P=\prod_{i\in I}P_i$, so $\cf P\le\#(Q)$.   If $\lambda<\omega$, then
every $Q_i$ must be just the set of maximal elements of $P_i$, so $Q$ is
the set of maximal elements of $P$, and $\cf P=\#(Q)$.   This deals with
case (i).

\medskip

{\bf (b)} $\cf P>\kappa$.   \Prf\ Set $J=\{i:i\in I$, $\cf P_i>1\}$.
If $\family{i}{J}{p_i}$ is any family in
$P$, then we can choose $q\in P$ such that $q(i)\not\le p_i(i)$ for
every $i\in J$;
accordingly $\{p_i:i\in J\}$ cannot be cofinal with $P$.\ \QeD\  So if
$\max(\omega,\lambda)\le\kappa$,

$$\eqalignno{\cf P
&\ge\kappa^+
=2^{\kappa}\cr
\displaycause{because we are assuming the generalized continuum
hypothesis}
&=2^{\max(\kappa,\lambda)}
\ge\#(\Cal P\{(i,q):i\in J,\,q\in Q_i\})
\ge\#(\prod_{i\in J}Q_i)
=\#(Q)\cr
\displaycause{because $\#(Q_i)=1$ for $i\in I\setminus J$}
&\ge\cf P,\cr}$$

\noindent and $\cf P=\kappa^+=\max(\kappa^+,\lambda^+)$, as required by
(iii).

\medskip

{\bf (c)} Note that $\cf P\ge\lambda$, because if
$R\subseteq P$ is cofinal with $P$ then $\{p(i):p\in R\}$ is cofinal
with $P_i$ for each $i$.   So if $\kappa$ is finite and $\lambda$ is
infinite,

\Centerline{$\lambda\le\cf P\le\#(Q)
\le\max(\omega,\sup_{i\in J}\#(Q_i))=\lambda$}

\noindent and $\cf P=\lambda$, as required by (ii).

\medskip

{\bf (d)} If $\lambda$ is infinite and $\kappa<\cf\lambda$ then every
function from $\kappa$ to
$\lambda$ is bounded above in $\lambda$.   So

$$\eqalignno{\lambda
&\le\cf P\le\#(Q)\le\#(\lambda^{\kappa})\cr
\displaycause{where $\lambda^{\kappa}$, for once, denotes the set of
functions from $\kappa$ to $\lambda$}
&=\#(\bigcup_{\zeta<\lambda}\zeta^{\kappa})
\le\max(\omega,\lambda,\sup_{\zeta<\lambda}\#(\zeta^{\kappa}))
\le\max(\omega,\lambda,\sup_{\zeta<\lambda}2^{\max(\zeta,\kappa)})
=\lambda,\cr}$$

\noindent again using GCH.   Thus in this case also we have
$\cf P=\lambda$, as required by (ii).

\medskip

{\bf (e)} So we are left with the case in which
$\cf\lambda=\theta\le\kappa<\lambda$.   Let
$\ofamily{\eta}{\theta}{\lambda_{\eta}}$ be a family of cardinals less
than $\lambda$ with supremum $\lambda$.

\medskip

\quad\grheada\ Suppose that we are in case (iii), that is,
$\#(\{i:i\in I$, $\cf P_i>\gamma\})\ge\theta$ for every
$\gamma<\lambda$.   Then $\cf P>\lambda$.   \Prf\ We can choose
$\ofamily{\eta}{\theta}{i(\eta)}$ inductively in $I$ so that
$\cf P_{i(\eta)}>\lambda_{\eta}$ and $i(\eta)\ne i(\xi)$ when
$\xi<\eta<\theta$.   If $\ofamily{\xi}{\lambda}{p_{\xi}}$ is any family
in $P$, we can find $q\in P$ such that
$q(i(\eta))\not\le p_{\xi}(i(\eta))$ for any $\eta<\theta$ and
$\xi<\lambda_{\eta}$, so that $q\not\le p_{\xi}$ for any $\xi<\lambda$.
As $\ofamily{\xi}{\lambda}{p_{\xi}}$ is arbitrary, $\cf P>\lambda$.\
\QeD\  Now

\Centerline{$\cf P\le\#(Q)\le\#(\lambda^{\kappa})
\le 2^{\max(\kappa,\lambda)}=\lambda^+\le\cf P$,}

\noindent so $\cf P=\lambda^+=\max(\kappa^+,\lambda^+)$, as required.

\medskip

\quad\grheadb\  Otherwise, we are in case (ii), and there is a cardinal
$\gamma<\lambda$ such that
$\#(K)<\theta$, where $K=\{i:i\in I$, $\cf P_i>\gamma\}$.   Then
$\sup_{i\in K}\cf P_i=\lambda$, so (d) tells us that
$\cf(\prod_{i\in K}P_i)=\lambda$.   On the other hand,

\Centerline{$\#(\prod_{i\in I\setminus K}\cf P_i)
\le 2^{\max(\gamma,\kappa)}\le\lambda$.}

\noindent Since we can identify $P$ with the product of
$\prod_{i\in K}P_i$ and $\prod_{i\in I\setminus K}P_i$,
$\cf P\le\#(\lambda\times\lambda)=\lambda$.   But we noted in
(c) that $\cf P\ge\lambda$, so $\cf P=\lambda$, as required.
This completes the proof.
}%end of proof of 513J

\leader{*513K}{}\dvAformerly{5{}13J}\cmmnt{ I remarked in the notes to
\S512 that
Galois-Tukey correspondences are not required to have any special
properties, and of course the same is true of Tukey functions.   But it
is also the case that the `natural' Tukey functions arising in Chapter
52 can in many cases be derived from Borel functions between Polish
spaces.   I now present some ideas taken from {\smc Solecki \&
Todor\v{c}evi\'c 04} which may be regarded as a partial explanation of
the phenomenon.

\medskip

\noindent}{\bf Definition} I will say that a {\bf metrizably compactly
based directed set} is a partially ordered set $P$ endowed with a
metrizable topology such that

(i) $p\vee q=\sup\{p,q\}$ is defined for all $p$, $q\in P$, and
$\vee:P\times P\to P$ is continuous;

(ii) $\{p:p\le q\}$ is compact for every $q\in P$;

(iii) every convergent sequence in $P$ has a subsequence which is bounded
above.

\noindent In this context, I will say that $P$ is `separable' or
`analytic' if it is separable, or analytic, in the topological sense.

\cmmnt{I leave it to you to check that many significant partially
ordered sets are compactly based in the sense defined here
(513Xj-513Xn, %513Xj 513Xk 513Xl 513Xm 513Xn
513Yg).
}%end of comment

\leader{*513L}{Proposition}\dvAformerly{5{}13K} Let $P$ be a metrizably
compactly based directed set.

(a) The ordering of $P$ is a closed subset of $P\times P$.

(b) $P$ is Dedekind complete.

(c)(i) A non-decreasing sequence in $P$ has an upper bound iff it is
topologically convergent, and in this case its supremum is its limit.

\quad(ii) A non-increasing sequence in $P$ converges topologically
to its infimum.

(d) If $p\in P$ and $\sequence{i}{p_i}$ is a sequence in $P$, then
$\sequence{i}{p_i}$ is topologically convergent to $p$ iff for every
$I\in[\Bbb N]^{\omega}$ there is a $J\in[I]^{\omega}$ such that
$p=\inf_{n\in\Bbb N}\sup_{i\in J\setminus n}p_i$.

(e) Suppose that $p\in P$ and a double sequence
$\langle p_{ni}\rangle_{n,i\in\Bbb N}$ in $P$ are such that
$\lim_{i\to\infty}p_{ni}=p_n$ is defined in $P$ and less than or equal
to $p$ for each $n$.   Then there is a $q\in P$ such that
$\{i:p_{ni}\le q\}$ is infinite for every $n\in\Bbb N$.

\proof{ Let $\rho$ be a metric on $P$ inducing its topology.

\medskip

{\bf (a)} We have only to observe that
$\{(p,q):p\le q\}=\{(p,q):p\vee q=q\}$.

\medskip

{\bf (b)} Suppose that $A\subseteq P$ is non-empty and bounded above.
Let $B$ be the set of upper bounds of $A$.   Then
$\Cal E=\{[p,q]:p\in A$, $q\in B\}$ is a non-empty family of compact
sets with the finite intersection property, because any non-empty finite
subset of $A$ has a least upper bound.   So there is a
$q_0\in\bigcap\Cal E$ and now $q_0$ must be the supremum of $A$.

\medskip

{\bf (c)(i)} Suppose that $\sequence{i}{p_i}$ is a non-decreasing
sequence in $P$.   ($\alpha$) If it has a topological limit $p$, then

\Centerline{$p\vee p_j=\lim_{i\to\infty}p_i\vee p_j
=\lim_{i\to\infty}p_i=p$}

\noindent for each $j$, so $p$ is an upper bound for
$\{p_i:i\in\Bbb N\}$;  while if $q$ is an upper bound for
$\{p_i:i\in\Bbb N\}$ then $p\le q$ by (a).   Thus
$p=\sup_{i\in\Bbb N}p_i$.   ($\beta$) If $\{p_i:i\in\Bbb N\}$ is bounded
above, then it has a least upper bound $p$, by (b).   Now
$\ocint{-\infty,p}$ is compact, therefore sequentially compact, and
every subsequence of $\sequence{i}{p_i}$ has a convergent
sub-subsequence;  by ($\alpha$), the limit of this sub-subsequence is
always its supremum, which must be $p$;  so $\sequence{i}{p_i}$ itself
converges to $p$.

\medskip

\quad{\bf (ii)} Suppose that $\sequence{i}{p_i}$ is a non-increasing
sequence in $P$.   Then it lies in the compact set $\ocint{-\infty,p_0}$
so has a convergent subsequence $\sequence{i}{p'_i}$ with limit $p$ say.
As in (i) just above,

\Centerline{$p\vee p'_j=\lim_{i\to\infty}p'_i\vee p'_j
=\lim_{i\to\infty}p'_j=p'_j$}

\noindent for each $j$, so $p$ is a lower bound for
$\{p'_i:i\in\Bbb N\}$;  while if $q$ is a lower bound for
$\{p'_i:i\in\Bbb N\}$ then $q\le p$ by (a).   Thus
$p=\inf_{i\in\Bbb N}p'_i=\inf_{i\in\Bbb N}p_i$.   What this shows is
that $\inf_{i\in\Bbb N}p_i$ is the only cluster point of
$\sequence{i}{p_i}$ and is therefore its topological limit.

\medskip

{\bf (d)(i)} Suppose that $p=\lim_{i\to\infty}p_i$.
Note first that if $q\in P$ then

\Centerline{$\limsup_{i\to\infty}\rho(q\vee p_i,p)
\le\lim_{i\to\infty}\rho(q\vee p_i,q\vee p)+\rho(q\vee p,p)
=\rho(q\vee p,p)$,}

\Centerline{$\limsup_{i\to\infty}\rho(p\vee q\vee p_i,p)
\le\rho((p\vee q)\vee p,p)=\rho(q\vee p,p)$}

\noindent because $\vee$ is continuous.   Now let $I\subseteq\Bbb N$ be
infinite.   By 513K(iii), there is an infinite $I'\subseteq I$ such that
$\{p_i:i\in I'\}$ is bounded above.
We can choose inductively a strictly increasing
sequence $\sequencen{i_n}$ in $I'$ such that

\Centerline{$\rho(\sup_{j\le n\le k}p_{i_n},p)<2^{-j}$,
\quad$\rho(p\vee\sup_{j\le n\le k}p_{i_n},p)<2^{-j}$}

\noindent whenever $j\le k$ in $\Bbb N$.   Set $J=\{i_n:n\in\Bbb N\}$;
then $\rho(\sup_{j\in J,m\le j\le k}p_j,p)<2^{-m}$ whenever
$m\le k\in\Bbb N$ and $[m,k]$ meets $J$.   For each $m$,
$q_m=\sup_{j\in J\setminus m}p_j$ is
defined in $P$, by (b) above;  moreover, (c-i) tells us that
$q_m=\lim_{k\to\infty}\sup_{j\in J,m\le j\le k}p_j$ so

\Centerline{$\rho(q_m,p)
=\lim_{k\to\infty}\rho(\sup_{j\in J,m\le j\le k}p_j,p)\le 2^{-m}$.}

\noindent But this means that $p$ is the topological limit of the
non-increasing sequence $\sequence{m}{q_m}$ and must be
$\inf_{m\in\Bbb N}q_m$.  Thus $\sequence{i}{p_i}$ satisfies the
condition proposed.

\medskip

\quad{\bf (ii)} Now suppose that for every $I\in[\Bbb N]^{\omega}$ there
is a $J\in[I]^{\omega}$ such that
$p=\inf_{n\in\Bbb N}\sup_{i\in J\setminus n}p_i$.
Then any convergent subsequence of $\sequence{i}{p_i}$ has limit $p$.
\Prf\ Suppose the subsequence is $\sequencen{p_{i_n}}$ where
$\sequencen{i_n}$ is strictly increasing.   Set $I=\{i_n:n\in\Bbb N\}$.
Then we must have an infinite $J\subseteq\Bbb N$ such that
$p=\inf_{m\in\Bbb N}\sup_{k\in J\setminus m}p_{i_k}$.   Now (i) tells us
that we also have an infinite $K\subseteq J$ such that the limit $p'$ of
$\sequence{n}{p_{i_n}}$ is
$\inf_{m\in\Bbb N}\sup_{k\in K\setminus m}p_{i_k}$.   Since
$\sup_{k\in K\setminus m}p_{i_k}\le\sup_{k\in J\setminus m}p_{i_k}$ for
every $m$, $p'\le p$.   On the other hand, we also have an infinite
$L\subseteq K$ such that
$p=\inf_{m\in\Bbb N}\sup_{k\in L\setminus m}p_{i_k}$;  so that
$p\le p'$ and $p=p'$.\ \Qed

Since the condition tells us also that every subsequence of
$\sequence{i}{p_i}$ has a sub-subsequence which is bounded above, and
therefore has a convergent sub-sub-subsequence, $p$ is
actually the limit of $\sequence{i}{p_i}$.

\medskip

{\bf (e)} Note first that if $\sequence{i}{q_i}$ is a sequence in $P$
converging to $q^*\in P$, and $\epsilon>0$, there is a $q'\in P$ such
that $\rho(q',q^*)\le\epsilon$ and $\{i:q_i\le q'\}$ is infinite.
\Prf\  By (d), there is an infinite $J\subseteq\Bbb N$ such that
$q^*=\inf_{n\in\Bbb N}\sup_{i\in J\setminus n}q_i$;  by (c-ii), we can
take $q'=\sup_{i\in J\setminus n}q_i$ for some $n$.\ \Qed

For $m$, $i\in\Bbb N$, set $q_{mi}=p\vee\sup_{n\le m}p_{ni}$.
Then $\lim_{i\to\infty}q_{mi}=p\vee\sup_{n\le m}p_n=p$ for each
$m$.   We can therefore find, for each $m\in\Bbb N$, a $q_m\in P$ such
that $\rho(q_m,p)\le 2^{-m}$ and $\{i:q_{mi}\le q_m\}$ is infinite.   As
$\sequence{m}{q_m}\to p$, there is a
$q\in P$ such that $\{m:q_m\le q\}$ is infinite.   Now, for any
$n\in\Bbb N$, there is an $m\ge n$ such that $q_m\le q$, so that

\Centerline{$\{i:p_{ni}\le q\}\supseteq\{i:q_{mi}\le q_m\}$}

\noindent is infinite.
}%end of proof of 513L

\leader{*513M}{Proposition}\dvAformerly{5{}13L} Let $P$ be a separable
metrizably compactly based directed set, and give the set
$\Cal C$ of closed subsets of $P$ its Vietoris topology.
Let $\Cal K_b\subseteq\Cal C$ be the family of non-empty compact subsets
of $P$ which are bounded above in $P$.   Then
$K\mapsto\sup K:\Cal K_b\to P$ is Borel measurable.

\proof{ Writing $\Cal K$ for the family of non-empty compact subsets of
$P$, we have a sequence $\sequencen{f_n}$ of Borel measurable functions
from $\Cal K$ to $P$ such that $K=\overline{\{f_n(K):n\in\Bbb N\}}$ for
every $K\in\Cal K$ (5A4Dc).
Set $g_n(K)=\sup_{i\le n}f_i(K)$ for each
$K\in\Cal K$ and $n\in\Bbb N$;   because $P$ is separable, every
$g_n$ is Borel measurable (put 418Bd and 418Ac together).   For
$K\in\Cal K_b$, $\sequencen{g_n(K)}$ is a non-decreasing bounded
sequence, so converges to $g(K)\in P$, by 513L(c-i);  now
$g:\Cal K_b\to P$ is Borel measurable (418Ba).   Since
$\{q:q\le g(K)\}$ is a closed set including $\{f_i(K):i\in\Bbb N\}$, it
includes $K$, and $g(K)$ is an upper bound for $K$;  because
$g(K)=\sup_{i\in\Bbb N}f_i(K)$, $g(K)=\sup K$.   So we have the result.
}%end of proof of 513M

\leader{*513N}{Lemma}\dvAformerly{5{}13M} Let $P$ and $Q$ be non-empty
metrizably compactly based directed sets
of which $P$ is separable, and $\phi:P\to Q$ a Tukey function.   Set

\Centerline{$R=\overline{\{(q,p):p\in P,\,q\in Q,\,\phi(p)\le q\}}$.}

\noindent Then

(a) $R[\,\ocint{-\infty,q}\,]$ is bounded above in $P$ for every
$q\in Q$;

(b) $R\subseteq Q\times P$ is usco-compact.

\proof{{\bf (a)} Because $P$ is non-empty, we need consider only the case
in which $R[\,\ocint{-\infty,q}\,]$ is non-empty.
Let $\sequencen{p_n}$ be a sequence running over a
dense subset of $R[\,\ocint{-\infty,q}\,]$.   For each $n\in\Bbb N$ we
have sequences $\sequence{i}{p_{ni}}$ in $P$ and $\sequencen{q_{ni}}$ in
$Q$ such that $\phi(p_{ni})\le q_{ni}$, $\lim_{i\to\infty}p_{ni}=p_n$
and $\lim_{i\to\infty}q_{ni}=q_n\le q$.   By 513Le, there is a $q'\in Q$
such that $I_n=\{i:q_{ni}\le q'\}$ is infinite for every $n\in\Bbb N$.
Because $\phi$ is a Tukey function, there is a $p'\in P$ such that
$p_{ni}\le p'$ whenever $n\in\Bbb N$ and $i\in I_n$.   But now
$\{p:p\le p'\}$ is closed, so it contains every $p_n$ and $p'$ is an upper
bound for $R[\,\ocint{-\infty,q}\,]$.

\medskip

{\bf (b)} In particular, for any $q\in Q$, $R[\{q\}]$ is bounded above
in $P$, therefore relatively compact;  since $R$ is closed, every
$R[\{q\}]$ is closed and therefore compact.   Now suppose that
$F\subseteq P$ is closed and that $\sequencen{q_n}$ is a sequence in
$R^{-1}[F]$ converging to $q\in Q$.   Then there is a $q^*\in Q$ such
that $J=\{n:q_n\le q^*\}$ is infinite.   For $n\in J$, let $p_n\in F$ be
such that $(q_n,p_n)\in R$.   Then $\{p_n:n\in J\}$ is included in the
order-bounded set $R^{-1}[\,\ocint{-\infty,q^*}\,]$, so is relatively
compact, and $\sequencen{p_n}$ has a cluster point $p$ say.   Of course
$p\in F$;  also $(q,p)$ is a cluster point of $\sequencen{(q_n,p_n)}$,
so belongs to $\overline{R}=R$, and $q\in R^{-1}[F]$.   As $q$ is
arbitrary, $R^{-1}[F]$ is closed;  as $F$ is arbitrary, $R$ is
usco-compact.
}%end of proof of 513N

\leader{*513O}{Theorem}\dvAformerly{5{}13N}\cmmnt{ ({\smc Solecki \&
Todor\v{c}evi\'c
04})} Let $P$ and $Q$ be metrizably compactly based directed sets such
that $P\prT Q$.   Let $\Sigma$ be the $\sigma$-algebra of subsets of $P$
generated by the Souslin-F sets.

(a) If $P$ is separable, there is a Borel measurable dual Tukey function
$\psi:Q\to P$.

(b) If $P$ is separable and $Q$ is analytic,
there is a $\Sigma$-measurable Tukey function $\phi:P\to Q$.

\proof{ If either $P$ or $Q$ is empty, so is the other, and the result
is trivial;  suppose that they are non-empty.

\medskip

{\bf (a)} Let $\phi_0:P\to Q$ be a Tukey function, and set
$R=\overline{\{(q,p):p\in P,\,q\in Q,\,\phi_0(p)\le q\}}$, so that $R$
is usco-compact (513N).
Let $\Cal C$ be the set of closed subsets of $P$ with its Vietoris
topology;  then $q\mapsto R[\{q\}]$ is Borel measurable (5A4Db).   Since
$\emptyset$ is an isolated point of $\Cal C$,
$Q_0=\{y:R[\{y\}]\ne\emptyset\}$ is a Borel set in $Q$.   If $q\in Q_0$,
then $R[\{q\}]$ is a non-empty compact subset of $P$ which is bounded
above (513Na), so 513M tells us that $q\mapsto\sup R[\{q\}]:Q_0\to P$ is
Borel measurable.   Fix any $p_0\in P$ and set $\psi(q)=\sup R[\{q\}]$
if $q\in Q_0$, $p_0$ if $q\in Q\setminus Q_0$.   Then $\psi$ is Borel
measurable.   If $p\in P$, $q\in Q$ and $\phi_0(p)\le q$, then
$(q,p)\in R$, $p\in R[\{q\}]$ and $p\le\psi(q)$;  thus $(\phi_0,\psi)$
is a Galois-Tukey connection and $\psi$ is a dual Tukey function.

\medskip

{\bf (b)} $R\subseteq Q\times P$ is a closed set (422Da) and $R[Q]=P$.
Because $P$ and $Q$ are separable and metrizable, $R$ can be obtained by
Souslin's operation from products of closed sets.
By the von Neumann-Jankow selection theorem (423M), there is a
$\Sigma$-measurable $\phi:P\to Q$ such that $(\phi(p),p)\in R$ for every
$p\in P$.   If $q\in Q$, then
$\{p:\phi(p)\le q\}\subseteq R[\,\ocint{-\infty,q}\,]$ is bounded above
in $P$, so $\phi$ is a Tukey function.
}%end of proof of 513O

\leader{513P}{}\dvAformerly{5{}13O}\cmmnt{ The last result in this
section is entirely
unconnected with the rest, and is a standard trick;  but it will be useful
later and contains an implicit challenge (513Yj).

\medskip

\noindent}{\bf Lemma} Let $P$ and $Q$ be
non-empty partially ordered sets, and
suppose that (i) every non-decreasing sequence in $P$ has an upper bound in
$P$ (ii) there is no strictly increasing family
$\ofamily{\xi}{\omega_1}{q_{\xi}}$ in $Q$.   Let $f:P\to Q$ be an
order-preserving function.   Then there is a $p\in P$ such that
$f(p')=f(p)$ whenever $p'\in P$ and $p'\ge p$.

\proof{ \Quer\ Otherwise, we can choose $\ofamily{\xi}{\omega_1}{p_{\xi}}$
inductively so that

\inset{$p_0\in P$,

$p_{\xi+1}\ge p_{\xi}$ and $f(p_{\xi+1})>f(p_{\xi})$ for every
$\xi<\omega_1$,

$p_{\xi}$ is an upper bound for $\{p_{\eta}:\eta<\xi\}$ for
every non-zero limit ordinal $\xi<\omega_1$.}

\noindent But now $\ofamily{\xi}{\omega_1}{f(p_{\xi})}$ is strictly
increasing, which is impossible.\ \Bang
}%end of proof of 513P

\exercises{\leader{513X}{Basic exercises (a)}
%\spheader 513Xa
Let $P$ be a partially ordered set, and $\Cal A$ the
family of subsets of $P$ which are not cofinal with $P$.   Show that
$(\Cal A,\not\ni,P)\prGT(P,\le,P)$.   Explain the relation of this fact to
511Xj, 513C(a-ii) and 513Xb.
%513C

\spheader 513Xb Let $P$ be a partially ordered set such that
$\bu P\ge\omega$.   Show that $\cf(\bu P)\ge\add P$.
%513C

\spheader 513Xc
Let $P$, $Q$ and $R$ be partially ordered sets.   (i)
Show that if $\phi_1:P\to Q$ and $\phi_2:Q\to R$ are Tukey functions,
then $\phi_2\phi_1:P\to R$ is a Tukey function.
(ii) Show that if $\psi_1:P\to Q$ and $\psi_2:Q\to R$ are dual Tukey
functions, then $\psi_2\psi_1:P\to R$ is a dual Tukey function.
%513D

\spheader 513Xd Let $P$ and $Q$ be partially ordered sets, and
$g:Q\to P$ a function.   Show that $g$ is a dual Tukey function iff for
every $p_0\in P$ there is a $q_0\in Q$ such that $g(q)\ge p_0$ for every
$q\ge q_0$.
%513D

\spheader 513Xe(i) Show that if $P$, $Q$ are partially ordered sets,
$P$ is Dedekind complete and $P\prT Q$, there is an order-preserving
dual Tukey function from $Q$ to $P$.
(ii) Set $P=[\{0,1,2\}]^{\le 2}$ and $Q=[\{0,1,2\}]^2$.   Show
that there is no order-preserving Tukey function from $P$ to $Q$.
%513E

\spheader 513Xf Suppose that $P$ is a partially ordered set and
$\add P=\cf P=\kappa\ge\omega$.    Show that $P\equivT\kappa$.
%513E

\spheader 513Xg Prove (a)-(d) of 513G directly, without mentioning Tukey
functions or Galois-Tukey connections.
%513G

\sqheader 513Xh(i) Show that if $P$ and $Q$ are two partially ordered
sets such that $\sat^{\uparrow}(P)=\#(P)^+=\#(Q)^+=\sat^{\uparrow}(Q)$
then $P$ and $Q$ are Tukey equivalent.   \Hint{if $B\subseteq Q$ is an
up-antichain, any injective function $\phi:P\to B$ is a Tukey function
from $P$ to $Q$.}   (ii) Give an example of such a pair $P$, $Q$ such
that $\frak m(P)\ne\frak m(Q)$ and $\bu P\ne\bu Q$.
%513G

\spheader 513Xi Let $P$, $Q_1$, $Q_2$ be partially ordered sets such
that $(P,\le,P)\prGT(Q_1,\le,Q_1)\ltimes(Q_2,\le\nobreak,Q_2)$ (definition:
512I).   Show that
$\add_{\omega}P\ge\min(\add_{\omega}Q_1,\add_{\omega}Q_2)$.
%513H

\spheader 513Xj Show that $\BbbN^{\Bbb N}$, with its usual ordering and
topology, is a metrizably compactly based directed set.
%513K

\spheader 513Xk Let $X$ be a set, $1\le p<\infty$ and $P$ the positive cone
$(\ell^p(X))^+$ of the Banach lattice $\ell^p(X)$, with the topology
induced by the norm of $\ell^p(X)$.   Show that $P$ is a metrizably
compactly based directed set.
%513K

\spheader 513Xl Let $\Cal Z$ be the ideal of subsets of $\Bbb N$ with
asymptotic density zero, with its natural ordering and the topology
induced by the metric
$(a,b)\mapsto\sup_{n\ge 1}\Bover1n\#((a\Bsymmdiff b)\cap n)$.   Show
that $\Cal Z$ is a metrizably compactly based directed set.
%513K

\spheader 513Xm Let $X$ be a metrizable space, and
$\Cal F$ the set of nowhere dense compact subsets of $X$.   Show that
$\Cal F$, with its natural ordering and its Vietoris topology, is a
metrizably compactly based directed set.   \Hint{use a Hausdorff metric.}
%513K

\spheader 513Xn Let $X$ be a metrizable space, $\Cal K$ the family
of compact subsets of $X$, and $\Cal I$ a $\sigma$-ideal of subsets of
$X$.   Show that $\Cal K\cap\Cal I$, with the natural partial order and
the Vietoris topology, is a metrizably compactly based directed set.
%513K

\spheader 513Xo Let $\familyiI{P_i}$ be a countable family of metrizably
compactly based directed sets, with product $P$.
Show that $P$ is metrizably compactly based.
%513K

\spheader 513Xp Let $P$ be a metrizably compactly based directed set.
(i) Show that $P$ is a lattice iff it has a least element.   (ii) Show
that if we adjoin a least element $-\infty$ to $P$ as an isolated point,
$P\cup\{-\infty\}$ is a metrizably compactly based directed set.
%513L

\spheader 513Xq\dvAnew{2014}
Let $\familyiI{P_i}$ be a family of partially ordered
sets, and $P$ their product.
(i) Show that $\cf P$ is at most the cardinal
product $\prod_{i\in I}\cf P_i$, with equality if $I$ is finite.
(ii) Show that if $P\ne\emptyset$ then $\sup_{i\in I}\cf P_i\le\cf P$.
(iii) Show that if $P\ne\emptyset$ and for every $i\in I$ there is a
$j\in I$ such that $\cf P_i<\cf P_j$, then $\sup_{i\in I}\cf P_i<\cf P$.
%513J out of order query

\leader{513Y}{Further exercises (a)}
%\spheader 513Ya
Show that for a cardinal $\kappa$, there is a partially ordered set $P$
such that $c^{\uparrow}(P)=\sat^{\uparrow}(P)=\kappa$ iff $\kappa$ is 
weakly inaccessible.   \Hint{for such a $\kappa$, take $X$ to be a product of
discrete spaces, one of each cardinality less than $\kappa$, and
$P$ the family of proper closed subsets of $X$.}
%513B

\spheader 513Yb Show that for any cardinal $\kappa>0$ there is a
supported relation $(A,R,B)$ such that $\sat(A,R,B)=\kappa$.
%513B

\spheader 513Yc For a non-empty upwards-directed set $P$, a topological
space $X$ and $A\subseteq X$, write $\text{cl}_{P}(A)$ for the set of
points $x\in X$ for which there is a function $f:P\to A$ such that
$x\in\overline{f[C]}$ for every cofinal set $C\subseteq P$;
equivalently, $f[[\Cal F^{\uparrow}(P)]]\to x$, where
$\Cal F^{\uparrow}(P)$ is the filter on $P$ generated by sets of the
form $\coint{p,\infty}$ as $p$ runs over $P$.   Now let $P$ and $Q$ be
upwards-directed sets.   Show that $P\prT Q$ iff
$\text{cl}_P(A)\subseteq\text{cl}_Q(A)$ for any subset $A$ of any
topological space.
%513D

\spheader 513Yd For partially ordered sets $P$ and $Q$, say that
$P\approx Q$ if there is a partially ordered set $R$ into which both $P$
and $Q$ can be embedded as cofinal subsets.
(i) Show that $P\approx Q$ iff there is a Galois-Tukey connection
$(\phi,\psi)$ from $(P,\le,P)$ to $(Q,\le,Q)$ such that $(\psi,\phi)$
is a Galois-Tukey connection from $(Q,\le,Q)$ to $(P,\le,P)$.
(ii) Show that if $P$, $R$ and
$R'$ are partially ordered sets such that $P$ can be embedded as a cofinal
subset into both $R$ and $R'$, then $R\approx R'$.   (iii) Show that
$\approx$ is an equivalence relation on the class of partially ordered
sets.   (iv) Show that if $\Cal P$ is a set of partially ordered sets,
and $P\approx P'$ for all $P$, $P'\in\Cal P$, then there is a partially
ordered set $R$ such that every member of $\Cal P$ can be embedded into $R$
as a cofinal set.   (v) Give an example of partially ordered sets $P$
and $Q$ such that $P\equivT Q$ but $P\not\approx Q$.
%513F mt51bits "co-cofinal sets"

\spheader 513Ye For a cardinal $\kappa$ and a supported relation
$(A,R,B)$ set
$\add_{<\kappa}(A,R,B)=\add(A,R^{\strprime},[B]^{<\kappa})$.   Which of
the ideas of 513I can be extended to the general context?
%513H

\spheader 513Yf\dvAformerly{5{}12Yb}
Show that there are two families
$\familyiI{(A_i,R_i,B_i)}$ and
$\familyiI{(C_i,S_i,D_i)}$ of supported relations, with simple products
$(A,R,B)$ and $(C,S,D)$ respectively, such that
$\cov(A_i,R_i,B_i)=\cov(C_i,S_i,D_i)$ for each $i$, but
$\cov(A,R,B)\ne\cov(C,S,D)$.   \Hint{examine the proof of 513J.}
%512H 513J mt51bits query out of order

\spheader 513Yg Let $X$ be a set and $U$ a solid linear subspace of
$\BbbR^X$ with an order-continuous norm under which it is a Banach
lattice.   Show that its positive cone, with its norm topology, is a
metrizably compactly based directed set.
%513Xk 513K

\spheader 513Yh Explore possible definitions of `compactly based'
partially ordered set which do not require the topology to be
metrizable.
%513L

\spheader 513Yi Let $P$ be an analytic metrizably compactly based
directed set.
Show that $P$ is Polish.   \Hint{{\smc Solecki \& Todor\v{c}evi\'c
04}.} %Theorem 4.1 show G_{\delta} in completion
%513O

\spheader 513Yj For partially ordered sets $P$ and $Q$, say that
$Q\maltese P$ if for every order-preserving $f:P\to Q$ there is a
$p\in P$ such that $f(p')=f(p)$ for every $p'\ge p$.   Explore the
properties of the relation $\maltese$.
%513P

}%end of exercises

\endnotes{
\Notesheader{513} Most of the first part of this section consists of
elementary verifications;  an exception is the Erd\H{o}s-Tarski theorem
on the cellularity and saturation of a partially ordered set
(513Bb-513Bc), which can equally well be regarded as a theorem about
topological spaces or Boolean algebras (see 514Da and 514Nc).   As usual,
I have presented the ideas of the last two sections in an ahistorical
manner;  the original objective of {\smc Tukey 40} was to classify
directed sets from the point of view of net-convergence (513Yc).

I have starred 513K-513O %513K 513L 513M 513N 513O
because I do not expect to rely on them in the rest of this work.
Nevertheless I think that they give a useful support to the ideas here,
particularly in the context of \S526, where these `compactly based'
partial orders appear naturally.   Note that 513Ld tells us that if a
directed set $P$ is metrizably compactly based, there is a unique
witnessing topology;  every topological property of $P$ must be a
reflection of a property of the ordering.
}%end of notes

\discrpage


