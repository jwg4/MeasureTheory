\frfilename{mt243.tex}
\versiondate{30.4.04}
     
\def\chaptername{Function spaces}
\def\sectionname{$L^{\infty}$}
     
\newsection{243}
     
The second of the classical Banach spaces of measure theory which I treat is the space
$L^{\infty}$.   As will appear below, $L^{\infty}$ is the polar
companion of $L^1$, the linked opposite;   for `ordinary' measure
spaces it is actually the dual of $L^1$ (243F-243G).
     
\leader{243A}{Definitions}  Let $(X,\Sigma,\mu)$ be any measure
space.    Let $\eusm L^{\infty}=\eusm L^{\infty}(\mu)$ be the set of
functions $f\in\eusm L^0=\eusm L^0(\mu)$ which are {\bf essentially bounded}, that is, such
that there is some $M\ge 0$ such that $\{x:x\in\dom f,\,|f(x)|\le M\}$ is conegligible, and write
     
\Centerline{$L^{\infty}=L^{\infty}(\mu)
=\{f^{\ssbullet}:f\in\eusm L^{\infty}(\mu)\}\subseteq L^0(\mu)$.}
     
Note that\cmmnt{ if $f\in\eusm L^{\infty}$, $g\in\eusm L^0$ and 
$g\eae f$, then $g\in\eusm L^{\infty}$;  thus} 
$\eusm L^{\infty}=\{f:f\in\eusm L^0,\,f^{\ssbullet}\in L^{\infty}\}$.
     
\vleader{72pt}{243B}{Theorem}  Let $(X,\Sigma,\mu)$ be any measure space.   Then
     
(a) $L^{\infty}=L^{\infty}(\mu)$ is a linear subspace of $L^0=L^0(\mu)$.
     
(b) If $u\in L^{\infty}$, $v\in L^0$ and
$|v|\le|u|$ then $v\in L^{\infty}$.   Consequently $|u|$, $u\vee v$,
$u\wedge v$, $u^+\cmmnt{\mskip5mu =u\vee 0}$ and 
$u^-\cmmnt{\mskip5mu =(-u)\vee 0}$
belong to $L^{\infty}$ for all $u$, $v\in L^{\infty}$.
     
(c) Writing $e=\chi X^{\ssbullet}$, the equivalence class in $L^0$ of
the constant function with value $1$, then an element $u$ of $L^0$
belongs to $L^{\infty}$ iff there is an $M\ge 0$ such that $|u|\le Me$.
     
(d) If $u$, $v\in L^{\infty}$ then $u\times v\in L^{\infty}$.
     
(e) If $u\in L^{\infty}$ and $v\in L^1=L^1(\mu)$ then 
$u\times v\in L^1$.
     
\proof{{\bf (a)} If $f$, $g\in\eusm
L^{\infty}=\eusm L^{\infty}(\mu)$ and $c\in\Bbb R$, then $f+g$,
$cf\in\eusm L^{\infty}$.
\Prf\ We have $M_1$, $M_2\ge 0$ such that $|f|\le M_1$ a.e.\ and 
$|g|\le M_2$ a.e.   Now
     
\Centerline{$|f+g|\le|f|+|g|\le M_1+M_2$ a.e.,
\quad $|cf|\le|c||M_1|$ a.e.,}
     
\noindent so $f+g$, $cf\in\eusm L^{\infty}$.\ \QeD\  It follows at once
that $u+v$, $cu\in L^{\infty}$ whenever $u$, $v\in L^{\infty}$ and
$c\in\Bbb R$.
     
\medskip
     
{\bf (b)(i)} Take $f\in\eusm L^{\infty}$, $g\in\eusm L^0=\eusm L^0(\mu)$
such that $u=f^{\ssbullet}$ and $v=g^{\ssbullet}$.   Then $|g|\leae|f|$.
Let $M\ge 0$ be such that $|f|\le M$ a.e.;  then $|g|\le M$ a.e., so
$g\in\eusm L^{\infty}$ and $v\in L^{\infty}$.
     
\medskip
     
\quad{\bf (ii)} Now $|\,|u|\,|=|u|$ so $|u|\in L^{\infty}$ whenever $u\in
L^{\infty}$.   Also $u\vee v=\bover12(u+v+|u-v|)$, $u\wedge
v=\bover12(u+v-|u-v|)$ belong to $L^{\infty}$ for all $u$, $v\in
L^{\infty}$.
     
\medskip
     
{\bf (c)}(i) If $u\in L^{\infty}$, take $f\in\eusm L^{\infty}$ such that
$f^{\ssbullet}=u$.   Then there is an $M\ge 0$ such that $|f|\le M$
a.e., so that $|f|\leae M\chi X$ and $|u|\le Me$.    (ii) Of
course $\chi X\in\eusm L^{\infty}$, so $e\in L^{\infty}$, and if 
$u\in L^0$ and $|u|\le Me$ then $u\in L^{\infty}$ by (b).
     
\medskip
     
{\bf (d)} $f\times g\in\eusm L^{\infty}$ whenever $f$, 
$g\in\eusm L^{\infty}$.   \Prf\ If $|f|\le M_1$ a.e.\ and $|g|\le M_2$ a.e., then
     
\Centerline{$|f\times g|=|f|\times|g|\le M_1M_2$ a.e.   \Qed}
     
\noindent So $u\times v\in L^{\infty}$ for all $u$, $v\in L^{\infty}$.
     
\medskip
     
{\bf (e)} If $f\in\eusm L^{\infty}$ and $g\in\eusm L^1=\eusm L^1(\mu)$,
then there is an $M\ge 0$ such that $|f|\le M$ a.e., so 
$|f\times g|\leae M|g|$;  because $M|g|$ is integrable and $f\times g$
is virtually measurable, $f\times g$ is integrable and 
$u\times v\in L^1$.
}%end of proof of 243B
     
\leader{243C}{The order structure of $L^{\infty}$}
Let $(X,\Sigma,\mu)$ be any measure space.   Then
$L^{\infty}=L^{\infty}(\mu)$, being a linear subspace of $L^0=L^0(\mu)$,
inherits a partial order which renders it a partially ordered linear
space\cmmnt{ (compare 242Ca)}.   Because $|u|\in L^{\infty}$ whenever
$u\in L^{\infty}$\cmmnt{ (243Bb)}, $u\wedge v$ and $u\vee v$ belong to
$L^{\infty}$ whenever $u$, $v\in L^{\infty}$, and $L^{\infty}$ is a Riesz space\cmmnt{ (compare 242Cd)}.
     
\cmmnt{The behaviour of }$L^{\infty}$\cmmnt{ as a Riesz space is
dominated by the fact
that it} has an {\bf order unit} $e$ with the property that
     
\Centerline{for every $u\in L^{\infty}$ there is an $M\ge 0$ such that
$|u|\le Me$\dvro{.}{}}

\cmmnt{\noindent (243Bc).}
     
\leader{243D}{The norm of $L^{\infty}$}  Let $(X,\Sigma,\mu)$ be any
measure space.
     
\header{243Da}{\bf (a)} For 
$f\in\eusm L^{\infty}=\eusm L^{\infty}(\mu)$,\cmmnt{ say that} the {\bf essential supremum} of $|f|$ is
     
\Centerline{$\esssup|f|=
\inf\{M:M\ge 0,\,\{x:x\in\dom f,\,|f(x)|\le M\}$ is conegligible$\}$.}
     
\noindent Then $|f|\le\esssup|f|$ a.e.   \prooflet{\Prf\ Set
$M=\esssup|f|$.
For each $n\in\Bbb N$, there is an $M_n\le M+2^{-n}$ such that 
$|f|\le M_n$ a.e.  Now
     
\Centerline{$\{x:|f(x)|\le M\}=\bigcap_{n\in\Bbb N}\{x:|f(x)|\le M_n\}$}
     
\noindent is conegligible, so $|f|\le M$ a.e.\ \Qed}
     
\header{243Db}{\bf (b)} If $f$, $g\in\eusm L^{\infty}$ and $f\eae g$,
then $\esssup|f|=\esssup|g|$.   Accordingly we may define a functional
$\|\,\|_{\infty}$ on $L^{\infty}=L^{\infty}(\mu)$ by setting
$\|u\|_{\infty}=\esssup|f|$ whenever $u=f^{\ssbullet}$.
     
\medskip
     
\spheader 243Dc\dvro{ For}{ From (a), we see that, for} any
$u\in L^{\infty}$,
$\|u\|_{\infty}=\min\{\gamma:|u|\le\gamma e\}$, where\cmmnt{, as
before,} $e=\chi X^{\ssbullet}\in L^{\infty}$.   \cmmnt{Consequently}
$\|\,\|_{\infty}$ is a norm on $L^{\infty}$.   \prooflet{\Prf(i) If
$u$, $v\in L^{\infty}$ then
     
     
\Centerline{$|u+v|\le|u|+|v|\le(\|u\|_{\infty}+\|v\|_{\infty})e$}
     
\noindent so $\|u+v\|_{\infty}\le\|u\|_{\infty}+\|v\|_{\infty}$.   (ii)
If $u\in L^{\infty}$ and $c\in\Bbb R$ then
     
\Centerline{$|cu|=|c||u|\le|c|\|u\|_{\infty}e$,}
     
\noindent so
$\|cu\|_{\infty}\le|c|\|u\|_{\infty}$.   (iii) If $\|u\|_{\infty}=0$,
there is an $f\in\eusm L^{\infty}$ such that $f^{\ssbullet}=u$ and
$|f|\le\|u\|_{\infty}$ a.e.;  now $f=0$ a.e.\ so $u=0$. \Qed}
     
\spheader 243Dd Note also that if $u\in L^0$, $v\in L^{\infty}$
and $|u|\le|v|$ then $|u|\le\|v\|_{\infty}e$ so $u\in L^{\infty}$ and
$\|u\|_{\infty}\le\|v\|_{\infty}$;  similarly,
     
\Centerline{$\|u\times v\|_{\infty}\le\|u\|_{\infty}\|v\|_{\infty}$,
\quad$\|u\vee v\|_{\infty}\le\max(\|u\|_{\infty},\|v\|_{\infty})$}
     
\noindent for all $u$, $v\in L^{\infty}$.   \cmmnt{Thus} $L^{\infty}$
is a commutative Banach algebra\cmmnt{ (2A4J)}.
     
\spheader 243De \cmmnt{Moreover,}

\Centerline{$|\int u\times v|\le\int|u\times v|
=\|u\times v\|_1\le\|u\|_1\|v\|_{\infty}$}
     
\noindent whenever $u\in L^1$ and $v\in L^{\infty}$\dvro{.}{, because
     
\Centerline{$|u\times v|
=|u|\times|v|\le|u|\times\|v\|_{\infty}e=\|v\|_{\infty}|u|$.}
}
     
\spheader 243Df
Observe that if $u$, $v$ are non-negative members of $L^{\infty}$ then
     
\Centerline{$\|u\vee v\|_{\infty}
=\max(\|u\|_{\infty},\|v\|_{\infty})$\dvro{.}{;}}
     
\cmmnt{\noindent this is because, for any $\gamma\ge 0$,
     
\Centerline{$u\vee v\le\gamma e\,\iff\,u\le\gamma e$ and 
$v\le\gamma e$.}
}
     
\leader{243E}{Theorem} For any measure space $(X,\Sigma,\mu)$,
$L^{\infty}=L^{\infty}(\mu)$ is a Banach lattice under
$\|\,\|_{\infty}$.
     
\proof{{\bf (a)} We already know that $\|u\|_{\infty}\le\|v\|_{\infty}$
whenever $|u|\le|v|$ (243Dd);  so we have just to check that
$L^{\infty}$ is complete under $\|\,\|_{\infty}$.   Let
$\sequencen{u_n}$ be a Cauchy sequence in
$L^{\infty}$.   For each $n\in\Bbb N$ choose $f_n\in\eusm
L^{\infty}=\eusm L^{\infty}(\mu)$
such that $f_n^{\ssbullet}=u_n$ in $L^{\infty}$.   For all $m$,
$n\in\Bbb
N$, $(f_m-f_n)^{\ssbullet}=u_m-u_n$.   Consequently
     
\Centerline{$E_{mn}=\{x:|f_m(x)-f_n(x)|>\|u_m-u_n\|_{\infty}\}$}
     
\noindent is negligible, by 243Da.
This means that
     
\Centerline{$E
=\bigcap_{n\in\Bbb N}\{x:x\in\dom f_n,\,|f_n(x)|\le\|u_n\|_{\infty}\}
  \setminus\bigcup_{m,n\in\Bbb N}E_{mn}$}
     
\noindent is conegligible.   But for every $x\in E$,
$|f_m(x)-f_n(x)|\le\|u_m-u_n\|_{\infty}$ for all $m$, $n\in\Bbb N$, so
that $\sequencen{f_n(x)}$ is a Cauchy sequence, with a limit in $\Bbb
R$.   Thus $f=\lim_{n\to\infty}f_n$ is defined almost everywhere.
Also, at least for $x\in E$,
     
\Centerline{$|f(x)|\le\sup_{n\in\Bbb N}\|u_n\|_{\infty}<\infty$,}
     
\noindent so $f\in\eusm L^{\infty}$ and
$u=f^{\ssbullet}\in L^{\infty}$.   If $m\in\Bbb N$, then, for every
$x\in E$,

\Centerline{$|f(x)-f_m(x)|\le\sup_{n\ge m}|f_n(x)-f_m(x)|
\le\sup_{n\ge m}\|u_n-u_m\|_{\infty}$,}
     
\noindent so
     
\Centerline{$\|u-u_m\|_{\infty}\le\sup_{n\ge m}\|u_n-u_m\|_{\infty}
\to 0$}
     
\noindent as $m\to\infty$, and $u=\lim_{m\to\infty}u_m$ in $L^{\infty}$.
As $\sequencen{u_n}$ is arbitrary, $L^{\infty}$ is complete.
}
     
\leader{243F}{The duality between $L^{\infty}$ and $L^1$}
Let $(X,\Sigma,\mu)$ be any measure
space.
     
     
\cmmnt{\header{243Fa}{\bf (a)}
I have already remarked that if $u\in L^{1}=L^{1}(\mu)$ and 
$v\in L^{\infty}=L^{\infty}(\mu)$, then $u\times v\in L^1$ and 
$|\int u\times v|\le\|u\|_1\|v\|_{\infty}$ (243Bd, 243De).
}
     
\header{243Fb}{\bf (b)} \dvro{We}{Consequently we} have a bounded linear
operator $T$ from $L^{\infty}$
to the normed space dual $(L^1)^*$ of $L^1$, given by writing
     
\Centerline{$(Tv)(u)=\int u\times v$ for all $u\in L^1$, 
$v\in L^{\infty}$.}   

\prooflet{\noindent\Prf\ (i)
By (a), $(Tv)(u)$ is well-defined for $u\in L^1$, $v\in L^{\infty}$.   (ii) If $v\in L^{\infty}$, $u$, $u_1$, $u_2\in L^1$ and $c\in\Bbb R$, then
     
$$\eqalign{(Tv)(u_1+u_2)
&=\int(u_1+u_2)\times v
=\int(u_1\times v)+(u_2\times v)\cr
&=\int u_1\times v+\int u_2\times v
=(Tv)(u_1)+(Tv)(u_2),\cr}$$
     
\Centerline{$(Tv)(cu)=\int cu\times v=\int c(u\times v)
=c\int u\times v=c(Tv)(u)$.}
     
\noindent This shows that $Tv:L^1\to\Bbb R$ is a linear functional for each $v\in L^{\infty}$.   (iii) Next, for any $u\in L^1$ and 
$v\in L^{\infty}$,
     
\Centerline{$|(Tv)(u)|=|\int u\times v|\le\|u\times v\|_1
\le\|u\|_1\|v\|_{\infty}$,}
     
\noindent as remarked in (a).   This means that $Tv\in (L^1)^*$ and
$\|Tv\|\le\|v\|_{\infty}$ for every $v\in L^{\infty}$.   (iv) If
$v$, $v_1$, $v_2\in L^{\infty}$, $u\in L^1$ and $c\in\Bbb R$, then
     
$$\eqalign{T(v_1+v_2)(u)
&=\int(v_1+v_2)\times u
=\int(v_1\times u)+(v_2\times u)\cr
&=\int v_1\times u+\int v_2\times u
=(Tv_1)(u)+(Tv_2)(u)\cr
&=(Tv_1+Tv_2)(u),\cr}$$
     
\Centerline{$T(cv)(u)=\int cv\times u=c\int v\times u
=c(Tv)(u)=(cTv)(u)$.}
     
\noindent As $u$ is arbitrary, $T(v_1+v_2)=Tv_1+Tv_2$ and $T(cv)=c(Tv)$;
thus $T:L^{\infty}\to(L^1)^*$ is linear.    (v) Recalling from
(iii) that
$\|Tv\|\le\|v\|_{\infty}$ for every $v\in L^{\infty}$, we see that
$\|T\|\le 1$.\ \Qed}
     
\header{243Fc}{\bf (c)} \dvro{We}{Exactly the same arguments show that
we} have a
linear operator $T':L^1\to (L^{\infty})^*$, given by writing
     
\Centerline{$(T'u)(v)=\int u\times v$ for all $u\in L^1$, 
$v\in L^{\infty}$,}
     
\noindent and\cmmnt{ that} $\|T'\|$ also is at most $1$.
     
\leader{243G}{Theorem} Let $(X,\Sigma,\mu)$ be a measure space, and
$T:L^{\infty}(\mu)\to(L^1(\mu))^*$ the canonical map described in 243F.
Then
     
(a) $T$ is injective iff $(X,\Sigma,\mu)$ is semi-finite, and in this
case is norm-preserving;
     
(b) $T$ is bijective iff $(X,\Sigma,\mu)$ is localizable, and in this
case is a normed space isomorphism.
     
\proof{{\bf (a)(i)} Suppose that $T$ is injective, and that
$E\in\Sigma$ has $\mu E=\infty$.   Then $\chi E$ is not equal almost everywhere to $\tbf{0}$, so $(\chi E)^{\ssbullet}\ne 0$ in $L^{\infty}$, and $T(\chi E)^{\ssbullet}\ne 0$;  let $u\in L^1$ be such that 
$T(\chi E)^{\ssbullet}(u)\ne 0$, that is,
$\int u\times(\chi E)^{\ssbullet}\ne 0$.   Express $u$ as
$f^{\ssbullet}$
where $f$ is integrable;  then $\int_Ef\ne 0$ so $\int_E|f|\ne 0$.   Let
$g$ be a simple function such that $0\le g\leae|f|$ and
$\int g>\int|f|-\int_E|f|$;  then $\int_Eg\ne 0$.   Express $g$ as
$\sum_{i=0}^na_i\chi E_i$ where $\mu E_i<\infty$ for each $i$;  then
$0\ne\sum_{i=0}^na_i\mu(E_i\cap E)$, so there is an $i\le n$ such that
$\mu(E\cap E_i)\ne 0$, and now $E\cap E_i$ is a measurable subset of $E$
of non-zero finite measure.
     
As $E$ is arbitrary, this shows that $(X,\Sigma,\mu)$ must be
semi-finite if $T$ is injective.
     
\medskip
     
\quad{\bf (ii)} Now suppose that $(X,\Sigma,\mu)$ is semi-finite, and
that $v\in L^{\infty}$ is non-zero.   Express $v$ as $g^{\ssbullet}$
where $g:X\to\Bbb R$ is measurable;  then
$g\in\eusm L^{\infty}$.   Take any
$a\in\ooint{0,\|v\|_{\infty}}$;  then $E=\{x:|g(x)|\ge a\}$ has non-zero
measure.   Let $F\subseteq E$ be a measurable set of non-zero finite
measure, and
set $f(x)=|g(x)|/g(x)$ if $x\in F$, $0$ otherwise;  then $f\in\eusm L^1$
and $(f\times
g)(x)\ge a$ for $x\in F$, so, setting $u=f^{\ssbullet}\in L^1$, we have
     
\Centerline{$(Tv)(u)=\int u\times v=\int f\times g\ge a\mu F=a\int|f|
=a\|u\|_1>0$.}
     
\noindent This shows that $\|Tv\|\ge a$;  as $a$ is arbitrary,
$\|Tv\|\ge\|v\|_{\infty}$.   We know already from 243F that
$\|Tv\|\le\|v\|_{\infty}$, so $\|Tv\|=\|v\|_{\infty}$ for every non-zero
$v\in L^{\infty}$;  the same is surely true for $v=0$, so $T$ is
norm-preserving and injective.
     
\medskip
     
{\bf (b)(i)} Using (a) and the definition of `localizable', we see
that under either of the conditions proposed $(X,\Sigma,\mu)$ is
semi-finite and $T$ is injective and norm-preserving.   I therefore have
to show just that it is surjective iff $(X,\Sigma,\mu)$ is localizable.
     
\medskip
     
\quad{\bf (ii)} Suppose that $T$ is surjective and that $\Cal
E\subseteq\Sigma$.   Let $\Cal F$ be the family of finite unions of
members of $\Cal E$, counting $\emptyset$ as the union of no members of
$\Cal E$, so that $\Cal F$ is closed under finite unions and, for any
$G\in\Sigma$, $E\setminus G$ is negligible for every $E\in\Cal E$ iff
$E\setminus G$ is negligible for every $E\in\Cal F$.
     
If $u\in L^1$, then $h(u)=\lim_{E\in\Cal F,E\uparrow}\int_Eu$ exists in
$\Bbb R$.   \prooflet{\Prf\ If $u$ is non-negative, then
     
     
\Centerline{$h(u)=\sup\{\int_Eu:E\in\Cal F\}\le\int u<\infty$.}
     
\noindent   For other $u$, we
can express $u$ as $u_1-u_2$, where $u_1$ and $u_2$ are non-negative,
and now $h(u)=h(u_1)-h(u_2)$.\ \Qed}
     
Evidently $h:L^1\to\Bbb R$ is linear, being a limit of the linear
functionals $u\mapsto\int_Eu$, and also
     
\Centerline{$|h(u)|\le\sup_{E\in\Cal F}|\int_Eu|\le\int|u|$}
     
\noindent for every $u$, so $h\in(L^1)^*$.
Since we are supposing that $T$ is surjective, there is a $v\in
L^{\infty}$ such that $Tv=h$.   Express $v$ as $g^{\ssbullet}$ where
$g:X\to\Bbb R$ is measurable and essentially bounded.   Set
$G=\{x:g(x)>0\}\in\Sigma$.
     
If $F\in\Sigma$ and $\mu F<\infty$, then
     
\Centerline{$\int_Fg=\int(\chi F)^{\ssbullet}\times g^{\ssbullet}
=(Tv)(\chi F)^{\ssbullet}
=h(\chi F)^{\ssbullet}
=\sup_{E\in\Cal F}\mu(E\cap F)$.}
\Quer\ If $E\in\Cal E$ and $E\setminus G$ is not negligible, then there
is a set $F\subseteq E\setminus G$ such that $0<\mu F<\infty$;  now
     
     
\Centerline{$\mu F=\mu(E\cap F)\le\int_Fg\le 0$,}
     
\noindent as $g(x)\le 0$ for $x\in F$.\ \BanG\  Thus $E\setminus G$ is
negligible for every $E\in\Cal E$.
     
Let $H\in\Sigma$ be such that $E\setminus H$ is negligible for every
$E\in\Cal E$.   \Quer\ If $G\setminus H$ is not negligible, there is a
set $F\subseteq G\setminus H$ of non-zero finite measure.   Now
     
     
\Centerline{$\mu(E\cap F)\le\mu(H\cap F)=0$}
     
\noindent for every $E\in\Cal E$, so $\mu(E\cap F)=0$ for every $E\in\Cal F$, and $\int_Fg=0$;  but $g(x)>0$ for every
$x\in F$, so $\mu F=0$, which is impossible.\ \BanG\   Thus 
$G\setminus H$ is negligible.
     
Accordingly $G$ is an essential supremum of $\Cal E$ in $\Sigma$.   As
$\Cal E$ is arbitrary, $(X,\Sigma,\mu)$ is localizable.
     
\medskip
     
\quad{\bf (iii)} For the rest of this proof, I will suppose that
$(X,\Sigma,\mu)$ is localizable and seek to show that $T$ is surjective.
     
Take $h\in(L^1)^*$ such that $\|h\|=1$.   Write
$\Sigma^f=\{F:F\in\Sigma,\,\mu F<\infty\}$, and for $F\in\Sigma^f$
define $\nu_F:\Sigma\to\Bbb R$ by setting
     
\Centerline{$\nu_FE=h(\chi(E\cap F)^{\ssbullet})$}
     
\noindent for every $E\in\Sigma$.   Then $\nu_F\emptyset=h(0)=0$, and if
$E$, $E'\in\Sigma$ are disjoint

$$\eqalign{\nu_FE+\nu_FE'
&=h(\chi(E\cap F)^{\ssbullet})+h(\chi(E'\cap F)^{\ssbullet})
=h((\chi(E\cap F)+\chi(E'\cap F))^{\ssbullet})\cr
&=h(\chi((E\cup E')\cap F)^{\ssbullet})
=\nu_F(E\cup E').\cr}$$

\noindent   Thus $\nu_F$ is additive.   Also
     
\Centerline{$|\nu_FE|\le\|\chi(E\cap F)^{\ssbullet}\|_1=\mu(E\cap F)$}

\noindent for every $E\in\Sigma$, so $\nu_F$ is truly continuous in the
sense of 232Ab.   By the Radon-Nikod\'ym theorem (232E), there is an
integrable function $g_F$ such that $\int_Eg_F=\nu_FE$ for every
$E\in\Sigma$;  we may take it that $g_F$ is measurable and has domain
$X$ (232He).
     
\medskip
     
\quad{\bf (iv)} It is worth noting that $|g_F|\le 1$ a.e.
\prooflet{\Prf\ If $G=\{x:g_F(x)>1\}$, then
     
\Centerline{$\int_Gg_F=\nu_FG\le\mu(F\cap G)\le\mu G$;}
     
\noindent  but this is possible only if $\mu G=0$.   Similarly, if
$G'=\{x:g_F(x)<-1\}$, then
     
     
\Centerline{$\int_{G'}g_F=\nu_FG'\ge-\mu G'$,}
     
\noindent so again $\mu G'=0$.\ \Qed}
     
\medskip
     
\quad{\bf (v)} If $F$, $F'\in\Sigma^f$, then $g_F=g_{F'}$ almost
everywhere in $F\cap F'$.   \Prf\ If $E\in\Sigma$ and 
$E\subseteq F\cap F'$, then
     
\Centerline{$\int_Eg_F=h(\chi(E\cap F)^{\ssbullet})
=h(\chi(E\cap F')^{\ssbullet})=\int_Eg_{F'}$.}
     
\noindent So 131Hb gives the result.\ \QeD\   213N (applied to
$\{g_F\restr F:F\in\Sigma^f\}$) now tells us
that, because $\mu$ is localizable, there is a measurable function
$g:X\to\Bbb R$ such that $g=g_F$ almost everywhere in $F$, for every
$F\in\Sigma^f$.
     
\medskip
     
\quad{\bf (vi)} For any $F\in\Sigma^f$, the set
     
\Centerline{$\{x:x\in F,\,|g(x)|>1\}\subseteq\{x:|g_F(x)|>1\}
\cup\{x:x\in F,\,g(x)\ne g_F(x)\}$}
     
\noindent is negligible;  because $\mu$ is semi-finite,
$\{x:|g(x)|>1\}$ is negligible, and $g\in\eusm L^{\infty}$, with
$\esssup|g|\le 1$.   Accordingly $v=g^{\ssbullet}\in L^{\infty}$, and we
may speak of $Tv\in(L^1)^*$.
     
\medskip
     
\quad{\bf (vii)} If $F\in\Sigma^f$, then

\Centerline{$\int_Fg=\int_Fg_F=\nu_FX=h(\chi F^{\ssbullet})$.}
     
\noindent It follows at once that

\Centerline{$(Tv)(f^{\ssbullet})=\int f\times g=h(f^{\ssbullet})$}
     
\noindent for every simple function $f:X\to\Bbb R$.
Consequently $Tv=h$, because both $Tv$ and $h$ are
continuous and the equivalence classes of simple functions form a dense
subset of $L^1$ (242Mb, 2A3Uc).   Thus $h=Tv$ is a value of $T$.
     
\medskip
     
\quad{\bf (viii)} The
argument as written above has assumed that $\|h\|=1$.   But of course
any non-zero member of $(L^1)^*$ is a scalar multiple of an element of
norm $1$, so is a value of $T$.   So $T:L^{\infty}\to(L^1)^*$ is indeed
surjective, and is therefore an isometric isomorphism, as claimed.
}%end of proof of 243G
     
\leader{243H}{}\cmmnt{ Recall that $L^0$ is always Dedekind
$\sigma$-complete and sometimes Dedekind complete (241G), while $L^1$ is
always Dedekind
complete (242H).   In this respect $L^{\infty}$ follows $L^0$.
     
\medskip
     
\noindent}{\bf Theorem} Let $(X,\Sigma,\mu)$ be a measure space.
     
(a) $L^{\infty}(\mu)$ is Dedekind $\sigma$-complete.
     
(b) If $\mu$ is localizable, $L^{\infty}(\mu)$ is Dedekind complete.
     
\proof{ These are both consequences of 241G.
If $A\subseteq L^{\infty}=L^{\infty}(\mu)$ is bounded above in
$L^{\infty}$, fix $u_0\in A$ and an upper bound $w_0$ of $A$ in
$L^{\infty}$.  If $B$ is the set of upper bounds for $A$ in
$L^0=L^0(\mu)$, then $B\cap L^{\infty}$ is the set of upper bounds for
$A$ in $L^{\infty}$.   Moreover, if $B$ has a least member $v_0$, then
we must have $u_0\le v_0\le w_0$, so that
     
\Centerline{$0\le v_0-u_0\le w_0-u_0\in L^{\infty}$}
     
\noindent and $v_0-u_0$, $v_0$ belong to $L^{\infty}$.   (Compare part
(a) of the proof of 242H.)   Thus $v_0=\sup A$ in $L^{\infty}$.
     
Now we know that $L^0$ is Dedekind $\sigma$-complete;  if $A\subseteq
L^{\infty}$ is a non-empty countable set which is bounded above in
$L^{\infty}$, it is surely bounded above in $L^0$, so has a supremum in
$L^0$ which is also its supremum in $L^{\infty}$.   As $A$ is arbitrary,
$L^{\infty}$ is Dedekind $\sigma$-complete.   While if $\mu$ is
localizable, we can argue in the same way with arbitrary non-empty
subsets of $L^{\infty}$ to see that $L^{\infty}$ is Dedekind complete
because $L^0$ is.
}
     
\leader{243I}{A dense subspace of
\dvrocolon{$L^{\infty}$}}\cmmnt{ In 242M and 242O I described
a couple of important dense linear subspaces of $L^1$ spaces.   The
position concerning $L^{\infty}$ is a little different.   However I can
describe one important dense subspace.
     
\medskip
     
\noindent}{\bf Proposition} Let $(X,\Sigma,\mu)$ be a measure space.
     
(a) Write $\eusm S$ for the space of `$\Sigma$-simple'
functions on $X$, that is, the space of functions from $X$ to $\Bbb R$
expressible as $\sum_{k=0}^na_k\chi E_k$ where $a_k\in\Bbb R$ and
$E_k\in\Sigma$ for every $k\le n$.   Then for every $f\in\eusm
L^{\infty}=\eusm L^{\infty}(\mu)$ and every $\epsilon>0$, there is a
$g\in\eusm S$ such that $\esssup|f-g|\le\epsilon$.
     
(b) $S=\{f^{\ssbullet}:f\in\eusm S\}$ is a
$\|\,\|_{\infty}$-dense linear subspace of $L^{\infty}=L^{\infty}(\mu)$.
     
(c) If $(X,\Sigma,\mu)$ is totally finite, then $\eusm S$ is
the space of $\mu$-simple functions, so $S$ becomes just the
space of equivalence classes of simple functions\cmmnt{, as in 242Mb}.
     
\proof{{\bf (a)} Let $\tilde f:X\to\Bbb R$ be a bounded measurable
function such that $f\eae\tilde f$.   Let $n\in\Bbb N$ be such that
$|f(x)|\le n\epsilon$ for every $x\in X$.   For $-n\le k\le n$ set
     
\Centerline{$E_k=\{x:k\epsilon\le \tilde f(x)<k+1)\epsilon$.}
     
\noindent Set
     
\Centerline{$g=\sum_{k=-n}^nk\epsilon\chi E_k\in\eusm S$;}
     
\noindent then $0\le\tilde f(x)-g(x)\le\epsilon$ for every $x\in X$, so
     
\Centerline{$\esssup|f-g|=\esssup|\tilde f-g|\le\epsilon$.}
     
\medskip
     
{\bf (b)} This follows immediately, as in 242Mb.
     
\medskip
     
{\bf (c)} also is elementary.
}%end of proof of 243I
     
     
\vleader{60pt}{243J}{Conditional expectations}\cmmnt{ Conditional 
expectations are so
important that it is worth considering their interaction with every new
concept.

\medskip

}{\bf (a)} If $(X,\Sigma,\mu)$ is any measure space, and
$\Tau$ is a $\sigma$-subalgebra of $\Sigma$, then the canonical
embedding $S:L^0(\mu\restrp\Tau)\to L^0(\mu)$\cmmnt{ (242Ja)} embeds
$L^{\infty}(\mu\restrp\Tau)$ as a subspace of $L^{\infty}(\mu)$, and
$\|Su\|_{\infty}=\|u\|_{\infty}$ for every $u\in
L^{\infty}(\mu\restrp\Tau)$.   \dvro{We}{As in 242Jb, we} can identify
$L^{\infty}(\mu\restrp\Tau)$ with its image in $L^{\infty}(\mu)$.
     
\header{243Jb}{\bf (b)} Now suppose that $\mu X=1$, and let
$P:L^1(\mu)\to L^1(\mu\restrp\Tau)$ be the conditional expectation
operator\cmmnt{ (242Jd)}.   Then
$L^{\infty}(\mu)$ is\cmmnt{ actually} a linear subspace of $L^1(\mu)$.
Setting $e=\chi X^{\ssbullet}\in L^{\infty}(\mu)$,\cmmnt{ we see that
$\int_Fe=(\mu\restrp\Tau)(F)$ for every $F\in\Tau$, so}
     
\Centerline{$Pe=\chi X^{\ssbullet}\in L^{\infty}(\mu\restrp\Tau)$.}
     
\noindent\cmmnt{If $u\in L^{\infty}(\mu)$, then setting
$M=\|u\|_{\infty}$ we
have $-Me\le u\le Me$, so $-MPe\le Pu\le MPe$, because $P$ is
order-preserving (242Je);  accordingly 
$\|Pu\|_{\infty}\le M=\|u\|_{\infty}$.   
Thus }$P\restr L^{\infty}(\mu):L^{\infty}(\mu)\to
L^{\infty}(\mu\restrp\Tau)$ is an operator of norm $1$.

If $u\in L^{\infty}(\mu\restrp\Tau)$, then $Pu=u$;  so 
$P[L^{\infty}]$ is the whole of $L^{\infty}(\mu\restrp\Tau)$.
     
\leader{243K}{Complex $L^{\infty}$}\cmmnt{ All the ideas needed to
adapt the work above to complex $L^{\infty}$ spaces have already appeared in 241J and 242P.}   Let $\eusm L^{\infty}_{\Bbb C}$ be
     
\Centerline{$\{f:f\in\eusm L^0_{\Bbb C},\,\esssup|f|<\infty\}
=\{f:\Real(f)\in\eusm L^{\infty},\,\Imag(f)\in\eusm L^{\infty}\}$.}
     
\noindent Then
     
\Centerline{$L^{\infty}_{\Bbb C}
=\{f^{\ssbullet}:f\in\eusm L^{\infty}_{\Bbb C}\}
=\{u:u\in L^0_{\Bbb C},\,\Real(u)\in L^{\infty},
  \,\Imag(u)\in L^{\infty}\}$.}
     
\noindent Setting
     
\Centerline{$\|u\|_{\infty}
=\||u|\|_{\infty}$\cmmnt{$=\esssup|f|$ whenever $f^{\ssbullet}=u$},}
     
\noindent we have a norm on $L^{\infty}_{\Bbb C}$ rendering it a Banach
space.   \cmmnt{We still have} $u\times v\in L^{\infty}_{\Bbb C}$ and
$\|u\times v\|_{\infty}\le\|u\|_{\infty}\|v\|_{\infty}$ for all $u$, $v\in L^{\infty}_{\Bbb C}$.
     
We now have a duality between $L^1_{\Bbb C}$ and $L^{\infty}_{\Bbb C}$
giving rise to a linear operator 
$T:L^{\infty}_{\Bbb C}\to(L^1_{\Bbb C})^*$ of norm at most $1$, defined by the formula
     
\Centerline{$(Tv)(u)=\int u\times v$ 
for every $u\in L^1$, $v\in L^{\infty}$.}
     
\noindent $T$ is injective iff the underlying measure space is
semi-finite, and is a bijection iff the underlying measure space is
localizable.   \prooflet{(This can of course be proved by re-working the
arguments
of 243G;  but it is perhaps easier to note that $T(\Real(v))=\Real(Tv)$,
$T(\Imag(v))=\Imag(Tv)$ for every $v$, so that the result for complex
spaces can be deduced from the result for real spaces.)}   \cmmnt{To
check that} $T$ is norm-preserving when it is injective\cmmnt{,
the quickest route seems to be to imitate the argument of (a-ii) of the
proof of 243G}.
     
\exercises{
\leader{243X}{Basic exercises (a)}
%\spheader 243Xa
Let $(X,\Sigma,\mu)$ be any measure space, and $\hat\mu$ the completion
of $\mu$ (212C, 241Xb).   Show that
$\eusm L^{\infty}(\hat\mu)=\eusm L^{\infty}(\mu)$ and $L^{\infty}(\hat\mu)=L^{\infty}(\mu)$.
%243A
     
\sqheader 243Xb Let $(X,\Sigma,\mu)$ be a non-empty measure
space.   Write $\eusm L^{\infty}_{\Sigma}$ for the space of
bounded $\Sigma$-measurable real-valued functions with domain $X$.   (i) Show that $L^{\infty}(\mu)
=\{f^{\ssbullet}:f\in\eusm L^{\infty}_{\Sigma}\}
\subseteq L^0=L^0(\mu)$.   (ii) Show that
$\eusm L^{\infty}_{\Sigma}$
is a Dedekind $\sigma$-complete Banach lattice if we give it the norm
     
\Centerline{$\|f\|_{\infty}=\sup_{x\in X}|f(x)|$ for every
$f\in\eusm L^{\infty}_{\Sigma}$.}
     
\noindent (iii) Show that for every $u\in L^{\infty}=L^{\infty}(\mu)$,
$\|u\|_{\infty}=\min\{\|f\|_{\infty}:
f\in\eusm L^{\infty}_{\Sigma},\,f^{\ssbullet}=u\}$.
%243D
     
\sqheader 243Xc Let $(X,\Sigma,\mu)$ be any measure space, and
$A$ a subset of $L^{\infty}(\mu)$.   Show that $A$ is bounded for the
norm $\|\,\|_{\infty}$ iff it is bounded above and below for the ordering of $L^{\infty}$.
%243D
     
\spheader 243Xd Let $(X,\Sigma,\mu)$ be any measure space, and
$A\subseteq L^{\infty}(\mu)$ a non-empty set with a least upper bound
$w$ in $L^{\infty}(\mu)$.   Show that 
$\|w\|_{\infty}\le\sup_{u\in A}\|u\|_{\infty}$.
%243D
     
\spheader 243Xe Let 
$\langle(X_i,\Sigma_i,\mu_i)\rangle_{i\in I}$ be a family of measure 
spaces, and $(X,\Sigma,\mu)$ their direct sum
(214L).   Show that the canonical isomorphism between $L^0(\mu)$ and
$\prod_{i\in I}L^0(\mu_i)$ (241Xd) induces an isomorphism between
$L^{\infty}(\mu)$ and the subspace
     
\Centerline{$\{u:u\in\prod_{i\in I}L^{\infty}(\mu_i),\,
\|u\|=\sup_{i\in I}\|u(i)\|_{\infty}<\infty\}$}
     
\noindent of $\prod_{i\in I}L^{\infty}(\mu_i)$.
%243D
     
\spheader 243Xf Let $(X,\Sigma,\mu)$ be any measure space, and
$u\in L^1(\mu)$.   Show that there is a $v\in L^{\infty}(\mu)$ such that
$\|v\|_{\infty}\le 1$ and $\int u\times v=\|u\|_1$.
%243F
     
\spheader 243Xg Let $(X,\Sigma,\mu)$ be a semi-finite measure
space and $v\in L^{\infty}(\mu)$.   Show that
     
\Centerline{$\|v\|_{\infty}
=\sup\{\int u\times v:u\in L^1,\,\|u\|_1\le 1\}
=\sup\{\|u\times v\|_1:u\in L^1,\,\|u\|_1\le 1\}$.}
%243F

\spheader 243Xh Give an example of a probability space
$(X,\Sigma,\mu)$ and a $v\in L^{\infty}(\mu)$ such that
$\|u\times v\|_1<\|v\|_{\infty}$ whenever $u\in L^1(\mu)$ and
$\|u\|_1\le 1$.
%243F
     
\spheader 243Xi Write out proofs of 243G adapted to the special
cases (i) $\mu X=1$ (ii) $(X,\Sigma,\mu)$ is $\sigma$-finite.
%243G
     
\spheader 243Xj Let $(X,\Sigma,\mu)$ be any measure space.
Show that $L^0(\mu)$ is Dedekind complete iff $L^{\infty}(\mu)$ is
Dedekind complete.
%243G
     
\spheader 243Xk Let $(X,\Sigma,\mu)$ be a totally finite measure
space and $\nu:\Sigma\to\Bbb R$ a functional.   Show that the following
are equiveridical:   (i) there is a continuous linear functional
$h:L^1(\mu)\to\Bbb R$ such that $h((\chi E)^{\ssbullet})=\nu E$ for
every $E\in\Sigma$ (ii) $\nu$ is additive and there is an $M\ge 0$ such
that $|\nu E|\le M\mu E$ for every $E\in\Sigma$.
%243G
     
\sqheader 243Xl  Let $X$ be any set, and let $\mu$ be counting
measure on $X$.
In this case it is customary  to write $\ell^{\infty}(X)$ for $\eusm
L^{\infty}(\mu)$, and to identify it with
$L^{\infty}(\mu)$.   Write out statements
and proofs of the results of this chapter adapted to this special
case -- if you like, with $X=\Bbb N$.   In particular, write out a
direct proof that $(\ell^1)^*$ can be identified with $\ell^{\infty}$.
What happens when $X$ has just two members? or three?
%243H
     
\spheader 243Xm Show that if $(X,\Sigma,\mu)$ is any measure
space and $u\in L^{\infty}_{\Bbb C}(\mu)$, then
     
\Centerline{$\|u\|_{\infty}=\sup\{\|\Real(\zeta u)\|_{\infty}:
\zeta\in\Bbb C,\,|\zeta|=1\}$.}
%243K
     
\spheader 243Xn Let $(X,\Sigma,\mu)$ and $(Y,\Tau,\nu)$ be measure
spaces, and $\phi:X\to Y$ an \imp\ function.   Show that 
$g\phi\in\eusm L^{\infty}(\mu)$ for every $g\in\eusm L^{\infty}(\nu)$, and that the map $g\mapsto g\phi$ induces a linear operator $T:L^{\infty}(\nu)\to L^{\infty}(\mu)$ defined by setting
$T(g^{\ssbullet})=(g\phi)^{\ssbullet}$ for every 
$g\in\eusm L^{\infty}(\nu)$.   (Compare 241Xg.)   Show that
$\|Tv\|_{\infty}=\|v\|_{\infty}$ for every $v\in L^{\infty}(\nu)$.
     
\vspheader{48pt}243Xo For $f$, $g\in C=C([0,1])$, the space of continuous 
real-valued functions on the unit interval $[0,1]$, say
     
\Centerline{$f\le g$ iff $f(x)\le g(x)$ for every $x\in [0,1]$,}
     
\Centerline{$\|f\|_{\infty}=\sup_{x\in [0,1]}|f(x)|$.}
     
\noindent Show that $C$ is a Banach lattice, and that moreover
     
\Centerline{$\|f\vee g\|_{\infty}=\max(\|f\|_{\infty},\|g\|_{\infty})$
whenever $f$, $g\ge 0$,}
     
\Centerline{$\|f\times g\|_{\infty}\le\|f\|_{\infty}\|g\|_{\infty}$ for
all $f$, $g\in C$,}
     
\Centerline{$\|f\|_{\infty}=\min\{\gamma:|f|\le\gamma\chi X\}$ for
every $f\in C$.}
%243K+
     
\leader{243Y}{Further exercises (a)} Let $(X,\Sigma,\mu)$ be a measure
space, and $Y$ a subset of
$X$;  write $\mu_Y$ for the subspace measure on $Y$.   Show that the
canonical map from $L^0(\mu)$ onto
$L^0(\mu_Y)$ (241Yg) induces a canonical map from
$L^{\infty}(\mu)$ onto $L^{\infty}(\mu_Y)$, which is norm-preserving iff
it is injective iff $Y$ has full outer measure.
     
}%end of exercises
     
\endnotes{
\Notesheader{243} I mention the formula
     
\Centerline{$\|u\vee v\|_{\infty}=\max(\|u\|_{\infty},\|v\|_{\infty})$
for $u$, $v\ge 0$}
     
\noindent (243Df) because while it does not characterize $L^{\infty}$
spaces among Banach lattices (see 243Xo), it is in a sense dual to the
characteristic property
     
\Centerline{$\|u+v\|_1=\|u\|_1+\|v\|_1$ for $u$, $v\ge 0$}
     
\noindent of the norm of $L^1$.   (I will return to this in Chapter 35 in the next volume.)
     
The particular set $\eusm L^{\infty}$ I have chosen (243A) is somewhat
arbitrary.   The space $L^{\infty}$ can very well
be described entirely as
a subspace of $L^0$, without going back to functions at all;  see 243Bc,
243Dc.   Just as with $\eusm L^0$ and $\eusm L^1$, there are occasions
when it would be simpler to work with the linear space of essentially
bounded measurable functions from $X$ to $\Bbb R$;  and we now have a
third obvious candidate, the linear space $\eusm
L^{\infty}_{\Sigma}$ of measurable functions from $X$ to $\Bbb R$
which are literally, rather than essentially, bounded, which is itself a
Banach lattice (243Xb).
     
I suppose the most important theorem of this section is 243G,
identifying $L^{\infty}$ with $(L^1)^*$.   This identification is the
chief reason for setting `localizable' measure spaces apart.   The
proof of 243Gb is long because it depends on two separate ideas.   The
Radon-Nikod\'ym theorem deals, in effect, with the totally finite case,
and then in parts (b-v) and (b-vi) of the proof localizability is used
to link the partial solutions $g_F$ together.   Exercise 243Xi is
supposed to
help you to distinguish the two operations.   The map
$T':L^1\to(L^{\infty})^*$ (243Fc) is also very interesting in its way,
but I shall leave it for Chapter 36.
     
243G gives another way of looking at conditional
expectation operators.   If $(X,\Sigma,\mu)$ is a probability space and
$\Tau$ is a $\sigma$-subalgebra of $\Sigma$, of course both $\mu$ and
$\mu\restrp\Tau$ are localizable, so $L^{\infty}(\mu)$ can be identified
with $(L^1(\mu))^*$ and $L^{\infty}(\mu\restrp\Tau)$ can be identified
with $(L^1(\mu\restrp\Tau))^*$.   Now we have the canonical embedding
$S:L^1(\mu\restrp\Tau)\to L^1(\mu)$ (242Jb)  which is a norm-preserving
linear operator, so gives rise to an adjoint  operator 
$S':L^1(\mu)^*\to L^1(\mu\restrp\Tau)^*$ defined by the formula
     
\Centerline{$(S'h)(v)=h(Sv)$ for all $v\in L^1(\mu\restrp\Tau)$, 
$h\in L^1(\mu)^*$.}
     
\noindent Writing $T_{\mu}:L^{\infty}(\mu)\to L^1(\mu)^*$ and
$T_{\mu\restrp\Tau}:L^{\infty}(\mu\restrp\Tau)\to L^1(\mu\restrp\Tau)^*$
for the canonical maps, we get a map
$Q=T_{\mu\restrp\Tau}^{-1}S'T_{\mu}:
L^{\infty}(\mu)\to L^{\infty}(\mu\restrp\Tau)$, defined by saying that
     
\Centerline{$\int Qu\times v=(T_{\mu\restrp\Tau}Qu)(v)
=(S'T_{\mu}u)(v)=(T_{\mu}v)(Su)=\int Su\times v=\int u\times v$}
     
\noindent whenever $v\in L^1(\mu\restrp\Tau)$ and 
$u\in L^{\infty}(\mu)$.   But this agrees with the formula of 242L:  we have
     
\Centerline{$\int Qu\times v=\int u\times v=\int P(u\times v)
=\int Pu\times v$.}
     
\noindent Because $v$ is arbitrary, we must have $Qu=Pu$ for every
$u\in L^{\infty}(\mu)$.   Thus a conditional expectation operator is, 
in a sense, the adjoint of the appropriate embedding operator.
     
The discussion in the last paragraph applies, of course, only to the
restriction $P\restr L^{\infty}(\mu)$ of the conditional expectation
operator to the $L^{\infty}$ space.   Because $\mu$ is totally finite,
$L^{\infty}(\mu)$ is a subspace of $L^1(\mu)$, and the real qualities of
the operator $P$ are related to its behaviour on the whole space $L^1$.
$P:L^1(\mu)\to L^1(\mu\restrp\Tau)$ can also be expressed as an adjoint
operator, but the expression needs more of the theory of Riesz spaces
than I have space for here.   I will return to this topic in Chapter 36.
}%end of notes
     
\discrpage
 

