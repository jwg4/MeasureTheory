\frfilename{mt481.tex}
\versiondate{4.9.09/30.11.12}
\copyrightdate{2000}

\def\Epsilon{\text{E}}
\def\Eta{\text{H}}

\def\chaptername{Gauge integrals}
\def\sectionname{Tagged partitions}

\newsection{481}

I devote this section to establishing some terminology (481A-481B,
481E-481G) %481E 481F 481G
and describing a variety of examples
(481I-481Q), %481I 481J 481K 481L 481M 481N 481O 481P 481Q
some of which will be elaborated later.   The clearest, simplest and
most important example is surely Henstock's integral on a closed bounded
interval (481J), so I recommend turning immediately to that paragraph
and keeping it in mind while studying the notation here.   It may also
help you to make sense of the definitions here if you glance at the
statements of some of the results in \S482;   in this section I give
only the formula defining gauge integrals (481C)\cmmnt{, with some
elementary examples of its use
(481Xb-481Xh)}.  %481Xb 481Xc 481Xd 481Xe 481Xf 481Xg 481Xh

\leader{481A}{Tagged partitions and Riemann sums} The common idea
underlying\cmmnt{ all the constructions of} this chapter is the following.   We
have a set $X$ and a functional $\nu$ defined on some family $\Cal C$ of
subsets of $X$.   We seek to define an integral
$\int fd\nu$, for functions $f$ with domain $X$, as a limit of
{\it finite} {\bf Riemann sums} $\sum_{i=0}^nf(x_i)\nu C_i$, where
$x_i\in X$ and $C_i\in\Cal C$ for $i\le n$.
\dvro{ A}{ There is no strict reason, at this stage, to forbid
repetitions in the string $(x_0,C_0),\ldots,(x_n,C_n)$, but also little
to be gained from allowing them, and it will simplify some of the
formulae below if I say from the outset
that a} {\bf tagged partition} on $X$ will be a finite subset $\pmb{t}$
of $X\times\Cal PX$.

\cmmnt{So one necessary element of the definition will be a
declaration of which tagged partitions
$\{(x_0,C_0),\ldots,\discretionary{}{}{}(x_n,C_n)\}$ will be
employed, in terms, for instance, of which sets $C_i$ are
permitted, whether they are allowed to overlap at
their boundaries, whether they are required to cover the space, and
whether each {\bf tag} $x_i$ is required to belong to the corresponding
$C_i$.}   The next element of the definition will be a description of a
filter $\Cal F$ on the set $T$ of tagged partitions, so that the
integral will be the limit (when it exists) of the sums along the
filter\cmmnt{, as in 481C below}.

\cmmnt{In the formulations studied in this chapter, the $C_i$ will
generally be disjoint, but this is not absolutely essential, and it is
occasionally
convenient to allow them to overlap in `small' sets, as in 481Ya.   In
some cases, we can restrict attention to families for which the $C_i$
are non-empty and have union $X$, so that $\{C_0,\ldots,C_n\}$ is a
partition of $X$ in the strict sense.}%end of comment

\woddheader{481B}{0}{0}{0}{90pt}

\leader{481B}{Notation}\cmmnt{ Let me immediately introduce notations
which will be in general use throughout the chapter.

\medskip

} {\bf (a)}\cmmnt{ First, a shorthand to describe a particular class of
sets of tagged partitions.}   If $X$ is a set, a {\bf straightforward
set of tagged partitions} on $X$ is a set of the form

\Centerline{$T=\{\pmb{t}:\pmb{t}\in[Q]^{<\omega},\,C\cap C'=\emptyset$
whenever $(x,C)$, $(x',C')$ are distinct members of $\pmb{t}\}$}

\noindent where $Q\subseteq X\times\Cal PX$;  I will say that $T$ is
{\bf generated} by $Q$.   \cmmnt{In this case, of
course, $Q$ can be recovered from $T$, since $Q=\bigcup T$.   Note that
no control is imposed on the tags at this point.   It remains
theoretically possible that a pair $(x,\emptyset)$ should belong to $Q$,
though in many applications this will be excluded in one way
or another.}

\spheader 481Bb If $X$ is a set and $\pmb{t}\subseteq X\times\Cal PX$ is
a tagged partition, \cmmnt{I write}

\Centerline{$W_{\pmb{t}}=\bigcup\{C:(x,C)\in\pmb{t}\}$.}

\spheader 481Bc If $X$ is a set, $\Cal C$ is a family of subsets of $X$,
$f$ and $\nu$ are real-valued functions, 
and $\pmb{t}\in[X\times\Cal C]^{<\omega}$ is a tagged
partition, then

\Centerline{$S_{\pmb{t}}(f,\nu)=\sum_{(x,C)\in\pmb{t}}f(x)\nu C$}

\noindent whenever $\pmb{t}\subseteq\dom f\times\dom\nu$.

\leader{481C}{Proposition} Let $X$ be a set, $\Cal C$ a family of
subsets of $X$, $T\subseteq[X\times\Cal C]^{<\omega}$ a
non-empty set of tagged partitions and
$\Cal F$ a filter on $T$.   For real-valued functions $f$ and
$\nu$, set

\Centerline{$I_{\nu}(f)=\lim_{\pmb{t}\to\Cal F}S_{\pmb{t}}(f,\nu)$}

\noindent if this is defined in $\Bbb R$.

(a) $I_{\nu}$ is a linear functional defined on a linear subspace of
$\Bbb R^X$.

(b) Now suppose that $\nu C\ge 0$ for every $C\in\Cal C$.   Then

\quad(i) $I_{\nu}$ is a positive
linear functional\cmmnt{ (definition:  351F)};

\quad(ii) if $f$, $g:X\to\Bbb R$ are such that $|f|\le g$ and
$I_{\nu}(g)$ is defined and equal to $0$, then $I_{\nu}(f)$ is defined
and equal to $0$.

\proof{{\bf (a)} We have only to observe that if $f$, $g$ are real-valued
functions and
$\alpha\in\Bbb R$, then

\Centerline{$S_{\pmb{t}}(f+g,\nu)
=S_{\pmb{t}}(f,\nu)+S_{\pmb{t}}(g,\nu)$,
\quad$S_{\pmb{t}}(\alpha f,\nu)=\alpha S_{\pmb{t}}(f,\nu)$}

\noindent whenever the right-hand sides are defined,
and apply 2A3Sf.

\medskip

{\bf (b)} If $g\ge 0$ in $\Bbb R^X$, that is, $g(x)\ge 0$ for every
$x\in X$, then $S_{\pmb{t}}(g,\nu)\ge 0$ for every $\pmb{t}\in T$, so
the limit $I_{\nu}(g)$, if defined, will also be non-negative.   Next,
if $|f|\le g$, then $|S_{\pmb{t}}(f,\nu)|\le S_{\pmb{t}}(g,\nu)$ for
every $\pmb{t}$, so if $I_{\nu}(g)=0$ then $I_{\nu}(f)$ also is zero.
}%end of proof of 481C

\leader{481D}{Remarks (a)} Functionals
$I_{\nu}=\lim_{\pmb{t}\to\Cal F}S_{\pmb{t}}(.,\nu)$, as described above,
are called {\bf gauge integrals}.

\cmmnt{\spheader 481Db In fact even greater generality is
possible at this point.   There is no reason why $f$ and $\nu$ should
take real values.   All we actually need is an interpretation of sums of
products $f(x)\times\nu C$ in a space in which we can define limits.
So for any linear spaces $U$, $V$ and $W$ with a bilinear functional
$\phi:U\times V\to W$ (253A) and a Hausdorff linear space topology on
$W$, we can set out to construct an integral of a function $f:X\to U$
with respect to a functional $\nu:\Cal C\to V$ as a limit of sums
$S_{\pmb{t}}(f,\nu)=\sum_{(x,C)\in\pmb{t}}\phi(f(x),\nu C)$ in $W$.   I
will not
go farther along this path here.   But it is worth noting that the
constructions of this chapter lead the way to interesting vector
integrals of many types.
}%end of comment

\spheader 481Dc An extension which is\cmmnt{, however,} sometimes useful 
is to allow $\nu$ to be undefined\cmmnt{ (or take values outside
$\Bbb R$, such as $\pm\infty$)} on part of $\Cal C$.   In this case, set
$\Cal C_0=\nu^{-1}[\Bbb R]$.   Provided that
$T\cap[X\times\Cal C_0]^{<\omega}$ belongs to $\Cal F$, we can still 
define $I_{\nu}$, and 481C will still be true.

\leader{481E}{Gauges}\cmmnt{ The most useful method (so far) of
defining filters on sets of
tagged partitions is the following.

\medskip

}{\bf (a)} If $X$ is a set, a {\bf gauge} on
$X$ is a subset $\delta$ of $X\times\Cal PX$.   For a gauge $\delta$, a
tagged partition $\pmb{t}$ is
{\bf $\delta$-fine} if $\pmb{t}\subseteq\delta$.   Now,
for a set $\Delta$ of gauges and a non-empty
set $T$ of tagged partitions, we can
seek to define a filter $\Cal F$ on $T$ as the filter generated by sets
of the form
$T_{\delta}=\{\pmb{t}:\pmb{t}\in T$ is $\delta$-fine$\}$ as $\delta$
runs over $\Delta$.   \cmmnt{Of course we shall need to establish that
$T$ and $\Delta$ are compatible in the sense that
$\{T_{\delta}:\delta\in\Delta\}$ has the finite intersection
property;  this will ensure that there is indeed a filter containing
every $T_{\delta}$ (4A1Ia).}

\cmmnt{In nearly all cases, $\Delta$ will be non-empty and
downwards-directed (that is, for any $\delta_1$, $\delta_2\in\Delta$
there will be a $\delta\in\Delta$ such that
$\delta\subseteq\delta_1\cap\delta_2$);  in this case, we shall
need only to establish that
$T_{\delta}$ is non-empty for every $\delta\in\Delta$.   Note
that the filter on $T$ generated by $\{T_{\delta}:\delta\in\Delta\}$
depends only on $T$ and the filter on $X\times\Cal PX$ generated by
$\Delta$.}

\spheader 481Eb\cmmnt{ The most important gauges (so far) are
`neighbourhood gauges'.}
If $(X,\frak T)$ is a topological space, a {\bf neighbourhood gauge} on
$X$ is a set expressible in the form
$\delta=\{(x,C):x\in X,\,C\subseteq G_x\}$
where $\family{x}{X}{G_x}$ is a family of open sets such that $x\in G_x$
for every $x\in X$.   It is useful to note (i) that the family
$\family{x}{X}{G_x}$ can be recovered from $\delta$\cmmnt{, since
$G_x=\bigcup\{A:(x,A)\in\delta\}$} (ii) that $\delta_1\cap\delta_2$ is a
neighbourhood gauge whenever $\delta_1$ and $\delta_2$ are.
When $(X,\rho)$ is a metric space, \dvro{we have}{we can define a
neighbourhood gauge $\delta_h$ from any function
$h:X\to\ooint{0,\infty}$, setting

\Centerline{$\delta_h
=\{(x,C):x\in X,\,C\subseteq X,\,\rho(y,x)<h(x)$ for every $y\in C\}$.}

\noindent The set of gauges expressible in this form is coinitial with
the set of all neighbourhood gauges and therefore defines the same
filter on any compatible set $T$ of tagged partitions.   Specializing
yet further, we can restrict attention to constant functions $h$,
obtaining} the {\bf uniform metric gauges}

\Centerline{$\delta_{\eta}
=\{(x,C):x\in X,\,C\subseteq X,\,\rho(x,y)<\eta$ for every $y\in C\}$}

\noindent for $\eta>0$, used in the Riemann integral\cmmnt{ (481I)}.
\cmmnt{(The use of the letter `$\delta$' to represent a gauge has
descended from its
traditional appearance in the definition of the Riemann integral.)}

\spheader 481Ec\dvAformerly{4{}82Jb} If $X$ is a set and
$\Delta\subseteq\Cal P(X\times\Cal PX)$ is a family of gauges on $X$, I
will say that $\Delta$ is {\bf countably full} if whenever
$\sequencen{\delta_n}$ is a sequence in $\Delta$, and $\phi:X\to\Bbb N$
is a function, then there is a $\delta\in\Delta$ such that
$(x,C)\in\delta_{\phi(x)}$ whenever $(x,C)\in\delta$.   I will say that
$\Delta$ is {\bf full} if whenever $\family{x}{X}{\delta_x}$ is a family
in $\Delta$, then there is a $\delta\in\Delta$ such that
$(x,C)\in\delta_x$ whenever $(x,C)\in\delta$.

Of course a full set of gauges is countably full.   Observe that if
$(X,\frak T)$ is any topological space, the set of all
neighbourhood gauges on $X$ is full.

\leader{481F}{Residual sets}\dvro{ Let $\frak R$ be a collection }{ The
versatility and power of the methods being introduced here derives from
the insistence on taking {\it finite} sums
$\sum_{(x,C)\in\pmb{t}}f(x)\nu C$, so
that all questions about convergence are concentrated in the final limit
$\lim_{{\pmb{t}}\to\Cal F}S_{\pmb{t}}(f,\nu)$.   Since in a given
Riemann sum we can look at only finitely many sets $C$ of finite
measure, we cannot insist, even when $\nu$ is Lebesgue measure on
$\Bbb R$, that $W_{\pmb{t}}$ should always be $X$.   There are many
other cases in which it is impossible or inappropriate to insist that
$W_{\pmb{t}}=X$ for every tagged partition in $T$.   We shall therefore
need to add something to the definition of the filter $\Cal F$ on $T$
beyond what is possible in the language of 481E.   In the examples
below, the extra condition will always be of the following form.   There
will be a collection $\frak R$ }%end of dvro
of {\bf residual families}
$\Cal R\subseteq\Cal PX$.   It will help to have a phrase corresponding
to the phrase `$\delta$-fine':  if $\Cal R\subseteq\Cal PX$, and
$\pmb{t}$ is a tagged partition on $X$, \cmmnt{I will say that}
$\pmb{t}$ is {\bf $\Cal R$-filling} if
$X\setminus W_{\pmb{t}}\in\Cal R$.   Now, given a family $\frak R$ of
residual sets, and a family $\Delta$ of gauges on $X$, we can seek to
define a filter $\Cal F(T,\Delta,\frak R)$ on $T$ as that generated by
sets of the form $T_{\delta}$, for $\delta\in\Delta$, and $T'_{\Cal R}$,
for $\Cal R\in\frak R$, where 

\Centerline{$T'_{\Cal R}
=\{\pmb{t}:\pmb{t}\in T$ is $\Cal R$-filling$\}.$}

\noindent When there is such a filter\cmmnt{, that is, the family
$\{T_{\delta}:\delta\in\Delta\}\cup\{T'_{\Cal R}:\Cal R\in\frak R\}$ has 
the finite
intersection property}, I will say that $T$ is {\bf compatible} with
$\Delta$ and $\frak R$.

\cmmnt{It is important here to note that we shall {\it not} suppose
that, for a typical residual family
$\Cal R\in\frak R$, subsets of members of $\Cal R$ again belong to
$\Cal R$;  there will frequently be a restriction on the `shape' of
members of $\Cal R$ as well as on their size.   On the other hand, it
will usually be helpful to arrange that $\frak R$ is a filter base, so
that (if $\Delta$ is also
downwards-directed, and neither $\Delta$ nor $\frak R$ is empty) 
we need only show that $T_{\delta}\cap T'_{\Cal R}$ is always
non-empty, and
$\{T_{\delta\Cal R}:\delta\in\Delta,\,\Cal R\in\frak R\}$ will be a base
for $\Cal F(T,\Delta,\frak R)$.}

\cmmnt{If the filter $\Cal F$ is defined as in 481Ea, with no mention
of a family $\frak R$, we can still bring the construction into the
framework considered here by setting $\frak R=\emptyset$.   If it is
convenient to define $T$ in terms which do not impose any requirement on
the sets $W_{\pmb{t}}$, but nevertheless we wish to restrict attention
to sums $S_{\pmb{t}}(f,\nu)$ for which the tagged partition covers the
whole space $X$, we can do so by setting $\frak R=\{\{\emptyset\}\}$.}

\leader{481G}{Subdivisions}\cmmnt{ When we come to analyse the
properties of integrals constructed by the method of 481C, there is an
important approach which depends on the following combination of features.}   I will say that
$(X,T,\Delta,\frak R)$ is a {\bf tagged-partition structure allowing
subdivisions}, {\bf witnessed by} $\Cal C$, if

\inset{(i) $X$ is a set.

(ii) $\Delta$ is a non-empty downwards-directed family of gauges on $X$.

(iii)($\alpha$) $\frak R$ is a non-empty downwards-directed collection
of families of subsets of $X$, all containing $\emptyset$;

\quad($\beta$) for every $\Cal R\in\frak R$ there is an
$\Cal R'\in\frak R$ such that $A\cup B\in\Cal R$ whenever $A$,
$B\in\Cal R'$ are disjoint.

(iv) $\Cal C$ is a family of subsets of $X$ such that
whenever $C$, $C'\in\Cal C$ then
$C\cap C'\in\Cal C$ and $C\setminus C'$ is expressible as the union of a
disjoint finite subset of $\Cal C$.

(v) Whenever $\Cal C_0\subseteq\Cal C$ is finite and $\Cal R\in\frak R$,
there is a finite set $\Cal C_1\subseteq\Cal C$, including $\Cal C_0$,
such that $X\setminus\bigcup\Cal C_1\in\Cal R$.

(vi) $T\subseteq[X\times\Cal C]^{<\omega}$ is\cmmnt{ (in the language
of 481Ba)} a non-empty straightforward set of tagged partitions on $X$.

(vii) Whenever $C\in\Cal C$, $\delta\in\Delta$ and
$\Cal R\in\frak R$ there is a $\delta$-fine tagged partition
$\pmb{t}\in T$ such that $W_{\pmb{t}}\subseteq C$ and
$C\setminus W_{\pmb{t}}\in\Cal R$.}

\leader{481H}{Remarks}\cmmnt{ {\bf (a)} Conditions (ii) and
(iii-$\alpha$) of 481G are included primarily for convenience, since
starting from any $\Delta$ and $\frak R$ we can find 
non-empty directed sets
leading to the same filter $\Cal F(T,\Delta,\frak R)$.   (iii-$\beta$),
on the other hand, is saying something new.

\medskip

}\cmmnt{{\bf (b)} It is important to note, in
(vii) of 481G, that the tags of $\pmb{t}$ there are {\it not} required
to belong to the set $C$.

\medskip

}\cmmnt{{\bf (c)} All the applications below will fall into one
of two classes.   In one type, the residual families $\Cal R\in\frak R$
will be families of `small' sets, in some recognisably measure-theoretic
sense, and, in particular, we shall have subsets of members of any
$\Cal R$ belonging to
$\Cal R$.   In the other type, (vii) of 481G will be true because we can
always find $\pmb{t}\in T$ such that $W_{\pmb{t}}=C$.

\medskip

} {\bf (d)}\cmmnt{ The following elementary fact got left out of
\S136 and Chapter 31.}   Let $\frak A$ be a Boolean algebra and
$C\subseteq\frak A$.   Set

\Centerline{$E
=\{\sup C_0:C_0\subseteq C$ is finite and disjoint$\}$.}

\noindent If $c\Bcap c'$ and $c\Bsetminus c'$ belong to $E$ for all $c$,
$c'\in C$, then $E$ is a subring of $\frak A$.
\prooflet{\Prf\ Write $\Cal D$ for the family of finite disjoint subsets
of $C$.   (i) If $C_0$, $C_1\in\Cal D$, then for $c\in C_0$,
$c'\in C_1$ there is a $D_{cc'}\in\Cal D$ with supremum
$c\Bcap c'$.   Now $D=\bigcup_{c\in C_0,c'\in C_1}D_{cc'}$ belongs to
$\Cal D$ and has supremum $(\sup C_0)\Bcap(\sup C_1)$.   Thus
$e\Bcap e'\in E$ for all $e$, $e'\in E$.   (ii) Of course
$e\Bcup e'\in E$ whenever $e$, $e'\in E$ and $e\Bcap e'=0$.
(iii) Again suppose that $C_0$, $C_1\in\Cal D$.   Then
$c\Bsetminus c'\in E$ for all
$c\in C_0$, $c'\in C_1$.   By (i), $c\Bsetminus\sup C_1\in E$ for every
$c\in C_0$;  by (ii), $(\sup C_0)\Bsetminus(\sup C_1)\in E$.   Thus
$e\Bsetminus e'\in E$ for all $e$, $e'\in E$.   (iv) Putting (ii) and
(iii) together, $e\Bsymmdiff e'\in E$ for all $e$, $e'\in E$.   (v) As
$0=\sup\emptyset$ belongs to $E$, $E$ is a subring of $\frak A$.\ \Qed}

In particular, if $\Cal C\subseteq\Cal PX$ has the properties in (iv) of
481G, then

\Centerline{$\Cal E
=\{\bigcup\Cal C_0:\Cal C_0\subseteq\Cal C$ is finite and disjoint$\}$}

\noindent is a ring of subsets of $X$.

\spheader 481He Suppose that $X$ is a set and that
$\frak R\subseteq\Cal P\Cal PX$ satisfies (iii) of 481G.   Then for
every $\Cal R\in\frak R$ there is a non-increasing sequence
$\sequencen{\Cal R_n}$ in
$\frak R$ such that $\bigcup_{i\le n}A_i\in\Cal R$ whenever
$A_i\in\Cal R_i$ for $i\le n$ and $\langle A_i\rangle_{i\le n}$ is
disjoint.   \prooflet{\Prf\ Take
$\Cal R_0\in\frak R$ such that $\Cal R_0\subseteq\Cal R$ and
$A\cup B\in\Cal R$ for all disjoint $A$, $B\in\Cal R_0$;  similarly, for
$n\in\Bbb N$, choose
$\Cal R_{n+1}\in\frak R$ such that $\Cal R_{n+1}\subseteq\Cal R_n$ and
$A\cup B\in\Cal R_n$ for all disjoint $A$, $B\in\Cal R_{n+1}$.   Now,
given that $A_i\in\Cal R_i$ for $i\le n$ and
$\langle A_i\rangle_{i\le n}$ is disjoint, we see by downwards induction
on $m$ that $\bigcup_{m<i\le n}A_i\in\Cal R_m$ for each $m\le n$, so
that $\bigcup_{i\le n}A_i\in\Cal R$.\ \Qed}

\spheader 481Hf If $(X,T,\Delta,\frak R)$ is a tagged-partition
structure allowing subdivisions, then $T$ is compatible with $\Delta$
and $\frak R$\cmmnt{ in the sense of 481F}.
\prooflet{\Prf\ $\emptyset\in T$ so $T$ is not empty.   Take
$\delta\in\Delta$ and $\Cal R\in\frak R$.   Let $\sequence{i}{\Cal R_i}$
be a sequence in $\frak R$ such that $\bigcup_{i\le n}A_i\in\Cal R$
whenever $A_i\in\Cal R_i$ for $i\le n$ and $\langle A_i\rangle_{i\le n}$
is disjoint ((e) above).   There is a finite
set $\Cal C_1\subseteq\Cal C$ such that
$X\setminus\bigcup\Cal C_1\in\Cal R_0$, by 481G(iv).   By (d), there is
a disjoint family $\Cal C_0\subseteq\Cal C$ such that
$\bigcup\Cal C_0=\bigcup\Cal C_1$;  enumerate $\Cal C_0$ as
$\ofamily{i}{n}{C_i}$.   For each $i<n$, there is a $\delta$-fine
$\pmb{t}_i\in T$ such that $W_{\pmb{t}_i}\subseteq C_i$ and
$C_i\setminus W_{\pmb{t}_i}\in\Cal R_{i+1}$, by 481G(vii).   Set
$\pmb{t}=\bigcup_{i<n}\pmb{t}_i$;  then $\pmb{t}\in T$ is $\delta$-fine,
and

\Centerline{$X\setminus W_{\pmb{t}}
=(X\setminus\bigcup\Cal C_1)
  \cup\bigcup_{i<n}(C_i\setminus W_{\pmb{t}_i})
\in\Cal R$}

\noindent by the choice of $\sequence{i}{\Cal R_i}$.   Thus we have a
$\delta$-fine $\Cal R$-filling member of $T$;  as $\delta$ and $\Cal R$
are arbitrary, $T$ is compatible with $\Delta$ and $\frak R$.\ \Qed
}%end of prooflet

\cmmnt{\spheader 481Hg For basic results which depend on `subdivisions'
as described in 481G(vii), see 482A-482B below.   A hypothesis
asserting the existence of a different
sort of subdivision appears in 482G(iv).}

\leader{481I}{}\cmmnt{ I now run through some simple examples of these
constructions, limiting myself for the moment to the definitions, the
proofs that $T$ is compatible with $\Delta$ and $\frak R$, and (when
appropriate) the proofs that the structures allow subdivisions.

\medskip

\noindent}{\bf The proper Riemann integral} Fix a
non-empty closed interval $X=[a,b]\subseteq\Bbb R$.   Write
$\Cal C$ for the set of all intervals (open, closed or
half-open, and allowing the empty set to count as an interval) 
included in $[a,b]$, and set
$Q=\{(x,C):C\in C,\,x\in\overline{C}\}$;  let $T$ be the straightforward
set of tagged partitions generated by $Q$.   Let $\Delta$ be the set of
uniform metric gauges on $X$, and $\frak R=\{\{\emptyset\}\}$.   Then
$(X,T,\Delta,\frak R)$ is a
tagged-partition structure allowing subdivisions, witnessed by $\Cal C$.
If $a<b$, then $\Delta$ is not countably full.

\proof{ (i), (iii) and (vi) of 481G are trivial and (ii), (iv) and (v)
are elementary.   As for (vii), given $\eta>0$ and $C\in\Cal C$, take a
disjoint family $\familyiI{C_i}$ of non-empty intervals of length less than
$2\eta$ covering $C$, and $x_i$ to be the midpoint of $C_i$ for
$i\in I$; then $\pmb{t}=\{(x_i,C_i):i\in I\}$ belongs to
$T$ and is $\delta_{\eta}$-fine, in the language of 481E, and
$W_{\pmb{t}}=C$.

Of course (apart from the trivial case $a=b$)
$\Delta$ is not countably full, since if we take $\delta_n$ to be
the gauge $\{(x,C):|x-y|<2^{-n}$ for every $y\in C\}$ and any unbounded
function $\phi:[a,b]\to\Bbb N$, there is no $\delta\in\Delta$ such that
$(x,C)\in\delta_{\phi(x)}$ whenever $(x,C)\in\Delta$.
}%end of proof of 481I

\leader{481J}{\bf The Henstock integral on a bounded
interval}\cmmnt{ ({\smc Henstock 63})} Take $X$, $\Cal C$, $T$
and $\frak R$ as in 481I.   This time, let $\Delta$ be the set of
{\it all} neighbourhood gauges on $[a,b]$.   Then
$(X,T,\Delta,\frak R)$ is a tagged-partition structure allowing
subdivisions, witnessed by $\Cal C$, and $\Delta$ is countably full.

\proof{ Again, only 481G(vii) needs more than a moment's consideration.
Take any
$C\in\Cal C$.   If $C=\emptyset$, then $\pmb{t}=\emptyset$ will suffice.
Otherwise, set $a_0=\inf C$, $b_0=\sup C$ and let $T_0$ be the
family of $\delta$-fine partitions $\pmb{t}\in T$ such that
$W_{\pmb{t}}$ is a relatively closed initial subinterval of $C$, that
is, is of the
form $C\cap[a_0,y_{\pmb{t}}]$ for some $y_{\pmb{t}}\in[a_0,b_0]$.   Set
$A=\{y_{\pmb{t}}:\pmb{t}\in T_0\}$.   I have to show that there is a
$\pmb{t}\in T_0$ such that $W_{\pmb{t}}=C$, that is, that $b_0\in A$.

Observe that there is an $\eta_0>0$ such that
$(a_0,A)\in\delta$ whenever
$A\subseteq[a,b]\cap[a_0-\eta_0,a_0+\eta_0]$, and now
$\{(a_0,[a_0,a_0+\eta_0]\cap C)\}$ belongs to $T_0$, so
$\min(a_0+\eta_0,b_0)\in A$ and $A$ is a non-empty subset of
$[a_0,b_0]$.
It follows that $c=\sup A$ is defined in $[a_0,b_0]$.   Let $\eta>0$ be
such that $(c,A)\in\delta$ whenever
$A\subseteq[a,b]\cap[c-\eta,c+\eta]$.   There is some
$\pmb{t}\in T_0$ such that $y_{\pmb{t}}\ge c-\eta$.   If
$y_{\pmb{t}}=b_0$, we can stop.   Otherwise, set
$C'=C\cap\ocint{y_{\pmb{t}},c+\eta}$.   Then $(c,C')\in\delta$ and
$c\in\overline{C'}$ and $C'\cap W_{\pmb{t}}$ is empty, so
$\pmb{t}'=\pmb{t}\cup\{(c,C')\}$ belongs to $T_0$ and
$y_{\pmb{t}'}=\min(c+\eta,b_0)$.   Since $y_{\pmb{t}'}\le c$, this shows
that $y_{\pmb{t}'}=c=b_0$ and again $b_0\in A$, as required.

$\Delta$ is full just because it is the family of neighbourhood gauges.
}%end of proof of 481J
% maybe {\smc Cousin 1895} relevant if I can be bothered to check

\leader{481K}{The Henstock integral on $\Bbb R$} This time, set
$X=\Bbb R$ and let $\Cal C$ be the family of all bounded
intervals in $\Bbb R$.   Let $T$ be the straightforward set of
tagged partitions generated by
$\{(x,C):C\in\Cal C$, $x\in\overline{C}\}$.   Following 481J, let $\Delta$ be the set of all neighbourhood
gauges on $\Bbb R$.   This time, set
$\frak R=\{\Cal R_{ab}:a\le b\in\Bbb R\}$, where
$\Cal R_{ab}=\{\Bbb R\setminus[c,d]:c\le a,\,d\ge b\}\cup\{\emptyset\}$.
Then $(X,T,\Delta,\frak R)$ is a tagged-partition structure allowing
subdivisions, witnessed by $\Cal C$.

\proof{ This time we should perhaps take a moment to look at
(iii) of 481G.   But all we need to note is that
$\Cal R_{ab}\cap\Cal R_{a'b'}=\Cal R_{\min(a,a'),\max(b,b')}$, and that
any two members of $\Cal R_{ab}$ have non-empty intersection.
Conditions (i), (ii), (iv), (v) and (vi) of 481G
are again elementary, so once more we are left with (vii).   But this
can be dealt with by exactly the same argument as in 481J.
}%end of proof of 481K

\leader{481L}{The symmetric Riemann-complete integral}\cmmnt{ (cf.\
{\smc Carrington 72}, chap.\ 3)} Again take $X=\Bbb R$, and $\Cal C$ the
set of all
bounded intervals in $\Bbb R$.   This time, take $T$ to be the
straightforward set of tagged partitions generated by the set
of pairs $(x,C)$ where $C\in\Cal C\setminus\{\emptyset\}$ and $x$ is the 
{\it midpoint} of
$C$.   As in 481K, take $\Delta$ to be the set of all neighbourhood
gauges on
$\Bbb R$;  but this time take $\frak R=\{\Cal R'_a:a\ge 0\}$, where
$\Cal R'_a=\{\Bbb R\setminus[-c,c]:c\ge a\}\cup\{\emptyset\}$.   Then
$T$ is compatible with $\Delta$ and $\frak R$.

\proof{ Take $\delta\in\Delta$ and $\Cal R\in\frak R$.   For $x\ge 0$,
let $\theta(x)>0$ be such that
$(x,D)\in\delta$ whenever $D\subseteq[x-\theta(x),x+\theta(x)]$
and $(-x,D)\in\delta$ whenever
$D\subseteq[-x-\theta(x),-x+\theta(x)]$.   Write $A$ for the set of
those $a>0$ such that there is a finite sequence $(a_0,\ldots,a_n)$
such that $0<a_0<a_1<\ldots<a_n=a$, $a_0\le\theta(0)$ and
$a_{i+1}-a_i\le 2\theta(\bover12(a_i+a_{i+1}))$ for $i<n$.

\Quer\ Suppose, if possible, that $A$ is bounded above.   Then
$c=\inf(\coint{0,\infty}\setminus\overline{A})$ is defined in
$\coint{0,\infty}$.   Observe that if $0<a\le\theta(0)$, then the
one-term sequence $\fraction{a}$ witnesses that $a\in A$.   
So $c\ge\theta(0)>0$.   Now there
must be $u$, $v$ such that $c<u<v<\min(c+\theta(c),2c)$ and
$\ooint{u,v}\cap A=\emptyset$;  on the other hand, the interval
$\ooint{2c-v,2c-u}$ must contain a point $x$ of $A$.   Set $y=2c-x$.
Then we can find $a_0<\ldots<a_n=x$ such that $0<a_0\le\theta(0)$ and
$a_{i+1}-a_i\le 2\theta(\bover12(a_i+a_{i+1}))$ for $i<n$;  setting
$a_{n+1}=y$, we see that $\langle a_i\rangle_{i\le n+1}$ witnesses that
$y\in A$, though $y\in\ooint{u,v}$.\ \Bang\

This contradiction shows that $A$ is unbounded above.   So now suppose
that $\Cal R=\Cal R_a$ where $a\ge 0$.   Take $a_0,\ldots,a_n$ such that
$0<a_0<\ldots<a_n$, $0<a_0\le\theta(0)$ and
$a_{i+1}-a_i\le 2\theta(\bover12(a_i+a_{i+1}))$ for $i<n$, and
$a_n\ge a$.
For $i<n$, set $x_i=\bover12(a_i+a_{i+1})$, $C_i=\ocint{a_i,a_{i+1}}$,
$x'_i=-x_i$, $C'_i=\coint{-a_{i+1},a_i}$.   Then $x_i$, $x'_i$ are the
midpoints of $C_i$, $C'_i$ and (by the choice of the function $\theta$)
$(x_i,C_i)\in\delta$, $(x'_i,C'_i)\in\delta$ for $i<n$.   So
if we set

\Centerline{$\pmb{t}
=\{(x_i,C_i):i<n\}\cup\{(x'_i,C'_i):i<n\}\cup\{(0,[-a_0,a_0])\}$}

\noindent we shall obtain a $\delta$-fine $\Cal R$-filling member of
$T$.

As $\Delta$ and $\frak R$ are both downwards-directed, this is enough to
show that $T$ is compatible with $\Delta$ and $\frak R$.
}%end of proof of 481L

\leader{481M}{The McShane integral on an 
interval}\cmmnt{ ({\smc McShane 73})} As in 481J, take
$X=[a,b]$ and let $\Cal C$ be the family of subintervals of
$[a,b]$.   This time, take $T$ to be the straightforward set of tagged
partitions generated by $Q=X\times\Cal C$\cmmnt{, so that {\it no}
condition is imposed relating the tags to their associated intervals}.
As in 481J, let $\Delta$ be the set of all neighbourhood gauges on $X$,
and $\frak R=\{\{\emptyset\}\}$.   \cmmnt{Proceed as before.   Since
the only change is that $Q$ and $T$ have been enlarged,}
$(X,T,\Delta,\frak R)$ is\cmmnt{ still} a tagged-partition structure
allowing subdivisions, witnessed by $\Cal C$.

\leader{481N}{The McShane integral on a topological
space}\cmmnt{ ({\smc Fremlin 95})} Now let
\ifdim\pagewidth=390pt\penalty-1000\fi
$(X,\frak T,\Sigma,\mu)$ be any effectively locally finite
$\tau$-additive topological measure space, and take
$\Cal C=\{E:E\in\Sigma,\,\mu E<\infty\}$,
$Q=X\times\Cal C$;  let $T$ be the straightforward set of tagged
partitions generated by $Q$.   Again let
$\Delta$ be the set of all neighbourhood gauges on $X$.   \cmmnt{This
time, define $\frak R$ as follows.}   For any set $E\in\Sigma$ of finite
measure and $\eta>0$, let $\Cal R_{E\eta}$ be the set
$\{F:F\in\Sigma,\,\mu(F\cap E)\le\eta\}$, and set
$\frak R=\{\Cal R_{E\eta}:\mu E<\infty,\,\eta>0\}$.   Then
$(X,T,\Delta,\frak R)$ is a tagged-partition structure allowing
subdivisions, witnessed by $\Cal C$.

\proof{ As usual, everything is elementary except perhaps 481G(vii).
But if $C\in\Cal C$, $\delta\in\Delta$, $E\in\Sigma$, $\mu E<\infty$ and
$\eta>0$, take for each $x\in X$ an open set $G_x$ containing $x$ such
that $(x,A)\in\delta$ whenever $A\subseteq G_x$.
$\{G_x:x\in X\}$ is an open cover of $X$, so by 414Ea there is a
finite family $\ofamily{i}{n}{x_i}$ in $X$ such that
$\mu(E\cap C\setminus\bigcup_{i<n}G_{x_i})\le\eta$;  setting
$C_i=C\cap G_{x_i}\setminus\bigcup_{j<i}G_{x_j}$ for
$i<n$, we get a $\delta$-fine tagged partition
$\pmb{t}=\{(x_i,C_i):i<n\}$ such that
$C\setminus W_{\pmb{t}}\in\Cal R_{E\eta}$.
}%end of proof of 481N

\leader{481O}{Convex partitions in $\BbbR^r$} Fix $r\ge 1$.   Let us say
that a {\bf convex polytope} in $\BbbR^r$ is a non-empty bounded set
expressible as the intersection of finitely many open or closed
half-spaces;  let
$\Cal C$ be the family of convex polytopes in $X=\BbbR^r$, and $T$ the
straightforward set of tagged partitions generated by
$\{(x,C):x\in\overline{C}\}$.   Let $\Delta$ be the set of
neighbourhood gauges on $\BbbR^r$.   For $a\ge 0$, let $\Cal C_a$ be the
set of closed convex polytopes $C\subseteq\BbbR^r$ such that, for some
$b\ge a$, $B(0,b)\subseteq C\subseteq B(0,2b)$, where $B(0,b)$ is the
ordinary Euclidean ball with centre $0$ and radius $b$;  set
$\Cal R_a=\{\BbbR^r\setminus C:C\in\Cal C_a\}\cup\{\emptyset\}$, and
$\frak R=\{\Cal R_a:a\ge 0\}$.   Then $(X,T,\Delta,\frak R)$ is a
tagged-partition structure allowing subdivisions, witnessed by $\Cal C$.

%what about Saks-regular partitions?  see \S475
%note that $\frak R$ looks just peculiar at the moment

\proof{ As usual, only 481G(vii) requires thought.

\medskip

{\bf (a)} We need a geometrical fact:  if $C\in\Cal C$,
$x\in\overline{C}$ and $y\in C$, then $\alpha y+(1-\alpha)x\in C$ for
every $\alpha\in\ocint{0,1}$.   \Prf\ The family of sets
$C\subseteq\BbbR^r$ with this property is closed under finite
intersections and contains all half-spaces.\ \QeD\   It follows that if
$C_1$, $C_2\in\Cal C$ are not disjoint, then
$\overline{C_1\cap C_2}=\overline{C}_1\cap\overline{C}_2$.

\medskip

{\bf (b)} Write $\Cal D\subseteq\Cal C$ for the family of products of
non-empty bounded intervals in $\Bbb R$.   The next step is to show that
if $D\in\Cal D$ and $\delta\in\Delta$, then there is a
$\delta$-fine tagged partition $\pmb{t}\in T$ such that $W_{\pmb{t}}=D$
and $\pmb{t}\subseteq\BbbR^r\times\Cal D$.
\Prf\ Induce on $r$.   For $r=1$ this is just 481J again.   For the
inductive step to $r+1$, suppose that
$D\subseteq\BbbR^{r+1}$ is a product of bounded intervals and that
$\delta$ is a neighbourhood gauge on $\BbbR^{r+1}$.   Identifying
$\BbbR^{r+1}$ with $\BbbR^r\times\Bbb R$, express $D$ as $D'\times L$,
where $D'\subseteq\BbbR^r$ is a product of bounded intervals and
$L\subseteq\Bbb R$ is a bounded interval.   For $y\in\BbbR^r$,
$\alpha\in\Bbb R$ let $G(y,\alpha)$, $H(y,\alpha)$ be open sets
containing $y$, $\alpha$ respectively such that
$((y,\alpha),A)\in\delta$ whenever
$A\subseteq G(y,\alpha)\times H(y,\alpha)$.
For $y\in\BbbR^r$, set $\delta_y=\{(\alpha,A):\alpha\in\Bbb
R,\,A\subseteq H(y,\alpha)\}$;  then $\delta_y$ is a neighbourhood gauge
on $\Bbb R$.

By the one-dimensional case there is a $\delta_y$-fine tagged partition
$\pmb{s}_y\in T_1$ such that $W_{\pmb{s}_y}=L$, where I write $T_1$ for
the set of tagged partitions used in 481K.
Set

\Centerline{$\delta'
=\{(y,A):y\in\BbbR^r,\,A\subseteq G(y,\alpha)$ for every
$(\alpha,F)\in\pmb{s}_y\}$.   }

$\delta'$ is a neighbourhood gauge on $\BbbR^r$.   By the inductive
hypothesis, there is a $\delta'$-fine tagged partition $\pmb{u}\in T_r$
such that $W_{\pmb{u}}=D'$, where here $T_r$ is the set of tagged
partitions on $\BbbR^r$ corresponding to the $r$-dimensional version of
this result.   Consider the family

\Centerline{$\pmb{t}
=\{((y,\alpha),E\times F):(y,E)\in\pmb{u},\,(\alpha,F)\in\pmb{s}_y\}$.}

\noindent For $(y,E)\in\pmb{u}$, $(\alpha,F)\in\pmb{s}_y$, we have

\Centerline{$y\in\overline{E}$,
\quad$E\subseteq G(y,\alpha)$,
\quad$\alpha\in\overline{F}$,
\quad$F\subseteq H(y,\alpha)$,}

\noindent so

\Centerline{$(y,\alpha)\in\overline{E\times F}$,
\quad$E\times F\subseteq G(y,\alpha)\times H(y,\alpha)$,}

\noindent and $((y,\alpha),E\times F)\in\delta$.
If $((y,\alpha),E\times F)$, $((y',\alpha'),E'\times F')$ are distinct
members of $\pmb{t}$, then either $(y,E)\ne(y',E')$ so
$E\cap E'=\emptyset$ and $(E\times F)\cap(E'\times F')$ is empty, or
$y=y'$ and $(\alpha,F)$, $(\alpha',F')$ are distinct members of
$\pmb{s}_y$, so that $F\cap F'=\emptyset$ and again $E\times F$,
$E'\times F'$ are disjoint.    Thus $\pmb{t}$ is a $\delta$-fine member
of $T_{r+1}$.   Finally,

\Centerline{$W_{\pmb{t}}
=\bigcup_{(y,E)\in\pmb{u}}\bigcup_{(\alpha,F)\in\pmb{s}_y}
   E\times F
=\bigcup_{(y,E)\in\pmb{u}}E\times L=D'\times L=D$.}

\noindent So the induction proceeds.\ \Qed

\medskip

{\bf (c)} Now suppose that $C_0$ is an arbitrary member of $\Cal C$ and
that $\delta$ is a neighbourhood gauge on $\BbbR^r$.   Set

\Centerline{$\delta'=\delta\cap\{(x,A):$ either $x\in\overline{C}_0$ or
$A\cap\overline{C}_0=\emptyset\}$.}

\noindent Then $\delta'$ is a neighbourhood gauge on $\BbbR^r$, being
the intersection of $\delta$ with the neighbourhood gauge associated
with the family $\family{x}{\BbbR^r}{U_x}$, where $U_x=\BbbR^r$ if
$x\in\overline{C}_0$, $\BbbR^r\setminus\overline{C}_0$ otherwise.   Let
$D\in\Cal D$ be such that $C_0\subseteq D$.   By (b), there is a
$\delta'$-fine tagged partition $\pmb{t}\in T$ such that
$W_{\pmb{t}}=D$.   Set
$\pmb{s}=\{(x,C\cap C_0):(x,C)\in\pmb{t},\,C\cap C_0\ne\emptyset\}$.
Since $\pmb{t}\subseteq\delta'$, $x\in\overline{C}_0$ whenever
$(x,C)\in\pmb{t}$ and
$C\cap C_0\ne\emptyset$.   By (a), $x\in\overline{C\cap C_0}$ for all
such pairs $(x,C)$;  and of course $(x,C\cap C_0)\in\delta$ for every
$(x,C)\in\pmb{t}$.   So $\pmb{s}$ belongs to $T$, and
$W_{\pmb{s}}=W_{\pmb{t}}\cap C_0=C_0$.   As $C_0$ and $\delta$ are
arbitrary, 481G(vii) is satisfied.
}%end of proof of 481O

\leader{481P}{Box products}\cmmnt{ (cf. {\smc Muldowney 87}, Prop.\
1)} Let $\familyiI{(X_i,\frak T_i)}$ be a non-empty family of non-empty
compact metrizable spaces with product $(X,\frak T)$.   Set
$\pi_i(x)=x(i)$ for $x\in X$ and $i\in I$.   For each $i\in I$, let
$\Cal C_i\subseteq\Cal PX_i$ be such that ($\alpha$) whenever $E$,
$E'\in\Cal C_i$ then
$E\cap E'\in\Cal C_i$ and $E\setminus E'$ is expressible as the union of
a disjoint finite subset of $\Cal C_i$ ($\beta$) $\Cal C_i$ includes a
base for $\frak T_i$.

Let $\Cal C$ be the set of subsets of $X$ of the form

\Centerline{$C=\{X\cap\bigcap_{i\in J}\pi_i^{-1}[E_i]:
J\in[I]^{<\omega},\,E_i\in\Cal C_i$ for every $i\in J\}$,}

\noindent and let $T$ be the straightforward set of tagged partitions
generated by $\{(x,C):C\in\Cal C,\,x\in\overline{C}\}$.   Let
$\Delta$ be the set of those neighbourhood gauges $\delta$ on $X$
defined by families $\family{x}{X}{G_x}$ of open sets such that, for
some countable $J\subseteq I$, every $G_x$ is determined by coordinates
in $J$\cmmnt{ (definition:  254M)}.   Then
$(X,T,\Delta,\{\{\emptyset\}\})$ is a tagged-partition structure
allowing subdivisions, witnessed by $\Cal C$.    $\Delta$ is countably
full;  $\Delta$ is full iff $I'=\{i:\#(X_i)>1\}$ is countable.

\proof{ Conditions (i), (iii) and (vi) of 481G are trivial, and (ii),
(iv) and (v) are elementary;  so we are left with (vii), as usual.
\Quer\ Suppose, if possible, that $C\in\Cal C$ and $\delta\in\Delta$ are
such that there is no $\delta$-fine $\pmb{t}\in T$ with $W_{\pmb{t}}=C$.
Let $\family{x}{X}{G_x}$ be the family of open sets determining
$\delta$, and $J\subseteq I$ a non-empty countable set such that $G_x$
is determined by coordinates in $J$ for every $x\in X$.   For $i\in J$,
let $\Cal C'_i\subseteq\Cal C_i$ be a countable base for $\frak T_i$
(4A2P(a-iii)), and take a sequence $\sequencen{(i_n,E_n)}$ running over
$\{(i,E):i\in J,\,E\in\Cal C'_i\}$.

Write $\Cal D=\{W_{\pmb{t}}:\pmb{t}\in T$ is $\delta$-fine$\}$.
Note that if $D_1$, $D_2\in\Cal D$ are disjoint then
$D_1\cup D_2\in\Cal D$.   So if
$D\in\Cal C\setminus\Cal D$ and $C\in\Cal C$, there must be some
$D'\in\Cal C\setminus\Cal D$ such that either $D'\subseteq D\cap C$ or
$D'\subseteq D\setminus C$, just because $\Cal C$ satisfies 481G(iv).
Now choose $\sequencen{C_n}$ inductively so that $C_0=C$ and

\Centerline{$C_n\in\Cal C\setminus\Cal D$,}

\Centerline{either $C_{n+1}\subseteq C_n\cap\pi_{i_n}^{-1}[E_n]$ or
$C_{n+1}\subseteq C_n\setminus\pi_{i_n}^{-1}[E_n]$}

\noindent for every $n\in\Bbb N$.   Because $X$, being a product of
compact spaces, is compact, there is an
$x\in\bigcap_{n\in\Bbb N}\overline{C_n}$.   We know that $G_x$ is
determined by coordinates in $J$, so
$G_x=\tilde\pi^{-1}[\tilde\pi[G_x]]$, where $\tilde\pi$ is the canonical
map from $X$ onto $Y=\prod_{i\in J}X_i$.   $V=\tilde\pi[G_x]$ is open,
so there must be a finite set $K\subseteq J$ and a family
$\family{i}{K}{V_i}$ such that $x(i)\in V_i\in\frak T_i$ for every
$i\in K$ and $\{y:y\in Y,\,y(i)\in V_i$ for every $i\in K\}$ is included
in $V$.   This means that
$\{z:z\in X,\,z(i)\subseteq V_i$ for every $i\in K\}$ is included in
$G_x$.   Now, for each $i\in K$, there is some $m\in\Bbb N$ such that
$i=i_m$ and
$x(i)\in E_m\subseteq V_i$.   Because $x\in\overline{C}_{m+1}$,
$\pi_{i_m}^{-1}[E_m]$ cannot be disjoint from $C_{m+1}$, and
$C_{m+1}\subseteq\pi_{i_m}^{-1}[E_m]\subseteq\pi_i^{-1}[V_i]$.

But this means that, for any $n$ large enough, $C_n\subseteq G_x$ and
$\pmb{t}=\{(x,C_n)\}$ is a $\delta$-fine member of $T$ with
$W_{\pmb{t}}=C_n$;  contradicting the requirement that
$C_n\notin\Cal D$.\ \Bang

This contradiction shows that 481G(vii) also is satisfied.

To see that $\Delta$ is countably full, note that if $\sequencen{\delta_n}$
is a sequence in $\Delta$, we have for each $n\in\Bbb N$ a countable set
$J_n\subseteq I$ and a family $\family{x}{X}{G_{nx}}$ of open sets, all
determined by coordinates in $J_n$, such
that $x\in G_{nx}$ and $(x,C)\in\delta_n$ whenever $x\in X$ and
$C\subseteq G_{nx}$.   Now, given $\phi:X\to\Bbb N$, set
$\delta=\{(x,C):x\in X$, $C\subseteq G_{\phi(x),x}\}$, and observe that
$\delta\in\Delta$ and that $(x,C)\in\delta_{\phi(x)}$ whenever
$(x,C)\in\Delta$.

If $I'$ is countable, then $\Delta$ is the set of all neighbourhood gauges
on $X$, so is full.   If $I'$ is uncountable, then for $j\in I'$ and
$x\in X$ choose a proper open subset $H_{jx}$ of $X_j$ containing
$x(j)$ and set $G_{jx}=\{y:y\in X$, $y(j)\in H_{jx}\}$.   For $j\in I'$ set
$\delta_j=\{(x,C):C\subseteq G_{jx}\}\in\Delta$.   Let $\phi:X\to I'$ be
any function such that $\phi[X]$ is uncountable;
then there is no $\delta\in\Delta$ such that
$(x,C)\in\delta_{\phi(x)}$ whenever $(x,C)\in\delta$, so
$\family{x}{X}{\delta_{\phi(x)}}$ witnesses that $\Delta$ is not full.
}%end of proof of 481P

\leader{481Q}{The approximately continuous Henstock
integral}\cmmnt{ ({\smc Gordon 94}, chap.\ 16)} Let $\mu$ be
Leb\discretionary{-}{}{}esgue
measure on $\Bbb R$.   As in 481K, let $\Cal C$ be the family of
non-empty bounded intervals in $\Bbb R$, $T$
the straightforward set of tagged partitions generated by
$\{(x,C):C\in\Cal C$, $x\in\overline{C}\}$, and
$\frak R=\{\Cal R_{ab}:a,\,b\in\Bbb R\}$, where
$\Cal R_{ab}=\{\Bbb R\setminus[c,d]:c\le a,\,d\ge b\}\cup\{\emptyset\}$
for $a$, $b\in\Bbb R$.

\cmmnt{This time, define gauges as follows.}   Let $\Epsilon$ be the
set of families $\pmb{e}=\family{x}{\Bbb R}{E_x}$ where every $E_x$ is a
measurable set containing $x$ such that $x$ is a density point of
$E_x$\cmmnt{ (definition:  223B)}.   For
$\pmb{e}=\family{x}{\Bbb R}{E_x}\in\Epsilon$, set

\Centerline{$\delta_{\pmb{e}}=\{(x,C):C\in\Cal C,
  \,x\in\overline{C},\,\inf C\in E_x$ and $\sup C\in E_x\}$.}

\noindent Set $\Delta=\{\delta_{\pmb{e}}:\pmb{e}\in\Epsilon\}$.   Then
$(X,T,\Delta,\frak R)$ is a tagged-partition structure allowing
subdivisions, witnessed by $\Cal C$, and $\Delta$ is full.

\proof{{\bf (a)} Turning to 481G, we find, as usual, that most of the
conditions are satisfied for elementary reasons.   Since we have here a
new kind of gauge, we had better check 481G(ii);  but if
$\pmb{e}=\family{x}{\Bbb R}{E_x}$ and
$\pmb{e}'=\family{x}{\Bbb R}{E'_x}$ both belong to $\Epsilon$, so does
$\pmb{e}\wedge\pmb{e}'=\family{x}{\Bbb R}{E_x\cap E'_x}$, because

$$\eqalign{\liminf_{\eta\downarrow 0}\Bover1{2\eta}
&\mu([x-\eta,x+\eta]\cap E_x\cap E'_x)\cr
&\ge\lim_{\eta\downarrow 0}\Bover1{2\eta}
  \bigl(\mu([x-\eta,x+\eta]\cap E_x)
  +\mu([x-\eta,x+\eta]\cap E'_x)-2\eta\bigr)
=1\cr}$$

\noindent for every $x$;  and now
$\delta_{\pmb{e}}\cap\delta_{\pmb{e}'}=\delta_{\pmb{e}\wedge\pmb{e}'}$
belongs to $\Delta$.   Everything else we have done before, except, of
course, (vii).

\medskip

{\bf (b)} So take any $\delta\in\Delta$;  express $\delta$ as
$\delta_{\pmb{e}}$, where
$\pmb{e}=\family{x}{\Bbb R}{E_x}\in\Epsilon$.   For $x$, $y\in\Bbb R$,
write $x\frown y$ if $x\le y$ and
$E_x\cap E_y\cap[x,y]\ne\emptyset$;  note that we always have
$x\frown x$.   For $x\in\Bbb R$, let $\eta_x>0$ be such that
$\mu(E_x\cap[x-\eta,x+\eta])\ge\bover53\eta$ whenever
$0\le\eta\le\eta_x$.

Fix $a<b$ in $\Bbb R$ for the moment.   Say that a finite string
$(x_0,\ldots,x_n)$ is `acceptable' if
$a\le x_0\frown\ldots\frown x_n\le b$ and
$\mu([x_0,x_n]\cap\bigcup_{i<n}E_{x_i}^+)\ge\bover12(x_n-x_0)$, where
$E_x^+=E_x\cap\coint{x,\infty}$ for $x\in\Bbb R$.   Observe that if
$(x_0,\ldots,x_m)$ and $(x_m,x_{m+1},\ldots,x_n)$ are both acceptable,
so is $(x_0,\ldots,x_n)$.   For $x\in[a,b]$, set

\Centerline{$h(x)=\sup\{x_n:(x,x_1,\ldots,x_n)$ is acceptable$\}$;}

\noindent this is defined in $[x,b]$ because the string $(x)$ is
acceptable.   If $a\le x<b$, then $(x,y)$ is acceptable whenever
$y\in E_x$ and $0\le y\le\min(b,x+\eta_x)$, so $h(x)>x$.
Now choose sequences $\sequence{i}{x_i}$, $\sequence{k}{n_k}$
inductively, as follows.   $n_0=0$ and $x_0=a$.   Given that
$x_i\frown x_{i+1}$ for $i<n_k$ and that $(x_{n_j},\ldots,x_{n_k})$ is
acceptable for every $j\le k$, let $n_{k+1}>n_k$,
$(x_{n_k+1},\ldots,x_{n_{k+1}})$ be such that
$(x_{n_k},\ldots,x_{n_{k+1}})$ is acceptable and
$x_{n_{k+1}}\ge\bover12(x_{n_k}+h(x_{n_k}))$;  then
$(x_{n_j},\ldots,x_{n_{k+1}})$ is acceptable for any $j\le k+1$;
continue.

At the end of the induction, set

\Centerline{$c=\sup_{i\in\Bbb N}x_i=\sup_{k\in\Bbb N}x_{n_k}$.}

\noindent Then there are infinitely many $i$ such that $x_i\frown c$.
\Prf\ For any $k\in\Bbb N$,

$$\eqalignno{\mu([x_{n_k},c]\cap\bigcup_{i\ge n_k}E_{x_i}^+)
&=\lim_{l\to\infty}
\mu\bigl([x_{n_k},x_{n_l}]\cap\bigcup_{n_k\le i<n_l}E_{x_i}^+\bigr)\cr
&\ge\lim_{l\to\infty}\Bover12(x_{n_l}-x_{n_k})\cr
\displaycause{because $(x_{n_k},\ldots,x_{n_l})$ is always an acceptable
string}
&=\Bover12(c-x_{n_k}).\cr}$$

\noindent But this means that if we take $k$ so large that
$x_{n_k}\ge c-\eta_c$, so that
$\mu(E_c\cap[x_{n_k},c])\ge\bover23(c-x_{n_k})$, there must be some
$z\in E_c\cap\bigcup_{i\ge n_k}E_i^+\cap[x_{n_k},c]$;  and if $i\ge n_k$
is such that
$z\in E_i^+$, then $z$ witnesses that $x_i\frown c$.   As $k$ is
arbitrarily large, we have the result.\ \Qed

\Quer\ If $c<b$, take any $y\in E_c$ such that $c<y\le\min(c+\eta_c,b)$.
Take $k\in\Bbb N$ such that
$x_{n_k}\ge c-\bover13(y-c)$, and $j\ge n_k$ such that $x_j\frown c$.
In this case, $(x_{n_k},x_{n_k+1},\ldots,x_j,c,y)$ is an acceptable
string, because

\Centerline{$\mu E_c^+\cap[x_{n_k},y]\ge\mu E_c\cap[c,y]
\ge\bover23(y-c)\ge\bover12(y-x_{n_k})$.}

\noindent But this means that $h(x_{n_k})\ge y$, so that

\Centerline{$x_{n_{k+1}}\le c<\Bover12(x_{n_k}+h(x_{n_k}))$,}

\noindent contrary to the choice of
$x_{n_k+1},\ldots,x_{n_{k+1}}$.\ \Bang

Thus $c=b$.   We therefore have a $j\in\Bbb N$ such that $x_j\frown b$,
and $a=x_0\frown\ldots\frown x_j\frown b$.

\medskip

{\bf (c)} Now suppose that $C\in\Cal C$.   Set $a=\inf C$ and
$b=\sup C$.   If $a=b$, then $\pmb{t}=\{(a,C)\}\in T$ and
$W_{\pmb{t}}=C$.   Otherwise, (b) tells us that we have $x_0,\ldots,x_n$
such that $a=x_0\frown\ldots\frown x_n=b$.   Choose
$a_i\in[x_{i-1},x_i]\cap E_{x_{i-1}}\cap E_{x_i}$ for $1\le i\le n$.
Set $C_i=\coint{a_i,a_{i+1}}$ for $1\le i<n$, $C_0=\coint{a,a_1}$,
$C_n=[a_n,b]$;  set $I=\{i:i\le n,\,C\cap C_i\ne\emptyset\}$;  and
check that $(x_i,C\cap C_i)\in\delta$ for $i\in I$, so that
$\pmb{t}=\{(x_i,C\cap C_i):i\in I\}$ is a $\delta$-fine member of $T$
with $W_{\pmb{t}}=C$.   As $C$ and $\delta$ are arbitrary, 481G(vii) is
satisfied.

\medskip

{\bf (d)} $\Delta$ is full.   \Prf\ Let $\family{x}{\Bbb R}{\delta'_x}$
be a
family in $\Delta$.   For each $x\in X$, there is a measurable set $E_x$
such that $x$ is a density point of $E_x$ and $(x,C)\in\delta'_x$ whenever
$C\in\Cal C$, $x\in\overline{C}$ and both $\inf C$, $\sup C$ belong to
$E_x$.   Set $\pmb{e}=\family{x}{\Bbb R}{E_x}\in\Epsilon$;
then $(x,C)\in\delta'_x$ whenever $(x,C)\in\delta_{\pmb{e}}$.\ \Qed
}%end of proof of 481Q

\exercises{\leader{481X}{Basic exercises (a)}
%\spheader 481Xa
Let $X$, $\Cal C$, $T$ and $\Cal F$ be as in 481C.   Show that if
$f:X\to\Bbb R$, $\mu:\Cal C\to\Bbb R$ and $\nu:\Cal C\to\Bbb R$ are
functions, then $I_{\mu+\nu}(f)=I_{\mu}(f)+I_{\nu}(f)$ whenever the
right-hand side is defined.
%481C

\sqheader 481Xb Let $I$ be any set.   Set
$T=\{\{(i,\{i\}):i\in J\}:J\in[I]^{<\omega}\}$,
$\delta=\{(i,\{i\}):i\in I\}$, $\Delta=\{\delta\}$.   For
$J\in[I]^{<\omega}$ set $\Cal R_J
=\{I\setminus K:J\subseteq K\in[I]^{<\omega}\}
  \cup\{\emptyset\}$;  set
$\frak R=\{\Cal R_J:J\in[I]^{<\omega}\}$.   Show that
$(I,T,\Delta,\frak R)$ is a tagged-partition structure
allowing subdivisions, witnessed by $[I]^{<\omega}$, and that $\Delta$ is
full.   Let
$\nu:[I]^{<\omega}\to\Bbb R$ be any additive functional.   Show that,
for a function $f:I\to\Bbb R$,
$\lim_{\pmb{t}\to\Cal F(T,\Delta,\frak R)}S_{\pmb{t}}(f,\nu)
=\sum_{i\in I}f(i)\nu(\{i\})$ if either exists in $\Bbb R$.
%481G

\sqheader 481Xc
Set $T=\{\{(n,\{n\}):n\in I\}:I\in[\Bbb Z]^{<\omega}\}$,
$\delta=\{(n,\{n\}):n\in\Bbb Z\}$, $\Delta=\{\delta\}$.   For
$I\in[\Bbb Z]^{<\omega}$ and $m$, $n\in\Bbb N$ set
$R_{mn}=\Bbb Z\setminus\{i:-m\le i\le n\}$,
$\Cal R'_n=\{R_{kl}:k,\,l\ge n\}\cup\{\emptyset\}$,
$\Cal R''_n=\{R_{kk}:k\ge n\}\cup\{\emptyset\}$,
$\frak R'=\{\Cal R'_n:n\in\Bbb N\}$,
$\frak R''=\{\Cal R''_n:n\in\Bbb N\}$.   Show that
$(\Bbb Z,T,\Delta,\frak R')$ and
$(\Bbb Z,T,\Delta,\frak R'')$ are tagged-partition structures
allowing subdivisions, witnessed by $[\Bbb Z]^{<\omega}$, and that $\Delta$
is full.   Let $\mu$ be
counting measure on $\Bbb Z$.   Show that, for a function
$f:\Bbb Z\to\Bbb R$,
(i) $\lim_{\pmb{t}\to\Cal F(T,\Delta,\frak R')}S_{\pmb{t}}(f,\nu)
=\lim_{m,n\to\infty}\sum_{i=-m}^nf(i)$ if either is defined in $\Bbb R$
(ii) $\lim_{\pmb{t}\to\Cal F(T,\Delta,\frak R'')}S_{\pmb{t}}(f,\mu)
=\lim_{n\to\infty}\sum_{i=-n}^nf(i)$ if either is defined in $\Bbb R$.
%481G

\spheader 481Xd Set $X=\Bbb N\cup\{\infty\}$, and let $T$ be the
straightforward set of tagged partitions generated by
$\{(n,\{n\}):n\in\Bbb N\}\cup\{(\infty,X\setminus n):n\in\Bbb N\}$
(interpreting a member of $\Bbb N$ as the set of its predecessors).
For $n\in\Bbb N$ set $\delta_n
=\{(i,\{i\}):i\in\Bbb N\}\cup\{(\infty,A):A\subseteq X\setminus n\}$;
set $\Delta=\{\delta_n:n\in\Bbb N\}$.   Show that
$(X,T,\Delta,\{\{\emptyset\}\})$ is a tagged-partition structure
allowing subdivisions, witnessed by
$\Cal C=[\Bbb N]^{<\omega}\cup\{X\setminus I:I\in[\Bbb N]^{<\omega}\}$, and
that $\Delta$ is full.
Let $h:\Bbb N\to\Bbb R$ be any function, and define
$\nu:\Cal C\to\Bbb R$ by setting $\nu I=\sum_{i\in I}h(i)$,
$\nu(X\setminus I)=-\nu I$ for $I\in[\Bbb N]^{<\omega}$.   Let
$f:X\to\Bbb R$ be any function such that $f(\infty)=0$.   Show that
$\lim_{\pmb{t}\to\Cal F(T,\Delta,\{\{\emptyset\}\})}S_{\pmb{t}}(f,\mu)
=\lim_{n\to\infty}\sum_{i=0}^nf(i)h(i)$ if either is defined in
$\Bbb R$.
%481G

\sqheader 481Xe Take $X$, $T$, $\Delta$ and $\frak R$ as in 481I.   Show
that if $\mu$ is Lebesgue measure on $[a,b]$ then the gauge integral
$I_{\mu}=\lim_{\pmb{t}\to\Cal F(T,\Delta,\frak R)}S_{\pmb{t}}(.,\mu)$ is
the ordinary Riemann integral $\Rint_a^b$ as described in
134K.   \Hint{show first that they agree on step-functions.}
%481I

\sqheader 481Xf Let $(X,\Sigma,\mu)$ be a semi-finite measure space, and
$\Sigma^f$ the family of measurable sets of finite measure.   Let
$T$ be the straightforward set of tagged partitions generated by
$\{(x,E):x\in E\in\Sigma^f\}$.   For $E\in\Sigma^f$ and $\epsilon>0$ set
$\Cal R_{E\epsilon}=\{F:F\in\Sigma,\,\mu(E\setminus F)\le\epsilon\}$;
set $\frak R=\{\Cal R_{E\epsilon}:E\in\Sigma^f,\,\epsilon>0\}$.   Let
$\frak E$ be the family of countable partitions of $X$ into measurable
sets, and set
$\delta_{\Cal E}=\bigcup_{E\in\Cal E}\{(x,A):x\in E,\,A\subseteq E\}$
for $\Cal E\in\frak E$,
$\Delta=\{\delta_{\Cal E}:\Cal E\in\frak E\}$.   Show that
$(X,T,\Delta,\frak R)$ is a
tagged-partition structure allowing subdivisions, witnessed by
$\Sigma^f$.      In what circumstances is
$\Delta$ full or countably full?
Show that, for a function $f:X\to\Bbb R$,
$\int fd\mu=\lim_{\pmb{t}\to\Cal F(T,\Delta,\frak R)}S_{\pmb{t}}(f,\mu)$
if either is defined in $\Bbb R$.   \Hint{when showing that if
$I_{\mu}(f)$ is defined then $f$ is $\mu$-virtually measurable, you will
need 413G or something similar;  compare 482E.}
%481N

\spheader 481Xg Let $(X,\Sigma,\mu)$ be a totally finite measure space,
and $T$ the straightforward set of tagged partitions generated by
$\{(x,E):x\in E\in\Sigma\}$.   Let $\frak E$ be the family of finite
partitions of $X$ into measurable sets, and set
$\delta_{\Cal E}=\bigcup_{E\in\Cal E}\{(x,A):x\in E,\,A\subseteq E\}$
for $\Cal E\in\frak E$, $\Delta=\{\delta_{\Cal E}:\Cal E\in\frak E\}$.
Show that $(X,T,\Delta,\{\{\emptyset\}\})$ is a
tagged-partition structure allowing subdivisions, witnessed by $\Sigma$.
In what circumstances is
$\Delta$ full or countably full?
Show that, for a
function $f:X\to\Bbb R$,
$I_{\mu}(f)=\lim_{\pmb{t}\to\Cal F(T,\Delta,\frak R)}S_{\pmb{t}}(f,\mu)$
is defined iff $f\in\eusm L^{\infty}(\mu)$ (definition:  243A), and
that then $I_{\mu}(f)=\int fd\mu$.
%481Xf, 481N

\spheader 481Xh Let $X$ be a zero-dimensional compact Hausdorff space
and $\Cal E$ the algebra of open-and-closed subsets of $X$.   Let $T$ be
the straightforward set of tagged partitions generated by
$\{(x,E):x\in E\in\Cal E\}$.   Let
$\Delta$ be the set of all neighbourhood gauges on $X$.   Show that
$(X,T,\Delta,\{\{\emptyset\}\})$ is a tagged-partition structure
allowing subdivisions, witnessed by $\Cal E$.   Now let
$\nu:\Cal E\to\Bbb R$ be an additive functional, and set $I_{\nu}(f)
=\lim_{\pmb{t}\to\Cal F(T,\Delta,\{\{\emptyset\}\})}
S_{\pmb{t}}(f,\nu)$ when $f:X\to\Bbb R$ is such that the limit is defined.   (i) Show that $I_{\nu}(\chi E)=\nu E$ for every $E\in\Cal E$.
(ii) Show that if $\nu$ is bounded then $I_{\nu}(f)$ is defined for every
$f\in C(X)$, and is equal to $\dashint fd\nu$ as defined in 363L, if we
identify $X$ with the Stone space of $\frak A$ and $C(X)$ with
$L^{\infty}(\frak A)$.
%+

\spheader 481Xi Let $X$ be a set, $\Delta$ a set of gauges on $X$,
$\frak R$ a collection of families of subsets of $X$, and $T$ a set of
tagged partitions on $X$ which is compatible with $\Delta$ and
$\frak R$.   Let $H\subseteq X$ be such that there is a
$\tilde\delta\in\Delta$ such that $H\cap A=\emptyset$ whenever
$x\in X\setminus H$ and $(x,A)\in\tilde\delta$, and set
$\delta_H=\{(x,A\cap H):x\in H,\,(x,A)\in\delta\}$ for
$\delta\in\Delta$, $\Delta_H=\{\delta_H:\delta\in\Delta\}$,
$\frak R_H=\{\{R\cap H:R\in\Cal R\}:\Cal R\in\frak R\}$,
$T_H=\{\{(x,C\cap H):(x,C)\in\pmb{t},\,x\in H\}:\pmb{t}\in T\}$.    Show
that $T_H$ is compatible with $\Delta_H$ and $\frak R_H$.
%+ subspaces

\spheader 418Xj\dvAnew{2012} Let $X$ be a set, $\Sigma$ an algebra of
subsets of $X$, and $\nu:\Sigma\to\coint{0,\infty}$ an additive functional.
Set $Q=\{(x,C):x\in C\in\Sigma\}$ and let $T$ be the straightforward set of
tagged partitions generated by $Q$.   Let $\Bbb E$ be the set of disjoint
families $\Cal E\subseteq\Sigma$ such that 
$\sum_{E\in\Cal E}\nu E=\nu X$, and 
$\Delta=\{\delta_{\Cal E}:\Cal E\in\Bbb E\}$, where

\Centerline{$\delta_{\Cal E}
=\{(x,C):(x,C)\in Q$ and there is an $E\in\Cal E$ such that
$C\subseteq E\}$}

\noindent for $\Cal E\in\Bbb E$.
Set $\frak R=\{\Cal R_{\epsilon}:\epsilon>0\}$ where
$\Cal R_{\epsilon}=\{E:E\in\Sigma$, $\nu E\le\epsilon\}$ for $\epsilon>0$.
Show that $(X,T,\Delta,\frak R)$ is a tagged-partition structure allowing
subdivisions, witnessed by $\Sigma$.

\leader{481Y}{Further exercises (a)}
%\spheader 481Ya
Suppose that $[a,b]$, $\Cal C$, $T$ and $\Delta$ are as in 481J.   Let
$T'\subseteq[[a,b]\times\Cal C]^{<\omega}$ be the set of tagged
partitions $\pmb{t}=\{(x_i,[a_i,a_{i+1}]):i<n\}$ where
$a=a_0\le x_0\le a_1\le x_2\le a_2\le\ldots\le x_{n-1}\le a_n=b$.
Show that $T'$, as well as $T$, is compatible with $\Delta$ in the sense
of 481Ea;  let $\Cal F'$, $\Cal F$ be the corresponding filters on $T'$
and $T$.   Show that if $\nu:\Cal C\to\Bbb R$ is a functional which is
additive in the sense that $\nu(C\cup C')=\nu C+\nu C'$ whenever $C$,
$C'$ are disjoint members of $\Cal C$ with union in $\Cal C$, and if
$\nu\{x\}=0$ for every $x\in[a,b]$ and $f:[a,b]\to\Bbb R$ is any
function, then $I'_{\nu}(f)=I_{\nu}(f)$ if either is defined, where

\Centerline{$I_{\nu}(f)=\lim_{\pmb{t}\to\Cal F}S_{\pmb{t}}(f,\nu)$,
\quad$I'_{\nu}(f)=\lim_{\pmb{t}\to\Cal F'}S_{\pmb{t}}(f,\nu)$}

\noindent are the gauge integrals associated with $(T,\Cal F)$ and
$(T',\Cal F')$.
%481J

\spheader 481Yb Let us say that a family $\frak R$ of residual families
is `the simple residual structure complementary to
$\Cal H\subseteq\Cal PX$' if
$\frak R=\{\Cal R_H:H\in\Cal H\}$, where
$\Cal R_H=\{X\setminus H':H\subseteq H'\in\Cal H\}\cup\{\emptyset\}$ for
each $H\in\Cal H$.   Suppose that, for each member $i$ of a non-empty
finite set $I$, $(X_i,T_i,\Delta_i,\frak R_i)$ is a tagged-partition
structure allowing subdivisions, witnessed by an upwards-directed family
$\Cal C_i\subseteq\Cal PX_i$, where $X_i$ is a topological space,
$\Delta_i$ is the set of all neighbourhood gauges on $X_i$, and
$\frak R_i$ is the simple residual structure complementary to
$\Cal C_i$.
Set $X=\prod_{i\in I}X_i$ and let $\Delta$ be the set of neighbourhood
gauges on $X$;   let $\Cal C$ be
$\{\prod_{i\in I}C_i:C_i\in\Cal C_i$ for each $i\in I\}$, and $\frak R$
the simple residual structure based on $\Cal C$;  and let $T$ be the
straightforward set of tagged partitions generated by
$\{(\familyiI{x_i},\prod_{i\in I}C_i):\{(x_i,C_i)\}\in T_i$ for every
$i\in I\}$.
Show that $(X,T,\Delta,\frak R)$ is a tagged-partition structure
allowing subdivisions, witnessed by $\Cal C$.
%481O 481P

\spheader 481Yc Give an example to show that, in 481Xi,
$(X,T,\Delta,\frak R)$ can be a tagged-partition structure allowing
subdivisions, while $(H,T_H,\Delta_H,\frak R_H)$ is not.
%481Xi mt48bits
}%end of exercises

\endnotes{
\Notesheader{481} In the examples above I have tried to give an idea of
the potential versatility of the ideas here.   Further examples may be
found in {\smc Henstock 91}.   The goal of 481A-481F is the formula
$I_{\nu}(f)
=\lim_{\pmb{t}\to\Cal F(T,\Delta,\frak R)}S_{\pmb{t}}(f,\nu)$ (481C, 481E);
the elaborate notation reflects the variety of the applications.   One
of these is a one-step definition of the ordinary integral (481Xf).   In
\S483 I will show that the Henstock integral (481J-481K) properly
extends the Lebesgue integral on $\Bbb R$.   In 481Xc I show how
adjusting $\frak R$ can change the class of integrable functions;  in
481Xd I show how a similar effect can sometimes be achieved by adding a
point at infinity and adjusting $T$ and $\Delta$.   As will become
apparent in later sections, one of the great strengths of gauge
integrals is their ability to incorporate special limiting processes.
Another is the fact that we don't need to assume that the functionals
$\nu$ are countably additive;  see 481Xd.   In the formulae of this
section, I don't even ask for finite additivity;  but of course the
functional $I_{\nu}$ is likely to have a rather small domain if $\nu$
behaves too erratically.

`Gauges', as I describe them here, have moved rather briskly forward
from the metric gauges $\delta_h$ (481Eb), which have sufficed for most
of the gauge integrals so far described.   But the generalization
affects only the notation, and makes it clear why so much of the theory
of the ordinary Henstock integral applies equally well to the
`approximately continuous Henstock integral' (481Q), for instance.
You will observe that the sets $Q$ of 481Ba are `gauges' in the
wide sense used here.   But (as the examples of this section show
clearly) we generally use them in a different way.

In ordinary measure theory, we have a fairly straightforward theory of
subspaces (\S214) and a rather deeper theory of product spaces
(chap.\ 25).   For gauge integrals, there are significant difficulties
in the theory of subspaces, some of which will appear in the next
section (see 482G-482H).   For closed subspaces, something can be done,
as in 481Xi;  but the procedure suggested there may lose some
essential element of the original tagged-partition structure (481Yc).
For products of gauge integrals, we do have a reasonably satisfying
version of Fubini's theorem (482M);  I offer 481O and 481P as
alternative approaches.   However, the example of the Pfeffer integral
(\S484) shows that other constructions may be more effective tools
for geometric measure theory.

You will note the concentration on `neighbourhood gauges' (481Eb)
in the work above.   This is partly because they are `full' in the sense of
481Ec.   As
will appear repeatedly in the next section, this flexibility in
constructing gauges is just what one needs when proving that functions
are gauge-integrable.

While I have used such phrases as `Henstock integral', `symmetric
Riemann-complete integral' above, I have not in fact discussed
integrals here, except in the exercises;  in most of the examples in
481I-481Q there is no mention of any functional $\nu$ from which a gauge
integral $I_{\nu}$ can be defined.   The essence of the method is that
we can set up a tagged-partition structure quite independently of any
set function, and it turns out that the properties of a gauge integral
depend more on this structure than on the measure involved.
}%end of notes

\discrpage


