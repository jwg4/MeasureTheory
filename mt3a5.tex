\frfilename{mt3a5.tex}
\versiondate{22.5.11}
\copyrightdate{1999}

\def\chaptername{Appendix}
\def\sectionname{Normed spaces}

\def\ref{\discrversionA{\medskip\noindent$<$\medskip}{}}
\def\Bourbaki{{\smc Bourbaki 87}}
%\def\Dugundji{{\smc Dugundji 66}}
\def\DS{{\smc Dunford \& Schwartz 57}}
%\def\Engelking{{\smc Engelking 89}}
%\def\Gaal{{\smc Gaal 64}}
\def\Gale{{\smc Gale 60}}
%\def\James{{\smc James 87}}
%\def\Kelley{{\smc Kelley 55}}
%\def\Kothe{{\smc K\"othe 69}}
\def\Lang{{\smc Lang 93}}
%\def\Pedersen{{\smc Pedersen 89}}
\def\Rudin{{\smc Rudin 91}}
\def\Schaefer{{\smc Schaefer 66}}
%\def\Schubert{{\smc Schubert 68}}
\def\Taylor{{\smc Taylor 64}}
\def\Wilansky{{\smc Wilansky 64}}

\newsection{3A5}

I run as quickly as possible over the results, nearly all of them
standard elements of any introductory course in functional analysis,
which I find myself calling on in this volume.   As in the corresponding
section of Volume 2 (\S2A4), a large proportion of these are valid for
both real and complex normed spaces, but as the present volume is almost
exclusively concerned with real linear spaces I leave this unsaid,
except in 3A5M, and if in doubt you may suppose for the time being that
scalars belong
to the field $\Bbb R$.   A couple of the most basic results will be used
in their complex forms in Volume 4.

%interpolate definition of Hamel basis for 376Ya??

\leader{3A5A}{The Hahn-Banach theorem:  analytic forms}\cmmnt{ The
Hahn-Banach theorem is
one of the central ideas of functional analysis, both finite- and
infinite-dimensional, and appears in a remarkable variety of forms.   I
list those formulations which I wish to quote, starting with those which
are more or less `analytic', according to the classification of
\Bourbaki.   Recall that if $U$ is a normed space I write $U^*$ for the
Banach space of bounded linear functionals on $U$.

\medskip

} {\bf (a)} Let $U$ be a linear space and $p:U\to\coint{0,\infty}$ a
functional such that $p(u+v)\le p(u)+p(v)$ and $p(\alpha u)=\alpha p(u)$
whenever $u$, $v\in U$ and $\alpha\ge 0$.   Then for any $u_0\in U$ there is a linear functional
$f:U\to\Bbb R$ such that $f(u_0)=p(u_0)$ and $f(u)\le p(u)$ for every
$u\in U$.
\prooflet{(\Rudin, 3.2;  \DS, II.3.10.)}
%\Bourbaki query
%\Lang query
%\Taylor not there
%cf. 4A4D
%for 3{}91E

\spheader 3A5Ab
Let $U$ be a normed space and $V$ a linear subspace of $U$.   Then for
any $f\in V^*$ there is a $g\in U^*$, extending $f$, with $\|g\|=\|f\|$.
\prooflet{(363R;  \Bourbaki, II.3.2;  \Rudin, 3.3;  \DS, II.3.11;
\Lang, p.\ 69;   \Wilansky, p.\ 66;  \Taylor, 3.7-B \& 4.3-A.)}
%3-63Q\Pedersen, 2.3.3

\spheader 3A5Ac If $U$ is a normed space and $u\in U$ there is an
$f\in U^*$ such that $\|f\|\le 1$ and $f(u)=\|u\|$.
\prooflet{(\Bourbaki, II.3.2;  \Rudin, 3.3;  \DS, II.3.14;  \Wilansky,
p.\ 67;  \Taylor, 3.7-C \& 4.3-B.)}
%3-91F\Pedersen, 2.3.4;

\spheader 3A5Ad If $U$ is a normed space and $V\subseteq U$ is a linear
subspace which is not dense, then there is a non-zero $f\in U^*$ such
that $f(v)=0$ for every $v\in V$.
\prooflet{(\Rudin, 3.5;  \DS, II.3.12;  \Taylor, 4.3-D.)}
%3-72A

\spheader 3A5Ae If $U$ is a normed space, $U^*$ separates the points of
$U$.
\prooflet{(\Rudin, 3.4;  \Lang, p.\ 70;  \DS, II.3.14.)}
%3-68

\leader{3A5B}{Cones (a)}
%\spheader 3A5Ba
Let $U$ be a linear space.  A {\bf convex cone} (with
apex $0$) is a set $C\subseteq U$ such that $\alpha u+\beta v\in C$
whenever $u$, $v\in C$ and $\alpha$, $\beta\ge 0$.   The intersection of
any family of convex cones is a convex cone, so for every subset $A$ of
$U$ there is a smallest convex cone including $A$.

\spheader 3A5Bb Let $U$ be a normed space.   Then the closure of a
convex cone is a convex cone.
\prooflet{(\Bourbaki, II.2.6;  \DS, V.2.1.)}

\leader{3A5C}{Hahn-Banach theorem:  geometric forms (a)} Let $U$ be a
normed space and $C\subseteq U$ a convex set such that $\|u\|\ge 1$ for
every $u\in C$.   Then there is an $f\in U^*$ such that $\|f\|\le 1$ and
$f(u)\ge 1$ for every $u\in C$.   \prooflet{(\DS, V.1.12.)}

\spheader 3A5Cb Let $U$ be a normed space and $C\subseteq U$ a non-empty
convex set such
that $0\notin\overline{C}$.   Then there is an $f\in U^*$ such that
$\inf_{u\in C}f(u)>0$.
\prooflet{(\Bourbaki, II.4.1; \Rudin, 3.4;  \Lang, p.\ 70;
\DS, V.2.12.)}
%3-56

\spheader 3A5Cc Let $U$ be a normed space, $C$ a closed convex subset of
$U$ containing $0$, and $u$ a point of $U\setminus C$.   Then there is
an $f\in U^*$ such that $f(u)>1$ and $f(v)\le 1$ for every $v\in C$.
\prooflet{(Apply (b) to $C-u$ to find a $g\in U^*$ such that
$g(u)<\inf_{v\in C}g(v)$ and now set $f=-\bover1{\alpha}g$ where
$g(u)<\alpha<\inf_{v\in C}g(v)$).}
%for 3-69H

\leader{3A5D}{Separation from finitely-generated cones} Let $U$ be a
linear space over $\Bbb R$ and $u$,
$v_0,\ldots,v_n$ points of $U$ such that $u$ does not belong to the
convex cone generated by $\{v_0,\ldots,v_n\}$.   Then there is a
linear functional $f:U\to\Bbb R$ such that $f(v_i)\ge 0$ for every $i$
and $f(u)<0$.

\proof{{\bf (a)} If $U$ is finite-dimensional this is covered by \Gale,
p.\ 56.

\medskip

{\bf (b)} For the general case, let $V$ be the linear subspace of $U$
generated by $u$, $v_0,\ldots,v_n$.   Then there is a linear functional
$f_0:V\to\Bbb R$ such that $f_0(u)<0\le f_0(v_i)$ for every $i$.   By
Zorn's Lemma, there is a maximal linear subspace $W\subseteq U$ such
that $W\cap V=\{0\}$.   Now $W+V=U$ (for if $u\notin W+V$, the linear
subspace $W'$ generated by $W\cup\{u\}$ still has trivial intersection
with $V$), so we have an extension of $f_0$ to a linear functional
$f:U\to \Bbb R$ defined
by setting $f(v+w)=f_0(v)$ whenever $v\in V$ and $w\in W$.   Now
$f(u)<0\le\min_{i\le n}f(v_i)$, as required.
}%end of proof of 3A5D
%for 3-51N

\leader{3A5E}{Weak topologies (a)} Let $U$ be any linear space over
$\Bbb R$ and $W$ a subset of the space $U'$ of all linear
functionals from $U$ to $\Bbb R$.   Then I write $\frak T_s(U,W)$ for
the linear space topology
defined by the method of 2A5B from the functionals
$u\mapsto|f(u)|$ as $f$ runs over $W$.
\cmmnt{ (\Bourbaki, II.6.2;  \Rudin, 3.10;  \DS, V.3.2;  \Taylor,
3.81.)}
%3-76O

\cmmnt{\spheader 3A5Eb I note that the weak topology of a normed space
$U$\cmmnt{ (2A5Ia)} is $\frak T_s(U,U^*)$, while the weak* topology
of $U^*$\cmmnt{ (2A5Ig)} is $\frak T_s(U^*,W)$ where $W$ is the
canonical image of $U$ in $U^{**}$.
\cmmnt{(\Rudin, 3.14.)}
}%end of comment

\spheader 3A5Ec Let $U$ and $V$ be linear spaces over $\Bbb R$ and
$T:U\to V$ a linear operator.   If $W\subseteq U'$ and $Z\subseteq V'$
are such that $gT\in W$ for every $g\in Z$, then $T$ is continuous for
$\frak T_s(U,W)$ and $\frak T_s(V,Z)$.
\cmmnt{(\Bourbaki, II.6.4.)}

\spheader 3A5Ed If $U$ and $V$ are normed spaces and $T:U\to V$ is a
bounded linear operator then we have an {\bf adjoint}
%\cmmnt{ (or {\bf conjugate}, or {\bf dual})}
operator $T':V^*\to U^*$ defined by
saying that $T'g=gT$ for every $g\in V^*$.   $T'$ is linear and is
continuous for the weak* topologies of $U^*$ and $V^*$.
\prooflet{(\Bourbaki, II.6.4;  \DS, \S VI.2;  \Taylor, 4.5.)}
%for 3-67L, 3-93;  \Pedersen, 2.4.12

\spheader 3A5Ee If $U$ is a normed space and $A\subseteq U$ is convex,
then the closure of $A$ for the norm topology is the same as the closure
of $A$ for the weak topology of $U$.   In particular, norm-closed convex
subsets (for instance, norm-closed linear
subspaces) of $U$ are closed for the weak topology.
\prooflet{(\Rudin, 3.12;  \Lang, p.\ 88;  \DS, V.3.13.)}
%for 3-67L, 3-72R

\leader{3A5F}{Weak* topologies:  Theorem} If $U$ is a normed space,
the unit ball of $U^*$ is compact and Hausdorff for the weak*
topology.
\prooflet{(\Rudin, 3.15;  \Lang, p.\ 71;  \DS, V.4.2;
\Taylor, 4.61-A.)}
%for 3-63P\Pedersen, 2.5.2;

\leader{3A5G}{Reflexive spaces (a)} A normed space $U$ is
{\bf reflexive} if every member of $U^{**}$ is of the form
$f\mapsto f(u)$ for some $u\in U$.

\spheader 3A5Gb A normed space is reflexive iff bounded sets are
relatively weakly compact. \prooflet{(\DS, V.4.8;  \Taylor, 4.61-C.)}
%for 3-72

\spheader 3A5Gc If $U$ is a reflexive space, $\sequencen{u_n}$ is a
bounded sequence in $U$ and $\Cal F$ is an ultrafilter on
$\Bbb N$, then $\lim_{n\to\Cal F}u_n$ is defined in $U$ for the weak
topology.
\prooflet{(Use (b) and 2A3Se.)}
%for 3-72A

\leader{3A5H}{(a) Uniform Boundedness Theorem} Let $U$ be a Banach
space, $V$ a normed space, and $A\subseteq\eurm B(U;V)$ a set such that
$\{Tu:T\in A\}$ is bounded in $V$ for every $u\in U$.   Then $A$ is
bounded in $\eurm B(U;V)$.
\prooflet{(\Rudin, 2.6;  \DS, II.3.21;  \Taylor, 4.4-E.)}
%3{}76N 4{}61

\spheader 3A5Hb {\bf Corollary} If $U$ is a
normed space and $A\subseteq U$ is such that $f[A]$ is bounded for every
$f\in U^*$, then $A$ is bounded.
\prooflet{(\Wilansky, p.\ 117;  \Taylor, 4.4-AS.)}
%3{}76
Consequently any relatively weakly compact set in $U$ is bounded.
\prooflet{(\Rudin, 3.18.)}
%3{}56

\leader{*3A5I}{Strong operator topologies}\dvAformerly{3{}A5M}
If $U$ and $V$ are normed spaces,
the {\bf strong operator topology} on $\eurm B(U;V)$ is that defined by
the seminorms $T\mapsto\|Tu\|$ as $u$ runs over $U$.
If $U$ is a Banach space, $V$ is a normed space and
$A\subseteq\eurm B(U;V)$,
then $A$ is relatively compact for the strong operator topology
iff $\{Tu:T\in A\}$ is relatively compact in $V$ for every $u\in U$.
\prooflet{(Put 3A5Ha and 2A3R together.)}

\leader{3A5J}{Completions}\dvAformerly{3{}A5I} Let $U$ be a normed space.

\spheader 3A5Ja\cmmnt{ Recall that} $U$ has a metric $\rho$ associated 
with the norm\cmmnt{ (2A4Bb)}, and\cmmnt{ that} the topology defined by 
$\rho$ is a
linear space topology\cmmnt{ (2A5D, 2A5B)}.   This topology defines a
uniformity $\Cal W$\cmmnt{ (3A4Ad)} which is also the uniformity
defined by $\rho$\cmmnt{ (3A4Bd)}.   The norm itself is a uniformly
continuous function from $U$ to $\Bbb R$\prooflet{ (because
$|\|u\|-\|v\||\le\|u-v\|$ for all $u$, $v\in U$)}.

\spheader 3A5Jb Let $(\hat U,\hat\Cal W)$ be the uniform space
completion of $(U,\Cal W)$\cmmnt{ (3A4H)}.   Then addition and scalar
multiplication and the norm extend uniquely to make $\hat U$ a Banach
space.  \prooflet{(\Schaefer, I.1.5;  \Lang, p.\ 78.)}

\spheader 3A5Jc If $U$ and $V$ are Banach spaces with dense linear
subspaces $U_0$ and $V_0$, then any norm-preserving isomorphism between
$U_0$ and $V_0$ extends uniquely to a norm-preserving isomorphism
between $U$ and $V$\prooflet{ (use 3A4G)}.
%for 3-63I

\leader{3A5K}{Normed algebras}\dvAformerly{3{}A5J} If $U$ is a normed
algebra\cmmnt{ (2A4J)}, its multiplication, regarded as a function
from $U\times U$ to $U$, is continuous.
\prooflet{(\Wilansky, p.\ 259.)}
%for 3-63I

\leader{3A5L}{Compact operators}\dvAformerly{3{}A5K} 
Let $U$ and $V$ be Banach spaces.

\spheader 3A5La A linear operator $T:U\to V$ is
{\bf compact}
%\cmmnt{ or {\bf completely continuous}}
if $\{Tu:\|u\|\le 1\}$ is relatively compact in $V$ for the topology
defined by the norm of $V$.

\spheader 3A5Lb A linear operator $T:U\to V$ is {\bf weakly compact} if
$\{Tu:\|u\|\le 1\}$ is relatively weakly compact in $V$.
Of course compact operators are weakly compact;
\cmmnt{because weakly compact sets must be norm-bounded (3A5Hb),}
weakly compact operators are bounded.

\leader{3A5M}{Hilbert spaces}\dvAformerly{3{}A5L}\cmmnt{ I mentioned the 
phrases `inner
product space', `Hilbert space' briefly in 244N and 244P, without
explanation, as I did not there rely on any of the abstract theory of
these spaces.   For the main result of \S396 we need one of their
fundamental properties, so I now skim over the definitions.

\medskip

} {\bf (a)} An {\bf inner product space} is a linear space $U$ over
$\RoverC$ together with an operator
$\innerprod{\,\,}{\,\,}:U\times U\to\RoverC$ such that

\Centerline{$\innerprod{u_1+u_2}{v}
=\innerprod{u_1}{v}+\innerprod{u_2}{v}$,
\quad$\innerprod{\alpha u}{v}=\alpha \innerprod{u}{v}$,
\quad$\innerprod{u}{v}=\overline{\innerprod{v}{u}}$}

\noindent (the complex conjugate of $\innerprod{v}{u}$),

\Centerline{$\innerprod{u}{u}\ge 0$,
\quad$u=0$ whenever $\innerprod{u}{u}=0$}

\noindent for all $u$, $u_1$, $u_2$, $v\in U$ and $\alpha \in\RoverC$.

\spheader 3A5Mb If $U$ is any inner product space, we have a norm on $U$
defined by setting $\|u\|=\sqrt{\innerprod{u}{u}}$ for every $u\in U$.
\prooflet{(\Taylor, 3.2-B.)}

\spheader 3A5Mc A {\bf Hilbert space} is an inner product space which
is\cmmnt{ a Banach space under the norm of (b) above, that is, is}
complete in the metric defined from its norm.

\spheader 3A5Md If $U$ is a Hilbert space, $C\subseteq U$ is a non-empty
closed convex set, and $u\in U$, then there is a unique $v\in C$ such
that $\|u-v\|=\inf_{w\in C}\|u-w\|$.  \prooflet{(\Taylor, 4.81-A;  compare
244Yn.)}

\leader{*3A5N}{Bounded sets in linear topological
spaces}\dvAnew{2009}\cmmnt{ There is a
point in \S377 %377E
where a concept from the general theory of linear topological spaces
helps an idea to flow more freely.}
Let $U$ be a linear topological space over $\RoverC$.

\spheader 3A5Na\dvAformerly{4{}A4Bf} A set
$A\subseteq U$ is {\bf bounded} if for every neighbourhood $G$ of $0$ 
there is an $n\in\Bbb N$ such that $A\subseteq nG$.

\spheader 3A5Nb If $A\subseteq U$ is bounded, then

\quad(i) every subset of $A$ is bounded;

\quad(ii) the closure of $A$ is bounded;

\quad(iii) $\alpha A$ is bounded for every $\alpha\in\RoverC$;

\quad(iv) $A\cup B$ and $A+B$ are bounded for every bounded $B\subseteq U$;

\quad(v)\dvAformerly{4{}A4Bg} if $V$ is another linear topological space, 
and $T:U\to V$ is a continuous linear operator, then $T[A]$ is bounded.

\spheader 3A5Nc If $A\subseteq U$ is relatively compact, it is bounded.

\spheader 3A5Nd If $U$ is a normed space, and $A\subseteq U$, then the
following are equiveridical:

\quad(i) $A$ is bounded in the sense of (a) above for the norm topology of
$U$;

\quad(ii) $A$ is bounded in the sense of 2A4Bc\cmmnt{, that is,
$\{\|u\|:u\in A\}$ is bounded above in $\Bbb R$};

\quad(iii) $A$ is bounded for the weak topology of $U$.

\proof{{\smc Schaefer 66}, \S I.5;
{\smc K\"othe 69}, \S15.6.   For (d-iii), use 3A5Hb.}

\discrpage

