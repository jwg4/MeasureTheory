\frfilename{mt391.tex}
\versiondate{5.9.07}
\copyrightdate{2000}

\def\chaptername{Measurable algebras}
\def\sectionname{Kelley's theorem}

\newsection{391}

In this section I introduce the notion of `measurable algebra' (391B),
which will be the subject of the whole chapter once the trivial
construction of 391A has been dealt with.   I show that for \wsid\
algebras countable additivity can be left to look after itself, and all
we need to find is a strictly positive finitely additive functional
(391D).   I give Kelley's criterion for the existence of such a
functional (391H-391J).

\leader{391A}{Proposition} Let $\frak A$ be any Dedekind
$\sigma$-complete Boolean algebra.   Then there is a function
$\bar\mu:\frak A\to[0,\infty]$ such that $(\frak A,\bar\mu)$ is a
measure algebra.

\proof{ Set $\bar\mu 0=0$, $\bar\mu a=\infty$ for
$a\in\frak A\setminus\{0\}$.
}%end of proof of 391A

\leader{391B}{Definition (a)} I will call a Boolean algebra $\frak A$ {\bf
measurable} if there is a functional
$\bar\mu:\frak A\to\coint{0,\infty}$ such that $(\frak A,\bar\mu)$ is a
totally finite measure algebra.

\cmmnt{In this case, if $\bar\mu\ne 0$, then it has a scalar multiple
with total mass $1$.   So a Boolean algebra $\frak A$ is measurable iff
either it is $\{0\}$ or there is a functional $\bar\mu$ such that
$(\frak A,\bar\mu)$ is a probability algebra.
}%end of comment

\spheader 391Bb\dvAnew{2007} 
I will call a Boolean algebra $\frak A$ {\bf chargeable} if
there is an additive functional $\nu:\frak A\to\coint{0,\infty}$ which is
{\bf strictly positive}, that is, $\nu a>0$ for every non-zero
$a\in\frak A$.

\cmmnt{Of course a measurable algebra is chargeable.}

\spheader 391Bc\dvAnew{2007} 
I will call a Boolean algebra {\bf nowhere measurable} if
none of its non-zero principal ideals are measurable algebras.

\vleader{48pt}{391C}{Proposition} Let $\frak A$ be a Boolean algebra.

(a) The following are equiveridical:  (i) there is a functional
$\bar\mu:\frak A\to[0,\infty]$ such that $(\frak A,\bar\mu)$ is a
semi-finite measure algebra;  (ii) $\frak A$ is Dedekind
$\sigma$-complete and $\{a:a\in\frak A,\,\frak A_a$ is measurable$\}$ is
order-dense in $\frak A$\cmmnt{, writing $\frak A_a$ for the principal
ideal generated by $a$}.

(b) The following are equiveridical:  (i) there is a functional
$\bar\mu:\frak A\to[0,\infty]$ such that $(\frak A,\bar\mu)$ is a
localizable measure algebra;  (ii) $\frak A$ is Dedekind
complete and $\{a:a\in\frak A,\,\frak A_a$ is measurable$\}$ is
order-dense in $\frak A$.

\proof{{\bf (a)} (i)$\Rightarrow$(ii):  if $(\frak A,\bar\mu)$ is a
semi-finite measure algebra, then $\frak A^f=\{a:\bar\mu a<\infty\}$ is
order-dense in $\frak A$ and $\frak A_a$ is measurable for every
$a\in\frak A^f$.

(ii)$\Rightarrow$(i):  setting $D=\{a:a\in\frak A,\,\frak A_a$ is
measurable$\}$, $D$ is order-dense, so there is a partition of unity
$C\subseteq D$ (313K).   For each $c\in C$, choose $\bar\mu_c$ such
that $(\frak A_c,\bar\mu_c)$ is a totally finite measure algebra.   Set
$\bar\mu a=\sum_{c\in C}\bar\mu_c(a\Bcap c)$ for every $a\in\frak A$;
then it is easy to check that $(\frak A,\bar\mu)$ is a semi-finite
measure algebra.
\medskip

{\bf (b)} Follows immediately.
}%end of proof of 391C

\leader{391D}{Theorem}\cmmnt{ ({\smc Kantorovich Vulikh \& Pinsker 50})}
Let $\frak A$ be a Boolean algebra.   Then the following are equiveridical:

(i) $\frak A$ is measurable;

(ii) $\frak A$ is Dedekind $\sigma$-complete, \wsid\ and chargeable.

\proof{{\bf (i)$\Rightarrow$(ii)} Put the definition together with
322C(b)-(c) (for Dedekind completeness) and 322F (for weak
$(\sigma,\infty)$-distributivity).

\medskip

{\bf (ii)$\Rightarrow$(i)}
Given that (ii) is satisfied, let $M$ be the $L$-space of bounded additive
functionals on $\frak A$, $M_{\tau}\subseteq M$ the band of completely
additive functionals, and $P_{\tau}:M\to M_{\tau}$ the band projection
(362Bd).   Let $\nu:\frak A\to\coint{0,\infty}$ be a strictly positive
additive functional, and set $\bar\mu=P_{\tau}(\nu)$.
Then $\bar\mu$ is strictly
positive.   \Prf\ If $c\in\frak A$ is non-zero, there is an
upwards-directed set $A$, with supremum $c$, such that
$\bar\mu c=\sup_{a\in A}\nu c$ (362D);  as $\nu$ is strictly positive and
$A$ contains a non-zero element, $\bar\mu c>0$.\ \QeD\  Of course $\bar\mu$
is countably additive, so witnesses that $\frak A$ is measurable.
}%end of proof of 391D

\leader{391E}{}\cmmnt{ Thus we are led naturally to the question:
which Boolean algebras carry strictly positive {\it finitely} additive
functionals?   The Hahn-Banach theorem, suitably applied, gives some
sort of answer to this question.   For the sake of applications later
on, I give two general results on the existence of additive functionals
related to given functionals.

\medskip

\noindent}{\bf Theorem} Let $\frak A$ be a Boolean algebra, not $\{0\}$,
and $\phi:\frak A\to[0,1]$ a functional.   Then the following are
equiveridical:

(i) there is a finitely additive functional $\nu:\frak A\to[0,1]$ such
that $\nu 1=1$ and $\nu a\le\phi a$ for every $a\in\frak A$;

(ii) whenever $\langle a_i\rangle_{i\in I}$ is a finite indexed family
in $\frak A$, $m\in\Bbb N$ and $\sum_{i\in I}\chi a_i\ge m\chi 1$ in
$S=S(\frak A)$\cmmnt{ (definition:  361A)}, then
$\sum_{i\in I}\phi a_i\ge m$.

\proof{{\bf (a)(i)$\Rightarrow$(ii)} If $\nu:\frak A\to[0,1]$ is a
finitely additive functional such that $\nu 1=1$ and $\nu a\le\phi a$
for every $a\in\frak A$, let $h:S\to\Bbb R$ be the positive linear
functional corresponding to $\nu$ (361G).   Now if $\familyiI{a_i}$ is a
finite family in $\frak A$ and $\sum_{i\in I}\chi a_i\ge m\chi 1$, then

$$\eqalign{\sum_{i\in I}\phi a_i
&\ge\sum_{i\in I}\nu a_i
=\sum_{i\in I}h(\chi a_i)\cr
&=h(\sum_{i\in I}\chi a_i)
\ge h(m\chi 1)
=m.\cr}$$

\noindent As $\familyiI{a_i}$ is arbitrary, (ii) is true.

\medskip

{\bf (b)(ii)$\Rightarrow$(i)} Now suppose that $\phi$ satisfies (ii).
For $u\in S$, set

\Centerline{$p(u)
=\inf\{\sum_{i=0}^n\alpha_i\phi a_i:
a_0,\ldots,a_n\in\frak A,\,\alpha_0,\ldots,\alpha_n\ge 0,\,
\sum_{i=0}^n\alpha_i\chi a_i\ge u\}$.}

\noindent Then it is easy to check that $p(u+v)\le p(u)+p(v)$ for all
$u$, $v\in S$, and that $p(\alpha u)=\alpha p(u)$ for all $u\in S$,
$\alpha\ge 0$.   Also $p(\chi 1)\ge 1$.   \Prf\Quer\ If not, there are
$a_0,\ldots,a_n\in\frak A$ and $\alpha_0,\ldots,\alpha_n\ge 0$ such that
$\chi 1\le\sum_{i=0}^n\alpha_i\chi a_i$ but
$\sum_{i=0}^n\alpha_i\phi a_i<1$.   Increasing each $\alpha_i$ slightly
if necessary, we may suppose that every $\alpha_i$ is rational;  let
$m\ge 1$ and $k_0,\ldots,k_n\in\Bbb N$ be such that $\alpha_i=k_i/m$ for
each $i\le n$.

Set $K=\{(i,j):0\le i\le n,\,1\le j\le k_i\}$, and for $(i,j)\in K$ set
$a_{ij}=a_i$.   Then

\Centerline{$\sum_{(i,j)\in K}\chi a_{ij}
=\sum_{i=0}^nk_i\chi a_i
=m\sum_{i=0}^n\alpha_i\chi a_i
\ge m\chi 1$,}

\noindent but

\Centerline{$\sum_{(i,j)\in K}\phi a_{ij}
=\sum_{i=0}^nk_i\phi a_i
=m\sum_{i=0}^n\alpha_i\phi a_i
<m$,}

\noindent which is supposed to be impossible.\ \Bang\Qed

By the Hahn-Banach theorem, in the form 3A5Aa, there is a linear
functional $h:S\to\Bbb R$ such that $h(\chi 1)=p(\chi 1)\ge 1$ and
$h(u)\le p(u)$ for
every $u\in S$.   In particular, $h(\chi a)\le\phi b$ whenever
$a\Bsubseteq b\in\frak A$.   Set $\nu a=h(\chi a)$ for $a\in\frak A$;
then $\nu:\frak A\to\coint{0,\infty}$ is an additive functional,
$\nu 1\ge 1$ and
$\nu a\le\phi b$ whenever $a\Bsubseteq b$ in $\frak A$.   We do not know
whether $\nu$ is positive, but if we define $\nu^+$ as in 362Ab, we
shall have a non-negative additive functional such that

\Centerline{$\nu^+a=\sup_{b\Bsubseteq a}\nu b\le\phi a$}

\noindent for every $a\in\frak A$, and

\Centerline{$1\le\nu 1\le\nu^+1\le\phi 1\le 1$,}

\noindent so $\nu^+$ witnesses that (i) is true.
}%end of proof of 391E

\leader{391F}{Theorem} Let $\frak A$ be a Boolean algebra, not $\{0\}$,
and $\psi:A\to[0,1]$ a functional, where $A\subseteq\frak A$.   Then the
following are equiveridical:

(i) there is a finitely additive functional
$\nu:\frak A\to[0,1]$ such that $\nu 1=1$ and $\nu a\ge\psi a$ for every
$a\in A$;

(ii) whenever $\langle a_i\rangle_{i\in I}$ is a finite indexed family
in $A$, there is a set $J\subseteq I$ such that
$\#(J)\ge\sum_{i\in I}\psi a_i$ and $\inf_{i\in J}a_i\ne 0$.

\cmmnt{\medskip

\noindent{\bf Remark} In (ii) here, we may have to interpret the infimum
of the empty set in $\frak A$ as $1$.
}%end of comment

\proof{{\bf (a)} We apply 391E to $\phi$, where

$$\eqalign{\phi a&=1-\psi(1\Bsetminus a)\text{ if }a\in\frak A
  \text{ and }1\Bsetminus a\in A,\cr
&=1\text{ for other }a\in\frak A.\cr}$$

\medskip

{\bf (b)} Suppose that (i) here is true of $\psi$.   Then 391E(i) is true
of $\phi$.   \Prf\ Let $\nu:\frak A\to[0,1]$ be an additive
functional such that $\nu 1=1$ and $\nu a\ge\psi a$ for every 
$a\in\frak A$.   If $a\in\frak A$ and $1\Bsetminus a\in A$, then

\Centerline{$\nu a=1-\nu(1\Bsetminus a)\le 1-\psi(1\Bsetminus a)=\phi a$;}

\noindent for other $a\in\frak A$, $\nu a\le 1=\phi a$.\ \Qed

\medskip

{\bf (c)} Suppose that 391E(i) is true of $\phi$.   Then (i) here is true
of $\psi$.   \Prf\ There is an additive functional $\nu:\frak A\to[0,1]$
such that $\nu 1=1$ and $\nu a\le\phi a$ for every $a\in\frak A$;  in this
case, for $a\in A$,

\Centerline{$\nu a=1-\nu(1\Bsetminus a)\ge 1-\phi(1\Bsetminus a)=\psi a$.
\Qed}

\medskip

{\bf (d)} Suppose that (ii) here is true of $\psi$, and that
$\langle a_i\rangle_{i\in I}$ is a finite family
in $\frak A$ such that $\sum_{i\in I}\chi a_i\ge m\chi 1$, while
$\sum_{i\in I}\phi a_i=\beta$.   Set 
$K=\{i:i\in I$, $1\Bsetminus a_i\in A\}$.

\Centerline{$\sum_{i\in K}\psi(1\Bsetminus a_i)
=\sum_{i\in K}(1-\phi a_i)
=\#(K)-\sum_{i\in I}\phi a_i+\#(I\setminus K)
=\#(I)-\beta$,}

\noindent so there is a set $J\subseteq K$ such that
$\#(J)\ge\#(I)-\beta$ and $\inf_{i\in J}(1\Bsetminus a_i)=c\ne 0$.   Now
$c\Bcap a_i=0$ for $i\in J$, so

\Centerline{$m\chi c\le\sum_{i\in I}\chi(a_i\Bcap c)
=\sum_{i\in I\setminus J}\chi(a_i\Bcap c)\le\#(I\setminus J)\chi c$}

\noindent and $m\le\#(I)-\#(J)\le\beta$.   As $\familyiI{a_i}$ is
arbitrary, 391E(ii) is true of $\phi$.

\medskip

{\bf (e)} Suppose that 391E(ii) is true of $\phi$, and that
$\familyiI{a_i}$ is a family in $A$.   Set

\Centerline{$\beta=\sum_{i\in I}\phi(1\Bsetminus a_i)
=\#(I)-\sum_{i\in I}\psi a_i$}

\noindent and let $k$ be the least integer greater than $\beta$.   Since
$\sum_{i\in I}\phi(1\Bsetminus a_i)<k$,
$\sum_{i\in I}\chi(1\Bsetminus a_i)\not\ge k\chi 1$, that is,
$\sum_{i\in I}\chi a_i\not\le(\#(I)-k)\chi 1$.   But this means that
there must be some $J\subseteq I$ such
that $\#(J)>\#(I)-k$ and $\inf_{i\in J}a_i\ne 0$.   Now

\Centerline{$\sum_{i\in I}\psi a_i=\#(I)-\beta\le\#(I)-(k-1)\le\#(J)$.}

\noindent As $\familyiI{a_i}$ is arbitrary, (ii) here is true of $\psi$.

\medskip

{\bf (f)} Since we know that 391E(i)$\Leftrightarrow$391E(ii), 
we can conclude that (i) and (ii) here are equiveridical.
}%end of proof of 391F

\leader{391G}{Corollary} Let $\frak A$ be a Boolean algebra, $\frak B$ a
subalgebra of $\frak A$, and $\nu_0:\frak B\to\Bbb R$ a non-negative
finitely additive functional.   Then there is a non-negative finitely
additive functional $\nu:\frak A\to\Bbb R$ extending $\nu_0$.

\proof{{\bf (a)} Suppose first that $\nu_01=1$.   Set $\psi b=\nu_0b$
for every $b\in\frak B$.   Then $\psi$ must satisfy the condition (ii)
of 391F when regarded as a functional defined on a subset of $\frak B$;
but this means that
it satisfies the same condition when regarded as a functional
defined on a subset of
$\frak A$.   So there is a non-negative finitely additive functional
$\nu:\frak A\to\Bbb R$ such that $\nu 1=1$ and $\nu b\ge\nu_0b$ for
every $b\in\frak B$.   In this case

\Centerline{$\nu b=1-\nu(1\Bsetminus b)\le 1-\nu_0(1\Bsetminus
b)=\nu_0b\le\nu b$}

\noindent for every $b\in\frak B$, so $\nu$ extends $\nu_0$.

\medskip

{\bf (b)} For the general case, if $\nu_01=0$ then $\nu_0$ must be the
zero functional on $\frak B$, so we can take $\nu$ to be the zero functional
on $\frak A$;  and if $\nu_01=\gamma>0$, we apply (a) to
$\gamma^{-1}\nu_0$.
}%end of proof of 391G

\leader{391H}{Definition} Let $\frak A$ be a Boolean algebra, and
$A\subseteq\frak A\setminus\{0\}$ any non-empty set.   The {\bf
intersection number} of $A$ is the largest $\delta\ge 0$ such that
whenever $\langle a_i\rangle_{i\in I}$ is a finite family in $A$, with
$I\ne\emptyset$, there is a $J\subseteq I$ such that
$\#(J)\ge\delta\#(I)$ and $\inf_{i\in J}a_i\ne 0$.

\cmmnt{\medskip

\noindent{\bf Remarks (a)} It is essential to note that in the
definition here the $\langle a_i\rangle_{i\in I}$ are indexed families,
with repetitions allowed;  see 391Xi.

\medskip

{\bf (b)} I spoke perhaps rather glibly of `the largest $\delta$ such
that $\ldots$';  you may prefer to write

\Centerline{$\delta=\inf\{\sup_{\emptyset\ne
J\subseteq\{0,\ldots,n\},\inf_{j\in J}a_j\ne
0}\Bover{\#(J)}{n+1}:a_0,\ldots,a_n\in A\}$.}
}%end of comment

\leader{391I}{Proposition} Let $\frak A$ be a Boolean algebra and
$A\subseteq\frak A\setminus\{0\}$ any non-empty set.   Write $C$ for the
set of non-negative finitely additive functionals $\nu:\frak A\to[0,1]$
such that $\nu 1=1$.    Then the intersection number of $A$ is precisely
$\max_{\nu\in C}\inf_{a\in A}\nu a$.

\proof{ Write $\delta$ for the intersection number of $A$, and $\delta'$
for $\sup_{\nu\in C}\inf_{a\in A}\nu a$.

\medskip

{\bf (a)} For any $\gamma<\delta'$, we can
find a $\nu\in C$ such that $\nu a\ge\gamma$ for every $a\in A$.   So if
we set $\psi a=\gamma$ for every $a\in A$, $\psi$ satisfies condition
(i) of 391F.   But this means that if $\langle a_i\rangle_{i\in I}$ is
any finite family in $A$, there must be a $J\subseteq I$ such that
$\inf_{i\in J}a_i\ne
0$ and $\#(J)\ge\gamma\#(I)$.   Accordingly $\gamma\le\delta$;  as
$\gamma$ is arbitrary, $\delta'\le\delta$.

\medskip

{\bf (b)} Define $\psi:A\to[0,1]$ by setting $\psi a=\delta$ for every
$a\in A$.   If $\langle a_i\rangle_{i\in I}$ is a finite indexed family
in $A$, there is a $J\subseteq I$ such that $\#(J)\ge\delta\#(I)$ and
$\inf_{i\in
J}a_i\ne 0$;  but $\delta\#(I)=\sum_{i\in I}\psi a_i$, so this means
that condition (ii) of 391F is satisfied.   So there is a $\nu\in C$
such that
$\nu a\ge\delta$ for every $a\in A$;  and $\nu$ witnesses not only that
$\delta'\ge\delta$, but that the supremum is a maximum.
}%end of proof of 391I

\leader{391J}{Theorem} Let $\frak A$ be a Boolean algebra.   Then the
following are equiveridical:

(i) $\frak A$ is chargeable;

(ii) either $\frak A=\{0\}$ or $\frak A\setminus\{0\}$ is expressible as
a countable union of sets with non-zero intersection numbers.

\proof{{\bf (i)$\Rightarrow$(ii)} If there is a strictly positive
finitely additive functional $\nu$ on $\frak A$, and $\frak A\ne\{0\}$,
set $A_n=\{a:\nu a\ge 2^{-n}\nu 1\}$ for every $n\in\Bbb N$;  then
(applying 391I to the functional $\bover1{\nu 1}\nu$) we see that every
$A_n$ has intersection number at least $2^{-n}$, while $\frak
A\setminus\{0\}=\bigcup_{n\in\Bbb N}A_n$ because $\nu$ is strictly
positive, so (ii) is satisfied.

\medskip

{\bf (ii)$\Rightarrow$(i)} If $\frak A\setminus\{0\}$ is expressible as
$\bigcup_{n\in\Bbb N}A_n$, where each $A_n$ has intersection number
$\delta_n>0$, then for each $n$ choose a finitely additive functional
$\nu_n$ on $\frak A$ such that $\nu_n1=1$ and $\nu_na\ge\delta_n$
for every $a\in A_n$.   Setting $\nu a=\sum_{n=0}^{\infty}2^{-n}\nu_na$
for
every $a\in\frak A$, $\nu$ is a strictly positive additive functional on
$\frak A$, and (i) is true.
}%end of proof of 391J

\leader{391K}{Corollary} Let $\frak A$ be a Boolean algebra.   Then
$\frak A$ is measurable iff it is Dedekind $\sigma$-complete and
\wsid\ and either $\frak A=\{0\}$ or
$\frak A\setminus\{0\}$ is expressible as a countable union of sets with
non-zero intersection numbers.

\proof{ Put 391D and 391J together.
}%end of proof of 391K

\exercises{
\leader{391X}{Basic exercises (a)}
%\spheader 391Xa
Show
that a chargeable Boolean algebra is ccc, so is Dedekind complete iff it
is Dedekind $\sigma$-complete.
%391D

\spheader 391Xb Show (i) that any subalgebra of a chargeable Boolean
algebra is chargeable (ii) that a countable simple product of chargeable
Boolean algebras is chargeable (iii) that any free product of chargeable
Boolean algebras is chargeable.
%391Xa

\spheader 391Xc(i) Let $\frak A$ be a Boolean algebra with a chargeable
order-dense subalgebra.   Show that $\frak A$ is chargeable.   (ii) Show
that the
Dedekind completion of a chargeable Boolean algebra is chargeable.
%391Xa

\spheader 391Xd(i) Show that the algebra of open-and-closed subsets of
$\{0,1\}^{I}$ is chargeable for any set $I$.   (ii) Show that the
regular open algebra of $\Bbb R$ is chargeable.
%391Xa

\spheader 391Xe(i) Show that any principal ideal of a chargeable
Boolean algebra is chargeable.   (ii) Let $\frak A$ be a chargeable
Boolean algebra and $\Cal I$ an order-closed ideal of $\frak A$.   Show
that $\frak A/\Cal I$ is chargeable.
%391Xa

\sqheader 391Xf Show that a Boolean algebra is chargeable iff it is
isomorphic to a subalgebra of a measurable algebra.   \Hint{324O, 392H.}
%391Xa

\spheader 391Xg Let $\frak A$ be a Boolean algebra.   Show that the
following are equiveridical:  (i) $\frak A$ is chargeable and \wsid;  (ii)
there is a strictly positive countably additive functional on $\frak A$;
(iii) there is a strictly positively completely additive functional on
$\frak A$.
%391D

\spheader 391Xh Explain how to use the Hahn-Banach theorem to prove
391G directly, without passing through 391F.   \Hint{$S(\frak B)$ can be
regarded as a subspace of $S(\frak A)$.}
%391G

\sqheader 391Xi Take $X=\{0,1,2,3\}$, $\frak A=\Cal PX$,
$A=\{\{0,1\},\{0,2\},\{0,3\},\{1,2,3\}\}$.
Show that the intersection number of $A$ is $\bover35$.   \Hint{use
391I.}   Show that if $a_0,\ldots,a_n$ are {\it distinct} members of $A$
then there is a set $J\subseteq\{0,\ldots,n\}$, with
$\#(J)\ge\bover23(n+1)$, such that $\inf_{j\in J}a_j\ne 0$.
%391H

\spheader 391Xj
Let $\frak A$ be a Boolean algebra.   For non-empty
$A\subseteq\frak A\setminus\{0\}$ write $\delta(A)$ for the intersection
number of $A$.   Show that for any non-empty
$A\subseteq\frak A\setminus\{0\}$,
$\delta(A)=\inf\{\delta(I):I$ is a non-empty finite subset of $A\}$.
%391H

\spheader 391Xk Let $\frak A$ be a Boolean algebra, not $\{0\}$.   For
$a_0,\ldots,a_n\in\frak A$ set
$t(a_0,\ldots,a_n)=\max\{m:m\in\Bbb N$, $m\chi 1\le\sum_{i=0}^n\chi a_i\}$.
Let $A\subseteq\frak A$ be non-empty.   Show that

$$\eqalign{\sup\{\Bover1{n+1}&t(a_0,\ldots,a_n):a_0,\ldots,a_n\in A\}\cr
&=\min\{\sup_{a\in A}\nu a:\nu\text{ is a non-negative additive
functional on }\frak A,\,\nu 1=1\}.\cr}$$

\noindent(This is the {\bf Kelley covering number} of $A$.)
%391I

\spheader 391Xl Let $\frak A$ be a Boolean algebra.   (i) Show that the
following are equiveridical:  ($\alpha$) there is a functional $\bar\mu$
such that $(\frak A,\bar\mu)$ is a localizable measure algebra;  ($\beta$)
$L^{\infty}(\frak A)$ is a perfect Riesz space (definition:  356J).
(ii) Show that in this case $\frak A$ is a measurable algebra iff it is
ccc.
%391C

\leader{391Y}{Further exercises (a)}
%\spheader 391Ya
Show that in 391D and 391K we can replace `weakly
$(\sigma,\infty)$-distributive' by `weakly $\sigma$-distributive'.
%391D

\spheader 391Yb Show that $\Cal P\Bbb N$ is
chargeable but that the quotient algebra
$\Cal P\Bbb N/[\Bbb N]^{<\omega}$ is not ccc, therefore not chargeable.
%391Xa

\spheader 391Yc(i)
Show that if $X$ is a separable topological space, then its regular
open algebra is chargeable.
(ii) Let $\langle X_i\rangle_{i\in I}$ be any family of
topological spaces with chargeable regular open algebras.   Show that
their product has a chargeable regular open algebra.
%391Xa

\spheader 391Yd Let $\mu$ be Lebesgue measure on $[0,1]$, and $\Sigma$ its
domain.   Let $\Cal A$ be a non-empty family of non-empty subsets of $X$,
with intersection number $\delta$,
and let $\Cal W$ be the family of those sets 
$W\in\Cal PX\tensorhat\Sigma$ such that $W^{-1}[\{t\}]\in\Cal A$ for every
$t\in[0,1]$.   Set 
$\alpha=\inf_{W\in\Cal W}\sup_{x\in X}\mu W[\{x\}]$.   (i) Show that
$\alpha\le\delta$.   (ii) Give an example in which $\alpha<\delta$.

\spheader 391Ye\dvAnew{2011} 
Let $\frak A$ be a Boolean algebra, $\frak B$ a subalgebra
of $\frak A$, $U$ a linear space and $\nu_0:\frak B\to U$ an additive
functional.   Show that there is an additive functional 
$\nu:\frak A\to U$ extending $\nu_0$.   \Hint{361F.}
%391G out of order query
}%end of exercises

\cmmnt{\Notesheader{391} By the standards of this volume, this is an
easy section;  I note that I have hardly called on anything after
Chapter 32, except for a reference to the construction $S(\frak A)$ in
\S361.   I do ask for a bit of functional analysis (the Hahn-Banach
theorem) in 391E.

391J-391K are due to {\smc Kelley 59};
condition (ii) of 391J is called {\bf Kelley's criterion}.   It provides
some sort of answer to the question `which Boolean algebras carry
strictly positive finitely additive functionals?', but leaves quite open
the possibility that there is some more abstract criterion which is also
necessary and sufficient.   It is indeed a non-trivial exercise to find
any ccc Boolean algebra which does not carry a strictly positive
finitely additive functional.   The first example published seems to
have been that of {\smc Gaifman 64}, which is described in
{\smc Comfort \& Negrepontis 82}.   But for the purposes of this book
Gaifman's example
has been superseded by Talagrand's example, presented in \S394.

Kelley's criterion is a little unsatisfying.
It is undoubtedly important (see 392F
below), but at the same time the structure of the criterion -- a special
sequence of subsets of $\frak A$ -- is rather close to the structure of
the conclusion;  after all, one is, or can be represented by, a function
from $\frak A\setminus\{0\}$ to $\Bbb N$, while the other is a function
from $\frak A$ to $\Bbb R$.   Also the actual intersection number of a
family $A\subseteq\frak A\setminus\{0\}$ can be hard to calculate;  as
often as not, the best method is to look at the additive functionals
on $\frak A$ (see 391Xi).
}%end of notes

\discrpage

