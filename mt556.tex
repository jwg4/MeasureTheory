\frfilename{mt556.tex}
\versiondate{3.1.15}
\copyrightdate{2007}

\def\chaptername{Forcing}
\def\sectionname{Forcing with Boolean subalgebras}

\newsection{556}

I propose
now to describe a completely different way in which forcing can be
used to throw light on problems in measure theory.   Rather than finding
forcing models of new mathematical universes, we look for models which will
express structures of the ordinary universe in new ways.   The problems to
which this approach seems to be most relevant are those centered on
invariant algebras:  in ergodic theory, fixed-point algebras;  in the
theory of relative independence, the core $\sigma$-algebras.

Most of the section is taken up with development of basic machinery.
The strategic plan is straightforward enough:  given a
specific Boolean algebra
$\frak C$ which seems to be central to a question in hand, force with
$\frak C\setminus\{0\}$, and translate the question into a question
in the forcing language.   In order to do this, we need an efficient scheme
for automatic translation.   This is what
556A-556L % 556A 556B 556C 556D 556E 556G 556H 556I 556J 556K 556L
and 556O are setting up.   The translation has to work both ways, since we
need to be able to deduce properties of the ground model from properties of
the forcing model.

There are four actual theorems for which I offer proofs by these methods.
The first three are 556M
(a strong law of large numbers), 556N (Dye's theorem on orbit-isomorphic
measure-preserving transformations) and
556P (Kawada's theorem on invariant measures).   In each of these,
the aim is
to prove a general form of the theorem from the classical special case in
which the algebra $\frak C$ is trivial.   In the final example 556S
(I.Farah's description of the Dedekind completion of the asymptotic density
algebra $\frak Z$), we have a natural subalgebra $\frak C$
of $\frak Z$ and a structure in the corresponding
forcing universe to which we can apply Maharam's theorem.

\vleader{60pt}{556A}{Forcing with Boolean subalgebras}
Let $\frak A$ be a Boolean algebra, not $\{0\}$, and
$\frak C$ a subalgebra of $\frak A$.   Let $\Bbb P$ be the forcing notion
$\frak C^+=\frak C\setminus\{0\}$, active downwards.

\spheader 556Aa If $a\in\frak A$, the
{\bf forcing name for $a$ over $\frak C$} will be the $\Bbb P$-name

\Centerline{$\dot a
=\{(\check b,p):p\in\frak C^+$, $b\in\frak A$, $p\Bcap b\Bsubseteq a\}$.}

\spheader 556Ab If $\frak B$ is a Boolean subalgebra of $\frak A$ including
$\frak C$, then the {\bf forcing name for $\frak B$ over $\frak C$}
will be the
$\Bbb P$-name $\{(\dot b,1):b\in\frak B\}$, where here
$1=1_{\frak A}=1_{\frak B}=1_{\frak C}$.

\spheader 556Ac For each of the binary
operations $\frmedcirc=\Bcap$, $\Bcup$, $\Bsymmdiff$,
$\Bsetminus$ on $\frak A$, the {\bf forcing name for $\frmedcirc$ over
$\frak C$} will be the $\Bbb P$-name

\Centerline{$\dot\frmedcirc=\{(((\dot a,\dot b),(a\frmedcirc b)^{\centerdot}),1):
a$, $b\in\frak A\}$.}

\spheader 556Ad The {\bf forcing name for $\Bsubseteq$ over $\frak C$}
will be the $\Bbb P$-name

\Centerline{$\dot{\Bsubseteq}=\{((\dot a,\dot b),1):a$, $b\in\frak A$,
$a\Bsubseteq b\}$.}

\spheader 556Ae Let $\pi:\frak A\to\frak A$ be a ring
homomorphism such that $\pi c\Bsubseteq c$
for every $c\in\frak C$.   In this case, I
will say that the {\bf forcing name for $\pi$ over $\frak C$} is the
$\Bbb P$-name $\{((\dot a,(\pi a)^{\centerdot}),1):a\in\frak A\}$.

\spheader 556Af Now suppose that $\frak A$ is Dedekind $\sigma$-complete.
For $u\in L^0(\frak A)$, the {\bf forcing name for $u$ over $\frak C$} will
be the $\Bbb P$-name $\{((\check\alpha,\Bvalue{u>\alpha}^{\centerdot}),1):
\alpha\in\Bbb Q\}$.

\medskip

\noindent{\bf Remark}\cmmnt{ We shall need to agree on what it is that
the formula $L^0(\frak A)$
abbreviates.}   The primary definition in 364Aa speaks of functions from
$\Bbb R$ to $\frak A$.   Because $\Bbb R$ is inadequately absolute this is
not convenient here, and I will move to the alternative version\cmmnt{ in
364Af}:   a member $u$ of
$L^0(\frak A)$ is a family $\family{\alpha}{\Bbb Q}{\Bvalue{u>\alpha}}$
in $\frak A$ such that

\Centerline{$\Bvalue{u>\alpha}
=\sup_{\beta\in\Bbb Q,\beta>\alpha}\Bvalue{u>\beta}$ for every
$\alpha\in\Bbb Q$,}

\Centerline{$\inf_{n\in\Bbb N}\Bvalue{u>n}=0$,\quad
$\sup_{n\in\Bbb N}\Bvalue{u>-n}=1$.}

\vleader{72pt}{556B}{Theorem}
Let $\frak A$ be a Boolean algebra, not $\{0\}$, and
$\frak C$ a subalgebra of $\frak A$.   Let $\Bbb P$ be the forcing notion
$\frak C^+$, active downwards, and
$\dot\frak A$ the forcing name for $\frak A$ over $\frak C$.

(a) If $p\in\frak C^+$, $a$, $b\in\frak A$ and $\dot a$, $\dot b$ are the
forcing names of $a$, $b$ over $\frak C$, then

\Centerline{$p\VVdP\,\dot a=\dot b$}

\noindent iff $\upr(p\Bcap(a\Bsymmdiff b),\frak C)=0$, that is,
for every $q$
stronger than $p$ there is an $r$ stronger than $q$ such that
$r\Bcap a=r\Bcap b$.   In particular,

\Centerline{$p\VVdP\,\dot a=\dot b$}

\noindent whenever $p\Bcap a=p\Bcap b$.

(b) Writing $\dot\frmedcirc$ for the forcing name for $\frmedcirc$ over $\frak C$,

\Centerline{$\VVdP\,\dot{\frmedcirc}$ is a binary operation on
$\dot\frak A$ and $\dot a\,\dot\frmedcirc\,\dot b=(a\frmedcirc b)^{\centerdot}$}

\noindent for each of the binary
operations $\frmedcirc=\Bcap$, $\Bcup$, $\Bsymmdiff$
and $\Bsetminus$ and all $a$, $b\in\frak A$.

(c) All the standard identities translate.   For instance,

\Centerline{$\VVdP\,x\dot\Bcap(y\dot\Bsymmdiff z)
  =(x\dot\Bcap y)\dot\Bsymmdiff(x\dot\Bcap z)$ for all $x$, $y$,
  $z\in\dot\frak A$.}

(d)

\doubleinset{$\VVdP\,\dot\frak A$, with the operations $\dot\Bsymmdiff$,
$\dot\Bcap$, $\dot\Bcup$ and $\dot\Bsetminus$, is a Boolean algebra,
with zero $\dot 0$ and identity $\dot 1$.}

(e)(i) Writing $\dot{\Bsubseteq}$ for the forcing name for $\Bsubseteq$
over $\frak C$,

\Centerline{$\VVdP\,\dot{\Bsubseteq}$ is the inclusion relation in the
Boolean algebra $\dot\frak A$.}

\quad(ii) For $a$, $b\in\frak A$ and $p\in\frak C^+$,

\Centerline{$p\VVdP\,\dot a\dot{\Bsubseteq}\dot b$}

\noindent iff $\upr(p\Bcap a\Bsetminus b,\frak C)=0$.

(f) If $\frak B$ is a Boolean subalgebra of $\frak A$ including $\frak C$,
then

\Centerline{$\VVdP\,\dot\frak B$ is a Boolean subalgebra of $\dot\frak A$.}

\proof{{\bf (a)(i)} Recall that
$\upr(a,\frak C)=\inf\{c:a\Bsubseteq c\in\frak C\}$ if the infimum is
defined in $\frak C$ (313S).
So $\upr(a,\frak C)=0$ iff for every
non-zero $c\in\frak C$ there is a $c'\in\frak C$ such that $a\Bsubseteq c'$
and $c\notBsubseteq c'$;  that is, for every non-zero $c\in\frak C$ there
is a non-zero $c'\in\frak C$ such that $c'\Bsubseteq c\Bsetminus a$.   In
the present context, we see that for $p\in\frak C^+$ and $a$,
$b\in\frak A$, $\upr(p\Bcap(a\Bsymmdiff b),\frak C)=0$ iff
for every $q$ stronger than $p$ there is an $r$ stronger than $q$ such that
$r\Bcap(a\Bsymmdiff b)=0$.

\medskip

\quad{\bf (ii)} Suppose that $p\Bcap a=p\Bcap b$,
that $q\in\frak C^+$ is stronger than $p$, and that $\dot x$ is a
$\Bbb P$-name such that $q\VVdP\,\dot x\in\dot a$.   Then there are an
$r\in\frak C^+$, a $d\in\frak A$ such that $(\check d,r)\in\dot a$, and an
$s$ stronger than both $r$ and $q$ such that $s\VVdP\,\dot x=\check d$.
In this case

\Centerline{$s\Bcap d
\Bsubseteq p\Bcap r\Bcap d\Bsubseteq p\Bcap a\Bsubseteq b$,}

\noindent so $(\check d,s)\in\dot b$ and

\Centerline{$s\VVdP\,\dot x=\check d\in\dot b$.}

\noindent As $q$ and $\dot x$ are arbitrary,

\Centerline{$p\VVdP\,\dot a$ is a subset of $\dot b$;}

\noindent similarly,

\Centerline{$p\VVdP\,\dot b$ is a subset of $\dot a$ and $\dot b=\dot a$.}

\medskip

\quad{\bf (iii)} If $\upr(p\Bcap(a\Bsymmdiff b),\frak C)=0$, then for every
$q$ stronger than $p$ there is an $r$ stronger than $q$ such that
$r\Bcap a=r\Bcap b$ and $r\VVdP\,\dot a=\dot b$, by (ii).
As $q$ is arbitrary, $p\VVdP\,\dot a=\dot b$.

\medskip

\quad{\bf (iv)} Now suppose that $p\VVdP\,\dot a=\dot b$ and that $q$ is
stronger than $p$.   Then $(\check a,q)\in\dot a$, so
$q\VVdP\,\check a\in\dot a=\dot b$.   There must therefore be a
$(\check d,r)\in\dot b$ and an $s$ stronger than both $r$ and $q$ such
that $s\VVdP\,\check a=\check d$;  in this case $d=a$,
$s\Bcap a\Bsubseteq r\Bcap d\Bsubseteq b$ and $s\Bcap a\Bsetminus b=0$.

As $q$ is arbitrary, $\upr(p\Bcap(a\Bsetminus b),\frak C)=0$.   Similarly,
$\upr(p\Bcap(b\Bsetminus a),\frak C)=0$.   By 313Sb,
$\upr(p\Bcap(a\Bsymmdiff b),\frak C)=0$.

\medskip

{\bf (b)} Of course

\Centerline{$\VVdP\,\dot{\frmedcirc}
\subseteq(\dot\frak A\times\dot\frak A)\times\dot\frak A$,}

\noindent just because

\Centerline{$\VVdP\,\dot a\in\dot\frak A$}

\noindent for every $a\in\frak A$.   To see that $\dot\frmedcirc$ is a name for
a function with domain $\dot A\times\dot A$, use 5A3H.   If
$(((\dot a_1,\dot b_1),(a_1\frmedcirc b_1)^{\centerdot}),1)$ and
$(((\dot a_2,\dot b_2),(a_2\frmedcirc b_2)^{\centerdot}),1)$ are two members of
$\dot\frmedcirc$, and $p\in\frak C^+$ is such that

\Centerline{$p\VVdP\,(\dot a_1,\dot b_1)=(\dot a_2,\dot b_2)$,}

\noindent then

$$\eqalign{\upr(p\Bcap((a_1\frmedcirc b_1)\Bsymmdiff(a_2\frmedcirc b_2)))
&\Bsubseteq\upr(p\Bcap((a_1\Bsymmdiff a_2)\Bcup(b_1\Bsymmdiff b_2)))\cr
&=\upr(p\Bcap(a_1\Bsymmdiff a_2))\Bcup\upr(p\Bcap(b_1\Bsymmdiff b_2))
=0\cr}$$

\noindent by (a) above and 313Sb again.   So

\Centerline{$p\VVdP\,(a_1\frmedcirc b_1)^{\centerdot}
=(a_2\frmedcirc b_2)^{\centerdot}$}

\noindent by (a) in the other direction.   Thus the condition of
5A3H(a-ii) is satisfied, and

\Centerline{$\VVdP\,\dot\frmedcirc$ is a function,}

\noindent while of course

\Centerline{$\VVdP\,\dot a\dot\frmedcirc\dot b=(a\frmedcirc b)^{\centerdot}$}

\noindent for all $a$, $b\in\frak A$.   Moreover, setting
$\dot A=\{((\dot a,\dot b),1):a$, $b\in\frak A\}$, 5A3Hb tells us that

\Centerline{$\VVdP\,\dom\dot\frmedcirc=\dot A=\dot\frak A\times\dot\frak A$,
so $\dot\frmedcirc$ is a binary operation on $\dot\frak A$.}

\woddheader{556B}{4}{2}{2}{30pt}

{\bf (c)} I work through only the given example.
Suppose that $p\in\frak C^+$ and that $\dot x$, $\dot y$ and
$\dot z$ are $\Bbb P$-names such that

\Centerline{$p\VVdP\,\dot x$, $\dot y$, $\dot z\in\dot\frak A$.}

\noindent If $q$ is stronger than $p$, there are an $r$ stronger than
$q$ and $a$, $b$, $c\in\frak A$ such
that

\Centerline{$r\VVdP\,\dot x=\dot a$, $\dot y=\dot b$ and $\dot z=\dot c$.}

\noindent Then

\Centerline{$r\VVdP\,\dot y\dot\Bsymmdiff\dot z=\dot b\dot\Bsymmdiff\dot c
  =(b\Bsymmdiff c)^{\centerdot}$,}

\Centerline{$r\VVdP\,
  \dot x\dot\Bcap(\dot y\dot\Bsymmdiff\dot z)
  =(a\Bcap(b\Bsymmdiff c))^{\centerdot}
  =((a\Bcap b)\Bsymmdiff(a\Bcap c))^{\centerdot}
  =(\dot x\dot\Bcap\dot y)\dot\Bsymmdiff(\dot x\dot\Bcap\dot z)$.}

\noindent As $q$ is arbitrary,

\Centerline{$p\VVdP\,
  \dot x\dot\Bcap(\dot y\dot\Bsymmdiff\dot z)
  =(\dot x\dot\Bcap\dot y)\dot\Bsymmdiff(\dot x\dot\Bcap\dot z)$;}

\noindent as $p$, $\dot x$, $\dot y$ and $\dot z$ are arbitrary,

\Centerline{$\VVdP\,x\dot\Bcap(y\dot\Bsymmdiff z)
  =(x\dot\Bcap y)\dot\Bsymmdiff(x\dot\Bcap z)$ for all $x$, $y$,
  $z\in\dot\frak A$.}

\medskip

{\bf (d)} This is now elementary, amounting to repeated use of the
technique in (c).

\medskip

{\bf (e)(i)} It will be enough to show that

\Centerline{$\VVdP$ for all $x$, $y\in\dot\frak A$,
$x\dot{\Bsubseteq}y\iff x\dot\Bcap y=x$.}

\noindent\Prf\ Suppose that $p\in\frak C^+$ and that $\dot x$,
$\dot y$ are $\Bbb P$-names such that

\Centerline{$p\VVdP\,\dot x$, $\dot y\in\dot\frak A$.}

\noindent($\alpha$) Suppose that $p\VVdP\,\dot x\dot{\Bsubseteq}\dot y$.
If $q$ is stronger than $p$, there are an $r$ stronger than $q$ and $a$,
$b\in\frak A$ such that $a\Bsubseteq b$ and

\Centerline{$r\VVdP\,\dot x=\dot a$ and $\dot y=\dot b$.}

\noindent Now

\Centerline{$r\VVdP\,\dot x\dot\Bcap\dot y
=(a\Bcap b)^{\centerdot}=\dot a=\dot x$;}

\noindent as $q$ is arbitrary, $p\VVdP\,\dot x\dot\Bcap\dot y=\dot x$.
($\beta$) Conversely, suppose that
$p\VVdP\,\dot x\dot\Bcap\dot y=\dot x$.   If $q$ is stronger than $p$ there
are $r$ stronger than $q$ and $a$, $b\in\frak A$ such that

\Centerline{$r\VVdP\,\dot x=\dot a$, $\dot y=\dot b$,
  $(a\Bcap b)^{\centerdot}=\dot a$;}

\noindent now $(((a\Bcap b)^{\centerdot},\dot b),1)\in\dot{\Bsubseteq}$,
so $\VVdP\,(a\Bcap b)^{\centerdot}\dot{\Bsubseteq}\dot b$ and

\Centerline{$r\VVdP\,\dot x=\dot a
=(a\Bcap b)^{\centerdot}\dot{\Bsubseteq}\dot b=\dot y$.}

\noindent As $q$ is arbitrary. $p\VVdP\,\dot x\dot{\Bsubseteq}\dot y$.

As $p$, $\dot x$ and $\dot y$ are arbitrary,

\Centerline{$\VVdP$ for $x$, $y\in\dot\frak A$,
$x\dot{\Bsubseteq}y\iff x\dot\Bcap y=x$.  \Qed}

\medskip

\quad{\bf (ii)} Now, for $a$, $b\in\frak A$ and $p\in\frak C^+$,

$$\eqalign{p\VVdP\,&\dot a\dot{\Bsubseteq}\dot b\cr
&\text{iff }p\VVdP\,\dot a\dot\Bcap\dot b=\dot a\cr
&\text{iff }p\VVdP\,(a\Bcap b)^{\centerdot}=\dot a\cr
&\text{iff }\upr(p\Bcap(a\Bsymmdiff(a\Bcap b)),\frak C)=0\cr
&\text{iff }\upr(p\Bcap a\Bsetminus b,\frak C)=0.\cr}$$

\medskip

{\bf (f)} This should now be easy.   As $\dot\frak B\subseteq\dot\frak A$,
$\VVdP\,\dot\frak B\subseteq\dot\frak A$.
If $p\in\frak C^+$ and $\dot x$, $\dot y$
are $\Bbb P$-names such that $p\VVdP\,\dot x$, $\dot y\in\dot\frak B$, then
for every $q$ stronger than $p$ there are $r$ stronger than $q$ and $a$,
$b\in\frak B$ such that $r\VVdP\,\dot x=\dot a$ and $\dot y=\dot b$.   In
this case

\Centerline{$r\VVdP\,
\dot x\dot\Bcap\dot y=(a\Bcap b)^{\centerdot}\in\dot\frak B$,
$\dot x\dot\Bsymmdiff\dot y=(a\Bsymmdiff b)^{\centerdot}\in\dot\frak B$;}

\noindent as $q$ is arbitrary,

\Centerline{$p\VVdP\,
\dot x\dot\Bcap\dot y$, $\dot x\dot\Bsymmdiff\dot y\in\dot\frak B$.}

\noindent As $p$, $\dot x$ and $\dot y$ are arbitrary,

\Centerline{$\VVdP\,\dot\frak B$ is a subring of $\dot\frak A$;}

\noindent as we also have
$\VVdP\,\dot 1\in\dot\frak B$, we get

\Centerline{$\VVdP\,\dot\frak B$ is a subalgebra
of $\dot\frak A$.}
}%end of proof of 556B

\leader{556C}{Theorem} Let $\frak A$ be a Boolean algebra, not $\{0\}$, and
$\frak C$ a subalgebra of $\frak A$.   Let $\Bbb P$ be the forcing notion
$\frak C^+$, active downwards, $\dot\frak A$ the
forcing name for $\frak A$ over $\frak C$, and
$\pi:\frak A\to\frak A$ a ring homomorphism such that
$\pi c\Bsubseteq c$ for every $c\in\frak C$;  write $\dot\pi$ for
the forcing name for $\pi$ over $\frak C$.

(a)(i)

\Centerline{$\VVdP\,\dot\pi$ is a ring homomorphism from $\dot\frak A$
to itself}

\noindent and

\Centerline{$\VVdP\,\dot\pi(\dot a)=(\pi a)^{\centerdot}$}

\noindent for every $a\in\frak A$.

\quad(ii) If $\pi$ is injective, $\VVdP\,\dot\pi$ is injective.

\quad(iii) If $\phi:\frak A\to\frak A$ is another ring homomorphism such
that $\phi c\Bsubseteq c$ for every $c\in\frak C$, with corresponding
forcing name $\dot\phi$, then

\Centerline{$\VVdP\,\dot\pi\dot\phi=(\pi\phi)^{\centerdot}$.}

(b) Now suppose that $\pi$ is a Boolean homomorphism.

\quad(i) $\pi c=c$ for every $c\in\frak C$.

\quad(ii) $\VVdP\,\dot\pi$ is a Boolean homomorphism.

\quad(iii) If $\pi$ is surjective, $\VVdP\,\dot\pi$ is surjective.

\quad(iv) If $\pi\in\Aut\frak A$ then

\Centerline{$\VVdP\,\dot\pi$ is a Boolean automorphism and
$(\dot\pi)^{-1}=(\pi^{-1})^{\centerdot}$.}

\quad(v) If the fixed-point subalgebra of $\pi$ is $\frak C$ exactly, then

\Centerline{$\VVdP$ the fixed-point subalgebra of $\dot\pi$
is $\{0,1\}$.}

\proof{{\bf (a)(i)}\grheada\
It will help to note straight away that
$\pi c=c\Bcap\pi 1$ for every $c\in\frak C$.   \Prf\
The hypothesis is that
$\pi c\Bsubseteq c$;  because $\pi$ is a ring homomorphism,
$\pi c\Bsubseteq\pi 1$, so $\pi c\Bsubseteq c\Bcap\pi 1$.
Since also

\Centerline{$\pi c=\pi 1\Bsetminus\pi(1\Bsetminus c)
\Bsupseteq\pi 1\Bsetminus(1\Bsetminus c)
=c\Bcap\pi 1$,}

\noindent we have equality.\ \QeD\  Consequently

\Centerline{$c\Bcap\pi a=c\Bcap\pi(1\Bcap a)=c\Bcap\pi 1\Bcap\pi a
=\pi c\Bcap\pi a=\pi(c\Bcap a)$}

\noindent whenever $c\in\frak C$ and $a\in\frak A$.

\medskip

\qquad\grheadb\ $\VVdP\,\dot\pi$ is a function from $\dot\frak A$ to
itself.   \Prf\ Of course
$\VVdP\,\dot\pi\subseteq\dot\frak A\times\dot\frak A$.
Suppose that $p\in\frak C^+$ and that
$((\dot a,(\pi a)^{\centerdot}),1)$,
$((\dot b,(\pi b)^{\centerdot}),1)$ are two members of $\dot\pi$ such that
$p\VVdP\,\dot a=\dot b$.   Then for every $q$ stronger than $p$ there is an
$r$ stronger than $q$ such that $r\Bcap a=r\Bcap b$ (556Ba), in which case

\Centerline{$r\Bcap\pi a=\pi(r\Bcap a)=\pi(r\Bcap b)=r\Bcap\pi b$.}

\noindent This shows that
$p\VVdP\,(\pi a)^{\centerdot}=(\pi b)^{\centerdot}$, by 556Ba in the other
direction.   As $a$ and $b$ are arbitrary, the condition of 5A3H is
satisfied, with $\dot A$ there exactly equal to $\dot\frak A$ here,
and

\Centerline{$\VVdP\,\dot\pi$ is a function with domain $\dot\frak A$.
\text{\Qed}}

If $a\in\frak A$ then $((\dot a,(\pi a)^{\centerdot}),1)\in\dot\pi$ so

\Centerline{$\VVdP\,(\dot a,(\pi a)^{\centerdot})\in\dot\pi$ and
$\dot\pi(\dot a)=(\pi a)^{\centerdot}$.}

\woddheader{556C}{4}{2}{2}{42pt}

\qquad\grheadc\ $\VVdP\,\dot\pi$ is a ring homomorphism.   \Prf\
Writing $\frmedcirc$ for either $\Bcap$ or $\Bsymmdiff$,

$$\eqalignno{\VVdP\,\dot\pi(\dot a\,\dot\frmedcirc\,\dot b)
&=\dot\pi(a\frmedcirc b)^{\centerdot}
=(\pi(a\frmedcirc b))^{\centerdot}\cr
&=(\pi a\frmedcirc\pi b)^{\centerdot}
=(\pi a)^{\centerdot}\,\dot\frmedcirc\,(\pi b)^{\centerdot}
=(\dot\pi\dot a)\,\dot\frmedcirc\,(\dot\pi\dot b)\cr}$$

\noindent for all $a$, $b\in\frak A$.   If now $p\in\frak C^+$ and
$\dot x$, $\dot y$ are $\Bbb P$-names such that

\Centerline{$p\VVdP\,\dot x,\,\dot y\in\dot\frak A$,}

\noindent then for any $q$ stronger than $p$ there are $r$ stronger than
$q$ and $a$, $b\in\frak A$ such that

\Centerline{$r\VVdP\,\dot x=\dot a$ and $\dot y=\dot b$,}

\noindent in which case

\Centerline{$r\VVdP\,\dot\pi(\dot x\,\dot\frmedcirc\,\dot b)
=\dot\pi(\dot a\,\dot\frmedcirc\,\dot b)
=\dot\pi(\dot a)\,\dot\frmedcirc\,\dot\pi(\dot b)
=\dot\pi(\dot x)\,\dot\frmedcirc\,\dot\pi(\dot y)$.}

\noindent As $q$ is arbitrary,

\Centerline{$p\VVdP\,\dot\pi(\dot x\,\dot\frmedcirc\,\dot y)
=\dot\pi(\dot x)\,\dot\frmedcirc\,\dot\pi(\dot y)$;}

\noindent as $p$, $\dot x$ and $\dot y$ are arbitrary,

\Centerline{$\VVdP\,\dot\pi(x\,\dot\frmedcirc\,y)
=(\dot\pi x)\,\dot\frmedcirc\,(\dot\pi y)$ for all $x$, $y\in\dot\frak A$.}

\noindent As this is true for both $\dot\frmedcirc=\dot\Bcap$ and
$\dot\frmedcirc=\dot\Bsymmdiff$,

\Centerline{$\VVdP\,\dot\pi$ is a ring homomorphism.  \Qed}

\medskip

\quad{\bf (ii)}
Let $p\in\frak C^+$ and a $\Bbb P$-name $\dot x$ be such that

\Centerline{$p\VVdP\,\dot x\in\dot\frak A$ and $\dot\pi\dot x=0$.}

\noindent For any $q$ stronger than $p$, there are an $r$ stronger than $q$
and an $a\in\frak A$ such that

\Centerline{$r\VVdP\,\dot a=\dot x$, therefore
$0=\dot\pi\dot a=(\pi a)^{\centerdot}$.}

\noindent By 556Ba, there is an $s$ stronger than $r$ such that
$s\Bcap\pi a=0$;  since $\pi s\Bsubseteq s$,
$\pi(s\Bcap a)=0$.   As $\pi$ is injective, $s\Bcap a=0$ and (using 556Ba
again)

\Centerline{$s\VVdP\,\dot x=\dot a=0$.}

\noindent As $q$ is arbitrary, $p\VVdP\,\dot x=0$;  as $p$ and $\dot x$ are
arbitrary, $\VVdP\,\dot\pi$ is injective.

\medskip

\quad{\bf (iii)} Suppose that $p\in\frak C^+$ and that
$\dot x$ is a $\Bbb P$-name such
that $p\VVdP\,\dot x\in\dot\frak A$.   For any $q$ stronger than $p$,
there are an $r$ stronger than $q$ and an
$a\in\frak A$ such that $r\VVdP\,\dot x=\dot a$, so that

\Centerline{$r\VVdP\,\dot\pi(\dot\phi(\dot x))
=\dot\pi(\dot\phi(\dot a))
=\dot\pi((\phi a)^{\centerdot})
=(\pi\phi a)^{\centerdot}
=(\pi\phi)^{\centerdot}(\dot a)
=(\pi\phi)^{\centerdot}(\dot x)$.}

\noindent As $q$ is arbitrary,

\Centerline{$p\VVdP\,\dot\pi(\dot\phi(\dot x))
=(\pi\phi)^{\centerdot}(\dot x)$;}

\noindent as $p$ and $\dot x$ are arbitrary,
$\VVdP\,\dot\pi\dot\phi=(\pi\phi)^{\centerdot}$.

\medskip

{\bf (b)(i)} I observed in (a-i-$\alpha$) above that $\pi c=c\Bcap\pi 1$
for every $c\in\frak C$, so if $\pi 1=1$ we shall have $\pi c=c$ for every
$c\in\frak C$.

\medskip

\quad{\bf (ii)} I pointed out in 556Bd that

\Centerline{$\VVdP\,\dot 1$ is the identity of $\dot\frak A$,}

\noindent and we now have

\Centerline{$\VVdP\,\dot\pi(\dot 1)=(\pi 1)^{\centerdot}=\dot 1$,
so $\dot\pi$ is a Boolean homomorphism.}

\medskip

\quad{\bf (iii)} Let $p\in\frak C^+$ and a $\Bbb P$-name $\dot x$
be such that $p\VVdP\,\dot x\in\dot\frak A$.
For any $q$ stronger than $p$, there are an $r$ stronger than $q$
and an $a\in\frak A$
such that $r\VVdP\,\dot a=\dot x$.   Now there is a $b\in\frak A$ such that
$a=\pi b$, in which case

\Centerline{$r\VVdP\,\dot x=\dot a=\dot\pi\dot b\in\dot\pi[\dot\frak A]$.}

\noindent As $q$ is arbitrary, $p\VVdP\,\dot x\in\dot\pi[\dot\frak A]$;
as $p$ and $\dot x$ are arbitrary, $\VVdP\,\dot\pi$ is surjective.

\medskip

\quad{\bf (iv)} By (a-iii),

\Centerline{$\VVdP\,\dot\pi(\pi^{-1})^{\centerdot}
=(\pi^{-1})^{\centerdot}\dot\pi=\dot\iota$}

\noindent where $\iota:\frak A\to\frak A$ is the identity automorphism.
But

\Centerline{$\VVdP\,\dot\iota$ is the identity on $\dot\frak A$.}

\noindent\Prf\ If $p\in\frak C^+$ and a $\Bbb P$-name $\dot x$ are such
that
$p\VVdP\,\dot x\in\dot\frak A$, then for any $q$ stronger than $p$ there
are an $r$ stronger than $q$ and an $a\in\frak A$ such that

\Centerline{$r\VVdP\,\dot x=\dot a=(\iota a)^{\centerdot}
=\dot\iota\dot a=\dot\iota\dot x$.}

\noindent As $q$ is arbitrary, $p\VVdP\,\dot\iota\dot x=\dot x$;  as
$p$ and $\dot x$ are arbitrary, $\VVdP\,\dot\iota$ is the identity.\ \QeD\
Putting these together,

\Centerline{$\VVdP\,(\pi^{-1})^{\centerdot}$ is the inverse of $\dot\pi$.}

\medskip

\quad{\bf (v)}
Suppose that $p\in\frak C^+$ and a $\Bbb P$-name $\dot x$ are such that

\Centerline{$p\VVdP\,\dot x\in\dot\frak A$ and $\dot\pi\dot x=\dot x$.}

\noindent For any $q$ stronger than $p$ there are an $r$ stronger than $q$
and an $a\in\frak A$ such that

\Centerline{$r\VVdP\,\dot a=\dot x=\dot\pi\dot x=\dot\pi\dot a
=(\pi a)^{\centerdot}$,}

\noindent and an $s$ stronger than $r$ such that
$s\Bcap a=s\Bcap\pi a$.   Now
$s\Bcap a=\pi(s\Bcap a)$ and
$s\Bcap a\in\frak C$.   If $s\Bcap a=0$, set $s'=s$;
then $s'\VVdP\,\dot x=0$.
Otherwise, set $s'=s\Bcap a$;  then $s'\VVdP\,\dot x=1$.   Thus in either
case we have an $s'$ stronger than $q$ such that
$s'\VVdP\,\dot x\in\{0,1\}$.   As $q$ is arbitrary,
$p\VVdP\,\dot x\in\{0,1\}$;  as $p$ and $\dot x$ are arbitrary,
$\VVdP\,\dot\pi$ has fixed-point subalgebra $\{0,1\}$.
}%end of proof of 556C

\leader{556D}{Regularly embedded \dvrocolon{subalgebras}}\cmmnt{ I am
trying to set these results out in maximal
generality, as usual.   However it seems that we need to move almost at
once to the case in which our subalgebra is regularly embedded, and we
have more effective versions of 556Ba and 556B(e-ii).

\medskip

\noindent}{\bf Proposition} Let $\frak A$ be a Boolean algebra,
not $\{0\}$, and   $\frak C$ a regularly embedded subalgebra of $\frak A$.
Let $\Bbb P$ be the
forcing notion $\frak C^+$, active downwards, and
for $a\in\frak A$ let $\dot a$ be the forcing name for $a$ over $\frak C$.

(a) For $p\in\frak C^+$ and $a$, $b\in\frak A$,

\Centerline{$p\VVdP\,\dot a=\dot b$}

\noindent iff $p\Bcap a=p\Bcap b$.

(b) Let $\dot{\Bsubseteq}$ be the forcing name for $\Bsubseteq$ over
$\frak C$.   Then for $p\in\frak C^+$ and $a$, $b\in\frak A$,

\Centerline{$p\VVdP\,\dot a\dot{\Bsubseteq}\dot b$}

\noindent iff $p\Bcap a\Bsubseteq b$.

\proof{ The point is just that $\upr(a,\frak C)=0$ only when $a=0$, because
infima in $\frak C$ are also infima in $\frak A$ (313N);  so that
$\upr(p\Bcap a\Bsetminus b,\frak C)=0$ iff $p\Bcap a\Bsubseteq b$, and
$\upr(p\Bcap(a\Bsymmdiff b),\frak C)=0$ iff $p\Bcap a=p\Bcap b$.
}%end of proof of 556D

\leader{556E}{Proposition} Let $\frak A$ be a Boolean algebra,
not $\{0\}$, $\frak C$ a
regularly embedded subalgebra of $\frak A$, $\Bbb P$ the forcing notion
$\frak C^+$, active downwards,
and $\dot\frak A$ the forcing name for $\frak A$ over
$\frak C$;  for $a\in\frak A$, write $\dot a$ for the forcing name for $a$
over $\frak C$.

(a) Let $\dot A$ be a $\Bbb P$-name, and set

\Centerline{$B
=\{q\Bcap a:q\in\frak C^+$, $a\in\frak A$, $q\VVdP\,\dot a\in\dot A\}$.}

\noindent Then for $d\in\frak A$ and $p\in\frak C^+$,

\Centerline{$p\VVdP\,\dot d$ is an upper bound for
$\dot A\cap\dot\frak A$}

\noindent iff $p\Bcap b\Bsubseteq d$ for every $b\in B$, and

\Centerline{$p\VVdP\,\dot d=\sup(\dot A\cap\dot\frak A)$}

\noindent iff $p\Bcap d=\sup_{b\in B}p\Bcap b$.

\wheader{556E}{0}{0}{0}{36pt}

(b)(i) If $\familyiI{a_i}$ is a family in $\frak A$ with supremum $a$, then

\Centerline{$\VVdP\,\dot a
=\sup_{i\in\check I}\dot a_i$.\footnote{See 5A3E for a note on the
interpretation of formulae of this kind.}}

\quad(ii) If $\familyiI{a_i}$ is a family in $\frak A$ with infimum $a$,
then

\Centerline{$\VVdP\,\dot a=\inf_{i\in\check I}\dot a_i$.}


(c) $\VVdP\,\sat(\dot\frak A)
\le\sat(\frak A)\var2spcheck$.\footnote{\smallerfonts{Of course I am not
asserting here that `$\eightVVdash_{\Bbb P}\,\sat(\frak A)\var2spcheck$
is a cardinal', only that
`$\eightVVdash_{\Bbb P}\,\sat(\dot\frak A)$ is a cardinal and
$\sat(\frak A)\var2spcheck$ is an ordinal'.}}

(d) $\VVdP\,\tau(\dot\frak A)\le\tau(\frak A)\var2spcheck$.

\wheader{556E}{0}{0}{0}{36pt}
\proof{{\bf (a)(i)} \Quer\ Suppose, if possible, that $b\in B$,
$p\Bcap b\notBsubseteq d$ and

\Centerline{$p\VVdP\,\dot d$ is an upper bound for
$\dot A\cap\dot\frak A$.}

\noindent Let $q\in\frak C^+$,
$a\in\frak A$ be such that $b=q\Bcap a$ and $q\VVdP\,\dot a\in\dot A$.
Then $p\Bcap q\ne 0$, so $p\Bcap q\in\frak C^+$ and

\Centerline{$p\Bcap q\VVdP\,\dot a\in\dot A\cap\dot\frak A$, therefore
$\dot a\dot{\Bsubseteq}\dot d$.}

\noindent It follows that $p\Bcap q\Bcap a\Bsubseteq d$ (556Db);  but
this contradicts the choice of $p$ and $b$.\ \Bang

Thus $p\Bcap b\Bsubseteq d$ whenever $b\in B$ and $p\VVdP\,\dot d$ is an
upper bound for $\dot A\cap\dot\frak A$.

\medskip

\quad{\bf (ii)} Next, suppose that $p\in\frak C^+$ and $d\in\frak A$ are
such that $p\Bcap b\Bsubseteq d$ for every $b\in B$.
Suppose that $q$ is stronger than $p$ and that $\dot x$ is a $\Bbb P$-name
such that $q\VVdP\,\dot x\in\dot A\cap\dot\frak A$.
If $r$ is stronger than $q$, there
are an $s$ stronger than $r$ and an $a\in\frak A$ such that
$s\VVdP\,\dot x=\dot a$.   In this case, $s\Bcap a\in B$ and
$s\Bcap a=p\Bcap s\Bcap a\Bsubseteq d$, so

\Centerline{$s\VVdP\,\dot x=\dot a
=(s\Bcap a)^{\centerdot}\dot{\Bsubseteq}\dot d$.}

\noindent As $r$ is arbitrary, $q\VVdP\,\dot x\dot{\Bsubseteq}\dot d$;  as
$q$ and $\dot x$ are arbitrary,

\Centerline{$p\VVdP\,\dot d$ is an upper bound for $\dot A$.}

\medskip

\quad{\bf (iii)} Putting these together, we see that
$p\Bcap b\Bsubseteq d$ for every $b\in B$
iff $p\VVdP\,\dot d$ is an upper bound for $\dot A\cap\dot\frak A$.

\medskip

\quad{\bf (iv)} Now suppose that
$p\VVdP\,\dot d=\sup(\dot A\cap\dot\frak A)$.   We know that $d$, and
therefore $p\Bcap d$, is an upper bound of $\{p\Bcap b:b\in B\}$.
If $e$ is any other upper bound of $\{p\Bcap b:b\in B\}$, then

\Centerline{$p\VVdP\,\dot e$ is an upper bound of $\dot A$, so
$\dot d\dot{\Bsubseteq}\dot e$}

\noindent and $p\Bcap d\Bsubseteq e$, by 556Db again;  thus
$p\Bcap d=\sup_{b\in B}p\Bcap b$.

\medskip

\quad{\bf (v)} Finally, suppose that $p\Bcap d=\sup_{b\in B}p\Bcap b$.
Suppose that $q$ is stronger than $p$ and that $\dot x$ is a
$\Bbb P$-name such that

\Centerline{$q\VVdP\,\dot x\in\dot\frak A$ is an upper bound of
$\dot A\cap\dot\frak A$.}

\noindent If $r$ is stronger than $q$, there are a
$s$ stronger than $r$
and a $c\in\frak A$ such that $s\VVdP\,\dot x=\dot c$.   In this case, by
(i), we must have $s\Bcap b\Bsubseteq c$ for every $b\in B$;
accordingly $s\Bcap d\Bsubseteq c$ (313Ba), so that

\Centerline{$s\VVdP\,\dot d\dot{\Bsubseteq}\dot c=\dot x$.}

\noindent As $r$ is arbitrary,

\Centerline{$q\VVdP\,\dot d\dot{\Bsubseteq}\dot x$;}

\noindent as $q$ and $\dot x$ are arbitrary,

\Centerline{$p\VVdP\,\dot d=\sup(\dot A\cap\dot\frak A)$.}

\medskip

{\bf (b)(i)} Of course

\Centerline{$\VVdP\,\dot a_i\dot{\Bsubseteq}\dot a$}

\noindent for every $i\in I$, so that

\Centerline{$\VVdP\,\dot a$ is an upper bound for
$\{\dot a_i:i\in\check I\}$.}

\noindent (Formally speaking:  if $p\in\frak C^+$ and $\dot x$ is a
$\Bbb P$-name such that $p\VVdP\,\dot x\in\{\dot a_i:i\in\check I\}$, then
for any $q$ stronger than $p$ there are an $r$ stronger than $q$ and an
$i\in I$ such that $r\VVdP\,\dot x=\dot a_i\dot{\Bsubseteq}\dot a$;  hence
$p\VVdP\,\dot x\dot{\Bsubseteq}\dot a$.)
In the other direction, suppose that $p\in\frak C^+$ and that
$\dot x$ is a $\Bbb P$-name such that

\Centerline{$p\VVdP\,\dot a_i\Bsubseteq\dot x\in\dot\frak A$ for every
$i\in\check I$.}

\noindent For any $q$ stronger than $p$ there are an $r$ stronger than $q$
and a $b\in\frak A$ such that $r\VVdP\,\dot x=\dot b$.   Now, for any
$i\in I$,

\Centerline{$r\VVdP\,\varchecki\in\check I$,
$\dot a_i\dot{\Bsubseteq}\dot x=\dot b$}

\noindent and therefore $r\Bcap a_i\Bsubseteq b$.   As $i$ is arbitrary,
$r\Bcap a\Bsubseteq b$ and $r\VVdP\,\dot a\dot{\Bsubseteq}\dot x$.
As $q$ is arbitrary, $p\VVdP\,\dot a\dot{\Bsubseteq}\dot x$; as $p$ and
$\dot x$ are arbitrary,

\Centerline{$\VVdP\,\dot a$ is the least upper bound of
$\{\dot a_i:i\in\check I\}$.}

\medskip

\quad{\bf (ii)} Now

$$\eqalign{\VVdP\,\inf_{i\in\check I}\dot a_i
&=1\dot\Bsetminus\sup_{i\in\check I}(1\dot\Bsetminus a_i)
=1\dot\Bsetminus\sup_{i\in\check I}(1\Bsetminus a_i)^{\centerdot}\cr
&=1\dot\Bsetminus(\sup_{i\in I}(1\Bsetminus a_i))^{\centerdot}
=(1\Bsetminus(\sup_{i\in I}(1\Bsetminus a_i)))^{\centerdot}
=(\inf_{i\in I}a_i)^{\centerdot}.\cr}$$

\medskip

{\bf (c)} \Quer\ Otherwise, there are a $p\in\frak C^+$ and a family
$\ofamily{\xi}{\kappa}{\dot x_{\xi}}$, where $\kappa=\sat(\frak A)$,
such that

\Centerline{$p\VVdP\,\dot x_{\xi}\in\dot\frak A^+$ for every
$\xi<\check\kappa$ and $\dot x_{\xi}\dot\Bcap\dot x_{\eta}=0$ whenever
$\xi<\eta<\check\kappa$.}

\noindent For each $\xi<\kappa$ choose $q_{\xi}$ stronger than $p$ and
$a_{\xi}\in\frak A$ such that $q_{\xi}\VVdP\,\dot x_{\xi}=\dot a_{\xi}$.
Then $q_{\xi}\VVdP\,\dot a_{\xi}\ne 0$, so $b_{\xi}=q_{\xi}\Bcap a_{\xi}$
is non-zero.   As $\sat(\frak A)=\kappa$, there must be $\xi<\eta<\kappa$
such that $b_{\xi}\Bcap b_{\eta}\ne 0$.   Set $r=q_{\xi}\Bcap q_{\eta}$;
then $r\in\frak C^+$ is stronger than $p$ and

\Centerline{$r\VVdP\,\dot x_{\xi}\dot\Bcap\dot x_{\eta}
  =\dot a_{\xi}\dot\Bcap\dot a_{\eta}
  =(a_{\xi}\Bcap a_{\eta})^{\centerdot}
  \ne 0$}

\noindent by 556Da, because $r\Bcap a_{\xi}\Bcap a_{\eta}\ne 0$.\ \Bang

\medskip

{\bf (d)} Let $A\subseteq\frak A$ be a set of size $\kappa=\tau(\frak A)$
which $\tau$-generates $\frak A$.   Let $\dot A$ be the $\Bbb P$-name
$\{(\dot a,1):a\in A\}$;  then

\Centerline{$\VVdP\,\dot A\,\,\tau$-generates $\dot\frak A$ and
$\#(\dot A)\le\check\kappa$.}

\noindent\Prf\ (i) Suppose that $p\in\frak C^+$ and $\dot x$, $\dot\frak B$
are $\Bbb P$-names such that

\Centerline{$p\VVdP\,\dot\frak B$ is an order-closed subalgebra of
$\dot\frak A$ including $\dot A$, and $\dot x\in\dot\frak A$.}

\noindent Consider
$\frak D=\{a:a\in\frak A$, $p\VVdP\,\dot a\in\dot\frak B\}$.   Then
$\frak D$ is a subalgebra of $\frak A$, by 556Bb,
and is order-closed by (b)
here;  also $A\subseteq\frak D$, so $\frak D=\frak A$.   Next, for any $q$
stronger than $p$ there are an $r$ stronger than $q$ and an $a\in\frak A$
such that $r\VVdP\,\dot x=\dot a$;  since $a\in\frak D$,
$p\VVdP\,\dot a\in\dot\frak B$ and

\Centerline{$r\VVdP\,\dot x\in\dot\frak B$.}

\noindent As $p$, $\dot x$ and $\dot\frak B$ are arbitrary,

\Centerline{$\VVdP\,\dot A\,\,\tau$-generates $\dot\frak A$.}

\noindent (ii) If $\ofamily{\xi}{\kappa}{a_{\xi}}$ enumerates $A$, then

\Centerline{$\VVdP\,\dot A=\{\dot a_{\xi}:\xi<\check\kappa\}$ and
$\#(\dot A)\le\#(\check\kappa)\le\check\kappa$.  \Qed}

\noindent Accordingly

\Centerline{$\VVdP\,\tau(\dot\frak A)\le\check\kappa$.}
}%end of proof of 556E

\leader{556F}{Quotient \dvrocolon{forcing}}\cmmnt{ In 556A-556B I have
gone to pains to describe names
$\dot\frak A,\dot\Bsymmdiff,\dot\Bcap,\dot 0,\dot 1$ constituting a Boolean
algebra.   Of course we also have much simpler names
$\check\frak A,\check\Bsymmdiff,\check\Bcap,\check 0,\check 1$
also constituting
a Boolean algebra in the forcing language, and these must obviously be
related in some way to the construction here.
I think the details are worth bringing into the open.

\medskip

\noindent}{\bf Proposition}
Let $\frak A$ be a Boolean algebra, not $\{0\}$, and
$\frak C$ a subalgebra of $\frak A$.   Let $\Bbb P$ be the forcing notion
$\frak C^+$, active downwards, and $\dot\frak A$ the
forcing name for $\frak A$ over $\frak C$.

(a) Consider the $\Bbb P$-names

\Centerline{$\dot\psi=\{((\check a,\dot a),1):a\in\frak A\}$,
\quad$\dot{\Cal I}
=\{(\check a,p):p\in\frak C^+$, $a\in\frak A$, $p\Bcap a=0\}$.}

\noindent Then

\Centerline{$\VVdP\,\dot\psi$ is a Boolean homomorphism from
$\check\frak A$ onto
$\dot\frak A$, and its kernel is $\dot{\Cal I}$.}

(b) Now suppose that $\frak C$ is regularly embedded in
$\frak A$.   Set
$\dot\Bbb Q=(\dot\frak A^+,\dot{\Bsubseteqshort},\dot 1,\check{\downarrow})$
and let $\Bbb P*\dot\Bbb Q$ be the iterated forcing notion\cmmnt{ (5A3O)}.
Then $\RO(\Bbb P*\dot\Bbb Q)$ is isomorphic to the Dedekind
completion of $\frak A$.

(c) Suppose that $\frak C$ is regularly embedded in $\frak A$ and that
$\frak B$ is a Boolean algebra such that

\Centerline{$\VVdP\,\dot\frak A\cong\check\frak B$.}

\noindent Then the Dedekind completion $\widehat{\frak A}$ of $\frak A$ is
isomorphic to the Dedekind completion $\frak C\tensorhat\frak B$ of the
free product $\frak C\otimes\frak B$ of $\frak C$ and $\frak B$.

\wheader{556F}{0}{0}{0}{36pt}

\proof{{\bf (a)(i)} It is elementary that

\Centerline{$\VVdP\,\dot\psi:\check{\frak A}\to\dot\frak A$ is a
surjective function}

\noindent just because $\dot\frak A=\{(\dot a,1):a\in\frak A\}$.
By 556Bb,

\Centerline{$\VVdP\,\dot\psi$ is a ring homomorphism;
being surjective, it is a Boolean homomorphism.}

\medskip

\quad{\bf (ii)}\grheada\ If $p\in\frak C^+$ and
$\dot x$ is a $\Bbb P$-name such that $p\VVdP\,\dot x\in\dot{\Cal I}$, then
there are a $q\in\frak C^+$, an $a\in\frak A$, and an $r$ stronger than
both $p$ and $q$ such that

\Centerline{$q\Bcap a=0$,
\quad$r\VVdP\,\dot x=\check a$.}

\noindent In this case, $r\Bcap a=0$ so, by 556Ba,

\Centerline{$r\VVdP\,0=\dot a=\dot\psi(\check a)=\dot\psi(\dot x)$.}

\noindent As $p$ and $\dot x$ are arbitrary,

\Centerline{$\VVdP\,\dot{\Cal I}$ is included in the kernel of $\dot\psi$.}

\medskip

\qquad\grheadb\ If $p\in\frak C^+$ and $\dot x$ is a $\Bbb P$-name such
that

\Centerline{$p\VVdP\,\dot x\in\check\frak A$ and $\dot\psi(\dot x)=0$,}

\noindent then there are an
$a\in\frak A$ and a $q$ stronger than $p$ such that

\Centerline{$q\VVdP\,\dot x=\check a$ and
$\dot a=\dot\psi(\check a)=\dot\psi(\dot x)=0$.}

\noindent Now there is an $r$ stronger than $q$ such that $r\Bcap a=0$, so
that $(\check a,r)\in\dot\Cal I$ and

\Centerline{$r\VVdP\,\dot x=\check a\in\dot\Cal I$.}

\noindent As $p$ and $\dot x$ are arbitrary,

\Centerline{$\VVdP$ the kernel of $\dot\psi$ is included in $\dot{\Cal I}$,
so they coincide.}

\medskip

{\bf (b)(i)} In order to use the description of iterated forcing in 5A3O,
we need to set out an exact $\Bbb P$-name
for $\dot\frak A^+$.   If we say that $\dot\frak A^+$ abbreviates
$\{x:x\in\dot\frak A$, $x\ne 0\}$, and use the formulation of
Comprehension in {\smc Kunen 80}, Theorem 4.2, we get

\Centerline{$\dot\frak A^+
=\{(\dot x,p):\dot x\in\dom\dot\frak A$, $p\in\frak C^+$,
$p\VVdP\,\dot x\in\dot\frak A$ and $\dot x\ne 0\}$.}

\noindent Now 556Ab specifies $\dom\dot\frak A$ to be
$\{\dot a:a\in\frak A\}$, so we get

$$\eqalign{\dot\frak A^+
&=\{(\dot a,p):a\in\frak A,\,p\in\frak C^+,\,
   p\VVdP\,\dot a\in\dot\frak A\text{ and }\dot a\ne 0\}\cr
&=\{(\dot a,p):a\in\frak A,\,p\in\frak C^+,\,
   p\VVdP\,\dot a\ne 0\},\cr}$$

\Centerline{$\dom\dot\frak A^+
=\{\dot a:a\in\frak A$, $\notVVdash_{\Bbb P}\,\dot a=0\}
=\{\dot a:a\in\frak A^+\}$}

\noindent by 556Da.

\medskip

\quad{\bf (ii)} 5A3O now tells us that the underlying set of
$\Bbb P*\dot\Bbb Q$ is to be

\Centerline{$P
=\{(p,\dot a):p\in\frak C^+$, $a\in\frak A^+$, $p\VVdP\,\dot a\ne 0\}$.}

\noindent For $p\in\frak C^+$ and $a\in\frak A$,

$$\eqalignno{p\VVdP\,\dot a\ne 0
&\iff\text{ for every }q\text{ stronger than }p,
  \,q\notVVdash_{\Bbb P}\,\dot a=0\cr
&\iff\text{ for every non-zero }q\Bsubseteq p,
  \,q\Bcap a\ne 0\cr}$$

\noindent (556Da).   So $P$ is just

\Centerline{$\{(p,\dot a):p\in\frak C^+$, $a\in\frak A$, $q\Bcap a\ne 0$
  whenever $q\in\frak C$ and $0\ne q\Bsubseteq p\}$.}

\noindent Next, for $(p,\dot a)$, $(q,\dot b)\in P$,

$$\eqalign{(p,\dot a)\text{ is stronger than }(q,\dot b)
&\iff p\Bsubseteq q\text{ and }p\VVdP\,\dot a\dot{\Bsubseteq}\dot b\cr
&\iff p\Bsubseteq q\text{ and }p\Bcap a\Bsubseteq b\cr}$$

\noindent (556Db).

\medskip

\quad{\bf (iii)} We can define a function $f:P\to\frak A^+$ by setting

\Centerline{$f(p,\dot a)=p\Bcap a$}

\noindent whenever $(p,\dot a)\in P$.   \Prf\ If you look at the definition
of $\dot a$ in 556A, you will see that $((b,1),1)=(\check b,1)$
belongs to $\dot a$ iff
$b\Bsubseteq a$, so that $\dot a=\dot b$ only when $a=b$;  thus $f$ is a
function from $P$ to $\frak A$.
And the definition of $P$ ensures that $f(p,\dot a)\ne 0$
whenever $(p,\dot a)\in P$.\ \Qed

\medskip

\quad{\bf (iv)}\grheada\
If $(p,\dot a)$ is stronger than $(q,\dot b)$ in $P$, then
$p\Bsubseteq q$ and $p\Bcap a\Bsubseteq b$, so
$f(p,\dot a)\Bsubseteq f(q,b)$.

\medskip

\qquad\grheadb\ If $a\in\frak A^+$, then (because $\frak C$ is regularly
embedded) $C=\{q:a\Bsubseteq q\in\frak C\}$ does not have infimum $0$ in
$\frak C$;  let $p\in\frak C^+$ be a lower bound for $C$.   Then
$(p,\dot a)\in P$, and $f(p,\dot a)\Bsubseteq a$.   Thus $f[P]$ is
order-dense in $\frak A$.

\medskip

\qquad\grheadc\ If $(p,\dot a)$, $(q,\dot b)$ are incompatible in
$P$, then $f(p,\dot a)\Bcap f(q,\dot b)=0$.   \Prf\Quer\ Otherwise,
$c=p\Bcap a\Bcap q\Bcap b$ is non-zero.   Let $r\in\frak C^+$ be such that
$(r,\dot c)\in P$.   Since $r\Bsetminus(p\Bcap q)\VVdP\,\dot c=0$,
$r\Bsubseteq p\Bcap q$;  since $c\Bsubseteq a\Bcap b$,
$(r,\dot c)$ is stronger than both $(p,\dot a)$ and
$(q,\dot b)$, which is supposed to be impossible.\ \Bang\Qed

\medskip

\quad{\bf (v)} Thinking of $\frak A^+$ as an order-dense subset of
$\widehat{\frak A}$, and of $f$ as a function from $P$ to
$\widehat{\frak A}^+$, 514Sa tells us that

\Centerline{$\RO(\Bbb P*\dot\Bbb Q)
=\RO(P)\cong\widehat{\frak A}$,}

\noindent as claimed.

\medskip

{\bf (c)} For free products of Boolean algebras, see \S315;  for Dedekind
completions, see \S314.   This part can be regarded as a corollary of (b)
(see 556Ya-556Yb), but can also be approached directly, as follows.

\medskip

\quad{\bf (i)} Let $\dot\theta$ be a $\Bbb P$-name such that

\Centerline{$\VVdP\,\dot\theta:\dot\frak A\to\check\frak B$ is an
isomorphism.}

\noindent Set

\Centerline{$R=\{(p,b,a):p\in\frak C^+$, $b\in\frak B^+$, $a\in\frak A^+$,
$a\Bsubseteq p$ and $p\VVdP\,\dot\theta(\dot a)=\check b\}$,}

\noindent and give $R$ the ordering induced by the product partial ordering
of $\frak C^+\times\frak B^+\times\frak A^+$.

\medskip

\quad{\bf (ii)} $\RO^{\downarrow}(R)\cong\frak C\tensorhat\frak B$.   \Prf\
Define $f:R\to(\frak C\tensorhat\frak B)^+$ by setting
$f(p,b,a)=p\otimes b$.

\medskip

\qquad\grheada\ Of course $f(p,b,a)\Bsubseteq f(p',b',a')$ whenever
$(p,b,a)\le(p',b',a')$.

\medskip

\qquad\grheadb\ If $(p_0,b_0,a_0)$ and $(p_1,b_1,a_1)$ belong to $R$ and
$f(p_0,b_0,a_0)\Bcap f(p_1,b_1,a_1)\ne 0$, set $p=p_0\Bcap p_1$,
$b=b_0\Bcap b_1$ and $a=a_0\Bcap a_1$.   Then $p\in\frak C^+$,
$b\in\frak B^+$, $a\Bsubseteq p$ and

\Centerline{$p\VVdP\,\dot\theta(\dot a)
=\dot\theta(\dot a_0\dot\Bcap\dot a_1)
=\dot\theta(\dot a_0)\check\Bcap\dot\theta(\dot a_1)
=\check b_0\check\Bcap\check b_1
=\check b$.}

\noindent As $p\VVdP\,\dot\theta(\dot a)\ne\check 0$, $a$ cannot be $0$,
and $(p,b,a)\in R$;  so $(p_0,b_0,a_0)$ and $(p_1,b_1,a_1)$ are downwards
compatible in $R$.

\medskip

\qquad\grheadc\ If $d\in(\frak C\tensorhat\frak B)^+$, there are
$p_0\in\frak C^+$, $b\in\frak B^+$ such that
$p_0\otimes b\Bsubseteq d$.   Now there is a $\Bbb P$-name $\dot x$ such
that

\Centerline{$\VVdP\,\dot x\in\dot\frak A$ and
$\dot\theta(\dot x)=\check b$.}

\noindent Let $p$ stronger than $p_0$ and $a_0\in\frak A$ be such that
$p\VVdP\,\dot a_0=\dot x$, and set $a=p\Bcap a_0$;  then

\Centerline{$p\VVdP\,\dot a=\dot a_0$ so
$\dot\theta(\dot a)=\dot\theta(\dot a_0)=\check b$.}

\noindent As in ($\beta$), it follows that $a\ne 0$, so that
$(p,b,a)\in R$;  now $f(p,b,a)\Bsubseteq d$.   As $d$ is arbitrary,
$f[R]$ is order-dense in $\frak C\tensorhat\frak B$.

\medskip

\qquad\grheadd\ Thus $f$ satisfies the conditions of 514Sa
and $\RO^{\downarrow}(R)\cong\frak C\tensorhat\frak B$.\ \Qed

\medskip

\quad{\bf (iii)} $\RO^{\downarrow}(R)\cong\widehat{\frak A}$.   \Prf\
Define $g:R\to\widehat{\frak A}^+$ by setting $g(p,b,a)=a$ for
$(p,b,a)\in R$.

\medskip

\qquad\grheada\ Of course $g(p,b,a)\Bsubseteq g(p',b',a')$ whenever
$(p,b,a)\le(p',b',a')$ in $R$.

\medskip

\qquad\grheadb\ Suppose that $(p_0,b_0,a_0)$, $(p_1,b_1,a_1)\in R$ and that
$a=a_0\Bcap a_1\ne 0$.   Set $p=p_0\Bcap p_1$ and $b=b_0\Bcap b_1$.   Then
$p\Bsupseteq a\ne 0$ and

\Centerline{$p\VVdP\,\dot\theta(\dot a)=\check b$}

\noindent as in (ii-$\beta$).   Since $p\Bcap a\ne 0$,
$p\notVVdash_{\Bbb P}\,\dot a=\dot 0$ (556Da), so
$p\notVVdash_{\Bbb P}\,\check b=\check 0$ and $b\ne 0$.   Thus
$(p,b,a)\in R$ and $(p_0,b_0,a_0)$, $(p_1,b_1,a_1)$ are compatible
downwards in $R$.

\medskip

\qquad\grheadc\ If $d\in\widehat{\frak A}^+$, there is an $a_0\in\frak A^+$
such that $a_0\Bsubseteq d$.   In this case,
$\notVVdash_{\Bbb P}\,\dot a_0=\dot 0$ so
there is a $p_0\in\frak C^+$
such that $p_0\VVdP\,\dot a_0\ne\dot 0$.   Now
$p_0\VVdP\,\dot\theta(\dot a_0)\in\check\frak B$ so there are a $p$
stronger than $p_0$ and a $b\in\frak B$ such that
$p\VVdP\,\dot\theta(\dot a_0)=\check b$.   Set $a=p\Bcap a_0$;  then

\Centerline{$p\VVdP\,\dot a=\dot a_0\ne\dot 0$ and
$\check b=\dot\theta(\dot a)\ne\check 0$.}

\noindent Consequently $a$ and $b$ are both non-zero and $(p,b,a)\in R$,
while $g(p,b,a)\Bsubseteq d$.

\medskip

\qquad\grheadd\ Thus $g$ satisfies the conditions of 514Sa
and $\RO^{\downarrow}(R)\cong\widehat{\frak A}$.\ \Qed

\medskip

\quad{\bf (iv)} Putting these together, $\frak C\tensorhat\frak B$ and
$\widehat{\frak A}$ are isomorphic.
}%end of proof of 556F

\leader{556G}{Proposition} Let $\frak A$ be a
Dedekind complete Boolean algebra, not $\{0\}$,
$\frak C$ a regularly embedded
subalgebra of $\frak A$, $\Bbb P$ the forcing notion
$\frak C^+$, active downwards,
and $\dot\frak A$ the forcing name for $\frak A$ over $\frak C$.

(a) Whenever $p\in\frak C^+$ and $\dot x$ is a $\Bbb P$-name such that

\Centerline{$p\VVdP\,\dot x\in\dot\frak A$,}

\noindent there is an $a\in\frak A$ such that

\Centerline{$p\VVdP\,\dot x=\dot a$,}

\noindent where $\dot a$ is the forcing name for $a$ over $\frak C$.

(b) $\VVdP\,\dot\frak A$ is Dedekind complete.

\proof{{\bf (a)} Set

\Centerline{$B=\{q\Bcap b:q\in\frak C^+$ is stronger than $p$, $b\in\frak A$,
$q\VVdP\,\dot b=\dot x\}$,
\quad$a=\sup B$.}

\noindent Then 556Ea tells us that

\Centerline{$p\VVdP\,\dot a=\sup\{\dot x\}=\dot x$.}

\medskip

{\bf (b)} Suppose that $p\in\frak C^+$ and that $\dot A$ is a
$\Bbb P$-name such that $p\VVdP\,\dot A\subseteq\dot\frak A$.   Set

\Centerline{$B=\{q\Bcap a:a\in\frak A$, $q\in\frak C^+$ and
$q\VVdP\,\dot a\in\dot A\}$,
\quad$d=\sup B$.}

\noindent Then $p\Bcap d=\sup_{b\in B}p\Bcap b$, so
$p\VVdP\,\dot d=\sup\dot A$,
by 556Ea.   As $p$ and $\dot A$ are arbitrary,

\Centerline{$\VVdP\,\dot\frak A$ is Dedekind complete.}
}%end of proof of 556G

\leader{556H}{$L^0(\frak A)$:  Proposition} Let $\frak A$ be a
Dedekind complete Boolean algebra, not $\{0\}$,
$\frak C$ a regularly embedded
subalgebra of $\frak A$, $\Bbb P$ the forcing notion
$\frak C^+$, active downwards,
and $\dot\frak A$ the forcing name for $\frak A$ over
$\frak C$.   For $a\in\frak A$ let $\dot a$ be the forcing name for $a$
over $\frak C$.

(a)(i) For every $u\in L^0(\frak A)$,

\Centerline{$\VVdP\,\dot u\in L^0(\dot\frak A)$}

\noindent where $\dot u$ is the forcing name for $u$ over $\frak C$.

\quad(ii) If $u$, $v\in L^0(\frak A)$ and $\VVdP\,\dot u=\dot v$, then
$u=v$.

(b) For $u$, $v\in L^0(\frak A)$ and $\alpha\in\Bbb R$,

$$\eqalign{\VVdP\,\dot u+\dot v&=(u+v)^{\centerdot},\cr
-\dot u&=(-u)^{\centerdot}, \cr
\dot u\vee\dot v&=(u\vee v)^{\centerdot},\cr
\dot u\times\dot v&=(u\times v)^{\centerdot},\cr
\check\alpha\dot u&=(\alpha u)^{\centerdot}.\cr}$$

\noindent If $u\le v$, then $\VVdP\,\dot u\le\dot v$.

(c) If $\familyiI{u_i}$ is a family in $L^0(\frak A)$
with supremum $u\in L^0(\frak A)$, then

\Centerline{$\VVdP\,\dot u=\sup_{i\in\check I}\dot u_i$
in $L^0(\dot\frak A)$.}

(d) If $p\in\frak C^+$ and $\dot w$ is a
$\Bbb P$-name such that $p\VVdP\,\dot w\in L^0(\dot\frak A)$, then there is
a $u\in L^0(\frak A)$ such that

\Centerline{$p\VVdP\,\dot w=\dot u$.}

(e) If $\sequencen{u_n}$ is a sequence in $L^0(\frak A)$, then the
following are equiveridical:

\quad(i) $\sequencen{u_n}$ is order*-convergent to
$0$\cmmnt{ (definition: 367A)},

\quad(ii) $\VVdP\,\sequencen{\dot u_n}$ is order*-convergent to $0$.

\proof{{\bf (a)(i)} Examining the definition in 556Af, we see that we have

\Centerline{$\VVdP\,\dot u$ is a function from $\Bbb Q$ to $\dot\frak A$
and $\dot u(\check\alpha)=\Bvalue{u>\alpha}^{\centerdot}$}

\noindent for every $\alpha\in\Bbb Q$.   Now 556Eb tells us that,
for every $\alpha\in\Bbb Q$,

$$\eqalignno{\VVdP\,\dot u(\check\alpha)
&=\Bvalue{u>\alpha}^{\centerdot}
=(\sup_{\beta\in\Bbb Q,\beta>\alpha}\Bvalue{u>\beta})^{\centerdot}
=\sup_{\beta\in\Bbb Q,\beta>\check\alpha}\Bvalue{u>\beta}^{\centerdot}
=\sup_{\beta\in\Bbb Q,\beta>\check\alpha}\dot u(\check\beta),\cr
0
&=(\inf_{n\in\Bbb N}\Bvalue{u>n})^{\centerdot}
=\inf_{n\in\Bbb N}\Bvalue{u>n}^{\centerdot}
=\inf_{n\in\Bbb N}\dot u(\check n),\cr
1
&=(\sup_{n\in\Bbb N}\Bvalue{u>-n})^{\centerdot}
=\sup_{n\in\Bbb N}\Bvalue{u>-n}^{\centerdot}
=\sup_{n\in\Bbb N}\dot u(-\check n),\cr}$$

\noindent so

\Centerline{$\VVdP\,\dot u\in L^0(\dot\frak A)$,}

\noindent and I can write $\Bvalue{\dot u>\check\alpha}$ for the
$\Bbb P$-name $\dot u(\check\alpha)$, so that

\Centerline{$\VVdP\,\Bvalue{\dot u>\check\alpha}
=\Bvalue{u>\alpha}^{\centerdot}$}

\noindent for every $\alpha\in\Bbb Q$.

\medskip

\quad{\bf (ii)} For any $\alpha\in\Bbb Q$,

\Centerline{$\VVdP\,\Bvalue{u>\alpha}^{\centerdot}
=\Bvalue{\dot u>\check\alpha}
=\Bvalue{\dot v>\check\alpha}
=\Bvalue{u>\alpha}^{\centerdot}$.}

\noindent By 556Da, $\Bvalue{u>\alpha}=\Bvalue{v>\alpha}$.   As $\alpha$ is
arbitrary, $u=v$.

\medskip

{\bf (b)(i)} Suppose $u$, $v\in L^0(\frak A)$.   By 364D,
we have

\Centerline{$\Bvalue{u+v>\alpha}
=\sup_{\beta\in\Bbb Q}\Bvalue{u>\beta}\Bcap\Bvalue{v>\alpha-\beta}$}

\noindent for every $\alpha\in\Bbb Q$.   If $\alpha$, $\beta\in\Bbb Q$,

\Centerline{$\VVdP\,\Bvalue{\dot u>\check\beta}
   \dot\Bcap\Bvalue{\dot v>\check\alpha-\check\beta}
=\Bvalue{u>\beta}^{\centerdot}\dot\Bcap\Bvalue{v>\alpha-\beta}^{\centerdot}
=(\Bvalue{u>\beta}\Bcap\Bvalue{v>\alpha-\beta})^{\centerdot}$.}

\noindent Taking the supremum over $\beta$, as in 556E(b-i),

$$\eqalign{\VVdP\,\Bvalue{\dot u+\dot v>\check\alpha}
&=\sup_{\beta\in\Bbb Q}\Bvalue{\dot u>\beta}
  \dot\Bcap\Bvalue{\dot v>\check\alpha-\beta}
=\sup_{\beta\in\check\Bbb Q}\Bvalue{\dot u>\beta}
  \dot\Bcap\Bvalue{\dot v>\check\alpha-\beta}\cr
&=(\sup_{\beta\in\Bbb Q}\Bvalue{u>\beta}
   \Bcap\Bvalue{v>\alpha-\beta})^{\centerdot}
=\Bvalue{u+v>\alpha}^{\centerdot}
=\Bvalue{(u+v)^{\centerdot}>\check\alpha}\cr}$$

\noindent for every $\alpha\in\Bbb Q$, and

\Centerline{$\VVdP\,\dot u+\dot v=(u+v)^{\centerdot}$.}

\medskip

\quad{\bf (ii)} Concerning $u\vee v$, we have

$$\eqalign{\VVdP\,\Bvalue{(u\vee v)^{\centerdot}>\check\alpha}
&=\Bvalue{u\vee v>\alpha}^{\centerdot}
=\Bvalue{u>\alpha}^{\centerdot}
  \dot\Bcup\Bvalue{v>\alpha}^{\centerdot}\cr
&=\Bvalue{\dot u>\check\alpha}\dot\Bcup\Bvalue{\dot v>\check\alpha}
=\Bvalue{\dot u\vee\dot v>\check\alpha}\cr}$$

\noindent for every $u$, $v\in L^0(\frak A)$ and $\alpha\in\Bbb Q$, so

\Centerline{$\VVdP\,\dot u\vee\dot v=(u\vee v)^{\centerdot}$;}

\noindent it follows that if $u\ge 0$ then
$\VVdP\,\dot u=\dot u\vee 0\ge 0$.

\medskip

\quad{\bf (iii)}
If $u$, $v\in L^0(\frak A)^+$, $\alpha\in\Bbb Q$
and $\alpha\ge 0$, then, just as in (i),

$$\eqalign{\VVdP\,\Bvalue{\dot u\times\dot v>\check\alpha}
&=\sup_{\beta\in\Bbb Q,\beta>0}\Bvalue{\dot u>\beta}
  \dot\Bcap\Bvalue{\dot v>\Bover{\check\alpha}{\beta}}
=(\sup_{\beta\in\Bbb Q,\beta>0}\Bvalue{u>\beta}
   \Bcap\Bvalue{v>\Bover{\alpha}{\beta}})^{\centerdot}\cr
&=\Bvalue{u\times v>\alpha}^{\centerdot}
=\Bvalue{(u\times v)^{\centerdot}>\check\alpha};\cr}$$

\noindent so $\VVdP\,\dot u\times\dot v=(u\times v)^{\centerdot}$.
Using the
distributive law we see that the same is true for all $u$,
$v\in L^0(\frak A)$.

\wheader{556H}{6}{2}{2}{36pt}

\quad{\bf (iv)} Take $\alpha\in\Bbb R$ and set
$w=\alpha\chi 1\in L^0(\frak A)$.   If $\beta\in\Bbb Q$ and $\beta<\alpha$,
then

\Centerline{$\VVdP\,\Bvalue{\check\alpha\chi 1>\check\beta}
=1=\dot 1=\Bvalue{w>\beta}^{\centerdot}
=\Bvalue{\dot w>\check\beta}$;}

\noindent while if $\beta\ge\alpha$,

\Centerline{$\VVdP\,\Bvalue{\check\alpha\chi 1>\check\beta}
=0=\dot 0=\Bvalue{w>\beta}^{\centerdot}
=\Bvalue{\dot w>\check\beta}$.}

\noindent So

\Centerline{$\VVdP\,\Bvalue{\check\alpha\chi 1>\beta}
=\Bvalue{\dot w>\beta}\text{ for every }
  \beta\in\Bbb Q,\text{ and }
\check\alpha\chi 1=\dot w=(\alpha\chi 1)^{\centerdot}$.}

\noindent Putting this together with (iii), we have

\Centerline{$\VVdP\,\check\alpha\dot u
=(\check\alpha\chi 1)\times\dot u
=(\alpha\chi 1)^{\centerdot}\times\dot u
=(\alpha\chi 1\times u)^{\centerdot}
=(\alpha u)^{\centerdot}$}

\noindent for every $u\in L^0(\frak A)$.   In particular, taking
$\alpha=-1$, $\VVdP\,-\dot u=(-u)^{\centerdot}$.

\medskip

\quad{\bf (v)} Finally, if $u\le v$ then $u\vee v=v$, so

\Centerline{$\VVdP\,\dot u\vee\dot v=\dot v$ and $\dot u\le\dot v$.}

\medskip

{\bf (c)(i)} It will help to note that the criterion in
364L(a-ii)

\inset{if $A\subseteq L^0(\frak A)$ is
non-empty, then $v\in L^0(\frak A)$ is the supremum of $A$ in
$L^0(\frak A)$ iff $\Bvalue{v>\alpha}=\sup_{u\in A}\Bvalue{u>\alpha}$ in
$\frak A$ for every $\alpha\in\Bbb R$}

\noindent can be replaced by

\inset{if $A\subseteq L^0(\frak A)$ is
non-empty, then $v\in L^0(\frak A)$ is the supremum of $A$ in
$L^0(\frak A)$ iff $\Bvalue{v>\alpha}=\sup_{u\in A}\Bvalue{u>\alpha}$ in
$\frak A$ for every $\alpha\in\Bbb Q$.}

\noindent\Prf\ If the weaker condition is satisfied, and $\alpha\in\Bbb R$,
then

$$\eqalign{\Bvalue{v>\alpha}
&=\sup_{\beta\in\Bbb Q,\beta\ge\alpha}\Bvalue{v>\beta}
=\sup_{\beta\in\Bbb Q,\beta\ge\alpha}\sup_{u\in A}\Bvalue{u>\beta}\cr
&=\sup_{u\in A}\sup_{\beta\in\Bbb Q,\beta\ge\alpha}\Bvalue{u>\beta}
=\sup_{u\in A}\Bvalue{u>\alpha}.  \text{ \Qed}\cr}$$

\medskip

\quad{\bf (ii)} Now 556E(b-i) tells us that

\Centerline{$\VVdP\,\Bvalue{\dot u>\check\alpha}
=\sup_{i\in\check I}\Bvalue{\dot u_i>\check\alpha}$}

\noindent for every $\alpha\in\Bbb Q$, so

\Centerline{$\VVdP\,\dot u=\sup_{i\in\check I}\dot u_i$.}

\medskip

{\bf (d)} For each $\alpha\in\Bbb Q$ we have an $a_{\alpha}\in\frak A$ such
that $p\VVdP\,\dot a_{\alpha}=\Bvalue{\dot w>\check\alpha}$ (556Ga);  since
$p\VVdP\,\dot a=(a\Bcap p)^{\centerdot}$ for every $a\in\frak A$, we can
suppose that $a_{\alpha}\Bsubseteq p$ for every $\alpha$.   Now we find
that if $\alpha\in\Bbb Q$ and
$b_{\alpha}=\sup_{\beta\in\Bbb Q,\beta>\alpha}a_{\alpha}$, then

\Centerline{$p\VVdP\,\dot b_{\alpha}
=\sup_{\beta\in\Bbb Q,\beta>\check\alpha}\Bvalue{\dot w>\beta}
=\dot a_{\alpha}$,}

\noindent so $a_{\alpha}=b_{\alpha}$.   Similarly, if
$b=\inf_{n\in\Bbb N}a_n$ and $c=\sup_{n\in\Bbb N}a_{-n}$,

$$\eqalign{p\VVdP\,\dot b
&=\inf_{n\in\Bbb N}\Bvalue{\dot w>n}=0,\cr
\dot c&=\sup_{n\in\Bbb N}\Bvalue{\dot w>-n}=1\cr}$$

\noindent and $b=0$, $c=p$.   It is now easy to check that
there is a $u\in L^0(\frak A)$ such that

$$\eqalign{\Bvalue{u>\alpha}
&=a_{\alpha}\text{ if }\alpha\in\Bbb Q
  \text{ and }\alpha>0,\cr
&=a_{\alpha}\Bcup(1\Bsetminus p)
  \text{ for other }\alpha\in\Bbb Q,\cr}$$

\noindent and that $p\VVdP\,\dot u=\dot w$.

\medskip

{\bf (e)} Recall that
$\sequencen{u_n}$ order*-converges to $0$ iff $\sequencen{u_n}$ is
order-bounded and
$0=\inf_{n\in\Bbb N}\sup_{m\ge n}|u_n|$ (367G);  and we shall have a
similar formulation in the forcing language.   So if $\sequencen{u_n}$
order*-converges to $0$, then

\Centerline{$\VVdP\,\sup_{m\ge\check n}|\dot u_m|
=(\sup_{m\ge n}|u_m|)^{\centerdot}$}

\noindent for every $n\in\Bbb N$, and

\Centerline{$\VVdP\,\inf_{n\in\Bbb N}\sup_{m\ge n}|\dot u_m|
=(\inf_{n\in\Bbb N}\sup_{m\ge n}|u_m|)^{\centerdot}=0$, so
$\sequencen{\dot u_n}\to^*0$.}

Conversely, if $\VVdP\,\sequencen{\dot u_n}$ order*-converges to
$0$, then $\VVdP\,\sequencen{\dot u_n}$ is order-bounded, and there is a
$\Bbb P$-name $\dot w$ such that

\Centerline{$\VVdP\,\dot w\in L^0(\dot\frak A)$,
$|\dot u_n|\le\dot w$ for every $n\in\Bbb N$.}

\noindent By (d), there is a $v\in L^0(\frak A)$ such that
$\VVdP\,\dot w=\dot v$, so that

\Centerline{$\VVdP\,(v\vee|u_n|)^{\centerdot}=\dot w\vee|\dot u_n|=\dot v$
for every $n\in\Bbb N$}

\noindent and $|u_n|\le v$ for every $n$ (use (a-ii)).
We can therefore repeat the calculation just above to see that

\Centerline{$\VVdP\,(\inf_{n\in\Bbb N}\sup_{m\ge n}|u_m|)^{\centerdot}
=\inf_{n\in\Bbb N}\sup_{m\ge n}|\dot u_m|=0$,}

\noindent so that $\inf_{n\in\Bbb N}\sup_{m\ge n}|u_m|=0$ and
$\sequencen{u_n}$ order*-converges to $0$.
}%end of proof of 556H

\leader{556I}{Proposition} Let $\frak A$ be a Boolean algebra,
not $\{0\}$, and $\frak C$ a regularly embedded
subalgebra of $\frak A$.   Let $\Bbb P$ be the forcing notion
$\frak C^+$, active downwards, and
$\pi:\frak A\to\frak A$ a Boolean homomorphism fixing every point of
$\frak C$;  let $\dot\pi$ be the forcing name for $\pi$ over $\frak C$.

(a) $\pi$ is injective iff $\VVdP\,\dot\pi$ is injective.

(b) If $\pi$ is order-continuous, then

\Centerline{$\VVdP\,\dot\pi$ is order-continuous.}

(c) If $\pi$ has a support $\supp\pi$\cmmnt{ (definition:  381Bb)}, then

\Centerline{$\VVdP\,(\supp\pi)^{\centerdot}$ is the support of $\dot\pi$.}

\proof{{\bf (a)} We saw in 556C(a-ii) that if $\pi$ is injective then
$\VVdP\,\dot\pi$ is injective.   Now suppose
that $\pi$ is not injective;  let $a\in\frak A^+$ be such that $\pi a=0$.
Then $\VVdP\,\dot\pi\dot a=0$.
$1\Bcap a\ne 0$, so $\notVVdash_{\Bbb P}\,\dot a=0$, by 556Da, and

\Centerline{$\notVVdash_{\Bbb P}\,\dot\pi$ is injective.}

\medskip

{\bf (b)} Take $p\in\frak C^+$ and a $\Bbb P$-name $\dot A$ such that

\Centerline{$p\VVdP\,\dot A\subseteq\dot\frak A$ and $\sup\dot A=1$.}

\noindent Set
$B=\{q\Bcap a:q\in\frak C^+$, $a\in\frak A$, $q\VVdP\,\dot a\in\dot A\}$.
Then $\sup_{b\in B}p\Bcap b=p\Bcap 1=p$, by 556Ea.
Because $\pi$ is order-continuous,

\Centerline{$p\Bcap 1=p
=\pi p=\sup_{b\in B}\pi(p\Bcap b)=\sup_{b\in B}p\Bcap\pi b$.}

\noindent Consider

\Centerline{$C=\{q\Bcap a:q\in\frak C^+$, $a\in\frak A$,
$q\VVdP\,\dot a\in\dot\pi[\dot A]\}$.}

\noindent Then $\pi[B]\subseteq C$.   \Prf\ If $q\in\frak C^+$, $a\in\frak A$
and $q\VVdP\,\dot a\in\dot A$, then

\Centerline{$q\VVdP\,(\pi a)^{\centerdot}
=\dot\pi\dot a\in\dot\pi[\dot A]$}

\noindent so

\Centerline{$\pi(q\Bcap a)=q\Bcap\pi a\in C$.  \Qed}

\noindent Accordingly

\Centerline{$\{p\Bcap c:c\in C\}\supseteq\{p\Bcap\pi b:b\in B\}$}

\noindent must have supremum $p$, and $p\VVdP\,\sup\dot\pi[\dot A]=1$.

As $p$ and $\dot A$ are arbitrary,

\Centerline{$\VVdP\,\sup\dot\pi[\dot A]=1$ whenever
$\dot A\subseteq\frak A$ and $\sup\dot A=1$,
so $\dot\pi$ is order-continuous}

\noindent (313L(b-iii)).

\medskip

{\bf (c)(i)} $\VVdP$ if $x\in\dot\frak A$ and
$x\dot\Bcap(\supp\pi)^{\centerdot}=0$ then $\dot\pi x=x$.
\Prf\ Take $p\in\frak C^+$ and a $\Bbb P$-name $\dot x$ such that

\Centerline{$p\VVdP\,\dot x\in\dot\frak A$
and $\dot x\dot\Bcap(\supp\pi)^{\centerdot}=0$.}

\noindent For any $q$ stronger than $p$ there are an $r$ stronger than $q$
and an $a\in\frak A$ such that

\Centerline{$r\VVdP\,\dot x=\dot a$, $(a\Bcap\supp\pi)^{\centerdot}=0$;}

\noindent now $r\Bcap a\Bcap\supp\pi=0$ (556Da).   In this case,

\Centerline{$r\VVdP\,\dot\pi\dot x=\dot\pi(r\Bcap a)^{\centerdot}
=(\pi(r\Bcap a))^{\centerdot}=(r\Bcap a)^{\centerdot}=\dot x$.}

\noindent As $q$ is arbitrary, $p\VVdP\,\dot\pi\dot x=\dot x$;  as $p$ and
$\dot x$ are arbitrary, we have the result.\ \Qed

Now 381Ei, applied in the forcing language, tells us that

\Centerline{$\VVdP\,(\supp\pi)^{\centerdot}$ supports $\dot\pi$.}

\medskip

\quad{\bf (ii)} $\VVdP$ if $x\in\dot\frak A$ supports $\dot\pi$, then
$x\dot{\Bsupseteq}(\supp\pi)^{\centerdot}$.
\Prf\ Take $p\in\frak C^+$ and a $\Bbb P$-name $\dot x$ such that

\Centerline{$p\VVdP\,\dot x\in\dot\frak A$ supports $\dot\pi$.}

\noindent Then for any $q$ stronger than $p$ we have an $r$ stronger than
$q$ and an $a\in\frak A$ such that $r\VVdP\,\dot x=\dot a$.
Set $b=a\Bcup(1\Bsetminus r)$;  then
$r\VVdP\,\dot b=\dot a$ supports $\dot\pi$.   \Quer\ If $b$ does not
support $\pi$, then there is a non-zero $d\Bsubseteq 1\Bsetminus b$ such
that $d\Bcap\pi d=0$ (381Ei again).   Since $r\Bcap d=d\ne 0$, there is an
$s$ stronger than $r$ such that $s\VVdP\,\dot d\ne 0$.   Now

\Centerline{$s\VVdP\,\dot d\dot\Bcap\dot\pi\dot d
=(d\Bcap\pi d)^{\centerdot}=0$, while $\dot d\dot\Bcap\dot a=0$ and
$\dot a$ supports $\dot\pi$,}

\noindent which is impossible.\ \Bang

So $b\Bsupseteq\supp\pi$ and

\Centerline{$r\VVdP\,\dot x
=\dot b\dot{\Bsupseteq}(\supp\pi)^{\centerdot}$.}

\noindent As $q$ is arbitrary,

\Centerline{$p\VVdP\,\dot x\dot{\Bsupseteq}(\supp\pi)^{\centerdot}$;}

\noindent as $p$ and $\dot x$ are arbitrary, we have the result.\ \Qed

Putting this together with (i),

\Centerline{$\VVdP\,(\supp\pi)^{\centerdot}$ is the least element of
$\dot\frak A$
supporting $\dot\pi$, and is the support of $\dot\pi$.}
}%end of proof of 556I

\leader{556J}{Theorem} Let $\frak A$ be a Dedekind complete
Boolean algebra, not $\{0\}$, and $\frak C$ a regularly embedded
subalgebra of $\frak A$.   Let $\Bbb P$ be the forcing notion
$\frak C^+$, active downwards, and $\dot\frak A$ the
forcing name for $\frak A$ over $\frak C$.

(a) If $\dot\theta$ is a $\Bbb P$-name such that

\Centerline{$\VVdP\,\dot\theta$ is a ring homomorphism from
$\dot\frak A$ to itself,}

\noindent then there is a unique
ring homomorphism $\pi:\frak A\to\frak A$ such that $\pi c\Bsubseteq c$ for
every $c\in\frak C$ and

\Centerline{$\VVdP\,\dot\theta=\dot\pi$,}

\noindent where $\dot\pi$ is the forcing name for $\pi$ over $\frak C$.

(b)(i) If

\Centerline{$\VVdP\,\dot\theta$ is a Boolean homomorphism,}

\noindent then $\pi$ is a Boolean homomorphism, and $\pi c=c$ for every
$c\in\frak C$.

\quad(ii) If

\Centerline{$\VVdP\,\dot\theta$ is a Boolean automorphism,}

\noindent that $\pi$ is a Boolean automorphism.

\wheader{556J}{0}{0}{0}{48pt}

\proof{{\bf (a)(i)} For each $a\in\frak A$, 556Ga tells us that there is
a $b\in\frak A$ such that

\Centerline{$\VVdP\,\dot\theta(\dot a)=\dot b$;}

\noindent by 556Da, this defines $b$ uniquely, so we have a
unique function $\pi:\frak A\to\frak A$ defined by the rule

\Centerline{for every $a\in\frak A$, $\VVdP\,\dot\theta(\dot a)
=(\pi a)^{\centerdot}$.}

\medskip

\quad{\bf (ii)} Now, for $\frmedcirc=\Bsymmdiff$ or $\frmedcirc=\Bcap$,
and $a$, $b\in\frak A$,

$$\eqalignno{\VVdP\,(\pi(a\frmedcirc b))^{\centerdot}
&=\dot\theta((a\frmedcirc b)^{\centerdot})
=\dot\theta(\dot a\,\dot\frmedcirc\,\dot b)\cr
&=\dot\theta\dot a\,\dot\frmedcirc\,\dot\theta\dot b
=(\pi a)^{\centerdot}\,\dot\frmedcirc\,(\pi b)^{\centerdot}
=(\pi a\frmedcirc\pi b)^{\centerdot}\cr}$$

\noindent and $\pi(a\frmedcirc b)=\pi a\frmedcirc\pi b$.
So $\pi$ is a ring homomorphism.

\medskip

\quad{\bf (iii)} If $c\in\frak C$ then $\pi c\Bsubseteq c$.
\Prf\ If $c=1$ this is trivial.   Otherwise,

\Centerline{$1\Bsetminus c\VVdP\,\dot c=0$,
$(\pi c)^{\centerdot}=\dot\theta 0=0$,}

\noindent so $(1\Bsetminus c)\Bcap\pi c=0$ and $\pi c\Bsubseteq c$.\ \Qed

\medskip

\quad{\bf (iv)} We can therefore speak of the forcing name $\dot\pi$
(556Ae, 556C).
If $p\in\frak C^+$ and $\dot x$ is a $\Bbb P$-name such that
$p\VVdP\,\dot x\in\dot\frak A$, let $a\in\frak A$ be such that
$p\VVdP\,\dot x=\dot a$ (556Ga again);  then

\Centerline{$p\VVdP\,\dot\theta(\dot x)=\dot\theta(\dot a)
=(\pi a)^{\centerdot}=\dot\pi(\dot a)=\dot\pi(\dot x)$.}

\noindent As $p$ and $\dot x$ are arbitrary,

\Centerline{$\VVdP\,\dot\theta=\dot\pi$.}

\medskip

{\bf (b)(i)} If $\VVdP\,\dot\theta$ is a Boolean homomorphism, then

\Centerline{$\VVdP\,(\pi 1)^{\centerdot}=\dot\theta\dot 1=\dot 1$}

\noindent and $\pi 1=1$.   Now

\Centerline{$\pi c\Bsubseteq c=1\Bsetminus(1\Bsetminus c)
\Bsubseteq 1\Bsetminus\pi(1\Bsetminus c)
=\pi 1\Bsetminus(\pi 1\Bsetminus\pi c)
=\pi c$}

\noindent so $\pi c=c$ for every $c\in\frak C$.

\medskip

\quad{\bf (ii)} If $\VVdP\,\dot\theta$ is a Boolean automorphism, then the same
arguments tell us that there is a Boolean homomorphism
$\phi:\frak A\to\frak A$
such that $\phi c=c$ for every $c\in\frak C$ and
$\VVdP\,\dot\phi=\dot\theta^{-1}$.   But in this case

\Centerline{$\VVdP\,(\pi\phi)^{\centerdot}=\dot\pi\dot\phi
=\dot\theta\dot\theta^{-1}=\dot\iota$}

\noindent where $\iota$ is the identity automorphism on $\frak A$;  by the
uniqueness of the representing homomorphisms of $\frak A$, $\pi\phi=\iota$.
Similarly, $\phi\pi=\iota$ and $\phi=\pi^{-1}$, so that $\pi$ is an
automorphism.
}%end of proof of 556J

\leader{556K}{Theorem} Let $(\frak A,\bar\mu)$ be a probability algebra,
and $\frak C$ a closed subalgebra of $\frak A$;  let $\Bbb P$ be the
forcing notion $\frak C^+$,
active downwards, and $\dot\frak A$ the forcing name for $\frak A$
over $\frak C$.   We can identify $\frak C$ with the regular open algebra
$\RO(\Bbb P)$\cmmnt{ (514Sb)}.
For $u\in L^0(\frak C)$ write $\vec u$ for the
corresponding $\Bbb P$-name for a real number as described in 5A3L.

(a)(i) For each $a\in\frak A$ there is a
$u_a\in L^1(\frak C,\bar\mu\restrp\frak C)$ defined by saying that
$\int_cu_a=\bar\mu(a\Bcap c)$ for every $c\in\frak C$.

\quad(ii) If $p\in\frak C^+$ and $a$, $b\in\frak A$ are such that

\Centerline{$p\VVdP\,\dot a=\dot b$}

\noindent(where $\dot a$, $\dot b$ are the forcing names for $a$, $b$ over
$\frak C$), then

\Centerline{$p\VVdP\,\vec u_a=\vec u_b$.}

\woddheader{556K}{0}{0}{0}{30pt}

(b) There is a $\Bbb P$-name $\dot{\bar\mu}$ such that

\Centerline{$\VVdP\,(\dot\frak A,\dot{\bar\mu})$ is a probability algebra,}

\noindent and

\Centerline{$\VVdP\,\dot{\bar\mu}\dot a=\vec u_a$}

\noindent whenever $a\in\frak A$ and $\dot a$ is the corresponding
forcing name over $\frak C$.

(c) If $\pi:\frak A\to\frak A$ is a measure-preserving Boolean
homomorphism such that $\pi c=c$ for every $c\in\frak C$, and
$\dot\pi$ the corresponding forcing name over $\frak C$,
then

\Centerline{$\VVdP\,\dot\pi:\dot\frak A\to\dot\frak A$ is
measure-preserving.}

(d) If $\dot\phi$ is a $\Bbb P$-name such that

\Centerline{$\VVdP\,\dot\phi:\dot\frak A\to\dot\frak A$ is a
measure-preserving Boolean automorphism}

\noindent then there is a measure-preserving Boolean automorphism
$\pi:\frak A\to\frak A$ such that $\pi c=c$ for every $c\in\frak C$ and

\Centerline{$\VVdP\,\dot\phi=\dot\pi$.}

(e) If $v\in L^1(\frak A,\bar\mu)$ and
$u\in L^1(\frak C,\bar\mu\restrp\frak C)$ is its conditional expectation
on $\frak C$, then

\Centerline{$\VVdP\,\dot v\in L^1(\dot\frak A,\dot{\bar\mu})$ and
$\int\dot v\,d\dot{\bar\mu}=\vec u$.}

\proof{{\bf (a)(i)} This is just the Radon-Nikod\'ym theorem (365E).

\medskip

\quad{\bf (ii)} If $p\VVdP\,\dot a=\dot b$, then $p\Bcap a=p\Bcap b$
(556Da).   Consequently

\Centerline{$\int_cu_a\times\chi p
=\int_{c\Bcap p}u_a=\bar\mu(c\Bcap p\Bcap a)
=\bar\mu(c\Bcap p\Bcap b)=\int_cu_b\times\chi p$}

\noindent whenever $c\in\frak C$, and
$u_a\times\chi p=u_b\times\chi p$;  by 5A3M,

\Centerline{$p\VVdP\,\vec u_a=\vec u_b$.}

\medskip

{\bf (b)(i)} Note first the elementary properties of the conditional
expectation
$a\mapsto u_a:\frak A\to L^1(\frak C,\bar\mu\restrp\frak C)$:
it is additive and positive and
order-continuous, and $0\le u_a\le\chi 1$ for every $a$.   (To extract
these facts efficiently from the presentation in \S365, note that
$u_a=P(\chi a)$, where
$P:L^1(\frak A,\bar\mu)\to L^1(\frak C,\bar\mu\restrp\frak C)$ is the
conditional expectation operator of 365Q\formerly{3{}65R}.)   
In particular,

\Centerline{$\VVdP\,\vec u_a\in[0,1]$}

\noindent for every $a\in\frak A$.   It is also worth observing that
if $c\in\frak C$ and $a\in\frak A$ then $u_{a\Bcap c}=u_a\times\chi c$
(see 365Oc\formerly{3{}65Pc}).

\medskip

\quad{\bf (ii)} Now consider the $\Bbb P$-name

\Centerline{$\dot{\bar\mu}=\{((\dot a,\vec u_a),1):a\in\frak A\}$.}

\noindent We have quite a lot to check, of course.   First,
$\dot{\bar\mu}$ is a name for a function with domain $\dot\frak A$.
\Prf\ If $((\dot a,\vec u_a),1)$ and $((\dot b,\vec u_b),1)$ are two
members of $\dot{\bar\mu}$, and $p\in\frak C^+$ is such that
$p\VVdP\,\dot a=\dot b$, then $p\Bcap a=p\Bcap b$, so
$p\VVdP\,\vec u_a=\vec u_b$, by (a-ii) above.
By 5A3H, $\VVdP\,\dot{\bar\mu}$ is a function.   Also
$\VVdP\,\dom\dot{\bar\mu}=\dot A$, where
$\dot A=\{(\dot a,1):a\in\frak A\}=\dot\frak A$.\ \Qed

\medskip

\quad{\bf (iii)} We have

\Centerline{$\VVdP\,\dot{\bar\mu}\dot a=\vec u_a\in[0,1]$}

\noindent for every $a\in\frak A$, so

\Centerline{$\VVdP\,\dot{\bar\mu}$ is a function from $\dot\frak A$ to
$[0,1]$.}

\noindent Since $u_1=\chi 1$,

\Centerline{$\VVdP\,\dot{\bar\mu} 1=\vec u_1=1$.}

\medskip

\quad{\bf (iv)} Next, $\VVdP\,\dot{\bar\mu}$ is additive.   \Prf\ Suppose that
$p\in\frak C^+$ and $\dot x$, $\dot y$ are $\Bbb P$-names such that

\Centerline{$p\VVdP\,\dot x$, $\dot y\in\dot\frak A$ are disjoint.}

\noindent By 556Ga there are $a$, $b\in\frak A$ such that

\Centerline{$p\VVdP\,\dot x=\dot a$, $\dot y=\dot b$,
$(a\Bcap b)^{\centerdot}=\dot x\dot\Bcap\dot y=0$.}

\noindent So $p\Bcap a\Bcap b=0$ and

\Centerline{$\chi p\times u_{a\Bcup b}
=u_{p\Bcap(a\Bcup b)}
=u_{p\Bcap a}+u_{p\Bcap b}
=\chi p\times u_a+\chi p\times u_b
=\chi p\times(u_a+u_b)$;}

\noindent it follows that

$$\eqalignno{p\VVdP\,\dot{\bar\mu}(\dot x\dot\Bcup\dot y)
&=\dot{\bar\mu}(\dot a\dot\Bcup\dot b)
=\dot{\bar\mu}(a\Bcup b)^{\centerdot}
=\vec u_{a\Bcup b}
=(u_a+u_b)\sspvec\cr
\displaycause{5A3M}
&=\vec u_a+\vec u_b
=\dot{\bar\mu}\dot x+\dot{\bar\mu}\dot y.\cr}$$

\noindent As $p$, $\dot x$ and $\dot y$ are arbitrary,

\Centerline{$\VVdP\,\dot{\bar\mu}$ is additive.  \Qed}

\medskip

\quad{\bf (v)} Suppose that $p\in\frak C^+$ and that $\dot A$ is a
$\Bbb P$-name such that

\Centerline{$p\VVdP\,\dot A\subseteq\dot\frak A$ is closed under
$\dot\Bcup$ and has supremum $1$.}

\noindent Then for every rational number $\alpha<1$ there are an
$r\in\frak C^+$, stronger than $p$, and a $d\in\frak A$ such that

\Centerline{$r\VVdP\,\dot d\in\dot A$ and $\dot{\bar\mu}\dot d
\ge\check\alpha$.}

\noindent\Prf\ Set

\Centerline{$B=\{q\Bcap a:q\in\frak C^+$, $a\in\frak A$,
$q\VVdP\,\dot a\in\dot A\}$,}

\noindent so that $p\Bsubseteq\sup B$ (556Ea).   Because $\bar\mu$ is
completely additive, there are $b_0,\ldots,b_{n-1}\in B$ such that
$\bar\mu(p\Bcap\sup_{i<n}b_i)>\alpha\bar\mu p$.   Express each
$b_i$ as $q_i\Bcap a_i$ where $q_i$ is stronger than $p$ and
$q_i\VVdP\,\dot a_i\in\dot A$.   For $J\subseteq n$ set

\Centerline{$c_J
=p\Bcap\inf_{i\in J}q_i\Bsetminus\sup_{i\in n\setminus J}q_i$,
\quad$d_J=\sup_{i\in J}a_i$;}

\noindent then $\langle c_J\rangle_{J\subseteq n}$ is disjoint and

\Centerline{$p\Bcap\sup_{i<n}b_i
=\sup_{\emptyset\ne J\subseteq n}c_J\Bcap d_J$.}

\noindent Accordingly

\Centerline{$\sum_{\emptyset\ne J\subseteq n}\bar\mu(c_J\Bcap d_J)
>\alpha\bar\mu p$}

\noindent and there must be a non-empty $J\subseteq n$ such that
$c_J\ne 0$ and

\Centerline{$\alpha\bar\mu c_J<\bar\mu(c_J\Bcap d_J)=\int_{c_J}u_{d_J}$.}

\noindent So $r=c_J\Bcap\Bvalue{u_{d_J}\ge\alpha}$ is non-zero.
Set $d=d_J$;  then

\Centerline{$r\VVdP\,\dot a_i\in\dot A$ for every $i\in\check J$, therefore
$\dot d=\sup_{i\in\check J}\dot a_i\in\dot A$,}

\noindent and

\Centerline{$r\VVdP\,\dot{\bar\mu}\dot d=\vec u_{d_J}\ge\check\alpha$.
\Qed}

\medskip

\quad{\bf (vi)} It follows that

\Centerline{$\VVdP\,\dot{\bar\mu}$ is completely additive.}

\noindent\Prf\ Suppose that $p\in\frak C^+$ and that $\dot A$ is a
$\Bbb P$-name such that

\Centerline{$p\VVdP\,\dot A\subseteq\dot\frak A$ is closed under
$\dot\Bcup$ and has supremum $1$.}

\noindent Then for every rational $\alpha<1$ and every $q$ stronger than
$p$ there is an $r$ stronger than $q$ such that

\Centerline{$r\VVdP$ there is an $x\in\dot A$ such that
$\dot{\bar\mu} x\ge\check\alpha$;}

\noindent as $q$ is arbitrary,

\Centerline{$p\VVdP$ there is an $x\in\dot A$ such that
$\dot{\bar\mu} x\ge\check\alpha$;}

\noindent as $\alpha$ is arbitrary,

\Centerline{$p\VVdP\,\sup_{x\in\dot A}\dot{\bar\mu} x=1$.}

\noindent As $p$ and $\dot A$ are arbitrary,

\Centerline{$\VVdP\,\sup_{x\in A}\dot{\bar\mu} x=1$ whenever
$A\subseteq\dot\frak A$ is closed under $\dot\Bcup$ and has supremum $1$.}

\noindent We know that

\Centerline{$\VVdP\,\dot{\bar\mu}$ is additive and $\dot{\bar\mu} 1=1$,}

\noindent so we can turn this over to get

\doubleinset{$\VVdP\,\inf_{x\in A}\dot{\bar\mu} x=0$ whenever
$A\subseteq\dot\frak A$ is closed under $\dot\Bcap$ and has infimum $0$,
therefore $\dot{\bar\mu}$ is completely additive.  \Qed}

\medskip

\quad{\bf (vii)} Since we already know that

\Centerline{$\VVdP\,\dot\frak A$ is Dedekind complete}

\noindent (556Gb), we have all the elements needed for

\Centerline{$\VVdP\,(\dot\frak A,\dot{\bar\mu})$ is a probability algebra.}

\medskip

{\bf (c)} The point is that if $a\in\frak A$ then $u_{\pi a}=u_a$.
\Prf\ For any $c\in\frak C$,

\Centerline{$\int_cu_{\pi a}
=\bar\mu(c\Bcap\pi a)
=\bar\mu(\pi(c\Bcap a))
=\bar\mu(c\Bcap a)
=\int_cu_a$.  \Qed}

\noindent Now suppose that $p\in\frak C^+$ and $\dot x$ is a
$\Bbb P$-name such that $p\VVdP\,\dot x\in\dot\frak A$.
Then there is an $a\in\frak A$ such that $p\VVdP\,\dot a=\dot x$
(556Ga again), and

\Centerline{$p\VVdP\,\dot{\bar\mu}(\dot\pi\dot x)
=\dot{\bar\mu}(\dot\pi\dot a)
=\dot{\bar\mu}(\pi a)^{\centerdot}
=\vec u_{\pi a}
=\vec u_a
=\dot{\bar\mu}\dot x$.}

\noindent As $p$ and $\dot x$ are arbitrary,

\Centerline{$\VVdP\,\dot\pi$ is measure-preserving.}

\medskip

{\bf (d)} By 556J, there is a unique $\pi\in\Aut\frak A$
such that $\pi c=c$ for every $c\in\frak C$ and
$\VVdP\,\dot\phi=\dot\pi$.   In this case, for any $a\in\frak A$,

\Centerline{$\VVdP\,\vec u_a
=\dot{\bar\mu}\dot a
=\dot{\bar\mu}(\dot\phi\dot a)
=\dot{\bar\mu}(\dot\pi\dot a)
=\dot{\bar\mu}(\pi a)^{\centerdot}
=\vec u_{\pi a}$.}

\noindent So $u_a=u_{\pi a}$ (5A3M again) and

\Centerline{$\bar\mu a=\int u_a=\int u_{\pi a}=\bar\mu(\pi a)$.}

\noindent Thus $\pi$ is measure-preserving.

\medskip

{\bf (e)} Let
$P:L^1(\frak A,\bar\mu)\to L^1(\frak C,\bar\mu\restrp\frak C)$ be the
conditional expectation operator, and let $U$ be the set of those
$v\in L^1(\frak A,\bar\mu)$ such that

\Centerline{$\VVdP\,\dot v\in L^1(\dot\frak A,\dot{\bar\mu})$ and
$\int\dot v\,d\dot{\bar\mu}=\vec{Pv}$.}

\noindent By (a), $\chi a\in U$ for every $a\in\frak A$;   by 556Hb, $U$
is closed under addition and rational multiplication;  by 556Hc
$\sup_{n\in\Bbb N}v_n\in U$ for every non-decreasing sequence
$\sequencen{v_n}$ in $U$.   So $U=L^1(\frak A,\bar\mu)$, as required.
}%end of proof of 556K

\leader{556L}{Relatively independent subalgebras} Let $(\frak A,\bar\mu)$
be a probability algebra and $\frak C$ a closed subalgebra
of $\frak A$;  let
$\Bbb P$ be the forcing notion $\frak C^+$, active downwards.
Let $\dot{\bar\mu}$ be the
forcing name for $\bar\mu$ described in 556K, so that
$\VVdP\,(\dot\frak A,\dot{\bar\mu})$ is a probability algebra.

(a) For a
subalgebra $\frak B$ of $\frak A$ including $\frak C$, let $\dot\frak B$ be
the forcing name for $\frak B$ over $\frak C$.
If $\familyiI{\frak B_i}$ is a family of subalgebras of $\frak A$
including $\frak C$, then $\familyiI{\frak B_i}$ is relatively
independent over $\frak C$\cmmnt{ (definition:  458La)} iff

\Centerline{$\VVdP\,\family{i}{\check I}{\dot\frak B_i}$ is stochastically
independent in $\dot\frak A$.}

(b) If $\familyiI{v_i}$ is a family in $L^0(\frak A)$ which is relatively
independent over $\frak C$, then

\Centerline{$\VVdP\,\family{i}{\check I}{\dot v_i}$ is
stochastically independent}

\noindent (writing $\dot v_i$ for the forcing name for $v_i$ over
$\frak C$).

\proof{{\bf (a)(i)} Suppose that $\familyiI{\frak B_i}$ is relatively
independent over $\frak C$.   Let $p\in\frak C^+$ and $\dot J$,
$\family{j}{\dot J}{\dot x_j}$ be $\Bbb P$-names such that

\Centerline{$p\VVdP\,\dot J\in[\check I]^{<\omega}$ is non-empty,
$\dot x_j\in\dot\frak B_j$ for every $j\in\dot J$.}

\noindent Then for every $q$ stronger than $p$ there are
an $r$ stronger than $q$ and a family $\family{j}{J}{b_j}$ such that $J$ is
a non-empty
finite subset of $I$, $b_j\in\frak B_j$ for every $j\in J$, and

\Centerline{$r\VVdP\,\dot J=\check J$ and $\dot x_j=\dot b_j$ for every
$j\in\check J$.}

\noindent Set $a=\inf_{j\in J}b_j$.
For each $j\in J$ let
$u_{b_j}\in L^1(\frak C,\bar\mu\restrp\frak C)$ be the conditional
expectation of $\chi b_j$ on $\frak C$, as in 556Ka;  then
$u_a=\prod_{i\in J}u_{b_j}$, because $\familyiI{\frak B_i}$ is relatively
stochastically independent.   But this means that

$$\eqalign{r\VVdP\,\dot{\bar\mu}(\inf_{j\in\dot J}\dot x_j)
&=\dot{\bar\mu}(\inf_{j\in\check J}\dot b_j)
=\dot{\bar\mu}\dot a
=\vec u_a\cr
&=\prod_{j\in\check J}\vec u_{b_j}
=\prod_{j\in\check J}\dot{\bar\mu}\dot b_j
=\prod_{j\in\dot J}\dot{\bar\mu}\dot x_j.\cr}$$

\noindent As $q$ is arbitrary,

\Centerline{$p\VVdP\,\dot{\bar\mu}(\inf_{j\in\dot J}\dot x_j)
=\prod_{j\in\dot J}\dot{\bar\mu}\dot x_j;$}

\noindent as $p$ and $\family{j}{\dot J}{\dot x_j}$ are arbitrary,

\Centerline{$\VVdP\,\family{i}{\check I}{\dot\frak B_i}$ is independent.}

\medskip

\quad{\bf (ii)} Now suppose that

\Centerline{$\VVdP\,\family{i}{\check I}{\dot\frak B_i}$ is independent.}

\noindent Take a finite set $J\subseteq I$ and
$\family{j}{J}{b_j}\in\prod_{j\in J}\frak B_j$.   Again set
$a=\inf_{j\in J}b_j$ and let $u_{b_j}$ be
the conditional expectation of $\chi b_j$ on $\frak C$ for each $j$.
Then

$$\eqalign{\VVdP\,(\prod_{j\in J}u_{b_j})\sspvec
&=\prod_{j\in\check J}\vec u_{b_j}
=\prod_{j\in\check J}\dot{\bar\mu}\dot b_j\cr
&=\dot{\bar\mu}(\inf_{j\in\check J}\dot b_j)
=\dot{\bar\mu}(\inf_{j\in J}b_j)^{\centerdot}
=\vec u_a,\cr}$$

\noindent so $\prod_{j\in J}u_{b_j}=u_a$.   As $\family{j}{J}{b_j}$ is
arbitrary, $\familyiI{\frak B_i}$ is relatively independent over $\frak C$.

\medskip

{\bf (b)} For each $i\in I$, let $\frak A_i$ be the closed subalgebra of
$\frak A$ generated by $\{\Bvalue{v_i>\alpha}:\alpha\in\Bbb Q\}$, and
$\frak B_i$ the closed subalgebra of $\frak A$ generated by
$\frak A_i\cup\frak C$.   Then $\familyiI{\frak B_i}$ is relatively
independent over $\frak C$ (458Ld), so
$\VVdP\,\family{i}{\check I}{\dot\frak B_i}$ is independent, by (a) here.
Now we have

\Centerline{$\VVdP\,\Bvalue{\dot v_i>\alpha}
=\Bvalue{v_i>\alpha}^{\centerdot}
\in\dot\frak B_i$}

\noindent whenever $\alpha\in\Bbb Q$ and $i\in I$, so

\Centerline{$\VVdP\,\Bvalue{\dot v_i>\alpha}\in\dot\frak B_i$
for every $\alpha\in\Bbb Q$ and $i\in\check I$, and
$\family{i}{\check I}{\dot v_i}$ is independent.}
}%end of proof of 556L

\leader{556M}{Laws of large numbers}\cmmnt{ As an elementary example
to show that we can
use this machinery to extend a classical result, I give the
following.}   Consider the two statements

\inset{($\ddagger$) Let $(X,\Sigma,\mu)$ be a probability space, $\Tau$ a
$\sigma$-subalgebra of $\Sigma$, and $\sequencen{f_n}$ a sequence in
$\eusm L^2(\mu)$ such that $\sequencen{f_n}$ is relatively independent
over $\Tau$ and $\int_Ff_nd\mu=0$ for every $n\in\Bbb N$ and every
$F\in\Tau$.   Suppose that $\sequencen{\beta_n}$ is a non-decreasing
sequence in $\ooint{0,\infty}$, diverging to $\infty$, such that
$\sum_{n=0}^{\infty}\Bover1{\beta_n^2}\|f_n\|_2^2<\infty$.
Then $\lim_{n\to\infty}\Bover1{\beta_n}\sum_{i=0}^nf_i=0$ a.e.}

\noindent and

\inset{($\dagger$) Let $(X,\Sigma,\mu)$ be a probability space and
$\sequencen{f_n}$ an independent sequence in
$\eusm L^2(\mu)$ such that $\int f_nd\mu=0$ for every $n\in\Bbb N$.
Suppose that $\sequencen{\beta_n}$ is a non-decreasing
sequence in $\ooint{0,\infty}$, diverging to $\infty$, such that
$\sum_{n=0}^{\infty}\Bover1{\beta_n^2}\|f_n\|_2^2<\infty$.
Then $\lim_{n\to\infty}\Bover1{\beta_n}\sum_{i=0}^nf_i=0$ a.e.}

\noindent\cmmnt{In 273D I presented ($\dagger$) as the basic strong
law of large
numbers from which the other standard forms could be derived.
($\ddagger$) may be found in Volume 4 as an exercise
(458Yd).   }\dvro{Then if there is a proof of ($\dagger$), there must be
a proof of ($\ddagger$).}{What I propose to do is to show how
($\ddagger$) can be deduced\cmmnt{, not exactly from
($\dagger$), but} from ($\dagger$) in a forcing model\cmmnt{;  relying
on the
fundamental theorem of forcing to confirm that if ($\dagger$) is true in
its ordinary sense, then its interpretation in any forcing language will
again be true}.}

\proof{{\bf (a)} In order to avoid
explanations involving names for real numbers, it seems helpful to re-word
$(\ddagger)$.   Consider the version

\inset{$(\ddagger)_1$ Let $(X,\Sigma,\mu)$ be a probability space, $\Tau$ a
$\sigma$-subalgebra of $\Sigma$, and $\sequencen{f_n}$ a sequence in
$\eusm L^2(\mu)$ such that $\sequencen{f_n}$ is relatively independent
over $\Tau$ and $\int_Ff_nd\mu=0$ for every $n\in\Bbb N$ and every
$F\in\Tau$.   Suppose that $\sequencen{\beta_n}$ is a non-decreasing
sequence in $\Bbb Q\cap\ooint{0,\infty}$, diverging to $\infty$, such that
$\sum_{n=0}^{\infty}\Bover1{\beta_n^2}\|f_n\|_2^2<\infty$.
Then $\lim_{n\to\infty}\Bover1{\beta_n}\sum_{i=0}^nf_i=0$ a.e.}

\noindent Then $(\ddagger)_1$ implies ($\ddagger$).   \Prf\ Given the
structure of ($\ddagger$), with general $\beta_n>0$, let
$\delta_n\in\Bbb Q\cap\ocint{0,\beta_n}$ be such that

\Centerline{$\Bover1{\delta_n^2}\|f_n\|_2^2
\le\Bover1{\beta_n^2}\|f_n\|_2^2+2^{-n}$}

\noindent for every $n$.   Set $\gamma_n=\sup_{m\le n}\delta_m$ for each
$n$;  then $\sequencen{\gamma_n}$ is a non-decreasing sequence in
$\Bbb Q\cap\ooint{0,\infty}$ and
$\sum_{n=0}^{\infty}\Bover1{\gamma_n^2}\|f_n\|_2^2$ is finite, so
$(\ddagger)_1$ tells us that

\Centerline{$\lim_{n\to\infty}\Bover1{\beta_n}\sum_{i=0}^nf_i
=\lim_{n\to\infty}\Bover{\gamma_n}{\beta_n}\cdot\Bover1{\gamma_n}
\sum_{i=0}^nf_i=0$ a.e.   \Qed}

\medskip

{\bf (b)} Now formulate the assertions $(\ddagger)_1$ and
($\dagger$) in terms of measure algebras;  we get

\inset{$(\ddagger)'$ Let $(\frak A,\bar\mu)$ be a probability algebra,
$\frak C$ a closed subalgebra of $\frak A$, and $\sequencen{v_n}$ a
sequence in $L^2(\frak A,\bar\mu)$ such that $\sequencen{v_n}$ is
relatively independent over $\frak C$ and $Pv_n=0$ for every
$n\in\Bbb N$, where
$P:L^1(\frak A,\bar\mu)\to L^1(\frak C,\bar\mu\restrp\frak C)$
is the conditional expectation operator.
Suppose that $\sequencen{\beta_n}$ is a non-decreasing
sequence in $\Bbb Q\cap\ooint{0,\infty}$, diverging to $\infty$, such that
$\sum_{n=0}^{\infty}\Bover1{\beta_n^2}\|v_n\|_2^2<\infty$.
Then $\sequencen{\Bover1{\beta_n}\sum_{i=0}^nv_i}$ is order*-convergent to
$0$.}

\noindent and

\inset{$(\dagger)'$ Let $(\frak A,\bar\mu)$ be a probability algebra
and $\sequencen{v_n}$ an
independent sequence in $L^2(\frak A,\bar\mu)$ such that
$\int v_nd\bar\mu=0$ for every $n$.
Suppose that $\sequencen{\beta_n}$ is a non-decreasing
sequence in $\ooint{0,\infty}$, diverging to $\infty$, such that
$\sum_{n=0}^{\infty}\Bover1{\beta_n^2}\|v_n\|_2^2<\infty$.
Then $\sequencen{\Bover1{\beta_n}\sum_{i=0}^nv_i}$ is order*-convergent to
$0$.}

\noindent (As usual, the conversions are just a matter of applying the
Loomis-Sikorski theorem, with 367F to translate order*-convergence in $L^0$
into almost-everywhere convergence of functions.)

\medskip

{\bf (c)} Assuming $(\dagger)'$, take a structure
$(\frak A,\bar\mu,\frak C,\sequencen{v_n},\sequencen{\beta_n})$ as in
$(\ddagger)'$, let $\Bbb P$ be the forcing notion $\frak C^+$, active
downwards, and consider the corresponding forcing names $\dot\frak A$,
$\dot{\bar\mu}$ and $\sequencen{\dot v_n}$.   Let
$P:L^1(\frak A,\bar\mu)\to L^1(\frak C,\bar\mu\restrp\frak C)$ be the
conditional expectation operator.   For each $n\in\Bbb N$,

\Centerline{$\VVdP\,\dot v_n\times\dot v_n
=(v_n\times v_n)^{\centerdot}\in L^1(\dot\frak A,\dot{\bar\mu})$,
\quad$\|\dot v_n\|_2^2
=\int\dot v_n^2d\dot{\bar\mu}=(P(v_n^2))\sspvec$}

\noindent by 556Hb and 556Ke.   Now

\Centerline{$\sum_{n=0}^{\infty}
  \Bover1{\beta_n}\int P(v_n^2)d(\bar\mu\restrp\frak C)
\le\sum_{n=0}^{\infty}\Bover1{\beta_n}\|v_n\|_2^2
<\infty$,}

\noindent so

\Centerline{$v=\sum_{i=n}^{\infty}\Bover1{\beta_n}P(v_n^2)$}

\noindent is defined in $L^0(\frak C)$, and

\Centerline{$\VVdP\,\sum_{n=0}^{\infty}\Bover1{\check\beta_n}
   \|\dot v_n\|_2^2
\le\vec v$ is finite.}

\noindent At the same time,

\Centerline{$\VVdP\,\int\dot v_nd\dot{\bar\mu}=\vec{Pv_n}=0$ for
every $n\in\Bbb N$,}

\noindent and by 556Lb

\Centerline{$\VVdP\,\sequencen{\dot v_n}$ is independent.}

Applying $(\dagger)'$ in the forcing language,

\Centerline{$\VVdP\,
\sequencen{(\Bover1{\check\beta_n}\sum_{i=0}^nv_i)^{\centerdot}}
=\sequencen{\Bover1{\check\beta_n}\sum_{i=0}^n\dot v_i}$
order*-converges to $0$ in $L^0(\dot\frak A)$,}

\noindent so $\sequencen{\Bover1{\beta_n}\sum_{i=0}^nv_i}$ order*-converges
to $0$ in $L^0(\frak A)$, by 556He.

Thus $(\ddagger)'$ is true, and we're home.
}%end of proof of 556M

\leader{556N}{Dye's theorem}\cmmnt{ Now for something from Volume 3.}
Let me state two versions of Dye's theorem\cmmnt{ (388L)}:  the
`full' version

\inset{($\ddagger$)
Let $(\frak A,\bar\mu)$ be a probability algebra of countable Maharam type,
$\frak C$ a closed subalgebra of $\frak A$, and $\pi_1$, $\pi_2$ two
measure-preserving automorphisms of $\frak A$ with fixed-point algebra
$\frak C$.   Then there is a measure-preserving automorphism $\phi$ of
$\frak A$ such that $\phi c=c$ for every $c\in\frak C$ and
$\pi_1$ and $\phi\pi_2\phi^{-1}$ generate the same
full subgroups of $\Aut\frak A$.}

\noindent and the `simple' version

\inset{($\dagger$) Let $(\frak A,\bar\mu)$ be a probability algebra of
countable Maharam type, and $\pi_1$, $\pi_2$ two ergodic
measure-preserving automorphisms of $\frak A$.
Then there is a measure-preserving automorphism $\phi$ of
$\frak A$ such that $\pi_1$ and $\phi\pi_2\phi^{-1}$ generate the same
full subgroups of $\Aut\frak A$.}

\noindent Here\cmmnt{ also} the machinery of this section provides a
proof of ($\ddagger$) from ($\dagger$).

\proof{{\bf (a)} Assume ($\dagger$).   Take $(\frak A,\bar\mu)$,
$\frak C$, $\pi_1$ and $\pi_2$ as in ($\ddagger$).   Let
$\Bbb P$ be the forcing notion $\frak C^+$, active downwards,
and let $\dot\frak A$, $\dot\pi_1$ and $\dot\pi_2$ be
the forcing names for $\frak A$, $\pi_1$ and $\pi_2$ over $\frak C$.
By 556C(b-iv), 556C(b-v), 556Gb, 556Kb and 556Kc, and using 372Pc,

\doubleinset{$\VVdP$ there is a measure on $\dot\frak A$ with respect to
which it is a probability measure and $\dot\pi_1$ and $\dot\pi_2$ are
measure-preserving automorphisms with fixed-point subalgebra $\{0,1\}$,
so are ergodic, because $\dot\frak A$ is Dedekind complete.}

\noindent By 556Ed,

\Centerline{$\VVdP\,\dot\frak A$ has countable Maharam type.}

\noindent By ($\dagger$), applied in the forcing universe,

\doubleinset{$\VVdP$ there is a measure-preserving automorphism $\theta$ of
$\dot\frak A$ such that $\dot\pi_1$ and $\theta\dot\pi_2\theta^{-1}$
generate the same full subgroups of $\Aut\dot\frak A$.}

\noindent Let $\dot\theta$ be a $\Bbb P$-name such that

\doubleinset{$\VVdP\,\dot\theta$ is a measure-preserving automorphism of
$\dot\frak A$ such that $\dot\pi_1$ and
$\dot\theta\dot\pi_2\dot\theta^{-1}$
generate the same full subgroups of $\Aut\dot\frak A$.}

\noindent
By 556Kd, there is a $\phi\in\Aut_{\bar\mu}\frak A$ such that
$\phi c=c$ for every $c\in\frak C$ and
$\VVdP\,\dot\theta=\dot\phi$, so that, setting $\pi_3=\phi\pi_2\phi^{-1}$,

\Centerline{$\VVdP\,\dot\pi_1$ and $\dot\pi_3$ generate the same full
subgroups of $\Aut\dot\frak A$}

\noindent (using 556C(a-iii) and 556C(b-iv)).

\medskip

{\bf (b)} Since

\Centerline{$\VVdP\,\dot\pi_3$ belongs to the full subgroup of
$\Aut\dot\frak A$ generated by $\dot\pi_1$,}

\noindent we can apply 381I(c-iv) in the forcing language to get

\Centerline{$\VVdP\,\inf_{n\in\Bbb Z}\supp(\dot\pi_1^n\dot\pi_3)=0$.}

\noindent Now by 556Ic we know that

\Centerline{$\VVdP\,\supp(\dot\pi_1^n\dot\pi_3)
=(\supp(\pi_1^n\pi_3))^{\centerdot}$}

\noindent for every $n\in\Bbb Z$ (of course we need to check that
$\VVdP\,\dot\pi_1^n\dot\pi_3=(\pi_1^n\pi_3)^{\centerdot}$;  but this is
easily deduced from 556C(a-iii),
an induction on $n$ for $n\ge 0$, and 556C(b-iv)).
So

$$\eqalignno{\VVdP\,(\inf_{n\in\Bbb Z}\supp(\pi_1^n\pi_3))^{\centerdot}
&=\inf_{n\in\Bbb Z}(\supp(\pi_1^n\pi_3)^{\centerdot})\cr
\displaycause{556E(b-ii)}
&=\inf_{n\in\Bbb Z}\supp(\pi_1^n\pi_3)^{\centerdot}
=\inf_{n\in\Bbb Z}\supp(\dot\pi_1^n\dot\pi_3)
=0.\cr}$$

\noindent By 556Da, as usual,
$\inf_{n\in\Bbb Z}\supp(\pi_1^n\pi_3)=0$;  by 381I(c-iv), in the other
direction and in the ordinary universe, $\pi_3$ belongs to the full
subgroup of $\Aut\frak A$ generated by $\pi_1$.   Similarly, $\pi_1$
belongs to the full subgroup generated by $\pi_3$, so $\pi_1$ and $\pi_3$
generate the same full subgroups, as required by ($\ddagger$).
}%end of proof of 556N

\leader{556O}{}\cmmnt{ For the next result, I prepare the ground with a
note on `full local semigroups' as defined in
\S395.

\medskip

\noindent}{\bf Lemma} Let $\frak A$ be a Dedekind complete Boolean algebra,
not $\{0\}$, and $\frak C$ a regularly embedded subalgebra of
$\frak A$;  let $\Bbb P$ be the forcing notion $\frak C^+$, active
downwards.   Let $\dot\frak A$
be the forcing name for $\frak A$ over $\frak C$, and for a ring
homomorphism $\pi:\frak A\to\frak A$ such that $\pi c\Bsubseteq c$ for
every $c\in\frak C$ let $\dot\pi$ be the forcing name for
$\pi$ over $\frak C$.   Let $G$ be a subgroup of $\Aut\frak A$ such that
every point of $\frak C$ is fixed by every member of $G$,
and $\dot G$ the $\Bbb P$-name $\{(\dot\pi,1):\pi\in G\}$.

(a) $\VVdP\,\dot G$ is a subgroup of $\Aut\dot\frak A$.

(b) If $\phi:\frak A\to\frak A$ is a ring homomorphism such that
$\phi c\Bsubseteq c$ for every $c\in\frak C$, and

\Centerline{$\VVdP\,\dot\phi$ belongs to the full local semigroup generated
by $\dot G$,}

\noindent then $\phi$ belongs to the full local semigroup generated by $G$.

\proof{{\bf (a)(i)} If $p\in\frak C^+$ and $\dot x$ is a $\Bbb P$-name such
that
$p\VVdP\,\dot x\in\dot G$, then for every $q$ stronger than $p$ there must
be an $r$ stronger than $q$ and a $\pi\in G$ such that

\Centerline{$r\VVdP\,\dot x=\dot\pi\in\Aut\dot\frak A$.}

\noindent (556C(b-iv)).
As $q$ is arbitrary, $p\VVdP\,\dot x\in\Aut\dot\frak A$;  as $p$
and $\dot x$ are arbitrary,

\Centerline{$\VVdP\,\dot G\subseteq\Aut\dot\frak A$.}

\medskip

\quad{\bf (ii)} Writing $\iota$ for the identity automorphism of $\frak A$,

\Centerline{$\VVdP\,\dot\iota\in\dot G$ is the identity automorphism of
$\dot\frak A$}

\noindent (see part (b-iv) of the proof of 556C).
If $p\in\frak C^+$ and $\dot x$, $\dot y$ are $\Bbb P$-names
such that $p\VVdP\,\dot x$, $\dot y\in\dot G$, then for every $q$ stronger
than $p$ there are $r$ stronger than $q$ and $\pi_1$, $\pi_2\in G$ such
that

\Centerline{$r\VVdP\,\dot x=\dot\pi_1$, $\dot y=\dot\pi_2$,
$\dot x\cdot\dot y=\dot\pi_1\dot\pi_2=(\pi_1\pi_2)^{\centerdot}
\in\dot G$,
$\dot x^{-1}=(\dot\pi_1)^{-1}=(\pi_1^{-1})^{\centerdot}\in\dot G$}

\noindent (556C(a-iii), 556C(b-iv)),
because $\pi_1\pi_2$ and $\pi_1^{-1}$ belong to $G$.   As $q$ is arbitrary,

\Centerline{$p\VVdP\,\dot x\cdot\dot y$ and $\dot x^{-1}$ belong to
$\dot G$;}

\noindent as $p$, $\dot x$ and $\dot y$ are arbitrary,

\Centerline{$\VVdP\,\dot G$ is a subgroup of $\Aut\dot\frak A$.}

\medskip

{\bf (b)} Take any non-zero $a\in\frak A$.   Then there is a
$p\in\frak C^+$ such that $p\VVdP\,\dot a\ne 0$ (556Da).   Since

\Centerline{$p\VVdP\,\dot\phi$ belongs to the full local semigroup
generated by $\dot G$,}

\noindent there must be $\Bbb P$-names $\dot x$, $\dot\theta$ such that

\Centerline{$p\VVdP\,\dot x\in\dot\frak A\setminus\{0\}$,
$\dot x\dot{\Bsubseteq}\dot a$,
$\dot\theta\in\dot G$,
$\dot\theta y=\dot\phi y$ whenever $y\dot{\Bsubseteq}\dot x$}

\noindent (395B(a-ii)).   Now there are a $q$ stronger than $p$ and
$b\in\frak A$, $\pi\in G$ such that

\Centerline{$q\VVdP\,\dot b=\dot x$, $\dot\pi=\dot\theta$.}

\noindent Since $q\VVdP\,\dot b\ne 0$, $q\Bcap b\ne 0$.   Suppose that
$d\Bsubseteq q\Bcap b$.   Then

\Centerline{$q\VVdP\,\dot d\dot{\Bsubseteq}\dot x$, so
$(\pi d)^{\centerdot}=\dot\pi\dot d
=\dot\theta\dot d=\dot\phi\dot d=(\phi d)^{\centerdot}$}

\noindent and

$$\eqalignno{\pi d
&=q\Bcap\pi d\cr
\displaycause{see (a-i-$\alpha$) of the proof of 556C}
&=q\Bcap\phi d\cr
\displaycause{556Da, because $\phi d\Bsubseteq\phi q\Bsubseteq q$}
&=\phi d.\cr}$$

\noindent Thus $\pi$ and $\phi$ agree on the principal ideal
$\frak A_{q\Bcap b}$, while $q\Bcap b\Bsubseteq a$ is non-zero.   As $a$
is arbitrary, $\phi$ belongs to the full local semigroup generated by
$G$, by 395B(a-ii) in the other direction.
}%end of proof of 556O

\vleader{72pt}{556P}{Kawada's theorem} In the same way as in 556M and 556N,
we have two versions of 395P:

\inset{($\ddagger$) Let $\frak A$ be a Dedekind complete Boolean
algebra and $G$ a fully non-paradoxical subgroup of $\Aut\frak A$, with
fixed-point subalgebra $\frak C$, such that $\frak C$ is a measurable
algebra.   Then there is a strictly positive $G$-invariant countably
additive real-valued functional on $\frak A$.}

\noindent and

\inset{($\dagger$) Let $\frak A$ be a Dedekind
complete Boolean algebra such that $\Aut\frak A$ has a subgroup $G$
which is ergodic and fully non-paradoxical.
Then there is a strictly positive $G$-invariant countably
additive real-valued functional on $\frak A$.}

\noindent Once again,\cmmnt{ I claim that} we can prove ($\ddagger$) from
($\dagger$).

\proof{{\bf (a)} Take $\frak A$, $G$ and $\frak C$ as in ($\ddagger$).
If $\frak A=\{0\}$, the result is trivial;  so let us suppose from now on
that $\frak A\ne\{0\}$.
Let $\bar\lambda$ be a functional such that $(\frak C,\bar\lambda)$ is a
probability algebra.
Let $\Bbb P$ be the forcing notion $\frak C^+$, active downwards,
and let $\dot\frak A$ be the forcing name for
$\frak A$ over $\frak C$;  for $\pi\in\Aut\frak A$ let $\dot\pi$ be the
forcing name for $\pi$ over $\frak C$.   Let $\dot G$ be the
$\Bbb P$-name $\{(\dot\pi,1):\pi\in G\}$.

\medskip

{\bf (b)} $\VVdP\,\dot G$ is an ergodic subgroup of $\Aut\dot\frak A$.
\Prf\ I noted in 556Oa that

\Centerline{$\VVdP\,\dot G$ is a subgroup of $\Aut\dot\frak A$.}

\noindent  For its ergodicity, copy the argument of 556C(b-v).
Suppose that $p\in\frak C^+$ and $\dot x$ is a $\Bbb P$-name
such that

\Centerline{$p\VVdP\,\dot x\in\dot\frak A$ and $\theta(\dot x)=\dot x$ for
every $\theta\in\dot G$.}

\noindent For any $q$ stronger than $p$ there are an $r$ stronger than $q$
and an $a\in\frak A$ such that $r\VVdP\,\dot x=\dot a$.   Take any
$\pi\in G$.   Then

\Centerline{$r\VVdP\,\dot\pi\in\dot G$, $(\pi a)^{\centerdot}
=\dot\pi\dot x=\dot x=\dot a$,}

\noindent so

\Centerline{$\pi(r\Bcap a)=r\Bcap\pi a=r\Bcap a$}

\noindent (556Da).   As $\pi$ is arbitrary, $r\Bcap a\in\frak C$.   If
$r\Bcap a\ne 0$, then $r\Bcap a\VVdP\,\dot a=1$;  if $r\Bcap a=0$, then
$r\VVdP\,\dot a=0$.   In either case, we have an $s$ stronger than $r$
such that $s\VVdP\,\dot x\in\{0,1\}$.   As $q$ is arbitrary,
$p\VVdP\,\dot x\in\{0,1\}$;  as $p$ and $\dot x$ are arbitrary,

\doubleinset{$\VVdP\,\dot G$ has fixed-point subalgebra $\{0,1\}$, so
is ergodic, because $\dot\frak A$ is Dedekind complete}

\noindent (556Gb, 395Gf).  \Qed

\medskip

{\bf (c)} $\VVdP\,\dot G$ is fully non-paradoxical.

\Prf\ {\bf (i)} \Quer\ Otherwise,

\Centerline{$\notVVdash_{\Bbb P}\,\dot G$ satisfies condition (i) of 395D,}

\noindent and there must be a $p\in\frak C^+$ and $\Bbb P$-names
$\dot\theta$, $\dot x$ such that

\doubleinset{$p\VVdP\,\dot x\in\dot\frak A\setminus\{1\}$,
$\dot\theta$ is a Boolean homomorphism from
$\dot\frak A$ to the principal ideal generated by $\dot x$, and
$\dot\theta$ belongs to the full local semigroup generated by $\dot G$.}

\noindent In order to apply 556J and 556O as stated
we need a $\Bbb P$-name $\dot\theta_1$
such that $\VVdP\,\dot\theta_1$ is a ring homomorphism.
If $p=1$, take
$\dot\theta_1=\dot\theta$;  otherwise, take $\dot\theta_1$ such that

\Centerline{$p\VVdP\,\dot\theta_1=\dot\theta$,
\quad$1\Bsetminus p\VVdP\,\dot\theta_1$ is the identity automorphism.}

\noindent Then

\doubleinset{$\VVdP\,\dot\theta_1:\dot\frak A\to\dot\frak A$ is a ring
homomorphism belonging to the full local semigroup
generated by $\dot G$.}

\medskip

\quad{\bf (ii)} By 556J there is a unique ring homomorphism
$\phi:\frak A\to\frak A$ such that $\phi c\Bsubseteq c$ for every
$c\in\frak C$ and $\VVdP\,\dot\theta_1=\dot\phi$.   By 556Ob, $\phi$
belongs to the full local semi-group generated by $G$.   Since $G$ is fully
non-paradoxical, $\phi 1=1$ and $\VVdP\,\dot\theta_11=\dot\phi 1=1$.
But $p\VVdP\,\dot\theta_11=\dot x\ne 1$.\ \Bang\Qed

\medskip

{\bf (d)} Applying ($\dagger$) in the forcing language, we see that

\doubleinset{$\VVdP$ there is a strictly positive $\dot G$-invariant
countably additive functional on $\dot\frak A$, therefore there is a
there is a strictly positive $\dot G$-invariant
countably additive functional on $\dot\frak A$ taking values in $[0,1]$.}

\noindent Let $\dot\nu$ be a $\Bbb P$-name such that

\doubleinset{$\VVdP\,\dot\nu$ is a strictly positive $\dot G$-invariant
$[0,1]$-valued countably additive functional on $\dot\frak A$.}

\noindent For each $a\in\frak A$,
$\VVdP\,\dot\nu\dot a\in[0,1]$, so there
is a unique $u_a\in L^0(\frak C)^+$ such that

\Centerline{$\VVdP\,\dot\nu\dot a=\vec u_a$}

\noindent (5A3M once more), and $0\le u_a\le\chi 1$.
Set $\mu a=\int u_ad\bar\lambda$ for $a\in\frak A$.

\medskip

{\bf (e)} $\mu$ is a strictly positive $G$-invariant
countably additive functional on $\frak A$.

\medskip

\Prf\ {\bf (i)} If $a$, $b\in\frak A$ are disjoint,

\Centerline{$\VVdP\,\dot a\dot{\Bcap}\dot b=0$, so
$\vec u_a+\vec u_b=\dot\nu\dot a+\dot\nu\dot b
=\dot\nu(\dot a\dot\Bcup\dot b)=\dot\nu(a\Bcup b)^{\centerdot}
=\vec u_{a\Bcup b}$}

\noindent (using 556Bb);  it follows that $u_a+u_b=u_{a\Bcup b}$ and
$\mu a+\mu b=\mu(a\Bcup b)$.   Thus $\mu$ is additive.

\medskip

\quad{\bf (ii)} If $\sequencen{a_n}$ is a
non-decreasing family in $\frak A$ with supremum $a$, then, by 556Be and
556Eb,

\doubleinset{$\VVdP\,\sequencen{\dot a_n}$ is a non-decreasing sequence
in $\dot\frak A$ with supremum $\dot a$, so
$\sequencen{\vec u_{a_n}}=\sequencen{\dot\nu\dot a_n}$ is a non-decreasing
sequence in $[0,1]$ with supremum $\vec u_a=\dot\nu\dot a$.}

\noindent Now $\sequencen{u_{a_n}}$ is a non-decreasing sequence in
$L^0(\frak C)$ with supremum $u_a$ (5A3Ld), so
$\mu a=\sup_{n\in\Bbb N}\mu a_n$.   Thus $\mu$ is countably additive.

\wheader{556P}{6}{2}{2}{36pt}
\quad{\bf (iii)} Because $u_a\ge 0$, $\mu a\ge 0$ for every $a\in\frak A$.
If $\mu a=0$, then $u_a=0$ so

\Centerline{$\VVdP\,\dot\nu\dot a=\vec u_a=0$, therefore $\dot a=0$,
because $\dot\nu$ is strictly positive,}

\noindent and $a=0$ (556Da).   Thus $\mu$ is strictly positive.

\medskip

\quad{\bf (iv)} Suppose that $\pi\in G$ and $a\in\frak A$.   Then

\Centerline{$\VVdP\,\dot\pi\in\dot G\text{ and }\dot\nu\text{ is }
  \dot G\text{-invariant, so }
\vec u_{\pi a}
=\dot\nu(\pi a)^{\centerdot}
=\dot\nu(\dot\pi\dot a)
=\dot\nu\dot a
=\vec u_a$.}

\noindent So $u_{\pi a}=u_a$ and $\mu(\pi a)=\mu a$.   Thus $\mu$ is
$G$-invariant.\ \Qed

Accordingly $\mu$ is a functional as required by ($\ddagger$).
}%end of proof of 556P

\leader{556Q}{}\cmmnt{ For the final application of the methods of this
section, I turn to a result of a quite different kind.   Here the structure
under consideration, the asymptotic density algebra $\frak Z$, is off the
main line of this treatise, but has some important measure-theoretic
properties (see \S491);  and it turns out that there is a remarkable
identification of its Dedekind completion (556S) which can be established
by applying Maharam's theorem in a suitable forcing universe of the kind
considered here.   I start with a couple of easy lemmas, one just a
restatement of ideas from Volume 3, and the other a straightforward
property of a basic class of forcing notions.

\medskip

\noindent}{\bf Lemma} (a) Let $\frak A$ be a Boolean algebra and
$\bar\mu:\frak A\to[0,1]$ a strictly positive additive functional such that
$\bar\mu 1=1$.   Suppose that
whenever $\sequencen{a_n}$ is a non-increasing sequence in $\frak A$,
there is an $a\in\frak A$ such that $a\Bsubseteq a_n$ for every $n$ and
$\bar\mu a=\inf_{n\in\Bbb N}\bar\mu a_n$.   Then $(\frak A,\bar\mu)$ is a
probability algebra.

(b) Let $(\frak A,\bar\mu)$ be a probability algebra.   Suppose that
$\kappa\ge\tau(\frak A)$ is an infinite cardinal and that
$\ofamily{\xi}{\kappa}{e_{\xi}}$
is a family in $\frak A$ such that
$\bar\mu(\inf_{\xi\in K}e_{\xi})=2^{-\#(K)}$ for every finite
$K\subseteq I$.   Then $(\frak A,\bar\mu)$ is isomorphic to the measure
algebra $(\frak B_{\kappa},\bar\nu_{\kappa})$ of the usual measure on
$\{0,1\}^{\kappa}$.

\proof{{\bf (a)} Let $A\subseteq\frak A$ be a non-empty
countable set.   Let $\sequencen{a_n}$ be a sequence running over $A$, and
set $b_n=\inf_{i\le n}a_i$ for each $n$.
There is a $b\in\frak A$, a lower
bound for $\{b_n:n\in\Bbb N\}$ and therefore for $A$, such that
$\bar\mu b=\inf_{n\in\Bbb N}\bar\mu b_n$.
If $c\in\frak A$ is any lower bound for $A$, then $b\Bcup c\Bsubseteq b_n$
for every $n$, so

\Centerline{$\bar\mu b+\bar\mu(c\Bsetminus b)
=\bar\mu(b\Bcup c)\le\inf_{n\in\Bbb N}\bar\mu b_n=\bar\mu b$,}

\noindent and $\bar\mu(c\Bsetminus b)=0$;
as $\bar\mu$ is strictly positive,
$c\Bsubseteq b$.   Thus $b=\inf A$.   As $A$ is
arbitrary, $\frak A$ is Dedekind $\sigma$-complete.   But this is the only
clause missing from the definition of `probability algebra'.

\medskip

{\bf (b)} By 331Ja,
$\tau(\frak A_d)\ge\kappa$ for every non-zero $d\in\frak A$.   So
$\frak A$ is \Mth, with Maharam type $\kappa$, and
$(\frak A,\bar\mu)\cong(\frak B_{\kappa},\bar\mu_{\kappa})$ (331I).
}%end of proof of 556Q

\leader{556R}{Proposition} Let $\Bbb P$ be a countably closed forcing
notion.   Then, for any set $I$, writing $(\frak B_I,\bar\nu_I)$ for the
measure algebra of the usual measure on $\{0,1\}^I$,

\Centerline{$\VVdP\,\,(\frak B_{\check I},\bar\nu_{\check I})
\cong(\check\frak B_I,\check{\bar\nu}_I)$.}

\proof{ If $I$ is finite, this is elementary (and does not rely on $\Bbb P$
being countably closed), so I shall suppose that $I$ is infinite.

\medskip

{\bf (a)}

\doubleinset{$\VVdP$ if $\sequencen{x_n}$ is a non-increasing sequence in
$\check\frak B_I$, there is an $x\in\check\frak B_I$ such that
$x\check{\Bsubseteq}x_n$ for every $n$ and
$\check{\bar\nu}_I(x)=\inf_{n\in\Bbb N}\check{\bar\nu}_I(x_n)$.}

\noindent\Prf\ Let $p$ be a condition of $\Bbb P$ and
$\sequencen{\dot x_n}$ a sequence of $\Bbb P$-names such that

\Centerline{$p\VVdP\,\dot x_n\in\check\frak B_I$ and
$\dot x_{n+1}\check{\Bsubseteq}\dot x_n$}

\noindent for every $n$.   If $q$ is stronger
than $p$, we can choose $\sequencen{q_n}$, $\sequencen{b_n}$
inductively so that $q_0=q$ and, for each $n$, $q_{n+1}$ is stronger than
$q_n$, $b_n\in\frak B_I$ and $q_{n+1}\VVdP\,\dot x_n=\check b_n$.
In this case,

\Centerline{$q_{n+1}\VVdP\,\check b_{n+1}=\dot x_{n+1}
  \check{\Bsubseteq}\dot x_n=\check b_n$,}

\noindent so $b_{n+1}\Bsubseteq b_n$ for each $n$.   Setting
$b=\inf_{n\in\Bbb N}b_n$, $\bar\nu_Ib=\inf_{n\in\Bbb N}\bar\nu_Ib_n$.
Also, because $\Bbb P$ is countably closed, there is a condition $r$
stronger than any $q_n$.   So

\Centerline{$r\VVdP\,\check b\check{\Bsubseteq}\check b_n=\dot x_n$
for every $n\in\Bbb N$,
$\check{\bar\nu}_I(\check b)
=\inf_{n\in\Bbb N}\check{\bar\nu}_I(\dot x_n)$.}

\noindent As $q$ is arbitrary,

\doubleinset{$p\VVdP$ there is an $x\in\check\frak B_I$ such that
$x\check{\Bsubseteq}\dot x_n$ for every $n$ and
$\check{\bar\nu}_I(x)=\inf_{n\in\Bbb N}\check{\bar\nu}_I(\dot x_n)$.}

\noindent As $p$ and $\sequencen{\dot x_n}$ are arbitrary,

\doubleinset{$\VVdP$ if $\sequencen{x_n}$ is a non-increasing sequence in
$\check\frak B_I$, there is an $x\in\check\frak B_I$ such that
$x\check{\Bsubseteq}x_n$ for every $n$ and
$\check{\bar\nu}_I(x)=\inf_{n\in\Bbb N}\check{\bar\nu}_I(x_n)$.  \Qed}

Since we certainly have

\doubleinset{$\VVdP\,\check\frak B_I$ is a Boolean algebra and
$\check{\bar\nu}:\check\frak B_I\to[0,1]$ is a strictly positive additive
functional such that $\check{\bar\nu}_I1=1$,}

\noindent 556Qa, applied in the forcing universe, tells us that

\Centerline{$\VVdP\,(\check\frak B,\check{\bar\nu}_I)$ is a
probability algebra.}

\medskip

{\bf (b)} Let $\familyiI{e_i}$ be the standard generating family in
$\frak B_I$.   Then
$\bar\nu_I(\inf_{i\in K}e_i)=2^{-\#(K)}$ for every finite set
$K\subseteq I$, so

\Centerline{$\VVdP\,
\check{\bar\nu}_I(\inf_{i\in K}\check e_i)=2^{-\#(K)}$
for every finite set $K\subseteq\check I$.}

\noindent Next, if $\frak D$ is the
subalgebra of $\frak B_I$ generated by $\{e_i:i\in I\}$, then $\frak D$ is
dense in $\frak B_I$ for the measure metric.   Now

\doubleinset{$\VVdP\,\check\frak D$
is the subalgebra of $\check\frak B_I$ generated by
$\{\check e_i:i\in\check I\}$ and $\check\frak D$ is metrically dense in
$\check\frak B_I$, so $\tau(\check\frak B_I)\le\#(\check I)$.
By 556Q, $(\check\frak B_I,\check{\bar\nu}_I)
\cong(\frak B_{\#(\check I)},\bar\nu_{\#(\check I)})
\cong(\frak B_{\check I},\bar\nu_{\check I})$,}

\noindent as required.
}%end of proof of 556R

\leader{556S}{Theorem}\cmmnt{ ({\smc Farah 06})} Let
$\Cal Z$ be the ideal of subsets of $\Bbb N$
with asymptotic density $0$ and $\frak Z$ the asymptotic density algebra
$\Cal P\Bbb N/\Cal Z$.   Then the Dedekind completion of
$\frak Z$ is isomorphic to the Dedekind
completion of the free product
$(\Cal P\Bbb N/[\Bbb N]^{<\omega})\otimes\frak B_{\frakc}$.

\proof{{\bf (a)} For $n\in\Bbb N$, set $I_n=\{i:2^n\le i<2^{n+1}\}$, so
that $\sequencen{I_n}$ is a partition of $\Bbb N\setminus\{0\}$, and
$\#(I_n)=2^n$ for every $n\in\Bbb N$.   Recall that

\Centerline{$\Cal Z
=\{J:J\subseteq\Bbb N$, $\lim_{n\to\infty}2^{-n}\#(J\cap I_n)=0\}$}

\noindent (491Ab).   The notation of this proof will be slightly less
appalling if I write $b_J$ for  $J^{\ssbullet}\in\frak Z$ when
$J\subseteq\Bbb N$ and $c_K$ for
$(\bigcup_{n\in K}I_n)^{\ssbullet}$ when $K\subseteq\Bbb N$.

Set

\Centerline{$\frak C=\{c_K:K\subseteq\Bbb N\}$.}

\noindent Because $K\mapsto c_K:\Cal P\Bbb N\to\frak Z$ is a Boolean
homomorphism, $\frak C$ is a subalgebra of $\frak Z$.   Now
$\frak C\cong\Cal P\Bbb N/[\Bbb N]^{<\omega}$.   \Prf\
If $K\subseteq\Bbb N$, then

\Centerline{$c_K=0
\iff\bigcup_{n\in K}I_n\in\Cal Z
\iff K$ is finite.}

\noindent So the Boolean homomorphism $K\mapsto c_K$ induces a Boolean
isomorphism
$\pi:\Cal P\Bbb N/[\Bbb N]^{<\omega}\to\frak C$ defined by setting
$\pi(K^{\ssbullet})=c_K$ for every $K\subseteq\Bbb N$.\ \Qed

\woddheader{556S}{0}{0}{0}{36pt}

For $p\in\frak C^+$, set

\Centerline{$\Cal F_p=\{K:K\subseteq\Bbb N$, $p\Bsubseteq c_K\}$,}

\noindent so that $\Cal F_p$ is a filter on $\Bbb N$ containing every
cofinite set.   Note that if $p\Bsubseteq q$ then $\Cal F_p$ is finer than
$\Cal F_q$.

\medskip

{\bf (b)} $\frak C$ is regularly embedded in $\frak Z$.   \Prf\ Suppose
that $A\subseteq\frak C$ has infimum $0$ in $\frak C$, and that
$b\in\frak Z^+$.   Let $J_0\in\Cal P\Bbb N\setminus\Cal Z$ be such that
$b=b_{J_0}$.   Then $\limsup_{n\to\infty}2^{-n}\#(J_0\cap I_n)>0$,
so there is an $\epsilon>0$ such that
$K=\{n:\#(J_0\cap I_n)\ge 2^n\epsilon\}$ is infinite.   $c_K$ cannot be a
lower bound of $A$ in $\frak C$, so there is an $L\subseteq\Bbb N$ such
that $c_L\in A$ and $c_K\notBsubseteq c_L$, that is, $K\setminus L$ is
infinite.   Set $J=\bigcup_{n\in K\setminus L}J_0\cap I_n$;  then
$\#(J\cap I_n)\ge 2^n\epsilon$ for infinitely many $n$, so
$J\notin\Cal Z$ and $0\ne b_J\Bsubseteq b$.   On the other hand,
$b_J\Bcap c_L=0$.   So $b\notBsubseteq c_L$ and $b$ is not a lower bound of
$A$ in $\frak Z$.   As $b$ is arbitrary, $A$ has infimum $0$ in $\frak Z$;
as $A$ is arbitrary, the embedding $\frak C\embedsinto\frak Z$ is
order-continuous (313L(b-v)), and $\frak C$ is regularly embedded in
$\frak Z$.\ \Qed.

\medskip

{\bf (c)} Let $\Bbb P$ be the forcing notion $\frak C^+$, active downwards.
Then $\Bbb P$ is countably closed.   \Prf\ Let $\sequence{m}{p_m}$ be a
non-increasing sequence in $\frak C^+$.   For each $m\in\Bbb N$, let
$K_m\subseteq\Bbb N$ be such that $p_m=c_{K_m}$.
Then $K_{m+1}\setminus K_m$ is finite for each $m$.   Let
$\sequence{k}{n_k}$ be a strictly increasing sequence such that
$n_k\in K_m$ whenever $m\le k\in\Bbb N$, and set
$K=\{n_k:k\in\Bbb N\}$;  then $c_K$ belongs to
$\frak C^+$, and $c_K\Bsubseteq p_m$ for every $m\in\Bbb N$.\ \Qed

\medskip

{\bf (d)(i)} Let $\dot\frak Z$ be the forcing name for $\frak Z$ over
$\frak C$, and for $b\in\frak Z$ let $\dot b$ be the forcing name for $b$
over $\frak C$.   Let $\dot\nu$ be the $\Bbb P$-name

\Centerline{$\{((\dot b_J,\check\alpha),p):
  p\in\frak C^+,\,J\subseteq\Bbb N,\,
  \lim_{n\to\Cal F_p}2^{-n}\#(J\cap I_n)$ is defined and equal to
  $\alpha\}$.}

\medskip

\quad{\bf (ii)} $\VVdP\,\dot\nu$ is a function.   \Prf\
Suppose that $(J_0,\alpha_0,p_0)$ and
$(J_1,\alpha_1,p_1)\in\Cal P\Bbb N\times\Bbb R\times\frak C^+$ are such
that

\Centerline{$\lim_{n\to\Cal F_{p_0}}2^{-n}\#(J_0\cap I_n)=\alpha_0$,
\quad$\lim_{n\to\Cal F_{p_1}}2^{-n}\#(J_1\cap I_n)=\alpha_1$,}

\noindent and that $p\in\frak C^+$, $p\Bsubseteq p_0\Bcap p_1$ and
$p\VVdP\,\dot b_{J_0}=\dot b_{J_1}$.
Then $p\Bcap b_{J_0}=p\Bcap b_{J_1}$ (556Da).
Express $p$ as $c_K$, where $K\subseteq\Bbb N$;  then
$\bigcup_{n\in K}I_n\cap(J_0\symmdiff J_1)\in\Cal Z$, so
$\lim_{n\in K,n\to\infty}2^{-n}\#(I_n\cap(J_0\symmdiff J_1))=0$, that is,
$\lim_{n\to\Cal F_p}2^{-n}\#(I_n\cap(J_0\symmdiff J_1))=0$.   But this
means that

\Centerline{$\alpha_0
=\lim_{n\to\Cal F_p}2^{-n}\#(I_n\cap J_0)
=\lim_{n\to\Cal F_p}2^{-n}\#(I_n\cap J_1)
=\alpha_1$,}

\noindent and surely $p\VVdP\,\check\alpha_0=\check\alpha_1$.   Thus the
condition of 5A3H is satisfied and

\Centerline{$\VVdP\,\dot\nu$ is a function.  \Qed}

\medskip

\quad{\bf (iii)} $\VVdP\,\dom\dot\nu=\dot\frak Z$.   \Prf\ Setting

\Centerline{$\dot A
=\{(\dot b_J,p):p\in\frak C^+,\,J\subseteq\Bbb N,\,
  \lim_{n\to\Cal F_p}2^{-n}\#(J\cap I_n)$ is defined$\}$,}

\noindent 5A3H tells us that $\VVdP\,\dom\dot\nu=\dot A$.   Of course
$\VVdP\,\dot A\subseteq\dot\frak Z$ just because
$\VVdP\,\dot b_J\in\dot\frak Z$ for every
$J\subseteq\Bbb N$.   In the other direction, if $p\in\frak C^+$ and
$\dot x$ is a $\Bbb P$-name such that $p\VVdP\,\dot x\in\dot\frak Z$, there
are a $q$ stronger than $p$ and a $b\in\frak Z$ such that
$q\VVdP\,\dot x=\dot b$.   Express $q$ as
$c_K$ and $b$ as $b_J$ where
$K\subseteq\Bbb N$ is infinite and $J\subseteq\Bbb N$.   Then there is an
infinite $L\subseteq K$ such that
$\lim_{n\in L,n\to\infty}2^{-n}\#(J\cap I_n)$ is defined, that is,
$(\dot b,r)\in\dot A$, where $r=c_L$.  So
$r\VVdP\,\dot x=\dot b\in\dot A$.   As $p$ and $\dot x$ are arbitrary,

\Centerline{$\VVdP\,\dot\frak Z\subseteq\dot A$ and
$\dom\dot\nu=\dot\frak Z$. \Qed}

\medskip

\quad{\bf (iv)} Of course $\lim_{n\to\Cal F_p}2^{-n}\#(J\cap I_n)$, if it
is defined, must belong to $[0,1]$.   So

\Centerline{$\VVdP\,\dot\nu$ is a function from $\dot\frak Z$ to $[0,1]$.}

\noindent Next, $((\dot 1,\check 1),1)\in\dot\nu$
(if you can work out how to interpret
each $1$ in this formula), so $\VVdP\,\dot\nu\dot 1=\check 1=1$.

\medskip

\quad{\bf (v)} $\VVdP\,\dot\nu$ is additive.   \Prf\ Suppose that
$p\in\frak C^+$ and that $\dot x_0$, $\dot x_1$
are $\Bbb P$-names such that

\Centerline{$p\VVdP\,\dot x_0$, $\dot x_1\in\dot\frak Z$ are disjoint.}

\noindent If $p_1$ is stronger than $p$ there are $q_0$, $q'_0$, $q_1$,
$r\in\Bbb P$,
$J_0$, $J_1\subseteq\Bbb N$ and $\alpha_0$, $\alpha_1\in\Bbb R$ such that

\Centerline{$((\dot b_{J_0},\check\alpha_0),q_0)
\in\dot\nu$,
\quad$q'_0$ is stronger than both $q_0$ and $p_1$,
\quad$q'_0\VVdP\,\dot b_{J_0}=\dot x_0$,}

\Centerline{$((\dot b_{J_1},\check\alpha_1),q_1)
\in\dot\nu$,
\quad$r$ is stronger than both $q_1$ and $q'_0$,
\quad$r\VVdP\,\dot b_{J_1}=\dot x_1$.}

\noindent As $r\VVdP\,(b_{J_0}\Bcap b_{J_1})^{\centerdot}=0$,
$r\Bcap b_{J_0}\Bcap b_{J_1}=0$.   Express $r$ as
$c_K$, where $K\in[\Bbb N]^{\omega}$.
Then $J_0\cap J_1\cap\bigcup_{n\in K}I_n\in\Cal Z$, so
$\lim_{n\to\Cal F_{r}}2^{-n}\#(J_0\cap J_1\cap I_n)=0$.   At the same
time,

\Centerline{$\lim_{n\to\Cal F_{r}}2^{-n}\#(J_0\cap I_n)
=\lim_{n\to\Cal F_{q_0}}2^{-n}\#(J_0\cap I_n)=\alpha_0$,}

\Centerline{$\lim_{n\to\Cal F_{r}}2^{-n}\#(J_1\cap I_n)
=\lim_{n\to\Cal F_{q_1}}2^{-n}\#(J_1\cap I_n)=\alpha_1$,}

\noindent so

$$\eqalign{\lim_{n\to\Cal F_{r}}&2^{-n}\#((J_0\cup J_1)\cap I_n)\cr
&=\lim_{n\to\Cal F_{r}}2^{-n}\#(J_0\cap I_n)
  +\lim_{n\to\Cal F_{r}}2^{-n}\#(J_1\cap I_n)
  -\lim_{n\to\Cal F_{r}}2^{-n}\#(J_0\cap J_1\cap I_n)\cr
&=\alpha_0+\alpha_1\cr}$$

\noindent and
$(((\dot b_{J_0\cup J_1},(\alpha_0+\alpha_1)\var2spcheck),r)\in\dot\nu$.
Accordingly

\Centerline{$r\VVdP\,\dot\nu(\dot x_0\dot{\Bcup}\dot x_1)
=\dot\nu(\dot b_{J_0\cup J_1})
=(\alpha_0+\alpha_1)\var2spcheck
=\check\alpha_0+\check\alpha_1
=\dot\nu(\dot x_0)+\dot\nu(\dot x_1)$,}

\noindent while $r\Bsubseteq p_1$.   As $p_1$ is arbitrary,

\Centerline{$p\VVdP\,\dot\nu(\dot x_0\dot{\Bcup}\dot x_1)
 =\dot\nu(\dot x_0)+\dot\nu(\dot x_1)$.}

\noindent As $p$, $\dot x_0$ and $\dot x_1$ are arbitrary,

\Centerline{$\VVdP\,\dot\nu$ is additive.  \Qed}

\medskip

\quad{\bf (vi)} $\VVdP\,\dot\nu$ is strictly positive.   \Prf\ Let
$p\in\frak C^+$ and a $\Bbb P$-name $\dot x$ be such that
$p\VVdP\,\dot x\in\dot\frak Z$ and $\dot x\ne\dot 0$.
If $q$ is stronger than $p$ there are a $q'$ stronger than $q$ and a
$J\subseteq\Bbb N$ such that
$q'\VVdP\,\dot x=\dot b_J$.   Express $q'$ as
$ c_K$ where $K\in[\Bbb N]^{\omega}$.
As $q'\VVdP\,\dot b_J\ne\dot 0$,
$q'\Bcap b_J\ne 0$ and $\bigcup_{n\in K}I_n\cap J\notin\Cal Z$.
Accordingly $\limsup_{n\in K,n\to\infty}2^{-n}\#(I_n\cap J)>0$ and there is
an infinite $L\subseteq K$ such that
$\alpha=\lim_{n\in L,n\to\infty}2^{-n}\#(I_n\cap J)$ is defined and greater
than $0$.   Set $r=c_L$;  then
$((\dot b_J,\check\alpha),r)\in\dot\nu$, so

\Centerline{$r\VVdP\,\dot\nu(\dot x)=\dot\nu(\dot b_J)
=\check\alpha>0$,}

\noindent while $r$ is stronger than $q$.   As $q$ is arbitrary,
$p\VVdP\,\dot\nu(\dot x)>0$;  as $p$ and $\dot x$ are arbitrary,
$\VVdP\,\dot\nu$ is strictly positive.\ \Qed

\medskip

\quad{\bf (vii)}

$$\eqalign{\VVdP\,\,&\text{if }\sequence{k}{x_k}
\text{ is a non-increasing sequence in }\dot\frak Z\cr
&\mskip20mu\text{ there is an }x\in\dot\frak Z
\text{ such that }x\dot{\Bsubseteq}x_k\text{ for every }k
\text{ and }\dot\nu(x)=\inf_{k\in\Bbb N}\dot\nu(x_k).\cr}$$

\noindent\Prf\ Let $p\in\frak C^+$ and a sequence $\sequence{k}{\dot x_k}$
of $\Bbb P$-names be such that

\Centerline{$p\VVdP\,\sequencen{\dot x_n}$ is a
non-increasing sequence in $\dot\frak Z$.}

\noindent Let $q$ be stronger than $p$.
Then we can choose $\sequence{k}{q_k}$,
$\sequence{k}{q'_k}$, $\sequence{k}{q''_k}$,
$\sequence{k}{J_k}$ and $\sequence{k}{\alpha_k}$
inductively so that $q'_0=q$ and

\Centerline{$q''_k$ is stronger than $q'_k$, $J_k\subseteq\Bbb N$ and
$q''_k\VVdP\,\dot x_k=\dot b_{J_k}$,}

\Centerline{$q_k$ is stronger than $q''_k$, $\alpha_k\in[0,1]$,
$\lim_{n\to\Cal F_{q_k}}2^{-n}\#(J_k\cap I_n)=\alpha_k$}

\noindent (compare (iii) above),

\Centerline{$q'_{k+1}=q_k$}

\noindent for every $k\in\Bbb N$.

Because $\Bbb P$ is countably closed,
there is an $r\in\frak C^+$ stronger than every $q_k$.   In this case,
$((\dot b_{J_k},\check\alpha_k),r)\in\dot\nu$ for every
$k$, so

\Centerline{$r\VVdP\,\inf_{k\in\Bbb N}\dot\nu(\dot x_k)
=\inf_{k\in\Bbb N}\check\alpha_k=\check\alpha$}

\noindent where $\alpha=\inf_{k\in\Bbb N}\alpha_k$.
Express $r$ as $c_K$ where
$K\subseteq\Bbb N$ is infinite.   For each $k\in\Bbb N$,

\Centerline{$r\VVdP\,\dot b_{J_{k+1}}=\dot x_{k+1}\dot{\Bsubseteq}\dot x_k
   =\dot b_{J_k}$,}

\noindent so $r\Bcap b_{J_{k+1}}\Bsetminus b_{J_k}=0$ and

\Centerline{$\lim_{n\in K,n\to\infty}2^{-n}\#(I_n\cap J_{k+1}\setminus J_k)
=0$,}

\noindent while

\Centerline{$\lim_{n\in K,n\to\infty}2^{-n}\#(I_n\cap J_k)
=\lim_{n\to\Cal F_{q_k}}2^{-n}\#(I_n\cap J_k)
=\alpha_k\ge\alpha$.}

\noindent We can therefore find a strictly increasing sequence
$\sequence{k}{n_k}$ in $K$ such that

\Centerline{$2^{-n_k}\#(I_{n_k}\cap J'_k)\ge\alpha-2^{-k}$}

\noindent for every $k$, where $J'_k=\bigcap_{j\le k}J_j$.   Set
$r'=(\bigcup_{k\in\Bbb N}I_{n_k})^{\ssbullet}$ and
$J=\bigcup_{k\in\Bbb N}I_{n_k}\cap J'_k$.
Then $J\setminus J_k$ is finite, so
$r'\VVdP\,\dot b_J\dot{\Bsubseteq}\dot x_k$ for every
$k$.   Also $((\dot b_J,\check\alpha),r')\in\dot\nu$, so

\Centerline{$r'\VVdP\,\dot\nu(\dot b_J)
=\check\alpha=\inf_{k\in\Bbb N}\dot\nu(\dot x_k)$.}

Thus

\Centerline{$r'\VVdP\,\text{ there is a lower bound }x\text{ for }
\{\dot x_k:k\in\Bbb N\}\text{ such that }
\dot\nu(x)=\inf_{k\in\Bbb N}\dot\nu(\dot x_k)$.}

\noindent As $q$ is arbitrary,

\Centerline{$p\VVdP\,\text{ there is a lower bound }x\text{ for }
\{\dot x_k:k\in\Bbb N\}\text{ such that }
\dot\nu(x)=\inf_{k\in\Bbb N}\dot\nu(\dot x_k)$.}

\noindent As $p$ and $\sequence{k}{\dot x_k}$ are arbitrary,

\doubleinset{$\VVdP$ if $\sequence{k}{x_k}$ is a non-increasing sequence
in $\dot\frak Z$ there is an $x\in\dot\frak Z$ such that
$x\dot{\Bsubseteq}x_k$ for every $k$ and
$\dot\nu(x)=\inf_{k\in\Bbb N}\dot\nu(x_k)$.  \Qed}

\medskip

\quad{\bf (viii)}

\doubleinset{$\VVdP$ there is a family $\family{L}{\Cal P\Bbb N}{x_L}$
in $\dot\frak A$ such that $\dot\nu(\inf_{L\in\Cal L}x_L)=2^{-\#(\Cal L)}$
for every finite set $\Cal L\subseteq\Cal P\Bbb N$.}

\noindent\Prf\ Let $\family{L}{\Cal P\Bbb N}{M_L}$ be an almost disjoint
family of infinite subsets of $\Bbb N$ (5A1Fa).   For each $n\in\Bbb N$,
let $\ofamily{i}{n}{K_{ni}}$ be a family of subsets of $I_n$ such that
$\#(\bigcap_{i\in J}K_{ni})=2^{n-\#(J)}$ for every non-empty set
$J\subseteq n$;  such a family exists because $\#(I_n)=2^n$.   For
$L\subseteq\Bbb N$, set

\Centerline{$A_L=\bigcup_{n\in\Bbb N,n>\min M_L}K_{n,\max(n\cap M_L)}$,
\quad$a_L=b_{A_L}\in\frak Z$.}

\noindent If $\Cal L\subseteq\Cal P\Bbb N$ is finite and not empty, let
$n_0\in\Bbb N$ be such that $M_L\cap M_{L'}\subseteq n_0$ whenever $L$,
$L'\in\Cal L$ are distinct, and $n_1\ge n_0$ such that
$M_L\cap n_1\setminus n_0\ne\emptyset$ for every $L\in\Cal L$.
Then $\max(n\cap M_L)\ne\max(n\cap M_{L'})$ whenever $L$, $L'\in\Cal L$ are
distinct and $n\ge n_1$.   So

$$\eqalign{\lim_{n\to\infty}2^{-n}\#(I_n\cap\bigcap_{L\in\Cal L}A_L)
&=\lim_{n\to\infty}
  2^{-n}\#(I_n\cap\bigcap_{L\in\Cal L}K_{n,\max(n\cap M_L)})\cr
&=\lim_{n\to\infty}2^{-n}2^{n-\#(\Cal L)}
=2^{-\#(\Cal L)}.\cr}$$

\noindent Of course the same formula is valid when $\Cal L=\emptyset$.

It follows that

\Centerline{$\VVdP\,\dot\nu(\inf_{L\in\check\Cal L}\dot a_L)
=2^{-\#(\check\Cal L)}$}

\noindent for every finite $\Cal L\subseteq\Cal P\Bbb N$.   Accordingly

\Centerline{$\VVdP\,\dot\nu(\inf_{L\in\Cal L}\dot a_L)=2^{-\#(\Cal L)}$
for every finite set $\Cal L\subseteq(\Cal P\Bbb N)\var2spcheck$.}

\noindent But we know also that

\Centerline{$\VVdP\,\Cal P\Bbb N=(\Cal P\Bbb N)\var2spcheck$}

\noindent (5A3Qb).   So the family $\family{L}{\Cal P\Bbb N}{\dot a_L}$ of
$\Bbb P$-names, when interpreted as a $\Bbb P$-name
$\family{L}{(\Cal P\Bbb N)\var2spcheck}{\dot a_L}$ as in 5A3Eb, can also be
regarded as a $\Bbb P$-name for a function defined on the whole power
set of the set of natural numbers.   If we do this, we get

\Centerline{$\VVdP\,\dot\nu(\inf_{L\in\Cal L}\dot a_L)=2^{-\#(\Cal L)}$
for every finite set $\Cal L\subseteq\Cal P\Bbb N$,}

\noindent witnessing the truth of the result we seek.\ \Qed

\medskip

\quad{\bf (ix)} $\VVdP\,\#(\dot\frak A)\le\frak c$.   \Prf\ Since

\Centerline{$\dot\frak Z=\{(\dot a,1):a\in\frak Z\}
=\{(\dot b_J,1):J\in\Cal P\Bbb N\}$}

\noindent (556Ab), we get

\Centerline{$\VVdP\,\dot\frak Z
=\{\dot b_J:J\in(\Cal P\Bbb N)\var2spcheck\}$,
so $\#(\dot\frak Z)\le\#((\Cal P\Bbb N)\var2spcheck)
\le\#(\Cal P\Bbb N)=\frak c$.  \Qed}

\medskip

{\bf (e)} Assembling the facts in (d), we see that

\Centerline{$\VVdP\,(\dot\frak Z,\dot\nu)$ satisfies the conditions of
556Q with $\kappa=\frak c$, so $\dot\frak Z\cong\frak B_{\frakc}$.}

\noindent But we also have

\Centerline{$\VVdP\,\frak B_{\frakc}$ is isomorphic to
$\frak B_{\Cal P\Bbb N}=\frak B_{(\Cal P\Bbb N)\var2spcheck}
\cong(\frak B_{\Cal P\Bbb N})\var2spcheck$}

\noindent by 556R.   As $\frak C$ is regularly embedded in $\frak Z$,
we can apply 556Fc to see that $\widehat{\frak Z}$
is isomorphic to the Dedekind completed free product
$\frak C\tensorhat\frak B_{\Cal P\Bbb N}$ and therefore to
$(\Cal P\Bbb N/[\Bbb N]^{<\omega})\tensorhat\frak B_{\frakc}$, by (a).

This completes the proof.
}%end of proof of 556S

\exercises{\leader{556X}{Basic exercises (a)}
%\spheader 556Xa
Let $\frak A$ be a Boolean algebra, not $\{0\}$, and
$\frak C$ a Boolean subalgebra of $\frak A$ which is not regularly
embedded;  let $\Bbb P$ be the forcing notion $\frak C^+$, active
downwards, and let $\dot\frak A$ be the forcing
name for $\frak A$ over $\frak C$.   Show that there is an
$a\in\frak A\setminus\{0\}$ such that $\VVdP\,\dot a=0$, where $\dot a$ is
the forcing name for $a$ over $\frak C$.
%556D

\spheader 556Xb Let $\Bbb P$ be a countably closed forcing notion.
(i) Show that $\VVdP\,\omega_1=\check\omega_1$.
(ii) Show that
$\VVdP\,[\check I]^{\le\omega}=([I]^{\le\omega})\var2spcheck$
for every set $I$.
(iii) Let $\frak A$ be a Dedekind $\sigma$-complete Boolean algebra.
Show that $\VVdP\,\check\frak A$ is Dedekind $\sigma$-complete.
(iv) Let $(X,\rho)$ be a complete metric space.   Show that
$\VVdP\,(\check X,\check\rho)$ is a complete metric space.
%556R

\spheader 556Xc Show that the Dedekind completion $\widehat{\frak Z}$ of
the asymptotic density algebra is a homogeneous Boolean algebra.
\Hint{316Q, 316P.}
%556S   316P:  completion of homog is homog.  316Q:  free product of
%homog is homog

\leader{556Y}{Further exercises (a)}
%\spheader 556Ya
Let $\Bbb P$ be a forcing notion, and $\dot\Bbb Q_1$, $\dot\Bbb Q_2$ two
$\Bbb P$-names for forcing notions such that

\Centerline{$\VVdP\,\RO(\dot\Bbb Q_1)\cong\RO(\dot\Bbb Q_2)$.}

\noindent Show that
$\RO(\Bbb P*\dot\Bbb Q_1)\cong\RO(\Bbb P*\dot\Bbb Q_2)$.
%556F

\spheader 556Yb Let $\Bbb P$ and $\Bbb Q$ be forcing notions.   Show that
$\RO(\Bbb P*\check\Bbb Q)\cong\RO(\Bbb P)\tensorhat\RO(\Bbb Q)$.
%556F

\spheader 556Yc
Give an example of a Dedekind $\sigma$-complete Boolean
algebra $\frak A$ with an order-closed subalgebra $\frak C$ such that

\Centerline{$\VVdP\,\dot\frak A$ is not Dedekind $\sigma$-complete,}

\noindent where
$\Bbb P$ is the forcing notion $\frak C^+$, active downwards,
and $\dot\frak A$ is the forcing name for $\frak A$ over $\frak C$.
%556G mt55bits

\spheader 556Yd Show that the argument of 556Q is sufficient to take us
from ($\dagger$) there to Theorem 395N, as well as to 395P.
%556Q

\spheader 556Ye Show that if the Proper Forcing Axiom is true then
the asymptotic density algebra $\frak Z$ is not homogeneous.   \Hint{5A6H.}
%556S

\spheader 556Yf\dvAnew{2015} Let $(\frak A,\bar\mu)$ be a probability
algebra, $\frak C$ a closed subalgebra and $\Bbb P$ the forcing notion
$\frak C^+$ active downwards.   Set $q(t)=-t\ln t$ for $t>0$,
$0$ for $t\le 0$ (cf.\ 385A).
Let $A$ be a finite partition of unity in $\frak A$, and $\dot A$ the
$\Bbb P$-name $\{(\dot a,1):a\in A\}$.
(i) Confirm that the definition of
$q$ can be interpreted in the forcing universe $V^{\Bbb P}$.
(ii) Show
that if $u\in L^0(\frak C)$ then $\VVdP\,(\bar q(u))\sspvec=q(\vec u)$.
(iii) Set $v=\sum_{a\in A}\bar q(P\chi a)$ where $P$ is the conditional
expectation associated with $\frak C$ (cf.\ 385D).   Show that

\Centerline{$\VVdP\,\dot A$ is a partition of unity
in $(\dot{\frak A},\dot{\bar\mu})$ and its entropy is $\vec v$.}

\noindent(iv) Re-examine Lemma 385Ga in the light of this.
}%end of exercises

\endnotes{
\Notesheader{556} I did not positively instruct you to do so in the
introduction to this section, but I expect that most readers will
have passed rather quickly over the nineteen %\query check
$^{\centerdot}$-infested pages up to
556L, and looked at the target theorems from 556M on.   In each of the
first three we have a pair ($\ddagger$), ($\dagger$) of propositions,
($\dagger$)
being the special case of ($\ddagger$) in which an algebra $\Tau$ or
$\frak C$ is trivial.   If, as I hope, you are already acquainted with at
least one of the assertions ($\ddagger$), you will know that it can be
proved by essentially the same methods as the corresponding ($\dagger$),
but with some non-trivial technical changes.   These technical changes,
already incorporated in the proofs of 388L/556N and 395P/556P in
Volume 3, and indicated in \S458 for 458Yd/556M,
certainly do not amount to nineteen pages of mathematics in total;
moreover, they explore ideas which are of independent interest.
So I cannot on this evidence claim that the approach gives quick proofs of
otherwise inaccessible results.

What I do claim is that the general method gives a way of
looking at a recurrent phenomenon.   Throughout the theory of
measure-preserving transformations, ergodic transformations have a special
place;  and one comes to expect that once one has answered a question for
ergodic transformations, the general case will be easy to determine.
Similarly, every theorem about independent random variables ought to have a
form applying to relatively independent variables.   Indeed there are
standard techniques for developing such extensions, based on
disintegrations, as in \S\S458-459.   What I have tried to do here is to
develop a completely different approach, and in the process to indicate a
new aspect of the theory of forcing.   I note that the method demands
preliminary translations into the language of measure algebras,
which suits my prejudices as already expressed at length in Volume 3.

The message is that everything works.   There are no royal roads in
mathematics, and to use this one you will have to master some non-trivial
machinery.   But perhaps just knowing that a machine exists will give
you the confidence to attack similar problems in your own way.   I offer an
example in 556Yf.   Note that this depends on the fact that the ordinary
functions of elementary calculus have definitions which can be interpreted
in any forcing universe.

The great bulk of the work of this section consists of routine checks that
natural formulae are in fact valid.   You will see that some simple ideas
recur repeatedly, but the details demand a certain amount of attention.
At the very beginning, in finding a forcing name $\dot a$ for an
element of a Boolean algebra, we have to take care that we are exactly
following our preferred formulation of what a name `is'.   (If my preferred
formulation is not yours, you have some work to do, but it should not be
difficult, and might be enlightening.)   It is not
surprising that regularly embedded subalgebras have a special status
(556D);  it is worth taking a moment to think about why it matters so much
(556Xa).   In 556H, I do not think it is obvious that $\frak A$ must be
Dedekind complete, rather than just Dedekind $\sigma$-complete,
to make the ideas work in the straightforward way that they do (556Yc).
When we come to measure algebras (556K), we need to be sure that we have a
description of forcing names for real numbers which is compatible with the
apparatus here.   Again and again, we have sentences with clauses in both
the forcing language and in ordinary language, and we must keep the pieces
properly segregated in our minds.

The last fifth of the section (556Q-556S) %556Q 556R 556S
is quite hard work for the result we get, but I think it is particularly
instructive, in that it cannot be regarded as a technical extension of a
simpler and more important
result.   It is a good example of a theorem proved by a method
unavoidably dependent on the Forcing Theorem (5A3D), and for which it is
not at all clear that a proof avoiding forcing can be made manageably
simple.   Such a proof must exist, but the obvious route to it involves
teasing out the requisite parts of the proof of Maharam's theorem, and
translating them into properties of the set

\Centerline{$\{(J,\alpha,K):K\in[\Bbb N]^{\omega}$,
  $J\subseteq\Bbb N$,
  $\lim_{n\in K,n\to\infty}2^{-n}\#(J\cap I_n)=\alpha\}$}

\noindent as in part (d) of the proof of 556S, but going very much farther.
My own experience is that facing up to such challenges is often profitable,
but for the moment I am happy to present an adaptation of Farah's
original proof.

An easy corollary of Theorem 556S is that $\widehat{\frak Z}$ is
homogeneous (556Xc).   This is striking in view of the fact that $\frak Z$
itself may or may not be homogeneous.   If the continuum hypothesis is
true, then $\frak Z$ is indeed homogeneous ({\smc Farah 03}); %8.2
but if
the Proper Forcing Axiom is true, then $\frak Z$ is {\it not} homogeneous
(556Ye), even though its completion is.
}%end of notes

\discrpage



%}}}}}

