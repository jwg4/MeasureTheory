\frfilename{mt5a2.tex}
\versiondate{6.6.11}
\copyrightdate{2007}

\def\chaptername{Appendix}
\def\sectionname{Pcf theory}

\newsection{5A2}

In \S\S542-543 I call on some results from Shelah's pcf theory.   As
I have still not found any satisfactory textbook for this material, I copy out
part of the appendix of {\smc Fremlin 93}, itself drawn largely from
{\smc Burke \& Magidor 90}.

\leader{5A2A}{Reduced products}\cmmnt{ We need the following
elementary generalization of the construction in 351M.}   Let
$\familyiI{P_i}$ be a family of partially ordered sets with product
$P$.

\spheader 5A2Aa Let $\Cal F$ be a filter on $I$.   We have an equivalence
relation $\equiv_{\Cal F}$ on $P$, given by saying that
$f\equiv_{\Cal F}g$ if $\{i:f(i)=g(i)\}\in \Cal F$.   I write
$P|\Cal F$
for the set of equivalence classes under this relation, the
{\bf partial order reduced product} of $\langle P_i\rangle_{i\in I}$ modulo
$\Cal F$.   Now $P|\Cal F$ is again a partially ordered set, writing

\Centerline{$f^{\bullet}\le g^{\bullet}
\iff f\le_{\Cal F}g\iff\{i:f(i)\le g(i)\}\in\Cal F$.}

\noindent Observe that if every $P_i$ is totally ordered and $\Cal F$
is an ultrafilter, then $P|\Cal F$ is totally ordered.

\spheader 5A2Ab Suppose that $P_i\ne\emptyset$ for every $i\in I$.
For any filter $\Cal F$ on $I$ we have

%$$\eqalign{\min_{i\in I}\add P_i
%&=\add P
%\le\sup_{F\in\Cal F}\add(\prod_{i\in F}P_i)\cr
%&=\sup_{F\in\Cal F}\min_{i\in F}\add P_i
%\le\add(P|\Cal F),\cr}$$

\Centerline{$\min_{i\in I}\add P_i=\add P
\le\sup_{F\in\Cal F}\add(\prod_{i\in F}P_i)
=\sup_{F\in\Cal F}\min_{i\in F}\add P_i
\le\add(P|\Cal F)$,}

\Centerline{$\cf(P|\Cal F)
\le\min_{F\in\Cal F}\cf(\prod_{i\in F}P_i)\le\cf P$.}

\noindent\prooflet{\Prf\ By 511Hg,
$\add(\prod_{i\in F}P_i)=\min_{i\in F}\add P_i$ for any $F\in\Cal F$,
and in
particular when $F=I$.   For $p\in P|\Cal F$ choose $f_p\in P$ such that
$f_p^{\ssbullet}=p$.   If $F\in\Cal F$ then $p\mapsto f_p\restr F$ is a
Tukey function from $P|\Cal F$ to $\prod_{i\in F}P_i$, so
$P|\Cal F\prT\prod_{i\in F}P_i$ and 513Ee tells us that
$\add(\prod_{i\in F}P_i)\le\add(P|\Cal F)$ and
$\cf(P|\Cal F)\le\cf(\prod_{i\in F}P_i)$.   Also $f\mapsto f\restr F$ is a
dual Tukey function from $P$ to $\prod_{i\in F}P_i$, so
$\prod_{i\in F}P_i\prT P$ and $\cf(\prod_{i\in I}P_i)\le\cf P$.\ \Qed}

\spheader 5A2Ac Note that if $\Cal F$, $\Cal G$ are filters on $I$
and $\Cal F\subseteq\Cal G$, then $\add(P|\Cal F)\le\add(P|\Cal G)$
and $\cf(P|\Cal F)\ge\cf(P|\Cal G)$.   \prooflet{\Prf\ If
$f\le_{\Cal F}g$ then $f\le_{\Cal G}g$.   So we have a canonical surjective
order-preserving map $\psi:P|\Cal F\to P|\Cal G$ given by saying that
$\psi(\pi_{\Cal F}(f))=\pi_{\Cal G}(f)$ for every $f\in P$, where
$\pi_{\Cal F}(f)$, $\pi_{\Cal G}(f)$ are the equivalence classes of $f$ in
$P|\Cal F$ and $P|\Cal G$ respectively.   By
513E(b-iii), $\psi$ is a dual Tukey function, so
$P|\Cal G\prT P|\Cal F$ and we can use 513Ee again.\ \Qed}
%5{}42D

\leader{5A2B}{Theorem} %RVMC A1I
Let $\lambda>0$ be a cardinal and
$\langle\theta_{\zeta}\rangle_{\zeta<\lambda}$ a family of regular
infinite cardinals, all greater than $\lambda$.   Set
$P=\prod_{\zeta<\lambda}\theta_{\zeta}$.
For any filter $\Cal F$ on $\lambda$, let $P|\Cal F$ be the corresponding
reduced product and $\pi_{\Cal F}:P\to P|\Cal F$
the canonical map.   For any cardinal $\delta$ set

\Centerline{$\frak F_{\delta}=\{\Cal F:\Cal F$ is an ultrafilter on
$\lambda$, $\cf(P|\Cal F)=\delta\},$}

\Centerline{$\frak F^*_{\delta}
=\bigcup_{\delta'\ge\delta}\frak F_{\delta'}$;}

\noindent if $\frak F^*_{\delta}\ne\emptyset$, let $\Cal G_{\delta}$
be the filter $\bigcap\frak F^*_{\delta}$.   Now

(a) if $\frak F^*_{\delta}\ne\emptyset$, then
$\add(P|\Cal G_{\delta})\ge\delta$;

(b) for every $\delta$ there is a set
$F\in[P]^{\le\delta}$ such that $\pi_{\Cal F}[F]$ is
cofinal with $P|\Cal F$ for every $F\in\frak F_{\delta}$;

(c) $\frak F_{\cf P}\ne\emptyset$.

\proof{ The case of finite $\lambda$ is trivial throughout,
as then

\Centerline{$\cf P=\max_{\zeta<\lambda}\theta_{\zeta}$,}

\Centerline{$\frak F_{\delta}
=\{\Cal F:$ there is a $\zeta<\lambda$ such that
$\{\zeta\}\in\Cal F$ and $\theta_{\zeta}=\delta\}$,}

\Centerline{$\frak F^*_{\delta}
=\{\Cal F:\{\zeta:\theta_{\zeta}\ge\delta\}\in\Cal F\}$,}

\Centerline{$\Cal G_{\delta}
=\{G:\{\zeta:\theta_{\zeta}\ge\delta\}\subseteq G\subseteq\lambda\}$.}

\noindent So henceforth let us take it that $\lambda$ is infinite.

If $\Cal F$ is an ultrafilter on $\lambda$, then
$P|\Cal F$ is a non-empty totally ordered set with no
greatest member, so its additivity and cofinality are the same;
thus

\Centerline{$\frak F_{\delta}=\{\Cal F:\Cal F$ is an ultrafilter on
$\lambda$, $\add(P|\Cal F)=\delta\}$}

\noindent for every $\delta$, and

\Centerline{$\min_{\zeta\in F}\theta_{\zeta}
\le\delta\le\cf(\prod_{\zeta\in F}\theta_{\zeta})$}

\noindent whenever $F\in\Cal F\in\frak F_{\delta}$, by 5A2Ab.

Write
$L=\{\zeta:\zeta<\lambda$, $\theta_{\zeta}=\lambda^+\}$,
$M=\lambda\setminus L$.   If $\Cal F$ is an ultrafilter on $\lambda$
and $L\in\Cal F$, then
$\cf(\prod_{\zeta\in L}\theta_{\zeta})=\lambda^+$, because the set of
constant functions is cofinal with $\prod_{\zeta\in L}\theta_{\zeta}$,
so $\cf(P|\Cal F)$ must be $\lambda^+$;  otherwise, $M\in\Cal F$ and
$\cf(P|\Cal F)>\lambda^+$.

\medskip

{\bf (a)} Set $\delta'=\add(P|\Cal G_{\delta})$.

\medskip

\quad{\bf (i)} $\delta'$ is a regular infinite cardinal (513C(a-i)) and

\Centerline{$\delta'\ge\min_{\zeta<\lambda}\theta_{\zeta}
>\lambda$}

\noindent by 5A2Ab again.
If $\delta=\lambda^+$ then of course $\delta'\ge\delta$;  so
suppose that $\delta>\lambda^+$.   In this case $L\notin\Cal F$ for
any $\Cal F\in\frak F^*_{\delta}$, so
$M\in\Cal G_{\delta}$ and
$\delta'\ge\min_{\zeta\in M}\theta_{\zeta}>\lambda^+$.

\medskip

\quad{\bf (ii)} \Quer\ If $\delta'<\delta$ then (translating 513C(a-i)
into a statement about the pre-order
$\le_{\Cal G_{\delta}}$) there is a family
$\langle f_{\alpha}\rangle_{\alpha<\delta'}$ in $P$ such that
$f_{\alpha}\le_{\Cal G_{\delta}}f_{\beta}$ whenever
$\alpha\le\beta<\delta'$ but
there is no $f\in P$ such that $f_{\alpha}\le_{\Cal G_{\delta}}f$
for every $\alpha<\delta'$.   Choose $h_{\xi}\in P$,
$\alpha_{\xi}<\delta'$
inductively, for $\xi<\lambda^+$, as follows.  $h_0=f_0$.
Given $h_{\xi}$, set

\Centerline{$B_{\xi\alpha}
=\{\zeta:\zeta\in M,\,h_{\xi}(\zeta)\ge f_{\alpha}(\zeta)\}$}

\noindent for each $\alpha<\delta'$;  let
$\alpha_{\xi}<\delta'$ be such that
$f_{\alpha_{\xi}}\not\le_{\Cal G_{\delta}}h_{\xi}$,
so that $B_{\xi\alpha}\notin\Cal G_{\delta}$ when
$\alpha_{\xi}\le\alpha<\delta'$.   Choose
$\Cal F_{\xi}\in\frak F^*_{\delta}$ such that
$B_{\xi,\alpha_{\xi}}\notin\Cal F_{\xi}$.
Now, because $\cf(P|\Cal F_{\xi})\ge\delta>\delta'$, there
is an $h_{\xi+1}\in P$ such that $f_{\alpha}\le_{\Cal F_{\xi}}h_{\xi+1}$
for every $\alpha<\delta'$;  we may take $h_{\xi+1}\ge h_{\xi}$.

For non-zero limit ordinals $\xi<\lambda^+$ take
$h_{\xi}(\zeta)=\sup_{\eta<\xi}h_{\eta}(\zeta)$ for every
$\zeta<\lambda$.

Set $\alpha=\sup_{\xi<\lambda^+}\alpha_{\xi}<\delta'$.   Then
$\langle B_{\xi\alpha}\rangle_{\xi<\lambda^+}$ is a non-decreasing
family in $\Cal P\lambda$.   So there must be a $\xi<\lambda^+$ such
that $B_{\xi\alpha}=B_{\xi+1,\alpha}$.

By the choice of $h_{\xi+1}$, $B_{\xi+1,\alpha}\in\Cal F_{\xi}$.
So $B_{\xi\alpha}\in\Cal F_{\xi}$ and
$f_{\alpha}\le_{\Cal F_{\xi}}h_{\xi}$.   Because $\alpha\ge\alpha_{\xi}$,
$f_{\alpha_{\xi}}\le_{\Cal G_{\delta}}f_{\alpha}$, so

\Centerline{$f_{\alpha_{\xi}}\le_{\Cal F_{\xi}}f_{\alpha}
\le_{\Cal F_{\xi}}h_{\xi}$}

\noindent and $B_{\xi,\alpha_{\xi}}\in\Cal F_{\xi}$;  contrary to the
choice of $\Cal F_{\xi}$.\ \Bang

\medskip

{\bf (b)(i)}
If $\frak F_{\delta}=\emptyset$, we can take $F=\emptyset$;  so
suppose that $\frak F_{\delta}$ is non-empty.
As in (a-i) above, we must have
$\delta\ge\min_{\zeta<\lambda}\theta_{\zeta}>\lambda$, and the case
$\delta=\lambda^+$ is again elementary.   \Prf\ Take $F$ to be the set
of constant functions with values less than $\lambda^+$.   If
$\Cal F\in\frak F_{\delta}$ then $M\notin\Cal F$ and $L\in\Cal F$.
So for any $h\in P$ we have $\alpha=\sup_{\zeta\in L}h(\zeta)<\lambda^+$,
and there is an $f\in F$ such that $f(\zeta)=\alpha$ for every
$\zeta$, in which case $h\le_{\Cal F}f$;  thus $\pi_{\Cal F}[F]$ is cofinal
with $P|\Cal F$, as required.\ \Qed

So suppose from now on
that $\delta>\lambda^+$, so that $M\in\Cal F$ for every
$\Cal F\in\frak F_{\delta}$.   Of course $\delta$, being the
cofinality of a non-empty totally ordered set
with no greatest member, is regular (511He-511Hf, 513C(a-i)).

\medskip

\quad{\bf (ii)} \Quer\ Suppose, if possible, that there is no $F$ of the
required type.   In this case, we can find families
$\langle f_{\xi\alpha}\rangle_{\xi<\lambda^+,\alpha<\delta}$ in $P$ and
$\langle\Cal F_{\xi}\rangle_{\xi<\lambda^+}$ in $\frak F_{\delta}$ such
that

\inset{($\alpha$) $f_{\eta\alpha}\le_{\Cal F_{\xi}}f_{\xi 0}$ whenever
$\alpha<\delta$, $\eta<\xi<\lambda^+$;

($\beta$) $\{\pi_{\Cal F_{\xi}}(f_{\xi\alpha}):\alpha<\delta\}$
is cofinal with $P|\Cal F_{\xi}$ for every $\xi<\lambda^+$;

($\gamma$) $f_{\eta\alpha}\le f_{\xi\alpha}$ whenever
$\alpha<\delta$, $\eta\le\xi<\lambda^+$;

($\delta$) if $\xi<\lambda^+$, $\alpha<\delta$ and
$\cf\alpha=\lambda^+$ then

\Centerline{$f_{\xi\alpha}(\zeta)
=\min\{\sup_{\beta\in C}f_{\xi\beta}(\zeta):C
  \text{ is a closed cofinal set in }\alpha\}$}

\noindent for every $\zeta\in M$;

($\epsilon$) $f_{\xi\beta}\le_{\Cal F_{\xi}}f_{\xi\alpha}$
whenever $\xi<\lambda^+$, $\beta\le\alpha<\delta$.}

\noindent\Prf\ Construct
$\langle f_{\xi\alpha}\rangle_{\xi<\lambda^+,\alpha<\delta}$ inductively,
taking $\lambda^+\times\delta$ with its lexicographic well-ordering (that
is, $(\xi,\alpha)\le(\eta,\beta)$ if either $\xi<\eta$ or $\xi=\eta$ and
$\alpha<\beta$).   Given that
$\langle f_{\eta\beta}\rangle_{(\eta,\beta)<(\xi,\alpha)}$ satisfies the
inductive hypothesis so far, proceed according to the nature of $\alpha$,
as follows.

{\it Zero} If $\alpha=0$, then, because $\#(\xi\times\delta)\le\delta$, the
counter-hypothesis tells us that there is an
$\Cal F_{\xi}\in\frak F_{\delta}$ such that
$\{\pi_{\Cal F_{\xi}}(f_{\eta\beta}):\eta<\xi$, $\beta<\delta\}$ is
not cofinal with $P|\Cal F_{\xi}$.
Accordingly we can find $f_{\xi 0}\in P$ such that

\Centerline{$f_{\eta\alpha}\le_{\Cal F_{\xi}}f_{\xi 0}$ whenever
$\eta<\xi$ and $\alpha<\delta$,}

\noindent and because $\add P\ge\delta>\#(\xi)$, we can also insist that

\Centerline{$f_{\eta 0}\le f_{\xi 0}$ whenever $\eta<\xi$.}

\noindent Now take a family
$\langle g_{\xi\beta}\rangle_{\beta<\delta}$ in $P$ such that
$\{\pi_{\Cal F_{\xi}}(g_{\xi\beta}):\beta<\delta\}$ is cofinal with
$P|\Cal F_{\xi}$.

{\it Successor}  If $\alpha=\beta+1$ is a successor ordinal, set

\Centerline{$f_{\xi\alpha}(\zeta)
=\max(f_{\xi\beta}(\zeta),g_{\xi\beta}(\zeta),
    \sup_{\eta<\xi}f_{\eta\alpha}(\zeta))$ for every $\zeta<\lambda$;}

\noindent this is acceptable because $\cf\theta_{\zeta}>\lambda$ for every
$\zeta$.

{\it Cofinality $\lambda^+$} If
$\cf\alpha=\lambda^+$, set

$$\eqalign{f_{\xi\alpha}(\zeta)
&=\sup_{\eta<\xi}f_{\eta\alpha}(\zeta)\text{ if }\zeta\in L,\cr
&=\min\{\sup_{\beta\in C}f_{\xi\beta}(\zeta):
  C\text{ is a closed cofinal set in }\alpha\}
  \text{ if }\zeta\in M.\cr}$$

\noindent This time, note that if $\zeta\in M$, then
$f_{\xi\alpha}(\zeta)<\theta_{\zeta}$ because
there is a closed cofinal set in $\alpha$ with cardinal
$\lambda^+<\theta_{\zeta}$.

{\it Otherwise} If $\alpha$ is a non-zero
limit ordinal and $\cf\alpha\ne\lambda^+$, choose $f_{\xi\alpha}$ such that

\Centerline{$f_{\eta\alpha}\le f_{\xi\alpha}$ for every $\eta<\xi$,}

\Centerline{$f_{\xi\beta}\le_{\Cal F_{\xi}}f_{\xi\alpha}$ for every
$\beta<\alpha$;}

\noindent this is possible because $\add P$ and $\add(P|\Cal F_{\xi})$ are
both at least $\lambda^+>\max(\#(\xi),\#(\alpha))$.

Now let us work through the list of conditions to be satisfied.

($\alpha$) is written into the case $\alpha=0$ of the induction.

($\beta$) Because $g_{\xi\alpha}\le f_{\xi,\alpha+1}$ for every $\alpha$
and $\{\pi_{\Cal F_{\xi}}(g_{\xi\alpha}):\alpha<\delta\}$ is cofinal with
$P|\Cal F_{\xi}$,
$\{\pi_{\Cal F_{\xi}}(f_{\xi\alpha}):\alpha<\delta\}$ is cofinal with
$P|\Cal F_{\xi}$.

($\gamma$) The construction ensures that we shall have
$f_{\eta\alpha}(\zeta)\le f_{\xi\alpha}(\zeta)$ in all the required cases
except possibly when $\cf\alpha=\lambda^+$ and $\zeta\in M$.   But in this
case, taking $\eta<\xi$ and a closed cofinal set $C\subseteq\alpha$
such that $f_{\xi\alpha}(\zeta)=\sup_{\beta\in C}f_{\xi\beta}(\zeta)$, the
inductive hypothesis will assure us that

\Centerline{$f_{\eta\alpha}(\zeta)
\le\sup_{\beta\in C}f_{\eta\beta}(\zeta)
\le\sup_{\beta\in C}f_{\xi\beta}(\zeta)=f_{\xi\alpha}(\zeta)$,}

\noindent so there is no problem.

($\delta$) is written into the formula for the inductive step when
$\cf\alpha=\lambda^+$.

($\epsilon$) We certainly have $f_{\xi\alpha}\le f_{\xi,\alpha+1}$, so
$f_{\xi\alpha}\le_{\Cal F_{\xi}}f_{\xi,\alpha+1}$, for
every $\alpha$.   If $\cf\alpha=\lambda^+$, then
because the intersection of
fewer than $\cf\alpha$ closed cofinal subsets of $\alpha$ is again a
closed cofinal set in $\alpha$ (4A1Bd), there
will be a closed cofinal set $C\subseteq\alpha$ such that
$f_{\xi\alpha}(\zeta)=\sup_{\beta\in C}f_{\xi\beta}(\zeta)$ for every
$\zeta\in M$.   So $f_{\xi\beta}\le_{\Cal F_{\xi}}f_{\xi\alpha}$ for every
$\beta\in C$;  by the inductive hypothesis,
$f_{\xi\beta}\le_{\Cal F_{\xi}}f_{\xi\alpha}$ for every
$\beta<\alpha$.   For other limit ordinals $\alpha$, we have
$f_{\xi\beta}\le_{\Cal F_{\xi}}f_{\xi\alpha}$ for every
$\beta<\alpha$ directly from the choice of $f_{\xi\alpha}$.

So the procedure works.\ \Qed

\medskip

\quad{\bf (iii)} The next step is to find a non-decreasing family
$\langle h_{\eta}\rangle_{\eta<\lambda^+}$ in $P$ and a strictly
increasing family
$\langle\gamma(\eta)\rangle_{\eta<\lambda^+}$ in $\delta$ such that

\inset{$\gamma(\eta)=\sup_{\eta'<\eta}\gamma(\eta')$ whenever
$\eta<\lambda^+$ is a limit ordinal (so $\gamma(0)=0$);

$f_{\xi,\gamma(\eta)}(\zeta)<h_{\eta}(\zeta)$ whenever
$\xi$, $\eta<\lambda^+$ and $\zeta\in M$ (choosing $h_{\eta}$);

$h_{\eta}\le_{\Cal F_{\xi}}f_{\xi,\gamma(\eta+1)}$ whenever
$\xi$, $\eta<\lambda^+$ (choosing $\gamma(\eta+1)$).}

\noindent Set $h(\zeta)=\sup_{\eta<\lambda^+}h_{\eta}(\zeta)$
for $\zeta\in M$, $h(\zeta)=0$ for $\zeta\in L$,
$\alpha=\sup_{\eta<\lambda^+}\gamma(\eta)<\delta$ (because
$\delta=\cf\delta>\lambda^+$);  then $\cf\alpha=\lambda^+$ and
$\{\gamma(\eta):\eta<\lambda^+\}$ is a closed cofinal set in
$\alpha$.   Thus

\Centerline{$f_{\xi\alpha}(\zeta)
\le\sup_{\eta<\lambda^+}f_{\xi,\gamma(\eta)}(\zeta)\le h(\zeta)$}

\noindent for every $\xi<\lambda^+$ and $\zeta\in M$, by (ii-$\delta$).
So if we set

\Centerline{$A_{\xi}=\{\zeta:\zeta\in M,\,
f_{\xi\alpha}(\zeta)=h(\zeta)\}$}

\noindent for each $\xi<\lambda^+$, we shall have
$A_{\eta}\subseteq A_{\xi}$ whenever
$\eta\le\xi<\lambda^+$, by (ii-$\gamma$).

\medskip

\quad{\bf (iv)} As $\#(M)\le\lambda$, there must be some
$\xi<\lambda^+$ such that $A_{\xi}=A_{\xi+1}$.   Let
$C\subseteq\alpha$ be a closed cofinal set such that

\Centerline{$f_{\xi+1,\alpha}(\zeta)
=\sup_{\beta\in C}f_{\xi+1,\beta}(\zeta)$}

\noindent for every $\zeta\in M$.
Set $C'=\gamma^{-1}[C]$.   Then $C'$ is a closed cofinal subset of
$\lambda^+$.   \Prf\ It is closed because $\gamma:\lambda^+\to\alpha$ is
order-continuous, therefore continuous (4A2Ro).   Next, $\gamma[\lambda^+]$
is closed and cofinal in
$\alpha$, while $\cf\alpha=\lambda^+$ is uncountable, so
$C\cap\gamma[\lambda^+]$ is cofinal with $\alpha$ and $\gamma[\lambda^+]$
(4A1Bd again), and $C'$ is cofinal with $\lambda^+$.\ \QeD\

For each $\eta\in C'$ write $\eta'$ for the next member of $C'$
greater than $\eta$;  then

\Centerline{$h_{\eta}\le_{\Cal F_{\xi+1}}f_{\xi+1,\gamma(\eta+1)}
\le_{\Cal F_{\xi+1}}f_{\xi+1,\gamma(\eta')}$,}

\Centerline{$f_{\xi\alpha}\le_{\Cal F_{\xi+1}}f_{\xi+1,0}
=f_{\xi+1,\gamma(0)}$}

\noindent so there is a $\zeta_{\eta}\in M$ such that

\Centerline{$h_{\eta}(\zeta_{\eta})
\le f_{\xi+1,\gamma(\eta')}(\zeta_{\eta})$,
\quad$f_{\xi\alpha}(\zeta_{\eta})\le f_{\xi+1,\gamma(0)}(\zeta_{\eta})
<h_0(\zeta_{\eta})\le h(\zeta_{\eta})$.}

\noindent Let $\zeta\in M$ be such that

\Centerline{$B=\{\eta:\eta\in C',\,\zeta_{\eta}=\zeta\}$}

\noindent is cofinal with $\lambda^+$.
Then $f_{\xi\alpha}(\zeta)<h(\zeta)$ so
$\zeta\notin A_{\xi}$.   On the other hand,

\Centerline{$f_{\xi+1,\alpha}(\zeta)
=\sup_{\beta\in C}f_{\xi+1,\beta}(\zeta)
\ge\sup_{\eta\in B}f_{\xi+1,\gamma(\eta')}(\zeta)
\ge\sup_{\eta\in B}h_{\eta}(\zeta)=h(\zeta)$}

\noindent because $\ofamily{\eta}{\lambda^+}{h_{\eta}}$ is non-decreasing.
So $\zeta\in A_{\xi+1}$;  which is impossible.\ \Bang

This contradiction completes the proof of (b).

\medskip

{\bf (c)(i)} Set $\Delta=\{\delta:\frak F_{\delta}\ne\emptyset\}$,
$\Cal G=\bigcup_{\delta\in\Delta}\Cal G_{\delta}$.   Since
$\Cal G_{\delta}\subseteq\Cal G_{\delta'}$ for $\delta\le\delta'$ in
$\Delta$,
$\Cal G$ is a filter on $\lambda$ and there is an ultrafilter
$\Cal H$ on $\lambda$ including $\Cal G$.   For any
$\delta\in\Delta$, $\Cal H\supseteq\Cal G_{\delta}$, so

\Centerline{$\cf(P|\Cal H)=\add(P|\Cal H)
\ge\add(P|\Cal G_{\delta})\ge\delta$,}

\noindent using 5A2Ac and (a) above.   Consequently
$\delta^*=\cf(P|\Cal H)$ is the greatest element of $\Delta$.

\medskip

\quad{\bf (ii)} For each $\delta\le\delta^*$ choose a set
$F_{\delta}\in[P]^{\le\delta}$ such that $\pi_{\Cal F}[F_{\delta}]$ is
cofinal with
$P|\Cal F$ for every $\Cal F\in\frak F_{\delta}$ (using (b) above).
Set $F=\bigcup_{\delta\in\Delta}F_{\delta}$ and

\Centerline{$G=\{\sup I:I\in[F]^{<\omega}\}\subseteq P$.}

\noindent Then $\#(G)\le\delta^*$.   I claim that $G$ is cofinal with
$P$.   \Prf\Quer\ Suppose otherwise;  take $h\in P$ such
that $h\not\le g$ for every $g\in G$.   Write

\Centerline{$A_g=\{\zeta:h(\zeta)>g(\zeta)\}$}

\noindent for each $g\in G$.   Because $G$ is upwards-directed,
$\{A_g:g\in G\}$ is a filter base, and there is an ultrafilter $\Cal
F$
on $\lambda$ containing every $A_g$.   Now there is a
$\delta\in\Delta$
such that $\Cal F\in\frak F_{\delta}$, so that
$\pi_{\Cal F}[F_{\delta}]$
is cofinal with $P|\Cal F$, and there is an $f\in F_{\delta}$
such that $h\le_{\Cal F}f$.   But in this
case $A=\{\zeta:h(\zeta)\le f(\zeta)\}$
and $A_f=\lambda\setminus A$ both belong to $\Cal F$.\ \Bang\Qed

\medskip

\quad{\bf (iii)}
Accordingly $\cf P\le\#(G)\le\delta^*$.   But also of course
$\delta^*=\cf(P|\Cal H)\le\cf P$,
so $\delta^*=\cf P$.   Now we
have $\Cal H\in\frak F_{\delta^*}=\frak F_{\cf P}$.
}%end of proof of 5A2B

\leader{5A2C}{Theorem} Let $\lambda>0$ be a cardinal and
$\langle\theta_{\zeta}\rangle_{\zeta<\lambda}$ a family of regular
infinite cardinals, all greater than $\lambda$.   Set
$P=\prod_{\zeta<\lambda}\theta_{\zeta}$.
Let $\Cal F$ be an ultrafilter on $\lambda$
and $\kappa$ a regular
infinite cardinal with $\lambda<\kappa\le\cf(P|\Cal F)$.
Then there is a
family $\langle\theta'_{\zeta}\rangle_{\zeta<\lambda}$ of regular
infinite cardinals such that $\lambda<\theta'_{\zeta}\le\theta_{\zeta}$ for
every $\zeta<\lambda$ and $\cf(P'|\Cal F)=\kappa$, where
$P'=\prod_{\zeta<\lambda}\theta'_{\zeta}$.

\proof{{\bf (a)}
If $\lambda$ is finite, there is a $\zeta<\lambda$ such that
$\{\zeta\}\in\Cal F$, $\cf(P|\Cal F_{\zeta})=\theta_{\zeta}$ and we just
have to take $\theta'_{\zeta}=\kappa$;   
if $\kappa=\lambda^+$ we may take $\theta'_{\zeta}=\lambda^+$ for
every $\zeta$;  if $\kappa=\cf(P|\Cal F)$ we may take
$\theta'_{\zeta}=\theta_{\zeta}$;  so let us assume that
$\omega_1\le\lambda^+<\kappa<\cf(P|\Cal F)$.   In this case
$M=\{\zeta:\zeta<\lambda$, $\theta_{\zeta}>\lambda^+\}$ must belong
to $\Cal F$.

\medskip

{\bf (b)} For each ordinal $\gamma<\kappa$ choose a
relatively closed cofinal set $C_{\gamma}\subseteq\gamma$ with
$\text{otp}(C_{\gamma})=\cf\gamma$.
Choose families $\langle f_{\alpha}\rangle_{\alpha<\kappa}$,
$\langle g_{\alpha\gamma}\rangle_{\alpha,\gamma<\kappa}$ in $P$
inductively, as follows.
Given $\langle f_{\beta}\rangle_{\beta<\alpha}$, where
$\alpha<\kappa$,
and $\gamma<\kappa$, define $g_{\alpha\gamma}\in P$ by setting

$$\eqalign{g_{\alpha\gamma}(\zeta)
&=\sup\{f_{\beta}(\zeta):\beta\in C_{\gamma}\cap\alpha\}+1
\text{ if this is less than }\theta_{\zeta},\cr
&=0\text{ otherwise}.\cr}$$

\noindent Now choose $f_{\alpha}\in P$ such that

\Centerline{$f_{\beta}\le_{\Cal F}f_{\alpha}\enskip\forall\enskip
\beta<\alpha$,
\quad$g_{\alpha\gamma}\le_{\Cal F}f_{\alpha}\enskip\forall\enskip
\gamma<\kappa$;}

\noindent this is possible because $\kappa<\cf(P|\Cal F)$.
Observe that if $\alpha=\beta+1$ then $C_{\alpha}=\{\beta\}$ so that
$g_{\alpha\alpha}=f_{\beta}+1$ and
$f_{\alpha}\not\le_{\Cal F}f_{\beta}$.   Continue.

\wheader{5A2C}{6}{2}{2}{36pt}

{\bf (c)} Suppose that for each $\zeta<\lambda$ we are given a set
$S_{\zeta}\subseteq \theta_{\zeta}$ with $\#(S_{\zeta})\le\lambda$.
Then there is an $\alpha<\kappa$ such that

\Centerline{for every $h\in\prod_{\zeta<\lambda}S_{\zeta}$,
if $f_{\alpha}\le_{\Cal F}h$ then
$f_{\beta}\le_{\Cal F}h$ for every $\beta<\kappa$.}

\noindent\Prf\Quer\ If not, then (because $\kappa$ is regular) we can
find a family $\langle h_{\xi}\rangle_{\xi<\kappa}$ in
$\prod_{\zeta<\lambda}S_{\zeta}$ and a strictly increasing family
$\langle\phi(\xi)\rangle_{\xi<\kappa}$ in $\kappa$ such that

\Centerline{$f_{\phi(\xi)}\le_{\Cal F}h_{\xi}
\le_{\Cal F}f_{\phi(\xi+1)}$ for all $\xi<\kappa$,}

\Centerline{$\phi(\xi)=\sup_{\eta<\xi}\phi(\eta)$ for limit ordinals
$\xi<\kappa$.}

\noindent Set

\Centerline{$C=\{\xi:\xi<\kappa,\,\phi(\xi)=\xi\}$,}

\noindent so that $C$ is a closed cofinal set in $\kappa$.   Let
$\alpha\in C$ be such that $\alpha=\sup(C\cap\alpha)$ and
$\cf\alpha=\lambda^+$.
Then (because $\lambda^+\ge\omega_1$) $C\cap C_{\alpha}$ is cofinal
with $\alpha$.

For $\beta\in C\cap C_{\alpha}$ and $\zeta<\lambda$ we have

\Centerline{$\#(C_{\alpha}\cap\beta)
\le\text{otp}(C_{\alpha}\cap\beta)<\text{otp}(C_{\alpha})
=\lambda^+\le \theta_{\zeta}$,}

\noindent so

\Centerline{$\theta_{\zeta}
>\sup_{\xi\in C_{\alpha}\cap\beta}f_{\xi}(\zeta)+1
=g_{\beta\alpha}(\zeta)$.}

\noindent Now

\Centerline{$g_{\beta\alpha}\le_{\Cal F}f_{\beta}
=f_{\phi(\beta)}\le_{\Cal F}h_{\beta}\le_{\Cal F}f_{\phi(\beta+1)}
\le_{\Cal F}f_{\beta'}$,}

\noindent where $\beta'$ is the next member of $C\cap C_{\alpha}$
greater than $\beta$.
So there is a $\zeta_{\beta}<\lambda$ such that

\Centerline{$g_{\beta\alpha}(\zeta_{\beta})
\le h_{\beta}(\zeta_{\beta})\le f_{\beta'}(\zeta_{\beta})$.}

\noindent Because $\lambda<\cf\alpha$ there is a $\zeta<\lambda$
such that

\Centerline{$B=\{\beta:\beta\in C\cap C_{\alpha},\,
\zeta_{\beta}=\zeta\}$}

\noindent is cofinal with $\alpha$.   But now observe that if
$\beta$, $\gamma\in B$ and $\beta'<\gamma$ then
$\beta'\in C\cap C_{\alpha}\cap\gamma$ so

\Centerline{$h_{\beta}(\zeta)\le f_{\beta'}(\zeta)<g_{\gamma\alpha}
(\zeta)\le h_{\gamma}(\zeta).$}

\noindent It follows that

\Centerline{$\lambda^+=\#(B)
=\#(\{h_{\beta}(\zeta):\beta\in B\})\le\#(S_{\zeta})
\le\lambda$,}

\noindent which is absurd.\ \Bang\Qed

\medskip

{\bf (d)} Consequently
$E=\{f_{\alpha}^{\ssbullet}:\alpha<\kappa\}$
has a least upper bound in $P|\Cal F$.   \Prf\Quer\
If not, choose a family
$\langle h_{\xi}\rangle_{\xi<\lambda^+}$ in $P$ inductively,
as follows.   Because $\kappa<\cf(P|\Cal F)$, there is an
$h_0\in P$ such that
$f_{\alpha}\le_{\Cal F}h_0$ for every $\alpha<\kappa$.   Given
$h_{\xi}$
such that $h_{\xi}^{\bullet}$ is an upper bound
for $E$, then $h_{\xi}^{\bullet}$ cannot be the least upper bound of
$E$, so there is an $h_{\xi+1}\in P$ such that $h_{\xi+1}^{\bullet}$
is an upper bound of $E$ strictly less than $h_{\xi}^{\bullet}$.
For non-zero limit ordinals $\xi<\lambda^+$, set

\Centerline{$S_{\xi\zeta}=\{h_{\eta}(\zeta):\eta<\xi\}
\subseteq\theta_{\zeta}$}

\noindent for each $\zeta<\lambda$.   By (c) above, there is an
$\alpha_{\xi}<\kappa$ such that

\Centerline{for every $h\in\prod_{\zeta<\lambda}S_{\xi\zeta}$
either $f_{\alpha_{\xi}}\not\le_{\Cal F}h$ or
$f_{\alpha}\le_{\Cal F}h\enskip\forall\enskip\alpha<\kappa$.}

\noindent Set

\Centerline{$h_{\xi}(\zeta)=\min(\{\eta:\eta\in S_{\xi\zeta},\,
f_{\alpha_{\xi}}(\zeta)\le\eta\}\cup\{h_0(\zeta)\})\in S_{\xi\zeta}$}

\noindent for each $\zeta<\lambda$.   Then
$f_{\alpha_{\xi}}\le_{\Cal F}h_{\xi}$ (because
$f_{\alpha_{\xi}}(\zeta)\le h_{\xi}(\zeta)$ whenever
$f_{\alpha_{\xi}}(\zeta)\le h_0(\zeta)$) and
$h_{\xi}\in\prod_{\zeta<\lambda}S_{\xi\zeta}$, so
$f_{\alpha}\le_{\Cal F}h_{\xi}$ for every $\alpha<\kappa$ and
$h_{\xi}^{\bullet}$ is an upper bound for $E$.   Also, if
$\eta<\xi$, then $h_{\xi}(\zeta)\le h_{\eta}(\zeta)$ whenever
$f_{\alpha_{\xi}}(\zeta)\le h_{\eta}(\zeta)$, so $h_{\xi}\le_{\Cal F}
h_{\eta}$.   Continue.

Having got the family $\langle h_{\xi}\rangle_{\xi<\lambda^+}$, set

\Centerline{$S_{\zeta}=\bigcup_{\xi<\lambda^+}S_{\xi\zeta}
=\{h_{\xi}(\zeta):\xi<\lambda^+\}\subseteq\theta_{\zeta}$}

\noindent for each $\zeta<\lambda$.   For each $\alpha<\kappa$,
$\zeta<\lambda$ set

\Centerline{$g_{\alpha}(\zeta)=\min(\{\eta:f_{\alpha}(\zeta)
\le\eta\in S_{\zeta}\}\cup\{h_0(\zeta)\})\in S_{\zeta}$.}

\noindent Then, by the same arguments as above,

\Centerline{$f_{\alpha}\le_{\Cal F}g_{\alpha}
\le_{\Cal F}h_{\xi}$ for every $\alpha<\kappa$,
$\xi<\lambda^+$.}

For each $\alpha<\kappa$ there is a non-zero limit ordinal $\xi<\lambda^+$
such that $g_{\alpha}(\zeta)\in S_{\xi\zeta}$ for every
$\zeta<\lambda$, because $\ofamily{\xi}{\lambda^+}{S_{\xi\zeta}}$ is
non-decreasing for each $\zeta$.
Because $\lambda^+<\kappa$ there is a non-zero
limit ordinal $\xi<\lambda^+$ such that

\Centerline{$A
=\{\alpha:g_{\alpha}(\zeta)\in S_{\xi\zeta}
   \enskip\forall\enskip\zeta<\lambda\}$}

\noindent is cofinal with $\kappa$.   In particular, there is an
$\alpha\in A$ such that $\alpha\ge\alpha_{\xi}$.   In this case

\Centerline{$f_{\alpha_{\xi}}\le_{\Cal F}f_{\alpha}
\le_{\Cal F}g_{\alpha}\le_{\Cal F}h_{\xi+1}
\le_{\Cal F}h_{\xi}\not\le_{\Cal F}h_{\xi+1}$,}

\noindent so there is a $\zeta<\lambda$ such that

\Centerline{$f_{\alpha_{\xi}}(\zeta)\le f_{\alpha}(\zeta)
\le g_{\alpha}(\zeta)\le h_{\xi+1}(\zeta)<h_{\xi}(\zeta)$.}

\noindent But now observe that

\Centerline{$f_{\alpha_{\xi}}(\zeta)\le g_{\alpha}(\zeta)
\in S_{\xi\zeta}$}

\noindent so $h_{\xi}(\zeta)\le g_{\alpha}(\zeta)<h_{\xi}(\zeta)$,
which is absurd.\ \Bang\Qed

\medskip

{\bf (e)} Let $g\in P$ be such that $g^{\bullet}=\sup E$ in
$P|\Cal F$
and $g(\zeta)>0$ for every $\zeta<\lambda$.  For each $\zeta<\lambda$
set $\hat\theta_{\zeta}=\cf g(\zeta)<\theta_{\zeta}$
and choose a cofinal set $D_{\zeta}\subseteq g(\zeta)$ of order type
$\hat\theta_{\zeta}$.   For $\alpha<\kappa$ and $\zeta<\lambda$ set

\Centerline{$\hat g_{\alpha}(\zeta)
=\min\{\eta:f_{\alpha}(\zeta)\le\eta\in D_{\zeta}\}$}

\noindent if $f_{\alpha}(\zeta)<g(\zeta)$,  $\min D_{\zeta}$
otherwise.   Then $\hat g_{\alpha}\le_{\Cal F}\hat g_{\beta}$
whenever $\alpha\le\beta<\kappa$.
Also if $h\in Q =\prod_{\zeta<\lambda}D_{\zeta}$ then
$h^{\bullet}<g^{\bullet}$ so there is an $\alpha<\kappa$ such that
$h^{\bullet}\le f_{\alpha}^{\bullet}\le\hat g_{\alpha}^{\bullet}$.
Thus $\{\hat g_{\alpha}^{\bullet}:\alpha<\kappa\}$ is cofinal with
$\{h^{\bullet}:h\in Q\}$.

\medskip

{\bf (f)}
Because each $D_{\zeta}$ is order-isomorphic to $\hat\theta_{\zeta}$,
we can identify $Q$ with
$\hat P=\prod_{\zeta<\lambda}\hat\theta_{\zeta}$,
and see that $\cf(\hat P|\Cal F)=\cf\{h^{\ssbullet}:h\in Q\}$
is either $1$ or $\kappa$.
But of course the former is impossible, because it could be so only
if $\{\zeta:g(\zeta)\text{ is a successor ordinal}\}$ belonged to
$\Cal F$,
and in this case there would have to be an $\alpha<\kappa$ such
that $g\le_{\Cal F}f_{\alpha}$;  but we saw in (b) above that
$f_{\alpha+1}\not\le_{\Cal F}f_{\alpha}$.

Accordingly $\cf(\hat P|\Cal F)=\kappa$.

\medskip

{\bf (g)} It may be that some of the $\hat\theta_{\zeta}$ are
less than or equal to $\lambda$.   But taking
$I=\{\zeta:\hat\theta_{\zeta}\le\lambda\}$, we have $I\notin\Cal F$.
\Prf\Quer\  If $I\in\Cal F$,
then for each $\zeta\in I$ set $S_{\zeta}=D_{\zeta}$ and for
$\zeta\in\lambda\setminus I$ set $S_{\zeta}=\{0\}$.   By (c), there
is an $\alpha<\kappa$ such that

\Centerline{for every $h\in\prod_{\zeta<\lambda}S_{\zeta}$,
if $f_{\alpha}\le_{\Cal F}h$ then
$f_{\beta}\le_{\Cal F}h\enskip\forall\enskip\beta<\kappa$.}

\noindent But as $f_{\alpha+1}\le_{\Cal F}g$, and $I\in\Cal F$, there
must be an $h\in\prod_{\zeta<\lambda}S_{\zeta}$ such that
$f_{\alpha}\le_{\Cal F}h$, and now $g\le_{\Cal F}h$ because
$g^{\bullet}$ is the
least upper bound of $E$;  but $h(\zeta)<g(\zeta)$ for every
$\zeta\in I$, so this is impossible.\ \Bang\Qed

So $\{\zeta:\hat\theta_{\zeta}>\lambda\}\in\Cal F$.   But this means
that if we set

$$\eqalign{\theta'_{\zeta}
&=\hat\theta_{\zeta}\text{ when }\hat\theta_{\zeta}>\lambda,\cr
&=\theta_{\zeta}\text{ when }\hat\theta_{\zeta}\le\lambda\cr}$$

\noindent and $P'=\prod_{\zeta<\lambda}\theta'_{\zeta}$,
then $P'|\Cal F\cong\hat P|\Cal F$ so $\cf(P'|\Cal F)=\kappa$, as
required.
}%end of proof of 5A2C
%5{}42D

\leader{5A2D}{Definitions (a)} Let $\alpha$, $\beta$, $\gamma$ and
$\delta$ be cardinals.   \cmmnt{Following {\smc Shelah 92} and
{\smc Shelah 94},} I write

\Centerline{$\covSh(\alpha,\beta,\gamma,\delta)$}

\noindent for the least cardinal of any family
$\Cal E\subseteq[\alpha]^{<\beta}$ such that for every
$A\in[\alpha]^{<\gamma}$ there is a $\Cal D\in[\Cal E]^{<\delta}$ with
$A\subseteq\bigcup\Cal D$.
In the trivial cases in which there is no such family
$\Cal E$ I write $\covSh(\alpha,\beta,\gamma,\delta)=\infty$.

\medskip

{\bf (b)} For cardinals $\alpha$, $\gamma$
write $\Theta(\alpha,\gamma)$ for the supremum of all cofinalities

\Centerline{$\cf(\prod_{\zeta<\lambda}\theta_{\zeta})$}

\noindent for families $\langle\theta_{\zeta}\rangle_{\zeta<\lambda}$
such that $\lambda<\gamma$ is a cardinal, every
$\theta_{\zeta}$ is a regular infinite cardinal, and
$\lambda<\theta_{\zeta}<\alpha$ for every $\zeta<\lambda$.
\cmmnt{(This carries some of the same information as the cardinal
pp$_{\kappa}(\alpha)$ of {\smc Shelah 94}, p.\ 41.)}
%5{}23M 5{}42D

\medskip

\wheader{5A2D}{0}{0}{0}{60pt}
\noindent{\bf Remarks\cmmnt{ (i)}}\cmmnt{ Immediately from the definitions,
we see that}

\Centerline{$\covSh(\alpha,\beta',\gamma,\delta')
\le\covSh(\alpha',\beta,\gamma',\delta)$,
\quad$\Theta(\alpha,\gamma)\le\Theta(\alpha',\gamma')$}

\noindent whenever $\alpha\le\alpha'$, $\beta\le\beta'$,
$\gamma\le\gamma'$ and $\delta\le\delta'$.

\cmmnt{\medskip

{\bf (ii)} The definition of $\Theta$ demands a moment's thought in
trivial cases.   If $\gamma=0$ there is no $\lambda<\gamma$, so we are
taking the supremum of an empty set of cofinalities, and
$\Theta(\alpha,0)=0$ for every $\alpha$.   If $\gamma>0$
then we are allowed $\lambda=0$ and
$\prod_{\zeta<\lambda}\theta_{\zeta}=\{\emptyset\}$, so
$\Theta(\alpha,\gamma)\ge 1$ for every $\alpha$.   If $\gamma>1$ we are
allowed $\lambda=1$, so $\Theta(\alpha^+,\gamma)\ge\alpha$ for every
infinite $\alpha$.
}%end of comment

%if $\gamma>1$ and $\alpha$ an uncountable limit cardinal, then
%$\Theta(\alpha,\gamma)\ge\alpha$ (5{}42Dd)

\vleader{48pt}{5A2E}{Lemma} %RVMC 7Q
Let $\alpha$, $\beta$, $\gamma$, $\gamma'$ and
$\delta$ be cardinals.

(a) If $\gamma\le\gamma'\le\beta$ and $\delta\ge 2$ then

\Centerline{$\covSh(\alpha,\beta,\gamma,\delta)
\le\cff[\alpha]^{<\gamma'}\le\#([\alpha]^{<\gamma'})$.}

(b) If either $\omega\le\gamma\le\cf\alpha$ or
$\omega\le\cf\alpha<\cf\delta$ then

\Centerline{$\covSh(\alpha,\beta,\gamma,\delta)
\le\max(\alpha,
  \sup_{\theta<\alpha}\covSh(\theta,\beta,\gamma,\delta))$.}

\proof{{\bf (a)}
If $\Cal E$ is a cofinal subset of $[\alpha]^{<\gamma'}$ of
cardinal $\cff[\alpha]^{<\gamma'}$, then $\Cal E$ witnesses that
$\covSh(\alpha,\gamma',\gamma',\delta)\le\cff[\alpha]^{<\gamma}$.   Now

\Centerline{$\covSh(\alpha,\beta,\gamma,\delta)
\le\covSh(\alpha,\gamma',\gamma',\delta)\le\cff[\alpha]^{<\gamma'}$.}

\medskip

{\bf (b)} Set $\kappa
=\max(\alpha,\sup_{\theta<\alpha}\covSh(\theta,\beta,\gamma,\delta))$
and $\lambda=\cf\alpha$.   Let $\ofamily{\xi}{\lambda}{\zeta_{\xi}}$
enumerate a cofinal subset of $\alpha$.   For each $\xi<\lambda$, let
$\Cal E_{\xi}\subseteq[\zeta_{\xi}]^{<\beta}$ be a set of size at most
$\kappa$ such that for every $A\in[\zeta_{\xi}]^{<\gamma}$ there is a
$\Cal D\in[\Cal E_{\xi}]^{<\delta}$ such that
$A\subseteq\bigcup\Cal D$.   Set
$\Cal E=\bigcup_{\xi<\lambda}\Cal E_{\xi}$, so that
$\Cal E\subseteq[\alpha]^{<\beta}$ has cardinal at most $\kappa$.
Take $A\in[\alpha]^{<\gamma}$.

If $\omega\le\gamma\le\lambda$ then $\sup A<\alpha$ and there is a
$\xi<\lambda$ such that $A\subseteq\zeta_{\xi}$.   Now there is a
$\Cal D\in[\Cal E_{\xi}]^{<\delta}\subseteq[\Cal E]^{<\delta}$ such
that $A\subseteq\bigcup\Cal D$.

If $\omega\le\lambda<\cf\delta$, then for each $\xi<\lambda$ there is
a $\Cal D_{\xi}\in[\Cal E_{\xi}]^{<\delta}$ such that
$A\cap\zeta_{\xi}\subseteq\bigcup\Cal D_{\xi}$.   Set
$\Cal D=\bigcup_{\xi<\lambda}\Cal D_{\xi}$;  because
$\lambda<\cf\delta$, $\Cal D\in[\Cal E]^{<\delta}$, while

\Centerline{$A=\bigcup_{\xi<\lambda}A\cap\zeta_{\xi}
\subseteq\bigcup\Cal D$.}

Thus in either case $\Cal E$ witnesses that
$\covSh(\alpha,\beta,\gamma,\delta)\le\kappa$.
}%end of proof of 5A2E
%5{}41S 5{}42D 5{}42E

\leader{5A2F}{Lemma} Let $\alpha$, $\gamma$ be cardinals.   If
$\alpha\le 2^{\gamma}$, then $\Theta(\alpha,\gamma)\le 2^{\gamma}$.

\proof{ If $\gamma\le\omega$ then
$\Theta(\alpha,\gamma)\le\max(1,\alpha)\le 2^{\gamma}$.
If $\gamma>\omega$,
$\lambda<\gamma$ and $\theta_{\zeta}<2^{\gamma}$ for every
$\zeta<\lambda$, then

\Centerline{$\#(\prod_{\zeta<\lambda}\theta_{\zeta})
\le(2^{\gamma})^{\gamma}=2^{\gamma}$.}
}%end of proof of 5A2F

\leader{5A2G}{Theorem} %RVCM A1K
For any cardinals $\alpha$ and $\gamma$,

\Centerline{$\covSh(\alpha,\gamma,\gamma,\omega_1)
\le\max(\omega,\alpha,\Theta(\alpha,\gamma))$.}

\proof{{\bf (a)} To begin with (down to the end of (f) below) let us
suppose that we have
$\alpha\ge\gamma=\gamma_0^+>\cf\alpha>\omega$, and set
$\kappa=\max(\alpha,\Theta(\alpha,\gamma))$.

Take a family $\Cal E\subseteq[\alpha]^{\le\gamma_0}$ such that

\inset{(i) $\Cal E$ contains all singleton subsets of $\alpha$;

(ii) $\Cal E$ contains a cofinal subset of $\alpha$;

(iii) If $E\in\Cal E$ then $\{\xi:\xi+1\in E\}\in\Cal E$;

(iv) if $E\in\Cal E$ then
there is an $F\in\Cal E$ such that $\sup(F\cap\xi)=\xi$ whenever
$\xi\in E$ and $\omega\le\cf\xi\le\gamma_0$;

(v) if $E\in\Cal E$ then $\{\xi:\xi\in E,\,
\cf\xi>\gamma_0\}\in\Cal E$;

(vi) if $E\in\Cal E$ and $\cf(\prod_{\eta\in E}\eta)\le\kappa$,
then $\{g:g\in\prod_{\eta\in E}\eta,\,g[E]\in\Cal E\}$ is cofinal with
$\prod_{\eta\in E}\eta$;

(vii) $\#(\Cal E)\le\kappa$.}

\noindent To see that this can be done, observe that whenever
$E\in[\alpha]^{\le\gamma_0}$ there is an $F\in[\alpha]^{\le\gamma_0}$
such that $\sup(F\cap\xi)=\xi$ whenever $\xi\in E$ and
$\omega\le\cf\xi\le\gamma_0$;  thus condition (iv) can be
achieved, like conditions
(iii) and (v), by ensuring that $\Cal E$ is closed under suitable
functions
from $[\alpha]^{\le\gamma_0}$ to itself;  while condition (vi)
requires that for each $E\in\Cal E$ we have  an appropriate
family of size at most $\kappa$ included in $\Cal E$.

Write $\Cal J$ for the $\sigma$-ideal of $\Cal P\alpha$ generated by
$\Cal E$.   Note that if $A\in\Cal J$ then $\{\xi:\xi+1\in A\}$ belongs to
$\Cal J$, by (iii).

\medskip

{\bf (b)} \Quer\ If
$\covSh(\alpha,\gamma,\gamma,\omega_1)>\kappa$, there must be a set
in $[\alpha]^{\le\gamma_0}$
not covered by any sequence from $\Cal E$, that is, not belonging to
$\Cal J$;  that is, there is a
function $f:\gamma_0\to\alpha$ such that $f[\gamma_0]\notin\Cal J$.
Accordingly $\Cal I=\{f^{-1}[E]:E\in\Cal J\}$
is a proper $\sigma$-ideal of $\Cal P\gamma_0$.
By condition (a-i), $\Cal I$ contains all singletons in
$\Cal P\gamma_0$.

Let $H$ be the set of all functions $h:\gamma_0\to\alpha$ such that
$f(\xi)\le h(\xi)$ for every $\xi<\gamma_0$ and
$h[\gamma_0]\in\Cal J$.   Because
$\Cal E$ contains a cofinal set $C\subseteq\alpha$ (condition (a-ii)),
we can find an
$h\in H$;  just take $h:\gamma_0\to C$ such that $f(\xi)\le h(\xi)$
for every $\xi$.

\medskip

{\bf (c)} Because $\Cal I$ is a proper
$\sigma$-ideal, there cannot be any sequence
$\langle h_n\rangle_{n\in\Bbb N}$ in $H$ such that
$\{\xi:h_{n+1}(\xi)\ge h_n(\xi)\}\in\Cal I$ for every $n\in\Bbb N$.
Consequently there is an $h^*\in H$ such that

\Centerline{$\{\xi:h(\xi)\ge h^*(\xi)\}\notin\Cal I$ for every $h\in H$.}

\noindent We know that $h^*[\gamma_0]\in\Cal J$;  let
$\langle E_n\rangle_{n\in\Bbb N}$ be a sequence in $\Cal E$ covering
$h^*[\gamma_0]$.
For $\xi<\gamma_0$ write $\theta_{\xi}=\cf h^*(\xi)$, so that
each $\theta_{\xi}$ is 0 or
1 or a regular infinite cardinal less than $\alpha$.   Set

\Centerline{$I=\{\xi:\xi<\gamma_0$, $f(\xi)=h^*(\xi)\}$,}

\Centerline{$I'=\{\xi:\xi<\gamma_0,\,f(\xi)<h^*(\xi),\,
\theta_{\xi}=1\}$,}

\Centerline{$I_n=\{\xi:\xi<\gamma_0$,
$f(\xi)<h^*(\xi)$, $\omega\le\theta_{\xi}\le\gamma_0$,
$h^*(\xi)\in E_n\}$ for $n\in\Bbb N$,}

\Centerline{$J_n=\{\xi:\xi<\gamma_0$, $f(\xi)<h^*(\xi)$,
$\gamma_0<\theta_{\xi}$, $h^*(\xi)\in E_n\}$ for $n\in\Bbb N$.}

\noindent Note that if $\theta_{\xi}=0$ then $h^*(\xi)=0=f(\xi)$, so
$I$, $I'$, $\sequencen{I_n}$ and $\sequencen{J_n}$ constitute a cover
of $\gamma_0$.

\medskip

{\bf (d)} For each $n\in\Bbb N$ set
$G_n=\{\eta:\eta\in E_n$, $\cf\eta>\gamma_0\}\in\Cal E$;  note that
$h^*(\xi)\in G_n$ for $\xi\in J_n$.   Then
$\cf(\prod_{\eta\in G_n}\eta)\le\Theta(\alpha,\gamma)$.   \Prf\
For $\eta\in G_n$ set $\theta'_{\eta}=\cf\eta$;  then $\theta'_{\eta}$
is a regular cardinal and $\#(G_n)\le\gamma_0<\theta'_{\eta}<\alpha$
for each $\eta\in G_n$.   If for each $\eta\in G_n$ we choose a
cofinal set $C_{\eta}\subseteq\eta$ of order type $\theta'_{\eta}$,
then

\Centerline{$\cf(\prod_{\eta\in G_n}\eta)
=\cf(\prod_{\eta\in G_n}C_{\eta})
=\cf(\prod_{\eta\in G_n}\theta'_{\eta})
\le\Theta(\alpha,\gamma)$}

\noindent by the definition of $\Theta(\alpha,\gamma)$.\ \Qed

Consequently, by (a-vi),

\Centerline{$\{g:g\in\prod_{\eta\in G_n}\eta,\,g[G_n]\in\Cal E\}$}

\noindent is cofinal with $\prod_{\eta\in G_n}\eta$.

\medskip

{\bf (e)} Define $h:\gamma_0\to\alpha$ as follows.

\quad(i) If $\xi\in I$ set $h(\xi)=h^*(\xi)$.

\quad(ii) If $\xi\in I'$ let $h(\xi)$ be the predecessor of
$h^*(\xi)$.

\quad(iii) For each $n\in\Bbb N$ take $F_n\in\Cal E$
such that $\eta=\sup(F_n\cap\eta)$ whenever $\eta\in E_n$ and
$\omega\le\cf\eta\le\gamma_0$.   If
$\xi\in I_n\setminus\bigcup_{m<n}I_m$, take $h(\xi)\in F_n$ such that
$f(\xi)<h(\xi)<h^*(\xi)$.

\quad(iv) For each $n\in\Bbb N$ and $\eta\in G_n$ set

\Centerline{$g^*(\eta)=\sup\{f(\xi):\xi<\gamma_0,\,h^*(\xi)=\eta\}$.}

\noindent Then $g^*(\eta)<\eta$, because $\gamma_0<\cf\eta$.   By
(d), there is a $g_n\in\prod_{\eta\in G_n}\eta$ such that
$g_n[G_n]\in\Cal E$
and $g^*(\eta)<g_n(\eta)$ for every $\eta\in G_n$.   So for
$\xi\in J_n\setminus\bigcup_{m<n}J_m$ we may set
$h(\xi)=g_n(h^*(\xi))$
and see that

\Centerline{$f(\xi)\le g^*h^*(\xi)<g_nh^*(\xi)=h(\xi)<h^*(\xi)$,}

\noindent while $h(\xi)\in g_n[G_n]$.

\medskip

{\bf (f)} Now we see that

\Centerline{$h[\gamma_0]\subseteq h^*[\gamma_0]\cup\{\eta:\eta+1\in
h^*[\gamma_0]\}\cup\bigcup_{n\in\Bbb N}F_n\cup\bigcup_{n\in\Bbb N}
g_n[G_n]\in\Cal J$,}

\noindent while $f(\xi)\le h(\xi)$ for every $\xi<\gamma_0$, so
$h\in H$.   Consequently

\Centerline{$I=\{\xi:h(\xi)\ge h^*(\xi)\}\notin\Cal I$.}

\noindent But also

\Centerline{$f[I]\subseteq h^*[\gamma_0]\in\Cal J$,}

\noindent so $I\in\Cal I$, which is absurd.  \Bang

\medskip

{\bf (g)} Thus the special case described in (a) is dealt with, and we
may return to the general case.   I proceed by induction on $\alpha$
for fixed $\gamma$.

\medskip

\quad{\bf (i)} To start the induction, observe that
if either $\alpha\le\omega$ or $\gamma\le\omega$ or $\alpha<\gamma$,
then

\Centerline{$\covSh(\alpha,\gamma,\gamma,\omega_1)
\le\cff[\alpha]^{<\gamma}\le\max(\alpha,\omega)$.}

\medskip

\quad{\bf (ii)} For the inductive step to $\alpha$ when {\it either}
$\cf\alpha\ge\gamma\ge\omega$ {\it or} $\cf\alpha=\omega$,
5A2Eb tells us that

$$\eqalign{\covSh(\alpha,\gamma,\gamma,\omega_1)
&\le\max(\alpha,
  \sup_{\alpha'<\alpha}\covSh(\alpha',\gamma,\gamma,\omega_1))\cr
&\le\max(\omega,\alpha,\sup_{\alpha'<\alpha}\Theta(\alpha',\gamma))
\le\max(\omega,\alpha,\Theta(\alpha,\gamma))\cr}$$

\noindent by the inductive hypothesis.

\medskip

\quad{\bf (iii)} For the inductive step to $\alpha$ when
$\omega<\cf\alpha<\gamma\le\alpha$, observe that

\Centerline{$[\alpha]^{<\gamma}
=\bigcup_{\delta<\gamma}[\alpha]^{\le\delta}$.}

\noindent For each $\delta<\gamma$ we have a set
$\Cal E_{\delta}\subseteq [\alpha]^{\le\delta}$ such that
$\#(\Cal E_{\delta})
\le\covSh(\alpha,\delta^+,\delta^+,\omega_1)$ and every member of
$[\alpha]^{\le\delta}$
can be covered by a sequence from $\Cal E_{\delta}$.   Set
$\Cal E=\bigcup_{\cf\alpha\le\delta<\gamma}\Cal E_{\delta}$;  then
$\Cal E\subseteq[\alpha]^{<\gamma}$ and every member of
$[\alpha]^{<\gamma}$ can be covered by a sequence from $\Cal E$.
So

$$\eqalignno{\covSh(\alpha,\gamma,\gamma,\omega_1)
&\le\#(\Cal E)
\le\max(\gamma,\sup_{\cf\alpha\le\delta<\gamma}
  \covSh(\alpha,\delta^+,\delta^+,\omega_1))\cr
&\le\max(\gamma,\alpha,
  \sup_{\cf\alpha\le\delta<\gamma}\Theta(\alpha,\delta^+))\cr
\displaycause{by (a)-(f) above}
&\le\max(\alpha,\Theta(\alpha,\gamma)).\cr}$$

This completes the proof.

\medskip

\noindent{\bf Remark} This is taken from {\smc Shelah 94}, Theorem
II.5.4,
where a stronger result is proved, giving an exact description of
many of the numbers
$\covSh(\alpha,\beta,\gamma,\delta)$ in terms of cofinalities of
reduced products $\prod_{\zeta<\lambda}\theta_{\zeta}|\Cal F$.
}%end of proof of 5A2G
%5{}42D 5{}42E

\leader{5A2H}{Lemma} %RVMC A1L
Let $\gamma$ be an infinite regular cardinal and
$\alpha\ge\Theta(\gamma,\gamma)$ a cardinal.   Then
$\Theta(\Theta(\alpha,\gamma),\gamma)\le\Theta(\alpha,\gamma)$.

%do we really need \gamma to be regular?

\proof{{\bf (a)} The case $\gamma=\omega$ is elementary, since
$\Theta(\alpha,\omega)\le\alpha$ for every cardinal $\alpha$.   So we may
suppose that $\gamma$ is uncountable.   If $\Theta(\alpha,\gamma)\le\alpha$
the result is immediate;  so we may suppose that
$\alpha<\Theta(\alpha,\gamma)$, in which case
$\Theta(\gamma,\gamma)<\Theta(\alpha,\gamma)$ and $\gamma<\alpha$.

\medskip

{\bf (b)} \Quer\ Suppose, if possible, that
$\Theta(\alpha,\gamma)<\Theta(\Theta(\alpha,\gamma),\gamma)$.
There must be a family $\langle\theta_{\zeta}\rangle_{\zeta<\lambda}$
of regular infinite cardinals such that $\lambda<\gamma$,
$\lambda<\theta_{\zeta}<\Theta(\alpha,\gamma)$ for every
$\zeta<\lambda$,
and $\cf(\prod_{\zeta<\lambda}\theta_{\zeta})>\Theta(\alpha,\gamma)$.
As $\Theta(\alpha,\gamma)\ge 1$, $\lambda\ne 0$.
By 5A2Bc, there is an ultrafilter $\Cal F$ on $\lambda$ such that
$\cf(\prod_{\zeta<\lambda}\theta_{\zeta}|\Cal F)>\Theta(\alpha,\gamma)$.
Set $L=\{\zeta:\zeta<\lambda,\,\theta_{\zeta}<\alpha\}$;  then
$\cf(\prod_{\zeta\in L}\theta_{\zeta})\le\Theta(\alpha,\gamma)$,
so $L\notin\Cal F$ and $M=\lambda\setminus L\in\Cal F$.   Let $\Cal
F'$ be the induced ultrafilter on $M$, so that
$\prod_{\zeta<\lambda}\theta_{\zeta}|\Cal F
\cong\prod_{\zeta\in M}\theta_{\zeta}|\Cal F'$, and
$\cf(\prod_{\zeta\in M}\theta_{\zeta}|\Cal
F')>\Theta(\alpha,\gamma)$.   For each $\zeta\in M$, we have
$\theta_{\zeta}<\Theta(\alpha,\gamma)$, so there must be a family
$\langle\theta_{\zeta\eta}\rangle_{\eta<\lambda_{\zeta}}$ of regular
infinite cardinals with $\lambda_{\zeta}<\gamma$,
$\lambda_{\zeta}<\theta_{\zeta\eta}<\alpha$ for every
$\eta<\lambda_{\zeta}$ and
$\theta_{\zeta}\le\cf(\prod_{\eta<\lambda_{\zeta}}\theta_{\zeta\eta})$.
Again by 5A2Bc, %RVMC A1Ic
there is an ultrafilter $\Cal F_{\zeta}$ on $\lambda_{\zeta}$ such
that $\theta_{\zeta}
\le\cf(\prod_{\eta<\lambda_{\zeta}}\theta_{\zeta\eta}|\Cal F_{\zeta})$.
Because

\Centerline{$\lambda_{\zeta}<\gamma\le\alpha\le\theta_{\zeta}$,}

\noindent 5A2C tells us that there
is a family $\langle\theta'_{\zeta\eta}\rangle_{\eta<\lambda_{\zeta}}$
of regular infinite cardinals such that
$\lambda_{\zeta}<\theta'_{\zeta\eta}\le \theta_{\zeta\eta}$ for every
$\eta$ and $\theta_{\zeta}
=\cf(\prod_{\eta<\lambda_{\zeta}}\theta'_{\zeta\eta}|\Cal F_{\zeta})$.

\woddheader{5A2H}{4}{2}{2}{66pt}
{\bf (c)} Set

\Centerline{$I=\{(\zeta,\eta):\zeta\in M,\,\eta<\lambda_{\zeta}\}$,}

\Centerline{$\Cal H
=\{H:H\subseteq I,\,
  \{\zeta:\{\eta:(\zeta,\eta)\in H\}\in\Cal F_{\zeta}\}
\in\Cal F'\}$,}

\Centerline{$P=\prod_{(\zeta,\eta)\in I}\theta'_{\zeta\eta}$.}

\noindent Then $\Cal H$ is an ultrafilter on $I$, and
$\cf(P|\Cal H)\ge\cf(\prod_{\zeta\in M}\theta_{\zeta}|\Cal F')$.
\Prf\ Let $F\subseteq P$ be a set with cardinal $\cf(P|\Cal H)$ such
that
$\{f^{\bullet}:f\in F\}$ is cofinal with $P|\Cal H$.   For
$f\in P$ and $\zeta\in M$, define
$f_{\zeta}\in\prod_{\eta<\lambda_{\zeta}}\theta'_{\zeta\eta}$
by setting $f_{\zeta}(\eta)=f(\zeta,\eta)$ for each
$\eta<\lambda_{\zeta}$, and let $f_{\zeta}^{\bullet}$ be the image of
$f_{\zeta}$ in
$\prod_{\eta<\lambda_{\zeta}}\theta'_{\zeta\eta}|\Cal F_{\zeta}$.
For each $\zeta\in M$ let
$\langle u_{\zeta\xi}\rangle_{\xi<\theta_{\zeta}}$ be a strictly
increasing cofinal family in the totally ordered set
$\prod_{\eta<\lambda_{\zeta}}\theta'_{\zeta\eta}|\Cal F_{\zeta}$.
Now, for $f\in F$, take a function
$g_{f}\in\prod_{\zeta\in M}\theta_{\zeta}$
such that $f_{\zeta}^{\bullet}\le u_{\zeta,g_f(\zeta)}$ for every
$\zeta\in M$.

If $g\in\prod_{\zeta\in M}\theta_{\zeta}$, then we can find an $h\in
P$
such that $h_{\zeta}^{\bullet}=u_{\zeta,g(\zeta)}$ for each
$\zeta\in M$.   Let $f\in F$ be such that $h\le_{\Cal H}f$.
Then

\Centerline{$\{\zeta:g(\zeta)\le g_f(\zeta)\}
\supseteq\{\zeta:h_{\zeta}^{\bullet}\le f_{\zeta}^{\bullet}\}
\in\Cal F'$,}

\noindent so $g\le_{\Cal F'}g_f$.
Accordingly $\{g_f:f\in F\}$ is cofinal with
$\prod_{\zeta\in M}\theta_{\zeta}|\Cal F'$ and
$\cf(\prod_{\zeta\in M}\theta_{\zeta}|\Cal F')\le\#(F)=\cf(P|\Cal
H)$, as
claimed.\ \Qed

\medskip

{\bf (d)} Thus $\cf(P|\Cal H)>\Theta(\alpha,\gamma)$.   Set
$J=\{(\zeta,\eta):(\zeta,\eta)\in I,\,\theta'_{\zeta\eta}\ge\gamma\}$.
Because $\gamma$ is regular, $\#(J)\le\#(I)<\gamma$, so
$\cf(\prod_{(\zeta,\eta)\in J}\theta'_{\zeta\eta})
\le\Theta(\alpha,\gamma)$, and $J\notin\Cal H$.   It follows that
$K=I\setminus J\in\Cal H$.   Set
$M'=\{\zeta:\zeta\in M,\,\{\eta:(\zeta,\eta)\in K\}\in\Cal F_{\zeta}\}
\in\Cal F'$.

If $\zeta\in M'$, then $F=\{\eta:\eta<\lambda_{\zeta}$,
$\theta'_{\zeta\eta}<\gamma\}$ belongs to $\Cal F_{\zeta}$, so

\Centerline{$\theta_{\zeta}\le\cf(\prod_{\eta\in F}\theta'_{\zeta\eta})
\le\Theta(\gamma,\gamma)\le\alpha\le\theta_{\zeta}$}

\noindent by the definition of $M$.
So in fact $\theta_{\zeta}=\alpha$ for $\zeta\in M'$ and we have

\Centerline{$\Theta(\alpha,\gamma)
<\cf(\prod_{\zeta\in M}\theta_{\zeta}|\Cal F')
\le\cf(\prod_{\zeta\in M'}\theta_{\zeta})
=\cf(\prod_{\zeta<\delta}\alpha)$,}

\noindent where $\delta=\#(M')$, while at the same time $\alpha$
is infinite and regular.

But if $\alpha$ is infinite and regular and $\delta<\alpha$,

\Centerline{$\cf(\prod_{\zeta<\delta}\alpha)\le\alpha$,}

\noindent so $\Theta(\alpha,\gamma)<\alpha$;  which is impossible.\ \Bang

This contradiction completes the proof.
}%end of proof of 5A2H

\leader{5A2I}{Lemma} %RVMC A1M
Let $\alpha$ and $\gamma$ be cardinals.   Set
$\delta=\sup_{\alpha'<\alpha}\Theta(\alpha',\gamma)$.

(a) If $\cf\alpha\ge\gamma$ then
$\Theta(\alpha,\gamma)\le\max(\alpha,\delta)$.

(b) If $\cf\alpha<\gamma$ then
$\Theta(\alpha,\gamma)\le\max(\alpha,\delta^{\cf\alpha})$, where
$\delta^{\cf\alpha}$ is the cardinal power.
%5{}42

\proof{ Let $\langle\theta_{\zeta}\rangle_{\zeta<\lambda}$
be a family of regular infinite
cardinals with $\lambda<\theta_{\zeta}<\alpha$
for each $\zeta$ and $\lambda<\gamma$.

\medskip

{\bf case 1} If $\alpha'=\sup_{\zeta<\lambda}\theta_{\zeta}$ is less than
$\alpha$, set

\Centerline{$I=\{\zeta:\zeta<\lambda,\,\theta_{\zeta}<\alpha'\}$,}

\Centerline{$J=\{\zeta:\zeta<\lambda,\,\theta_{\zeta}=\alpha'\}$.}

\noindent Then

\Centerline{$\cf(\prod_{\zeta\in I}\theta_{\zeta})
\le\Theta(\alpha',\gamma)\le\delta$,
\quad$\cf(\prod_{\zeta\in J}\theta_{\zeta})
\le\max(1,\alpha')\le\alpha$.}

\noindent If $\lambda=0$ then
$\cf(\prod_{\zeta<\lambda}\theta_{\zeta})=1\le\alpha$;  if $\lambda>0$ then
$\alpha$ is infinite and

\Centerline{$\cf(\prod_{\zeta<\lambda}\theta_{\zeta})
\le\max(\omega,\cf(\prod_{\zeta\in I}\theta_{\zeta}),
   \cf(\prod_{\zeta\in J}\theta_{\zeta}))
\le\max(\alpha,\delta)$.}

This is enough to deal with (a).

\medskip

{\bf case 2} If $\alpha'=\alpha=0$ then $\lambda=0=\delta$, so

\Centerline{$\cf(\prod_{\zeta<\lambda}\theta_{\zeta})=1
=\delta^{\cf\alpha}$.}

\noindent So (b) is true if $\alpha=0$.

\medskip

{\bf case 3} If $\alpha'=\alpha>0$ and $\cf\alpha<\gamma$, then
$\lambda>0$ and $\alpha$ is a supremum of strictly smaller infinite
cardinals, so must be uncountable.   Let
$\langle\alpha_{\xi}\rangle_{\xi<\cf\alpha}$ be a strictly increasing
family of cardinals with supremum $\alpha$, starting from
$\alpha_0=0$ and $\alpha_1=\omega$ and with
$\alpha_{\xi}=\sup_{\eta<\xi}\alpha_{\eta}$ for non-zero limit ordinals
$\xi<\cf\alpha$.   Set

\Centerline{$P_{\xi}
=\prod_{\zeta<\lambda,\alpha_{\xi}\le\theta_{\zeta}<\alpha_{\xi+1}}
   \theta_{\zeta}$}

\noindent for each $\xi<\cf\alpha$.   Then

\Centerline{$\cf P_{\xi}\le\Theta(\alpha_{\xi+1},\gamma)\le\delta$}

\noindent for each $\xi<\cf\alpha$, so

\Centerline{$\cf(\prod_{\zeta<\lambda}\theta_{\zeta})
=\cf(\prod_{\xi<\cf\alpha}P_{\xi})\le\delta^{\cf\alpha}$.}

\noindent Putting this together with case 1, we have a proof of (b) when
$\alpha>0$.
}%end of proof of 5A2I

\leaveitout{\exercises{\leader{5A2X}{Basic exercises (a)}

\leader{5A2Y}{Further exercises (a)}

}%end of exercises

\endnotes{
\Notesheader{5A2}

}%end of notes
}%end of leaveitout

\discrpage


