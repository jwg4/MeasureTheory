Chapter 46: Pointwise compact sets of measurable functions

\chapintrosection{15.12.00}{576}{}

\section{461}{Barycenters and Choquet's theorem}{9.7.08}{576}{}
{Barycenters;  elementary properties;  sufficient conditions for
existence;  closed convex hulls of compact sets;  \Krein's theorem;
existence and uniqueness of measures on sets of extreme points;  ergodic
functions and extreme measures.}

\section{462}{Pointwise compact sets of continuous
functions}{30.6.07}{588}{}
{Angelic spaces;  the topology of pointwise convergence on $C(X)$;  weak
convergence and weakly compact sets in $C_0(X)$;  Radon measures on
$C(X)$;  separately continuous functions;  convex hulls.}

\section{463}{$\frak{T}_p$ and $\frak{T}_m$}{1.2.13}{594}{}
{Pointwise convergence and convergence in measure on spaces of
measurable functions;  compact and sequentially compact sets;  perfect
measures and Fremlin's Alternative;  separately continuous
functions.}

\section{464}{Talagrand's measure}{25.5.13}{604}{}
{The usual measure on
$\Cal{P}I$;  the intersection of a sequence of non-measurable filters;
Talagrand's measure;  the $L$-space of additive functionals on
$\Cal{P}I$;  measurable and purely non-measurable functionals.}

\ifdim\pagewidth>467pt\fontdimen3\tenrm=2pt\fi

\section{465}{Stable sets}{28.1.06}{616}{}
{Stable sets of functions;  elementary properties;  pointwise
compactness;  pointwise convergence and convergence in measure;   a law
of large numbers;  stable sets and uniform convergence in the strong law
of large numbers;  convex hulls;
stable sets in $L^0$ and $L^1$;  *R-stable sets.}
\fontdimen3\tenrm=1.67pt

\section{466}{Measures on linear topological spaces}{2.8.13}{644}{}
{Quasi-Radon measures for weak and strong topologies;  Kadec norms;
constructing weak-Borel measures;  characteristic functions of measures
on locally convex spaces;  universally measurable linear operators; 
Gaussian measures on linear topological spaces.}

\section{*467}{Locally uniformly rotund norms}{13.1.10}{653}{}
{Locally uniformly rotund norms;  separable normed spaces;  long
sequences of projections;  K-countably determined spaces;  weakly
compactly generated spaces;  Banach lattices with
order-continuous norms;  Eberlein compacta and Schachermeyer's theorem.}

\wheader{}{10}{4}{4}{100pt}

  Chapter 47:  Geometric measure theory

\chapintrosection{25.9.04}{665}{}

\section{471}{Hausdorff measures}{23.1.06}{665}{}
{Metric outer measures;  Increasing Sets Lemma;  analytic spaces;  inner
regularity;  Vitali's theorem and a density theorem;  Howroyd's
theorem.}

\section{472}{Besicovitch's Density Theorem}{22.3.11}{681}{}
{Besicovitch's Covering Lemma;  Besicovitch's Density Theorem;  *a
maximal theorem.}

\section{473}{Poincar\'e's inequality}{25.7.11}{688}{}
{Differentiable and Lipschitz functions;  smoothing by convolution;  the
Gagliardo-\vthsp{}Nirenberg-\vthsp{}Sobolev inequality;
Poin\discretionary{-}{}{}car\'e's inequality for balls.}

\section{474}{The distributional perimeter}{17.11.12}{700}{}
{The divergence of a vector field;  sets with locally finite perimeter,
perimeter measures and outward-normal functions;  the reduced boundary;
invariance under isometries;  isoperimetric inequalities;  Federer
exterior normals;  the Compactness Theorem.}

\ifdim\pagewidth>467pt\fontdimen3\tenrm=2pt\fi

\section{475}{The essential boundary}{24.1.13}{721}{}
{Essential interior, closure and boundary;  the reduced boundary, the
essential boundary and perimeter measures;  
characterizing sets with locally finite perimeter;
the Divergence Theorem;   calculating perimeters from cross-sectional
counts, and an integral-geometric formula;  
Cauchy's Perimeter Theorem;  the Isoperimetric Theorem for
convex sets.}

\fontdimen3\tenrm=1.67pt

\section{476}{Concentration of measure}{10.11.07}{744}{}
{Vietoris and Fell topologies;  concentration by partial
reflection;  concentration of measure in $\eightBbb{R}^r$;
the Isoperimetric
Theorem;  concentration of measure on spheres.}

\section{477}{Brownian motion}{2.1.10}{753}{}
{Brownian motion as a stochastic process;  Wiener measure on
$C(\coint{0,\infty})_0$;  *as a limit of random walks;
Brownian motion in $\eightBbb{R}^r$;   invariant
transformations of Wiener measure on
$C(\coint{0,\infty};\eightBbb{R}^r)_0$;  Wiener measure is strictly
positive;  the strong Markov property;  hitting times;  almost every
Brownian path is nowhere differentiable;  almost every Brownian path has
zero two-dimensional Hausdorff measure.}

\section{478}{Harmonic functions}{4.6.09}{778}{}
{Harmonic and superharmonic functions;  a maximal principle;  $f$ is
superharmonic iff $\nabla^2f\le0$;  the Poisson kernel and
harmonic functions
with given values on a sphere;  smoothing by convolution;  Brownian motion
and Dynkin's formula;  Brownian motion and superharmonic functions;
recurrence and divergence of Brownian motion;
harmonic measures and Dirichlet's problem;  disintegrating harmonic
measures over intermediate boundaries;  hitting probabilities.}

\section{479}{Newtonian capacity}{15.2.10}{803}{}
{Defining Newtonian capacity from Brownian hitting probabilities, and
equilibrium measures from harmonic measures;  submodularity and
sequential order-continuity;
extending Newtonian capacity to Choquet-Newton capacity;   Newtonian
potential and energy
of a Radon measure;  Riesz kernels and their Fourier transforms;
energy and $(r-1)$-potentials;  alternative definitions of 
capacity and equilibrium measures;  analytic sets of finite capacity;
polar sets;  general sets of finite capacity;
Brownian hitting probabilities and equilibrium potentials;
Hausdorff measure;  self-intersecting Brownian paths;  a discontinuous
equilibrium potential;  yet another definition of Newtonian capacity; 
capacity and volume;  a measure on the set of closed subsets of
$\eightBbb{R}^r$.}


\wheader{}{10}{4}{4}{100pt}

Chapter 48:  Gauge integrals

\chapintrosection{9.5.11}{845}{}

\section{481}{Tagged partitions}{4.9.09}{845}{}
{Tagged partitions and Riemann sums;  gauge integrals;  gauges;
residual sets;  subdivisions;  examples (the Riemann integral, the
Henstock integral, the symmetric Riemann-complete integral, the McShane
integral, box products, the approximately continuous Henstock
integral).}

\ifdim\pagewidth>467pt\fontdimen3\tenrm=2pt\fi

\section{482}{General theory}{11.5.10}{855}{}
{Saks-Henstock lemma;  when gauge-{\vthsp}integrable functions are 
measurable;  when integrable functions are gauge-integrable;
$I_{\nu}(f\times{\chi}H)$;  improper integrals;
integrating derivatives;  B.Levi's theorem;  Fubini's theorem.}

\fontdimen3\tenrm=1.67pt

\section{483}{The Henstock integral}{6.9.10}{869}{}
{The Henstock and Lebesgue integrals;  indefinite Henstock integrals;
Saks-Henstock lemma;  Fundamental Theorem of Calculus;  the Perron
integral;  $\|f\|_H$ and $HL^1$;  AC$_*$ and ACG$_*$ functions.}

\section{484}{The Pfeffer integral}{21.1.10}{885}{}
{The Tamanini-Giacomelli theorem;
%or Congedo-Tamanini
a family of tagged-partition structures;  the Pfeffer integral;  the
Saks-Henstock indefinite integral of a Pfeffer integrable function;
Pfeffer's Divergence
Theorem;  differentiating the indefinite integral;  invariance under
lipeomorphisms.}

\wheader{}{10}{4}{4}{100pt}

 Chapter 49:  Further topics

\chapintrosection{27.9.02}{903}{}

\section{491}{Equidistributed sequences}{14.8.08}{903}{}
{The asymptotic density ideal $\Cal{Z}$;  equidistributed sequences;
when equidistributed sequences exist;
$\frak{Z}=\Cal{P}\eightBbb{N}/\Cal{Z}$;
effectively regular measures;  equidistributed sequences and induced
embeddings of measure algebras in $\frak{Z}$.}

\section{492}{Combinatorial concentration of measure}{30.12.06}{921}{}
{The Hamming metric;
concentration of measure in product spaces;  concentration of measure
in permutation groups.}

\section{493}{Extremely amenable groups}{4.1.13}{928}{}
{Extremely amenable groups;  concentrating additive functionals;
measure algebras under $\Bsymmdiff$;  $L^0$;
isometry groups of spheres in inner product spaces;
locally compact groups.}

\section{494}{Groups of measure-preserving automorphisms}
{17.5.13}{936}{}
{Weak and uniform topologies on $\AmuA$;  a weakly mixing automorphism
which is not mixing;  full subgroups and fixed-point
subalgebras;  extreme
amenability;  automatic continuity;  algebraic cofinality.}

\section{495}{Poisson point processes}{20.12.08}{967}{}
{Poisson distributions;  Poisson point processes;  disintegrations;
transforming disjointness into stochastic independence;  representing
Poisson point processes by Radon measures;
exponential distributions and Poisson point processes on
$\coint{0,\infty}$.}

\section{496}{Maharam submeasures}{27.5.09}{989}{}
{Submeasures;  totally finite Radon submeasures;  Souslin's operation;
(K-)analytic spaces;  product submeasures.}

\section{497}{Tao's proof of Szemer\'edi's theorem}{7.12.10}{998}{}
{$\Tau$-removable intersections;  and relative independence;
permutation-invariant measures on $\Cal{P}([I]^{<\omega})$;
and $\Tau$-removable intersections;
the Hypergraph Removal Lemma;  Szemer\'edi's theorem;  a multiple
recurrence theorem.}

\section{498}{Cubes in product spaces}{4.9.08}{1010}{}
{Subsets of measure algebras with non-zero infima;  product sets
included in given sets of positive measure.}

\wheader{}{10}{4}{4}{100pt}

Appendix to Volume 4

\chapintrosection{3.9.13}{1013}{}

\section{4A1}{Set theory}{27.1.13}{1013}{}
{Cardinals;  closed cofinal sets and stationary sets;  $\Delta$-system
lemma;  free sets;  Ramsey's theorem;  the Marriage Lemma again;  filters;
normal ultrafilters;  Ostaszewski's $\clubsuit$;
the size of $\sigma$-algebras.}

\section{4A2}{General topology}{21.4.13}{1017}{}
{Glossary;  general constructions;  F$_{\sigma}$, G$_{\delta}$, zero and
cozero sets;  weight;  countable chain condition;
separation axioms;  compact and locally compact spaces;  Lindel\"of
spaces;   Stone-\v{C}ech compactifications;  uniform spaces;
first-{\vthsp}countable, sequential,
countably tight, metrizable spaces;  countable networks;
second-countable spaces;  separable
metrizable spaces;  Polish spaces;  order topologies;  Vietoris and Fell
topologies.}

\section{4A3}{Topological $\sigma$-algebras}{11.10.07}{1040}{}
{Borel $\sigma$-algebras;  measurable functions;  hereditarily
Lindel\"of spaces;  second-countable spaces;  Polish spaces;
$\omega_1$;  Baire $\sigma$-algebras;  product spaces;  compact spaces;
Baire-property algebras;  cylindrical $\sigma$-algebras;  spaces of
\cadlag{} functions.}

\section{4A4}{Locally convex spaces}{19.6.13}{1050}{}
{Linear topological spaces;  locally convex spaces;  Hahn-Banach
theorem;  normed spaces;  inner product spaces;  max-flow min-cut
theorem.}

\section{4A5}{Topological groups}{4.8.13}{1056}{}
{Group actions;  topological groups;  uniformities;  quotient groups;
metrizable groups.}

\section{4A6}{Banach algebras}{8.12.10}{1063}{}
{Stone-Weierstrass theorem (fourth form);  multiplicative linear
functionals;  spectral radius;  invertible elements;  exponentiation; 
Arens multiplication.}

\section{4A7}{`Later editions only'}{5.10.13}{1067}{}
{Items recently interpolated into other volumes.}

\wheader{}{10}{2}{2}{100pt}

% Concordance to Part II \pagereference{499}{}

\medskip

References for Volume 4 \vtmpb{3.9.03}\pagereference{1068}{}

\medskip

Index to Volumes 1-4

\qquad Principal topics and results \pagereference{1075}{}

\qquad General index \pagereference{1087}{}

%573 pages

