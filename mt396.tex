\frfilename{mt396.tex}
\versiondate{15.8.08}
\copyrightdate{1997}

\def\chaptername{Measurable algebras}
\def\sectionname{The Hajian-Ito theorem}

\newsection{396}

In the notes to the last section, I said that the argument there
short-circuits if we are told that we are dealing with a measurable
algebra.   The point is that in this case there is a much simpler
criterion for the existence of a $G$-invariant measure (396B(ii)), with
a proof which is independent of \S395 in all its non-trivial parts,
which makes it easy to prove that non-paradoxicality is sufficient as
well as necessary.

\leader{396A}{Lemma} Let $(\frak A,\bar\mu)$ be a localizable measure
algebra.

(a) Let $\pi\in\Aut\frak A$ be a Boolean automorphism\cmmnt{ (not
necessarily measure-preserving)}, and $T_{\pi}$ the corresponding
Riesz homomorphism from $L^0=L^0(\frak A)$ to itself\cmmnt{ (364P)}.
Then there is a unique $w_{\pi}\in(L^0)^+$ such that $\int w_{\pi}\times
v=\int T_{\pi}v$ for every $v\in(L^0)^+$.

(b) If $\phi$, $\pi\in\Aut\frak A$ then $w_{\pi\phi}=w_{\phi}\times
T_{\phi^{-1}}w_{\pi}$.

(c) For each $\pi\in\Aut\frak A$ we have a norm-preserving isomorphism
$U_{\pi}$ from $L^2=L^2(\frak A,\bar\mu)$ to itself defined by
setting

\Centerline{$U_{\pi}v=T_{\pi}v\times\sqrt{w_{\pi^{-1}}}$}

\noindent for every $v\in L^2$, and $U_{\pi\phi}=U_{\pi}U_{\phi}$ for
all $\pi$, $\phi\in\Aut\frak A$.

\proof{{\bf (a)} Set $\bar\nu a=\bar\mu(\pi a)$ for $a\in\frak A$.
Then $(\frak A,\bar\nu)$ is a semi-finite measure algebra.   \Prf\
$\bar\nu 0=\bar\mu 0=0$;  if $\sequencen{a_n}$ is a disjoint sequence in
$\frak A$ with supremum $a$, then $\sequencen{\pi a_n}$ is disjoint and
(because $\pi$ is sequentially order-continuous)
$a=\sup_{n\in\Bbb N}\pi a_n$, so
$\bar\nu a=\sum_{n=0}^{\infty}\bar\nu a_n$;   if $a\ne 0$ then
$\pi a\ne 0$ so $\bar\nu a>0$.    Thus $(\frak A,\bar\nu)$ is a measure
algebra.   If $a\ne 0$ there is a $b\Bsubseteq\pi a$ such that
$0<\bar\mu b<\infty$, and now $\pi^{-1}b\Bsubseteq a$ and
$0<\bar\nu(\pi^{-1}b)<\infty$;  thus $\bar\nu$ is semi-finite.\ \Qed

By 365T, there is a unique $w_{\pi}\in(L^0)^+$ such that
$\int_aw_{\pi}=\bar\mu(\pi a)$ for every $a\in\frak A$.   If we look at

\Centerline{$W=\{v:v\in (L^0)^+$, $\int v\times w_{\pi}
=\int T_{\pi}v\}$,}

\noindent we see that $W$ contains $\chi a$ for every $a\in\frak A$,
that $v+v'\in W$ and $\alpha v\in W$ whenever $v$, $v'\in W$ and
$\alpha\ge 0$, and that $\sup_{n\in\Bbb N}v_n\in W$ whenever
$\sequencen{v_n}$ is a non-decreasing sequence in $W$ which is bounded
above in $L^0$.   By 364Jd, $W=(L^0)^+$, as required.

\medskip

{\bf (b)} For any $v\in(L^0)^+$,

$$\eqalignno{\int w_{\pi\phi}\times v
&=\int T_{\pi\phi}v
=\int T_{\pi}T_{\phi}v\cr
\noalign{\noindent (364Pe)}
&=\int w_{\pi}\times T_{\phi}v
=\int T_{\phi}(T_{\phi^{-1}}w_{\pi}\times v)\cr
\noalign{\noindent (recalling that $T_{\phi}$ is multiplicative)}
&=\int w_{\phi}\times T_{\phi^{-1}}w_{\pi}\times v.\cr}$$

\noindent As $v$ is arbitrary (and $(\frak A,\bar\mu)$ is semi-finite),
$w_{\pi\phi}=w_{\phi}\times T_{\phi^{-1}}w_{\pi}$.

\woddheader{396A}{6}{2}{2}{36pt}

{\bf (c)(i)} For any $v\in L^0$,

\Centerline{$\int(T_{\pi}v\times\sqrt{w_{\pi^{-1}}})^2
=\int T_{\pi}v^2\times w_{\pi^{-1}}
=\int T_{\pi^{-1}}T_{\pi}v^2
=\int v^2$.}

\noindent So $U_{\pi}v\in L^2$
and $\|U_{\pi}v\|_2=\|v\|_2$ whenever $v\in L^2$, and $U_{\pi}$ is a
norm-preserving operator on $L^2$.

\medskip

\quad{\bf (ii)} Now consider $U_{\pi\phi}$.   For any $v\in L^2$, we
have

$$\eqalignno{U_{\pi}U_{\phi}v
&=T_{\pi}(T_{\phi}v\times\sqrt{w_{\phi^{-1}}})
      \times\sqrt{w_{\pi^{-1}}}\cr
&=T_{\pi}T_{\phi}v
  \times\sqrt{T_{\pi}w_{\phi^{-1}}\times w_{\pi^{-1}}}\cr
\noalign{\noindent (using 364Pd)}
&=T_{\pi\phi}v\times\sqrt{w_{\phi^{-1}\pi^{-1}}}\cr
\noalign{\noindent (by (b) above)}
&=U_{\pi\phi}v.\cr}$$

\noindent So $U_{\pi\phi}=U_{\pi}U_{\phi}$.

\medskip

\quad{\bf (iii)} Writing $\iota$ for the identity operator on $\frak A$,
we see that $T_{\iota}$ is the identity operator on $L^0$,
$w_{\iota}=\chi 1$ and $U_{\iota}$ is the identity operator on $L^2$.
Since $U_{\pi^{-1}}U_{\pi}=U_{\pi}U_{\pi^{-1}}=U_{\iota}$,
$U_{\pi}:L^2\to L^2$ is an isomorphism, with inverse $U_{\pi^{-1}}$, for
every $\pi\in\Aut\frak A$.
}%end of proof of 396A

\leader{396B}{Theorem}\cmmnt{ ({\smc Hajian \& Ito 69})} Let $\frak A$
be a measurable algebra and $G$ a subgroup of $\Aut\frak A$.   Then the
following are equiveridical:

(i) there is a $G$-invariant functional $\bar\nu$ such that
$(\frak A,\bar\nu)$ is a totally finite measure algebra;

(ii) whenever $a\in\frak A\setminus\{0\}$ and $\sequencen{\pi_n}$ is a
sequence in $G$, $\sequencen{\pi_na}$ is not disjoint;

(iii) $G$ is fully non-paradoxical\cmmnt{ (definition:  395E)}.

\proof{{\bf (a)} (i)$\Rightarrow$(iii) by the argument of 395F, and
(iii)$\Rightarrow$(ii) by the criterion (ii) of 395E.   So for the rest
of the proof I assume that (ii) is true and seek to prove (i).

\medskip

{\bf (b)} Let $\bar\mu$ be such that $(\frak A,\bar\mu)$ is a totally
finite measure algebra.   If $a\in\frak A\setminus\{0\}$, then
$\inf_{\pi\in G}\bar\mu(\pi a)>0$.   \Prf\Quer\ Otherwise, let
$\sequencen{\pi_n}$ be a sequence in $G$ such that
$\bar\mu\pi_na\le 2^{-n}$ for each $n\in\Bbb N$.   Set
$b_n=\sup_{k\ge n}\pi_ka$ for each
$n$;  then $\inf_{n\in\Bbb N}b_n=0$, so that

\Centerline{$\inf_{n\in\Bbb N}\pi b_n=0$,
\quad$\lim_{n\to\infty}\bar\mu(\pi\pi_na)=0$}

\noindent for every $\pi\in\Aut\frak A$.   Choose $\sequence{i}{n_i}$
inductively so that

\Centerline{$\bar\mu(\pi_{n_i}^{-1}\pi_{n_j}a)\le 2^{-j-2}\bar\mu a$}

\noindent whenever $i<j$.   Set

\Centerline{$c=a\Bsetminus\sup_{i<j}\pi_{n_i}^{-1}\pi_{n_j}a$.}

\noindent Because

\Centerline{$\sum_{j=1}^{\infty}\sum_{i=0}^{j-1}
  \bar\mu(\pi_{n_i}^{-1}\pi_{n_j}a)
<\bar\mu a$,}

\noindent $c\ne 0$, while $\pi_{n_i}c\Bcap\pi_{n_j}c=0$ whenever $i<j$,
contrary to the hypothesis (ii).\ \Bang\Qed

\medskip

{\bf (c)} For each $\pi\in G$, define $w_{\pi}\in L^0=L^0(\frak A)$ and
$U_{\pi}:L^2\to L^2$ as in 396A, where $L^2=L^2(\frak A,\bar\mu)$.   If
$a\in\frak A\setminus\{0\}$, then
$\inf_{\pi\in G}\int_a\sqrt{w_{\pi}}>0$.   \Prf\Quer\ Otherwise, there
is a sequence $\sequencen{\pi_n}$ in $G$ such that
$\int_av_n\le 4^{-n-2}\bar\mu a$ for every $n$, where
$v_n=\sqrt{w_{\pi_n}}$.   In this case,
$\bar\mu(a\Bcap\Bvalue{v_n\ge 2^{-n}})\le 2^{-n-2}\bar\mu a$ for every
$n$, so that $b=a\Bsetminus\sup_{n\in\Bbb N}\Bvalue{v_n\ge 2^{-n}}$ is
non-zero.   But now

\Centerline{$\bar\mu(\pi_nb)=\int_bw_{\pi_n}
=\int_bv_n^2\le 4^{-n}\bar\mu b\to 0$}

\noindent as $n\to\infty$, contradicting (b) above.\ \Bang\Qed

\medskip

{\bf (d)} Write $e=\chi 1$ for the standard weak order unit of $L^0$ or
$L^2$.   Let $C\subseteq L^2$ be the convex hull of $\{U_{\pi}e:\pi\in
G\}$.   Then $C$ and its norm closure $\overline{C}$ are $G$-invariant
in the sense that $U_{\pi}v\in C$, $U_{\pi}v'\in\overline{C}$ whenever
$v\in C$, $v'\in\overline{C}$ and $\pi\in G$.   By 3A5Md, there is a
unique $u_0\in\overline{C}$ such that $\|u_0\|_2\le\|u\|_2$ for every
$u\in\overline{C}$.   Now if $\pi\in G$, $U_{\pi}u_0\in\overline{C}$,
while $\|U_{\pi}u_0\|_2=\|u_0\|_2$; so $U_{\pi}u_0=u_0$.   Also, if
$a\in\frak A\setminus\{0\}$,

$$\eqalignno{\int_au_0
&\ge\inf_{u\in\overline{C}}\int_au
=\inf_{u\in C}\int_au\cr
\noalign{\noindent (because $u\mapsto\int_au$ is $\|\,\|_2$-continuous)}
&=\inf_{\pi\in G}\int_aU_{\pi}e
=\inf_{\pi\in G}\int_aT_{\pi}e\times\sqrt{w_{\pi^{-1}}}\cr
&=\inf_{\pi\in G}\int_a\sqrt{w_{\pi^{-1}}}
=\inf_{\pi\in G}\int_a\sqrt{w_{\pi}}
>0\cr}$$

\noindent by (c).   So $\Bvalue{u_0>0}=1$.

\medskip

{\bf (e)} For $a\in\frak A$, set $\bar\nu a=\int_au_0^2$.   Because
$u_0\in L^2$, $\bar\nu$ is a non-negative countably additive functional
on $\frak A$;  because $\Bvalue{u_0^2>0}=\Bvalue{u_0>0}=1$, $\bar\nu$ is
strictly positive, and $(\frak A,\bar\nu)$ is a totally finite measure
algebra.   Finally, $\bar\nu$ is $G$-invariant.   \Prf\ If $a\in\frak A$
and $\pi\in G$, then

$$\eqalignno{\bar\nu(\pi a)
&=\int_{\pi a}u_0^2
=\int u_0^2\times\chi(\pi a)
=\int T_{\pi}(T_{\pi^{-1}}u_0^2\times\chi a)\cr
&=\int w_{\pi}\times T_{\pi^{-1}}u_0^2\times\chi a
=\int_a(T_{\pi^{-1}}u_0\times\sqrt{w_{\pi}})^2\cr
&=\int_a(U_{\pi^{-1}}u_0)^2
=\int_au_0^2
=\bar\nu a.\text{ \Qed}\cr}$$

\noindent So (i) is true.
}%end of proof of 396B

\cmmnt{
\leader{396C}{Remark} If $\frak A$ is a Boolean algebra and $G$
a subgroup of $\Aut\frak A$, a non-zero element $a$ of $\frak A$ is
called {\bf weakly wandering} if there is a sequence $\sequencen{\pi_n}$
in $G$ such that $\sequencen{\pi_na}$ is disjoint.   Thus condition (ii)
of 396B may be read as `there is no weakly wandering element of
$\frak A$'.
}%end of comment

\exercises{\leader{396X}{Basic exercises (a)}
%\spheader 396Xa
Let $(\frak A,\bar\mu)$ be a totally finite measure algebra, and
$\pi:\frak A\to\frak A$ an order-continuous Boolean homomorphism.   Let
$T_{\pi}:L^0(\frak A)\to L^0(\frak A)$ be the corresponding Riesz
homomorphism.   Show that there is a unique $w_{\pi}\in L^1(\frak
A,\bar\mu)$ such that $\int T_{\pi}v=\int v\times w_{\pi}$ for every
$v\in L^0(\frak A)^+$.
%396A

\spheader 396Xb In 396A, show that the map
$\pi\mapsto U_{\pi}:\Aut\frak A\to\eurm B(L^2;L^2)$ is injective.
%396A

\spheader 396Xc Let $\frak A$ be a measurable algebra and $G$ a subgroup
of $\Aut\frak A$.   Suppose that there is a strictly positive
$G$-invariant finitely additive functional on $\frak A$.   Show that
there is a $G$-invariant $\bar\mu$ such that  $(\frak A,\bar\mu)$ is a
totally finite measure algebra.
%396B

\leader{396Y}{Further exercises (a)}
Let $\frak A$ be a \wsid\ Dedekind complete Boolean algebra and $G$ a
subgroup of $\Aut\frak A$.   For $a$, $b\in\frak A$, say that $a$ and
$b$ are {\bf $G$-equidecomposable} if there are {\it finite} partitions
of unity $\familyiI{a_i}$ in $\frak A_a$ and $\familyiI{b_i}$ in
$\frak A_b$, and a family $\familyiI{\pi_i}$ in $G$, such that
$\pi_ia_i=b_i$
for every $i\in I$.   Show that the following are equiveridical:  (i)
$G$ is fully non-paradoxical in the sense of 395E;  (ii) if
$\sequencen{a_n}$ is a disjoint sequence of mutually
$G$-equidecomposable elements of $\frak A$, they must all be $0$.
%396B
}%end of exercises

\endnotes{
\Notesheader{396}
I have separated these few pages from \S395 partly because \S395 was
already up to full weight and partly in order that the ideas here should
not be entirely overshadowed by those of the earlier section.   It will
be evident that the construction of the $U_{\pi}$ in 396A, providing us
with a faithful representation, acting on a Hilbert space, of the whole
group $\Aut\frak A$, is a basic tool for the study of that group.
}%end of notes

\frnewpage


