\frfilename{mt247.tex} 
\versiondate{26.8.13} 
\copyrightdate{1995} 
      
\def\chaptername{Function spaces} 
\def\sectionname{Weak compactness in $L^1$} 
      
\newsection{247} 
      
I now come to the most striking feature of uniform integrability:  it 
provides a description of the relatively weakly compact subsets of 
$L^1$ (247C).   I have put this into a separate section because it 
demands some knowledge of functional analysis -- in 
particular, of course, of weak topologies on Banach spaces. 
I will try to give an account in terms which are accessible to 
novices in the theory of normed spaces because the result is 
essentially measure-theoretic, as 
well as being of vital importance to applications in probability theory. 
I have written out the essential definitions in  
\S\S2A3-2A5.  %2A3 2A4 2A5 
      
\leader{247A}{}\cmmnt{ Part of the argument of the main theorem below 
will run more smoothly if I separate out an idea which is, in effect, a 
simple special case of a theme which has been running through the 
exercises of this chapter (241Yg, 242Yb, 243Ya, 244Yd). 
      
\medskip 
      
\noindent}{\bf Lemma} Let $(X,\Sigma,\mu)$ be a measure space, and $G$ 
any member of $\Sigma$.   Let $\mu_G$ be the subspace measure on 
$G$\cmmnt{, so 
that $\mu_GE=\mu E$ when $E\subseteq G$ and $E\in\Sigma$}.   Set 
      
\Centerline{$U=\{u:u\in L^1(\mu),\,u\times\chi G^{\ssbullet}=u\} 
\subseteq L^1(\mu)$.} 
      
\noindent Then we have 
an isomorphism $S$ between the ordered normed spaces $U$ and 
$L^1(\mu_G)$, given by writing 
      
\Centerline{$S(f^{\ssbullet})=(f\restr G)^{\ssbullet}$} 
      
\noindent for every $f\in\eusm L^1(\mu)$ such that $f^{\ssbullet}\in 
U$. 
      
\proof{ Of course I should remark explicitly that $U$ is a linear 
subspace of $L^1(\mu)$.   I have discussed integration over subspaces in 
\S\S131 and 214;  in particular, I noted that $f\restr G$ is integrable, 
and that 
      
\Centerline{$\int|f\restr G|d\mu_G 
=\int|f|\times\chi G\,d\mu\le\int|f|d\mu$} 
      
\noindent for every $f\in\eusm L^1(\mu)$ (131Fa).   If $f$, 
$g\in\eusm L^1(\mu)$ and 
$f=g\,\,\mu$-a.e., then $f\restr G=g\restr G\,\,\mu_G$-a.e.;  so the 
proposed formula for $S$ does indeed define a map from $U$ to 
$L^1(\mu_G)$. 
      
Because 
      
\Centerline{$(f+g)\restr G=(f\restr G)+(g\restr G)$, 
\quad$(cf)\restr G=c(f\restr G)$} 
      
\noindent for all $f$, $g\in\eusm L^1(\mu)$ and all $c\in\Bbb R$, $S$ is 
linear.   Because 
      
\Centerline{$f\le g\,\,\mu$-a.e.$\,\Longrightarrow\, 
f\restr G\le g\restr G\,\,\mu_G$-a.e.,} 
      
\noindent $S$ is order-preserving.   Because $\int|f\restr 
G|d\mu_G\le\int|f|d\mu$ for every $f\in\eusm L^1(\mu)$, 
$\|Su\|_1\le\|u\|_1$ for every $u\in U$. 
      
To see that $S$ is surjective, take any $v\in L^1(\mu_G)$.   Express $v$ 
as $g^{\ssbullet}$ where $g\in\eusm L^1(\mu_G)$.   By 131E, 
$f\in\eusm L^1(\mu)$, where $f(x)=g(x)$ for $x\in\dom g$, $0$ for 
$x\in X\setminus G$;  so that $f^{\ssbullet}\in U$ and 
$f\restr G=g$ and $v=S(f^{\ssbullet})\in S[U]$. 
      
To see that $S$ is norm-preserving, note that, 
for any $f\in\eusm L^1(\mu)$, 
      
\Centerline{$\int|f\restr G|d\mu_G 
=\int |f|\times\chi G\,d\mu$,} 
      
\noindent so that if $u=f^{\ssbullet}\in U$ we shall have 
      
\Centerline{$\|Su\|_1 
=\int|f\restr G|d\mu_G 
=\int|f|\times\chi G\,d\mu 
=\|u\times\chi G^{\ssbullet}\|_1 
=\|u\|_1$.} 
} 
      
\leader{247B}{Corollary} Let $(X,\Sigma,\mu)$ be any measure space, and 
let $G\in\Sigma$ be a measurable set expressible as a countable union 
of sets of finite measure.   Define $U$ as in 247A, and let 
$h:L^1(\mu)\to\Bbb R$ be any continuous linear functional.   Then there 
is a $v\in L^{\infty}(\mu)$ such that $h(u)=\int u\times v\,d\mu$ for 
every $u\in U$. 
      
\proof{ Let $S:U\to L^1(\mu_G)$ be the isomorphism described in 247A. 
Then $S^{-1}:L^1(\mu_G)\to U$ is linear and continuous, so 
$h_1=hS^{-1}$ belongs to the normed space dual $(L^1(\mu_G))^*$ of 
$L^1(\mu_G)$.   Now of course $\mu_G$ is $\sigma$-finite, therefore 
localizable (211L), so 243Gb tells us that there is a 
$v_1\in L^{\infty}(\mu_G)$ such that 
      
\Centerline{$h_1(u)=\int u\times v_1d\mu_G$} 
      
\noindent for every $u\in L^1(\mu_G)$. 
      
Express $v_1$ as $g_1^{\ssbullet}$ where $g_1:G\to\Bbb R$ is a 
bounded measurable function.   Set $g(x)=g_1(x)$ for $x\in G$, $0$ for 
$x\in X\setminus G$;  then $g:X\to\Bbb R$ is a bounded measurable 
function, and $v=g^{\ssbullet}\in L^{\infty}(\mu)$.   If $u\in U$, 
express $u$ as $f^{\ssbullet}$ where $f\in\eusm L^1(\mu)$;  then 
      
$$\eqalignno{h(u) 
&=h(S^{-1}Su) 
=h_1((f\restr G)^{\ssbullet}) 
=\int(f\restr G)\times g_1d\mu_G\cr 
&=\int(f\times g)\restr G\,d\mu_G 
=\int f\times g\times\chi G\,d\mu 
=\int f\times g\,d\mu 
=\int u\times v.\cr}$$ 
      
\noindent As $u$ is arbitrary, this proves the result. 
}  
      
\leader{247C}{Theorem} Let $(X,\Sigma,\mu)$ be any measure space and $A$ 
a subset of $L^1={L}^1(\mu)$.    Then $A$ is uniformly integrable iff it 
is relatively compact in $L^1$ for the weak topology of $L^1$. 
      
\proof{{\bf (a)} Suppose that $A$ is relatively 
compact for the weak topology.   I seek to show that it satisfies the 
condition (iii) of 246G. 
      
\medskip 
      
\quad{\bf (i)} If $F\in\Sigma$, then surely 
$\sup_{u\in A}|\int_Fu|<\infty$, because $u\mapsto\int_Fu$ belongs to 
$(L^1)^*$, 
and if $h\in (L^1)^*$ then the image of any relatively weakly compact 
set under $h$ must be bounded (2A5Ie). 
      
\medskip 
      
\quad{\bf (ii)} Now suppose that 
$\sequencen{F_n}$ is a disjoint sequence in $\Sigma$.   \Quer\ Suppose, 
if possible, that 
\discrcenter{468pt}{$\sequencen{\sup_{u\in A}|\int_{F_n}u|}$ }does not converge to $0$. 
Then there is a strictly increasing sequence 
$\sequence{k}{n(k)}$ in $\Bbb N$ such that 
      
\Centerline{$\gamma 
=\Bover12\inf_{k\in\Bbb N}\sup_{u\in A}|\int_{F_{n(k)}}u|>0$.} 
      
\noindent    For each $k$, choose $u_k\in A$ such that 
$|\int_{F_{n(k)}}u_k|\ge\gamma$.    Because $A$ is 
relatively compact for the weak topology, there is a cluster point $u$ 
of $\sequence{k}{u_{k}}$ in $L^1$ for the weak topology (2A3Ob). 
Set $\eta_j=2^{-j}\gamma/6>0$ for each $j\in\Bbb N$. 
      
We can now choose a strictly increasing sequence $\sequence{j}{k(j)}$ 
inductively so that, for each $j$, 
      
\Centerline{$\int_{F_{n(k(j))}}(|u|+\sum_{i=0}^{j-1}|u_{k(i)}|) 
\le\eta_j$} 
      
\Centerline{$\sum_{i=0}^{j-1} 
|\int_{F_{n(k(i))}}u-\int_{F_{n(k(i))}}u_{k(j)}|\le\eta_j$} 
      
\noindent for every $j$, interpreting $\sum_{i=0}^{-1}$ as $0$.   \Prf\ 
Given $\langle k(i)\rangle_{i<j}$, set $v^*=|u| 
+\sum_{i=0}^{j-1}|u_{k(i)}|$;  then 
$\lim_{k\to\infty}\int_{F_{n(k)}}v^*=0$, by Lebesgue's Dominated 
Convergence Theorem or otherwise, so there is a $k^*$ such that 
$k^*>k(i)$ for every $i<j$ and $\int_{F_{n(k)}}v^*\le\eta_j$ for every 
$k\ge k^*$.   Next, 
      
\Centerline{$w\mapsto\sum_{i=0}^{j-1} 
|\int_{F_{n(k(i))}}u-\int_{F_{n(k(i))}}w|:L^1\to\Bbb R$} 
      
\noindent is continuous for the weak topology of $L^1$ and zero at $u$, 
and $u$ belongs 
to every weakly open set containing $\{u_{k}:k\ge k^*\}$, so there is a 
$k(j)\ge k^*$ such that 
$\sum_{i=0}^{j-1} 
|\int_{F_{n(k(i))}}u-\int_{F_{n(k(i))}}u_{k(j)}|<\eta_j$, which 
continues the construction. \Qed 
      
Let $v$ be any cluster point in $L^1$, for the weak topology, of 
$\sequence{j}{u_{k(j)}}$.   Setting $G_i=F_{n(k(i))}$, we have 
$|\int_{G_i}u-\int_{G_i}u_{k(j)}|\le\eta_j$ whenever $i<j$, so 
$\lim_{j\to\infty}\int_{G_i}u_{k(j)}$ exists $=\int_{G_i}u$ for each 
$i$, and $\int_{G_i}v=\int_{G_i}u$ for every $i$;  setting 
$G=\bigcup_{i\in\Bbb N}G_i$, 
      
\Centerline{$\int_Gv=\sum_{i=0}^{\infty}\int_{G_i}v 
=\sum_{i=0}^{\infty}\int_{G_i}u=\int_Gu$,} 
      
\noindent by 232D, because $\sequence{i}{G_i}$ is disjoint. 
      
For each $j\in\Bbb N$, 
      
$$\eqalignno{\sum_{i=0}^{j-1}|\int_{G_i}u_{k(j)}| 
&+\sum_{i=j+1}^{\infty}|\int_{G_i}u_{k(j)}|\cr 
&\le\sum_{i=0}^{j-1}\int_{G_i}|u| 
+\sum_{i=0}^{j-1}|\int_{G_i}u-\int_{G_i}u_{k(j)}| 
+\sum_{i=j+1}^{\infty}\int_{G_i}|u_{k(j)}|\cr 
&\le\sum_{i=0}^{j-1}\eta_i+\eta_j+\sum_{i=j+1}^{\infty}\eta_i 
=\sum_{i=0}^{\infty}\eta_i 
=\Bover{\gamma}{3}.\cr}$$ 
      
\noindent On the other hand, $|\int_{G_j}u_{k(j)}|\ge\gamma$.   So 
      
\Centerline{$|\int_Gu_{k(j)}| 
=|\sum_{i=0}^{\infty}\int_{G_i}u_{k(j)}| 
\ge\Bover23\gamma$.} 
      
This is true for every $j$;  because every weakly open set containing 
$v$ meets $\{u_{k(j)}:j\in\Bbb N\}$, $|\int_Gv|\ge\bover23\gamma$ and 
$|\int_Gu|\ge\bover23\gamma$.   On the other hand, 
      
\Centerline{$|\int_Gu| 
=|\sum_{i=0}^{\infty}\int_{G_i}u| 
\le\sum_{i=0}^{\infty}\int_{G_i}|u| 
\le\sum_{i=0}^{\infty}\eta_i 
=\Bover{\gamma}{3}$,} 
      
\noindent which is absurd.  \Bang 
      
This contradiction shows that 
$\lim_{n\to\infty}\sup_{u\in A}|\int_{F_n}u|=0$.   As $\sequencen{F_n}$ 
is arbitrary, $A$ satisfies 
the condition 246G(iii) and is uniformly integrable. 
      
\medskip 
      
{\bf (b)} Now assume that $A$ is uniformly integrable.  I seek a weakly 
compact set $C\supseteq A$. 
      
\medskip 
      
\quad{\bf (i)} For 
each $n\in\Bbb N$, choose $E_n\in\Sigma$, $M_n\ge 0$ such that 
$\mu E_n<\infty$ and $\int(|u|-M_n\chi E_n^{\ssbullet})^+\le 2^{-n}$ for 
every $u\in A$.   Set 
      
\Centerline{$C=\{v:v\in L^1,\,|\int_Fv|\le M_n\mu(F\cap E_n) 
  +2^{-n}\Forall n\in\Bbb N,\,F\in\Sigma\}$,} 
      
\noindent and note that $A\subseteq C$, because if $u\in A$ and 
$F\in\Sigma$, 
      
\Centerline{$|\int_Fu| 
\le\int_F(|u|-M_n\chi E_n^{\ssbullet})^++\int_FM_n\chi E_n^{\ssbullet} 
\le 2^{-n}+M_n\mu(F\cap E_n)$} 
      
\noindent for every $n$.   Observe also that $C$ is $\|\,\|_1$-bounded, 
because 
      
\Centerline{$\|u\|_1\le 2\sup_{F\in\Sigma}|\int_Fu| 
\le 2\sup_{F\in\Sigma}(1+M_0\mu(F\cap E_0))\le 2(1+M_0\mu E_0)$} 
      
\noindent for every $u\in C$ (using 246F). 
      
\medskip 
      
\quad{\bf (ii)} Because I am seeking to prove this theorem for 
arbitrary measure spaces $(X,\Sigma,\mu)$, I cannot use 243G to identify 
the dual of $L^1$.   Nevertheless, 247B above shows that 243Gb it is 
`nearly' valid, in the following 
sense:  if $h\in(L^1)^*$, there is a $v\in L^{\infty}$ such that 
$h(u)=\int u\times v$ for every $u\in C$.  \Prf\ Set 
$G=\bigcup_{n\in\Bbb N}E_n\in\Sigma$, and define $U\subseteq L^1$ as 
in 247A-247B.   By 247B, there is a $v\in L^{\infty}$ such that 
$h(u)=\int u\times v$ for every $u\in U$.   But if $u\in C$, 
we can express $u$ as $f^{\ssbullet}$ where $f:X\to\Bbb R$ is 
measurable.   If $F\in\Sigma$ and $F\cap G=\emptyset$, then 
      
\Centerline{$|\int_Ff|=|\int_Fu|\le 2^{-n}+M_n\mu(F\cap E_n) 
=2^{-n}$} 
      
\noindent for every $n\in\Bbb N$, so $\int_Ff=0$;  it follows that $f=0$ 
a.e.\ on $X\setminus G$ (131Fc), so that $f\times\chi G\eae f$ and 
$u=u\times\chi G^{\ssbullet}$, that is, $u\in U$, and 
$h(u)=\int u\times v$, as required. 
\Qed 
            
\medskip 
      
\quad{\bf (iii)} So we may proceed, having an adequate description, not 
of $(L^1(\mu))^*$ itself, but of its action on $C$. 
      
Let $\Cal F$ be any ultrafilter on $L^1$ containing $C$ (see 2A3R). 
For each $F\in\Sigma$, set 
      
\Centerline{$\nu F=\lim_{u\to\Cal F}\int_Fu$;} 
      
\noindent because 
      
\Centerline{$\sup_{u\in C}|\int_Fu|\le\sup_{u\in C}\|u\|_1<\infty$,} 
      
\noindent this is well-defined in $\Bbb R$ (2A3S(e-ii)).   If $E$, $F$ 
are disjoint 
members of $\Sigma$, then $\int_{E\cup F}u=\int_Eu+\int_Fu$ for every 
$u\in C$, so 
      
\Centerline{$\nu(E\cup F) 
=\lim_{u\to\Cal F}\int_{E\cup F}u 
=\lim_{u\to\Cal F}\int_{E}u+\lim_{u\to\Cal F}\int_{F}u 
=\nu E+\nu F$} 
      
\noindent (2A3Sf).   Thus 
$\nu:\Sigma\to\Bbb R$ is additive.   Next, it is truly continuous with 
respect to $\mu$.   \Prf\ Given $\epsilon>0$, take $n\in\Bbb N$ such 
that $2^{-n}\le\bover12\epsilon$, set $\delta=\epsilon/2(M_n+1)>0$ and 
observe that 
      
\Centerline{$|\nu F|\le\sup_{u\in C}|\int_Fu| 
\le 2^{-n}+M_n\mu(F\cap E_n)\le\epsilon$} 
      
\noindent whenever $\mu(F\cap E_n)\le\delta$.\ \Qed\   By the 
Radon-Nikod\'ym theorem (232E), there is an $f_0\in{\eusm L}^1$ such 
that $\int_Ff_0=\nu F$ for every $F\in\Sigma$.    Set 
$u_0=f_0^{\ssbullet}\in L^1$.   If $n\in\Bbb N$, $F\in\Sigma$ then 
      
\Centerline{$|\int_Fu_0| 
=|\nu F|\le\sup_{u\in C}|\int_Fu|\le 2^{-n}+M_n\mu(F\cap E_n)$,} 
      
\noindent  so $u_0\in C$. 
      
\medskip 
      
\quad{\bf (iv)} Of course the point is that $\Cal F$ converges to $u_0$. 
\Prf\ Let $h\in(L^1)^*$.   Then there is a $v\in L^{\infty}$ such that 
$h(u)=\int u\times v$ for every $u\in C$.   Express $v$ as 
$g^{\ssbullet}$, where $g:X\to\Bbb R$ is bounded and 
$\Sigma$-measurable. 
Let $\epsilon>0$.   Take $a_0\le a_1\le\ldots\le a_n$ such that 
$a_{i+1}-a_i\le\epsilon$ for each $i$ while $a_0\le g(x)<a_n$ for each 
$x\in X$.   Set $F_i=\{x:a_{i-1}\le g(x)<a_i\}$ for $1\le i\le n$, and 
set $\tilde g=\sum_{i=1}^na_i\chi F_i$, $\tilde v=\tilde g^{\ssbullet}$; 
then 
$\|\tilde v-v\|_{\infty}\le\epsilon$.      We have 
      
$$\eqalign{\int u_0\times\tilde v 
&=\sum_{i=1}^na_i\int_{F_i}u 
=\sum_{i=1}^na_i\nu F_i\cr 
&=\sum_{i=1}^{n}a_i\lim_{u\to\Cal F}\int_{F_i}u 
=\lim_{u\to\Cal F}\sum_{i=1}^na_i\int_{F_i}u 
=\lim_{u\to\Cal F}\int u\times\tilde v.\cr}$$ 
      
\noindent Consequently 
      
$$\eqalign{\limsup_{u\to\Cal F}|\int u\times v-\int u_0\times v| 
&\le|\int u_0\times v-\int u_0\times \tilde v| 
+\sup_{u\in C}|\int u\times v-\int u\times\tilde v|\cr 
&\le\|u_0\|_1\|v-\tilde v\|_{\infty} 
+\sup_{u\in C}\|u\|_1\|v-\tilde v\|_{\infty}\cr 
&\le2\epsilon\sup_{u\in C}\|u\|_1.\cr}$$ 
      
\noindent As $\epsilon$ is arbitrary, 
      
$$\eqalign{\limsup_{u\to\Cal F}|h(u)-h(u_0)| 
&=\limsup_{u\to\Cal F}|\int u\times v-\int u_0\times v| 
=0.\cr}$$ 
      
\noindent As $h$ is arbitrary, 
$u_0$ is a limit of $\Cal F$ in $C$ for the weak topology of $L^1$.\ \Qed 
 
As $\Cal F$ is arbitrary, $C$ is weakly compact in $L^1$, and the proof is 
complete. 
}%end of proof of 247C 
      
\leader{247D}{Corollary} Let $(X,\Sigma,\mu)$ and $(Y,\Tau,\nu)$ be any 
two measure spaces, and $T:L^1(\mu)\to L^1(\nu)$ a continuous linear 
operator.   Then $T[A]$ is a uniformly integrable subset of $L^1(\nu)$ 
whenever $A$ is a uniformly integrable subset of $L^1(\mu)$. 
      
\proof{ The point is that $T$ is continuous for the respective weak 
topologies (2A5If).   If $A\subseteq L^1(\mu)$ is uniformly 
integrable, then there is a weakly compact $C\supseteq A$, by 247C; 
$T[C]$, being the image of a compact set under a continuous map, 
must be weakly compact (2A3N(b-ii));  so $T[C]$ and $T[A]$ are 
uniformly integrable by the other half of 247C. 
}%end of proof of 247D 
      
\cmmnt{ 
\leader{247E}{Complex $L^1$} There are no difficulties, and no 
surprises, in proving 247C for $L^1_{\Bbb C}$.   If we follow the same 
proof, everything works, but of course we must remember to change the 
constant when applying 246F, or rather 246K, in part (b-i) of the 
proof. 
} 
      
\exercises{ 
\leader{247X}{Basic exercises $\pmb{>}$(a)} Let $(X,\Sigma,\mu)$ be any 
measure space.  Show that if 
$A\subseteq L^1=L^1(\mu)$ is relatively weakly compact, then $\{v:v\in 
L^1,\,|v|\le |u|$ for some $u\in A\}$ is 
relatively weakly compact. 
%247C 
      
\spheader 247Xb Let $(X,\Sigma,\mu)$ be a measure space.   On 
$L^1=L^1(\mu)$ define pseudometrics $\rho_F$, $\rho'_w$ for 
$F\in\Sigma$, $w\in L^{\infty}(\mu)$ by setting 
$\rho_F(u,v)=|\int_Fu-\int_Fv|$, $\rho'_w(u,v)=|\int u\times w-\int 
v\times w|$ for $u$, $v\in L^1$.   Show that on any $\|\,\|_1$-bounded 
subset of $L^1$, the topology defined by $\{\rho_F:F\in\Sigma\}$ agrees 
with the topology generated by $\{\rho'_w:w\in L^{\infty}\}$. 
%247C 
      
\sqheader 247Xc Show that for any set $X$ a subset of $\ell^1=\ell^1(X)$ 
is compact for the weak topology of $\ell^1$ iff it is compact for the 
norm topology of $\ell^1$.   \Hint{246Xd.} 
%247C 
      
\spheader 247Xd Use the argument of (a-ii) in the proof of 247C to show 
directly that if $A\subseteq\ell^1(\Bbb N)$ is weakly compact then 
$\inf_{n\in\Bbb N}|u_n(n)|=0$ for any sequence $\sequencen{u_n}$ in $A$. 
%247C 
      
\spheader 247Xe Let $(X,\Sigma,\mu)$ and $(Y,\Tau,\nu)$ be measure 
spaces, and $T:L^2(\nu)\to L^1(\mu)$ any bounded linear operator.   Show 
that $\{Tu:u\in L^2(\nu),\,\|u\|_2\le 1\}$ is uniformly integrable in 
$L^1(\mu)$.   \Hint{use 244K to see that $\{u:\|u\|_2\le 1\}$ is weakly 
compact in $L^2(\nu)$.} 
%247C
      
\leader{247Y}{Further exercises (a)} 
%\spheader 247Ya 
Let $(X,\Sigma,\mu)$ be a measure space.   Take $1<p<\infty$ and 
$M\ge 0$ and set $A=\{u:u\in L^p=L^p(\mu),\,\|u\|_p\le M\}$.   Write 
$\frak S_A$ for the topology of convergence in measure on $A$, that is, the 
subspace topology induced by the topology of convergence in measure on 
$L^0(\mu)$.    Show that if $h\in(L^p)^*$ then $h\restr A$ is continuous 
for $\frak S_A$;   so that if $\frak T$ is the weak topology on $L^p$, 
then the subspace topology $\frak T_A$ is included in $\frak S_A$. 
%247C
      
\spheader 247Yb Let $(X,\Sigma,\mu)$ be a measure space and 
$\sequencen{u_n}$ a sequence in $L^1=L^1(\mu)$ such that 
$\lim_{n\to\infty}\int_Fu_n$ is defined for every $F\in\Sigma$.   Show 
that $\{u_n:n\in\Bbb N\}$ is weakly convergent.  \Hint{246Yh.}   
%Find 
%an alternative argument relying on 2A5J and the result of 246Yj. (?)
%247C
}%end of exercises 
      
\endnotes{\Notesheader{247} In 247D and 247Xa I try to suggest the power 
of the identification between weak compactness and uniform 
integrability.   That a continuous image of a weakly compact set should 
be weakly compact is a commonplace of functional analysis;  that the 
solid hull of a uniformly integrable set should be uniformly integrable 
is immediate from the definition.   But I see no simple arguments to 
show that a continuous image of a uniformly integrable set should be 
uniformly integrable, or that the solid hull of a weakly compact set 
should be relatively weakly compact.   (Concerning the former, an 
alternative route does exist;  see 371Xf in the next volume.) 
      
I can distinguish two important ideas in the proof of 247C.   The first, 
in (a-ii) of the proof, is a careful manipulation of sequences;  it is 
the argument needed to show that a weakly compact subset of $\ell^1$ is 
norm-compact.   (You may find it helpful to write out a solution to 
247Xd.)   The $F_{n(k)}$ and $u_k$ are chosen to mimic the situation in 
which we have a sequence in $\ell^1$ such that $u_k(k)=1$ for each $k$. 
The $k(i)$ are chosen so that the `hump' moves sufficiently rapidly 
along for $u_{k(j)}(k(i))$ to be very small whenever $i\ne j$. 
But this means that $\sum_{i=0}^{\infty}u_{k(j)}(k(i))$ (corresponding 
to $\int_Gu_{k(j)}$ in the proof) is always substantial, while 
$\sum_{i=0}^{\infty}v(k(i))$ will be small for any proposed cluster 
point $v$ of $\sequence{j}{u_{k(j)}}$.   I used similar techniques in 
\S246;  compare 246Yg. 
      
In the other half of the proof of 247C, the strategy is clearer. 
Members of $L^1$ correspond to truly continuous functionals on $\Sigma$; 
the uniform integrability of $C$ makes the corresponding set of 
functionals `uniformly truly continuous', so that any limit 
functional will also be truly continuous and will give us a member of 
$L^1$ via the Radon-Nikod\'ym theorem.   A straightforward approximation 
argument ((b-iv) in the proof, and 247Xb) shows that 
$\lim_{u\in\Cal F}\int u\times w=\int v\times w$ for every 
$w\in L^{\infty}$.   For 
localizable measures $\mu$, this would complete the proof.   For the 
general case, we need another step, here done in 247A-247B;  a uniformly 
integrable subset of $L^1$ effectively lives on a $\sigma$-finite part 
of the measure space, so that we can ignore the rest of the measure and 
suppose that we have a localizable measure space. 
      
The conditions (ii)-(iv) of 246G make it plain that weak compactness 
in $L^1$ can be effectively discussed in terms of sequences;  see also 
246Yh.   I should remark that this is a general feature of weak 
compactness in Banach spaces (2A5J).   Of course the disjoint-sequence 
formulations in 246G are characteristic of $L^1$ -- I mean that while 
there are similar results applicable elsewhere (see {\smc Fremlin 74}, 
chap.\ 8), the ideas are clearest and most dramatically expressed in 
their application to $L^1$. 
}%end of notes 
      
\frnewpage 
      
