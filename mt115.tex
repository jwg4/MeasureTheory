\frfilename{mt115.tex}
\versiondate{21.7.05}
\copyrightdate{1994}

\def\chaptername{Measure spaces}
\def\sectionname{Lebesgue measure on $\BbbR^r$}


\newsection{115}
\def\headlinesectionname{Lebesgue measure on $\eightBbb R^r$}

Following the very abstract ideas of \S\S111-113, there is an urgent
need for non-trivial examples of measure spaces.   By far the most
important examples are the Euclidean spaces $\BbbR^r$ with Lebesgue
measure, and I now
proceed to a definition of these measures (115A-115E), with a few of
their basic properties.   Except at one point (in the proof of the
fundamental lemma 115B) this section does not rely essentially on \S114;
but nevertheless most students encountering Lebesgue measure for the
first time will find it easier to work through the one-dimensional case
carefully before embarking on the multi-dimensional case.

\leader{115A}{Definitions (a)} For practically the whole of this section
(the exception is the proof of Lemma 115B) $r$ will denote a fixed
integer greater than or equal to $1$.    I will use Roman letters $a$,
$b$, $c$, $d$, $x$, $y$ to denote members of $\BbbR^r$, and Greek letters for their coordinates, so that $a=(\alpha_1,\ldots,\alpha_r)$,
$b=(\beta_1,\ldots,\beta_r)$, $x=(\xi_1,\ldots,\xi_r)$.

\header{115Ab}{\bf (b)} For the purposes of this section, a {\bf
half-open interval} in $\BbbR^r$ is a set of the form
$\coint{a,b}=\{x:\alpha_i\le \xi_i<\beta_i\Forall i\le
r\}$, where $a$, $b\in\BbbR^r$.
Observe that I allow $\beta_i\le \alpha_i$ in this formula;  if this
happens for any $i$, then $\coint{a,b}=\emptyset$.



\header{115Ac}{\bf (c)} If $I=\coint{a,b}\subseteq \BbbR^r$ is a
half-open interval, then either
$I=\emptyset$ or

\Centerline{$\alpha_i=\inf\{\xi_i:x\in I\}$,
\quad$\beta_i=\sup\{\xi_i:x\in I\}$}

\noindent for every $i\le r$;  in the latter case, the expression of $I$
as a half-open interval is unique.
We may therefore define the {\bf $r$-dimensional volume} $\lambda I$ of
a half-open interval $I$ by setting

\Centerline{$\lambda\emptyset = 0$,
\quad $\lambda\coint{a,b}=\prod_{i=1}^r\beta_i-\alpha_i$ if
$\alpha_i<\beta_i$ for every $i$.}



\leader{115B}{Lemma} If $I\subseteq\BbbR^r$ is a half-open interval and
$\langle I_j\rangle_{j\in\Bbb N}$ is a sequence of half-open intervals
covering $I$, then $\lambda I\le\sum_{j=0}^{\infty}\lambda I_j$.

\proof{ The proof is by induction on $r$.   For this proof only,
therefore, I write $\lambda_r$ for the function defined on the half-open
intervals of $\BbbR^r$ by the formula of 115Ac.

\medskip

{\bf (a)} The argument for $r=1$, starting the induction, is similar to
the inductive step;  but rather than establish a suitable convention to
set up a trivial case $r=0$, or ask you to work out the details yourself,
I refer you to 114B, which is exactly the case $r=1$.

\medskip

{\bf (b)} For the inductive step to $r+1$, where $r\ge 1$, take a
half-open interval $I\subseteq\BbbR^{r+1}$ and $\sequence{j}{I_j}$ a
sequence of half-open intervals covering $I$.   If
$I=\emptyset$ then of course $\lambda_{r+1}
I=0\le\sum_{j=0}^{\infty}\lambda_{r+1}
I_j$.   Otherwise, express
$I$ as $\coint{a,b}$, where $\alpha_i<\beta_i$ for $i\le r+1$, and each
$I_j$ as $\coint{a^{(j)},b^{(j)}}$.   Write
$\zeta=\prod_{i=1}^r\beta_i-\alpha_i$, so that
$\lambda_{r+1}I=\zeta(\beta_{r+1}-\alpha_{r+1})$.   Fix $\epsilon>0$.
For each $\xi\in \Bbb R$ let $H_{\xi}$ be the
half-space $\{x:\xi_{r+1}<\xi\}$, and consider the set

\Centerline{$A=\{\xi:\alpha_{r+1}\le \xi\le \beta_{r+1},\,
\zeta(\xi-\alpha_{r+1})
\le(1+\epsilon)\sum_{j=0}^{\infty}\lambda_{r+1}(I_j\cap H_{\xi})\}$.}

\noindent (Note that
$I_j\cap H_{\xi}
=\hbox{$\bigl[$}a^{(j)},\tilde b^{(j)}\hbox{$\bigr[$}$,
where $\tilde\beta^{(j)}_i=\beta^{(j)}_i$ for $i\le r$ and
$\tilde\beta^{(j)}_{r+1}=\min(\beta^{(j)}_{r+1},\xi)$, so
$\lambda_{r+1}(I_j\cap H_{\xi})$ is always defined.)
We have $\alpha_{r+1}\in A$, because

\Centerline{$\zeta(\alpha_{r+1}-\alpha_{r+1})=0
\le(1+\epsilon)\sum_{j=0}^{\infty}\lambda_{r+1}(I_j\cap H_{\alpha_{r+1}})$,}

\noindent and of course $A\subseteq[\alpha_{r+1},\beta_{r+1}]$, so
$\gamma=\sup A$ is defined, and
belongs to $[\alpha_{r+1},\beta_{r+1}]$.

\medskip

{\bf (c)} We find now that $\gamma\in A$.

$$\eqalign{\Prf\ \zeta(\gamma-\alpha_{r+1})
&=\sup_{\xi\in A}\zeta(\xi-\alpha_{r+1})\cr
&\le(1+\epsilon)\sup_{\xi\in A}
  \sum_{j=0}^{\infty}\lambda_{r+1}(I_j\cap H_{\xi})
\le(1+\epsilon)\sum_{j=0}^{\infty}\lambda_{r+1}(I_j\cap H_{\gamma}).
\text{ \Qed}\cr}$$

\wheader{115B}{6}{2}{2}{12pt}

{\bf (d)} \Quer\ Suppose, if possible, that $\gamma<\beta_{r+1}$.   Then
$\gamma\in\coint{\alpha_{r+1},\beta_{r+1}}$.   Set

\Centerline{$J=\{x:x\in\BbbR^r,\,(x,\gamma)\in I\}
=\coint{a',b'}$,}

\noindent where $a'=(\alpha_1,\ldots,\alpha_r)$,
$b'=(\beta_1,\ldots,\beta_r)$, and for each $j\in\Bbb N$ set

\Centerline{$J_j=\{x:x\in\BbbR^r,\,(x,\gamma)\in I_j\}$.}

\noindent Because $I\subseteq\bigcup_{j\in\Bbb N}I_j$, we must have
$J\subseteq\bigcup_{j\in\Bbb N}J_j$.   Of course both $J$ and the $J_j$
are half-open intervals in $\BbbR^r$.    (This is one of the places
where it is helpful to count the empty set as a half-open interval.)
By the inductive hypothesis,
$\zeta=\lambda_rJ\le\sum_{j=0}^{\infty}\lambda_rJ_j$.   As $\zeta>0$,
there is an $m\in\Bbb N$ such that
$\zeta\le(1+\epsilon)\sum_{j=0}^m\lambda_rJ_j$.
Now for each $j\le m$, either $J_j=\emptyset$ or
$\alpha^{(j)}_{r+1}\le\gamma<\beta^{(j)}_{r+1}$;  set

\Centerline{$\xi=\min(\{\beta_{r+1}\}\cup\{\beta^{(j)}_{r+1}:j\le
m,\,J_j\ne\emptyset\})>\gamma$.}

\noindent Then

\Centerline{$\lambda_{r+1}(I_j\cap H_{\xi})
\ge\lambda_{r+1}(I_j\cap H_{\gamma})+(\xi-\gamma)\lambda_rJ_j$}

\noindent for every $j\le m$ such that $J_j$ is non-empty, and therefore
for every $j$.   Consequently

$$\eqalign{\zeta(\xi-\alpha_{r+1})
&=\zeta(\gamma-\alpha_{r+1})+\zeta(\xi-\gamma)\cr
&\le(1+\epsilon)\sum_{j=0}^{\infty}\lambda_{r+1}(I_j\cap H_{\gamma})
+(1+\epsilon)(\xi-\gamma)\sum_{j=0}^m\lambda_rJ_j\cr
&\le(1+\epsilon)\sum_{j=m+1}^{\infty}\lambda_{r+1}(I_j\cap H_{\gamma})
+(1+\epsilon)\sum_{j=0}^m\lambda_{r+1}(I_j\cap H_{\xi})\cr
&\le(1+\epsilon)\sum_{j=0}^{\infty}\lambda_{r+1}(I_j\cap H_{\xi}),\cr}$$

\noindent and $\xi\in A$, which is impossible.\ \Bang

\medskip

{\bf (e)} We conclude that $\gamma=\beta_{r+1}$, so that
$\beta_{r+1}\in A$ and

\Centerline{$\lambda_{r+1}I=\zeta(\beta_{r+1}-\alpha_{r+1})
\le(1+\epsilon)\sum_{j=0}^n\lambda_{r+1}(I_j\cap H_{\beta_{r+1}})
\le(1+\epsilon)\sum_{j=0}^{\infty}\lambda_{r+1}I_j$.}

\noindent As $\epsilon$ is arbitrary,

\Centerline{$\lambda_{r+1}I\le\sum_{j=0}^{\infty}\lambda_{r+1}I_j$,}

\noindent as claimed.
}%end of proof of 115B

\cmmnt{\medskip
\noindent{\bf Remark} This proof is hard work, and not everybody makes
such a
\ifUSEnglish{large}\else\fi
mouthful of it.   What is perhaps a more conventional approach is
sketched in 115Ya, using the Heine-Borel
theorem to reduce the problem to one of
finite covers, and then (very often) saying that it is trivial.   I do
not use this method, partly because we do not need the Heine-Borel
theorem elsewhere in this volume (though we shall certainly need it in
Volume 2, and I write out a proof in 2A2F), and partly because I do not
agree that the lemma is trivial when we have a finite sequence
$I_0,\ldots,I_m$ covering $I$.   I invite you to consider this for
yourself.   It seems to me that any rigorous argument must involve an
induction on the dimension, which is what I provide here.   Of course
dealing throughout with an infinite sequence makes it a little harder to
keep track of what we are doing, and I note that in fact there is a
crucial step which necessitates truncation of the sequence;  I mean the
formula

\Centerline{$\xi=\min(\{\beta_{r+1}\}\cup\{\beta^{(j)}_{r+1}:
                                          j\le m,\,J_j\ne\emptyset\})$}

\noindent in part (d) of the proof.   We certainly cannot take
$\xi=\inf\{\beta^{(j)}_{r+1}:j\in\Bbb N,\,J_j\ne\emptyset\}$, since this
is very likely to be equal to $\gamma$.   Accordingly I need some excuse
for truncating, which is in the sentence

\Centerline{As $\zeta>0$, there is an $m\in\Bbb N$ such that
$\zeta\le(1+\epsilon)\sum_{j=0}^m\lambda_rJ_j$.}

\noindent And that step is the reason for introducing the slack
$\epsilon$ into the definition of the set $A$ at the beginning of the
proof.   Apart from this modification, the structure of the argument is
supposed to reflect that of 114B;  so I hope you can use the simpler
formulae of 114B as a guide here.
}%end of comment

\leader{115C}{Definition} Now, and for the rest of this section, define
$\theta:\Cal P(\BbbR^r)\to[0,\infty]$ by
writing

$$\eqalign{\theta A=\inf\{\sum_{j=0}^{\infty}\lambda I_j:\langle
I_j\rangle_{j\in\Bbb N}
\text{ is a sequence of }&\text{half-open intervals}\cr
&\text{such that }
A\subseteq\bigcup_{j\in\Bbb N}I_j\}.\cr}$$


\cmmnt{\noindent Observe that every $A$ can be covered by some sequence
of half-open intervals -- e.g.,
$A\subseteq\bigcup_{n\in\Bbb N}\coint{-\tbf{n},\tbf{n}}$, writing
$\tbf{n}=(n,n,\ldots,n)\in\BbbR^r$;  so that if we interpret
the sums in $[0,\infty]$, as in 112Bc above, we always have a non-empty
set to take the infimum of, and $\theta A$ is always defined in
$[0,\infty]$.
}%end of comment

This function $\theta$ is called {\bf Lebesgue outer measure} on
$\BbbR^{r}$;  the phrase is justified by (a) of the next proposition.


\leader{115D}{Proposition} (a) $\theta$ is an outer measure on
$\BbbR^r$.

(b) $\theta I=\lambda I$ for every half-open interval
$I\subseteq\BbbR^r$.

\proof{{\bf (a)(i)} $\theta$ takes values in $[0,\infty]$
because every $\theta A$ is the infimum of a non-empty subset of
$[0,\infty]$.

\medskip

\quad{\bf (ii)} $\theta\emptyset=0$ because (for instance) if we set
$I_j=\emptyset$ for every $j$, then every $I_j$ is a half-open interval
(on the convention I am using),
$\emptyset\subseteq\bigcup_{j\in\Bbb N}I_j$
and $\sum_{j=0}^{\infty}\lambda I_j=0$.

\medskip

\quad{\bf (iii)} If $A\subseteq B$ then whenever
$B\subseteq\bigcup_{j\in\Bbb N}I_j$ we have
$A\subseteq\bigcup_{j\in\Bbb N}I_j$, so $\theta A$ is the infimum of a set at least as large as that
involved in the definition of $\theta B$, and $\theta A\le\theta B$.

\medskip

\quad{\bf (iv)} Now suppose that $\langle A_n\rangle_{n\in\Bbb N}$ is a
sequence of subsets of $\BbbR^r$, with union $A$.   For any
$\epsilon>0$,
we can choose, for each $n\in\Bbb N$, a sequence $\langle
I_{nj}\rangle_{j\in\Bbb N}$ of half-open intervals such that
$A_n\subseteq\bigcup_{j\in\Bbb N}I_{nj}$ and
$\sum_{j=0}^{\infty}\lambda I_{nj}\le\theta A_n+2^{-n}\epsilon$.
(You should perhaps check that this formulation is valid whether
$\theta A_n$
is finite or infinite.)   Now by 111F(b-ii) there is a
bijection from $\Bbb N$ to $\Bbb N\times\Bbb N$;  express
this in the form $m\mapsto(k_m,l_m)$.   Then we find that

\Centerline{$\sum_{m=0}^{\infty}\lambda I_{k_m,l_m}
=\sum_{n=0}^{\infty}\sum_{j=0}^{\infty}\lambda I_{nj}$.}

\noindent (To see this, note that because every $\lambda I_{nj}$ is
greater than or equal to $0$, and $m\mapsto(k_m,l_m)$ is a bijection,
both sums are equal to

\Centerline{$\sup_{K\subseteq\Bbb N\times\Bbb N\text{ is finite}}
\sum_{(n,j)\in K}\lambda I_{nj}$.}

\noindent Or look at the argument written out in 114D.)   But now
$\langle I_{k_m,l_m}\rangle_{m\in\Bbb N}$ is a
sequence of half-open intervals and

\Centerline{$A=\bigcup_{n\in\Bbb N}A_n
\subseteq\bigcup_{n\in\Bbb N}\bigcup_{j\in\Bbb N}I_{nj}
=\bigcup_{m\in\Bbb N}I_{k_m,l_m}$,}

\noindent so

$$\eqalign{\theta A
&\le\sum_{m=0}^{\infty}\lambda I_{k_m,l_m}
=\sum_{n=0}^{\infty}\sum_{j=0}^{\infty}\lambda I_{nj}\cr
&\le\sum_{n=0}^{\infty}(\theta A_n+2^{-n}\epsilon)
=\sum_{n=0}^{\infty}\theta A_n+\sum_{n=0}^{\infty}2^{-n}\epsilon
=\sum_{n=0}^{\infty}\theta A_n+2\epsilon.\cr}$$

\noindent Because $\epsilon$ is arbitrary, $\theta
A\le\sum_{n=0}^{\infty}\theta A_n$ (again, you should check that this
is valid whether or not $\sum_{n=0}^{\infty}\theta A_n$ is finite).
As $\langle A_n\rangle_{n\in\Bbb N}$ is arbitrary, $\theta$ is an outer
measure.

\medskip

{\bf (b)} Because we can always take $I_0=I$, $I_j=\emptyset$ for
$j\ge 1$, to obtain a sequence of half-open intervals covering $I$ with
$\sum_{j=0}^{\infty}\lambda I_j=\lambda I$, we surely have $\theta
I\le\lambda I$.   For the reverse inequality, use 115B;  if
$I\subseteq\bigcup_{j\in\Bbb N}I_j$, then $\lambda
I\le\sum_{j=0}^{\infty}\lambda I_j$;   as
$\langle I_j\rangle_{j\in\Bbb N}$ is arbitrary, $\theta I\ge\lambda I$ and $\theta I=\lambda I$, as
required.
}%end of proof of 115D


\leader{115E}{Definition} Because Lebesgue outer
measure\cmmnt{ (115C)} is\cmmnt{ indeed}
an outer measure\cmmnt{ (115Da)},
we may use it to construct a measure $\mu$, using \Caratheodory's
method\cmmnt{ (113C)}.
This measure is {\bf Lebesgue measure on $\BbbR^r$}.
The sets $E$ for which $\mu E$ is defined\cmmnt{ (that is, for
which $\theta(A\cap E)+\theta(A\setminus E)=\theta A$ for every
$A\subseteq\BbbR^r$)} are called {\bf Lebesgue measurable}.

Sets which are negligible for $\mu$ are called {\bf Lebesgue
negligible}\cmmnt{;  note that these are just the sets $A$ for which
$\theta A=0$, and are all Lebesgue measurable (113Xa)}.


\leader{115F}{Lemma} If $i\le r$ and $\xi\in\Bbb R$, then
$H_{i\xi}=\{y:\eta_i<\xi\}$ is Lebesgue measurable.

\proof{ Write $H$ for $H_{i\xi}$.

\medskip

{\bf (a)} The point is that $\lambda I=\lambda(I\cap
H)+\lambda(I\setminus H)$ for every half-open interval
$I\subseteq\BbbR^r$.   \Prf\ If either $I\subseteq H$ or $I\cap
H=\emptyset$, this is trivial.   Otherwise, $I$ must be of the form
$\coint{a,b}$, where $\alpha_i<\xi<\beta_i$.   Now $I\cap H=\coint{a,x}$
and
$I\setminus H=\coint{y,b}$, where $\xi_j=\beta_j$ for $j\ne i$,
$\xi_i=\xi$, $\eta_j=\alpha_j$ for $j\ne i$, $\eta_i=\xi$, so both are
half-open intervals, and

$$\eqalign{\lambda(I\cap H)+\lambda(I\setminus H)
&=(\xi-\alpha_i)\prod_{j\ne i}(\beta_j-\alpha_j)
+(\beta_i-\xi)\prod_{j\ne i}(\beta_j-\alpha_j)\cr
&=(\beta_i-\alpha_i)\prod_{j\ne i}(\beta_j-\alpha_j)
=\lambda I.\text{ \Qed}\cr}$$

\medskip

{\bf (b)} Now suppose that $A$ is any subset of $\BbbR^r$, and
$\epsilon>0$.   Then we can find a sequence $\langle
I_j\rangle_{j\in\Bbb N}$
of half-open intervals such that $A\subseteq\bigcup_{j\in\Bbb N}I_j$ and
$\sum_{j=0}^{\infty}\lambda I_j\le\theta A+\epsilon$.   In this case,
$\langle I_j\cap H\rangle_{j\in\Bbb N}$ amd
$\langle I_j\setminus H\rangle_{j\in\Bbb N}$ are sequences of half-open
intervals, $A\cap H\subseteq\bigcup_{j\in\Bbb N}(I_j\cap H)$ and
$A\setminus H\subseteq\bigcup_{j\in\Bbb N}(I_j\setminus H)$.   So

$$\eqalign{\theta(A\cap H)+\theta(A\setminus H)
&\le\sum_{j=0}^{\infty}\lambda(I_j\cap
H)+\sum_{j=0}^{\infty}\lambda(I_j\setminus H) \cr
&=\sum_{j=0}^{\infty}\lambda I_j
\le\theta A+\epsilon.\cr}$$

\noindent Because $\epsilon$ is arbitrary,
$\theta(A\cap H)+\theta(A\setminus H)\le\theta A$;
because $A$ is arbitrary, $H$ is measurable, as remarked in 113D.
}%end of proof of 115F

\leader{115G}{Proposition} All Borel subsets of $\BbbR^r$ are Lebesgue
measurable;  in particular, all open sets, and all sets of the following
classes, together with countable unions of them:

\inset{open intervals
$\ooint{a,b}=\{x:x\in\BbbR^r,\,
\alpha_i<\xi_i<\beta_i\Forall i\le r\}$,
where $-\infty\le\alpha_i<\beta_i\le\infty$ for each $i\le r$;

closed intervals $[a,b]
=\{x:x\in\BbbR^r,\,\alpha_i\le\xi_i\le\beta_i
  \Forall i\le r\}$, where
$-\infty<\alpha_i<\beta_i<\infty$ for each $i\le r$.}

\noindent We have\cmmnt{ moreover} the following formula for the
measures of such sets, writing $\mu$ for Lebesgue measure:

\Centerline{$\mu\ooint{a,b}=\mu[a,b]=\prod_{i=1}^r\beta_i-\alpha_i$}

\noindent whenever $a\le b$ in $\BbbR^r$.   Consequently every
countable subset of $\BbbR^r$ is measurable and of zero measure.

\proof{{\bf (a)} I show first that all open subsets of $\BbbR^r$
are measurable.   \Prf\ Let $G\subseteq\BbbR^r$ be open.   Let
$K\subseteq\BbbQ^r\times\BbbQ^r$ be the set of pairs $(c,d)$ of
$r$-tuples of rational
numbers such that $\coint{c,d}\subseteq G$.   Now by the remarks in
111E-111F
-- specifically, 111Eb, showing that $\Bbb Q$ is countable, 111F(b-iii),
showing that the product of two countable sets is countable, and
111F(b-i), showing that subsets of countable sets are countable -- we
see, inducing
on $r$, that $\BbbQ^r$ is countable, and that $K$
is countable.   Also, every $\coint{c,d}$ is measurable, being

\Centerline{$\bigcap_{i\le r}H_{i\delta_i}\setminus H_{i\gamma_i}$,}

\noindent in the language of 115F, if $c=(\gamma_1,\ldots,\gamma_r)$ and
$d=(\delta_1,\ldots,\delta_r)$.   So, by 111Fa,
$G'=\bigcup_{(r,s)\in K}\coint{r,s}$ is measurable.

By the definition of $K$, $G'\subseteq G$.   On the other hand, if
$x\in G$, there is an $\epsilon>0$ such that $y\in G$ whenever
$\|y-x\|<\epsilon$.
Now for each $i$ there are rational numbers $\gamma_i$, $\delta_i$ such
that $\gamma_i\le\xi_i<\delta_i$ and
$\delta_i-\gamma_i\le\Bover{\epsilon}{\sqrt r}$.   If $y\in\coint{c,d}$
then $|\eta_i-\xi_i|<\Bover{\epsilon}{\sqrt r}$ for every $i$ so
$\|y-x\|<\epsilon$ and $y\in G$.   Accordingly $(c,d)\in K$ and
$x\in\coint{c,d}\subseteq G'$.
As $x$ is arbitrary, $G=G'$ and $G$ is measurable.\ \Qed

\medskip

{\bf (b)} Now the family $\Sigma$ of Lebesgue measurable sets is a
$\sigma$-algebra of subsets of $\BbbR^r$ including the family of open
sets, so must contain every Borel set, by the definition of Borel set
(111G).

\medskip

{\bf (c)} Of the types of interval considered, all the open intervals
are actually open sets, so are surely Borel.   A closed interval $[a,b]$ is expressible as the intersection
$\bigcap_{n\in\Bbb N}\ooint{a-2^{-n}\tbf{1},b+2^{-n}\tbf{1}}$ of a sequence of open intervals, so is Borel.

\medskip

{\bf (d)} To compute the measures, we already know from 115Db that
$\mu\coint{a,b}=\prod_{i=1}^r\beta_i-\alpha_i$ if $a\le b$.   For the
other types of bounded
interval, it is enough to note that if $-\infty<\alpha_i<\beta_i<\infty$
for every $i$, then


\Centerline{$\coint{a+\epsilon\tbf{1},b}
\subseteq\ooint{a,b}\subseteq[a,b]\subseteq
\coint{a,b+\epsilon\tbf{1}}$}

\noindent whenever $\epsilon>0$ in $\Bbb R$.   So

\Centerline{$\mu\ooint{a,b}\le\mu[a,b]
\le\inf_{\epsilon>0}\mu\coint{a,b+\epsilon\tbf{1}}
=\inf_{\epsilon>0}\prod_{i=1}^r(\beta_i-\alpha_1+\epsilon)
=\prod_{i=1}^r\beta_i-\alpha_i$.}

\noindent If $\beta_i=\alpha_i$ for any $i$, then we must have

\Centerline{$\mu\ooint{a,b}=\mu[a,b]=
0=\prod_{i=1}^r\beta_i-\alpha_i$.}

\noindent If $\beta_i>\alpha_i$ for every $i$, then set
$\epsilon_0=\min_{i\le r}\beta_i-\alpha_i>0$;  then

$$\eqalign{\mu[a,b]\ge\mu\ooint{a,b}
&\ge\sup_{0<\epsilon\le\epsilon_0}\mu\coint{a+\epsilon\tbf{1},b}\cr
&=\sup_{0<\epsilon\le\epsilon_0}
  \prod_{i=1}^r(\beta_i-\alpha_i-\epsilon)
=\prod_{i=1}^r\beta_i-\alpha_i.\cr}$$

\noindent So in this case

\Centerline{$\prod_{i=1}^r\beta_i-\alpha_i\le\mu\ooint{a,b}
\le\mu[a,b]\le\prod_{i=1}^r\beta_i-\alpha_i$}

\noindent and

\Centerline{$\mu\ooint{a,b}
=\mu[a,b]=\prod_{i=1}^r\beta_i-\alpha_i$.}

\medskip

{\bf (e)} By (d), $\mu\{a\}=\mu[a,a]=0$ for every $a$.   If
$A\subseteq\BbbR^r$ is countable, it is either empty or expressible as
$\{a_n:n\in\Bbb N\}$.   In the former case $\mu A=\mu\emptyset=0$;  in
the latter, $A=\bigcup_{n\in\Bbb N}\{a_n\}$ is Borel and $\mu
A\le\sum_{n=0}^{\infty}\mu\{a_n\}=0$.
}%end of proof of 115G

\exercises{
\leader{115X}{Basic exercises}  If you skipped \S114, you should now
return to 114X and assure yourself that you can do the exercises there as
well as those below.

\header{115Xa}{\bf (a)} Show that if $I$, $J$ are half-open intervals in
$\BbbR^r$, then $I\setminus J$ is expressible as the union of at most
$2r$
disjoint half-open intervals.   Hence show that (i) any finite union of
half-open intervals is expressible as a finite union of disjoint
half-open intervals (ii) any countable union of half-open intervals is
expressible as the union of a disjoint sequence of half-open intervals.

\sqheader 115Xb Write $\theta$ for Lebesgue outer measure, $\mu$
for Lebesgue
measure on $\BbbR^r$.   Show that $\theta A=\inf\{\mu E:E$ is Lebesgue
measurable, $A\subseteq E\}$ for every $A\subseteq\BbbR^r$.
\Hint{consider sets $E$ of the form $\bigcup_{j\in\Bbb N}I_j$, where
$\sequence{j}{I_j}$ is a sequence of half-open intervals.}

\header{115Xc}{\bf (c)} Let $E\subseteq\BbbR^r$ be a set of finite
measure for Lebesgue
measure $\mu$.   Show that for every $\epsilon>0$ there is a disjoint
family $I_0,\ldots,I_n$ of half-open intervals such that
$\mu(E\symmdiff\bigcup_{j\le n}I_j)\le\epsilon$.   \Hint{let
$\langle J_j\rangle_{j\in\Bbb N}$ be a sequence of half-open intervals
such that $E\subseteq\bigcup_{j\in\Bbb N}J_j$ and
$\sum_{j=0}^{\infty}\mu J_j\le\mu E+\bover{1}{2}\epsilon$.   Now take a
suitably large $m$ and express $\bigcup_{j\le m}J_j$ as a disjoint union
of half-open intervals.}

\sqheader 115Xd Suppose that $c\in\BbbR^r$.   (i) Show that
$\theta(A+c)=\theta A$ for every $A\subseteq\BbbR^r$, where
$A+c=\{x+c:x\in A\}$.   (ii)
Show that if $E\subseteq \BbbR^r$ is measurable
so is $E+c$, and that in this case $\mu(E+c)=\mu E$.

\header{115Xe}{\bf (e)} Suppose that $\gamma>0$.   (i) Show that
$\theta(\gamma A)=\gamma^r\theta A$ for every $A\subseteq\BbbR^r$, where
$\gamma A=\{\gamma x:x\in A\}$.   (ii) Show that if $E\subseteq \BbbR^r$ is
measurable so is $\gamma E$, and that in this case
$\mu(\gamma E)=\gamma^r\mu E$

\vleader{48pt}{115Y}{Further exercises (a)}
%\spheader 115Ya
(i) Suppose that $M$ is a strictly positive integer and $k_i$, $l_i$ are
integers for $1\le i\le r$.   Set $\alpha_i=k_i/M$ and
$\beta_i=l_i/M$ for each $i$, and $I=\coint{a,b}$.   Show that
$\lambda I=\#(J)/M^r$, where $J$ is $\{z:z\in\BbbZ^r$, $\bover1Mz\in I\}$.
(ii) Show that if a
half-open interval $I\subseteq\BbbR^r$ is covered by a {\it finite}
sequence $I_0,\ldots,I_m$ of half-open intervals, and all the coordinates
involved in specifying the intervals $I$, $I_0,\ldots,I_m$ are rational,
then $\lambda I\le\sum_{j=0}^m\lambda I_j$.   (iii) Assuming the
Heine-Borel theorem in the form

\inset{\noindent
whenever $[a,b]$ is a closed interval in $\BbbR^r$ which is
covered by a sequence $\sequence{j}{\ooint{a^{(j)},b^{(j)}}}$ of open
intervals, there is an $m\in\Bbb N$ such that
$[a,b]\subseteq\bigcup_{j\le m}\ooint{a^{(j)},b^{(j)}}$,}

\noindent prove 115B.   \Hint{if
$\coint{a,b}\subseteq\bigcup_{j\in\Bbb N}\coint{a^{(j)},b^{(j)}}$,
replace $\coint{a,b}$ by a smaller closed
interval and each $\coint{a^{(j)},b^{(j)}}$ by a larger open interval,
changing the volumes by adequately small amounts.}
%{\smc Facenda \& Freniche 02}

\spheader 115Yb(i)
Show that if $A\subseteq\BbbR^r$ and $\epsilon>0$, there is
an open set $G\supseteq A$ such that $\theta G\le\theta A+\epsilon$,
where $\theta$ is Lebesgue outer measure.   (ii) Show that if
$E\subseteq\BbbR^r$ is Lebesgue measurable and $\epsilon>0$, there is an
open set $G\supseteq E$ such that $\mu(G\setminus E)\le\epsilon$, where
$\mu$ is Lebesgue measure.   \Hint{consider first the case of
bounded $E$.}   (iii) Show that if $E\subseteq\BbbR^r$ is Lebesgue
measurable, there are Borel sets $H_1$, $H_2$ such that $H_1\subseteq
E\subseteq H_2$ and $\mu(H_2\setminus E)=\mu(E\setminus H_1)=0$.
\Hint{use (ii) to find $H_2$, and then consider the complement of
$E$.}

\spheader 115Yc Write $\theta$ for Lebesgue outer measure on
$\BbbR^r$.   Show
that a set $E\subseteq\BbbR^r$ is Lebesgue measurable iff
$\theta([-\tbf{n},\tbf{n}]\cap E)+\theta([-\tbf{n},\tbf{n}]\setminus
E)=(2n)^r$ for every $n\in\Bbb N$, writing $\tbf{n}=(n,\ldots,n)$.   \Hint{use 115Yb to show that
for each $n$ there are
measurable sets $F_n$, $H_n$ such that
$F_n\subseteq[-\tbf{n},\tbf{n}]\cap
E\subseteq H_n$ and $H_n\setminus F_n$ is negligible.}

\header{115Yd}{\bf (d)} Assuming that there is a set $A\subseteq\Bbb R$
which is not a Borel set, show that there is a family $\Cal E$ of
half-open intervals in $\BbbR^2$ such that $\bigcup\Cal E$ is not a
Borel set.   \Hint{consider $\Cal
E=\{\coint{\xi,1+\xi}\times\coint{-\xi,1-\xi}:\xi\in A\}$.}

\header{115Ye}{\bf (e)} Let $X$ be a set and $\Cal A$ a {\bf semiring} of
subsets of $X$, that is, a family of subsets of $X$ such that

\inset{$\emptyset\in\Cal A$,}

\inset{$E\cap F\in\Cal A$ for all $E$, $F\in\Cal A$,}

\inset{whenever $E$, $F\in\Cal A$ there are disjoint
$E_0,\ldots,E_n\in\Cal A$ such that $E\setminus F=E_0\cup\ldots\cup
E_n$.}

\noindent Let $\lambda:\Cal A\to[0,\infty]$ be a functional such that

\inset{$\lambda\emptyset=0$,}

\inset{$\lambda E=\sum_{i=0}^{\infty}\lambda E_i$ whenever $E\in\Cal A$
and $\sequence{i}{E_i}$ is a disjoint sequence in $\Cal A$ with union
$E$.}

\noindent Show that there is a measure $\mu$ on $X$ extending $\lambda$.
\Hint{use the method of 113Yi.}
}%end of exercises

\endnotes{
\Notesheader{115} In the notes to \S114 I ran over the methods so far
available to us for the construction of measure spaces.   To the list
there we can now add Lebesgue measure on $\BbbR^r$.

If you look back at \S114, you will see that I have deliberately copied
the exposition there.   I hope that this duplication will help you to see
the essential elements of the method, which are three:  a primitive
concept of volume (114A/115A);  countable subadditivity (114B/115B);  and
measurability of building blocks (114F/115F).

Concerning the `primitive concept of volume' there is not much to be
said.   The ideas of length of an interval, area of a rectangle and
volume of a cuboid go back to the beginning of mathematics.   I use
`half-open intervals', as defined in 114Aa/115Ab, for purely technical
reasons, because they fit together neatly (see 115Xa and 115Ye);
if we started with `open' or
`closed' intervals the method would still work.   One thing is perhaps
worth mentioning:  the blocks I use are all upright, with edges parallel
to the coordinate axes.   It is in fact a non-trivial exercise to prove
that a block in any other orientation has the right Lebesgue measure, and
I delay this until Chapter 26.   For the moment we are looking for the
shortest safe path to a precise definition, and the fact that rotating a
set doesn't change its Lebesgue measure will have to wait.

The big step is `countable subadditivity':  the fact that if one block is
covered by a sequence of other blocks, its volume is less than or equal
to the sum of theirs.   This is surely necessary if blocks are to be
measurable with the right measures, by 112Cd.   (What is remarkable is
that it is so nearly sufficient.)   Here we have some work to do, and in
the $r$-dimensional case there is a substantial hill to climb.   You can
do the climb in two stages if you look up the Heine-Borel theorem
(115Ya);  but as I try to explain in the remarks following 115B, I do not
think that this route avoids any of the real difficulties.

The third thing we must check is that blocks are measurable in the
technical sense described by \Caratheodory's theorem.   This is because
they are obtainable by the operations of intersection and union and
complementation from half-spaces, and half-spaces are measurable for very
straightforward reasons (114F/115F).   Now we are well away, and I do
very little more, only checking that open sets, and therefore Borel sets,
are measurable, and that closed and open intervals have the right
measures (114G/115G).   Some more properties of Lebesgue measure can be
found in \S134.   But every volume, if not quite every chapter, of this
treatise will introduce further features of this extraordinary
construction.
}%end of notes

\frnewpage


