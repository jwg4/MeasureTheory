\frfilename{mt262.tex}
\versiondate{11.8.15}
\copyrightdate{1995}

\def\dist{\mathop{\text{dist}}}

\def\chaptername{Change of variable in the integral}
\def\sectionname{Lipschitz and differentiable functions}

\newsection{262}

In preparation for the main work of this chapter in \S263, I devote a
section to two important classes of functions between Euclidean spaces.
What we really need is the essentially elementary material down to 262I,
together with the technical lemma 262M and its corollaries.   Theorem
262Q is not relied on in this volume, though I believe that it makes the
patterns which will develop more natural and comprehensible.

As in \S261, $r$ (and here also $s$) will be a strictly positive integer,
and `measurable', `negligible', `integrable' will refer to Lebesgue measure
unless otherwise stated.

\leader{262A}{Lipschitz functions} Suppose that
$\phi:D\to\BbbR^s$ is a function, where $D\subseteq\BbbR^r$.   
\cmmnt{We say that }$\phi$ is {\bf $\gamma$-Lipschitz}, where
$\gamma\in\coint{0,\infty}$, if

\Centerline{$\|\phi(x)-\phi(y)\|\le \gamma \|x-y\|$}

\noindent for all $x$, $y\in D$, writing
$\|x\|=\sqrt{\xi_1^2+\ldots+\xi_r^2}$ if
$x=(\xi_1,\ldots,\xi_r)\in\BbbR^r$,
$\|z\|=\sqrt{\zeta_1^2+\ldots+\zeta_s^2}$ if
$z=(\zeta_1,\ldots,\zeta_s)\in\BbbR^s$.   In this case, $\gamma$ is
{\bf a Lipschitz constant for $\phi$}.

A {\bf Lipschitz function} is a function $\phi$ which is
$\gamma$-Lipschitz for some $\gamma\ge 0$.
\cmmnt{Note that in this case $\phi$ has a least
Lipschitz constant (since if $A$ is the set of Lipschitz constants for
$\phi$, and $\gamma_0=\inf A$, then $\gamma_0$ is a Lipschitz constant
for $\phi$).}
\dvAnew{2015}Evidently a Lipschitz function is (uniformly) continuous.

\leader{262B}{}\cmmnt{ We need the following easy facts.

\medskip

\noindent}{\bf Lemma} Let $D\subseteq\BbbR^r$ be a set and
$\phi:D\to\BbbR^s$ a function.

(a)  $\phi$ is Lipschitz iff $\phi_i:D\to\Bbb R$
is Lipschitz for every $i$, writing
$\phi(x)=(\phi_1(x),\ldots,\phi_s(x))$ for every $x\in
D=\dom\phi\subseteq\BbbR^r$.

(b) In this case, there is a Lipschitz function
$\tilde\phi:\BbbR^r\to\BbbR^s$ extending $\phi$.

(c) If $r=s=1$ and $D=[a,b]$ is an interval,
then $\phi$ is Lipschitz iff it is absolutely continuous and has a
bounded derivative.

\proof{{\bf (a)} For any $x$, $y\in D$ and $i\le s$,

\Centerline{$|\phi_i(x)-\phi_i(y)|\le\|\phi(x)-\phi(y)\|
\le\sqrt{s}\sup_{j\le s}|\phi_j(x)-\phi_j(y)|$,}

\noindent so any Lipschitz constant for $\phi$ will be a Lipschitz
constant for every $\phi_i$, and if $\gamma_j$ is a Lipschitz constant
for $\phi_j$ for each $j$, then $\sqrt{s}\sup_{j\le s}\gamma_j$ will be
a Lipschitz constant for $\phi$.

\medskip

{\bf (b)} By (a), it is
enough to consider the case $s=1$, for if every $\phi_i$ has a Lipschitz
extension $\tilde\phi_i$, we can set
$\tilde\phi(x)=(\tilde\phi_1(x),\ldots,\tilde\phi_s(x))$ for every $x$
to obtain a Lipschitz extension of $\phi$.
Taking $s=1$, then, note that the case $D=\emptyset$ is trivial;  so
suppose that $D\ne\emptyset$.   Let $\gamma$ be a Lipschitz constant for
$\phi$, and write

\Centerline{$\tilde\phi(z)=\sup_{y\in D}\phi(y)-\gamma\|y-z\|$}

\noindent for every $z\in\BbbR^r$.   If $x\in D$, then, for any
$z\in\BbbR^r$ and $y\in D$,

\Centerline{$\phi(y)-\gamma\|y-z\|
\le\phi(x)+\gamma\|y-x\|-\gamma\|y-z\|
\le\phi(x)+\gamma\|z-x\|$,}

\noindent so that $\tilde\phi(z)\le\phi(x)+\gamma\|z-x\|$;  this shows,
in particular, that $\tilde\phi(z)<\infty$.   Also, if $z\in D$, we must
have

\Centerline{$\phi(z)-\gamma\|z-z\|\le\tilde\phi(z)
\le\phi(z)+\gamma\|z-z\|$,}

\noindent so that $\tilde\phi$ extends $\phi$.   Finally, if $w$,
$z\in\BbbR^r$ and $y\in D$,

\Centerline{$\phi(y)-\gamma\|y-w\|
\le\phi(y)-\gamma\|y-z\|+\gamma\|w-z\|
\le\tilde\phi(z)+\gamma\|w-z\|$;}

\noindent and taking the supremum over $y\in D$,

\Centerline{$\tilde\phi(w)\le\tilde\phi(z)+\gamma\|w-z\|$.}

\noindent As $w$ and $z$ are arbitrary, $\tilde\phi$ is Lipschitz.

\medskip

{\bf (c)(i)} Suppose that $\phi$ is $\gamma$-Lipschitz.  If $\epsilon>0$
and
$a\le a_1\le b_1\le\ldots\le a_n\le b_n\le b$ and
$\sum_{i=1}^nb_i-a_i\le\epsilon/(1+\gamma)$, then

\Centerline{$\sum_{i=1}^n|\phi(b_i)-\phi(a_i)|
\le\sum_{i=1}^n\gamma|b_i-a_i|
\le\epsilon$.}

\noindent As $\epsilon$ is arbitrary, $\phi$ is absolutely continuous.
If $x\in[a,b]$ and $\phi'(x)$ is defined, then

\Centerline{$|\phi'(x)|
=\lim_{y\to x}\Bover{|\phi(y)-\phi(x)|}{|y-x|}\le\gamma$,}

\noindent so $\phi'$ is bounded.

\medskip

\quad{\bf (ii)} Now suppose that $\phi$ is absolutely continuous and
that $|\phi'(x)|\le\gamma$ for every $x\in\dom\phi'$, where
$\gamma\ge 0$.   Then whenever $a\le x\le y\le b$,

\Centerline{$|\phi(y)-\phi(x)|=|\int_x^y\phi'|\le\int_x^y|\phi'|
\le\gamma(y-x)$}

\noindent (using 225E for the first equality).   As $x$ and $y$ are
arbitrary, $\phi$ is $\gamma$-Lipschitz.
}%end of proof of 262B

\cmmnt{
\leader{262C}{Remark} The argument for (b) above shows that if
$\phi:D\to\Bbb R$ is a Lipschitz function, where
$D\subseteq\Bbb R^r$, then $\phi$ has an extension to $\BbbR^r$ with the same Lipschitz
constants.   In fact it is the case that if $\phi:D\to\BbbR^s$ is a
Lipschitz function, then $\phi$ has
an extension to $\tilde\phi:\BbbR^r\to\BbbR^s$ with the same Lipschitz
constants;  this is `Kirszbraun's theorem' ({\smc Kirszbraun 34}, or
{\smc Federer 69}, 2.10.43).
}%end of comment

\leader{262D}{Proposition} If $\phi:D\to\BbbR^r$ is a $\gamma$-Lipschitz
function, where $D\subseteq\BbbR^r$, then
$\mu^*\phi[A]\le\gamma^r\mu^*A$ for every $A\subseteq D$, where $\mu$ is
Lebesgue measure on $\BbbR^r$.   In particular,
$\phi[D\cap A]$ is negligible for every negligible set $A\subseteq\BbbR^r$.

\proof{ Let $\epsilon>0$.   By 261F, there is a sequence
$\sequencen{B_n}=\sequencen{B(x_n,\delta_n)}$ of closed balls in
$\Bbb R^r$, covering $A$, such that
$\sum_{n=0}^{\infty}\mu B_n\le\mu^* A+\epsilon$ and
$\sum_{n\in\Bbb N\setminus K}\mu B_n\le\epsilon$, where
$K=\{n:n\in\Bbb N,\,x_n\in A\}$.   Set

\Centerline{$L=\{n:n\in\Bbb N\setminus K,\,B_n\cap D\ne\emptyset\}$,}

\noindent and for $n\in L$ choose $y_n\in D\cap B_n$.   Now set

$$\eqalign{B'_n
&=B(\phi(x_n),\gamma\delta_n)\text{ if }n\in K,\cr
&=B(\phi(y_n),2\gamma\delta_n)\text{ if }n\in L,\cr
&=\emptyset\text{ if }n\in\Bbb N\setminus(K\cup L).\cr}$$

\noindent Then $\phi[B_n\cap D]\subseteq B'_n$ for every $n$, so
$\phi[D\cap A]\subseteq\bigcup_{n\in\Bbb N}B'_n$, and

$$\eqalign{\mu^*\phi[A\cap D]
&\le\sum_{n=0}^{\infty}\mu B'_n
=\gamma^r\sum_{n\in K}\mu B_n+2^r\gamma^r\sum_{n\in L}\mu B_n\cr
&\le\gamma^r(\mu^*A+\epsilon)+2^r\gamma^r\epsilon.\cr}$$

\noindent As $\epsilon$ is arbitrary, $\mu^*\phi[A\cap
D]\le\gamma^r\mu^*A$, as claimed.
}%end of proof of 262D

\vleader{72pt}{262E}{Corollary} Let $\phi:D\to\BbbR^r$ be an injective
Lipschitz function, where $D\subseteq\BbbR^r$, and $f$ a measurable
function from a subset of $\BbbR^r$ to $\Bbb R$.

(a) If $\phi^{-1}$ is defined almost everywhere in a subset $H$ of
$\BbbR^r$ and $f$ is defined almost everywhere in $\BbbR^r$, then
$f\phi^{-1}$ is defined almost everywhere in $H$.

(b) If $E\subseteq D$ is Lebesgue measurable then $\phi[E]$ is
measurable.

(c) If $D$ is measurable then $f\phi^{-1}$ is measurable.

\proof{ Set

\Centerline{$C=\dom(f\phi^{-1})
=\{y:y\in\phi[D],\,\phi^{-1}(y)\in\dom f\}=\phi[D\cap\dom f]$.}

\medskip

{\bf (a)} Because $f$ is defined almost everywhere,
$\phi[D\setminus\dom f]$
is negligible.   But now

\Centerline{$C=\phi[D]\setminus \phi[D\setminus\dom f]
=\dom\phi^{-1}\setminus\phi[D\setminus\dom f]$,}

\noindent so

\Centerline{$H\setminus C
\subseteq(H\setminus\dom\phi^{-1})\cup\phi[D\setminus\dom f]$}

\noindent is negligible.

\medskip

{\bf (b)} Now suppose that $E\subseteq D$ and that $E$ is measurable.
Let $\sequencen{F_n}$ be a sequence of closed bounded
subsets of $E$ such that $\mu(E\setminus\bigcup_{n\in\Bbb N}F_n)=0$
(134Fb).   Because $\phi$ is Lipschitz, it is continuous, so
$\phi[F_n]$ is compact, therefore closed, therefore measurable for every
$n$ (2A2E, 115G);  also $\phi[E\setminus\bigcup_{n\in\Bbb N}F_n]$
is negligible, by 262D, therefore measurable.   So

\Centerline{$\phi[E]
=\phi[E\setminus\bigcup_{n\in\Bbb N}F_n]
  \cup\bigcup_{n\in\Bbb N}\phi[F_n]
$}

\noindent is measurable.

\medskip

{\bf (c)} For any $a\in\Bbb R$, take a measurable set
$E\subseteq\BbbR^r$
such that $\{x:f(x)\ge a\}=E\cap\dom f$.   Then

\Centerline{$\{y:y\in C,\,f\phi^{-1}(y)\ge a\}
=C\cap\phi[D\cap E]$.}

\noindent But $\phi[D\cap E]$ is measurable, by (b), so
$\{y:f\phi^{-1}(y)\ge a\}$ is relatively measurable in $C$.   As $a$ is
arbitrary, $f\phi^{-1}$ is measurable.
}%end of proof of 262E

\leader{262F}{\dvrocolon{Differentiability}}\cmmnt{ I come now to the
class of functions whose properties will take up most of the rest of the
chapter.

\medskip

\noindent}{\bf Definitions}  Suppose that $\phi$ is a
function from a subset $D=\dom\phi$ of $\BbbR^r$ to $\BbbR^s$.

\spheader 262Fa $\phi$  is {\bf differentiable} at $x\in D$ if there is a
real $s\times r$ matrix $T$ such that

\Centerline{$\lim_{y\to x}
\Bover{\|\phi(y)-\phi(x)-T(y-x)\|}{\|y-x\|}=0$;}

\noindent in this case we may write $T=\phi'(x)$.

\spheader 262Fb I will say that $\phi$ is {\bf differentiable relative to
its domain} at
$x$, and that $T$ is {\bf a} derivative of $\phi$ at $x$, if
$x\in D$ and for every $\epsilon>0$ there is a $\delta>0$ such that
$\|\phi(y)-\phi(x)-T(y-x)\|\le\epsilon\|y-x\|$ for every
$y\in B(x,\delta)\cap D$.

\cmmnt{
\vleader{72pt}{262G}{Remarks (a)} The standard definition in 262Fa,
involving an all-sided limit
`$\lim_{y\to x}$', implicitly requires $\phi$ to be defined on some
non-trivial ball centered on $x$, so that we can calculate
$\phi(y)-\phi(x)-T(y-x)$ for all $y$ sufficiently near $x$.   It has the
advantage that the derivative $T=\phi'(x)$ is uniquely defined (because
if $\lim_{z\to \tbf{0}}\Bover{\|T_1z-T_2z\|}{\|z\|}=0$ then

\Centerline{$\Bover{\|(T_1-T_2)z\|}{\|z\|}=\lim_{\alpha\to
0}\Bover{\|T_1(\alpha
z)-T_2(\alpha z)\|}{\|\alpha z\|}=0$}

\noindent for every non-zero $z$, so $T_1-T_2$ must be the zero matrix).
For our purposes here, there is some advantage in relaxing this slightly
to the form
in 262Fb, so that we do not need to pay special attention to the
boundary of $\dom\phi$.   In particular we find that if
$T$ is a derivative of $\phi:D\to\BbbR^s$
relative to its domain at $x$, and $x\in D'\subseteq D$, then
$T$ is a derivative of $\phi\restr D'$,
relative to its domain, at $x$.

\header{262Gb}{\bf (b)} If you have not seen this concept of
`differentiability'
before, but have some familiarity with partial differentiation, it is
necessary to emphasize that the concept of `differentiable' function
(at least in the strict sense demanded by 262Fa) is strictly
stronger than the concept of `partially differentiable' function.
For purposes of computation, the most useful method of finding true
derivatives is through 262Id below.   For a simple
example of a function with a full set of partial derivatives, which is
not everywhere differentiable, consider $\phi:\BbbR^2\to\Bbb R$ defined
by

$$\eqalign{\phi(\xi_1,\xi_2)
&={{\xi_1\xi_2}\over{\xi_1^2+\xi_2^2}}
 \text{ if }\xi_1^2+\xi_2^2\ne 0,\cr
&=0\text{ if }\xi_1=\xi_2=0.\cr}$$

\noindent Then $\phi$ is not even continuous at $\tbf{0}$, although
both partial derivatives $\pd{\phi}{\xi_j}$ are defined everywhere.

\header{262Gc}{\bf (c)} In the definition above, I speak of a derivative
as being a
matrix.   Properly speaking,  the derivative of a function defined on a
subset of $\BbbR^r$ and taking values in $\BbbR^s$ should be thought
of as a bounded linear operator from $\BbbR^r$ to $\BbbR^s$;   the
formulation in terms of matrices is acceptable just because there is a
natural one-to-one correspondence between $s\times r$ real matrices and
linear operators from $\BbbR^r$ to $\BbbR^s$, and all these linear
operators are
bounded.   I use the `matrix' description because it makes certain
calculations more direct;  in particular, the relationship between
$\phi'$ and the partial derivatives of $\phi$ (262Ic), and the notion of
the determinant $\det\phi'(x)$, used throughout \S\S263 and 265.
}%end of comment

\leader{262H}{The norm of a matrix} Some of the calculations below will
rely on the notion of `norm' of a matrix.   The one I
will use\cmmnt{ (in fact, for our purposes here, any norm would do)}
is the `operator norm', defined by saying

\Centerline{$\|T\|=\sup\{\|Tx\|:x\in\BbbR^r,\,\|x\|\le 1\}$}

\noindent for any $s\times r$ matrix $T$.   \cmmnt{For the basic facts
concerning these norms, see 2A4F-2A4G.   The following will also be
useful.}

\header{262Ha}{\bf (a)} If\cmmnt{ all the coefficients of $T$ are
small, so is $\|T\|$;  in fact,
if} $T=\langle\tau_{ij}\rangle_{i\le s,j\le r}$\cmmnt{, and
$\|x\|\le 1$,} then
\cmmnt{$|\xi_j|\le 1$ for each $j$, so

\Centerline{$\|Tx\|
=\bigl(\sum_{i=1}^s(\sum_{j=1}^r\tau_{ij}\xi_j)^2\bigr)^{1/2}
\le \bigl(\sum_{i=1}^s(\sum_{j=1}^r|\tau_{ij}|)^2\bigr)^{1/2}
\le r\sqrt{s}\max_{i\le s,j\le r}|\tau_{ij}|$,}

\noindent and} $\|T\|\le r\sqrt{s}\max_{i\le s,j\le r}|\tau_{ij}|$.
\cmmnt{(This is a singularly crude inequality.   A better one is in
262Yb.   But it tells us, in particular, that $\|T\|$ is always finite.)
}

\header{262Hb}{\bf (b)}\dvro{ $|\tau_{ij}|\le\|T\|$ for all $i$, $j$.}
{ If $\|T\|$ is small, so are all the coefficients of $T$;  in fact,
writing $e_j$ for the $j$th unit vector of $\BbbR^r$, then the $i$th
coordinate of $Te_j$ is $\tau_{ij}$, so
$|\tau_{ij}|\le\|Te_j\|\le\|T\|$.}

\leader{262I}{Lemma} Let $\phi:D\to\BbbR^s$ be a function, where
$D\subseteq\BbbR^r$.   For $i\le s$ let $\phi_i:D\to\Bbb R$ be its $i$th
coordinate, so that $\phi(x)=(\phi_1(x),\ldots,\phi_s(x))$ for $x\in D$.

(a) If $\phi$ is differentiable relative to its domain
at $x\in D$, then $\phi$ is continuous at $x$.

(b) If $x\in D$, then $\phi$ is differentiable relative to its domain at
$x$ iff each $\phi_i$ is differentiable relative to its domain at $x$.

(c) If $\phi$ is differentiable at $x\in D$, then all the partial
derivatives $\pd{\phi_i}{\xi_j}$ of $\phi$ are defined at $x$, and the
derivative of $\phi$ at $x$ is the matrix
$\langle\pd{\phi_i}{\xi_j}(x)\rangle_{i\le s,j\le r}$.

(d) If all the partial derivatives $\pd{\phi_i}{\xi_j}$, for $i\le s$
and $j\le r$, are defined in a neighbourhood of $x\in D$ and are continuous at $x$, then $\phi$ is differentiable at $x$.

\proof{{\bf (a)} Let $T$ be a derivative of $\phi$ at $x$.   Applying
the definition 262Fb with
$\epsilon=1$, we see that there is a $\delta>0$ such that
\Centerline{$\|\phi(y)-\phi(x)-T(y-x)\|\le\|y-x\|$}

\noindent whenever $y\in D$ and $\|y-x\|\le\delta$.   Now

\Centerline{$\|\phi(y)-\phi(x)\|\le\|T(y-x)\|+\|y-x\|
\le(1+\|T\|)\|y-x\|$}

\noindent whenever $y\in D$ and $\|y-x\|\le\delta$, so $\phi$ is
continuous at $x$.

\medskip

{\bf (b)(i)} If $\phi$ is differentiable relative to its domain at
$x\in D$, let $T$ be a
derivative of $\phi$ at $x$.   For $i\le s$ let $T_i$ be
the $1\times r$ matrix consisting of the $i$th row of $T$.    Let
$\epsilon>0$.  Then we have a $\delta>0$ such that

$$\eqalign{|\phi_i(y)-\phi_i(x)-T_i(y-x)|
&\le\|\phi(y)-\phi(x)-T(y-x)\|\cr
&\le\epsilon\|y-x\|\cr}$$

\noindent whenever $y\in D$ and $\|y-x\|\le\delta$, so that $T_i$ is a
derivative of $\phi_i$ at $x$.

\medskip

\quad{\bf (ii)} If each $\phi_i$ is differentiable relative to its
domain at
$x$, with corresponding derivatives $T_i$, let $T$ be the $s\times r$
matrix with rows $T_1,\ldots,T_s$.   Given $\epsilon>0$, there is for
each $i\le s$ a $\delta_i>0$ such that

\Centerline{$|\phi_i(y)-\phi_i(x)-T_iy|\le\epsilon\|y-x\|$ whenever
$y\in D$, $\|y-x\|\le\delta_i$;}

\noindent set $\delta=\min_{i\le s}\delta_i>0$;  then if $y\in D$ and
$\|y-x\|\le\delta$, we shall have

\Centerline{$\|\phi(y)-\phi(x)-T(y-x)\|^2
=\sum_{i=1}^s|\phi_i(y)-\phi_i(x)-T_i(y-x)|^2
\le s\epsilon^2\|y-x\|^2$,}

\noindent so that

\Centerline{$\|\phi(y)-\phi(x)-T(y-x)\|
\le\epsilon\sqrt{s}\|y-x\|$.}

\noindent As $\epsilon$ is arbitrary, $T$ is a derivative of $\phi$ at
$x$.

\medskip

{\bf (c)} Set $T=\phi'(x)$.   We have

\Centerline{$\lim_{y\to x}\Bover{\|\phi(y)-\phi(x)-T(y-x)\|}{\|y-x\|}
=0$;}

\noindent fix $j\le r$, and consider $y=x+\eta e_j$, where
$e_j=(0,\ldots,0,1,0,\ldots,0)$ is the $j$th unit vector in $\BbbR^r$.
Then we must have

\Centerline{$\lim_{\eta\to 0}\Bover{\|\phi(x+\eta e_j)-\phi(x)-\eta
T(e_j)\|}{|\eta|}
=0$.}

\noindent Looking at the $i$th coordinate of
$\phi(x+\eta e_j)-\phi(x)-\eta T(e_j)$, we have


\Centerline{$|\phi_i(x+\eta e_j)-\phi_i(x)-\tau_{ij}\eta|
\le\|\phi(x+\eta e_j)-\phi(x)-\eta T(e_j)\|$,}

\noindent where $\tau_{ij}$ is the
$(i,j)$th coefficient of $T$;  so that

\Centerline{$\lim_{\eta\to 0}
\Bover{|\phi_i(x+\eta e_j)-\phi_i(x)-\tau_{ij}\eta|}{|\eta|}
=0$.}

\noindent  But this just says that the partial derivative
$\pd{\phi_i}{\xi_j}(x)$ exists and is equal to $\tau_{ij}$, as claimed.

\medskip

{\bf (d)} Now suppose that the partial derivatives $\pd{\phi_i}{\xi_j}$
are defined near $x$ and continuous at $x$.   Let $\epsilon>0$.   Let
$\delta>0$ be such that

\Centerline{$|\pd{\phi_i}{\xi_j}(y)-\tau_{ij}|\le\epsilon$}

\noindent whenever $\|y-x\|\le\delta$, writing
$\tau_{ij}=\pd{\phi_i}{\xi_j}(x)$.   Now suppose that
$\|y-x\|\le\delta$.   Set

\Centerline{$y=(\eta_1,\ldots,\eta_r)$,
\quad$x=(\xi_1,\ldots,\xi_r)$,}

\Centerline{$y_j=(\eta_1,\ldots,\eta_j,\xi_{j+1},\ldots,\xi_r)$ for
$0\le
j\le r$,}

\noindent so that $y_0=x$, $y_r=y$ and the line segment between
$y_{j-1}$ and $y_{j}$ lies wholly within $\delta$ of $x$ whenever
$1\le j\le r$, since if $z=(\zeta_1,\ldots,\zeta_r)$
lies on this segment then $\zeta_i$ lies
between $\xi_i$ and $\eta_i$ for every $i$.   By the ordinary mean value
theorem for differentiable real functions, applied to the function

\Centerline{$t\mapsto
\phi_i(\eta_1,\ldots,\eta_{j-1},t,\xi_{j+1},\ldots,\xi_r)$,}

\noindent there is for each $i\le s$, $j\le r$ a point $z_{ij}$ on the
line segment between $y_{j-1}$ and $y_j$ such that

\Centerline{$\phi_i(y_j)-\phi_i(y_{j-1})
=(\eta_j-\xi_j)\pd{\phi_i}{\xi_j}(z_{ij})$.}

\noindent But

\Centerline{$|\pd{\phi_i}{\xi_j}(z_{ij})-\tau_{ij}|\le\epsilon$,}
\noindent so

\Centerline{$|\phi_i(y_j)-\phi_i(y_{j-1})
-\tau_{ij}(\eta_j-\xi_j)|\le\epsilon|\eta_j-\xi_j|\le\epsilon\|y-x\|$.}

\noindent Summing over $j$,

\Centerline{$|\phi_i(y)-\phi_i(x)-\sum_{j=1}^r\tau_{ij}(\eta_j-\xi_j)|
\le r\epsilon\|y-x\|$}

\noindent for each $i$.   Summing the squares and taking the square
root,

\Centerline{$\|\phi(y)-\phi(x)-T(y-x)\|\le \epsilon r\sqrt{s}\|y-x\|$,}

\noindent where $T=\langle\tau_{ij}\rangle_{i\le s,j\le r}$.   And this
is true
whenever $\|y-x\|\le\delta$.   As $\epsilon$ is arbitrary, $\phi'(x)=T$
is defined.
}%end of proof of 262I

\cmmnt{
\leader{262J}{Remark} I am not sure if I ought to apologize for the
notation $\pd{}{\xi_j}$.   In such formulae as
$(\eta_j-\xi_j)\pd{\phi_i}{\xi_j}(z_{ij})$ above, the two appearances of
$\xi_j$ clash most violently.   But I do not think that any person of
good will is likely to be misled, provided that the labels $\xi_j$ (or
whatever symbols are used to represent the variables involved) are
adequately described when the domain of $\phi$ is first introduced (and
always remembering that in partial differentiation, we are not only
moving one variable -- a $\xi_j$ in the present context -- but holding
fixed some further list of variables, not listed in the notation).   I
believe that the traditional notation $\pd{}{\xi_j}$ has survived for
solid reasons, and I should like to offer a welcome to those who are
more comfortable with it than with any of the many alternatives which
have been proposed, but have never taken root.
}%end of comment

\cmmnt{
\leader{262K}{The Cantor function revisited} It is salutary to
re-examine the examples of 134H-134I in the light of the present
considerations.   Let $f:[0,1]\to[0,1]$ be the Cantor function (134H)
and set $g(x)=\bover12(x+f(x))$ for $x\in [0,1]$.   Then
$g:[0,1]\to[0,1]$ is a homeomorphism (134I);  set
$\phi=g^{-1}:[0,1]\to[0,1]$.   We see that if $0\le x\le y\le 1$ then
$g(y)-g(x)\ge\bover12(y-x)$;  equivalently, $\phi(y)-\phi(x)\le 2(y-x)$
whenever $0\le x\le y\le 1$, so that $\phi$ is a Lipschitz function,
therefore absolutely continuous (262Bc).   If
$D=\{x:\phi'(x)$ is defined$\}$, then
$[0,1]\setminus D$ is negligible (225Cb), so
$[0,1]\setminus\phi[D]=\phi[\,[0,1]\setminus D]$ is negligible (262Da).
I noted in 134I that there is a measurable function $h:[0,1]\to\Bbb R$
such that the composition $h\phi$ is not measurable;  now
$h(\phi\restr D)=(h\phi)\restr D$ cannot be measurable, even though
$\phi\restr D$ is differentiable.
}%end of comment

\leader{262L}{}\cmmnt{ It will be convenient to be able to call on the
following straightforward result.

\medskip

\noindent}{\bf Lemma} Suppose that $D\subseteq\BbbR^r$ and $x\in\BbbR^r$
are such that
$\lim_{\delta\downarrow 0}
 \Bover{\mu^*(D\cap B(x,\delta))}{\mu B(x,\delta)}=1$.   Then
$\lim_{z\to\tbf{0}}\Bover{\rho(x+z,D)}{\|z\|}=0$, where
$\rho(x+z,D)=\inf_{y\in D}\|x+z-y\|$.

\proof{ Let $\epsilon>0$.  Let $\delta_0>0$ be such that

\Centerline{$\mu^*(D\cap B(x,\delta))
>(1-(\Bover{\epsilon}{1+\epsilon})^r)\mu B(x,\delta)$}

\noindent whenever $0<\delta\le\delta_0$.   Take any $z$ such that
$0<\|z\|\le\delta_0/(1+\epsilon)$.   \Quer\ Suppose, if possible, that
$\rho(x+z,D)>\epsilon\|z\|$.   Then $B(x+z,\epsilon\|z\|)\subseteq
B(x,(1+\epsilon)\|z\|)\setminus D$, so

$$\eqalign{\mu^*(D\cap B(x,(1+\epsilon)\|z\|))
&\le\mu B(x,(1+\epsilon)\|z\|)-\mu B(x+z,\epsilon\|z\|)\cr
&=(1-(\Bover{\epsilon}{1+\epsilon})^r)\mu B(x,(1+\epsilon)\|z\|),\cr}$$

\noindent which is impossible, as $(1+\epsilon)\|z\|\le\delta_0$.\
\BanG\   Thus $\rho(x+z,D)\le\epsilon\|z\|$.   As $\epsilon$ is
arbitrary, this proves the result.
}%end of proof of 262L

\cmmnt{\medskip

\noindent{\bf Remark} There is a word for this;  see 261Yg.}

\leader{262M}{}\cmmnt{ I come now to the first result connecting
Lipschitz functions with differentiable functions.   I approach it
through a substantial lemma which will be the foundation of \S263.

\medskip

\noindent}{\bf Lemma} Let $\phi$ be a
function from a subset $D$ of $\BbbR^r$ to $\BbbR^s$ which is
differentiable at
each point of its domain.   For each $x\in D$ let $T(x)$ be a derivative
of $\phi$.   Let $M_{sr}$ be the set of $s\times r$ matrices and
$\zeta:A\to\ooint{0,\infty}$ a strictly positive function, where
$A\subseteq M_{sr}$ is a non-empty set containing $T(x)$ for every
$x\in D$.   Then we
can find sequences $\sequencen{D_n}$, $\sequencen{T_n}$ such that

(i) $\sequencen{D_n}$ is a partition of $D$ into sets which are
relatively measurable in $D$\cmmnt{, that is, are intersections of $D$ with measurable subsets of $\BbbR^r$};

(ii) $T_n\in A$ for every $n$;

(iii) $\|\phi(x)-\phi(y)-T_n(x-y)\|\le\zeta(T_n)\|x-y\|$ for every
$n\in\Bbb N$ and $x$, $y\in D_n$;

(iv) $\|T(x)-T_n\|\le\zeta(T_n)$ for every $x\in D_n$.

\proof{{\bf (a)} The first step is to note that there is a sequence
$\sequencen{S_n}$ in $A$ such that

\Centerline{$A
\subseteq\bigcup_{n\in\Bbb N}\{T:T\in M_{sr},\,\|T-S_n\|<\zeta(S_n)\}$.}

\noindent\Prf\ (Of course this is a standard result about
separable metric
spaces.) Write $Q$ for the set of matrices in $M_{sr}$ with rational
coefficients;  then there is a natural bijection between $Q$ and
$\Bbb Q^{sr}$, so $Q$ and $Q\times\Bbb N$ are countable.   Enumerate
$Q\times\Bbb N$ as $\sequencen{(R_n,k_n)}$.   For each $n\in\Bbb N$,
choose $S_n\in A$ by the rule

\quad --- if there is an $S\in A$ such that $\{T:\|T-R_n\|\le
2^{-k_n}\}\subseteq\{T:\|T-S\|<\zeta(S)\}$, take such an $S$ for $S_n$;

\quad --- otherwise, take $S_n$ to be any member of $A$.

\noindent I claim that this works.   For let $S\in A$.   Then
$\zeta(S)>0$;  take $k\in\Bbb N$ such that $2^{-k}<\zeta(S)$.   Take
$R^*\in Q$ such that $\|R^*-S\|<\min(\zeta(S)-2^{-k},2^{-k})$;  this
is possible because $\|R-S\|$ will be small whenever all the
coefficients
of $R$ are close enough to the corresponding coefficients of $S$
(262Ha), and we can find rational numbers to achieve this.   Let
$n\in\Bbb N$ be such that $R^*=R_n$ and $k=k_n$.   Then

\Centerline{$\{T:\|T-R_n\|\le 2^{-k_n}\}
\subseteq\{T:\|T-S\|<\zeta(S)\}$}

\noindent (because $\|T-S\|\le\|T-R_n\|+\|R_n-S\|$), so we must have
chosen $S_n$ by the first part of the rule above, and

\Centerline{$S\in\{T:\|T-R_n\|\le 2^{-k_n}\}
\subseteq\{T:\|T-S_n\|<\zeta(S_n)\}$.}

\noindent As $S$ is arbitrary, this proves the result.\ \Qed

\medskip

{\bf (b)} Enumerate $\Bbb Q^r\times\Bbb Q^r\times\Bbb N$ as
$\sequencen{(q_n,q'_n,m_n)}$.   For each $n\in\Bbb N$, set

$$\eqalign{H_n&=\{x:x\in[q_n,q'_n]\cap D,\,
\|\phi(y)-\phi(x)-S_{m_n}(y-x)\|\le\zeta(S_{m_n})\|y-x\|\cr
&\mskip370mu
    \text{for every }y\in[q_n,q'_n]\cap D\}\cr
&=[q_n,q'_n]\cap D\cap\bigcap_{y\in[q_n,q'_n]\cap D}
       \{x:x\in D,\cr
&\mskip270mu
    \|\phi(y)-\phi(x)-S_{m_n}(y-x)\|\le\zeta(S_{m_n})\|y-x\|\}.\cr}$$

\noindent Because $\phi$ is continuous, $H_n=D\cap\overline{H}_n$,
writing $\overline{H}_n$ for the closure of $H_n$, so $H_n$ is
relatively measurable in $D$.
Note that if $x$, $y\in H_n$, then $y\in D\cap[q_n,q'_n]$, so that

\Centerline{$\|\phi(y)-\phi(x)-S_{m_n}(y-x)\|\le\zeta(S_{m_n})\|y-x\|$.}


Set

\Centerline{$H_n'=\{x:x\in H_n,\,\|T(x)-S_{m_n}\|\le\zeta(S_{m_n})\}$.}

\medskip

{\bf (c)} $D=\bigcup_{n\in\Bbb N}H'_n$.   \Prf\ Let $x\in D$.   Then
$T(x)\in A$, so there is a $k\in\Bbb N$ such that
$\|T(x)-S_k\|<\zeta(S_k)$.   Let $\delta>0$ be such that

\Centerline{$\|\phi(y)-\phi(x)-T(x)(x-y)\|
\le(\zeta(S_k)-\|T(x)-S_k\|)\|x-y\|$}

\noindent whenever $y\in D$ and $\|y-x\|\le\delta$.   Then

$$\eqalign{\|\phi(y)-\phi(x)-S_k(x-y)\|
&\le(\zeta(S_k)-\|T(x)-S_k\|)\|x-y\|+\|T(x)-S_k\|\|x-y\|\cr
&\le\zeta(S_k)\|x-y\|\cr}$$

\noindent whenever $y\in D\cap B(x,\delta)$.   Let $q$, $q'\in\Bbb Q^r$
be such that $x\in[q,q']\subseteq B(x,\delta)$.   Let $n$ be such that
$q=q_n$, $q'=q'_n$ and $k=m_n$.   Then $x\in H'_n$.\ \Qed

\medskip

{\bf (d)} Write

\Centerline{$C_n=\{x:x\in H_n,\,\lim_{\delta\downarrow 0}
\Bover{\mu^*(H_n\cap B(x,\delta))}{\mu B(x,\delta)}=1\}$.}

\noindent Then $C_n\subseteq H'_n$.

\medskip

\Prf\ {\bf (i)} Take $x\in C_n$, and set $\tilde T=T(x)-S_{m_n}$.   I
have
to show that $\|\tilde T\|\le\zeta(S_{m_n})$.   Take $\epsilon>0$.   Let
$\delta_0>0$ be such that

\Centerline{$\|\phi(y)-\phi(x)-T(x)(y-x)\|\le\epsilon\|y-x\|$}

\noindent whenever $y\in D$ and $\|y-x\|\le\delta_0$.   Since

\Centerline{$\|\phi(y)-\phi(x)-S_{m_n}(y-x)\|\le\zeta(S_{m_n})\|y-x\|$}

\noindent whenever $y\in H_n$, we have

\Centerline{$\|\tilde T(y-x)\|\le(\epsilon+\zeta(S_{m_n}))\|y-x\|$}

\noindent whenever $y\in H_n$ and $\|y-x\|\le\delta_0$.

\medskip

\quad{\bf (ii)} By 262L, there is a $\delta_1>0$ such that
$(1+2\epsilon)\delta_1\le\delta_0$ and $\rho(x+z,H_n)\le\epsilon\|z\|$
whenever $0<\|z\|\le\delta_1$.   So if $\|z\|\le\delta_1$ there is a
$y\in
H_n$ such that $\|x+z-y\|\le 2\epsilon\|z\|$.   (If $z=0$ we can take
$y=x$.)   Now $\|x-y\|\le (1+2\epsilon)\|z\|\le\delta_0$, so

$$\eqalign{\|\tilde Tz\|
&\le\|\tilde T(y-x)\|+\|\tilde T(x+z-y)\|\cr
&\le(\epsilon+\zeta(S_{m_n}))\|y-x\|+\|\tilde T\|\|x+z-y\|\cr
&\le(\epsilon+\zeta(S_{m_n}))\|z\|
    +(\epsilon+\zeta(S_{m_n})+\|\tilde T\|)\|x+z-y\|\cr
&\le(\epsilon+\zeta(S_{m_n})+2\epsilon^2+2\epsilon\zeta(S_{m_n})
    +2\epsilon\|\tilde T\|)\|z\|.\cr}$$

\noindent And this is true whenever $0<\|z\|\le\delta_1$.
But multiplying this inequality by
suitable positive scalars we see that

\Centerline{$\|\tilde Tz\|
\le\bigl(\epsilon+\zeta(S_{m_n})+2\epsilon^2+2\epsilon\zeta(S_{m_n})
+2\epsilon\|\tilde T\|\bigr)\|z\|$}

\noindent for all $z\in\BbbR^r$, and

\Centerline{$\|\tilde T\|
\le\epsilon+\zeta(S_{m_n})+2\epsilon^2+2\epsilon\zeta(S_{m_n})
  +2\epsilon\|\tilde T\|$.}

\noindent As $\epsilon$ is
arbitrary, $\|\tilde T\|\le\zeta(S_{m_n})$, as claimed.\ \Qed

\medskip

{\bf (e)} By 261Da, $H_n\setminus C_n$ is negligible for every $n$, so
$H_n\setminus H'_n$ is negligible, and

\Centerline{$H'_n=D\cap(\overline{H}_n\setminus(H_n\setminus H'_n))$}

\noindent is relatively measurable in $D$.   Set

\Centerline{$D_n=H'_n\setminus\bigcup_{k<n}H'_k$,
\quad$T_n=S_{m_n}$}

\noindent for each $n$;  these serve.
}%end of proof of 262M


\leader{262N}{Corollary} Let $\phi$ be a function from a subset $D$ of
$\Bbb R^r$ to $\BbbR^s$, and suppose that $\phi$ is differentiable relative to
its domain at each point of $D$.   Then $D$ can be expressed as the
union of
a disjoint sequence $\sequencen{D_n}$ of relatively measurable subsets
of $D$ such that $\phi\restr D_n$ is Lipschitz for each $n\in\Bbb N$.

\proof{ In 262M, take $\zeta(T)=1$ for every $T\in A=M_{sr}$.   If $x$,
$y\in D_n$ then

$$\eqalign{\|\phi(x)-\phi(y)\|
&\le\|\phi(x)-\phi(y)-T_n(x-y)\|+\|T_n(x-y)\|\cr
&\le\|x-y\|+\|T_n\|\|x-y\|,\cr}$$

\noindent so $\phi\restr D_n$ is $(1+\|T_n\|)$-Lipschitz.
}%end of proof of 262N

\leader{262O}{Corollary} Suppose that $\phi$ is an injective function
from a measurable subset $D$ of $\BbbR^r$ to $\BbbR^r$, and that
$\phi$ is differentiable relative to its domain at every point of $D$.

(a) If $A\subseteq D$ is negligible, $\phi[A]$ is negligible.

(b) If $E\subseteq D$ is measurable, then $\phi[E]$ is
measurable.

(c) If $D$ is measurable and $f$ is a measurable function defined on a
subset of $\BbbR^r$, then $f\phi^{-1}$ is measurable.

(d) If $H\subseteq\BbbR^r$ and $\phi^{-1}$ is defined almost everywhere
in $H$, and if $f$ is a function defined almost everywhere in
$\Bbb R^r$, then $f\phi^{-1}$ is defined almost everywhere in $H$.

\proof{ Take a sequence $\sequencen{D_n}$ as in 262N,
and apply 262E to $\phi\restr D_n$ for each $n$.
}%end of proof of 262O

\leader{262P}{Corollary} Let $\phi$ be a function from a a subset $D$ of
$\BbbR^r$ to $\BbbR^s$, and suppose that $\phi$ is differentiable
relative to its domain, with a derivative $T(x)$, at each point $x\in
D$.   Then the function $x\mapsto T(x)$ is measurable in the sense that
$\tau_{ij}:D\to\Bbb R$ is measurable for all $i\le s$ and $j\le r$,
where $\tau_{ij}(x)$ is the
$(i,j)$th coefficient of the matrix $T(x)$ for all $i$, $j$ and $x$.

\proof{ For each $k\in\Bbb N$, apply 262M with $\zeta(T)=2^{-k}$ for
each $T\in A=M_{sr}$, obtaining sequences $\sequencen{D_{kn}}$ of
relatively measurable subsets of $D$ and $\sequencen{T_{kn}}$ in
$M_{sr}$.   Let
$\tau^{(kn)}_{ij}$ be the $(i,j)$th coefficient of $T_{kn}$.   Then we
have functions $f_{ijk}:D\to\Bbb R$ defined by setting

\Centerline{$f_{ijk}(x)=\tau^{(kn)}_{ij}$ if $x\in D_{kn}$.}

\noindent Because the $D_{kn}$ are relatively measurable, the $f_{ijk}$
are
measurable functions.   For $x\in D_{kn}$,
\Centerline{$|\tau_{ij}(x)-f_{ijk}(x)|\le\|T(x)-T_n\|\le 2^{-k}$,}

\noindent so $|\tau_{ij}(x)-f_{ijk}(x)|\le 2^{-k}$ for every $x\in D$,
and

\Centerline{$\tau_{ij}=\lim_{k\to\infty}f_{ijk}$}

\noindent is measurable, as claimed.
}%end of proof of 262P

\leader{*262Q}{}\cmmnt{ This concludes the part of the section
which is essential for the rest of the chapter.   However the main
results of \S263
will I think be better understood if you are aware of the fact that any
Lipschitz function is differentiable (relative to its domain) almost
everywhere in its domain.   I devote the next couple of pages to a proof
of this fact, which apart from its intrinsic interest is a useful
exercise.

\medskip

\noindent}{\bf Rademacher's theorem} Let $\phi$ be a Lipschitz function
from a subset
of $\BbbR^r$ to $\BbbR^s$.   Then $\phi$ is
differentiable relative to its domain almost everywhere in its domain.

\proof{{\bf (a)} By 262Ba and 262Ib, it will be enough to
deal with the
case $s=1$.   By 262Bb, there is a Lipschitz function
$\tilde\phi:\BbbR^r\to\Bbb R$ extending $\phi$;  now $\phi$ is
differentiable with respect to its domain at any point of $\dom\phi$ at
which $\tilde \phi$ is differentiable, so it will be enough if I can
show that $\tilde\phi$ is differentiable almost everywhere.   To make
the notation more agreeable to the eye, I will suppose that $\phi$
itself was defined everywhere in $\BbbR^r$.   Let $\gamma$ be a
Lipschitz constant for $\phi$.

The proof proceeds by induction on $r$.   If $r=1$, we have a Lipschitz
function $\phi:\Bbb R\to\Bbb R$;  now $\phi$ is absolutely
continuous in any bounded interval (262Bc), therefore differentiable
almost everywhere (225Cb).   Thus the
induction starts.   The rest of the proof is devoted to the inductive
step to $r>1$.

\medskip

{\bf (b)} The first step is to show that all the partial derivatives
$\pd{\phi}{\xi_j}$ are defined almost everywhere and are Borel
measurable.
\Prf\ Take $j\le r$.   For $q\in\Bbb Q\setminus\{0\}$ set

\Centerline{$\Delta_q(x)=\Bover{1}{q}(\phi(x+qe_j)-\phi(x))$,}

\noindent writing $e_j$ for the $j$th unit vector of $\BbbR^r$.
Because $\phi$ is continuous, so is $\Delta_q$, so that $\Delta_q$ is a
Borel
measurable function for each $q$.   Next, for any $x\in\Bbb
R^r$,

\Centerline{$D^+(x)=\limsup_{\delta\to 0}\Bover1{\delta}
(\phi(x+\delta e_j)-\phi(x))
=\lim_{n\to\infty}\sup_{q\in\Bbb Q,0<|q|\le 2^{-n}}\Delta_q(x)$,}

\noindent so that the set on which $D^+(x)$ is defined in $\Bbb R$ is
Borel and
$D^+$ is a Borel measurable function.
Similarly,

\Centerline{$D^-(x)=\liminf_{\delta\to 0}\Bover1{\delta}
(\phi(x+\delta e_j)-\phi(x))$}

\noindent is a Borel measurable function with Borel domain.   So

\Centerline{$E=\{x:\pd{\phi}{\xi_j}(x)\text{ exists in }\Bbb R\}
=\{x:D^+(x)=D^-(x)\in\Bbb R\}$}

\noindent is a Borel set, and $\pd{\phi}{\xi_j}$ is a Borel measurable
function.

On the other hand, if we identify $\BbbR^r$ with
$\BbbR^J\times\Bbb R$, taking $J$ to be
$\{1,\ldots,j-1,j+1,\ldots,r\}$, then we can think of
Lebesgue measure $\mu$
on $\BbbR^r$ as being the product of Lebesgue measure $\mu_J$ on
$\BbbR^J$ with Lebesgue measure $\mu_1$ on $\Bbb R$ (251N).   Now for
every $y\in\BbbR^J$ we have a function $\phi_y:\Bbb R\to\Bbb R$ defined
by writing

\Centerline{$\phi_y(\sigma)=\phi(y,\sigma)$,}

\noindent and $E$ becomes

\Centerline{$\{(y,\sigma):\phi_y'(\sigma)$ is defined$\}$,}

\noindent so that all the sections

\Centerline{$\{\sigma:(y,\sigma)\in E\}$}

\noindent are conegligible subsets of $\Bbb R$, because every
$\phi_y$ is Lipschitz, therefore differentiable almost everywhere, as
remarked in part (a) of the proof.   Since we know that $E$ is
measurable, it must be conegligible, by Fubini's theorem (apply 252D or
252F to the complement of $E$).
Thus $\pd{\phi}{\xi_j}$ is defined almost everywhere, as claimed.\ \Qed

Write

\Centerline{$H=\{x:x\in\BbbR^r,\,\pd{\phi}{\xi_j}(x)
\text{ exists for every }j\le r\}$,}

\noindent so that $H$ is a conegligible Borel set in $\BbbR^r$.

\medskip

{\bf (c)} For the rest of this proof, I fix on the natural
identification of $\BbbR^r$ with $\BbbR^{r-1}\times\Bbb R$,
identifying $(\xi_1,\ldots,\xi_r)$ with
$((\xi_1,\ldots,\xi_{r-1}),\xi_r)$.
For $x\in H$, let $T(x)$ be the $1\times r$ matrix
$(\pd{\phi}{\xi_1}(x),\ldots,\pd{\phi}{\xi_r}(x))$.

\medskip

{\bf (d)} Set

\Centerline{$H_1=\{x:x\in H,\,\lim_{u\to\tbf{0}\text{ in }\BbbR^{r-1}}
\Bover{|\phi(x+(u,0))-\phi(x)-T(x)(u,0)|}{\|u\|}=0\}$.}

\noindent I claim that $H_1$ is conegligible in $\BbbR^r$.
\Prf\ This is really the same idea as in (b).   For $x\in H$, $x\in H_1$
iff

\inset{for every $\epsilon>0$ there is a $\delta>0$ such that

\Centerline{$|\phi(x+(u,0))-\phi(x)-T(x)(u,0)|\le\epsilon\|u\|$}

\noindent whenever $\|u\|\le\delta$,}

\noindent that is, iff

\inset{for every $m\in\Bbb N$ there is an $n\in\Bbb N$ such that

\Centerline{$|\phi(x+(u,0))-\phi(x)-T(x)(u,0)|\le 2^{-m}\|u\|$}

\noindent whenever $u\in\Bbb Q^{r-1}$ and $\|u\|\le 2^{-n}$.}

\noindent But for any particular $m\in\Bbb N$ and $u\in\Bbb Q^{r-1}$ the
set

\Centerline{$\{x:|\phi(x+(u,0))-\phi(x)-T(x)(u,0)|\le 2^{-m}\|u\|\}$}

\noindent is measurable, indeed Borel, because all the functions
$x\mapsto\phi(x+(u,0))$, $x\mapsto\phi(x)$, $x\mapsto T(x)(u,0)$ are
Borel measurable.   So $H_1$ is of the form

\Centerline{$\bigcap_{m\in\Bbb N}\bigcup_{n\in\Bbb N}
\bigcap_{u\in\Bbb Q^{r-1},\|u\|\le 2^{-n}}E_{mnu}$}

\noindent where every $E_{mnu}$ is a measurable set, and $H_1$ is
therefore measurable.

Now however observe that for any $\sigma\in\Bbb R$, the function

\Centerline{$v\mapsto\phi_{\sigma}(v)=\phi(v,\sigma):
\BbbR^{r-1}\to\Bbb R$}

\noindent is Lipschitz, therefore (by the inductive hypothesis)
differentiable almost everywhere in $\BbbR^{r-1}$;  and that
$(v,\sigma)\in H_1$ iff $(v,\sigma)\in H$ and $\phi_{\sigma}'(v)$ is
defined.   Consequently
$\{v:(v,\sigma)\in H_1\}$ is conegligible whenever
$\{v:(v,\sigma)\in H\}$
is, that is, for almost every $\sigma\in \Bbb R$;  so that $H_1$, being
measurable, must be conegligible.\ \Qed

\medskip

{\bf (e)} Now, for $q$, $q'\in\Bbb Q$ and $n\in\Bbb N$, set

\Centerline{$F(q,q',n)
=\{x:x\in\BbbR^r$,
  $q\le\Bover{\phi(x+(\tbf{0},\eta))-\phi(x)}{\eta}\le q'$
  whenever $0<|\eta|\le 2^{-n}\}$,}

\Centerline{$F_*(q,q',n)
=\{x:x\in F(q,q',n),\,\lim_{\delta\downarrow 0}
 \Bover{\mu^*(F(q,q',n)\cap B(x,\delta))}{\mu B(x,\delta)}=1\}$.}

\noindent By 261Da, $F(q,q',n)\setminus F_*(q,q',n)$ is negligible for
all $q$, $q'$, $n$, so that

\Centerline{$H_2
=H_1\setminus\bigcup_{q,q'\in\Bbb Q,n\in\Bbb N}
  (F(q,q',n)\setminus F_*(q,q',n))$}

\noindent is conegligible.

\medskip

{\bf (f)} I claim that $\phi$ is differentiable at every point of $H_2$.
\Prf\ Take $x=(u,\sigma)\in H_2$.   Then $\alpha=\Pd{\phi}{\xi_r}(x)$
and $T=T(x)$ are defined.   Let $\gamma$ be a Lipschitz constant for
$\phi$.

Take $\epsilon>0$;  take $q$, $q'\in\Bbb Q$ such that
$\alpha-\epsilon\le q<\alpha<q'\le\alpha+\epsilon$.   There must be an
$n\in\Bbb N$ such that $x\in F(q,q',n)$;  consequently
$x\in F_*(q,q',n)$,
by the definition of $H_2$.   By 262L, there is a $\delta_0>0$ such that
$\rho(x+z,F(q,q',n))\le\epsilon\|z\|$ whenever $\|z\|\le\delta_0$.
Next, there is a $\delta_1>0$ such that
$|\phi(x+(v,0))-\phi(x)-T(v,0)|\le\epsilon\|v\|$ whenever
$v\in\BbbR^{r-1}$ and $\|v\|\le\delta_1$.   Set

\Centerline{$\delta=\min(\delta_0,\delta_1,2^{-n})/(1+2\epsilon)>0$.}

Suppose that $z=(v,\tau)\in\BbbR^r$ and that $\|z\|\le\delta$.   Because
$\|z\|\le\delta_0$ there is an $x'=(u',\sigma')\in F(q,q',n)$ such that
$\|x+z-x'\|\le2\epsilon\|z\|$;  set $x^*=(u',\sigma)$.   Now

\Centerline{$\max(\|u-u'\|,|\sigma-\sigma'|)\le
\|x-x'\|\le(1+2\epsilon)\|z\|\le\min(\delta_1,2^{-n})$}

\noindent and $x^*=x+(u'-u,0)$, so

\Centerline{$|\phi(x^*)-\phi(x)-T(x^*-x)|
\le\epsilon\|u'-u\|\le\epsilon(1+2\epsilon)\|z\|$.}

\noindent But also

\Centerline{$|\phi(x')-\phi(x^*)-T(x'-x^*)|
=|\phi(x')-\phi(x^*)-\alpha(\sigma'-\sigma)|
\le\epsilon|\sigma'-\sigma|\le\epsilon(1+ 2\epsilon)\|z\|$,}

\noindent because $x'\in F(q,q',n)$ and $|\sigma-\sigma'|\le 2^{-n}$,
so that (if $x'\ne x^*$)

\Centerline{$\alpha-\epsilon\le q
\le\Bover{\phi(x^*)-\phi(x')}{\sigma-\sigma'}\le q'\le\alpha+\epsilon$}

\noindent and

\Centerline{$\bigl|\Bover{\phi(x')-\phi(x^*)}
{\sigma'-\sigma}-\alpha\bigr|\le\epsilon$.}
\noindent Finally,

\Centerline{$|\phi(x+z)-\phi(x')|
\le\gamma\|x+z-x'\|\le2\gamma\epsilon\|z\|$,}

\Centerline{$|Tz-T(x'-x)|\le\|T\|\|x+z-x'\|\le2\epsilon\|T\|\|z\|$.}

Putting all these together,

$$\eqalign{|\phi(x+z)-\phi x-Tz|
&\le|\phi(x+z)-\phi(x')|
+|T(x'-x)-Tz|\cr
&\qquad
+|\phi(x')-\phi(x^*)-T(x'-x^*)|
+|\phi(x^*)-\phi(x)-T(x^*-x)|\cr
&\le2\gamma\epsilon\|z\|
+2\epsilon\|T\|\|z\|
+\epsilon(1+2\epsilon)\|z\|
+\epsilon(1+2\epsilon)\|z\|\cr
&=\epsilon(2\gamma+2\|T\|+2+4\epsilon)\|z\|.\cr}$$

\noindent And this is true whenever $\|z\|\le\delta$.   As $\epsilon$ is
arbitrary, $\phi$ is differentiable at $x$.\ \Qed

Thus $\{x:\phi$ is differentiable at $x\}$ includes $H_2$ and is
conegligible;  and the induction continues.
}%end of proof of 262Q

\exercises{
\leader{262X}{Basic exercises (a)} Let $\phi$ and $\psi$ be Lipschitz
functions from subsets of $\BbbR^r$ to $\BbbR^s$.   Show that
$\phi+\psi$ is a Lipschitz function from $\dom\phi\cap\dom\psi$ to
$\Bbb R^s$.
%262A

\header{262Xb}{\bf (b)} Let $\phi$ be a Lipschitz function from a subset
of $\BbbR^r$ to $\BbbR^s$, and $c\in\Bbb R$.   Show that $c\phi$ is a
Lipschitz function.
%262A

\header{262Xc}{\bf (c)} Suppose $\phi:D\to\BbbR^s$ and
$\psi:E\to\BbbR^q$ are Lipschitz functions,
where $D\subseteq\BbbR^r$ and $E\subseteq\BbbR^s$.   Show that the
composition $\psi\phi:D\cap\phi^{-1}[E]\to\BbbR^q$ is Lipschitz.
%262A

\header{262Xd}{\bf (d)} Suppose $\phi$, $\psi$ are functions from
subsets of $\BbbR^r$ to $\BbbR^s$, and suppose that
$x\in\dom\phi\cap\dom\psi$ is such that each function is differentiable
relative to its domain at $x$, with derivatives $S$, $T$ there.   Show
that $\phi+\psi$ is differentiable relative to its domain at $x$, and
that $S+T$ is a derivative of $\phi+\psi$ at $x$.
%262F

\header{262Xe}{\bf (e)} Suppose that $\phi$ is a function from a subset
of $\BbbR^r$ to $\BbbR^s$, and is differentiable relative to its
domain at $x\in\dom\phi$.   Show that $c\phi$ is differentiable relative
to its domain at $x$ for every $c\in\Bbb R$.
%262F

\sqheader 262Xf Suppose $\phi:D\to\BbbR^s$ and
$\psi:E\to\Bbb R^q$ are functions,
where $D\subseteq\BbbR^r$ and $E\subseteq\BbbR^s$;  suppose that
$\phi$ is differentiable relative to its domain at
$x\in D\cap\phi^{-1}[E]$, with an
$s\times r$ matrix $T$ a derivative there, and that $\psi$ is
differentiable
relative to its domain at $\phi(x)$, with a $q\times s$ matrix  $S$
a derivative there.   Show that the composition $\psi\phi$ is
differentiable relative to its
domain at $x$, and that the $q\times r$ matrix $ST$ is a derivative of
$\psi\phi$ at $x$.
%262F

\sqheader 262Xg Let $\phi:\BbbR^r\to\BbbR^s$ be a linear
operator, with associated matrix $T$.   Show that $\phi$ is
differentiable everywhere, with $\phi'(x)=T$ for every $x$.
%262F

\spheader 262Xh Let $G\subseteq\BbbR^r$ be a convex open set,
and $\phi:G\to\BbbR^s$ a function such that all the partial derivatives
$\pd{\phi_i}{\xi_j}$ are defined everywhere in $G$.   Show that $\phi$
is Lipschitz iff all the partial derivatives are bounded on $G$.
%262I

\spheader 262Xi Let $\phi:\BbbR^r\to\BbbR^s$ be a function.   Show that
$\phi$ is differentiable at $x\in\BbbR^r$ iff for every $m\in\Bbb N$
there are an $n\in\Bbb N$ and an $r\times s$ matrix $T$ with rational
coefficients
such that $\|\phi(y)-\phi(x)-T(y-x)\|\le 2^{-m}\|y-x\|$ whenever
$\|y-x\|\le 2^{-n}$.
%262I

\spheader 262Xj Suppose that $f$ is a real-valued function which is
integrable over $\BbbR^r$, and that $g:\BbbR^r\to\Bbb R$ is a bounded
differentiable function such that the partial derivative $\Pd{g}{\xi_j}$
is bounded, where $j\le r$.   Let $f*g$ be
the convolution of $f$ and $g$ (255E, 255L).
Show that $\Pd{}{\xi_j}(f*g)$
is defined everywhere and equal to $f*\Pd{g}{\xi_j}$.   \Hint{255Xd.}
%262+

\spheader 262Xk Let $(X,\Sigma,\mu)$ be a measure space,
$G\subseteq\BbbR^r$ an open set, and $f:X\times G\to\Bbb R$ a function.
Suppose that

\inset{(i) for every $x\in X$, $t\mapsto f(x,t):G\to\Bbb R$ is
differentiable;

(ii) there is an integrable function $g$ on $X$ such that
$|\pd{f}{\tau_j}(x,t)|\le g(x)$ whenever $x\in X$, $t\in G$ and
$j\le r$;

(iii) $\int|f(x,t)|\mu(dx)$ exists in $\Bbb R$ for every $t\in G$.}

\noindent Show that $t\mapsto\int f(x,t)\mu(dx):G\to\Bbb R$ is
differentiable.  \Hint{show first that, for a suitable $M$,
$|f(x,t)-f(x,t')|\le M|g(x)|\|t-t'\|$ for every  $t$, $t'\in G$ and
$x\in X$.}
%262+

\spheader 262Xl\dvAnew{2015} Let $f:[a,b]\to\Bbb R$ 
be an absolutely continuous
function, where $a\le b$, and $g:f[\,[a,b]\,]\to\Bbb R$ a Lipschitz
function.   Show that $gf$ is absolutely continuous.
%262Bc out of order query

\leader{262Y}{Further exercises (a)}
%\spheader 262Ya
Let $L$ be the space of all Lipschitz functions from
$\BbbR^r$ to $\BbbR^s$, and for $\phi\in L$ set

\Centerline{$\|\phi\|
=\|\phi(0)\|+\min\{\gamma:\gamma\in\coint{0,\infty}$,
$\|\phi(y)-\phi(x)\|\le\gamma\|y-x\|$ for every $x$, $y\in\BbbR^r\}$.}

\noindent Show that $(L,\|\,\,\|)$ is a Banach space.
%262A

\spheader 262Yb Show that if
$T=\langle\tau_{ij}\rangle_{i\le s,j\le r}$ is an $s\times r$ matrix
then the operator norm $\|T\|$, as defined in 262H, is at most
$\sqrt{\sum_{i=1}^s\sum_{j=1}^r\tau_{ij}^2}$.
%262H

\spheader 262Yc Let $\phi:D\to\Bbb R$ be any function, where
$D\subseteq\BbbR^r$.   Show that $H=\{x:x\in D,\,\phi$ is differentiable
relative to its domain at $x\}$ is relatively measurable in $D$, and
that $\pd{\phi}{\xi_j}\restr H$ is measurable for every $j\le r$.
%262I

\spheader 262Yd\dvAnew{2015}
Let $\phi:D\to\Bbb R$ be a function, where
$D\subseteq\BbbR^r$.
(i) Show that if $\phi$ is measurable then all its partial derivatives
are measurable.
(ii) Show that if $\phi$ is Borel
measurable then all its partial derivatives are Borel measurable.
%262Yc 262I 225Yg

\spheader 262Ye A function $\phi:\BbbR^r\to\Bbb R$ is {\bf smooth} if
all its partial derivatives
$\Bover{\partial\ldots\partial\phi}
{\partial\xi_i\partial\xi_j\ldots\partial\xi_l}$ are defined everywhere
in $\BbbR^r$ and are
continuous.   Show that if $f$ is integrable over $\BbbR^r$ and
$\phi:\BbbR^r\to\Bbb R$ is smooth and has bounded support then the
convolution $f*\phi$ is smooth.   \Hint{262Xj, 262Xk.}
%262Xk 262+

\spheader 262Yf For $\delta>0$ set
$\tilde\phi_{\delta}(x) =e^{1/(\delta^2-\|x\|^2)}$ if $\|x\|<\delta$,
$0$ if $\|x\|\ge\delta$;  set
$\alpha_{\delta}=\int\tilde\phi_{\delta}(x)dx$,
$\phi_{\delta}(x)=\alpha_{\delta}^{-1}\tilde\phi_{\delta}(x)$ for every
$x$.  (i) Show that $\phi_{\delta}:\BbbR^r\to\Bbb R$ is smooth and has
bounded support.   (ii) Show that if $f$ is integrable over $\BbbR^r$
then $\lim_{\delta\downarrow 0}\int|f(x)-(f*\phi_{\delta})(x)|dx=0$.
\Hint{start with continuous functions $f$ with bounded support, and use
242O.}
%262Ye 262+

\spheader 262Yg Show that if
$f$ is integrable over $\BbbR^r$ and $\epsilon>0$ there is a smooth
function $h$ with bounded support such that $\int|f-h|\le\epsilon$.
\Hint{{\it either} reduce to the case in which $f$ has bounded support
and use 262Yf {\it or} adapt the method of 242Xi.}
%262Yf 262+

\spheader 262Yh Suppose that $f$ is a real function which is integrable
over every bounded subset of $\BbbR^r$.   (i) Show that $f\times\phi$ is
integrable whenever $\phi:\BbbR^r\to\Bbb R$ is a smooth function with
bounded support.   (ii) Show that if $\int f\times\phi=0$ for every
smooth function with bounded support then $f=0$ a.e.   ({\it Hint\/}:
show that
$\int_{B(x,\delta)}f=0$ for every $x\in\BbbR^r$ and $\delta>0$, and use
261C.   Alternatively show that $\int_Ef=0$ first for $E=[b,c]$, then
for open sets $E$, then for arbitrary measurable sets $E$.)
%262Ye 262+

\spheader 262Yi Let $f$ be integrable over $\BbbR^r$, and for $\delta>0$
let $\phi_{\delta}:\BbbR^r\to\Bbb R$ be the function of 262Yf.   Show
that $\lim_{\delta\downarrow 0}(f*\phi_{\delta})(x)=f(x)$ for every $x$
in the Lebesgue set of $f$.   ({\it Hint\/}:  261Ye.)
%262Yf 262+

}%end of exercises

\endnotes{
\Notesheader{262} The emphasis of this section has turned out to be on
the connexions between the concepts of `Lipschitz function' and
`differentiable function'.   It is the delight of classical real
analysis
that such intimate relationships arise between concepts which belong to
different categories.   `Lipschitz functions' clearly belong to the
theory of metric spaces (I will return to this in \S264), while
`differentiable functions' belong to the theory of differentiable
manifolds, which is outside the scope of this volume.
I have written this section out carefully just in case there are readers
who have so far missed the theory of differentiable mappings between
multi-dimensional Euclidean spaces;  but it also gives me a chance to
work through the notion of `function differentiable relative to its
domain', which will make it possible in the next section to ride
smoothly past a variety of problems arising at boundaries.   The
difficulties I am concerned with arise in the first place with such
functions as the polar-coordinate transformation

\Centerline{$(\rho,\theta)\mapsto (\rho\cos\theta,\rho\sin\theta):
\{(0,0)\}\cup(\ooint{0,\infty}\times\ocint{-\pi,\pi})\to\BbbR^2$.}

\noindent In order to make this a bijection we have to do something
rather arbitrary, and the domain of the transformation cannot be an open
set.   On the definitions I am using, this function is differentiable
relative to its domain at every point of its domain, and we can apply
such results as 262O uninhibitedly.   You will observe that in this case
the non-interior points of the domain form a negligible set
$\{(0,0)\}\cup(\ooint{0,\infty}\times\{\pi\})$, so we can expect to be
able to ignore them;
and for most of the geometrically straightforward transformations that
the theory is applied to, judicious excision of negligible sets will
reduce problems to the case of honestly differentiable functions with
open domains.   But while open-domain theory will deal with a large
proportion of the most important examples, there is a danger that you
would be left with real misapprehensions concerning the scope of these
methods.

The essence of differentiability is that a differentiable function
$\phi$ is
approximable, near any given point of its domain, by an affine function.
The idea of 262M is to describe a widely effective method of dissecting
$D=\dom\phi$ into countably many pieces on each of which $\phi$ is
well controlled.   This will be applied in \S\S263 and 265 to investigate
the measure of $\phi[D]$;  but we already have several straightforward
consequences (262N-262P).

I have offered a number of results suggesting that (on the definitions I
have chosen) a derivative can be expected to share at least some
`descriptive' properties
with the original function;  see 222Yd, 225J, 225Yg, 262Yc, 262Yd.
For partial derivatives, there are complications concerning their domains
(419Yd, 431Yd) which do not arise with full derivatives (225J, 262Yc).
}%end of notes

\discrpage

