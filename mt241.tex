\frfilename{mt241.tex}
\versiondate{6.11.03}
\copyrightdate{1995}

\loadeusm

\def\chaptername{Function spaces}
\def\sectionname{$\eusm L^0$ and $L^0$}

\newsection{241}

The chief aim of this chapter is to discuss the spaces $L^1$,
$L^{\infty}$ and $L^p$ of the following three sections.   However it
will be convenient to regard all these as subspaces of a larger space
$L^0$ of equivalence classes of (virtually) measurable functions, and
I have collected in this section the basic facts concerning the ordered
linear space $L^0$.

It is almost the first principle of measure theory that sets of measure
zero can often be ignored;  the phrase `negligible set' itself
asserts this principle.   Accordingly, two functions which agree almost
everywhere may often (not always!) be treated as identical.   A suitable
expression of this idea is to form the space of equivalence classes of
functions, saying that two functions are equivalent if they agree on a
conegligible set.   This is the basis of all the constructions of this
chapter.   It is a remarkable fact that the spaces of equivalence
classes so constructed are actually better adapted to certain problems
than the spaces of functions from which they are derived, so that once
the technique has been mastered it is easier to do one's
thinking in the more abstract spaces.


\leader{241A}{The space $\eusm L^0$:  Definition}
\cmmnt{ It is time to give a
name to a set of functions which has already been used more than once.}
Let $(X,\Sigma,\mu)$ be a measure space.
I write $\eusm L^0$, or $\eusm L^0(\mu)$, for the space of
real-valued functions $f$
defined on conegligible subsets of $X$ which are virtually
measurable\cmmnt{, that is, such that $f\restr E$ is measurable for
some conegligible set $E\subseteq X$.   Recall that $f$ is
$\mu$-virtually measurable iff it is $\hat\Sigma$-measurable,
where $\hat\Sigma$ is the completion of $\Sigma$ (212Fa)}.

\leader{241B}{Basic properties}\cmmnt{ If $(X,\Sigma,\mu)$ is any
measure
space, then we have the following facts, corresponding to the
fundamental properties of measurable functions listed in \S121 of
Volume 1.   I work through them in order, so that if you have Volume 1
to hand you can see what has to be missed out.

\medskip

}{\bf (a)} A constant real-valued function defined almost everywhere in $X$ belongs to $\eusm L^0$\prooflet{ (121Ea)}.

\spheader 241Bb $f+g\in\eusm L^0$ for all $f$, $g\in\eusm L^0$\prooflet{ (for if
$f\restr F$ and $g\restr G$ are measurable, then
$(f+g)\restr(F\cap G)=(f\restr F)+(g\restr G)$ is measurable)(121Eb)}.

\spheader 241Bc $cf\in\eusm L^0$ for all $f\in\eusm L^0$,
$c\in\Bbb R$\prooflet{ (121Ec)}.

\spheader 241Bd $f\times g\in\eusm L^0$ for all $f$,
$g\in\eusm L^0$\prooflet{ (121Ed)}.

\spheader 241Be If $f\in\eusm L^0$ and $h:\Bbb R\to\Bbb R$ is Borel measurable, then
$hf\in\eusm L^0$\prooflet{ (121Eg)}.

\spheader 241Bf If $\sequencen{f_n}$ is a sequence in $\eusm L^0$ and
$f=\lim_{n\to\infty}f_n$ is defined\cmmnt{ (as a real-valued
function)}
almost everywhere in $X$, then $f\in\eusm L^0$\prooflet{ (121Fa)}.

\spheader 241Bg If $\sequencen{f_n}$ is a sequence in $\eusm L^0$ and
$f=\sup_{n\in\Bbb N}f_n$ is defined\cmmnt{ (as a real-valued
function)}
almost everywhere in $X$, then $f\in\eusm L^0$\prooflet{ (121Fb)}.

\spheader 241Bh If $\sequencen{f_n}$ is a sequence in $\eusm L^0$ and
$f=\inf_{n\in\Bbb N}f_n$ is defined\cmmnt{ (as a real-valued
function)}
almost everywhere in $X$, then $f\in\eusm L^0$\prooflet{ (121Fc)}.

\spheader 241Bi If $\sequencen{f_n}$ is a sequence in $\eusm L^0$ and
$f=\limsup_{n\to\infty}f_n$ is defined\cmmnt{ (as a real-valued
function)} almost everywhere in $X$, then
$f\in\eusm L^0$\prooflet{ (121Fd)}.

\spheader 241Bj If $\sequencen{f_n}$ is a sequence in $\eusm L^0$ and
$f=\liminf_{n\to\infty}f_n$ is defined\cmmnt{ (as a real-valued
function)} almost everywhere in $X$, then
$f\in\eusm L^0$\prooflet{ (121Fe)}.

\spheader 241Bk $\eusm L^0$ is just the set of real-valued functions, defined on
subsets of $X$, which are equal almost everywhere
to some $\Sigma$-measurable function from $X$ to $\Bbb R$.
\prooflet{\Prf\ (i) If $g:X\to\Bbb R$ is $\Sigma$-measurable and
$f\eae g$, then $F=\{x:x\in\dom f,\,f(x)=g(x)\}$ is conegligible and
$f\restr F=g\restr F$ is measurable (121Eh), so $f\in\eusm L^0$.   (ii)
If $f\in\eusm L^0$, let $E\subseteq X$ be a conegligible set such that
$f\restr E$ is measurable.   Then $D=E\cap\dom f$ is conegligible and
$f\restr D$ is measurable, so there is a measurable $h:X\to\Bbb R$
agreeing with $f$ on $D$ (121I);  and $h\eae f$.\ \Qed}

\leader{241C}{The space $L^0$:  Definition} Let $(X,\Sigma,\mu)$ be any
measure space.    Then $\eae$ is an equivalence relation on $\eusm L^0$.
Write $L^0$, or
$L^0(\mu)$, for the set of equivalence classes in $\eusm L^0$ under
$\eae$.   For $f\in\eusm L^0$, write $f^{\ssbullet}$ for its equivalence
class in $L^0$.

\leader{241D}{The linear structure of $L^0$} Let $(X,\Sigma,\mu)$ be any
measure space, and set $\eusm L^0=\eusm L^0(\mu)$, $L^0=L^0(\mu)$.

\spheader 241Da If $f_1$, $f_2$, $g_1$, $g_2\in\eusm L^0$,
$f_1\eae f_2$ and
$g_1\eae g_2$ then $f_1+g_1\eae f_2+g_2$.   Accordingly we may define
addition on $L^0$ by setting
$f^{\ssbullet}+g^{\ssbullet}=(f+g)^{\ssbullet}$
for all $f$, $g\in\eusm L^0$.

\spheader 241Db If $f_1$, $f_2\in\eusm L^0$ and $f_1\eae f_2$,
then $cf_1\eae cf_2$ for every $c\in\Bbb R$.   Accordingly we may define
scalar multiplication on $L^0$ by setting
$c\cdot f^{\ssbullet}=(cf)^{\ssbullet}$ for
all $f\in\eusm L^0$ and $c\in\Bbb R$.


\spheader 241Dc\cmmnt{ Now} $L^0$ is a linear space over $\Bbb R$, with zero $\tbf{0}^{\ssbullet}$, where $\tbf{0}$ is the function with domain $X$ and
constant value $0$, and negatives $-(f^{\ssbullet})=(-f)^{\ssbullet}$.
\prooflet{\Prf\ (i)

\Centerline{$f+(g+h)=(f+g)+h$ for all $f$, $g$, $h\in\eusm L^0$,}

\noindent so


\Centerline{$u+(v+w)=(u+v)+w$ for all $u$, $v$, $w\in L^0$.}

(ii)

\Centerline{$f+\tbf{0}=\tbf{0}+f=f$ for every $f\in\eusm L^0$,}

\noindent so

\Centerline{$u+\tbf{0}^{\ssbullet}=\tbf{0}^{\ssbullet}+u=u$ for every
$u\in L^0$.}

(iii)

\Centerline{$f+(-f)\eae\tbf{0}$ for every $f\in\eusm L^0$,}

\noindent so
\Centerline{$f^{\ssbullet}+(-f)^{\ssbullet}=\tbf{0}^{\ssbullet}$ for
every $f\in\eusm L^0$.}

(iv)

\Centerline{$f+g=g+f$ for all $f$, $g\in\eusm L^0$,}

\noindent so

\Centerline{$u+v=v+u$ for all $u$,
$v\in L^0$.}

(v)

\Centerline{$c(f+g)=cf+cg$ for all $f$, $g\in\eusm L^0$ and
$c\in\Bbb R$,}

\noindent so

\Centerline{$c(u+v)=cu+cv$ for all $u$, $v\in L^0$ and $c\in\Bbb
R$.}

(vi)

\Centerline{$(a+b)f=af+bf$ for all $f\in\eusm L^0$ and $a$,
$b\in\Bbb R$,}

\noindent so

\Centerline{$(a+b)u=au+bu$ for all $u\in L^0$ and $a$, $b\in\Bbb R$.}

(vii)

\Centerline{$(ab)f=a(bf)$ for all $f\in\eusm L^0$ and $a$,
$b\in\Bbb R$,}


\noindent so


\Centerline{$(ab)u=a(bu)$ for all $u\in L^0$ and $a$, $b\in\Bbb R$.}

(viii)

\Centerline{$1f=f$ for all $f\in\eusm L^0$,}

\noindent so

\Centerline{$1u=u$ for all $u\in L^0$. \Qed}
}%end of prooflet

\leader{241E}{The order structure of $L^0$}  Let $(X,\Sigma,\mu)$ be any
measure space and set $\eusm L^0=\eusm L^0(\mu)$, $L^0=L^0(\mu)$.

\spheader 241Ea If $f_1$, $f_2$, $g_1$, $g_2\in\eusm L^0$,
$f_1\eae f_2$,
$g_1\eae g_2$ and $f_1\leae g_1$, then $f_2\leae g_2$.
Accordingly we may define a relation $\le$ on $L^0$ by saying that
$f^{\ssbullet}\le g^{\ssbullet}$ iff $f\leae g$.

\spheader 241Eb\cmmnt{ Now} $\le$ is a partial order on $L^0$.
\prooflet{\Prf\ (i) If $f$, $g$, $h\in\eusm L^0$ and $f\leae g$ and
$g\leae h$, then $f\leae h$.   Accordingly $u\le w$ whenever $u$, $v$,
$w\in L^0$, $u\le v$ and $v\le w$.   (ii) If $f\in \eusm L^0$ then
$f\leae f$;  so $u\le u$ for every $u\in L^0$.   (iii) If $f$,
$g\in\eusm L^0$ and $f\leae g$ and $g\leae f$, then $f\eae g$, so if
$u\le v$ and $v\le u$ then $u=v$.\ \Qed}

\spheader 241Ec\cmmnt{ In fact} $L^0$, with $\le$, is a {\bf partially
ordered linear
space}, that is, a (real) linear space with a partial order $\le$
such that

\qquad if $u\le v$ then $u+w\le v+w$ for every $w$,

\qquad if $0\le u$ then $0\le cu$ for every $c\ge 0$.

\noindent\prooflet{\Prf\ (i) If $f$, $g$, $h\in\eusm L^0$ and
$f\leae g$, then $f+h\leae g+h$.  (ii) If $f\in\eusm L^0$ and
$f\ge 0$ a.e.,
then $cf\ge 0$ a.e.\ for every $c\ge 0$.\ \Qed}

\spheader 241Ed\cmmnt{ More:}  $L^0$ is a {\bf Riesz space} or {\bf
vector lattice}, that is, a partially ordered linear space such that
$u\vee v=\sup\{u,v\}$ and $u\wedge v=\inf\{u,v\}$ are defined for all
$u$, $v\in L^0$.
\prooflet{\Prf\ Take $f$, $g\in \eusm L^0$ such that $f^{\ssbullet}=u$
and $g^{\ssbullet}=v$.  Then $f\vee g$, $f\wedge g\in\eusm L^0$, writing

\Centerline{$(f\vee g)(x)=\max(f(x),g(x))$, \quad $(f\wedge
g)(x)=\min(f(x),g(x))$}

\noindent for $x\in\dom f\cap\dom g$.   (Compare 241Bg-h.)   Now, for
any $h\in\eusm L^0$, we have

\Centerline{$f\vee g\leae h\iff f\leae h$ and $g\leae h$,}

\Centerline{$h\leae f\wedge g\iff h\leae f$ and $h\leae g$,}

\noindent so for any $w\in L^0$ we have

\Centerline{$(f\vee g)^{\ssbullet}\le w\iff u\le w$ and $v\le w$,}

\Centerline{$w\le(f\wedge g)^{\ssbullet}\iff w\le u$ and $w\le v$.}

\noindent Thus we have

\Centerline{$(f\vee g)^{\ssbullet}=\sup\{u,v\}=u\vee v$,
\quad$(f\wedge g)^{\ssbullet}=\inf\{u,v\}=u\wedge v$}

\noindent in $L^0$.\ \Qed}

\spheader 241Ee In particular, for any $u\in L^0$ we can speak
of $|u|=u\vee(-u)$;  if $f\in\eusm L^0$ then
$|f^{\ssbullet}|=|f|^{\ssbullet}$.

If\cmmnt{ $f$, $g\in\eusm L^0$,} $c\in\Bbb R$ then

\prooflet{
\Centerline{$|cf|=|c||f|$, \quad$f\vee g=\Bover12(f+g+|f-g|)$,}

\Centerline{
$f\wedge g=\Bover12(f+g-|f-g|)$, \quad$|f+g|\leae|f|+|g|$,}

\noindent so
}

\Centerline{$|cu|=|c||u|$, \quad$u\vee v=\Bover12(u+v+|u-v|)$,}

\Centerline{$u\wedge v=\Bover12(u+v-|u-v|)$, \quad$|u+v|\le|u|+|v|$}

\noindent for all $u$, $v\in L^0$.

\spheader 241Ef\cmmnt{ A special notation is often useful.}   
If $f$ is a real-valued function, set $f^+(x)=\max(f(x),0)$,
$f^-(x)=\max(-f(x),0)$ for $x\in\dom f$, so that

\Centerline{$f=f^+-f^-$, \quad$|f|=f^++f^-=f^+\vee f^-$,}

\noindent all these
functions being defined on $\dom f$.   In $L^0$, the corresponding
operations are $u^+=u\vee 0$, $u^-=(-u)\vee 0$, and we have

\Centerline{$u=u^+-u^-$,
\quad$|u|=u^++u^-=u^+\vee u^-$,
\quad$u^+\wedge u^-=0$.}

\spheader 241Gg\dvro{ If}{ It is perhaps obvious, but I say it anyway:
if} $u\ge 0$ in $L^0$, then there is an $f\ge 0$ in $\eusm L^0$ such
that $f^{\ssbullet}=u$.   \prooflet{\Prf\ Take any $g\in\eusm L^0$ such
that $u=g^{\ssbullet}$, and set $f=g\vee\tbf{0}$.\ \Qed}


\leader{241F}{Riesz spaces}\cmmnt{ There is an extensive abstract
theory of
Riesz spaces, which I think it best to leave aside for the moment;  a
general account may be found in {\smc Luxemburg \& Zaanen 71} and 
{\smc Zaanen 83};  my own book {\smc Fremlin 74} covers
the elementary material, and Chapter 35 in the next volume repeats the
most essential ideas.   For our purposes here we need only a few
definitions and some simple results which are most easily proved for the
special cases in which we need them, without reference to the general
theory.

\medskip

} {\bf (a)} A Riesz space $U$ is {\bf Archimedean} if
whenever $u\in U$, $u>0$\cmmnt{ (that is, $u\ge 0$ and $u\ne 0$),} and
$v\in U$, there is an $n\in\Bbb N$ such that $nu\not\le v$.

\spheader 241Fb A Riesz space $U$ is {\bf Dedekind
$\sigma$-complete}\cmmnt{ (or {\bf $\sigma$-order-complete}, or {\bf
$\sigma$-complete})} if every non-empty countable set $A\subseteq U$
which is bounded above has a least upper bound in $U$.

\spheader 241Fc A Riesz space $U$ is {\bf Dedekind complete} (or
{\bf order complete}, or {\bf complete}) if every
non-empty set $A\subseteq U$ which is bounded above in $U$ has a least
upper bound in $U$.

\leader{241G}{}\cmmnt{ Now we have the following important properties
of $L^0$.

\medskip

\noindent}{\bf Theorem} Let $(X,\Sigma,\mu)$ be a measure space.   Set
$L^0=L^0(\mu)$.

(a) $L^0$ is Archimedean and Dedekind $\sigma$-complete.

(b) If $(X,\Sigma,\mu)$ is
semi-finite, then $L^0$ is Dedekind complete iff $(X,\Sigma,\mu)$ is
localizable.

\proof{ Set $\eusm L^0=\eusm L^0(\mu)$.

\medskip

{\bf (a)(i)} If $u$, $v\in L^0$ and $u>0$, express $u$ as
$f^{\ssbullet}$ and $v$ as $g^{\ssbullet}$ where $f$, $g\in\eusm L^0$.
Then $E=\{x:x\in\dom f,\,f(x)>0\}$ is not negligible.   So there is an
$n\in\Bbb N$ such that

\Centerline{$E_n=\{x:x\in\dom f\cap\dom g,\,nf(x)>g(x)\}$}

\noindent is not negligible,  since $E\cap\dom
g\subseteq\bigcup_{n\in\Bbb N}E_n$.   But now $nu\not\le v$.
As $u$ and $v$ are arbitrary, $L^0$ is Archimedean.

\medskip

\quad{\bf (ii)} Now let $A\subseteq L^0$ be a non-empty countable set
with an upper bound $w$ in $L^0$.   Express $A$ as
$\{f_n^{\ssbullet}:n\in\Bbb N\}$ where $\sequencen{f_n}$ is a sequence
in $\eusm L^0$, and $w$ as $h^{\ssbullet}$ where $h\in\eusm L^0$.   Set
$f=\sup_{n\in\Bbb N}f_n$.   Then we have $f(x)$ defined in $\Bbb R$ at
any point $x\in\dom h\cap\bigcap_{n\in\Bbb N}\dom f_n$ such that
$f_n(x)\le h(x)$ for every $n\in\Bbb N$, that is, for almost every $x\in
X$;  so $f\in\eusm L^0$ (241Bg).   Set $u=f^{\ssbullet}\in L^0$.   If
$v\in L^0$, say $v=g^{\ssbullet}$ where $g\in\eusm L^0$, then

$$\eqalign{u_n\le v&\text{ for every }n\in\Bbb N\cr
&\iff\text{ for every }n\in\Bbb N,\,f_n\leae g\cr
&\iff\text{ for almost every }x\in X,\,f_n(x)\le g(x)
      \text{ for every }n\in\Bbb N\cr
&\iff f\leae g
\iff u\le v.\cr}$$

\noindent Thus $u=\sup_{n\in\Bbb N}u_n$ in $L^0$.   As $A$ is arbitrary,
$L^0$ is Dedekind $\sigma$-complete.

\medskip

{\bf (b)(i)} Suppose that $(X,\Sigma,\mu)$ is localizable.   Let
$A\subseteq L^0$ be any non-empty set with an upper bound $w_0\in L^0$.
Set

\Centerline{$\Cal A=\{f:f$ is a measurable function from $X$ to
$\Bbb R,\,f^{\ssbullet}\in A\}$;}

\noindent then every member of $A$ is of the form $f^{\ssbullet}$ for
some $f\in\Cal A$ (241Bk).     For each $q\in\Bbb Q$, let $\Cal E_q$ be
the family of subsets of $X$ expressible in the form
$\{x:f(x)\ge q\}$ for some $f\in\Cal A$;  then $\Cal
E_q\subseteq\Sigma$.   Because $(X,\Sigma,\mu)$ is localizable, there is
a set $F_q\in\Sigma$ which is an essential supremum for $\Cal E_q$.
For $x\in X$, set

\Centerline{$g^*(x)=\sup\{q:q\in\Bbb Q,\,x\in F_q\}$,}

\noindent allowing $\infty$ as the supremum of a set which is not
bounded above, and $-\infty$ as $\sup\emptyset$.   Then

\Centerline{$\{x:g^*(x)>a\}=\bigcup_{q\in\Bbb Q,q>a}F_q\in\Sigma$}

\noindent for every $a\in\Bbb R$.

If $f\in\Cal A$, then $f\leae g^*$.   \Prf\ For each $q\in\Bbb Q$, set

\Centerline{$E_q=\{x:f(x)\ge q\}\in\Cal E_q$;}

\noindent then $E_q\setminus F_q$ is negligible.   Set
$H=\bigcup_{q\in\Bbb Q}(E_q\setminus F_q)$.   If $x\in X\setminus H$,
then

\Centerline{$f(x)\ge q\Longrightarrow g^*(x)\ge q$,}

\noindent so $f(x)\le g^*(x)$;  thus $f\leae g^*$.\ \Qed

If $h:X\to\Bbb R$ is measurable and $u\le h^{\ssbullet}$ for every $u\in
A$, then $g^*\leae h$.   \Prf\ Set $G_q=\{x:h(x)\ge q\}$ for each
$q\in\Bbb Q$.   If $E\in\Cal E_q$, there is an $f\in\Cal A$ such that
$E=\{x:f(x)\ge q\}$;  now $f\leae h$, so
$E\setminus G_q\subseteq\{x:f(x)>h(x)\}$ is negligible.   Because $F_q$
is an
essential supremum for $\Cal E_q$, $F_q\setminus G_q$ is negligible;
and this is true for every $q\in\Bbb Q$.   Consequently

\Centerline{$\{x:h(x)<g^*(x)\}
\subseteq\bigcup_{q\in\Bbb Q}F_q\setminus G_q$}

\noindent is negligible, and $g^*\leae h$.\ \Qed

Now recall that we are assuming that $A\ne\emptyset$ and that $A$ has an
upper bound $w_0\in L^0$.   Take any $f_0\in\Cal A$ and a measurable
$h_0:X\to\Bbb R$ such that $h_0^{\ssbullet}=w_0$;  then $f\leae h_0$
for every $f\in \Cal A$, so $f_0\leae g^*\leae h_0$, and $g^*$ must be
finite a.e.   Setting $g(x)=g^*(x)$ when $g^*(x)\in\Bbb R$, we have
$g\in\eusm L^0$ and $g\eae g^*$, so that

\Centerline{$f\leae g\leae h$}

\noindent whenever $f$, $h$ are measurable functions from $X$ to $\Bbb
R$, $f^{\ssbullet}\in A$ and $h^{\ssbullet}$ is an upper bound for $A$;
that is,

\Centerline{$u\le g^{\ssbullet}\le w$}

\noindent whenever $u\in A$ and $w$ is an upper bound for $A$.   But
this means that $g^{\ssbullet}$ is a least upper bound for $A$ in $L^0$.
As $A$ is arbitrary, $L^0$ is Dedekind complete.

\medskip

{\bf (ii)} Suppose that $L^0$ is Dedekind complete.   We are assuming
that $(X,\Sigma,\mu)$ is
semi-finite.   Let $\Cal E$ be any subset of $\Sigma$.   Set

\Centerline{$A=\{0\}\cup\{(\chi E)^{\ssbullet}:E\in\Cal E\}\subseteq
L^0$.}

\noindent Then $A$ is bounded above by $(\chi X)^{\ssbullet}$ so has a
least upper bound $w\in L^0$.   Express $w$ as $h^{\ssbullet}$ where
$h:X\to\Bbb R$ is measurable, and set $F=\{x:h(x)>0\}$.   Then $F$ is an
essential supremum for $\Cal E$ in $\Sigma$.   \Prf\ {\bf
(${\alpha}$)} If $E\in\Cal E$, then $(\chi E)^{\ssbullet}\le w$ so
$\chi E\leae h$, that is, $h(x)\ge 1$ for almost every $x\in E$, and
$E\setminus F\subseteq\{x:x\in E,\,h(x)<1\}$ is negligible.
{\bf (${\beta}$)} If $G\in\Sigma$ and $E\setminus G$ is negligible
for every $E\in\Cal E$, then $\chi E\leae\chi G$ for every
$E\in\Cal E$, that is, $(\chi E)^{\ssbullet}\le(\chi G)^{\ssbullet}$ for
every
$E\in\Cal E$;  so $w\le(\chi G)^{\ssbullet}$, that is, $h\leae\chi G$.
Accordingly $F\setminus G\subseteq\{x:h(x)>(\chi G)(x)\}$ is
negligible.\ \Qed

As $\Cal E$ is arbitrary, $(X,\Sigma,\mu)$ is localizable.
}%end of proof of 241G

\leader{241H}{The multiplicative structure of $L^0$}  Let
$(X,\Sigma,\mu)$ be any measure space;  write $L^0=L^0(\mu)$,
$\eusm L^0=\eusm L^0(\mu)$.

\spheader 241Ha If $f_1$, $f_2$, $g_1$, $g_2\in\eusm L^0$,
$f_1\eae f_2$ and
$g_1\eae g_2$ then $f_1\times g_1\eae f_2\times g_2$.   Accordingly we
may define multiplication on $L^0$ by setting
$f^{\ssbullet}\times g^{\ssbullet}=(f\times g)^{\ssbullet}$ for all $f$,
$g\in\eusm L^0$.

\spheader 241Hb\dvro{ For}{ It is
now easy to check that, for} all $u$, $v$, $w\in L^0$ and $c\in\Bbb R$,

\qquad $u\times(v\times w)=(u\times v)\times w$,

\qquad $u\times e=e\times u=u$,

\noindent where $e=\chi X^{\ssbullet}$ is the
equivalence class of the function with constant value $1$,

\qquad $c(u\times v)=cu\times v=u\times cv$,

\qquad $u\times(v+w)=(u\times v)+(u\times w)$,

\qquad $(u+v)\times w=(u\times w)+(v\times w)$,

\qquad $u\times v=v\times u$,

\qquad $|u\times v|=|u|\times |v|$,

\qquad $u\times v=0$ iff  $|u|\wedge|v|=0$,

\qquad $|u|\le|v|$ iff there is a $w$ such that
$|w|\le e$ and $u=v\times w$.

\leader{241I}{The action of Borel functions on $L^0$} Let
$(X,\Sigma,\mu)$ be a measure space and $h:\Bbb R\to\Bbb R$ a Borel
measurable function.   Then\cmmnt{ $hf\in\eusm L^0=\eusm L^0(\mu)$ for
every $f\in\eusm L^0$\prooflet{ (241Be)} and $hf\eae hg$ whenever
$f\eae g$.   So} we have a function $\bar h:L^0\to L^0$ defined by
setting
$\bar h(f^{\ssbullet})=(hf)^{\ssbullet}$ for every $f\in\eusm L^0$.
\cmmnt{For instance, if $u\in L^0$ and $p\ge 1$, we can consider
$|u|^p=\bar h(u)$ where $h(x)=|x|^p$ for $x\in\Bbb R$.}



\leader{241J}{Complex $L^0$}\cmmnt{ The ideas of this chapter, like
those of Chapters 22-23, are
often applied to spaces based on complex-valued functions instead of
real-valued functions.}   Let $(X,\Sigma,\mu)$ be a measure space.

\spheader 241Ja We may write
$\eusm L^0_{\Bbb C}=\eusm L^0_{\Bbb C}(\mu)$ for the space of
complex-valued functions $f$ such that $\dom f$ is a conegligible subset of $X$ and there is a conegligible subset
$E\subseteq X$ such that $f\restr E$ is measurable;  that is, such that
the real and imaginary parts of $f$ both belong to $\eusm L^0(\mu)$.
\cmmnt{Next,} $L^0_{\Bbb C}=L^0_{\Bbb C}(\mu)$ will be the space of
equivalence classes in $\eusm L^0_{\Bbb C}$ under the equivalence
relation $\eae$.

\spheader 241Jb Using just the same formulae as in 241D, it is
easy to describe addition and scalar multiplication rendering $L^0_{\Bbb
C}$ a linear space over $\Bbb C$.   We\cmmnt{ no longer have quite the
same kind of order structure, but we} can identify a `real part', being

\Centerline{$\{f^{\ssbullet}:f\in\eusm L^0_{\Bbb C}$ is real a.e.$\}$,}

\noindent \cmmnt{obviously }identifiable with the real linear space
$L^0$, and
corresponding maps $u\mapsto\Real(u)$, $u\mapsto\Imag(u):L^0_{\Bbb C}\to
L^0$ such that $u=\Real(u)+i\Imag(u)$ for every $u$.   Moreover, we have
a notion of `modulus', writing

\Centerline{$|f^{\ssbullet}|=|f|^{\ssbullet}\in L^0$ for every
$f\in\eusm L^0_{\Bbb C}$,}

\noindent satisfying the basic relations $|cu|=|c||u|$,
$|u+v|\le|u|+|v|$ for $u$, $v\in L^0_{\Bbb C}$ and
$c\in\Bbb C$\cmmnt{, as in 241Ef}.
We do\cmmnt{ of course} still have a multiplication on $L^0_{\Bbb C}$,
for which all the formulae in 241H are still valid.

\spheader 241Jc\cmmnt{ The following fact is useful.} For any $u\in
L^0_{\Bbb C}$, $|u|$ is the supremum in $L^0$ of $\{\Real(\zeta
u):\zeta\in\Bbb C,\,|\zeta|=1\}$.  \prooflet{\Prf\ (i) If $|\zeta|=1$,
then $\Real(\zeta u)\le|\zeta u|=|u|$.   So $|u|$ is an upper bound of
$\{\Real(\zeta u):|\zeta|=1\}$.   (ii) If $v\in L^0$ and $\Real(\zeta
u)\le v$ whenever $|\zeta|=1$, then express $u$, $v$ as $f^{\ssbullet}$,
$g^{\ssbullet}$ where $f:X\to\Bbb C$ and $g:X\to\Bbb R$ are measurable.
For any $q\in\Bbb Q$, $x\in X$ set $f_q(x)=\Real(e^{iqx}f(x))$.   Then
$f_q\leae g$.   Accordingly $H=\{x:f_q(x)\le g(x)$ for every
$q\in\Bbb Q\}$ is conegligible.   But of course
$H=\{x:|f(x)|\le g(x)\}$, so
$|f|\leae g$ and $|u|\le v$.   As $v$ is arbitrary, $|u|$ is the
least upper bound of $\{\Real(\zeta u):|\zeta|=1\}$.\ \Qed}

\exercises{
\leader{241X}{Basic exercises $\pmb{>}$(a)}
%\spheader 241Xa
Let $X$ be a set, and let $\mu$ be
counting measure on $X$ (112Bd).   Show that $L^0(\mu)$ can be
identified with $\eusm L^0(\mu)=\Bbb R^X$.
%241C

\sqheader 241Xb Let $(X,\Sigma,\mu)$ be a measure space and
$\hat\mu$ the completion of $\mu$.   Show that
$\eusm L^0(\mu)=\eusm L^0(\hat\mu)$ and $L^0(\mu)=L^0(\hat\mu)$.
%241C

\spheader 241Xc Let $(X,\Sigma,\mu)$ be a measure space.   (i)
Show that for every $u\in L^0(\mu)$ we may define an outer measure
$\theta_u:\Cal P\Bbb R\to[0,\infty]$ by writing
$\theta_u(A)=\mu^*f^{-1}[A]$ whenever $A\subseteq\Bbb R$ and
$f\in\eusm L^0(\mu)$ is such that $f^{\ssbullet}=u$.    (ii) Show that
the measure defined from
$\theta_u$ by \Caratheodory's method measures
every Borel subset of $\Bbb R$.
%241C

\spheader 241Xd Let $\langle(X_i,\Sigma_i,\mu_i)\rangle_{i\in I}$ be a
family of measure spaces, with direct sum $(X,\Sigma,\mu)$ (214L).
(i) Writing $\phi_i:X_i\to X$ for the canonical maps (in the
construction of 214L, $\phi_i(x)=(x,i)$ for $x\in X_i$), show that
$f\mapsto\langle f\phi_i\rangle_{i\in I}$ is a bijection between
$\eusm L^0(\mu)$ and $\prod_{i\in I}\eusm L^0(\mu_i)$.   (ii) Show that
this corresponds to a bijection between $L^0(\mu)$ and
$\prod_{i\in I}L^0(\mu_i)$.
%241C

\spheader 241Xe Let $U$ be a Dedekind $\sigma$-complete Riesz
space and $A\subseteq U$ a non-empty countable set which is bounded
below in $U$.   Show that $\inf A$ is defined in $U$.
%241F

\spheader 241Xf Let $U$ be a Dedekind complete Riesz space and
$A\subseteq U$ a non-empty set which is bounded below in $U$.   Show
that $\inf A$ is defined in $U$.
%241F

\spheader 241Xg Let $(X,\Sigma,\mu)$ and $(Y,\Tau,\nu)$ be measure
spaces, and $\phi:X\to Y$ an \imp\ function.   (i) Show that we have a
map $T:L^0(\nu)\to L^0(\mu)$ defined by setting
$Tg^{\ssbullet}=(g\phi)^{\ssbullet}$ for every $g\in\eusm L^0(\nu)$.
(ii) Show that
$T$ is linear, that $T(v\times w)=Tv\times Tw$ for all $v$,
$w\in L^0(\nu)$, and that $T(\sup_{n\in\Bbb N}v_n)=\sup_{n\in\Bbb N}Tv_n$
whenever $\sequencen{v_n}$ is a sequence in $L^0(\nu)$ with an upper
bound in $L^0(\nu)$.
%241H

\sqheader 241Xh Let $(X,\Sigma,\mu)$ be a measure space.
Suppose that $r\ge 1$ and that $h:\BbbR^r\to\Bbb R$ is a Borel
measurable function.   Show that there is a function
$\bar h:L^0(\mu)^r\to L^0(\mu)$ defined by writing

\Centerline{$\bar h(f_1^{\ssbullet},\ldots,f_r^{\ssbullet})
=(h(f_1,\ldots,f_r))^{\ssbullet}$}

\noindent for $f_1,\ldots,f_r\in{\eusm L}^0(\mu)$.
%241I

\spheader 241Xi Let $(X,\Sigma,\mu)$ be a measure space and $g$, $h$,
$\sequencen{g_n}$ Borel measurable functions from $\Bbb R$ to itself;
write $\bar g$, $\bar h$, $\bar g_n$ for the corresponding functions
from $L^0=L^0(\mu)$ to itself (241I).   (i) Show that

\Centerline{$\bar g(u)+\bar h(u)=\overline{g+h}(u)$,
\quad$\bar g(u)\times\bar h(u)=\overline{g\times h}(u)$,
\quad$\bar g(\bar h(u))=\overline{gh}(u)$}

\noindent for every $u\in L^0$.   (ii) Show that if $g(t)\le h(t)$ for
every $t\in\Bbb R$, then $\bar g(u)\le\bar h(u)$ for every $u\in L^0$.
(iii) Show that if $g$ is non-decreasing, then $\bar g(u)\le\bar g(v)$
whenever $u\le v$ in $L^0$.   (iv) Show that if $h(t)=\sup_{n\in\Bbb
N}g_n(t)$ for every $t\in\Bbb R$, then $\bar h(u)=\sup_{n\in\Bbb N}\bar
g_n(u)$ in $L^0$ for every $u\in L^0$.
%241I

\leader{241Y}{Further exercises (a)}
%\spheader 241Ya
Let $U$ be any Riesz space.   For $u\in U$ write
$|u|=u\vee(-u)$, $u^+=u\vee 0$, $u^-=(-u)\vee 0$.   Show that, for any
$u$, $v\in U$,

\Centerline{$u=u^+-u^-$,\quad$|u|=u^++u^-=u^+\vee u^-$,
\quad$u^+\wedge u^-=0$,}

\Centerline{$u\vee v=\Bover12(u+v+|u-v|)=u+(v-u)^+$,}

\Centerline{$u\wedge v=\Bover12(u+v-|u-v|)=u-(u-v)^+$,}

\Centerline{$|u+v|\le|u|+|v|$.}
%241E

\spheader 241Yb Let $U$ be a partially ordered linear space and
$N$ a linear subspace of $U$ such that whenever $u$, $u'\in N$ and
$u'\le v\le u$ then $v\in N$.   (i) Show that the linear space quotient
$U/N$ is a
partially ordered linear space if we say that
$u^{\ssbullet}\le v^{\ssbullet}$ in $U/N$ iff there is a $w\in N$ such
that $u\le v+w$ in
$U$.   (ii) Show that in this case $U/N$ is a Riesz space if $U$ is a
Riesz space and $|u|\in N$ for every $u\in N$.
%241E

\spheader 241Yc Let $(X,\Sigma,\mu)$ be a measure space.   Write
$\eusm L^0_{\Sigma}$ for the space of all measurable functions
from $X$ to $\Bbb R$, and $\eusm N$ for the subspace of
$\eusm L^0_{\Sigma}$ consisting of measurable functions which are
zero almost everywhere.   (i) Show that $\eusm L^0_{\Sigma}$ is a
Dedekind $\sigma$-complete Riesz space.   (ii) Show that $L^0(\mu)$ can
be identified, as ordered linear space, with the quotient
$\eusm L^0_{\Sigma}/\eusm N$ as defined in 241Yb above.
%241Yb 241E

\spheader 241Yd Show that any Dedekind
$\sigma$-complete Riesz space is Archimedean.
%241F

\spheader 241Ye A Riesz space $U$ is said to have the
{\bf countable sup property} if for every $A\subseteq U$ with a least
upper bound in $U$, there is a countable $B\subseteq A$ such that
$\sup B=\sup A$.   Show that if $(X,\Sigma,\mu)$ is a
semi-finite measure space, then it is
$\sigma$-finite iff $L^0(\mu)$ has the countable sup property.
%241G

\spheader 241Yf
Let $(X,\Sigma,\mu)$ be a measure space and $\tilde\mu$ the c.l.d.\
version of $\mu$ (213E).   (i) Show that
$\eusm L^0(\mu)\subseteq\eusm L^0(\tilde\mu)$.   (ii) Show that this
inclusion defines a linear operator
$T:L^0(\mu)\to L^0(\tilde\mu)$ such that $T(u\times v)=Tu\times Tv$ for
all $u$, $v\in L^0(\mu)$.   (iii) Show that whenever $v>0$ in
$L^0(\tilde\mu)$ there is a $u\ge 0$ in $L^0(\mu)$ such that
$0<Tu\le v$.   (iv) Show that
$T(\sup A)=\sup T[A]$ whenever $A\subseteq L^0(\mu)$ is a non-empty set
with a least upper bound in $L^0(\mu)$.   (v) Show that $T$ is injective
iff $\mu$ is semi-finite.   (vi) Show that if $\mu$ is localizable, then
$T$ is an isomorphism for the linear and order structures of $L^0(\mu)$
and $L^0(\tilde\mu)$.   \Hint{213Hb.}
%241H

\spheader 241Yg Let $(X,\Sigma,\mu)$ be a measure space and $Y$
any subset of $X$;  let $\mu_Y$ be the subspace measure on $Y$.   (i)
Show that $\eusm L^0(\mu_Y)=\{f\restrp Y:f\in\eusm L^0(\mu)\}$.   (ii)
Show that there is a canonical surjection $T:L^0(\mu)\to L^0(\mu_Y)$
defined by setting $T(f^{\ssbullet})=(f\restrp Y)^{\ssbullet}$ for every $f\in\eusm L^0(\mu)$, which is linear and multiplicative and preserves finite suprema and infima, so that (in particular)
$T(|u|)=|Tu|$ for every $u\in L^0(\mu)$.   (iii) Show that $T$ is injective iff $Y$ has full outer measure.
%241H

\spheader 241Yh Suppose, in 241Yg, that $Y\in\Sigma$.
Explain how $L^0(\mu_Y)$ may be identified (as ordered
linear space) with the subspace
$\{u:u\times\chi(X\setminus Y)^{\ssbullet}=0\}$ of $L^0(\mu)$.
%241Yg 241H

\spheader 241Yi Let $(X,\Sigma,\mu)$ be a measure space, and
$h:\Bbb R\to\Bbb R$ a non-decreasing function which is continuous on the
left.   Show that if $A\subseteq L^0=L^0(\mu)$ is a non-empty set with a
supremum $v\in L^0$, then $\bar h(v)=\sup_{u\in A}\bar h(u)$, where
$\bar h:L^0\to L^0$ is the function described in 241I.
%241I
}%end of exercises

\endnotes{
\Notesheader{241} As hinted in 241Ya and 241Yd, the elementary properties of
the space $L^0$ which take up most of this section are strongly
interdependent;  it is not difficult to develop a theory of `Riesz
algebras' to incorporate the ideas of 241H into the
rest.   (Indeed, I sketch such a theory in \S352 in the next volume, under
the name `$f$-algebra'.)

If we write $\eusm L^0_{\Sigma}$ for the space of measurable
functions from $X$ to $\Bbb R$, then $\eusm L^0_{\Sigma}$ is
also a Dedekind
$\sigma$-complete Riesz space, and $L^0$ can be identified with the
quotient $\eusm L^0_{\Sigma}/\eusm N$, writing $\eusm N$ for the
set of functions in $\eusm L^0_{\Sigma}$ which are zero almost
everywhere.   (To do this properly, we need a theory of
quotients of ordered linear spaces;  see 241Yb-241Yc above.)   Of course
$\eusm L^0$, as I define it, is not quite a linear space.   I choose the
slightly more awkward description of $L^0$ as a space of equivalence
classes in $\eusm L^0$ rather than in $\eusm L^0_{\Sigma}$
because it frequently happens in practice that a member of $L^0$ arises
from a
member of $\eusm L^0$ which is either not defined at every point of the
underlying space, or not quite measurable;  and to adjust such a
function so that it becomes a member of $\eusm L^0_{\Sigma}$,
while trivial, is an arbitrary process which to my mind is liable to
distort the true nature of such a construction.   Of course the same
argument could be used in favour of a slightly larger space, the space
$\eusm L^0_{\infty}$ of $\mu$-virtually measurable
$[-\infty,\infty]$-valued functions defined and finite almost
everywhere, relying on 135E rather than on 121E-121F.   But I
maintain that the operation of restricting a function in
$\eusm L^0_{\infty}$ to the set on which it is finite is {\it not}
arbitrary, but canonical and entirely natural.

Reading the exposition above -- or, for that matter, scanning the rest
of this chapter -- you are sure to notice a plethora of $^{\ssbullet}$s,
adding a distinctive character to the pages which, I expect you will
feel, is disagreeable to the eye and daunting, or at any rate wearisome,
to the spirit.   Many, perhaps most, authors prefer to simplify the
typography by using the same symbol for a function in $\eusm L^0$ or
$\eusm L^0_{\Sigma}$ and for its equivalence class in $L^0$;
and indeed it is common to use syntax which does not distinguish between
them either, so that an object which has been defined as a
member of $L^0$ will suddenly become a function with actual values at
points of the underlying measure space.   I prefer to maintain a rigid
distinction;   you must choose for yourself whether to follow me.
Since I have chosen the more cumbersome form, I suppose the burden of
proof is on me, to justify my decision.   (i) Anyone would agree that
there is at least a formal difference between a function and a set of
functions.   This by itself does not justify insisting on the difference
in every sentence;  mathematical exposition would be impossible if we
always insisted on consistency in such questions as whether (for
instance) the number $3$ belonging to the set $\Bbb N$ of natural
numbers is exactly the same object as the number $3$ belonging to the
set $\Bbb C$ of complex numbers, or the ordinal $3$.   But the
difference between an object and a set to which it belongs is a
sufficient difference in kind to make any confusion extremely dangerous,
and while I agree that you can study this topic without using different
symbols for $f$ and $f^{\ssbullet}$, I do not think you can ever safely
escape a mental distinction for more than a few lines of argument.
(ii) As a teacher, I have to say that quite a few students, encountering
this material for the first time, are misled by any failure to make the
distinction between $f$ and $f^{\ssbullet}$ into believing that no
distinction need be made;   and -- as a teacher -- I always
insist on a student convincing me, by correctly writing out the more
pedantic forms of the arguments for a few weeks, that he understands the
manipulations necessary, before I allow him to go his own way.
(iii) The reason why it {\it is} possible to evade the distinction in
certain types of argument is just that the Dedekind
$\sigma$-complete Riesz space $\eusm L^0_{\Sigma}$ parallels the
Dedekind $\sigma$-complete Riesz space $L^0$ so closely that any
proposition involving only countably many members of these spaces is
likely to be valid in one if and only if it is valid in the other.   In
my view, the
implications of this correspondence are at the very heart of measure
theory.   I prefer therefore to keep it constantly conspicuous,
reminding myself through symbolism that every theorem has a Siamese
twin, and rising to each challenge to express the twin theorem in an
appropriate language.   (iv) There are ways in which 
$\eusm L^0_{\Sigma}$ and $L^0$ are
actually very different, and many interesting ideas can be expressed
only in a language which keeps them clearly separated.

For more than half my life now I have felt that these points between
them are
sufficient reason for being consistent in maintaining
the formal distinction between $f$ and $f^{\ssbullet}$.   You may feel
that in (iii) and (iv) of the last paragraph I am trying to have things
both ways;  I am arguing that both the similarities and the differences
between $L^0$ and $\eusm L^0$ support my case.   Indeed that is
exactly my position.   If they were totally different, using the same
language for both would not give rise to confusion;  if they were
essentially the same, it would not matter if we were sometimes unclear
which we were talking about.
}%end of notes

\discrpage

