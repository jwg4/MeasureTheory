\frfilename{mt311.tex} 
\versiondate{15.10.08} 
\copyrightdate{1996} 
 
\def\chaptername{Boolean algebras} 
\def\sectionname{Boolean algebras} 
 
\newsection{311} 
 
In this section I try to give a sufficient notion of the character of 
abstract Boolean algebras to make the calculations which will appear on 
almost every page of this volume seem both elementary and natural. 
The principal result is of course M.H.Stone's theorem:  every Boolean 
algebra can be expressed as an algebra of sets (311E).   So the section 
divides naturally into the first part, proving Stone's theorem, and the 
second, consisting of elementary consequences of the theorem and a 
little practice in using the insights it offers. 
 
\leader{311A}{Definitions (a)} A {\bf Boolean ring} is a ring 
$(\frak A,+,.)$ in which $a^2=a$ for every $a\in\frak A$. 
 
\header{311Ab}{\bf (b)} A {\bf Boolean algebra} is a Boolean ring 
$\frak A$ with a multiplicative identity $1=1_{\frak A}$;  I allow $1=0$ 
in this context. 
 
\cmmnt{\medskip 
 
\noindent{\bf Remark} For notes on those parts of the elementary theory 
of rings which we shall need, see \S3A2. 
 
I hope that the rather arbitrary use of the word 
`algebra' here will give no difficulties;  it gives me the 
freedom to insist that the ring $\{0\}$ should be accepted as a Boolean 
algebra. 
} 
 
\leader{311B}{Examples (a)} For any set $X$, $(\Cal PX,\symmdiff,\cap)$ 
is a Boolean algebra;  its zero is $\emptyset$ and its multiplicative 
identity is $X$.   \prooflet{\Prf\ We have to check the following, which 
are all easily established, using Venn diagrams or otherwise: 
 
\inset{$A\symmdiff B\subseteq X$ for all $A$, $B\subseteq X$,} 
 
\inset{$(A\symmdiff B)\symmdiff C=A\symmdiff(B\symmdiff C)$ for all $A$, 
$B$, $C\subseteq X$,} 
 
\noindent so that $(\Cal PX,\symmdiff)$ is a semigroup; 
 
\inset{$A\symmdiff\emptyset=\emptyset\symmdiff A=A$ for every 
$A\subseteq X$,} 
 
\noindent so that $\emptyset$ is the identity in $(\Cal PX,\symmdiff)$; 
 
\inset{$A\symmdiff A=\emptyset$ for every $A\subseteq X$,} 
 
\noindent so that every element of $\Cal PX$ is its own inverse in 
$(\Cal PX,\symmdiff)$, and $(\Cal PX,\symmdiff)$ is a group; 
 
\inset{$A\symmdiff B=B\symmdiff A$ for all $A$, $B\subseteq X$,} 
 
\noindent so that $(\Cal PX,\symmdiff)$ is an abelian group; 
 
\inset{$A\cap B\subseteq X$ for all $A$, $B\subseteq X$,} 
 
\inset{$(A\cap B)\cap C=A\cap(B\cap C)$ for all $A$, $B$, $C\subseteq 
X$,} 
 
\noindent so that $(\Cal PX,\cap)$ is a semigroup; 
 
\inset{$A\cap(B\symmdiff C)=(A\cap B)\symmdiff(A\cap C)$, $(A\symmdiff 
B)\cap C=(A\cap C)\symmdiff(B\cap C)$ for all $A$, $B$, $C\subseteq 
X$,} 
 
\noindent so that $(\Cal PX,\symmdiff,\cap)$ is a ring; 
 
\inset{$A\cap A=A$ for every $A\subseteq X$,} 
 
\noindent so that $(\Cal PX,\symmdiff,\cap)$ is a Boolean ring; 
 
\inset{$A\cap X=X\cap A=A$ for every $A\subseteq X$,} 
 
\noindent so that $(\Cal PX,\symmdiff,\cap)$ is a Boolean algebra and 
$X$ is its identity.\ \Qed} 
 
\header{311Bb}{\bf (b)} Recall that an `algebra of subsets of 
$X$'\cmmnt{ (136E)} is a family $\Sigma\subseteq\Cal PX$ such that 
$\emptyset\in\Sigma$, $X\setminus E\in\Sigma$ for every $E\in\Sigma$, 
and $E\cup F\in\Sigma$ for all $E$, $F\in\Sigma$.    In this case 
$(\Sigma,\symmdiff,\cap)$ is a Boolean algebra with zero $\emptyset$ and 
identity $X$.   \prooflet{\Prf\ If $E$, $F\in\Sigma$, then 
 
\Centerline{$E\cap F=X\setminus((X\setminus E)\cup(X\setminus 
F))\in\Sigma$,} 
 
\Centerline{$E\symmdiff F=(E\cap(X\setminus F))\cup(F\cap(X\setminus 
E))\in\Sigma$.} 
 
\noindent Because $\emptyset$ and $X=X\setminus\emptyset$ both belong to 
$\Sigma$, we can work through the identities in (a) above to see that 
$\Sigma$, like $\Cal PX$, is a Boolean algebra.\ \Qed} 
 
\header{311Bc}{\bf (c)} Consider the ring $\Bbb Z_2=\{0,1\}$, with its 
ring operations $+_2$, $\cdot$ given by setting 
 
\Centerline{$0+_20=1+_21=0$, 
\quad $0+_21=1+_20=1$,} 
 
\Centerline{$0\cdot 0=0\cdot 1=1\cdot 0=0$, 
\quad$1\cdot 1=1$.} 
 
\noindent\cmmnt{I leave it to you to check, if you have not seen it 
before, that this is a ring. }Because $0\cdot 0=0$ and $1\cdot 1=1$, it 
is a Boolean algebra. 
 
 
\leader{311C}{Proposition} Let $\frak A$ be a Boolean ring. 
 
(a) $a+a=0$\cmmnt{, that is, $a=-a$,} for every $a\in\frak A$. 
 
(b) $ab=ba$ for all $a$, $b\in\frak A$. 
 
\proof{{\bf (a)} If $a\in\frak A$, then 
 
\Centerline{$a+a=(a+a)(a+a)=a^2+a^2+a^2+a^2=a+a+a+a$,} 
 
\noindent so we must have $0=a+a$. 
 
\medskip 
 
{\bf (b)} Now for any $a$, $b\in\frak A$, 
 
\Centerline{$a+b=(a+b)(a+b)=a^2+ab+ba+b^2=a+ab+ba+b$,} 
 
\noindent so 
 
\Centerline{$0=ab+ba=ab+ab$} 
 
\noindent and $ab=ba$. 
}%end of proof of 311C 
 
\leader{311D}{Lemma} Let $\frak A$ be a Boolean ring, $I$ an ideal of 
$\frak A$\cmmnt{ (3A2E)}, and $a\in\frak A\setminus I$.   Then there 
is a ring homomorphism $\phi:\frak A\to\Bbb Z_2$ such that $\phi a=1$ 
and $\phi d=0$ for every $d\in I$. 
 
\proof{{\bf (a)} Let $\Cal I$ be the family of those ideals $J$ of 
$\frak A$ which include $I$ and do not contain $a$.   Then $\Cal I$ has 
a maximal element $K$ say.   \Prf\ Apply Zorn's lemma.  Since $I\in\Cal 
I$, $\Cal I\ne\emptyset$.   If $\Cal J$ is a non-empty totally ordered 
subset of $\Cal I$, then set $J^*=\bigcup\Cal J$.   If $b$, $c\in J^*$ 
and $d\in\frak A$, then there are $J_1$, $J_2\in\Cal J$ such that $b\in 
J_1$ and $c\in J_2$;  now $J=J_1\cup J_2$ is equal to one of $J_1$, 
$J_2$, so belongs to $\Cal J$, and $0$, $b+c$, $bd$ all belong to $J$, 
so all belong to $J^*$.   Thus $J^*\normalsubgroup\frak A$;  of course 
$I\subseteq J^*$ and $a\notin J^*$, so $J^*\in\Cal I$ and is an upper 
bound for $\Cal J$ in $\Cal I$.   As $\Cal J$ is arbitrary, the 
hypotheses of Zorn's lemma are satisfied and $\Cal I$ has a maximal 
element.\ \Qed 
 
\medskip 
 
{\bf (b)} For $b\in\frak A$ set $K_b=\{d:d\in\frak A,\,bd\in K\}$.   The 
following are easy to check: 
 
\quad{(i)} $K\subseteq K_b$ for every $b\in\frak A$, because $K$ is an 
ideal. 
 
 
\quad{(ii)} $K_b\normalsubgroup\frak A$ for every $b\in\frak A$.   \Prf\ 
$0\in K\subseteq K_b$.   If $d$, $d'\in K_b$ and $c\in\frak A$ then 
 
\Centerline{$b(d+d')=bd+bd'$,\quad $b(dc)=(bd)c$} 
 
\noindent belong to $K$, so $d+d'$, $dc\in K_b$.\ \Qed 
 
 
\quad{(iii)} If $b\in\frak A$ and $a\notin K_b$, then $K_b\in\Cal I$ so 
$K_b=K$. 
 
\quad{(iv)} Now $a^2=a\notin K$, so $a\notin K_a$ and $K_a=K$. 
 
\quad (v) If $b\in\frak A\setminus K$ then $b\notin K_a$, that is, 
$ba=ab\notin K$, and $a\notin K_b$;  consequently $K_b=K$. 
 
\quad (vi) If $b$, $c\in\frak A\setminus K$ then $c\notin K_b$ so 
$bc\notin K$. 
 
\quad (vii) If $b$, $c\in\frak A\setminus K$ then 
 
\Centerline{$bc(b+c)=b^2c+bc^2=bc+bc=0\in K$,} 
 
\noindent so $b+c\in K_{bc}$.   By (vi) and (v), $K_{bc}=K$ so $b+c\in 
K$. 
 
\medskip 
 
{\bf (c)} Now define $\phi:\frak A\to\Bbb Z_2$ by setting $\phi d=0$ if 
$d\in K$, $\phi d=1$ if $d\in\frak A\setminus K$. 
Then $\phi$ is a ring homomorphism.   \Prf\ 
 
\quad(i) If $b$, $c\in K$ then $b+c$, $bc\in K$ so 
 
\Centerline{$\phi(b+c)=0=\phi b+_2\phi c$, 
\quad$\phi(bc)=0=\phi b\,\phi c$.} 
 
\quad(ii) If $b\in K$, $c\in\frak A\setminus K$ then 
 
\Centerline{$c=(b+b)+c=b+(b+c)\notin K$} 
 
\noindent so $b+c\notin K$, while $bc\in K$, so 
 
\Centerline{$\phi(b+c)=1=\phi b+_2\phi c$, 
\quad$\phi(bc)=0=\phi b\,\phi c$.} 
 
\quad(iii) Similarly, 
 
\Centerline{$\phi(b+c)=1=\phi b+_2\phi c$, 
\quad$\phi(bc)=0=\phi b\,\phi c$} 
 
\noindent if $b\in\frak A\setminus K$ and $c\in K$. 
 
\quad(iv) If $b$, $c\in\frak A\setminus K$, then by 
(b-vi) and (b-vii) we have $b+c\in K$, $bc\notin K$ so 
 
\Centerline{$\phi(b+c)=0=\phi b+_2\phi c$, 
\quad$\phi(bc)=1=\phi b\,\phi c$.} 
 
\noindent Thus $\phi$ is a ring homomorphism.\ \Qed 
 
\medskip 
 
{\bf (d)} Finally, if $d\in I$ then $d\in K$ so $\phi d=0$;  and $\phi 
a=1$ because $a\notin K$. 
}%end of proof of 311D 
 
\leader{311E}{M.H.Stone's theorem:  first form} Let $\frak A$ 
be any Boolean ring, and let $Z$ be the set of ring homomorphisms from 
$\frak A$ onto $\Bbb Z_2$.   Then we have an injective ring homomorphism 
$a\mapsto\widehat{a}:\frak A\to\Cal PZ$, setting 
$\widehat{a}=\{z:z\in Z,\,z(a)=1\}$. 
If $\frak A$ is a Boolean algebra, then $\widehat{1}_{\frak A}=Z$. 
 
\proof{{\bf (a)} If $a$, $b\in\frak A$, then 
 
\Centerline{$\widehat{a\hbox{+}b}=\{z:z(a\hbox{+}b)=1\} 
=\{z:z(a)+_2z(b)=1\} 
=\{z:\{z(a),z(b)\}=\{0,1\}\} 
=\widehat{a}\symmdiff\widehat{b}$,} 
 
\Centerline{$\widehat{ab} 
=\{z:z(ab)=1\} 
=\{z:z(a)z(b)=1\} 
=\{z:z(a)=z(b)=1\} 
=\widehat{a}\cap \widehat{b}$.} 
 
\noindent Thus $a\mapsto\widehat{a}$ is a ring homomorphism. 
 
\medskip 
 
{\bf (b)} If $a\in\frak A$ and $a\ne 0$, then by 311D, with $I=\{0\}$, 
there is a $z\in Z$ such that $z(a)=1$, that is, $z\in\widehat{a}$;  so 
that $\widehat{a}\ne\emptyset$.   This shows that the kernel of 
$a\mapsto\widehat{a}$ is 
$\{0\}$, so that the homomorphism is injective (3A2Db). 
 
\medskip 
 
{\bf (c)} If $\frak A$ is a Boolean algebra, and $z\in Z$, then there is 
some $a\in\frak A$ such that $z(a)=1$, so that 
$z(1_{\frak A})z(a)=z(1_{\frak A}a)\ne 0$ and 
$z(1_{\frak A})\ne 0$;  thus $\widehat{1}_{\frak A}=Z$. 
}%end of proof of 311E 
 
\leader{311F}{Remarks\cmmnt{ (a)}} 
For any Boolean ring $\frak A$, I will say 
that the {\bf Stone space} of $\frak A$ is the set $Z$ of non-zero ring 
homomorphisms from $\frak A$ to $\Bbb Z_2$, and the canonical map 
$a\mapsto\widehat{a}:\frak A\to\Cal PZ$ is the {\bf Stone 
representation}. 
 
\cmmnt{ 
\header{311Fb}{\bf (b)} Because the map 
$a\mapsto\widehat{a}:\frak A\to\Cal PZ$ is an injective ring 
homomorphism, $\frak A$ is isomorphic, as 
Boolean ring, to its image $\Cal E=\{\widehat{a}:a\in\frak A\}$, which 
is a subring of $\Cal PZ$.   Thus the Boolean rings $\Cal PX$ of 311Ba 
are leading examples in a very strong sense. 
 
\header{311Fc}{\bf (c)} I have taken the set $Z$ of the Stone 
representation to be actually the set of homomorphisms from $\frak A$ 
onto $\Bbb Z_2$.   Of course we could equally well take any set which is 
in a natural one-to-one correspondence with $Z$;  a popular choice is 
the set of maximal ideals of $\frak A$, since a subset of $\frak A$ is a 
maximal ideal iff it is the kernel of a member of $Z$, which is then 
uniquely defined. 
}%end of comment 
 
\vleader{60pt}{311G}{The operations $\Bcup$, $\Bsetminus$, 
$\Bsymmdiff$ on a Boolean ring} 
Let $\frak A$ be a Boolean ring. 
 
\header{311Ga}{\bf (a)}\ifresultsonly{ Set}\else{ 
Using the Stone representation, we can see that the elementary 
operations $\cup$, $\cap$, $\setminus$, $\symmdiff$ of set theory all 
correspond to operations on $\frak A$.   If we set}\fi 
 
\Centerline{$a\Bcup b=a+b+ab$,\quad$a\Bcap b=ab$,\quad $a\Bsetminus 
b=a+ab$,\quad $a\Bsymmdiff b=a+b$} 
 
\noindent for $a$, $b\in \frak A$\ifresultsonly{.}\else{, then we see 
that 
 
\Centerline{$\widehat{a\closeBcup b} 
=\widehat{a}\symmdiff\widehat{b}\symmdiff(\widehat{a}\cap\widehat{b}) 
=\widehat{a}\cup\widehat{b}$,} 
 
\Centerline{$\widehat{a\closeBcap b}=\widehat{a}\cap\widehat{b}$,} 
 
\Centerline{$\widehat{a\closeBsetminus 
b}=\widehat{a}\setminus\widehat{b}$,} 
 
\Centerline{$\widehat{a\closeBsymmdiff 
b}=\widehat{a}\symmdiff\widehat{b}$.} 
 
\noindent Consequently all the familiar rules for manipulation of 
$\cap$, $\cup$, etc.\ will apply also to $\Bcap$, $\Bcup$, and we shall 
have, for instance, 
 
\Centerline{$a\Bcap(b\Bcup c)=(a\Bcap b)\Bcup(a\Bcap c)$, 
\quad $a\Bcup(b\Bcap c)=(a\Bcup b)\Bcap(a\Bcup c)$} 
 
\noindent for any members $a$, $b$, $c$ of any Boolean ring $\frak A$. 
}\fi 
 
\header{311Gb}{\bf (b)}\cmmnt{ Still importing terminology from 
elementary set theory,} I 
will say that a set $A\subseteq \frak A$ is {\bf disjoint} if 
$a\Bcap b=0$, that is, $ab=0$, for all distinct $a$, $b\in A$;  and that 
an indexed family $\langle a_i\rangle_{i\in I}$ in $\frak A$ is {\bf 
disjoint} if $a_i\Bcap a_j=0$ for all distinct $i$, $j\in I$. 
(\cmmnt{Just as 
I allow $\emptyset$ to be a member of a disjoint family of sets, }I 
allow $0\in A$ or $a_i=0$\cmmnt{ in the present context}.) 
 
\header{311Gc}{\bf (c)} A {\bf partition of unity} in $\frak A$ will be 
{\it either} a disjoint set $C\subseteq\frak A$ such that there is no 
non-zero $a\in\frak A$ such that $a\Bcap c=0$ for every $c\in C$ {\it 
or} a disjoint family $\langle c_i\rangle_{i\in I}$ in 
$\frak A$ such that there is no non-zero $a\in\frak A$ such that 
$a\Bcap c_i=0$ for every $i\in I$.   \cmmnt{(In the first case I allow 
$0\in C$, and in the second I allow $c_i=0$.)} 
 
\spheader 311Gd Note that a set $C\subseteq\frak A$ is a partition of 
unity iff $C\cup\{0\}$ is a maximal disjoint set.   \prooflet{\Prf\ 
If $C$ is a partition of unity and $a\in\frak A\setminus(C\cup\{0\})$, then 
there must be a $c\in C$ such that $a\Bcap c\ne 0$, so that $C\cup\{0,a\}$ 
is not disjoint;  thus $C\cup\{0\}$ is a maximal disjoint set.   If 
$C\cup\{0\}$ is a maximal disjoint set, and $a\in\frak A\setminus\{0\}$, 
then either $a\in C$ and $a\Bcap a\ne 0$, or $C\cup\{0,a\}$ is not 
disjoint, so there is a $c\in C$ such that $a\Bcap c\ne 0$;  thus $C$ is a 
partition of unity.\ \Qed} 
 
If $A\subseteq\frak A$ is any disjoint set, there is a partition of unity 
including $A$.   \prooflet{\Prf\ Apply Zorn's Lemma to  
$\{C:C$ is a disjoint set including $A\}$.\ \Qed} 
 
\spheader 311Ge If $C$ and $D$ are two partitions of unity, I say that 
$C$ {\bf refines} $D$ if for every $c\in C$ there is a $d\in D$ such 
that $cd=c$\cmmnt{ (that is, $c\Bsubseteq d$ in the language of 311H 
below)}.   Note that if $C$ refines $D$ and $D$ refines $E$ then $C$ 
refines $E$.   \prooflet{\Prf\ If $c\in C$, there is a $d\in D$ such 
that $cd=c$;  now there is an $e\in E$ such that $de=d$;  in this case, 
 
\Centerline{$ce=(cd)e=c(de)=cd=c$;} 
 
\noindent as $c$ is arbitrary, $C$ refines $E$.\ \Qed} 
 
\leader{311H}{The order structure of a Boolean ring} Again treating a 
Boolean ring $\frak A$ as an algebra of sets, it 
has a natural ordering, setting 
$a\Bsubseteq b$ if $ab=a$, so that $a\Bsubseteq b$ iff 
$\widehat{a}\subseteq\widehat{b}$.   This translation makes it obvious 
that $\Bsubseteq$ 
is a partial order on $\frak A$, with least element $0$, and with 
greatest element $1$ iff $\frak A$ is a Boolean algebra.   Moreover, 
$\frak A$ is a lattice\cmmnt{ (definition:  2A1Ad)}, with 
$a\Bcup b=\sup\{a,b\}$ and
$a\Bcap b=\inf\{a,b\}$ for all $a$, $b\in\frak A$.   Generally, for 
$a_0,\ldots,a_n\in\frak A$, 
 
\Centerline{$\sup_{i\le n}a_i=a_0\Bcup\ldots\Bcup a_n$, 
\quad$\inf_{i\le n}a_i=a_0\Bcap\ldots\Bcap a_n$;} 
 
\noindent suprema and infima of finite subsets of $\frak A$ correspond to 
unions and intersections of the corresponding families in the Stone 
space.   \cmmnt{(But suprema and infima of {\it infinite} subsets of 
$\frak A$ are a very different matter;  see \S313 below.) 
 
It may be obvious, but it is nevertheless vital to recognise that when 
$\frak A$ is a ring of sets then $\Bsubseteq$ agrees with $\subseteq$. 
} 
 
\leader{311I}{The topology of a Stone space:  Theorem} Let $Z$ be the 
Stone space of a Boolean ring $\frak A$, and let $\frak T$ be 
 
\Centerline{$\{G:G\subseteq Z$ and for every $z\in G$ there is an 
$a\in\frak A$ such that $z\in\widehat{a}\subseteq G\}$.} 
 
\noindent Then $\frak T$ is a topology on $Z$, under which $Z$ 
is a locally compact zero-dimensional Hausdorff space, and 
$\Cal E=\{\widehat{a}:a\in\frak A\}$ is precisely the set 
of compact open subsets of $Z$.   $\frak A$ is a Boolean algebra iff $Z$ 
is compact. 
 
\proof{{\bf (a)} Because $\Cal E$ is 
closed under $\cap$, and $\bigcup\Cal E=Z$ (recall that $Z$ is the set 
of surjective homomorphisms from $\frak A$ to $\Bbb Z_2$, so that every 
$z\in Z$ is somewhere non-zero and belongs to some $\widehat{a}$), 
$\Cal E$ is a topology base, and $\frak T$ is a topology. 
 
\medskip 
 
{\bf (b)} $\frak T$ is Hausdorff.   \Prf\ Take any distinct $z$, 
$w\in Z$.   Then there is an $a\in\frak A$ such that $z(a)\ne w(a)$; 
let us take it that $z(a)=1$, $w(a)=0$.   There is also a $b\in\frak A$ 
such that $w(b)=1$, so that $w(b+ab)=w(b)+_2w(a)w(b)=1$ and 
$w\in(b+ab)\sphat\,$;  also 
 
\Centerline{$a(b+ab)=ab+a^2b=ab+ab=0$,} 
 
\noindent so 
 
\Centerline{$\widehat{a}\cap(b+ab)\sphat\; 
=(a(b+ab))\sphat\;=\widehat{0}=\emptyset$,} 
 
\noindent and $\widehat{a}$, $(b+ab)\sphat\,$ are disjoint members of 
$\frak T$ containing $z$, $w$ respectively.\ \Qed 
 
\medskip 
 
{\bf (c)} If $a\in\frak A$ then $\widehat{a}$ is compact.   \Prf\ 
Let $\Cal F$ be an ultrafilter on $Z$ containing $\widehat{a}$.   For 
each $b\in\frak A$, $z_0(b)=\lim_{z\to\Cal F}z(b)$ must be defined in 
$\Bbb Z_2$, since one of the sets $\{z:z(b)=0\}$, $\{z:z(b)=1\}$ must 
belong to $\Cal F$.   If $b$, $c\in\frak A$, then the set 
 
\Centerline{$F=\{z:z(b)=z_0(b),\,z(c)=z_0(c),\,z(b+c)=z_0(b+c),\, 
z(bc)=z_0(bc)\}$} 
 
\noindent belongs to $\Cal F$, so is not empty;  take any $z_1\in F$; 
then 
 
\Centerline{$z_0(b+c)=z_1(b+c)=z_1(b)+_2z_1(c)=z_0(b)+_2z_0(c)$,} 
 
\Centerline{$z_0(bc)=z_1(bc)=z_1(b)z_1(c)=z_0(b)z_0(c)$.} 
 
\noindent As $b$, $c$ are arbitrary, $z_0:\frak A\to\Bbb Z_2$ is a 
ring homomorphism.  Also $z_0(a)=1$, because $\widehat{a}\in\Cal F$, 
so $z_0\in\widehat{a}$.   Now let $G$ be any open subset of $Z$ 
containing $z_0$;  then there is a $b\in\frak A$ such that 
$z_0\subseteq\widehat{b}\subseteq G$;  since 
$\lim_{z\to\Cal F}z(b)=z_0(b)=1$, we must have 
$\widehat{b}=\{z:z(b)=1\}\in\Cal F$ and $G\in\Cal F$.   Thus $\Cal F$ 
converges to $z_0$.   As $\Cal F$ is arbitrary, $\widehat{a}$ is 
compact (2A3R).\ \Qed 
 
\medskip 
 
{\bf (d)} This shows that $\widehat{a}$ is a compact open set for every 
$a\in\frak A$.   Moreover, since every point of $Z$ belongs to some 
$\widehat{a}$, every point of $Z$ has a compact neighbourhood, and $Z$ 
is locally compact. 
Every $\widehat{a}$ is closed (because it is compact, or otherwise), so 
$\Cal E$ is a base for $\frak T$ consisting of open-and-closed sets, and 
$\frak T$ is zero-dimensional. 
 
\medskip 
 
{\bf (e)} Now suppose that $E\subseteq Z$ is an open compact set.   If 
$E=\emptyset$ then $E=\widehat{0}$.   Otherwise, set 
 
\Centerline{$\Cal G 
=\{\widehat{a}:a\in\frak A,\,\widehat{a}\subseteq E\}$.} 
 
\noindent Then $\Cal G$ is a family of open subsets of $Z$ and 
$\bigcup\Cal G=E$, because $E$ is open.   But $E$ is also compact, so 
there is a finite $\Cal G_0\subseteq\Cal G$ such that 
$E=\bigcup\Cal G_0$.   Express $\Cal G_0$ as 
$\{\widehat{a}_0,\ldots,\widehat{a}_n\}$.   Then 
 
\Centerline{$E=\widehat{a}_0\cup\ldots\cup\widehat{a}_n 
=(a_0\Bcup\ldots\Bcup a_n)\sphat\,$.} 
 
\noindent This shows that every compact open subset of $Z$ is of the 
form $\widehat{a}$ for some $a\in\frak A$. 
 
\medskip 
 
{\bf (f)} Finally, if $\frak A$ is a Boolean algebra then 
$Z=\widehat{1}$ is compact, by (c);  while if $Z$ is compact then (e) 
tells us that $Z=\widehat{a}$ for some $a\in\frak A$, and of course 
this $a$ must be a multiplicative identity for $\frak A$, so that 
$\frak A$ is a Boolean algebra. 
}%end of proof of 311I 
 
\leader{311J}{}\cmmnt{ We have a kind of converse of Stone's theorem. 
 
\medskip 
 
\noindent}{\bf Proposition} Let $X$ be a locally compact 
zero-dimensional Hausdorff space.   Then the set $\frak A$ of 
open-and-compact subsets of $X$ is a subring of $\Cal PX$.   If $Z$ is 
the Stone space of $\frak A$, there is a unique homeomorphism 
$\theta:Z\to X$ such that $\widehat{a}=\theta^{-1}[a]$ for every 
$a\in\frak A$. 
 
\proof{{\bf (a)} Because $X$ is Hausdorff, all its compact sets are 
closed, so every member of $\frak A$ is closed.   Consequently 
$a\cup b$, $a\setminus b$, $a\cap b$ and $a\symmdiff b$ belong to 
$\frak A$ for all $a$, 
$b\in\frak A$, and $\frak A$ is a subring of $\Cal PX$. 
 
It will be helpful to know that $\frak A$ is a base for the topology of 
$X$.   \Prf\ If $G\subseteq X$ is open and $x\in G$, then (because $X$ 
is locally compact) there is a compact set $K\subseteq X$ such that 
$x\in\interior K$;  now (because $X$ is zero-dimensional) there is an 
open-and-closed set $a\subseteq X$ such that 
$x\in a\subseteq G\cap\interior K$;  because $a$ is a closed subset of a 
compact subset of $X$, it is compact, and belongs to $\frak A$, while 
$x\in a\subseteq G$.\ \Qed 
 
\medskip 
 
{\bf (b)} Let $R\subseteq Z\times X$ be the relation 
 
\Centerline{$\{(z,x)$: for every $a\in\frak A$, $x\in a\iff z(a)=1\}$.} 
 
\noindent Then $R$ is the graph of a bijective function $\theta:Z\to X$. 
 
\medskip 
 
\Prf\ {\bf (i)} If $z\in Z$ and $x$, $x'\in X$ are distinct, then, 
because $X$ is Hausdorff, there is an open set 
$G\subseteq X$ containing $x$ and not containing $x'$;  because 
$\frak A$ is a base for the topology of $X$, there is an $a\in\frak A$ 
such that $x\in a\subseteq G$, so that $x'\notin a$.   Now either 
$z(a)=1$ and $(z,x')\notin R$, or $z(a)=0$ and $(z,x)\notin R$.   Thus 
$R$ is the graph of a function $\theta$ with domain included in $Z$ and 
taking values in $X$. 
 
\medskip 
 
\quad{\bf (ii)} If $z\in Z$, 
there is an $a_0\in\frak A$ such that $z(a_0)=1$.    Consider 
$\Cal A=\{a:z(a)=1\}$.   This is a family of closed subsets of $X$ 
containing the compact set $a_0$, and $a\cap b\in\Cal A$ for all $a$, 
$b\in\Cal A$.   So $\bigcap\Cal A$ is not empty (3A3Db);  take 
$x\in\bigcap\Cal A$.   Then $x\in a$ whenever $z(a)=1$.   On the other 
hand, if $z(a)=0$, then 
 
\Centerline{$z(a_0\setminus a)=z(a_0\symmdiff(a\cap a_0)) 
=z(a_0)+_2z(a_0)z(a)=1$,} 
 
\noindent so $x\in a_0\setminus a$ and 
$x\notin a$.   Thus $(z,x)\in R$ and $\theta(z)=x$ is defined.   As $z$ 
is arbitrary, the domain of $\theta$ is the whole of $Z$. 
 
\medskip 
 
\quad{\bf (iii)} If $x\in X$, define $z:\frak A\to\Bbb Z_2$ by setting 
$z(a)=1$ if $x\in a$, $0$ otherwise. 
It is elementary to check that $z$ is a ring homomorphism 
form $\frak A$ to $\Bbb Z_2$.    To see that it takes the value $1$, 
note that because $\frak A$ is a base for the topology of $X$ there 
is an $a\in\frak A$ such that $x\in a$, so that $z(a)=1$. 
So $z\in Z$, and of course $(z,x)\in R$.   As $x$ is arbitrary, $\theta$ 
is surjective. 
 
\medskip 
 
\quad{\bf (iv)} If $z$, $z'\in Z$ and $\theta(z)=\theta(z')$, then, for 
any $a\in\frak A$, 
 
\Centerline{$z(a)=1 
\iff\theta(z)\in a 
\iff\theta(z')\in a 
\iff z'(a)=1$,} 
 
\noindent so $z=z'$.   Thus $\theta$ is injective.\ \Qed 
 
\medskip 
 
{\bf (c)} For any $a\in\frak A$, 
 
\Centerline{$\theta^{-1}[a] 
=\{z:\theta(z)\in a\}=\{z:z(a)=1\}=\widehat{a}$.} 
 
\noindent It follows that $\theta$ is a homeomorphism.   \Prf\ (i) If 
$G\subseteq X$ is open, then (because $\frak A$ is a base for the 
topology of $X$) $G=\bigcup\{a:a\in\frak A,\,a\subseteq G\}$ and 
 
\Centerline{$\theta^{-1}[G] 
=\bigcup\{\theta^{-1}[a]:a\in\frak A,\,a\subseteq G\} 
=\bigcup\{\widehat{a}:a\in\frak A,\,a\subseteq G\}$} 
 
\noindent is an open subset of $Z$.   As $G$ is arbitrary, $\theta$ is 
continuous.   (ii) On the other hand, if $G\subseteq X$ and 
$\theta^{-1}[G]$ is open, then $\theta^{-1}[G]$ is of the form 
$\bigcup_{a\in\Cal A}\widehat{a}$ for some $\Cal A\subseteq\frak A$, so 
that $G=\bigcup\Cal A$ is an open set in $X$.   Accordingly $\theta$ is 
a homeomorphism.\ \Qed 
 
\medskip 
 
{\bf (d)} Finally, I must check the uniqueness of $\theta$. 
But of course if $\tilde\theta:Z\to X$ is any function such that 
$\tilde\theta^{-1}[a]=\widehat{a}$ for every $a\in\frak A$, then the 
graph of $\tilde\theta$ must be $R$, so $\tilde\theta=\theta$. 
}%end of proof of 311J 
 
\cmmnt{ 
\leader{311K}{Remark} Thus we have a correspondence between 
Boolean rings and zero-dimensional locally compact Hausdorff spaces 
which is (up to isomorphism, on the one hand, and homeomorphism, on the 
other) one-to-one.   Every property of Boolean rings which we study will 
necessarily correspond to some property of zero-dimensional locally 
compact Hausdorff spaces.} 
 
\leader{311L}{Complemented distributive \dvrocolon{lattices}}\cmmnt{ I 
have introduced Boolean algebras through the theory of rings;  this 
seems to be the quickest route to them from an ordinary undergraduate 
course in abstract algebra.   However there are alternative approaches, 
taking the order structure rather than the algebraic operations as 
fundamental, 
and for the sake of an application in Chapter 35 I give the details of 
one of these. 
 
\medskip 
 
\noindent}{\bf Proposition} Let $\frak A$ be a lattice such that 
 
\inset{(i) $(a\vee b)\wedge c=(a\wedge c)\vee(b\wedge c)$ 
for all $a$, $b$, $c\in\frak A$; 
 
(ii) there is a permutation $a\mapsto a':\frak A\to\frak A$ which is 
order-reversing, that is, $a\le b$ iff $b'\le a'$, and such that $a''=a$ 
for every $a$; 
 
(iii) $\frak A$ has a least element $0$ and $a\wedge a'=0$ for every 
$a\in\frak A$.} 
 
\noindent Then $\frak A$ has a Boolean algebra structure for which 
$a\Bsubseteq b$ iff $a\le b$. 
 
\proof{{\bf (a)} Write $1$ 
for $0'$;  if $a\in\frak A$, then $a'\ge 0$ so $a=a''\le 0'=1$, and $1$ 
is the greatest element of $\frak A$. 
 
If $a$, $b\in\frak A$ then, because $'$ is an order-reversing permutation, 
$a'\vee b'=(a\wedge b)'$.   \Prf\ For $c\in\frak A$, 
 
$$\eqalign{a'\vee b'\le c 
&\iff a'\le c\,\,\&\,\,b'\le c 
\iff c'\le a\,\,\&\,\,c'\le b\cr 
&\iff c'\le a\wedge b 
\iff (a\wedge b)'\le c. \text{ \Qed}\cr}$$ 
 
\noindent Similarly, $a'\wedge b'=(a\vee b)'$.   If $a$, $b$, 
$c\in\frak A$ then 
 
\Centerline{$(a\wedge b)\vee c=((a'\vee b')\wedge c')' 
=((a'\wedge c')\vee(b'\wedge c'))' 
=(a\vee c)\wedge(b\vee c)$.} 
 
\medskip 
 
{\bf (b)} Define addition and multiplication on $\frak A$ by setting 
 
\Centerline{$a+b=(a\wedge b')\vee(a'\wedge b)$, 
\quad $ab=a\wedge b$} 
 
\noindent for $a$, $b\in \frak A$. 
 
\wheader{311L}{6}{2}{2}{36pt}
 
{\bf (c)(i)} If $a$, $b\in\frak A$ then 
 
$$\eqalign{(a+b)' 
&=(a'\vee b)\wedge(a\vee b') 
=(a'\wedge a)\vee(a'\wedge b')\vee(b\wedge a)\vee(b\wedge b')\cr 
&=0\vee(a'\wedge b')\vee(b\wedge a) 
=(a'\wedge b')\vee(a\wedge b).\cr}$$ 
 
\noindent So if $a$, $b$, $c\in\frak A$ then 
 
$$\eqalign{(a+b)+c 
&=((a+b)\wedge c')\vee((a+b)'\wedge c)\cr 
&=(((a\wedge b')\vee(a'\wedge b))\wedge c') 
   \vee(((a'\wedge b')\vee(a\wedge b))\wedge c)\cr 
&=(a\wedge b'\wedge c')\vee(a'\wedge b\wedge c') 
   \vee(a'\wedge b'\wedge c)\vee(a\wedge b\wedge c);\cr}$$ 
 
\noindent as this last formula is symmetric in $a$, $b$ and $c$, it is 
also equal to $a+(b+c)$.   Thus addition is associative. 
 
\medskip 
 
\quad{\bf (ii)} For any $a\in\frak A$, 
 
\Centerline{$a+0=0+a=(a'\wedge 0)\vee(a\wedge 0')=0\vee(a\wedge 1)=a$,} 
 
\noindent so $0$ is the additive identity of $\frak A$.   Also 
 
\Centerline{$a+a=(a\wedge a')\vee(a'\wedge a)=0\vee 0=0$} 
 
\noindent so each element of $\frak A$ is its own additive inverse, and 
$(\frak A,+)$ is a group.   It is abelian because $\vee$ and $\wedge$ 
are commutative. 
 
\medskip 
 
{\bf (d)} Because $\wedge$ is associative and commutative, 
$(\frak A,\cdot)$ 
is a commutative semigroup;  also $1$ is its identity, because 
$a\wedge 1=a$ for every $a\in\frak A$.   As for the distributive law in 
$\frak A$, 
 
 
$$\eqalign{ab+ac 
&=(a\wedge b\wedge(a\wedge c)')\vee((a\wedge b)'\wedge a\wedge c)\cr 
&=(a\wedge b\wedge(a'\vee c'))\vee((a'\vee b')\wedge a\wedge c)\cr 
&=(a\wedge b\wedge a')\vee(a\wedge b\wedge c') 
   \vee(a'\wedge a\wedge c)\vee(b'\wedge a\wedge c)\cr 
&=(a\wedge b\wedge c') 
   \vee(b'\wedge a\wedge c)\cr 
&=a\wedge((b\wedge c')\vee(b'\wedge c)) 
=a(b+c)\cr}$$ 
 
\noindent for all $a$, $b$, $c\in\frak A$.  Thus $(\frak A,+,\cdot)$ is 
a ring;  because $a\wedge a=a$ for every $a$, it is a Boolean ring. 
 
\medskip 
 
{\bf (e)} For $a$, $b\in\frak A$, 
 
\Centerline{$a\Bsubseteq b\iff ab=a\iff a\wedge b=a\iff a\le b$,} 
 
\noindent so the order relations of $\frak A$ coincide. 
}%end of proof of 311L 
 
\cmmnt{\medskip 
 
\noindent{\bf Remark} It is the case that the Boolean algebra structure 
of $\frak A$ is uniquely determined by its order structure, but I delay 
the proof to the next section (312M). 
}%end of comment 
 
 
\exercises{ 
\leader{311X}{Basic exercises (a)} Let $A_0,\ldots,A_n$ be sets. 
Show that 
 
\Centerline{$A_0\symmdiff\ldots\symmdiff A_n 
=\{x:\#(\{i:i\le n,\,x\in A_i\})$ is odd$\}$.} 
%311B 
 
\header{311Xb}{\bf (b)} Let $X$ be a set, and $\Sigma\subseteq\Cal PX$. 
Show that the following are equiveridical: 
(i) $\Sigma$ is an algebra of subsets of $X$; 
(ii) $\Sigma$ is a subring of $\Cal PX$ (that is, contains 
$\emptyset$ and is closed under $\symmdiff$ and $\cap$) and contains 
$X$; 
(iii) $\emptyset\in\Sigma$, $X\setminus E\in\Sigma$ for every 
$E\in\Sigma$, and $E\cap F\in\Sigma$ for all $E$, $F\in\Sigma$. 
%311B 
 
\spheader 311Xc Let $\frak A$ be any Boolean ring.   Let 
$a\mapsto a'$ be any bijection between $\frak A$ and a set $B$ disjoint 
from $\frak A$.   Set $\frak B=\frak A\cup B$, and extend the addition 
and multiplication of $\frak A$ to form binary operations on $\frak B$ 
by using the formulae 
 
\Centerline{$a+b'=a'+b=(a+b)'$,\quad $a'+b'=a+b$,} 
 
\Centerline{$a'b=b+ab$,\quad $ab'=a+ab$, 
$a'b'=(a+b+ab)'$.} 
 
\noindent Show that $\frak B$ is a Boolean algebra and that $\frak A$ is 
an ideal in $\frak B$. 
%311C 
 
\sqheader 311Xd Let $\frak A$ be a Boolean ring, and $K$ a 
finite subset of $\frak A$.   Show that the subring of $\frak A$ 
generated by $K$ has 
at most $2^{2^{\#(K)}-1}$ members.   \Hint{count its minimal non-zero
elements.} 
%311C 
 
\sqheader 311Xe Show that any finite Boolean ring is isomorphic to 
$\Cal PX$ for some finite set $X$ (and, in particular, is a Boolean 
algebra). 
%311Xd, 311C 
 
\spheader 311Xf Let $\frak A$ be any Boolean ring.   Show that 
 
\Centerline{$a\Bcup(b\Bcap c)=(a\Bcup b)\Bcap(a\Bcup c)$,\quad 
$a\Bcap(b\Bcup c)=(a\Bcap b)\Bcup(a\Bcap c)$} 
 
\noindent  for all $a$, $b$, $c\in\frak A$ directly from the definitions 
in 311G, without using Stone's theorem. 
%311G 
 
\sqheader 311Xg Let $\frak A$ be any Boolean ring.   Show that 
if we regard the Stone space $Z$ of $\frak A$ as a subset of 
$\{0,1\}^{\frak A}$, then the topology of $Z$ (311I) is just the 
subspace topology induced by the ordinary product topology of 
$\{0,1\}^{\frak A}$. 
%311I 
 
\spheader 311Xh Let $I$ be any set, and set $X=\{0,1\}^I$ with its usual 
topology (3A3K).   Show that for a subset $E$ of $X$ the following are 
equiveridical:  (i) $E$ is open-and-compact;  (ii) $E$ is determined by 
coordinates in a finite subset of $I$ (definition:  254M);  (iii) $E$ 
belongs to the algebra of subsets of $X$ generated by $\{E_i:i\in I\}$, 
where $E_i=\{x:x(i)=1\}$ for each $i$. 
%311J 
 
\spheader 311Xi Let $(\frak A,\le)$ be a lattice such that ($\alpha$) 
$\frak A$ has a least element $0$ and a greatest element $1$ ($\beta$) 
for every $a$, $b$, $c\in\frak A$, $a\vee(b\wedge c)=(a\vee 
b)\wedge(a\vee c)$ and $a\wedge(b\vee c)=(a\wedge b)\vee(a\wedge c)$ 
($\gamma$) for every $a\in\frak A$ there is an $a'\in\frak A$ such that 
$a\vee a'=1$ and $a\wedge a'=0$.   Show that there is a Boolean algebra 
structure on $\frak A$ for which $\le$ agrees with $\Bsubseteq$. 
%311L 
 
\leader{311Y}{Further exercises (a)} Let $\frak A$ be a Boolean ring, 
and $\frak B$ the Boolean algebra constructed by the method of 311Xc. 
Show that the Stone space of $\frak B$ can be identified with the 
one-point compactification (3A3O) of the Stone space of $\frak A$. 
%311C, 311Xc 
 
\spheader 311Yb Let $(\frak A,\vee,\wedge,0,1)$ be such that (i) 
$(\frak A,\vee)$ is a commutative semigroup with identity $0$ (ii) 
$(\frak A,\wedge)$ is a commutative semigroup with identity $1$ (iii) 
$a\wedge(b\vee c)=(a\wedge b)\vee(a\wedge c)$, 
$a\vee(b\wedge c)=(a\vee b)\wedge(a\vee c)$ for all $a$, $b$, 
$c\in\frak A$ (iv) $a\vee a=a\wedge a=a$ for every $a\in\frak A$ 
(v) for every $a\in\frak A$ there is an 
$a'\in\frak A$ such that $a\vee a'=1$ and $a\wedge a'=0$.   Show that there 
is a Boolean algebra structure on $\frak A$ for which $\vee=\Bcup$, 
$\wedge=\Bcap$. 

\spheader 311Yc Let $(\frak A,\vee,')$ be such that (i) 
$(\frak A,\vee)$ is a 
non-empty commutative semigroup (ii) $':\frak A\to\frak A$ is
a function (iii) $((a\vee b)'\vee(a\vee b')')'=a$ for all $a$,
$b\in\frak A$.
Show that there is a Boolean algebra structure on $\frak A$ for which
$\vee=\Bcup$ and $'$ is complementation.
\Hint{{\smc McCune 97}.}

\spheader 311Yd\dvAnew{2010} Let $P$ be a distributive lattice, and $Z$ the
set of surjective lattice homomorphisms from $P$ to $\{0,1\}$.   Show that
there is a sublattice of $\Cal PZ$ isomorphic to $P$.
%311I mt31bits out of order query
}%end of exercises 
 
\endnotes{ 
\Notesheader{311} My aim in this section has been to get as quickly as 
possible to Stone's theorem, since this is surely the best route to a 
picture of general Boolean algebras;  they are isomorphic to algebras of 
sets.   This means that all their elementary algebraic 
properties -- indeed, 
all their first-order properties -- can be effectively studied 
in the context of elementary set theory.   In 311G-311H I describe a few 
of the ways in which the Stone representation suggests algebraic 
properties of Boolean algebras. 
 
You should not, however, come too readily to the conclusion that Boolean 
algebras will never be able to surprise you.   In this book, in 
particular, we shall need to work a good deal with suprema and infima of 
infinite sets in Boolean algebras, for the ordering of 311H;  and even 
though this corresponds to the ordering $\subseteq$ of ordinary sets, we 
find that $(\sup A)\sphat\,$ is sufficiently different from 
$\bigcup_{a\in A}\widehat{a}$ to need new kinds of intuition.   (The 
point is that 
$\bigcup_{a\in A}\widehat{a}$ is an open set in the Stone space, but 
need not 
be compact if $A$ is infinite, so may well be smaller than 
$(\sup A)\sphat\,$, even when $\sup A$ is defined in $\frak A$.)   There 
is also 
the fact that Stone's theorem depends crucially on a fairly strong form 
of the axiom of choice (employed through Zorn's Lemma in the argument of 
311D).   Of course I shall be using the axiom of choice without scruple 
throughout this volume.   But it should be clear that such results as 
312B-312C in the next section cannot possibly need the axiom of choice 
for their proofs, and that 
to use Stone's theorem in such a context is slightly misleading. 
 
Nevertheless, it is so useful to be able to regard a Boolean algebra as 
an algebra of sets -- especially when dealing with only finitely many 
elements of the algebra at a time -- that henceforth I will almost 
always use the symbols $\Bsymmdiff$, $\Bcap$ for the addition and 
multiplication of a Boolean ring, and will use $\Bcup$, $\Bsetminus$, 
$\Bsubseteq$ without further comment, just as if I were considering 
$\cup$, $\setminus$ and $\subseteq$ in the Stone space.   (In 311Gb I 
have given a definition of `disjointness' in a Boolean algebra based 
on the same idea.)   Even without 
the axiom of choice this approach can be justified, once we have 
observed that finitely-generated Boolean algebras are finite 
(311Xd), since relatively elementary methods show that any finite 
Boolean algebra is isomorphic to $\Cal PX$ for some finite set $X$. 
 
I have taken a Boolean algebra to be a particular kind of commutative 
ring with identity.   Of course there are other approaches.   If we wish 
to think of the order relation as primary, then 311L and 311Xi are 
reasonably natural.   Other descriptions can be based on a list of the 
properties of the binary operations $\Bcup$, $\Bcap$ and the 
complementation operation $a\mapsto a'=1\Bsetminus a$, as in 311Yb.   (The
hardest I know of is in 311Yc.)   I 
give extra space to 311L only because this is well adapted to 
an application in 352Q below. 
}%end of notes 
 
\discrpage 
 
