\frfilename{mt257.tex} 
\versiondate{14.8.13} 
\copyrightdate{1995} 
 
\def\chaptername{Product measures} 
\def\sectionname{Convolutions of measures} 
 
\newsection{257} 
 
The ideas of this chapter can be brought together in a satisfying way in 
the theory of convolutions of Radon measures, 
which will be useful in \S272 and again in 
\S285.   I give just the definition (257A) and the central property 
(257B) of the convolution of totally finite Radon measures, with a few 
corollaries and a note on the relation between convolution of functions 
and convolution of measures (257F). 
 
\leader{257A}{Definition}  Let $r\ge 1$ be an 
integer and $\nu_1$, $\nu_2$ two totally finite Radon measures on 
$\Bbb R^r$.   Let $\lambda$ be the product measure on 
$\BbbR^r\times\BbbR^r$\cmmnt{;   then $\lambda$ also is a (totally 
finite) Radon measure, by 256K}.   Define 
$\phi:\BbbR^r\times\BbbR^r\to\BbbR^r$ by setting 
$\phi(x,y)=x+y$\cmmnt{;  then 
$\phi$ is continuous, therefore measurable in the sense of 
256G}.   The {\bf convolution} 
of $\nu_1$ and $\nu_2$, $\nu_1*\nu_2$, is the image measure 
$\lambda\phi^{-1}$;  \cmmnt{by 256G,} this is a Radon measure. 
 
Note that if $\nu_1$ and $\nu_2$ are Radon probability measures, then 
$\lambda$ and $\nu_1*\nu_2$ are also probability measures. 
 
\leader{257B}{Theorem} Let $r\ge 1$ be an integer, and $\nu_1$ and 
$\nu_2$ two totally finite Radon measures on $\BbbR^r$;  let 
$\nu=\nu_1*\nu_2$ be their convolution, and $\lambda$ their product on 
$\BbbR^r\times\BbbR^r$.   Then for any real-valued function $h$ defined 
on a subset of $\BbbR^r$, 
 
\Centerline{$\int h(x+y)\lambda(d(x,y))$ exists $=\int h(x)\nu(dx)$} 
 
\noindent if either integral is defined in $[-\infty,\infty]$. 
%will this work for $[-\infty,\infty]$-valued h?   
 
\proof{ Apply 235J with $J(x,y)=1$, $\phi(x,y)=x+y$ for all $x$, 
$y\in\BbbR^r$. 
}%end of proof of 257B 
 
\leader{257C}{Corollary} Let $r\ge 1$ be an integer, and $\nu_1$, 
$\nu_2$ two totally finite Radon measures on $\BbbR^r$;  let 
$\nu=\nu_1*\nu_2$ be their convolution, and $\lambda$ their product on 
$\BbbR^r\times\BbbR^r$;  write $\Lambda$ for the domain of $\lambda$. 
Let $h$ be a $\Lambda$-measurable function 
defined $\lambda$-almost everywhere in $\BbbR^r$.     Suppose that any 
one of the integrals 
 
\Centerline{$\iint|h(x+y)|\nu_1(dx)\nu_2(dy)$, 
\quad$\iint|h(x+y)|\nu_2(dy)\nu_1(dx)$, 
\quad$\int h(x+y)\lambda(d(x,y))$} 
 
\noindent exists and is finite.   Then $h$ is $\nu$-integrable and 
 
\Centerline{$\int h(x)\nu(dx)=\iint h(x+y)\nu_1(dx)\nu_2(dy) 
=\iint h(x+y)\nu_2(dy)\nu_1(dx)$.} 
 
\proof{ Put 257B together with Fubini's and Tonelli's theorems (252H). 
}%end of proof of 257C 
 
\vleader{48pt}{257D}{Corollary} If $\nu_1$ and $\nu_2$ are totally finite Radon 
measures on $\BbbR^r$, then $\nu_1*\nu_2=\nu_2*\nu_1$. 
 
\proof{ For any Borel set $E\subseteq\BbbR^r$, apply 257C to $h=\chi E$ 
to see that 
 
$$\eqalign{(\nu_1*\nu_2)(E) 
&=\iint\chi E(x+y)\nu_1(dx)\nu_2(dy) 
=\iint\chi E(x+y)\nu_2(dy)\nu_1(dx)\cr 
&=\iint\chi E(y+x)\nu_2(dy)\nu_1(dx) 
=(\nu_2*\nu_1)(E).\cr}$$ 
 
\noindent Thus $\nu_1*\nu_2$ and $\nu_2*\nu_1$ agree on the Borel sets 
of $\BbbR^r$;  because they are both Radon measures, they must be 
identical (256D). 
}%end of proof of 257D 
 
\leader{257E}{Corollary} If $\nu_1$, $\nu_2$ and $\nu_3$ are totally 
finite Radon measures on $\BbbR^r$, then 
$(\nu_1*\nu_2)*\nu_3=\nu_1*(\nu_2*\nu_3)$. 
 
\proof{ For any Borel set $E\subseteq\BbbR^r$, apply 257B to $h=\chi E$ 
to see that 
 
$$\eqalignno{((\nu_1*\nu_2)*\nu_3)(E) 
&=\iint\chi E(x+z)(\nu_1*\nu_2)(dx)\nu_3(dz)\cr 
&=\iiint\chi E(x+y+z)\nu_1(dx)\nu_2(dy)\nu_3(dz)\cr 
\noalign{\noindent (because $x\mapsto \chi E(x+z)$ is Borel measurable 
for every $z$)} 
&=\iint\chi E(x+y)\nu_1(dx)(\nu_2*\nu_3)(dy)\cr 
\noalign{\noindent (because $(x,y)\mapsto\chi E(x+y)$ is 
Borel measurable, so $y\mapsto\int\chi E(x+y)\nu_1(dx)$ is 
$(\nu_2*\nu_3)$-integrable)} 
&=(\nu_1*(\nu_2*\nu_3))(E).\cr}$$ 
 
\noindent Thus $(\nu_1*\nu_2)*\nu_3$ and $\nu_1*(\nu_2*\nu_3)$ agree on 
the Borel sets of $\BbbR^r$;  because they are both Radon measures, 
they must be identical. 
}%end of proof of 257E 
 
\leader{257F}{Theorem} Suppose that $\nu_1$ and $\nu_2$ are totally 
finite Radon measures on $\BbbR^r$ which are indefinite-integral 
measures over Lebesgue measure $\mu$.   Then $\nu_1*\nu_2$ also is an 
indefinite-integral measure over $\mu$;  if $f_1$ and $f_2$ are 
Radon-Nikod\'ym derivatives of $\nu_1$, $\nu_2$ respectively, then 
$f_1*f_2$ is a Radon-Nikod\'ym derivative of $\nu_1*\nu_2$. 
 
\proof{ By 255H/255L, $f_1*f_2$ is integrable with 
respect to $\mu$, 
with $\int f_1*f_2d\mu=1$, and of course $f_1*f_2$ is non-negative.   If 
$E\subseteq\BbbR^r$ is a Borel set, 
 
$$\eqalignno{\int_Ef_1*f_2d\mu 
&=\iint\chi E(x+y)f_1(x)f_2(y)\mu(dx)\mu(dy)\cr 
\noalign{\noindent (255G)} 
&=\iint\chi E(x+y)f_2(y)\nu_1(dx)\mu(dy)\cr 
\noalign{\noindent (because $x\mapsto\chi E(x+y)$ is Borel measurable)} 
&=\iint\chi E(x+y)\nu_1(dx)\nu_2(dy)\cr 
\noalign{\noindent (because $(x,y)\mapsto\chi E(x+y)$ is 
Borel measurable, so $y\mapsto\int\chi E(x+y)\nu_1(dx)$ 
is $\nu_2$-integrable)} 
&=(\nu_1*\nu_2)(E).\cr}$$ 
 
\noindent So $f_1*f_2$ is a Radon-Nikod\'ym derivative of $\nu$ with 
respect to $\mu$, by 256J. 
}%end of proof of 257F 
 
\exercises{\leader{257X}{Basic exercises $\pmb{>}$(a)} 
%\spheader 257Xa 
Let $r\ge 1$ be an integer.   Let $\delta_0$ be the 
Dirac measure on $\BbbR^r$ 
concentrated at $0$.   Show that $\delta_0$ is a Radon probability 
measure on $\BbbR^r$ and that 
$\delta_0*\nu=\nu$ for every totally finite Radon measure on $\BbbR^r$. 
%257A 
 
\spheader 257Xb Let $\mu$ and $\nu$ be totally finite Radon measures on 
$\BbbR^r$, and $E$ any set measured by their convolution $\mu*\nu$. 
Show that $\int\mu(E-y)\nu(dy)$ is defined 
in $[0,\infty]$ and equal to $(\mu*\nu)(E)$. 
%257B 
 
\spheader 257Xc Let $\nu_1,\ldots,\nu_n$ be totally finite Radon 
measures on $\BbbR^r$, and let $\nu$ be the convolution 
$\nu_1*\ldots*\nu_n$ (using 257E to see that such a bracketless 
expression is legitimate).   Show that 
 
\Centerline{$\int h(x)\nu(dx)=\int\ldots\int 
h(x_1+\ldots+x_n)\nu_1(dx_1)\ldots\nu_n(dx_n)$} 
 
\noindent for every $\nu$-integrable function $h$. 
%257E 
 
\spheader 257Xd Let $\nu_1$ and $\nu_2$ be totally finite Radon measures 
on $\BbbR^r$, with supports $F_1$, $F_2$ (256Xf).   Show that the 
support of $\nu_1*\nu_2$ is $\overline{\{x+y:x\in F_1,\,y\in F_2\}}$. 
%257E 
 
\sqheader 257Xe Let $\nu_1$ and $\nu_2$ be totally finite Radon measures 
on $\BbbR^r$, and suppose that $\nu_1$ has a Radon-Nikod\'ym derivative 
$f$ with respect 
to Lebesgue measure $\mu$.   Show that $\nu_1*\nu_2$ has a 
Radon-Nikod\'ym derivative $g$, 
where $g(x)=\int f(x-y)\nu_2(dy)$ for $\mu$-almost every $x\in\BbbR^r$. 
%257F 
 
\spheader 257Xf Suppose that $\nu_1$, $\nu_2$, $\nu_1'$ and $\nu_2'$ are 
totally finite Radon measures on $\BbbR^r$, and that $\nu_1'$, $\nu_2'$ 
are absolutely continuous with respect to $\nu_1$, $\nu_2$ respectively.   Show that 
$\nu_1'*\nu_2'$ is absolutely continuous with respect to $\nu_1*\nu_2$. 
%257F 
 
\leader{257Y}{Further exercises (a)} 
Let $M$ be the space of countably additive 
functionals defined on the algebra $\Cal B$ of Borel subsets of 
$\Bbb R$, with its norm $\|\nu\|=|\nu|(\Bbb R)$ (see 231Yh).   (i) Show 
that we have a unique bilinear operator $*:M\times M\to M$ such that 
$(\mu_1\restr\Cal B)*(\mu_2\restr\Cal B)=(\mu_1*\mu_2)\restr\Cal B$ for 
all totally finite Radon measures $\mu_1$, $\mu_2$ on $\Bbb R$.  (ii) 
Show that $*$ is commutative and associative.   (iii) Show that 
$\|\nu_1*\nu_2\|\le\|\nu_1\|\|\nu_2\|$ for all $\nu_1$, $\nu_2\in M$, so 
that $M$ is a Banach algebra under this multiplication.   (iv) Show that 
$M$ has a multiplicative identity.   (v) Show that $L^1(\mu)$ can be 
regarded as a closed subalgebra of $M$, where $\mu$ is Lebesgue measure 
on $\BbbR^r$ (cf.\ 255Xi). 
%257A 
 
\spheader 257Yb Let us say that a {\bf Radon 
measure on $\ocint{-\pi,\pi}$} is a 
complete measure $\nu$ on $\ocint{-\pi,\pi}$ 
such that (i) every Borel subset of 
$\ocint{-\pi,\pi}$ belongs to the domain $\Sigma$
of $\mu$ (ii) for every $E\in\Sigma$ there 
are Borel sets $E_1$, $E_2$ such that $E_1\subseteq E\subseteq E_2$ 
and $\nu(E_2\setminus E_1)=0$ (iii) every compact subset 
of $\ocint{-\pi,\pi}$ has finite measure.   Show that for any two 
totally finite Radon measures $\nu_1$, $\nu_2$ on $\ocint{-\pi,\pi}$ 
there is a unique totally finite Radon measure $\nu$ on 
$\ocint{-\pi,\pi}$ such that 
 
\Centerline{$\int h(x)\nu(dx)=\int h(x+_{2\pi}y)\nu_1(dx)\nu_2(dy)$} 
 
\noindent for every $\nu$-integrable function $h$, where $+_{2\pi}$ is 
defined as in 255Ma. 
}%end of exercises 
 
\endnotes{ 
\Notesheader{257} 
Of course convolution of functions and convolution of measures are very 
closely connected;  the obvious link being 257F, but the correspondence 
between 255G and 257B is also very marked.   In effect, they give us 
the same notion of convolution $u*v$ when $u$, $v$ are positive members 
of $L^1$ and $u*v$ is interpreted in $L^1$ rather than as a function 
(257Ya).   But 
we should have to go rather deeper than the arguments here to 
find ideas in the theory of convolution of measures to correspond to 
such results as 255K.   I will return to questions of this type in  
\S444 in Volume 4. 
 
All the theorems of this section can be extended to general abelian 
locally compact Hausdorff topological groups;  but for such generality 
we need much more advanced ideas (see \S444), and for the moment I leave 
only the 
suggestion in 257Yb that you should try to adapt the ideas here to 
$\ocint{-\pi,\pi}$ or $S^1$. 
}%end of comment 
 
\discrpage 
 
