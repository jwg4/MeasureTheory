\frfilename{mt42.tex}
\versiondate{30.9.08}
\copyrightdate{1998}

\def\chaptername{Descriptive set theory}
\def\sectionname{Introduction}

\newchapter{42}

At this point, I interpolate an auxiliary chapter, in the same spirit as
Chapters 31 and 35 in the last volume.   As with Boolean algebras and
Riesz spaces, it is not just that descriptive set theory provides
essential tools for modern measure theory;  it also offers deep
intuitions, and for this reason demands study well beyond an occasional
glance at an appendix.   Several excellent accounts have been published;
the closest to what we need here is probably {\smc Rogers 80}; at a
deeper level we have {\smc Moschovakis 80}, and an admirable recent
treatment is {\smc Kechris 95}.   Once again, however, I indulge myself
by extracting those parts of the theory which I shall use directly,
giving proofs and exercises adapted to the ideas I am trying to
emphasize in this volume and the next.

The first section describes Souslin's operation and its basic
set-theoretic properties up to first steps in the theory of `constituents'
(421N-421Q), %421N 421O 421P 421Q
mostly steering away from topological ideas,
but with some remarks on $\sigma$-algebras and Souslin-F sets.   \S422
deals with usco-compact relations and K-analytic spaces, working through
the topological properties which will be useful later, and giving a
version of the First Separation Theorem (422I-422J).   \S423 looks at
`analytic' or `Souslin' spaces, treating them primarily as a special
kind of K-analytic space, with the von Neumann-Jankow selection theorem
(423N).   \S424 is devoted to `standard Borel spaces';  it is largely a
series of easy applications of results in \S423, but there is a
substantial theorem on Borel measurable actions of Polish groups (424H).
Finally, I add a note on A.T\"ornquist's theorem on representation of
groups of automorphisms of quotient algebras (425D).

\discrpage


