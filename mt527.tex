\frfilename{mt527.tex}
\versiondate{10.1.10}
\copyrightdate{2002}

\def\chaptername{Cardinal functions of measure theory}
\def\sectionname{Skew products of ideals}

\def\CalMk{\Cal{M}\eurm{k}}
\def\CalSmz{\Cal{S}\eurm{mz}}
\def\rti{right-{\vthsp}translation-{\vthsp}invariant}

\newsection{527}

The methods of this chapter can be applied to a large proportion of the
partially ordered sets which arise in analysis.   In this section I
look at skew products of ideals,
constructed by a method suggested by Fubini's theorem and the
Kuratowski-Ulam theorem (527E).

\leader{527A}{Notation} If $(X,\Sigma,\mu)$ is a measure space,
$\Cal N(\mu)$ will be the null ideal of $\mu$;
$\Cal N$ will be the null ideal of Lebesgue measure on $\Bbb R$.
If $X$ is a topological space, $\Cal B(X)$ will be the Borel
$\sigma$-algebra of $X$ and $\Cal M(X)$ the $\sigma$-ideal of meager
subsets of $X$;  $\Cal M$ will be the ideal
$\Cal M(\Bbb R)$ of meager subsets of $\Bbb R$.

\vleader{48pt}{527B}{Skew products of ideals} Suppose that
$\Cal I\normalsubgroup\Cal PX$ and $\Cal J\normalsubgroup\Cal PY$ are
ideals of subsets of sets $X$, $Y$ respectively.

\spheader 527Ba I will write $\Cal I\ltimes\Cal J$  for their
{\bf skew product}
$\{W:W\subseteq X\times Y$, $\{x:W[\{x\}]\notin\Cal J\}\in\Cal I\}$.
\cmmnt{(This use of the symbol $\ltimes$ is unconnected with the usage
in \S512 except by the vaguest of analogies.)}
\cmmnt{It is easy to check that}
$\Cal I\ltimes\Cal J\normalsubgroup\Cal P(X\times Y)$.

\cmmnt{Similarly, }$\Cal I\rtimes\Cal J$ will be
$\{W:W\subseteq X\times Y$, $\{y:W^{-1}[\{y\}]\notin\Cal I\}\in\Cal J\}$.

\spheader 527Bb Suppose that $X$ and $Y$ are not empty and that $\Cal I$
and $\Cal J$ are proper ideals.   Then

\Centerline{$\add(\Cal I\ltimes\Cal J)=\min(\add\Cal I,\add\Cal J)$,
\quad$\cf(\Cal I\ltimes\Cal J)\ge\max(\cf\Cal I,\cf\Cal J)$,}

\Centerline{$\non(\Cal I\ltimes\Cal J)=\max(\non\Cal I,\non\Cal J)$,
\quad$\cov(\Cal I\ltimes\Cal J)=\min(\cov\Cal I,\cov\Cal J)$.}

\noindent\prooflet{\Prf\ {\bf (i)} If $\ofamily{\xi}{\kappa}{W_{\xi}}$
is a family in $\Cal I\ltimes\Cal J$ with $\kappa<\min(\add\Cal
I,\add\Cal J)$, set $W=\bigcup_{\xi<\kappa}W_{\xi}$.
For each $\xi<\kappa$, $H_{\xi}=\{x:W_{\xi}[\{x\}]\notin\Cal J\}$
belongs to $\Cal I$;  as $\kappa<\add\Cal I$,
$H=\bigcup_{\xi<\kappa}H_{\xi}\in\Cal I$.   For any $x\notin H$,
$W[\{x\}]=\bigcup_{\xi<\kappa}W_{\xi}[\{x\}]\in\Cal J$ because
$\kappa<\add\Cal J$;  so $W\in\Cal I\ltimes\Cal J$.   As
$\ofamily{\xi}{\kappa}{W_{\xi}}$ is arbitrary,
$\add(\Cal I\ltimes\Cal J)\ge\min(\add\Cal I,\add\Cal J)$.

In the other direction, as $X\notin\Cal I$, $\Cal J=\{F:F\subseteq Y$,
$X\times F\in\Cal I\ltimes\Cal J\}$,
so $F\mapsto X\times F$ is a Tukey function from $\Cal J$ to $\Cal
I\ltimes\Cal J$ and
$\add\Cal J\ge\add(\Cal I\ltimes\Cal J)$, $\cf\Cal J\le\cf(\Cal
I\ltimes\Cal J)$.   Similarly, $E\mapsto E\times Y$ is a Tukey function
from $\Cal I$ to
$\Cal I\ltimes\Cal J$ and $\add\Cal I\ge\add(\Cal I\ltimes\Cal J)$,
$\cf\Cal I\le\cf(\Cal I\ltimes\Cal J)$.

\medskip

\quad{\bf (ii)} Let $A\subseteq X$ and $B\subseteq Y$ be such that
$A\notin\Cal I$, $B\notin\Cal J$, $\#(A)=\non\Cal I$
and $\#(B)=\non\Cal J$.   Then $A\times B\notin\Cal I\ltimes\Cal J$, so
$\non(\Cal I\ltimes\Cal J)\le\#(A\times B)$.   But note that as $\Cal I$
and $\Cal J$ are ideals,
$A$ and $B$ are either singletons or infinite;  so $\#(A\times
B)=\max(\#(A),\#(B))$ and
$\non(\Cal I\ltimes\Cal J)\le\max(\non\Cal I,\non\Cal J)$.

In the other direction, take any
$W\in\Cal P(X\times Y)\setminus(\Cal I\ltimes\Cal J)$.   Set
$E=\{x:W[\{x\}]\notin\Cal J\}$.
Then $\#(E)\ge\non\Cal I$ and $\#(W[\{x\}])\ge\non\Cal J$ for every
$x\in E$, so $\#(W)\ge\max(\non\Cal I,\non\Cal J)$;  as $W$ is
arbitrary,
$\non(\Cal I\ltimes\Cal J)\ge\max(\non\Cal I,\non\Cal J)$.

\medskip

\quad{\bf (iii)} If $\Cal A\subseteq\Cal I$ covers $X$, then
$\{A\times Y:A\in\Cal A\}\subseteq\Cal I\ltimes\Cal J$ covers
$X\times Y$;  so
$\cov(\Cal I\ltimes\Cal J)\le\cov\Cal I$.
Similarly, $\cov(\Cal I\ltimes\Cal J)\le\cov\Cal J$.

Now suppose that $\Cal W\subseteq\Cal I\ltimes\Cal J$ and that
$\#(\Cal W)<\min(\cov\Cal I,\cov\Cal J)$.   For each $W\in\Cal W$ set
$E_W=\{x:W[\{x\}]\notin\Cal J\}$;  then
$E_W\in\Cal I$ for every $W$, so there is an $x\in
X\setminus\bigcup_{W\in\Cal W}E_W$, because $\#(\Cal W)<\cov\Cal I$.
Now $W[\{x\}]\in\Cal J$ for every $W$, so
there is a $y\in Y\setminus\bigcup_{W\in\Cal W}W[\{x\}]$, because
$\#(\Cal W)<\cov\Cal J$.   In this case
$(x,y)\in(X\times Y)\setminus\bigcup\Cal W$.   As $\Cal W$ is arbitrary,
$\cov(\Cal I\ltimes\Cal J)\ge\min(\cov\Cal I,\cov\Cal J)$ and we have
equality.\ \Qed}%end of prooflet

\spheader 527Bc \cmmnt{The idea of the operation $\ltimes$ here is
that we iterate notions of `negligible set' in a way indicated by
Fubini's theorem:  a measurable subset of $\BbbR^2$
is negligible iff almost every vertical section is negligible, that is,
iff it belongs to $\Cal N\ltimes\Cal N$.   However it is immediately
apparent that $\Cal N\ltimes\Cal N$
contains many non-measurable sets, and indeed many sets of full outer
measure (527Xa).   We are therefore led to the following idea.}
If $\Lambda$ is a family of subsets of $X\times Y$,
write $\Cal I\ltimes_{\Lambda}\Cal J$ for the ideal generated by
$(\Cal I\ltimes\Cal J)\cap\Lambda$.
Note that if $\kappa\le\min(\add\Cal I,\add\Cal J)$ and
$\bigcup\Cal W\in\Lambda$
for every $\Cal W\in[\Lambda]^{<\kappa}$, then
$\add(\Cal I\ltimes_{\Lambda}\Cal J)\ge\kappa$;  in particular,
$\Cal I\ltimes_{\Lambda}\Cal J$ will be a $\sigma$-ideal whenever
$\Cal I$ and $\Cal J$ are $\sigma$-ideals and $\Lambda$ is a
$\sigma$-algebra of subsets of $X\times Y$.
\cmmnt{Typical applications will be with $\Lambda$ a Borel
$\sigma$-algebra or an algebra
of the form $\Sigma\tensorhat\Tau$.   Thus 252F tells us that

\inset{\noindent if $(X,\Sigma,\mu)$ and $(Y,\Tau,\nu)$ are measure
spaces with c.l.d.\ product $(X\times Y,\Lambda,\lambda)$ then
$\Cal N(\mu)\ltimes_{\Lambda}\Cal N(\nu)\subseteq\Cal N(\lambda)$.}

\noindent If $\mu$ and $\nu$ are $\sigma$-finite then we get

\Centerline{$\Cal N(\lambda)
=\Cal N(\mu)\ltimes_{\Sigma\tensorhat\Tau}\Cal N(\nu)$}

\noindent (252C).
If we take $\Cal B=\Cal B(\BbbR^2)$ to be the Borel $\sigma$-algebra of
$\BbbR^2$,
then all four ideals $\Cal N\ltimes_{\Cal B(\BbbR^2)}\Cal N$,
$\Cal M\ltimes_{\Cal B(\BbbR^2)}\Cal M$,
$\Cal M\ltimes_{\Cal B(\BbbR^2)}\Cal N$ and
$\Cal N\ltimes_{\Cal B(\BbbR^2)}\Cal M$ become interesting.
In the next few paragraphs I will sketch some of the ideas needed to
deal with ideals of these kinds.
}%end of comment

\leader{527C}{}\cmmnt{ We are already familiar with
$\Cal N\ltimes_{\Cal B(\BbbR^2)}\Cal N$;  I begin by repeating a result
from \S417 in this language.

\medskip

\noindent}{\bf Theorem} Let $(X,\frak T,\Sigma,\mu)$ and
$(Y,\frak S,\Tau,\nu)$ be $\sigma$-finite effectively locally finite
$\tau$-additive topological measure spaces.
Let $\tilde\lambda$ be the $\tau$-additive product measure on
$X\times Y$\cmmnt{ (417C, 417G)}.   Then
$\Cal N(\mu)\ltimes_{\Cal B(X\times Y)}\Cal N(\nu)
=\Cal N(\tilde\lambda)$.

\proof{ Completing $\mu$ and $\nu$ does not change $\Cal N(\mu)$ or
$\Cal N(\nu)$, and leaves $\mu$ and $\nu$ effectively locally finite and
$\tau$-additive;
so it also does not change $\tilde\lambda$ (use the uniqueness assertion
in 417Da).   We may therefore assume that $\mu$ and
$\nu$ are complete.   Now 417H tells us that a Borel subset of
$X\times Y$ is $\tilde\lambda$-negligible iff it belongs to
$\Cal N(\mu)\ltimes\Cal N(\nu)$.   On the other hand, because $\mu$ and
$\nu$ are $\sigma$-finite, so is
$\tilde\lambda$, and every $\tilde\lambda$-negligible set is included in
a $\tilde\lambda$-negligible Borel set.   \Prf\ Suppose that
$W\in\Cal N(\tilde\lambda)$.   Let $\sequencen{W_n}$
be a cover of $X\times Y$ by sets of finite measure.   Because
$\tilde\lambda$ is inner regular with respect to the Borel sets (417Da),
we can find $V_n\in\Cal B(X\times Y)$ such
that $V_n\subseteq W_n\setminus W$ and
$\tilde\lambda V_n=\tilde\lambda W_n$ for each $n$.   Now

\Centerline{$W\subseteq(X\times Y)\setminus\bigcup_{n\in\Bbb N}V_n
\in\Cal N(\tilde\lambda)\cap\Cal B(X\times Y)$.  \Qed}

\noindent So $\Cal N(\mu)\ltimes_{\Cal B(X\times Y)}\Cal N(\nu)
=\Cal N(\tilde\lambda)$.
}%end of proof of 527C

\leader{527D}{}\cmmnt{ The case
$\Cal M\ltimes_{\Cal B(\BbbR^2)}\Cal M$ is also well known.

\medskip

\noindent}{\bf Theorem} Let $X$ and $Y$ be topological spaces, with
product $X\times Y$.   Write
$\Cal M^*=\Cal M(X)\ltimes_{\Cal B(X\times Y)}\Cal M(Y)$ and
$\Cal M_1^*=\Cal M(X)\ltimes_{\widehat{\Cal B}(X\times Y)}\Cal M(Y)$,
writing $\widehat{\Cal B}(X\times Y)$ for the Baire-property algebra of
$X\times Y$.

(a) If $\Cal M(X\times Y)\subseteq\Cal M_1^*$, then
$\Cal M^*=\Cal M_1^*=\Cal M(X\times Y)$.

(b) Let $\frak G$ be the category algebra of $Y$\cmmnt{ (514I)}.   If
$\pi(\frak G)<\add\Cal M(X)$ then $\Cal M^*=\Cal M(X\times Y)$.

\proof{{\bf (a)(i)} \Quer\ If $\Cal M_1^*\ne\Cal M(X\times Y)$,
there is a set
$W\in\Cal M_1^*\setminus\Cal M(X\times Y)$;  take
$W_1\in\widehat{\Cal B}(X\times Y)\cap(\Cal M(X)\ltimes\Cal M(Y))$
such that $W_1\supseteq W$.   By 4A3Ra, there is an open set
$V\subseteq X\times Y$ such that $W_1\symmdiff V$ is meager and
$V\cap V'$ is empty whenever $V'\subseteq X\times Y$
is open and $V'\cap W_1$ is meager.   As $W_1\notin\Cal M(X\times Y)$,
$V$ cannot be empty;  let $G\subseteq X$, $H\subseteq Y$ be non-empty
open sets such that $G\times H\subseteq V$.
In this case, $G\times H$ cannot be meager, so neither $G$ nor $H$ can
be meager.   (If $F\subseteq X$ is nowhere dense, then $F\times Y$ is
nowhere dense in $X\times Y$;  so
$M\times Y\in\Cal M(X\times Y)$ whenever $M\in\Cal M(X)$;  as
$G\times Y\notin\Cal M(X\times Y)$, $G\notin\Cal M(X)$.)   But now we
see that

\Centerline{$\{x:(G\times H)[\{x\}]\notin\Cal M(Y)\}=G\notin\Cal M(X)$,}

\noindent so that $G\times H\notin\Cal M_1^*$;  but
$(G\times H)\setminus W_1$ is meager, so belongs to $\Cal M_1^*$, and
$W_1$ is also supposed to belong to $\Cal M_1^*$.\ \Bang

\medskip

\quad{\bf (ii)} So $\Cal M_1^*=\Cal M(X\times Y)$.   Of course
$\Cal M^*\subseteq\Cal M_1^*$ just because
$\Cal B(X\times Y)\subseteq\widehat{\Cal B}(X\times Y)$.   In the other
direction, if $W\in\Cal M(X\times Y)$ there is a meager
F$_{\sigma}$ set $W'\supseteq W$.   Now $W'$ is a Borel set in
$\Cal M(X\times Y)=\Cal M_1^*$, so $W'\in\Cal M(X)\ltimes\Cal M(Y)$ and
witnesses that $W\in\Cal M^*$.   Thus $\Cal M(X\times Y)\subseteq\Cal M^*$
and the three classes are equal.

\medskip

{\bf (b)} By (a), I have only to show that $W\in\Cal M^*$ whenever
$W\subseteq X\times Y$ is meager.   Let
$D\subseteq\frak G\setminus\{0\}$ be an order-dense subset with cardinal
$\pi(\frak G)$.
Let $H$ be the smallest comeager regular open subset of $Y$, so that an
open subset of $Y$ is meager iff it is disjoint from $H$ (4A3Ra).   For
each $d\in D$ let $V_d\subseteq Y$
be an open set such that $V_d^{\ssbullet}=d$ in $\frak G$;  since
$H^{\ssbullet}=1$, we may suppose that $V_d\subseteq H$.   Observe that
if $F\subseteq Y$ is a non-meager closed set,
then there is a $d\in D$ such that $0\ne d\Bsubseteq F^{\ssbullet}$ in
$\frak G$, in which case $V_d\setminus F$ is meager;  as
$V_d\subseteq H$, $V_d\subseteq F$.

If $W\subseteq X\times Y$ is a nowhere dense closed set, it belongs to
$\Cal M^*$.   \Prf\   Set $E=\{x:W[\{x\}]$ is not meager$\}$.   For each
$d\in D$, the set

\Centerline{$E_d=\{x:V_d\subseteq W[\{x\}]\}=\{x:(x,y)\in W$ for every
$y\in V_d\}$}

\noindent is a closed set in $X$ and $E_d\times V_d\subseteq W$;  so
$\interior E_d\times V_d$ is an open subset of $W$.   As $W$ is nowhere
dense, and $V_d\ne\emptyset$, $\interior E_d$ must be
empty, and $E_d\in\Cal M(X)$.   Next, $E=\bigcup_{d\in D}E_d$ and
$\#(D)=\pi(\frak G)<\add\Cal M(X)$, so $E\in\Cal M(X)$ and
$W\in\Cal M^*$.\ \Qed

Since $\Cal M^*$ is a $\sigma$-ideal, it follows that every meager
subset of $X\times Y$ belongs to $\Cal M^*$, as required.
}%end of proof of 527D

\leader{527E}{Corollary} If $X$ and $Y$ are separable metrizable spaces,
then $\Cal M(X\times Y)=\Cal M(X)\ltimes_{\Cal B(X\times Y)}\Cal M(Y)$.

\proof{ $\pi(\frak C)\le\pi(Y)\le w(Y)\le\omega<\add\Cal M(X)$ (514Ja,
5A4Ba, 4A2P(a-i)).
}%end of proof of 527E

\cmmnt{\medskip

\noindent{\bf Remark} The case $X=Y=\Bbb R$ is the {\bf Kuratowski-Ulam
theorem}.
}%end of comment

\leader{527F}{}\cmmnt{ If we mix measure and category, as in
$\Cal M\ltimes_{\Cal B(\BbbR^2)}\Cal N$ and
$\Cal N\ltimes_{\Cal B(\BbbR^2)}\Cal M$, we encounter some new
phenomena.
To deal with the first we need the following, which is important for
other reasons.

\medskip

\noindent}{\bf Lemma}\cmmnt{ (see {\smc Cicho\'n \& Pawlikowski 86})}
Let $X$ be a set, $\Sigma$ a $\sigma$-algebra of subsets of $X$, and
$\Cal I$ a $\sigma$-ideal of subsets of $X$ generated by
$\Sigma\cap\Cal I$;  suppose that
the quotient algebra $\Sigma/\Sigma\cap\Cal I$ is non-zero, atomless and
has countable $\pi$-weight.
Let $Y$ be a set, $\Tau$ a $\sigma$-algebra of subsets of $Y$, and
$\sequencen{\Cal H_n}$ a sequence of finite covers of $Y$ by members of
$\Tau$.   Set

\Centerline{$\Cal H^*_n=\{\bigcup_{m\ge n}H_m:H_m\in\Cal
H_m\cup\{\emptyset\}$ for every $m\ge n\}$}

\noindent for each $n\in\Bbb N$.   Then there is a sequence
$\sequencen{W_n}$ of subsets of $\NN\times X\times Y$ such that

(i) for every $n\in\Bbb N$, $W_n$ is expressible as the union of a
sequence of sets of the form $I\times E\times F$ where $I\subseteq\NN$
is open-and-closed, $E\in\Sigma$ and $F\in\Tau$;

(ii) whenever $n\in\Bbb N$, $\alpha\in\NN$ and $x\in X$ then
$\{y:(\alpha,x,y)\in W_n\}\in\Cal H_n^*$;

(iii) setting $W=\bigcap_{n\in\Bbb N}W_n$, the set
$\{(\alpha,x):\alpha\in\NN$, $x\in X$, $(\alpha,x,f(x))\notin W\}$
belongs to $[\NN]^{\le\omega}\ltimes\Cal I$ for every
$(\Sigma,\Tau)$-measurable function $f:X\to Y$.

\proof{ If $X\in\Cal I$ or $Y$ is empty, we can take every $W_n$ to be $\emptyset$;
suppose otherwise.

\medskip

{\bf (a)} Set $S=\bigcup_{n\in\Bbb N}\BbbN^n$.
There is a family $\family{\sigma}{S}{U_{\sigma}}$ such that

\inset{every $U_{\sigma}$ belongs to $\Sigma\setminus\Cal I$,

for every $\sigma\in S$,
$\sequence{i}{U_{\sigma^{\smallfrown}\fraction{i}}}$ is
a disjoint sequence of subsets of $U_{\sigma}$ and
$U_{\sigma}\setminus\bigcup_{i\in\Bbb N}
  U_{\sigma^{\smallfrown}\fraction{i}}\in\Cal I$,

for every $E\in\Sigma\setminus\Cal I$ there is a $\sigma\in S$
such that $U_{\sigma}\setminus E\in\Cal I$.}

\noindent\Prf\ Let $D$ be a countable order-dense set in
$\frak A=\Sigma/\Sigma\cap\Cal I$.   Then the subalgebra $\frak B$ of
$\frak A$ generated by $D$ is countable and atomless and non-trivial.
Let $\Cal E$ be the algebra of subsets of $\Cal P(\NN)$ generated by the
sets
$I_{\sigma}=\{\alpha:\sigma\subseteq\alpha\in\NN\}$ for $\sigma\in S$.
This is also an atomless countable Boolean algebra, and must therefore
be isomorphic to $\frak B$ (316M).   Let $\pi:\Cal E\to\frak B$ be an
isomorphism, and set $b_{\sigma}=\pi I_{\sigma}$ for each
$\sigma\in S$.   Set $U_{\emptyset}=X$ and
for $n\in\Bbb N$, $\sigma\in\BbbN^n$ choose a disjoint sequence
$\sequence{i}{U_{\sigma^{\smallfrown}\fraction{i}}}$ of subsets of
$U_{\sigma}$ such that
$U_{\sigma^{\smallfrown}\fraction{i}}^{\ssbullet}
=b_{\sigma^{\smallfrown}\fraction{i}}$ for
every $i$.   This construction ensures that
$U_{\sigma}\in\Sigma\setminus\Cal I$ for every $\sigma$.
If $E\in\Sigma\setminus\Cal I$, there must be a non-zero $d\in D$ such
that
$d\Bsubseteq E^{\ssbullet}$;  now $\pi^{-1}d\in\Cal
E\setminus\{\emptyset\}$, so there is a $\sigma\in S$ such that
$I_{\sigma}\Bsubseteq\pi^{-1}d$, $b_{\sigma}\Bsubseteq E^{\ssbullet}$
and $U_{\sigma}\setminus E\in\Cal I$.
Finally, if $\sigma\in S$, set
$E=U_{\sigma}\setminus\bigcup_{i\in\Bbb N}
U_{\sigma^{\smallfrown}\fraction{i}}$;
then for every $\tau\in S$ either $\tau\subseteq\sigma$ and
$U_{\tau}\setminus E\supseteq U_{\sigma^{\smallfrown}\fraction{0}}
\notin\Cal I$, or
$U_{\tau}\cap U_{\sigma}=\emptyset$ and
$U_{\tau}\setminus E\supseteq U_{\tau}\notin\Cal I$,
or there is an $i\in\Bbb N$ such that
$\tau\supseteq\sigma^{\smallfrown}\fraction{i}$ and again
$U_{\tau}\setminus E\subseteq U_{\tau}\notin\Cal I$.   This means that
$E$ must belong to $\Cal I$, so that $\family{\sigma}{S}{U_{\sigma}}$
has all the required properties.\ \Qed

\medskip

{\bf (b)} Enumerate $S$ as $\sequence{k}{\tau_k}$.   Let
$\sequencen{H_n}$ be a sequence running over
$\bigcup_{n\in\Bbb N}\Cal H_n$.   For $n\in\Bbb N$, set

\Centerline{$K_n
=\{(\sigma,k):\sigma\in\BbbN^{n+2}$, $k<\#(\tau_{\sigma(n)})$,
$\sigma(n+1)=\#(\tau_k)$, $H_{\tau_{\sigma(n)}(k)}\in\Cal H_n\}$,}

\Centerline{$V_n
=\bigcup_{(\sigma,k)\in K_n}\{(\alpha,x,y):
  \tau_k\subseteq\alpha\in\NN$, $x\in U_{\sigma}$,
  $y\in H_{\tau_{\sigma(n)}(k)}\}$.}

\noindent  If $\alpha\in\NN$ and $x\in X$ the section
$\{y:(\alpha,x,y)\in V_n\}$ is either empty or
$H_{\tau_{\sigma(n)}(k)}$ where $\sigma\in\BbbN^{n+2}$,
$x\in U_{\sigma}$ and $\tau_k=\alpha\restr\sigma(n+1)$;
in either case it belongs to $\Cal H_n\cup\{\emptyset\}$.

So if we now set
$W_n=\bigcup_{m\ge n}V_m$, $W_n$ satisfies (i) and (ii) for every $n$.

\medskip

{\bf (c)} Set $W=\bigcap_{n\in\Bbb N}W_n$.   \Quer\ Suppose, if
possible, that $f:X\to Y$ is a
$(\Sigma,\Tau)$-measurable function such that
$\{(\alpha,x):(\alpha,x,f(x))\notin W\}
  \notin[\NN]^{\le\omega}\ltimes\Cal I$.   Note that

\Centerline{$\{V:V\subseteq\NN\times X\times Y$,
$\{x:(\alpha,x,f(x))\in V\}\in\Sigma$ for every $\alpha\in\NN\}$}

\noindent is a $\sigma$-algebra of subsets of $\NN\times X\times Y$
containing $I\times E\times F$ whenever $I$ is open-and-closed,
$E\in\Sigma$ and $F\in\Tau$, so
contains every $V_n$ and every $W_n$.

\wheader{527F}{0}{0}{0}{48pt}
Set

\Centerline{$A_0=\{\alpha:\alpha\in\NN$,
$\{x:(\alpha,x,f(x))\notin W\}\notin\Cal I\}$,}

\noindent so that $A_0$ is uncountable.   For each $\alpha\in A_0$,

\Centerline{$\bigcup_{n\in\Bbb N}\{x:(\alpha,x,f(x))\notin W_n\}
=\{x:(\alpha,x,f(x))\notin W\}$}

\noindent does not belong to $\Cal I$.   So there is an $n\in\Bbb N$
such that

\Centerline{$A_1
=\{\alpha:\alpha\in A_0$,
  $\{x:(\alpha,x,f(x))\notin W_n\}\notin\Cal I\}$}

\noindent is uncountable.   For each $\alpha\in A_1$, set
$G_{\alpha}=\{x:(\alpha,x,f(x))\notin W_n\}$;  then
$G_{\alpha}\in\Sigma\setminus\Cal I$, so there is a $\sigma\in S$ such
that
$U_{\sigma}\setminus G_{\alpha}\in\Cal I$.   Let $\sigma\in S$ be such
that

\Centerline{$A_2
=\{\alpha:\alpha\in A_1$, $U_{\sigma}\setminus G_{\alpha}\in\Cal I\}$}

\noindent is uncountable.   Set $m=\max(n,\#(\sigma))$, so that
$U_{\sigma}\cap\{x:(\alpha,x,f(x))\in V_m\}\in\Cal I$ for every
$\alpha\in A_2$.   Set $M=\#(\Cal H_m)$.

Take $k\in\Bbb N$ such that $\#(\{\alpha\restr k:\alpha\in A_2\})\ge M$.
Let $\langle\alpha_i\rangle_{i<M}$ be a family in $A_2$ such that
$\alpha_i\restr k\ne\alpha_j\restr k$ for distinct $i$, $j<M$;  let
$\langle r_i\rangle_{i<M}$, $\langle l_i\rangle_{i<M}$ be such that
$\alpha_i\restr k=\tau_{r_i}$ for each $i$ and
$\Cal H_m=\{H_{l_i}:i<M\}$.
Let $s\in\Bbb N$ be such that $\tau_s(r_i)$ is defined and equal to
$l_i$ for $i<M$.   Let $\sigma'\in\BbbN^{m+2}$ be such that
$\sigma'\supseteq\sigma$, $\sigma'(m)=s$ and $\sigma'(m+1)=k$.   Then
$U_{\sigma'}\notin\Cal I$ and
$U_{\sigma'}\setminus G_{\alpha}\in\Cal I$ for every $\alpha\in A_2$.

Suppose that $i<M$ and $x\in U_{\sigma'}$.   Then

\Centerline{$\{y:(\alpha_i,x,y)\in V_m\}=H_{\tau_{\sigma'(m)}}(j)
=H_{\tau_s(j)}$}

\noindent where $(\sigma',j)\in K_m$, that is, $j$ is such that
$\tau_j\subseteq\alpha_i$ and $\#(\tau(j))=\sigma'(m+1)=k$.   Thus
$j=r_i$, $\tau_s(j)=l_i$ and
$\{y:(\alpha_i,x,y)\in V_m\}=H_{l_i}$.   But this means that, for any
$x\in U_{\sigma'}$,

\Centerline{$\bigcup_{i<M}\{y:(\alpha_i,x,y)\in V_m\}
=\bigcup_{i<M}H_{l_i}=Y$}

\noindent contains $f(x)$;  that is,
$U_{\sigma'}\subseteq\bigcup_{i<M}\{x:(\alpha_i,x,f(x))\in V_m\}$.
On the other hand,

\Centerline{$U_{\sigma'}\cap\{x:(\alpha_i,x,f(x))\in V_m\}
\subseteq U_{\sigma}\cap\{x:(\alpha_i,x,f(x))\in V_m\}\in\Cal I$}

\noindent for each $i<M$, while $U_{\sigma'}$ itself does not belong to
$\Cal I$.   So this is impossible.\ \Bang

Thus $\sequencen{W_n}$ satisfies (iii).
}%end of proof of 527F

\vleader{48pt}{527G}{Theorem} Let $X$ be a set, $\Sigma$ a $\sigma$-algebra of
subsets of $X$, and $\Cal I$ a $\sigma$-ideal of subsets of $X$ which is
generated by $\Sigma\cap\Cal I$;  suppose that the quotient algebra
$\Sigma/\Sigma\cap\Cal I$ is non-zero, atomless and has countable
$\pi$-weight.
Let $(Y,\Tau,\nu)$ be an atomless perfect semi-finite measure space such that
$\nu Y>0$.    Set
$\Cal K=\Cal I\ltimes_{\Sigma\tensorhat\Tau}\Cal N(\nu)$.   Then
$[\frak c]^{\le\omega}\prT\Cal K$, so $\add\Cal K=\omega_1$ and
$\cf\Cal K\ge\frak c$.

\proof{{\bf (a)} To begin with (down to the end of (d)) suppose that $\nu$ is totally
finite.   Because $\nu$ is atomless, we can for each $n\in\Bbb N$
find a finite cover $\Cal H_n$ of $Y$ by measurable sets with measures
at most $2^{-n}$.   Let $\Tau_0$ be the
$\sigma$-algebra generated by $\bigcup_{n\in\Bbb N}\Cal H_n$, so that
$\Tau_0$ is a $\sigma$-subalgebra of $\Tau$.
Construct $\sequencen{\Cal H^*_n}$,
$\sequencen{W_n}$ and $W$ from $\sequencen{\Cal H_n}$ as in
527F.   Then if
$f:X\to Y$ is $(\Sigma,\Tau_0)$-measurable,
$\{(\alpha,x):(\alpha,x,f(x))\notin W\}\in[\NN]^{\le\omega}\ltimes\Cal I$.
Note that $\nu H\le 2^{-n+1}$ for every $H\in\Cal H_n^*$,
so $\nu\{y:(\alpha,x,y)\in W_n\}\le 2^{-n+1}$ for every $\alpha\in\NN$,
$x\in X$ and $n\in\Bbb N$.   For each $\alpha\in\NN$ set
$K_{\alpha}=\{(x,y):(\alpha,x,y)\in W\}$.  Observe that
$K_{\alpha}\in\Sigma\tensorhat\Tau$ because
$W\in\Cal B(\NN)\tensorhat\Sigma\tensorhat\Tau$, and that

\Centerline{$\nu K_{\alpha}[\{x\}]
\le\inf_{n\in\Bbb N}\nu\{y:(\alpha,x,y)\in W_n\}=0$}

\noindent for every $x\in X$, so $K_{\alpha}\in\Cal K$ for every
$\alpha\in\NN$.

\medskip

{\bf (b)} Set $\hat\Sigma=\{E\symmdiff M:E\in\Sigma$, $M\in\Cal I\}$.
Then $\hat\Sigma$ is a $\sigma$-algebra of subsets of $X$ (cf.\ 212Ca)
and $\Cal I$ is a
$\sigma$-ideal in $\hat\Sigma$;  also the identity embedding of $\Sigma$
in $\hat\Sigma$ induces an isomorphism between $\Sigma/\Sigma\cap\Cal I$
and $\hat\Sigma/\Cal I$
(cf.\ 322Da).   Consequently $\hat\Sigma/\Cal I$ has countable
$\pi$-weight, therefore is ccc, and $\hat\Sigma$ is closed under
Souslin's operation (431G).  

\medskip

{\bf (c)} Let $A\subseteq\NN$ be an uncountable set, and
$V\in\Sigma\tensorhat\Tau$ a set disjoint from
$\bigcup_{\alpha\in A}K_{\alpha}$.  (I aim to show that
$(X\times Y)\setminus V\notin\Cal I\ltimes\Cal N(\nu)$.)   There must be
sequences $\sequencen{C_n}$ in $\Sigma$, $\sequencen{F_n}$ in $\Tau$
such that $V$ belongs to the $\sigma$-algebra
generated by $\{C_n\times F_n:n\in\Bbb N\}$;  we may of course arrange
that $\bigcup_{n\in\Bbb N}\Cal H_n\subseteq\{F_n:n\in\Bbb N\}$.   Let
$\Tau_1$ be the
$\sigma$-subalgebra of $\Tau$ generated by $\{F_n:n\in\Bbb N\}$, so that
$\Tau_0\subseteq\Tau_1$ and $V\in\Sigma\tensorhat\Tau_1$.
Let $g:Y\to\Bbb R$ be the Marczewski functional defined by setting
$g=\sum_{n=0}^{\infty}3^{-n}\chi F_n$.
Because $\nu$ is perfect, there is a Borel set $H\subseteq g[Y]$ such
that $g^{-1}[H]$ is conegligible.   Let $h:H\to Y$ be any function such
that $g(h(t))=t$ for every
$t\in H$;  note that $h$ is $(\Cal B(H),\Tau_1)$-measurable, where $\Cal B(H)$
is the Borel $\sigma$-algebra of $H$, just because
$\overline{g[F_n]}\cap\overline{g[Y\setminus F_n]}$
is empty for every $n$.   Set $V_0=\{(x,t):x\in X$, $t\in H$,
$(x,h(t))\in V\}$;  then $V_0\in\Sigma\tensorhat\Cal B(H)$.
It follows that $V_0$ belongs to the class of sets obtainable by
Souslin's operation from sets of the form $E\times F$ where $E\in\Sigma$
and $F\subseteq H$ is relatively closed in $H$.
(Use 421F.)   Set $\tilde E=V_0^{-1}[H]$.   Because
$H$ is analytic and $\hat\Sigma$ is closed under Souslin's operation,
$\tilde E\in\hat\Sigma$ and there is a
$(\hat\Sigma,\Cal B(H))$-measurable function $f_1:\tilde E\to H$ such
that $(x,f_1(x))\in V_0$ for every $x\in \tilde E$ (423M).   Now
$f_2=hf_1:\tilde E\to Y$ is $(\hat\Sigma,\Tau_1)$-measurable and
$(x,f_2(x))\in V$ for every $x\in \tilde E$.

For every $n\in\Bbb N$, $E_n=f_1^{-1}[F_n]$ belongs to $\hat\Sigma$, so
there is an $E_n'\in\Sigma$ such that $E_n\symmdiff E'_n\in\Cal I$.
Similarly, there is an $\tilde E'\in\Sigma$
such that $\tilde E\symmdiff\tilde E'\in\Cal I$.   Because $\Cal I$ is
generated by $\Sigma\cap\Cal I$, there is an $M_0\in\Sigma\cap\Cal I$ including
$(\tilde E\symmdiff\tilde E')\cup\bigcup_{n\in\Bbb N}(E_n\symmdiff E'_n)$.
Now $\tilde E\setminus M_0=\tilde E'\setminus M_0$
belongs to $\Sigma$.   Set $f_3=f_2\restr\tilde E\setminus M_0$;  then
$f_3^{-1}[F_n]=E'_n\setminus M_0\in\Sigma$ for every $n$, so $f_3$ is
$(\Sigma,\Tau_1)$-measurable.   Take any
$y_0\in Y$, and set $f(x)=f_3(x)$ if $x\in\tilde E\setminus M_0$, $y_0$
for other $x\in X$;  then $f$ is $(\Sigma,\Tau_1)$-measurable, therefore
$(\Sigma,\Tau_0)$-measurable.

The set $\{(\alpha,x):(\alpha,x,f(x))\notin W\}$ belongs to
$[\NN]^{\le\omega}\ltimes\Cal I$, so there must be an $\alpha\in A$ such
that $M_1=\{x:(\alpha,x,f(x))\notin W\}$ belongs
to $\Cal I$.   \Quer\ Suppose, if possible, that
$(X\times Y)\setminus V\in\Cal I\ltimes\Cal N(\nu)$.
Then there must be an $x\in X\setminus(M_0\cup M_1)$
such that $V[\{x\}]$ is conegligible.   In this case,
$V[\{x\}]\cap g^{-1}[H]$ is conegligible, so is not empty, and there
is a $y\in V[\{x\}]\cap g^{-1}[H]$.   Consider
$y'=h(g(y))$;  then $g(y')=g(y)$, so $\{n:y'\in F_n\}=\{n:y\in F_n\}$,
and $\{F:y\in F\iff y'\in F\}$ is a $\sigma$-algebra of subsets of $Y$
containing every $F_n$ and therefore
containing $V[\{x\}]$.   So $y'\in V[\{x\}]$ and $(x,g(y))\in V_0$.
This means that $x\in\tilde E$;  as $x\notin M_0$, $f(x)=f_3(x)=f_2(x)$
and $(x,f(x))\in V$.   On the other
hand, $x\notin M_1$, so $(\alpha,x,f(x))\in W$ and $(x,f(x))\in
K_{\alpha}$;  contradicting the choice of $V$ as a set disjoint from
$K_{\alpha}$.\ \Bang

This shows that $(X\times Y)\setminus V\notin\Cal I\ltimes\Cal N(\nu)$.
As $V$ is arbitrary, $\bigcup_{\alpha\in A}K_{\alpha}\notin\Cal K$.

\medskip

{\bf (d)} This is true for every uncountable $A\subseteq\NN$.   But this
means that $A\mapsto\bigcup_{\alpha\in A}K_{\alpha}$ is a Tukey function
from $[\NN]^{\le\omega}$ to
$\Cal K$, and $[\frak c]^{\le\omega}\cong[\NN]^{\le\omega}\prT\Cal K$.

\medskip

{\bf (e)} Thus the theorem is true if $\nu Y$ is finite.   For the
general case, let $Y_0\in\Tau$ be such that $0<\nu Y_0<\infty$.
Then the subspace measure $\nu_{Y_0}$ is still atomless and perfect
(214Ka, 451Dc), so $[\frak c]^{\le\omega}\prT\Cal K_0$, where
$\Cal K_0=\Cal I\ltimes_{\Sigma\tensorhat(\Tau\cap\Cal PY_0)}
\Cal N(\nu_{Y_0})$.
But $\Cal K_0=\Cal K\cap\Cal P(X\times Y_0)$,
so the identity map from $\Cal K_0$ to
$\Cal K$ is a Tukey function, and

\Centerline{$[\frak c]^{\le\omega}\prT\Cal K_0\prT\Cal K$}

\noindent in this case also.
It follows at once that $\add\Cal K\le\add[\frak c]^{\le\omega}=\omega_1$,
so that $\add\Cal K=\omega_1$, and that
$\cf\Cal K\ge\cff[\frak c]^{\le\omega}=\frak c$.
}%end of proof of 527G

\leader{527H}{Corollary} $\Cal M\ltimes_{\Cal B(\BbbR^2)}\Cal N
\equivT[\frak c]^{\le\omega}$.

\proof{ By 527G, $[\frak c]^{\le\omega}\prT\Cal M\ltimes_{\Cal B(\BbbR^2)}\Cal N$.
In the other direction, all we need to observe is
that $\#(\Cal B(\BbbR^2))=\frak c$.
Let $\ofamily{\xi}{\frakc}{W_{\xi}}$ run over
$\Cal B(\BbbR^2)\cap(\Cal M\ltimes\Cal N)$, and for
$V\in\Cal M\ltimes_{\Cal B(\BbbR^2)}\Cal N$
choose $\xi_V<\frak c$ such
that $V\subseteq W_{\xi_V}$;  then
$V\mapsto\{\xi_V\}:\Cal M\ltimes_{\Cal B(\BbbR^2)}\Cal N\to[\frak c]^{\le\omega}$ is a
Tukey function, so $\Cal M\ltimes_{\Cal B(\BbbR^2)}\prT[\frak c]^{\le\omega}$.
}%end of proof of 527H

\leader{527I}{}\cmmnt{ I now turn to the ideal
$\Cal N\ltimes_{\Cal B(\BbbR^2)}\Cal M$.

\medskip

\noindent}{\bf Lemma} Let $X$ be a set, $\Sigma$ a $\sigma$-algebra of
subsets of $X$, and $Y$ a topological space with a countable $\pi$-base
$\Cal H$.   Let $\Cal W$ be the family of subsets of $X\times Y$ of the
form $\bigcup_{H\in\Cal H}E_H\times H$, where $E_H\in\Sigma$ for every
$H\in\Cal H$, and
$\Cal D_0$ the family of sets $D\subseteq X\times Y$ such that
$(X\times Y)\setminus D\in\Cal W$ and $D[\{x\}]$ is nowhere dense for
every $x\in X$;  let $\Cal L_0$ be the $\sigma$-ideal of subsets of
$X\times Y$ generated by $\Cal D_0$.   Then
$\Sigma\tensorhat\Cal B(Y)
\subseteq\{W\symmdiff L:W\in\Cal W$, $L\in\Cal L_0\}$.

\proof{ Write $\Cal V=\{W\symmdiff L:W\in\Cal W$, $L\in\Cal L_0\}$.   Then $\Cal W$ and
$\Cal V$ are closed under countable unions.   Next, $(X\times Y)\setminus W\in\Cal V$
for every $W\in\Cal W$.   \Prf\ Express $W$ as $\bigcup_{H\in\Cal H}E_H\times H$ where
$E_H\in\Sigma$ for every $H\in\Cal H$.   For $H\in\Cal H$, set

\Centerline{$F_H=X\setminus\bigcup\{E_{H'}:H'\in\Cal H$, $H'\cap H\ne\emptyset\}
\in\Sigma$,}

\noindent and set $W'=\bigcup_{H\in\Cal H}F_H\times H$.   Then $W'$ and $W\cup W'$
belong to $\Cal W$.   Set
$D=(X\times Y)\setminus(W\cup W')$.   If $x\in X$ and $G\subseteq Y$ is a non-empty
open set, let $H\subseteq G$ be a non-empty member of $\Cal H$.
Then either $x\in F_H$ and $H$ is a non-empty subset of
$G\setminus D[\{x\}]$, or there is an $H'\in\Cal H$ such
that $H\cap H'\ne\emptyset$ and $x\in E_{H'}$, in which case
$H\cap H'$ is a non-empty subset of $G\setminus D[\{x\}]$.   As $G$ is
arbitrary, $D[\{x\}]$ is nowhere dense;  as $x$ is arbitrary, $D\in\Cal D$.
But now observe that $(X\times Y)\setminus W=W'\symmdiff D$ belongs to $\Cal V$.\ \Qed

It follows that the complement of any member of $\Cal V$ belongs to $\Cal V$, so
$\Cal V$ is a $\sigma$-algebra.   Now $E\times G\in\Cal V$ for every $E\in\Sigma$ and
open $G\subseteq Y$.   \Prf\ For $H\in\Cal H$, set $E_H=E$ if $H\subseteq G$,
$\emptyset$ otherwise;  set $W=\bigcup_{H\in\Cal H}E_H\times H\in\Cal W$.   Then
$W\subseteq E\times G$.   But, defining $W'$ from $W$ as just above, we see that
$W'$ is disjoint from $E\times G$.   So

\Centerline{$(E\times G)\symmdiff W\subseteq(X\times Y)\setminus(W\cup W')\in\Cal D$}

\noindent and $E\times G\in\Cal V$.\ \Qed

Accordingly $\Cal V$ includes the $\sigma$-algebra generated by
$\{E\times G:E\in\Sigma$, $G\subseteq Y$ is open$\}$, which is
$\Sigma\tensorhat\Cal B(Y)$.
}%end of proof of 527I

\leader{527J}{Theorem}\cmmnt{ (see {\smc Fremlin 91})} Let $X$ be a
topological space and $\mu$ a $\sigma$-finite quasi-Radon measure on $X$
with countable Maharam
type;  let $Y$ be a topological space of countable $\pi$-weight.   Then
$\Cal N(\mu)\ltimes_{\Cal B(X\times Y)}\Cal M(Y)\prT\Cal N$.

%will "countable network" do for Y ?

\proof{ Write $\Cal L$ for $\Cal N(\mu)\ltimes_{\Cal B(X\times Y)}\Cal M(Y)$,
and fix a countable $\pi$-base $\Cal H$ for the topology of $Y$.

\medskip

{\bf (a)} We need to know that for every Borel set $V\subseteq X\times Y$
there are sets $V'$, $V''\in\Cal B(X)\tensorhat\Cal B(Y)$
such that $V'\subseteq V\subseteq V''$ and
$V''\setminus V'\in\Cal L$.   \Prf\ Let $\Cal V^*$ be the family of
all subsets of $X\times Y$ with this property.   Because
$\Cal B(X)\tensorhat\Cal B(Y)$ is a $\sigma$-algebra and $\Cal L$ is a $\sigma$-ideal
of sets, $\Cal V^*$ is a $\sigma$-algebra.   If $W\subseteq X\times Y$
is open, set

\Centerline{$U_H=\bigcup\{G:G\subseteq X$ is open, $G\times H\subseteq W\}$,
\quad$U'_H=\{x:H\cap W[\{x\}]\ne\emptyset\}$}

\noindent for $H\in\Cal H$, so all the $U_H$ and $U'_H$ are
open ($U'_H$ is just the projection of the open set
$W\cap(X\times H)$).
Set $V_1=\bigcup_{H\in\Cal H}U_H\times H$ and
$V_2=\bigcup_{H\in\Cal H}((X\setminus U'_H)\times H)$.   Then $V_1$
and $V_2$ both belong to $\Cal B(X)\tensorhat\Cal B(Y)$,
$V_1\subseteq W$ and $W\cap V_2=\emptyset$.

Let $x\in X$.   \Quer\ If the open set $V_1[\{x\}]\cup V_2[\{x\}]$ is
not dense, there is a non-empty $H\in\Cal H$ disjoint from
both $V_1[\{x\}]$ and $V_2[\{x\}]$.   In this case $x$ must belong to $U'_H$,
and there is a point $y\in H\cap W[\{x\}]$.   $(x,y)$ belongs to the
open set $(X\times H)\cap W$, so there
are open sets $G\subseteq X$, $\tilde H\subseteq Y$ such that
$(x,y)\in G\times\tilde H\subseteq(X\times H)\cap W$.   Now there is an $H'\in\Cal H$
such that $\emptyset\ne H'\subseteq\tilde  H$, in which
case $x\in U_{H'}$.   But this will mean that $H'\subseteq V_1[\{x\}]$ and
$H'$ is a non-empty subset of $H\cap V_1[\{x\}]$, which is
impossible.\ \Bang

Thus $V_1[\{x\}]\cup V_2[\{x\}]$ is dense for every $x$, and if we set
$V_3=(X\times Y)\setminus V_2$ we shall have
$V_3\setminus V_1\in\Cal L$, while both $V_1$ and $V_3$ belong to
$\Cal B(X)\tensorhat\Cal B(Y)$, and $V_1\subseteq W\subseteq V_3$.   So
$W\in\Cal V^*$.   This is true for every open set $W\subseteq X\times Y$,
so the $\sigma$-algebra $\Cal V^*$ must contain every Borel set, as required.\ \Qed

It follows that every member of $\Cal L$ is included in a member of
$\Cal L\cap(\Cal B(X)\tensorhat\Cal B(Y))$.   \Prf\ If $V\in\Cal L$
there is a Borel set $V'\supseteq V$
which belongs to $\Cal L$, and now there is a set
$V''\in\Cal B(X)\tensorhat\Cal B(Y)$ such that $V''\supseteq V'$ and
$V''\setminus V'\in\Cal L$, in which case $V''\supseteq V$ also must
belong to $\Cal L$.\ \Qed

Thus $\Cal L=\Cal N(\mu)\ltimes_{\Cal B(X)\tensorhat\Cal B(Y)}\Cal M(Y)$.

\medskip

{\bf (b)} To begin with let us suppose
that $X$ is compact and metrizable, $\mu$ is totally finite and $Y$ is a
Baire space.

\medskip

\quad{\bf (i)}
Taking $\Sigma=\Cal B(X)$, define $\Cal W$, $\Cal D_0$ and $\Cal L_0$ as in 527I.
Now let $\Cal D$ be the family of closed subsets belonging to $\Cal D_0$, and
$\Cal L_1$ the $\sigma$-ideal of subsets of $X\times Y$ generated by
$\{E\times Y:E\in\Cal N(\mu)\}\cup\Cal D$.

\medskip

\quad {\bf (ii)} $\Cal D_0\subseteq\Cal L_1$.
\Prf\ If $D\in\Cal D_0$, express $(X\times Y)\setminus D$ as
$\bigcup_{H\in\Cal H}E_H\times H$ where $E_H\in\Cal B(X)$ for every $H\in\Cal H$.
Because $\mu$ is totally finite, $\mu$ is outer
regular with respect to the open sets (412Wb).
So for each $n\in\Bbb N$ we can find a family
$\family{H}{\Cal H}{G_{nH}}$ of open sets in $X$ such that $E_H\subseteq G_{nH}$ for
every $H$ and $\sum_{H\in\Cal H}\mu(G_{nH}\setminus E_H)\le 2^{-n}$.   Set
$D_n=(X\times Y)\setminus\bigcup_{H\in\Cal H}(G_{nH}\times H)$.   Then $D_n$ is closed
and $D_n\subseteq D\in\Cal D_0$ so $D_n\in\Cal D$.   Set
$E=\bigcap_{n\in\Bbb N}\bigcup_{H\in\Cal H}(G_{nH}\setminus E_H)$;   then
$E\in\Cal N(\mu)$ and

\Centerline{$D\subseteq (E\times Y)\cup\bigcup_{n\in\Bbb N}D_n\in\Cal L_1$. \Qed}

\medskip

\quad {\bf (iii)} Of course every member of $\Cal D$ belongs to $\Cal L$, so
$\Cal L_1\subseteq\Cal L$.   But in fact $\Cal L=\Cal L_1$.   \Prf\ If $V\in\Cal L$,
there is a $V'\in(\Cal N(\mu)\ltimes\Cal M(Y))\cap(\Cal B(X)\tensorhat\Cal B(Y))$
such that $V\subseteq V'$, by (a).   By 527I, we can
express $V'$ as $W\symmdiff L$ where $W\in\Cal W$ and $L\in\Cal L_0$.   By (ii),
$\Cal L_0\subseteq\Cal L_1$, so $W\in\Cal L$.   There is therefore a negligible set
$E\subseteq X$ such that $W[\{x\}]$ is meager for every $x\in X\setminus E$.
But $W[\{x\}]$ is always open, and $Y$ is a Baire
space, so $W\subseteq E\times Y\in\Cal L_1$.   Accordingly $V'$ and $V$ belong to
$\Cal L_1$.   As $V$ is arbitrary, $\Cal L\subseteq\Cal L_1$.\ \Qed

\medskip

\quad{\bf (iv)} Let $\Cal G$ be a countable base for the topology of $X$ containing
$X$.   Let
$\Cal U_0$ be the family of those sets $U\subseteq X\times Y$ such that $U$
is expressible as a finite union of sets of the form $G\times H$ where $G\in\Cal G$
and $H\in\Cal H$, and $\Cal U$ the set of those $U\in\Cal U_0$ such that
$\pi_1[U]=X$, where $\pi_1$ is the projection from $X\times Y$ onto $X$.
Consider

\Centerline{$\Cal D'=\{D:D\subseteq X\times Y$, for every $U_0\in\Cal U$
there is a $U\in\Cal U$ such that $U\subseteq U_0\setminus D\}$.}

$\Cal D\subseteq\Cal D'$.   \Prf\
Suppose that $D\in\Cal D$ and $U_0\in\Cal U$, and consider
$\Cal U_1=\{U:U\in\Cal U_0$, $U\subseteq U_0\setminus D\}$.   For every
$x\in X$ the section $U_0[\{x\}]$ is open and not empty and the section
$D[\{x\}]$ is
nowhere dense, so there is a $y$ such that $(x,y)\in U_0\setminus D$;
now there are $G\in\Cal G$, containing $x$, and
an open $H$ containing $y$ such that $G\times H\subseteq U_0\setminus D$.
Let $H'\in\Cal H$ be such that $\emptyset\ne H'\subseteq H$.
Then $U=G\times H'\in\Cal U_1$ and $x\in\pi_1[U]$.   As $x$ is arbitrary,
$\{\pi_1[U]:U\in\Cal U_1\}$ is an open cover of $X$;  as $X$ is compact and $\Cal U_1$
is upwards-directed, there is a $U\in\Cal U_1$ such that $\pi_1[U]=X$;  in which case
$U\in\Cal U$ and $U\subseteq U_0\setminus D$.   As $U$ is arbitrary, $D\in\Cal D'$;
as $D$ is arbitrary, $\Cal D\subseteq\Cal D'$.\ \Qed

$\Cal D$ is cofinal with $\Cal D'$.   \Prf\ Let $D\in\Cal D'$.
For each $H\in\Cal H\setminus\{\emptyset\}$, $X\times H\in\Cal U$, so
there is a $U_H\in\Cal U$ such that $U_H\subseteq(X\times H)\setminus D$;
try
$D_1=(X\setminus Y)\setminus\bigcup_{H\in\Cal H\setminus\{\emptyset\}}U_H$.
$D_1$ is closed.   Since $\Cal U\subseteq\Cal U_0\subseteq\Cal W$,
$(X\times Y)\setminus D_1\in\Cal W$.   If $x\in X$, then $D_1[\{x\}]$ is a
closed set not including any member of the $\pi$-base $\Cal H$, so is
nowhere dense in $Y$;  thus $D_1\in\Cal D_0$ and (being closed) belongs to
$\Cal D$.   Of course $D\subseteq D_1$.   As
$D$ is arbitrary, $\Cal D$ is cofinal with $\Cal D'$.\ \Qed

\medskip

\quad{\bf (v)} Because $\Cal U$ is countable, 526Hd tells us that
$\Cal D'\prT\CalNwd$, where $\CalNwd$ is the ideal of nowhere dense
subsets of $\NN$;  while of course $\Cal D\equivT\Cal D'$ (513E(d-ii)).
Let $\phi:\Cal L\to\Cal N(\mu)\times\Cal D^{\Bbb N}$ be such that if
$\phi(V)=(E,\sequencen{D_n})$ then
$V\subseteq(E\times Y)\cup\bigcup_{n\in\Bbb N}D_n$;
such a function exists by (iii), and is evidently a Tukey function.

Note that the measure algebra of $\mu$, being a totally finite measure
algebra with countable Maharam type, can be regularly embedded in the
measure algebra of Lebesgue measure on either $[0,1]$ or on $\Bbb R$.
Consequently $\Cal N(\mu)\prT\Cal N$ (524K) and

\Centerline{$\Cal L\prT\Cal N(\mu)\times\Cal D^{\Bbb N}
\prT\Cal N\times\CalNwd^{\Bbb N}\cong\Cal N\times\CalNwd$}

\noindent (513Eg, 526Ha).   Accordingly

$$\eqalignno{(\Cal L,\subseteq,\Cal L)
&\equivGT(\Cal L,\subseteq^{\strprime},[\Cal L]^{\le\omega})\cr
\displaycause{513Id}
&\prGT(\Cal N\times\CalNwd,\le^{\strprime},
  [\Cal N\times\CalNwd]^{\le\omega})\cr
\displaycause{512Gb}
&\equivGT(\Cal N,\subseteq^{\strprime},[\Cal N]^{\le\omega})
  \times(\CalNwd,\subseteq^{\strprime},[\CalNwd]^{\le\omega})\cr
\displaycause{512Hd}
&\equivGT(\Cal N,\subseteq\Cal N)
  \times(\Cal M,\subseteq,\Cal M)\cr
\displaycause{513Id, 526Hb, 512Hb}
&\prGT(\Cal N,\subseteq,\Cal N)\times(\Cal N,\subseteq,\Cal N)\cr
\displaycause{522P}
&\equivGT(\Cal N,\subseteq,\Cal N)\cr}$$

\noindent (513Eh), and $\Cal L\prT\Cal N$.

\medskip

{\bf (c)} This proves the theorem when $X$ is compact and metrizable,
$\mu$ is totally finite and $Y$ is a Baire space.   Now suppose that $Y$
is still
a Baire space, while $(X,\mu)$ is any totally finite quasi-Radon measure
space with countable Maharam type.

\medskip

\quad{\bf (i)} There is a compact metrizable Radon measure space
$(Z,\lambda)$ such that $\lambda$ and $\mu$ have isomorphic measure
algebras.   \Prf\ Because the measure
algebra $(\frak A,\bar\mu)$ of $\mu$ is totally finite, it is isomorphic
to the simple product of a countable family
$\familyiI{(\frak A_i,\bar\mu_i)}$ of homogeneous
totally finite measure algebras (332B).   Because $\mu$ has countable
Maharam type, every $\frak A_i$ is either $\{0\}$, $\{0,1\}$ or
isomorphic to the measure algebra of Lebesgue
measure on an interval;  in any case it is isomorphic to the measure
algebra of a compact Radon measure space $(Z_i,\lambda_i)$.   Take
$(Z',\lambda')$ to be the direct sum
of the measure spaces $\familyiI{(Z_i,\lambda_i)}$;  then the measure
algebra of $(Z',\lambda')$ is isomorphic to $\frak A$.   If we give $Z'$
its disjoint union topology, it is
a locally compact $\sigma$-compact metrizable space, and its one-point
compactification $Z$ is second-countable, therefore metrizable;  taking
$\lambda$ to be the trivial extension
of $\lambda'$, $(Z,\lambda)$ is a compact metrizable Radon measure space
with measure algebra $(\frak B,\bar\lambda)\cong(\frak A,\bar\mu)$.\
\Qed

\medskip

\quad{\bf (ii)} Let $f:X\to Z$ be an \imp\ function inducing an
isomorphism $\pi:\frak B\to\frak A$ of the measure algebras (416Wb).
By 418J, $f$ is almost
continuous, so there is a Borel measurable function which is equal
almost everywhere to $f$;  this function will still represent $\pi$, so
we may suppose that $f$ itself is Borel
measurable.   Now if $V\in\Cal B(X)\tensorhat\Cal B(Y)$, there is a
$V'\in\Cal B(Z)\tensorhat\Cal B(Y)$ such that
$\{x:V[\{x\}]\ne V'[\{f(x)\}]\}\in\Cal N(\lambda)$.
\Prf\ Let $\tilde{\Cal V}$ be the family of subsets $V$ of $X\times Y$
for which there is a $V'\in\Cal B(Z)\tensorhat\Cal B(Y)$ such that
$\{x:V[\{x\}]\ne V'[\{f(x)\}]\}\in\Cal N(\lambda)$.
Then $\tilde{\Cal V}$ is a $\sigma$-algebra.   If $E\in\Cal B(X)$ and
$H\in\Cal B(Y)$,
then there must be an $F\in\Cal B(Z)$ such that
$F^{\ssbullet}=\pi E^{\ssbullet}$ in $\frak B$, so that
$E\symmdiff f^{-1}[F]\in\Cal N(\mu)$;  now $F\times H$ witnesses that
$E\times H$
belongs to $\tilde{\Cal V}$.   Accordingly $\tilde{\Cal V}$ must include
$\Cal B(X)\tensorhat\Cal B(Y)$.\ \Qed

\medskip

\quad{\bf (iii)} We know that $\Cal N(\mu)\prT\Cal N(\lambda)$
(524Sa), so there is a Tukey function
$\theta:\Cal N(\mu)\to\Cal N(\lambda)$.   Set
$\Cal L'=\Cal N(\lambda)\ltimes_{\Cal B(Z)\tensorhat\Cal B(Y)}\Cal M(Y)$.
Define a function $\phi:\Cal L\to\Cal L'$ as
follows.   First, for $V\in\Cal L$, choose
$\phi_0(V)\in\Cal L\cap(\Cal B(X)\tensorhat\Cal B(Y))$ including $V$
((a) above).   Next, by (ii) here, we can
choose $\phi_1(V)\in\Cal B(Z)\tensorhat\Cal B(Y)$
such that $N_V=\{x:\phi_0(V)[\{x\}]\ne\phi_1(V)[\{f(x)\}]\}$ belongs to
$\Cal N(\mu)$.   Set
$F=\{z:z\in Z$, $\phi_1(V)[\{z\}]$ is not meager$\}$;
then $F$ is a Borel set, by 4A3Sa, and
$f^{-1}[F]\subseteq N_V\cup\{x:\phi_0[\{x\}]\notin\Cal M(Y)\}\in\Cal N(\mu)$;
so $F\in\Cal N(\lambda)$ and $\phi_1(V)\in\Cal L'$.
Finally, set
$\phi(V)=(\theta(N_V)\times Y)\cup\phi_1(V)\in\Cal L'$.

$\phi$ is a Tukey function from $\Cal L$ to $\Cal L'$.   \Prf\ Take
$W\in\Cal L'$ and consider $\Cal E=\{V:V\in\Cal L$, $\phi(V)\subseteq
W\}$.
If $Y=\emptyset$ then of course $\Cal E$ is bounded above in $\Cal L$.
Otherwise, $N^*=\{z:W[\{z\}]=Y\}$ must be negligible, and
$\theta(N_V)\subseteq N^*$ for every $V\in\Cal E$;
because $\theta$ is a
Tukey function, $\tilde N=\bigcup\{N_V:V\in\Cal E\}$ is negligible.
Take $W_1\in\Cal L'\cap(\Cal B(Z)\tensorhat\Cal B(Y))$ including $W$,
and set $\tilde W=\{(x,y):(f(x),y)\in W_1\}$;
then $\tilde W\in\Cal B(X)\tensorhat\Cal B(Y)$ because $f$ is Borel
measurable.   As

\Centerline{$\{x:\tilde W[\{x\}]\notin\Cal M(Y)\}
=f^{-1}\{z:W_1[\{z\}]\notin\Cal M(Y)\}$}

\noindent is negligible, $\tilde W\in\Cal L$.   So $V_0=(N^*\times
Y)\cup\tilde W$ belongs to $\Cal L$.   Now take any $V\in\Cal E$.   If
$x\in X\setminus N^*$, then $x\notin N_V$, so

\Centerline{$V[\{x\}]\subseteq\phi_0(V)[\{x\}]
=\phi_1(V)[\{f(x)\}]\subseteq W[\{f(x)\}]\subseteq W_1[\{f(x)\}]
=\tilde W[\{x\}]=V_0[\{x\}]$.}

\noindent This shows that $V\subseteq V_0$;  as $V$ is arbitrary, $V_0$
is an upper bound for $\Cal E$ in $\Cal L$;  as $W$ is arbitrary, $\phi$
is a Tukey function.\ \Qed

\medskip

\quad{\bf (iv)} By (b), we know that $\Cal L'\prT\Cal N$, so (iii)
tells us that $\Cal L\prT\Cal N$, and the theorem is true in this case
also.

\medskip

{\bf (d)} We are nearly home.   If $Y$ is a Baire space and $(X,\mu)$ is
a $\sigma$-finite quasi-Radon measure space with countable Maharam type,
which is not totally finite, then
there is a measurable function $f:X\to\ooint{0,\infty}$ such that
$\int fd\mu=1$ (215B(ix)).   Let $\nu$ be the indefinite-integral
measure defined by $f$.   Then $\nu$ has the same
negligible sets as $\mu$ (234Lc),
and is a quasi-Radon measure (415Ob), so

\Centerline{$\Cal L=\Cal N(\nu)\ltimes_{\Cal B(X\times Y)}\Cal M(Y)
\prT\Cal N$,}

\noindent by (c).

\medskip

{\bf (e)} Finally, suppose that $Y$ is not a Baire space.   In this
case, let $H^*$ be the minimal
comeager regular open subset of $Y$ (4A3Ra again), and set
$\Cal L^*=\Cal N(\mu)\ltimes_{\Cal B(X\times H^*)}\Cal M(H^*)$.
Then $\Cal L\prT\Cal L^*$.
\Prf\ For every $V\in\Cal L$, let
$V'$ be such that
$V\subseteq V'\in\Cal B(X\times Y)\cap(\Cal N(\mu)\ltimes\Cal M(Y))$, and set
$\phi(V)=V'\cap(X\times H^*)$.   Then
$\phi(V)$ is a Borel subset of $X\times H^*$,
and $\phi(V)[\{x\}]=V'[\{x\}]\cap H^*$ is meager in $H^*$ whenever
$V'[\{x\}]$ is meager in $Y$, so $\phi(V)\in\Cal L^*$.   To see that
$\phi:\Cal L:\to\Cal L^*$ is
a Tukey function, take any $W\in\Cal L^*$.   There is a Borel set
$W'\in\Cal L^*$ including $W$, and now $V'=W'\cup(X\times(Y\setminus H^*))$
is a Borel subset of $X\times Y$;  since $V'[\{x\}]$ is meager in $Y$
whenever $W'[\{x\}]$ is meager in $H^*$, $V'\in\Cal L$.   Of course $V'$
is an upper bound
of $\{V:V\in\Cal L$, $\phi(V)\subseteq W\}$;  as $W$ is arbitrary,
$\phi$ is a Tukey function and $\Cal L\prT\Cal L^*$.\ \Qed

By (d), $\Cal L\prT\Cal N$ in this case also, and the proof is complete.
}%end of proof of 527J

\leader{527K}{Corollary}
$\Cal N\ltimes_{\Cal B(\BbbR^2)}\Cal M\equivT\Cal N$.

\proof{ By 527J, $\Cal N\ltimes_{\Cal B(\BbbR^2)}\Cal M\prT\Cal N$.   On
the other hand, $E\mapsto E\times\Bbb R$ is a Tukey function from
$\Cal N$ to $\Cal N\ltimes_{\Cal B(\BbbR^2)}\Cal M$, so
$\Cal N\prT\Cal N\ltimes_{\Cal B(\BbbR^2)}\Cal M$.
}%end of proof of 527K

\leader{527L}{}\cmmnt{ There are some interesting questions concerning
the saturation of skew products.
Here I give two results which will be useful later.

\medskip

\noindent}{\bf Theorem} Let $X$ be a set, $\Sigma$ a $\sigma$-ideal of
subsets of $X$, and $\Cal I\normalsubgroup\Cal PX$ a $\sigma$-ideal;
suppose that $\Sigma/\Sigma\cap\Cal I$ is
ccc.   Let $(Y,\Tau,\nu)$ be a $\sigma$-finite measure space.   Then
$(\Sigma\tensorhat\Tau)
/(\Sigma\tensorhat\Tau)\cap(\Cal I\ltimes\Cal N(\nu))$ is ccc.

\proof{{\bf (a)} The case $\nu Y=0$ is trivial, as then
$\Cal I\ltimes\Cal N(\nu)=\Cal P(X\times Y)$.
Otherwise, there is a probability measure on $Y$ with the same
domain and null ideal as $\nu$ (215B(vii)),
so we may suppose that $\nu Y=1$.

\medskip

{\bf (b)} The family $\Cal W$ of sets $W\subseteq X\times Y$ such that
$W[\{x\}]\in\Tau$ for every $x\in X$ and $x\mapsto\nu W[\{x\}]$ is
$\Sigma$-measurable is a Dynkin class (definition:  136A),
and contains $E\times F$
whenever $E\in\Sigma$ and $F\in\Tau$;  by the Monotone Class Theorem
(136B) it includes $\Sigma\tensorhat\Tau$.

\medskip

{\bf (c)} Now suppose that
$\ofamily{\xi}{\omega_1}{W_{\xi}}$ is a disjoint family in
$\Sigma\tensorhat\Tau$.
For $n\in\Bbb N$ and $\xi<\kappa$ set

\Centerline{$E_{n\xi}=\{x:\nu W_{\xi}[\{x\}]\ge 2^{-n}\}$;}

\noindent then
$\#(\{\xi:x\in E_{n\xi}\})\le 2^{-n}$ for every $x\in X$.
It follows that $A_n=\{\xi:\xi<\omega_1$, $E_{n\xi}\notin\Cal I\}$ is
countable.   \Prf\Quer\ Otherwise, write $\frak A$ for the ccc
algebra $\Sigma/\Sigma\cap\Cal I$, and
$a_{\xi}=E_{n\xi}^{\ssbullet}$ for $\xi<\omega_1$.   Then $\frak A$ is
Dedekind complete;  set $b_{\xi}=\sup_{\xi\le\eta<\omega_1}a_{\eta}$ for
$\xi<\omega_1$ and $b=\inf_{\xi<\omega_1}b_{\xi}$.   Because $\frak A$ is
ccc, there is a $\zeta<\omega_1$ such that $b=b_{\xi}$ for every
$\xi\ge\zeta$;  because $A_n$ is uncountable, $b\ne 0$.
Choose $\sequence{i}{c_i}$ and $\sequence{i}{\xi_i}$ inductively such that
$c_0=b$ and, given that $0\ne c_i\Bsubseteq b$,
$\xi_i$ is to be such
that $c_{i+1}=a_{\xi_i}\Bcap c_i\ne 0$ and $\xi_i>\xi_j$ for every $j<i$.

\noindent Now $\inf_{i\le 2^{n}}a_{\xi_i}\Bsupseteq c_{2^n+1}$
is non-zero, so there is an
$x\in\bigcap_{i\le 2^n}E_{n\xi_i}$;  but this is impossible.\ \Bang\Qed

\medskip

{\bf (d)} This is true for every $n\in\Bbb N$, so there is a $\xi<\omega_1$
such that $\xi\notin A_n$ for every $n$, that is, $E_{n\xi}\in\Cal I$ for
every $n$.   But in this case

\Centerline{$\{x:W_{\xi}[\{x\}]\notin\Cal N(\nu)\}
=\bigcup_{n\in\Bbb N}E_{n\xi}$}

\noindent belongs to $\Cal I$ and $W_{\xi}\in\Cal I\ltimes\Cal N(\nu)$.
As $\ofamily{\xi}{\omega_1}{W_{\xi}}$ is arbitrary,
$(\Sigma\tensorhat\Tau)\cap(\Cal I\ltimes\Cal N(\nu))$ is
$\omega_1$-saturated in $\Sigma\tensorhat\Tau$ and
$(\Sigma\tensorhat\Tau)/
(\Sigma\tensorhat\Tau)\cap(\Cal I\ltimes\Cal N(\nu))$ is ccc (316C).
}%end of proof of 527L

\leader{527M}{}\cmmnt{ The next result
provides me with an opportunity to introduce a concept which will
be needed in \S546.

\medskip

\noindent}{\bf Definition} A Boolean algebra $\frak A$ is {\bf harmless}
(cf.\ {\smc Just 92}) if it is ccc
and whenever $\frak B$ is a countable subalgebra of
$\frak A$, there is a countable subalgebra $\frak C$ of $\frak A$ such that
$\frak B\subseteq\frak C$ and $\frak C$ is regularly embedded in $\frak A$.

\leader{527N}{Lemma} (a) If $\frak A$ is a Boolean algebra and $\frak D$
is a harmless
order-dense subalgebra of $\frak A$, then $\frak A$ is harmless.

(b) If $\frak A$ is a Dedekind complete Boolean algebra, then it is
harmless iff
every order-closed subalgebra of $\frak A$ with countable Maharam type
has countable $\pi$-weight.

(c) For any set $I$, the regular open algebra $\RO(\{0,1\}^I)$
of $\{0,1\}^I$ is harmless, so the category algebra of $\{0,1\}^I$ is
harmless.
%also any product of spaces with countable \pi-weight

(d) If $\frak A$ has countable $\pi$-weight it is harmless.

(e) If $\frak A$ is a harmless Boolean algebra, $\frak B$ is a Boolean
algebra and $\pi:\frak A\to\frak B$ is a surjective order-continuous
Boolean homomorphism, then
$\frak B$ is harmless.   In particular,
any principal ideal of a harmless Boolean algebra is harmless.

\proof{{\bf (a)} By 513E(e-iii), $\frak A$ is ccc.
Let $\frak B$ be a countable subalgebra of $\frak A$.   For each
$b\in\frak B$ let
$D_b\subseteq\frak D$ be a countable set with supremum $b$ (313K, 316E).
Let $\frak D_0$
be the subalgebra of $\frak D$ generated by $\bigcup_{b\in\frak B}D_b$.
Then $\frak D_0$ is countable, so there is a countable subalgebra
$\frak D_1$ of $\frak D$, including $\frak D_0$, which is regularly
embedded in $\frak D$.   Let $\frak C$ be the subalgebra of $\frak A$
generated by $\frak B\cup\frak D_1$.   Then $\frak C$ is countable.   Now
every member of $\frak C$ is the supremum of the members of
$\frak D_1$ it includes.  \Prf\ Set

\Centerline{$C
=\{c:c\in\frak C$, $c=\sup\{d:d\in\frak D_1$, $d\Bsubseteq c\}
=\inf\{d:d\in\frak D_1$, $c\Bsubseteq d\}\}$.}

\noindent Then $C$ is
closed under union (use 313Bd) and complementation (313A), and includes
$\frak B\cup\frak D_1$, so $C=\frak C$.\ \Qed

It follows that $\frak C$ is regularly embedded in $\frak A$, because if
$C\subseteq\frak C$ has supremum $1$ in $\frak C$ then
$\bigcup_{c\in C}\{d:d\in\frak D_1$, $d\Bsubseteq c\}$ must have supremum
$1$ in $\frak C$ and therefore in $\frak D_1$ (because
$\frak D_1\subseteq\frak C$) and in $\frak D$ (because $\frak D_1$ is
regularly embedded in $\frak D$) and in $\frak A$ (because $\frak D$ is
regularly embedded in $\frak A$).   But this means that $\sup C$ must be
$1$ in $\frak A$.   As $C$ is arbitrary, $\frak C$ is
regularly embedded.   As $\frak B$ is arbitrary, $\frak A$ is harmless.

\medskip

{\bf (b)(i)} Suppose that $\frak A$ is harmless and that
$\frak B\subseteq\frak A$ is
an order-closed subalgebra of countable Maharam type.
Let $B\subseteq\frak B$ be a
countable set which $\tau$-generates $\frak B$,
and $\frak B_0$ the algebra generated
by $B$;  let $\frak C$ be a countable subalgebra of $\frak A$,
including $\frak B_0$,
which is regularly embedded in $\frak A$.   Let $\frak D$ be the set

\Centerline{$\{d:d\in\frak A$, $d=\sup\{c:c\in\frak C$, $c\Bsubseteq d\}
=\inf\{c:c\in\frak C$, $d\Bsubseteq c\}\}$.}

\noindent Then $\frak D$ is an order-closed subalgebra of $\frak A$.
\Prf\ As in (a) just above, it is a subalgebra.
If $D\subseteq\frak D$ is a non-empty set with supremum
$a$ in $\frak A$, set $C=\{c:c\in\frak C$, $c\Bsubseteq a\}$,
$C'=\{c:c\in\frak C$, $a\Bsubseteq c\}$.   Then
$a$ is an upper bound for $C$ and a lower bound for $C'$.   \Quer\ If either $a$ is
not the least upper bound of $C$, or $a$ is not the greatest lower bound of $C'$, then
$A=\{c'\Bsetminus c:c'\in C'$, $c\in C\}$ is a subset of $\frak C$ with a non-zero lower
bound in $\frak A$, so $A$ has a non-zero lower bound $c^*$ in $\frak C$.
Now if $d\in D$, $c\in\frak C$ and $c\Bsubseteq d$, then $c\in C$ so $c\Bcap c^*=0$;
as $d=\sup\{c:c\in\frak C$, $c\subseteq d\}$, $d\Bcap c^*=0$.   This is true for every
$d\in D$, so $a\Bcap c^*=0$ and $1\Bsetminus c^*\in C'$;  but $c^*$ was chosen to be
included in every member of $C'$.\ \BanG\  Thus $a\in\frak D$;  as $D$ is arbitrary,
$\frak D$ is order-closed in $\frak A$.\ \Qed

Now $B\subseteq\frak C\subseteq\frak D$.
As $\frak B$ is regularly embedded in $\frak A$ (314Ga),
$\frak B\cap\frak D$ is an order-closed subalgebra of $\frak B$
including $B$, so is the whole of $\frak B$,
and $\frak B\subseteq\frak D$.   It
follows that $\pi(\frak B)\le\pi(\frak D)$ (514Eb).   But
$\frak C$ is countable and order-dense in $\frak D$,
so $\pi(\frak D)$ and $\pi(\frak B)$ are countable.
As $\frak B$ is arbitrary, $\frak A$ satisfies the declared condition.

\medskip

\quad{\bf (ii)} Now suppose that $\frak A$ satisfies the condition.   Note first that
$\frak A$ is ccc.   \Prf\Quer\ Suppose, if possible, otherwise;  let
$\ofamily{\xi}{\omega_1}{a_{\xi}}$ be a disjoint family in
$\frak A\setminus\{0\}$.   Replacing $a_0$ by
$a_0\Bcup(1\Bsetminus\sup_{\xi<\omega_1}a_{\xi})$ if necessary, we
may suppose that $\sup_{\xi<\omega_1}a_{\xi}=1$.   The map
$I\mapsto\sup_{\xi\in I}a_{\xi}:\Cal P\omega_1\to\frak A$
is an injective order-continuous
Boolean homomorphism, so its image $\frak B$ is an order-closed subalgebra of
$\frak A$ isomorphic to $\Cal P\omega_1$.   Now
$\tau(\frak B)=\tau(\Cal P\omega_1)=\omega$ (514Ef, or otherwise),
but $\pi(\frak B)=\omega_1$;  which is supposed to be impossible.\
\Bang\Qed

If $\frak B$ is a countable subalgebra of $\frak A$,
let $\frak B_1$ be the order-closed
subalgebra of $\frak A$ which it generates.   Then
$\tau(\frak B_1)\le\omega$ so
$\pi(\frak B_1)\le\omega$, and there is a countable subalgebra $\frak C$ of
$\frak B_1$ which is order-dense in $\frak B_1$;  of course we may suppose that
$\frak B\subseteq\frak C$.   Now the identity maps from $\frak C$ to $\frak B_1$ and
from $\frak B_1$ to $\frak A$ are both order-continuous, so their composition also is,
and $\frak C$ is regularly embedded in $\frak A$.   As $\frak B$ is arbitrary,
$\frak A$ is harmless.

\medskip

{\bf (c)} All regular open algebras are Dedekind complete.
If $\frak B\subseteq\RO(\{0,1\}^I)$ is an order-closed subalgebra with
countable
Maharam type, let $\sequencen{G_n}$ be a sequence in $\frak B$ which
$\tau$-generates $\frak B$.   Every regular open subset of $\{0,1\}^I$ is
determined by coordinates in some countable set (4A2E(b-i)), so there is a
countable $J\subseteq I$ such that every $G_n$ is determined by coordinates
in $J$.   Let $\pi_J:\{0,1\}^I\to\{0,1\}^J$ be the restriction map;  then
we have an injective order-continuous Boolean homomorphism
$H\mapsto\pi_J^{-1}[H]:\RO(\{0,1\}^J)\to\RO(\{0,1\}^I)$
(4A2B(f-iii)).   Let $\frak C$ be the
image of this homomorphism, so that $\frak C$ is an order-closed subalgebra of
$\RO(\{0,1\}^I)$.   If $H_n=\pi_J[G_n]$
then $H_n$ is regular and open for each $n$ (4A2B(f-iii) again), so
$G_n=\pi_J^{-1}[H_n]\in\frak C$;  accordingly $\frak B\subseteq\frak C$.   Now
$\frak B$ is an order-closed subalgebra of $\frak C$ so

\Centerline{$\pi(\frak B)\le\pi(\frak C)=\pi(\{0,1\}^J)\le\omega$.}

\noindent As $\frak B$ is arbitrary, $\RO(\{0,1\}^I)$ satisfies the 
condition of (b) and is harmless.

Of course it follows at once that the category algebra is harmless, because
it is isomorphic to the regular open algebra (514If-514Ig).

\medskip

{\bf (d)} Let $D$ be a countable order-dense set in $\frak A$.   If $\frak B$ is a
countable subalgebra of $\frak A$, let $\frak C$ be the subalgebra of $\frak A$
generated by $D\cup\frak B$;  then $\frak C$ is countable, includes $\frak B$ and is
order-dense, therefore regularly embedded in $\frak A$.   As $\frak B$ is arbitrary,
$\frak A$ is harmless.

\medskip

{\bf (e)} Let $\frak D\subseteq\frak B$ be a countable subalgebra.   Because
$\pi[\frak A]=\frak B$, there is a countable subalgebra $\frak C$ of $\frak A$ such
that $\pi[\frak C]=\frak D$.   Let $\frak C_1\supseteq\frak C$ be a countable
regularly embedded subalgebra of $\frak A$.   Then
$\frak D_1=\pi[\frak C_1]$ is regularly
embedded in $\frak B$.   \Prf\ Let $D\subseteq\frak D_1$ be a non-empty set such
that $1$ is not the least upper bound of $D$ in $\frak B$.   Set
$C=\frak C_1\cap\pi^{-1}[D\cup\{0\}]$;  then $1$ is not the least upper bound of
$\pi[C]$ in $\frak B$, so (because $\pi$ is order-continuous)
$1$ is not the least upper bound of $C$ in $\frak A$.
Because $\frak C_1$ is regularly embedded in $\frak A$, there is a non-zero
$c_0\in\frak C_1$ such that $c_0\Bcap c=0$ for every $c\in C$.   In particular,
$c_0\notin C$ and $\pi c_0\ne 0$.   But we also have $\pi c\Bcap\pi c_0=0$ for every
$c\in C$, that is, $d\Bcap\pi c_0=0$ for every $d\in D$, and $1$ is not the least
upper bound of $D$ in $\frak D_1$.   As $D$ is arbitrary, $\frak D_1$ is regularly
embedded.\ \QeD\   Of course $\frak D_1$ is countable.   As $\frak D$ is arbitrary,
$\frak B$ is harmless.

If $c\in\frak A$ then $a\mapsto a\Bcap c$ is an order-continuous homomorphism onto the
principal ideal $\frak A_c$ generated by $c$, so $\frak A_c$ is harmless.
}%end of proof of 527N

\leader{527O}{Theorem} Let $(X,\Sigma,\mu)$ be a $\sigma$-finite measure
space and
$Y$ a topological space such that the category algebra $\frak G$ of $Y$ is
harmless.   Write $\Cal L$ for
 $(\Sigma\tensorhat\Cal B(Y))\cap(\Cal N(\mu)\ltimes\Cal M(Y))$ and
$\frak A$ for the measure algebra of $\mu$.   Then
$\frak C=(\Sigma\tensorhat\Cal B(Y))/\Cal L$
is ccc, and is isomorphic to the Dedekind completion of the
free product $\frak A\otimes\frak G$.
If neither $\frak A$ nor $\frak G$ is trivial,
the isomorphism corresponds to embeddings
$E^{\ssbullet}\mapsto(E\times Y)^{\ssbullet}:\frak A\to\frak C$ and
$F^{\ssbullet}\mapsto(X\times F)^{\ssbullet}:\frak B\to\frak C$.

\proof{ Write $\frak S$ for the topology of $Y$.

\medskip

{\bf (a)} Let $\Cal W$ be the family of all sets of the form
$\bigcup_{n\in\Bbb N}E_n\times H_n$, where $E_n\in\Sigma$ and $H_n\subseteq Y$ is
open for every $n$.   Then for any $W\in\Cal W$ there is a $W'\in\Cal W$ such that
$W'\symmdiff((X\times Y)\setminus W)\in\Cal L$.   \Prf\ Express
$W$ as $\bigcup_{n\in\Bbb N}E_n\times H_n$ where $E_n\in\Sigma$ and $H_n\in\frak S$
for each $n$.   Let
$\frak D$ be the order-closed subalgebra of $\frak G$ generated by
$\{H_n^{\ssbullet}:n\in\Bbb N\}$.   Because $\frak G$ is harmless and Dedekind
complete, $\pi(\frak D)\le\omega$ (527Nb);  let $\sequencen{G_n}$
be a sequence in $\frak S$
such that $\{G_n^{\ssbullet}:n\in\Bbb N\}$ is a $\pi$-base for
$\frak D$;  we may suppose that any non-empty open subset of any $G_n$ is non-meager.
Let $\frak S_1$ be the second-countable topology on $Y$ generated by
$\{H_n:n\in\Bbb N\}\cup\{G_n:n\in\Bbb N\}$, and
$\Cal B_1(Y)\subseteq\Cal B(Y)$ the corresponding Borel
$\sigma$-algebra.   Then $V^{\ssbullet}\in\frak D$ for every
$V\in\frak S_1$, because
$V$ is the union of a countable family of sets all with images in
$\frak D$.   If $V\in\frak S_1$ is dense for $\frak S_1$, and
$n\in\Bbb N$ is such that $G_n$ is non-empty, $V\cap G_n\ne\emptyset$ so
$V^{\ssbullet}\Bcap G_n^{\ssbullet}\ne 0$,
by the choice of the $G_n$.   But this means that $V^{\ssbullet}=1$, that
is, $V$ is comeager for the original topology of $Y$.

Now $W$ and $(X\times Y)\setminus W$ belong to
$\Sigma\tensorhat\Cal B_1(Y)$.   By 527I, there are $W'$ and
$\sequencen{D_n}$ such that

\inset{$((X\times Y)\setminus W)\symmdiff W'
\subseteq\bigcup_{n\in\Bbb N}D_n$,

$W'$ is expressible as $\bigcup_{n\in\Bbb N}F_n\times V_n$ where
$F_n\in\Sigma$ and $V_n\in\frak S_1$ for every $n$,

every $D_n$ belongs to $\Sigma\tensorhat\Cal B_1(Y)$,

for every $x\in X$ and $n\in\Bbb N$,
$D_n[\{x\}]$ is closed and nowhere dense for $\frak S_1$.}

\noindent Evidently $W'\in\Cal W$;  but we have just seen that sets which are closed
and nowhere dense for $\frak S_1$ are meager for $\frak S$.   So every $D_n$ belongs
to $\Cal L$ and $((X\times Y)\setminus W)\symmdiff W'\in\Cal L$.\ \Qed

\medskip

{\bf (b)} It follows (as in the proof of 527I) that
$\Cal V=\{W\symmdiff D:W\in\Cal W$, $D\in\Cal L\}$ is a $\sigma$-algebra of sets, and as
$E\times H\in\Cal W$ for every $E\in\Sigma$ and $H\in\frak S$,
$\Cal V=\Sigma\tensorhat\Cal B(Y)$.

\medskip

{\bf (c)} $\frak C$ is ccc.
\Prf\Quer\ Otherwise, there is a disjoint family $\ofamily{\xi}{\omega_1}{e_{\xi}}$
in $\frak C\setminus\{0\}$.   For each $\xi<\omega_1$, there is a
$V_{\xi}\in(\Sigma\tensorhat\Cal B(Y))\setminus\Cal L$ such that
$V_{\xi}^{\ssbullet}=e_{\xi}$, and a $W_{\xi}\in\Cal W$ such that
$V_{\xi}\symmdiff W_{\xi}\in\Cal L$.   Express $W_{\xi}$ as
$\bigcup_{n\in\Bbb N}E_{\xi n}\times H_{\xi n}$;  as $W_{\xi}\notin\Cal L$, there must
be an $n_{\xi}$ such that $E_{\xi}=E_{\xi,n_{\xi}}\notin\Cal N(\mu)$ and
$H_{\xi}=H_{\xi,n_{\xi}}$ is non-meager.    Since the measure algebra of
$\mu$ satisfies Knaster's condition (525Tb), there is an uncountable
$A\subseteq\omega_1$ such that $E_{\xi}\cap E_{\eta}\notin\Cal N(\mu)$
for all $\xi$, $\eta\in A$;  because
$\frak G$ is ccc, there are distinct $\xi$, $\eta\in A$ such that
$H_{\xi}\cap H_{\eta}$ is non-meager.   But also

\Centerline{$(E_{\xi}\cap E_{\eta})\times(H_{\xi}\cap H_{\eta})
\subseteq W_{\xi}\cap W_{\eta}\in\Cal L$}

\noindent because
$(W_{\xi}\cap W_{\eta})^{\ssbullet}=e_{\xi}\Bcap e_{\eta}=0$.
So this is impossible.\ \Bang\Qed

Thus $\frak C$ is ccc.   As it is Dedekind $\sigma$-complete (314C), it is Dedekind
complete (316Fa).

\medskip

{\bf (d)} If either $\mu X=0$ or $Y$ is meager, then
$\frak A\otimes\frak G$ and $\frak C$ are trivially isomorphic, and we can
stop.   Otherwise,
the map $E\mapsto(E\times Y)^{\ssbullet}:\Sigma\to\frak C$ is a Boolean
homomorphism with kernel $\Sigma\cap\Cal N(\mu)$, so induces a Boolean
homomorphism
$\pi_1:\frak A\to\frak C$.   Similarly, we have a Boolean homomorphism
$\pi_2:\frak G\to\frak C$ defined by setting
$\pi_2(F^{\ssbullet})=(X\times F)^{\ssbullet}$ for $F\in\Cal B(Y)$.
These now give
us a Boolean homomorphism $\phi:\frak A\otimes\frak G\to\frak C$ defined
by saying that

\Centerline{$\psi(E^{\ssbullet}\otimes F^{\ssbullet})
=\pi_1(E^{\ssbullet})\Bcap\pi_2(F^{\ssbullet})=(E\times F)^{\ssbullet}$}

\noindent for $E\in\Sigma$ and $F\in\Cal B(Y)$ (315Jb).   If
$E\in\Sigma\setminus\Cal N(\mu)$ and $F\in\Cal B(Y)\setminus\Cal M(X)$,
then $E\times F\notin\Cal L$;  so $\phi$ is injective
(use 315Kb).   If $c\in\frak C$ is
non-zero, it is expressible as $W^{\ssbullet}$ for some
$W\in\Cal W\setminus\Cal L$;
there must now be $E\in\Sigma$, $F\in\Cal B(Y)$ such that
$E\times F\subseteq W$ and $E\times F\notin\Cal L$, so that
$\phi(E^{\ssbullet}\otimes F^{\ssbullet})$ is
non-zero and included in $w$.
Thus $\phi[\frak A\otimes\frak G]$ is isomorphic to
$\frak A\otimes\frak G$ and is an order-dense
subalgebra of the Dedekind complete Boolean algebra $\frak C$;
it follows that
$\frak C$ can be identified with the Dedekind completion of
$\frak A\otimes\frak G$.
}%end of proof of 527O

\exercises{\leader{527X}{Basic exercises $\pmb{>}$(a)}
%\sqheader 527Xa
Show that there is a set belonging to $\Cal N\ltimes\Cal N$ which has
full outer measure for Lebesgue measure in the plane.   \Hint{enumerate
the compact non-negligible subsets of the plane as
$\ofamily{\xi}{\frak c}{K_{\xi}}$ (4A3Fa);  note that the projection
$L_{\xi}$ of $K_{\xi}$ onto the first
coordinate is always non-negligible, therefore uncountable, therefore of
cardinal $\frak c$ (423K);
choose $s_{\xi}\in L_{\xi}\setminus\{s_{\eta}:\eta<\xi\}$ and
$t_{\xi}\in K_{\xi}[\{s_{\xi}\}]$ for each $\xi$;  consider
$\{(s_{\xi},t_{\xi}):\xi<\frak c\}$.}
%527B

\sqheader 527Xb Show that there is a unique construction of iterated
skew products $\Cal I_0\ltimes\Cal I_1\ltimes\ldots\ltimes\Cal I_n$ such
that

(i) whenever $X_0,\ldots,X_n$ are sets and $\Cal I_j$ is an ideal of
subsets of $X_j$ for every $j$, then $\Cal I_0\ltimes\ldots\ltimes\Cal
I_n$ is an ideal of subsets of
$X_0\times\ldots\times X_n$;

(ii) whenever $X_0,\ldots,X_n$ are sets, $\Cal I_j$ is an ideal of
subsets of $X_j$ for every $j$, and $k<n$, then the natural
identification of $X_0\times\ldots\times X_n$
with $(X_0\times\ldots\times X_k)\times(X_{k+1}\times\ldots\times X_n)$
identifies $\Cal I_0\ltimes\ldots\ltimes\Cal I_n$ with
$(\Cal I_0\ltimes\ldots\ltimes I_k)\ltimes(\Cal
I_{k+1}\ltimes\ldots\ltimes I_n)$ as defined in 527B.
%527B

\spheader 527Xc Complete the analysis in 527Bb by describing what
happens if one of $X$, $Y$ is empty or one of the ideals is not proper.
%527B

\spheader 527Xd\dvAnew{2010}
Let $X$ be a set, $\Sigma$ a $\sigma$-algebra of subsets of
$X$, and $\Cal I$ an ideal of subsets of $X$;  let $Y$ be a topological
space, $\Cal B$ its Borel $\sigma$-algebra, $\widehat{\Cal B}$ its
Baire-property algebra, and $\Cal M$ its meager ideal.   Show that
$\Cal I\ltimes_{\Sigma\tensorhat\Cal B}\Cal M
=\Cal I\ltimes_{\Sigma\tensorhat\widehat{\Cal B}}\Cal M$.
%527B

\sqheader 527Xe Let $Z$ be the Stone space of the measure algebra of
Lebesgue measure on $[0,1]$, and $f:Z\to[0,1]$ the canonical \imp\
continuous function (416V).
Let $F\subseteq[0,1]$ be a nowhere dense set which is not negligible,
and set $W=\{(x,z):x\in[0,1]$, $z\in Z$, $x+f(z)\in F\}$.   Show that
$W$ is a nowhere dense closed
set in $[0,1]\times Z$ but does not belong to
$\Cal M([0,1])\ltimes\Cal M(Z)$.
\Hint{meager subsets of $Z$ are negligible (321K).}
%527D

\sqheader 527Xf Suppose that $I$ and $J$ are sets,
$X=\{0,1\}^I$ and $Y=\{0,1\}^J$.   Show that
$\Cal M(X)\ltimes_{\Cal B(X\times Y)}\Cal M(Y)=\Cal M(X\times Y)$.
%527D

\spheader 527Xg Write $\frak X$ for the class of topological spaces
which have category algebras which are atomless and with countable
$\pi$-weight.   (i) Show that $\Bbb R$, with the right-facing Sorgenfrey
topology, belongs to $\frak X$.   (ii) Show that the split interval
(419L) belongs to $\frak X$.   (iii) Show that if the regular open
algebra of a topological space $X$ is atomless and has countable
$\pi$-weight, then $X\in\frak X$.   (iv) Show that any open subspace of
a space in $\frak X$ belongs to $\frak X$.   (v) Show that any dense
subspace of a space in $\frak X$ belongs to $\frak X$.
(vi) Show that any comeager subspace of a space in $\frak X$ belongs to
$\frak X$.
(vii) Show that the product of countably many spaces in $\frak X$
belongs to $\frak X$.
%527F

\spheader 527Xh Let $P$ be a partially ordered set.   Show that if
$\kappa\ge\cf P$ and $\lambda\le\add P$ then $P\prT[\kappa]^{<\lambda}$.
%527H

\spheader 527Xi Let $X$ be a topological space with a $\sigma$-finite
measure $\mu$ such that $\mu$ has countable Maharam type and every
measurable set can be expressed as the
symmetric difference of a Borel set and a negligible set.   Let
$Y$ be a topological space with a countable $\pi$-base.   Show that
$\Cal N(\mu)\ltimes_{\Cal B(X\times Y)}\Cal M(Y)\prT\Cal N(\mu)\times\Cal M$.
%527J

\leader{527Y}{Further exercises (a)}
%\spheader 527Ya
Show that
$\Cal I\ltimes\Cal J\ne\Cal I\rtimes\Cal J$ for any of the four cases in
which $\{\Cal I,\Cal J\}\subseteq\{\Cal M,\Cal N\}$.
%527B

\spheader 527Yb Let $X$ be a set, $\Sigma$ a $\sigma$-algebra of subsets of $X$ and
$\Cal I$ a $\sigma$-ideal of $\Sigma$ such that $\Sigma/\Cal I$ is ccc.
Let $(Y,\Tau,\nu)$ be a probability space.   Show that
$(\Sigma\tensorhat\Tau)/(\Sigma\tensorhat\Tau)\cap(\Cal I\ltimes\Cal N(\nu))$ is ccc.
\Hint{show that if $V\in\Sigma\tensorhat\Tau$ then $x\mapsto\nu V[\{x\}]$ is
$\Sigma$-measurable, and hence that there is no uncountable disjoint family
in $(\Sigma\tensorhat\Tau)\setminus(\Cal I\ltimes\Cal N(\nu))$.}
%527G

\spheader 527Yc\dvAnew{2010}
Let $(Y,\frak T)$ be a topological space.   Show that there
is a topology $\frak S$ on $Y$, coarser than $\frak T$, such that the
weight of $(Y,\frak S)$ is equal to the $\pi$-weight of $(Y,\frak T)$, and
the two topologies have the same nowhere dense sets, the same meager ideal
and the same Baire-property algebras.
%527J

\spheader 527Yd\discrversionA{\footnote{Revised 2010.}}{} Let
$\familyiI{\frak A_i}$ be any family of harmless
Boolean algebras all satisfying Knaster's condition,
and $\frak A$ their free product (315I).   Show that $\frak A$ is harmless.
%527N

\spheader 527Ye Let $\mu$ be a $\sigma$-finite Borel probability measure
on a topological space $X$, and $Y$ a topological
space such that its category algebra is harmless.   Show that
$\Cal B(X\times Y)/\Cal B(X\times Y)\cap(\Cal N(\mu)\ltimes\Cal M(Y))$ can
be identified with the Dedekind completion of
$\frak A\otimes\frak G$, where $\frak A$ is the measure algebra of $\mu$
and $\frak G$ is the category algebra of $Y$.
%527O
}%end of exercises

\endnotes{
\Notesheader{527} Skew products of ideals have been used many times for
special purposes, and we are approaching the point at which it would be
worth developing a general theory
of such products.   I am not really attempting to do this here, though
the language of 527B is supposed to point to the right questions.   My
primary aim in this section is
to show that $\Cal M\ltimes_{\Cal B(\BbbR^2)}\Cal N$ and
$\Cal N\ltimes_{\Cal B(\BbbR^2)}\Cal M$ are very different (527H, 527K).
Of course the difference appears
only when the continuum hypothesis is false (513Xf, 527Xh).

The version of the Kuratowski-Ulam theorem given in 527D is a natural
one from the point of view of this chapter, but you should be aware that
there are many more cases in which $\Cal M^*=\Cal M(X\times Y)$;  see 527Xf
and {\smc Fremlin Natkaniec \& Rec{\l}aw 00}.
The statement of 527J includes the phrase `quasi-Radon measure'.
Actually we do not really need either $\tau$-additivity or inner
regularity with respect to closed sets.
What we need is a measure $\mu$ such that $\Cal N(\mu)\prT\Cal N$ and
the Borel sets generate the measure algebra (527Xi).   The argument for
527J betrays its origin in the case
$X=Y=[0,1]$, which is of course also the natural home of
527C-527F.   %527C 527D 527E 527F
Some of the complications of the argument are due to its being written out
for spaces of countable $\pi$-weight;  an alternative approach would
start with a reduction to the case in which $Y$ is second-countable
(527Yc).

It is interesting that all four of the quotient algebras

\Centerline{$\Cal B(\BbbR^2)/\Cal B(\BbbR^2)\cap(\Cal M\ltimes\Cal M)$,
\quad$\Cal B(\BbbR^2)/\Cal B(\BbbR^2)\cap(\Cal M\ltimes\Cal N)$,}

\Centerline{$\Cal B(\BbbR^2)/\Cal B(\BbbR^2)\cap(\Cal N\ltimes\Cal M)$,
\quad$\Cal B(\BbbR^2)/\Cal B(\BbbR^2)\cap(\Cal N\ltimes\Cal N)$}

\noindent are ccc (see 527E, 527Yb, 527O, 527Bc and also 527L).
This should not be taken for
granted;  for a variety of examples of quotient algebras associated with
$\sigma$-ideals see {\smc Fremlin 03}.

}%end of notes
\discrpage

\leaveitout{Pawlikowski 86:  " \Cal R*\Cal C  can be completely embedded
in  \Cal C*\Cal R " (cor. 3.3)
}
