\frfilename{mt471.tex}
\versiondate{23.1.06}
\copyrightdate{2000}

\def\chaptername{Geometric measure theory}
\def\sectionname{Hausdorff measures}

\def\NN{\BbbN^{\Bbb N}}

\newsection{471}

I begin the chapter by returning to a class of measures which we have
not examined since
Chapter 26.  The primary importance of these measures is in studying the
geometry of Euclidean space;  in \S265 I looked briefly at their use in
describing surface measures, which will reappear in \S475.   Hausdorff
measures are also one of the basic tools in
the study of fractals, but for such applications I must refer you to
{\smc Falconer 90} and {\smc Mattila 95}.
All I shall attempt to do here is to indicate some of the principal
ideas which are applicable to general metric spaces, and to look at some
special properties of Hausdorff measures related to the concerns of this
chapter and of \S261.

\leader{471A}{Definition} Let $(X,\rho)$ be a metric space and
$r\in\ooint{0,\infty}$.
For $\delta>0$ and $A\subseteq X$, set

$$\eqalign{\theta_{r\delta}A
=\inf\{\sum_{n=0}^{\infty}(\diam D_n)^r:\sequencen{D_n}
&\text{ is a sequence of subsets of }X\text{ covering }A,\cr
&\qquad\qquad\qquad\quad
  \diam D_n\le\delta\text{ for every }n\in\Bbb N\}.\cr}$$

\noindent\cmmnt{(As in \S264, take $\diam\emptyset=0$ and
$\inf\emptyset=\infty$.) }It will be useful to note that every
$\theta_{r\delta}$ is an outer measure.   Now set

\Centerline{$\theta_rA=\sup_{\delta>0}\theta_{r\delta}A$}

\noindent for $A\subseteq X$;  $\theta_r$ also is an outer measure on
$X$\cmmnt{, as in
264B};  this is {\bf $r$-dimensional Hausdorff outer measure} on $X$.
Let $\mu_{Hr}$ be the measure defined by \Caratheodory's method from
$\theta_r$;  $\mu_{Hr}$ is {\bf $r$-dimensional Hausdorff measure} on
$X$.

\cmmnt{\medskip

\noindent{\bf Notation} It may help if I list some notation already used
elsewhere.   Suppose that $(X,\rho)$ is a metric space.    I write

\Centerline{$B(x,\delta)=\{y:\rho(y,x)\le\delta\}$,
\quad$U(x,\delta)=\{y:\rho(y,x)<\delta\}$}

\noindent for $x\in X$, $\delta\ge 0$;  recall that $U(x,\delta)$ is
open (2A3G).   For $x\in X$ and $A$, $A'\subseteq X$ I write

\Centerline{$\rho(x,A)=\inf_{y\in A}\rho(x,y)$,
\quad$\rho(A,A')=\inf_{y\in A,z\in A'}\rho(y,z)$;}

\noindent for definiteness, take $\inf\emptyset$ to be $\infty$, as
before.}%end of comment

\leader{471B}{Definition} Let $(X,\rho)$ be a metric space.   An outer
measure $\theta$ on $X$ is a {\bf metric outer measure} if
$\theta (A_1\cup A_2)=\theta A_1+\theta A_2$ whenever $A_1$,
$A_2\subseteq X$ and $\rho(A_1,A_2)>0$.

\leader{471C}{Proposition} Let $(X,\rho)$ be a metric space and $\theta$
a metric outer measure on $X$.  Let $\mu$ be the measure on $X$ defined
from $\theta$ by \Caratheodory's method.   Then $\mu$ is a topological
measure.

\proof{ (Compare 264E, part (b) of the proof.)  Let $G\subseteq X$ be
open, and $A$ any subset of $X$ such that $\theta A<\infty$.   Set

\Centerline{$A_n=\{x:x\in A,\,\rho(x,A\setminus G)\ge 2^{-n}\}$,}

\Centerline{$B_0=A_0$,
\quad$B_n=A_n\setminus A_{n-1}$ for $n>1$.}

\noindent Observe that $A_n\subseteq A_{n+1}$ for every $n$ and
$\bigcup_{n\in\Bbb N}A_n=\bigcup_{n\in\Bbb N}B_n=A\cap G$.   The point
is that if $m$, $n\in\Bbb N$ and $n\ge m+2$, and if $x\in B_m$ and
$y\in B_n$, then there is a $z\in A\setminus G$ such that
$\rho(y,z)<2^{-n+1}\le 2^{-m-1}$, while $\rho(x,z)$ must be at least
$2^{-m}$, so $\rho(x,y)\ge\rho(x,z)-\rho(y,z)\ge 2^{-m-1}$.   Thus
$\rho(B_m,B_n)>0$ whenever $n\ge m+2$.   It follows
that for any $k\ge 0$

\Centerline{$\sum_{m=0}^k\theta B_{2m}
=\theta(\bigcup_{m\le k}B_{2m})\le\theta(A\cap G)<\infty$,}

\Centerline{$\sum_{m=0}^k\theta B_{2m+1}
=\theta(\bigcup_{m\le k}B_{2m+1})\le\theta(A\cap G)<\infty$.}

\noindent Consequently $\sum_{n=0}^{\infty}\theta B_n<\infty$.

But now, given $\epsilon>0$, there is an $m$ such that
$\sum_{n=m}^{\infty}\theta B_m\le\epsilon$, so that

$$\eqalignno{\theta(A\cap G)+\theta(A\setminus G)
&\le\theta A_m+\sum_{n=m}^{\infty}\theta B_n+\theta(A\setminus G)\cr
&\le\epsilon+\theta A_m+\theta(A\setminus G)
=\epsilon+\theta(A_m\cup(A\setminus G))\cr
\noalign{\noindent (since $\rho(A_m,A\setminus G)\ge 2^{-m}$)}
&\le\epsilon+\theta A.\cr}$$

\noindent As $\epsilon$ is arbitrary,
$\theta(A\cap G)+\theta(A\setminus G)\le\theta A$.   As $A$ is
arbitrary, $G$ is measured by $\mu$;  as $G$ is arbitrary, $\mu$ is a
topological measure.
}%end of proof of 471C

\leader{471D}{Theorem} Let $(X,\rho)$ be a metric space and $r>0$.   Let
$\mu_{Hr}$ be $r$-dimensional Hausdorff measure on $X$, and $\Sigma$ its
domain;  write $\theta_r$ for $r$-dimensional Hausdorff outer measure on
$X$\cmmnt{, as defined in 471A}.

(a) $\mu_{Hr}$ is a topological measure.

(b) For every $A\subseteq X$ there is a G$_{\delta}$ set
$H\supseteq A$ such that $\mu_{Hr}H=\theta_rA$.

(c) $\theta_r$ is the outer measure defined from $\mu_{Hr}$\cmmnt{
(that is, $\theta_r$ is a regular outer measure)}.

(d) $\Sigma$ is closed under Souslin's operation.

(e) $\mu_{Hr}E=\sup\{\mu_{Hr}F:F\subseteq E$ is closed$\}$ whenever
$E\in\Sigma$ and $\mu_{Hr}E<\infty$.

(f) If $A\subseteq X$ and $\theta_rA<\infty$ then $A$ is separable and
the set of isolated points of $A$ is $\mu_{Hr}$-negligible.

(g) $\mu_{Hr}$ is atomless.

(h) If $\mu_{Hr}$ is totally finite it is a quasi-Radon measure.

\proof{{\bf (a)} The point is that $\theta_r$, as defined in 471A, is a
metric outer measure.   \Prf\ (Compare 264E, part (a) of the proof.)
Let $A_1$, $A_2$ be subsets of $X$ such that $\rho(A_1,A_2)>0$.   Of
course $\theta_r(A\cup B)\le\theta_rA+\theta_rB$, because $\theta_r$ is an
outer measure.   For the reverse inequality, we may suppose that
$\theta_r(A\cup B)<\infty$, so that $\theta_rA$ and $\theta_rB$ are both
finite.   Let $\epsilon>0$ and let $\delta_1$, $\delta_2>0$ be such that

\Centerline{$\theta_rA_1+\theta_rA_2
\le\theta_{r\delta_1}A_1+\theta_{r\delta_2}A_2+\epsilon$,}

\noindent defining the $\theta_{r\delta_i}$ as in 471A.
Set $\delta=\min(\delta_1,\delta_2,\bover12\rho(A_1,A_2))>0$ and let
$\sequencen{D_n}$ be a sequence of sets of diameter at most $\delta$,
covering $A_1\cup A_2$, and such that $\sum_{n=0}^{\infty}(\diam D_n)^r
\le\theta_{r\delta}(A_1\cup A_2)+\epsilon$.   Set

\Centerline{$K=\{n:D_n\cap A_1\ne\emptyset\}$,
\quad$L=\{n:D_n\cap A_2\ne\emptyset\}$.}

\noindent Because $\rho(x,y)>\diam D_n$ whenever $x\in A_1$, $y\in A_2$
and $n\in\Bbb N$,
$K\cap L=\emptyset$;  and of course $A_1\subseteq\bigcup_{n\in K}D_k$,
$A_2\subseteq\bigcup_{n\in L}D_n$.   Consequently

$$\eqalign{\theta_rA_1+\theta_rA_2
&\le\epsilon+\theta_{r\delta_1}A_1+\theta_{r\delta_2}A_2
\le\epsilon+\sum_{n\in K}(\diam D_n)^r+\sum_{n\in L}(\diam D_n)^r\cr
&\le\epsilon+\sum_{n=0}^{\infty}(\diam D_n)^r
\le 2\epsilon+\theta_{r\delta}(A_1\cup A_2)
\le 2\epsilon+\theta_r(A_1\cup A_2).\cr}$$

\noindent As $\epsilon$ is arbitrary,
$\theta_r(A_1\cup A_2)\ge\theta_rA_1+\theta_rA_2$.   The reverse
inequality is true just because $\theta_r$ is an outer measure, so
$\theta_r(A_1\cup A_2)=\theta_rA_1+\theta_rA_2$.   As $A_1$ and $A_2$
are
arbitrary, $\theta_r$ is a metric outer measure.\ \Qed

Now 471C tells us that $\mu_{Hr}$ must be a topological measure.

\medskip

{\bf (b)} (Compare 264Fa.)   If $\theta_rA=\infty$ this is trivial.
Otherwise, for each $n\in\Bbb N$, let $\sequence{i}{D_{ni}}$ be a
sequence of sets of diameter at most $2^{-n}$ such that
$A\subseteq\bigcup_{i\in\Bbb N}D_i$ and
$\sum_{i=0}^{\infty}(\diam D_{ni})^r\le\theta_{r,2^{-n}}(A)+2^{-n}$,
defining $\theta_{r,2^{-n}}$
as in 471A.   Let $\eta_{ni}\in\ocint{0,2^{-n}}$ be such that
$(2\eta_{ni}+\diam D_{ni})^r\le 2^{-n-i}+(\diam D_{ni})^r$, and set
$G_{ni}=\{x:\rho(x,D_{ni})<\eta_{ni}\}$ for all $n$, $i\in\Bbb N$;  then
$G_{ni}=\bigcup_{x\in D_{ni}}U(x,\eta_{ni})$ is an open set including
$D_{ni}$ and $(\diam G_{ni})^r\le 2^{-n-i}+(\diam D_{ni})^r$.   Set

\Centerline{$H=\bigcap_{n\in\Bbb N}\bigcup_{i\in\Bbb N}G_{ni}$,}

\noindent so that $H$ is a G$_{\delta}$ set including $A$.

For any $\delta>0$, there is an $n\in\Bbb N$ such that
$3\cdot 2^{-n}\le\delta$, so that
$\diam G_{mi}\le\diam D_{mi}+2\eta_{mi}\le\delta$ for every
$i\in\Bbb N$ and $m\ge n$, and

$$\eqalign{\theta_{r\delta}H
&\le\sum_{i=0}^{\infty}(\diam G_{mi})^r
\le\sum_{i=0}^{\infty}2^{-m-i}+(\diam D_{mi})^r\cr
&\le 2^{-m+1}+\theta_{r,2^{-m}}(A)+2^{-m}
\le 2^{-m+2}+\theta_rA\cr}$$

\noindent for every $m\ge n$.   Accordingly
$\theta_{r\delta}H\le\theta_rA$ for every $\delta>0$, so
$\theta_rH\le\theta_rA$.   Of course this means that
$\theta_rH=\theta_rA$;  and since, by (a), $\mu_{Hr}$ measures $H$, we
have $\mu_{Hr}H=\theta_rA$, as required.

\medskip

{\bf (c)} (Compare 264Fb.)   If $A\subseteq X$,

$$\eqalignno{\theta_rA&\ge\mu_{Hr}^*A\cr
\displaycause{by (b)}
&=\inf\{\theta_rE:A\subseteq E\in\Sigma\}
\ge\theta_rA.\cr}$$

\medskip

{\bf (d)} Use 431C.

\medskip

{\bf (e)} By (b), there is a Borel set $H\supseteq E$ such that
$\mu_{Hr}H=\mu_{Hr}E$, and now there is a Borel set
$H'\supseteq H\setminus E$ such that
$\mu_{Hr}H'=\mu_{Hr}(H\setminus E)=0$, so that
$G=H\setminus H'$ is a Borel set included in $E$ and
$\mu_{Hr}G=\mu_{Hr}E$.   Now $G$ is a Baire set (4A3Kb), so is Souslin-F
(421L), and $\mu_{Hr}G=\sup_{F\subseteq G\text{ is closed}}\mu_{Hr}F$,
by 431E.

\medskip

{\bf (f)} For every $n\in\Bbb N$, there must be a sequence
$\sequence{i}{D_{ni}}$ of sets of diameter at most
$2^{-n}$ covering $A$;  now if $D\subseteq A$ is a countable set which
meets $D_{ni}$ whenever $i$, $n\in\Bbb N$ and
$A\cap D_{ni}\ne\emptyset$, $D$ will be dense in $A$.   Now if $A_0$ is
the set of isolated points in $A$, it is still separable (4A2P(a-iv));
but as the only dense subset of $A_0$ is itself, it is countable.
Since $\theta_{r\delta}\{x\}=(\diam\{x\})^r=0$ for every $\delta>0$,
$\mu_{Hr}\{x\}=0$ for every $x\in X$, and $A_0$ is negligible.

\medskip

{\bf (g)} In fact, if $A\subseteq X$ and $\theta_rA>0$, there are
disjoint $A_0$, $A_1\subseteq A$ such that $\theta_rA_i>0$ for both $i$.
\Prf\ (i) Suppose first that $A$ is not separable.   For each
$n\in\Bbb N$, let $D_n\subseteq A$ be a maximal set such that
$\rho(x,y)\ge 2^{-n}$ for all distinct $x$, $y\in D_n$;  then
$\bigcup_{n\in\Bbb N}D_n$ is dense in $A$, so there is some $n\in\Bbb N$
such that $D_n$ is uncountable;  if we take $A_1$, $A_2$ to be disjoint
uncountable subsets of $D_n$, then $\theta_rA_1=\theta_rA_2=\infty$.
(ii) If $A$ is separable, then set $\Cal G=\{G:G\subseteq X$ is open,
$\theta_r(A\cap G)=0\}$.   Because $A$ is hereditarily Lindel\"of
(4A2P(a-iii)), there is a countable subset $\Cal G_0$ of $\Cal G$ such
that $A\cap\bigcup\Cal G=A\cap\bigcup\Cal G_0$ (4A2H(c-i)), so
$A\cap\bigcup\Cal G$ is negligible and $A\setminus\bigcup\Cal G$ has at
least two points $x_0$, $x_1$.   If we set
$A_i=U(x_i,\bover12\rho(x_i,x_{1-i}))$ for each $i$, these are disjoint
subsets of $A$ of non-zero outer measure.\ \Qed

\medskip

{\bf (h)} If $\mu_{Hr}$ is totally finite, then it is inner regular with
respect to the closed sets, by (e).   Also, because $X$ must be
separable, by (f), therefore hereditarily Lindel\"of, $\mu_{Hr}$ must be
$\tau$-additive (414O).   Finally, $\mu_{Hr}$ is complete just
because it is defined by \Caratheodory's method.   So $\mu_{Hr}$ is a
quasi-Radon measure.
}%end of proof of 471D

\leader{471E}{Corollary} If $(X,\rho)$ is a metric space, $r>0$ and
$Y\subseteq X$ then $r$-dimensional Hausdorff measure $\mu^{(Y)}_{Hr}$
on $Y$ extends the subspace measure $(\mu^{(X)}_{Hr})_Y$ on $Y$ induced
by $r$-dimensional Hausdorff measure $\mu^{(X)}_{Hr}$ on $X$;  and if
either $Y$ is measured by $\mu^{(X)}_{Hr}$ or $Y$ has finite
$r$-dimensional Hausdorff outer measure in $X$, then
$\mu^{(Y)}_{Hr}=(\mu^{(X)}_{Hr})_Y$.

\proof{ Write $\theta_r^{(X)}$ and $\theta_r^{(Y)}$ for the two
$r$-dimensional Hausdorff outer measures.

If $A\subseteq Y$ and $\sequencen{D_n}$ is any sequence of subsets of
$X$ covering $A$, then $\sequencen{D_n\cap Y}$ is a sequence of subsets
of $Y$ covering $A$, and $\sum_{n=0}^{\infty}(\diam(D_n\cap Y))^r
\le\sum_{n=0}^{\infty}(\diam D_n)^r$;  moreover, when calculating
$\diam(D_n\cap Y)$, it doesn't matter whether we use the metric $\rho$
on $X$ or the subspace metric $\rho\restrp Y\times Y$ on $Y$.   What
this means is that, for any $\delta>0$, $\theta_{r\delta}A$ is the same
whether calculated in $Y$ or in $X$, so that
$\theta^{(Y)}_rA=\sup_{\delta>0}\theta_{r\delta}A=\theta_r^{(X)}A$.

Thus $\theta^{(Y)}_r=\theta^{(X)}_r\restrp\Cal PY$.   Also, by 471Db,
$\theta^{(X)}_r$ is a regular outer measure.   So 214Hb gives the
results.
}%end of proof of 471E

\leader{471F}{Corollary} Let $(X,\rho)$ be an analytic metric
space\cmmnt{ (that is, a metric space in which the
topology is analytic in the sense of \S423)}, and write $\mu_{Hr}$ for
$r$-dimensional Hausdorff measure on $X$.
Suppose that $\nu$ is a locally finite indefinite-integral measure over
$\mu_{Hr}$.   Then $\nu$ is a Radon measure.

\proof{ Since $\dom\nu\supseteq\dom\mu_{Hr}$, $\nu$ is a topological measure.
Because $X$ is separable, therefore hereditarily Lindel\"of, $\nu$ is
$\sigma$-finite and $\tau$-additive, therefore locally determined and
effectively locally finite.   Next, it is inner regular with respect to
the closed sets.   \Prf\ Let $f$ be a Radon-Nikod\'ym derivative of
$\nu$.   If $\nu E>0$, there is an $E'\subseteq E$ such that

\Centerline{$0<\nu E'=\biggerint f\times\chi E'\,d\mu_{Hr}<\infty$.}

\noindent There is a $\mu_{Hr}$-simple function $g$ such that
$g\le f\times\chi E'\,\,\mu_{Hr}$-a.e.\ and $\int g\,d\mu_{Hr}>0$;
setting $H=E'\cap\{x:g(x)>0\}$, $\nu_{Hr}H<\infty$.
Now there is a closed set $F\subseteq H$ such that $\mu_{Hr}F>0$, by
471De, and in this case $\nu F\ge\int_Fg\,d\mu_{Hr}>0$.   By 412B, this
is enough to show that $\nu$ is inner regular with respect to the closed
sets.\ \Qed

Since $\nu$ is complete (234I\formerly{2{}34A}), it is a
quasi-Radon measure, therefore a Radon measure (434Jf, 434Jb).
}%end of proof of 471F

\leader{471G}{Increasing Sets Lemma}\cmmnt{ ({\smc Davies 70})} Let
$(X,\rho)$ be a metric space and $r>0$.

(a) Suppose that $\delta>0$ and that $\sequencen{A_n}$ is a
non-decreasing sequence of subsets of $X$ with union $A$.   Then
$\theta_{r,6\delta}(A)
\le(5^r+2)\sup_{n\in\Bbb N}\theta_{r\delta}A_n$.

(b) Suppose that $\delta>0$ and that $\sequencen{A_n}$ is a
non-decreasing sequence of subsets of $X$ with union $A$.   Then
$\theta_{r\delta}A\le\sup_{n\in\Bbb N}\theta_{r\delta}A_n$.

\proof{{\bf (a)} If $\sup_{n\in\Bbb N}\theta_{r\delta}A_n=\infty$ this
is trivial;  suppose otherwise.

\medskip

\quad{\bf (i)} Take any
$\gamma>\gamma'>\sup_{n\in\Bbb N}\theta_{r\delta}A_n$.   For each
$i\in\Bbb N$, let $\zeta_i\in\ocint{0,\bover14\delta}$ be such that
$(\alpha+2\zeta_i)^r\le\alpha^r+2^{-i-1}(\gamma-\gamma')$ whenever
$0\le\alpha\le\delta$.   For each
$n\in\Bbb N$, there is a sequence $\sequence{i}{C_{ni}}$ of
sets covering $A_n$ such that $\diam C_{ni}\le\delta$ for every $i$ and
$\sum_{i=0}^{\infty}(\diam C_{ni})^r<\gamma'$;  let
$\sequence{i}{\gamma_{ni}}$ be such that
$\diam C_{ni}\le\gamma_{ni}\le\delta$ and $\gamma_{ni}>0$ for
every $i$ and $\sum_{i=0}^{\infty}\gamma_{ni}^r\le\gamma'$.   Since
$\sum_{i=0}^{\infty}\gamma_{ni}^r$ is finite,
$\lim_{i\to\infty}\gamma_{ni}=0$.   Because $\gamma_{ni}>0$ for every $i$,
we may rearrange the sequences $\sequence{i}{C_{ni}}$,
$\sequence{i}{\gamma_{ni}}$ in such a way that
$\gamma_{ni}\ge\gamma_{n,i+1}$ for each $i$.
%(We may have singleton $C_{ni}$.)

In this case, $\lim_{i\to\infty}\sup_{n\in\Bbb N}\gamma_{ni}=0$.
\Prf\

\Centerline{$(i+1)\gamma_{ni}^r\le\sum_{j=0}^i\gamma_{nj}^r\le\gamma$}

\noindent for every $n$, $i\in\Bbb N$.\ \Qed

\medskip

\quad{\bf (ii)} By Ramsey's theorem (4A1G, with $n=2$), there is an
infinite set $I\subseteq\Bbb N$ such that

\inset{for all $i$, $j\in\Bbb N$ there is an $s\in\Bbb N$ such that
either $C_{mi}\cap C_{nj}=\emptyset$ whenever $m$, $n\in I$ and
$s\le m<n$ or $C_{mi}\cap C_{nj}\ne\emptyset$ whenever $m$, $n\in I$ and
$s\le m<n$,}

\inset{for each $i\in\Bbb N$,
$\alpha_i=\lim_{n\in I,n\to\infty}\gamma_{ni}$ is defined in $\Bbb R$.}

\noindent (Apply 4A1Fb with the families

$$\eqalign{\Cal J_{ij}&=\{J:J\in[\Bbb N]^{\omega},
\text{ either }C_{mi}\cap C_{nj}=\emptyset
  \text{ whenever }m,\,n\in J\text{ and }m<n\cr
&\mskip150mu\text{ or }C_{mi}\cap C_{nj}\ne\emptyset
  \text{ whenever }m,\,n\in J\text{ and }m<n\}\cr
\Cal J'_{iq}&=\{J:J\in[\Bbb N]^{\omega},
\text{ either }\gamma_{ni}\le q\text{ for every }n\in J\cr
&\mskip200mu\text{ or }\gamma_{ni}\ge q\text{ for every }n\in J\}\cr}$$

\noindent for $i$, $j\in\Bbb N$ and $q\in\Bbb Q$.)

Of course $\alpha_j\le\alpha_i\le\delta$ whenever
$i\le j$, because $\gamma_{nj}\le\gamma_{ni}\le\delta$ for
every $n$.   Set $D_{ni}=\{x:\rho(x,C_{ni})\le 2\alpha_i+2\zeta_i\}$ for
all $n$, $i\in\Bbb N$, and
$D_i=\bigcup_{s\in\Bbb N}\bigcap_{n\in I\setminus s}D_{ni}$ for
$i\in\Bbb N$.   (I am identifying each $s\in\Bbb N$ with the set of its
predecessors.)

\medskip

\quad{\bf (iii)} Set

\Centerline{$L=\{(i,j):i,\,j\in\Bbb N,\,
  \Forall s\in\Bbb N\,\Exists m,\,n\in I,\,s\le m<n$
  and $C_{mi}\cap C_{nj}\ne\emptyset\}$.}

\noindent If $(i,j)\in L$ then there is an $s\in\Bbb N$ such that
$C_{mi}\subseteq D_{\min(i,j)}$ whenever $m\in I$ and
$m\ge s$.   \Prf\ By the choice of $I$, we know that there is an
$s_0\in\Bbb N$ such
that $C_{mi}\cap C_{nj}\ne\emptyset$ whenever $m$, $n\in I$ and
$s_0\le m<n$.   Let $s_1\ge s_0$ be such that

\Centerline{$\gamma C_{mi}\le\alpha_i+\min(\zeta_i,\zeta_j)$,
\quad$\gamma C_{mj}\le\alpha_j+\min(\zeta_i,\zeta_j)$}

\noindent whenever $m\in I$ and $m\ge s_1$.   Take $m_0\in I$ such that
$m_0\ge s_1$, and set $s=m_0+1$.   Let $m\in I$ be such that $m\ge s$.

\medskip

\qquad\grheada\ Suppose that $i\le j$ and $x\in C_{mi}$.   Take any
$n\in I$ such that $m\le n$.   Then there is an $n'\in I$ such that
$n<n'$.   We know that
$C_{mi}\cap C_{n'j}$ and $C_{ni}\cap C_{n'j}$ are both non-empty.   So

\Centerline{$\rho(x,C_{ni})\le\diam C_{mi}+\diam C_{n'j}
\le\gamma_{mi}+\gamma_{n'j}
\le\alpha_i+\zeta_i+\alpha_j+\zeta_i\le 2\alpha_i+2\zeta_i$}

\noindent and $x\in D_{ni}$.   This is true for all $n\in I$ such that
$n\ge m$, so $x\in D_i$.   As $x$ is arbitrary, $C_{mi}\subseteq D_i$.

\medskip

\qquad\grheadb\ Suppose that $j\le i$ and $x\in C_{mi}$.   Take any
$n\in I$ such that $n>m$.   Then $C_{mi}\cap C_{nj}$ is not empty, so

\Centerline{$\rho(x,C_{nj})
\le\diam C_{mi}\le\gamma_i\le\alpha_i+\zeta_j\le\alpha_j+\zeta_j$}

\noindent and $x\in D_{nj}$.   As $x$ and $n$ are arbitrary,
$C_{mi}\subseteq D_j$.

Thus $C_{mi}\subseteq D_{\min(i,j)}$ in both cases.\ \Qed

\medskip

\quad{\bf (iv)} Set

\Centerline{$D=\bigcup_{i\in\Bbb N}D_i$,
\quad$J=\{i:i\in\Bbb N,\,\Exists s\in\Bbb N$, $C_{ni}\subseteq D$
whenever $n\in I$ and $n\ge s\}$.}

\noindent If $i\in\Bbb N\setminus J$ and $j\in\Bbb N$, then (iii) tells
us that $(i,j)\notin L$, so there is some $s\in\Bbb N$ such that
$C_{mi}\cap C_{nj}=\emptyset$ whenever $m$,
$n\in I$ and $s\le m<n$.

\medskip

\quad{\bf (v)} For $l\in\Bbb N$,
$\mu^*_{Hr}(A_l\setminus D)\le 2\gamma$.   \Prf\ Let $\epsilon>0$.
Then there is a
$k\in\Bbb N$ such that $\gamma_{ni}\le\epsilon$ whenever $n\in\Bbb N$
and $i>k$.   Next, there is an $s\in\Bbb N$ such that

\Centerline{$C_{ni}\subseteq D$ whenever $i\le k$, $i\in J$, $n\in I$
and $s\le n$,}

\Centerline{$C_{mi}\cap C_{nj}=\emptyset$ whenever $i$, $j\le k$,
$i\notin J$, $m$, $n\in I$ and $s\le m<n$.}

\noindent Take $m$, $n\in I$ such that $\max(l,s)\le m<n$.   Then

$$\eqalign{A_l\setminus D
&=\bigcup_{i\in\Bbb N}A_l\cap C_{mi}\setminus D\cr
&\subseteq\bigcup_{i\le k}(A_n\cap C_{mi}\setminus D)
  \cup\bigcup_{i>k}C_{mi}\cr
&\subseteq\bigcup_{i\le k,i\notin J}(A_n\cap C_{mi})
  \cup\bigcup_{i>k}C_{mi}\cr
&\subseteq\bigcup_{i\le k,i\notin J,j\le k}(C_{mi}\cap C_{nj})
  \cup\bigcup_{j>k}C_{nj}
  \cup\bigcup_{i>k}C_{mi}\cr
&=\bigcup_{j>k}C_{nj}\cup\bigcup_{i>k}C_{mi}.\cr}$$

\noindent Since $\diam C_{nj}\le\gamma_j\le\epsilon$ and
$\diam C_{mi}\le\gamma_i\le\epsilon$ for all $i$, $j>k$,

\Centerline{$\theta_{r\epsilon}(A_l\setminus D)
\le\sum_{i=k+1}^{\infty}\gamma_{ni}^r
  +\sum_{i=k+1}^{\infty}\gamma_{mi}^r
\le 2\gamma$.}

\noindent This is true for every $\epsilon>0$, so
$\mu^*_{Hr}(A_l\setminus D)\le 2\gamma$, as claimed.\ \Qed

\medskip

\quad{\bf (vi)} This is true for each $l\in\Bbb N$.   But this means
that $\mu_{Hr}^*(A\setminus D)\le 2\gamma$ (132Ae).
Now $\theta_{r,6\delta}D\le 5^r\gamma$.   \Prf\ For each $i\in\Bbb N$,

$$\eqalign{\diam D_i
&\le\lim_{n\in I,n\to\infty}\diam D_{ni}
\le\lim_{n\in I,n\to\infty}\diam C_{ni}+4\alpha_i+4\zeta_i\cr
&\le\lim_{n\in I,n\to\infty}\gamma_{ni}+4\alpha_i+4\zeta_i
=5\alpha_i+4\zeta_i\le 6\delta.\cr}$$

\noindent Next, for any $k\in\Bbb N$,

\Centerline{$\sum_{i=0}^k\alpha_i^r
=\lim_{n\in I,n\to\infty}\sum_{i=0}^k\gamma_{ni}^r
\le\gamma'$,}

\noindent so

$$\eqalignno{\sum_{i=0}^k(\diam D_i)^r
&\le 5^r\sum_{i=0}^k(\alpha_i+2\zeta_i)^r
\le 5^r(\sum_{i=0}^k\alpha_i^r+2^{-i-1}(\gamma-\gamma'))\cr
\displaycause{by the choice of the $\zeta_i$}
&\le 5^r\gamma.\cr}$$

\noindent Letting $k\to\infty$,

\Centerline{$\theta_{r,6\delta}D\le\sum_{i=0}^{\infty}(\diam D_i)^r
\le 5^r\gamma$.   \Qed}

Putting these together,

\Centerline{$\theta_{r,6\delta}A
\le\theta_{r,6\delta}D+\theta_{r,6\delta}(A\setminus D)
\le\theta_{r,6\delta}D+\mu^*_{Hr}(A\setminus D)
\le(5^r+2)\gamma$.}

\noindent As $\gamma$ is arbitrary, we have the preliminary result (a).

\medskip

{\bf (b)} Now let us turn to the sharp form (b).   Once again, we may
suppose that $\sup_{n\in\Bbb N}\theta_{r\delta}A_n$ is finite.

\medskip

\quad{\bf (i)} Take $\gamma$, $\gamma'$ such that
$\sup_{n\in\Bbb N}\theta_{r\delta}A_n<\gamma'<\gamma$.   As in (a)(i)
above, we can find a family
$\langle C_{ni}\rangle_{n,i\in\Bbb N}$ such that

\Centerline{$A_n\subseteq\bigcup_{i\in\Bbb N}C_{ni}$,}

\Centerline{$\diam C_{ni}\le\delta$ for every $i\in\Bbb N$,}

\Centerline{$\sum_{i=0}^{\infty}(\diam C_{ni})^r\le\gamma'$}

\noindent for each $n$, and

\Centerline{$\lim_{i\to\infty}\sup_{n\in\Bbb N}\diam C_{ni}=0$.}

\noindent Replacing each $C_{ni}$ by its closure if necessary, we may
suppose that every $C_{ni}$ is a Borel set.

Let $Q\subseteq X$ be a countable set which meets $C_{ni}$ whenever $n$,
$i\in\Bbb N$ and $C_{ni}$ is not empty.   This time, let
$I\subseteq\Bbb N$ be an infinite set such that

\Centerline{$\alpha_i=\lim_{n\in I,n\to\infty}\diam C_{ni}$ is defined
in $[0,\delta]$ for every $i\in\Bbb N$,}

\Centerline{$\lim_{n\in I,n\to\infty}\rho(z,C_{ni})$ is defined in
$[0,\infty]$ for every $i\in\Bbb N$ and every $z\in Q$.}

\noindent (Take $\rho(z,\emptyset)=\infty$ if any of the $C_{ni}$ are
empty.)   It will be helpful to note straight away that the limit
$\lim_{n\in I,n\to\infty}\rho(x,C_{ni})$ is defined in $[0,\infty]$ for
every $i\in\Bbb N$ and $x\in\overline{Q}$.   \Prf\ If
$\lim_{n\in I,n\to\infty}\rho(y,C_{ni})=\infty$ for some $y\in Q$, then
$\lim_{n\in I,n\to\infty}\rho(x,C_{ni})\penalty-100=\infty$,
and we can stop.
Otherwise, for any $\epsilon>0$, there are a $z\in Q$ such that
$\rho(x,z)\le\epsilon$ and an $s\in\Bbb N$ such that $C_{mi}$ is not
empty and $|\rho(z,C_{mi})-\rho(z,C_{ni})|\le\epsilon$ whenever $m$,
$n\in I\setminus s$;  in which case
$|\rho(x,C_{mi})-\rho(x,C_{ni})|\le 3\epsilon$ whenever $m$,
$n\in I\setminus s$.   As $\epsilon$ is arbitrary,
$\lim_{n\in I,n\to\infty}\rho(x,C_{ni})$ is defined in $\Bbb R$.\ \Qed

Let $\Cal F$ be a non-principal ultrafilter on $\Bbb N$ containing $I$,
and for $i\in\Bbb N$ set

\Centerline{$D_i=\{x:\lim_{n\to\Cal F}\rho(x,C_{ni})=0\}$.}

\noindent Set $D=\bigcup_{i\in\Bbb N}\overline{D}_i$.   (Actually it is
easy to check that every $D_i$ is closed.)

\medskip

\quad{\bf (ii)} Set

\Centerline{$A^*=\bigcup_{m\in\Bbb N}\bigcap_{n\in I\setminus m}
  \bigcup_{i\in\Bbb N}C_{ni}\setminus D$;}

\noindent note that $A^*$ is a Borel set.   For $k$, $m\in\Bbb N$, set

\Centerline{$A^*_{km}
=\bigcap_{n\in I\setminus m}\bigcup_{i\ge k}C_{ni}$.}

\noindent For fixed $k$, $\sequence{m}{A^*_{km}}$ is a non-decreasing
sequence of sets.   Also its union includes $A^*$.   \Prf\ Take
$x\in A^*$.

\medskip

\qquad\grheada\Quer\ If $x\notin\overline{Q}$, let $\epsilon>0$ be such
that $Q\cap B(x,\epsilon)=\emptyset$.   Let $l\in\Bbb N$ be such that
$\diam C_{ni}\le\epsilon$ whenever $n\in\Bbb N$ and $i\ge l$;   then
$x\notin C_{ni}$ whenever $n\in\Bbb N$ and $i\ge l$.   Let $m\in\Bbb N$
be such that $x\in\bigcup_{i\in\Bbb N}C_{ni}$ whenever $n\in I$ and
$n\ge m$.   Then $x\in\bigcup_{i<l}C_{ni}$ whenever $n\in I$ and
$n\ge m$.   But this means that there must be some $i<l$ such that
$\{n:x\in C_{ni}\}\in\Cal F$ and $x\in D_i\subseteq D$;  which is
impossible.\ \Bang

\medskip

\qquad\grheadb\ Thus $x\in\overline{Q}$, so
$\lim_{n\in I,n\to\infty}\rho(x,C_{ni})$ is defined for each $i$ (see
(i) above), and must be greater than $0$, since $x\notin D_i$.   In
particular, there is an $s\in\Bbb N$ such that $x\notin C_{ni}$ whenever
$i<k$ and $n\in I\setminus s$;   there is also an $m\in\Bbb N$ such that
$x\in\bigcap_{n\in I\setminus m}\bigcup_{i\in\Bbb N}C_{ni}$;  so that
$x\in A^*_{k,\max(s,m)}$.   As $x$ is arbitrary,
$A^*\subseteq\bigcup_{m\in\Bbb N}A^*_{km}$.\ \Qed

\medskip

\quad{\bf (iii)} $\mu_{Hr}A^*$ is finite.   \Prf\ Take any $\epsilon>0$.
Let $k\in\Bbb N$ be such that $\diam C_{ni}\le\epsilon$ whenever
$i\ge k$ and $n\in\Bbb N$.   For any $m\in I$,
$\theta_{r\epsilon}A^*_{km}
\le\sum_{i=k}^{\infty}(\diam C_{mi})^r\le\gamma$.   By (a),

$$\eqalignno{\theta_{r,6\epsilon}A^*
&\le(5^r+2)\sup_{m\in\Bbb N}\theta_{r\epsilon}A^*_{km}\cr
&=(5^r+2)\sup_{m\in I}\theta_{r\epsilon}A^*_{km}
\le(5^r+2)\gamma.\cr}$$

\noindent As $\epsilon$ is arbitrary,
$\mu_{Hr}A^*\le(5^r+2)\gamma<\infty$.\ \Qed

\medskip

\quad{\bf (iv)} Actually,
$\mu_{Hr}A^*\le\gamma'-\sum_{i=0}^{\infty}\alpha_i^r$.  \Prf\Quer\
Suppose, if possible, otherwise.   Take $\beta$ such that
$\gamma'-\sum_{i=0}^{\infty}\alpha_i^r<\beta<\mu_{Hr}A^*$.
For $x\in A^*$ and $k\in\Bbb N$, set
$f_k(x)=\min\{n:n\in I$, $x\in A^*_{kn}\}$;   then $\sequence{k}{f_k}$
is a non-decreasing sequence of Borel measurable functions from $A^*$ to
$\Bbb N$.   Choose $\sequence{k}{s_k}$ inductively so that

\Centerline{$\mu_{Hr}\{x:x\in A^*$, $f_j(x)\le s_j$ for every $j\le k\}
>\beta$}

\noindent for every $k\in\Bbb N$.   Set
$\tilde A=\{x:x\in A^*$, $f_j(x)\le s_j$ for every $j\in\Bbb N\}$;
because $\mu_{Hr}A^*$ is finite, $\mu_{Hr}\tilde A\ge\beta$.   Take
$\epsilon>0$ such that
$\theta_{r\epsilon}\tilde A>\gamma'-\sum_{i=0}^{\infty}\alpha_i^r$.
Let $k\in\Bbb N$ be such that
$\theta_{r\epsilon}\tilde A+\sum_{i=0}^{k-1}\alpha_i^r>\gamma'$ and
$\diam C_{ni}\le\epsilon$ whenever $n\in\Bbb N$ and $i\ge k$.
Take $n\in I$ such that $n\ge s_j$ for every $j\le k$ and
$\theta_{r\epsilon}\tilde A+\sum_{i=0}^{k-1}(\diam C_{ni})^r>\gamma'$.
If $x\in\tilde A$, then

\Centerline{$f_k(x)\le s_k\le n$,
\quad$x\in A^*_{kn}\subseteq\bigcup_{i\ge k}C_{ni}$,}

\noindent so
$\theta_{r\epsilon}\tilde A\le\sum_{i=k}^{\infty}(\diam C_{ni})^r$;
but this means that $\sum_{i=0}^{\infty}(\diam C_{ni})^r>\gamma'$,
contrary to the choice of the $C_{ni}$.\ \Bang\Qed

\medskip

\quad{\bf (v)} Now observe that

\Centerline{$A
\subseteq\bigcup_{m\in\Bbb N}\bigcap_{n\ge m}\bigcup_{i\in\Bbb N}C_{ni}
\subseteq A^*\cup D$.}

\noindent Moreover, for any $i\in\Bbb N$,
$\diam\overline{D}_i\le\alpha_i\le\delta$.   \Prf\ If $x$, $y\in D_i$
then for every $\epsilon>0$

\Centerline{$\rho(x,C_{ni})\le\epsilon$,
\quad$\rho(y,C_{ni})\le\epsilon$,
\quad$\diam C_{ni}\le\alpha_i+\epsilon$}

\noindent for all but finitely many $n\in I$.   So
$\rho(x,y)\le\alpha_i+3\epsilon$.   As $x$, $y$ and $\epsilon$ are
arbitrary, $\diam\overline{D}_i=\diam D_i\le\alpha_i$.   Of course
$\alpha_i\le\delta$ because $\diam C_{ni}\le\delta$ for every $n$.\ \Qed

Now

\Centerline{$\theta_{r\delta}D
\le\sum_{i=0}^{\infty}(\diam\overline{D}_i)^r
\le\sum_{i=0}^{\infty}\alpha_i^r$.}

\noindent Putting this together with (iv),

\Centerline{$\theta_{r\delta}A
\le\theta_{r\delta}D+\theta_{r\delta}A^*
\le\theta_{r\delta}D+\mu_{Hr}A^*
\le\gamma$.}

\noindent As $\gamma$ and $\gamma'$ are arbitrary,

\Centerline{$\theta_{r\delta}A
\le\sup_{n\in\Bbb N}\theta_{r\delta}A_n$,}

\noindent as required.
}%end of proof of 471G

\leader{471H}{Corollary} Let $(X,\rho)$ be a metric space, and $r>0$.
For $A\subseteq X$, set

\Centerline{$\theta_{r\infty}A
=\inf\{\sum_{n=0}^{\infty}(\diam D_n)^r:\sequencen{D_n}$ is a sequence
of subsets of $X$ covering $A\}$.}

\noindent Then $\theta_{r\infty}$ is a Choquet capacity on $X$.

\proof{{\bf (a)} Of course $0\le\theta_{r\infty}A\le\theta_{r\infty}B$
whenever $A\subseteq B\subseteq X$.

\medskip

{\bf (b)} Suppose that $\sequencen{A_n}$ is a non-decreasing sequence of
subsets of $A$ with union $A$.   By (a),
$\gamma=\lim_{n\to\infty}\theta_{r\infty}A_n$ is defined
and less than or equal to $\theta_{r\infty}A$.   If $\gamma=\infty$, of
course it is equal to $\theta_{r\infty}A$.   Otherwise,
take $\beta=(\gamma+1)^{1/r}$.   For $n$, $k\in\Bbb N$ there is a
sequence
$\sequence{i}{D_{nki}}$ of sets, covering $A_n$, such that
$\sum_{i=0}^{\infty}(\diam D_{nki})^r\le\gamma+2^{-k}$.   But in this
case $\diam D_{nki}\le\beta$ for all
$n$, $k$ and $i$, so the $D_{nki}$ witness that
$\theta_{r\beta}A_n\le\gamma$.   By 471Gb,
$\gamma\ge\theta_{r\beta}A\ge\theta_{r\infty}A$ and again we have
$\gamma=\theta_{r\infty}A$.

\medskip

{\bf (c)} Let $K\subseteq X$ be any set, and suppose that
$\gamma>\theta_{r\infty}K$.   Let $\sequencen{D_n}$ be a sequence of
sets, covering $K$, such that
$\sum_{n=0}^{\infty}(\diam D_n)^r<\gamma$.   Let
$\sequencen{\epsilon_n}$ be a sequence of strictly positive real numbers
such that
$\sum_{n=0}^{\infty}(\diam D_n+2\epsilon_n)^r\le\gamma$.   Set
$G_n=\{x:\rho(x,D_n)<\epsilon_n\}$ for each $n$;  then $G_n$ is open and
$\diam G_n\le\diam D_n+2\epsilon_n$.
So $G=\bigcup_{n\in\Bbb N}G_n$ is an open set including $K$, and
$\sequencen{G_n}$ witnesses that $\theta_{r\infty}G\le\gamma$.   As $K$
and $\gamma$ are arbitrary,
condition (iii) of 432Ja is satisfied and $\theta_{r\infty}$ is a
Choquet capacity.
}%end of proof of 471H

\medskip

\noindent{\bf Remark} $\theta_{r\infty}$ is {\bf $r$-dimensional
Hausdorff capacity} on $X$.

\leader{471I}{Theorem} Let $(X,\rho)$ be a metric space, and
$r>0$.   Write $\mu_{Hr}$ for $r$-dimensional Hausdorff measure on
$X$.   If $A\subseteq X$ is analytic, then $\mu_{Hr}A$ is defined and
equal to $\sup\{\mu_{Hr}K:K\subseteq A$ is compact$\}$.

\proof{{\bf (a)} Before embarking on the main line of the proof, it will
be convenient to set out a preliminary result.   For $\delta>0$,
$n\in\Bbb N$, $B\subseteq X$ set

\Centerline{$\theta^{(n)}_{r\delta}(B)
=\inf\{\sum_{i=0}^n(\diam D_i)^r:
B\subseteq\bigcup_{i\le n}D_i,\,\diam D_i\le\delta$
for every $i\le n\}$,}

\noindent taking $\inf\emptyset=\infty$ as usual.   Then
$\theta_{r\delta}B\le\theta^{(n)}_{r\delta}(B)$ for
every $n$.   Now the point is that $\theta^{(n)}_{r\delta}(B)
=\sup\{\theta^{(n)}_{r\delta}(I):I\subseteq B$ is finite$\}$.   \Prf\
Set $\gamma=\sup_{I\in[B]^{<\omega}}\theta^{(n)}_{r\delta}(I)$.   Of
course $\gamma\le\theta^{(n)}_{r\delta}(B)$.   If $\gamma=\infty$ there
is nothing more to say.   Otherwise, take any $\gamma'>\gamma$.   For
each $I\in[B]^{<\omega}$, we have a function $f_I:I\to\{0,\ldots,n\}$
such that $\sum_{i\in J}\rho(x_i,y_i)^r\le\gamma'$ whenever
$J\subseteq\{0,\ldots,n\}$ and $x_i$, $y_i\in I$ and
$f_I(x_i)=f_I(y_i)=i$ for every $i\in J$, while $\rho(x,y)\le\delta$
whenever $x$, $y\in I$ and $f_I(x)=f_I(y)$.   Let $\Cal F$ be an
ultrafilter on $[B]^{<\omega}$ such that
$\{I:x\in I\in[B]^{<\omega}\}\in\Cal F$ for every $x\in B$ (4A1Ia).
Then for every $x\in B$ there is an $f(x)\in\{0,\ldots,n\}$ such that
$\{I:x\in I\in[B]^{<\omega},\,f_I(x)=f(x)\}\in\Cal F$.   Set
$D_i=f^{-1}[\{i\}]$ for $i\le n$.   If
$x$, $y\in B$ and $f(x)=f(y)$, there is an $I\in[B]^{<\omega}$
containing both $x$ and $y$ such that $f_I(x)=f(x)=f(y)=f_I(y)$, so that
$\rho(x,y)\le\delta$;  thus $\diam D_i\le\delta$ for each $i$.   If
$J\subseteq\{0,\ldots,n\}$ and
for each $i\in J$ we take $x_i$, $y_i\in D_i$, then there is an
$I\in[B]^{<\omega}$ such that $f_I(x_i)=f_I(y_i)=i$ for every $i\in J$,
so $\sum_{i\in J}\rho(x_i,y_i)^r\le\gamma'$.   This means that
$\sum_{i\le n}(\diam D_i)^r\le\gamma'$, so that
$\theta^{(n)}_{r\delta}(B)\le\gamma'$.   As $\gamma'$ is arbitrary,
$\theta^{(n)}_{r\delta}(B)\le\gamma$, as claimed.\ \Qed

\medskip

{\bf (b)} Now let us turn to the set $A$.
Because $A$ is Souslin-F (422Ha), $\mu_{Hr}$ measures $A$ (471Da,
471Dd).   Set
$\gamma=\sup\{\mu_{Hr}K:K\subseteq A$ is compact$\}$.

\Quer\ Suppose, if possible, that $\mu_{Hr}A>\gamma$.   Take
$\gamma'\in\ooint{\gamma,\mu_{Hr}A}$.  Let $\delta>0$ be such that
$\gamma'<\theta_{r\delta}A$.   Let $f:\NN\to A$ be a continuous
surjection.   For
$\sigma\in S=\bigcup_{n\in\Bbb N}\BbbN^n$, set

\Centerline{$F_{\sigma}=\{\phi:\phi\in\NN$, $\phi(i)\le\sigma(i)$ for
every $i<\#(\sigma)\}$,}

\noindent so that $f[F_{\emptyset}]=A$.   Now choose $\psi\in\NN$ and
a sequence $\sequencen{I_n}$ of finite subsets of $\NN$ inductively, as
follows.   Given that $I_j\subseteq F_{\psi\restr n}$ for every $j<n$
and that $\theta_{r\delta}(f[F_{\psi\restr n}])>\gamma'$, then
$\theta^{(n)}_{r\delta}(f[F_{\psi\restr n}])>\gamma'$, so by (a) above
there is a finite subset $I_n$ of $F_{\psi\restr n}$ such that
$\theta^{(n)}_{r\delta}(f[I_n])\ge\gamma'$.   Next,

$$\eqalignno{\lim_{i\to\infty}\theta_{r\delta}
  f[F_{(\psi\restr n)^{\smallfrown}\fraction{i}}]
&=\theta_{r\delta}
  (\bigcup_{i\in\Bbb N}f[F_{(\psi\restr n)^{\smallfrown}\fraction{i}}])\cr
\displaycause{by 471G}
&=\theta_{r\delta}f[F_{\psi\restr n}]>\gamma',\cr}$$

\noindent so we can take $\psi(n)$ such
that $\bigcup_{j\le n}I_j\subseteq F_{\psi\restr n+1}$ and
$\theta_{r\delta}f[F_{\psi\restr n+1}]>\gamma'$, and continue.

At the end of the induction, set $K=\{\phi:\phi\le\psi\}$.   Then $f[K]$
is a compact subset of $A$, and
$I_n\subseteq K$ for every $n\in\Bbb N$, so

\Centerline{$\theta^{(n)}_{r\delta}(f[K])
\ge\theta^{(n)}_{r\delta}(f[I_n])\ge\gamma'$}

\noindent for every $n\in\Bbb N$.   On the
other hand, $\mu_{Hr}(f[K])\le\gamma$, so there is a sequence
$\sequence{i}{D_i}$ of sets, covering $f[K]$, all of diameter less than
$\delta$, such that $\sum_{i=0}^{\infty}(\diam D_i)^r<\gamma'$.
Enlarging the $D_i$ slightly if need be, we may suppose that they are
all open.   But in this case there is some finite $n$ such that
$K\subseteq\bigcup_{i\le n}D_i$, and
$\theta^{(n)}_{r\delta}(K)\le\sum_{i=0}^n(\diam D_i)^r<\gamma'$;  which
is impossible.\ \Bang

This contradiction shows that $\mu_{Hr}A=\gamma$, as required.
}%end of proof of 471I

\leader{471J}{Proposition} Let $(X,\rho)$ and $(Y,\sigma)$ be metric
spaces, and $f:X\to Y$ a $\gamma$-Lipschitz function, where $\gamma\ge 0$.
If $r>0$ and $\theta^{(X)}_r$,
$\theta^{(Y)}_r$ are the $r$-dimensional Hausdorff outer measures on $X$
and $Y$ respectively, then
$\theta^{(Y)}_rf[A]\le\gamma^r\theta^{(X)}_rA$ for every $A\subseteq X$.

\proof{ (Compare 264G.)  Let $\delta>0$.   Set
$\eta=\delta/(1+\gamma)$ and consider
$\theta^{(X)}_{r\eta}:\Cal PX\to[0,\infty]$, defined as in 471A.   We
know that $\theta^{(X)}_rA\ge\theta^{(X)}_{r\eta}A$, so there is a
sequence $\sequencen{D_n}$ of sets, all of
diameter at most $\eta$, covering $A$, with
$\sum_{n=0}^{\infty}(\diam D_n)^r\le\theta^{(X)}_rA+\delta$.
Now $\phi[A]\subseteq\bigcup_{n\in\Bbb N}\phi[D_n]$ and

\Centerline{$\diam\phi[D_n]
\le\gamma\diam D_n\le\gamma\eta\le\delta$}

\noindent for every $n$.   Consequently

\Centerline{$\theta^{(Y)}_{r\delta}(\phi[A])
\le\sum_{n=0}^{\infty}(\diam\phi[D_n])^r
\le\sum_{n=0}^{\infty}\gamma^r(\diam D_n)^r
\le\gamma^r(\theta_r^{(X)}A+\delta)$,}

\noindent and

\Centerline{$\theta^{(Y)}_r(\phi[A])
=\lim_{\delta\downarrow 0}\theta^{(Y)}_{r\delta}(\phi[A])
\le\gamma^r\theta_r^{(X)}A$,}

\noindent as claimed.
}%end of proof of 471J

\leader{471K}{Lemma} Let $(X,\rho)$ be a metric space, and $r>0$.   Let
$\mu_{Hr}$ be $r$-dimensional Hausdorff measure on $X$.   If
$A\subseteq X$, then $\mu_{Hr}A=0$ iff for every $\epsilon>0$ there is a
countable family $\Cal D$ of sets, covering $A$, such that
$\sum_{D\in\Cal D}(\diam D)^r\le\epsilon$.

\proof{ If $\mu_{Hr}A=0$ and $\epsilon>0$, then, in the language of
471A, $\theta_{r1}A\le\epsilon$, so there is a sequence
$\sequencen{D_n}$ of sets covering $A$ such that
$\sum_{n=0}^{\infty}(\diam D_n)^r\le\epsilon$.

If the condition is satisfied, then for any $\epsilon$, $\delta>0$ there
is a countable family $\Cal D$ of sets, covering $A$, such that
$\sum_{D\in\Cal D}(\diam D)^r\le\min(\epsilon,\delta^r)$.   If $\Cal D$
is infinite, enumerate it as $\sequencen{D_n}$;  if it is finite,
enumerate it as $\langle D_n\rangle_{n<m}$ and set $D_n=\emptyset$ for
$n\ge m$.   Now $A\subseteq\bigcup_{n\in\Bbb N}D_n$ and
$\diam D_n\le\delta$ for every $n\in\Bbb N$, so
$\theta_{r\delta}A\le\sum_{n=0}^{\infty}(\diam D_n)^r\le\epsilon$.
As $\epsilon$ is arbitrary, $\theta_{r\delta}A=0$;  as $\delta$ is
arbitrary, $\theta_rA=0$;  it follows at once that $\mu_{Hr}A$ is
defined and is zero (113Xa).
}%end of proof of 471K

\leader{471L}{Proposition} Let $(X,\rho)$ be a metric space and
$0<r<s$.   If $A\subseteq X$ is such that $\mu^*_{Hr}A$ is finite, then
$\mu_{Hs}A=0$.

\proof{ Let $\epsilon>0$.   Let $\delta>0$ be such that
$\delta^{s-r}(1+\mu^*_{Hr}A)\le\epsilon$.   Then there is a sequence
$\sequencen{A_n}$ of sets of diameter at most $\delta$ such that
$A\subseteq\bigcup_{n\in\Bbb N}A_n$ and
$\sum_{n=0}^{\infty}(\diam A_n)^r\le 1+\mu^*_{Hr}A$.   But now, by the
choice of $\delta$, $\sum_{n=0}^{\infty}(\diam A_n)^s\le\epsilon$.   As
$\epsilon$ is arbitrary, $\mu_{Hs}A=0$, by 471K.
}%end of proof of 471L

\leader{471M}{}\cmmnt{ There is a generalization of the density
theorems of \S\S223 and 261 for general Hausdorff measures, which (as
one expects) depends on a kind of Vitali theorem.   I will use the
following notation for the next few paragraphs.

\medskip

\noindent}{\bf Definition} If $(X,\rho)$ is a metric space and
$A\subseteq X$, write $A^{\sim}$ for
$\{x:x\in X,\,\rho(x,A)\le 2\diam A\}$, where
$\rho(x,A)=\inf_{y\in A}\rho(x,y)$.   (\cmmnt{Following the
conventions of 471A, }$\emptyset^{\sim}=\emptyset$.)

\leader{471N}{Lemma} Let $(X,\rho)$ be a metric space.
Let $\Cal F$ be a family of subsets of $X$ such that
$\{\diam F:F\in\Cal F\}$ is bounded.   Set

\Centerline{$Y
=\bigcap_{\delta>0}\bigcup\{F:F\in\Cal F,\,\diam F\le\delta\}$.}

\noindent Then there is a disjoint family $\Cal I\subseteq\Cal F$ such
that

(i) $\bigcup\Cal F\subseteq\bigcup_{F\in\Cal I}F^{\sim}$;

(ii) $Y
\subseteq\overline{\bigcup\Cal J}
  \cup\bigcup_{F\in\Cal I\setminus\Cal J}F^{\sim}$
for every $\Cal J\subseteq\Cal I$.

\proof{{\bf (a)} Let $\gamma$ be an upper bound for
$\{\diam F:F\in\Cal F\}$.   Choose
$\sequencen{\Cal I_n}$, $\sequencen{\Cal J_n}$ inductively, as follows.
$\Cal I_0=\emptyset$.   Given $\Cal I_n$, set
$\Cal F_n'=\{F:F\in\Cal F,\,\diam F\ge 2^{-n}\gamma,\,
F\cap\bigcup\Cal I_n=\emptyset\}$, and let $\Cal J_n\subseteq\Cal F'_n$
be a maximal disjoint set;  now set $\Cal I_{n+1}=\Cal I_n\cup\Cal J_n$,
and continue.

At the end of the induction, set

\Centerline{$\Cal I'=\bigcup_{n\in\Bbb N}\Cal I_n$,
\quad$\Cal I
=\Cal I'\cup\{\{x\}:x\in F\setminus\bigcup\Cal I',\,\{x\}\in\Cal F\}$.}

\noindent The construction ensures that every $\Cal I_n$ is a disjoint
subset of $\Cal F$, so $\Cal I'$ and $\Cal I$ are also disjoint
subfamilies of $\Cal F$.

\medskip

{\bf (b)} \Quer\ Suppose, if possible, that there is a point $x$ in
$\bigcup\Cal F\setminus\bigcup_{F\in\Cal I}F^{\sim}$.   Let $F\in\Cal F$
be such that $x\in F$.   Since $x\notin\bigcup\Cal I'$ and
$\{x\}\notin\Cal I$, $\{x\}\notin\Cal F$, and $\diam F>0$;  let
$n\in\Bbb N$ be such that $2^{-n}\gamma\le\diam F\le 2^{-n+1}\gamma$.
If $F\notin\Cal F'_n$, there is a $D\in\Cal I_n$ such that
$F\cap D\ne\emptyset$;  otherwise, since $\Cal J_n$ is maximal and
$F\notin\Cal J_n$, there is a $D\in\Cal J_n$ such that
$F\cap D\ne\emptyset$.   In either case, we have a $D\in\Cal I$ such
that $F\cap D\ne\emptyset$ and $\diam F\le 2\diam D$.   But in this case
$\rho(x,D)\le\diam F\le 2\diam D$ and $x\in D^{\sim}$, which is
impossible.\ \Bang

\medskip

{\bf (c)} \Quer\ Suppose, if possible, that there are a point $x\in Y$
and a set $\Cal J\subseteq\Cal I$ such that
$x\notin\overline{\bigcup\Cal J}
  \cup\bigcup_{F\in\Cal I\setminus\Cal J}F^{\sim}$.   Then there is an
$F\in\Cal F$ such that $x\in F$ and
$\diam F<\rho(x,\bigcup\Cal J)$, so that
$F\cap\bigcup\Cal J=\emptyset$.   As in (b), $F$ cannot be $\{x\}$, and
there must be an $n\in\Bbb N$ such that
$2^{-n}\gamma<\diam F\le 2^{-n+1}\gamma$.   As in (b), there must be a
$D\in\Cal I_{n+1}$ such that $F\cap D\ne\emptyset$, so that
$x\in D^{\sim}$;  and as $D$ cannot belong to $\Cal J$, we again have a
contradiction.\ \Bang
}%end of proof of 471N

\leader{471O}{Lemma} Let $(X,\rho)$ be a metric space, and $r>0$.
Suppose that $A$, $\Cal F$ are such that

\inset{(i) $\Cal F$ is a family of closed subsets of $X$ such that
$\sum_{n=0}^{\infty}(\diam F_n)^r$ is finite for every disjoint sequence
$\sequencen{F_n}$ in $\Cal F$,

(ii) for every $x\in A$, $\delta>0$ there is an $F\in\Cal F$ such that
$x\in F$ and $0<\diam F\le\delta$.}

\noindent Then there is a countable disjoint family
$\Cal I\subseteq\Cal F$ such
that $A\setminus\bigcup\Cal I$ has zero $r$-dimensional Hausdorff
measure.

\proof{ Replacing $\Cal F$ by $\{F:F\in\Cal F,\,0<\diam F\le 1\}$ if
necessary, we may suppose that $\sup_{F\in\Cal F}\diam F$ is finite and
that $\diam F>0$ for every $F\in\Cal F$.   Take a disjoint family
$\Cal I\subseteq\Cal F$ as in 471N.   If $\Cal I$ is finite, then
$A\subseteq Y\subseteq\bigcup\Cal I$, where $Y$ is defined as in 471N,
so we can stop.   Otherwise, hypothesis (i) tells us that
$\{F:F\in\Cal I,\,\diam F\ge\delta\}$ is finite for every $\delta>0$, so
$\Cal I$ is countable;  enumerate it as $\sequencen{F_n}$;  we must have
$\sum_{n=0}^{\infty}(\diam F_n)^r<\infty$.   Since
$\diam F_n^{\sim}\le 5\diam F_n$ for every $n$,
$\sum_{n=0}^{\infty}(\diam F_n^{\sim})^r$ is finite, and
$\inf_{n\in\Bbb N}\sum_{i=n}^{\infty}(\diam F_i^{\sim})^r=0$.   But now
observe that the construction ensures that
$A\setminus\bigcup\Cal I\subseteq\bigcup_{i\ge n}F_i^{\sim}$ for every
$n\in\Bbb N$.   By 471K, $\mu_{Hr}(A\setminus\bigcup\Cal I)=0$, as
required.
}%end of proof of 471O

\leader{471P}{Theorem} Let $(X,\rho)$ be a metric space, and $r>0$.
Let $\mu_{Hr}$ be $r$-dimensional Hausdorff measure on $X$.   Suppose
that $A\subseteq X$ and $\mu_{Hr}^*A<\infty$.

(a) $\lim_{\delta\downarrow 0}
  \sup\{\Bover{\mu_{Hr}^*(A\cap D)}{(\diam D)^r}:
    x\in D,\,0<\diam D\le\delta\}
=1$ for $\mu_{Hr}$-almost every $x\in A$.

(b) $\limsup_{\delta\downarrow 0}
  \Bover{\mu_{Hr}^*(A\cap B(x,\delta))}{\delta^r}\ge 1$ for
$\mu_{Hr}$-almost every $x\in A$.   So

\Centerline{$2^{-r}
\le\limsup_{\delta\downarrow 0}
  \Bover{\mu_{Hr}^*(A\cap B(x,\delta))}{(\diam B(x,\delta))^r}
\le 1$}

\noindent for $\mu_{Hr}$-almost every $x\in A$.

(c) If $A$ is measured by $\mu_{Hr}$, then

\Centerline{$\lim_{\delta\downarrow 0}
  \sup\{\Bover{\mu_{Hr}^*(A\cap D)}{(\diam D)^r}:
    x\in D,\,0<\diam D\le\delta\}
=0$}

\noindent for $\mu_{Hr}$-almost every $x\in X\setminus A$.

\proof{{\bf (a)(i)} Note first that as the quantities

\Centerline{$\sup\{\Bover{\mu_{Hr}^*(A\cap D)}{(\diam D)^r}:
x\in D,\,0<\diam D\le\delta\}$}

\noindent decrease with $\delta$, the limit is defined in $[0,\infty]$
for every $x\in X$.   Moreover, since $\diam D=\diam\overline{D}$ and
$\mu_{Hr}^*(A\cap\overline{D})\ge\mu_{Hr}^*(A\cap D)$ for every $D$,

$$\eqalign{&\sup\{\Bover{\mu_{Hr}^*(A\cap D)}{(\diam D)^r}:x
\in D\subseteq X,\,0<\diam D\le\delta\}\cr
&\qquad\qquad\qquad
=\sup\{\Bover{\mu_{Hr}^*(A\cap F)}{(\diam F)^r}:
    F\subseteq X\text{ is closed},\,x\in F,\,0<\diam F\le\delta\}\cr}$$

\noindent for every $x$ and $\delta$.

\medskip

\quad{\bf (ii)} Fix $\epsilon$ for the moment, and set

\Centerline{$A_{\epsilon}=\{x:x\in A,\,\lim_{\delta\downarrow 0}
  \sup\{\Bover{\mu_{Hr}^*(A\cap D)}{(\diam D)^r}:
    x\in D,\,0<\diam D\le\delta\}
>1+\epsilon\}$.}

\noindent Then $\theta_{r\eta}(A)
\le\mu_{Hr}^*A-\Bover{\epsilon}{1+\epsilon}\mu^*_{Hr}A_{\epsilon}$ for
every
$\eta>0$, where $\theta_{r\eta}$ is defined in 471A.   \Prf\ Let
$\Cal F$ be the family

\Centerline{$\{F:F\subseteq X$ is closed, $0<\diam F\le\eta$,
$(1+\epsilon)(\diam F)^r\le\mu_{Hr}^*(A\cap F)\}$.}

\noindent Then every member of $A_{\epsilon}$ belongs to sets in
$\Cal F$ of arbitrarily small diameter.   Also, if $\sequencen{F_n}$ is
any disjoint sequence in $\Cal F$,

\Centerline{$\sum_{n=0}^{\infty}(\diam F_n)^r
\le\sum_{n=0}^{\infty}\mu^*_{Hr}(A\cap F_n)
\le\mu^*_{Hr}A<\infty$}

\noindent because every $F_n$, being closed, is measured by $\mu_{Hr}$.
(If you like, $F_n\cap A$ is measured by the subspace measure on $A$ for
every $n$.)   So 471O tells us that there is a countable disjoint family
$\Cal I\subseteq\Cal F$ such that $A_{\epsilon}\setminus\bigcup\Cal I$
is negligible, and
$\mu_{Hr}^*A_{\epsilon}=\mu_{Hr}^*(A_{\epsilon}\cap\bigcup\Cal I)$.

Because $\theta_{r\eta}$ is an outer measure and
$\theta_{r\eta}\le\mu_{Hr}^*$,

$$\eqalignno{\theta_{r\eta}A
&\le\theta_{r\eta}(A\cap\bigcup\Cal I)
  +\theta_{r\eta}(A\setminus\bigcup\Cal I)
\le\sum_{F\in\Cal I}(\diam F)^r+\mu_{Hr}^*(A\setminus\bigcup\Cal I)\cr
\displaycause{because $\Cal I$ is countable}
&\le\Bover1{1+\epsilon}\mu_{Hr}^*(A\cap\bigcup\Cal I)
  +\mu_{Hr}^*(A\setminus\bigcup\Cal I)
=\mu_{Hr}^*A
  -\Bover{\epsilon}{1+\epsilon}\mu_{Hr}^*(A\cap\bigcup\Cal I)\cr
&\le\mu_{Hr}^*A-\Bover{\epsilon}{1+\epsilon}
      \mu_{Hr}^*(A_{\epsilon}\cap\bigcup\Cal I)
=\mu_{Hr}^*A
  -\Bover{\epsilon}{1+\epsilon}\mu_{Hr}^*A_{\epsilon},\cr}$$

\noindent as claimed.\ \Qed

\medskip

\quad{\bf (iii)} Taking the supremum as $\eta\downarrow 0$,
$\mu_{Hr}^*A
\le\mu_{Hr}^*A-\Bover{\epsilon}{1+\epsilon}\mu_{Hr}^*A_{\epsilon}$ and
$\mu_{Hr}A_{\epsilon}=0$.

\woddheader{471P}{0}{0}{0}{40pt}

This is true for any $\epsilon>0$.   But

\Centerline{$\{x:x\in A,\,\lim_{\delta\downarrow 0}
  \sup\{\Bover{\mu_{Hr}^*(A\cap D)}{(\diam D)^r}:
    x\in D,\,0<\diam D\le\delta\}
>1\}$}

\noindent is just $\bigcup_{n\in\Bbb N}A_{2^{-n}}$, so is negligible.

\medskip

\quad{\bf (iv)} Next, for $0<\epsilon\le 1$, set

$$\eqalign{A'_{\epsilon}
=\{x:x\in A,\,\mu_{Hr}^*(A\cap D)&\le(1-\epsilon)(\diam D)^r\cr
&\text{ whenever }x\in D\text{ and }0<\diam D\le\epsilon\}.\cr}$$

\noindent Then $A'_{\epsilon}$ is negligible.   \Prf\ Let
$\sequencen{D_n}$ be any sequence of sets of diameter at most $\epsilon$
covering $A'_{\epsilon}$.   Set
$K=\{n:D_n\cap A'_{\epsilon}\ne\emptyset\}$.   Then

$$\eqalign{\mu_{Hr}^*A'_{\epsilon}
&\le\sum_{n\in K}\mu_{Hr}^*(A\cap D_n)\cr
&\le(1-\epsilon)\sum_{n\in K}(\diam D_n)^r
\le(1-\epsilon)\sum_{n=0}^{\infty}(\diam D_n)^r.\cr}$$

\noindent As $\sequencen{D_n}$ is arbitrary,

\Centerline{$\mu_{Hr}^*A'_{\epsilon}
\le(1-\epsilon)\theta_{r\epsilon}A'_{\epsilon}
\le(1-\epsilon)\mu_{Hr}^*A'_{\epsilon}$,}

\noindent and $\mu_{Hr}^*A'_{\epsilon}$ (being finite) must be zero.\
\Qed

This means that

\Centerline{$\{x:x\in A,\,\lim_{\delta\downarrow 0}
  \sup\{\Bover{\mu_{Hr}^*(A\cap D)}{(\diam D)^r}:
    x\in D,\,0<\diam D\le\delta\}<1\}
\subseteq\bigcup_{n\in\Bbb N}A'_{2^{-n}}$}

\noindent is also negligible, and we have the result.

\medskip

{\bf (b)} We need a slight modification of the argument in (a)(iv).
This time, for $0<\epsilon\le 1$, set

\Centerline{$\tilde A'_{\epsilon}
=\{x:x\in A,\,\mu_{Hr}^*(A\cap B(x,\delta))\le(1-\epsilon)\delta^r$
whenever $0<\delta\le\epsilon\}$.}

\noindent Then $\mu_{Hr}^*\tilde A'_{\epsilon}\le\epsilon$.   \Prf\ Note
first that, as $\mu_{Hr}\{x\}=0$ for every $x$,
$\mu_{Hr}^*(A\cap B(x,\delta))\le(1-\epsilon)\delta^r$ whenever
$x\in\tilde A'_{\epsilon}$ and $0\le\delta\le\epsilon$.   Let
$\sequencen{D_n}$ be a sequence of sets of diameter at most $\epsilon$
covering $\tilde A'_{\epsilon}$.   Set
$K=\{n:D_n\cap\tilde A'_{\epsilon}\ne\emptyset\}$, and for $n\in K$
choose $x_n\in D_n\cap\tilde A'_{\epsilon}$ and set
$\delta_n=\diam D_n$.   Then
$D_n\subseteq B(x_n,\delta_n)$ and $\delta_n\le\epsilon$ for each $n$,
so $\tilde A'_{\epsilon}\subseteq\bigcup_{n\in K}B(x_n,\delta_n)$ and

$$\eqalign{\mu_{Hr}^*\tilde A'_{\epsilon}
&\le\sum_{n\in K}\mu_{Hr}^*(\tilde A'_{\epsilon}\cap B(x_n,\delta_n))\cr
&\le\sum_{n\in K}(1-\epsilon)\delta_n^r
\le(1-\epsilon)\sum_{n=0}^{\infty}(\diam D_n)^r.\cr}$$

\noindent As $\sequencen{D_n}$ is arbitrary,
$\mu_{Hr}^*\tilde A'_{\epsilon}
\le(1-\epsilon)\mu_{Hr}^*\tilde A'_{\epsilon}$ and
$\tilde A'_{\epsilon}$ must be negligible.\ \Qed

Now

\Centerline{$\{x:x\in A,\,\limsup_{\delta\downarrow 0}
  \Bover{\mu_{Hr}^*(A\cap B(x,\delta))}{\delta^r}<1\}
=\bigcup_{n\in\Bbb N}\tilde A'_{2^{-n}}$}

\noindent is negligible.   As for the second formula, we need note only
that $\diam B(x,\delta)\le 2\delta$ for every $x\in X$,
$\delta>0$ to obtain the first inequality, and apply (a) to obtain the
second.

\medskip

{\bf (c)} Let $\epsilon>0$.   This time, write $\tilde A_{\epsilon}$ for

\Centerline{$\{x:x\in X,\,\lim_{\delta\downarrow 0}
  \sup\{\Bover{\mu_{Hr}^*(A\cap D)}{(\diam D)^r}:
    x\in D,\,0<\diam D\le\delta\}
>\epsilon\}$.}

\noindent Let $E\subseteq A$ be a closed set such that
$\mu(A\setminus E)\le\epsilon^2$ (471De).   For $\eta>0$, let
$\Cal F_{\eta}$ be the family

\Centerline{$\{F:F\subseteq X\setminus E$ is closed, $0<\diam F\le\eta$,
$\mu_{Hr}(A\cap F)\ge\epsilon(\diam F)^r\}$.}

\noindent Just as in (a) above, every point in
$\tilde A_{\epsilon}\setminus E$
belongs to members of $\Cal F_{\eta}$ of arbitrarily small diameter.
If $\familyiI{F_i}$ is a countable disjoint family in $\Cal F_{\eta}$,

\Centerline{$\sum_{i\in I}(\diam F_i)^r
\le\Bover1{\epsilon}\mu_{Hr}(A\setminus E)\le\epsilon$}

\noindent is finite.   There is therefore a countable disjoint family
$\Cal I_{\eta}\subseteq\Cal F_{\eta}$ such that
$\mu_{Hr}((\tilde A_{\epsilon}\setminus E)
  \setminus\bigcup\Cal I_{\eta})=0$.
If $\theta_{r\eta}$ is the outer measure defined in 471A, we have

$$\eqalign{\theta_{r\eta}(\tilde A_{\epsilon}\setminus A)
&\le\theta_{r\eta}(\bigcup\Cal I_{\eta})
  +\theta_{r\eta}(\tilde A_{\epsilon}
      \setminus(E\cup\bigcup\Cal I_{\eta}))\cr
&\le\sum_{F\in\Cal I_{\eta}}(\diam F)^r
  +\mu_{Hr}^*(\tilde A_{\epsilon}\setminus(E\cup\bigcup\Cal I_{\eta}))
\le\epsilon.\cr}$$

\noindent As $\eta$ is arbitrary,
$\mu_{Hr}^*(\tilde A_{\epsilon}\setminus A)\le\epsilon$.   But now

\Centerline{$\{x:x\in X\setminus A,\,\lim_{\delta\downarrow 0}
  \sup\{\Bover{\mu_{Hr}^*(A\cap D)}{(\diam D)^r}:
    x\in D,\,0<\diam D\le\delta\}
>0\}$}

\noindent is $\bigcup_{n\in\Bbb N}\tilde A_{2^{-n}}\setminus A$, and is
negligible.
}%end of proof of 471P

\leader{471Q}{}\cmmnt{ I now come to a remarkable fact about Hausdorff
measures on analytic spaces:  their Borel versions are semi-finite
(471S).   We need some new machinery.

\medskip

\noindent}{\bf Lemma} Let $(X,\rho)$ be a metric space, and $r>0$,
$\delta>0$.   Suppose that $\theta_{r\delta}X$, as defined in 471A, is
finite.

(a) There is a non-negative additive functional $\nu$ on $\Cal PX$ such
that $\nu X=5^{-r}\theta_{r\delta}X$ and
$\nu A\le(\diam A)^r$ whenever $A\subseteq X$ and
$\diam A\le\bover15\delta$.

(b) If $X$ is compact, there is a Radon measure $\mu$ on $X$ such that
$\mu X=5^{-r}\theta_{r\delta}X$ and $\mu G\le(\diam G)^r$ whenever
$G\subseteq X$ is open and $\diam G\le\bover15\delta$.

\proof{{\bf (a)} I use 391E.   If $\theta_{r\delta}X=0$ the result is
trivial.   Otherwise, set $\gamma=5^r/\theta_{r\delta}X$ and define
$\phi:\Cal PX\to[0,1]$ by setting $\phi A=\min(1,\gamma(\diam A)^r)$ if
$\diam A\le\bover15\delta$, $1$ for other $A\subseteq X$.   Now

\inset{whenever $\familyiI{A_i}$ is a finite family of subsets of $X$,
$m\in\Bbb N$ and $\sum_{i\in I}\chi A_i\ge m\chi X$,
then $\sum_{i\in I}\phi A_i\ge m$.}

\noindent\Prf\   Discarding any
$A_i$ for which $\phi A_i=1$, if necessary, we may suppose that
$\diam A_i\le\bover15\delta$ and $\phi A_i=\gamma(\diam A_i)^r$ for
every $i$.   Choose
$\langle I_j\rangle_{j\le m}$, $\langle J_j\rangle_{j<m}$ inductively,
as follows.   $I_0=I$.   Given that $j<m$ and that $I_j\subseteq I$ is
such that $\sum_{i\in I_j}\chi A_i\ge(m-j)\chi X$, apply 471N to
$\{A_i:i\in I_j\}$ to find $J_j\subseteq I_j$ such that
$\family{i}{J_j}{A_i}$ is disjoint and
$\bigcup_{i\in I_j}A_i\subseteq\bigcup_{i\in J_j}A_j^{\sim}$.   Set
$I_{j+1}=I_j\setminus J_j$.   Observe that
$\sum_{i\in J_j}\chi A_j\le\chi X$, so
$\sum_{i\in I_{j+1}}\chi A_i\ge(m-j-1)\chi X$ and the induction
proceeds.

Now note that, for each $j<m$, $\family{i}{J_j}{A_i^{\sim}}$ is a cover
of $\bigcup_{i\in I_j}A_i=X$ by sets of diameter at most $\delta$.   So
$\sum_{i\in J_j}(\diam A_i^{\sim})^r\ge\theta_{r\delta}X$ for each
$j$, and $\sum_{i\in I}(\diam A_i^{\sim})^r\ge m\theta_{r\delta}X$.
Accordingly

$$\eqalign{\sum_{i\in I}\phi A_i
&=\gamma\sum_{i\in I}(\diam A_i)^r
\ge 5^{-r}\gamma\sum_{i\in I}(\diam A_i^{\sim})^r\cr
&\ge 5^{-r}m\gamma\theta_{r\delta}X
=m.  \text{ \Qed}\cr}$$

By 391E, there is an additive functional $\nu_0:\Cal PX\to[0,1]$
such that $\nu_0X=1$ and $\nu_0A\le\phi A$ for every $A\subseteq X$.
Setting $\nu=5^{-r}\theta_{r\delta}X\nu_0$, we have the result.

\medskip

{\bf (b)} Now suppose that $X$ is compact.   By 416K, there is a Radon
measure $\mu$ on $X$ such that $\mu K\ge\nu K$ for every compact
$K\subseteq X$ and $\mu G\le\nu G$ for every open $G\subseteq X$.
Because $X$ itself is compact, $\mu X=\nu X=5^{-r}\theta_{r\delta}X$.
If $G$ is open and $\diam G\le\bover15\delta$,

\Centerline{$\mu G\le\nu G\le(\diam G)^r$,}

\noindent as required.
}%end of proof of 471Q

\leader{471R}{Lemma}\cmmnt{ ({\smc Howroyd 95})} Let $(X,\rho)$ be a
compact metric space and $r>0$.   Let $\mu_{Hr}$ be $r$-dimensional
Hausdorff measure on $X$.   If $\mu_{Hr}X>0$, there is a Borel set
$H\subseteq X$ such that $0<\mu_{Hr}H<\infty$.

\proof{{\bf (a)} Let $\delta>0$ be such that $\theta_{r,5\delta}(X)>0$,
where $\theta_{r,5\delta}$ is defined as in 471A.   Then there is a
family $\Cal V$ of open subsets of $X$ such that (i) $\diam V\le\delta$
for every $V\in\Cal V$ (ii) $\{V:V\in\Cal V,\,\diam V\ge\epsilon\}$ is
finite for every $\epsilon>0$ (iii) whenever $A\subseteq X$ and
$0<\diam A<\bover14\delta$ there is a $V\in\Cal V$ such that
$A\subseteq V$ and $\diam V\le 8\diam A$.   \Prf\ For each $k\in\Bbb N$,
let $I_k$ be a finite subset of $X$ such that
$X=\bigcup_{x\in I_k}B(x,2^{-k-2}\delta)$;  now set
$\Cal V=\{U(x,2^{-k-1}\delta):k\in\Bbb N,\,x\in I_k\}$.   Then $\Cal V$
is a family of open
sets and (i) and (ii) are satisfied.   If $A\subseteq X$ and
$0<\diam A<\bover14\delta$, let $k\in\Bbb N$ be such that
$2^{-k-3}\delta\le\diam A<2^{-k-2}\delta$.   Take $x\in I_k$ such that
$B(x,2^{-k-2}\delta)\cap A\ne\emptyset$;  then
$A\subseteq U(x,2^{-k-1}\delta)\in\Cal V$ and
$\diam U(x,2^{-k-1}\delta)\le 2^{-k}\delta\le 8\diam A$.\ \Qed

In particular, $\{V:V\in\Cal V,\,\diam V\le\epsilon\}$ covers $X$ for
every $\epsilon>0$.

\medskip

{\bf (b)} Set

\Centerline{$P=\{\mu:\mu$ is a Radon measure on $X$,
$\mu V\le(\diam V)^r$ for every $V\in\Cal V\}$.}

\noindent $P$ is non-empty (it contains the zero measure, for instance).
Now if $G\subseteq X$ is open,
$\mu\mapsto\mu G$ is lower semi-continuous for the narrow topology
(437Jd), so $P$ is a closed set in the narrow topology on the set of
Radon measures on
$X$, which may be identified with a subset of $C(X)^*$ with its
weak* topology (437Kc).   Moreover, since there is a finite subfamily of
$\Cal V$ covering
$X$, $\gamma=\sup\{\mu X:\mu\in P\}$ is finite, and $P$ is compact
(437Pb/437Rf).   Because $\mu\mapsto\mu X$ is continuous,
$P_0=\{\mu:\mu\in P,\,\mu X=\gamma\}$ is non-empty.   Of course $P$ and
$P_0$ are both convex, and $P_0$, like $P$, is compact.   By
the \Krein-Mil'man theorem (4A4Gb), applied in $C(X)^*$, $P$ has an
extreme point $\nu$ say.

Note next that $\theta_{r,5\delta}(X)$ is certainly finite, again
because $X$ is compact.   By 471Qb, $\gamma>0$, and $\nu$ is
non-trivial.   For any $\epsilon>0$, there is a finite cover of $X$ by
sets in $\Cal V$ of diameter at most $\epsilon$, which have measure at
most $\epsilon^r$ (for $\nu$);  so $\nu$ is atomless.   In particular,
$\nu\{x\}=0$ for every $x\in X$.

\medskip

{\bf (c)} For $\epsilon>0$, set

\Centerline{$G_{\epsilon}
=\bigcup\{V:V\in\Cal V,\,0<\diam V\le\epsilon$ and
  $\nu V\ge\bover12(\diam V)^r\}$.}

\noindent Then $G_{\epsilon}$ is $\nu$-conegligible.  \Prf\Quer\
Otherwise, $\nu(X\setminus G_{\epsilon})>0$.   Because
$\Cal V'_{\epsilon}=\{V:V\in\Cal V,\,\diam V>\epsilon\}$ is finite,
there is a Borel set $E\subseteq X\setminus G_{\epsilon}$ such that
$\nu E>0$ and, for every $V\in\Cal V'_{\epsilon}$, either $E\subseteq V$
or $E\cap V=\emptyset$.   Because $\nu$ is atomless, there is a
measurable set
$E_0\subseteq E$ such that $\nu E_0=\bover12\nu E$ (215D);  set
$E_1=E\setminus E_0$.

Define Radon measures $\nu_0$, $\nu_1$ on $X$ by setting

\Centerline{$\nu_i(F)=2\nu(F\cap E_i)+\nu(F\setminus E)$}

\noindent whenever $\nu$ measures $F\setminus E_{1-i}$, for each $i$
(use 416S if you feel the need to check that this defines a Radon
measure on the definitions of this book).   If $V\in\Cal V$, then, by
the choice of $E$,

\inset{{\it either} $E\subseteq V$ and $\nu_iV=\nu V\le(\diam V)^r$

{\it or} $E\cap V=\emptyset$ and $\nu_iV=\nu V\le(\diam V)^r$

{\it or} $0<\diam V\le\epsilon$ and $\nu V<\bover12(\diam V)^r$, in
which case $\nu_iV\le 2\nu V\le(\diam V)^r$

{\it or} $\diam V=0$ and $\nu_iV=\nu V=0=(\diam V)^r$.}

\noindent So both $\nu_i$ belong to $P$ and therefore to $P_0$, since
$\nu_iX=\nu X=\gamma$.   But $\nu=\bover12(\nu_0+\nu_1)$ and
$\nu_0\ne\nu_1$, so this is impossible, because $\nu$ is supposed to be
an extreme point of $P_0$.\ \Bang\Qed

\medskip

{\bf (d)} Accordingly, setting $H=\bigcap_{n\in\Bbb N}G_{2^{-n}}$,
$\nu H=\nu X=\gamma$.   Now examine $\mu_{Hr}H$.

\medskip

\quad{\bf (i)} $\mu_{Hr}H\ge 8^{-r}\gamma$.   \Prf\ Let
$\sequencen{A_n}$ be a sequence of sets covering $H$ with
$\diam A_n\le\bover18\delta$ for every $n$.   Set $K=\{n:\diam A_n>0\}$,
$H'=H\cap\bigcup_{n\in K}A_n$;  then $H\setminus H'$ is countable, so
$\nu H'=\nu H$.   For each $n\in K$, let $V_n\in\Cal V$ be such that
$A_n\subseteq V_n$ and $\diam V_n\le 8\diam A_n$ ((a) above).   Then

$$\eqalignno{\sum_{n=0}^{\infty}(\diam A_n)^r
&=\sum_{n\in K}(\diam A_n)^r
\ge 8^{-r}\sum_{n\in K}(\diam V_n)^r\cr
&\ge 8^{-r}\sum_{n\in K}\nu V_n
\ge 8^{-r}\nu H'
=8^{-r}\gamma.\cr}$$

\noindent As $\sequencen{A_n}$ is arbitrary,

\Centerline{$8^{-r}\gamma\le\theta_{r,\delta/8}(H)\le\mu_{Hr}^*H
=\mu_{Hr}H$.  \Qed}

\medskip

\quad{\bf (ii)} $\mu_{Hr}H\le 2\gamma$.   \Prf\ Let $\eta>0$.   Set
$\Cal F=\{\overline{V}:V\in\Cal V,\,0<\diam V\le\eta,\,
\nu V\ge\bover12(\diam V)^r\}$.   Then $\Cal F$ is a family of closed
subsets of $X$, and (by the definition of $G_{\epsilon}$) every member
of $H$ belongs to members of $\Cal F$ of arbitrarily small diameter.
Also $\nu F\ge\bover12(\diam F)^r$ for every $F\in\Cal F$, so

\Centerline{$\sum_{n=0}^{\infty}(\diam F_n)^r
\le 2\sum_{n=0}^{\infty}\nu F_n<\infty$}

\noindent for any disjoint sequence $\sequencen{F_n}$ in $\Cal F$.
By 471O, there is a countable disjoint family $\Cal I\subseteq\Cal F$
such that $\mu_{Hr}(H\setminus\bigcup\Cal I)=0$.   Accordingly

\Centerline{$\theta_{r\eta}(H)
\le\sum_{F\in\Cal I}(\diam F)^r+\theta_{r\eta}(H\setminus\bigcup\Cal I)
\le\sum_{F\in\Cal I}2\nu F
\le 2\gamma$.}

\noindent As $\eta$ is arbitrary, $\mu_{Hr}H=\mu_{Hr}^*H\le 2\gamma$.\
\Qed

\medskip

{\bf (e)} But this means that we have found a Borel set $H$ with
$0<\mu_{Hr}H<\infty$, as required.
}%end of proof of 471R

\leader{471S}{Theorem}\cmmnt{ ({\smc Howroyd 95})} Let $(X,\rho)$ be
an analytic metric space, and $r>0$.   Let
$\mu_{Hr}$ be $r$-dimensional Hausdorff measure on $X$, and $\Cal B$ the
Borel $\sigma$-algebra of $X$.   Then the Borel measure
$\mu_{Hr}\restr\Cal B$ is semi-finite and tight\cmmnt{ (that is, inner
regular with respect to the closed compact sets)}.

\proof{ Suppose that $E\in\Cal B$ and $\mu_{Hr}E>0$.   Since
$E$ is analytic (423Eb), 471I above tells us that there is a compact set
$K\subseteq E$ such that $\mu_{Hr}K>0$.   Next, by 471R, there is
a Borel set $H\subseteq K$ such that $0<\mu_{Hr}H<\infty$.   (Strictly
speaking, $\mu_{Hr}H$ here should be calculated as the $r$-dimensional
Hausdorff measure of $H$ defined by the subspace metric
$\rho\restr K\times K$ on $K$.   By 471E we do not need to distinguish
between this and the $r$-dimensional measure calculated from $\rho$
itself.)   By 471I again (applied to the subspace metric on $H$), there
is a compact set $L\subseteq H$ such that $\mu_{Hr}L>0$.

Thus $E$ includes a non-negligible compact set of finite measure.   As
$E$ is arbitrary, this is enough to show both that
$\mu_{Hr}\restr\Cal B$ is semi-finite and that it is tight.
}%end of proof of 471S

\leader{471T}{Proposition} Let $(X,\rho)$ be a metric space, and $r>0$.

(a) If $X$ is analytic and
$\mu_{Hr}X>0$, then for every $s\in\ooint{0,r}$
there is a non-zero Radon measure $\mu$ on $X$ such that
$\biggeriint\Bover1{\rho(x,y)^s}\mu(dx)\mu(dy)<\infty$.

(b) If there is a non-zero topological measure $\mu$ on $X$ such that
$\biggeriint\Bover1{\rho(x,y)^r}\mu(dx)\mu(dy)$ is finite, then
$\mu_{Hr}X=\infty$.

\proof{{\bf (a)} By 471S, there is a compact set $K\subseteq X$ such that
$\mu_{Hr}K>0$.   Set $\delta=5\diam K$ and define $\theta_{r\delta}$ as in
471A.  Then $\theta_{r\delta}K>0$, by 471K, and
$\theta_{r\delta}K\le(\diam K)^r<\infty$.   By 471Qb, there is
a Radon measure $\nu$ on $K$ such that $\nu K>0$ and $\nu G\le(\diam G)^r$
whenever $G\subseteq K$ is relatively
open;  consequently $\nu^*A\le(\diam A)^r$ for
every $A\subseteq K$.   Now, for any $y\in X$,

$$\eqalign{\int_K\Bover1{\rho(x,y)^s}\nu(dx)
&=\int_0^{\infty}\nu\{x:x\in K,\,\Bover1{\rho(x,y)^s}\ge t\}dt
=\int_0^{\infty}\nu\{x:x\in K,\,\rho(x,y)\le\Bover1{t^{1/s}}\}dt\cr
&=\int_0^{\infty}\nu(K\cap B(y,\Bover1{t^{1/s}}))dt
\le\int_0^{\infty}(\diam(K\cap B(y,\Bover1{t^{1/s}})))^rdt\cr
&\le\int_0^{\infty}(\min(\diam K,\Bover2{t^{1/s}}))^rdt
\le 2^r\int_0^{\infty}\min((\diam K)^r,\Bover1{t^{r/s}})dt
<\infty\cr}$$

\noindent because $r>s$.   It follows at once that
$\int_K\int_K\Bover1{\rho(x,y)^s}\nu(dx)\nu(dy)$ is finite.   Taking $\mu$
to be the extension of $\nu$ to a Radon measure on $X$ for which
$X\setminus K$ is negligible, we have an appropriate $\mu$.

\medskip

{\bf (b)(i)} We can suppose that $X$ is separable (471Df).
Since the integrand
is strictly positive, $\mu$ must be $\sigma$-finite, so that
there is no difficulty with the repeated integral.   Replacing $\mu$ by
$\mu\LLcorner F$ for some set $F$ of non-zero finite measure, we can
suppose that $\mu$ is totally finite;  and replacing $\mu$ by a scalar
multiple of itself, we can suppose that it is a probability measure.

\medskip

\quad{\bf (ii)} Let $\epsilon>0$.   Let $H$ be the conegligible set
$\{y:\biggerint\Bover1{\rho(x,y)^r}\mu(dx)<\infty\}$.
For any $y\in X$, $\mu\{y\}=0$, so

\Centerline{$\lim_{\delta\downarrow 0}
\int_{B(y,\delta)}\Bover1{\rho(x,y)^r}\mu(dx)=0$}

\noindent for every $y\in H$.   For each $\delta>0$,

\Centerline{$(x,y)\mapsto\Bover{\chi B(y,\delta)(x)}{\rho(x,y)^r}:
X\times X\to[0,\infty]$}

\noindent is Borel measurable, so

\Centerline{$y\mapsto\int_{B(y,\delta)}\Bover1{\rho(x,y)^r}\mu(dx):
X\to[0,\infty]$}

\noindent is Borel measurable (252P, applied to the restriction of $\mu$ to
the Borel $\sigma$-algebra of $X$).   There is therefore a $\delta>0$ such
that $E=\{y:y\in H$,
$\biggerint_{B(y,\delta)}\Bover1{\rho(x,y)^r}\mu(dy)\le\epsilon\}$ has
measure
$\mu E\ge\bover12$.   Note that if $C\subseteq X$ has diameter less than or
equal to $\delta$ and meets $E$ then
$\mu\overline{C}\le\epsilon(\diam C)^r$.   \Prf\ Set $\gamma=\diam C$ and
take $y\in C\cap E$.   If $C=\{y\}$ then $\mu\overline{C}=0$.   Otherwise,

\Centerline{$\mu\overline{C}
\le\mu B(y,\gamma)
\le\gamma^r\int_{B(y,\delta)}\Bover1{\rho(x,y)^r}\mu(dx)
\le\gamma^r\epsilon$.  \Qed}

Now suppose that $E\subseteq\bigcup_{i\in I}C_i$ where
$\diam C_i\le\delta$ for every $i$, and each $C_i$ is either empty or meets
$E$.   Then

\Centerline{$\Bover12\le\mu E\le\sum_{i=0}^{\infty}\mu\overline{C}_i
\le\sum_{i=0}^{\infty}\epsilon(\diam C_i)^r$.}

\noindent As $\sequence{i}{C_i}$ is arbitrary,
$\epsilon\mu_{Hr}E\ge\Bover12$ and $\mu_{Hr}X\ge\Bover1{2\epsilon}$.
As $\epsilon$ is arbitrary, $\mu_{Hr}X=\infty$.
}%end of proof of 471T

\exercises{\leader{471X}{Basic exercises (a)}
%\spheader 471Xa
Define a metric $\rho$ on $X=\{0,1\}^{\Bbb N}$ by
setting $\rho(x,y)=2^{-n}$ if $x\restr n=y\restr n$ and $x(n)\ne y(n)$.
Show that the usual measure $\mu$ on $X$ is one-dimensional Hausdorff
measure.   \Hint{$\diam F\ge\mu F$ for every closed set $F\subseteq X$.}
%471A

\spheader 471Xb Let $(X,\rho)$ be a metric space and $r>0$;  let
$\mu_{Hr}$, $\theta_{r\infty}$ be $r$-dimensional Hausdorff measure and
capacity on $X$.   (i) Show that, for $A\subseteq X$, $\mu_{Hr}A=0$ iff
$\theta_{r\infty}A=0$.   (ii)  Suppose that $E\subseteq X$ and
$\delta>0$ are such that
$\delta\mu_{Hr}E<\theta_{r\infty}E$.   Show that there is a closed set
$F\subseteq E$ such that $\mu_{Hr}F>0$ and
$\delta\mu_{Hr}(F\cap G)\le(\diam G)^r$ whenever $\mu_{Hr}$ measures
$G$.   \Hint{show that
$\{G^{\ssbullet}:\theta_{r\infty}G<\delta\mu_{Hr}G\}$ cannot
be order-dense in the measure algebra of $\mu_{Hr}$.   This is a version
of `Frostman's Lemma'.}   (iii) Let $\Cal C$ be the family of closed
subsets of $X$, with its Vietoris topology.
Show that $\theta_{r\infty}\restr\Cal C$ is upper semi-continuous.
%471H

\spheader 471Xc Let $(X,\rho)$ be an analytic metric space, $(Y,\sigma)$
a metric space, and $f:X\to Y$ a Lipschitz function.   Show that if $r>0$
and $A\subseteq X$ is measured by Hausdorff $r$-dimensional measure on
$X$, with finite measure, then $f[A]$ is measured by Hausdorff
$r$-dimensional measure on $Y$.
%471J

\spheader 471Xd Let $(X,\rho)$ be a metric space and $r>0$.   Show that
a set $A\subseteq X$ is negligible for Hausdorff $r$-dimensional measure
on $X$ iff there is a sequence $\sequencen{A_n}$ of subsets of $X$ such
that $\sum_{n=0}^{\infty}(\diam A_n)^r$ is finite and
$A\subseteq\bigcap_{n\in\Bbb N}\bigcup_{m\ge n}A_m$.
%471K

\spheader 471Xe Let $(X,\rho)$ be a metric space.   (i) Show that there
is a unique $\dim_H(X)\in[0,\infty]$ such that the $r$-dimensional
Hausdorff measure of $X$ is infinite if $0<r<\dim_H(X)$, zero if
$r>\dim_H(X)$.   ($\dim_H(X)$ is the {\bf Hausdorff dimension} of $X$.)
(ii) Show that if $\sequencen{A_n}$ is any sequence of subsets of $X$,
then $\dim_H(\bigcup_{n\in\Bbb N}A_n)=\sup_{n\in\Bbb N}\dim_H(A_n)$.
%471L

\spheader 471Xf Let $(X,\rho)$ be a metric space, and $\mu$ any
topological measure on $X$.   Suppose that $E\subseteq X$ and that
$\mu E$ is defined and finite.   (i) Show that
$(x,\delta)\mapsto\mu(E\cap B(x,\delta)):
X\times\coint{0,\infty}\to\Bbb R$ is upper semi-continuous.  (ii) Show
that $x\mapsto\limsup_{\delta\downarrow 0}
\Bover1{\delta^r}\mu(E\cap B(x,\delta)):X\to[0,\infty]$ is Borel
measurable, for every $r\ge 0$.   (iii) Show that if $X$ is separable,
then
$\mu B(x,\delta)>0$ for every $\delta>0$, for almost every $x\in X$.
%471P

\spheader 471Xg Give $\Bbb R$ its usual metric.   Let $C\subseteq\Bbb R$
be the Cantor set, and $r=\ln 2/\ln 3$.   Show that

\Centerline{$\liminf_{\delta\downarrow 0}
  \Bover{\mu_{Hr}(C\cap B(x,\delta))}{(\diam B(x,\delta))^r}
\le 2^{-r}$}

\noindent for every $x\in\Bbb R$.
%471P

\spheader 471Xh Let $(X,\rho)$ be a metric space and $r>0$.   Let
$\mu_{Hr}$ be $r$-dimensional Hausdorff measure on $X$ and
$\tilde\mu_{Hr}$ its c.l.d.\ version (213D-213E).   Show that
$\tilde\mu_{Hr}$ is inner regular with respect to the closed sets, and
that $\tilde\mu_{Hr}A=\mu_{Hr}A$ for every analytic set $A\subseteq X$.
%471D, 471Q

\spheader 471Xi Suppose that $g:\Bbb R\to\Bbb R$ is continuous and
non-decreasing,
and that $\nu$ is the corresponding Lebesgue-Stieltjes measure
(114Xa).   Define $\rho(x,y)=|x-y|+\sqrt{|g(x)-g(y)|}$ for $x$, $y\in\Bbb R$.
Show that $\rho$ is a metric on $\Bbb R$ defining the usual topology.   Show that $\nu$ is $2$-dimensional Hausdorff measure for the metric $\rho$.
%471A

\leader{471Y}{Further exercises (a)}
%\spheader 471Ya
The next few exercises (down to 471Yd) will be based on the
following.    Let $(X,\rho)$ be a metric space and
$\psi:\Cal PX\to[0,\infty]$ a function such that $\psi\emptyset=0$ and
$\psi A\le\psi A'$ whenever $A\subseteq A'\subseteq X$.   Set

$$\eqalign{\theta_{\psi\delta}A
=\inf\{\sum_{n=0}^{\infty}\psi D_n:\sequencen{D_n}
&\text{ is a sequence of subsets of }X\text{ covering }A,\cr
&\qquad\qquad\qquad\quad
  \diam D_n\le\delta\text{ for every }n\in\Bbb N\}\cr}$$

\noindent for $\delta>0$, and
$\theta_{\psi}A=\sup_{\delta>0}\theta_{\psi\delta}A$ for $A\subseteq X$
Show that $\theta_{\psi}$ is a metric outer measure.   Let
$\mu_{\psi}$ be the measure defined from $\theta_{\psi}$ by
\Caratheodory's method.
%471A

\spheader 471Yb Suppose that
$\psi A=\inf\{\psi E:E$ is a Borel set including $A\}$ for every
$A\subseteq X$.   Show that $\theta_{\psi}=\mu_{\psi}^*$ and that
$\mu_{\psi}E=\sup\{\mu_{\psi}F:F\subseteq E$ is closed$\}$ whenever
$\mu_{\psi}E<\infty$.
%471D, 471Ya

\spheader 471Yc Suppose that $X$ is separable and that there is a
$\beta\ge 0$ such that
$\psi A^{\sim}\le\beta\psi A$ for every $A\subseteq X$, where
$A^{\sim}$ is defined in 471M.   (i) Suppose that $A\subseteq X$ and
$\Cal F$ is a family of closed subsets of $X$ such that
$\sum_{n=0}^{\infty}\psi F_n$ is finite for every disjoint sequence
$\sequencen{F_n}$ in $\Cal F$ and for every $x\in A$, $\delta>0$ there
is an $F\in\Cal F$ such that $x\in F$ and $0<\diam F\le\delta$.   Show
that there is a disjoint family $\Cal I\subseteq\Cal F$ such that
$\mu_{\psi}(A\setminus\bigcup\Cal I)=0$.   (ii) Suppose that $\delta>0$
and that $\theta_{\psi\delta}(X)<\infty$.   Show that there is a
non-negative additive functional $\nu$ on $\Cal PX$ such that
$\nu X=\bover1{\beta}\theta_{\psi\delta}(X)$ and $\nu A\le\psi A$
whenever $A\subseteq X$ and $\diam A\le\bover15\delta$.
(iii) Now suppose that for every $x\in X$, $\epsilon>0$ there is a
$\delta>0$ such that $\psi B(x,\delta)\le\epsilon$.
Show that if $X$ is compact and $\mu_{\psi}X>0$ there is a
compact set $K\subseteq X$ such that $0<\mu_{\psi}K<\infty$.
%471O, 471Q, 471R, 471Ya  mt43bits

\spheader 471Yd State and prove a version of 471P appropriate to this
context.
%471P, 471Ya, 471Yc

\spheader 471Ye Give an example of a set $A\subseteq\BbbR^2$ which is
measured by Hausdorff 1-dimensional measure on $\BbbR^2$ but is such
that its projection onto the first coordinate is not measured by
Hausdorff 1-dimensional measure on $\Bbb R$.
%471Xc, 471J

\spheader 471Yf Show that the space $(X,\rho)$ of 471Xa can be
isometrically embedded
as a subset of a metric space $(Y,\sigma)$ in such a way that (i)
$\diam B(y,\delta)=2\delta$ for every $y\in Y$ and $\delta\ge 0$ (ii)
$Y\setminus X$ is countable.   Show that if $\mu_{H1}$ is
one-dimensional Hausdorff measure on $Y$, then
$\mu_{H1}B(y,\delta)\le\delta$ for every $y\in Y$ and $\delta\ge 0$, so
that

\Centerline{$\limsup_{\delta\downarrow 0}
  \Bover{\mu_{H1}B(y,\delta)}{\diam B(x,\delta)}
=\Bover12$}

\noindent for every $y\in Y$.
%471P

\spheader 471Yg Let $\sequencen{k_n}$ be a sequence in
$\Bbb N\setminus \{0,1,2,3\}$ such that
$\sum_{n=0}^{\infty}\Bover1{k_n}<\infty$.   Set
$X=\prod_{n\in\Bbb N}k_n$.
% (regarding each $k_n$ as the set of its predecessors).   %471G
Set $m_0=1$, $m_{n+1}=k_0k_1\ldots k_n$ for
$n\in\Bbb N$.   Define a metric $\rho$ on $X$ by saying that

$$\eqalign{\rho(x,y)
&=1/2m_n\text{ if }n=\min\{i:x(i)\ne y(i)\}
  \text{ and }\min(x(n),y(n))=0,\cr
&=1/m_n\text{ if }n=\min\{i:x(i)\ne y(i)\}
  \text{ and }\min(x(n),y(n))>0.\cr}$$

\noindent Let $\nu$ be the product measure on $X$ obtained by giving
each factor $k_n$ the uniform probability measure in which each
singleton set has measure $1/k_n$.   (i) Show that if $A\subseteq X$
then $\nu^*A\le\diam A$.   (ii) Show that $\nu$ is one-dimensional
Hausdorff measure on $X$.   (iii) Set
$E=\bigcup_{n\in\Bbb N}\{x:x\in X,\,x(n)=0\}$.   Show that $\nu E<1$.
(iv) Show that

\Centerline{$\limsup_{\delta\downarrow 0}
\Bover{\nu(E\cap B(x,\delta))}{\nu B(x,\delta)}\ge\Bover12$}

\noindent for every $x\in X$.   (v) Show that there is a family $\Cal F$
of closed balls in $X$ such that every point of $X$ is the centre of
arbitrarily small members of $\Cal F$, but $\nu(\bigcup\Cal I)<1$ for
any disjoint subfamily $\Cal I$ of $\Cal F$.
%471P   mt43bits

\spheader 471Yh Let $\rho$ be the metric on $\{0,1\}^{\Bbb N}$ defined
in 471Xa.
Show that for any integer $s\ge 1$ there is a bijection
$f:[0,1]^s\to\{0,1\}^{\Bbb N}$ such that whenever $0<r\le s$,
$\mu^*_{Hr}$ is Hausdorff $r$-dimensional
outer measure on $[0,1]^s$ (for its usual metric) and
$\tilde\mu^*_{H,r/s}$ is Hausdorff $\bover{r}{s}$-dimensional measure on
$\{0,1\}^{\omega}$, then there is an
$\alpha>0$ such that
$\mu_{Hr}^*f^{-1}[A]\le\mu^*_{H,r/s}A\le\alpha\mu_{Hr}^*f^{-1}[A]$
for every $A\subseteq\{0,1\}^{\Bbb N}$.
%+

\spheader 471Yi Let $(X,\rho)$ be a metric space, and $r>0$;  let $s\ge
1$ be an integer.   Write $\mu_{Hr}^{(X)}$ and
$\mu_{H,r+s}^{(X\times\BbbR^s)}$ for
Hausdorff $r$-dimensional measure on $X$ and Hausdorff $(r+s)$-dimensional
measure on $X\times\BbbR^s$ respectively, and $\mu_{Ls}$ for Lebesgue
measure on $\BbbR^s$.
(i) Show that there are non-zero constants $c$, $c'$ such that
$c\mu_{H,r+s}^{(X\times\BbbR^s)}(E\times F)
\le\mu_{Hr}^{(X)}(E)\cdot\mu_{Ls}(F)
\le c'\mu_{H,r+s}^{(X\times\BbbR^s)}(E\times F)$ for all Borel
sets $E\subseteq X$, $F\subseteq\BbbR^s$.   \Hint{{\smc Federer 69},
2.10.45.}   (ii) Write $\lambda$ for the c.l.d.\ product of
$\mu_{Hr}^{(X)}$ and $\mu_{Ls}$, and
$\tilde\mu_{H,r+s}^{(X\times\BbbR^s)}$ for the c.l.d.\ version of
$\mu_{H,r+s}^{(X\times\BbbR^s)}$.   Show that these have the same domain
$\Lambda$ and that
$c\tilde\mu_{H,r+s}^{(X\times\BbbR^s)}(W)\le\lambda W\le
c'\tilde\mu_{H,r+s}^{(X\times\BbbR^s)}(W)$ for every $W\in\Lambda$.
%+

\spheader 471Yj Let $(X,\rho)$ be a metric space and $0\le s<t$.
Suppose that there is an analytic set $A\subseteq X$ such that
$\mu_{Ht}A>0$.
Show that there is a Borel surjection $f:X\to\Bbb R$ such that
$\mu_{Hs}f^{-1}[\{\alpha\}]\ge 1$ for every $\alpha\in\Bbb R$.
%471S

\spheader 471Yk Let $(X,\rho)$ be a separable metric space and
$r>0$.   Suppose that there is an atomless Borel probability measure $\mu$
on $X$ such that $\iint\Bover1{\rho(x,y)^r}\mu(dx)\mu(dy)$ is finite.
Show that $X$ has infinite $r$-dimensional Hausdorff measure.
%the integral is called the `$r$-energy' of $\mu$

\spheader 471Yl\dvAnew{2010} Let $(X,\rho)$ be a metric space, and $\mu$,
$\nu$ two non-zero quasi-Radon measures on $X$
such that $\mu B(x,\delta)=\mu B(y,\delta)$
and $\nu B(x,\delta)=\nu B(y,\delta)$ for all $\delta>0$ and $x$, $y\in X$.
Show that $\mu$ is a multiple of $\nu$.   \Hint{442B.}
%442B out of order query mt47bits
}%end of exercises

\endnotes{
\Notesheader{471} In the exposition above, I have worked throughout with
simple $r$-dimensional measures for $r>0$.   As noted in 264Db, there
are formulae in which it is helpful to interpret $\mu_{H0}$ as counting
measure.   More interestingly, when we come to use Hausdorff measures to
give us information about the geometric structure of an object (e.g., in
the observation that the Cantor set has $\ln 2/\ln 3$-dimensional
Hausdorff measure $1$, in 264J), it is sometimes useful to refine the
technique by using other functionals than $A\mapsto(\diam A)^r$ in the
basic formulae of 264A or 471A.   The most natural generalization is to
functionals of the form $\psi A=h(\diam A)$ where
$h:[0,\infty]\to[0,\infty]$ is a non-decreasing function (264Yo).   But
it is easy to see that many of the arguments are valid in much greater
generality, as in 471Ya-471Yc.   %471Ya 471Yb 471Yc 471Yc
For more in these directions see {\smc Rogers 70} and {\smc Federer 69}.

In the context of this book, the most conspicuous peculiarity of
Hausdorff measures is that they are often very far from being
semi-finite.   (This is trivial for non-separable spaces, by 471Df.
That Hausdorff one-dimensional measure on a subset of $\BbbR^2$ can be
purely infinite is not I think obvious;  I gave an example in 439H.)
The response I ordinarily recommend in such cases is to take the c.l.d.\
version.   But then of course we need to know just what effect this will
have.   In geometric applications, one usually begins by checking that
the sets one is interested in have $\sigma$-finite measure, and that
therefore no problems arise;  but it is a striking fact that Hausdorff
measures behave relatively well on analytic sets, even when not
$\sigma$-finite, provided we ask exactly the right questions (471I,
471S, 471Xh).

The geometric applications of Hausdorff measures, naturally, tend to
rely heavily on density theorems;  it is therefore useful to know that
we have effective versions of Vitali's theorem available in this context
(471N-471O), leading to a general density theorem (471P) similar to that
in 261D;  see also 472D below.   I note that 471P is useful only
after we have
been able to concentrate our attention on a set of finite measure.   And
traps remain.   For instance, the formulae of 261C-261D cannot be
transferred to the present context without re-evaluation (471Yg).

}%end of notes

\discrpage

