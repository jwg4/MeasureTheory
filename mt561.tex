\frfilename{mt561.tex}
\versiondate{8.9.13}
\copyrightdate{2006}

\def\chaptername{Choice and determinacy}
\def\sectionname{Analysis without choice}

\newsection{561}

Elementary courses in analysis
are often casual about uses of weak forms of choice;  a typical argument
runs `for every $\epsilon>0$ there is an $a\in A$ such that
$|a-x|\le\epsilon$, so there is a sequence in $A$ converging to $x$'.
This is a direct call on the countable axiom of choice:  setting
$A_n=\{a:a\in A$, $|a-x|\le 2^{-n}\}$, we are told that every $A_n$ is
non-empty, and conclude that $\prod_{n\in\Bbb N}A_n$ is non-empty.   In the
present section I will abjure such methods and investigate what can
still be done with the ideas important in measure theory.
We have useful partial versions of Tychonoff's theorem (561D), Baire's
theorem (561E), Stone's theorem (561F) and Kakutani's theorem on the
representation of $L$-spaces (561H);  moreover, there is a direct
construction of Haar measures, regarded as linear functionals (561G).

{\it Unless explicitly stated otherwise}, throughout this section (and the
next four) I am working entirely without any form of the axiom of choice.

\leader{561A}{Set theory without choice}\cmmnt{ In \S\S1A1 and 2A1
I tried to lay out some of the basic ideas of set theory without
appealing to the axiom of choice except when this was clearly necessary.}
The most obvious point is that in the absence of choice

\Centerline{\bf the union of a sequence of countable sets need not be
countable\dvro{.}{}}

\noindent\cmmnt{(see the note in 1A1G).
In fact {\smc Feferman \& Levy 63} (see
{\smc Jech 73}, 10.6) described a
model of set theory in which $\Bbb R$ is the union of a sequence of
countable sets.   But not all is lost.   }The elementary arguments of 1A1E
still give

\Centerline{$\Bbb N\simeq\Bbb Z\simeq\Bbb N\times\Bbb N\simeq\Bbb Q$;}

\cmmnt{\noindent there is no difficulty in extending them to show such
things as}

\Centerline{$\Bbb N\simeq[\Bbb N]^{<\omega}
\simeq\bigcup_{n\ge 1}\BbbN^n\simeq\BbbQ^r\times\BbbQ^r$}

\noindent for every integer $r\ge 1$.   The Schr\"oder-Bernstein theorem
survives\cmmnt{ (the method in 344D is easily translated back into its
original form as a proof of the ordinary Schr\"oder-Bernstein theorem)}.
Consequently\cmmnt{ we still have enough bijections to establish}

\Centerline{$\Bbb R\simeq\Cal P\Bbb N\simeq\{0,1\}^{\Bbb N}
\simeq\Cal P(\Bbb N\times\Bbb N)
\simeq(\Cal P\Bbb N)^{\Bbb N}\simeq\BbbR^{\Bbb N}\simeq\NN$.}

\noindent\cmmnt{Cantor's theorem
that }$X\not\simeq\Cal PX$\cmmnt{ is unaffected}, so\cmmnt{ we
still know that} $\Bbb R$ is not countable.

We can still use transfinite recursion\cmmnt{;  see 2A1B}.
We still have a class
$\On$ of von Neumann ordinals such that every well-ordered set is
isomorphic to exactly one
ordinal\cmmnt{ (2A1Dg)} and\cmmnt{ therefore} equipollent with
exactly one initial ordinal\cmmnt{ (2A1Fb)}.
I will say that a set $X$ is {\bf well-orderable} if there is a
well-ordering of $X$.   The standard
arguments for Zermelo's Well-Ordering Theorem\cmmnt{ (2A1K)} now
tell us that for any set $X$ the following are equiveridical:

\inset{(i) $X$ is well-orderable;

(ii) $X$ is equipollent with some ordinal;

(iii) there is an injective function from $X$ into a well-orderable set;

(iv) there is a choice function for $\Cal PX\setminus\{\emptyset\}$}

\noindent (that is, a function $f$ such that
$f(A)\in A$ for every non-empty $A\subseteq X$).
What this means is that if we are given a family $\familyiI{A_i}$
of non-empty sets, and $X=\bigcup_{i\in I}A_i$ is
well-orderable\cmmnt{ (e.g., because it is countable)}, then
$\prod_{i\in I}A_i$ is not empty\cmmnt{
(it contains $\familyiI{f(A_i)}$ where $f$ is a function as in (iv)
above)}.

Note\cmmnt{ also} that while we still have a first uncountable ordinal
$\omega_1$\cmmnt{ (the set of
countable ordinals)}, it can have countable cofinality\cmmnt{ (561Ya)}.
The union of a sequence of finite sets need not be
countable\cmmnt{ ({\smc Jech 73}, \S5.4)};   but the union
of a sequence of finite subsets of a given totally ordered set {\it is}
countable\cmmnt{, because we can use the total ordering to simultaneously
enumerate each of the finite sets in ascending order}.   Consequently, if
$\gamma:\omega_1\to\Bbb R$ is a monotonic function there is a
$\xi<\omega_1$ such that $\gamma(\xi+1)=\gamma(\xi)$.
\prooflet{\Prf\ It is enough to consider the case in which $\gamma$ is
non-decreasing.   Set

\Centerline{$A_n=\{\xi:\gamma(\xi)+2^{-n}\le\gamma(\xi+1)\le n\}$.}

\noindent Then $A_n$ has at most $2^n\max(0,n-\gamma(0))$ members, so is
finite;  consequently $\bigcup_{n\in\Bbb N}A_n$ is countable, and there is
a $\xi\in\omega_1\setminus\bigcup_{n\in\Bbb N}A_n$.   Of course we now find
that $\gamma(\xi+1)=\gamma(\xi)$.\ \Qed}

\cmmnt{
\leader{561B}{Real analysis without choice}
In fact all the standard theorems of elementary
real and complex analysis are
essentially unchanged.   The kind of tightening required in some proofs, to
avoid explicit dependence on the existence of sequences, is similar to the
adaptations needed when we move to general topological spaces.   For
instance, we must define `compactness' in terms of open covers;
compactness and sequential compactness, even for subsets of $\Bbb R$, may
no longer coincide (561Xc).
But we do still have the Heine-Borel theorem in the form `a subset of
$\BbbR^r$ is compact iff it is closed and bounded' (provided, of course,
that we understand that `closed' is not the same thing as `sequentially
closed');  see the proof in 2A2F.
}%end of comment

\leader{561C}{}\cmmnt{ Some new difficulties arise when we move away
from `concrete' questions like the Prime Number Theorem and start looking
at general metric spaces, or even general subsets of $\Bbb R$.   For
instance, a subset of $\Bbb R$, regarded as a topological space, must be
second-countable but need not be separable.   However we can go a long way
if we take care.   The following is an elementary example which will be
useful below.

\medskip

\noindent}{\bf Lemma}\dvArevised{2013}
Let $\Cal E$ be the set of non-empty closed subsets
of $\NN$.   Then there is a family $\family{F}{\Cal E}{f_F}$ such that,
for each $F\in\Cal E$, $f_F$ is a continuous function from $\NN$ to $F$
and $f_F(\alpha)=\alpha$ for every $\alpha\in F$.

\proof{ For $F\in\Cal E$,
set $T_F=\{\alpha\restr n:\alpha\in F$, $n\in\Bbb N\}$.
If $\alpha\in\NN\setminus F$ then, because $F$ is closed,
there is some $n\in\Bbb N$ such that $\beta\restr n\ne\alpha\restr n$ for
any $\beta\in F$, that is, $\alpha\restr n\notin T_F$.   For
$\sigma\in T_F$ define $\beta_{F\sigma}\in\NN$ inductively by saying that

$$\eqalign{\beta_{F\sigma}(n)&=\sigma(n)\text{ if }n<\#(\sigma),\cr
&=\inf\{i:\text{ there is some }\alpha\text{ such that }
\beta_{F\sigma}\restr n\subseteq\alpha\in F\text{ and }\alpha(n)=i\}
\text{ otherwise},\cr}$$

\noindent counting $\inf\emptyset$ as $0$ if necessary.
We see that in fact $\beta_{F\sigma}\restr n\in T_F$ for
every $n\in\Bbb N$, so that $\beta_{F\sigma}\in F$.

We can therefore define $f_F:\NN\to\NN$ by setting

$$\eqalign{f_F(\alpha)&=\alpha\text{ if }\alpha\in F,\cr
&=\beta_{F,\alpha\restr n}\text{ for the largest }n
\text{ such that }\alpha\restr n\in T_F\text{ otherwise}.\cr}$$

\noindent (Because $F$ is not empty, the empty sequence $\alpha\restr 0$
belongs to $T_F$ for every $\alpha\in\NN$.)   We see that
$f_F(\alpha)\in F$ for all $F$ and $\alpha$, and $f_F(\alpha)=\alpha$ if
$\alpha\in F$.   To see that $f_F$ is always 
continuous, note that in fact
if $\alpha\in\NN\setminus F$, and $n$ is the largest integer such that
$\alpha\restr n$ belongs to $T$, then $f_F(\beta)=f_F(\alpha)$ whenever
$\beta\restr n+1=\alpha\restr n+1$, so $f_F$ is continuous at $\alpha$.
While if $\alpha\in F$, $n\in\Bbb N$ and $\beta\restr n=\alpha\restr n$,
then either $\beta\in F$ so $f_F(\beta)\restr n=f_F(\alpha)\restr n$, or
$f_F(\beta)=\beta_{F\sigma}$ where
$\alpha\restr n\subseteq\sigma\subseteq\beta_{F\sigma}$, and again
$f_F(\beta)\restr n=\beta_{\sigma}\restr n=\alpha\restr n$.   So we have a
suitable family of functions.
}%end of proof of 561C

\leader{561D}{Tychonoff's theorem}\cmmnt{ It is a classic
result ({\smc Kelley 50}) that Tychonoff's theorem, in a general form, is
actually equivalent to the axiom of choice.   But nevertheless we have
useful partial results which do not depend on the axiom of choice.   The
following will help in the proofs of 561F and 563I.

\medskip

\noindent{\bf Theorem}} Let $\familyiI{X_i}$ be a family of compact
topological spaces such that $I$ is well-orderable.   For each $i\in I$ let
$\Cal E_i$ be the family of non-empty closed subsets of $X_i$, and suppose
that there is a choice function for $\bigcup_{i\in I}\Cal E_i$.   Then
$X=\prod_{i\in I}X_i$ is compact.

\proof{ Since $I$ is well-orderable, we may suppose that $I=\kappa$ for
some initial ordinal $\kappa$.   Fix a choice function $\psi$ for
$\bigcup_{\xi<\kappa}\Cal E_i$.
For $\xi<\kappa$ write $\pi_{\xi}:X\to X_{\xi}$ for the coordinate map.
If $X$ is empty
the result is trivial.   Otherwise, let $\Cal F$ be any family of closed
subsets of $X$ with the finite intersection property.   I
seek to define a non-decreasing family
$\langle\Cal F_{\xi}\rangle_{\xi\le\kappa}$ of filters on $X$ such that
the image filter $\pi_{\xi}[[\Cal F_{\xi+1}]]$ (2A1Ib) is convergent for
each $\xi<\kappa$.   Start with $\Cal F_0$ the filter generated by
$\Cal F$.   Given $\Cal F_{\xi}$,
let $F_{\xi}$ be the set of cluster points of $\pi_{\xi}[[\Cal F_{\xi}]]$;
because $X_{\xi}$ is compact, this is a non-empty closed subset of
$X_{\xi}$, and $x_{\xi}=\psi(F_{\xi})$ is defined.   Let
$\Cal F_{\xi+1}$ be the filter on $X$ generated by

\Centerline{$\Cal F_{\xi}
\cup\{\pi_{\xi}^{-1}[U]:U$ is a neighbourhood of $x_{\xi}$ in $X_{\xi}\}$.}

\noindent For limit ordinals $\xi\le\kappa$, let $\Cal F_{\xi}$ be the
filter on $X$ generated by $\bigcup_{\eta<\xi}\Cal F_{\eta}$.

Now $\Cal F_{\kappa}$ is a filter including $\Cal F$ converging to
$x=\ofamily{\xi}{\kappa}{x_{\xi}}$, and $x$ must belong to $\bigcap\Cal F$.
As $\Cal F$ is arbitrary, $X$ is compact.
}%end of proof of 561D

\cmmnt{\medskip

\noindent{\bf Remark} The point of the condition
`there is a choice function for $\bigcup_{i\in I}\Cal E_i$'
is that it is
satisfied if every $X_i$ is the unit interval $[0,1]$, for instance;
we could take
$\psi(E)=\min E$ for non-empty closed sets $E\subseteq[0,1]$.
You will have no difficulty in devising other examples, using the
technique of the proof above, or otherwise.   Note that 561C shows that
there is a choice function for the family $\Cal E$
of non-empty closed subsets of
$\NN$, since we can use the function $F\mapsto f_F(\tbf{0})$ where
$\family{F}{\Cal E}{f_F}$ is the family of functions defined there.
}

\leader{561E}{Baire's theorem} (a) Let $(X,\rho)$ be a complete metric
space with a well-orderable dense subset.   Then $X$ is a Baire space.

(b) Let $X$ be a compact Hausdorff space
with a well-orderable $\pi$-base.   Then $X$ is a Baire space.

\proof{{\bf (a)} Let $D$ be a dense subset of $X$ with a well-ordering
$\preccurlyeq$.   If $\sequencen{G_n}$ is a sequence of dense
open subsets of $X$, and $G$ is a non-empty open set, define
$\sequencen{H_n}$, $\sequencen{x_n}$ and $\sequencen{\epsilon_n}$
inductively, as follows.   $H_0=G$.   Given $H_n$,
$x_n$ is to be the $\preccurlyeq$-first point of $H_n$.
Given $x_n$ and $H_n$, $\epsilon_n$ is to be the first rational number
in $\ocint{0,2^{-n}}$ such that $B(x_n,\epsilon_n)\subseteq H_n$.
(I leave it to you to decide which rational numbers come first.)
Now set $H_{n+1}=\{y:y\in G_n$, $\rho(y,x_n)<\epsilon_n\}$;  continue.

At the end of the induction, $\sequencen{x_n}$ is a Cauchy sequence so has
a limit $x$ in $X$.   Since $x_n\in H_n\subseteq H_m$ whenever $m\le n$,
$x\in\overline{H}_{n+2}\subseteq H_{n+1}\subseteq G_n$ for every $n$, and
$x$ witnesses that $G\cap\bigcap_{n\in\Bbb N}G_n$ is non-empty.
As $G$ and $\sequencen{G_n}$ are arbitrary, $X$ is a Baire space.

\medskip

{\bf (b)} Let $\Cal U$ be a $\pi$-base for the topology of $X$, not
containing $\emptyset$, with a
well-ordering $\preccurlyeq$.   If $\sequencen{G_n}$ is a sequence of dense
open subsets of $X$, and $G$ is a non-empty open set, define
$\sequencen{U_n}$ in $\Cal U$ inductively by saying that

\inset{$U_0$ is the
$\preccurlyeq$-first member of $\Cal U$ included in $G$,

$U_{n+1}$ is the $\preccurlyeq$-first member of $\Cal U$ such that
$\overline{U}_{n+1}\subseteq U_n\cap G_n$}

\noindent for each $n$.   Then $\bigcap_{n\in\Bbb N}\overline{U}_n$ is
non-empty and included in $G\cap\bigcap_{n\in\Bbb N}G_n$.
}%end of proof of 561E

\leader{561F}{Stone's Theorem} Let $\frak A$ be a well-orderable Boolean
algebra.   Then there is a compact Hausdorff
Baire space $Z$ such that $\frak A$
is isomorphic to the algebra of open-and-closed subsets of $Z$.

\proof{ As in 311E, let $Z$ be the set of ring homomorphisms from $\frak A$
onto $\Bbb Z_2$.   Writing $\Bbb B$ for the set of finite subalgebras of
$\frak A$, $Z=\bigcap_{\frak B\in\Bbb B}Z_{\frak B}$ where

\Centerline{$Z_{\frak B}
=\{z:z\in\Bbb Z_2^{\frak A}$, $z\restrp\frak B$ is a Boolean
homomorphism$\}$.}

\noindent So $Z$ is a closed subset of the compact Hausdorff
space $\{0,1\}^{\frak A}$, and is compact.   Setting
$\widehat{a}=\{z:z\in Z$, $z(a)=1\}$, the map $a\mapsto\widehat{a}$ is a
Boolean homomorphism from $\frak A$ to the algebra $\frak E$ of
open-and-closed subsets of $Z$.
If $a\in\frak A\setminus\{0\}$ and $\frak B$ is a finite
subalgebra of $\frak A$, then the subalgebra $\frak C$ generated by
$\{a\}\cup\frak B$ is still finite, and there is a Boolean homomorphism
$w:\frak C\to\Bbb Z_2$ such that $w(a)=1$;  extending $w$ arbitrarily to a
member of $\{0,1\}^{\frak A}$, we obtain a $z\in Z_{\frak B}$ such that
$z(a)=1$;  as $\frak B$ is arbitrary, there is a $z\in Z$ such that
$z(a)=1$.   So the map $a\mapsto\widehat{a}$ is injective.   If
$G\subseteq Z$ is open and $z\in G$, there must be a finite set
$A\subseteq\frak A$ such that $G$ includes
$\{z':z'\in Z$, $z'\restr A=z\restr A\}$;  in this case, setting
$c=\inf\{a:a\in A$, $z(a)=1\}\Bsetminus\sup\{a:a\in A$, $z(a)=0\}$,
$z\in\widehat{c}\subseteq G$.   It follows that any member of $\frak E$
is of the form $\widehat{a}$ for some $a\in\frak A$, so that
$a\mapsto\widehat{a}$ is an isomorphism between $\frak A$ and $\frak E$.

Because $\frak A$ is well-orderable, $\Bbb Z_2^{\frak A}$ and $Z$ have
well-orderable bases, and $Z$ is a Baire space, by 561E.
}%end of proof of 561F

\leader{561G}{Haar\dvrocolon{ measure}}\cmmnt{ Now I come to something
which demands a rather less sketchy treatment.

\medskip

\noindent}{\bf Theorem} Let $X$ be a completely regular
locally compact Hausdorff topological group.

(i) There is a non-zero
left-translation-invariant positive linear functional on $C_k(X)$.

(ii) If $\phi$, $\phi'$ are non-zero
left-translation-invariant positive linear functionals on $C_k(X)$ then
each is a scalar multiple of the other.
%maybe drop "Hausdorff"

\proof{ %{\smc Naimark 70}, VI.27.5 p 361
{\bf (a)} Write $\Phi$ for $\{g:g\in C_k(X)^+$, $g(e)=\|g\|_{\infty}=1\}$
where $e$ is the identity of $X$.  For $f\in C_k(X)^+$ and $g\in\Phi$, set

$$\eqalign{\high{f:g}
&=\inf\{\sum_{i=0}^n\alpha_i:\alpha_0,\ldots,\alpha_n\ge 0\cr
&\mskip100mu \text{ and there are }a_0,\ldots,a_n\in X\text{ such that }
f\le\sum_{i=0}^n\alpha_ia_i\action_l g\},\cr}$$

\noindent writing $(a\action_lg)(x)=g(a^{-1}x)$ as in
4A5Cc\dvAformerly{4{}41Ac}.
We have to confirm that this infimum is always defined in
$\coint{0,\infty}$.   \Prf\ Set $K=\overline{\{x:f(x)>0\}}$ and
$U=\{x:g(x)>\bover12\}$, so that $K$ is compact,
$U$ is open and $U\ne\emptyset$.
Then $K\subseteq\bigcup_{a\in X}aU$, so there are $a_0,\ldots,a_n\in X$
such that $K\subseteq\bigcup_{i\le n}a_iU$.   In this case

\Centerline{$f
\le\sum_{i=0}^n2\|f\|_{\infty}a_i\action_l g$}

\noindent and $\high{f:g}\le 2(n+1)\|f\|_{\infty}$.\ \QeD\

It is now easy to check that

\Centerline{$\high{a\action_l f:g}=\high{f:g}$,
\quad$\high{f_1+f_2:g}\le\high{f_1:g}+\high{f_2:g}$,}

\Centerline{$\high{\alpha f:g}=\alpha\high{f:g}$,
\quad$\|f\|_{\infty}\le\high{f:g}$,
\quad$\high{f:h}\le\high{f:g}\high{g:h}$}

\noindent whenever $f$, $f_1$, $f_2\in C_k(X)^+$,
$g$, $h\in\Phi$, $a\in X$
and $\alpha\in\coint{0,\infty}$.  (Compare part (c) of the proof of 441C.)

\medskip

{\bf (b)} Fix $g_0\in\Phi$, and for $g\in\Phi$ set

$$\psi_g(f)=\bover{\high{f:g}}{\high{g_0:g}}$$

\noindent for $f\in C_k(X)$.   Then

\Centerline{$\psi_g(a\action_l f)=\psi_g(f)$,
\quad$\psi_g(f_1+f_2)\le\psi_g(f_1)+\psi_g(f_2)$,}

\Centerline{$\psi_g(\alpha f)=\alpha\psi_g(f),
\quad\psi_g(f)\le\high{f:g_0}$,}

\Centerline{$\psi_g(f)\le\psi_g(h)\high{f:h}$,
\quad$1\le\psi_g(h)\high{g_0:h}$}

\noindent whenever $f$, $f_1$, $f_2\in C_k(X)^+$, $h\in\Phi$,
$a\in X$ and $\alpha\ge 0$.   For a neighbourhood $U$ of the identity $e$
of $X$, write $\Phi_U$ for the set of those $g\in\Phi$ such that
$g(x)=0$ for every $x\in X\setminus U$;  because $X$ is locally compact and
completely regular, $\Phi_U\ne\emptyset$.

\medskip

{\bf (c)(i)} If $f_0,\ldots,f_m\in C_k(X)^+$ and $\epsilon>0$, there
is a neighbourhood $U$ of $e$ such that

\Centerline{$\sum_{j=0}^m\psi_g(f_j)
\le\psi_g(\sum_{j=0}^mf_j)+\epsilon$}

\noindent whenever $g\in\Phi_U$.
\Prf\ Set $f=\sum_{j=0}^mf_j$.   Let $K$ be the compact set
$\overline{\{x:f(x)\ne 0\}}$, and let $\hat f\in C_k(X)$ be
such that $\chi K\le\hat f$.   Let $\eta>0$ be such that

\Centerline{$(1+(m+1)\eta)(\psi_g(f)+\eta\high{\hat f:g_0})
\le\psi_g(f)+\epsilon$,}

\noindent and set $f^*=f+\eta\hat f$.
Then we can express each $f_j$ as
$f^*\times h_j$ where $h_j\in C_k(X)^+$ and $\sum_{j=0}^mh_j\le\chi X$.
Let $U$ be a neighbourhood of $e$ such that $|h_j(x)-h_j(y)|\le\eta$
whenever $x^{-1}y\in U$ and $j\le m$ (compare 4A5Pa).

Take $g\in\Phi_U$.   Let $\alpha_0,\ldots,\alpha_n\ge 0$ and
$a_0,\ldots,a_n\in X$ be such that $f^*\le\sum_{i=0}^n\alpha_ia_i\action_l g$
and $\sum_{i=0}^n\alpha_i\le\high{f^*:g}+\eta$.
Then, for any $x\in X$ and $j\le m$,

\Centerline{$f_j(x)=f^*(x)h_j(x)
\le\sum_{i=0}^n\alpha_ig(a_i^{-1}x)h_j(x)
\le\sum_{i=0}^n\alpha_ig(a_i^{-1}x)(h_j(a_i)+\eta)$}

\noindent because if $i$ is such that $g(a_i^{-1}x)\ne 0$ then
$a_i^{-1}x\in U$ and $h_j(x)\le h_j(a_i)+\eta$.   So
$\high{f_j:g}\le\sum_{i=0}^n\alpha_i(h_j(a_i)+\eta)$.   Summing over $j$,

\Centerline{$\sum_{j=0}^m\high{f_j:g}
\le\sum_{i=0}^n\alpha_i(1+(m+1)\eta)$}

\noindent because $\sum_{j=0}^mh_j(a_i)\le 1$ for every $i$.
As $\alpha_0,\ldots,\alpha_n$ and $a_0,\ldots,a_n$ are arbitrary,

\Centerline{$\sum_{j=0}^m\high{f_j:g}
\le(1+(m+1)\eta)\high{f^*:g}
\le(1+(m+1)\eta)(\high{f:g}+\eta\high{\hat f:g})$,}

\noindent and

$$\eqalign{\sum_{j=0}^m\psi_g(f_i)
&\le(1+(m+1)\eta)(\psi_g(f)+\eta\psi_g(\hat f))\cr
&\le(1+(m+1)\eta)(\psi_g(f)+\eta\high{\hat f:g_0})
\le\psi_g(f)+\epsilon\cr}$$

\noindent as required.\ \Qed

\medskip

\quad{\bf (ii)} If $f_0,\ldots,f_m\in C_k(X)^+$, $M\ge 0$ and $\epsilon>0$,
there is a neighbourhood $U$ of $e$ such that

\Centerline{$\sum_{j=0}^m\psi_g(\gamma_jf_j)
\le\psi_g(\sum_{j=0}^m\gamma_jf_j)+\epsilon$}

\noindent whenever $g\in\Phi_U$ and $\gamma_0,\ldots,\gamma_m\in[0,M]$.
\Prf\ Let $\eta>0$ be such that
$\eta(1+\sum_{j=0}^m\high{f_j:\nobreak g_0})\le\epsilon$.
By (i), applied finitely often,
there is a neighbourhood $U$ of $e$ such that

\Centerline{$\sum_{j=0}^m\psi_g(\gamma_jf_j)
\le\psi_g(\sum_{j=0}^m\gamma_jf_j)+\eta$}

\noindent whenever $g\in\Phi_U$ and $\gamma_0,\ldots,\gamma_m\in[0,M]$ are
multiples of $\eta$.   Now, given arbitrary
$\gamma_0,\ldots,\gamma_m\in[0,M]$ and $g\in\Phi_U$,
let $\gamma_0',\ldots,\gamma_m'$ be
multiples of $\eta$ such that $\gamma_j'\le\gamma_j<\gamma_j'+\eta$ for
each $j$.   Then

$$\eqalign{\sum_{j=0}^m\psi_g(\gamma_jf_j)
&\le\sum_{j=0}^m\psi_g(\gamma'_jf_j)+\eta\psi_g(f_j)\cr
&\le\psi_g(\sum_{j=0}^m\gamma'_jf_j)+\eta+\eta\sum_{j=0}^m\psi_g(f_j)
\le\psi_g(\sum_{j=0}^m\gamma_jf_j)+\epsilon\cr}$$

\noindent as required.\ \Qed

\medskip

{\bf (d)} (We are coming to the magic bit.)
Suppose that $f\in C_k(X)^+$, $\epsilon>0$ and that $U$ is a
neighbourhood of $e$ such that $|f(x)-f(y)|\le\epsilon$ whenever
$x^{-1}y\in U$.   Then if $g\in\Phi_U$ and $\gamma>\epsilon$ there are
$\alpha_0,\ldots,\alpha_n\ge 0$ and $a_0,\ldots,a_n\in X$ such that
$\|f-\sum_{i=0}^n\alpha_ia_i\action_l g\|_{\infty}\le\gamma$.
\Prf\ For all $x$, $y\in X$ we have

\Centerline{$(f(x)-\epsilon)g(x^{-1}y)
\le f(y)g(x^{-1}y)\le(f(x)+\epsilon)g(x^{-1}y)$.}

\noindent Let $\eta>0$ be such that
$\eta(1+\high{f:\Reverse{g}})\le\gamma-\epsilon$,
where $\Reverse{g}(x)=g(x^{-1})$ for $x\in X$.
Let $V$ be an open neighbourhood of $e$ such that $|g(x)-g(y)|\le\eta$
whenever $xy^{-1}\in V$.   Then we have $a_0,\ldots,a_n$ such that
$\bigcup_{i\le n}a_iV\supseteq\overline{\{x:f(x)\ne 0\}}$, and
$h_0,\ldots,h_n\in C_k(X)^+$ such that $\sum_{i=0}^nh_i(x)=1$ whenever
$f(x)>0$, while $h_i(x)=0$ if $i\le n$ and $x\notin a_iV$.
By (c-ii), there is an $h\in\Phi$ such that

\Centerline{$\sum_{i=0}^n\psi_h(\gamma_if\times h_i)
\le\psi_h(\sum_{i=0}^n\gamma_if\times h_i)+\eta$}

\noindent whenever $0\le\gamma_i\le\high{g_0:\Reverse{g}}$ for each $i$.

Now, for $i\le n$ and $x$, $y\in X$,

\Centerline{$h_i(y)f(y)(g(a_i^{-1}x)-\eta)
\le h_i(y)f(y)g(y^{-1}x)
\le h_i(y)f(y)(g(a_i^{-1}x)+\eta)$.}

\noindent Accordingly

$$\eqalign{(f(x)-\epsilon)(x\action_l\Reverse{g})(y)
&=(f(x)-\epsilon)g(y^{-1}x)
\le f(y)g(y^{-1}x)
=\sum_{i=0}^nh_i(y)f(y)g(y^{-1}x)\cr
&\le\sum_{i=0}^nh_i(y)f(y)(g(a_i^{-1}x)+\eta)
=\eta f(y)+\sum_{i=0}^nh_i(y)f(y)g(a_i^{-1}x);\cr}$$

\noindent similarly,

\Centerline{$(f(x)+\epsilon)(x\action_l\Reverse{g})(y)
\ge\sum_{i=0}^nh_i(y)f(y)g(a_i^{-1}x)-\eta f(y)$.}

Fixing $x$ for the moment, and applying the functional $\psi_h$ to the
expressions here (regarded as functions of $y$), we get

\Centerline{$(f(x)-\epsilon)\psi_h(\Reverse{g})
\le\eta\psi_h(f)+\psi_h(\sum_{i=0}^ng(a_i^{-1}x)f\times h_i)$}

\noindent so

$$\eqalign{f(x)-\gamma
&\le f(x)-\epsilon-\eta\high{f:\Reverse{g}}
\le f(x)-\epsilon-\eta\Bover{\psi_h(f)}{\psi_h(\Reverse{g})}\cr
&\le\psi_h(\sum_{i=0}^n
  \Bover{g(a_i^{-1}x)}{\psi_h(\Reverse{g})}f\times h_i)
\le\sum_{i=0}^n\Bover{g(a_i^{-1}x)}{\psi_h(\Reverse{g})}\psi_h(f\times h_i)
=\sum_{i=0}^n\alpha_ig(a_i^{-1}x)\cr}$$

\noindent where $\alpha_i=\Bover{\psi_h(f\times h_i)}{\psi_h(\Reverse{g})}$.
On the other side,

\Centerline{$(f(x)+\epsilon)\psi_h(\Reverse{g})+\eta\psi_h(f)
\ge\psi_h(\sum_{i=0}^ng(a_i^{-1}x)f\times h_i)$,}

\noindent so

$$\eqalignno{f(x)+\gamma
&\ge f(x)+\epsilon+\eta\Bover{\psi_h(f)}{\psi_h(\Reverse{g})}+\eta\cr
&\ge\psi_h(\sum_{i=0}^n\Bover{g(a_i^{-1}x)}{\psi_h(\Reverse{g})}f\times h_i)
  +\eta
\ge\sum_{i=0}^n\Bover{g(a_i^{-1}x)}{\psi_h(\Reverse{g})}\psi_h(f\times h_i)\cr
\displaycause{because
$\Bover{g(a_i^{-1}x)}{\psi_h(\Reverse{g})}\le\high{g_0:\Reverse{g}}$ for every $i$}
&=\sum_{i=0}^n\alpha_ig(a_i^{-1}x).}$$

\noindent All this is valid for every $x\in X$;  so

\Centerline{$\|f-\sum_{i=0}^n\alpha_ia_i\action_l g\|_{\infty}\le\gamma$.
\Qed}

\medskip

{\bf (e)} For any $f\in C_k(X)^+$ and $\epsilon>0$ there are a
$\gamma\ge 0$ and a neighbourhood $U$ of $e$ such that
$|\psi_h(f)-\gamma|\le\epsilon$ for every $h\in\Phi_U$.   \Prf\ Let $V$ be
a compact neighbourhood of $0$ and $K=\overline{\{x:f(x)+g_0(x)>0\}}$;
let $f^*\in C_k(X)$ be such that $\chi(KV^{-1}V)\le f^*\le\chi X$.
Let $\delta$, $\eta>0$ be such that

\Centerline{$\delta(1+2(\delta+\high{f:g_0}))\le\epsilon$,
\quad$\delta\le\bover12$,
\quad$\eta(1+\high{f^*:g_0})\le\delta$.}

\noindent By (d), there are $g\in\Phi_V$,
$\alpha_0,\ldots,\alpha_n,\beta_0,\ldots,\beta_m\ge 0$ and
$a_0,\ldots,a_n,b_0,\ldots,b_n\in X$ such that

\Centerline{$\|f-\sum_{i=0}^n\alpha_ia_i\action_l g\|_{\infty}\le\eta$,
\quad$\|g_0-\sum_{j=0}^m\beta_jb_j\action_l g\|_{\infty}\le\eta$.}

\noindent
We can suppose that all the $a_i$, $b_j$ belong to $KV^{-1}$, since
$g(a^{-1}x)=0$ if $x\in K$ and $a\notin KV^{-1}$;  consequently

\Centerline{$|f-\sum_{i=0}^n\alpha_ia_i\action_l g|\le\eta f^*$,
\quad$|g_0-\sum_{j=0}^m\beta_jb_j\action_l g|\le\eta f^*$.}

\noindent Set $\alpha=\sum_{i=0}^n\alpha_i$, $\beta=\sum_{j=0}^m\beta_j$
and $\gamma=\Bover{\alpha}{\beta}$.   ($\beta$ is non-zero because
$\|g_0\|_{\infty}=1$ and $\eta\|f^*\|_{\infty}\le\bover12$.)

Let $U\subseteq V$ be a neighbourhood of $e$ such that

\Centerline{$\sum_{i=0}^n\alpha_i\psi_h(a_i\action_l g)
\le\psi_h(\sum_{i=0}^n\alpha_ia_i\action_l g)+\eta$,}

\Centerline{$\sum_{j=0}^m\beta_j\psi_h(b_j\action_l g)
\le\psi_h(\sum_{j=0}^m\beta_jb_j\action_l g)+\eta$}

\noindent for every $h\in\Phi_U$  ((c) above).   Take any $h\in\Phi_U$.
Then

$$\eqalign{|\psi_h(f)-\alpha\psi_h(g)|
&=|\psi_h(f)-\sum_{i=0}^n\alpha_i\psi_h(a_i\action_l g)|
\le|\psi_h(f)-\psi_h(\sum_{i=0}^n\alpha_ia_i\action_l g)|+\eta\cr
&\le\eta\psi_h(f^*)+\eta
\le\eta\high{f^*:g_0}+\eta;\cr}$$

\noindent similarly,

\Centerline{$|1-\beta\psi_h(g)|
=|\psi_h(g_0)-\beta\psi_h(g)|\le\eta(\high{f^*:g_0}+1)$.}

\noindent But this means that

$$\eqalign{|\psi_h(f)-\gamma|
&\le\eta(1+\high{f^*:g_0})+|\alpha\psi_h(g)-\gamma|\cr
&\le\delta+\gamma|\beta\psi_h(g)-1|
\le\delta(1+\gamma).\cr}$$

\noindent Consequently

\Centerline{$\gamma\le\Bover{\psi_h(f)+\delta}{1-\delta}
\le 2(\delta+\high{f:g_0})$,}

\Centerline{$|\psi_h(f)-\gamma|\le\delta(1+2(\delta+\high{f:g_0}))
\le\epsilon$,}

\noindent as required.\ \Qed

\medskip

{\bf (f)} We are nearly home.   Let $\Cal F$ be the filter on $\Phi$
generated by $\{\Phi_U:U$ is a neighbourhood of $e\}$.   By (e),
$\phi(f)=\lim_{h\to\Cal F}\psi_h(f)$ is defined for every $f\in C_k(X)^+$.
From the formulae in (b) we have

\Centerline{$\phi(a\action_l f)=\phi(f)$,
\quad$\phi(f_1+f_2)\le\phi(f_1)+\phi(f_2)$,
\quad$\phi(\alpha f)=\alpha\phi(f)$}

\noindent whenever $f$,
$f_1$, $f_2\in C_k(X)^+$, $a\in X$ and $\alpha\ge 0$.
By (c-i), we have $\phi(f_1)+\phi(f_2)\le\phi(f_1+f_2)$ for all $f_1$,
$f_2\in C_k(X)^+$.   So $\phi$ is additive and extends to an invariant
positive linear functional on $C_k(X)$ which is non-zero because
$\phi(g_0)=1$.

\medskip

{\bf (g)} As for uniqueness, we can repeat the arguments in (e).
Suppose that $\phi'$ is another
left-translation-invariant positive linear functional on $C_k(X)$ such that
$\phi'(g_0)=1$, and $f\in C_k(X)^+$.   Let
$K$ be the closure of ${\{x:f(x)+g_0(x)>0\}}$ and $V$ a compact
neighbourhood
of $e$;  let $f^*\in C_k(X^*)$ be such that $\chi(KV^{-1}V)\le f^*$.
Take $\epsilon>0$.   Let $\delta$, $\eta>0$ be such that

\Centerline{$\delta\le\Bover12$,
\quad$\delta(1+2(\phi(f)+1))\le\epsilon$,
\quad$\delta(1+2(\phi'(f)+1))\le\epsilon$,}

\Centerline{$\eta\phi(f^*)\le\delta$,
\quad$\eta\phi'(f^*)\le\delta$.}

\noindent Then there
is a neighbourhood $U$ of $e$, included in $V$, such that
$|f(x)-f(y)|\le\eta$
and $|g_0(x)-g_0(y)|\le\eta$ whenever $x^{-1}y\in U$.   By (d),
there are $g\in\Phi_V$,
$\alpha_0,\ldots,\alpha_n,\beta_0,\ldots,\beta_m\ge 0$ and
$a_0,\ldots,a_n,b_0,\ldots,b_m\in X$ such that

\Centerline{$|f(x)-\sum_{i=0}^n\alpha_i(a_i\action_l g)(x)|\le\eta$,
\quad$|g_0(x)-\sum_{j=0}^m\beta_j(b_j\action_l g)(x)|\le\eta$}

\noindent for every $x\in X$;  as in (e), we may suppose that every $a_i$,
$b_j$ belongs to $KV^{-1}$ so that

\Centerline{$|f-\sum_{i=0}^n\alpha_ia_i\action_l g|\le\eta f^*$,
\quad$|g_0-\sum_{j=0}^m\beta_jb_j\action_l g|\le\eta f^*$.}

\noindent Consequently, setting $\alpha=\sum_{i=0}^n\alpha_i$,
$\beta=\sum_{i=0}^n\beta_i$ and $\gamma=\alpha/\beta$,

\Centerline{$|\phi(f)-\alpha\phi(g)|
=|\phi(f)-\sum_{i=0}^n\alpha_i\phi(a_i\action_l g)|
\le\eta\phi(f^*)\le\delta$,}

\Centerline{$|1-\beta\phi(g)|
=|\phi(g_0)-\sum_{j=0}^m\beta_j\phi(b_j\action_l g)|
\le\eta\phi(f^*)\le\delta$.}

\noindent So

\Centerline{$|\phi(f)-\gamma|
\le\eta\phi(f^*)+\gamma|\beta\phi(g)-1|
\le\eta\phi(f^*)(1+\gamma)
\le\delta(1+\gamma)$}

\noindent and

\Centerline{$\gamma\le\Bover{\phi(f)+\delta}{1-\delta}\le 2(\phi(f)+1)$,}

\Centerline{$|\phi(f)-\gamma|\le\delta(1+2(\phi(f)+\delta))
\le\epsilon$.}

\noindent Similarly, $|\phi'(f)-\gamma|\le\epsilon$ and
$|\phi(f)-\phi'(f)|\le 2\epsilon$.   As $\epsilon$ and $f$ are arbitrary,
$\phi=\phi'$.
}%end of proof of 561G

\leader{561H}{Kakutani's theorem} (a) Let $U$ be an Archimedean Riesz space
with a weak order unit.
Then there are a Dedekind complete Boolean algebra
$\frak A$ and an order-dense Riesz subspace of $L^0(\frak A)$, containing
$\chi 1$, which is isomorphic to $U$.

(b) Let $U$ be an $L$-space with a weak order unit $e$.   Then there is a
totally finite measure algebra $(\frak A,\bar\mu)$ such that $U$ is
isomorphic, as normed Riesz space, to $L^1(\frak A,\bar\mu)$, and we can
choose the isomorphism to match $e$ with $\chi 1$.

\proof{ All the required ideas are in Volume 3;  but we have quite a lot of
checking to do.

\medskip

{\bf (a)(i)} The first step is to observe that, for any Dedekind
$\sigma$-complete Boolean algebra $\frak A$, the definition of
$L^0=L^0(\frak A)$ in 364A gives no difficulties, and that the
formulae of 364D
can be used to define a Riesz space structure on $L^0$.   \Prf\
I recall the formulae in question:

\Centerline{$\Bvalue{u>\alpha}=\sup_{\beta>\alpha}\Bvalue{u>\beta}$
for every $\alpha\in\Bbb R$,}

\Centerline{$\inf_{\alpha\in\Bbb R}\Bvalue{u>\alpha}=0$,
\quad $\sup_{\alpha\in\Bbb R}\Bvalue{u>\alpha}=1$,}

\Centerline{$\Bvalue{u+v>\alpha}
=\sup_{q\in\Bbb Q}\Bvalue{u>q}\Bcap\Bvalue{v>\alpha-q}$,}

\noindent whenever $u$, $v\in L^0$ and $\alpha\in\Bbb R$,

\Centerline{$\Bvalue{\gamma u>\alpha}=\Bvalue{u>\bover{\alpha}{\gamma}}$}

\noindent whenever $u\in L^0$, $\gamma\in\ooint{0,\infty}$ and
$\alpha\in\Bbb R$.
The distributive laws in 313A-313B are enough to ensure that $u+v$ and
$\gamma u$, so defined, belong to $L^0$, and also that
$u+v=v+u$, $u+(v+w)=(u+v)+w$, $\gamma(u+v)=\gamma u+\gamma v$ for $u$, $v$,
$w\in L^0$ and $\gamma>0$.   Defining $\tbf{0}\in L^0$ by saying that

\Centerline{$\Bvalue{\tbf{0}>\alpha}=1$ if $\alpha<0$, $0$ if
$\alpha\ge 0$,}

\noindent we can check that $u+0=u$ for every $u$.   Defining $-u\in L^0$
by saying that

\Centerline{$\Bvalue{-u>\alpha}
=\sup_{q\in\Bbb Q,q>\alpha}1\Bsetminus\Bvalue{u>-q}$}

\noindent for $u\in L^0$ and $\alpha\in\Bbb R$, we find (again using the
distributive laws, of course) that $u+(-u)=\tbf{0}$;  we can now define
$\gamma u$, for $\gamma\le 0$, by saying that $0\cdot u=\tbf{0}$ and
$\gamma u=(-\gamma)(-u)$ if $\gamma<0$, and we shall have a linear space.
Turning to the ordering, it is nearly trivial to check that the definition

\Centerline{$u\le v\iff\Bvalue{u>\alpha}\Bsubseteq\Bvalue{v>\alpha}$ for
every $\alpha\in\Bbb R$}

\noindent gives us a partially ordered linear space.   It is a Riesz space
because the formula

\Centerline{$\Bvalue{u\vee v>\alpha}
=\Bvalue{u>\alpha}\Bcup\Bvalue{v>\alpha}$}

\noindent defines a member of $L^0$ which must be $\sup\{u,v\}$ in $L^0$.\
\Qed

We need to know that if $\frak A$ is Dedekind
complete, so is $L^0$;  the argument of 364M still applies.
Note also that $a\mapsto\chi a:\frak A\to L^0$ is
order-continuous, as in 364Jc.

\medskip

\quad{\bf (ii)} Now suppose that $U$ is an Archimedean Riesz space with an
order unit $e$.   Let $\frak A$ be the band algebra of $U$ (353B).
Then we can argue as in 368E, but with the simplification
that the maximal disjoint set $C$ in $U^+\setminus\{0\}$ is just $\{e\}$,
to see that we have an injective Riesz homomorphism $T:U\to L^0(\frak A)$
defined by taking $\Bvalue{Tu>\alpha}$ to be the band generated by
$e\wedge(u-\alpha e)^+$ (or, if you prefer, by $(u-\alpha e)^+$, since it
comes to the same thing).   We shall have $T[U]$ order-dense, as before,
with $Te=\chi 1$.

\medskip

{\bf (b)(i)} Again, the bit we have to concentrate on is the check that,
starting from a totally finite measure algebra $(\frak A,\bar\mu)$, we can
define $L^1(\frak A,\bar\mu)$ as in 365A.
We have to be a bit careful, because already in Proposition 321C there is
an appeal to AC$(\omega)$ (see 561Yi(vi));  
but I think we need to know very little about
measure algebras to get through the arguments here.   Of course another
difficulty arises at once in 365A, because I write

\Centerline{$\|u\|_1
=\int_0^{\infty}\bar\mu\Bvalue{|u|>\alpha}\,d\alpha$,}

\noindent and say that the integration is with respect to `Lebesgue
measure', which won't do, at least until I redefine Lebesgue integration as
in \S565.   But we are integrating a monotonic
function, so the integral can be thought of as an improper Riemann
integral;  if you like,

\Centerline{$\|u\|_1
=\lim_{n\to\infty}2^{-n}\sum_{i=1}^{4^n}\bar\mu\Bvalue{|u|>2^{-n}i}
=\sup_{n\in\Bbb N}2^{-n}\sum_{i=1}^{4^n}\bar\mu\Bvalue{|u|>2^{-n}i}$.}

\noindent Next, we can't use the Loomis-Sikorski theorem to prove 365C, and
have to go back to first principles.   To see that $\|\,\|_1$ is
subadditive, and additive on $(L^0)^+$, look first at `simple' non-negative
$u$, expressed as $u=\sum_{i=0}^n\alpha_i\chi a_i$, and check that
$\int u=\|u\|_1=\sum_{i=0}^n\alpha_i\bar\mu a_i$;  now confirm that
every element of $(L^0)^+$ is expressible as the supremum of a
non-decreasing sequence of such elements, and that $\|\,\|_1$ is
sequentially order-continuous on the left on $(L^0)^+$.
(We need 321Be.)
This is enough to show that $L^1$ is a solid linear subspace of $L^0$ with
a Riesz norm and a sequentially order-continuous integral.
(I do {\it not} claim, yet, that $L^1$ is an $L$-space, because
I do not know, in the absence of countable choice, that every Cauchy
filter on $L^1$ converges.)

\medskip

\quad{\bf (ii)} Now let $U$ be an $L$-space with a weak order unit $e$.
As in (a), let $\frak A$ be the band algebra of $U$ and $T:U\to L^0$ an
injective Riesz homomorphism onto an order-dense Riesz subspace of $L^0$
with $Te=\chi 1$.   Now $U$ is Dedekind complete (354N, 354Ee).
Consequently $T[U]$ must be solid in $L^0$ (353K).

\medskip

\quad{\bf (iii)} For $a\in\frak A$, set $\bar\mu a=\|T^{-1}(\chi a)\|$.
Because the map $a\mapsto T^{-1}\chi a$ is additive and order-continuous
and injective, $(\frak A,\bar\mu)$ is a measure algebra;  indeed,
$\bar\mu$ is actually order-continuous.   So we have a space
$L^1=L^1(\frak A,\bar\mu)$.   Because $\bar\mu$ is order-continuous,
364L(a-ii) tells us that $\|w\|_1=\sup_{v\in B}\|v\|_1$ whenever
$B\subseteq L^0$ is a non-empty upwards-directed set in $L^0$
with supremum $w$ in $L^0$.

Writing $S\subseteq L^0$ for
the linear span of $\{\chi a:a\in\frak A\}$, we see that
$\|w\|_1=\|T^{-1}w\|$ for every $w\in S$.   Since $S$ is order-dense in
$L^0$ it is order-dense in $L^1$, and $T^{-1}[S]$ is order-dense in $U$,
therefore norm-dense (354Ef).

\medskip

\quad{\bf (iv)}
$Tu\in L^1$ for every $u\in U^+$.   \Prf\ For $n\in\Bbb N$ set
$a_n=\Bvalue{Tu>2^n}\Bsetminus\Bvalue{Tu>2^{n+1}}$, $u_n=T^{-1}(\chi a_n)$.
Set $w_n=\sum_{i=0}^n2^i\chi a_i$ for $n\in\Bbb N$.   Then $w_n\le Tu$
and $\|w_n\|_1=\|T^{-1}w_n\|\le\|u\|$ for every $n$.   By 364L(a-i),
$w=\sup_{n\in\Bbb N}w_n$ is defined in $L^0$, and
$\|w\|_1=\sup_{n\in\Bbb N}\|w_n\|_1$ is finite.   But $Tu\le 2w+\chi 1$, so
$Tu\in L^1$.\ \QeD\

\medskip

\quad{\bf (v)} If $w\in(L^1)^+$ there is a $v\in U^+$ such that $w=Tv$ and
$\|v\|=\|w\|_1$.   \Prf\ Consider
$A=\{u:u\in T^{-1}[S]$, $Tu\le w\}$.   This is upwards-directed and
norm-bounded, so has a supremum $v$ in $U$ (354N again), and
$Tv\ge w'$ whenever $w'\in S$ and $w'\le w$.   But $S$ is order-dense in
$L^0$ so $Tv\ge w$.   Because $T$ is order-continuous, (iii) tells us that

\Centerline{$\|Tv\|_1=\sup_{u\in A}\|Tu\|_1=\sup_{u\in A}\|u\|=\|v\|$,}

\noindent while surely $\|w\|_1\ge\sup_{u\in A}\|Tu\|_1$.   So $Tv=w$.\
\Qed

\medskip

\quad{\bf (vi)} Putting (iv) and (v) together, we see that $T[U]=L^1$ and
that $T$ is a normed Riesz space isomorphism, as required.
}%end of proof of 561H

\vleader{72pt}{561I}{Hilbert spaces:  Proposition}
Let $U$ be a Hilbert space.

(a) If $C\subseteq U$ is a non-empty closed convex set then for
any $u\in U$ there is a unique $v\in C$ such that
$\|u-v\|=\min\{\|u-w\|:w\in C\}$.

(b) Every closed linear subspace of $U$ is the image of an orthogonal
projection, that is, has an orthogonal complement.

(c) Every member of $U^*$ is of the form
$u\mapsto\innerprod{u}{v}$ for some $v\in U$.

(d) $U$ is reflexive.

(e) If $C\subseteq U$ is a norm-closed convex set then it is
weakly closed.

\proof{{\bf (a)} Set $\gamma=\inf\{\|u-w\|:w\in C\}$ and let $\Cal F$ be
the filter on $U$ generated by sets of the form
$F_{\epsilon}=\{w:w\in C$, $\|u-w\|\le\gamma+\epsilon\}$ for $\epsilon>0$.
Then $\Cal F$ is Cauchy.   \Prf\ Suppose that $\epsilon>0$ and $w_1$,
$w_2\in F_{\epsilon}$.   Then

$$\eqalignno{\|w_1-w_2\|^2
&=2\|u-w_1\|^2+2\|u-w_2\|^2-\|2u-w_1-w_2\|^2
\le 4(\gamma+\epsilon)^2-4\gamma^2\cr
\displaycause{because $\Bover12(w_1+w_2)\in C$}
&=8\gamma\epsilon+4\epsilon^2.\cr}$$

\noindent So

\Centerline{$\inf_{F\in\Cal F}\diam F=\inf_{\epsilon>0}\diam F_{\epsilon}
=0$.   \Qed}

We therefore have a limit $v$ of $\Cal F$, which is in $C$ because $C$ is
closed, and $\|u-v\|=\lim_{w\to\Cal F}\|u-w\|=\gamma$.
If now $w$ is any other member of $C$, $\|u-\Bover12(v+w)\|\ge\gamma$ so
$\|u-w\|>\gamma$.

\medskip

{\bf (b)} Let $V$ be a closed linear subspace of $U$.   By (a), we have a
function $P:U\to V$ such that $Pu$ is the unique closest element of $V$ to
$u$, that is, $\|u-Pu\|\le\|u-Pu+\alpha v\|$ for every $v\in V$ and
$\alpha\in\Bbb R$.   It follows that $\innerprod{u-Pu}{v}=0$ for every
$v\in V$, that is, that $u-Pu\in V^{\perp}$.   As $u$ is arbitrary,
$U=V+V^{\perp}$;  as $V\cap V^{\perp}=\{0\}$, $P$ must be the projection
onto $V$ with kernel $V^{\perp}$, and is an orthogonal projection.

\medskip

{\bf (c)} Take $f\in U^*$.   If $f=0$ then $f(u)=\innerprod{u}{0}$ for
every $u$.   Otherwise, set $C=\{w:f(w)=0\}$.   Then $C$ is a proper closed
linear subspace of $U$.   Take any $u_0\in U\setminus C$.
Let $v_0$ be the point of $C$ nearest to $u_0$, and consider $u_1=u_0-v_0$.
Then $0$ is the point of $C$ nearest to $u_1$, so that
$\innerprod{u}{u_1}=0$ for every $u\in C$.   Set
$v=\Bover{f(u_1)}{\|u_1\|^2}u_1$;  then $\innerprod{u}{v}=0$ for every
$u\in C$, while $f(v)=\innerprod{v}{v}$.   So $f(u)=\innerprod{u}{v}$ for
every $u\in U$.

\medskip

{\bf (d)} From (c) it follows that we can identify $U^*$ with $U$ and
therefore $U^{**}$ also becomes identified with $U$.

\medskip

{\bf (e)} If $C$ is empty this is trivial.   Otherwise, take any
$u\in U\setminus C$.   Let $v$ be the point of $C$ nearest to $U$, and set
$f(w)=\innerprod{w}{u-v}$ for $w\in U$.   Then $f(w)\le f(v)<f(u)$ for
every $w\in C$.   So $u$ does not belong to the weak closure of $C$;  as
$u$ is arbitrary, $C$ is weakly closed.
}%end of proof of 561I

\exercises{\leader{561X}{Basic exercises (a)}
%\spheader 561Xa
Let $X$ be any set.   (i) Show
that $\ell^p(X)$, for $1\le p\le\infty$, is a Banach space.
(ii) Show that $\ell^1(X)^*$ can be identified with $\ell^{\infty}(X)$.
(iii) Show that if $1<p<\infty$ and $\bover1p+\bover1q=1$ then
$\ell^p(X)^*$ can be identified with $\ell^q(X)$.
%561B

\spheader 561Xb Let $X$ be any topological space.   Show that $C_b(X)$,
with $\|\,\|_{\infty}$, is a Banach space.
%561B

\spheader 561Xc Suppose that there is
an infinite subset $X$ of $\Bbb R$ with no infinite countable subset
({\smc Jech 73}, \S10.1).   Show that $X$ is sequentially closed
but not closed, second-countable but not separable,
sequentially compact but not compact, sequentially complete
(that is, every Cauchy sequence converges)
but not complete.   Show that the topology of $\Bbb R$ is not countably
tight.
%561B

\sqheader 561Xd (i) Let $C$ be the set of those
$R\subseteq\Bbb N\times\Bbb N$ which are total orderings of subsets of
$\Bbb N$.   Show that $C$ is a closed subset of
$\Cal P(\Bbb N\times\Bbb N)$ with its usual topology.
(ii) For $\xi<\omega_1$, let $C_{\xi}$ be the set of those
$R\in C$ which are well-orderings of order type $\xi$ of subsets of
$\Bbb N$.   Show that $C_{\xi}$ is a Borel subset of
$\Cal P(\Bbb N\times\Bbb N)$.   \Hint{induce on $\xi$.}
(iii) Show that there is an injective function from $\omega_1$ to the
Borel $\sigma$-algebra $\Cal B(\Bbb R)$ of $\Bbb R$.
%561B

\spheader 561Xe(i) Show that every non-empty closed subset of $\NN$ has a
lexicographically-first member.   (ii) Show that if a T$_1$
topological space $X$ is a continuous image
of $\NN$, then there is an injection from $X$ to $\Cal P\Bbb N$.
%561C

\spheader 561Xf Let $X$ be a topological space.
(i) Show that if $X$ is separable, then
$X^{\Bbb R}$ is separable.  (ii) Show that if $X$ has a countable network,
then $X^{\Bbb R}$ has a countable network.
%561C

\spheader 561Xg(i) Show that a locally compact Hausdorff space is regular.
(ii) Show that a compact regular space is normal.
%561D

\spheader 561Xh Let $U$ be a normed space with a well-orderable subset $D$
such that the linear span of $D$ is dense in $U$.
(i) Show that if $V$ is a
linear subspace of $U$ and $f\in V^*$, there is a $g\in U^*$, extending
$f$, with the same norm as $f$.   (ii) Show that the unit ball
$B$ of $U^*$ is
weak*-compact and has a well-orderable base for its topology.   (iii) Show
that if $K\subseteq B$ is weak*-closed then $K$ has an extreme point.
%561D

\spheader 561Xi Let $(X,\rho)$ be a separable
compact metric space, and $G$ the
isometry group of $X$ with its topology of pointwise convergence (441G).
Show that $G$ is compact.   \Hint{$X^{\Bbb N}$ is compact.}
%561D

\spheader 561Xj Let $X$ be a regular topological space and $A$ a subset of
$X$.   Show that the following are equiveridical:  (i) $A$ is relatively
compact in $X$;  (ii) $\overline{A}$ is compact;  (iii) every filter on $X$
containing $A$ has a cluster point.
%561D

\sqheader 561Xk Let $(X,\frak T)$ be a regular second-countable
topological space, and write $\frak S$ for the usual topology on
$\BbbR^{\Bbb N}$.   (i) Show that there are a continuous function
$f:X\to\BbbR^{\Bbb N}$ and a function $\phi:\frak T\to\frak S$ such that
$G=f^{-1}[\phi(G)]$ for every $G\in\frak T$.   (ii) Show that if $X$ is
Hausdorff it is metrizable.
%561E 561Xl

\sqheader 561Xl Let $X$ be a regular second-countable topological space,
$\Cal C$ the family of closed subsets of $X$, and $\Cal D$ the set of
disjoint pairs
$(F_0,F_1)\in\Cal C\times\Cal C$.   (i) Show that $X$ is
normal, and that there
is a function $\psi:\Cal D\to\Cal C$ such that
$F_0\subseteq\interior\psi(F_0,F_1)$ and $F_1\cap\psi(F_0,F_1)=\emptyset$
whenever $(F_0,F_1)\in\Cal D$.
(ii) Show that there is a function $\phi:\Cal D\to C(X)$ such
$\phi(F_0,F_1)(x)=0$, $\phi(F_0,F_1)(y)=1$ whenever $(F_0,F_1)\in\Cal D$,
$x\in F_0$ and $y\in F_1$.
%561E

\spheader 561Xm Let $X$ be a well-orderable discrete abelian group.   Show
that its dual group, as defined in 445A, is a completely regular compact
Hausdorff group.
%561G

\spheader 561Xn Let $U$ be a Riesz space with a Riesz norm.   Let
$\Delta:U^+\to\coint{0,\infty}$ be such that
($\alpha$) $\Delta$ is non-decreasing,
($\beta$) $\Delta(\alpha u)=\alpha\Delta(u)$ whenever
$u\in U^+$ and $\alpha\ge 0$,
($\gamma$) $\Delta(u+v)=\Delta(u)+\Delta(v)$ whenever $u\wedge v=0$
($\delta$) $|\Delta(u)-\Delta(v)|\le\|u-v\|$ for all $u$, $v\in U^+$.
Show that $\Delta$ has an  to a member of $U^*$.
%561H

\spheader 561Xo Let $U$ be an $L$-space.
Show that $\|u\|=\sup\{f(u):f\in U^*$, $\|f\|\le 1\}$ for every $u\in U$.
%561H

\spheader 561Xp Let $(\frak A,\bar\mu)$ be a measure algebra.   Show that
$L^1(\frak A,\bar\mu)$ is a Dedekind $\sigma$-complete Riesz space and a
sequentially complete normed space.
%561H

\spheader 561Xq Let $\frak A$ be a Boolean algebra.   Show that there are a
set $X$, an algebra $\Cal E$ of subsets of $X$ and a surjective Boolean
homomorphism from $\Cal E$ onto $\frak A$.   \Hint{566L.}
%561H

\spheader 561Xr Let $U$ be a Hilbert space.
(i) Show that a bounded sequence
$\sequencen{u_n}$ in $U$ is weakly convergent in $U$ iff
$\lim_{n\to\infty}\innerprod{u_n}{u_m}$ is defined for every
$m\in\Bbb N$.
(ii) Show that the unit ball
of $U$ is sequentially compact for the weak topology.   (iii)
Show that if
$T:U\to U$ is a self-adjoint compact linear operator, then $T[U]$ is
included in the closed linear span of $\{Tv:v$ is an eigenvector of $T\}$.
\Hint{reduce to the case in which $U$ is separable, and show that there is
then a sequence $\sequencen{u_n}$ in the unit ball $B$ of $U$ such that
$\lim_{n\to\infty}\innerprod{Tu_n}{u_n}=\sup_{u\in B}\innerprod{Tu}{u}$.}
%561I

\spheader 561Xs\dvAnew{2013} 
In 561C, show that $(F,\alpha)\mapsto f_F(\alpha):\Cal E\times\NN\to\NN$ is
continuous if $\Cal E$ is given its Vietoris topology (4A2T) and
$\NN$ its usual topology.
%561C out of order query

\leader{561Y}{Further exercises (a)}
%\spheader 561Ya
Suppose that there is a sequence $\sequencen{A_n}$ of
countable sets such that $\bigcup_{n\in\Bbb N}A_n=\Bbb R$.   Show that
$\cf\omega_1=\omega$.
%561A

\spheader 561Yb(i) Show that there is a bijection between $\omega_1$ and
$\Bbb N\times\omega_1$.   (ii) Show that $\omega_2$ is not expressible as the
union of a sequence of
countable sets.   (iii) Show that $\Cal P\omega_1$ is not expressible as the
union of a sequence of
countable sets.   (iv) Show that $\Cal P(\Cal P\Bbb N)$ is not
expressible as the union of a sequence of countable sets.
%561Ya 561A

\spheader 561Yc Suppose there is a countable family of doubleton
sets with no choice function ({\smc Jech 73}, \S5.5).   Show that
(i) there is
a set $I$ such that $\{0,1\}^I$ has an open-and-closed set which is not
determined by coordinates in any countable subset of $I$ (ii) there is a
compact metrizable space which is not ccc, therefore not second-countable
(iii) there is a complete totally bounded metric space which is neither ccc
nor compact (iv) there is a probability algebra which is not ccc.
%561D mt56bits

\spheader 561Yd Let $X$ be a metrizable space.   Show that it
is second-countable iff it has a countable $\pi$-base
iff it has a countable network.
%561E

\spheader 561Ye(i) Let $(X,\rho)$ be a complete metric space.
Show that $X$ has a well-orderable dense subset iff it has a well-orderable
base iff it has a well-orderable $\pi$-base iff it has a well-orderable
network iff there is a choice function
for the family of its non-empty closed subsets.
(ii) Let $X$ be a locally compact Hausdorff space.
($\alpha$) Show that if it
has a well-orderable $\pi$-base then it has a well-orderable dense subset.
($\beta$) Show that if it has a well-orderable base then
it is completely regular and there is a choice
function for the family of its non-empty closed subsets.
%561E 561D

\spheader 561Yf Let $X$ be a metrizable space.
Show that every continuous real-valued
function defined on a closed subset of $X$ has an extension to
a continuous real-valued function on $X$.
%561E mt56bits

\spheader 561Yg(i) Show that if $\frak A$ is a Boolean algebra,
there is an essentially unique Dedekind complete Boolean algebra
$\widehat{\frak A}$ in which $\frak A$ can
be embedded as an order-dense subalgebra.   (ii) Show that if $\frak A$ and
$\frak B$ are two Boolean algebras and $\pi:\frak A\to\frak B$ is an
order-continuous Boolean homomorphism, $\pi$ has a unique extension to an
order-continuous Boolean homomorphism from $\widehat{\frak A}$ to
$\widehat{\frak B}$.   \Hint{take $\widehat{\frak A}$ to be the set of
pairs $(A,A')$ of subsets of $\frak A$ such that $A$ is the set of lower
bounds of $A'$ and $A'$ is the set of upper bounds of $A$.}
%3{}14T 561H

\spheader 561Yh Let $\familyiI{\frak A_i}$ be a family of Boolean algebras.
(i) Show that there is an essentially unique structure
$(\frak A,\familyiI{\varepsilon_i})$ such that ($\alpha$) $\frak A$ is a
Boolean algebra ($\beta$) $\varepsilon_i:\frak A_i\to\frak A$ is a Boolean
homomorphism for every $i$ ($\gamma$) whenever $\frak B$ is a Boolean
algebra
and $\phi_i:\frak A_i\to\frak B$ is a Boolean homomorphism for every $i$,
there is a unique Boolean homomorphism $\pi:\frak A\to\frak B$ such that
$\pi\varepsilon_i=\phi_i$ for every $i$.   (ii) Show that if
$\nu_i:\frak A_i\to\Bbb R$ is additive, with $\nu_i1=1$,
for every $i\in I$, there is a
unique additive $\nu:\frak A\to\Bbb R$ such that
$\nu(\inf_{i\in J}\varepsilon_i(a_i))=\prod_{i\in J}\nu_ia_i$ whenever
$J\subseteq I$ is finite and $a_i\in\frak A_i$ for $i\in J$.
%3{}15 561H

\spheader 561Yi Suppose that there is an infinite totally ordered set $I$
with no countably infinite subset ({\smc Jech 73}, \S10.1).   Let $\Cal E$
be the algebra of subsets of $\{0,1\}^I$ determined by coordinates in
finite sets.   (i) Show that the union of any countable family of finite 
subsets of $I$ is finite.   
(ii) Show that $\Cal E$ has no countably infinite subset, so
that every finitely additive real-valued functional on $\Cal E$ is
countably additive.   (iii) Show that there is no infinite disjoint
family in $\Cal E$.
(iv) Show that $\Cal E$ is Dedekind complete.   (v) Show that there
is a functional $\mu_1$ such that $(\Cal E,\mu_1)$ 
is a probability algebra and
$\mu_1$ is order-continuous.   (vi) Show that there is a functional
$\mu_2$ such that $(\Cal E,\mu_2)$ is a probability algebra and $\mu_2$ is
not order-continuous.
%3{}91  56bits

\spheader 561Yj Let $(X,\rho)$ be a complete metric space with a
well-orderable base.   Show that a subset of $X$ is compact iff it is
sequentially compact iff it is closed and totally bounded.
%561Ye 563Ff
}%end of exercises

\endnotes{
\Notesheader{561}
The arguments of this section will I hope give an idea of
the kind of discipline which will be imposed for the rest of the chapter.
Apart from anything else, we have to fix on the correct definitions.
Typically, when defining something like `compactness' or `completeness',
the definition to use is that which is most useful in the most general
context;  so that even in metrizable spaces we should prefer filters to
sequences (cf.\ 561Xc).

We can distinguish two themes in the methods I have used here.   First, in
the presence of a well-ordering we can hope to adapt the standard attack on
a problem;  see 561D-561F.  %561D, 561E, 561F.
Second, if (in the presence of the
axiom of choice) there is a {\it unique} solution to a problem, then we can
hope that it is still a unique solution without choice.   This is what
happens in 561G and also in 561Ia-561Ic.   In 561I we just go through the
usual arguments with a little more care.   In 561G (taken from
{\smc Naimark 70}) we need new ideas.   But in the key step, part (d) of
the proof, the two variables $x$ and $y$ reflect an adaptation of a
repeated-integration argument as in \S442.   Note that the scope of 561G
may be limited if we have fewer locally compact groups than we expect.

A regular second-countable Hausdorff space is
metrizable (561Yf).   But it may not be separable (561Xc).
We do not have Urysohn's Lemma in its usual form, so cannot be sure that
a locally compact Hausdorff space is completely regular;  a topological
group has left, right and bilateral uniformities, but a uniformity need not
be defined by pseudometrics and a uniform space need not be completely
regular.   So in such results as 561G we may need an extra `completely
regular' in the hypotheses.

I give a version of Kakutani's theorem (561H) to show that some of the
familiar patterns are distorted in possibly unexpected ways, and that
occasionally it is the more abstract parts of the theory which survive
best.
I suppose I ought to remark explicitly that I define `measure algebra'
exactly as in 321A:  a Dedekind $\sigma$-complete Boolean algebra with a
strictly positive countably additive $[0,\infty]$-valued functional.
I do not claim that every $\sigma$-finite measure algebra is
either localizable or ccc (561Yc),
nor that every measure algebra can be represented in terms of a measure
space.   I set up a construction of a normed Riesz space
$L^1(\frak A,\bar\mu)$, but do not claim that this is an $L$-space.
However, if we start from an $L$-space $U$
with a weak order unit, we can build
a measure on its band algebra and proceed to an $L^1(\frak A,\bar\mu)$
which is isomorphic to $U$ (and is therefore an $L$-space).
}%end of notes


\discrpage

