\frfilename{mt539.tex}
\versiondate{24.5.14}
\copyrightdate{2007}

\def\chaptername{Topologies and measures III}
\def\sectionname{Maharam submeasures}

\newsection{539}

Continuing the work of \S\S392-394 %392 393 394
and 496, I return to Maharam
submeasures and the forms taken by the ideas of the present volume in this
context.   At least for countably generated algebras, and in some cases
more generally, many of the methods of Chapter 52 can be applied
(539B-539K). %539B 539C 539D 539G 539H 539J 539K
In 539L-539N  %539L 539M 539N
I give the main result of {\smc Balcar Jech \& Pazak 05} and
{\smc Veli\v{c}kovi\'c 05}:  it is consistent
to suppose that every Dedekind complete ccc \wsid\ Boolean algebra is a
Maharam algebra.   In 539R-539U %539R 539S 539T 539U
I introduce the idea of `exhaustivity rank' of an exhaustive submeasure.

\cmmnt{\leader{539A}{The story so far} As submeasures have hardly
appeared before in this volume, I begin by repeating some of
the essential ideas.

\spheader 539Aa If $\frak A$ is a Boolean algebra, a
{\bf submeasure} on $\frak A$ is a functional $\nu:\frak A\to[0,\infty]$
such that $\nu 0=0$, $\nu a\le\nu b$ whenever $a\Bsubseteq b$, and
$\nu(a\Bcup b)\le\nu a+\nu b$ for all $a$, $b\in\frak B$
(392A);  it is {\bf totally finite} if $\nu 1<\infty$.
If $\nu$ is a submeasure defined on an algebra of subsets of a set $X$, I
say that the {\bf null ideal} of $\nu$ is the ideal
$\Cal N(\nu)$ of subsets of $X$
generated by $\{E:\nu E=0\}$ (496Bc).
A submeasure $\nu$ on a Boolean algebra $\frak A$
is {\bf exhaustive} if $\lim_{n\to\infty}\nu a_n=0$ for every
disjoint sequence $\sequencen{a_n}$ in $\frak A$;
it is {\bf uniformly exhaustive} if for every $\epsilon>0$ there
is an $n\in\Bbb N$ such that there is no disjoint family
$a_0,\ldots,a_n$ with $\nu a_i\ge\epsilon$ for every $i\le n$ (392Bc).
A {\bf Maharam submeasure} is a totally finite
sequentially order-continuous submeasure (393A);
a Maharam submeasure on a Dedekind
$\sigma$-complete Boolean algebra is exhaustive (393Bc).

\spheader 539Ab
A {\bf Maharam algebra} is a Dedekind $\sigma$-complete Boolean algebra
with a
strictly positive Maharam submeasure.   Any Maharam algebra is ccc and
\wsid\ (393Eb).   A Maharam algebra is measurable
iff it carries a strictly positive uniformly exhaustive submeasure (393D).
If $\nu$ is any Maharam submeasure on a Dedekind $\sigma$-complete Boolean
algebra $\frak A$, its Maharam algebra is the quotient
$\frak A/\{a:\nu a=0\}$ (496Ba).

\spheader 539Ac If $\nu$ is any
strictly positive totally finite submeasure on a Boolean algebra
$\frak A$, there is an associated metric $(a,b)\mapsto\nu(a\Bsymmdiff b)$
on $\frak A$;
the completion $\widehat{\frak A}$
of $\frak A$ under this metric is a Boolean algebra (392Hc).
If $\nu$ is exhaustive, then
$\widehat{\frak A}$ is a Maharam algebra (393H).   If $\nu$ and $\nuprime$ are
both strictly positive Maharam submeasures on the same Maharam algebra
$\frak A$, $\nu$ is absolutely continuous with respect to $\nuprime$ (393F).
Consequently the associated metrics are uniformly equivalent, and $\frak A$
has a canonical topology and uniformity, its {\bf Maharam-algebra topology}
and {\bf Maharam-algebra uniformity} (393G).

\spheader 539Ad Let $\frak A$ be a Boolean algebra.

\medskip

\quad{\bf (i)} A sequence $\sequencen{a_n}$ in $\frak A$
order*-converges to $a\in\frak A$ (definition:  367A)
iff there is a partition $B$ of unity in
$\frak A$ such that $\{n:b\Bcap(a_n\Bsymmdiff a)\ne 0\}$ is finite for
every $b\in B$ (393Ma).

\medskip

\quad{\bf (ii)} The {\bf order-sequential topology} on $\frak A$
is the topology for which the closed sets are just the sets closed under
order*-convergence (393L).

\medskip

\quad{\bf (iii)}
If $\frak A$ is ccc and Dedekind $\sigma$-complete, a subalgebra of
$\frak A$
is order-closed iff it is closed for the order-sequential topology (393O).

\medskip

\quad{\bf (iv)} If $\frak A$ is ccc and \wsid, then the closure of a
set $A\subseteq\frak A$ for the order-sequential topology is the set of
order*-limits of sequences in $A$ (393Pb).

\medskip

\quad{\bf (v)}
If $\frak A$ is a Maharam algebra, then its Maharam-algebra topology
is its order-sequential topology (393N).

\medskip

\quad{\bf (vi)}
If $\frak A$ is a Dedekind $\sigma$-complete ccc \wsid\ Boolean algebra,
and $\{0\}$
is a G$_{\delta}$ set for the order-sequential topology, then $\frak A$ is
a Maharam algebra (393Q).

\spheader 539Ae
It was a long-outstanding problem (the `Control Measure Problem') whether
every Maharam algebra is in fact a measurable algebra;  this was solved by
a counterexample in {\smc Talagrand 08}, described in \S394.

\spheader 539Af
If $X$ is a Hausdorff space, a {\bf totally finite Radon submeasure} on
$X$ is a
totally finite submeasure $\nu$ defined on a $\sigma$-algebra $\Sigma$ of
subsets of $X$ such that (i) if $E\subseteq F\in\Sigma$ and $\nu F=0$ then
$E\in\Sigma$ (ii) every open set belongs to $\Sigma$ (iii) if $E\in\Sigma$
and $\epsilon>0$ there is a compact set $K\subseteq E$ such that
$\nu(E\setminus K)\le\epsilon$ (496C).
Every totally finite Radon submeasure is a Maharam submeasure (496Da).
If $X$ is a Hausdorff space and $\nu$
is a totally finite Radon submeasure on $X$,
a set $E\in\dom\nu$ is {\bf self-supporting} if $\nu(E\cap G)>0$ whenever
$G\subseteq X$ is an open set meeting $E$.   If
$E\in\dom\nu$ and $\epsilon>0$, there is a compact self-supporting
$K\subseteq E$ such that $\nu(E\setminus K)\le\epsilon$ (496Dd).

Let $\nu$ be a strictly positive Maharam submeasure on a
Dedekind $\sigma$-complete Boolean algebra $\frak A$.
Let $Z$ be the Stone space
of $\frak A$, and write $\widehat{a}$ for the open-and-closed subset of $Z$
corresponding to each $a\in\frak A$.
Then there is a unique totally finite Radon
submeasure $\nuprime$ on $Z$ such that
$\nuprime\widehat{a}=\nu a$ for every $a\in\frak A$; the null ideal of $\nuprime$
is the nowhere dense ideal of $Z$ (496G).

\spheader 539Ag For a cardinal $\kappa$, I write $\Cal N_{\kappa}$ for the
null ideal of the usual measure on $\{0,1\}^{\kappa}$;
$\Cal N\cong\Cal N_{\omega}$ will be the null ideal of Lebesgue measure on
$\Bbb R$, and $\Cal M$ the meager ideal of $\Bbb R$.
}%end of comment

\leader{539B}{Proposition} Let $\frak A$ be a Maharam algebra,
$\tau(\frak A)$ its Maharam type and $d_{\frak T}(\frak A)$ its
topological density for its Maharam-algebra topology.
Then $\tau(\frak A)\le d_{\frak T}(\frak A)\le\max(\omega,\tau(\frak A))$.

\proof{ Recall that the Maharam-algebra topology is the order-sequential
topology (539A(d-v)).   $\frak A$ is ccc and \wsid\ (539Ab), so if
$D\subseteq\frak A$ is topologically dense, then
every element of $\frak A$ is expressible as the order*-limit
$\inf_{n\in\Bbb N}\sup_{m\ge n}a_m$ of some sequence $\sequencen{a_n}$ in
$D$ (539A(d-iv)).   In this case
$D\,\,\tau$-generates $\frak A$ and $\tau(\frak A)\le\#(D)$;
accordingly $\tau(\frak A)\le d_{\frak T}(\frak A)$.   If
$D\subseteq\frak A\,\,\tau$-generates $\frak A$, let $\frak B$ be the
subalgebra of $\frak A$ generated by $D$ and $\overline{\frak B}$ its
topological closure.   Then $\overline{\frak B}$ is order-closed (because
$\frak A$ is ccc), so is the whole of $\frak A$, and
$d_{\frak T}(\frak A)\le\#(\frak B)\le\max(\omega,\#(D))$;  accordingly
$d_{\frak T}(\frak A)\le\max(\omega,\tau(\frak A))$.
}%end of proof of 539B

\leader{539C}{Theorem} Let $\frak A$ be a Maharam algebra.

(a)

\Centerline{$(\frak A^+,\Bsupseteqshort^{\strprime},
[\frak A^+]^{\le\max(\omega,\tau(\frak A))})
\prGT(\Pou(\frak A),\sqsubseteq^*,\Pou(\frak A))$\dvro{.}{,}}

\cmmnt{\noindent where $\frak A^+=\frak A\setminus\{0\}$,
$(\frak A^+,\Bsupseteqshort^{\strprime},[\frak A^+]^{\le\kappa})$ is
defined as in 512F, and $(\Pou(\frak A),\sqsubseteq^*)$
as in 512Ee.}

(b) $\Pou(\frak A)\prT\Cal N_{\tau(\frak A)}$.

\proof{ If $\frak A=\{0\}$ these are both trivial;  suppose otherwise.
Fix a strictly positive Maharam submeasure $\nu$ on $\frak A$ such that
$\nu 1=1$.   Let $\frak B$ be a subalgebra of $\frak A$ which is
dense in $\frak A$ for the metric $(a,b)\mapsto\nu(a\Bsymmdiff b)$ and has
cardinal at most $\kappa=\max(\omega,\tau(\frak A))$ (539B).

\medskip

{\bf (a)(i)}
For $a\in\frak A^+$ choose $\phi(a)\in\Pou(\frak A)$ as follows.   Start by
taking $d_n\in\frak B$, for $n\in\Bbb N$, such that
$\nu(d_n\Bsymmdiff(1\Bsetminus a))\le 2^{-n-2}\nu a$ for each $n$;  set
$b_n=d_n\Bsetminus\sup_{i<n}b_i$ for $n\in\Bbb N$,
$a'=1\Bsetminus\sup_{n\in\Bbb N}b_n=1\Bsetminus\sup_{n\in\Bbb N}d_n$;
then every $b_n$ belongs to $\frak B$,

\Centerline{$\nu(a'\Bsetminus a)
\le\inf_{n\in\Bbb N}\nu((1\Bsetminus d_n)\setminus a)
\le\inf_{n\in\Bbb N}\nu(d_n\Bsymmdiff(1\Bsetminus a))
=0$,}

\Centerline{$\nu(a\Bsetminus a')\le\sum_{n=0}^{\infty}\nu(a\Bcap d_n)
<\nu a$,}

\noindent so $0\ne a'\Bsubseteq a$.   Now set
$\phi(a)=\{a'\}\Bcup\{b_n:n\in\Bbb N\}$.

\medskip

\quad{\bf (ii)} For $C\in\Pou(\frak A)$, set

\Centerline{$\psi(C)=\{c\Bcap b:c\in C$, $b\in\frak B\}\setminus\{0\}
\in[\frak A^+]^{\le\kappa}$.}

\medskip

\quad{\bf (iii)} Suppose that $a\in\frak A^+$, $C\in\Pou(\frak A)$ and
$\phi(a)\sqsubseteq^*C$.   Then there is a $b\in\psi(C)$ such that
$b\Bsubseteq a$.   \Prf\ Let $c\in C$ be such that $c\Bcap a'\ne 0$, where
$a'$ is defined as in (i) above.   Then
$B=\{b:b\in\phi(a)\setminus\{a'\}$, $c\Bcap b\ne 0\}$ is a finite subset of
$\frak B$, so $\sup B\in\frak B$ and $c\Bsetminus\sup B\in\psi(C)$.
But $c\Bsetminus\sup B=c\Bcap a'\Bsubseteq a$.\ \QeD\  Thus
$a\mskip5mu\Bsupseteqshort^{\strprime}\mskip5mu\psi(C)$.

As $a$ is arbitrary, $(\phi,\psi)$ is a Galois-Tukey connection from
$(\frak A^+,\Bsupseteqshort^{\strprime},
[\frak A^+]^{\le\kappa})$ to
$(\Pou(\frak A),\penalty-100\sqsubseteq^*\penalty+100,\Pou(\frak A)$, and
$(\frak A^+,\Bsupseteqshort^{\strprime},
\penalty-100[\frak A^+]^{\le\kappa})
\prGT(\Pou(\frak A),\sqsubseteq^*,\Pou(\frak A)$.

\medskip

{\bf (b)(i)} If $\tau(\frak A)$ is finite,
then $\frak A$ is purely atomic and
$\Pou(\frak A)$ has an upper bound in itself,
as does $\Cal N_{\kappa}$;  so the
result is trivial.   Accordingly we may suppose henceforth that
$\tau(\frak A)=\kappa$ is infinite.

\medskip

\quad{\bf (ii)} If $C\in\Pou(\frak A)$,
there is a sequence $\sequencen{b_n}$ in $\frak B$ such
that $\nu b_n\le 4^{-n}$ for every $n\in\Bbb N$ and
$\{c:c\in C$, $c\notBsubseteq\sup_{i\ge n}b_i\}$ is finite for every
$n\in\Bbb N$.   \Prf\ If $C$ is finite this is trivial.   Otherwise,
set $\epsilon_n=4^{-n}/(n+2)$ for each $n\in\Bbb N$, and
enumerate $C$ as $\sequencen{c_n}$.   Let $\sequencen{k(n)}$ be a strictly
increasing sequence such that $\nu c'_n\le\epsilon_n$ for every $n$, where
$c'_n=\sup_{i\ge k(n)}c_i$;  choose $\sequencen{b_n}$ in $\frak B$
inductively so that

\Centerline{$\nu(b_n\symmdiff\sup_{j\le n}
   (c'_j\Bsetminus\sup_{j\le i<n}b_i))
\le\epsilon_{n+1}$}

\noindent for each $n\in\Bbb N$.   Then we see by induction on $n$ that

\Centerline{$\nu(c'_j\setminus\sup_{j\le i<n}b_i)\le\epsilon_n$}

\noindent whenever $j\le n$ in $\Bbb N$, and therefore that

\Centerline{$\nu b_n\le\epsilon_{n+1}+(n+1)\epsilon_n\le 4^{-n}$}

\noindent for each $n$;  while $c'_j\Bsubseteq\sup_{i\ge j}b_i$ for every
$j$, so

\Centerline{$1\Bsetminus\sup_{i\ge n}b_i\Bsubseteq 1\Bsetminus c'_n
=\sup_{i<k(n)}c_i$}

\noindent meets only finitely many members of $C$, for every $n$.\ \Qed

\medskip

\quad{\bf (iii)} Now fix on an enumeration $\ofamily{\xi}{\kappa}{b_{\xi}}$
of $\frak B$.   Consider the $\kappa$-localization relation
$(\kappa^{\Bbb N},\subseteq^*,\Cal S_{\kappa})$ (522K).
We see from (ii) that we can find a function
$\phi:\Pou(\frak A)\to\kappa^{\Bbb N}$ such that

\Centerline{$\nu b_{\phi(C)(n)}\le 4^{-n}$ for every $n\in\Bbb N$,}

\Centerline{$1\Bsetminus\sup_{i\ge n}b_{\phi(C)(i)}$ meets only finitely
many members of $C$, for every $n\in\Bbb N$.}

\noindent Next, define $\psi:\Cal S_{\kappa}\to\Pou(\frak A)$ as follows.
Given $S\in\Cal S_{\kappa}$, set $a_0(S)=1$,

\Centerline{$a_{n+1}(S)
=\sup_{m\ge n}\sup\{b_{\xi}:(m,\xi)\in S$, $\nu b_{\xi}\le 4^{-m}\}$}

\noindent for each $n$;  then
$\nu a_{n+1}(S)\le\sum_{m=n}^{\infty}2^{-m}=2^{-n+1}$ for every $n$, so
$\psi(S)=\{a_n(S)\Bsetminus a_{n+1}(S):n\in\Bbb N\}$
is a partition of unity in $\frak A$.

\medskip

\quad{\bf (iv)} Suppose that $C\in\Pou(\frak A)$ and $S\in\Cal S_{\kappa}$ are such
that $\phi(C)\subseteq^*S$.   In this case there is an $m\in\Bbb N$ such
that $(n,\phi(C)(n))\in S$ for every $n\ge m$.   Since
$\nu b_{\phi(C)(n)}\le 4^{-n}$ for every $n$,
$\sup_{i\ge n}b_{\phi(C)(i)}\Bsubseteq a_{n+1}(S)$
and $1\Bsetminus a_{n+1}(S)$ meets only finitely many members of
$C$, for every $n\ge m$.   Thus every member of $\psi(S)$ meets only
finitely many members of $C$, and $C\sqsubseteq^*\psi(S)$.

This shows that $(\phi,\psi)$ is a Galois-Tukey connection from
$(\Pou(\frak A),\sqsubseteq^*,\Pou(\frak A))$ to
$(\kappa^{\Bbb N},\subseteq^*,\Cal S_{\kappa})$, and
$(\Pou(\frak A),\sqsubseteq^*,\Pou(\frak A))
\prGT(\kappa^{\Bbb N},\subseteq^*,\Cal S_{\kappa})$.
On the other side, we know already that
$(\kappa^{\Bbb N},\subseteq^*,\Cal S_{\kappa})
\prGT(\Cal N_{\kappa},\subseteq,\penalty-50\Cal N_{\kappa})$ (524G);  so
$(\Pou(\frak A),\sqsubseteq^*,\Pou(\frak A))
\prGT(\Cal N_{\kappa},\penalty-100\subseteq\penalty+100,\Cal N_{\kappa})$,
that is, $\Pou(\frak A)\prT\Cal N_{\kappa}$.
}%end of proof of 539C

\leader{539D}{Corollary} Let $\frak A$ be a Maharam algebra.

(a) $\pi(\frak A)\le\max(\cff[\tau(\frak A)]^{\le\omega},\cf\Cal N)$.

(b) If $\tau(\frak A)\le\omega$, then $\wdistr(\frak A)\ge\add\Cal N$.

\proof{ Set $\kappa=\tau(\frak A)$.

\medskip

{\bf (a)} If $\pi(\frak A)$ is countable, or
$\pi(\frak A)\le\cff[\kappa]^{\le\omega}$, we can stop.   Otherwise,
$\kappa$ is infinite and

$$\eqalign{\max(\omega,\kappa)
&\le\max(\omega,\cff[\kappa]^{\le\omega})
<\pi(\frak A)\cr
&=\cov(\frak A^+,\Bsupseteqshort,\frak A^+)
\le\max(\omega,\kappa,
   \cov(\frak A^+,\Bsupseteqshort^{\strprime},[\frak A^+]^{\le\kappa}))
   \cr}$$

\noindent (512Gf), so

$$\eqalignno{\pi(\frak A)
&\le\cov(\frak A^+,\Bsupseteqshort^{\strprime},[\frak A^+]^{\le\kappa})
\le\cov(\Pou(\frak A),\sqsubseteq^*,\Pou(\frak A))\cr
\displaycause{539Ca, 512Da}
&=\cf\,\Pou(\frak A)
\le\cf\Cal N_{\kappa}\cr
\displaycause{539Cb, 513E(e-i)}
&=\max(\cff[\kappa]^{\le\omega},\cf\Cal N)\cr}$$

\noindent (523N).

\medskip

{\bf (b)} If $\kappa$ is finite, $\wdistr(\frak A)=\infty$ and we can stop.
Otherwise, $\kappa=\omega$ and

$$\eqalignno{\wdistr(\frak A)
&=\add\Pou(\frak A)
\Displaycause{512Ee}
\ge\add\Cal N_{\kappa}
\Displaycause{539Cb, 513E(e-ii)}
=\add\Cal N.
\cr}$$
}%end of proof of 539D

\leader{539E}{Proposition}\cmmnt{ ({\smc Veli\v{c}kovi\'c 05},
{\smc Balcar Jech \& Paz\'ak 05})} If
$\frak A$ is an atomless Maharam algebra, not $\{0\}$, there is a sequence
$\sequencen{a_n}$ in $\frak A$ such that $\sup_{n\in I}a_n=1$ and
$\inf_{n\in I}a_n=0$ for every infinite $I\subseteq\Bbb N$.

\proof{ Fix a strictly positive Maharam submeasure
$\nu$ on $\frak A$.

\medskip

{\bf (a)} If $\sequencen{a_n}$ is a sequence in $\frak A$ such that
$\delta=\inf_{n\in\Bbb N}\nu a_n$ is greater than $0$, there
are a non-zero $d\in\frak A$ and an infinite $I\subseteq\Bbb N$ such
that $d\Bsubseteq\sup_{i\in J}a_i$ for every infinite
$J\subseteq I$.   \Prf\Quer\ Otherwise, set $b_J=\sup_{i\in J}a_i$ for
$J\subseteq\Bbb N$.
Choose $\ofamily{\xi}{\omega_1}{I_{\xi}}$,
$\ofamily{\xi}{\omega_1}{c_{\xi}}$ and
$\ofamily{\xi}{\omega_1}{d_{\xi}}$ inductively, as follows.
$I_0=\Bbb N$.   The inductive hypothesis will be that $I_{\xi}$ is an
infinite subset of $\Bbb N$,
$I_{\xi}\setminus I_{\eta}$ is finite whenever $\eta\le\xi$, and
$c_{\xi}\Bcap b_{I_{\xi+1}}=0$ for every $\xi<\omega_1$.
Given $\langle I_{\eta}\rangle_{\eta\le\xi}$ where $\xi<\omega_1$,
set $d_{\xi}=\inf_{n\in\Bbb N}b_{I_{\xi}\setminus n}$.   Since
$\nu b_J\ge\delta$ for every non-empty $J\subseteq\Bbb N$,
$\nu d_{\xi}\ge\delta$
and $d_{\xi}\ne 0$.   By hypothesis, there is an infinite
$I_{\xi+1}\subseteq I_{\xi}$ such that
$c_{\xi}=d_{\xi}\Bsetminus b_{I_{\xi+1}}$
is non-zero.
Given $\ofamily{\eta}{\xi}{I_{\eta}}$ where $\xi<\omega_1$ is a non-zero
limit ordinal, let $I_{\xi}$ be an infinite set such that
$I_{\xi}\setminus I_{\eta}$ is finite for every $\eta<\xi$, and
continue.

Now observe that if $\eta<\xi<\omega_1$, $I_{\xi}\setminus I_{\eta}$ is
finite, so that there is an $n\in\Bbb N$ such that
$I_{\xi}\setminus n\subseteq I_{\eta+1}$, and

\Centerline{$c_{\xi}\Bsubseteq d_{\xi}
\Bsubseteq b_{I_{\xi}\setminus n}\Bsubseteq b_{I_{\eta+1}}$}

\noindent is disjoint from $c_{\eta}$.   But this means that
$\ofamily{\xi}{\omega_1}{c_{\xi}}$ is disjoint, which is impossible,
because $\frak A$ is ccc.\ \Bang\Qed

\medskip

{\bf (b)} Let us say that a Boolean algebra $\frak B$ {\bf splits
reals} if there is a sequence $\sequencen{b_n}$ in $\frak B$ such
that $\sup_{n\in I}b_n=1$ and $\inf_{n\in I}b_n=0$ for every infinite
$I\subseteq\Bbb N$.   Now the set of those
$d\in\frak A$ such that the principal ideal $\frak A_d$ generated by
$d$ splits reals is order-dense in $\frak A$.  \Prf\
Let $a\in\frak A^+$.

\medskip

\quad{\bf case 1} If $\nu\restrp\frak A_a$ is uniformly exhaustive, then
$\frak A_a$ is measurable (539Ab).
Let $\bar\mu$ be a probability measure on
$\frak A_a$;  because $\frak A_a$, like $\frak A$, is atomless, there
is a stochastically independent family $\sequencen{a_n}$ in
$\frak A_a$ with $\bar\mu a_n=\bover12$ for every $n$, and now
$\sequencen{a_n}$ witnesses that $\frak A_a$ splits reals.

\medskip

\quad{\bf case 2} If $\nu\restrp\frak A_a$ is not uniformly exhaustive,
let $\langle b_{ni}\rangle_{i\le n\in\Bbb N}$ be a family of elements
of $\frak A_a$ such that $\langle b_{ni}\rangle_{i\le n}$ is disjoint
for each $n$ and
$\epsilon=\inf_{i\le n\in\Bbb N}\nu b_{ni}$ is greater than $0$.   There is
a family $\ofamily{\xi}{\omega_1}{f_{\xi}}$ in
$\prod_{n\in\Bbb N}\{0,\ldots,n\}$ such that
$\{n:f_{\xi}(n)=f_{\eta}(n)\}$ is finite whenever
$\eta<\xi<\omega_1$.   (For each $\xi<\omega_1$ let
$\theta_{\xi}:\xi\to\Bbb N$ be injective.   Now define
$\ofamily{\xi}{\omega_1}{f_{\xi}}$ inductively by saying that

\Centerline{$f_{\xi}(n)
=\min(\Bbb N\setminus\{f_{\eta}(n):\eta<\xi$, $\theta_{\xi}(\eta)<n\})$}

\noindent for every $\xi<\omega_1$ and $n\in\Bbb N$.)

\Quer\ If for every
$\xi<\omega_1$ and $I\in[\Bbb N]^{\omega}$ there is a
$J\in[I]^{\omega}$ such that
$\inf_{i\in J}b_{i,f_{\xi}(i)}\ne 0$, choose
$\ofamily{\xi}{\omega_1}{I_{\xi}}$ inductively so that
$I_{\xi}\in[\Bbb N]^{\omega}$, $I_{\xi}\setminus I_{\eta}$ is finite
for every $\eta<\xi$, and
$c_{\xi}=\inf_{i\in I_{\xi}}b_{i,f_{\xi}(i)}$ is non-zero for every
$\xi<\omega_1$.   Then
whenever $\eta<\xi$ the set $I_{\xi}\cap I_{\eta}$ is infinite, so
there is an
$i\in I_{\xi}\cap I_{\eta}$ such that $f_{\xi}(i)\ne f_{\eta}(i)$;
now $c_{\xi}\Bcap c_{\eta}
\Bsubseteq b_{i,f_{\xi}(i)}\Bcap b_{i,f_{\eta}(i)}=0$.
But this means that we have an uncountable
disjoint family in $\frak A_a$, which is impossible, because
$\frak A$ is ccc.\ \Bang

Thus we have a $\xi<\omega_1$ and an infinite $I\subseteq\Bbb N$ such
that $\inf_{i\in J}d_i=0$ for every infinite $J\subseteq I$, where
$d_i=b_{i,f_{\xi}(i)}$ for $i\in I$.   Next, applying (a) to
$\familyiI{d_i}$, we have an infinite $K\subseteq I$ and a $d\ne 0$
such that $d=\sup_{i\in J}d_i$ for every infinite $J\subseteq K$.
But this means that
$\family{i}{K}{d\Bcap d_i}$ witnesses that $\frak A_d$ splits reals;
while $d\Bsubseteq a$.

As $a$ is arbitrary, we have the result.\ \Qed

\medskip

{\bf (c)} By 313K, there is a partition $D$ of unity in $\frak A$
such that
$\frak A_d$ splits reals for every $d\in D$;  choose a sequence
$\sequencen{a_{dn}}$ in $\frak A_d$ witnessing this for each $d\in D$.
Set $a_n=\sup_{d\in D}a_{dn}$ for each $n$.   If $I\subseteq\Bbb N$ is
infinite, then

\Centerline{$\sup_{n\in I}a_n=\sup_{d\in D}\sup_{n\in I}a_{dn}
=\sup D=1$,}

\noindent while

\Centerline{$d\Bcap\inf_{n\in I}a_n=\inf_{n\in I}a_{dn}=0$}

\noindent for every $d\in D$, so $\inf_{n\in I}a_n=0$.   Thus
$\sequencen{a_n}$ witnesses that $\frak A$ splits reals, as claimed.
}%end of proof of 539E

\leader{539F}{Definition}\cmmnt{ For the next result I need a name for
one more cardinal between $\omega_1$ and $\frak c$.}   The {\bf
splitting number} $\frak s$ is the least cardinal of any family
$\Cal A\subseteq\Cal P\Bbb N$ such that for every infinite
$I\subseteq\Bbb N$ there is an $A\in\Cal A$ such that $I\cap A$ and
$I\setminus A$ are both infinite.

\leader{539G}{Proposition} Let $X$ be a set, $\Sigma$ a $\sigma$-algebra of
subsets of $X$, and $\nu$ an atomless Maharam submeasure on $\Sigma$.
Let $\Cal M$ be the ideal of meager subsets of $\Bbb R$.

(a) $\non\Cal N(\nu)\ge\max(\frak s,\frakmctbl)$.

(b) $\cov\Cal N(\nu)\le\non\Cal M$.

\proof{ If $\nu X=0$, these are both trivial;  suppose otherwise.

\medskip

{\bf (a)(i)} Suppose that $D\subseteq X$ and $\#(D)<\frakmctbl$.
For any $\epsilon>0$, there is an $F\in\Sigma$ such that
$D\subseteq F$ and $\nu F\le\epsilon$.   \Prf\ By 393I,
there is for each
$n\in\Bbb N$ a finite partition $\Cal E_n$ of $X$ into members of $\Sigma$
such that $\nu E\le 2^{-n-1}\epsilon$ for each $E\in\Cal E_n$.
Express each $\Cal E_n$ as
$\{E_{ni}:i<k(n)\}$.   For $x\in D$,
let $f_x\in\prod_{n\in\Bbb N}k(n)$ be such that $x\in E_{n,f_x(n)}$ for
every $n$.   Because $\#(D)<\frakmctbl$, there is an
$f\in\NN$ such that $f\cap f_x\ne\emptyset$ for every $x\in D$
(522Sb);  we may suppose that $f(n)<k(n)$ for every
$n$.   Set $F=\bigcup_{n\in\Bbb N}E_{n,f(n)}$;  this works.\ \Qed

Applying this repeatedly, we get a sequence
$\sequencen{F_n}$ in $\Sigma$ such that $D\subseteq F_n$ and
$\nu F_n\le 2^{-n}$ for every $n$;  now $F=\bigcap_{n\in\Bbb N}F_n$
includes $D$ and belongs to $\Cal N(\nu)$.   As $D$ is arbitrary,
$\non\Cal N(\nu)\ge\frakmctbl$.

\medskip

\quad{\bf (ii)} Set $\frak A=\Sigma/\Sigma\cap\Cal N(\nu)$, and
define $\bar\nu:\frak A\to\coint{0,\infty}$ by setting
$\bar\nu E^{\ssbullet}=\nu E$ for every $E\in\Sigma$.   Then $\bar\nu$
is a strictly positive atomless Maharam submeasure on $\frak A$.
By 539E, there is a sequence $\sequencen{a_n}$ in $\frak A$ such that
$\sup_{n\in I}a_n=1$ and $\inf_{n\in I}a_n=0$ for every infinite
$I\subseteq\Bbb N$.   For each $n\in\Bbb N$, let $E_n\in\Sigma$ be such
that $E_n^{\ssbullet}=a_n$.

Suppose that $D\subseteq X$ and $\#(D)<\frak s$.   For $x\in D$, set
$A_x=\{n:x\in E_n\}$.   Because $\#(D)<\frak s$, there is an infinite
$I\subseteq\Bbb N$ such that one of $I\cap A_x$, $I\setminus A_x$ is
finite for every $x\in D$.   Set

\Centerline{$F
=\bigcup_{m\in\Bbb N}\bigl((X\setminus\bigcup_{n\in I\setminus m}E_n)
  \cup(\bigcap_{n\in I\setminus m}E_n)\bigr)$;}

\noindent then

\Centerline{$F^{\ssbullet}
=\sup_{m\in\Bbb N}\bigl((1\Bsetminus\sup_{n\in I\setminus m}a_n)
  \cup(\inf_{n\in I\setminus m}a_n)\bigr)=0$,}

\noindent so $F\in\Cal N(\nu)$, while $D\subseteq F$.   As $D$ is
arbitrary, $\non\Cal N(\nu)\ge\frak s$.

\medskip

{\bf (b)} Let $\sequencen{k(n)}$, $\langle E_{ni}\rangle_{i<k(n)}$ and
$\family{x}{X}{f_x}$ be as in (a-i) above, with $\epsilon=1$.   Give
$Z=\prod_{n\in\Bbb N}k(n)$ its compact metrizable product topology.
By 522Wb, there is a family $\ofamily{\xi}{\non\Cal M}{g_{\xi}}$ in
$Z$ such that $\{g_{\xi}:\xi<\non\Cal M\}$ is non-meager.   For each
$f\in Z$, the set

\Centerline{$H(f)=\bigcap_{m\in\Bbb N}\bigcup_{n\ge m}
  \{g:g\in Z$, $g(n)=f(n)\}$}

\noindent is comeager in $Z$, so contains some $g_{\xi}$;  turning this
round, $Z=\bigcup_{\xi<\non\Cal M}H(g_{\xi})$.   Consider the sets
$F_{\xi}=\{x:x\in X$, $f_x\in H(g_{\xi})\}$;  then
$X=\bigcup_{\xi<\non\Cal M}F_{\xi}$, while

\Centerline{$\nu F_{\xi}
\le\inf_{m\in\Bbb N}\sum_{n=m}^{\infty}\nu E_{n,g_{\xi}(n)}
=0$}

\noindent for every $\xi$.   So $\cov\Cal N(\nu)\le\non\Cal M$.
}%end of proof of 539G

\leader{539H}{Corollary} Let $\frak A$ be an atomless Maharam
algebra, not $\{0\}$.   Then $d(\frak A)\ge\max(\frak s,\frakmctbl)$.

\proof{ Let $Z$ be the Stone space of $\frak A$ and $\nuprime$ the totally
finite Radon submeasure on $Z$ corresponding to a strictly positive
Maharam submeasure $\nu$ on $\frak A$ (539Af), so that $\Cal N(\nuprime)$ is
the ideal of meager subsets of $Z$.
Note that the meager sets of $Z$ are all nowhere dense, because $\frak A$
is \wsid\ (316I).   Because $\frak A$ is atomless, so are $\nu$ and $\nuprime$.
As every meager subset of
$Z$ is nowhere dense (and $Z\ne\emptyset$), no dense set can be meager, and

$$\eqalignno{d(\frak A)
&=d(Z)
\Displaycause{514Bd}
\ge\non\Cal N(\nuprime)\ge\max(\frak s,\frakmctbl)
\cr}$$

\noindent by 539Ga.
}%end of proof of 539H

\leader{539I}{Corollary}\dvAformerly{5{}55L} Suppose that
$\#(X)<\max(\frak s,\frakmctbl)$, where $\frak s$ is the splitting number.
Let $\Sigma$ be a $\sigma$-algebra of subsets of $X$ such that
$(X,\Sigma)$ is countably separated\cmmnt{, in the sense that there is a
sequence in $\Sigma$ separating the points of $X$},
and $\Cal I$ a $\sigma$-ideal of
$\Sigma$ containing singletons.   Then there is no non-zero Maharam
submeasure on $\Sigma/\Cal I$.

\proof{{\bf (a)}
Let $\mu$ be a Maharam submeasure on $\Sigma/\Cal I$.   Then we
have a Maharam submeasure $\nu$ on $\Sigma$ defined by setting
$\nu E=\mu E^{\ssbullet}$ for every $E\in\Sigma$, and $\nu\{x\}=0$ for
every $x\in X$.

\medskip

{\bf (b)} $\nu$ is atomless.   \Prf\ Let $\sequencen{E_n}$ be a sequence in
$\Sigma$ separating the points of $X$, and $F\in\Sigma$ such that
$\nu F>0$.
Choose $\sequencen{F_n}$ inductively so that $F_0=F$ and, given that
$\nu F_n>0$, $F_{n+1}$ is either $F_n\cap E_n$ or $F_n\setminus E_n$ and
$\nu F_{n+1}>0$.   Then $\bigcap_{n\in\Bbb N}F_n$ has at most one member,
so $\lim_{n\to\infty}\nu F_n=0$, and there is an $n$ such that
$\nu F_n=\nu(F\cap F_n)$ and $\nu(F\setminus F_n)$ are non-zero.\ \Qed

\medskip

{\bf (c)} By 539Ga,

\Centerline{$\non\Cal N(\nu)\ge\max(\frak s,\frakmctbl)>\#(X)$}

\noindent and $\nu X=0$, so $\mu$ is identically $0$.
}%end of proof of 539I

\leader{539J}{Theorem} (a) Let $\nu$ be a totally finite Radon submeasure
on a Hausdorff
space $X$\cmmnt{ (539Af)} and $\frak A$ its Maharam algebra.
Then $\Cal N(\nu)\prT\Pou(\frak A)$.

(b) Let $\nu$ be a totally finite Radon submeasure on a Hausdorff
space $X$ and $\frak A$ its Maharam algebra.

\quad(i) $\wdistr(\frak A)\le\add\Cal N(\nu)$.

\quad(ii) $\tau(\frak A)\le w(X)$.

\quad(iii)
$\cf\Cal N(\nu)\le\max(\cff[\tau(\frak A)]^{\le\omega},\cf\Cal N)$.

\quad(iv) If $\tau(\frak A)\le\omega$\cmmnt{ (e.g., because $X$ is
second-countable)}, then
$\add\Cal N(\nu)\ge\add\Cal N$ and $\cf\Cal N(\nu)\le\cf\Cal N$.

\proof{{\bf (a)} For $E\in\Cal N(\nu)$, let $\Cal K_E$ be a maximal
disjoint family of compact sets of non-zero submeasure disjoint from $E$,
and set $C_E=\{K^{\ssbullet}:K\in\Cal K_E\}$.   Because $\nu$ is inner
regular with respect to the compact sets, $C_E\in\Pou(\frak A)$.
Now $E\mapsto C_E:\Cal N(\nu)\to\Pou(\frak A)$ is a Tukey function.   \Prf\
Suppose that $\Cal E\subseteq\Cal N(\nu)$ and $D\in\Pou(\frak A)$ are such
that $C_E\sqsubseteq^*D$ for every $E\in\Cal E$;  take any $\epsilon>0$.
Because $D$ is
countable, we have a countable partition $\Cal H$ of $X$ into measurable
sets such that $D=\{H^{\ssbullet}:H\in\Cal H\}$.   Because $\nu$ is inner
regular with respect to the self-supporting compact sets (539Af),
we can find a self-supporting compact set $K\subseteq X$ such that
$\nu(X\setminus K)\le\epsilon$ and $K$ is covered by finitely many members
of $\Cal H$;  consequently $K^{\ssbullet}$ meets only finitely many members
of $D$.

If $E\in\Cal E$, then $K^{\ssbullet}$ meets only finitely many members of
$C_E$, so there is a finite set $\Cal K'_E\subseteq\Cal K_E$ such that
$K\setminus K_E$ is negligible, where $K_E=\bigcup\Cal K'_E$.   But $K_E$
is compact and $K$ is self-supporting, so $K\subseteq K_E$ and
$K\cap E=\emptyset$.

This means that $\bigcup\Cal E\subseteq X\setminus K$ is included in an
open set of submeasure at most $\epsilon$.   This is true for every
$\epsilon>0$, so $\bigcup\Cal E$ is included in a negligible G$_{\delta}$
set and belongs to $\Cal N(\nu)$;  that is, $\Cal E$ is bounded above in
$\Cal N(\nu)$.
As $\Cal E$ is arbitrary, $E\mapsto C_E$ is a Tukey function.\ \Qed

\medskip

{\bf (b)(i)} Putting (a) and 513E(e-ii) together,

\Centerline{$\wdistr(\frak A)=\add\Pou(\frak A)\le\add\Cal N(\nu)$.}

\medskip

\quad{\bf (ii)} If $\Cal U$ is a base for the topology of $X$ with
$\#(\Cal U)=w(X)$, consider $D=\{U^{\ssbullet}:U\in\Cal U\}$ and the
order-closed subalgebra $\frak B$ of $\frak A$ generated by $D$;  note that
$\frak B$ is closed for the order-sequential (or Maharam-algebra)
topology of $\frak A$ (539Ad).
Let $\Cal E$ be the algebra of sets generated by $\Cal U$.
If $F\in\dom\nu$ and $\epsilon>0$, there are compact sets
$K\subseteq F$, $L\subseteq X\setminus F$ such that
$\nu(X\setminus(K\cup L))\le\epsilon$.   There is an $E\in\Cal E$ such
that $K\subseteq E\subseteq X\setminus L$, so
$\nu(E\symmdiff F)\le\epsilon$.   Now $E^{\ssbullet}\in\frak B$ and
$\bar\nu(F^{\ssbullet}\Bsymmdiff E^{\ssbullet})\le\epsilon$;  as $\epsilon$
is arbitrary, $F^{\ssbullet}\in\frak B$;  as $F$ is arbitrary,
$\frak B=\frak A$ and $\frak A$ is $\tau$-generated by $D$.   This means
that $\tau(\frak A)\le\#(D)\le w(X)$, as required.

\medskip

\quad{\bf (iii)} Setting $\kappa=\tau(\frak A)$, (a) and 539Cb tell
us that
$\Cal N(\nu)\prT\Cal N_{\kappa}$, where $\Cal N_{\kappa}$ is the null ideal
of the usual measure on $\{0,1\}^{\kappa}$.
So $\add\Cal N(\nu)\ge\add\Cal N_{\kappa}$ and

\Centerline{$\cf\Cal N(\nu)\le\cf\Cal N_{\kappa}
\le\max(\cff[\kappa]^{\le\omega},\cf\Cal N)$}

\noindent (513E(e-i), 523N).

\medskip

\quad{\bf (iv)} If $\kappa\le\omega$ then
$\Cal N_{\kappa}\prT\Cal N$ so $\add\Cal N(\nu)\ge\add\Cal N$ and
$\cf\Cal N(\nu)\le\cf\Cal N$.
}%end of proof of 539J

\leader{539K}{}\cmmnt{ We can approach precalibers by some of the same
combinatorial methods as before.

\wheader{539K}{6}{2}{2}{72pt}

\noindent}{\bf Proposition} Let $\frak A$ be a Boolean algebra and $\nu$ an
exhaustive submeasure on $\frak A$.

(a) Let $\sequence{i}{a_i}$ be a
sequence in $\frak A$ such that $\inf_{i\in\Bbb N}\nu a_i>0$.

\quad (i) There is an infinite $I\subseteq\Bbb N$ such that
$\{a_i:i\in I\}$ is centered.

\quad(ii) For every $k\in\Bbb N$ there are an $S\in[\Bbb N]^{\omega}$ and
a $\delta>0$ such that
$\nu(\inf_{i\in J}a_i)\ge\delta$ for every $J\in[S]^k$.

(b) Suppose that $\ofamily{\xi}{\kappa}{a_{\xi}}$ is a
family in $\frak A$ such that $\inf_{\xi<\kappa}\nu a_{\xi}>0$,
where $\kappa$ is a regular uncountable cardinal.
Then for every $k\in\Bbb N$ there are a stationary set
$S\subseteq\kappa$ and a $\delta>0$ such that
$\nu(\inf_{i\in J}a_i)\ge\delta$ for every $J\in[S]^k$.

(c) If $\nu$ is strictly positive, then $(\kappa,\kappa,k)$ is
a precaliber triple of $\frak A$ for every regular uncountable
cardinal $\kappa$ and every $k\in\Bbb N$;  in particular, $\frak A$
satisfies Knaster's condition.

\proof{{\bf (a)(i)} This is 392J.

\medskip

\quad{\bf (ii)} Induce on $k$.   The cases $k=0$, $k=1$ are
trivial.   For the inductive step to $k+1$, let
$M\in[\Bbb N]^{\omega}$ and $\delta>0$ be such that
$\nu(\inf_{i\in J}a_i)\ge\delta$ for every $J\in[M]^k$.   \Quer\
Suppose, if possible, that
for every $S\in[M]^{\omega}$ and $\gamma>0$ there is a
$J\in[S]^{k+1}$ such that
$\nu(\inf_{i\in J}a_i)<\gamma$.   Using Ramsey's theorem (4A1G) repeatedly,
we can find
$\sequencen{I_n}$ such that $I_0\in[M]^{\omega}$,
$I_{n+1}\in[I_n]^{\omega}$, $r_n=\min I_n\notin I_{n+1}$ and
$\nu(\inf_{i\in J}a_i)\le 2^{-n-2}\delta$ for every $n\in\Bbb N$ and
$J\in[I_n]^{k+1}$.   Set $S=\{r_n:n\in\Bbb N\}$.
If $J\in[S]^k$ and $\min J=r_n$, then $J\cup\{r_m\}\in[I_m]^{k+1}$, so
$\nu(\inf_{i\in J}a_i\Bcap a_{r_m})\le 2^{-m-2}\delta$, for every
$m<n$.   It follows that
$\nu(\inf_{i\in J}a_i\Bcap\sup_{m<n}a_{r_m})\le\bover12\delta$ and
$\nu(\inf_{i\in J}a_i\Bsetminus\sup_{m<n}a_{r_m})\ge\bover12\delta$.
But this means that
$\nu c_n\ge\bover12\delta$ where
$c_n=a_{r_n}\Bsetminus\sup_{m<n}a_{r_m}$ for each $n$.   As
$\sequencen{c_n}$ is disjoint, this is impossible.\ \Bang

Thus we can find $\gamma>0$ and $S\in[M]^{\omega}$ such that
$\nu(\inf_{i\in J}a_i)\ge\gamma$ for every $J\in[S]^{k+1}$, and the
induction continues.

\medskip

{\bf (b)} Again induce on $k$.   The cases $k=0$, $k=1$ are
trivial.   For the inductive step to $k+1\ge 2$, write
$c_J=\inf_{i\in J}a_i$ for $J\in[\kappa]^{<\omega}$.   We know from
the inductive hypothesis that there are a stationary set
$S\subseteq\kappa$ and a $\delta>0$ such that $\nu c_J\ge 3\delta$
for every $J\in[S]^k$.   For each $\xi\in S$, choose
$m(\xi)\in\Bbb N$ and $\ofamily{i}{m(\xi)}{J_{\xi i}}$ as follows.   Given
$\ofamily{i}{j}{J_{\xi i}}$, where $j\in\Bbb N$, choose, if possible,
$J_{\xi j}\in[S\cap\xi]^k$
such that $\nu(c_{J_{\xi j}}\Bcap c_{J_{\xi i}})\le 2^{-i}\delta$ for
every $i<j$ and
$\nu(a_{\xi}\Bcap c_{J_{\xi j}})\le 2^{-j}\delta$;  if this is not
possible, set $m(\xi)=j$ and stop.   Now the point is that we always
do have to stop.   \Prf\Quer\ Otherwise, set
$d_i=c_{J_{\xi i}}$ for each $i\in\Bbb N$.   Because
$J_{\xi i}\in[S]^k$, $\nu d_i\ge 3\delta$ for each $i$;  also
$\nu(d_i\Bcap d_j)\le 2^{-i}\delta$ for $i<j$;  so $\nu d'_j\ge\delta$,
where $d'_j=d_j\Bsetminus\sup_{i<j}d_i$ for each $j$.   But now
$\sequence{j}{d'_j}$ is disjoint and $\nu$ is not exhaustive.\
\Bang\Qed

At the end of the process, we have $m(\xi)$ and
$\ofamily{i}{m(\xi)}{J_{\xi i}}$ for each
$\xi\in S$.   By the Pressing-Down Lemma (4A1Cc), there are $\tilde m$ and
$\ofamily{i}{\tilde m}{\tilde J_i}$ such that
$S'=\{\xi:\xi\in S$, $m(\xi)=\tilde m$, $J_{\xi i}=\tilde J_i$ for
every $i<\tilde m\}$ is
stationary in $\kappa$.   \Quer\ Suppose, if possible, that
$I\in[S']^{k+1}$ and
$\nu c_I\le 2^{-\tilde m}\delta$.   Set $\xi=\max I$,
$J=I\setminus\{\xi\}$, $\eta=\min I\in J$.
Then $J\in[S\cap\xi]^k$.   For each $i<\tilde m=m(\xi)$,

\Centerline{$\nu(c_J\Bcap c_{J_{\xi i}})
\le\nu(a_{\eta}\Bcap c_{J_{\xi i}})
=\nu(a_{\eta}\Bcap c_{J_{\eta i}})\le 2^{-i}\delta$,}

\noindent while

\Centerline{$\nu(a_{\xi}\Bcap c_J)=\nu c_I\le 2^{-\tilde m}\delta$.}

\noindent  But this means that we could have extended the sequence
$\ofamily{i}{\tilde m}{J_{\xi i}}$ by setting $J_{\xi\tilde m}=J$.\
\Bang

So $S'$ and $2^{-\tilde m}\delta$ provide the next step in the
induction.

\medskip

{\bf (c)} This is now immediate from (b).
}%end of proof of 539K

\leader{539L}{}\cmmnt{ I come now to the work of
{\smc Balcar Jech \& Paz\'ak 05}, based on the characterizations of Maharam
algebras set out in \S393.

\woddheader{539L}{4}{2}{2}{60pt}

\noindent}{\bf Lemma}\cmmnt{ ({\smc Quickert 02})} Let
$\frak A$ be a Boolean
algebra, and $\Cal I$ the family of countable subsets $I$ of
$\frak A$ for which there is a partition $C$ of unity
such that $\{a:a\in I$, $a\Bcap c\ne 0\}$ is finite for every $c\in C$.

(a) $\Cal I$ is an ideal of $\Cal P\frak A$ including
$[\frak A]^{<\omega}$.

(b) If $A\subseteq\frak A^+$ is such that $A\cap I$ is
finite for every $I\in\Cal I$, and
$B=\{b:b\Bsupseteq a$ for some $a\in A\}$, then $B\cap I$ is finite
for every $I\in\Cal I$.

(c) If $\frak A$ is ccc, then there is no uncountable
$B\subseteq\frak A$ such that $[B]^{\le\omega}\subseteq\Cal I$.

(d) If $\frak A$ is ccc and \wsid, $\Cal I$ is a
$p$-ideal\cmmnt{ (definition:  5A6Ga)}.

\proof{{\bf (a)} Of course every finite subset of $\frak A$
belongs to $\Cal I$.   If $I_0$, $I_1\in\Cal I$ and
$J\subseteq I_0\cup I_1$, then
$J\in[\frak A]^{\le\omega}$.   For each $j$, we have a partition $C_j$
of unity in $\frak A$ such that $\{a:a\in I_j$, $a\Bcap c\ne 0\}$ is
finite for every $c\in C_j$.   Set
$C=\{c_0\Bcap c_1:c_0\in C_0$, $c_1\in C_1\}$;  then $C$ is a
partition
of unity in $\frak A$ and $\{a:a\in J$, $a\Bcap c\ne 0\}$ is finite
for every $c\in C$.

\medskip

{\bf (b)} Take $I\in\Cal I$.   Set $J=B\cap I$.   For each
$b\in J$, let $a_b\in A$ be such that $a_b\Bsubseteq b$.   Let $C$ be
a partition of unity such that $\{b:b\in I$, $b\Bcap c\ne 0\}$ is finite
for every $c\in C$;  then $\{a_b:b\in J$, $a_b\Bcap c\ne 0\}$ is
finite
for every $c\in C$, so $\{a_b:b\in J\}$ belongs to $\Cal I$ and must be
finite.   \Quer\ If $J$ is infinite, there is an $a\in A$ such that
$K=\{b:b\in J$, $a=a_b\}$ is infinite;  but in this case there is a
$c\in C$ such that $a\Bcap c\ne 0$ and $b\Bcap c\ne 0$ for every
$b\in K$.\ \BanG\  So $J$ is finite, as claimed.

\medskip

{\bf (c)} Let $\widehat{\frak A}$ be the Dedekind completion of
$\frak A$ (314U).   Let $B\subseteq\frak A$ be an uncountable set, and
$\ofamily{\xi}{\omega_1}{b_{\xi}}$ a family of distinct elements of
$B$.   Set
$d=\inf_{\xi<\omega_1}\sup_{\xi\le\eta<\omega_1}b_{\eta}$, taken in
$\widehat{\frak A}$.   Then (because $\widehat{\frak A}$ is ccc, by
514Ee) $d=\sup_{\xi\le\eta<\omega_1}b_{\eta}$ for
some $\xi$ (316E);  in particular, $d\ne 0$.   Next, we can find a strictly
increasing sequence $\sequencen{\xi_n}$ in $\omega_1$ such that
$d\Bsubseteq\sup_{\xi_n\le\eta<\xi_{n+1}}b_{\eta}$ for every
$n\in\Bbb N$.   Set
$I=\{b_{\eta}:\eta<\sup_{n\in\Bbb N}\xi_n\}\in[B]^{\le\omega}$.   If
$C$ is any partition of unity in $\frak A$, there must be some $c\in C$
such that $c\Bcap d\ne 0$, and now $\{a:a\in I$, $a\Bcap c\ne 0\}$ is
infinite.   So $I\notin\Cal I$.

\medskip

{\bf (d)} Let $\sequencen{I_n}$ be a sequence in $\Cal I$.
For each $n\in\Bbb N$, let $C_n$ be a partition of unity such
that $\{a:a\in I_n$, $a\Bcap c\ne 0\}$ is finite for every $c\in C_n$.
Let $D$ be a partition of unity such that
$\{c:c\in C_n$, $c\Bcap d\ne 0\}$ is finite for every $d\in D$ and
$n\in\Bbb N$.   Then

\Centerline{$\{a:a\in I_n$, $a\Bcap d\ne 0\}
\subseteq\bigcup_{c\in C_n,c\Bcap d\ne 0}
  \{a:a\in I_n$, $a\Bcap c\ne 0\}$}

\noindent is finite for every $d\in D$ and $n\in\Bbb N$.   Let
$\sequencen{d_n}$ be a sequence running over $D\cup\{\emptyset\}$ and
set $I=\bigcup_{n\in\Bbb N}\{a:a\in I_n$, $a\Bcap d_i=0$ for every
$i\le n\}$.   Then

\Centerline{$I_n\setminus I\subseteq\bigcup_{i\le n}\{a:a\in I_n$,
$a\Bcap d_i\ne\emptyset\}$}

\noindent is finite for each $n$.   Also

\Centerline{$\{a:a\in I$, $a\Bcap d_n\ne 0\}
\subseteq\bigcup_{i<n}\{a:a\in I_i$, $a\Bcap d_n\ne 0\}$}

\noindent is finite for each $n$, so $I\in\Cal I$.
}%end of proof of 539L

\medskip

\noindent{\bf Remark}\cmmnt{ In this context,} $\Cal I$ is called
{\bf Quickert's ideal}.

\leader{539M}{Lemma} Let $\frak A$
be a \wsid\ ccc Dedekind $\sigma$-complete Boolean algebra, and suppose
that $\frak A^+$ is expressible as $\bigcup_{k\in\Bbb N}D_k$
where no infinite subset of any $D_k$ belongs to Quickert's ideal
$\Cal I$.   Then $\frak A$ is a Maharam algebra.

\proof{ The point is that if $\sequencen{a_n}$ is a
sequence in $\frak A$ which order*-converges to $0$, then
$\{a_n:n\in\Bbb N\}\in\Cal I$ (539A(d-i)).   So no
sequence in any $D_k$ can order*-converge to $0$.   Because $\frak A$
is \wsid\ and ccc, $0$ does not belong to the closure $\overline{D}_k$ of
$D_k$ for the order-sequential topology on $\frak A$
(539A(d-iv)).   So $\frak A^+=\bigcup_{k\in\Bbb N}\overline{D}_k$
is F$_{\sigma}$ and $\{0\}$ is G$_{\delta}$ for the order-sequential
topology.   It follows that $\frak A$ is a Maharam algebra (539A(d-vi)).
}%end of proof of 539M

\leader{539N}{Theorem} ({\smc Balcar Jech \& Paz\'ak 05},
{\smc Veli\v{c}kovi\'c 05}) Suppose that
Todor\v{c}evi\'c's $p$-ideal dichotomy\cmmnt{ (5A6Gb)} is true.
Then every Dedekind
$\sigma$-complete ccc \wsid\ Boolean algebra is a Maharam algebra.

\proof{ Let $\frak A$ be a Dedekind $\sigma$-complete ccc \wsid\
Boolean algebra.   Let $\Cal I$ be Quickert's ideal on $\frak A$;
then $\Cal I$ is a $p$-ideal (539Ld).
By 539Lc, there is no $B\in[\frak A]^{\omega_1}$ such that
$[B]^{\le\omega}\subseteq\Cal I$.
We are assuming that Todor\v{c}evi\'c's $p$-ideal dichotomy is true;  so
$\frak A$ must be expressible as $\bigcup_{n\in\Bbb N}D_n$ where no
infinite subset of any $D_n$ belongs to
$\Cal I$.   By 539M, $\frak A$ is a Maharam algebra.
}%end of proof of 539N

\leader{539O}{Corollary} Suppose that
Todor\v{c}evi\'c's $p$-ideal dichotomy is true.   Let $\frak A$ be a
Dedekind complete Boolean algebra such
that every countably generated order-closed subalgebra of $\frak A$ is a
measurable algebra.   Then $\frak A$ is a measurable algebra.

\proof{{\bf (a)} $\frak A$ is ccc.   \Prf\Quer\ Otherwise, let
$\ofamily{\xi}{\omega_1}{a_{\xi}}$ be a disjoint family of
non-zero elements of $\frak A$.
Let $f:\omega_1\to\{0,1\}^{\Bbb N}$ be an injective function, and set
$b_n=\sup\{a_{\xi}:\xi<\omega_1$, $f_{\xi}(n)=1\}$ for each $n$;  let
$\frak B$ be the order-closed subalgebra of $\frak A$ generated by
$\{b_n:n\in\Bbb N\}\cup\{\sup_{\xi<\omega_1}a_{\xi}\}$.
Then $a_{\xi}\in\frak B$ for every $\xi<\omega_1$, so $\frak B$ is not ccc;
but $\frak B$ is supposed to be measurable.\ \Bang\Qed

\medskip

{\bf (b)} $\frak A$ is \wsid.   \Prf\ Let $\sequencen{C_n}$ be a sequence
of partitions of unity in $\frak A$.   As $\frak A$ is ccc, every $C_n$ is
countable;  let $\frak B$ be the order-closed subalgebra of $\frak A$
generated by $\bigcup_{n\in\Bbb N}C_n$.   Then $\frak B$ is measurable,
therefore \wsid, and there is a partition $D$ of unity in $\frak B$ such
that $\{c:c\in C_n$, $c\Bcap d\ne 0\}$ is finite for every $n\in\Bbb N$ and
$d\in D$.   As $\frak B$ is order-closed, $D$ is still a partition of unity
in $\frak A$.   As $\sequencen{C_n}$ is arbitrary, $\frak A$ is \wsid.\
\Qed

\medskip

{\bf (c)} By 539N, $\frak A$ is a Maharam algebra;  let $\nu$ be a strictly
positive Maharam submeasure on $\frak A$.   Now $\nu$ is uniformly
exhaustive.   \Prf\Quer\ Otherwise, there are $\epsilon>0$ and a family
$\langle a_{ni}\rangle_{i\le n\in\Bbb N}$ in $\frak A$ such that
$\langle a_{ni}\rangle_{i\le n}$ is disjoint for every $n\in\Bbb N$ and
$\nu a_{ni}\ge\epsilon$ whenever $i\le n\in\Bbb N$.   Let $\frak B$ be the
order-closed subalgebra of $\frak A$ generated by
$\{a_{ni}:i\le n\in\Bbb N\}$.   Then $\frak B$ is a measurable algebra;
let $\bar\mu$ be a functional such that $(\frak B,\bar\mu)$ is a totally
finite measure algebra.   Since $\bar\mu$ and $\nu\restr B$ are both
strictly positive Maharam submeasures on $\frak B$, $\nu$ is absolutely
continuous with respect to $\bar\mu$ (539Ac).
But $\nu a_{ni}\ge\epsilon$ for every $n$ and
$i$, while $\inf_{i\le n\in\Bbb N}\bar\mu a_{ni}$ must be zero.\ \Bang\Qed

\medskip

{\bf (d)} So $\frak A$ is a Dedekind $\sigma$-complete Boolean algebra
with a strictly positive uniformly exhaustive Maharam submeasure, and is a
measurable algebra (539Ab).
}%end of proof of 539O

\leader{539P}{}\cmmnt{ I should say at once that 539N-539O really do need
some special axiom.   In fact the following example was found at the very
beginning of the study of Maharam algebras.

\medskip

\noindent}{\bf Souslin algebras:  Proposition}
Suppose that $T$ is a well-pruned
Souslin tree\cmmnt{ (554Yc, 5A1Dd)}, and set $\frak A=\RO^{\uparrow}(T)$.

(a) $\frak A$ is Dedekind complete, ccc and \wsid.

(b) If $\frak B$ is an order-closed subalgebra of $\frak A$ and
$\tau(\frak B)\le\omega$, then $\frak B\cong\Cal PI$ for some countable set
$I$;  in particular, $\frak B$ is a measurable algebra.

(c)\cmmnt{ ({\smc Maharam 47})} The only Maharam submeasure on
$\frak A$ is identically zero.

\proof{{\bf (a)(i)} $\frak A$ is Dedekind complete just because it is a
regular open algebra.

\medskip

\quad{\bf (ii)} $T$ is upwards-ccc, so $\frak A$ is ccc, by 514Nc.

\medskip

\quad{\bf (iii)} For $t\in T$, set
$\widehat{t}=\interior\overline{\coint{t,\infty}}\in\frak A$;  then
$\{\widehat{t}:t\in T\}$ is order-dense in $\frak A$.   Let
$r:T\to\On$ be the rank function of $T$ (5A1Da).   For each
$\xi<\omega_1$, $A_{\xi}=\{\widehat{t}:t\in T$, $r(t)=\xi\}$
is a partition of
unity in $\frak A$.   \Prf\ If $r(t)=r(t')$ and $t\ne t'$ then
$\coint{t,\infty}\cap\coint{t',\infty}=\emptyset$ so
$\widehat{t}\cap\widehat{t'}=0$ in $\frak A$;  thus $A_{\xi}$ is disjoint.
If $a\in\frak A\setminus\{0\}$, there is an $s\in T$ such that
$\widehat{s}\Bsubseteq a$;   if $r(s)\ge\xi$, there is a $t\le s$ such that
$r(t)=\xi$, and $a\Bcap\widehat{t}\ne 0$;  if $r(s)<\xi$, there is a
$t\ge s$ such that $r(t)=\xi$ (because $T$ is well-pruned),
and $\widehat{t}\Bsubseteq a$.   Thus $\sup A_{\xi}=1$ in $\frak A$.\ \Qed

If $A\subseteq\frak A$ is a partition of unity, there is a $\xi<\omega_1$
such that $A_{\xi}$ refines $A$ in the sense that every member of $A_{\xi}$
is included in some member of $A$ (see 311Ge).  \Prf\
$B=\{\widehat{t}:t\in T$, $\widehat{t}\Bsubseteq a$ for some $a\in A\}$ is
order-dense in $\frak A$, so there is a partition $C$ of unity included in
$B$;  $C$ is countable;  let $D\subseteq T$ be a countable set such that
$C=\{\widehat{t}:t\in D\}$;  set $\xi=\sup_{t\in D}r(t)$.\ \Qed

Of course $A_{\eta}$ refines $A_{\xi}$ whenever $\xi\le\eta<\omega_1$.
So if $\sequencen{C_n}$ is a sequence of partitions of unity in $\frak A$,
there is a $\xi<\omega_1$ such that $A_{\xi}$ refines $C_n$ for every
$n\in\Bbb N$, and then $\{c:c\in C_n$, $a\Bcap c\ne 0\}$ has just one
member for
every $a\in A_{\xi}$ and $n\in\Bbb N$.   As $\sequencen{C_n}$ is
arbitrary, $\frak A$ is \wsid.

\medskip

{\bf (b)} If $B\subseteq\frak A$ is a countable set $\tau$-generating
$\frak B$, there is a countable set $D\subseteq T$ such that
$b=\sup\{\widehat{t}:t\in D$, $\widehat{t}\Bsubseteq b\}$ for every
$b\in B$.   Now $\xi=\sup\{r(t):t\in D\}$ is countable, and
$b=\sup\{a:a\in A_{\xi}$, $a\Bsubseteq b\}$ for every
$b\in B$, so $\frak B$ is included in the order-closed subalgebra $\frak C$
of $\frak A$ generated by $A_{\xi}$.   Of course $A_{\xi}$ is order-dense
in $\frak C$.   For $a\in A_{\xi}$, set
$b_a=\inf\{b:b\in\frak B$, $b\Bsupseteq a\}$;  then every $b_a$ is an atom
in $\frak B$ and $\{b_a:a\in A_{\xi}\}$ is order-dense in $\frak B$, so
$\frak B$ is purely atomic.   As $\frak B$ is ccc, the set $I$ of its atoms
is countable;  being Dedekind complete, $\frak B$ is isomophic to
$\Cal PI$.  %316Xf

\medskip

{\bf (c)} Let $\nu$ be a Maharam submeasure on $\frak A$.   Then for every
$\epsilon>0$ there is a $\xi<\omega_1$ such that $\nu a\le\epsilon$ for
every $a\in A_{\xi}$.   \Prf\ Set

\Centerline{$T'=\{t:\nu\widehat{t}\ge\epsilon\}$.}

\noindent Then $T'$ is a subtree of $T$ and $\{t:t\in T'$, $r(t)=\xi\}$ is
finite for every $\xi<\omega_1$, because $\nu$ is exhaustive.   Also $T'$,
like $T$, can have no uncountable branches.   It follows that the height of
$T'$ is countable (5A1D(b-i)),
that is, that there is a $\xi<\omega_1$ such
that $r(t)<\xi$ for every $t\in T'$ and $\nu a\le\epsilon$ for every
$a\in A_{\xi}$.\ \Qed

As this is true for every $\epsilon>0$, there is actually a $\xi<\omega_1$
such that $\nu a=0$ for every $a\in A_{\xi}$.   But as $A_{\xi}$ is a
countable partition of unity and $\nu$ is a Maharam submeasure, $\nu 1=0$
and $\nu$ is identically zero.
}%end of proof of 539P

\leader{539Q}{Reflection principles }\cmmnt{In 539O, we have a theorem
of the type `if every small subalgebra of $\frak A$ is $\ldots$,
then $\frak A$ is $\ldots$'.   There was a similar result in 518I, and we
shall have another in 545G.   Here I collect some simple facts which are
relevant to the present discussion.

\medskip

}{\bf (a)} If $\frak A$ is a Boolean algebra and every subset of
$\frak A$ of
cardinal $\omega_1$ is included in a ccc subalgebra of $\frak A$, then
$\frak A$ is ccc.   \prooflet{(For there can be no disjoint set with cardinal
$\omega_1$.)}

\spheader 539Qb If $\frak A$ is ccc and every countable subset of
$\frak A$ is included
in a \wsid\ subalgebra of $\frak A$, then $\frak A$ is \wsid.
\prooflet{\Prf\ If $C_n$ is a partition of unity in $\frak A$
for every $n$, set

\Centerline{$D=\{d:\{c:c\in C_n$, $c\Bcap d\ne 0\}$ is finite
for every $n\in\Bbb N\}$.}

\noindent\Quer\ If $D$ is not order-dense in $\frak A$, take
$a\in\frak A^+$
such that $d\notBsubseteq a$ for every $d\in D$.   Let $\frak B$ be a
\wsid\ subalgebra of $\frak A$ including
$\{a\}\cup\bigcup_{n\in\Bbb N}C_n$.   Then every $C_n$ is a partition of
unity in $\frak B$, so there is a partition $B$ of unity in $\frak B$ such
that $B\subseteq D$.   But now $a\in\frak B^+$ so there is a $b\in B$ such
that $a\Bcap b\ne 0$ and $a\Bcap b\in D$.\ \Bang

So $D$ is order-dense in $\frak A$ and includes a partition of unity
in $\frak A$.   As $\sequencen{C_n}$ is arbitrary, $\frak A$ is \wsid.\
\Qed}

\spheader 539Qc\dvArevised{2014} If every countable subset of
$\frak A$ is included in a subalgebra of $\frak A$ with the
$\sigma$-interpolation property, then $\frak A$ has the
$\sigma$-interpolation property.
\prooflet{\Prf\ If $A$, $B\subseteq\frak A$ are countable and
$a\Bsubseteq b$ whenever $a\in A$ and $b\in B$, let $\frak B$ be a
subalgebra of $\frak A$, including $A\cup B$,
with the $\sigma$-interpolation property;  then there is a $c\in\frak B$
such that $a\Bsubseteq c\Bsubseteq b$ for every $a\in A$ and $b\in B$.\
\Qed}

\spheader 539Qd If $\frak A$ is a Maharam algebra and every countably
generated closed subalgebra of $\frak A$ is a measurable algebra, then
$\frak A$ is measurable.
\prooflet{(This is part (c) of the proof of 539O.)}

\spheader 539Qe
Suppose that Todor\v{c}evi\'c's $p$-ideal dichotomy is true.   Let
$\frak A$ be a Boolean algebra such that every subset of $\frak A$ of
cardinal at most $\omega_1$ is included in a subalgebra of $\frak A$ which
is a Maharam algebra.   Then $\frak A$ is a Maharam algebra.
\prooflet{\Prf\ By (a),
$\frak A$ is ccc;  by (c), $\frak A$ is Dedekind complete;  by (b),
$\frak A$ is \wsid;  by 539N, $\frak A$ is a Maharam algebra.\ \Qed}

\spheader 539Qf
Suppose that Todor\v{c}evi\'c's $p$-ideal dichotomy is true.   Let
$\frak A$ be a Boolean algebra such that every subset of $\frak A$ of
cardinal at most $\frak c$ is included in a subalgebra of $\frak A$ which
is a measurable algebra.   Then $\frak A$ is measurable.
\prooflet{\Prf\
By (a), $\frak A$ is ccc.   So if $\frak B$ is a countably generated
order-closed subalgebra, it has cardinal $\frak c$, and is included in a
measurable subalgebra $\frak C$ of $\frak A$.   Now $\frak B$ is
order-closed in $\frak C$, so is itself a measurable algebra.   By 539O,
$\frak A$ also is measurable.\ \Qed}

\spheader 539Qg
On the other hand,\cmmnt{ {\smc Farah \& Veli\v{c}kovi\'c 06} show
that} if
$\kappa$ is an infinite cardinal such that $2^{\kappa}=\kappa^+$,
$\square_{\kappa}$\cmmnt{ (5A6D)}
is true and the cardinal power $\kappa^{\omega}$ is
equal to $\kappa$, then there is a Dedekind complete Boolean algebra
$\frak A$, with cardinal $\kappa^+$, such that every order-closed subalgebra
of $\frak A$ with cardinal at most $\kappa$ is a measurable algebra, but
$\frak A$ is not a measurable
algebra (and therefore is not a Maharam algebra\cmmnt{,
by (d) above}). %539Qd
In particular, this can\cmmnt{ easily} be the case with $\kappa=\frak c$.

\leader{539R}{Exhaustivity \dvrocolon{rank}}\cmmnt{ While we now know
that there are
non-measurable Maharam algebras, we know practically nothing about their
structure.   I introduce the following idea as a possible tool for
investigation.

\medskip

\noindent}{\bf Definitions} Suppose that $\frak A$ is a
Boolean algebra and $\nu$ an exhaustive
submeasure on $\frak A$.   For $\epsilon>0$, say that
$a\preccurlyeq_{\epsilon}b$ if either $a=b$ or $a\Bsubseteq b$ and
$\nu(b\Bsetminus a)>\epsilon$.
Then $\preccurlyeq_{\epsilon}$ is a well-founded partial order on
$\frak A$\cmmnt{ (use 5A1Cc;
if $\sequencen{a_n}$ were strictly decreasing for $\preccurlyeq_{\epsilon}$, then
$\sequencen{a_n\Bsetminus a_{n+1}}$ would be disjoint, with
$\nu(a_n\Bsetminus a_{n+1})\ge\epsilon$ for every $n$)}.   Let
$r_{\epsilon}:\frak A\to\On$
be the corresponding rank function\cmmnt{, so that

\Centerline{$r_{\epsilon}(a)=\sup\{r_{\epsilon}(b)+1:b\Bsubseteq a$,
$\nu(a\Bsetminus b)>\epsilon\}$}

\noindent for every $a\in\frak A$ (5A1Cb)}.   Now the
{\bf exhaustivity rank} of $\nu$ is $\sup_{\epsilon>0}r_{\epsilon}(1)$.

\leader{539S}{Elementary facts} Let $\frak A$ be a Boolean algebra with
an exhaustive submeasure $\nu$ and associated rank functions
$r_{\epsilon}$ for $\epsilon>0$.

\spheader 539Sa $r_{\delta}(a)\le r_{\epsilon}(b)$ whenever
$\nu(a\Bsetminus b)\le\delta-\epsilon$.
\prooflet{\Prf\ Induce on $r_{\epsilon}(b)$.   If $r_{\epsilon}(b)=0$,
then $\nu b\le\epsilon$ so $\nu a\le\delta$ and $r_{\delta}(a)=0$.
For the inductive step to $r_{\epsilon}(b)=\xi$,
if $c\Bsubseteq a$ and $\nu(a\Bsetminus c)>\delta$ then
$\nu(b\Bsetminus c)>\epsilon$ and
$r_{\epsilon}(b\Bcap c)<\xi$.   Also
$\nu(c\Bsetminus b)\le\delta-\epsilon$ so, by the inductive
hypothesis, $r_{\delta}(c)\le r_{\delta}(b\Bcap c)<\xi$;  as $c$ is
arbitrary,
$r_{\delta}(a)\le\xi$ and the induction continues.\ \QeD\ }\cmmnt{ In
particular,}

\Centerline{$r_{\epsilon}(a)\le r_{\epsilon}(b)$ if $a\Bsubseteq b$,
\quad$r_{\delta}(a)\le r_{\epsilon}(a)$ if $\epsilon\le\delta$.}

\spheader 539Sb If $a$, $b\in\frak A$ are disjoint and $\epsilon>0$, then
$r_{\epsilon}(a\Bcup b)$ is at least the ordinal sum
$r_{\epsilon}(a)+r_{\epsilon}(b)$.
\prooflet{\Prf\ Induce on $r_{\epsilon}(b)$.   If $r_{\epsilon}(b)=0$, the
result is immediate from (a) above.   For the inductive step to
$r_{\epsilon}(b)=\xi$, we have for any $\eta<\xi$ a $c\Bsubseteq b$
such that $\nu(b\Bsetminus c)>\epsilon$ and
$\eta\le r_{\epsilon}(c)<\xi$.   Now
$r_{\epsilon}(a\Bcup c)\ge r_{\epsilon}(a)+\eta$,
by the inductive hypothesis, and
$\nu((a\Bcup b)\Bsetminus(a\Bcup c))>\epsilon$, so
$r_{\epsilon}(a\Bcup b)>r_{\epsilon}(a)+\eta$;  as $\eta$ is arbitrary,
$r_{\epsilon}(a\Bcup b)\ge r_{\epsilon}(a)+\xi$ and the induction
continues.\ \Qed}

\leaveitout{\spheader 539Sc
If $\nuprime$ is another exhaustive submeasure on $\frak A$
with rank functions $r'_{\epsilon}$, and $\nu a\le\alpha\nuprime a$
for every $a\in\frak A$, where $\alpha>0$, then
$r_{\alpha\epsilon}(a)\ge r'_{\epsilon}(a)$ for every $a\in\frak A$
and $\epsilon>0$ (induce on $r'_{\epsilon}(a)$, as usual).
%do we want this?
}

\leader{539T}{The rank of a Maharam algebra} Note that the rank
function $r_{\epsilon}$ associated with an exhaustive submeasure
$\nu$ depends only on the set $\{a:\nu a>\epsilon\}$.   In
particular, if $\nu$ and $\nuprime$ are exhaustive submeasures on a
Boolean algebra $\frak A$ and
$\nu a\le\epsilon$ whenever $\nuprime a\le\delta$, then
$r^{(\nu)}_{\epsilon}(a)\le r^{(\nu')}_{\delta}(a)$ for every
$a\in\frak A$.   If $\frak A$ is a Maharam algebra, then any two
Maharam submeasures on $\frak A$ are mutually absolutely continuous
\cmmnt{(539Ac)}, so have the same exhaustivity rank;
I will call this the
{\bf Maharam submeasure rank} of $\frak A$, $\Mhsr(\frak A)$.
Note that if $a\in\frak A$ then $\Mhsr(\frak A_a)\le\Mhsr(\frak A)$.

If $\frak A$ is a measurable algebra, $\Mhsr(\frak A)\le\omega$\cmmnt{,
because if $\mu$ is an additive functional and $\epsilon>0$,
then $\mu a>\epsilon r_{\epsilon}^{(\mu)}(a)$ for every $a\in\frak A$}.
More generally,
for any uniformly exhaustive submeasure $\nu$ and $\epsilon>0$,
$r^{(\nu)}_{\epsilon}(a)$ is finite, being
the maximal size of any disjoint set
consisting of elements, included in $a$,
of submeasure greater than $\epsilon$.

\leader{539U}{Theorem} Suppose that $\frak A$ is a non-measurable
Maharam algebra.   Then
$\Mhsr(\frak A)$ is at least the ordinal power $\omega^{\omega}$.

\proof{ Let $\nu$ be a strictly positive Maharam submeasure on $\frak A$.

\medskip

{\bf (a)} For the time being (down to the end of (d) below), assume that
$\frak A$ is nowhere measurable (definition:  391Bc).
For $a\in\frak A$, set

\Centerline{$\check\nu a
=\inf_{n\in\Bbb N}\sup\{\min_{i\le n}\nu a_i:
    a_0,\ldots,a_n\Bsubseteq a$ are disjoint$\}$.}

\noindent Then $\check\nu$ is a Maharam submeasure.   \Prf\ Of course
$\check\nu 0=0$ and $\check\nu a\le\check\nu b$ whenever
$a\Bsubseteq b$.   If $a$, $b\in\frak A$ and $\epsilon>0$, then there
are $n_0$, $n_1\in\Bbb N$ such that whenever $\familyiI{c_i}$ is a
disjoint family in $\frak A$, then
$\#(\{i:\nu(c_i\Bcap a)\ge\check\nu a+\epsilon\})\le n_0$ and
$\#(\{i:\nu(c_i\Bcap b)\ge\check\nu b+\epsilon\})\le n_1$.   So

\Centerline{$\#(\{i:\nu(c_i\Bcap(a\Bcup b))
\ge\check\nu a+\check\nu b+2\epsilon\})\le n_0+n_1$.}

\noindent It follows that
$\check\nu(a\Bcup b)\le\check\nu a+\check\nu b+2\epsilon$;  as
$\epsilon$, $a$ and $b$ are arbitrary, $\check\nu$ is a submeasure.
Because $\check\nu\le\nu$, $\check\nu$ is a Maharam submeasure.\ \Qed

\medskip

{\bf (b)} Because $\frak A$ is nowhere measurable, $\check\nu$ is strictly
positive.   \Prf\ If $a\in\frak A\setminus\{0\}$, the principal ideal
$\frak A_a$ is not measurable, so the Maharam submeasure
$\nu\restrp\frak A_a$ cannot be uniformly exhaustive;  that is, there is an
$\epsilon>0$ such that there are arbitrarily long disjoint strings
$\langle a_i\rangle_{i\le n}$ in $\frak A_a$ with $\nu a_i\ge\epsilon$ for
every $i\le n$.   But this means that $\check\nu a\ge\epsilon>0$.\ \Qed

\medskip

{\bf (c)} Let $r_{\epsilon}$,
$\check r_{\epsilon}$ be the rank functions associated with
$\nu$ and $\check\nu$.   Then
$r_{\epsilon}(a)$ is at least the ordinal product
$\omega\cdot\check r_{\epsilon}(a)$ whenever $a\in\frak A$ and
$\epsilon>0$.
\Prf\ Induce on $\check r_{\epsilon}(a)$.   If
$\check r_{\epsilon}(a)=0$, the result is trivial.   For the
inductive step to $\check r_{\epsilon}(a)=\xi+1$, take
$b\Bsubseteq a$ such that $\check\nu b>\epsilon$ and
$\check r_{\epsilon}(a\Bsetminus b)=\xi$.
Then for every $n\in\Bbb N$ there are disjoint
$b_0,\ldots,b_n\Bsubseteq b$ such that $\nu b_i>\epsilon$ for every
$i$, and $r_{\epsilon}(b)\ge\omega$;  by the inductive hypothesis,
$r_{\epsilon}(a\Bsetminus b)\ge\omega\cdot\xi$;  by 539Sb,
$r_{\epsilon}(a)\ge\omega\cdot\xi+\omega=\omega\cdot(\xi+1)$, and the
induction proceeds.   The inductive step to non-zero limit $\xi$ is
elementary.\ \Qed

\medskip

{\bf (d)} Now

$$\eqalignno{\Mhsr(\frak A)
&=\sup_{\epsilon>0}r_{\epsilon}(1)
\ge\sup_{\epsilon>0}\omega\cdot\check r_{\epsilon}(1)
=\omega\cdot\sup_{\epsilon>0}\check r_{\epsilon}(1)\cr
\displaycause{5A1Bb}
&=\omega\cdot\Mhsr(\frak A);\cr}$$

\noindent as $\Mhsr(\frak A)>0$, $\Mhsr(\frak A)\ge\omega^{\omega}$
(5A1Bc).

\medskip

{\bf (e)} For the general case,
let $a\in\frak A^+$ be such that the
principal ideal $\frak A_a$ is nowhere measurable.   Then
$\Mhsr(\frak A)\ge\Mhsr(\frak A_a)\ge\omega^{\omega}$.
}%end of proof of 539U

\exercises{\leader{539X}{Basic exercises (a)}
%\spheader 539Xa
Let $\frak A$ be a Maharam algebra.    Show that
$\link_n(\frak A)\le\max(\omega,\tau(\frak A))$ for every $n\ge 2$.
%539B

\spheader 539Xb Show that, in the language of \S522,
$\frak p\le\frak s\le\min(\non\Cal N,\non\Cal M,\frak d)$.
%539F 53bits

\spheader 539Xc Let $\frak A$ be a Maharam algebra.   (i) Show that if

\inset{($\alpha$) $\cff[\lambda]^{\le\omega}\le\lambda^+$ for every
cardinal $\lambda\le\tau(\frak A)$,

($\beta$) $\square_{\lambda}$ is true for every uncountable cardinal
$\lambda\le\tau(\frak A)$ of countable cofinality,}

\noindent then $\FN(\frak A)\le\FN(\Cal P\Bbb N)$, with equality unless
$\frak A$ is finite.   \Hint{518D, 518I.}   (ii) Show that if
$\#(\frak A)\le\omega_2$ and $\FN(\Cal P\Bbb N)=\omega_1$, then
$\frak A$ is tightly $\omega_1$-filtered.   \Hint{518M.}
%539H

\spheader 539Xd Let $X$ be a set, $\Sigma$ a $\sigma$-algebra of subsets of
$X$, and $\nu:\Sigma\to\coint{0,\infty}$ a non-zero Maharam submeasure;
set $\Cal I=\{E:E\in\Sigma$, $\nu E=0\}$ and $\frak A=\Sigma/\Cal I$.
Suppose that $\#(\frak A)\le\omega_2$ and $\FN(\Cal P\Bbb N)=\omega_1$.
Show that there is a lifting for $\nu$, that is, a Boolean homomorphism
$\theta:\frak A\to\Sigma$ such that $(\theta a)^{\ssbullet}=a$ for every
$a\in\frak A$.   \Hint{518L.}
%539Xc 539H out of order query

\spheader 539Xe Let $\frak A$ be a Boolean algebra, $\nu$ an exhaustive
submeasure on $\frak A$, and $\sequence{i}{a_i}$ a sequence in $\frak A$
such that $\inf_{i\in\Bbb N}\nu a_i>0$.   Let $\Cal F$ be a Ramsey
ultrafilter on $\Bbb N$.   (i) Show that there is an $I\in\Cal F$ such
$\inf_{i,j\in I}\nu(a_i\Bcap a_j)>0$.   (ii) Show that for every
$k\in\Bbb N$ there is an $I\in\Cal F$ such that
$\inf\{\nu(\inf_{i\in K}a_i):K\in[I]^k\}>0$.
(iii) Show that there is an $I\in\Cal F$ such that
$\{a_i:i\in I\}$ is centered.  \Hint{538Hc.}
%539K for (iii), work in $\widehat{\frak A}$

\spheader 539Xf Let $X$ be a set, $\Sigma$ a $\sigma$-algebra of
subsets of $X$, and $\Cal I\normalsubgroup\Cal PX$ a $\sigma$-ideal;
suppose that $\Sigma/\Sigma\cap\Cal I$ is ccc.
Let $Y$ be a set, $\Tau$ a $\sigma$-algebra of subsets of $Y$, and
$\nu:\Tau\to\coint{0,\infty}$ a Maharam submeasure;
let $\Cal I\ltimes\Cal N(\nu)$ be the skew product
as defined in 527B.   Show that $(\Sigma\tensorhat\Tau)
/(\Sigma\tensorhat\Tau)\cap(\Cal I\ltimes\Cal N(\nu))$ is ccc.
\Hint{527L.}
%527L

\leader{539Y}{Further exercises (a)}
%\spheader 539Ya
Let $\frak A$ be a Dedekind $\sigma$-complete Boolean
algebra with a countable $\sigma$-generating set (331E), and $\nu$
a Maharam submeasure on $\frak A$.   Set $\Cal I=\{a:\nu a=0\}$.
Show that $\Cal I\prT\Cal N$.
%539J 341Yf 496K 539Ja 539Cb

\spheader 539Yb\dvArevised{2014}
Let $X$ be a set, $\Sigma$ a $\sigma$-algebra of subsets of
$X$, and $\Cal I$ a proper $\sigma$-ideal of subsets of $X$ generated by
$\Sigma\cap\Cal I$;  let $\Sigma_L$ be the algebra of Lebesgue measurable
subsets of $\Bbb R$.   Write $\frak A$ for $\Sigma/\Sigma\cap\Cal I$,
$\Cal L$ for $(\Sigma\tensorhat\Sigma_L)\cap(\Cal I\ltimes\Cal N)$ and
$\frak C$ for $\Sigma\tensorhat\Sigma_L/\Cal L$.
(i) Show that $c(\frak C)=\max(\omega,c(\frak A))$ and
$\tau(\frak C)=\max(\omega,\tau(\frak A))$.
(ii) Show that $\frak C$ is \wsid\ iff $\frak A$ is.
% what about $\wdistr(\frak C)=\min(\cf\Cal N,\wdistr(\frak A))$?.
(iii) Show that $\frak C$ is measurable iff $\frak A$ is.
(iv) Show that $\frak C$ is a Maharam algebra iff $\frak A$ is.
% is $\Mhsr(\frak C)=\omega\cdot\Mhsr(\frak A)$?.\query
% not if \frak A measurable!  or \max(\omega,\Mhsr(\frak A)) ??
%539P 53bits

\spheader 539Yc Let $\frak A$ be a Boolean algebra with a
strictly positive Maharam submeasure $\hat\nu$, and $\frak B$ a
subalgebra of $\frak A$ which is dense for the associated metric (539Ac);
set $\nu=\hat\nu\restrp\frak B$, so
that $\nu$ is an exhaustive submeasure on $\frak B$.   For $\epsilon>0$ let
$r_{\epsilon}:\frak B\to\On$ and
$\hat r_{\epsilon}:\frak A\to\On$ be the rank functions
associated with $\nu$ and $\hat\nu$ respectively.   Show that

\Centerline{$r_{\delta}(b)\le\hat r_{\delta}(b)
\le r_{\epsilon}(b)$}

\noindent whenever $b\in\frak B$ and $0<\epsilon<\delta$.
%539S

\spheader 539Yd Let $\frak A$ be an infinite Maharam algebra.
Show that $\Mhsr(\frak A)<\tau(\frak A)^+$.
%539Xa 539Yc 539S

\spheader 539Ye (J.Kupka) Let $\nu$ be a
totally finite submeasure on a Boolean algebra $\frak A$, and set

\Centerline{$\check\nu a
=\inf_{n\in\Bbb N}\sup\{\min_{i\le n}\nu a_i:
    a_0,\ldots,a_n\Bsubseteq a$ are disjoint$\}$.}

\noindent for $a\in\frak A$, as in the proof of 539U.   Show that {\it
either} $\check\nu\ge\bover13\nu$ {\it or} there is a non-zero additive
$\mu:\frak A\to\coint{0,\infty}$ such that $\mu a\le\nu a$ for every
$a\in\frak A$.  \Hint{392D.}
%539U

\spheader 539Yf Show that the exhaustive submeasures constructed by
Talagrand's method, as described in \S394,
have exhaustivity rank at most the ordinal power $\omega^{\omega^2}$.
%n06204 539U

\spheader 539Yg\dvAnew{2014}
Suppose that $\frak A$ is a non-measurable Maharam algebra.
Show that $\Mhsr(\frak A)=\omega\cdot\Mhsr(\frak A)$.
%539U
}%end of exercises

\leader{539Z}{Problems (a)}
%\spheader 539Za
Let $\nu$ be a non-zero totally finite Radon submeasure on a Hausdorff
space $X$.   Must there be a lifting for $\nu$?
\cmmnt{that is, writing $\Sigma$ for the domain of $\nu$, must there be a
Boolean homomorphism $\phi:\Sigma\to\Sigma$ such that
$\nu(E\symmdiff\phi E)=0$ for every $E\in\Sigma$ and $\phi E=\emptyset$
whenever $\nu E=0$?}

\spheader 539Zb Can a non-measurable Maharam algebra $\frak A$ have
Maharam submeasure rank different from $\omega^{\omega^2}$?
%539U 539Yf

\endnotes{
\Notesheader{539} During the growth of this treatise, the
sections on Maharam submeasures were twice transformed by
new discoveries, and I naturally hope that the work I have just presented
will be similarly outdated before too long.
In the pages above I have tried in
the first place to show how the cardinal functions of chapters 51 and 52
can be applied in this more general context.   With minor refinements of
technique, we can go a fair way.   Because we know we have at least two
non-trivial atomless Maharam algebras of countable type, we are led to a
more detailed analysis, as in 539Ca and 539J.

Equally instructive are the
apparent limits to what the methods can achieve, which mostly point to
remaining areas of obscurity.   I say `remaining';  but what is most
conspicuous about the present situation is our nearly total ignorance
concerning the structure of non-measurable Maharam algebras.   Talagrand's
construction, as described in \S394, gives us a family of such algebras,
but so far we can answer hardly any of the most elementary questions
about them (539Yf, 394Z).

The message of {\smc Balcar Jech \& Paz\'ak 05} is that a Dedekind
complete, ccc, \wsid\ Boolean algebra is `nearly' a Maharam algebra.   Any
further condition (e.g., the $\sigma$-finite chain condition, as in
393S) is likely to render it a Maharam algebra;
and with a little help from an extra axiom of set theory, it is
already necessarily a Maharam algebra
(539N).   Similarly, much of the work of the last sixty years on
submeasures suggests that exhaustive submeasures are `nearly'
uniformly exhaustive, and that an extra condition (e.g., sub- or
super-modularity) is enough to tip the balance (413Yf).
At both boundaries, there are few examples to limit
conjectures about further conditions on which such results might be based.
Besides 539P and Talagrand's examples, we have a further important
possibility of a not-quite-Maharam algebra in 555K below.
}%end of notes

\discrpage

