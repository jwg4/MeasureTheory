\frfilename{mt452.tex}
\versiondate{6.11.08}
\copyrightdate{2002}

\def\chaptername{Perfect measures, disintegrations and processes}
\def\sectionname{Integration and disintegration of measures}

\def\undtheta{\underline{\theta}}
\def\undphi{\underline{\phi}}
\def\undpsi{\underline{\psi}}

\newsection{452}

A standard method of defining measures is through a formula

\Centerline{$\mu E=\int\mu_yE\,\nu(dy)$}

\noindent where $(Y,\Tau,\nu)$ is a measure space and
$\family{y}{Y}{\mu_y}$ is a family of measures on another set $X$.   In
practice these constructions commonly involve technical problems
concerning the domain of $\mu$\cmmnt{ (as in 452Xi)}, which is why I
have hardly used them so
far in this treatise.   There are not-quite-trivial examples in 417Yb,
434R and 436F, and the indefinite-integral measures of \S234 can also be
expressed in this way\cmmnt{ (452Xf)};  for a case in which this
approach is worked out fully, see 453N.
%Losert
But when a formula of this kind is valid, as in Fubini's
theorem, it is likely to be so useful that it dominates
further investigation of the topic.   In this section I give one of the
two most important theorems guaranteeing the existence of appropriate
families $\family{y}{Y}{\mu_y}$ when $\mu$ and $\nu$ are given (452I);
the other will follow in the next section\cmmnt{ (453K)}.   They
both suppose that we are provided with a suitable function $f:X\to Y$,
and rely heavily on the Lifting Theorem\cmmnt{ (\S341)} and on
considerations of inner regularity from Chapter 41.

The formal definition of a `disintegration' (which is nearly the same
thing as a `regular conditional probability') is in 452E.   The main
theorem depends, for its full generality, on the concept of `countably
compact measure'\cmmnt{ (451B)}.   It can be strengthened when $\mu$
is actually a Radon measure (452O).

The greater part of the section is concerned with general
disintegrations, in which the measures $\mu_y$ are supposed to be
measures on $X$ and are not necessarily
related to any particular structure on $X$.   However a natural,
and obviously important, class of applications has $X=Y\times Z$ and
each $\mu_y$ based on
the section $\{y\}\times Z$, so that it can be regarded as a measure on
$Z$.   Mostly there is very little more to be said in this case (see
452B-452D);  %452B 452C 452D
but in 452M we find that there is an interesting variation in the way that
countable compactness can be used.

\leader{452A}{Lemma} Let $(Y,\Tau,\nu)$ be a measure space, $X$ a
set, and $\family{y}{Y}{\mu_y}$ a family of measures on $X$.
Let $\Cal A$ be the family of subsets $A$ of $X$ such that
$\theta E=\int\mu_yE\,\nu(dy)$ is defined in $\Bbb R$.   Suppose that
$X\in\Cal A$.

(a) $\Cal A$ is a Dynkin class.

(b) If $\Sigma$ is any $\sigma$-subalgebra of $\Cal A$ then
$\mu=\theta\restr\Sigma$ is a measure on $X$.

(c) Suppose now that every $\mu_y$ is complete.
If, in (b), $\hat\mu$ is the completion of $\mu$ and $\hat\Sigma$ its
domain, then $\hat\Sigma\subseteq\Cal A$ and
$\hat\mu=\theta\restr\hat\Sigma$.

\proof{ For (a) and (b), we have only to look at the definitions of
`Dynkin class' and `measure' and apply the elementary properties of the
integral.    For (c),
if $E\in\hat\Sigma$, then there are $E'$, $E''\in\Sigma$ such that
$E'\subseteq E\subseteq E''$ and $\theta E'=\theta E''$.   So
$\mu_yE'=\mu_yE''$ for
$\nu$-almost every $y$;  since all the $\mu_y$ are supposed to be
complete, $\mu_yE$ is defined and equal to $\mu_yE'$ for almost every
$y$, and
$\theta E$ is defined and equal to $\theta E'=\mu E'=\hat\mu E$.
}%end of proof of 452A

\leader{452B}{Theorem} (a) Let $X$ be a set, $(Y,\Tau,\nu)$ a
measure space, and $\family{y}{Y}{\mu_y}$ a family of
measures on $X$ such that $\int\mu_yX\,\nu(dy)$ is defined and finite.
Let $\Cal E$ be
a family of subsets of $X$, closed under finite intersections, such that
$\int\mu_yE\,\nu(dy)$ is defined in $\Bbb R$ for every $E\in\Cal E$.

\quad(i) If $\Sigma$ is the $\sigma$-algebra of subsets of $X$ generated
by $\Cal E$, we have a totally finite measure $\mu$ on $X$, with domain
$\Sigma$, given by the
formula $\mu E=\int\mu_yE\,\nu(dy)$ for every $E\in\Sigma$.

\quad(ii) If $\hat\mu$ is the completion of $\mu$ and $\hat\Sigma$ its
domain, then $\int\hat\mu_yE\,\nu(dy)$ is defined and equal to
$\hat\mu E$ for every $E\in\hat\Sigma$,
where $\hat\mu_y$ is the completion of $\mu_y$ for each $y\in Y$.

(b) Let $Z$ be a set, $(Y,\Tau,\nu)$ a measure space, and
$\family{y}{Y}{\mu_y}$ a family of measures on $Z$ such that
$\int\mu_yZ\,\nu(dy)$ is defined and finite.   Let
$\Cal H$ be a family of subsets of $Z$,
closed under finite intersections, such that
$\int\mu_yH\,\nu(dy)$ is
defined in $\Bbb R$ for every $H\in\Cal H$.

\quad(i) If $\Upsilon$ is the $\sigma$-algebra of subsets of $Z$
generated by $\Cal H$, we have a
totally finite measure $\mu$ on $Y\times Z$, with domain
$\Tau\tensorhat\Upsilon$, defined by setting
$\mu E=\int\mu_yE[\{y\}]\nu(dy)$ for every
$E\in\Tau\tensorhat\Upsilon$.

\quad(ii) If $\hat\mu$ is the completion of $\mu$ and $\hat\Sigma$ its
domain, then $\int\hat\mu_yE[\{y\}]\nu(dy)$ is defined and equal to
$\hat\mu E$ for every $E\in\hat\Sigma$,
where $\hat\mu_y$ is the completion of $\mu_y$ for each $y\in Y$.

\proof{{\bf (a)} Define $\Cal A\subseteq\Cal PX$ as in 452A.
Then $\Cal E\subseteq\Cal A$, so by the Monotone Class Theorem (136B)
$\Sigma\subseteq\Cal A$ and we have
(i).   Applying 452Ac to $\family{y}{Y}{\hat\mu_y}$ we have (ii).

\medskip

{\bf (b)} Set $X=Y\times Z$.  For $y\in Y$, let $\mu'_y$ be the measure
on $X$ defined by setting $\mu'_yE=\mu_yE[\{y\}]$ whenever this is
defined;  that is, $\mu'_y$ is the
image of $\mu_y$ under the function $z\mapsto(y,z):Z\to X$.   Set
$\Cal E=\{F\times H:F\in\Tau$, $H\in\Cal H\}$.
Then $\Cal E$ is a family of subsets of $X$ closed under
finite intersections, and

\Centerline{$\int\mu'_y(F\times H)\nu(dy)=\int\chi F(y)\mu_yH\,\nu(dy)$}

\noindent is defined whenever $F\in\Tau$ and $H\in\Cal H$.   By (a), we
have a measure $\mu$ on $X$, with domain the $\sigma$-algebra $\Sigma$
generated by $\Cal E$, defined by writing

\Centerline{$\mu E=\int\mu'_yE\,\nu(dy)=\int\mu_yE[\{y\}]\nu(dy)$}

\noindent for every $E\in\Sigma$.   Of course $\Sigma$ includes
$\Tau\tensorhat\Upsilon$ (the set $\{H:Y\times H\in\Sigma\}$ is a
$\sigma$-algebra of subsets of $Z$ including
$\Cal H$, so includes $\Upsilon$) and is therefore equal to
$\Tau\tensorhat\Upsilon$.

\wheader{452B}{0}{0}{0}{30pt}
This proves (i).   If now $E\in\hat\Sigma$, (a-ii) tells us that

\Centerline{$\hat\mu
E=\int\hat\mu'_yE\,\nu(dy)=\int\hat\mu_yE[\{y\}]\nu(dy)$.}
}%end of proof of 452B

\vleader{72pt}{452C}{Theorem} (a) Let $Y$ be a topological space, $\nu$ a
$\tau$-additive topological measure on $Y$, $(X,\frak T)$ a
topological space, and $\family{y}{Y}{\mu_y}$
a family of $\tau$-additive topological measures on $X$ such that
$\int\mu_yX\,\nu(dy)$ is defined and finite.
Suppose that there is a base $\Cal U$ for $\frak T$, closed under finite
unions, such
that $y\mapsto\mu_yU$ is lower semi-continuous for every $U\in\Cal U$.

\quad(i) We can define a $\tau$-additive Borel measure
$\mu$ on $X$ by writing
$\mu E\penalty-100=\penalty-100\int\mu_yE\,\nu(dy)$ for every Borel set
$E\subseteq X$.

\quad(ii) If $\hat\mu$ is the completion of $\mu$ and $\hat\Sigma$ its
domain, then $\int\hat\mu_yE\,\nu(dy)$ is defined and equal to
$\hat\mu E$ for every $E\in\hat\Sigma$,
where $\hat\mu_y$ is the completion of $\mu_y$ for each $y\in Y$.

(b) Let $Y$ be a topological space, $\nu$ a $\tau$-additive topological
measure on $Y$, $(Z,\frak U)$ a topological space, and
$\family{y}{Y}{\mu_y}$
a family of $\tau$-additive topological measures on $Z$ such that
$\int\mu_yZ\,\nu(dy)$ is defined and finite.
Suppose that there is a base $\Cal V$ for $\frak U$, closed under finite
unions, such
that $y\mapsto\mu_yV$ is lower semi-continuous for every $V\in\Cal V$.

\quad(i) We can define a $\tau$-additive Borel measure $\mu$ on
$Y\times Z$ by writing $\mu E=\int\mu_yE[\{y\}]\nu(dy)$ for
every Borel set $E\subseteq Y\times Z$.

\quad(ii) If $\hat\mu$ is the completion of $\mu$ and $\hat\Sigma$ its
domain, then $\int\hat\mu_yE[\{y\}]\nu(dy)$ is defined and equal to
$\hat\mu E$ for every $E\in\hat\Sigma$,
where $\hat\mu_y$ is the completion of $\mu_y$ for each $y\in Y$.

\proof{{\bf (a)} For $A\subseteq X$, set $f_A(y)=\mu_yA$ when this is
defined.   We may suppose that $\emptyset\in\Cal U$.
If $\Cal W\subseteq\Cal U$ is a non-empty upwards-directed set with
union $G$, $\family{W}{\Cal W}{f_W}$ is an upwards-directed family of
lower semi-continuous functions
with supremum $f_G$, because every $\mu_y$ is $\tau$-additive.   So
$f_G$ is lower semi-continuous, and also
$\int f_Gd\nu=\sup_{W\in\Cal W}\int f_Wd\nu$, by 414Ba.
Taking $\Cal E$ to be the family of open subsets of $X$ in 452Ba, we see
that we have a $\tau$-additive Borel measure $\mu$ on $X$ such that
$\mu E=\int\mu_yE\,\nu(dy)$ for every Borel set $E\subseteq X$.
Moreover, if $\Cal G$ is a non-empty upwards-directed family of open
subsets of $X$ with union $G^*$, then
$\Cal W=\{W:W\in\Cal U$, $W\subseteq G$ for some $G\in\Cal G\}$ is an
upwards-directed family with union $G^*$, so

\Centerline{$\mu G^*=\int f_{G^*}d\nu
=\sup_{W\in\Cal W}\int f_Wd\nu\le\sup_{G\in\Cal G}\mu G\le\mu G^*$.}

\noindent As $\Cal G$ is arbitrary, $\mu$ is $\tau$-additive.   This
proves (i);  (ii) follows immediately, as in 452Ba.

\medskip

{\bf (b)} Let $\Cal U$ be the family of sets expressible as
$\bigcup_{i\le n}H_i\times V_i$ where $H_i\subseteq Y$ is open and
$V_i\in\Cal V$ for every $i\le n$.   Because $\Cal V$ is a base for
$\frak U$, $\Cal U$ is a base for the topology of $X=Y\times Z$.   For
$y\in Y$ let $\mu'_y$ be the
measure on $X$ defined by saying that $\mu'_yE=\mu_yE[\{y\}]$ whenever
this is defined.   Then $\mu'_y$ is a $\tau$-additive topological
probability measure on $X$,
by 418Ha or otherwise.   If $U\in\Cal U$, $y\mapsto\mu'_yU$ is lower
semi-continuous.   \Prf\  Express $U$ as $\bigcup_{i\le n}H_i\times V_i$
where $H_i\subseteq Y$
is open and $V_i\in\Cal V$ for each $i$.    Suppose that $y\in Y$ and
$\gamma<\mu_yU$.   Set $I=\{i:i\le n$, $y\in H_i\}$,
$H=Y\cap\bigcap_{i\in I}H_i$
and $V=\bigcup_{i\in I}V_i$.   Then $U[\{y\}]=V\subseteq U[\{y'\}]$ for
every $y'\in H$.   Also $H'=\{y':\mu_{y'}V>\gamma\}$ is a neighbourhood
of $y$.   So $H\cap H'$ is
a neighbourhood of $y$, and $\mu'_{y'}U>\gamma$ for every
$y'\in H\cap H'$.   As $y$ and $\gamma$ are arbitrary, we have the result.\
\Qed

Now applying (a) to $\family{y}{Y}{\mu'_y}$ we see that (b) is true.
}%end of proof of 452C

\leader{452D}{Theorem} (a) Let $(Y,\frak S,\Tau,\nu)$ be a Radon
measure space, $(X,\frak T)$ a topological space, and
$\family{y}{Y}{\mu_y}$
a uniformly tight\cmmnt{ (definition:  437O)} family of Radon
measures on $X$ such that $\int\mu_yX\,\nu(dy)$ is defined and finite.
Suppose that there is a base $\Cal U$ for $\frak T$, closed under finite
unions, such
that $y\mapsto\mu_yU$ is lower semi-continuous for every $U\in\Cal U$.
Then we have a totally finite Radon measure $\tilde\mu$ on $X$ defined by
saying that that $\tilde\mu E=\int\mu_yE\,\nu(dy)$ whenever $\tilde\mu$
measures $E$.

(b) Let $(Y,\frak S,\Tau,\nu)$ be a Radon measure space, $(Z,\frak U)$ a
topological space, and $\family{y}{Y}{\mu_y}$
a uniformly tight family of Radon measures on $Z$ such that
$\int\mu_yZ\,\nu(dy)$ is defined and finite.
Suppose that there is a base $\Cal V$ for $\frak U$, closed under finite
unions, such
that $y\mapsto\mu_yV$ is lower semi-continuous for every $V\in\Cal V$.
Then we have a totally finite Radon measure $\tilde\mu$ on $Y\times Z$
such that $\tilde\mu E=\int\mu_yE[\{y\}]\nu(dy)$ whenever $\tilde\mu$
measures $E$.

\proof{ I take the two parts together.   By 452C we have a
$\tau$-additive Borel measure $\mu$ satisfying the
appropriate formula.
Now for any $\epsilon>0$ there is a compact set $K\subseteq X$ such that
$\mu K\ge\mu X-2\epsilon$.   \Prf\ In (a), take $\eta>0$ such that
$\int\min(\eta,\mu_yX)\nu(dy)\le 2\epsilon$, and $K$ such that
$\mu_y(X\setminus K)\le\eta$ for every $y\in Y$.
In (b), take $\eta>0$
such that $\int\min(\eta,\mu_yZ)\nu(dy)\le\epsilon$.   Now let
$K_1\subseteq Y$ and $K_2\subseteq Z$ be compact sets such that

\Centerline{$\int_{K_1}\mu_yZ\,\nu(dy)\ge\int_Y\mu_yZ\,\nu(dy)-\epsilon$,
\quad$\mu_y(Z\setminus K_2)\le\eta$ for every $y\in Y$.}

\noindent Then $K=K_1\times K_2$
is compact and

$$\eqalign{\mu((Y\times Z)\setminus K)
&\le\int_{Y\setminus K_1}\mu_yZ\,\nu(dy)
  +\int_{K_1}\mu_y(Z\setminus K_2)\nu(dy)\cr
&\le\epsilon+\int_{K_1}\min(\eta,\mu_yZ)\nu(dy)
\le 2\epsilon.  \text{ \Qed}\cr}$$

Since $\mu$ is totally finite it is surely locally finite and
effectively locally finite, so the conditions of 416F(iv) are satisfied
and the c.l.d.\ version
$\tilde\mu$ of $\mu$ is a Radon measure on $X$.   But of course
$\tilde\mu$ is just the completion of $\mu$, so 452C(a-ii) or 452C(b-ii)
tells us that the declared formula also applies to $\tilde\mu$.
}%end of proof of 452D

\leader{452E}{}\cmmnt{ All the constructions above can be thought of
as special cases of the following.

\medskip

\noindent}{\bf Definition}  Let $(X,\Sigma,\mu)$ and $(Y,\Tau,\nu)$ be
measure spaces.   A {\bf disintegration} of $\mu$ over $\nu$ is a family
$\family{y}{Y}{\mu_y}$ of measures on $X$ such that
$\int\mu_yE\,\nu(dy)$
is defined in $[0,\infty]$ and equal to $\mu E$ for every $E\in\Sigma$.
If $f:X\to Y$ is an \imp\ function, a disintegration
$\family{y}{Y}{\mu_y}$ of $\mu$ over $\nu$ is {\bf consistent} with $f$
if, for each $F\in\Tau$, $\mu_yf^{-1}[F]=1$ for $\nu$-almost every
$y\in F$.   $\family{y}{Y}{\mu_y}$ is {\bf strongly consistent} with $f$
if, for almost every $y\in Y$, $\mu_y$ is a probability measure for which
$f^{-1}[\{y\}]$ is conegligible.

\cmmnt{A trivial example of a disintegration is when $\nu$ is a
probability measure and $\mu_y=\mu$ for every $y$.   Of course this is
of little interest.   The archetypal disintegration is 452Bb when all
the $\mu_y$ are the same, in which case Fubini's theorem tells us that
we are looking at a product measure on $X=Y\times Z$.
If $\mu$ is a probability measure then this disintegration
is strongly consistent.

The phrase {\bf regular conditional probability} is used for special
types of disintegration;  typically, when $\mu$ and $\nu$ and every
$\mu_y$ are probabilities, and sometimes supposing that every $\mu_y$
has the same domain as $\mu$.   I have seen the word {\bf decomposition}
used for what I call a disintegration.
}%end of comment

\leader{452F}{Proposition} Let $(X,\Sigma,\mu)$ and $(Y,\Tau,\nu)$ be
measure spaces and $\family{y}{Y}{\mu_y}$ a disintegration of $\mu$ over
$\nu$.   Then $\iint f(x)\mu_y(dx)\nu(dy)$ is defined and equal to
$\int fd\mu$ for every $[-\infty,\infty]$-valued function $f$ such that
$\int fd\mu$ is defined in $[-\infty,\infty]$.

\proof{{\bf (a)} Suppose first that $f$ is non-negative.   Let
$H\in\Sigma$ be a conegligible set such that $f\restr H$ is
$\Sigma$-measurable.   For $n\in\Bbb N$ set

\Centerline{$E_{nk}=\{x:x\in H,\,2^{-n}k\le f(x)\}$ for $k\ge 1$,
\quad $f_n=2^{-n}\sum_{k=1}^{4^n}\chi E_{nk}$.}

\noindent Then $\sequencen{f_n}$ is a non-decreasing sequence of
functions with $\lim_{n\to\infty}f_n(x)=f(x)$ for every $x\in H$.   Now
$\int\mu_y(X\setminus H)\nu(dy)=0$, so $X\setminus H$ is
$\mu_y$-negligible for almost every $y$.   Set

\Centerline{$V=\{y:\mu_y(X\setminus H)=0$, $E_{nk}\in\dom\mu_y$ for
every $n\in\Bbb N$, $k\ge 1\}$;}

\noindent then $V$ is $\nu$-conegligible.   For $y\in V$,

\Centerline{$\int fd\mu_y
=\lim_{n\to\infty}\int f_nd\mu_y
=\lim_{n\to\infty}2^{-n}$$\sum_{k=1}^{4^n}\mu_yE_{nk}$,}

\noindent while each function $y\mapsto\mu_yE_{nk}$ is $\nu$-virtually
measurable, so $y\mapsto\int fd\mu_y$ is $\nu$-virtually measurable
and

$$\eqalign{\iint fd\mu_y\nu(dy)
&=\lim_{n\to\infty}\iint f_nd\mu_y\nu(dy)
=\lim_{n\to\infty}2^{-n}\sum_{k=1}^{4^n}\int\mu_yE_{nk}\,\nu(dy)\cr
&=\lim_{n\to\infty}2^{-n}\sum_{k=1}^{4^n}\mu E_{nk}
=\lim_{n\to\infty}\int f_nd\mu
=\int fd\mu.\cr}$$

\medskip

{\bf (b)} For general $f$ we now have

$$\eqalign{\iint f(x)\mu_y(dx)\nu(dy)
&=\iint f^+(x)\mu_y(dx)\nu(dy)-\iint f^-(x)\mu_y(dx)\nu(dy)\cr
&=\int f^+d\mu-\int f^-d\mu
=\int fd\mu,\cr}$$

\noindent where $f^+$, $f^-$ are the positive and negative parts of $f$.
}%end of proof of 452F

\medskip

\noindent{\bf Remark} When $X=Y\times Z$ and our disintegration is a
family $\family{y}{Y}{\mu'_y}$ of measures on $X$ defined from a family
$\family{y}{Y}{\mu_y}$ of probability measures
on $Z$\cmmnt{, as in 452Bb,} we can more naturally write
$\int f(y,z)\mu_y(dz)$ in place of $\int f(x)\mu'_y(dx)$, and we get

\Centerline{$\iint f(y,z)\mu_y(dz)\nu(dy)=\int fd\mu$ whenever the
latter is defined in $[-\infty,\infty]$\dvro{.}{}}

\cmmnt{\noindent as in 252B.}

\leader{452G}{}\cmmnt{ The most useful theorems about disintegrations
of course involve some restrictions on their form, most commonly
involving consistency with some kind of projection.   I clear the path
with statements of some elementary facts.

\medskip

\noindent}{\bf Proposition} Let $(X,\Sigma,\mu)$ and $(Y,\Tau,\nu)$ be
measure spaces, $f:X\to Y$ an \imp\ function, and $\family{y}{Y}{\mu_y}$
a disintegration of $\mu$ over $\nu$.

(a) If $\family{y}{Y}{\mu_y}$ is consistent with $f$, and $F\in\Tau$,
then $\mu_yf^{-1}[F]=\chi F(y)$ for $\nu$-almost every $y\in Y$;  in
particular, almost every $\mu_y$ is a probability measure.

(b) If $\family{y}{Y}{\mu_y}$ is strongly consistent with $f$ it is
consistent with $f$.

(c) If $\nu$ is countably separated\cmmnt{ (definition: 343D)} and
$\family{y}{Y}{\mu_y}$ is consistent with $f$, then it is strongly
consistent with $f$.

\proof{{\bf (a)} We have $\mu_yf^{-1}[F]=1$ for almost every $y\in F$.
Since also

\Centerline{$\mu_y(X\setminus f^{-1}[F])=\mu_yf^{-1}[Y\setminus F]=1$,
\quad$\mu_yX=\mu_yf^{-1}[Y]=1$}

\noindent for almost every $y\in Y\setminus F$, $\mu_yf^{-1}[F]=0$ for
almost every $y\in X\setminus F$.

\medskip

{\bf (b)} If $F\in\Tau$, then $f^{-1}[F]\supseteq f^{-1}[\{y\}]$ is $\mu_y$-conegligible for almost every $y\in F$;  since we are also
told that $\mu_yX=1$ for almost every $y$, $\mu_yf^{-1}[F]=1$ for almost
every $y\in F$.

\medskip

{\bf (c)} There is a countable $\Cal F\subseteq\Tau$ separating the
points of $Y$;  we may suppose that $Y\in\Cal F$ and that
$Y\setminus F\in\Cal F$ for every $F\in\Cal F$.   Now

\Centerline{$H_F=F\setminus\{y:\mu_yf^{-1}[F]$ is defined and equal to
$1\}$}

\noindent is negligible for every $F\in\Cal F$, so that

\Centerline{$Z
=Y\setminus\bigcup_{F\in\Cal F}H_F$}

\noindent is conegligible.   For $y\in Z$, set
$\Cal F_y=\{F:y\in F\in\Cal F\}$;  then

\Centerline{$\{y\}=\bigcap\Cal F_y$,
\quad$f^{-1}[\{y\}]=\bigcap\{f^{-1}[F]:F\in\Cal F_y\}$,}

\noindent while $\mu_yf^{-1}[F]=1$ for every $F\in\Cal F_y$.   Because
$\Cal F_y$ is countable, $\mu_yf^{-1}[\{y\}]=1$.   This is true for
almost every $y$, so $\family{y}{Y}{\mu_y}$ is strongly consistent with
$f$.
}%end of proof of 452G

\leader{452H}{Lemma} Let $(X,\Sigma,\mu)$ and $(Y,\Tau,\nu)$ be
probability spaces, and
$T:L^{\infty}(\mu)\to L^{\infty}(\nu)$ a positive linear operator such
that $T(\chi X^{\ssbullet})=\chi Y^{\ssbullet}$ and $\int Tu=\int u$
whenever $u\in L^{\infty}(\mu)^+$.   Let $\Cal K$ be a countably compact
class of subsets of $X$, closed under finite unions and countable
intersections, such that $\mu$ is inner regular with respect to
$\Cal K$.   Then there is a disintegration $\family{y}{Y}{\mu_y}$ of
$\mu$ over $\nu$ such that

(i) $\mu_y$ is a complete probability measure on $X$,
inner regular with respect to $\Cal K$ and measuring every member of
$\Cal K$, for every $y\in Y$;

(ii) setting $h_g(y)=\int g\,d\mu_y$ whenever
$g\in\eusm L^{\infty}(\mu)$ and $y\in Y$ are such that the integral is
defined, $h_g\in\eusm L^{\infty}(\nu)$ and
$T(g^{\ssbullet})=h_g^{\ssbullet}$ for every
$g\in\eusm L^{\infty}(\mu)$.

\proof{{\bf (a)} Completing $\nu$ does not change $\eusm L^{\infty}(\nu)$
or $L^{\infty}(\nu)$, nor does it change the families which are
disintegrations over
$\nu$;  so we may assume throughout that $\nu$ is complete.   It
therefore has a lifting $\theta:\frak B\to\Tau$, where $\frak B$ is the
measure algebra of $\nu$, which gives rise to a Riesz homomorphism $S$
from $L^{\infty}(\nu)\cong L^{\infty}(\frak B)$ to the space
$L^{\infty}(\Tau)$ of bounded $\Tau$-measurable real-valued functions on
$Y$ such that $(Sv)^{\ssbullet}=v$ for every $v\in L^{\infty}(\nu)$
(363I, 363F, 363H).

\medskip

{\bf (b)} For $y\in Y$ and $E\in\Sigma$, set
$\psi_yE=(ST(\chi E^{\ssbullet}))(y)$.   Because
$0\le T(\chi E^{\ssbullet})\le\chi Y^{\ssbullet}$ in
$L^{\infty}(\nu)$, $0\le\psi_yE\le 1$.   The maps

\Centerline{$E\mapsto\chi E\mapsto\chi E^{\ssbullet}
\mapsto T(\chi E^{\ssbullet})\mapsto ST(\chi E^{\ssbullet})$}

\noindent are all additive, so $\psi_y:\Sigma\to[0,1]$ is additive for
each $y\in Y$.   For fixed $E\in\Sigma$,

\Centerline{$\mu E=\int\chi E\,d\mu=\int(\chi E^{\ssbullet})
=\int T(\chi E^{\ssbullet})
=\int ST(\chi E^{\ssbullet})=\int\psi_yE\,\nu(dy)$.}

\medskip

{\bf (c)} Recall that $\mu$ is supposed to be inner
regular with respect to the countably compact class $\Cal K$.   By
413Sa, there is for every $y\in Y$ a complete measure $\mu'_y$ on
$X$ such that $\mu'_yX\le\psi_yX\le 1$, $\Cal K\subseteq\dom\mu'_y$, and
$\mu'_yK\ge\psi_yK$ for every $K\in\Cal K\cap\Sigma$.

\medskip

{\bf (d)} Now, for any fixed $E\in\Sigma$, $\mu'_yE$ is defined and
equal to $\psi_yE$ for almost every $y\in Y$.   \Prf\ Let
$\sequencen{K_n}$, $\sequencen{K'_n}$ be sequences in $\Cal K\cap\Sigma$
such that $K_n\subseteq E$ and $K'_n\subseteq X\setminus E$ for every
$n$, while $\mu E=\sup_{n\in\Bbb N}\mu K_n$ and
$\mu(X\setminus E)=\sup_{n\in\Bbb N}\mu K'_n$.   Set
$L=\bigcup_{n\in\Bbb N}K_n$,
$L'=\bigcap_{n\in\Bbb N}(X\setminus K'_n)$.   Then both $L$ and $L'$
belong to the domain of every $\mu'_y$, and

$$\eqalign{\sup_{n\in\Bbb N}\psi_yK_n
&\le\sup_{n\in\Bbb N}\mu'_yK_n
\le\mu'_yL
\le\mu'_yL'\cr
&\le\inf_{n\in\Bbb N}\mu'_y(X\setminus K'_n)
=\mu'_yX-\sup_{n\in\Bbb N}\mu'_yK'_n
\le 1-\sup_{n\in\Bbb N}\psi_yK'_n\cr}$$

\noindent for every $y$.   On the other hand,

$$\eqalign{\int(1-\sup_{n\in\Bbb N}\psi_yK'_n)\nu(dy)
&\le\nu Y-\sup_{n\in\Bbb N}\int\psi_y(K'_n)\nu(dy)
=\mu X-\sup_{n\in\Bbb N}\mu K'_n
=\mu E\cr
&=\sup_{n\in\Bbb N}\mu K_n
=\sup_{n\in\Bbb N}\int\psi_yK_n\nu(dy)
\le\int\sup_{n\in\Bbb N}\psi_yK_n\nu(dy).\cr}$$

\noindent So

\Centerline{$\sup_{n\in\Bbb N}\psi_yK_n=\mu'_yL=\mu'_yL'
=1-\sup_{n\in\Bbb N}\psi_yK'_n$}

\noindent for almost every $y$.   Because $L\subseteq E\subseteq L'$ and
$\mu'_y$ is complete, $E\in\dom\mu'_y$ and

\Centerline{$\mu'_yE=1-\sup_{n\in\Bbb N}\psi_yK'_n
\ge 1-\psi_y(X\setminus E)\ge\psi_yE$}

\noindent for almost every $y\in Y$.   Similarly,
$\mu'_y(X\setminus E)\ge\psi_y(X\setminus E)$ for almost every $y$.
But as

\Centerline{$\mu'_yE+\mu'_y(X\setminus E)=\mu'_yX\le\psi_yX\le 1$}

\noindent whenever the left-hand side is defined, we must have
$\mu'_yE=\psi_yE$ for almost every $y$, as claimed.\ \Qed

It follows at once that

\Centerline{$\int\mu'_yE\,\nu(dy)=\int\psi_yE\,\nu(dy)=\mu E$}

\noindent for every $E\in\Sigma$, and $\family{y}{Y}{\mu'_y}$ is a
disintegration of $\mu$ over $\nu$.

\medskip

{\bf (e)} At this point observe that

\Centerline{$\int\mu'_yX\,\nu(dy)
=\mu X=\int\chi X^{\ssbullet}=\int T(\chi X^{\ssbullet})
=\nu Y$,}

\noindent so $F_0=\{y:\mu'_yX<1\}$ is negligible.   Taking any $y_0\in
Y\setminus F_0$ and setting

$$\eqalign{\mu_y&=\mu'_{y_0}\text{ for }y\in F_0\cr
&=\mu'_y\text{ for }y\in Y\setminus F_0,\cr}$$

\noindent we find ourselves with a disintegration $\family{y}{Y}{\mu_y}$
of $\mu$ over $\nu$ with the same properties as $\family{y}{Y}{\mu'_y}$,
but now consisting entirely of probability measures.

\medskip

{\bf (f)} For $g\in\eusm L^{\infty}(\mu)$, set $h_g(y)=\int g\,d\mu_y$
whenever $y\in Y$ is such that the integral is defined.   Consider the
set $V$ of those $g\in\eusm L^{\infty}(\mu)$ such that
$h_g\in\eusm L^{\infty}(\nu)$ and $Tg^{\ssbullet}=h_g^{\ssbullet}$ in
$L^{\infty}(\nu)$.   If $E\in\Sigma$, then $h_{\chi E}(y)=\psi_yE$ for
almost every $y$, so

\Centerline{$h_{\chi E}^{\ssbullet}
=(ST(\chi E^{\ssbullet}))^{\ssbullet}
=T(\chi E^{\ssbullet})$;}

\noindent accordingly $\chi E\in V$.   It is easy to check that $V$ is
closed under addition and scalar multiplication, so it contains all
simple functions.   Next, if $\sequencen{g_n}$ is a non-decreasing
sequence of simple functions with limit $g\in\eusm L^{\infty}(\nu)$,
then $h_g=\sup_{n\in\Bbb N}h_{g_n}$ wherever the right-hand side is
defined.   Also $T$ is order-continuous, because it preserves integrals,
so

\Centerline{$Tg^{\ssbullet}=\sup_{n\in\Bbb N}Tg_n^{\ssbullet}
=\sup_{n\in\Bbb N}h_{g_n}^{\ssbullet}=h_g^{\ssbullet}$}

\noindent and $g\in V$.   Finally, if $g\in\eusm L^{\infty}(\mu)$ is
zero almost everywhere, there is a negligible $E\in\Sigma$ such that
$g(x)=0$ for every $x\in X\setminus E$;  $\mu_yE=0$ for almost every
$y$, so $h_g(y)=\int g\,d\mu_y=0$ for almost every $y$ and again
$g\in V$.   Putting these together, we see that
$V=\eusm L^{\infty}(\nu)$, as required by (ii) as stated above.
}%end of proof of 452H

\leader{452I}{Theorem}\cmmnt{ ({\smc Pachl 78})} Let $(X,\Sigma,\mu)$
be a non-empty countably compact
measure space, $(Y,\Tau,\nu)$ a $\sigma$-finite measure space, and
$f:X\to Y$ an \imp\ function.   Then there is a disintegration
$\family{y}{Y}{\mu_y}$ of $\mu$ over $\nu$, consistent with $f$, such
that $\mu_y$ is a complete probability measure on $X$
for every $y\in Y$.   Moreover,

(i) if $\Cal K$ is a countably compact class of subsets of $X$ such that
$\mu$ is inner regular with respect to $\Cal K$, then we can arrange
that $\Cal K\subseteq\dom\mu_y$ for every $y\in Y$;

(ii) if, in (i), $\Cal K$ is closed under finite unions and countable
intersections, then we can arrange that $\Cal K\subseteq\dom\mu_y$ and
$\mu_y$ is inner regular with respect to $\Cal K$ for every $y\in Y$.

\proof{{\bf (a)} Consider first the case in which $\nu$ and $\mu$ are
probability measures and we are provided with a class $\Cal K$ as in
(ii).   In this case, for each $u\in L^{\infty}(\mu)$,
$F\mapsto\int_{f^{-1}[F]}u$ is countably additive.   So we have an
operator $T:L^{\infty}(\mu)\to L^{\infty}(\nu)$ defined by saying that
$\int_FTu=\int_{f^{-1}[F]}u$ whenever $u\in L^{\infty}(\mu)$ and
$F\in\Tau$.   Of course $T$ is linear and positive and $\int Tu=\int u$
whenever $u\in L^{\infty}(\mu)$.

By 452H, there is a disintegration $\family{y}{Y}{\mu_y}$ of $\mu$ over
$\nu$ such that

\inset{($\alpha$) for every $y\in Y$, $\mu_yX=1$,
$\Cal K\subseteq\dom\mu_y$ and $\mu_y$ is inner regular with respect to
$\Cal K$;

($\beta$) $T(g^{\ssbullet})=h_g^{\ssbullet}$ whenever
$g\in\eusm L^{\infty}(\mu)$ and $h_g(y)=\int g\,d\mu_y$ when the
integral is defined.}

\noindent If now $F\in\Tau$, set $g=\chi f^{-1}[F]$ in ($\beta$);  then
$Tg^{\ssbullet}$ is defined by saying that

\Centerline{$\int_HTg^{\ssbullet}=\int_{f^{-1}[H]}g
=\mu f^{-1}[F\cap H]=\nu(F\cap H)$}

\noindent for every $H\in\Tau$, so that
$Tg^{\ssbullet}=\chi F^{\ssbullet}$ and we must have $\mu_yf^{-1}[F]=1$
for almost every $y\in F$.   Thus $\family{y}{Y}{\mu_y}$ is a consistent
distribution.

\medskip

{\bf (b)} The theorem is formulated in a way to make it quotable in parts
without committing oneself to a particular class $\Cal K$.   But if we
are given a class satisfying (i), we can extend it to one satisfying
(ii), by 413R;  and if we are told only that $\mu$ is countably
compact, we know from the definition that we shall be able to choose a
countably compact class satisfying (i).

\medskip

{\bf (c)} This proves the theorem on the assumption that $\mu$ and $\nu$
are probability measures.   If $\mu X=\nu Y=0$ then the result is
trivial, as we can take every $\mu_y$ to be the zero measure.
Otherwise, because
$\nu$ is $\sigma$-finite, there is a partition $\sequencen{Y_n}$ of $Y$
into measurable sets of finite measure.   Let $\sequencen{\gamma_n}$ be
a sequence of strictly positive real numbers such that
$\sum_{n=0}^{\infty}\gamma_n\nu Y_n=1$, and write

\Centerline{$\nuprime F=\sum_{n=0}^{\infty}\gamma_n\nu(F\cap Y_n)$ for
$F\in\Tau$,}

\Centerline{$\mu'E=\sum_{n=0}^{\infty}\gamma_n\mu(E\cap X_n)$ for
$E\in\Sigma$.}

\noindent It is easy to check ($\alpha$) that $\nuprime$ and $\mu'$ are
probability measures ($\beta$) that $f$ is \imp\ for $\mu'$ and
$\nuprime$ ($\gamma$) that if $\mu$ is inner regular with respect to
$\Cal K$ so is $\mu'$.   Note that
$\nuprime$ and $\nu$ have the same negligible sets.   By
(a)-(b), $\mu'$ has a disintegration $\family{y}{Y}{\mu_y}$ over
$\nuprime$ which is consistent with $f$, and (if appropriate) has the properties demanded in (i) or (ii).   Now, if $E\in\Sigma$,

$$\eqalignno{\mu E
&=\sum_{n=0}^{\infty}\gamma_n^{-1}\mu'(E\cap X_n)
=\sum_{n=0}^{\infty}\gamma_n^{-1}\int\mu_y(E\cap X_n)\nuprime(dy)\cr
&=\sum_{n=0}^{\infty}\gamma_n^{-1}\int_{Y_n}\mu_yE\,\nuprime(dy)\cr
\noalign{\noindent (because $\mu_yX=1$, $\mu_yX_n=(\chi Y_n)(y)$ for
$\nuprime$-almost every $y$, every $n$)}
&=\sum_{n=0}^{\infty}\int_{Y_n}\mu_yE\,\nu(dy)
=\sum_{n=0}^{\infty}\int\mu_y(E\cap X_n)\nu(dy)
=\int\mu_yE\,\nu(dy).\cr}$$

\noindent So $\family{y}{Y}{\mu_y}$ is a disintegration of $\mu$ over
$\nu$.   If $F\in\Tau$, then $\mu_yf^{-1}[F]=1$ for $\nuprime$-almost
every
$y$, that is, for $\nu$-almost every $y$, so $\family{y}{Y}{\mu_y}$ is
still consistent with $f$ with respect to the measure $\nu$.
}%end of proof of 452I

\cmmnt{
\leader{452J}{Remarks (a)} In the theorem above, I have
carefully avoided making any promises about the domains of the $\mu_y$
beyond that in (i).   If $\Sigma_0$ is the $\sigma$-algebra generated by
$\Cal K\cap\Sigma$, then whenever $E\in\Sigma$ there are $E'$,
$E''\in\Sigma_0$ such that $E'\subseteq E\subseteq E''$ and
$\mu(E''\setminus E')=0$.   (For $\mu$, like $\nu$, must be
$\sigma$-finite, so we can choose $E'$ to be a countable union of
members of $\Cal K\cap\Sigma$, and $E''$ to be the complement of such a
union.)   Thus we shall have a $\sigma$-algebra on which every $\mu_y$
is defined and which will be adequate to describe nearly everything
about $\mu$.   The example of Lebesgue measure on the square
(452E) shows that we cannot ordinarily expect the $\mu_y$ to be defined
on the whole of $\Sigma$ itself.   In many important cases, of course,
we can say more (452Xl).

\spheader 452Jb Necessarily (as remarked in the course of the
proof) $\mu_yX=1$ for almost every $y$.   In some
applications it seems right to change $\mu_y$ for a negligible set of
$y$'s so that every $\mu_y$ is a probability measure.   Of course this
cannot be done if $X=\emptyset\ne Y$, but this case is trivial (we
should have to have $\nu Y=0$).   In other cases, we can make sure that
any new $\mu_y$ is equal to some old one, so that a property required by
(i) or (ii) remains true of the new disintegration.   If we want to have
`$\mu_yf^{-1}[\{y\}]=\mu_yX=1$ for every $y\in Y$', strengthening `strongly consistent',
we shall of course have to begin by checking that $f$ is surjective.

\spheader 452Jc The question of whether `$\sigma$-finite' can
be weakened to `strictly localizable' in the hypotheses of 452I is
related to the Banach-Ulam problem (452Yb).   See also 452O.
}%end of comment

\leader{452K}{Example}\cmmnt{ The hypothesis `countably compact' in
452I is in fact essential (452Ye).   To see at least that it cannot be
omitted, we have the following elementary example.}
Set $Y=[0,1]$, and let $\nu$ be Lebesgue measure on $Y$, with domain
$\Tau$.   Let $X\subseteq[0,1]$ have outer measure $1$ and inner measure
$0$\cmmnt{ (134D, 419I)};  let $\mu$ be the subspace measure on $X$.
Set $f(x)=x$ for $x\in X$.   Then there is no disintegration
$\family{y}{Y}{\mu_y}$ of $\mu$ over $\nu$ which is consistent with $f$.

\prooflet{\Prf\Quer\ Suppose, if possible, that
$\family{y}{[0,1]}{\mu_y}$
is such a disintegration.   Then, in particular, the sets

\Centerline{$H_q=[0,q]\setminus\{y:X\cap
[0,q]\in\dom\mu_y,\,\mu_y(X\cap[0,q])=1\}$,}

\Centerline{$H'_q=[q,1]\setminus\{y:X\cap[q,1]\in\dom\mu_y,\,
\mu_y(X\cap[q,1])=1\}$}

\noindent are negligible for every $q\in[0,1]$.   Set
$G=[0,1]\setminus\bigcup_{q\in\Bbb Q\cap[0,1]}(H_q\cup H'_q)$, so that
$G$ is $\nu$-conegligible.   Then there must be some
$y\in G\setminus X$.   Now $\mu_y(X\cap[0,q'])=\mu_y(X\cap[q,1])=1$
whenever $q$,
$q'\in\Bbb Q$ and $0\le q<y<q'\le 1$, so that $\mu_y(X\cap\{y\})=1$.
But $X\cap\{y\}=\emptyset$.\ \Bang\Qed
}%end of prooflet

\leader{452L}{}\cmmnt{ The same ideas as in 452I can be used to prove
a result on the disintegration of measures on product spaces.   It will
help to have a definition.

\medskip

\noindent}{\bf Definition} Let $\familyiI{X_i}$ be a family of sets, and
$\lambda$ a measure on $X=\prod_{i\in I}X_i$.   For each $i\in I$ set
$\pi_i(x)=x(i)$ for $x\in X$.   Then the image measure
$\lambda\pi_i^{-1}$ is the {\bf marginal measure} of $\lambda$ on $X_i$.

\leader{452M}{}\cmmnt{ I return to the context of 452B-452D.

\medskip

\noindent}{\bf Theorem} Let $Y$ and $Z$ be sets and
$\Tau\subseteq\Cal PY$, $\Upsilon\subseteq\Cal PZ\,\,\sigma$-algebras.
Let $\mu$ be a
non-zero totally finite measure with domain $\Tau\tensorhat\Upsilon$,
and $\nu$ the marginal measure of $\mu$ on $Y$.   Suppose that the
marginal measure $\lambda$ of $\mu$ on $Z$ is inner regular with respect
to a
countably compact class $\Cal K\subseteq\Cal PZ$ which is closed under
finite unions and countable intersections.   Then there is a family
$\family{y}{Y}{\mu_y}$ of complete probability measures on $Z$, all
measuring every member of $\Cal K$ and inner regular with respect to
$\Cal K$, such that

\Centerline{$\mu E=\int\mu_yE[\{y\}]\nu(dy)$}

\noindent for every $E\in\Tau\tensorhat\Upsilon$, and

\Centerline{$\int fd\mu=\iint f(y,z)\mu_y(dz)\nu(dy)$}

\noindent whenever $f$ is a $[-\infty,\infty]$-valued function such that
$\int fd\mu$ is defined in $[-\infty,\infty]$.

\proof{{\bf (a)} To begin with, assume that $\mu$ is a probability
measure and that $\nu$ is complete.   Let $\frak B$ be the measure
algebra of $\nu$ and
$\theta:\frak B\to\Tau$ a lifting.   For $H\in\Upsilon$ and $F\in\Tau$
set $\nu_HF=\mu(F\times H)$;  then $\nu_H:\Tau\to[0,1]$ is countably
additive and
$\nu_HF\le\nu F$ for every $F\in\Tau$, so there is a $v_H\in L^1(\nu)$
such that $\int_Fv_H=\nu_HF$ for every $F\in\Tau$ and
$0\le v_H\le\chi 1$.   We can therefore think of $v_H$ as a member of
$L^{\infty}(\nu)\cong L^{\infty}(\frak B)$.   Let
$T:L^{\infty}(\frak B)\to L^{\infty}(\Tau)$ be the Riesz homomorphism
associated with $\theta$, and set $\psi_yH=(Tv_H)(y)$ for every
$y\in Y$.

Each $\psi_y:\Upsilon\to\coint{0,\infty}$ is finitely additive.   So we
have a complete measure $\mu_y$ on $Z$ such that $\mu_yZ\le\psi_yZ=1$,
$\Cal K\subseteq\dom\mu_y$, $\mu_y$ is inner regular with respect to
$\Cal K$ and $\mu_yK\ge\psi_yK$ for every $K\in\Cal K$ (413Sa, as before).

For $H\in\Upsilon$, $F\in\Tau$ we have

\Centerline{$\int_F\psi_yH\,\nu(dy)=\int_FTv_H
=\int_Fv_H=\nu_HF=\mu(F\times H)$.}

\noindent So

\Centerline{$
\underline{\intop}\mu_yK\cdot\chi F(y)\nu(dy)
\ge\int_F\psi_yK\,\nu(dy)=\mu(F\times K)$}

\noindent for every $K\in\Cal K$.   Now note that, for any
$H\in\Upsilon$
and $F\in\Tau$,

$$\eqalign{\mu(F\times H)
  -\sup_{K\in\Cal K,K\subseteq H}\mu(F\times K)
&=\inf_{K\in\Cal K,K\subseteq H}
  \mu(F\times(H\setminus K))\cr
&\le\inf_{K\in\Cal K,K\subseteq H}\lambda(H\setminus K)
=0\cr}$$

\noindent
because $\lambda$ is inner regular with respect to $\Cal K$ (and, like
$\mu$, is a probability measure).   So

$$\eqalign{\underline{\int}(\mu_y)_*H\cdot\chi F(y)\nu(dy)
&\ge\sup_{K\in\Cal K,K\subseteq H}
  \underline{\int}\mu_yK\cdot\chi F(y)\nu(dy)\cr
&\ge\sup_{K\in\Cal K,K\subseteq H}\mu(F\times K)
=\mu(F\times H).\cr}$$

\noindent In particular,

\Centerline{$\underline{\intop}(\mu_y)_*H\,\nu(dy)
\ge\mu(Y\times H)=\lambda H$,}

\noindent and similarly
$\underline{\int}(\mu_y)_*(Z\setminus H)\nu(dy)
\ge\lambda(Z\setminus H)$.

Taking $\nu$-integrable functions $g_1$, $g_2$ such that
$g_1(y)\le(\mu_y)_*H$ and $g_2(y)\le(\mu_y)_*(Z\setminus H)$ for almost
every $y$, $\int g_1d\nu=\underline{\int}(\mu_y)_*H\,\nu(dy)$
and $\int g_2d\nu=\underline{\int}(\mu_y)_*(Z\setminus H)\nu(dy)$
(133Ja), we must have

\Centerline{$g_1(y)+g_2(y)\le(\mu_y)_*H+(\mu_y)_*(Z\setminus H)
\le\mu_yZ\le 1$}

\noindent for almost every $y$, while $\int g_1+g_2\,d\nu\ge 1$;  so that,
for almost all $y$,

\Centerline{$g_1(y)+g_2(y)=(\mu_y)_*H+(\mu_y)_*(Z\setminus H)
=\mu_yZ=1$,}

\noindent and (because $\mu_y$ is complete) $\mu_yH$ is defined
and equal to $g_1(y)$ (413E).   It now follows that

\Centerline{$\int_F\mu_yH\,\nu(dy)
=\int_Fg_1(y)\nu(dy)
=\underline{\intop}(\mu_y)_*H\cdot\chi F(y)\nu(dy)
\ge\mu(F\times H)$}

\noindent for every $F\in\Tau$.   But since also

\Centerline{$\int_F\mu_y(Z\setminus H)\nu(dy)
\ge\mu(F\times(Z\setminus H))$,}

\Centerline{$\int_F\mu_yH+\mu_y(Z\setminus H)\,\nu(dy)
\le\nu F=\mu(F\times H)+\mu(F\times(Z\setminus H))$,}

\noindent we must actually have $\int_F\mu_yH\,\nu(dy)=\mu(F\times H)$.

All this is true whenever $F\in\Tau$ and $H\in\Upsilon$.   But now,
setting

\Centerline{$\Cal E
=\{E:E\in\Tau\tensorhat\Upsilon,\,
  \mu E=\int\mu_yE[\{y\}]\nu(dy)\}$,}

\noindent we see that $\Cal E$ is a Dynkin class and includes
$\Cal I=\{F\times H:F\in\Tau$, $H\in\Upsilon\}$, which is closed under
finite intersections;  so that the Monotone Class Theorem tells us that
$\Cal E$ includes the $\sigma$-algebra generated by $\Cal I$, and is the
whole of $\Tau\tensorhat\Upsilon$.

\medskip

{\bf (b)} The rest is just tidying up.
(i) The construction in (a) allows $\mu_yZ$ to be less than
$1$ for a
$\nu$-negligible set of $y$;  but of course all we have to do, if that
happens, is to amend $\mu_y$ arbitrarily on that set to any of the
`ordinary' values of $\mu_y$.
(ii) If the original measure $\nu$ is not complete, let $\hat\mu$ and
$\hat\nu$ be the completions of $\mu$ and $\nu$, and $\hat{\Tau}$ the
domain of $\hat\nu$.   The projection onto $Y$ is \imp\ for $\mu$ and
$\nu$, so is \imp\ for $\hat\mu$ and $\hat\nu$ (234Ba\formerly{2{}35Hc}),
and $\hat\mu$ measures every member of $\hat\Tau\tensorhat\Upsilon$;
set $\mu'=\hat\mu\restr\hat\Tau\tensorhat\Upsilon$.   Next, the marginal
measure of $\mu'$ on $Z$ is still $\lambda$ (since both must have domain
$\Upsilon$).   So we can apply (a) to $\mu'$ to get the result.   (iii)
If the original measure $\mu$ is not a probability measure, apply the
arguments so far to suitable scalar multiples of $\mu$ and $\nu$.

\medskip

{\bf (c)} Thus we have the formula

\Centerline{$\mu E=\int\mu_yE[\{y\}]\nu(dy)$}

\noindent for every $E\in\Tau\tensorhat\Upsilon$.   The second formula
announced follows as in the remark following 452F.
}%end of proof of 452M

\leader{452N}{Corollary} Let $Y$ and $Z$ be sets and
$\Tau\subseteq\Cal PY$, $\Upsilon\subseteq\Cal PZ\,\,\sigma$-algebras.
Let $\mu$ be a
probability measure with domain $\Tau\tensorhat\Upsilon$, and $\nu$ the
marginal measure of $\mu$ on $Y$.   Suppose that

\inset{{\it either} $\Upsilon$ is the Baire $\sigma$-algebra with
respect to a compact Hausdorff topology on $Z$

{\it or} $\Upsilon$ is the Borel $\sigma$-algebra with respect to an
analytic Hausdorff topology on $Z$

{\it or} $(Z,\Upsilon)$ is a standard Borel space.}

\noindent Then there is a family $\family{y}{Y}{\mu_y}$ of probability
measures on $Z$, all with domain $\Upsilon$, such that

\Centerline{$\mu E=\int\mu_yE[\{y\}]\nu(dy)$}

\noindent for every $E\in\Tau\tensorhat\Upsilon$, and

\Centerline{$\int fd\mu=\iint f(y,z)\mu_y(dz)\nu(dy)$}

\noindent whenever $f$ is a $[-\infty,\infty]$-valued function such that
$\int fd\mu$ is defined in $[-\infty,\infty]$.

\proof{ In each case, the marginal measure of $\mu$ on $Z$ is tight
(that is, inner regular with respect to the closed compact sets)
for a Hausdorff topology on
$Z$.   (Use 412D when $\Upsilon$ is the Baire $\sigma$-algebra on a
compact Hausdorff space and 433Ca when it is the Borel $\sigma$-algebra
on an analytic Hausdorff space;  when $(Z,\Upsilon)$ is a standard Borel
space, take any appropriate Polish topology on $Z$ and use 423Ba.)   So
452M tells us that we can achieve the formulae sought with
Radon probability measures $\mu_y$.   Since (in all three cases)
$\dom\mu_y$ will include $\Upsilon$ for every $y$, we can get the result
as stated by replacing each $\mu_y$ by $\mu_y\restr\Upsilon$.
}%end of proof of 452N

\leader{452O}{Proposition} Let $(X,\frak T,\Sigma,\mu)$ be a Radon
measure space, $(Y,\Tau,\nu)$ a strictly localizable measure space, and
$f:X\to Y$ an \imp\ function.   Then there is a disintegration
$\family{y}{Y}{\mu_y}$ of $\mu$ over $\nu$, consistent with $f$, such
that every $\mu_y$ is a Radon measure on $X$.

\proof{{\bf (a)} Let $\familyiI{Y_i}$ be a decomposition of $Y$.   For
each $i\in I$, let $\nu_i$ be the subspace measure on $Y_i$ and
$\lambda_i$ the subspace measure on $X_i=f^{-1}[Y_i]$.   Then
$f_i=f\restr X_i$ is \imp\ for $\lambda_i$ and $\nu_i$.   Let $\Cal K_i$
be the family of compact subsets of $X_i$;  of course $\Cal K_i$ is a
(countably) compact class and $\lambda_i$ is inner regular with respect
to $\Cal K_i$ (412Oa).   By 452I, we can choose, for each
$i\in I$, a disintegration $\family{y}{Y_i}{\tilde\mu_y}$ of $\lambda_i$
over $\nu_i$, consistent with $f\restr X_i$, such that
$\tilde\mu_y$ measures every compact subset of  $X_i$ and is
inner regular with respect to $\Cal K_i$ for every $y\in Y_i$.
Adjusting any which are not probability
measures, and completing them if necessary,
we can suppose that every $\tilde\mu_y$ is a
complete probability measure.   By
412Ja, $\tilde\mu_y$ measures every relatively closed subset of $X_i$
for every $y\in Y_i$.

For $i\in I$ and $y\in Y_i$, set

\Centerline{$\mu_yE=\tilde\mu_y(E\cap X_i)$}

\noindent whenever $E\subseteq X$ and $E\cap X_i$ is measured by
$\tilde\mu_y$.   Then $\mu_y$ is a complete totally finite measure on
$X$;  it is inner regular with respect to $\Cal K_i$ and measures every
closed subset of $X$.   It follows at once that it is tight and measures
every Borel set, that is, is a Radon measure on $X$.

\medskip

{\bf (b)} Now $\mu E=\int\mu_yE\,\nu(dy)$ for every $E\in\Sigma$.  \Prf\
$\bigcup_{i\in J}E\cap X_i=E\cap f^{-1}[\bigcup_{i\in J}Y_i]$  belongs
to $\Sigma$ for every
$J\subseteq I$.   By 451Q, $\mu E=\sum_{i\in I}\mu(E\cap X_i)$.  For
$i\in I$, we have
$\int_{Y_i}\tilde\mu_y(E\cap X_i)\nu_i(dy)=\mu(E\cap X_i)$.   So

$$\eqalign{\mu E
&=\sum_{i\in I}\mu(E\cap X_i)
=\sum_{i\in I}\int_{Y_i}\tilde\mu_y(E\cap X_i)\nu_i(dy)\cr
&=\sum_{i\in I}\int_{Y_i}\mu_yE\,\nu(dy)
=\int\mu_yE\,\nu(dy)\cr}$$

\noindent by 214N.\ \Qed

Thus $\family{y}{Y}{\mu_y}$ is a disintegration of $\mu$ over $\nu$.

\medskip

{\bf (c)} Finally, if $F\in\Tau$ and $i\in I$, then

$$\eqalign{Y_i\cap F\setminus&\{y:\mu_yf^{-1}[F]
  \text{ is defined and equal to }1\}\cr
&=(F\cap Y_i)\setminus\{y:y\in Y_i,\,
  \tilde\mu_yf^{-1}[F\cap Y_i]=1\}\cr}$$

\noindent is negligible for every $i$, so $\mu_yf^{-1}[F]=1$ for almost
every $y$.   Thus $\family{y}{Y}{\mu_y}$ is consistent with $f$.
}%end of proof of 452O

\leader{452P}{Corollary}\cmmnt{ (cf.\ {\smc Blackwell 56})} Let
$(X,\frak T,\Sigma,\mu)$ be a Radon measure space,
$(Y,\frak S,\Tau,\nu)$ an analytic Radon measure space and $f:X\to Y$ an
\imp\ function.   Then there is a disintegration $\family{y}{Y}{\mu_y}$
of $\mu$ over $\nu$, strongly consistent with $f$, such that every
$\mu_y$ is a Radon measure on $X$.

\proof{ By 433B, $\nu$ is countably separated;  now put 452O and 452Gc
together.
}%end of proof of 452P

\leader{452Q}{Disintegrations and conditional
\dvrocolon{expectations}}\cmmnt{ Fubini's theorem provides a
relatively
concrete description of the conditional expectation of a function on a
product of probability spaces with respect to the $\sigma$-algebra
defined by one of the factors, by means of the formula
$g(x,y)=\int f(x,z)dz$ (253H).   This generalizes straightforwardly to
measures with disintegrations, as follows.

\wheader{452Q}{6}{2}{2}{72pt}\noindent}{\bf Proposition} Let $(X,\Sigma,\mu)$ and $(Y,\Tau,\nu)$ be
probability spaces and $f:X\to Y$ an \imp\ function.   Suppose that
$\family{y}{Y}{\mu_y}$ is a disintegration of $\mu$ over $\nu$ which is
consistent with $f$, and that $g$ is a $\mu$-integrable real-valued
function.

(a) Setting $h_0(y)=\int g\,d\mu_y$ whenever $y\in Y$ and the integral
is defined in
$\Bbb R$, $h_0$ is a Radon-Nikod\'ym derivative of the functional
$F\mapsto\int_{f^{-1}[F]}g\,d\mu:\Tau\to\Bbb R$.

(b) Now suppose that $\nu$ is complete.   Setting
$h_1(x)=\int g\,d\mu_{f(x)}$ whenever $x\in X$ and the integral is
defined in $\Bbb R$, then
$h_1$ is a conditional expectation of $g$ on the $\sigma$-algebra
$\Sigma_0=\{f^{-1}[F]:F\in\Tau\}$.

\proof{{\bf (a)} If $F\in\Tau$, then $f^{-1}[F]$ is $\mu_y$-conegligible
for almost every $y\in F$, and $\mu_y$-negligible for almost every
$y\in Y\setminus F$, so
$\int g\times\chi f^{-1}[F]d\mu_y=h_0(y)\times\chi F(y)$ for almost
every $y$, and

\Centerline{$\int_Fh_0d\nu
=\iint g\times\chi f^{-1}[F]d\mu_y\nu(dy)
=\int_{f^{-1}[F]}g\,d\mu$}

\noindent (452F).   As $F$ is arbitrary, we have the result.

\medskip

{\bf (b)} Of course $\Sigma_0$ is a $\sigma$-algebra (111Xd), and it is
included in $\Sigma$ because $f$ is \imp.   By 452F,
$Y_0=\{y:g$ is $\mu_y$-integrable$\}$ is conegligible, so
$\dom h_1=f^{-1}[Y_0]$ is conegligible.   If $\alpha\in\Bbb R$, then

\Centerline{$F=\{y:y\in Y_0$, $\int g\,d\mu_y\ge\alpha\}$}

\noindent belongs to $\Tau$
because $y\mapsto\int g\,d\mu_y$ is $\nu$-virtually measurable
and $\nu$ is complete.   So

\Centerline{$\{x:x\in\dom h_1$, $h_1(x)\ge\alpha\}=f^{-1}[F]$}

\noindent belongs to $\Sigma_0$, and $h_1$ is $\Sigma_0$-measurable.
If $F\in\Tau$, then

$$\eqalignno{\int_{f^{-1}[F]}h_1\,d\mu
&=\int_{f^{-1}[F]}\int g\,d\mu_{f(x)}\mu(dx)
=\int_F\int g\,d\mu_y\nu(dy)\cr
\displaycause{235G\formerly{2{}35I}}
&=\int_Fh_0d\nu
=\int_{f^{-1}[F]}g\,d\mu\cr}$$

\noindent as in (a).
As $F$ is arbitrary, $h_1$ is a conditional expectation of $g$
on $\Sigma_0$, as claimed.
}%end of proof of 452Q

\leader{*452R}{}\cmmnt{ I take the opportunity to interpolate an
interesting result about countably compact measures.   It demonstrates
the power of 452I to work in unexpected ways.

\medskip

\noindent}{\bf Theorem}\cmmnt{ ({\smc Pachl 79})} Let $(X,\Sigma,\mu)$
be a countably compact
measure space, $(Y,\Tau,\nu)$ a strictly localizable measure space, and
$f:X\to Y$ an \imp\ function.   Then $\nu$ is countably compact.

\proof{{\bf (a)} For most of the proof (down to the end of (b) below) I
suppose that $\mu$ and $\nu$ are totally finite.

Let $Z$ be the Stone space of the Boolean algebra $\Tau$.   (I am {\it
not} using the measure algebra here!)   For $F\in\Tau$, let $F^*$ be the
corresponding open-and-closed subset of $Z$.   For each $y\in Y$, the
map $F\mapsto\chi F(y)$ is a Boolean homomorphism from $\Tau$ to
$\{0,1\}$, so belongs to $Z$;  define $g:Y\to Z$ by saying that
$g(y)(F)=\chi F(y)$ for $y\in Y$, $F\in\Tau$, that is,
$g^{-1}[F^*]=F$ for every $F\in\Tau$.   Let $\Cal Z$ be the family of
zero sets in $Z$, and $\Lambda$ the Baire $\sigma$-algebra of $Z$.

The set

\Centerline{$\{W:W\subseteq Z,\,g^{-1}[W]\in\Tau\}$}

\noindent is a $\sigma$-algebra of subsets of $Z$ containing all the
open-and-closed sets, so contains every zero set (4A3Oe) and
includes $\Lambda$.   Set $\lambda W=\nu g^{-1}[W]$ for $W\in\Lambda$.
Then $\lambda$ is a Baire measure on $Z$, so is inner regular with
respect to $\Cal Z$ (412D).

Set $h=gf:X\to Z$.   Then $h$ is a composition of \imp\ functions, so is
\imp.   By 452I, there is a disintegration $\family{z}{Z}{\mu_z}$ of
$\mu$ over $\lambda$ which is consistent with $h$.

\medskip

{\bf (b)} Let $\Cal K\subseteq\Cal PY$ be the family of sets

\Centerline{$\{g^{-1}[V]:V\in\Cal Z,\,\mu_zh^{-1}[V]=\mu_z X=1$ for
every $z\in V\}$.}

\medskip

\quad{\bf (i)} $\Cal K$ is a countably compact class of sets.   \Prf\
Let $\sequencen{K_n}$ be a sequence in $\Cal K$ such that $\bigcap_{i\le
n}K_i\ne\emptyset$ for each $n\in\Bbb N$.   For each $n\in\Bbb N$, let
$V_n\in\Cal Z$ be such that $K_n=g^{-1}[V_n]$ and
$\mu_zh^{-1}V_n=\mu_zX=1$ for every $z\in V_n$.   Then

\Centerline{$g^{-1}[\bigcap_{i\le n}V_i]=\bigcap_{i\le n}K_i
\ne\emptyset$}

\noindent for every $n\in\Bbb N$, so $\{V_n:n\in\Bbb N\}$ has the finite
intersection property and (because $Z$ is compact) there is a
$z\in\bigcap_{n\in\Bbb N}V_n$.   Now

\Centerline{$\mu_zh^{-1}[V_n]=\mu_zX=1$}

\noindent for every $n\in\Bbb N$, so

\Centerline{$\emptyset\ne\bigcap_{n\in\Bbb N}h^{-1}[V_n]
=f^{-1}[\bigcap_{n\in\Bbb N}K_n]$.}

\noindent Thus $\bigcap_{n\in\Bbb N}K_n$ is non-empty.   As
$\sequencen{K_n}$ is arbitrary, $\Cal K$ is a countably compact
class.\ \Qed

\medskip

\quad{\bf (ii)} $\nu$ is inner regular with respect to $\Cal K$.   \Prf\
Suppose that $F\in\Tau$ and $\gamma<\nu F$.   Choose a sequence
$\sequencen{V_n}$ in $\Cal Z$ as follows.   Start with $V_0=F^*$, so
that

\Centerline{$\lambda V_0=\nu g^{-1}[V_0]=\nu F>\gamma$.}

\noindent  Given that $V_n\in\Cal Z$ and $\lambda V_n>\gamma$, then we
know that $\mu_zh^{-1}[V_n]=\mu_zX=1$ for $\lambda$-almost every
$z\in V_n$;
because $\lambda$ is inner regular with respect to $\Cal Z$, there is a
$V_{n+1}\in\Cal Z$ such that $V_{n+1}\subseteq V_n$, $\lambda
V_{n+1}>\gamma$ and $\mu_zh^{-1}[V_n]=\mu_zX=1$ for every $z\in
V_{n+1}$.   Continue.

At the end of the induction, set $V=\bigcap_{n\in\Bbb N}V_n$.   Then
$V\in\Cal Z$.   If $z\in V$, then

\Centerline{$\mu_zh^{-1}[V]
=\lim_{n\to\infty}\mu_zh^{-1}[V_n]=1=\mu_zX$,}

\noindent so $g^{-1}[V]\in\Cal K$.   Because $V\subseteq V_0=F^*$,
$g^{-1}[V]\subseteq F$, and

\Centerline{$\nu g^{-1}[V]
=\lambda V
=\lim_{n\to\infty}\lambda V_n
\ge\gamma$.}

\noindent As $F$ and $\gamma$ are arbitrary, $\nu$ is inner regular with
respect to $\Cal K$.\ \Qed

Thus $\Cal K$ witnesses that $\nu$ is countably compact.

\medskip

{\bf (c)} For the general case, let $\familyiI{Y_i}$ be a decomposition
of $Y$.   For each $i\in I$, set $X_i=f^{-1}[Y_i]$;  let $\mu_i$ be the
subspace measure on $X_i$ and $\nu_i$ the subspace measure on $Y_i$.
Then $\mu_i$ is countably compact (451Db) and $f\restr X_i:X_i\to Y_i$
is \imp\ for $\mu_i$ and $\nu_i$, so $\nu_i$ is countably compact, by
(a)-(b) above.   Let $\Cal K_i\subseteq\Cal PY_i$ be a countably compact
class such that $\nu_i$ is inner regular with respect to $\Cal K_i$.
Then $\Cal K=\bigcup_{i\in I}\Cal K_i$ is a countably compact class
(because any sequence in $\Cal K$ with the finite intersection property
must lie within a single $\Cal K_i$).   By 413R, there is a
countably compact class $\Cal K^*\supseteq\Cal K$ which is closed under
finite unions;  by 412Aa, $\nu$ is inner regular with respect to
$\Cal K^*$, so is countably compact.   This completes the proof.
}%end of proof of 452R

\leader{*452S}{Corollary}\cmmnt{ ({\smc Pachl 78})} If
$(X,\Sigma,\mu)$ is a countably compact totally finite measure space,
and $\Tau$ is any $\sigma$-subalgebra of $\Sigma$, then $\mu\restrp\Tau$
is countably compact.

\leader{452T}{}\cmmnt{ In 452E, I remarked in passing that Fubini's theorem
on a product space $X=Y\times Z$ can be thought of as giving us a
disintegration of the product measure on $X$ over the factor measure on
$Y$.   There are other contexts in
which we find that a canonical disintegration is provided for a structure
$(X,\mu,Y,\nu)$ without calling
on the Lifting Theorem.   Here I will describe
an important case arising naturally in the theory of group actions.

\medskip

\noindent}{\bf Theorem} Let $X$ be a locally compact Hausdorff space, $G$ a
compact Hausdorff topological group and $\action$ a continuous action of
$G$ on $X$.   Suppose that $\mu$ is a $G$-invariant
Radon probability measure on $X$.
For $x\in X$, write $f(x)$ for the corresponding
orbit $\{a\action x:a\in G\}$ of the
action.   Let $Y=f[X]$ be the set of orbits, with the topology
$\{W:W\subseteq Y$, $f^{-1}[W]$ is open in $X\}$.
Write $\nu$ for the image measure $\mu f^{-1}$ on $Y$.

(a) $Y$ is locally compact and
Hausdorff, and $\nu$ is a Radon probability measure.

(b) For each $\pmb{y}\in Y$, there is a unique $G$-invariant Radon
probability $\mu_{\pmb{y}}$ on $X$ such that $\mu_{\pmb{y}}(\pmb{y})=1$.

(c) $\family{\pmb{y}}{Y}{\mu_{\pmb{y}}}$ is a
disintegration of $\mu$ over $\nu$, strongly consistent with $f$.

\proof{{\bf (a)} By 4A5Ja, $Y$ is locally compact and Hausdorff, and $f$ is
an open map.   By 418I, $\nu$ is a Radon measure.

\medskip

{\bf (b)} Let $\lambda$ be the unique Haar probability measure on $G$
(442Id).   By 443Ub-443Ud, applied to the action
$\action\restr G\times\pmb{y}$ of $G$ on $\pmb{y}$, we have a unique
$G$-invariant Radon probability measure $\mu'_{\pmb{y}}$ on $\pmb{y}$
defined by saying that $\mu'_{\pmb{y}}E=\lambda\{g:g\action x\in E\}$ for
every $x\in\pmb{y}$ and Borel set $E\subseteq\pmb{y}$.
Now $\mu_{\pmb{y}}$ must be the unique extension of
$\mu'_{\pmb{y}}$ to $X$.   Of course we still have
$\mu_{\pmb{y}}E=\lambda\{g:g\action x\in E\}$ for
every $x\in\pmb{y}$ and Borel set $E\subseteq X$.

\medskip

{\bf (c)(i)} Let $V\subseteq X$ be an open set, and set
$h_V(\pmb{y})=\mu_{\pmb{y}}V$ for $\pmb{y}\in Y$.   Then $h_V$ is
lower semi-continuous.   \Prf\ Suppose that $\pmb{y}\in Y$ and
$\alpha\in\Bbb R$ are such that $h_V(\pmb{y})>\alpha$.   Then there is a
compact set $K\subseteq V$ such that $\mu_{\pmb{y}}K>\alpha$.
Fix $x\in\pmb{y}$, and set $L=\{g:g\action x\in K\}$, so that $L$ is a
compact subset of $G$ and $\lambda L>\alpha$.   The set
$\{(g,x'):g\in L$, $x'\in X$, $g\action x'\notin V\}$ is closed in
$L\times X$, so its projection
$\{x':\,\Exists g\in L$, $g\action x'\notin V\}$ is closed
(4A2Gm) and
$U=\{x':g\action x'\in V$ for every $g\in L\}$ is open in $X$.   Now
$f[U]$ is open in $Y$, because $f$ is an open map.   Of course
$x\in U$ and $\pmb{y}\in f[U]$.   But if
$\pmb{y}'\in f[U]$, there is an $x'\in U$ such that
$f(x')=\pmb{y}'$, and now

\Centerline{$h_V(\pmb{y}')=\mu_{\pmb{y}'}V
=\lambda\{g:g\action x'\in V\}\ge\lambda L>\alpha$.}

\noindent As $\pmb{y}$ and $\alpha$ are arbitrary, $h_V$ is lower
semi-continuous.\ \Qed

\medskip

\quad{\bf (ii)} In particular, $h_V$ is Borel measurable;  because $f$ is
\imp\ for $\mu$ and $\nu$,

$$\eqalignno{\int h_V\,d\nu
&=\int h_V(f(x))\mu(dx)\cr
\displaycause{235G again}
&=\int\lambda\{g:g\action x\in V\}\mu(dx)
=\int\mu\{x:g\action x\in V\}\lambda(dg)\cr
\displaycause{by 417H,
because $\mu$ and $\lambda$ are totally finite Radon measures
and $\{(g,x):g\action x\in V\}$ is an open set in $G\times X$}
&=\int\mu(g^{-1}\action V)\lambda(dg)
=\int\mu V\lambda(dg)\cr
\displaycause{because $\mu$ is $G$-invariant}
&=\mu V.\cr}$$

By the Monotone Class Theorem, as usual, it follows that
$\int\mu_{\pmb{y}}E\,\nu(d\pmb{y})=\mu E$ for every Borel set
$E\subseteq X$
(apply 136C to $\mu$ and $E\mapsto\int\mu_{\pmb{y}}E\,\nu(d\pmb{y})$),
and therefore (because every $\mu_{\pmb{y}}$ is complete and $\mu$ is the
completion of a Borel measure) for every $E\in\dom\mu$.   %452Xg
So $\family{\pmb{y}}{Y}{\mu_{\pmb{y}}}$ is a
disintegration of $\mu$ over $\nu$.   Since

\Centerline{$\mu_{\pmb{y}}f^{-1}[\{\pmb{y}\}]
=\mu_{\pmb{y}}(\pmb{y})=1$}

\noindent for every $\pmb{y}\in Y$, the disintegration
is strongly consistent with $f$.
}%end of proof of 452T

\exercises{
\leader{452X}{Basic exercises (a)}
%\spheader 452Xa
Let $Y$ be a first-countable topological space, $\nu$ a topological
probability measure on $Y$, $Z$ a topological space, and
$\family{y}{Y}{\mu_y}$ a family of topological probability measures on
$Z$ such that $y\mapsto\mu_yV$ is lower semi-continuous for every open
set $V\subseteq Z$.
Show that there is a Borel probability measure $\mu$ on $Y\times Z$ such
that $\mu E=\int\mu_yE[\{y\}]\nu(dy)$ for every Borel set
$E\subseteq Y\times Z$.   \Hint{434R.}
%452C

\spheader 452Xb Let $(Y,\Tau,\nu)$ be a probability space, $Z$ a
topological space and $P$ the set of topological probability measures on
$Z$ with its narrow
topology (437Jd).   Let $y\mapsto\mu_y:Y\to P$ be a function which is
measurable in the sense of 411L.   Show that, writing $\Cal B(Z)$ for
the Borel $\sigma$-algebra of $Z$,
we have a probability measure $\mu$ defined on $\Tau\tensorhat\Cal B(Z)$
such that $\mu E=\int\mu_yE[\{y\}]\nu(dy)$ for every
$E\in\Tau\tensorhat\Cal B(Z)$.
%452Xa 452C

\spheader 452Xc Let $(Y,\Tau,\nu)$ be a probability space, $Z$ a
topological space and $P_{\CalBa}$ the set of Baire probability measures
on $Z$ with its vague
topology (437Jc).   Let $y\mapsto\mu_y:Y\to P_{\CalBa}$ be a measurable
function.
Show that, writing $\CalBa(Z)$ for the Baire $\sigma$-algebra of $Z$,
we have a probability measure $\mu$ defined on $\Tau\tensorhat\CalBa(Z)$
such that $\mu E=\int\mu_yE[\{y\}]\nu(dy)$ for every
$E\in\Tau\tensorhat\CalBa(Z)$.
%452Xa 452C 452Xb

\spheader 452Xd
Let $(Y,\frak S,\Tau,\nu)$ be a Radon probability space, $(X,\frak T)$ a
topological space, and $\family{y}{Y}{\mu_y}$
a family of Radon probability measures on $X$.
Suppose that (i) there is a base $\Cal U$ for $\frak T$, closed under
finite unions, such
that $y\mapsto\mu_yU$ is lower semi-continuous for every $U\in\Cal U$
(ii) $\nu$ is inner regular with respect to the family
$\{K:K\subseteq Y$, $\{\mu_y:y\in K\}$ is uniformly tight$\}$.
Show that we have a Radon probability measure $\tilde\mu$ on $X$
such that $\tilde\mu E=\int\mu_yE\nu(dy)$ whenever $\tilde\mu$
measures $E$.
%452D

\spheader 452Xe Let $(Y,\frak S,\Tau,\nu)$ be a Radon probability space,
$(Z,\frak U)$ a Prokhorov Hausdorff space (437U), and $P$ the space of
Radon probability measures
on $Z$ with its narrow topology.   Suppose that $y\mapsto\mu_y:Y\to P$
is almost continuous.
Show that we have a Radon probability measure $\tilde\mu$ on $Y\times Z$
such that $\tilde\mu E=\int\mu_yE[\{y\}]\nu(dy)$ whenever $\tilde\mu$
measures $E$.
%452D

\spheader 452Xf
Let $(X,\Tau,\nu)$ be a measure space, and $\mu$ an indefinite-integral
measure over $\nu$ (234J\formerly{2{}34B}).
Show that there is a disintegration
$\family{x}{X}{\mu_x}$ of $\mu$ over $\nu$ such that $\mu_x\{x\}=\mu_xX$
for every $x\in X$.
%452E

\sqheader 452Xg Let $(X,\Sigma,\mu)$ and $(Y,\Tau,\nu)$ be measure
spaces and $\family{y}{Y}{\mu_y}$ a disintegration of $\mu$ over $\nu$.
Show that $\family{y}{Y}{\hat\mu_y}$ is a disintegration of $\hat\mu$
over $\nu$, where $\hat\mu_y$ and $\hat\mu$ are the completions of
$\mu_y$ and $\mu$ respectively.
%452E proofs of 452T, 455E

\sqheader 452Xh Let $(X,\Sigma,\mu)$ and $(Y,\Tau,\nu)$ be measure
spaces, and $\nuprime$ an indefinite-integral measure over $\nu$,
defined
from a $\nu$-virtually measurable function $g:Y\to\coint{0,\infty}$.
Suppose that
$\family{y}{Y}{\mu_y}$ is a disintegration of $\mu$ over $\nuprime$.
Show
that $\family{y}{Y}{g(y)\mu_y}$ is a disintegration of $\mu$ over $\nu$.
%452E

\spheader 452Xi Let $(Y,\Tau,\nu)$ be a probability space, $X$ a set
and $\family{y}{Y}{\mu_y}$ a family of probability measures on $X$.
Set $\theta A=\overline{\int}\mu^*_y(A)\,\nu(dy)$ for every
$A\subseteq X$.   (i) Show that $\theta$ is an outer measure on $X$.
(ii) Let $\mu$ be the measure on $X$ defined from $\theta$ by
\Caratheodory's construction.   Show that $\family{y}{Y}{\mu_y}$ is a
disintegration of $\mu$ over $\nu$.   (iii) Suppose that $X=[0,1]^2$,
$\nu$ is Lebesgue measure on $[0,1]=Y$ and $\mu_yE=\nu\{x:(x,y)\in E\}$
whenever this is defined.   Show that, for any $E$ measured by $\mu$,
$\mu_yE\in\{0,1\}$ for $\nu$-almost every $y$.
%452E

\spheader 452Xj Explore connexions between 452F and the formula
$\int fd\mu=\iint fd\nu_z\lambda(dz)$ of 443Qe.
%452F

\spheader 452Xk Let $(X,\Sigma,\mu)$ be a countably compact
$\sigma$-finite measure space, $(Y,\Tau,\nu)$ a $\sigma$-finite measure
space, and $f:X\to Y$ a $(\Sigma,\Tau)$-measurable function such that
$f^{-1}[F]$ is $\mu$-negligible whenever $F\subseteq Y$ is
$\nu$-negligible.   Show that there is a disintegration
$\family{y}{Y}{\mu_y}$ of $\mu$ over $\nu$ such that, for each
$F\in\Tau$, $\mu_y(X\setminus f^{-1}[F])=0$ for almost every $y\in F$.
\Hint{Reduce to the case in which $\mu$ is totally finite, and
disintegrate $\mu$ over $\nuprime=(\mu f^{-1})\restrp\Tau$.}
%452I, 452Xh

\sqheader 452Xl Let $(X,\Sigma,\mu)$ be a non-empty countably compact
measure space such that $\Sigma$ is countably generated (as
$\sigma$-algebra), $(Y,\Tau,\nu)$ a $\sigma$-finite measure space,
and $f:X\to Y$ an \imp\ function.   (i) Show that there is a
disintegration
$\family{y}{Y}{\mu_y}$ of $\mu$ over $\nu$, consistent with $f$, such
that every
$\mu_y$ is a probability measure with domain $\Sigma$.   (ii) Show
that if $\family{y}{Y}{\mu'_y}$ is any other disintegration of $\mu$
over $\nu$ which is consistent with $f$, then $\mu_y=\mu'_y\restr\Sigma$
for almost every $y$.
%452J

\spheader 452Xm Let $(X,\Sigma)$ be a non-empty standard Borel space,
$\mu$ a measure with domain $\Sigma$, $(Y,\Tau,\nu)$ a $\sigma$-finite
measure space,
and $f:X\to Y$ an \imp\ function.   (i) Show that there is a
disintegration
$\family{y}{Y}{\mu_y}$ of $\mu$ over $\nu$, consistent with $f$, such
that every
$\mu_y$ is a probability measure with domain $\Sigma$.   (ii) Show
that if $\family{y}{Y}{\mu'_y}$ is any other disintegration of $\mu$
over $\nu$ which is consistent with $f$, then $\mu_y=\mu'_y\restr\Sigma$
for almost every $y$.
%452Xl 452J omittable

\spheader 452Xn Let $(X,\Sigma,\mu)$ be a totally finite countably
compact measure space and $\Tau\subseteq\Sigma$ a countably-generated
$\sigma$-algebra;  set $\nu=\mu\restrp\Tau$.   Show that there is a
disintegration $\family{x}{X}{\mu_x}$ of $\mu$ over $\nu$ such that
$\mu_xH_x=\mu_xX=1$ for every $x\in X$, where
$H_x=\bigcap\{F:x\in F\in\Tau\}$ for every $x$.   \Hint{apply 452I
with $Y=\{H_x:x\in X\}$.}
%452J

\spheader 452Xo Show that 452I can be deduced from 452M.   \Hint{start
with the case $\nu Y=1$;  set $\lambda W=\mu\{x:(x,f(x))\in W\}$ for
$W\in\Sigma\tensorhat\Tau$.}
%452M

\spheader 452Xp Show that, in 452M, we shall have
$\hat\mu E=\int\mu_y E[\{y\}]\nu(dy)$ whenever the completion
$\hat\mu$ of $\mu$ measures $E$.
%452M

\sqheader 452Xq Let $\Tau$ be the Borel $\sigma$-algebra of $[0,1]$,
$\nu$ the restriction of Lebesgue measure to $\Tau$, $Z\subseteq[0,1]$ a
set with inner measure
$0$ and outer measure $1$, and $\Upsilon$ the Borel $\sigma$-algebra of
$Z$.   Show that there is a probability measure $\mu$ on $[0,1]\times Z$
defined by setting
$\mu E=\nu^*\{y:(y,y)\in E\}$ for $E\in\Tau\tensorhat\Upsilon$.   Show
that there is no disintegration of $\mu$ over $\nu$ which is consistent
with the projection
$(y,z)\mapsto y$.
%452M

\sqheader 452Xr Let $(X,\Sigma,\mu)$ be a complete totally finite
countably compact measure space and $\Tau$ a
$\sigma$-subalgebra of $\Sigma$ containing all negligible sets.   Show
that there is a
family $\family{x}{X}{\mu_x}$ of probability measures on $X$ such that
(i) $x\mapsto\mu_xE$ is $\Tau$-measurable and
$\int\mu_xE\,\mu(dx)=\mu E$ for every $E\in\Sigma$ (ii) if $F\in\Tau$,
then $\mu_xF=1$ for almost
every $x\in F$.   Show that if $g$ is any $\mu$-integrable real-valued
function, then $g$ is $\mu_x$-integrable for almost every $x$, and
$x\mapsto\int g\,d\mu_x$ is a conditional expectation of $g$ on $\Tau$.
%452Q

\spheader 452Xs\dvAnew{2010}
Let $(X_0,\Sigma_0,\mu_0)$ and $(X_1,\Sigma_1,\mu_1)$ be
$\sigma$-finite measure spaces.
For each $i$, let $(Y_i,\Tau_i,\nu_i)$ be a measure space and
$\family{y}{Y_i}{\mu^{(i)}_{y}}$ a disintegration of $\mu_i$ over $\nu_i$.
Show that
$\family{(y_0,y_1)}{Y_0\times Y_1}{\mu^{(0)}_{y_0}\times\mu^{(1)}_{y_1}}$
is a disintegration of $\mu_0\times\mu_1$ over $\nu_0\times\nu_1$, where
each product here is a c.l.d.\ product measure.
%452E out of order query

\spheader 452Xt\dvAnew{2011}
In 452M, suppose that $Z$ is a metrizable space and $\Cal K$ is the family
of compact subsets of $Z$, and let $(Y,\hat\Tau,\hat\nu)$ be the completion
of $(Y,\Tau,\nu)$.   Show that $y\mapsto\mu_y$ is a $\hat\Tau$-measurable
function from $Y$ to the set of Radon probability measures on $Z$ with
its narrow topology.   \Hint{437Rh.}
%452M out of order query

\spheader 452Xu\dvAnew{2012} $SU(r)$, for $r\ge 2$,
is the set of $r\times r$ matrices $T$ with complex coefficients such that
$\det T=1$ and $TT^*=I$, where $T^*$ is the complex conjugate of the
transpose of $T$.   (i) Show that under the natural action
$(T,u)\mapsto Tu:SU(r)\times\Bbb C^r\to\Bbb C^r$ the orbits are the
spheres $\{u:u\dotproduct\bar u=\gamma\}$, for $\gamma>0$, together with
$\{0\}$.   (ii) Show that if a Borel set $C\subseteq\Bbb C^r$ is such that
$\gamma C\subseteq C$ for every $\gamma>0$, and $\mu_0$, $\mu_1$ are two
$SU(r)$-invariant Radon probability measures on $\Bbb C^r$ such that
$\mu_0\{0\}=\mu_1\{0\}$, than $\mu_0C=\mu_1C$.\footnote{I am grateful to
G.Vitillaro for bringing this to my attention.}
%452T

\leader{452Y}{Further exercises (a)}
%\spheader 452Ya
Let $Z$ be a set, $(Y,\Tau,\nu)$ a measure space, and
$\family{y}{Y}{\mu_y}$ a family of measures on $Z$.   Let $\Upsilon$
be a $\sigma$-algebra of subsets of $Z$ such that, for every
$H\in\Upsilon$, $y\mapsto\mu_yH:Y\to[0,\infty]$ is defined $\nu$-a.e.\
and is $\nu$-virtually measurable.
For $F\in\Tau$, set $\Cal H_F=\{H:H\in\Upsilon$, $\mu_yH$ is defined for
every $y\in F$ and $\sup_{y\in F}\mu_yH<\infty\}$.   Show that there is
a measure $\mu$
on $Y\times Z$, with domain $\Tau\tensorhat\Upsilon$, defined by setting

$$\eqalign{\mu E
&=\sup\{\sum_{i=0}^n\int_{F_i}\mu_y(E[\{y\}]\cap H_i)\nu(dy):
   F_0,\ldots,F_n\in\Tau\text{ are disjoint},\cr
&\hskip13em \nu F_i<\infty\text{ and }H_i\in\Cal H_{F_i}\text{ for every
}i\le n\}\cr}$$

\noindent for $E\in\Tau\tensorhat\Upsilon$.

\spheader 452Yb
Let $(X,\Sigma,\mu)$ be a
semi-finite countably compact measure space, $(Y,\Tau,\nu)$ a strictly
localizable measure space, and $f:X\to Y$ an \imp\ function.   Suppose
that the magnitude of $\nu$ (definition:  332Ga) is finite or a
measure-free cardinal (definition:  438A).   Show that there
is a disintegration $\family{y}{Y}{\mu_y}$ of $\mu$ over $\nu$ which is
consistent with $f$.
%452O

\spheader 452Yc Give an example to show that the phrase `strictly
localizable' in the statements of 452O and 452Yb cannot be dispensed
with.
%452O

\spheader 452Yd Give an example to show that, in 452M, we cannot
always arrange that $\Upsilon\subseteq\dom\mu_y$ for $\nu$-almost every
$y\in Y$.
%452M  %mt45bits

\spheader 452Ye Let $(X,\Sigma,\mu)$ be a probability space such that
whenever $(Y,\Tau,\nu)$ is a probability space and $f:X\to Y$ is an
\imp\ function, there is a disintegration $\family{y}{Y}{\mu_y}$ of
$\mu$ over $\nu$ which is consistent with $f$.
Show that $\mu$ is countably compact.   \Hint{452R, or {\smc Pachl 78}.}
%452R

\spheader 452Yf Let $X$ be a K-analytic Hausdorff space and $\mu$ a totally
finite measure on $X$
which is inner regular with respect to the closed sets.
Show that $\mu$ is countably compact.   \Hint{432D.}
%452S

%new 2008
\spheader 452Yg Let $X$ be a set, and $\familyiI{\mu_i}$ a family of
countably compact measures on $X$ with sum $\mu$
(234G\formerly{1{}12Ya}).   Show that if $\mu$ is
semi-finite, it is countably compact.
%452S

\spheader 452Yh\dvAnew{2012}
Let $X$ be a locally compact Hausdorff space, $G$ a
compact Hausdorff group, and $\action$ a continuous action of $G$ on
$X$.   Let $H$ be another group and $\varaction$ a continuous
action of $H$ on $X$ which commutes with $\action$ in the sense that
$g\action(h\varaction x)=h\varaction(g\action x)$ for all
$g\in G$, $h\in H$ and $x\in X$.   (i) Show that
$((g,h),x)\to g\action(h\varaction x):(G\times H)\times X\to X$ is a
continuous
action of the product group $G\times H$ on $X$.   (ii) Suppose that the
action in (i) is transitive.   Show that if $\mu$, $\mu'$ are $G$-invariant
Radon probability measures on $X$ and $E\subseteq X$ is a Borel set such
that $h\varaction E=E$ for every $h\in H$, then $\mu E=\mu'E$.
%452T 452Xu

}%end of exercises

\endnotes{\Notesheader{452}
452B and 452C correspond respectively to the ordinary and
$\tau$-additive product measures of \S\S251 and 417.   I have not
attempted to find a suitable general formulation
for the constructions when the measures involved are not totally finite.
In 452Ya I set out a possible version which at least agrees
with the c.l.d.\ product measure
when all the $\mu_y$ are the same.   Any product measure which has an
associated Fubini theorem can be expected to be generalizable in the
same way;  for instance, 434R becomes 452Xa.

The hypotheses in 452B are closely matched with the conclusion, and
clearly cannot be relaxed substantially if the theorem is to remain
true.   452C and 452D are a rather different matter.
While the condition `$y\mapsto\mu_yV$ is lower semi-continuous' is a
natural one, and plainly necessary for the argument given, the
integrated measure
$\mu$ can be $\tau$-additive or Radon for other reasons.   In
particular, the most interesting specific example in this book of a
Radon measure constructed through these
formulae (453N below) does not satisfy the lower semi-continuity
condition for the section measures.

Early theorems on disintegrations
concentrated on cases in which all the measure spaces involved were
`standard' in that the measures were defined on standard Borel algebras,
or were the completions of such measures.   Theorem 452I here
is the end (so far) of a long search for ways to escape from topological
considerations.   As usual, of course, the most important applications
(in probability theory) are still rooted in the standard case.   Being
countably separated, such spaces automatically yield disintegrations
which are concentrated on fibers, in the sense that
$\mu_yf^{-1}[\{y\}]=\mu_yX=1$ for almost every $y$ (452P).   The general
question of when we can expect to find disintegrations of this type is
an important one to which I will return in the next section.

452I and 452O, as stated, assume that the functions $f:X\to Y$
controlling the disintegrations are \imp.   In fact it is easy to weaken
this assumption (452Xk).
Note the constructions for conditional expectations in 452Q and 452Xr.

Obviously 452I and 452M are nearly the same theorem;  but I write out
formally independent proofs because the constructions needed to move
between them are not quite trivial.   In fact I think it is easier to
deduce 452I from 452M than the other way about (452Xo).   The point of
452N is that the spaces $(X,\Sigma)$ there have the `countably compact
measure property', that is, any totally finite measure with domain
$\Sigma$ is countably compact.   I will return to this in the exercises
to \S454 (454Xf {\it et seq.}).

The method of 452R, due to J.Pachl, may have inspired the proof of
(vi)$\Rightarrow$(i) in 343B.   In the general introduction to this work
I wrote `I have very little confidence in anything I have ever read
concerning the history of ideas'.   We have here a case indicating the
difficulties a historian faces.   I
proved 343B in the winter of 1996-97, while a guest of the University of
Wisconsin at Madison.   Around that time I was renewing my acquaintance
with {\smc Pachl 78}.   I know I ran my eye over the proof of 452R,
without, I may say, understanding it, as became plain when I came to
write the first draft of the present section in the summer of 1997;
whether I had
understood it twenty years earlier I do not know.   It is entirely
possible that a subterranean percolation of Pachl's idea was what
dislodged an obstacle to my attempts to prove 343B, but I was not at the
time conscious of any connexion.
}%end of notes

\discrpage

