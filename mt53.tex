\frfilename{mt53.tex} 
\versiondate{30.8.14} 
\copyrightdate{2007} 
 
\def\chaptername{Topologies and measures III} 
 
\newchapter{53} 
 
In this chapter I return to the concerns of earlier volumes, looking for 
results which can be expressed in the language so far developed in this 
volume.    
In Chapter 43 I examined relationships between measure-theoretic and 
topological properties.   The concepts we now have available (in 
particular, the notion of `precaliber') make it possible to extend this 
work in a new direction, seeking to understand the possible Maharam 
types of measures on a given topological space. 
\S531 deals with general Radon measures;  new patterns arise if we 
restrict ourselves to completion regular Radon measures (\S532). 
In \S533 I give a brief account of some further 
results depending on assumptions 
concerning the cardinals examined in Chapter 52, including  
notes on uniformly regular measures and a description 
of the cardinals $\kappa$ for which $\BbbR^{\kappa}$ is measure-compact 
(533J). 
 
In \S534 I set out the elementary theory of `strong measure zero' ideals in 
uniform spaces, concentrating on aspects which can be studied in terms of 
concepts already induced.   Here there are some very natural questions  
which have not I think been answered (534Z).   In the same section I run 
through elementary  
properties of Hausdorff measures when examined in the light of 
the concepts in Chapter 52.   In \S535 I look at 
liftings and strong liftings, extending the results of \S\S341 and 453; 
in particular, asking which 
non-complete probability spaces have liftings.   In \S536 I run over 
what is known about Alexandra Bellow's problem concerning pointwise 
compact sets of continuous functions, mentioned in \S463.   With a little 
help from special axioms, there are some striking possibilities concerning 
repeated integrals, which I examine in \S537. 
Moving into new territory, I devote a section (\S538) to a study of special 
types of filter on $\Bbb N$ associated with measure-theoretic phenomena, 
and to medial limits. 
In \S539, I complete my account of the result of B.Balcar, T.Jech and 
T.Paz\'ak that it is consistent to suppose that every Dedekind complete ccc 
\wsid\ Boolean algebra is a Maharam algebra, and work through applications 
of the methods of Chapter 52 to Maharam submeasures and algebras. 
 
\discrpage 
 
