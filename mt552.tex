\frfilename{mt552.tex}
\versiondate{29.1.14}
\copyrightdate{2007}

\Loadfourteens

\def\chaptername{Possible worlds}
\def\sectionname{Random reals I}

\def\BbbPk{\Bbb P_{\kappa}}
\def\eightVVdash{\mskip5mu\vrule height 6pt depth 2pt width 0.4pt
  \mskip2mu\vrule height 6pt depth 2pt width 0.4pt
  \vrule height 2.25pt depth -1.8pt width 3.2pt\mskip2mu}
\def\VVdPk{\VVdash_{\Bbb P_{\kappa}}}


\newsection{552}

From the point of view of a measure theorist, `random real forcing' has a
particular significance.   Because the forcing notions are defined directly
from the central structures of measure theory (552A), they can be
expected to provide worlds in which measure-theoretic questions are
answered.   Thus we find ourselves with many Sierpi\'nski sets (552E),
information on cardinal functions
(552C, 552F-552J), %552F 552G 552H 552I 552J
and theorems on extension of measures (552N).   But there is a
second reason why any measure theorist or probabilist should pay attention
to random real forcing.   Natural questions in the forcing language, when
translated into propositions about the ground model, are likely to hinge on
properties of measure algebras, giving us a new source of challenging
problems.   Perhaps the deepest intuitions are those associated with
iterated random real forcing (552P).

\leader{552A}{Notation (a)} As usual, if $\mu$ is a measure then
$\Cal N(\mu)$ will be its null ideal.   It
will be convenient to have a special notation for certain sets of
finite functions:  if $I$ is a set,
$\Fn_{<\omega}(I;\{0,1\})$ will be
$\bigcup_{K\in[I]^{<\omega}}\{0,1\}^K$.

For any set $I$ I\cmmnt{ will} write $\nu_I$
for the usual completion regular Radon probability
measure on $\{0,1\}^I$, $\Tau_I$ for its domain and
$(\frak B_I,\bar\nu_I)$ for its measure algebra;
$\CalBa_I\cmmnt{\mskip5mu=\CalBa(\{0,1\}^I)}$
will be the Baire $\sigma$-algebra of $\{0,1\}^I$.
\cmmnt{(It will sometimes be convenient, when
applying the results of \S551, to regard $\frak B_I$ as the quotient
$\CalBa_I/\CalBa_I\cap\Cal N(\nu_I)$.)}   \cmmnt{In this context,} I\cmmnt{ will}
write $\familyiI{e_i}$ for the standard generating family in
$\frak B_I$\cmmnt{ (525A)}.   $\Bbb P_I$ will be the forcing notion
$\frak B_I^+=\frak B_I\setminus\{0\}$, active downwards.
For a formula $\phi$
in the corresponding forcing language I\cmmnt{ will} write $\Bvalue{\phi}$ for the
truth value of $\phi$, interpreted as a member of
$\frak B_I$\cmmnt{ (5A3M)}.
\cmmnt{Note that as }$\Bbb P_I$\cmmnt{ is ccc (cf.\ 511Db),
it} preserves cardinals\cmmnt{ (5A3Nb)}.

As in \S551, the formulae $\nu_I$, $\frak B_I$ etc.\ are to be regarded as
formulae of set theory with one free variable into which the parameter $I$
has been substituted, so that we have corresponding names $\nu_{\dot I}$,
$\frak B_{\dot I}$ in any forcing language, and in particular (once the
context has established a forcing notion $\Bbb P$) we have $\Bbb P$-names
$\nu_{\check I}$, $\frak B_{\check I}$ for any ground-model set $I$.

\spheader 552Ab A great deal of the work of this chapter will involve
interpretations of names for standard objects\cmmnt{ (in particular, for
cardinals)} in forcing languages.
\cmmnt{Reflecting suggestions in 5A3G and 5A3N,}
I\cmmnt{ will} try to signal intended interpretations by using the superscript
$\var2spcheck$.
\cmmnt{Thus $\frak c$ will always be an abbreviation for `the
initial ordinal equipollent with the set of subsets of the natural
numbers', whether I am using the ordinary
language of set theory or speaking in a forcing language;  and
$\check{\frak c}$, in a forcing language, will refer to the name
$\{(\check\xi,\Bbbone):\xi<\frak c\}$, where it is to be understood that
the symbol $\frak c$ must now
be interpreted in the ordinary universe.   As I
shall avoid arguments involving more than one forcing notion (and, in
particular, iterated forcing), there will I hope be little scope for
confusion, even in such sentences as

\Centerline{$\VVdPk\,\frak b=\check{\frak b}$}

\noindent (552C).   The leading $\VVdPk$ declares that the rest of the
sentence is in the language of $\Bbb P_{\kappa}$-forcing;
the first $\frak b$, and the $\var2spcheck$, are therefore to be
interpreted in that language;  but the second $\frak b$, being subject to
the $\mskip5mu\var2spcheck$, is to be interpreted in the ground model.
(Many authors would write $\frak b^V$ at this point.)   Similarly, in

\Centerline{$\VVdPk\,2^{\check\lambda}=(\kappa^{\lambda})\var2spcheck$}

\noindent (552B), the subformula $\kappa^{\lambda}$ is to be interpreted in
the ordinary universe, but $2^{\check\lambda}=\#(\Cal P\check\lambda)$ is
to be interpreted in the forcing language.   I hope that
the resulting gains in directness and conciseness
will not be at the expense of leaving you uncertain of the meaning.}

\leader{552B}{Theorem} Suppose that $\lambda$ and $\kappa$ are infinite
cardinals.   Then

\Centerline{$\VVdPk\,2^{\check\lambda}=(\kappa^{\lambda})\var2spcheck$,}

\noindent where $\kappa^{\lambda}$ is the cardinal power\cmmnt{
(interpreted in the ordinary universe, of course)}.

\proof{{\bf (a)} Recall that $\#(\frak B_{\kappa})=\kappa^{\omega}$
(524Ma), so that

\Centerline{$\#(\frak B_{\kappa}^{\lambda})
=\#(\kappa^{\omega\times\lambda})$.}

If $\dot A$ is a $\BbbPk$-name for a subset of
$\check\lambda$, then we have a corresponding family
$\ofamily{\eta}{\lambda}{\Bvalue{\check\eta\in\dot A}}$ of truth values;
and if $\dot A$, $\dot B$ are two such names, and
$\Bvalue{\check\eta\in\dot A}=\Bvalue{\check\eta\in\dot B}$
for every $\eta<\lambda$, then

\Centerline{$\VVdPk\,\check\eta\in\dot A\iff\check\eta\in\dot B$}

\noindent for every $\eta<\lambda$, so

\Centerline{$\VVdPk\,\dot A=\dot B$.}

\noindent So

\Centerline{$\VVdPk\,2^{\check\lambda}=\#(\Cal P\check\lambda)
\le\#((\frak B_{\kappa}^{\lambda})\var2spcheck)=(\kappa^{\lambda})\var2spcheck$.}

\medskip

{\bf (b)} In the other direction,
consider first the case in which $\lambda\le\kappa$.
Let $F$ be the set of all functions from $\lambda$ to $\kappa$,
so that $\#(F)=\kappa^{\lambda}$.
Then there is a set $G\subseteq F$ such that
$\#(G)=\kappa^{\lambda}$ and $\{\eta:\eta<\lambda$, $f(\eta)\ne g(\eta)\}$
is infinite
whenever $f$, $g\in G$ are distinct.   \Prf\ If $\kappa=\kappa^{\lambda}$ we can take
$G$ to be the set of constant functions.   Otherwise, for $f$, $g\in F$,
say that $f=^*g$ if
$\{\eta:f(\eta)\ne g(\eta)\}$ is finite;  this is an equivalence relation.
Let $G\subseteq F$ be a set meeting
each equivalence class in just one element.   Then we have
$\#(\{g:g=^*f\})=\kappa<\kappa^{\lambda}$ for every $f\in F$, so $\#(G)=\kappa^{\lambda}$, as
required.\ \Qed

Let $\langle e_{\xi\eta}\rangle_{\xi<\kappa,\eta<\lambda}$ be a
stochastically independent family in $\frak B_{\kappa}$ of elements of
measure $\bover12$.   For $f\in G$ let $\dot A_f$ be a $\BbbPk$-name for
a subset of $\check\lambda$ such that

\Centerline{$\Bvalue{\check\eta\in\dot A_f}=e_{f(\eta),\eta}$}

\noindent for every $\eta<\lambda$.   If $f$, $g\in G$ are distinct, set
$I=\{\eta:f(\eta)\ne g(\eta)\}$;  then

\Centerline{$\Bvalue{\dot A_f\ne\dot A_g}
=\sup_{\eta<\lambda}e_{f(\eta),\eta}\Bsymmdiff e_{g(\eta),\eta}
=\sup_{\eta\in I}e_{f(\eta),\eta}\Bsymmdiff e_{g(\eta),\eta}
=1$}

\noindent because
$\family{\eta}{I}{e_{f(\eta),\eta}\Bsymmdiff e_{g(\eta),\eta}}$
is an infinite
stochastically independent family of elements of measure $\bover12$.

Thus in the forcing language we have a name for an injective function from
$\check G$ to $\Cal P\check\lambda$,
corresponding to the map $f\mapsto\dot A_f$
from $G$ to names of subsets of $\lambda$.   So

\Centerline{$\VVdPk\,2^{\check\lambda}\ge\#(\check G)
=(\kappa^{\lambda})\var2spcheck$.}

\noindent Putting this together with (a), we have

\Centerline{$\VVdPk\,2^{\check\lambda}=(\kappa^{\lambda})\var2spcheck$.}

\medskip

{\bf (c)} If $\lambda>\kappa$, then (in the ordinary universe)
$2^{\lambda}=\kappa^{\lambda}$.   Now

\Centerline{$\VVdPk\,(\Cal P\lambda)\var2spcheck
\subseteq\Cal P\check\lambda$,}

\noindent so

\Centerline{$\VVdPk\,(\kappa^{\lambda})\var2spcheck=\#((\Cal P\lambda)\var2spcheck)
\le\#(\Cal P\check\lambda)=2^{\check\lambda}$,}

\noindent and again we have

\Centerline{$\VVdPk\,2^{\check\lambda}=(\kappa^{\lambda})\var2spcheck$.}
}%end of proof of 552B

\leader{552C}{Theorem} Let $\kappa$ be any cardinal.   Then

\Centerline{$\VVdPk\,\frak b=\check{\frak b}$ and
$\frak d=\check{\frak d}$.}

\proof{{\bf (a)} The point is that if $\dot f$ is any $\BbbPk$-name for a
member of $\BbbN^{\Bbb N}$, then there is an $h\in\BbbN^{\Bbb N}$ such that

\Centerline{$\VVdPk\,\dot f\le^*\check h$,}

\noindent where I write $f\le^*g$ to mean that $\{n:g(n)<f(n)\}$ is finite,
as in 522C.   \Prf\ For $n$, $i\in\Bbb N$ set
$a_{ni}=\Bvalue{\dot f(\check n)=\varchecki}$.
Then $D_n=\{a_{ni}:i\in\Bbb N\}$ is a partition of unity in
$\frak B_{\kappa}$ for each $n\in\Bbb N$.   Because $\frak B_{\kappa}$ is
\wsid\ (322F), there is a partition of unity $D$ such that
$\{i:a_{ni}\Bcap d\ne 0\}$ is finite for each $n$ and each $d\in D$.
Let $\sequence{k}{d_k}$ be a sequence running over $D$ and take $h(n)$ such
that $a_{mi}\Bcap d_n=0$ whenever $m\le n$ and $i>h(n)$.   Now

\Centerline{$\Bvalue{\check h(\check m)<\dot f(\check m)}
=\Bvalue{h(m)\var2spcheck<\dot f(\check m)}
=\sup\{a_{mi}:i>h(m)\}\Bsubseteq 1\Bsetminus d_n$}

\noindent whenever $n\le m$.   So

$$\eqalign{\Bvalue{\check h(n)<\dot f(n)\text{ for infinitely many }n}
&=\inf_{n\in\Bbb N}\sup_{m\ge n}
  \Bvalue{h(m)\var2spcheck<\dot f(\check m)}\cr
&\Bsubseteq\inf_{n\in\Bbb N}1\Bsetminus d_n
=0,\cr}$$

\noindent that is,

\Centerline{$\VVdPk\,\dot f\le^*\check h$.  \Qed}

\medskip

{\bf (b)(i)} Let $\ofamily{\xi}{\lambda}{\dot f_{\xi}}$ be a family of
$\BbbPk$-names for members of $\BbbN^{\Bbb N}$, where $\lambda<{\frak b}$.
Then for each $\xi<\lambda$ we can find an $h_{\xi}\in\BbbN^{\Bbb N}$ such
that $\VVdPk\,\dot f_{\xi}\le^*\check h_{\xi}$.   As $\lambda<{\frak b}$, there
is an $h\in\BbbN^{\Bbb N}$ such that $h_{\xi}\le^*h$ for every
$\xi<\lambda$.   Now $\VVdPk\,\check h_{\xi}\le^*\check h$ for every $\xi$, so
$\VVdPk\,\dot f_{\xi}\le^*\check h$ for every $\xi$.   As $\lambda$ and
$\ofamily{\xi}{\lambda}{\dot f_{\xi}}$ are arbitrary,

\Centerline{$\VVdPk\,\frak b\ge\check{\frak b}$.}

\medskip

\quad{\bf (ii)} Let $\ofamily{\xi}{{\frak b}}{h_{\xi}}$ be a family in
$\BbbN^{\Bbb N}$ which has no $\le^*$-upper bound in $\BbbN^{\Bbb N}$.
Then

\Centerline{$\VVdPk\,\{\check h_{\xi}:\xi<\check{\frak b}\}$
has no $\le^*$-upper bound.}

\noindent\Prf\Quer\ Otherwise, there are a $\BbbPk$-name $\dot f$ for a
member of $\BbbN^{\Bbb N}$ and an $a\in\frak B_{\kappa}^+$ such that

\Centerline{$a\VVdPk\,\check h_{\xi}\le^*\dot f$ for every
$\xi<\check{\frak b}$.}

\noindent Now there is an $h\in\BbbN^{\Bbb N}$ such that
$\VVdPk\,\dot f\le^*\check h$.
There must be a $\xi<{\frak b}$ such that $h_{\xi}\not\le^*h$.   We have
$a\VVdPk\,\check h_{\xi}\le^*\dot f\le^*\check h$, so there are an $a'$,
stronger than $a$, and an $n\in\Bbb N$ such that

\Centerline{$a'\VVdPk\,\check h_{\xi}(i)\le\check h(i)$ for every
$i\ge\check n$.}

\noindent However, there is an $i\ge n$ such that $h(i)<h_{\xi}(i)$, in
which case

\Centerline{$\VVdPk\,\varchecki\ge\check n$ and
$\check h(\varchecki)<\check h_{\xi}(\varchecki)$;}

\noindent  which is impossible.\ \Bang\Qed

So

\Centerline{$\VVdPk\,\frak b\le\check{\frak b}$, therefore
$\frak b=\check{\frak b}$.}

\medskip

{\bf (c)(i)} Let $\ofamily{\xi}{\lambda}{\dot f_{\xi}}$ be a family of
$\BbbPk$-names for members of $\BbbN^{\Bbb N}$ where
$\lambda<\check{\frak d}$.
Then for each $\xi<\lambda$ we can find an $h_{\xi}\in\BbbN^{\Bbb N}$ such
that $\VVdPk\,\dot f_{\xi}\le^*\check h_{\xi}$.   As $\lambda<{\frak d}$, there
is an $h\in\BbbN^{\Bbb N}$ such that $h\not\le^*h_{\xi}$ for every
$\xi<\lambda$.   Now

\Centerline{$\VVdPk\,\check h\not\le^*\check h_{\xi}$ for every
$\xi<\check\lambda$,}

\noindent so

\Centerline{$\VVdPk\,\check h\not\le^*\dot f_{\xi}$ for every
$\xi<\check\lambda$.}

\noindent As $\lambda$ and
$\ofamily{\xi}{\lambda}{\dot f_{\xi}}$ are arbitrary,

\Centerline{$\VVdPk\,\frak d\ge\check{\frak d}$.}

\medskip

\quad{\bf (ii)} Let $\ofamily{\xi}{{\frak d}}{h_{\xi}}$ be a family in
$\BbbN^{\Bbb N}$ which is $\le^*$-cofinal with $\BbbN^{\Bbb N}$.
Then

\Centerline{$\VVdPk\,\{\check h_{\xi}:\xi<\check{\frak d}\}$ is
$\le^*$-cofinal with $\BbbN^{\Bbb N}$.}

\noindent\Prf\ Let $\dot f$ be a $\BbbPk$-name for a
member of $\BbbN^{\Bbb N}$.  There are an $h\in\BbbN^{\Bbb N}$ such that
$\VVdPk\,\dot f\le^*\check h$, and a $\xi<{\frak d}$ such that
$h\le^*h_{\xi}$.   In this case,

\Centerline{$\VVdPk\,\dot f\le^*\check h\le^*\check h_{\xi}$.  \Qed}

\noindent So

\Centerline{$\VVdPk\,\frak d\le\check{\frak d}$.}
}%end of proof of 552C

\vleader{36pt}{552D}{Lemma} Let $\lambda$ and $\kappa$ be infinite
cardinals, and $A$ any subset of $\{0,1\}^{\lambda}$.   Then

\Centerline{$\VVdPk\,\nu_{\check\lambda}^*(\check A)
=(\nu_{\lambda}^*A)\var2spcheck$.}

\proof{{\bf (a)} \Quer\ Suppose, if possible, that

\Centerline{$\neg\VVdPk\,(\nu_{\lambda}^*A)\var2spcheck
\le\nu_{\check\lambda}^*(\check A)$.}

\noindent Then there are an $a\in\frak B_{\kappa}^+$ and a $q\in\Bbb Q$ such that
$q<\nu_{\lambda}^*A$ and

\Centerline{$a\VVdPk\,\nu_{\check\lambda}^*(\check A)<\check q$.}

\noindent Let $\dot E$ be a $\BbbPk$-name such that

\Centerline{$a\VVdPk\,\check A\subseteq\dot E$,
$\dot E\in\CalBa_{\check\lambda}$ and
$\nu_{\check\lambda}\dot E<\check q$.}

\noindent Of course we can arrange that
$1\Bsetminus a\VVdPk\,\dot E=\emptyset$,
so that $\VVdPk\,\dot E\in\CalBa_{\check\lambda}$ and there is a
$W\in\Tau_{\kappa}\tensorhat\CalBa_{\lambda}$ such that
$\VVdPk\,\dot E=\vec W$ (551Fb).   Setting $h(x)=\nu_{\lambda}W[\{x\}]$ for
$x\in\{0,1\}^{\kappa}$,

\Centerline{$\VVdPk\,\vec h=\nu_{\check\lambda}\vec W$}

\noindent (551I(iii)), so

\Centerline{$a\VVdPk\,\vec h=\nu_{\check\lambda}\dot E<\check q$}

\noindent and $a\Bsubseteq\{x:h(x)<q\}^{\ssbullet}$.
Take $F\in\Tau_{\kappa}$
such that $F^{\ssbullet}=a$;  then $h(x)<q$ for almost every $x\in F$ and
$(\nu_{\kappa}\times\nu_{\lambda})(W\cap(F\times\{0,1\}^{\lambda}))
<q\nu_{\kappa}F$.

For each $y\in A$, let $e_y:\{0,1\}^{\kappa}\to\{0,1\}^{\lambda}$ be the
constant function with value $y$.   Then $\VVdPk\,\vec e_y=\check y$
(551Ce), so

\Centerline{$a\VVdPk\,\vec e_y\in\check A\subseteq\vec W$}

\noindent and $(x,y)\in W$ for almost every $x\in F$.   But if we set

\Centerline{$H=\{y:(x,y)\in W$ for $\nu_{\kappa}$-almost every $x\in F\}$,}

\noindent $H\in\Tau_{\lambda}$, $A\subseteq H$ and

\Centerline{$\nu_{\kappa}F\cdot\nu_{\lambda}H
\le(\nu_{\kappa}\times\nu_{\lambda})(W\cap(F\times\{0,1\}^{\lambda}))
<q\nu_{\kappa}F$.}

\noindent It follows that

\Centerline{$\nu_{\lambda}^*A\le\nu_{\lambda}H<q$,}

\noindent contrary to hypothesis.\ \BanG\  So

\Centerline{$\VVdPk\,(\nu_{\lambda}^*A)\var2spcheck
\le\nu_{\check\lambda}^*(\check A)$.}


\medskip

{\bf (b)} In the other direction, let $E\in\CalBa_{\lambda}$ be such that
$A\subseteq E$ and $\nu_{\lambda}E=\nu_{\lambda}^*A$, and consider
$W=\{0,1\}^{\kappa}\times E$.   Then

\Centerline{$\VVdPk\,\check A\subseteq\vec W$ and
$\nu_{\check\lambda}\vec W=(\nu_{\lambda}E)\var2spcheck$,}

\noindent so

\Centerline{$\VVdPk\,\dot\nu_{\lambda}^*\check A
\le(\nu_{\lambda}^*A)\var2spcheck$.}
}%end of proof of 552D

\leader{552E}{Theorem} Let $\kappa$ and $\lambda$ be infinite
cardinals, with $\kappa\ge\max(\omega_1,\lambda)$.
Then

\Centerline{$\VVdPk$ there is a strongly Sierpi\'nski set for
$\nu_{\check\lambda}$ of size $\check\kappa$.}

\proof{{\bf (a)} As $\kappa\ge\lambda$, $\Bbb P_{\kappa}$ is isomorphic to
$\Bbb P=\Bbb P_{\kappa\times\lambda}$.   For each $\xi<\kappa$, let
$f_{\xi}:\{0,1\}^{\kappa\times\lambda}\to\{0,1\}^{\lambda}$ be given by
setting $f_{\xi}(x)(\eta)=x(\xi,\eta)$ for every
$x\in\{0,1\}^{\kappa\times\lambda}$ and $\eta<\lambda$;  then, taking
$\vec f_{\xi}$ to be the $\Bbb P$-name defined by the
process of 551Cb,

\Centerline{$\VVdP\,\vec f_{\xi}\in\{0,1\}^{\check\lambda}$.}

\noindent If $\xi$, $\xi'<\kappa$ are distinct, then for any finite set
$I\subseteq\lambda$

$$\eqalign{\Bvalue{\vec f_{\xi}(\eta)
=\vec f_{\xi'}(\eta)\text{ for every }\eta\in\check I}
&=\{x:f_{\xi}(x)(\eta)=f_{\xi'}(x)(\eta)
  \text{ for every }\eta\in I\}^{\ssbullet}\cr
&=\{x:x(\xi,\eta)=x(\xi',\eta)\text{ for every }\eta\in I\}^{\ssbullet}\cr}$$

\noindent has measure $2^{-\#(I)}$, so, because $\lambda$ is infinite,
$\bar\nu\Bvalue{\vec f_{\xi}=\vec f_{\xi'}}=0$ and

\Centerline{$\VVdP\,\vec f_{\xi}\ne\vec f_{\xi'}$.}

\noindent So, taking $\dot A$ to be the $\Bbb P$-name
$\{(\vec f_{\xi},\Bbbone):\xi<\kappa\}$, we have

\Centerline{$\VVdP\,\dot A\subseteq\{0,1\}^{\check\lambda}$ has
cardinal $\check\kappa\ge\omega_1$, so is uncountable.}

\noindent(As remarked in 5A3Nb, we do not need to distinguish between
$\omega_1$ and $\check\omega_1$ in the last formula.)

\medskip

{\bf (b)} Let $r\ge 1$ be an integer and $\dot W$ a
$\Bbb P$-name such that

\Centerline{$\VVdP\,\dot W$ is a subset of
$(\{0,1\}^{\check\lambda})^{\check r}$
which is negligible for the usual measure.}

\noindent Then there is a Baire subset $W$ of
$\{0,1\}^{\kappa\times\lambda}\times(\{0,1\}^{\lambda})^r$,
negligible for the usual measure on this space, such that

\Centerline{$\VVdP\,\dot W\subseteq\vec W$}

\noindent (551J, applied to
$\{0,1\}^{\lambda\times r}\cong(\{0,1\}^{\lambda})^r$).
Let $J\subseteq\kappa$ be a countable set such that $W$ factors
through $\{0,1\}^{J\times\lambda}\times(\{0,1\}^{\lambda})^r$, that is,
there is a negligible Baire set
$W_1\subseteq\{0,1\}^{J\times\lambda}\times(\{0,1\}^{\lambda})^r$ such that
$W=\{(x,y):(x\restr J\times\lambda,y)\in W_1\}$.
If $\xi_0,\ldots,\xi_{r-1}$
are distinct elements of $\kappa\setminus J$, then

\Centerline{$\VVdP\,(\vec f_{\xi_0},\ldots,\vec f_{\xi_{r-1}})\notin\vec W$.}

\noindent\Prf\ Applying 551Ea to the function
$x\mapsto(f_{\xi_0}(x),\ldots,f_{\xi_{r-1}}(x))$, we have

\Centerline{$\Bvalue{(\vec f_{\xi_0},\ldots,\vec f_{\xi_{r-1}})\in\vec W}
=\{x:(x,(f_{\xi_0}(x),\ldots,f_{\xi_{r-1}}(x)))\in W\}^{\ssbullet}$.}

\noindent Set $K=J\cup\{\xi_0,\ldots,\xi_{r-1}\}$ and for
$w\in\{0,1\}^{K\times\lambda}$, $i<r$, $\eta<\lambda$
set $g_i(w)(\eta)=w(\xi_i,\eta)$.   Then
$w\mapsto(w\restr J\times\lambda,(g_0(w),\ldots,g_{r-1}(w))$ is a
measure space isomorphism
between $\{0,1\}^{K\times\lambda}$ and
$\{0,1\}^{J\times\lambda}\times(\{0,1\}^{\lambda})^r$, so

\Centerline{$W_2=\{w:w\in\{0,1\}^{K\times\lambda}$,
$(w\restr J\times\lambda,(g_0(w),\ldots,g_{r-1}(w))\in W_1\}$}

\noindent is negligible.   Consequently


$$\eqalign{\{x:(x,(f_{\xi_0}(x),\ldots,f_{\xi_{r-1}}(x)))\in W\}
&=\{x:(x\restr J\times\lambda,(f_{\xi_0}(x),\ldots,f_{\xi_{r-1}}(x)))
  \in W_1\}\cr
&=\{x:x\restr K\times\lambda\in W_2\}\cr}$$

\noindent is negligible and
$\Bvalue{(\vec f_{\xi_0},\ldots,\vec f_{\xi_{r-1}})\in\vec W}=0$, that is,

\Centerline{$\VVdP\,(\vec f_{\xi_0},\ldots,\vec f_{\xi_{r-1}})\notin\vec W$.
\Qed}

Now if we set $\dot B=\{(\vec f_{\xi},\Bbbone):\xi\in J\}$, we have

\doubleinset{$\VVdP\,\dot B$ is a countable subset of $\dot A$ and
$(x_0,\ldots,x_{r-1})\notin\dot W$ whenever $x_0,\ldots,x_{r-1}$ are distinct
members of $\dot A\setminus\dot B$.}

\noindent As $\dot W$ is arbitrary,

\Centerline{$\VVdP\,\dot A$ is a
strongly Sierpi\'nski set of size $\check\kappa$.}

\noindent As $\Bbb P$ is isomorphic to $\Bbb P_{\kappa}$,

\Centerline{$\VVdPk$ there is a
strongly Sierpi\'nski set for $\nu_{\check\lambda}$
of size $\check\kappa$.}
}%end of proof of 552E

\leader{552F}{Theorem} Let $\kappa$ and $\lambda$ be infinite cardinals.

(a) If either $\kappa$ or $\lambda$ is uncountable,

\Centerline{$\VVdPk\,\add\Cal N(\nu_{\check\lambda})=\omega_1$.}

(b) $\VVdash_{\Bbb P_{\omega}}\,\add\Cal N(\nu_{\omega})
=(\add\Cal N(\nu_{\omega}))\var2spcheck$.

\proof{{\bf (a)(i)} If $\lambda$ is uncountable, then

\Centerline{$\VVdPk\,\check\lambda$ is
uncountable, so $\add\Cal N(\nu_{\check\lambda})=\omega_1$}

\noindent (5A3Nb, 521Jb/523E).

\medskip

\quad{\bf (ii)} If $\kappa$ is uncountable, then

\Centerline{$\VVdPk$ there is a Sierpi\'nski set for $\nu_{\omega}$, so
$\omega_1\le\add\Cal N(\nu_{\check\lambda})
\le\add\Cal N(\nu_{\omega})=\omega_1$}

\noindent (552E, 523B, 537B(a-i)).
%523B Cichon

\medskip

{\bf (b)(i)}
Let $\ofamily{\xi}{\add\Cal N(\nu_{\omega})}{H_{\xi}}$ be a family of
negligible Borel sets in $\{0,1\}^{\omega}$ such that
$A=\bigcup_{\xi<\add\Cal N(\nu_{\omega})}H_{\xi}$ is not negligible.
Then 552D tells us that

\doubleinset{$\VVdash_{\Bbb P_{\omega}}\,
\check H_{\xi}$ is negligible for every
$\xi<(\add\Cal N(\nu_{\omega}))\var2spcheck$,
but $\check A=\bigcup_{\xi<(\add\Cal N(\nu_{\omega}))\var2spcheck}\check H_{\xi}$ is not,
so $\add\Cal N(\nu_{\omega})\le(\add\Cal N(\nu_{\omega}))\var2spcheck$.}

\medskip

\quad{\bf (ii)} \Quer\ If

\Centerline{$\neg\VVdash_{\Bbb P_{\omega}}\,
\add\Cal N(\nu_{\omega})\ge(\add\Cal N(\nu_{\omega}))\var2spcheck$,}

\noindent then there are an $a\in\frak B_{\kappa}^+$ and a
$\theta<\add\Cal N(\nu_{\omega})$ such that

\Centerline{$a\VVdash_{\Bbb P_{\omega}}\,
\add\Cal N(\nu_{\omega})=\check\theta$.}

\noindent Now there is a family $\ofamily{\xi}{\theta}{\dot W_{\xi}}$ of
$\BbbPk$-names such that

\Centerline{$a\VVdash_{\Bbb P_{\omega}}\,
\dot W_{\xi}\in\Cal N(\nu_{\omega})\Forall\xi<\check\theta$,
$\bigcup_{\xi<\check\theta}\dot W_{\xi}\notin\Cal N(\nu_{\omega})$.}

\noindent By 551J, there is for each $\xi<\theta$ a
$W_{\xi}\in\Tau_{\omega}\tensorhat\CalBa_{\omega}$ such that

\Centerline{$a\VVdash_{\Bbb P_{\omega}}\,
\dot W_{\xi}\subseteq\vec W_{\xi}$}

\noindent and all the vertical sections of every $W_{\xi}$ are negligible.   But
this means that $W_{\xi}$ is negligible for the product measure
$\nu_{\omega}\times\nu_{\omega}$.   Because

\Centerline{$\theta<\add\Cal N(\nu_{\omega})
=\add\Cal N(\nu_{\omega}\times\nu_{\omega})$,}

\noindent
$\bigcup_{\xi<\theta}W_{\xi}$ also is negligible, and there is a negligible
$W\in\Tau_{\omega}\tensorhat\CalBa_{\omega}$ including every $W_{\xi}$.
In this case, 551I(iii) tells us that

\Centerline{$\VVdash_{\Bbb P_{\omega}}\,\nu_{\omega}\vec W=0$,}

\noindent so

\Centerline{$a\VVdash_{\Bbb P_{\omega}}\,
\bigcup_{\xi<\check\theta}\dot W_{\xi}
\subseteq\bigcup_{\xi<\check\theta}\vec W_{\xi}
\subseteq\vec W$ is negligible.   \Bang}

Putting this together with (i),

\Centerline{$\VVdash_{\Bbb P_{\omega}}\,
  \add\Cal N(\nu_{\omega})=(\add\Cal N(\nu_{\omega}))\var2spcheck$.}
}%end of proof of 552F

\leader{552G}{Theorem} Let $\kappa$ and $\lambda$ be infinite cardinals.

(a) $\VVdPk\,\cov\Cal N(\nu_{\check\lambda})
\ge\max(\kappa,\cov\Cal N(\nu_{\lambda}))\var2spcheck$.

(b)\cmmnt{ ({\smc Pawlikowski 86})}
$\VVdPk\,\cov\Cal N(\nu_{\omega})\ge\frak b$.

(c)\cmmnt{ ({\smc Miller 82})}
If $\kappa\ge\frak c$ then
$\VVdPk\,\cov\Cal N(\nu_{\omega})=\frak c$.\cmmnt{\footnote{\smallerfonts
Remember that the final
$\frak c$ here is to be interpreted in the forcing language.}}

(d)\cmmnt{ ({\smc Miller 82})} Suppose that $\kappa$ and $\lambda$
are uncountable.   Then

\Centerline{$\VVdPk\,\cov\Cal N(\nu_{\check\lambda})
\le(\sup_{\delta<\kappa}\delta^{\omega})\var2spcheck$,}

\noindent where each $\delta^{\omega}$ is the cardinal power.

\proof{{\bf (a)(i)} If $\kappa=\omega$ then of course
$\VVdPk\,\cov\Cal N(\nu_{\check\lambda})\ge\check\kappa$.
If $\kappa$ is uncountable, then

\Centerline{$\VVdPk\,\cov\Cal N(\nu_{\check\kappa})\ge\check\kappa$}

\noindent by 552E and 537B(a-i), so by 523F we have

\Centerline{$\VVdPk\,\cov\Cal N(\nu_{\check\lambda})\ge\check\kappa$.}

\medskip

\quad{\bf (ii)} \Quer\ If

\Centerline{$\neg\VVdPk\,\cov\Cal N(\nu_{\check\lambda})
  \ge(\cov\Cal N(\nu_{\lambda}))\var2spcheck$}

\noindent then we have an $a\in\frak B_{\kappa}^+$, a cardinal
$\theta<\cov\Cal N(\nu_{\lambda})$ and a
family $\ofamily{\xi}{\theta}{\dot W_{\xi}}$ of $\BbbPk$-names such that

\Centerline{$a\VVdPk\,\{\dot W_{\xi}:\xi<\check\theta\}$ is a cover of
$\{0,1\}^{\check\lambda}$ by negligible sets.}

\noindent By 551J again, we have for each $\xi<\theta$ a
$(\nu_{\kappa}\times\nu_{\lambda})$-negligible set $W_{\xi}$ such that
$a\VVdPk\,\dot W_{\xi}\subseteq\vec W_{\xi}$.   Set

\Centerline{$V_{\xi}=\{y:y\in\{0,1\}^{\lambda}$, $W_{\xi}^{-1}[\{y\}]$ is not
$\nu_{\kappa}$-negligible$\}$;}

\noindent then $\nu_{\lambda}V_{\xi}=0$ for every $\xi<\theta$,
so there is a
$y\in\{0,1\}^{\lambda}\setminus\bigcup_{\xi<\theta}V_{\xi}$.
In this case, let
$e_y:\{0,1\}^{\kappa}\to\{0,1\}^{\lambda}$ be the constant function with
value $y$.   Then we have a $\BbbPk$-name $\vec e_y$ for a member of
$\{0,1\}^{\lambda}$, and for each $\xi<\theta$

\Centerline{$\Bvalue{\vec e_y\in\vec W_{\xi}}
=\{x:(x,e_y(x))\in W_{\xi}\}^{\ssbullet}
=W_{\xi}^{-1}[\{y\}]^{\ssbullet}=0$}

\noindent (551Ea).   So

\Centerline{$\VVdPk\,\vec e_y\notin\vec W_{\xi}$ for every
$\xi<\check\theta$,}

\noindent and

\Centerline{$a\VVdPk\,
\vec e_y\in\{0,1\}^{\check\lambda}
  \setminus\bigcup_{\xi<\check\theta}\dot W_{\xi}$.  \Bang}

We conclude that

\Centerline{$\VVdPk\,\cov\Cal N(\nu_{\check\lambda})
  \ge(\cov\Cal N(\nu_{\lambda}))\var2spcheck$.}

\medskip

{\bf (b)(i)} Set $S_2=\bigcup_{n\in\Bbb N}\{0,1\}^n$ and let
$\sequencen{(\sigma_n,\tau_n,k_n)}$ enumerate
$S_2\times S_2\times\Bbb N$ with cofinal repetitions.
Let $D$ be the set of those $\alpha\in\BbbN^{\Bbb N}$ such
that

\inset{$\sequencen{k_{\alpha(n)}}$ is strictly increasing,

$\#(\sigma_{\alpha(m)})\le\alpha(n)$ whenever $n\in\Bbb N$ and
$m<k_{\alpha(n+1)}$,

$\sum_{i=k_{\alpha(n)}}^{k_{\alpha(n+1)}-1}
2^{-\#(\sigma_{\alpha(i)})-\#(\tau_{\alpha(i)})}\le 4^{-n}$ for every
$n\in\Bbb N$.}

\noindent For $\alpha\in D$ set

\Centerline{$G_{\alpha}
=\bigcap_{n\in\Bbb N}\bigcup_{m\ge n}\{(u,y):u$, $y\in\{0,1\}^{\omega}$,
$u\supseteq\sigma_{\alpha(m)}$, $y\supseteq\tau_{\alpha(m)}\}$.}

\medskip

\quad{\bf (ii)} For every $\alpha\in D$, $G_{\alpha}$ is negligible for the
product measure on $\{0,1\}^{\omega}\times\{0,1\}^{\omega}$.   \Prf\
For any $n\in\Bbb N$, the measure of $G_{\alpha}$ is at most

$$\eqalign{\sum_{m=k_{\alpha(n)}}^{\infty}
  2^{-\#(\sigma_{\alpha(m)})-\#(\tau_{\alpha(m)})}
&\le\sum_{j=n}^{\infty}\sum_{m=k_{\alpha(j)}}^{k_{\alpha(j+1)}-1}
2^{-\#(\sigma_{\alpha(m)})-\#(\tau_{\alpha(m)})}\cr
&\le\sum_{j=n}^{\infty}4^{-j}
=\Bover43\cdot 4^{-n}.  \text{ \Qed}\cr}$$

\medskip

\quad{\bf (iii)} If $G\subseteq\{0,1\}^{\omega}\times\{0,1\}^{\omega}$ is
negligible, there is an $\alpha\in D$ such that $G\subseteq G_{\alpha}$.
\Prf\ For each $i\in\Bbb N$, let $H_i\supseteq G$ be an open set such that
$(\nu_{\omega}\times\nu_{\omega})(H_i)\le 2^{-i}$;  we can suppose that
$H_i$ is not open-and-closed.   $H_i$ can be expressed as
the union of a sequence of open-and-closed sets;  it can therefore be expressed
as the union of a disjoint sequence of open-and-closed sets;  each of these is
expressible as the union of a disjoint family of sets of the form
$\{u:\sigma\subseteq x\}\times\{y:\tau\subseteq y\}$ where
$\sigma$, $\tau\in S_2$;  so $H_i$ is expressible as
$\bigcup_{j\in\Bbb N}\{u:\sigma'_{ij}\subseteq u\}
\times\{y:\tau'_{ij}\subseteq y\}$, with

\Centerline{$\sum_{j\in\Bbb N}2^{-\#(\sigma'_{ij})-\#(\tau'_{ij})}\le 2^{-i}$.}

\noindent Re-indexing
$\langle (\sigma'_{ij},\tau'_{ij})\rangle_{i,j\in\Bbb N}$ as
$\sequence{m}{(\sigma''_m,\tau''_m)}$, we have

\Centerline{$G\subseteq\bigcap_{n\in\Bbb N}\bigcup_{m\ge n}
\{(u,y):\sigma''_m\subseteq u$, $\tau''_m\subseteq y\}$,}

\noindent and

\Centerline{$\sum_{m\in\Bbb N}2^{-\#(\sigma''_m)-\#(\tau''_m)}
\le\sum_{i=0}^{\infty}2^{-i}<\infty$.}

Let $\gamma:\Bbb N\to\Bbb N$ be a strictly increasing function such
that

\Centerline{$\sum_{m=\gamma(n)}^{\gamma(n+1)-1}
2^{-\#(\sigma''_m)-\#(\tau''_m)}
\le 4^{-n}$}

\noindent for every $n\in\Bbb N$.   Now choose $\sequencen{\alpha(n)}$ so that

\Centerline{$k_{\alpha(n)}=\gamma(n)$,
\quad$\sigma_{\alpha(n)}=\sigma''_n$,
\quad$\tau_{\alpha(n)}=\tau''_n$,
\quad$\alpha(n)\ge\#(\sigma''_m)$ whenever $m<\gamma(n+1)$}

\noindent for each $n\in\Bbb N$.   Then $\alpha\in D$ and
$G\subseteq G_{\alpha}$.\ \Qed

\medskip

\quad{\bf (iv)} Define $h:\BbbN^{\Bbb N}\to\BbbN^{\Bbb N}$ by setting
$h(\beta)(n)=n+\sum_{i=0}^n\beta(i)$ for $\beta\in\BbbN^{\Bbb N}$
and $n\in\Bbb N$.   For $\beta\in\BbbN^{\Bbb N}$ define
$f_{\beta}:\{0,1\}^{\omega}\to\{0,1\}^{\omega}$ by setting
$f_{\beta}(u)(n)=u(h(\beta)(n))$ for $\beta\in\BbbN^{\Bbb N}$ and $n\in\Bbb N$;
note that $f_{\beta}$ is continuous.
For $\alpha$, $\beta\in\BbbN^{\Bbb N}$ say that $\alpha\le^*\beta$ if
$\{n:\beta(n)<\alpha(n)\}$ is finite.

\medskip

\quad{\bf (v)} If $\alpha\in D$, $\beta\in\BbbN^{\Bbb N}$ and
$\alpha\le^*\beta$, then
$C=\{u:u\in\{0,1\}^{\omega}$, $(u,f_{\beta}(u))\in G_{\alpha}\}$ is
$\nu_{\omega}$-negligible.   \Prf\ Let $n_0$ be such that $\alpha(n)\le\beta(n)$
for $n\ge n_0$.   $C$ is just $\bigcap_{n\in\Bbb N}\bigcup_{m\ge n}C_m$
where

\Centerline{$C_m
=\{u:\sigma_{\alpha(m)}\subseteq u,\,
\tau_{\alpha(m)}\subseteq f_{\beta}(u)\}$}

\noindent for each $m$.
We know that $\sequence{j}{k_{\alpha(j)}}$ is strictly increasing;
if $m\ge k_{\alpha(n_0)}$, let $j\ge n_0$ be such that
$k_{\alpha(j)}\le m<k_{\alpha(j+1)}$, and set

$$\eqalign{C'_m
&=\{u:\sigma_{\alpha(m)}\subseteq u,\,
\tau_{\alpha(m)}(i)=f_{\beta}(u)(i)\text{ for }
j\le i<\#(\tau_{\alpha(m)})\}\cr
&=\{u:u(i)=\sigma_{\alpha(m)}(i)\text{ for }i<\#(\sigma_{\alpha(m)}),\cr
&\mskip100mu
u(h(\beta)(i))=\tau_{\alpha(m)}(i)\text{ for }
j\le i<\#(\tau_{\alpha(m)})\}\cr
&\supseteq C_m.\cr}$$

\noindent We know that

\Centerline{$\#(\sigma_{\alpha(m)})\le\alpha(j)\le\beta(j)
\le h(\beta)(i)$}

\noindent whenever $i\ge j$ and that $h(\beta)$ is a
strictly increasing function, so

\Centerline{$\nu_{\omega}C'_m
\le 2^{-\#(\sigma_{\alpha(m)})-\#(\tau_{\alpha(m)})+j}$.}

\noindent But this means that

$$\eqalign{\sum_{m=k_{\alpha(n_0)}}^{\infty}\nu_{\omega}C_m
&=\sum_{j=n_0}^{\infty}
\sum_{m=k_{\alpha(j)}}^{k_{\alpha(j+1)}-1}\nu_{\omega}C_m\cr
&\le\sum_{j=n_0}^{\infty}
2^{j}\sum_{m=k_{\alpha(j)}}^{k_{\alpha(j+1)}-1}
  2^{-\#(\sigma_{\alpha(m)})-\#(\tau_{\alpha(m)})}
\le\sum_{j=n_0}^{\infty}2^{j}\cdot 4^{-j}\cr}$$

\noindent is finite, and $C$ is negligible.\ \Qed

\medskip

\quad{\bf (vi)} Let $\Phi$ be the set of all continuous
functions from $\{0,1\}^{\kappa}$ to $\{0,1\}^{\omega}$, and
$\Cal E$ the set of $(\nu_{\kappa}\times\nu_{\omega})$-negligible sets in
$\Tau_{\kappa}\tensorhat\CalBa_{\omega}$;  let $R$ be the relation

\Centerline{$\{(W,g):W\in\Cal E$, $g\in\Phi$,
$\{x:(x,g(x))\in W\}\in\Cal N(\nu_{\kappa})\}$.}

\noindent Then $(\Cal E,R,\Phi)\prGT(\BbbN^{\Bbb N},\le^*,\BbbN^{\Bbb N})$.
\Prf\ For $W\in\Cal E$ set

$$\eqalign{V_W
&=\{(u,y):u,\,y\in\{0,1\}^{\omega},\cr
&\mskip150mu\{v:v\in\{0,1\}^{\kappa\setminus\omega},\,(u\cup v,y)\in W\}
\text{ is not }\nu_{\kappa\setminus\omega}\text{-negligible}\}.\cr}$$

\noindent Then $V_W$ is $(\nu_{\omega}\times\nu_{\omega})$-negligible;  by
(iii), we can find $\phi(W)\in D$ such that $V_W\subseteq G_{\phi(W)}$.
In the other direction, given $\beta\in\BbbN^{\Bbb N}$, define
$\psi(\beta)\in\Phi$ by saying that $\psi(\beta)(x)=f_{\beta}(x\restr\omega)$
for $x\in\{0,1\}^{\kappa}$.

If $W\in\Cal E$ and $\beta\in\BbbN^{\Bbb N}$
are such that $\phi(W)\le^*\beta$, we have $\nu_{\omega}C=0$ where
$C=\{u:(u,f_{\beta}(u))\in G_{\phi(W)}\}$, by (v).   But if
$C'=\{x:(x,\psi(\beta)(x))\in W\}$, and $u\in\{0,1\}^{\omega}\setminus C$, then

\Centerline{$\{v:v\in\{0,1\}^{\kappa\setminus\omega},\,u\cup v\in C'\}
=\{v:v\in\{0,1\}^{\kappa\setminus\omega},\,(u\cup v,f_{\beta}(u))\in W\}$}

\noindent must be $\nu_{\kappa\setminus\omega}$-negligible, since
$(u,f_{\beta}(u))\notin V_W$.   So $C'$ is negligible and
$(W,\psi(\beta))\in R$.   As $W$ and $\beta$ are arbitrary, $(\phi,\psi)$
is a Galois-Tukey connection and
$(\Cal E,R,\Phi)\prGT(\BbbN^{\Bbb N},\le^*,\penalty-100\BbbN^{\Bbb N})$.\
\Qed

Consequently $\add(\Cal E,R,\Phi)\ge\frak b$ (522C(i), 512Ea, 512Db).
%512Ea add P = add(P,\le,P)
%512Db add and \prGT
%522C(i)  add(\NN,\le^*)=\frak b

\medskip

\quad{\bf (vii)} By 552C, we do not need to distinguish between the
interpretations of $\frak b$ in the ordinary universe and in the
forcing language.   Suppose that $a\in\frak B_{\kappa}^+$ and that $\dot{\Cal A}$ is a
$\BbbPk$-name such that

\Centerline{$a\VVdPk\,\dot{\Cal A}\subseteq\Cal N(\nu_{\omega})$ and
$\#(\dot{\Cal A})<\frak b$.}

\noindent Then there are a $b\in\frak B_{\kappa}^+$, stronger than $a$, a cardinal
$\theta<\frak b$ and a family $\ofamily{\xi}{\theta}{\dot W_{\xi}}$ of
$\BbbPk$-names such that

\Centerline{$b\VVdPk\,\dot{\Cal A}=\{\dot W_{\xi}:\xi<\check\theta\}$.}

\noindent For each $\xi<\theta$,
we have a $(\nu_{\kappa}\times\nu_{\omega})$-negligible
$W_{\xi}\in\Tau_{\kappa}\tensorhat\CalBa_{\omega}$ such that
$b\VVdPk\,\dot W_{\xi}\subseteq\vec W_{\xi}$
(551J, as usual).   Each $W_{\xi}$ belongs to
$\Cal E$.   Since $\theta<\frak b\le\add(\Cal E,R,\Phi)$,
there is a $g\in\Phi$ such that
$(W_{\xi},g)\in R$ for every $\xi<\theta$, that is,
$\{x:(x,g(x))\in W_{\xi}\}$ is negligible for every $\xi<\theta$.
But this means that

\Centerline{$\VVdPk\,\vec g\in\{0,1\}^{\omega}\setminus\vec W_{\xi}$}

\noindent for every $\xi<\theta$.   So

\Centerline{$b\VVdPk\,\vec g\notin\bigcup\dot{\Cal A}$ and $\dot{\Cal A}$
does not cover $\{0,1\}^{\omega}$.}

\noindent As $a$ and $\dot{\Cal A}$ are arbitrary,

\Centerline{$\VVdPk\,\cov\Cal N(\nu_{\omega})\ge\frak b$.}

\medskip

{\bf (c)} Write $\theta$ for the cardinal power
$\kappa^{\omega}$, so that $\VVdPk\,\frak c=\check\theta$ (552B).
\Quer\ If

\Centerline{$\neg\VVdPk\,\cov\Cal N(\nu_{\omega})=\frak c$,}

\noindent then there must be an $a\in\frak B_{\kappa}^+$, a cardinal $\delta<\theta$
and a family $\ofamily{\xi}{\delta}{\dot W_{\xi}}$ of $\BbbPk$-names such
that

\Centerline{$a\VVdPk\,\{\dot W_{\xi}:\xi<\check\delta\}$ is a cover of
$\{0,1\}^{\omega}$ by negligible sets.}

\noindent For each $\xi<\delta$, let
$W_{\xi}\in\Tau_{\kappa}\tensorhat\CalBa_{\omega}$ be a
$(\nu_{\kappa}\times\nu_{\omega})$-negligible set such that
$a\VVdPk\,\dot W_{\xi}\subseteq\vec W_{\xi}$;  expanding it if
necessary, we can suppose that $W_{\xi}$ is a Baire set.   Let
$I_{\xi}\subseteq\kappa$ be a countable set
such that $(u,y)\in W_{\xi}$ whenever
$(x,y)\in W_{\xi}$, $u\in\{0,1\}^{\kappa}$ and
$u\restr I_{\xi}=x\restr I_{\xi}$.   Set

\Centerline{$W'_{\xi}=\{(v,y):(u,y)\in W_{\xi}$, $v\in\{0,1\}^{\kappa}$,
$\{\eta:\eta<\kappa$, $u(\eta)\ne v(\eta)\}\in[I_{\xi}]^{<\omega}\}$.}

\noindent Then $W'_{\xi}$ is still
$(\nu_{\kappa}\times\nu_{\omega})$-negligible.

Because $\kappa\ge\frak c$ and $\delta<\kappa^{\omega}$,
there is a countably infinite $K\subseteq\kappa$ such that $K\cap I_{\xi}$
is finite for every $\xi<\delta$ (5A1Fc).   Enumerate $K$ as
$\sequencen{\eta_n}$ and define $f:\{0,1\}^{\kappa}\to\{0,1\}^{\omega}$ by
setting $f(u)=\sequencen{u(\eta_n)}$ for $u\in\{0,1\}^{\kappa}$.

For each $\xi<\kappa$, $\{u:(u,f(u))\in W_{\xi}\}$ is
$\nu_{\kappa}$-negligible.   \Prf\ Set $J=\kappa\setminus K$,
so that $\{0,1\}^{\kappa}$ can be identified with
$\{0,1\}^J\times\{0,1\}^K$.   Because $I_{\xi}\setminus J$ is finite,
$W'_{\xi}$ is equal to

\Centerline{$\{(v,y):(u,y)\in W_{\xi}$, $v\in\{0,1\}^{\kappa}$,
$\{\eta:\eta<\kappa$, $u(\eta)\ne v(\eta)\}\in[J\cap I_{\xi}]^{<\omega}\}$}

\noindent and can be expressed as
$\{(u,y):(u\restr J,y)\in V\}$ where
$V\subseteq\{0,1\}^J\times\{0,1\}^{\omega}$ must be
negligible.   Now the map $u\mapsto(u\restr J,f(u)):
\{0,1\}^{\kappa}\to\{0,1\}^J\times\{0,1\}^{\omega}$ is just a copy of the
map $u\mapsto(u\restr J,u\restr K)$, so is a measure space isomorphism
between $\{0,1\}^{\kappa}$ and $\{0,1\}^J\times\{0,1\}^{\omega}$, and
$V'=\{u:u\in\{0,1\}^{\kappa}$, $(u\restr J,f(u))\in V\}$ is negligible.
But observe now that

\Centerline{$\{u:(u,f(u))\in W_{\xi}\}
\subseteq\{u:(u,f(u))\in W'_{\xi}\}
=\{u:(u\restr J,f(u))\in V\}=V'$}

\noindent is negligible.\ \Qed

Turn now to 551E.   In the language there, we have
$\Bvalue{\vec f\in\vec W_{\xi}}=0$, that is,
$\VVdPk\,\vec f\notin\vec W_{\xi}$ and $a\VVdPk\,\vec f\notin\dot W_{\xi}$.
So

\Centerline{$a\VVdPk\,
\bigcup_{\xi<\check\delta}\dot W_{\xi}\ne\{0,1\}^{\check\lambda}$,}

\noindent which is impossible.\ \Bang

So we have the result claimed.

\medskip

{\bf (d)(i)}
If $\cf\kappa>\omega$ then
$\sup_{\delta<\kappa}\delta^{\omega}=\kappa^{\omega}$;  but
this means that

\Centerline{$\VVdPk\,\cov\Cal N_{\check\lambda}
\le\frak c
=(\kappa^{\omega})\var2spcheck
=(\sup_{\delta<\kappa}\delta^{\omega})\var2spcheck$.}

\noindent
So henceforth suppose that $\cf\kappa=\omega$.   By 523B, we may also
assume that $\lambda=\omega_1$.
%523B Cichon

\medskip

\quad{\bf (ii)} Let $D$ be the set of all pairs $(\pmb{\xi},y)$ where
$\pmb{\xi}\in\omega_1^{\Bbb N}$ is one-to-one
and $y$ is a Baire measurable function from
$\{0,1\}^{\delta}$ to $\{0,1\}^{\omega}$ for some cardinal $\delta<\kappa$.
Then $\#(D)=\sup_{\delta<\kappa}\delta^{\omega}$ (use 5A4G(b-ii)).
For $(\pmb{\xi},y)\in D$, let
$W_{\pmb{\xi}y}\subseteq\{0,1\}^{\kappa}\times\{0,1\}^{\omega_1}$
be the set

\Centerline{$\{(u,v):\lim_{n\to\infty}
  \Bover1n\#(\{i:i<n,\,v(\xi_i)=y(u\restr\alpha)(i)\})
=\Bover12\}$,}

\noindent where $\pmb{\xi}=\sequence{i}{\xi_i}$ and
$\dom y=\{0,1\}^{\alpha}$.   Then $W_{\pmb{\xi}y}$ is a Baire set;  also
the vertical section
$W_{\pmb{\xi}y}[\{u\}]$ is $\nu_{\omega_1}$-conegligible for almost every
$u\in\{0,1\}^{\kappa}$.   \Prf\ The set

\Centerline{$V=\{x:x\in\{0,1\}^{\Bbb N}$,
$\lim_{n\to\infty}\Bover1n\#(\{i:i<n$, $x(i)=y(u\restr\alpha)(i)\})
=\Bover12\}$}

\noindent is conegligible in $\{0,1\}^{\Bbb N}$, by the strong law of large
numbers (273F).   But $W_{\pmb{\xi}y}[\{u\}]$ is the inverse image of $V$
under the \imp\ map $v\mapsto\sequence{i}{v(\xi_i)}$, so is
$\nu_{\omega_1}$-conegligible.\ \Qed

\medskip

\quad{\bf (iii)} Consequently

\Centerline{$\VVdPk\,\vec W_{\pmb{\xi}y}$ is conegligible in
$\{0,1\}^{\omega_1}$}

\noindent whenever $(\pmb{\xi},y)\in D$ (551I(iii)).   Now

\Centerline{$\VVdPk\,
\bigcap_{(\pmb{\xi},y)\in\check D}\vec W_{\pmb{\xi}y}$ is empty.}

\noindent\Prf\Quer\ Otherwise, there are an $a\in\frak B_{\kappa}^+$ and a
$\BbbPk$-name $\dot x$ such that

\Centerline{$a\VVdPk\,\dot x\in
\bigcap_{(\pmb{\xi},y)\in\check D}\vec W_{\pmb{\xi}y}$.}

\noindent Let $f:\{0,1\}^{\kappa}\to\{0,1\}^{\omega_1}$ be a
$(\Tau_{\kappa},\CalBa_{\omega_1})$-measurable function such
that $a\VVdPk\,\vec f=\dot x$ (551Cc).   Set
$\epsilon=\bover14\bar\nu_{\kappa}a$ and for each $\xi<\omega_1$ let
$E_{\xi}$ be an open-and-closed subset of $\{0,1\}^{\kappa}$ such that
$\bar\nu_{\kappa}(E_{\xi}\symmdiff\{u:f(u)(\xi)=1\})\le\epsilon$.
Let $\alpha_{\xi}<\kappa$ be such that $E_{\xi}$ is determined by
coordinates less than $\alpha_{\xi}$.   Because $\cf\kappa=\omega$,
there is a cardinal $\delta<\kappa$ such that
$A=\{\xi:\alpha_{\xi}\le\delta\}$ is infinite;  let
$\pmb{\xi}=\sequence{i}{\xi_i}$ enumerate a subset of $A$.   For each
$i\in\Bbb N$ let $F_i\subseteq\{0,1\}^{\delta}$ be an open-and-closed set
such that $E_{\xi_i}=\{u:u\restr\delta\in F_i\}$.   Define
$y:\{0,1\}^{\delta}\to\{0,1\}^{\Bbb N}$ by saying that
$y(v)(i)=\chi F_i(v)$ for $v\in\{0,1\}^{\delta}$ and $i\in\Bbb N$;  then
$y$ is Baire measurable, so $W_{\pmb{\xi}y}$ is defined and
$a\VVdPk\,\vec f\in\vec W_{\pmb{\xi}y}$.

Set $H=\{u:(u,f(u))\in W_{\pmb{\xi}y}\}$.   Then

\Centerline{$a\Bsubseteq\Bvalue{\vec f\in\vec W_{\pmb{\xi}y}}
=H^{\ssbullet}$}

\noindent so $\nu_{\kappa}H\ge\bar\nu_{\kappa}a=4\epsilon$.

But consider the sets

$$\eqalign{H_i
&=\{u:f(u)(\xi_i)=y(u\restr\delta)(i)\}
\cr&=\{0,1\}^{\kappa}
  \setminus(\{u:f(u)(\xi_i)=1\}\symmdiff\{u:y(u\restr\delta)(i)=1\})
\cr&=\{0,1\}^{\kappa}\setminus(E_{\xi_i}\symmdiff\{u:f(u)(\xi_i)=1\})\cr}$$

\noindent for $i\in\Bbb N$.   These all have measure at least $1-\epsilon$.
For $n\ge 1$ set

\Centerline{$\gamma_n
=\nu_{\kappa}\{u:\#(\{i:i<n,\,u\in H_i\})\le\Bover{2n}3\}$;}

\noindent  then

\Centerline{$n(1-\epsilon)
\le\sum_{i<n}\nu_{\kappa}H_i
=\int\#(\{i:i<n,\,u\in H_i\})\nu_{\kappa}(du)
\le\Bover{2n}3\gamma_n+n(1-\gamma_n)$}

\noindent and $\gamma_n\le 3\epsilon$.   So

$$\eqalign{H
&=\{u:\lim_{n\to\infty}\Bover1n\#(\{i:i<n,\,u\in H_i\})=\Bover12\}\cr
&\subseteq\bigcup_{m\in\Bbb N}\bigcap_{n\ge m}
  \{u:\#(\{i:i<n,\,u\in H_i\})\le\Bover23\}\cr}$$

\noindent has measure at most $3\epsilon$;  which is impossible.\
\Bang\Qed

\medskip

\quad{\bf (iv)} Thus

\Centerline{$\VVdPk\,
\bigcap_{(\pmb{\xi},y)\in\check D}\vec W_{\pmb{\xi}y}$ is empty and
$\{0,1\}^{\omega_1}$ can be covered by $(\sup_{\delta<\kappa}\delta^{\omega})\var2spcheck$ negligible sets,}

\noindent which is what we needed to know.
}%end of proof of 552G

\vleader{60pt}{552H}{Theorem}
Let $\kappa$ and $\lambda$ be infinite cardinals.

(a) $\VVdPk\,\non\Cal N(\nu_{\check\lambda})
\le(\non\Cal N(\nu_{\lambda}))\var2spcheck$.

\wheader{552H}{0}{0}{0}{36pt}
(b) If $\kappa\ge\max(\lambda,\omega_1)$ then

\Centerline{$\VVdPk\,\non\Cal N(\nu_{\check\lambda})=\omega_1$.}

(c)\cmmnt{ ({\smc Pawlikowski 86})}

\Centerline{$\VVdPk\,\non\Cal N(\nu_{\omega})\le\frak d$.}

\proof{{\bf (a)} Let $A\subseteq\{0,1\}^{\lambda}$ be a non-negligible set of
size $\non\Cal N(\nu_{\lambda})$.   Then 552D tells us that

\Centerline{$\VVdPk\,\check A$ is a non-negligible set of size
$(\non\Cal N(\nu_{\lambda}))\var2spcheck$, so $\non\Cal N(\nu_{\check\lambda})\le(\non\Cal N(\nu_{\lambda}))\var2spcheck$.}

\medskip

{\bf (b)} Put 552E and 537Ba together again, as in part (a) of the proof of
552F.

\medskip

{\bf (c)} Continue the argument from the end of (b-vi) of the proof of
552G above.    We have
$(\Cal E,R,\Phi)\prGT(\BbbN^{\Bbb N},\le^*,\BbbN^{\Bbb N})$, so
$\cov(\Cal E,R,\Phi)\le\frak d$ (522C(i), 512Ea, 512Da).
%522C(i)  cf(\NN,\le^*)=\frak d
%512Ea  cov(P,\le,P)=\cf P
%512Da  cov and \prGT
So there is a
family $\ofamily{\xi}{\frak d}{g_{\xi}}$ in $\Phi$ such that for
every $W\in\Cal E$ there is a $\xi<\frak d$ such that $(W,g_{\xi})\in R$,
that is, $\{x:(x,g_{\xi}(x))\in W\}$ is negligible, that is,
$\VVdPk\,\vec g_{\xi}\notin\vec W$.   Now

\Centerline{$\VVdPk\,\{\vec g_{\xi}:\xi<\frak d\}$ is not negligible.}

\noindent\Prf\Quer\ Otherwise, there are an $a\in\frak B_{\kappa}^+$ and a $\BbbPk$-name
$\dot W$ such that

\Centerline{$a\VVdPk\,\dot W$ is a negligible set containing $\vec g_{\xi}$ for
every $\xi<\frak d$.}

\noindent By 551J once again, there is a $W\in\Cal E$ such that
$a\VVdPk\,\dot W\subseteq\vec W$.   But now we have a $\xi<\frak d$ such that
$\VVdPk\,\vec g_{\xi}\notin\vec W$, which is impossible.\ \Bang\QeD\
So $\VVdPk\,\non\Cal N(\nu_{\omega})\le\frak d$.
}%end of proof of 552H

\leader{552I}{Theorem} Let $\kappa$ and $\lambda$ be infinite cardinals.
Set $\theta_0=\max(\cf\Cal N(\nu_{\omega}),\cff[\kappa]^{\le\omega},
\cff[\lambda]^{\le\omega})$.   Then

\Centerline{$\VVdPk\,\cf\Cal N(\nu_{\check\lambda})=\check\theta_0$.}

\proof{{\bf (a)} $\VVdPk\,\cf\Cal N(\nu_{\omega})
\ge\cff[\check\kappa]^{\le\omega}=(\cff[\kappa]^{\le\omega})\var2spcheck$.
\Prf\ If $\kappa=\omega$ this is trivial.   Otherwise it follows from
552E, % Sierpinski set of size \check\kappa
537B(a-ii) % \VVdPk\,\cf\Cal N(\nu_{\omega})
  %\ge\cff[\check\kappa]^{\le\omega}
and 5A3Nd.\ \Qed

\medskip

{\bf (b)} Set

\Centerline{$\theta_1=\cf\Cal N(\nu_{\lambda})
=\max(\cf\Cal N(\nu_{\omega}),\cff[\lambda]^{\le\omega})$}

\noindent (523N).   Then
$\VVdPk\,\cf\Cal N(\nu_{\check\lambda})\ge\check\theta_1$.
\Prf\Quer\ Otherwise, there are $a\in\frak B_{\kappa}$,
$\theta<\theta_1$ and a family $\ofamily{\xi}{\theta}{\dot W_{\xi}}$ of
$\Bbb P_{\kappa}$-names such that

\Centerline{$a\VVdPk\,\{\dot W_{\xi}:\xi<\check\theta\}$ is a cofinal
family in $\Cal N(\nu_{\check\lambda})$.}

\noindent For each $\xi$ choose a $(\nu_{\kappa}\times\nu_{\lambda})$-negligible
$W_{\xi}\in\Tau_{\kappa}\tensorhat\CalBa_{\lambda}$ such that
$a\VVdPk\,\dot W_{\xi}\subseteq\vec W_{\xi}$.   Then

\Centerline{$V_{\xi}
=\{y:y\in\{0,1\}^{\lambda}$, $W_{\xi}^{-1}[\{y\}]$ is not
$\nu_{\kappa}$-negligible$\}$}

\noindent is $\nu_{\lambda}$-negligible.   Because
$\theta<\cf\Cal N(\nu_{\lambda})$, there is a $V\in\Cal N(\nu_{\lambda})$ such
that $V\not\subseteq V_{\xi}$ for every $\xi<\theta$, and (enlarging $V$
slightly if necessary) we can arrange that
$V\in\CalBa_{\lambda}$.

Set $W=\{0,1\}^{\kappa}\times V$.   Then
$W\in\Tau_{\kappa}\tensorhat\CalBa_{\lambda}$ and every vertical
section of $W$ is negligible, so 551I tells us that

\Centerline{$\VVdPk\,\vec W$ is negligible in $\{0,1\}^{\check\lambda}$.}

\noindent Accordingly

\Centerline{$a\VVdPk$ there is a $\xi<\check\theta$ such that
$\vec W\subseteq\dot W_{\xi}\subseteq\vec W_{\xi}$,}

\noindent and there must be a $b\in\frak B_{\kappa}$,
stronger than $a$, and a
$\xi<\theta$ such that $b\VVdPk\,\vec W\subseteq\vec W_{\xi}$.   But now take
any point $y$ of $V\setminus V_{\xi}$ and consider the constant function
$e_y$ on $\{0,1\}^{\kappa}$ with value $y$.   Then
$\{x:(x,e_y(x))\in W\setminus W_{\xi}\}$
is conegligible, so 551E tells us that

\Centerline{$\VVdPk\,\vec e_y\in\vec W\setminus\vec W_{\xi}$, so
$\vec W\not\subseteq\vec W_{\xi}$,}

\noindent contrary to the choice of $\xi$.\ \Bang\Qed

\medskip

{\bf (c)} Now

\Centerline{$\VVdPk\,\cf\Cal N(\nu_{\check\lambda})
\ge\max(\cf\Cal N(\nu_{\omega}),\check\theta_1)
\ge\max(\cff[\check\kappa]^{\le\omega},\check\theta_1)
=\check\theta_0$.}

\medskip

{\bf (d)} In the other direction, let
$\mu=\nu_{\kappa}\times\nu_{\lambda}$ be the product measure on
$\{0,1\}^{\kappa}\times\{0,1\}^{\lambda}$.   Again by 523N,
$\cf\Cal N(\mu)=\theta_0$;  let $\ofamily{\xi}{\theta_0}{W_{\xi}}$ be a
cofinal family in $\Cal N(\mu)$ consisting of sets in
$\Tau_{\kappa}\tensorhat\CalBa_{\lambda}$.   By 551J,

\Centerline{$\VVdPk\,\{\vec W_{\xi}:\xi<\check\theta_1\}$ is cofinal with
$\Cal N(\nu_{\check\lambda})$,
so $\cf\Cal N(\nu_{\check\lambda})\le\check\theta_0$.}

\noindent Putting this together with (c),

\Centerline{$\VVdPk\,\cf\Cal N(\nu_{\check\lambda})=\check\theta_0$,}

\noindent and the proof is complete.
}%end of proof of 552I

\leaveitout{
\leader{}{Proposition} If $\frakmctbl=\theta_1$ and $\non\Cal M=\theta_2$
then

\Centerline{$\VVdash_{\Bbb P_{\omega}}\frakmctbl=\check\theta_1$ and
$\non\Cal M=\check\theta_2$.}

\proof{{\smc Bartoszy\'nski \& Judah 95}, 3.2.43.
}
}%end of leaveitout

\leader{552J}{Theorem} Let $\kappa$ and $\lambda$ be infinite cardinals;
set $\theta_0=\shr\Cal N(\nu_{\lambda})$ and let
$\theta_1$ be the cardinal power $\lambda^{\omega}$.
Then

\Centerline{$\VVdPk\,\check\theta_0
\le\shr\Cal N(\nu_{\check\lambda})\le\check\theta_1$.}

\proof{{\bf (a)} \Quer\ Suppose, if possible, that

\Centerline{$\neg\VVdPk\,
\check\theta_0\le\shr\Cal N(\nu_{\check\lambda})$.}

\noindent Then there are an
$a\in\frak B_{\kappa}^+$ and a cardinal $\theta'<\theta_0$ such that

\Centerline{$a\VVdPk\,\shr\Cal N(\nu_{\check\lambda})=\check\theta'$.}

\noindent Of course $\theta'$ is infinite.
Let $A\subseteq\{0,1\}^{\lambda}$ be such that
$\nu_{\lambda}^*A>0$ but $B\in\Cal N(\nu_{\lambda})$ for every
$B\in[A]^{\le\theta'}$.   By 552D,

\Centerline{$\VVdPk\,\nu_{\check\lambda}^*(\check A)>0$.}

\noindent
There must therefore be a $\BbbPk$-name $\dot B$ for a
subset of $\check A$ of size at most $\check\theta'$ such that

\Centerline{$a\VVdPk\,\nu_{\check\lambda}^*(\dot B)>0$.}

\noindent By 5A3Nc there is a $B\subseteq A$ such that
$\#(B)\le\max(\omega,\theta')=\theta'$ and

\Centerline{$a\VVdPk\,\dot B\subseteq\check B$.}

\noindent By 552D, in the other direction,

\Centerline{$a\VVdPk\,0<\nu_{\check\lambda}^*(\dot B)
\le\nu_{\check\lambda}^*(\check B)
=(\nu_{\lambda}^*B)\var2spcheck$}

\noindent and $\nu_{\lambda}^*B>0$, contrary to the choice of $A$.\ \Bang

\medskip

{\bf (b)} (In this part of the
proof it will be convenient to regard $\frak B_{\kappa}$ as the measure
algebra of $\nu_{\kappa}\restr\CalBa_{\kappa}$.)

\medskip

\quad{\bf (i)} \Quer\
Suppose, if possible, that

\Centerline{$\neg\VVdPk\,
\shr\Cal N(\nu_{\check\lambda})\le\check\theta_1$.}

\noindent Then there are an $a\in\frak B_{\kappa}^+$
and a $\BbbPk$-name $\dot A$ such that

\doubleinset{$a
\VVdPk\,\dot A$ is a non-negligible subset of $\{0,1\}^{\check\lambda}$
and every subset of $\dot A$ with cardinal at most
$\check\theta_1$ is negligible.}

\medskip

\quad{\bf (ii)} Let $\ofamily{\xi}{\kappa}{e_{\xi}}$
be the standard generating family in $\frak B_{\kappa}$.
Choose $\ofamily{\xi}{\theta_1^+}{f_{\xi}}$,
$\ofamily{\xi}{\theta_1^+}{J_{\xi}}$, $\ofamily{\xi}{\theta_1^+}{K_{\xi}}$,
$\ofamily{\xi}{\theta_1^+}{W_{\xi}}$ and $\ofamily{\xi}{\theta_1^+}{V_{\xi}}$
inductively, as follows.   $K_{\xi}=\bigcup_{\eta<\xi}J_{\eta}$.
Given that $\xi<\theta_1^+$ and that, for each $\eta<\xi$,
$f_{\eta}:\{0,1\}^{\kappa}\to\{0,1\}^{\lambda}$ is a
$(\CalBa_{\kappa},\CalBa_{\lambda})$-measurable function such that
$a\VVdPk\,\vec f_{\eta}\in\dot A$ (where $\vec f_{\eta}$ is
the $\Bbb P_{\kappa}$-name for a member of $\{0,1\}^{\check\lambda}$
as defined in 551Cb), then

\Centerline{$a\VVdPk\,\{\vec f_{\eta}:\eta<\check\xi\}$ is negligible,}

\noindent so by 551J there is a set
$W_{\xi}\in\CalBa_{\kappa}\tensorhat\CalBa_{\lambda}$,
negligible for the product measure on
$\{0,1\}^{\kappa}\times\{0,1\}^{\lambda}$,
such that $a\VVdPk\,\vec f_{\eta}\in\vec W_{\xi}$ for every $\eta<\xi$.

Set

$$\eqalign{V_{\xi}
&=\{(x,y):x\in\{0,1\}^{\kappa},\,
y\in\{0,1\}^{\lambda},\cr
&\mskip100mu\{t:t\in\{0,1\}^{\kappa\setminus K_{\xi}},\,
  ((x\restr K_{\xi})\cup t,y)\in W_{\xi}\}\text{ is
not }\nu_{\kappa\setminus K_{\xi}}\text{-negligible}\}.\cr}$$

\noindent Then
$V_{\xi}\in\CalBa_{\kappa}\tensorhat\CalBa_{\lambda}$ is
negligible, so $\VVdPk\,\vec V_{\xi}\in\Cal N(\nu_{\check\lambda})$ and
$a\VVdPk\,\dot A\not\subseteq\vec V_{\xi}$;  let
$f_{\xi}:\{0,1\}^{\kappa}\to\{0,1\}^{\lambda}$ be a
$(\CalBa_{\kappa},\CalBa_{\lambda})$-measurable function such that
$a\VVdPk\,\vec f_{\xi}\in\dot A\setminus\vec V_{\xi}$ (551Cc).   Let
$J_{\xi}\subseteq\kappa$ be a set with cardinal at most $\lambda$ such that
$\{x:f_{\xi}(x)(\zeta)=1\}$ is determined by coordinates in $J_{\xi}$ for every
$\zeta<\lambda$, and continue.

\medskip

\quad{\bf (iii)} If $\eta<\xi<\theta_1^+$ then
$a\VVdPk\,\vec f_{\eta}\in\vec V_{\xi}$.
\Prf\ As $J_{\eta}\subseteq K_{\xi}$,
we have a function $g:\{0,1\}^{K_{\xi}}\to\{0,1\}^{\lambda}$ such that
$f_{\eta}(x)=g(x\restr K_{\xi})$ for every $x\in\{0,1\}^{\lambda}$.   Now, for
any $s\in\{0,1\}^{K_{\xi}}$ and $y\in\{0,1\}^{\lambda}$, the set

\Centerline{$\{t:t\in\{0,1\}^{\kappa\setminus K_{\xi}}$,
$(s\cup t,y)\in W_{\xi}\setminus V_{\xi}\}$}

\noindent is $\nu_{\kappa\setminus K_{\xi}}$-negligible, so

\Centerline{$E=\{(s,t):(s\cup t,g(s))\in W_{\xi}\setminus V_{\xi}\}$}

\noindent is
$(\nu_{K_{\xi}}\times\nu_{\kappa\setminus K_{\xi}})$-negligible.
($E\in\CalBa_{K_{\xi}}\tensorhat\CalBa_{\kappa\setminus K_{\xi}}$ because
$W_{\xi}$ and $V_{\xi}$ belong to
$\CalBa_{\kappa}\tensorhat\CalBa_{\lambda}$
and $g$ is $(\CalBa_{K_{\xi}},\penalty-100\CalBa_{\lambda})$-measurable.)
But if we
identify $\{0,1\}^{K_{\xi}}\times\{0,1\}^{\kappa\setminus K_{\xi}}$
with $\{0,1\}^{\kappa}$, then
$E$ becomes $\{x:(x,f_{\eta}(x))\in W_{\xi}\setminus V_{\xi}\}$.   Now

$$\eqalignno{a
&\Bsubseteq\Bvalue{\vec f_{\eta}\in\vec W_{\xi}}
=\{x:(x,f_{\eta}(x))\in W_{\xi}\}^{\ssbullet}\cr
\displaycause{551Ea}
&\Bsubseteq\{x:(x,f_{\eta}(x))\in V_{\xi}\}^{\ssbullet}
=\Bvalue{\vec f_{\eta}\in\vec V_{\xi}},\cr}$$

\noindent and $a\VVdPk\,\vec f_{\eta}\in\vec V_{\xi}$.\ \Qed

\medskip

\quad{\bf (iv)} For each $\xi<\theta_1^+$, $V_{\xi}$ factors through
$\{0,1\}^{K_{\xi}}\times\{0,1\}^{\lambda}$ and belongs to
$\CalBa_{\kappa}\tensorhat\CalBa_{\lambda}$.
There is therefore a countable set $L_{\xi}\subseteq K_{\xi}$
such that $V_{\xi}$ factors through
$\{0,1\}^{L_{\xi}}\times\{0,1\}^{\lambda}$.   Let $S$ be the set
$\{\xi:\xi<\theta_1^+$, $\cf\xi>\omega\}$.
Because $\theta_1\ge\omega_1$, $S$ is stationary in $\theta_1^+$
(5A1Ac).   For each $\xi\in S$, let $g(\xi)<\xi$ be such that
$L_{\xi}\subseteq K_{g(\xi)}$.   By the Pressing-Down Lemma there is a
$\gamma<\theta_1^+$ such that $S'=\{\xi:\xi\in S$, $g(\xi)=\gamma\}$ is
stationary.

For $\xi\in S'$, we have a
$V'_{\xi}\in\CalBa_{K_{\gamma}}\tensorhat\CalBa_{\lambda}$ such that

\Centerline{$V_{\xi}
=\{(x,y):x\in\{0,1\}^{\kappa}$, $y\in\{0,1\}^{\lambda}$,
$(x\restr K_{\gamma},y)\in V'_{\xi}\}$.}

\noindent But $\#(K_{\gamma})\le\lambda$, so

\Centerline{$\#(\CalBa_{K_{\gamma}}\tensorhat\CalBa_{\lambda})
\le\lambda^{\omega}=\theta_1<\#(S')$,}

\noindent and there are $\xi$, $\eta\in S'$ such that $\eta<\xi$ and
$V'_{\eta}=V'_{\xi}$ and $V_{\eta}=V_{\xi}$.   But also

\Centerline{$a\VVdPk\,\vec f_{\eta}\in\vec V_{\xi}\setminus\vec V_{\eta}$,}

\noindent so this is impossible.\ \Bang
}%\end of proof of 552J

\leader{552K}{Lemma} Let $I$ be a set.   Let
$q:\Fn_{<\omega}(I;\{0,1\})\to\coint{0,\infty}$ be a
function such that $q(\emptyset)=1$ and

\Centerline{$q(z)=q(z\cup\{(i,0)\})+q(z\cup\{(i,1)\})$}

\noindent whenever $z\in\Fn_{<\omega}(I;\{0,1\})$ and
$i\in I\setminus\dom z$.   Then there is a
unique Radon measure $\mu$ on $\{0,1\}^I$ such that

\Centerline{$\mu\{x:z\subseteq x\in\{0,1\}^I\}=q(z)$}

\noindent for every $z\in\Fn_{<\omega}(I;\{0,1\})$.

\proof{{\bf (a)} For each $K\in[I]^{<\omega}$,
let $\mu_K$ be the measure on the
finite set $\{0,1\}^K$ defined by saying that
$\mu_KA=\sum_{z\in A}q(z)$  for every $A\subseteq\{0,1\}^K$.   For
$K\subseteq L\in[I]^{<\omega}$ set $f_{KL}(z)=z\restr K$ for
$z\in\{0,1\}^L$;  then $f_{KL}$ is \imp\ for $\mu_K$ and $\mu_L$.
\Prf\ It is enough to consider the case $L=K\cup\{i\}$ where
$i\in I\setminus K$.   In this case, for $A\subseteq\{0,1\}^K$,

$$\eqalign{\mu_Lf^{-1}[A]
&=\sum_{w\in f^{-1}[A]}q(w)
=\sum_{z\in A}q(z\cup\{(i,0)\})+q(z\cup\{(i,1)\})\cr
&=\sum_{z\in A}q(z)
=\mu_KA.  \text{ \Qed}\cr}$$

\medskip

{\bf (b)}
Let $\Cal E$ be the algebra of open-and-closed sets in $\{0,1\}^I$, that
is, the family $\{f_K^{-1}[A]:K\in[I]^{<\omega}$, $A\subseteq\{0,1\}^K\}$,
where $f_K(x)=x\restr K$ for $x\in\{0,1\}^I$.   Then we can define a
functional $\nu:\Cal E\to[0,1]$ by setting

\Centerline{$\nu f_K^{-1}[A]=\mu_KA$ whenever $K\in[I]^{<\omega}$,
$A\subseteq\{0,1\}^K$;}

\noindent by (a), this is well-defined.
By 416Qa, there is a unique Radon measure $\mu$ on
$\{0,1\}^I$ extending $\nu$, so that

\Centerline{$\mu\{x:z\subseteq x\}=\nu\{x:z\subseteq x\}=\mu_K\{z\}=q(z)$}

\noindent whenever $K\subseteq I$ is finite and $z\in\{0,1\}^K$.
}%end of proof of 552K

\leader{552L}{Lemma} Let $\theta$ be a regular infinite cardinal such
that the cardinal power $\delta^{\omega}$ is less than $\theta$ for every
$\delta<\theta$, and
$S\subseteq\theta$ a stationary set such that $\cf\xi>\omega$ for every
$\xi\in S$.   Let $\ofamily{\xi}{\theta}{M_{\xi}}$
be a family of sets of size less than $\theta$, and $I$ a set of size less
than $\theta$;  suppose that for each $i\in I$ we are given a function
$f_i$ with domain $S$ such that $f_i(\xi)\in\bigcup_{\eta<\xi}M_{\eta}$ for
every $\xi\in S$.   Then there is an $\omega_1$-complete filter $\Cal F$
on $\theta$, containing every closed cofinal subset of $\theta$, such that
for every $i\in I$ there is a $D\in\Cal F$ such that $D\subseteq S$ and
$f_i$ is constant on $D$.

\proof{{\bf (a)}
Set $M=\bigcup_{\xi<\theta}M_{\xi}$, so that $\#(M)\le\theta$;
let $\ofamily{\xi}{\theta}{x_{\xi}}$ run over $M$.   Set

\Centerline{$F^*=\{\xi:\xi<\theta$,
$\bigcup_{\eta<\xi}M_{\eta}=\{x_{\eta}:\eta<\xi\}\}$;}

\noindent then $F^*$ is a closed cofinal subset of $\theta$, because
$\theta$ is regular and uncountable.   Set $S_1=S\cap F^*$, so that $S_1$
is stationary.   For $\xi\in S_1$ and $i\in I$
let $h_{\xi}(i)<\xi$ be such that $f_i(\xi)=x_{h_{\xi}(i)}$.
For $J\in[I]^{\le\omega}$, $\xi\in S_1$ set

\Centerline{$D_{\xi J}
=\{\eta:\eta\in S\cap F^*$, $h_{\eta}\restr J=h_{\xi}\restr J\}$.}

\medskip

{\bf (b)} There is a $\xi\in S_1$ such that
$D_{\xi J}\cap F\ne\emptyset$ for every closed cofinal set
$F\subseteq\theta$ and every $J\in[I]^{\le\omega}$.   \Prf\Quer\ Otherwise,
for each $\xi\in S_1$ choose $J_{\xi}\in[I]^{\le\omega}$ and
a closed cofinal set $F_{\xi}$ not meeting $D_{\xi J_{\xi}}$.
Let $F$ be the diagonal intersection
$\{\xi:\xi<\theta$, $\xi\in F_{\eta}$ whenever $\eta\in S_1\cap\xi\}$,
so that $F$ is a closed cofinal set (4A1B(c-ii))
and $S_2=S_1\cap F$ is stationary.
For $\xi\in S_2$ let $g(\xi)<\xi$ be such that
$h_{\xi}[J_{\xi}]\subseteq g(\xi)$.   Then there is
a $\gamma<\theta$ such that $S_3=\{\xi:\xi\in S_2$, $g(\xi)=\gamma\}$ is
stationary, by the Pressing-Down Lemma (4A1Cc).   Now
$h_{\xi}\restr J_{\xi}\in[I\times\gamma]^{\le\omega}$ for every
$\xi\in S_1$, and $\#([I\times\gamma]^{\le\omega})
\le\max(\#(I),\gamma,\omega)^{\omega}<\theta$,
so there are $\xi$, $\eta\in S_3$ such that
$h_{\xi}\restr J_{\xi}=h_{\eta}\restr J_{\eta}$ and $\eta<\xi$.
But in this case we have $\xi\in F_{\eta}\cap D_{\eta J_{\eta}}$, which is
supposed to be impossible.\ \Bang\Qed

\medskip

{\bf (c)} If now $\sequencen{F_n}$ is any sequence of closed cofinal sets
in $\theta$, and $\sequencen{J_n}$ is any sequence in $[I]^{\le\omega}$,

\Centerline{$\bigcap_{n\in\Bbb N}D_{\xi J_n}\cap F_n=D_{\xi J}\cap F$}

\noindent is non-empty, where $J=\bigcup_{n\in\Bbb N}J_n$ and
$F=\bigcap_{n\in\Bbb N}F_n$.   So we have an $\omega_1$-complete filter
$\Cal F$ on $\theta$ generated by

\Centerline{$\{D_{\xi J}:J\in[I]^{\le\omega}\}
\cup\{F:F\subseteq\theta$ is closed and cofinal$\}$.}

\noindent If $i\in I$ then $f_i$ is constant on $D_{\xi,\{i\}}\in\Cal F$,
so we're done.
}%end of proof of 552L

\vleader{72pt}{552M}{Proposition} Let $\kappa$ and $\lambda$ be infinite
cardinals.   Then the following are equiveridical:

(i) if $\Cal A\subseteq\Cal P(\{0,1\}^{\kappa})$ and
$\#(\Cal A)\le\lambda$ then there is an extension of
$\nu_{\kappa}$ to a measure measuring every member of $\Cal A$;

(ii) for every function
$f:\{0,1\}^{\kappa}\to\{0,1\}^{(\kappa+\lambda)\setminus\kappa}$,
there is a Baire measure $\mu$ on $\{0,1\}^{\kappa+\lambda}$ such that
$\mu\{y:y\in\{0,1\}^{\kappa+\lambda}$, $z\subseteq y\}=2^{-\#(K)}$
whenever $K\in[\kappa]^{<\omega}$ and $z\in\{0,1\}^K$, and
$\mu^*\{x\cup f(x):x\in\{0,1\}^{\kappa}\}=1$;

(iii) if $(X,\Sigma,\mu)$ is a locally compact\cmmnt{ (definition:
342Ad)} semi-finite measure space with Maharam type at most $\kappa$,
$\Cal A\subseteq\Cal PX$ and $\#(\Cal A)\le\lambda$, then there is an
extension of $\mu$ to a measure measuring every member of $\Cal A$.

\proof{{\bf (i)$\Rightarrow$(ii)} Assume (i).   If
$f:\{0,1\}^{\kappa}\to\{0,1\}^{(\kappa+\lambda)\setminus\kappa}$
is a function, set
$\Cal A=\{\{x:f(x)(\xi)=1\}:\kappa\le\xi<\kappa+\lambda\}$, so that
$\Cal A$ is a family of subsets of $\{0,1\}^{\kappa}$ and
$\#(\Cal A)\le\lambda$.   Let $\nu$ be a measure on $\{0,1\}^{\kappa}$,
extending $\nu_{\kappa}$ and measuring every member of $\Cal A$.
Then $\{x:(x\cup f(x))(\xi)=1\}\in\dom\nu$ for every
$\xi\in\kappa+\lambda$, so we have a Baire measure $\mu$ on
$\{0,1\}^{\kappa+\lambda}$ defined by saying that
$\mu E=\nu\{x:x\cup f(x)\in E\}$ for Baire sets
$E\subseteq\{0,1\}^{\kappa+\lambda}$.   If $K\in[\kappa]^{<\omega}$ and
$z\in\{0,1\}^{\kappa}$, then

\Centerline{$\mu\{y:z\subseteq y\}=\nu\{x:z\subseteq x\cup f(x)\}
=\nu\{x:z\subseteq x\}=\nu_{\kappa}\{x:z\subseteq x\}
=2^{-\#(K)}$;}

\noindent while if $E\in\CalBa_{\kappa+\lambda}$ and
$x\cup f(x)\in E$ for every
$x\in\{0,1\}^{\kappa}$, then $\mu E=\nu\{0,1\}^{\kappa}=1$, so
$\mu^*\{x\cup f(x):x\in\{0,1\}^{\kappa}\}=1$.   Thus (ii) is true.

\medskip

{\bf (ii)$\Rightarrow$(i)} Assume (ii).   Let $\Cal A$ be a family of
subsets of $\{0,1\}^{\kappa}$ with $\#(\Cal A)\le\lambda$.   Let
$\ofamily{\eta}{\lambda}{A_{\eta}}$ run over $\Cal A\cup\{\emptyset\}$.
Define $f:\{0,1\}^{\kappa}\to\{0,1\}^{(\kappa+\lambda)\setminus\kappa}$
by saying that
$f(x)(\kappa+\eta)=(\chi A_{\eta})(x)$ whenever $\eta<\lambda$ and
$x\in\{0,1\}^{\kappa}$.   Let $\mu$ be a Baire measure on
$\{0,1\}^{\kappa+\lambda}$ satisfying the conditions of (ii).   Set
$g(x)=x\cup f(x)$ for $x\in\{0,1\}^{\kappa}$.
Because $g[\{0,1\}^{\kappa}]$ has full outer measure for
$\mu$, we have a measure $\nu$ on $\{0,1\}^{\kappa}$ such that
$\nu g^{-1}[E]=\mu E$ for every Baire set
$E\subseteq\{0,1\}^{\kappa+\lambda}$ (234F);
let $\hat\nu$ be the completion of $\nu$.   Now
$A_{\eta}=g^{-1}[\{y:y(\kappa+\eta)=1\}]$ is measured by $\nu$ and
$\hat\nu$.   Also

\Centerline{$\hat\nu\{x:z\subseteq x\}
=\nu\{x:z\subseteq x\}=\nu\{x:z\subseteq g(x)\}
=\mu\{y:z\subseteq y\}=2^{-\#(K)}$}

\noindent whenever
$K\in[\kappa]^{<\omega}$ and $z\in\{0,1\}^{\kappa}$, so $\hat\nu$ extends
$\nu_{\kappa}$ (254G) and is an extension of $\nu_{\kappa}$ measuring every
member of $\Cal A$.

\medskip

{\bf (i)$\Rightarrow$(iii)} Suppose that (i) is true.

\medskip

\quad\grheada\ Let $(X,\Sigma,\mu)$ be a compact probability space of
Maharam
type at most $\kappa$, and $\Cal A$ a family of subsets of $X$ of size at
most $\lambda$.   Then there is a function $h:\{0,1\}^{\kappa}\to X$ which
is \imp\ for $\nu_{\kappa}$ and $\mu$.   \Prf\ By 332P, the measure algebra
of $\mu$ can be embedded into $\frak B_{\kappa}$;  by 343B, this embedding
can be realized by an \imp\ function from $\{0,1\}^{\kappa}$ to $X$.\ \QeD\
Now $\Cal C=\{h^{-1}[A]:A\in\Cal A\}$ has cardinal at most $\lambda$,
so there is
an extension $\nu$ of $\nu_{\kappa}$ measuring every member of $\Cal C$;
and the image measure $\nu h^{-1}$ extends $\mu$ and measures every member
of $\Cal A$.

\medskip

\quad\grheadb\ It follows at once that if $(X,\Sigma,\mu)$ is a compact
totally finite measure space with Maharam
type at most $\kappa$, and $\Cal A$ a family of subsets of $X$ of size at
most $\lambda$, then $\mu$ can be extended to every member of $\Cal A$.
(If $\mu X=0$ this is trivial, and otherwise we can apply ($\alpha$) to a
scalar multiple of $\mu$.)

\medskip

\quad\grheadc\ Now suppose that $(X,\Sigma,\mu)$ is a locally compact
semi-finite measure space with Maharam
type at most $\kappa$, and $\Cal A$ a family of subsets of $X$ of size at
most $\lambda$.   In the measure algebra $(\frak A,\bar\mu)$ of $\mu$,
let $D$ be a
partition of unity consisting of elements of finite measure;  for $d\in D$
choose $E_d\in\Sigma$ such that $E_d^{\ssbullet}=d$.   If $G\in\Sigma$ then

\Centerline{$\mu G=\bar\mu G^{\ssbullet}
=\sum_{d\in D}\bar\mu(d\Bcap G^{\ssbullet})=\sum_{d\in D}\mu(E_d\cap G)$.}

\noindent For each $d\in D$, the subspace measure $\mu_{E_d}$ on $E_d$ is
compact and totally finite and has Maharam type at most $\kappa$ (put
331Hc and 322Ja together),
so by ($\beta$) can be extended to a measure
$\mu'_{E_d}$ measuring $A\cap E_d$ for every $A\in\Cal A$.   Set
$\mu'F=\sum_{d\in D}\mu'_{E_d}(F\cap E_d)$ whenever $F\subseteq X$ is such
that the sum is defined;  then $\mu'$ is a measure on $X$, extending $\mu$
and measuring every set in $\Cal A$, as required.

\medskip

{\bf (iii)$\Rightarrow$(i)} is trivial.
}%end of proof of 552M

\leader{552N}{Theorem}\cmmnt{ ({\smc Carlson 84})} Let $\kappa$ and
$\lambda$ be infinite cardinals such that $\kappa$ is greater than the
cardinal power $\lambda^{\omega}$.   Then

\doubleinset{$\VVdPk$ if $\Cal A\subseteq\Cal P(\{0,1\}^{\check\kappa})$
and $\#(\Cal A)\le\check\lambda$, there is an extension of
$\nu_{\check\kappa}$ to a measure measuring every member of $\Cal A$.}

%what if we have a family $\Cal A$ expressible as the union of
%at most  \check\lambda  well-ordered sets?  see 214P

\proof{{\bf (a)}
Let $\langle e_{\xi\zeta}\rangle_{\xi,\zeta<\kappa}$
be a re-indexing of the standard generating family in $\frak B_{\kappa}$.
For $J\subseteq\kappa\times\kappa$ let $\frak C_J$ be the closed subalgebra of
$\frak B_{\kappa}$ generated by $\{e_{\xi\zeta}:(\xi,\zeta)\in J\}$.
Recall that $\#(L^{\infty}(\frak C_J))\le\max(\omega,\#(J)^{\omega})$
for every $J$ (524Ma, 515Mb);
%515Mb  \#(L^0)=\/(\frak A)
we shall also need to know that
every element of $\frak B_{\kappa}$ belongs to $\frak C_J$ for
a smallest $J\subseteq\kappa\times\kappa$, and this
$J$ is countable (254Rc, 531Jb).

Set
$I=(\kappa+\lambda)\setminus\kappa$, where $\kappa+\lambda$ is the ordinal
sum, so that $I$ is disjoint from $\kappa$ and $\#(I)=\lambda$.
Let $\dot f$ be a $\BbbPk$-name for a function
from $\{0,1\}^{\check\kappa}$ to $\{0,1\}^{\check I}$.

For each $\xi<\kappa$ let $\dot x_{\xi}$ be a $\BbbPk$-name for a member of
$\{0,1\}^{\check\kappa}$ such that
$\Bvalue{\dot x_{\xi}(\check\zeta)=1}=e_{\xi\zeta}$
for every $\zeta<\kappa$.   For
$z\in\Fn_{<\omega}(\kappa+\lambda;\{0,1\})$ and $\xi<\kappa$, set

\Centerline{$a_{\xi z}
=\Bvalue{\check z\subseteq\dot x_{\xi}\cup\dot f(\dot x_{\xi})}$}

\noindent and let
$J_{\xi z}\subseteq\kappa\times\kappa$ be the smallest set such that
$a_{\xi z}\in\frak C_{J_{\xi z}}$.   Note that

\Centerline{$a_{\xi z}
=a_{\xi,z\restr I}
  \Bcap\inf_{\zeta\in\kappa\cap z^{-1}[\{1\}]}e_{\xi\zeta}
  \Bsetminus\sup_{\zeta\in\kappa\cap z^{-1}[\{0\}]}e_{\xi\zeta}$,}

\noindent so that

\Centerline{$J_{\xi z}
\subseteq J_{\xi,z\restr I}\cup(\{\xi\}\times\dom z)$.}

Set $\theta=(\lambda^{\omega})^+\le\kappa$.
For $\xi\le\theta$ let $L_0(\xi)\subseteq\kappa$ be the smallest set
such that $\xi\subseteq L_0(\xi)$ and
$J_{\eta w}\subseteq L_0(\xi)\times L_0(\xi)$
for every $\eta<\xi$ and $w\in\Fn_{<\omega}(I;\{0,1\})$;  set
$L(\xi)=L_0(\xi)\times L_0(\xi)$.   Then
$\#(L(\xi))\le\max(\omega,\lambda,\#(\xi))<\theta$ for every
$\xi<\theta$, and $L(\xi)=\bigcup_{\eta<\xi}L(\eta)$ for limit
$\xi\le\theta$.   Set

\Centerline{$D^*=\{\xi:\xi<\theta$ is a limit ordinal,
$\xi>\sup(\theta\cap L_0(\eta))$ for every $\eta<\xi\}$;}

\noindent then $D^*$ is a closed cofinal subset of $\theta$,
and $\xi\notin L_0(\xi)$ for every $\xi\in D^*$.   So

\Centerline{$S=\{\xi:\xi\in D^*$, $\cf(\xi\cap D^*)\ge\omega_1\}$}

\noindent is stationary in $\theta$.

\medskip

{\bf (b)} For $J\subseteq\kappa\times\kappa$, let
$P_J:L^1(\frak B_{\kappa})
\to L^1(\frak C_J)\subseteq L^1(\frak B_{\kappa})$ be the
conditional expectation operator defined by saying that
$P_Ju\in L^1(\frak C_J)$ and
$\int_cP_Ju=\int_cu$ whenever $c\in\frak C_J$ and
$u\in L^1(\frak B_{\kappa})$ (254R, 365Q\formerly{3{}65R}).
We need to know that $P_{J\cap J'}=P_JP_{J'}$ for all $J$,
$J'\subseteq\kappa$ (254Ra), and that $P_J(u\times v)=u\times P_Jv$
whenever $u\in L^1(\frak C_J)$,
$v\in L^1(\frak B_{\kappa})$ and $u\times v\in L^1(\frak B_{\kappa})$
(242L).   It follows that if $J$, $J'$,
$J''\subseteq\kappa\times\kappa$, $u\in L^1(\frak C_J)$ and
$J\cap J'=J\cap J''$, then

\Centerline{$P_{J'}(u)=P_{J'}P_J(u)=P_{J\cap J'}(u)=P_{J\cap J''}(u)
=P_{J''}(u)$.}

If $h:\kappa\times\kappa\to\kappa\times\kappa$ is any permutation, then we
have a corresponding measure-preserving automorphism
$\pi:\frak B_{\kappa}\to\frak B_{\kappa}$ defined by saying that
$\pi e_{\xi\zeta}=e_{\xi'\zeta'}$ if $(\xi',\zeta')=h(\xi,\zeta)$, and
a Banach lattice automorphism
$T:L^1(\frak B_{\kappa})\to L^1(\frak B_{\kappa})$ defined by
saying that $T(\chi a)=\chi\pi a$ for $a\in\frak B_{\kappa}$ (see
365N\formerly{3{}65O});  
note that $T\restr L^{\infty}(\frak B_{\kappa})$ is multiplicative
(compare the formulae in 365Nb and 364Pa).
%Now, for any $J\subseteq\kappa\times\kappa$,
%$P_{h[J]}=TP_JT^{-1}$.   \Prf\ Write $\phi:\frak C_J\to\frak B_{\kappa}$
%and $\psi:\frak C_{h[J]}\to\frak B_{\kappa}$ for the identity maps.
%Then $\psi\pi=\pi\phi:\frak C_J\to\frak B_{\kappa}$, because
%$\pi[\frak C_J]=\frak C_{h[J]}$.   By 365O,
%
%\Centerline{$P_{\pi}P_{\psi}=P_{\psi\pi}=P_{\pi\phi}=P_{\phi}P_{\pi}$,}
%
%\noindent where $P_{\pi}$, $P_{\psi}$, $P_{\phi}$ are the conditional
%expectation operators associated with the measure-preserving Boolean
%homomorphisms $\pi$, $\psi$ and $\phi$.   But $\pi$ is an automorphism, so
%$P_{\pi}$ is just $T^{-1}$ (365P), while in the language here we have
%$P_{\phi}=P_J$ and $P_{\psi}=P_{h[J]}$;  so
%$T^{-1}P_{h[J]}=P_JT^{-1}$, that is, $P_{h[J]}=TP_JT^{-1}$.\ \Qed

If $J\subseteq\kappa\times\kappa$ and $u\in L^1(\frak C_J)$, then
$Tu\in L^1(\frak C_{h[J]})$.    \Prf\ By 314H,
$\frak C_{h[J]}=\pi[\frak C_J]$;  now use the fact that
$\Bvalue{Tu>\alpha}=\pi\Bvalue{u>\alpha}$ for every $\alpha\in\Bbb R$.\
\QeD\   If
$h\restr J$ is the identity, then $\pi a=a$ for every $a\in\frak C_J$ and
$Tv=v$ for every $v\in L^{\infty}(\frak C_J)$.   Consequently $P_JT=P_J$.
\Prf\ If $u\in L^1(\frak B_{\kappa})$ and $c\in\frak C_J$ then

$$\eqalign{\int_cP_JTu
&=\int_cTu
=\int Tu\times\chi c
=\int T^{-1}(Tu\times\chi c)
\cr&=\int u\times T^{-1}\chi c
=\int u\times\chi c
=\int_cu;\cr}$$

\noindent as $P_JTu$ certainly belongs to $L^1(\frak C_J)$, it must be
equal to $P_Ju$.\ \Qed

\medskip

{\bf (c)} Fix $\xi\in S$ and $w\in\Fn_{<\omega}(I;\{0,1\})$ for the moment.
Because $\xi\cap D^*$ has
uncountable cofinality and $J_{\xi w}$ is countable, there is a
$g_w(\xi)\in\xi\cap D^*$ such that
$J_{\xi w}\cap L(\xi)\subseteq L(g_w(\xi))$ and
$\{\eta:\eta\in L_0(\xi)$, $(\xi,\eta)\in J_{\xi w}\}
\subseteq L_0(g_w(\xi))$.
Let $h_{\xi w}:\kappa\times\kappa\to\kappa\times\kappa$ be the involution
defined by saying that

$$\eqalign{h_{\xi w}(\eta,\zeta)
&=(g_w(\xi),\zeta)\text{ if }\eta=\xi,\cr
&=(\xi,\zeta)\text{ if }\eta=g_w(\xi),\cr
&=(\eta,\zeta)\text{ otherwise;}\cr}$$

\noindent note that

\Centerline{$h_{\xi w}[J_{\xi w}]\cap L(\xi)
\subseteq L(g_w(\xi))\cup(\{g_w(\xi)\}\times L_0(g_w(\xi))
\subseteq L(g_w(\xi)+1)$,}

\noindent while $h_{\xi w}$ is the identity on $L(g_w(\xi))$, since
neither $\xi$ nor $g_w(\xi)$ belongs to $L_0(g_w(\xi))$.   Let
$T_{\xi w}:L^1(\frak B_{\kappa})\to L^1(\frak B_{\kappa})$ be
the corresponding Banach lattice isomorphism defined as in (b) above.
Then (b) tells us that

\Centerline{$P_{L(g_w(\xi))}T_{\xi w}=P_{L(g_w(\xi))}$.}

Set

\Centerline{$u_{\xi w}=P_{L(g_w(\xi)+1)}T_{\xi w}(\chi a_{\xi w})
\in L^{\infty}(\frak C_{L(g_w(\xi)+1)})$.}

\medskip

{\bf (d)} Setting $M(\eta)
=\eta\times L^{\infty}(\frak C_{L(\eta)})$ for $\eta<\theta$, we
see that

\Centerline{$\#(M(\eta))\le\#(\eta)^{\omega}\le\lambda^{\omega}<\theta$}

\noindent whenever $\eta\ge 2$ (see (a) above), while
$(g_w(\xi),u_{\xi w})\in\bigcup_{\eta<\xi}M(\eta)$ whenever $\xi\in S$ and
$w\in\Fn_{<\omega}(I;\{0,1\})$.   Since
$\#(\Fn_{<\omega}(I;\{0,1\}))=\lambda<\theta$, 552L tells us that there is
an $\omega_1$-complete filter $\Cal F$ on $\theta$,
containing $\theta\setminus\zeta$
for every $\zeta<\theta$, such that for every
$w\in\Fn_{<\omega}(I;\{0,1\})$ there is a $D\in\Cal F$ such that
$D\subseteq S$ and $g_w$ and
$\xi\mapsto u_{\xi w}$ are constant on $D$.

\medskip

{\bf (e)} For $z\in\Fn_{<\omega}(\kappa+\lambda;\{0,1\})$ and
$\xi<\theta$, set $v_{\xi z}=P_{L(\xi)}(\chi a_{\xi z})$.
Then there is a
$D\in\Cal F$ such that $D\subseteq S$
and $\xi\mapsto v_{\xi z}$ is constant on $D$.
\Prf\ Set $z'=z\restr L_0(\theta)$,
$z''=z\restr\kappa\setminus L_0(\theta)$
and $w=z\restr I$, so that
$a_{\xi z}=a_{\xi z'}\Bcap a_{\xi z''}\Bcap a_{\xi w}$.   Set
$m=\#(z'')$, so that
$a_{\xi z''}=\Bvalue{\check z''\subseteq\dot x_{\xi}}$
has measure $2^{-m}$ for every $\xi$.   Also
$a_{\xi z''}\in\frak C_{\kappa\times(\kappa\setminus L_0(\theta))}$ is
stochastically independent of $\frak C_{L(\theta)}$, so
$P_{L(\theta)}(\chi a_{\xi z''})=2^{-m}\chi 1$;
while $a_{\xi z'}$ and $a_{\xi w}$ belong to
$\frak C_{L(\theta)}$, so, using the formulae in (b),

$$\eqalign{v_{\xi z}
&=P_{L(\xi)}(\chi a_{\xi z})
=P_{L(\xi)}P_{L(\theta)}(\chi a_{\xi z})\cr
&=P_{L(\xi)}P_{L(\theta)}
  (\chi a_{\xi z'}\times\chi a_{\xi z''}\times\chi a_{\xi w})\cr
&=P_{L(\xi)}(\chi a_{\xi z'}\times\chi a_{\xi w}
  \times P_{L(\theta)}(\chi a_{\xi z''}))\cr
&=2^{-m}P_{L(\xi)}(\chi a_{\xi z'}\times\chi a_{\xi w}).\cr}$$

Let $\xi_0<\theta$ be such that $\dom z'\subseteq L_0(\xi_0)$.   By (d),
there are $D_0\in\Cal F$, $\zeta<\theta$ and
$u\in L^{\infty}(\frak B_{\kappa})$ such that
$D_0\subseteq S$ and $g_w(\xi)=\zeta$ and
$u_{\xi w}=u$ for every $\xi\in D_0$.   For
$\xi\in D_0\setminus\xi_0$ take $h_{\xi w}$ and $T_{\xi w}$ as in (c).
Then, writing $\pi_{\xi w}$ for the measure-preserving
automorphism defined from $h_{\xi w}$ as in (b),

$$\eqalign{\pi_{\xi w}a_{\xi z'}
&=\pi_{\xi w}(\inf_{z'(\eta)=1}e_{\xi\eta}
  \Bsetminus\sup_{z'(\eta)=0}e_{\xi\eta})
=\inf_{z'(\eta)=1}\pi_{\xi w}e_{\xi\eta}
  \Bsetminus\sup_{z'(\eta)=0}\pi_{\xi w}e_{\xi\eta}\cr
&=\inf_{z'(\eta)=1}e_{\zeta\eta}
 \Bsetminus\sup_{z'(\eta)=0}e_{\zeta\eta}
=a_{\zeta z'};\cr}$$

\noindent consequently $T_{\xi w}(\chi a_{\xi z'})=\chi a_{\zeta z'}$.
Now

$$\eqalignno{P_{L(\xi)}(\chi a_{\xi z'}\times\chi a_{\xi w})
&=P_{L(\zeta)}(\chi a_{\xi z'}\times\chi a_{\xi w})\cr
\displaycause{because
$\chi a_{\xi z'}\times\chi a_{\xi w}
\in L^{\infty}(\frak C_{(\{\xi\}\times L_0(\xi))\cup J_{\xi w}})$, and
$(\{\xi\}\times L_0(\xi))\cup J_{\xi w})\cap L(\xi)
=J_{\xi w}\cap L(\xi)\subseteq L(\zeta)\subseteq L(\xi)$}
&=P_{L(\zeta)}T_{\xi w}(\chi a_{\xi z'}\times\chi a_{\xi w})\cr
\displaycause{see (c) above}
&=P_{L(\zeta)}
  (T_{\xi w}\chi a_{\xi z'}\times T_{\xi w}\chi a_{\xi w})\cr
\displaycause{because $T_{\xi w}$ is multiplicative on
$L^{\infty}(\frak B_{\kappa})$}
&=P_{L(\zeta)}
  (\chi a_{\zeta z'}\times T_{\xi w}\chi a_{\xi w})\cr
&=P_{L(\zeta)}P_{L(\xi)}
  (\chi a_{\zeta z'}\times T_{\xi w}\chi a_{\xi w})\cr
\displaycause{because $L(\zeta)\subseteq L(\xi)$}
&=P_{L(\zeta)}
  (\chi a_{\zeta z'}\times P_{L(\xi)}T_{\xi w}\chi a_{\xi w})\cr
\displaycause{because
$a_{\zeta z'}\in\frak C_{\{\zeta\}\times L_0(\xi)}
\subseteq\frak C_{L(\xi)}$}
&=P_{L(\zeta)}
  (\chi a_{\zeta z'}\times P_{L(\zeta+1)}T_{\xi w}\chi a_{\xi w})\cr
\displaycause{because
$T_{\xi w}\chi a_{\xi w}\in L^{\infty}(\frak C_{h_{\xi w}[J_{\xi w}]})$ and
$h_{\xi w}[J_{\xi w}]\cap L(\xi)\subseteq L(\zeta+1)$}
&=P_{L(\zeta)}(\chi a_{\zeta z'}\times u_{\xi w})
=P_{L(\zeta)}(\chi a_{\zeta z'}\times u).\cr}$$

Finally, we get

\Centerline{$v_{\xi z}
=2^{-m}P_{L(\xi)}(\chi a_{\xi z'}\times\chi a_{\xi w})
=2^{-m}P_{L(\zeta)}(\chi a_{\zeta z'}\times u)$}

\noindent for every $\xi\in D=D_0\setminus\xi_0$,
so we have the required constant value.\ \Qed

\medskip

{\bf (f)} For each $z\in\Fn_{<\omega}(\kappa+\lambda;\{0,1\})$ set
$v_z=\lim_{\xi\to\Cal F}v_{\xi z}$, the limit being defined in the strong
sense that $\{\xi:\xi\in S$, $v_{\xi z}=v_z\}$ belongs to $\Cal F$.
Observe that
$0\le v_z\le\chi 1$ and that $v_{\emptyset}=\chi 1$, because
$a_{\xi\emptyset}=1$ for every $\xi\in S$.   If
$z\in\Fn_{<\omega}(\kappa+\lambda;\{0,1\})$ and
$\eta\in(\kappa+\lambda)\setminus\dom z$,
then $a_{\xi,z\cup\{(\eta,0)\}}$ and $a_{\xi,z\cup\{(\eta,1)\}}$ are
disjoint and have union $a_{\xi z}$, so

\Centerline{$v_{\xi z}=P_{L(\xi)}(\chi a_{\xi z})
=P_{L(\xi)}
  (\chi a_{\xi,z\cup\{(\eta,0)\}}+\chi a_{\xi,z\cup\{(\eta,1)\}})
=v_{\xi,z\cup\{(\eta,0)\}}+v_{\xi,z\cup\{(\eta,1)\}}$}

\noindent for every $\xi\in S$, and
$v_z=v_{z\cup\{(\eta,0)\}}+v_{z\cup\{(\eta,1)\}}$.
Let $\vec v_z$ be the $\BbbPk$-name for a real number corresponding
to $v_z$ as defined in 5A3L.

\medskip

{\bf (g)} We have a $\BbbPk$-name $\dot\mu$ for a
Baire probability measure on $\{0,1\}^{\kappa+\lambda}$ such that

\Centerline{$\VVdPk\,
\dot\mu\{x:\check z\subseteq x\in\{0,1\}^{\check\kappa+\check\lambda}\}
=\vec v_z$}

\noindent for every $z\in\Fn_{<\omega}(\kappa+\lambda;\{0,1\})$.
\Prf\ Start by setting

\Centerline{$\dot\mu_0
=\{((\check z,\vec v_z),1_{\frak B_{\kappa}}):
z\in\Fn_{<\omega}(\kappa+\lambda;\{0,1\})\}$,}

\noindent so that $\dot\mu_0$ is a $\Bbb P_{\kappa}$-name for a function
from $\Fn_{<\omega}(\check\kappa+\check\lambda;\{0,1\})$ to $[0,1]$ and

\Centerline{$\VVdPk\,
\dot\mu_0\check z=\vec v_z$}

\noindent for every $z\in\Fn_{<\omega}(\kappa+\lambda;\{0,1\})$;  note that

\Centerline{$\VVdPk\,
  \check\kappa+\check\lambda=(\kappa+\lambda)\var2spcheck$
and $\Fn_{<\omega}(\check\kappa+\check\lambda;\{0,1\})
=(\Fn_{<\omega}(\kappa+\lambda;\{0,1\}))\var2spcheck$.}

\noindent Now (f) tells us that

\Centerline{$\VVdPk\,\dot\mu_0\emptyset=1$,}

$$\eqalign{\VVdPk\,\dot\mu_0(z)
&=\dot\mu_0(z\cup\{(\eta,0)\})+\dot\mu_0(z\cup\{(\eta,1)\})\cr
&\mskip100mu\text{whenever }
  z\in\Fn_{<\omega}(\check\kappa+\check\lambda;\{0,1\})
\text{ and }\eta\in(\check\kappa+\check\lambda)\setminus\dom z.\cr}$$

\noindent By 552K, copied into $V^{\BbbPk}$,

$$\eqalign{\VVdPk\,
&\text{ there is a Radon measure }
  \mu\text{ on }\{0,1\}^{\check\kappa+\check\lambda}\text{ such that }\cr
&\mskip150mu\mu\{x:z\subseteq x\}=\dot\mu_0(z)
  \text{ for every }
  z\in\Fn_{<\omega}(\check\kappa+\check\lambda;\{0,1\}).\cr}$$

\noindent In fact we don't really want the Radon measure here, but its
restriction to the Baire $\sigma$-algebra.
Let $\dot\mu$ be a $\BbbPk$-name for a Baire measure on
$\{0,1\}^{\check\kappa+\check\lambda}$ such that

\Centerline{$\VVdPk\,
\dot\mu\{x:\check z\subseteq x\in\{0,1\}^{\check\kappa+\check\lambda}\}
=\dot\mu_0(\check z)=\vec v_z$}

\noindent for every $z\in\Fn_{<\omega}(\kappa+\lambda;\{0,1\})$;
this is what we have been looking for.\ \Qed

\medskip

{\bf (h)} I had better check that

\Centerline{$\VVdPk\,\dot\mu\{x:\check z\subseteq x\}
=(2^{-\#(z)})\var2spcheck$}

\noindent whenever $z\in\Fn_{<\omega}(\kappa;\{0,1\})$.
\Prf\ If $z\in\Fn_{<\omega}(\kappa;\{0,1\})$ and
$\xi\in S$,
then $a_{\xi z}$ belongs to $\frak C_{\{\xi\}\times\kappa}$, so is
stochastically independent of $\frak C_{L(\xi)}$, and

\Centerline{$v_{\xi z}
=P_{L(\xi)}(\chi a_{\xi z})
=(\bar\nu_{\kappa}a_{\xi z})\chi 1=2^{-\#(z)}\chi 1$.}

\noindent Accordingly
$v_z=2^{-\#(z)}\chi 1$ and

\Centerline{$\VVdPk\,\dot\mu\{x:\check z\subseteq x\}=\vec v_z
=(2^{-\#(z)})\var2spcheck$.   \Qed}

\medskip

{\bf (i)} Finally, we come to the key fact:

\Centerline{$\VVdPk\,
\{\dot x_{\xi}\cup\dot f(\dot x_{\xi}):\xi<\check\theta\}$
has full outer measure for $\dot\mu$.}

\noindent\Prf\Quer\ Otherwise, there are a non-zero $b\in\frak B_{\kappa}$,
a rational number $q<1$ and a sequence $\sequencen{\dot C_n}$ of
$\BbbPk$-names for basic cylinder sets in
$\{0,1\}^{\check\kappa+\check\lambda}$ such that

\Centerline{$b\VVdPk\,\dot x_{\xi}\cup\dot f(\dot x_{\xi})
\in\bigcup_{n\in\Bbb N}\dot C_n$}

\noindent for every $\xi<\theta$, while also

\Centerline{$b\VVdPk\,\sum_{n=0}^{\infty}\dot\mu\dot C_n\le\check q$.}

\noindent Because $\frak B_{\kappa}$ is ccc, we can find for each
$n\in\Bbb N$ a partition $\sequence{i}{b_{ni}}$ of unity in
$\frak B_{\kappa}$, and a sequence
$\sequence{i}{z_{ni}}$ in $\Fn_{<\omega}(\kappa+\lambda;\{0,1\})$, such
that $b_{ni}\VVdPk\,\dot C_n=\{x:\check z_{ni}\subseteq x\}$ for each $i$.
Now

\Centerline{$\Bvalue{\dot x_{\xi}\cup\dot f(\dot x_{\xi})\in\dot C_n}
=\sup_{i\in\Bbb N}b_{ni}
  \Bcap\Bvalue{\check z_{ni}\subseteq\dot x_{ni}\cup\dot f(\dot x_{\xi}}
=\sup_{i\in\Bbb N}b_{ni}\Bcap a_{\xi,z_{ni}}$}

\noindent so we must have

\Centerline{$b\Bsubseteq\sup_{i,n\in\Bbb N}b_{ni}\Bcap a_{\xi,z_{ni}}$}

\noindent for every $\xi$.   At the same time,

\Centerline{$b_{ni}\VVdPk\,\dot\mu\dot C_n=\vec v_{z_{ni}}$}

\noindent for each $n$ and $i$, so
$\dot\mu\dot C_n$ is represented by
$\sum_{i\in\Bbb N}\chi b_{ni}\times v_{z_{ni}}$ in
$L^{\infty}(\frak B_{\kappa})$ and
$\min(1,\sum_{n=0}^{\infty}\dot\mu\dot C_n)$ is
represented by
$\chi 1\wedge\sum_{n,i\in\Bbb N}\chi b_{ni}\times v_{z_{ni}}$.
Since
$\Bvalue{\sum_{n=0}^{\infty}\dot\mu\dot C_n\le\check q}$ includes $b$,

\Centerline{$\sum_{n,i\in\Bbb N}\chi b\times\chi b_{ni}\times v_{z_{ni}}
\le q\chi b$.}

Let $J\subseteq\kappa\times\kappa$ be a countable set such that
$b\Bcap b_{ni}\in\frak C_J$ for every $n$, $i\in\Bbb N$.   Then,
because $\Cal F$ is $\omega_1$-complete and contains $S\setminus\zeta$ for
every $\zeta<\theta$, there is a $\xi\in S$ such that

\Centerline{$v_{\xi z_{ni}}=v_{z_{ni}}$ whenever $n$, $i\in\Bbb N$,}

\Centerline{$J\cap L(\theta)=J\cap L(\xi)$,
\quad$J\cap(\{\xi\}\times\kappa)=\emptyset$.}

\noindent Set $L=L(\theta)\cup(\{\xi\}\times\kappa)$.   If $n$, $i\in\Bbb N$, then

$$\eqalignno{\int\chi b\times\chi b_{ni}\times v_{z_{ni}}
&=\int\chi b\times\chi b_{ni}\times P_{L(\xi)}\chi a_{\xi,z_{ni}}\cr
&=\int P_{L(\xi)}(\chi b\times\chi b_{ni}
  \times P_{L(\xi)}\chi a_{\xi,z_{ni}})\cr
&=\int P_{L(\xi)}(\chi b\times\chi b_{ni})
  \times P_{L(\xi)}\chi a_{\xi,z_{ni}}\cr
&=\int P_{L(\xi)}(\chi b\times\chi b_{ni})
  \times\chi a_{\xi,z_{ni}}\cr
&=\int P_L(\chi b\times\chi b_{ni})
  \times\chi a_{\xi,z_{ni}}\cr
\displaycause{because $J\cap L=J\cap L(\xi)$}
&=\int P_L(\chi b\times\chi b_{ni}
  \times\chi a_{\xi,z_{ni}})\cr
\displaycause{because $J_{\xi,z_{ni}}
\subseteq J_{\xi,z_{ni}\restr I}\cup(\{\xi\}\times\kappa)
\subseteq L$, so $a_{\xi,z_{ni}}\in\frak C_L$}
&=\int\chi b\times\chi b_{ni}\times\chi a_{\xi,z_{ni}}.\cr}$$

\noindent Summing over $n$ and $i$, we have

$$\eqalignno{\bar\nu_{\kappa}b
&\le\sum_{n=0}^{\infty}\sum_{i=0}^{\infty}
  \bar\nu_{\kappa}(b\Bcap b_{ni}\Bcap a_{\xi,z_{ni}})
=\sum_{n=0}^{\infty}\sum_{i=0}^{\infty}\int\chi b\times\chi b_{ni}
  \times\chi a_{\xi,z_{ni}}\cr
&=\sum_{n=0}^{\infty}\sum_{i=0}^{\infty}\int\chi b\times\chi b_{ni}
  \times v_{z_{ni}}
\le\int q\chi b
=q\bar\nu_{\kappa}b,\cr}$$

\noindent which is impossible.\ \Bang\Qed

\medskip

{\bf (j)} What all this shows is that

\doubleinset{$\VVdPk$ for every
$f:\{0,1\}^{\check\kappa}
  \to\{0,1\}^{(\check\kappa+\check\lambda)\setminus\check\kappa}$
there is a Baire measure $\mu$
on $\{0,1\}^{\check\kappa+\check\lambda}$ such that
$\mu\{y:y\in\{0,1\}^{\check\kappa+\check\lambda},\,z\subseteq y\}
=2^{-\#(K)}$
whenever $K\in[\check\kappa]^{<\omega}$
and $z\in\{0,1\}^K$,
and $\mu^*\{x\cup f(x):x\in\{0,1\}^{\check\kappa}\}=1$.}

\noindent By 552M, copied into $V^{\BbbPk}$,

\doubleinset{$\VVdPk$ if $\Cal A\subseteq\Cal P(\{0,1\}^{\check\kappa})$
and $\#(\Cal A)\le\check\lambda$, there is an extension of
$\nu_{\check\kappa}$ to a measure measuring every member of $\Cal A$,}

\noindent as required.
}%end of proof of 552N

\leader{552O}{Proposition} Suppose that $(X,\Sigma,\mu)$ is a probability
space such that for every countable family $\Cal A$
of subsets of $X$ there is a measure on $X$
extending $\mu$ and measuring every member of $\Cal A$.

(a) If $Y$ is a universally negligible\cmmnt{ (definition:  439B)}
second-countable T$_0$ space, then $\#(Y)<\cov\Cal N(\mu)$.

(b) $\cov\Cal N(\mu)>\non\Cal N(\nu_{\omega})$.

\proof{{\bf (a)} \Quer\ Otherwise,
let $\family{y}{Y}{E_y}$ be a cover of $X$ by
$\mu$-negligible sets, and $f:X\to Y$ a function such that $x\in E_{f(x)}$
for every $x\in X$.   Let $\Cal U$ be a countable base for the topology of
$Y$ and $\Cal A=\{f^{-1}[U]:U\in\Cal U\}$;  let $\tilde\mu$
be a measure on $X$
extending $\mu$ and measuring every member of $\Cal A$.   Consider the
image measure $\tilde\mu f^{-1}$ on $Y$.
This measures every member of $\Cal U$ so measures every Borel set in $Y$;
let $\nu$ be its restriction to the Borel $\sigma$-algebra of $Y$.   Then
$\nu$ is a Borel probability measure on $Y$.   Take any $y\in Y$.
Because $Y$ has a T$_0$ topology, $\Cal U$ must separate the points of $Y$
and $\{y\}$ is a Borel set;  now

\Centerline{$\nu\{y\}=\tilde\mu f^{-1}[\{y\}]\le\tilde\mu^*E_y
\le\mu^*E_y=0$.}

\noindent So $\nu$ is zero on singletons and witnesses that $Y$ is not
universally negligible.\ \Bang

\medskip

{\bf (b)} By Grzegorek's theorem (439Fc), there is a universally negligible
set $Y\subseteq[0,1]$ with cardinal $\non\Cal N(\nu_{\omega})$.   (Recall
that the Lebesgue null ideal is isomorphic to $\Cal N(\nu_{\omega})$, as
noted in 522Wa.)
}%end of proof of 552O

%question:  do we really need "second-countable"?

\leader{552P}{Theorem} Let $\kappa$ and $\lambda$ be infinite
cardinals.   Then the iterated forcing notion
$\BbbPk*\Bbb P_{\check\lambda}$ has regular open algebra
isomorphic to $\frak B_{\max(\kappa,\lambda)}$.

\cmmnt{\medskip

\noindent{\bf Remark} Here $\Bbb P_{\check\lambda}$ represents a standard
$\BbbPk$-name for random real forcing;  see 551O.
}

\proof{ In Theorem 551Q, take $\Omega=\{0,1\}^{\kappa}$,
$\Sigma=\Tau_{\kappa}$, $\Cal I=\Cal N(\nu_{\kappa})$ and $I=\lambda$.
If we identify $\{0,1\}^{\kappa}\times\{0,1\}^{\lambda}$ with
$\{0,1\}^{\kappa+\lambda}$, where $\kappa+\lambda$ is the ordinal sum,
then $\Lambda=\Sigma\tensorhat\CalBa_{\lambda}$
becomes a $\sigma$-algebra intermediate between $\CalBa_{\kappa+\lambda}$
and $\Tau_{\kappa+\lambda}$, while

\Centerline{$\Cal J=\{W:W\in\Lambda$, $\nu_{\lambda}W[\{x\}]=0$ for
$\nu_{\kappa}$-almost every $x\in\{0,1\}^{\kappa}\}$}

\noindent is just $\Lambda\cap\Cal N(\nu_{\kappa+\lambda})$.   It follows
at once that the algebra $\frak A=\Lambda/\Cal J$ is isomorphic to
$\frak B_{\kappa+\lambda}$;  and 551Q tells us that
$\RO(\BbbPk*\Bbb P_{\check\lambda})$ is isomorphic to $\frak A$.   Since we
are supposing that $\kappa$ and $\lambda$ are infinite,
$\frak B_{\kappa+\lambda}\cong\frak B_{\max(\kappa,\lambda)}$ and we're
done.
}%end of proof of 552P

\exercises{\leader{552X}{Basic exercises (a)}
%\spheader 552Xa
Let $\kappa$ be an infinite cardinal.
Show that $\VVdPk\,\check\kappa^{\omega}=(\kappa^{\omega})\var2spcheck$,
where these are all cardinal powers.
%552B

\spheader 552Xb ({\smc Miller 82})
Suppose that $\frak c<\omega_{\omega}$.   Show that

\Centerline{$\VVdash_{\Bbb P_{\omega_{\omega}}}
\,\cov\Cal N(\nu_{\omega_1})=\omega_{\omega}
<\cov\Cal N(\nu_{\omega})$.}
%552G

\spheader 552Xc\dvAnew{2009}
Suppose that the continuum hypothesis is true.   Show that

\Centerline{$\VVdash_{\Bbb P_{\omega_2}}\,\,\frak c$ is a
precaliber of every measurable algebra.}

\noindent\Hint{525K.}
%552H out of order query

\spheader 552Xd\dvAnew{2010}
Describe Cicho\'n's diagram in the forcing universe
$V^{\Bbb P_{\omega_2}}$ (i) if we start with $\frak c=\omega_1$
(ii) if we start with $\frak m=\frak c=\omega_2$.   Locate the shrinking
number of Lebesgue measure in each case.
%552J

\spheader 552Xe\dvAnew{2010}
Suppose that the continuum hypothesis is true.   Show that

\doubleinset{$\VVdash_{\Bbb P_{\omega_2}}\,
\ci(\Cal P(\{0,1\}^{\omega})\setminus\Cal N(\nu_{\omega}))=\frak c$,
so there is a
family $\ofamily{\xi}{\frak c}{\nu_{\xi}}$ of additive functionals on
$\Cal P([0,1])$ such that $\sup_{\xi<\frak c}\nu_{\xi}A$ is the Lebesgue
outer measure of $A$ for every $A\subseteq[0,1]$.}

\noindent(See {\smc Lipecki 09}.)
%552J

\spheader 552Xf Suppose that the continuum hypothesis is true.
Show that there
is a sequence $\sequencen{A_n}$ of subsets of $[0,1]$ such that there is no
measure extending Lebesgue measure which measures every $A_n$.
\Hint{there is a sequence $\sequencen{f_n}$ of functions from $\omega_1$
to itself such that
$\{f_n(\xi):n\in\Bbb N\}=\{\eta:\eta\le\xi\}$ for every $\xi<\omega_1$.}
%552N

\leader{552Y}{Further exercises (a)}
%\spheader 552Ya
Let $\kappa$ and $\lambda$ be infinite cardinals, and $\mu$
a Baire measure on $\{0,1\}^{\lambda}$.   (i) Show that there is a
$\BbbPk$-name $\dot\mu$ for a Baire measure on $\{0,1\}^{\check\lambda}$
such that $\VVdPk\,\dot\mu\{x:\check z\subseteq x\}
=(\mu\{x:z\subseteq x\})\var2spcheck$
for every $z\in\Fn_{<\omega}(\lambda;\{0,1\})$.   (ii) Show that if
$A\subseteq\{0,1\}^{\lambda}$,
then $\VVdPk\,\dot\mu^*(\check A)=(\mu^*A)\var2spcheck$.
%552D, 552O

\spheader 552Yb(i) Show that for any non-zero
cardinals $\kappa$, $\lambda$ there are
cardinals $\theta^{\text{cov}}_{\kappa\lambda}$ and
$\theta^{\text{non}}_{\kappa\lambda}$ such that

\Centerline{$\VVdPk\,
\cov\Cal N(\nu_{\check\lambda})
=\check\theta^{\text{cov}}_{\kappa\lambda}$,
\quad$\non\Cal N(\nu_{\check\lambda})
=\check\theta^{\text{non}}_{\kappa\lambda}$.}

\noindent\Hint{if $\kappa$ is infinite, $\frak B_{\kappa}$ is homogeneous.}
(ii) Show that $\theta^{\text{cov}}_{\kappa\lambda}$ increases with
$\kappa$ and decreases with $\lambda$, while
$\theta^{\text{non}}_{\kappa\lambda}$ decreases with $\kappa$ and
increases with $\lambda$.   \Hint{552P.}
%552P 552G 552H
}%end of exercises


\endnotes{
\Notesheader{552} In any forcing model, all the open questions of ZFC
re-present themselves for our attention.   The first and most important
question concerns the continuum hypothesis, and in most cases we can say
something useful.   So I start with 552B:  `if you add $\kappa$ random
reals, then the continuum rises to $\kappa^{\omega}$'.   Any mnemonic of
this kind has to come with footnotes concerning the interpretation of the
terms, because we cannot rely on the formula
`$\kappa^{\omega}$' meaning the
same thing in the universe we start from and the forcing model we move to.
Indeed, in general forcing models, the symbol `$\kappa$' has to be watched,
since I normally reserve it for cardinals, and cardinals sometimes
collapse;  but here, at least, we have a ccc forcing notion, and cardinals
are preserved (5A3N).   Actually, `$\kappa^{\omega}$' also is safe in the
present context (552Xa);  but we find this out afterwards.

One of the central properties of random real
forcing concerns iteration:
if you do it twice, you still have random real forcing.   Of
course `iterated forcing', in a vast variety of forms, is an indispensable
technique, and two-stage forcing, as in 552P, is the easiest kind.   I do
not expect to quote this result very often in this book, but that is
because (for random reals) I am interested as much in the forcing notions
themselves, and the measure algebras which are their regular open algebras,
as in the propositions which are true in the forcing models.   So when I
see a proof which depends on repeated random real forcing my first impulse
is to examine the relevant properties of measure algebras, and this
generally leads to a direct proof in terms of single-stage forcing.
Note the form of Theorem 552P:  as in 551Q,
it does not claim that the iteration
$\BbbPk*\Bbb P_{\check\lambda}$ is isomorphic to
$\Bbb P_{\max(\kappa,\lambda)}$, but only that they have isomorphic regular
open algebras, and therefore lead to the same mathematical worlds (5A3I).

A typical example is in 552J.    Random real forcing does not change outer
measures (552D).   If we think of $\Bbb P_{\kappa}$ as an iteration
$\Bbb P_{\kappa\setminus J}*\Bbb P_{\check J}$, and we have a
$\Bbb P_{\kappa}$-name $\dot E$ for a `new' negligible set, then, thinking
in $V^{\Bbb P_{\check J}}$,
the set of members of $\{0,1\}^{\check\lambda}$ contained in
$\dot E$ will have to be negligible.
Back in the ordinary universe, we shall
have a $\Bbb P_J$-name for a negligible set containing every member of
$\{0,1\}^{\check\lambda}$ with a $\Bbb P_J$-name which belongs to $\dot E$.
In 552J, the idea is that if $\dot A$ is a set in
$V^{\Bbb P_{\kappa}}$ and every small subset of $\dot A$ is negligible
in $V^{\Bbb P_{\kappa}}$,
then at every stage the set of members of $\dot A$ which have
been named so far must be negligible in $V^{\Bbb P_{\kappa}}$,
just because there are not very
many names yet available, and therefore is also negligible
in the intermediate universe of the forcing notion $\Bbb P_{K_{\xi}}$.
This must be witnessed by a countable structure
in the intermediate universe, and the Pressing-Down Lemma tells us that
there is a stationary set of levels for which the same countable structure
will serve;  it follows easily that we have a name in $V^{\Bbb P_{\kappa}}$
for a negligible set including $\dot A$.   I invite you to seek out the
elements of the formal exposition in 552J which correspond to this
sketch.

552E can also be approached as a result about iterated random real forcing.
Here, $\dot A$ is just the set of `random reals' $\dot x_{\xi}$ built
directly from the regular open algebra $\frak B_{\kappa}$.   To see that
this is a Sierpi\'nski set, we need to look at a negligible set.   A
negligible set in
$V^{\Bbb P_{\kappa}}$ is included in one which has a name $\dot C$ in
$V^{\Bbb P_J}$ for some countable $J\subseteq\kappa$.   Thinking in
$V^{\Bbb P_J}$, all but countably many of the $\dot x_{\xi}$ are still
random, because they are the `random reals' of
$V^{\Bbb P_{\kappa\setminus J}}$, and therefore do not belong to $\dot C$.
The proof of 552E is no more than a formal elaboration of this idea, with
the extra technical device necessary to reach `strongly Sierpi\'nski'.

In 552C all we need to know is that $\BbbPk$ is weakly
$\sigma$-distributive, and the key fact is that for every name $\dot f$ for
a member of $\BbbN^{\Bbb N}$ there is an $h$ in the ordinary universe such
that $\VVdPk\,\dot f\le^*\check h$;
this is why such partial orders are sometimes
called `$\omega^{\omega}$-bounding'.    The rest of the argument is
based on the same ideas as part (d) of the proof of 5A3N.

In 552F-552J %552F 552G 552H 552I 552J
I list the results known to me about the
additivity, covering number, uniformity, cofinality and shrinking number
of the ideals
$\Cal N(\nu_{\lambda})$ after random real forcing.   Covering number,
uniformity and shrinking number are the
difficult ones, and even the most basic case, when $\lambda=\omega$ and
we are forcing with $\Bbb P_{\omega}$, seems not to have been completely
sorted out.   552Gb and 552Hc show that there is room for surprises.
My method throughout is to use the results of \S551 to relate
$\Cal N(\nu_{\lambda})$ in $V^{\Bbb P_{\kappa}}$ to
$\Cal N(\nu_{\kappa}\times\nu_{\lambda})$ in the original universe.  Given a
$\Bbb P_{\kappa}$-name $\dot W$ for a negligible set in
$\{0,1\}^{\check\lambda}$, we
have a negligible $W\subseteq\{0,1\}^{\kappa}\times\{0,1\}^{\lambda}$ such that
$\VVdPk\,\dot W\subseteq\vec W$, and then a negligible
$V\subseteq\{0,1\}^{\lambda}$, corresponding to the non-negligible horizontal
sections of $W$, such that $(\{0,1\}^{\lambda}\setminus V)\var2spcheck$ is
disjoint from $\vec W$ and $\dot W$ in $V^{\Bbb P_{\kappa}}$.

In 552K-552M %552K 552L 552M
I give some lemmas which apparently have nothing to do with forcing.   The
intention is to express as much as possible of the argument of Carlson's
theorem 552N as results in ZFC.   In this section I am taking forcing
arguments particularly laboriously;  but even when you have got to the
point where they seem elementary to you, I believe that it is still worth
while minimising the regions in which one has to deal with more than one
model of set theory at a time.
In 552M the parts (ii) and (iii) contrast oddly.   Part
(ii) is there to serve
as a combinatorial form of (i) which will be
accessible for the purposes of 552N.    Part (iii) is there to give
a notion of the scope of 552N, and in particular to show that in
random real models we have extension theorems for many measures not
obviously similar to the basic measures $\nu_{\kappa}$.   I have already
noted a similar result in 543G.

In \S439 I described a number of examples of probability spaces
$(X,\Sigma,\mu)$ with a countable family $\Cal A\subseteq\Cal PX$ such that
$\mu$ has no extension to a measure measuring every member of $\Cal A$.
In particular, as observed in 439Xk, Grzegorek's theorem 439Fc
gives us an example of a subspace of $[0,1]$ for which the subspace measure
fails to be extendable to some countably-generated $\sigma$-algebra.
These are ZFC examples;  we really do need something like `compactness'
in 552M(iii).

Note that {\smc Carlson 84} gives a rather sharper form of Theorem 552N,
carrying information about the covering numbers
of the measures constructed in $V^{\BbbPk}$.

}%end of notes

\discrpage

