\frfilename{mt261.tex} 
\versiondate{11.12.12} 
\copyrightdate{2000} 
      
\def\dist{\mathop{\text{dist}}} 
      
\def\chaptername{Change of variable in the integral} 
\def\sectionname{Vitali's theorem in $\BbbR^r$} 
      
\newsection{261} 
      
\def\headlinesectionname{Vitali's theorem in $\eightBbb R^r$} 
      
The main aim of this section is to give $r$-dimensional versions of 
Vitali's theorem and Lebesgue's Density Theorem, following 
ideas already presented in \S\S221 and 223.   I end with a proof that
Lebesgue outer measure can be defined in terms of coverings by balls
instead of by intervals (261F).
      
\cmmnt{ 
\leader{261A}{Notation} For most of this chapter, we shall be dealing 
with the geometry and 
measure of Euclidean space;  it will save space to fix some notation. 
      
Throughout this section and the two following, $r\ge 1$ will be an 
integer.  I will use Roman letters for members of $\BbbR^r$ and 
Greek letters for their coordinates, so that 
$a=(\alpha_1,\ldots,\alpha_r)$, etc.;  if you see any Greek letter with 
a subscript you should look first for a nearby vector of which it might 
be a coordinate.   The measure under consideration will nearly always be 
Lebesgue measure on $\Bbb R^r$;  so unless otherwise indicated  $\mu$ 
should be interpreted as 
Lebesgue measure, and $\mu^*$ as Lebesgue outer measure.   Similarly, 
$\int\ldots dx$ will always be integration with respect to Lebesgue 
measure (in a dimension determined by the context). 
      
For $x=(\xi_1,\ldots,\xi_r)\in\BbbR^r$, write 
$\|x\|=\sqrt{\xi_1^2+\ldots+\xi_r^2}$.   Recall that 
$\|x+y\|\le\|x\|+\|y\|$ (1A2C) and that $\|\alpha x\|=|\alpha|\|x\|$ 
for any vectors $x$, $y$ and scalar $\alpha$. 
      
I will use the same notation as in \S115 for `intervals', so 
that, in particular, 
      
\Centerline{$\coint{a,b} 
=\{x:\alpha_i\le\xi_i<\beta_i\Forall i\le r\}$,} 
      
\Centerline{$\ooint{a,b} 
=\{x:\alpha_i<\xi_i<\beta_i\Forall i\le r\}$,} 
      
\Centerline{$[a,b] 
=\{x:\alpha_i\le \xi_i\le \beta_i\Forall i\le r\}$} 
      
\noindent whenever $a$, $b\in\BbbR^r$. 
      
$\tbf{0}=(0,\ldots,0)$ will be the zero vector in $\BbbR^r$, and 
$\tbf{1}$ will be $(1,\ldots,1)$.   If 
$x\in\BbbR^r$ and $\delta>0$, $B(x,\delta)$ will be the closed ball 
with centre $x$ and radius $\delta$, that is, 
$\{y:y\in\BbbR^r,\,\|y-x\|\le\delta\}$. 
Note that $B(x,\delta)=x+B(\tbf{0},\delta)$;  so that by the 
translation-invariance of Lebesgue measure we have 
      
\Centerline{$\mu B(x,\delta)=\mu B(\tbf{0},\delta) 
=\beta_r\delta^r$,} 
      
\noindent where 
      
$$\eqalign{\beta_r 
&=\Bover{1}{k!}\pi^k\text{ if }r=2k\text{ is even},\cr 
&=\Bover{2^{2k+1}k!}{(2k+1)!}\pi^k\text{ if }r=2k+1\text{ is odd}\cr}$$ 
      
\noindent (252Q). 
}%end of comment 
      
\leader{261B}{Vitali's theorem in $\BbbR^r$} Let $A\subseteq\BbbR^r$ 
be any set, and $\Cal I$ a family of closed non-trivial\cmmnt{ (that 
is, non-singleton, or, equivalently, non-negligible)} 
balls in $\BbbR^r$ such that every point of $A$ is contained in 
arbitrarily small members of $\Cal I$.   Then there is a countable 
disjoint set $\Cal I_0\subseteq \Cal I$ such that 
$\mu(A\setminus\bigcup\Cal I_0)=0$. 
      
\proof{{\bf (a)} To begin with (down to the end of (f) below), suppose 
that $\|x\|<M$ for every $x\in A$, and 
set 
      
\Centerline{$\Cal 
I'=\{I:I\in\Cal I,\,I\subseteq B(\tbf{0},M)\}$.} 
      
\noindent   If there is a finite disjoint 
set $\Cal I_0\subseteq\Cal I'$ 
such that $A\subseteq\bigcup\Cal I_0$ (including the 
possibility that $A=\Cal I_0=\emptyset$), we can stop.   So let us 
suppose henceforth that there is no such $\Cal I_0$. 
      
\medskip 
      
{\bf (b)} In this case, if $\Cal I_0$ is any finite disjoint subset of 
$\Cal I'$, there is a $J\in\Cal I'$ which is disjoint from any member of 
$\Cal I_0$.   \Prf\ Take $x\in A\setminus\bigcup\Cal I_0$.   Because 
every member of $\Cal I_0$ is closed, there is a $\delta>0$ such that 
$B(x,\delta)$ does not meet any member of $\Cal I_0$, and as 
$\|x\|<M$ we can suppose that $B(x,\delta)\subseteq B(\tbf{0},M)$.   Let 
$J$ be a member of $\Cal I$, containing $x$, and of diameter at most 
$\delta$;  then $J\in\Cal I'$ and $J\cap\bigcup\Cal I_0=\emptyset$.\ 
\Qed 
      
\medskip 
      
{\bf (c)} We can therefore choose a sequence $\sequencen{\gamma_n}$ of 
real numbers and a disjoint sequence $\sequencen{I_n}$ in $\Cal I'$ 
inductively, as follows.   Given $\langle I_j\rangle_{j<n}$ (if $n=0$, 
this is the empty sequence, with no members), with $I_j\in\Cal I'$ for 
each $j<n$, and $I_j\cap I_k=\emptyset$ for $j<k<n$, set $\Cal 
J_n=\{I:I\in\Cal I',\,I\cap I_j=\emptyset$ for every $j<n\}$.   We know 
from (b) that $\Cal J_n\ne\emptyset$.   Set 
      
\Centerline{$\gamma_n=\sup\{\diam I:I\in\Cal J_n\}$;} 
      
\noindent then $\gamma_n\le 2M$, because every member of $\Cal J_n$ is 
included in $B(\tbf{0},M)$.   We can therefore find a set 
$I_n\in\Cal J_n$ such that $\diam I_n\ge{1\over 2}\gamma_n$, and this 
continues the induction. 
      
\medskip 
      
{\bf (e)} Because the $I_n$ are disjoint measurable subsets of 
the bounded set $B(\tbf{0},M)$, we have 
      
\Centerline{$\sum_{n=0}^{\infty}\mu I_n 
\le\mu B(\tbf{0},M)<\infty$,} 
      
\noindent and $\lim_{n\to\infty}\mu I_n=0$.   Also 
$\mu I_n\ge\beta_r(\bover14\gamma_n)^r$ for each $n$, so 
$\lim_{n\to\infty}\gamma_n=0$. 
      
Now define $I'_n$ to be the closed ball with the same centre as $I_n$ 
but five times the diameter, so 
that it contains every point within a distance $\gamma_n$ of $I_n$.   I 
claim that, for any $n$, 
$A\subseteq\bigcup_{j<n}{I}_j\cup\bigcup_{j\ge n}I_j'$. 
\Prf\Quer\ Suppose, if possible, otherwise.   Take any 
$x\in A\setminus(\bigcup_{j<n}{I}_j\cup\bigcup_{j\ge n}I'_j)$.   Let 
$\delta>0$ be such that 
      
      
\Centerline{$B(x,\delta) 
  \subseteq B(\tbf{0},M)\setminus\bigcup_{j<n}I_j$,} 
      
\noindent and let 
$J\in\Cal I$ be such that $x\in J\subseteq B(x,\delta)$.   Then 
      
\Centerline{$\lim_{m\to\infty}\gamma_m=0<\diam J$} 
      
\noindent (this is where we use the hypothesis that all the balls in 
$\Cal I$ are non-trivial);  let $m$ be the least integer 
greater than or equal to $n$ such that $\gamma_m<\diam J$.   In this 
case $J$ cannot belong to $\Cal J_m$, so there must be some $k<m$ such 
that $J\cap I_k\ne\emptyset$, because certainly $J\in\Cal I'$.   By the 
choice of $\delta$, $k$ cannot be less than $n$, so $n\le k<m$, and 
$\gamma_k\ge\diam J$.   So the distance from $x$ to the 
nearest point of $I_k$ is at most $\diam J\le \gamma_k$.   But this 
means 
that $x\in 
I'_k$;  which contradicts the choice of $x$.\ \Bang\Qed 
      
\medskip 
      
{\bf (f)} It follows that 
      
\Centerline{$\mu\sp*(A\setminus\bigcup_{j<n}I_j)\le\mu(\bigcup_{j\ge 
n}I'_j)\le\sum_{j=n}^{\infty}\mu I'_j\le 5^r\sum_{j=n}^{\infty}\mu 
I_j$.} 
      
\noindent As 
      
\Centerline{$\sum_{j=0}^{\infty}\mu I_j\le \mu B(\tbf{0},M)<\infty$,} 
      
\noindent $\lim_{n\to\infty}\mu\sp*(A\setminus\bigcup_{j<n}I_j)=0$ and 
      
\Centerline{$\mu(A\setminus\bigcup_{j\in\Bbb N}I_j)= 
\mu\sp*(A\setminus\bigcup_{j\in\Bbb N}I_j)=0$.} 
      
      
Thus in this case we may set $\Cal I_0=\{I_n:n\in\Bbb N\}$ to obtain 
a countable disjoint family in $\Cal I$ with $\mu(A\setminus\bigcup 
\Cal I_0)=0$. 
      
      
\medskip 
      
{\bf (g)}  This completes the proof if $A$ is bounded.   In general, set 
      
\Centerline{$U_n=\{x:x\in \BbbR^r,\,n<\|x\|<n+1\}$, 
\quad $A_n=A\cap U_n$, 
\quad$\Cal J_n=\{I:I\in\Cal I,\,I\subseteq U_n\}$,} 
      
\noindent for each $n\in\Bbb N$.   Then for each $n$ we see that every 
point of $A_n$ belongs to arbitrarily small members of $\Cal J_n$, so 
there is a countable disjoint $\Cal J'_n\subseteq\Cal J_n$ such that 
$A_n\setminus\bigcup\Cal J'_n$ is negligible.   Now (because the $U_n$ 
are disjoint) $\Cal I_0=\bigcup_{n\in\Bbb N}\Cal J'_n$ is disjoint, and 
of course $\Cal I_0$ is a countable subset of $\Cal I$;  moreover, 
      
\Centerline{$A\setminus\bigcup\Cal I_0 
\subseteq(\BbbR^r\setminus\bigcup_{n\in\Bbb N}U_n) 
\cup\bigcup_{n\in\Bbb N}(A_n\setminus\bigcup\Cal J'_n)$} 
      
\noindent is negligible.   (To see that $\Bbb 
R^r\setminus\bigcup_{n\in\Bbb N}U_n=\{x:\|x\|\in\Bbb N\}$ is negligible, 
note that for any $n\in\Bbb N$ the set 
      
\Centerline{$\{x:\|x\|=n\}\subseteq B(\tbf{0},n)\setminus 
B(\tbf{0},\delta n)$} 
      
\noindent has measure at most $\beta_rn^r-\beta_r(\delta n)^r$ for every 
$\delta\in\coint{0,1}$, so must be negligible.) 
}%end of proof of 261B 
      
      
\leader{261C}{}\cmmnt{ Just as in \S223, we can use the 
$r$-dimensional Vitali theorem to prove theorems on the approximation of 
functions by their local mean values. 
      
\medskip 
      
\noindent}{\bf Density Theorem in $\BbbR^r$:  integral form} Let $D$ be 
a subset of $\BbbR^r$, and $f$ a real-valued function which is 
integrable over $D$.   Then 
      
\Centerline{$f(x)=\lim_{\delta\downarrow 0}\Bover{1}{\mu B(x,\delta)} 
  \int_{D\cap B(x,\delta)}fd\mu$} 
      
\noindent for almost every $x\in D$. 
      
\proof{{\bf (a)} To begin with (down to the end of (b)), let us 
suppose that $D=\dom f=\BbbR^r$. 
      
Take $n\in\Bbb N$ and $q$, $q'\in\Bbb Q$ with $q<q'$, and set 
      
\Centerline{$A=A_{nqq'} 
=\{x:\|x\|\le n,\,f(x)\le q,\, 
  \limsup_{\delta\downarrow 0}\Bover{1}{\mu B(x,\delta)} 
  \int_{B(x,\delta)}fd\mu>q'\}$.} 
      
\noindent\Quer\ Suppose, if possible, that $\mu^*A>0$.      Let 
$\epsilon>0$ be such that $\epsilon(1+|q|)<(q'-q)\mu^* A$, and let 
$\eta\in\ocint{0,\epsilon}$ be such that $\int_{E}|f|\le\epsilon$ 
whenever $\mu E\le\eta$ (225A).   Let $G\supseteq A$ be an open set of 
measure at most 
$\mu^*A+\eta$ (134Fa).   Let $\Cal I$ be the set of 
non-trivial closed 
balls $B\subseteq G$ such that $\bover{1}{\mu B}\int_Bfd\mu\ge q'$. 
Then every point of $A$ is contained in (indeed, is the centre 
of) arbitrarily small members of $\Cal I$.   So there is a countable 
disjoint set $\Cal I_0\subseteq\Cal I$ such that 
$\mu(A\setminus\bigcup\Cal I_0)=0$, by 261B;  set $H=\bigcup\Cal I_0$. 
      
Because $\int_{I}fd\mu\ge q'\mu I$ for each $I\in\Cal I_0$, we 
have 
      
\Centerline{$\int_{H}fd\mu=\sum_{I\in\Cal I_0}\int_Ifd\mu 
\ge q'\sum_{I\in\Cal I_0}\mu I=q'\mu H\ge q'\mu^* A$.} 
      
\noindent Set 
      
\Centerline{$E=\{x:x\in G,\,f(x)\le q\}$.} 
      
\noindent Then $E$ is measurable, and $A\subseteq E\subseteq G$; 
so 
      
\Centerline{$\mu^* A\le\mu E\le\mu G\le\mu^* A+\eta 
\le\mu^* A+\epsilon$.} 
      
\noindent  Also 
      
\Centerline{$\mu(H\setminus E)\le\mu G-\mu E\le\eta$,} 
      
\noindent so by 
the choice of $\eta$, $\int_{H\setminus E}f\le\epsilon$ and 
      
$$\eqalignno{\int_{H}f 
&\le\epsilon+\int_{H\cap E}f 
\le\epsilon+q\mu(H\cap E)\cr 
&\le\epsilon+q\mu^*A+|q|(\mu(H\cap E)-\mu^* A) 
\le q\mu^* A+\epsilon(1+|q|)\cr 
\displaycause{because $\mu^*A=\mu^*(A\cap H)\le\mu(H\cap E)\le\mu E$} 
&<q'\mu^* A 
\le\int_{H}f,\cr}$$ 
      
\noindent which is impossible.\ \Bang 
      
Thus $A_{nqq'}$ is negligible.   This is true for all $q<q'$ and all 
$n$, so 
      
\Centerline{$A^*=\bigcup_{q,q'\in\Bbb Q,q<q'} 
  \bigcup_{n\in\Bbb N}A_{nqq'}$} 
      
\noindent is negligible.   But 
      
\Centerline{$f(x) 
\ge\limsup_{\delta\downarrow 0}\Bover{1}{\mu B(x,\delta)} 
  \int_{B(x,\delta)}f$} 
      
\noindent for every $x\in\BbbR^r\setminus A^*$, that is, for almost 
all $x\in \BbbR^r$. 
      
\medskip 
      
{\bf (b)} Similarly, or applying this result to $-f$. 
      
\Centerline{$f(x) 
\le\liminf_{\delta\downarrow 0}\Bover{1}{\mu B(x,\delta)} 
  \int_{B(x,\delta)}f$} 
      
\noindent for almost every $x$, so 
      
\Centerline{$f(x) 
=\lim_{\delta\downarrow 0}\Bover{1}{\mu B(x,\delta)} 
  \int_{B(x,\delta)}f$} 
      
\noindent for almost every $x$. 
      
\medskip 
      
{\bf (c)} For the (superficially) more general case enunciated in the 
theorem, let $\tilde f$ be a $\mu$-integrable function extending 
$f\restr D$, defined everywhere on $\BbbR^r$, and such that 
$\int_F\tilde f=\int_{D\cap F}f$ for every measurable 
$F\subseteq\BbbR^r$ (applying 214Eb to $f\restr D$).   Then 
      
\Centerline{$f(x)=\tilde f(x) 
=\lim_{\delta\downarrow 0}\Bover{1}{\mu B(x,\delta)} 
  \int_{B(x,\delta)}\tilde f 
=\lim_{\delta\downarrow 0}\Bover{1}{\mu B(x,\delta)} 
  \int_{D\cap B(x,\delta)}f$} 
      
\noindent for almost every $x\in D$. 
}%end of proof of 261C 
      
\leader{261D}{Corollary} (a) If $D\subseteq\BbbR^r$ is any set, then 
      
\Centerline{$\lim_{\delta\downarrow 0}\Bover{\mu^*(D\cap B(x,\delta))} 
{\mu B(x,\delta)}=1$} 
      
\noindent for almost every $x\in D$. 
      
(b) If $E\subseteq\BbbR^r$ is a measurable set, then 
      
\Centerline{$\lim_{\delta\downarrow 0}\Bover{\mu(E\cap B(x,\delta))} 
{\mu B(x,\delta)}=\chi E(x)$} 
      
\noindent for almost every $x\in \BbbR^r$. 
      
      
(c) If $D\subseteq\BbbR^r$ and $f:D\to\Bbb R$ is any function, then for 
almost every $x\in D$, 
      
\Centerline{$\lim_{\delta\downarrow 0} 
\Bover{\mu^*(\{y:y\in D,\,|f(y)-f(x)|\le\epsilon\}\cap 
B(x,\delta))} 
{\mu B(x,\delta)}=1$} 
      
\noindent for every $\epsilon>0$. 
      
(d) If $D\subseteq\BbbR^r$ and $f:D\to\Bbb R$ is measurable, then for 
almost every $x\in D$, 
      
\Centerline{$\lim_{\delta\downarrow 0} 
\Bover{\mu^*(\{y:y\in D,\,|f(y)-f(x)|\ge\epsilon\}\cap 
B(x,\delta))} 
{\mu B(x,\delta)}=0$} 
      
\noindent for every $\epsilon>0$. 
      
\proof{{\bf (a)} Apply 261C with $f=\chi B(\tbf{0},n)$ to see that, for 
any $n\in\Bbb N$, 
      
\Centerline{$\lim_{\delta\downarrow 0}\Bover{\mu^*(D\cap B(x,\delta))} 
{\mu B(x,\delta)}=1$} 
      
\noindent for almost every $x\in D$ with $\|x\|<n$. 
      
\medskip 
      
{\bf (b)} Apply (a) to $E$ to see that 
      
\Centerline{$\liminf_{\delta\downarrow 0}\Bover{\mu(E\cap 
B(x,\delta))}{\mu B(x,\delta)}\ge\chi E(x)$} 
      
\noindent for almost every $x\in\BbbR^r$, and to $E'=\BbbR^r\setminus 
E$ to see that 
      
\Centerline{$\limsup_{\delta\downarrow 0}\Bover{\mu(E\cap 
B(x,\delta))}{\mu B(x,\delta)} 
=1-\liminf_{\delta\downarrow 0} 
  \Bover{\mu(E'\cap B(x,\delta))}{\mu B(x,\delta)} 
\le 1-\chi E'(x)=\chi E(x)$} 
      
\noindent for almost every $x$. 
      
\medskip 
      
{\bf (c)} For $q$, $q'\in\Bbb Q$, set 
      
\Centerline{$D_{qq'}=\{x:x\in D,\,q\le f(x)\le q'\}$,} 
      
\Centerline{$C_{qq'}=\{x:x\in D_{qq'},\,\lim_{\delta\downarrow 0} 
\Bover{\mu^*(D_{qq'}\cap B(x,\delta))}{\mu B(x,\delta)}=1\}$;} 
      
\noindent now set 
      
\Centerline{$C 
=D\setminus\bigcup_{q,q'\in\Bbb Q}(D_{qq'}\setminus C_{qq'})$,} 
      
\noindent so that $D\setminus C$ is negligible.   If $x\in C$ and 
$\epsilon>0$, then there are $q$, $q'\in\Bbb Q$ such that 
$f(x)-\epsilon\le q\le f(x)\le q'\le f(x)+\epsilon$, and now 
$x\in C_{qq'}$;  accordingly
      
\Centerline{$\liminf_{\delta\downarrow 0} 
\Bover{\mu^*\{y:y\in D\cap B(x,\delta),\,|f(y)-f(x)|\le\epsilon\}} 
  {\mu B(x,\delta)} 
\ge\liminf_{\delta\downarrow 0} 
  \Bover{\mu^*(D_{qq'}\cap B(x,\delta))}{\mu B(x,\delta)}=1$,} 
      
\noindent so 
      
\Centerline{$\lim_{\delta\downarrow 0} 
\Bover{\mu^*\{y:y\in D\cap B(x,\delta),\,|f(y)-f(x)|\le\epsilon\}} 
{\mu B(x,\delta)} 
=1$.} 
      
\medskip 
      
{\bf (d)} Define $C$ as in (c).   We know from (a) that $\mu(D\setminus 
C')=0$, where 
      
\Centerline{$C'=\{x:x\in D,\lim_{\delta\downarrow 0}\Bover{\mu^*(D\cap 
B(x,\delta))} 
{\mu B(x,\delta)}=1\}$. 
} 
      
\noindent If $x\in C\cap C'$ and $\epsilon>0$, we know from (c) that 
      
 \Centerline{$\lim_{\delta\downarrow 0} 
\Bover{\mu^*\{y:y\in D\cap B(x,\delta),\,|f(y)-f(x)|\le\epsilon/2\}} 
{\mu B(x,\delta)} 
=1$.} 
      
\noindent But because $f$ is measurable, we have 
      
$$\eqalign{\mu^*\{y:y\in &D\cap B(x,\delta),\, 
|f(y)-f(x)|\ge\epsilon\}\cr 
&+\mu^*\{y:y\in D\cap B(x,\delta),\,|f(y)-f(x)|\le\Bover12\epsilon\} 
\le\mu^*(D\cap B(x,\delta))\cr}$$ 
      
\noindent for every $\delta>0$.   Accordingly 
      
$$\eqalign{\limsup_{\delta\downarrow 0} 
&\Bover{\mu^*\{y:y\in D\cap B(x,\delta),\,|f(y)-f(x)|\ge\epsilon\}} 
{\mu B(x,\delta)}\cr 
&\le\lim_{\delta\downarrow 0} 
  \Bover{\mu^*(D\cap B(x,\delta))}{\mu B(x,\delta)} 
-\lim_{\delta\downarrow 0} 
  \Bover{\mu^*\{y:y\in D\cap B(x,\delta),\,|f(y)-f(x)|\le\epsilon/2\}} 
  {\mu B(x,\delta)} 
=0,\cr}$$ 
      
\noindent and 
      
\Centerline{$\lim_{\delta\downarrow 0} 
\Bover{\mu^*\{y:y\in D\cap B(x,\delta),\,|f(y)-f(x)|\ge\epsilon\}} 
{\mu B(x,\delta)}=0$} 
      
\noindent for every $x\in C\cap C'$, that is, for almost every $x\in D$. 
}%end of proof of 261D 
      
\leader{261E}{Theorem} Let $f$ be a locally integrable function defined on 
a conegligible subset of $\BbbR^r$.   Then 
      
\Centerline{$\lim_{\delta\downarrow 0}\Bover1{\mu B(x,\delta)} 
\int_{B(x,\delta)}|f(y)-f(x)|dy=0$} 
      
\noindent for almost every $x\in\BbbR^r$. 
      
\proof{ (Compare 223D.) 
      
\medskip 
      
{\bf (a)} Fix $n\in\Bbb N$ for the moment, and set $G=\{x:\|x\|<n\}$. 
For each $q\in\Bbb Q$, set $g_q(x)=|f(x)-q|$ 
for $x\in G\cap\dom f$;  then $g_q$ is integrable over $G$, and 
      
\Centerline{$\lim_{\delta\downarrow 0}\Bover1{\mu B(x,\delta)} 
\int_{G\cap B(x,\delta)}g_q=g_q(x)$} 
      
\noindent for almost every $x\in G$, by 261C.   Setting 
      
\Centerline{$E_q=\{x:x\in G\cap\dom f,\,\lim_{\delta\downarrow 0} 
\Bover1{\mu B(x,\delta)}\int_{G\cap B(x,\delta)}g_q=g_q(x)\}$,} 
      
\noindent we have $G\setminus E_q$ negligible for every $q$, so 
$G\setminus E$ is 
negligible, where $E=\bigcap_{q\in\Bbb Q}E_q$.   Now 
      
\Centerline{$\lim_{\delta\downarrow 0}\Bover1{\mu B(x,\delta)} 
\int_{G\cap B(x,\delta)}|f(y)-f(x)|dy=0$} 
      
\noindent for every $x\in E$.   \Prf\ Take $x\in E$ and $\epsilon>0$. 
Then there is a $q\in\Bbb Q$ such that $|f(x)-q|\le\epsilon$, so that 
      
\Centerline{$|f(y)-f(x)|\le|f(y)-q|+\epsilon=g_q(y)+\epsilon$} 
      
\noindent for every $y\in G\cap\dom f$, and 
      
$$\eqalign{\limsup_{\delta\downarrow 0}\Bover1{\mu B(x,\delta)} 
\int_{G\cap B(x,\delta)}|f(y)-f(x)|dy 
&\le\limsup_{\delta\downarrow 0}\Bover1{\mu B(x,\delta)} 
\int_{G\cap B(x,\delta)}g_q(y)+\epsilon\,dy\cr 
&=\epsilon+g_q(x) 
\le 2\epsilon.\cr}$$ 
      
\noindent As $\epsilon$ is arbitrary, 
      
\Centerline{$\lim_{\delta\downarrow 0}\Bover1{\mu B(x,\delta)} 
\int_{G\cap B(x,\delta)}|f(y)-f(x)|dy=0$,} 
      
\noindent as required.\ \Qed 
      
\medskip 
      
{\bf (b)} Because $G$ is open, 
 
\Centerline{$\lim_{\delta\downarrow 0}\Bover1{\mu B(x,\delta)} 
\int_{B(x,\delta)}|f(y)-f(x)|dy 
=\lim_{\delta\downarrow 0}\Bover1{\mu B(x,\delta)} 
\int_{G\cap B(x,\delta)}|f(y)-f(x)|dy 
=0$} 
      
\noindent for almost every $x\in G$.   As $n$ is arbitrary, 
 
\Centerline{$\lim_{\delta\downarrow 0}\Bover1{\mu B(x,\delta)} 
\int_{B(x,\delta)}|f(y)-f(x)|dy=0$} 
      
\noindent for almost every $x\in\BbbR^r$. 
}%end of proof of 261E 
      
\medskip 
\noindent{\bf Remark} The set 
      
\Centerline{$\{x:x\in\dom f,\, 
  \lim_{\delta\downarrow 0}\Bover1{\mu B(x,\delta)} 
\int_{B(x,\delta)}|f(y)-f(x)|dy=0\}$} 
      
\noindent is sometimes called the {\bf Lebesgue set} of $f$. 
      
\leader{261F}{}\cmmnt{ Another very useful consequence of 261B is the 
following. 
      
\medskip 
      
\noindent}{\bf Proposition} Let $A\subseteq\BbbR^r$ be any set, and 
$\epsilon>0$.   Then there is a sequence $\sequencen{B_n}$ of closed 
balls in $\BbbR^r$, all of radius at most $\epsilon$, such that 
$A\subseteq\bigcup_{n\in\Bbb N}B_n$ and 
$\sum_{n=0}^{\infty}\mu B_n\le\mu^*A+\epsilon$.   Moreover, we may 
suppose that the balls in the 
sequence whose centres do not lie in $A$ have measures summing to at 
most $\epsilon$. 
      
\proof{{\bf (a)} The first step is the 
obvious remark that if $x\in\BbbR^r$, $\delta>0$ then the half-open 
cube $I=\coint{x,x+\delta\tbf{1}}$ is a subset of the ball 
$B(x,\delta\sqrt{r})$, 
which has measure $\gamma_r\delta^r=\gamma_r\mu I$, where 
$\gamma_r=\beta_rr^{r/2}$.   It follows that if $G\subseteq\BbbR^r$ is 
any open set, then $G$ can be covered by a sequence of balls of total 
measure at most $\gamma_r\mu G$.   \Prf\ If $G$ is empty, we can take 
all the balls to be singletons.   Otherwise, for each $k\in\Bbb N$, set 
      
\Centerline{$Q_k 
=\{z:z\in \Bbb Z^r,\,\coint{2^{-k}z,2^{-k}(z+\tbf{1})}\subseteq G\}$,} 
      
\Centerline{$E_k 
=\bigcup_{z\in Q_k}\coint{2^{-k}z,2^{-k}(z+\tbf{1}})$.} 
      
\noindent Then $\sequence{k}{E_k}$ is a non-decreasing sequence of sets 
with union $G$, and $E_0$ and each of the differences $E_{k+1}\setminus 
E_k$ is expressible as a disjoint union of half-open cubes.   Thus $G$ 
also is expressible as a disjoint union of a sequence $\sequencen{I_n}$ 
of half-open cubes.   Each $I_n$ is covered by a ball $B_n$ of measure 
$\gamma_r\mu I_n$;  so that $G\subseteq\bigcup_{n\in\Bbb N}B_n$ and 
      
\Centerline{$\sum_{n=0}^{\infty}\mu B_n 
\le\gamma_r\sum_{n=0}^{\infty}\mu I_n=\gamma_r\mu G$.   \Qed} 
      
\medskip 
      
{\bf (b)} It follows at once that if $\mu A=0$ then for any $\epsilon>0$ 
there is a sequence $\sequencen{B_n}$ of balls covering $A$ of measures 
summing to at most $\epsilon$, because there is certainly an open set 
including $A$ with measure at most $\epsilon/\gamma_r$. 
      
\medskip 
      
{\bf (c)} Now take any set $A$, and $\epsilon>0$.   Let $G\supseteq A$ 
be an open set with $\mu G\le\mu^*A+\bover12\epsilon$.   Let $\Cal I$ be 
the family of non-trivial closed balls included in $G$, of radius at 
most $\epsilon$ and with centres in $A$.   Then every point of $A$ 
belongs to arbitrarily small members of $\Cal I$, so there is a 
countable disjoint $\Cal I_0\subseteq\Cal I$ such that 
$\mu(A\setminus\bigcup\Cal I_0)=0$.   Let $\sequencen{B'_n}$ be a 
sequence of balls covering $A\setminus\bigcup\Cal I_0$ with 
$\sum_{n=0}^{\infty}\mu 
B'_n\le\min(\bover12\epsilon,\beta_r\epsilon^r)$;  these surely all have 
radius at most $\epsilon$.   Let $\sequencen{B_n}$ be a sequence 
amalgamating $\Cal I_0$ with $\sequencen{B'_n}$;  then 
$A\subseteq\bigcup_{n\in\Bbb N}B_n$, every $B_n$ has radius at most 
$\epsilon$ and 
      
\Centerline{$\sum_{n=0}^{\infty}\mu B_n 
=\sum_{B\in\Cal I_0}\mu B+\sum_{n=0}^{\infty}\mu B'_n 
\le\mu G+\Bover12\epsilon 
\le\mu A+\epsilon$,} 
      
\noindent while the $B_n$ whose centres do not lie in $A$ must come from 
the sequence $\sequencen{B'_n}$, so their measures sum to at most 
$\bover12\epsilon\le\epsilon$. 
}%end of proof of 261F 
      
\cmmnt{\medskip 
      
\noindent{\bf Remark} In fact we can (if $A$ is not empty) arrange that 
the centre of {\it every} $B_n$ belongs to $A$.   This is an easy 
consequence of Besicovitch's Covering Lemma (see \S472 in Volume 4). 
}%end of comment 
      
\exercises{ 
\leader{261X}{Basic exercises (a)} Show that 261C is valid for 
any locally integrable real-valued function $f$;  in particular, for any 
$f\in\eusm L^p(\mu_D)$ for any $p\ge 1$, writing $\mu_D$ for the 
subspace measure on $D$. 
%261C 
      
\spheader 261Xb Show that 261C, 261Dc, 261Dd and 261E are valid for 
complex-valued functions $f$. 
%261E 
      
\sqheader 261Xc Take three disks in the plane, each touching the other 
two, so that they enclose an open region $R$ with three cusps.   In $R$ 
let $D$ be a disk tangent to each of the three original disks, and 
$R_0$, $R_1$, $R_2$ the three components of $R\setminus D$.   In each 
$R_j$ let $D_j$ be a disk tangent to each of the disks bounding $R_j$, 
and $R_{j0}$, $R_{j1}$, $R_{j2}$ the three components of $R_j\setminus 
D_j$.   Continue, obtaining 27 regions at the next step, 81 regions at 
the next, and so on. 
      
Show that the total area of the residual regions converges to zero as 
the process continues indefinitely.   \Hint{compare with the process in 
the proof of 261B.} 
%261B 
      
\leader{261Y}{Further exercises (a)} 
%\spheader 261Ya 
Formulate an abstract definition of 
`Vitali cover', meaning a family of sets satisfying the conclusion of 
261B in some sense, and corresponding generalizations of 261C-261E, 
covering (at least) (b)-(d) below. 
      
\header{261Yb}{\bf (b)} For $x\in\BbbR^r$, $k\in\Bbb N$ let $C(x,k)$ be 
the half-open 
cube of the form $\coint{2^{-k}z,2^{-k}(z+\tbf{1})}$, with $z\in \Bbb 
Z^r$, containing $x$.   Show that if $f$ is an integrable function on 
$\BbbR^r$ then 
      
\Centerline{$\lim_{k\to\infty}2^{kr}\int_{C(x,k)}f=f(x)$} 
      
\noindent for almost every $x\in\BbbR^r$. 
      
\spheader 261Yc Let $f$ be a real-valued function which is integrable 
over $\BbbR^r$.   Show that 
      
\Centerline{$\lim_{\delta\downarrow 
0}\Bover{1}{\delta^r}\int_{\coint{x,x+\delta\tbf{1}}}f=f(x)$} 
      
\noindent for almost every $x\in\BbbR^r$. 
      
\spheader 261Yd Give $X=\{0,1\}^{\Bbb N}$ its usual measure $\nu$ 
(254J).   For $x\in X$, $k\in\Bbb N$ set 
$C(x,k)=\{y:y\in X,\,y(i)=x(i)$ for $i<k\}$.   Show that if $f$ is any 
real-valued function which is integrable over $X$ then 
$\lim_{k\to\infty}2^k\int_{C(x,k)}fd\nu=f(x)$, 
$\lim_{k\to\infty}2^k\int_{C(x,k)}|f(y)-f(x)|\nu(dy)=0$ for almost every 
$x\in X$. 
      
\spheader 261Ye Let $f$ be a real-valued function which is integrable 
over $\BbbR^r$, and $x$ a point in the Lebesgue set of $f$.   Show that 
for every $\epsilon>0$ there is a $\delta>0$ such that 
\discrcenter{390pt}{$|f(x)-\int f(x-y)g(\|y\|)dy|\le\epsilon$ }whenever 
$g:\coint{0,\infty}\to\coint{0,\infty}$ is a non-increasing function 
such that $\int_{\Bbb R^r}g(\|y\|)dy=1$ and 
$\int_{B(\tbf{0},\delta)}g(\|y\|)dy\ge 1-\delta$.   \Hint{223Yg.} 
      
\spheader 261Yf Let $\frak T$ be the family of those measurable sets 
$G\subseteq\BbbR^r$ such that 
\discrcenter{390pt}{$\lim_{\delta\downarrow 0} 
\Bover{\mu(G\cap B(x,\delta))}{\mu B(x,\delta)}=1$ }for every $x\in G$. 
Show that $\frak T$ is a topology on $\BbbR^r$, the {\bf density 
topology} of $\BbbR^r$.   Show that a function $f:\BbbR^r\to\Bbb R$ is 
measurable iff it is $\frak T$-continuous at almost every point of 
$\Bbb R^r$. 
      
\spheader 261Yg A set $A\subseteq\BbbR^r$ is said to be {\bf porous} at 
$x\in\BbbR^r$ if $\limsup_{y\to x}\Bover{\rho(y,A)}{\|y-x\|}>0$,  
writing $\rho(y,A)=\inf_{z\in A}\|y-z\|$ (or $\infty$ if $A$ is empty). 
(i) Show that if $A$ is porous at all its points then it is negligible. 
(ii) Show that in the construction of 261B the residual set 
$A\setminus\bigcup\Cal I_0$ will be porous at all its points. 
 
\spheader 261Yh Let $A\subseteq\BbbR^r$ be a bounded set and $\Cal I$ a 
non-empty family of non-trivial closed balls covering $A$.   Show that 
for any $\epsilon>0$ there are disjoint $B_0,\ldots,B_n\in\Cal I$ such 
that $\mu^*A\le(3+\epsilon)^r\sum_{k=0}^n\mu B_k$. 
      
\spheader 261Yi Let $(X,\rho)$ be a metric space and $A\subseteq X$ any 
set, $x\mapsto\delta_x:A\to\coint{0,\infty}$ any bounded function. 
Show that if $\gamma>3$ then there is an $A'\subseteq A$ such that (i) 
$\rho(x,y)>\delta_x+\delta_y$ for all distinct $x$, $y\in A'$ (ii) 
$\bigcup_{x\in A}B(x,\delta_x) 
\subseteq\bigcup_{x\in A'}B(x,\gamma\delta_x)$, writing $B(x,\alpha)$ 
for the closed ball $\{y:\rho(y,x)\le\alpha\}$. 
      
\spheader 261Yj(i) Let $\Cal C$ be the family of those measurable sets 
$C\subseteq\BbbR^r$ such that 
\discrcenter{390pt}{$\limsup_{\delta\downarrow 0} 
\Bover{\mu(C\cap B(x,\delta))}{\mu B(x,\delta)}>0$ }for every $x\in C$. 
Show that $\bigcup\Cal C_0\in\Cal C$ for every $\Cal C_0\subseteq\Cal C$.
\Hint{215B(iv).}   
(ii) Show that any union of non-trivial closed balls in 
$\BbbR^r$ is Lebesgue measurable.   (Compare 
415Ye in Volume 4.) 
      
\spheader 261Yk Suppose that $A\subseteq\BbbR^r$ and that $\Cal I$ is a 
family of closed subsets of $\BbbR^r$ such that 
      
\inset{\noindent for every $x\in A$ there is an $\eta>0$ such that for 
every $\epsilon>0$ there is an $I\in\Cal I$ such that $x\in I$ and 
$0<\eta(\diam I)^r\le\mu I\le\epsilon$.} 
      
\noindent Show that there is a countable disjoint set 
$\Cal I_0\subseteq\Cal I$ such that 
$A\setminus\bigcup\Cal I_0$ is negligible. 
%261B 
 
\spheader 261Yl Let $\frak T'$ be the family of measurable sets 
$G\subseteq\BbbR^r$ such that whenever $x\in G$ and $\epsilon>0$ 
there is a $\delta>0$ such that $\mu(G\cap I)\ge(1-\epsilon)\mu I$ whenever 
$I$ is an interval containing $x$ and included in $B(x,\delta)$.    
Show that  
$\frak T'$ is a topology on $\BbbR^r$ intermediate between the density 
topology (261Yf) and the Euclidean topology. 
}%end of exercises 
      
\endnotes{ 
\Notesheader{261} 
In the proofs of 261B-261E above, I have done my best to 
follow the lines of the one-dimensional case;  this section amounts to a 
series of generalizations of the work of \S\S221 and 223. 
      
It will be clear that 
the idea of 261A/261B can be used on other shapes than balls.   To make 
it work in the form above, we need a family $\Cal I$ such that there is 
a constant $K$ for which 
      
\Centerline{$\mu I'\le K\mu I$} 
      
\noindent for every $I\in \Cal I$, where we write 
      
\Centerline{$I'=\{x:\inf_{y\in I}\|x-y\|\le\diam(I)\}$.} 
      
\noindent Evidently this will be true for many classes $\Cal I$ 
determined by the shapes of the sets involved;  for instance, if 
$E\subseteq\BbbR^r$ is any bounded set of strictly positive measure, 
the family $\Cal I=\{x+\delta E:x\in\BbbR^r,\,\delta>0\}$ will satisfy 
the condition. 
      
In 261Ya I challenge you to find an appropriate generalization of the 
arguments depending on the conclusion of 261B. 
      
Another way of using 261B is to say that because sets 
can be essentially covered by {\it disjoint} sequences of balls, it 
ought to be possible to use balls, rather than half-open intervals, in 
the definition of Lebesgue measure on $\BbbR^r$.   This is indeed so 
(261F).  The difficulty in using balls in the basic 
definition comes right at the start, in proving that if a ball is 
covered by finitely many balls then the sum of the volumes of the 
covering balls is at least the volume of the covered ball.    (There is 
a trick, using the compactness of closed balls and the openness of open 
balls, to extend such a proof to infinite covers.)   Of course you could 
regard this fact as `elementary', on the ground that Archimedes 
would have noticed if it weren't true, but nevertheless it would be 
something of a challenge to prove it, unless you were willing to wait 
for a version of Fubini's theorem, as some authors do. 
      
I have given the results in 261C-261D for arbitrary subsets $D$ of  
$\Bbb R^r$ not because I have any applications in mind in which 
non-measurable subsets are significant, but because I wish to make it 
possible to notice when measurability matters.   Of course it is 
necessary to interpret the integrals $\int_Dfd\mu$ in the way laid down 
in \S214.   The game is given away in part (c) of the proof of 261C, 
where I rely on the fact that if $f$ is integrable over $D$ then there 
is an integrable $\tilde f:\BbbR^r\to\Bbb R$ such that 
$\int_F\tilde f=\int_{D\cap F}f$ for every measurable 
$F\subseteq\BbbR^r$.   In 
effect, for all the questions dealt with here, we can replace $f$, $D$ 
by $\tilde f$, $\BbbR^r$. 
      
The idea of 261C is that, for almost every $x$, $f(x)$ is approximated 
by its mean value on small balls $B(x,\delta)$, ignoring the missing 
values on $B(x,\delta)\setminus(D\cap\dom f)$;  261E is a sharper 
version of the same idea. 
The formulae of 261C-261E mostly involve the expression 
$\mu B(x,\delta)$.   Of course this is just $\beta_r\delta^r$.   But I 
think that leaving it unexpanded is actually more illuminating, as well 
as avoiding sub- and superscripts, since it makes it clearer what these 
density theorems are really about.   In \S472 of Volume 4 I will revisit  
this material, showing that a surprisingly large proportion of the ideas  
can be applied to arbitrary Radon measures on $\BbbR^r$, even though 
Vitali's theorem (in the form stated here) is no longer valid. 
      
}%end of notes 
      
      
\discrpage 
      
