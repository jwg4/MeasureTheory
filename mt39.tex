\frfilename{mt39.tex} 
\versiondate{17.11.10} 
\copyrightdate{2010} 
 
\def\chaptername{Measurable algebras} 
 
\newchapter{39} 
 
In the final chapter of this volume, I present results connected with 
the following question:  which algebras can appear as the underlying 
Boolean algebras of measure algebras?   Put in this form, there is a 
trivial answer (391A).   The proper question is rather:  which algebras 
can appear as the underlying Boolean algebras of semi-finite measure 
algebras?   This is easily reducible to the question:  which algebras 
can appear as the underlying Boolean algebras of probability algebras? 
Now in one sense Maharam's theorem (\S332) gives us the answer exactly: 
they are the countable simple products of the measure algebras of 
$\{0,1\}^{\kappa}$ for cardinals $\kappa$.   But if we approach from 
another direction, things are more interesting.   Probability algebras 
share a very large number of very special properties.   Can we find a 
selection of these properties which will be sufficient to force an 
abstract Boolean algebra to be a probability 
algebra when endowed with a suitable functional? 
 
No fully satisfying answer to this question is known.   But in exploring 
the possibilities we encounter some interesting and important ideas. 
In \S391 I discuss algebras which have strictly positive additive 
real-valued functionals;  for such algebras, weak 
$(\sigma,\infty)$-distributivity is necessary and sufficient for the 
existence of a measure;  so we are led to look for conditions sufficient 
to ensure that there is a strictly positive additive functional.   A 
slightly different approach lies through the concept of `submeasure'. 
Submeasures arise naturally in the theories of topological Boolean 
algebras (393J), topological Riesz spaces (393K) and vector measures 
(394P), and on any given algebra there is a strictly positive 
`uniformly exhaustive' 
submeasure iff there is a strictly positive additive functional;  this 
is the Kalton-Roberts theorem (392F).    
 
Submeasures in general are common, but correspondingly limited in what they 
can tell us about a structure in the absence of further properties.   
Uniformly exhaustive submeasures are not far from additive functionals. 
An intermediate class, the `exhaustive' submeasures, has been intensively 
studied, originally in the hope that they might lead to characterizations 
of measurable algebras, but more recently for their own sake.   Just as 
additive functionals lead to measurable algebras, totally 
finite exhaustive submeasures lead to `Maharam algebras' (\S393).   For 
many years it was not known whether every exhaustive submeasure was 
exhaustive (equivalently, whether every Maharam algebra was a measurable 
algebra);  an example was eventually found by M.Talagrand, and is presented 
in \S394.
 
In \S395, I look at a characterization  
of measurable algebras in terms of the special properties 
which the automorphism group of a measure algebra must have (Kawada's 
theorem, 395Q).   \S396 complements the previous section by looking 
briefly at the subgroups of an automorphism group $\Aut\frak A$ which 
can appear as groups of measure-preserving automorphisms. 
 
\discrpage 
 
