\wheader{}{10}{4}{4}{100pt}

Introduction to Volume 4 \vtmpb{7.4.03}\pagereference{12}{}

\wheader{}{10}{4}{4}{100pt}

Chapter 41:  Topologies and measures I

\chapintrosection{17.4.10}{14}{}

\section{411}{Definitions}{31.12.08}{14}{}
{Topological, inner regular, $\tau$-additive, outer regular,
locally finite, effectively locally finite, quasi-Radon, Radon,
completion regular, Baire, Borel and strictly positive measures;
measurable and almost continuous functions;   self-supporting sets and
supports of measures;  Stone spaces;  Dieudonn\'e's measure.}

\section{412}{Inner regularity}{31.1.05}{20}{}
{Exhaustion;  Baire measures;  Borel measures on metrizable
spaces;  completions and c.l.d.\ versions;  complete locally determined
spaces;  \imp\ functions;  subspaces;  indefinite-integral measures;
products;  outer regularity.}

\section{413}{Inner measure constructions}{25.2.05}{31}{}
{Inner measures;  constructing a measure from an inner measure;
the inner measure defined by a measure;  complete locally determined
spaces;  extension of functionals to measures;  countably compact
classes;  constructing measures dominating given functionals.}

\section{414}{$\tau$-additivity}{26.1.10}{50}{}
{Semi-continuous functions;  supports;  strict
localizability;  subspace measures;  regular topologies;  density
topologies;  lifting topologies.}

\allowmorestretch{468}{

\section{415}{Quasi-Radon measure spaces}{7.4.05}{57}{}
{Strict localizability;  subspaces;  regular topologies;
hereditarily Lindel\"of spaces;  products of separable metrizable
spaces;  comparison and specification of quasi-Radon
measures;  construction of quasi-Radon measures extending given
functionals;  indefinite-integral measures;  $L^p$ spaces;  Stone
spaces.}
}%end of allowmorestretch

\section{416}{Radon measure spaces}{24.6.05}{71}{}
{Radon and quasi-Radon measures;  specification of Radon measures;
c.l.d.\ versions of Borel measures;  locally compact topologies;
constructions of Radon
measures extending or dominating given functionals;  additive
functionals on Boolean algebras and Radon measures on Stone spaces;
subspaces;
products;  Stone spaces of measure algebras;  compact and perfect
measures;  representation
of homomorphisms of measure algebras.}

\section{417}{$\tau$-additive product measures}{9.3.10}{85}{}
{The product of two effectively locally finite $\tau$-additive
measures;  the product of many $\tau$-additive probability measures;
Fubini's theorem;  generalized associative law;  measures on subproducts
as image measures;  products of strictly positive measures;  quasi-Radon
and Radon product measures;  when `ordinary' product
measures are $\tau$-additive;  continuous functions and Baire
$\sigma$-algebras in product spaces.}

\section{418}{Measurable functions and almost continuous functions}
{19.8.05}{106}{}
{Measurable functions;  into (separable) metrizable spaces;  and
image measures;  almost continuous functions;  continuity,
measurability, image measures;  expressing Radon measures as
images of Radon measures;  Prokhorov's theorem on projective limits of
Radon measures;  representing measurable functions into $L^0$ spaces.}

\section{419}{Examples}{2.12.05}{121}{}
{A nearly quasi-Radon measure;  a Radon measure space in which the
Borel sets are inadequate;  a nearly Radon measure;  the Stone space of
the Lebesgue measure algebra;  measures with domain $\Cal{P}\omega_1$;
notes on Lebesgue measure;  the split interval.}

\wheader{}{10}{4}{4}{100pt}

 Chapter 42: Descriptive set theory

\chapintrosection{5.5.02}{133}{}

\section{421}{Souslin's operation}{14.12.07}{133}{}
{Souslin's operation;  is idempotent;  as a projection operator;
Souslin-F sets;  *constituents.}

\section{422}{K-analytic spaces}{4.12.04}{142}{}
{Usco-compact relations;  K-analytic sets;  and Souslin-F sets;
*First Separation Theorem.}

\section{423}{Analytic spaces}{13.8.05}{148}{}
{Analytic spaces;  are K-analytic spaces with countable networks;
Souslin-F sets;   Borel measurable functions;  injective images of
Polish spaces;  non-Borel analytic sets;  a von Neumann-Jankow
selection theorem;  *constituents of coanalytic sets.}

\section{424}{Standard Borel spaces}{21.3.08}{158}{}
{Basic properties;  isomorphism types;  subspaces;  Borel
measurable actions of Polish groups.}

\section{*425}{Realization of automorphisms}{9.8.13}{164}{}
{Extending group actions;  T\"ornquist's theorem.}

\wheader{}{10}{4}{4}{100pt}

Chapter 43: Topologies and measures II

\chapintrosection{2.6.02}{173}{}

\section{431}{Souslin's operation}{9.4.05}{173}{}
{The domain of a complete locally determined measure is closed
under Souslin's operation;  the kernel of a Souslin scheme is
approximable from within;  Baire-property algebras.}

\section{432}{K-analytic spaces}{2.10.13}{177}{}
{Topological measures on K-analytic spaces;  extensions to Radon
measures;  expressing Radon measures as images of Radon measures;
Choquet capacities.}

\section{433}{Analytic spaces}{27.6.10}{182}{}
{Measures on spaces with countable networks;  inner regularity of Borel
measures;  expressing Radon measures as
images of Radon measures;  measurable and almost continuous functions;
the von Neumann-Jankow selection theorem;  products;  extension of
measures on $\sigma$-subalgebras;  standard Borel spaces.}

\section{434}{Borel measures}{1.9.04}{186}{}
{Classification of Borel measures;  Radon spaces;  universally
measurable sets and functions;  Borel-measure-compact,
Borel-measure-complete and pre-Radon spaces;  countable compactness and
countable tightness;
quasi-dyadic spaces and completion regular measures;  first-countable
spaces and Borel product measures.}

\section{435}{Baire measures}{16.8.08}{204}{}
{Classification of Baire measures;  extension of Baire measures to
Borel measures (Ma\v{r}\'{\i}k's theorem);  measure-compact spaces.}

\section{436}{Representation of linear functionals}{9.5.11}{209}{}
{Smooth and sequentially smooth linear functionals;  measures and
sequentially smooth functionals;  Baire measures;  
sequential spaces and products of Baire
measures;  quasi-Radon measures
and smooth functionals;  locally compact spaces and Radon measures.}

\section{437}{Spaces of measures}{5.11.12}{219}{}
{Smooth and sequentially smooth duals;  signed measures;  embedding
spaces of measurable functions in
the bidual of $C_b(X)$;  vague and narrow topologies;  product measures;
extreme points;  uniform tightness;  total variation metric,
Kantorovich-Rubinshtein metric;  invariant probability measures;
Prokhorov spaces.}

\section{438}{Measure-free cardinals}{13.12.06}{244}{}
{Measure-free cardinals;  point-finite families of sets with
measurable unions;  measurable functions into metrizable spaces;  Radon
and measure-compact metric spaces;  metacompact spaces;  hereditarily
weakly $\theta$-refinable spaces;  when $\frak{c}$ is measure-free.}

\section{439}{Examples}{24.9.04}{259}{}
{Measures on $[0,1]$ not extending to Borel measures;  universally
negligible sets;  Hausdorff measures are rarely semi-finite;  a smooth
linear functional not expressible as an integral;  a first-countable
non-Radon space;  Baire measures not extending to Borel measures;
$\BbbN^{\frak{c}}$ is not Borel-measure-compact;  the
Sorgenfrey line;  $\Bbb{Q}$ is not a Prokhorov space.}

\wheader{}{10}{4}{4}{100pt}

Chapter 44: Topological groups

\chapintrosection{14.10.02}{274}{}

\section{441}{Invariant measures on locally compact
spaces}{3.1.06}{274}{}
{Measures invariant under homeomorphisms;  Haar measures;
measures invariant under isometries.}

\section{442}{Uniqueness of Haar measure}{21.3.07}{283}{}
{Two (left) Haar measures are multiples of each other;  left and
right Haar measures;  Haar measurable and Haar negligible sets;  the
modular function of a group;  formulae for ${\int}f(x^{-1})dx$,
${\int}f(xy)dx$.}

\section{443}{Further properties of Haar measure}{14.1.13}{289}{}
{The Haar measure algebra of a group carrying Haar measures;
actions of the group on the Haar measure algebra;  locally compact
groups;  actions of the group on $L^0$ and $L^p$;  the bilateral
uniformity;  Borel sets are adequate;  completing the group;  expressing
an arbitrary Haar measure in terms of a Haar measure on a locally
compact group;  completion regularity of Haar measure;  invariant
measures on the set of left cosets of a closed subgroup of a locally
compact group;  modular functions of subgroups and quotient groups;
transitive actions of compact groups on compact spaces.}

\section{444}{Convolutions}{23.7.07}{314}{}
{Convolutions of quasi-Radon measures;  the Banach algebra of
signed $\tau$-additive measures;  continuous actions and corresponding
actions on $L^0(\nu)$
for an arbitrary quasi-Radon measure $\nu$;  convolutions of measures
and functions;  indefinite-integral measures over a Haar measure $\mu$;
convolutions of functions;  $L^p(\mu)$;  approximate identities;
convolution in $L^2(\mu)$.}

\section{445}{The duality theorem}{20.3.08}{336}{}
{Dual groups;  Fourier-Stieltjes transforms;  Fourier transforms;
identifying the dual group with the maximal ideal space of $L^1$;  the
topology of the dual group;  positive definite functions;  Bochner's
theorem;  the Inversion Theorem;  the Plancherel Theorem;  the Duality
Theorem.}

\section{446}{The structure of locally compact groups}{8.10.13}{357}{}
{Finite-dimensional representations separate the points of a
compact group;  groups with no small subgroups have $B$-sequences;
chains of subgroups.}

\ifdim\pagewidth>467pt\fontdimen3\tenrm=2pt\fi

\section{447}{Translation-invariant liftings}{7.1.10}{371}{}
{Translation-invariant liftings and lower densities;
Vitali's theorem and a
density theorem for groups with $B$-sequences;   Haar measures have
transla\discretionary{}{-}{}tion-{\vthsp}invariant liftings.}

\fontdimen3\tenrm=1.67pt

\section{448}{Polish group actions}
{12.4.13}{382}{}
{Countably full local semigroups of $\Aut\frak{A}$;
$\sigma$-equidecomposability;   countably non-paradoxical groups;
$G$-invariant additive functions from $\frak{A}$ to
$L^{\infty}(\frak{C})$;  measures invariant under Polish group actions
(the Nadkarni-Becker-Kechris theorem);  measurable liftings of $L^0$;
the Borel structure of $L^0$;  representing a Borel measurable action on
a measure algebra by a Borel
measurable action on a Polish space (Mackey's theorem).}

\section{449}{Amenable groups}{13.6.13}{397}{}
{Amenable groups;  permanence properties;  the greatest ambit of a
topological group;  locally compact
amenable groups;  Tarski's theorem;  discrete amenable groups; 
isometry-invariant extensions of Lebesgue measure.}

\wheader{}{10}{2}{2}{100pt}

Chapter 45: Perfect measures, disintegrations and processes

\chapintrosection{8.12.02}{420}{}

\section{451}{Perfect, compact and countably compact
measures}{8.11.07}{421}{}
{Basic properties of the three classes;  subspaces, completions,
c.l.d.\ versions, products;  measurable functions from compact
measure spaces to metrizable spaces;  *weakly
$\alpha$-favourable spaces.}

\section{452}{Integration and disintegration of
measures}{6.11.08}{434}{}
{Integrating families of measures;  $\tau$-additive and
Radon measures;  disintegrations and regular conditional probabilities;
disintegrating countably compact
measures;  disintegrating Radon measures;  *images of countably compact
measures.}

\section{453}{Strong liftings}{22.3.10}{449}{}
{Strong and almost strong liftings;  existence;  on product spaces;
disintegrations
of Radon measures over spaces with almost strong liftings;  Stone
spaces;  Losert's example.}

\section{454}{Measures on product spaces}{6.12.05}{461}{}
{Perfect, compact and countably compact measures on product
spaces;  extension of finitely additive functions with perfect countably
additive marginals;  Kolmogorov's extension theorem;  measures defined
from conditional distributions;  distributions of random processes;
measures on $C(T)$ for Polish $T$;  completion regular product measures.}

\section{455}{Markov and L\'evy processes}
{18.1.09}{470}{}
{Realization of a Markov process with given
conditional distributions;  the Markov property for stopping times taking
countably many values -- disintegrations and conditional expectations;
Radon conditional distributions;  narrowly continuous and uniformly
time-continuous systems of conditional distributions;  \cadlag{}
and \callal{} functions;  extending the distribution of a process to a
Radon measure;  when the subspace
measure on the \cadlag{} functions is quasi-Radon;
general stopping times, hitting times;  the strong Markov property;
independent
increments, L\'evy processes;  expressing the strong Markov property 
with an \imp\ function.}

\section{456}{Gaussian distributions}{19.5.10}{510}{}
{Gaussian distributions and processes;  covariance matrices, correlation
and independence;  supports;  universal Gaussian distributions;
cluster sets of $n$-dimensional processes;  $\tau$-additivity.}

\section{457}{Simultaneous extension of measures}{18.1.13}{527}{}
{Extending families of finitely additive functionals;  Strassen's
theorem;  extending families of measures;  examples;  the Wasserstein
metric.}

\section{458}{Relative independence and relative
products}{16.1.07}{543}{}
{Relatively independent algebras of measurable sets;
relative distributions and relatively independent random variables;
relatively independent
subalgebras of a probability algebra;  relative free products of
probability algebras;  relative products of probability spaces;
existence of relative products.}

\section{459}{Symmetric measures and exchangeable random
variables}{7.12.10}{562}{}
{Exchangeable families of \imp\ functions;
De Finetti's theorem;  countably compact symmetric measures on product
spaces disintegrate into product measures;  symmetric quasi-Radon
measures;  other actions of symmetric groups.}

% \wheader{}{10}{4}{4}{20pt}

% Concordance to Part I \pagereference{576}{}

%577 pages
