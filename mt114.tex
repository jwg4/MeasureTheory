\frfilename{mt114.tex}
\versiondate{14.6.05}
\copyrightdate{1994}

\def\chaptername{Measure spaces}
\def\sectionname{Lebesgue measure on $\Bbb R$}

\newsection{114}
\def\headlinesectionname{Lebesgue measure on ${\eightBbb R}$}

Following the very abstract ideas of \S\S111-113, we have an urgent
need for a non-trivial example of a measure space.   By far the most
important example is the real line with Lebesgue measure, and I now
proceed to a description of this measure (114A-114E), with a few of its
basic properties.

The principal ideas of this section are repeated in \S115, and if
you have encountered Lebesgue measure before, or feel confident in your
ability to deal with two- and three-dimensional spaces at the same time
as doing some difficult analysis, you could go directly to that
section, turning back to this one only when a specific reference is
given.

\leader{114A}{Definitions (a)} For the purposes of this section, a {\bf
half-open interval} in $\Bbb R$ is a set of the form
$\coint{a,b}=\{x:a\le x<b\}$, where $a$, $b\in\Bbb R$.

\cmmnt{Observe that I allow $b\le a$ in this formula;  in this case
$\coint{a,b}=\emptyset$ (see 1A1A).}

\header{114Ab}{\bf (b)}\cmmnt{ If $I\subseteq \Bbb R$ is a half-open
interval, then either
$I=\emptyset$ or $I=\coint{\inf I,\sup I}$, so that its endpoints are
well defined.}   We\cmmnt{ may therefore} define the {\bf length}
$\lambda I$ of a half-open interval $I$ by setting

\Centerline{$\lambda\emptyset = 0$,
\quad $\lambda\coint{a,b}=b-a$ if $a<b$.}

\leader{114B}{Lemma} If $I\subseteq\Bbb R$ is a half-open interval and
$\langle I_j\rangle_{j\in\Bbb N}$ is a sequence of half-open intervals
covering $I$, then $\lambda I\le\sum_{j=0}^{\infty}\lambda I_j$.

\proof{{\bf (a)}
If $I=\emptyset$ then of course $\lambda I=0\le\sum_{j=0}^{\infty}\lambda
I_j$.   Otherwise, take
$I=\coint{a,b}$, where $a<b$.    For each $x\in \Bbb R$ let $H_x$ be the
half-line $\ooint{-\infty,x}$, and consider the set

\Centerline{$A=\{x:a\le x\le b,\,
x-a\le\sum_{j=0}\sp{\infty}\lambda(I_j\cap
H_x)\}$.}

\noindent (Note that if $I_j=\coint{c_j,d_j}$ then $I_j\cap
H_x=\coint{c_j,\min(d_j,x)}$, so $\lambda(I_j\cap H_x)$ is always
defined.)   We have $a\in A$ (because
$a-a=0\le\sum_{j=0}\sp{\infty}\lambda(I_j\cap
H_a)$) and of course $A\subseteq[a,b]$, so $c=\sup A$ is defined, and
belongs to $[a,b]$.

\medskip

{\bf (b)} We find now that $c\in A$.

\prooflet{

$$\eqalign{\Prf\ c-a&=\sup_{x\in A}x-a\cr
&\le\sup_{x\in A}\sum_{j=0}^{\infty}\lambda(I_j\cap H_x)
\le\sum_{j=0}^{\infty}\lambda(I_j\cap H_c). \text{ \Qed}\cr}$$
}

\medskip

{\bf (c)} \Quer\ Suppose, if possible, that $c<b$.   Then $c\in
\coint{a,b}$, so there is some $k\in\Bbb N$ such that $c\in I_k$.
Express $I_k$ as $\coint{c_k,d_k}$;  then $x=\min(d_k,b)>c$.    For each
$j$, $\lambda(I_j\cap H_x)\ge\lambda(I_j\cap H_c)$, while

\Centerline{$\lambda(I_k\cap H_x)=\lambda(I_k\cap H_c)+x-c$.}

\noindent So

$$\eqalign{\sum_{j=0}^{\infty}\lambda(I_j\cap H_x)
&\ge\sum_{j=0}^{\infty}\lambda(I_j\cap H_c)+x-c\cr
&\ge c-a+x-c
=x-a,\cr}$$

\noindent so $x\in A$;  but $x>c$ and $c=\sup A$.  \Bang

\medskip

{\bf (d)} We conclude that $c=b$, so that $b\in A$ and

\Centerline{$b-a\le\sum_{j=0}^{\infty}\lambda(I_j\cap
H_b)\le\sum_{j=0}^{\infty}\lambda I_j$,}

\noindent as claimed.
}%end of proof of 114B


\vleader{36pt}{114C}{Definition} Now\cmmnt{, and for the rest of this
section,} define $\theta:\Cal P\Bbb R\to[0,\infty]$ by
writing

$$\eqalign{\theta A=\inf\{\sum_{j=0}\sp{\infty}\lambda I_j:\langle
I_j\rangle_{j\in\Bbb N}
\text{ is a sequence of }&\text{half-open intervals}\cr
&\text{such that }
A\subseteq\bigcup_{j\in\Bbb N}I_j\}.\cr}$$

\noindent\cmmnt{Observe that every $A$ can be covered by some sequence
of half-open intervals -- e.g.,
$A\subseteq\bigcup_{n\in\Bbb N}\coint{-n,n}$;  so that if we interpret
the sums in $[0,\infty]$, as in 112Bc above, we always have a non-empty
set to take the infimum of, and $\theta A$ is always defined in
$[0,\infty]$.
This function} $\theta$ is called {\bf Lebesgue outer measure} on
$\Bbb R$\cmmnt{;  the phrase is justified by (a) of the next proposition}.



\leader{114D}{Proposition} (a) $\theta$ is an outer measure on $\Bbb R$.

(b) $\theta I=\lambda I$ for every half-open interval
$I\subseteq\Bbb R$.

\proof{{\bf (a)(i)} $\theta$ takes values in $[0,\infty]$
because every $\theta A$ is the infimum of a non-empty subset of
$[0,\infty]$.

\medskip

\quad{\bf (ii)} $\theta\emptyset=0$ because (for instance) if we set
$I_j=\emptyset$ for every $j$, then every $I_j$ is a half-open interval
(on the convention I am using) and
$\emptyset\subseteq\bigcup_{j\in\Bbb N}I_j$, $\sum_{j=0}\sp{\infty}\lambda I_j=0$.

\medskip

\quad{\bf (iii)} If $A\subseteq B$ then whenever
$B\subseteq\bigcup_{j\in\Bbb N}I_j$ we have
$A\subseteq\bigcup_{j\in\Bbb N}I_j$, so $\theta A$ is the infimum of a set at least as large as that
involved in the definition of $\theta B$, and $\theta A\le\theta B$.

\medskip

\quad{\bf (iv)} Now suppose that $\langle A_n\rangle_{n\in\Bbb N}$ is a
sequence of subsets of $\Bbb R$, with union $A$.   For any $\epsilon>0$,
we can choose, for each $n\in\Bbb N$, a sequence $\langle
I_{nj}\rangle_{j\in\Bbb N}$ of half-open intervals such that
$A_n\subseteq\bigcup_{j\in\Bbb N}I_{nj}$ and
$\sum_{j=0}\sp{\infty}\lambda I_{nj}\le\theta A_n+2^{-n}\epsilon$.
(You should perhaps check that this formulation is valid whether $\theta
A_n$ is finite or infinite.)   Now by 111F(b-ii)
there is a bijection from $\Bbb N$ to $\Bbb N\times\Bbb N$;  express
this in the form $m\mapsto(k_m,l_m)$.   Then $\sequence{m}{I_{k_m,l_m}}$
is a sequence of half-open intervals, and

\Centerline{$A\subseteq\bigcup_{m\in\Bbb N}I_{k_m,l_m}$.}

\noindent \Prf\  If $x\in A=\bigcup_{n\in\Bbb N}A_n$ there must be an
$n\in\Bbb N$ such that $x\in A_n\subseteq\bigcup_{j\in\Bbb N}I_{nj}$, so
there is a $j\in\Bbb N$ such that $x\in I_{nj}$.   Now
$m\mapsto(k_m,l_m)$ is surjective, so there is an $m\in\Bbb N$ such that
$k_m=n$ and $l_m=j$, in which case $x\in I_{k_m,l_m}$.\ \Qed

Next,

\Centerline{$\sum_{m=0}\sp{\infty}\lambda I_{k_m,l_m}
\le\sum_{n=0}\sp{\infty}\sum_{j=0}\sp{\infty}\lambda I_{nj}$.}

\noindent \Prf\ If $M\in\Bbb N$, then
$N=\max(k_0,k_1,\ldots,k_M,l_0,l_1,\ldots,l_M)$ is finite;   because
every $\lambda I_{nj}$ is
greater than or equal to $0$, and any pair $(n,j)$ can appear at most
once as a $(k_m,l_m)$,

\Centerline{$\sum_{m=0}^M\lambda I_{k_m,l_m}
\le\sum_{n=0}^N\sum_{j=0}^N\lambda I_{nj}
\le\sum_{n=0}^N\sum_{j=0}^{\infty}\lambda I_{nj}
\le\sum_{n=0}^{\infty}\sum_{j=0}^{\infty}\lambda I_{nj}$.}

\noindent So

\Centerline{$\sum_{m=0}^{\infty}\lambda I_{k_m,l_m}
=\lim_{M\to\infty}\sum_{m=0}^M\lambda I_{k_m,l_m}
\le\sum_{n=0}^{\infty}\sum_{j=0}^{\infty}\lambda I_{nj}$.  \Qed}

Accordingly

$$\eqalign{\theta A
&\le\sum_{m=0}\sp{\infty}\lambda I_{k_m,l_m}\cr
&\le\sum_{n=0}\sp{\infty}\sum_{j=0}\sp{\infty}\lambda I_{nj}\cr
&\le\sum_{n=0}\sp{\infty}(\theta A_n+2^{-n}\epsilon)\cr
&=\sum_{n=0}\sp{\infty}\theta A_n+\sum_{n=0}\sp{\infty}2^{-n}\epsilon\cr
&=\sum_{n=0}\sp{\infty}\theta A_n+2\epsilon.\cr}$$

\noindent Because $\epsilon$ is arbitrary, $\theta
A\le\sum_{n=0}\sp{\infty}\theta A_n$ (again, you should check that this
is valid whether or not $\sum_{n=0}\sp{\infty}\theta A_n$ is finite).
As $\langle A_n\rangle_{n\in\Bbb N}$ is arbitrary, $\theta$ is an outer
measure.

\medskip

{\bf (b)} Because we can always take $I_0=I$, $I_j=\emptyset$ for
$j\ge 1$, to obtain a sequence of half-open intervals covering $I$ with
$\sum_{j=0}\sp{\infty}\lambda I_j=\lambda I$, we surely have $\theta
I\le\lambda I$.   For the reverse inequality, use 114B:  if
$I\subseteq\bigcup_{j\in\Bbb N}I_j$, then $\lambda
I\le\sum_{j=0}^{\infty}\lambda I_j$;   as
$\langle I_j\rangle_{j\in\Bbb N}$ is arbitrary, $\theta I\ge\lambda I$ and $\theta I=\lambda I$, as
required.
}%end of proof of 114D

\cmmnt{\medskip

\noindent{\bf Remark} There is an ungainly shift in the argument of
(a-iv) above, in the stage

\Centerline{`$\theta A\le\sum_{m=0}^{\infty}\lambda
I_{k_m,l_m}\le\sum_{n=0}^{\infty}\sum_{j=0}^{\infty}\lambda I_{nj}$'.}

\noindent I dare say you would have believed me if I had suppressed the
$k_m$, $l_m$ altogether and simply written `because
$A\subseteq\bigcup_{n,j\in\Bbb N}I_{nj}$, $\theta
A\le\sum_{n=0}^{\infty}\sum_{j=0}^{\infty}\lambda I_{nj}$'.   I hope that
you will not find it too demoralizing if I suggest that such a jump is
not quite safe.   My reasons for interpolating a name for a bijection
between $\Bbb N$ and $\Bbb N\times\Bbb N$, and taking a couple of lines
to say explicitly that $\sum_{m=0}^{\infty}\lambda
I_{k_m,l_m}\le\sum_{n=0}^{\infty}\sum_{j=0}^{\infty}\lambda I_{nj}$, are
the following.   To start with, there is the formal point that the
definition 114C demands a simple sequence, not a double sequence.   Is it
really obvious that it doesn't matter here?   If so, why?   There can be
no way to justify the shift which does not rely on the facts that
$\Bbb N\times\Bbb N$ is countable and every $\lambda I_{nj}$ is
non-negative.
If either of those were untrue, the method would be in grave danger of
failing.

At some point we shall certainly need to discuss sums over infinite index
sets other than $\Bbb N$, including uncountable index sets.   I have
already touched on these in 112Bd, and I will return to them in 226A in
Volume 2.
For the moment, I feel that we have quite enough new ideas to cope with,
and that what we need here is a reasonably honest expedient to deal with
the question immediately before us.

You may have noticed, or guessed, that some of the inequalities `$\le$'
here must actually be equalities;  if so, check your guess in 114Ya.
}%end of comment

\leader{114E}{Definition} Because Lebesgue outer
measure\cmmnt{ (114C)} is\cmmnt{ indeed} an
outer measure\cmmnt{ (114Da)},  we may use it to construct a measure
$\mu$, using \Caratheodory's method\cmmnt{ (113C)}.
This measure is {\bf Lebesgue measure on $\Bbb R$}.
The sets $E$ measured by $\mu$\cmmnt{ (that is, for
which $\theta(A\cap E)+\theta(A\setminus E)=\theta A$ for every
$A\subseteq\Bbb R$)} are called {\bf Lebesgue measurable}.

Sets which are negligible for $\mu$ are called {\bf Lebesgue
negligible}\cmmnt{;  note that these are just the sets $A$ for which
$\theta A=0$, and are all Lebesgue measurable (113Xa)}.



\leader{114F}{Lemma} Let $x\in\Bbb R$.   Then
$H_x=\ooint{-\infty,x}$ is
Lebesgue measurable for every $x\in\Bbb R$.

\proof{{\bf (a)} The point is that $\lambda I=\lambda(I\cap
H_x)+\lambda(I\setminus H_x)$ for every half-open interval
$I\subseteq\Bbb R$.   \Prf\ If either $I\subseteq H_x$ or $I\cap
H_x=\emptyset$, this is trivial.   Otherwise, $I$ must be of the form
$\coint{a,b}$, where $a<x<b$.   Now $I\cap H_x=\coint{a,x}$ and
$I\setminus H_x=\coint{x,b}$  are both half-open intervals, and

\Centerline{$\lambda(I\cap H_x)+\lambda(I\setminus H_x)=(x-a)+(b-x)
=b-a=\lambda I$. \Qed}

\medskip

{\bf (b)} Now suppose that $A$ is any subset of $\Bbb R$, and
$\epsilon>0$.   Then we can find a sequence $\langle
I_j\rangle_{j\in\Bbb N}$
of half-open intervals such that $A\subseteq\bigcup_{j\in\Bbb N}I_j$ and
$\sum_{j=0}^{\infty}\lambda I_j\le\theta A+\epsilon$.   Now
$\langle I_j\cap H_x\rangle_{j\in\Bbb N}$ and
$\langle I_j\setminus  H_x\rangle_{j\in\Bbb N}$ are sequences of
half-open intervals and
$A\cap H_x\subseteq\bigcup_{j\in\Bbb N}(I_j\cap H_x)$,
$A\setminus H_x\subseteq\bigcup_{j\in\Bbb N}(I_j\setminus H_x)$.   So

$$\eqalign{\theta(A\cap H_x)+\theta(A\setminus H_x)
&\le\sum_{j=0}^{\infty}\lambda(I_j\cap
H_x)+\sum_{j=0}^{\infty}\lambda(I_j\setminus H_x) \cr
&=\sum_{j=0}^{\infty}\lambda I_j
\le\theta A+\epsilon.\cr}$$

\noindent Because $\epsilon$ is arbitrary,
$\theta(A\cap H_x)+\theta(A\setminus H_x)\le\theta A$;
because $A$ is arbitrary,
$H_x$ is measurable, as remarked in 113D.
}%end of proof of 114F

\leader{114G}{Proposition} All Borel subsets of $\Bbb R$ are Lebesgue
measurable;  in particular, all open sets, and all sets of the following
classes, together with countable unions of them:

(i) open intervals $\left]a,b\right[$, $\left]-\infty,b\right[$,
$\left]a,\infty\right[$, $\left]-\infty,\infty\right[$, where
$a<b\in\Bbb R$;

(ii) closed intervals $[a,b]$, where $a\le b\in\Bbb R$;

(iii) half-open intervals $\left[a,b\right[$, $\left]a,b\right]$,
$\left]-\infty,b\right]$, $\left[a,\infty\right[$, where $a<b$ in
$\Bbb R$.

\noindent We have\cmmnt{ moreover} the following formula for the measures of such sets, writing $\mu$ for Lebesgue measure:

\Centerline{$\mu\left]a,b\right[=\mu[a,b]
=\mu\left[a,b\right[=\mu\left]a,b\right]=b-a$}

\noindent whenever $a\le b$ in $\Bbb R$, while all the unbounded
intervals have infinite measure.   It follows that every countable
subset of $\Bbb R$ is measurable and of zero measure.


\proof{{\bf (a)} I show first that all open subsets of $\Bbb R$
are measurable.   \Prf\ Let $G\subseteq\Bbb R$ be open.   Let
$K\subseteq\Bbb Q\times\Bbb Q$ be the set of pairs $(q,q')$ of rational
numbers such that $\coint{q,q'}\subseteq G$.   Now by the remarks in
111E-111F
-- specifically, 111Eb, showing that $\Bbb Q$ is countable, 111F(b-iii),
showing that products of countable sets are countable, and 111F(b-i),
showing that subsets of countable sets are countable -- we see that $K$
is countable.   Also, every $\coint{q,q'}$ is measurable, being
$H_{q'}\setminus H_q$ in the language of 114F.   So, by 111Fa,
$G'=\bigcup_{(q,q')\in K}\coint{q,q'}$ is measurable.

By the definition of $K$, $G'\subseteq G$.   On the other hand, if
$x\in G$, there is an $\epsilon>0$
such that $\ooint{x-\epsilon,x+\epsilon}\subseteq G$.
Now there are rational numbers $q\in\ocint{x-\epsilon,x}$ and
$q'\in\ocint{x,x+\epsilon}$, so that $(q,q')\in K$ and
$x\in\coint{q,q'}\subseteq G'$.
As $x$ is arbitrary, $G=G'$ and $G$ is measurable.\ \Qed

\medskip

{\bf (b)} Now the family $\Sigma$ of Lebesgue measurable sets is a
$\sigma$-algebra of subsets of $\Bbb R$ including the family of open
sets, so must contain every Borel set, by the definition of Borel set
(111G).
\medskip

{\bf (c)} Of the types of interval considered, all the open intervals
are actually open sets, so are surely Borel.   The complement of a
closed interval is expressible as the union of at most two open
intervals, so is Borel, and the closed interval, being the complement of
a Borel set, is Borel.   A bounded half-open interval is expressible as
the intersection of an open interval with a closed interval, so is
Borel;  and finally an unbounded interval of the form
$\left]-\infty,b\right]$ or $\left[a,\infty\right[$ is the complement of
an open interval, so is also Borel.

\medskip

{\bf (d)} To compute the measures, we already know from 114Db that

\Centerline{$\mu\coint{a,b}=\theta\coint{a,b}=b-a$}

\noindent if $a\le b$.   For the other types of bounded
interval, it is enough to note that $\mu\{a\}=0$ for every $a\in\Bbb R$,
as the different intervals differ only by one or two points;  and this
is so because $\{a\}\subseteq\coint{a,a+\epsilon}$, so
$\mu\{a\}\le\epsilon$, for every $\epsilon>0$.

As for the unbounded intervals, they include arbitrarily long half-open
intervals, so must have infinite measure.

\medskip

{\bf (e)} As just remarked, $\mu\{a\}=0$ for every $a$.   If
$A\subseteq\Bbb R$ is countable, it is either empty or expressible as
$\{a_n:n\in\Bbb N\}$.   In the former case $\mu A=\mu\emptyset=0$;  in
the latter, $A=\bigcup_{n\in\Bbb N}\{a_n\}$ is Borel and $\mu
A\le\sum_{n=0}^{\infty}\mu\{a_n\}=0$.
}%end of proof of 114G

\exercises{
\leader{114X}{Basic exercises $\pmb{>}$(a)} Let $g:\Bbb R\to\Bbb R$ be
any non-decreasing function.   For
half-open intervals $I\subseteq\Bbb R$ define $\lambda_gI$ by setting

\Centerline{$\lambda_g\emptyset=0$,\quad
$\lambda_g\coint{a,b}
=\lim_{x\uparrow b}g(x)-\lim_{x\uparrow a}g(x)$}

\noindent if $a<b$.   For any set $A\subseteq\Bbb R$ set

$$\eqalign{\theta_gA=\inf\{\sum_{j=0}\sp{\infty}\lambda_gI_j:\langle
I_j\rangle_{j\in\Bbb N}
\text{ is a sequence of }&\text{half-open intervals}\cr
&\text{such that }
A\subseteq\bigcup_{j\in\Bbb N}I_j\}.\cr}$$

\noindent Show that $\theta_g$ is an outer measure on $\Bbb R$.   Let
$\mu_g$ be the measure defined from $\theta_g$ by \Caratheodory's
method;  show that $\mu_gI$ is defined and equal to $\lambda_gI$ for every half-open interval
$I\subseteq\Bbb R$, and that every Borel subset of $\Bbb R$ is
in the domain of $\mu_g$.

($\mu_g$ is the {\bf Lebesgue-Stieltjes} measure associated with $g$.)

\spheader 114Xb At which point would the argument of 114Xa break down if
we wrote $\lambda_g\coint{a,b}=g(b)-g(a)$ instead of using the formula
given?

\sqheader 114Xc Write $\theta$ for Lebesgue outer measure and $\mu$
for Lebesgue
measure on $\Bbb R$.   Show that $\theta A=\inf\{\mu E:E$ is Lebesgue
measurable, $A\subseteq E\}$ for every $A\subseteq\Bbb R$.
\Hint{Consider sets $E$ of the form $\bigcup_{j\in\Bbb N}I_j$, where
$\sequence{j}{I_j}$ is a sequence of half-open intervals.}

\spheader 114Xd Let $X$ be a set, $\Cal I$ a family of subsets
of $X$ such
that $\emptyset\in\Cal I$, and $\lambda:\Cal I\to\coint{0,\infty}$ a
function such that $\lambda\emptyset=0$.   Define
$\theta:\Cal PX\to[0,\infty]$ by writing

\Centerline{$\theta A=\inf\{\sum_{j=0}\sp{\infty}\lambda I_j:\langle
I_j\rangle_{j\in\Bbb N}
\text{ is a sequence in }\Cal I\text{ such that }
A\subseteq\bigcup_{j\in\Bbb N}I_j\}$,}

\noindent interpreting $\inf\emptyset$ as $\infty$, so that
$\theta A=\infty$
if $A$ is not covered by any sequence in $\Cal I$.   Show that $\theta$
is an outer measure on $X$.

\spheader 114Xe Let $E\subseteq\Bbb R$ be a set of finite
measure for Lebesgue
measure $\mu$.   Show that for every $\epsilon>0$ there is a disjoint
family $I_0,\ldots,I_n$ of half-open intervals such that
$\mu(E\symmdiff\bigcup_{j\le n}I_j)\le\epsilon$.   \Hint{let
$\langle J_j\rangle_{j\in\Bbb N}$ be a sequence of half-open intervals
such that $E\subseteq\bigcup_{j\in\Bbb N}J_j$ and
$\sum_{j=0}^{\infty}\mu J_j\le\mu E+\bover{1}{2}\epsilon$.   Now take a
suitably large $m$ and express $\bigcup_{j\le m}J_j$ as a disjoint union
of half-open intervals.}

\sqheader 114Xf Write $\theta$ for Lebesgue outer measure and $\mu$
for Lebesgue
measure on $\Bbb R$.   Suppose that $c\in\Bbb R$.   Show that
$\theta(A+c)=\theta A$ for every $A\subseteq\Bbb R$, where
$A+c=\{x+c:x\in A\}$.   Show that if $E\subseteq \Bbb R$ is measurable
so is $E+c$, and that in this case $\mu(E+c)=\mu E$.

\spheader 114Xg Write $\theta$ for Lebesgue outer measure and $\mu$
for Lebesgue
measure on $\Bbb R$.   Suppose that $c>0$.   Show that
$\theta(cA)=c\theta(A)$ for every $A\subseteq\Bbb R$, where
$cA=\{cx:x\in A\}$.
Show that if $E\subseteq \Bbb R$ is measurable so is $cE$,
and that in this case $\mu(cE)=c\mu E$.

\leader{114Y}{Further exercises (a)}
%\spheader 114Ya
In (a-iv) of the proof of 114D, show that $\sum_{m=0}^{\infty}\lambda
I_{k_m,l_m}$ is actually equal to
$\sum_{n=0}^{\infty}\sum_{j=0}^{\infty}\lambda I_{nj}$.

\spheader 114Yb Let $g$, $h:\Bbb R\to\Bbb R$ be two non-decreasing
functions, with sum $g+h$;  let $\mu_g$, $\mu_h$, $\mu_{g+h}$ be the
corresponding Lebesgue-Stieltjes measures (114Xa).   Show that

\Centerline{$\dom\mu_{g+h}=\dom\mu_g\cap\dom\mu_h$,
\quad$\mu_{g+h}E=\mu_gE+\mu_hE$ for every $E\in\dom\mu_{g+h}$.}
\spheader 114Yc Let $\sequencen{g_n}$ be a sequence of non-decreasing
functions from $\Bbb R$ to $\Bbb R$, and suppose that
$g(x)=\sum_{n=0}^{\infty}g_n(x)$ is defined and finite for every
$x\in\Bbb R$.   Let $\mu_{g_n}$, $\mu_g$ be the corresponding
Lebesgue-Stieltjes measures.   Show that

\Centerline{$\dom\mu_g=\bigcap_{n\in\Bbb N}\dom\mu_{g_n}$,
\quad$\mu_gE=\sum_{n=0}^{\infty}\mu_{g_n}E$ for every $E\in\dom\mu_g$.}

\spheader 114Yd(i) Show that if $A\subseteq\Bbb R$
and $\epsilon>0$, there is
an open set $G\supseteq A$ such that $\theta G\le\theta A+\epsilon$,
where $\theta$ is Lebesgue outer measure.  (ii) Show that if
$E\subseteq\Bbb R$ is Lebesgue measurable and $\epsilon>0$, there is an
open set $G\supseteq E$ such that $\mu(G\setminus E)\le\epsilon$, where
$\mu$ is Lebesgue measure.   \Hint{consider first the case of
bounded $E$.}   (iii) Show that if $E\subseteq\Bbb R$ is Lebesgue
measurable, there are Borel sets $H_1$, $H_2$ such that $H_1\subseteq
E\subseteq H_2$ and $\mu(H_2\setminus E)=\mu(E\setminus H_1)=0$.
\Hint{use (ii) to find $H_2$, and then consider the complement of
$E$.}

\spheader 114Ye Write $\theta$ for Lebesgue outer measure on
$\Bbb R$.   Show
that a set $E\subseteq\Bbb R$ is Lebesgue measurable iff
$\theta([-n,n]\cap E)+\theta([-n,n]\setminus E)=2n$ for every
$n\in\Bbb N$.   \Hint{Use 114Yd to show that for each $n$ there are
measurable sets $F_n$, $H_n$ such that $F_n\subseteq[-n,n]\cap
E\subseteq H_n$ and $H_n\setminus F_n$ is negligible.}

\spheader 114Yf Repeat 114Xc and 114Yd-114Ye for the
Lebesgue-Stieltjes measures of 114Xa.

\spheader 114Yg Write $\Cal B$ for the $\sigma$-algebra of Borel
subsets of
$\Bbb R$, and let $\nu:\Cal B\to[0,\infty]$ be a measure.   Let
$g$, $\lambda_g$, $\theta_g$ and $\mu_g$ be as in 114Xa.   Show that if
$\nu I=\lambda_gI$ for every half-open interval $I$, then $\nu E=\mu_gE$
for every $E\in\Cal B$.   \Hint{first consider open sets $E$, and
then use 114Yd(i) as extended in 114Yf.}

\spheader 114Yh Write $\Cal B$ for the $\sigma$-algebra of Borel
subsets of
$\Bbb R$, and let $\nu:\Cal B\to[0,\infty]$ be a measure such that
$\nu[-n,n]<\infty$ for every $n\in\Bbb N$.   Show that there is a
function $g:\Bbb R\to\Bbb R$ which is non-decreasing, continuous on the
left and
such that $\nu E=\mu_gE$ for every $E\in\Cal B$, where $\mu_g$ is
defined as in 114Xa.   Is $g$ unique?

\spheader 114Yi Write $\Cal B$ for the $\sigma$-algebra of Borel
subsets of
$\Bbb R$, and let $\nu_1$, $\nu_2$ be measures with domain $\Cal B$ such
that $\nu_1 I=\nu_2I<\infty$ for every half-open interval
$I\subseteq\Bbb R$.     Show that $\nu_1E=\nu_2E$ for every $E\in\Cal
B$.

\spheader 114Yj Let $\Cal E$ be any family of half-open
intervals in $\Bbb R$.   Show that (i) there is a countable
$\Cal C\subseteq\Cal E$ such that $\bigcup\Cal E=\bigcup\Cal C$
(definition:
1A1F) (ii) that $\bigcup\Cal E$ is a Borel set, so is Lebesgue
measurable (iii) that there is a disjoint sequence $\sequencen{I_n}$ of
half-open intervals in $\Bbb R$ such that $\bigcup\Cal
E=\bigcup_{n\in\Bbb N}I_n$.

\spheader 114Yk Show that for almost every $x\in\Bbb R$ (as measured by
Lebesgue measure) the set

\Centerline{$\{(m,n):m\in\Bbb Z,\,n\in\Bbb N\setminus\{0\},
\,|x-\Bover{m}{n}|\le\Bover{1}{n^3}\}$}

\noindent is finite.   \Hint{estimate the outer measure of
$\bigcup_{n\ge n_0}\bigcup_{|m|\le kn}
[\bover{m}{n}-\bover1{n^3},\bover{m}{n}+\bover1{n^3}]$ for $n_0$,
$k\ge 1$.}   Repeat with $2+\epsilon$ in the place of $3$.

\spheader 114Yl Write $\mu$ for Lebesgue measure on $\Bbb R$.   Show that
there is a countable family $\Cal F$ of Lebesgue measurable subsets of
$\Bbb R$ such that whenever $\mu E$ is defined and finite, and
$\epsilon>0$, there is an $F\in\Cal F$ such that
$\mu(E\symmdiff F)\le\epsilon$.   \Hint{in 114Xe, show that we can take the
$I_j$ to have rational endpoints.}
%114Xe
}%end of exercises

\endnotes{
\Notesheader{114} My own interests are in \lq abstract'
measure theory, and from the point of view of the structure of this
treatise, the chief object of this section is the description of a
non-trivial measure space to provide a focus for the general theorems
which follow.   Let me enumerate the methods of constructing measure
spaces already available to us.   (a) We have the point-supported measures of
112Bd;  in some ways, these are trivial;  but they do
occur in applications, and, just because they are generally easy to deal
with, it is often right to test any new ideas on them.   (b) We have
Lebesgue measure on $\Bbb R$;  a straightforward generalization of the
construction yields the Lebesgue-Stieltjes measures (114Xa).   (c) Next,
we have ways of building new measures from old, starting with subspace
measures (113Yb), image measures (112Xf) and sums of
measures (112Yf).   Perhaps the most important of these is
`Lebesgue measure on $[0,1]$', I call it $\mu_1$ for the moment, where
the domain of $\mu_1$ is $\{E:E\subseteq[0,1]$ is Lebesgue
measurable$\}=\{E\cap[0,1]:E\subseteq\Bbb R$ is Lebesgue measurable$\}$,
and $\mu_1E$ is just the Lebesgue measure of $E$ for each
$E\in\dom\mu_1$.   In fact the image measures of Lebesgue measure on
$[0,1]$ include a very large proportion of the probability measures
(that is, measures giving measure $1$ to the whole space) of importance
in ordinary applications.

Of course Lebesgue measure is not only the dominant guiding example for
general measure theory, but is itself the individual measure of greatest
importance for applications.   For this reason it would be
possible -- though in my view narrow-minded -- to read
chapters 12-13 of this
volume, and a substantial proportion of Volume 2, as if they applied only
to Lebesgue measure on $\Bbb R$.   This is, indeed, the context in which
most of these results were first developed.   I believe, however, that it
is often the case in mathematics, that one's understanding of a
particular construction is deepened and strengthened by an acquaintance
with related objects, and that one of the ways to an appreciation of the
nature of Lebesgue measure is through a study of its properties in the
more abstract context of general measure theory.

For any proper investigation of the applications of Lebesgue measure
theory we must wait for Volume 2.   But I include 114Yk as a hint of one
of the ways in which this theory can be used.
}%end of notes
\discrpage
