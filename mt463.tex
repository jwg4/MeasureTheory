\frfilename{mt463.tex}
\versiondate{1.2.13}
\copyrightdate{2001}

\def\chaptername{Pointwise compact sets of measurable functions}
\def\sectionname{$\frak T_p$ and $\frak T_m$}

\newsection{463}
\def\headlinesectionname{$\eightfrak T_p$ and
$\eightfrak T_m$}

We are now ready to start on the central ideas of this chapter with an
investigation of sets of measurable functions which are compact for the
topology of pointwise convergence.   Because `measurability' is, from
the point of view of this topology on $\BbbR^X$, a rather arbitrary
condition, we are looking at compact subsets of a topologically
irregular subspace of $\BbbR^X$;  there are consequently relatively few
of them, and (under a variety of special circumstances, to be examined
later in the chapter and also in Volume 5) they have some striking
special properties.

The presentation here is focused on the relationship between the two
natural topologies on any space of measurable functions, the `pointwise'
topology $\frak T_p$ and the topology $\frak T_m$ of convergence in
measure (463A).   In this section I begin with results which apply to
any $\sigma$-finite measure space (463B-463H) before turning to some
which apply to perfect measure spaces (463I-463L) -- in particular, to
Lebesgue measure.   These lead to some interesting properties of
separately continuous functions (463M-463N).

\leader{463A}{Preliminaries} Let $(X,\Sigma,\mu)$ be a measure space, and
$\eusm L^0\cmmnt{\mskip5mu =\eusm L^0(\Sigma)}$ the space of all
$\Sigma$-measurable functions from $X$ to $\Bbb R$\cmmnt{, so that
$\eusm L^0$ is a linear subspace of $\BbbR^X$}.   On $\eusm L^0$ we
shall be concerned with two\cmmnt{ very different} topologies.   The
first is the topology $\frak T_p$ of pointwise
convergence\cmmnt{ (462Ab)};  the second is the topology $\frak T_m$
of\cmmnt{ (local)} convergence in measure\cmmnt{ (245A)}.   Both
are linear space topologies.   \prooflet{\Prf\ For $\frak T_p$ I have
already noted this in 462Ab.   For $\frak T_m$, repeat the argument of
245Da;  $\frak T_m$ is defined by the functionals
$f\mapsto\int_F\min(1,|f|)d\mu$, where $\mu F<\infty$, and these are
F-seminorms 
(definition:  2A5B\footnote{Later editions only; see \S4A7.}).\ \QeD}   
$\frak T_p$ is
Hausdorff\cmmnt{ (3A3Id)} and locally convex\cmmnt{ (4A4Ce);  only
in exceptional circumstances is either true of $\frak T_m$}.
\cmmnt{However, $\frak T_m$ can easily be pseudometrizable (if, for
instance, $\mu$ is $\sigma$-finite, as in 245Eb), while $\frak T_p$ is
not, except in nearly trivial cases.}

Associated with the topology of pointwise convergence on $\BbbR^X$ is
the usual topology of $\Cal PX$\cmmnt{ (4A2A)};  the map
$\chi:\Cal PX\to\BbbR^X$ is a homeomorphism between $\Cal PX$ and its
image $\{0,1\}^X\subseteq\BbbR^X$.

\cmmnt{$\frak T_m$ is intimately associated with the topology of
convergence in measure on $L^0=L^0(\mu)$ (\S245).}   A subset of
$\eusm L^0$ is open for $\frak T_m$ iff it is of the form
$\{f:f^{\ssbullet}\in G\}$ for some open set $G\subseteq L^0$;
\cmmnt{consequently,} a
subset $K$ of $\eusm L^0$ is compact, or separable, for $\frak T_m$ iff
$\{f^{\ssbullet}:f\in K\}$ is compact or separable for the topology of
convergence in measure on $L^0$.

\cmmnt{It turns out that the identity map from $(\eusm L^0,\frak T_p)$
to $(\eusm L^0,\frak T_m)$ is sequentially continuous (463B).   Only in
nearly trivial cases is it actually continuous
(463Xa(i)), and it is similarly rare for the reverse map from $(\eusm
L^0,\frak T_m)$ to $(\eusm L^0,\frak T_p)$ to be continuous (463Xa(ii)).
If, however, we relativise both topologies to a
$\frak T_p$-compact subset of $\eusm L^0$, the situation becomes very
different, and there are many important cases in which the topologies
are comparable.}

\leader{463B}{Lemma} Let $(X,\Sigma,\mu)$ be a measure space, and
$\eusm L^0$ the space of $\Sigma$-measurable real-valued functions on
$X$.   Then every pointwise convergent sequence in $\eusm L^0$ is
convergent in measure to the same limit.

\proof{ 245Ca.
}%end of proof of 463B

\leader{463C}{Proposition}\cmmnt{ ({\smc Ionescu Tulcea 73})} Let
$(X,\Sigma,\mu)$ be a measure space, and $\eusm L^0$ the space of
$\Sigma$-measurable real-valued functions on $X$.   Write $\frak T_p$,
$\frak T_m$ for the topologies of pointwise convergence and convergence
in measure on $\eusm L^0$;  for $A\subseteq\eusm L^0$, write
$\frak T_p^{(A)}$, $\frak T_m^{(A)}$ for the corresponding subspace
topologies.

(a) If $A\subseteq\eusm L^0$ and $\frak T_p^{(A)}$ is metrizable, then the
identity map from $A$ to itself is
$(\frak T_p^{(A)},\frak T_m^{(A)})$-continuous.

(b) Suppose that $\mu$ is semi-finite.   Then, for any
$A\subseteq\eusm L^0$, $\frak T_m^{(A)}$ is Hausdorff iff whenever $f$, $g$
are distinct
members of $A$ the set $\{x:f(x)\ne g(x)\}$ is non-negligible.

(c) Suppose that $K\subseteq\eusm L^0$ is such that $\frak T_p^{(K)}$ is
compact and metrizable.   Then $\frak T_p^{(K)}=\frak T_m^{(K)}$ iff
$\frak T_m^{(K)}$ is Hausdorff.

(d) Suppose that $\mu$ is $\sigma$-finite, and that
$K\subseteq\eusm L^0$ is $\frak T_p$-sequentially compact.   Then
$\frak T_p^{(K)}=\frak T_m^{(K)}$ iff $\frak T_m^{(K)}$ is Hausdorff, and in this
case $\frak T_p^{(K)}$ is compact and metrizable.

(e) Suppose that $K\subseteq\eusm L^0$ is such that $\frak T_p^{(K)}$ is
compact and metrizable.   Then whenever $\epsilon>0$ and $E\in\Sigma$ is
a non-negligible measurable set, there is a non-negligible measurable
set $F\subseteq E$ such that $|f(x)-f(y)|\le\epsilon$ whenever $f\in K$
and $x$, $y\in F$.

\proof{{\bf (a)} All we need is to remember that sequentially continuous
functions from metrizable spaces are continuous (4A2Ld), and apply 463B.

\medskip

{\bf (b)} $\frak T_m^{(A)}$ is Hausdorff iff for any distinct $f$, $g\in A$
there is a measurable set $F$ of finite measure such that
$\int_F\min(1,|f-g|)d\mu>0$, that is,
$\mu\{x:x\in F,\,f(x)\ne g(x)\}>0$;  because $\mu$ is semi-finite, this
happens iff $\mu\{x:f(x)\ne g(x)\}>0$.

\medskip

{\bf (c)} If $\frak T_p^{(K)}=\frak T_m^{(K)}$ then of course $\frak T_m^{(K)}$
is Hausdorff, because $\frak T_p^{(K)}$ is.   If $\frak T_m^{(K)}$ is
Hausdorff then the identity map $(K,\frak T_p^{(K)})\to(K,\frak T_m^{(K)})$ is
an injective function from a compact space to a Hausdorff space and (by
(a)) is continuous, therefore a homeomorphism, so the two topologies are
equal.

\medskip

{\bf (d)} If $\frak T_p^{(K)}=\frak T_m^{(K)}$ then $\frak T_m^{(K)}$ must be
Hausdorff, just as in (c).   So let us suppose that $\frak T_m^{(K)}$ is
Hausdorff.   Note that, by 245Eb, the topology of convergence in measure
on $L^0$ is metrizable;  in terms of $\eusm L^0$, this says just that
the topology of convergence in measure on $\eusm L^0$ is
pseudometrizable.   So $\frak T_m^{(K)}$ is Hausdorff and pseudometrizable,
therefore metrizable (4A2La).

We are told that any sequence in $K$ has a $\frak T_p^{(K)}$-convergent
subsequence.   But this subsequence is now $\frak T_m^{(K)}$-convergent
(463B), so $\frak T_m^{(K)}$ is sequentially compact;  being metrizable, it
is compact (4A2Lf).   Moreover, the same is true of any
$\frak T_p^{(K)}$-closed subset of $K$, so every $\frak T_p^{(K)}$-closed set
is $\frak T_m^{(K)}$-compact, therefore $\frak T_m^{(K)}$-closed.   Thus the
identity map from $(K,\frak T_m^{(K)})$ to $(K,\frak T_p^{(K)})$ is
continuous.   Since $\frak T_m^{(K)}$ is compact and $\frak T_p^{(K)}$ is
Hausdorff, the two topologies are equal;  and, in particular, 
$\frak T_p^{(K)}$ is compact and metrizable.

\medskip

{\bf (e)} Let $\rho$ be a metric on $K$ inducing the topology 
$\frak T_p^{(K)}$.   Let $D\subseteq K$ be a countable dense set.   For each
$n\in\Bbb N$, set

\Centerline{$G_n=\{x:|f(x)-g(x)|\le\Bover13\epsilon$ whenever $f$, $g\in
D$ and $\rho(f,g)\le 2^{-n}\}$.}

\noindent Because $D$ is countable, $G_n$ is measurable.   Now
$\bigcup_{n\in\Bbb N}G_n=X$.   \Prf\Quer\ If $x\in
X\setminus\bigcup_{n\in\Bbb N}G_n$, then for each $n\in\Bbb N$ we can
find $f_n$, $g_n\in D$ such that $\rho(f_n,g_n)\le 2^{-n}$ and
$|f_n(x)-g_n(x)|\ge\bover13\epsilon$.   Because $K$ is compact, there is
a strictly increasing sequence $\sequence{k}{n_k}$ such that
$\sequence{k}{f_{n_k}}$ and $\sequence{k}{g_{n_k}}$ are both convergent
to $f$, $g$ say.   Now

\Centerline{$\rho(f,g)=\lim_{k\to\infty}\rho(f_{n_k},g_{n_k})=0$,
\quad$|f(x)-g(x)|
=\lim_{k\to\infty}|f_{n_k}(x)-g_{n_k}(x)|\ge\Bover13\epsilon$,}

\noindent so $f=g$ while $f(x)\ne g(x)$, which is impossible.\ \Bang\Qed

There is therefore some $n\in\Bbb N$ such that $\mu(E\cap G_n)>0$.
Since $K$, being compact, is totally bounded for $\rho$, there is a
finite set $D'\subseteq D$ such that every member of $D$ is within a
distance of $2^{-n}$ of some member of $D'$.   Now there is a measurable
set $F\subseteq E\cap G_n$ such that $\mu F>0$ and
$|g(x)-g(y)|\le\bover13\epsilon$ whenever $g\in D'$ and $x$, $y\in F$.
So $|f(x)-f(y)|\le\epsilon$ whenever $f\in D$ and $x$, $y\in F$.   But
as $D$ is dense in $K$, $|f(x)-f(y)|\le\epsilon$ whenever $f\in K$ and
$x$, $y\in F$, as required.
}%end of proof of 463C

\leader{463D}{Lemma} Let $(X,\Sigma,\mu)$ be a measure space, and
$\eusm L^0$ the space of $\Sigma$-measurable real-valued functions on
$X$.   Write $\frak T_p$ for the topology of pointwise
convergence on $\eusm L^0$.   Suppose that
$K\subseteq\eusm L^0$ is $\frak T_p$-compact and that there is no
$\frak T_p$-continuous surjection from any closed subset of $K$ onto
$\{0,1\}^{\omega_1}$.   If $E\in\Sigma$ has finite measure, then every sequence in $K$ has a subsequence which is convergent almost everywhere in $E$.

\proof{{\bf (a)} Let $\sequencen{f_n}$ be a sequence in $K$.   Set
$q(x)=\max(1,2\sup_{n\in\Bbb N}|f_n(x)|)$ for each $x\in X$.
For any infinite $I\subseteq\Bbb N$, set

\Centerline{$g_I=\liminf_{i\to I}f_i
=\sup_{n\in\Bbb N}\inf_{i\in I,i\ge n}f_i$,}

\Centerline{$h_I=\limsup_{i\to I}f_i
=\inf_{n\in\Bbb N}\sup_{i\in I,i\ge n}f_i$;}

\noindent because $\sup_{f\in K}|f(x)|$ is surely finite for each
$x\in X$, $g_I$ and $h_I$ are defined in $\eusm L^0$, and $g_I\le h_I$.
For $f\in\eusm L^0$ set $\tau'(f)=\int_E\min(1,|f|/q)$, and for
$I\in[\Bbb N]^{\omega}$ (the set of infinite subsets of $\Bbb N$) set
$\Delta(I)=\tau'(h_I-g_I)$.   Since $h_I-g_I\le q$,
$\Delta(I)=\int_E(h_I-g_I)/q$.   If $I$, $J\in[\Bbb N]^{\omega}$ and
$J\setminus I$ is finite, then $g_I\le g_J\le h_J\le h_I$, so
$\Delta(J)\le\Delta(I)$, with equality only when $g_I=g_J$ a.e.\ on $E$
and $h_I=h_J$ a.e.\ on $E$.

\medskip

{\bf (b)} There is a $J\in[\Bbb N]^{\omega}$ such that
$\Delta(I)=\Delta(J)$ for every $I\in[J]^{\omega}$.   \Prf\ For
$J\in[\Bbb N]^{\omega}$, set
$\underline{\Delta}(J)=\inf\{\Delta(I):I\in[J]^{\omega}\}$.   Choose
$\sequencen{I_n}$ in $[\Bbb N]^{\omega}$ inductively in such a way that
$I_{n+1}\subseteq I_n$ and
$\Delta(I_{n+1})\le\underline{\Delta}(I_n)+2^{-n}$ for every $n$.   If
we now set

\Centerline{$J=\{\min\{i:i\in I_n,\,i\ge n\}:n\in\Bbb N\}$,}

\noindent $J\subseteq\Bbb N$ will be an infinite set and
$J\setminus I_n$ is finite for every $n$.   If $I\in[J]^{\omega}$ then,
for every $n$,

\Centerline{$\Delta(J)\le\Delta(I_{n+1})
\le\underline{\Delta}(I_n)+2^{-n}
\le\Delta(I_n\cap I)+2^{-n}=\Delta(I)+2^{-n}$;}

\noindent as $n$ and $I$ are
arbitrary, $\underline{\Delta}(J)=\Delta(J)$, as required.\ \Qed

Now for any $I\in[J]^{\omega}$ we have 
$g_I=g_J$ a.e.\ on $E$ and $h_I=h_J$ a.e.\ on $E$.

\medskip

{\bf (c)} $\Delta(J)=0$.   \Prf\Quer\ Otherwise,
$F=\{x:x\in E,\,g_J(x)<h_J(x)\}$ has positive measure.   Write $K_0$ for
\penalty-100$\bigcap_{n\in\Bbb N}\overline{\{f_i:i\in J,\,i\ge n\}}$, the
closure being taken for $\frak T_p$, so that $K_0$ is
$\frak T_p$-compact.   Let
$\Cal A$ be the family of sets $A\subseteq F$ such that whenever $L$,
$M\subseteq A$ are finite and disjoint there is an $f\in K_0$ such that
$f(x)=g_J(x)$ for $x\in L$ and $f(x)=h_J(x)$ for $x\in M$.   Then
$\Cal A$ has a maximal member $A_0$ say.   If we define
$\phi:\eusm L^0\to[0,1]^{A_0}$ by setting
$\phi(f)(x)=\Bover{\med(0,f(x)-g_J(x),h_J(x)-g_J(x))}{h_J(x)-g_J(x)}$ for
$x\in A_0$ and $f\in\eusm L^0$,
$\phi[K_0]$ is a compact subset of $[0,1]^{A_0}$, and whenever $L$,
$M\subseteq A_0$ are finite there is a $g\in\phi[K_0]$ such that
$g(x)=0$ for $x\in L$ and $g(x)=1$ for $x\in M$.   This means that
$\phi[K_0]\cap\{0,1\}^{A_0}$ is dense in $\{0,1\}^{A_0}$ and must
therefore be the whole of $\{0,1\}^{A_0}$.   So $\{0,1\}^{A_0}$ is a
continuous image of a closed subset of $K$.

Since $\{0,1\}^{\omega_1}$ is not a continuous image of a closed subset of
$K$, it is not a continuous image of $\{0,1\}^{A_0}$, and cannot be
homeomorphic to $\{0,1\}^A$ for any $A\subseteq A_0$.   Thus no subset of
$A_0$ can have cardinal $\omega_1$ and $A_0$ is countable.

For each pair $L$, $M$ of disjoint finite
subsets of $A_0$, we have a cluster point $f_{LM}$ of
$\family{j}{J}{f_j}$ such that $f_{LM}(x)=g_J(x)$ for $x\in L$ and
$f_{LM}(x)=h_J(x)$ for $x\in M$.   Let $I(L,M)$ be an infinite subset of
$J$ such that
$\lim_{i\to\infty,i\in I(L,M)}f_i(x)=f_{LM}(x)$ for every $x\in A_0$.
Then $g_{I(L,M)}=g_J$ and $h_{I(L,M)}=h_J$ almost everywhere in $E$.
Because $\mu F>0$ and $A_0$ has only countably many finite subsets,
there is a $y\in F$ such that $g_{I(L,M)}(y)=g_J(y)$ and
$h_{I(L,M)}(y)=h_J(y)$ whenever $L$ and $M$ are disjoint finite subsets
of $A_0$.

What this means is that if $L$ and $M$ are disjoint finite subsets of
$A_0$, then there are infinite sets $I'$, $I''\subseteq I(L,M)$ such
that $\lim_{i\to\infty,i\in I'}f_i(y)=g_J(y)$ and
$\lim_{i\to\infty,i\in I''}f_i(y)=h_J(y)$;  so that there are $f'$,
$f''\in K_0$ such that

\Centerline{$f'(x)=g_J(x)$ for $x\in L\cup\{y\}$,
\quad$f'(x)=h_J(x)$ for $x\in M$,}

\Centerline{$f''(x)=g_J(x)$ for $x\in L$,
\quad$f''(x)=h_J(x)$ for $x\in M\cup\{y\}$.}

\noindent But this means that $A_0\cup\{y\}\in\Cal A$, and also that $y\notin A_0$;
and $A_0$ was chosen to be maximal.\ \Bang\Qed

\medskip

{\bf (d)} So $\int_E(h_J-g_J)/q=0$ and $g_J=h_J$ almost everywhere in
$E$.   But if we enumerate $J$ in ascending order as
$\sequence{i}{n_i}$,
$g_J=\liminf_{i\to\infty}f_{n_i}$ and $h_J=\limsup_{i\to\infty}f_{n_i}$,
so $\sequence{i}{f_{n_i}}$ converges almost everywhere in $E$.
}%end of proof of 463D

\leader{463E}{Proposition} Let $(X,\Sigma,\mu)$ be a measure space, and
$\eusm L^0$ the space of $\Sigma$-measurable real-valued functions on
$X$.   Write $\frak T_p$, $\frak T_m$ for the topologies of pointwise
convergence and convergence in measure on $\eusm L^0$.   Suppose that
$K\subseteq\eusm L^0$ is $\frak T_p$-compact and that there is no
$\frak T_p$-continuous surjection from any closed subset of $K$ onto
$\omega_1+1$ with its order topology.   Then the identity map from
$(K,\frak T_p)$ to $(K,\frak T_m)$ is continuous.

\proof{{\bf (a)} It is worth noting straight away that
$\xi\mapsto\chi\xi:\omega_1+1\to\{0,1\}^{\omega_1}$ is a homeomorphism
between $\omega_1+1$ and a subspace of $\{0,1\}^{\omega_1}$.   So our
hypothesis tells us that there is no continuous surjection from any
closed subset of $K$ onto $\{0,1\}^{\omega_1}$, and therefore none onto
$\{0,1\}^A$ for any uncountable $A$.

\medskip

{\bf (b)} \Quer\ Suppose, if possible, that the identity map from
$(K,\frak T_p)$ to $(K,\frak T_m)$ is not continuous at $f_0\in K$.
Then there are an $E\in\Sigma$, of finite measure, and an $\epsilon>0$
such that $C=\{f:f\in K,\,\tau_E(f-f_0)\ge\epsilon\}$ meets every
$\frak T_p$-neighbourhood of $f$, where $\tau_E(f)=\int_E\min(1,|f|)$
for every $f\in\eusm L^0$, and there is an ultrafilter $\Cal F$ on
$\eusm L^0$ which contains $C$ and converges to $f_0$ for $\frak T_p$.
Consider the map $\psi:\eusm L^0\to L^0(\mu_E)$, where $\mu_E$ is the
subspace measure on $E$, defined by setting
$\psi(f)=(f\restr E)^{\ssbullet}$ for $f\in\eusm L^0$.
We know from 463D that every sequence in $K$ has a subsequence convergent
almost everywhere in $E$, so every sequence in $\psi[K]$ has a
subsequence which is convergent for the topology of convergence in
measure on $L^0(\mu_E)$.   Since this is metrizable, $\psi[K]$ is
relatively compact in $L^0(\mu_E)$ (4A2Le), and the image filter
$\psi[[\Cal F]]$ has a limit $v\in L^0(\mu_E)$.   Let $f_1\in\eusm L^0$
be such that $\psi(f_1)=v$.

For any countable set $A\subseteq X$ there is a $g\in C$ such that
$g\restr A=f_0\restr A$ and $g=f_1$ almost everywhere in $E$.   \Prf\ If
$X=\emptyset$ this is trivial, so we may, if necessary, enlarge $A$ by
one point so that it is not empty.      Let $\sequencen{x_n}$ be a
sequence running over $A$.   Then for each $n\in\Bbb N$ the set

\Centerline{$\{g:g\in C,\,|g(x_i)-f_0(x_i)|\le 2^{-n}$ for every
$i\le n$, $\tau_E(g,f_1)\le 2^{-n}\}$}

\noindent belongs to $\Cal F$, so is not empty;  take $g_n$ in this set.
Let $g\in K$ be any cluster point of $\sequencen{g_n}$.    Since
$\sequencen{g_n}$ converges to $f_1$ almost everywhere in $E$, $g=f_1$
a.e.\ on $E$ and $\sequencen{g_n}$ converges to $g$ almost everywhere in
$E$.   Consequently $\tau_E(g-f_0)=\lim_{n\to\infty}\tau_E(g_n-f_0)$, by
the dominated convergence theorem, and $g\in C$.   
Since $\sequencen{g_n(x_i)}$
converges to $f_0(x_i)$ for every $i$, $g\restr A=f_0\restr A$.   So we
have the result.\ \Qed

In particular, there is a $g\in C$ such that $g=f_1$ a.e.\ on $E$, so
$\tau_E(f_1-f_0)=\tau(g-f_0)\ge\epsilon$ and
$F=\{x:x\in E,\,f_0(x)\ne f_1(x)\}$ has non-zero measure.   Now choose
$\ofamily{\xi}{\omega_1}{g_{\xi}}$ in $K$ and
$\ofamily{\xi}{\omega_1}{x_{\xi}}$ in $F$ inductively so that

\Centerline{$g_{\xi}\in C$,
\quad $g_{\xi}=f_1$ almost everywhere in $E$,
\quad $g_{\xi}(x_{\eta})=f_0(x_{\eta})$ for $\eta<\xi$}

\noindent (choosing $g_{\xi}$),

\Centerline{$x_{\xi}\in F$,
\quad $g_{\eta}(x_{\xi})=f_1(x_{\xi})$ for $\eta\le\xi$}

\noindent (choosing $x_{\xi}$).
If we now set $A=\{x_{\xi}:\xi<\omega_1\}$,

\Centerline{$K_1=\bigcap_{\xi\le\eta<\omega_1}\{f:f\in K$, either
$f(x_{\xi})=f_0(x_{\xi})$ or $f(x_{\eta})=f_1(x_{\eta})\}$,}

\noindent then $K_1$ is a closed subset of $K$ containing every
$g_{\xi}$ and also $f_0$.   But if we look at $\{f\restr A:f\in K_1\}$,
this is homeomorphic to $\omega_1+1$;   which is supposed to be
impossible.\ \Bang

So we conclude that the identity map from $(K,\frak T_p)$ to
$(K,\frak T_m)$ is continuous.
}%end of proof of 463E

\leader{463F}{Corollary} Let $(X,\Sigma,\mu)$ be a measure space, and
$\eusm L^0$ the space of $\Sigma$-measurable real-valued functions on
$X$.   Write $\frak T_p$, $\frak T_m$ for the topologies of pointwise
convergence and convergence in measure on $\eusm L^0$.   Suppose that
$K\subseteq\eusm L^0$ is compact
and countably tight for $\frak T_p$.   Then the identity map from
$(K,\frak T_p)$ to $(K,\frak T_m)$ is continuous.   If $\frak T_m$ is
Hausdorff on $K$, the two topologies coincide on $K$.

\proof{ Since $\omega_1+1$ is not countably tight (the top point
$\omega_1$ is not in the closure of any countable subset of $\omega_1$),
$\omega_1+1$ is not a continuous image of any closed subset of $K$
(4A2Kb), and we can apply 463E to see that the identity map is
continuous.   It follows at once that if $\frak T_m$ is Hausdorff on
$K$, then the topologies coincide.
}%end of proof of 463F

\vleader{48pt}{463G}{Theorem}\cmmnt{ ({\smc Ionescu Tulcea 74})} Let
$(X,\Sigma,\mu)$ be a $\sigma$-finite measure space, and $K$ a convex
set of measurable functions from $X$ to $\Bbb R$ such that (i) $K$ is
compact for the topology $\frak T_p$ of pointwise convergence (ii)
$\{x:f(x)\ne g(x)\}$ is not
negligible for any distinct $f$, $g\in K$.   Then $K$ is metrizable for
$\frak T_p$, which agrees with the topology of convergence in
measure on $K$.

\proof{{\bf (a)} Let $\sequencen{f_n}$ be a sequence in $K$.   Then it
has a pointwise convergent subsequence.   \Prf\ Because $K$ is compact,
we surely have $\sup_{f\in K}|f(x)|<\infty$ for every $x\in X$.
Let $\sequence{k}{X_k}$ be a sequence of measurable sets
of finite measure covering $X$, and set

\Centerline{$Y_k=\{x:x\in X_k$, $|f_n(x)|\le k$ for every $n\in\Bbb N\}$}

\noindent for $k\in\Bbb N$,

\Centerline{$q=\sum_{k=0}^{\infty}\Bover1{2^k(1+\mu Y_k)}\chi Y_k$,}

\noindent so that $q$ is a strictly positive measurable function and

\Centerline{$\|f_n\times q\|_2\le\sum_{k=0}^{\infty}\Bover{k}{2^k}=2$}

\noindent for every $n$.

Set $K'=\{f\times q:f\in K\}$, so that $K'$ is another convex
pointwise compact set of measurable functions, this time all dominated
by $q'$, so that $K'\subseteq\eusm L^2(\mu)$.   Setting
$g_n=f_n\times q$, the sequence $\sequencen{g_n^{\ssbullet}}$ of
equivalence
classes is a norm-bounded sequence in the Hilbert space $L^2=L^2(\mu)$.
It therefore has a weakly convergent subsequence
$\sequence{i}{g^{\ssbullet}_{n(i)}}$ say (4A4Kb), with limit $v$.

\Quer\ Suppose, if possible, that $\sequence{i}{f_{n(i)}}$ is not
pointwise convergent.   Then there must be some $x_0\in X$ such that
$\liminf_{i\to\infty}f_{n(i)}(x_0)<\limsup_{i\to\infty}f_{n(i)}(x_0)$;
let $\alpha<\beta$ in $\Bbb R$ be such that
$I=\{i:f_{n(i)}(x_0)\le\alpha\}$, $I'=\{i:f_{n(i)}(x_0)\ge\beta\}$ are
both infinite.   In this case
$v$ belongs to the weak closures of both
$D=\{g_{n(i)}^{\ssbullet}:i\in I\}$ and
$D'=\{g_{n(i)}^{\ssbullet}:i\in I'\}$.   It must therefore
belong to the {\it norm} closures of their convex hulls $\Gamma(D)$,
$\Gamma(D')$ (4A4Ed).   Accordingly we can find $v_n\in\Gamma(D)$,
$v'_n\in\Gamma(D')$ such that $\|v-v_n\|_2\le 3^{-n}$,
$\|v-v'_n\|_2\le 3^{-n}$ for every $n\in\Bbb N$.

Setting $A=\{f_{n(i)}:i\in I\}$, $A'=\{f_{n(i)}:i\in I'\}$, we see that
there must be $h_n\in\Gamma(A)$, $h'_n\in\Gamma(A')$ such that
$v_n=(h_n\times q)^{\ssbullet}$, $v'_n=(h'_n\times
q)^{\ssbullet}$ for every $n\in\Bbb N$.   Now if $g:X\to\Bbb R$ is
a measurable function such that $g^{\ssbullet}=v$, and 
$\tilde h=g/q$, we have

$$\eqalign{\mu\{x:|\tilde h(x)-h_n(x)|\ge\Bover1{2^nq(x)}\}
&=\mu\{x:|g(x)-(h_n\times q)(x)|\ge 2^{-n}\}\cr
&\le 4^n\|v-v_n\|_2^2
\le 2^{-n}\cr}$$

\noindent for every $n\in\Bbb N$, and $h_n\to\tilde h$ a.e.   Similarly,
$h'_n\to\tilde h$ a.e.

At this point, recall that $K$ is supposed to be convex, so all the
$h_n$, $h'_n$ belong to $K$.   Let $\Cal F$ be any non-principal
ultrafilter on $\Bbb N$.   Because $K$ is pointwise compact,
$h=\lim_{n\to\Cal F}h_n$ and $h'=\lim_{n\to\Cal F}h'_n$ are both defined
in $K$ for the topology of pointwise convergence.   For any $x$ such
that
$\lim_{n\to\infty}h_n(x)=\tilde h(x)$, we surely have $h(x)=\tilde
h(x)$;  so $h\eae\tilde h$.   Similarly, $h'\eae\tilde h$, and
$h\eae h'$.

Now at last we apply the hypothesis that distinct members of $K$ are not
equal almost everywhere, to see that $h=h'$.   But if we look at what
happens at the distinguished point $x_0$ above, we see that
$f(x_0)\le\alpha$ for every $f\in A$, so that $f(x_0)\le\alpha$ for
every $f\in\Gamma(A)$, $h_n(x_0)\le\alpha$ for every $n\in\Bbb N$, and
$h(x_0)\le\alpha$;  and similarly $h'(x_0)\ge\beta$.   So $h\ne h'$,
which is absurd.\ \Bang

This contradiction shows that $\sequence{i}{f_{n(i)}}$ is pointwise
convergent, and is an appropriate subsequence.\ \Qed

\medskip

{\bf (b)} Now 463Cd tells us that $K$ is metrizable for $\frak T_p$, and
that $\frak T_p$ agrees on $K$ with the topology of convergence in
measure.
}%end of proof of 463G

\leader{463H}{Corollary} Let $(X,\frak T,\Sigma,\mu)$ be a
$\sigma$-finite topological measure space in which $\mu$ is strictly
positive.   Suppose that

\inset{\noindent
whenever $h\in\BbbR^X$ is such that $h\restr Q$ is continuous
for every relatively countably compact $Q\subseteq X$, then $h$ is
continuous.}

\noindent If $K\subseteq C_b(X)$ is a norm-bounded $\frak T_p$-compact
set, then it is $\frak T_p$-metrizable.

\proof{ By 462L, the $\frak T_p$-closed convex hull
$\overline{\Gamma(K)}$ of $K$ in $C(X)$ is $\frak T_p$-compact.
Because $\mu$ is strictly positive, $\mu\{x:f(x)\ne g(x)\}>0$ whenever
$f$ and $g$ are distinct continuous real-valued functions on $X$.   So
the result is immediate from 463G.
}%end of proof of 463H

\vleader{60pt}{463I}{Lemma} Let $(X,\Sigma,\mu)$ be a perfect probability
space, and $\sequencen{E_n}$ a sequence in $\Sigma$.   Suppose that
there is an $\epsilon>0$ such that

\Centerline{$\epsilon\mu F\le\liminf_{n\to\infty}\mu(F\cap E_n)
\le\limsup_{n\to\infty}\mu(F\cap E_n)\le(1-\epsilon)\mu F$}

\noindent for
every $F\in\Sigma$.   Then $\sequencen{E_n}$ has a subsequence
$\sequence{k}{E_{n_k}}$ such that $\mu_*A=0$ and $\mu^*A=1$ for any
cluster point $A$ of $\sequence{k}{E_{n_k}}$ in $\Cal PX$;  in
particular, $\sequence{k}{E_{n_k}}$ has no measurable cluster point.

\proof{{\bf (a)} If we replace $\mu$ by its completion, we do not change
$\mu^*$ and $\mu_*$ (212Ea, 413Eg), so we may suppose that $\mu$ is
already complete.

\medskip

{\bf (b)} Suppose that $\sequencen{E_n}$ is actually stochastically
independent, with $\mu E_n=\bover1k$ for every $n$, where $k\ge 2$ is an
integer.   In this case $\mu^*A=1$ for any cluster point $A$ of
$\sequencen{E_n}$.

\medskip

\Prf\ {\bf (i)} There is a non-principal ultrafilter $\Cal F$ on
$\Bbb N$ such that $A=\lim_{n\to\Cal F}E_n$ in $\Cal PX$ (4A2F(a-ii));  
that
is, $\chi A(x)=\lim_{n\to\Cal F}\chi E_n(x)$ for every $x\in X$;  that
is, $A=\{x:x\in X,\,\{n:x\in E_n\}\in\Cal F\}$.

\medskip

\quad{\bf (ii)} We have a measurable function
$\phi:X\to Y=\{0,1\}^{\Bbb N}$ defined by setting
$\phi(x)(n)=(\chi E_n)(x)$ for every $x\in X$,
$n\in\Bbb N$.   Because $\mu$ is complete and perfect and totally
finite, and $\{0,1\}^{\Bbb N}$ is compact and metrizable, the image
measure $\nu=\mu\phi^{-1}$ is a Radon measure (451O).   For any basic open set of the form $H=\{y:y(i)=\epsilon_i$ for
every $i\le n\}$, $\mu\phi^{-1}[H]=\tilde\nu H$, where $\tilde\nu$ is
the product measure corresponding to the measure $\nu_0$ on $\{0,1\}$
for which $\nu_0\{1\}=\bover1k$, $\nu_0\{0\}=\bover{k-1}{k}$.   Since
$\tilde\nu$ also is a Radon measure (416U), $\tilde\nu=\nu$ (415H(v)).

Set $B=\{y:y\in Y$, $\{n:y(n)=1\}\in\Cal F\}$, so that $\phi^{-1}[B]=A$
and $B$ is determined by coordinates in $\{n,n+1,\ldots\}$ for every
$n\in\Bbb N$.   By 451Pc, $\mu^*A=\nu^*B$;  by the zero-one law
(254Sa), $\nu^*B$ must be either 0 or 1.   So $\mu^*A$ is either 0 or 1.

\medskip

\quad{\bf (iii)} To see that $\mu^*A$ cannot be $0$, we repeat the
arguments of (ii) from the other side, as follows.   Let $\lambda_0$ be
the uniform probability measure on $\{0,1,\ldots,k-1\}$, giving measure
$\bover1k$ to each point;  let $\lambda$ be the corresponding product
measure on
$Z=\{0,\ldots,k-1\}^{\Bbb N}$.   Let $\psi:Z\to Y$ be defined by setting
$\psi(z)(n)=1$ if $z(n)=0$, $0$ otherwise;  then $\psi$ is \imp\ (254G).
Since $\lambda$ is a Radon measure and $\psi$ is continuous,
$\lambda\psi^{-1}$ is a Radon measure on $Y$ and must be equal to $\nu$.
Accordingly $\nu^*B=\lambda^*\psi^{-1}[B]$, by 451Pc again or
otherwise.

\medskip

\quad{\bf (iv)}
We have a measure space automorphism $\theta:Z\to Z$ defined by setting
$\theta(z)(n)=z(n)+_k1$ for every $z\in Z$, $n\in\Bbb N$, where $+_k$
is addition mod $k$.   So, writing $C=\psi^{-1}[B]$,
$\lambda^*C=\lambda^*\theta^i[C]$ for every $i\in\Bbb N$.   Now, for
$z\in Z$,

\Centerline{$\{n:z(n)=0\}\in\Cal F
\iff\{n:\psi(z)(n)=1\}\in\Cal F
\iff\psi(z)\in B
\iff z\in C$.}

\noindent But for any $z\in Z$, there is some $i<k$ such that
$\{n:z(n)=i\}\in\Cal F$, so that $\theta^{k-i}(z)\in C$.   Thus
$\bigcup_{i<k}\theta^i[C]=Z$ and
$\sum_{i=0}^{k-1}\lambda^*\theta^i[C]\ge 1$ and $\lambda^*C>0$.   But
this means that $\mu^*A=\nu^*B=\lambda^*C$ is non-zero, and $\mu^*A$
must be $1$.\ \Qed

\medskip

{\bf (c)} Now return to the general case considered in (a).   Note first
that $\mu$ is atomless, because if $\mu F>0$ there is some $n\in\Bbb N$
such that $0<\mu(F\cap E_n)<\mu F$.

Let $k\ge 2$ be such that $\bover1k<\epsilon$.   Then there are a
strictly increasing sequence $\sequence{i}{m(i)}$ in $\Bbb N$ and a
stochastically independent sequence $\sequence{i}{F_i}$ in $\Sigma$ such
that $F_i\subseteq E_{m(i)}$ and $\mu F_i=\bover1k$ for every
$i\in\Bbb N$.   \Prf\ Choose $\sequence{i}{m(i)}$, $F_i$ inductively, as
follows.   Let $\Sigma_i$ be the (finite) algebra generated by
$\{F_j:j<i\}$.   Choose $m(i)$ such that $m(i)>m(j)$ for any $j<i$ and
$\mu(F\cap E_{m(i)})\ge\bover1k\mu F$ for every $F\in\Sigma_i$.   List
the atoms of $\Sigma_i$ as $G_{i0},\ldots,G_{ip_i}$, and choose
$F_{ir}\subseteq E_{m(i)}\cap G_{ir}$ such that
$\mu F_{ir}=\bover1k\mu G_{ir}$, for each
$r\le p_i$;  215D tells us that this is possible.   Set
$F_i=\bigcup_{r\le p_i}F_{ir}$;  then $\mu(F_i\cap F)=\bover1k\mu F$ for
every $F\in\Sigma_i$, and $F_i\subseteq E_{m(i)}$.   Continue.
It is easy to check that $\mu(F_{i_1}\cap\ldots\cap F_{i_r})=1/k^r$
whenever $i_1<\ldots<i_r$, so that $\sequence{i}{F_i}$ is stochastically
independent.\ \Qed

If $A$ is a cluster point of $\sequence{i}{E_{m(i)}}$, then there is a
non-principal ultrafilter $\Cal F$ on $\Bbb N$ such that
$A=\lim_{i\to\Cal F}E_{m(i)}$ in $\Cal PX$.   In this case,
$A\supseteq A'$, where $A'=\lim_{i\to\Cal F}F_i$.
But (b) tells us that $\mu^*A'$
must be $1$, so $\mu^*A=1$.

\medskip

{\bf (d)} Thus we have a subsequence $\sequence{i}{E_{m(i)}}$ of
$\sequencen{E_n}$ such that any cluster point of
$\sequence{i}{E_{m(i)}}$ has outer measure $1$.   But the same argument
applies to $\sequence{i}{X\setminus E_{m(i)}}$ to show that there is a
strictly increasing sequence $\sequence{k}{i_k}$ such that every cluster
point of $\sequence{k}{X\setminus E_{m(i_k)}}$ has outer measure $1$.
Since complementation is a homeomorphism of $\Cal PX$, $\mu^*(X\setminus
A)=1$, that is, $\mu_*A=0$, for every cluster point $A$ of
$\sequence{k}{E_{m(i_k)}}$.   So if we set $n_k=m(i_k)$, any cluster
point of $\sequence{k}{E_{n_k}}$ will have inner measure 0 and outer
measure 1, as claimed.
}%end of proof of 463I

\leader{463J}{Lemma} Let $(X,\Sigma,\mu)$ be a perfect probability
space, and $\sequencen{E_n}$ a sequence in $\Sigma$.   Then

\inset{{\it either} $\sequencen{\chi E_n}$ has a subsequence which is
convergent almost everywhere}

\inset{{\it or} $\sequencen{E_n}$ has a subsequence with no measurable
cluster point in $\Cal PX$.}

\proof{ Consider the sequence $\sequencen{\chi E_n^{\ssbullet}}$ in the
Hilbert space $L^2=L^2(\mu)$.   This is a norm-bounded sequence, so has
a weakly convergent subsequence $\sequence{i}{\chi E_{n_i}^{\ssbullet}}$
with limit $v$ say (4A4Kb again).   Express $v$ as $g^{\ssbullet}$ where
$g:X\to\Bbb R$ is $\Sigma$-measurable.

\woddheader{463J}{4}{2}{2}{30pt}

\quad{\bf case 1} Suppose that $g(x)\in\{0,1\}$ for almost every $x\in
X$;  set $F=\{x:g(x)=1\}$.    Then

\Centerline{$\lim_{i\to\infty}\int_F\chi E_{n_i}=\int_Fg=\mu F$,
\quad$\lim_{i\to\infty}\int_{X\setminus F}\chi E_{n_i}
=\int_{X\setminus F}g=0$.}

\noindent So, replacing $\sequence{i}{\chi E_{n_i}}$ with a
sub-subsequence if necessary, we may suppose that

\Centerline{$|\mu F-\int_F\chi E_{n_i}|\le 2^{-i}$,
\quad$|\int_{X\setminus F}\chi E_{n_i}|\le 2^{-i}$}

\noindent for every $i$.   But as $0\le \chi E_{n_i}\le 1$ everywhere,
we have $\int|\chi F-\chi E_{n_i}|\le 2^{-i+1}$ for every $i$, so that $\chi E_{n_i}\to\chi F$ a.e., and we
have a subsequence of $\sequencen{\chi E_n}$ which is convergent almost
everywhere.

\medskip

\quad{\bf case 2} Suppose that $\{x:g(x)\notin\{0,1\}\}$ has positive
measure.   Note that because
$\int_Fg=\lim_{i\to\infty}\mu(F\cap E_{n_i})$ lies between $0$ and
$\mu F$ for every $F\in\Sigma$,
$0\le g\le 1$ a.e., and $\mu\{x:0<g(x)<1\}>0$.   There is therefore an
$\epsilon>0$ such that $\mu G>0$, where
$G=\{x:\epsilon\le g(x)\le 1-\epsilon\}$.

Write $\mu_G$ for the subspace measure on $G$, and $\Sigma_G$ for its
domain;  set $\nu=(\mu G)^{-1}\mu_G$, so that $\nu$ is a probability
measure.   We know that $\mu_G$ is perfect (451Dc), so $\nu$ also is
(see the definition in 451Ad).   Now if $F\in\Sigma_G$,

\Centerline{$\lim_{i\to\infty}\nu(F\cap E_{n_i})
=(\mu G)^{-1}\int_Fg$}

\noindent lies between $\epsilon\mu F/\mu G=\epsilon\nu F$ and
$(1-\epsilon)\nu F$.

By 463I, there is a strictly increasing sequence $\sequence{k}{i(k)}$
such that $B\notin\Sigma_G$ whenever $B$ is a cluster point of
$\sequence{k}{G\cap E_{n_{i(k)}}}$ in $\Cal PG$.
If $A$ is any cluster point of $\sequence{k}{E_{n_{i(k)}}}$ in
$\BbbR^X$, then $A\cap G$ is a cluster point of $\sequence{k}{G\cap
E_{n_{i(k)}}}$ in $\Cal PG$, so cannot belong to $\Sigma_G$.   Thus
$A\notin\Sigma$.

So in this case we have a subsequence $\sequence{k}{E_{n_{i(k)}}}$ of
$\sequencen{E_n}$ which has no measurable cluster point.
}%end of proof of 463J

\leader{463K}{Fremlin's Alternative}\cmmnt{ ({\smc Fremlin 75a})} Let
$(X,\Sigma,\mu)$ be a perfect $\sigma$-finite measure space, and
$\sequencen{f_n}$ a sequence of real-valued measurable functions on $X$.
Then

\inset{{\it either} $\sequencen{f_n}$ has a subsequence which is
convergent almost everywhere}

\inset{{\it or} $\sequencen{f_n}$ has a subsequence with no measurable
cluster point in $\BbbR^X$.}

\proof{{\bf (a)} If
$\mu X=0$ then of course $\sequencen{f_n}$ itself is convergent a.e., so
we may suppose that $\mu X>0$.   If there is any $x\in X$ such that
$\sup_{n\in\Bbb N}|f_n(x)|=\infty$, then $\sequencen{f_n}$ has a
subsequence with no cluster point in $\BbbR^X$, measurable or
otherwise;
so we may suppose that $\sequencen{f_n}$ is bounded at each point of
$X$.

\medskip

{\bf (b)} Let $\lambda$ be the c.l.d.\ product of $\mu$ with Lebesgue measure on
$\Bbb R$, and $\Lambda$ its domain.   Then $\lambda$ is perfect (451Ic)
and also $\sigma$-finite (251K).   There is therefore a probability
measure $\nu$ on $X\times\Bbb R$ with the same domain and the same
negligible sets
as $\lambda$ (215B(vii)), so that $\nu$ also is perfect.
For any function $h\in\BbbR^X$, write

\Centerline{$\Omega(h)=\{(x,\alpha):x\in X,\,\alpha\le h(x)\}
\subseteq X\times\Bbb R$}

\noindent (compare 252N).

\medskip

{\bf (c)} By 463J, applied to the measure space
$(X\times\Bbb R,\Lambda,\nu)$ and the sequence
$\sequencen{\chi\Omega(f_n)}$, we have
a strictly increasing sequence $\sequence{i}{n(i)}$ such that either
$\sequence{i}{\chi\Omega(f_{n(i)})}$ is convergent $\nu$-a.e.\ or
$\sequence{i}{\Omega(f_{n(i)})}$ has no cluster point in $\Lambda$.

\medskip

\quad{\bf case 1} Suppose that $\sequence{i}{\chi\Omega(f_{n(i)})}$ is
convergent $\nu$-a.e.   Set

\Centerline{$W=\{(x,\alpha):
\lim_{i\to\infty}\chi\Omega(f_{n(i)})(x,\alpha)$ is defined$\}$.}

\noindent Then $W$ is $\lambda$-conegligible, so
$W^{-1}[\{\alpha\}]=\{x:(x,\alpha)\in W\}$ is $\mu$-conegligible for
almost every $\alpha$ (apply 252D to the complement of $W$).   Set
$D=\{\alpha:W^{-1}[\{\alpha\}]$ is $\mu$-conegligible$\}$, and let
$Q\subseteq D$ be a countable dense set;  then
$G=\bigcap_{\alpha\in Q}W^{-1}[\{\alpha\}]$ is $\mu$-conegligible.   But if $x\in G$, then for
any $\alpha\in Q$ the set
$\{i:f_{n(i)}(x)\ge\alpha\}=\{i:\chi\Omega(f_{n(i)})(x,\alpha)=1\}$
is either finite or has finite complement in $\Bbb N$, so
$\sequence{i}{f_{n(i)}(x)}$ must be convergent in $[-\infty,\infty]$.
Since $\sequencen{f_n(x)}$ is supposed to be bounded,
$\sequence{i}{f_{n(i)}(x)}$ is convergent in $\Bbb R$.   Thus in this
case we have an almost-everywhere-convergent subsequence of
$\sequencen{f_n}$.

\medskip

\quad{\bf case 2} Suppose that $\sequence{i}{\Omega(f_{n(i)})}$ has no
cluster point in $\Lambda$.   Let $h$ be any cluster point of
$\sequence{i}{f_{n(i)}}$ in $\BbbR^X$.   Then there is a non-principal
ultrafilter $\Cal F$ on $\Bbb N$ such that $h=\lim_{i\to\Cal F}f_{n(i)}$
in $\BbbR^X$.   Set $A=\lim_{i\to\Cal F}\Omega(f_{n(i)})$, so that
$A\notin\Lambda$.   If $x\in X$ and $\alpha\in\Bbb R$, then

\Centerline{$\alpha<h(x)\Longrightarrow\{i:\alpha<f_{n(i)}(x)\}\in\Cal F
\Longrightarrow(x,\alpha)\in A$,}

\Centerline{$h(x)<\alpha\Longrightarrow\{i:\alpha<f_{n(i)}(x)\}\notin\Cal F
\Longrightarrow(x,\alpha)\notin A$.}

\noindent Thus $\Omega'(h)\subseteq A\subseteq\Omega(h)$, where
$\Omega'(h)=\{(x,\alpha):\alpha<h(x)\}$.

\Quer\ If $h$ is $\Sigma$-measurable, then

\Centerline{$\Omega'(h)
=\bigcup_{q\in\Bbb Q}\{x:h(x)>q\}\times\ocint{-\infty,q}$,}

\Centerline{$\Omega(h)
=(X\times\Bbb R)\setminus\bigcup_{q\in\Bbb Q}
   \{x:h(x)<q\}\times\coint{q,\infty}$}

\noindent belong to $\Lambda$, and
$\lambda(\Omega(h)\setminus\Omega'(h))=0$ (because every vertical
section of $\Omega(h)\setminus\Omega'(h)$ is negligible).   But as
$\Omega'(h)\subseteq A\subseteq\Omega(h)$, $A\in\Lambda$ (remember that
product measures in this book are complete), which is
impossible.\ \Bang

Thus $h$ is not $\Sigma$-measurable.   As $h$ is arbitrary,
$\sequence{i}{f_{n(i)}}$ has no measurable cluster point in $\BbbR^X$.

So at least one of the envisaged alternatives must be true.
}%end of proof of 463K

\leader{463L}{Corollary} Let $(X,\Sigma,\mu)$ be a perfect
$\sigma$-finite measure space.   Write $\eusm L^0\subseteq\BbbR^X$ for
the space of real-valued $\Sigma$-measurable functions on $X$.

(a) If $K\subseteq\eusm L^0$ is relatively countably compact for the
topology $\frak T_p$ of pointwise convergence on $\eusm L^0$, then every
sequence in
$K$ has a subsequence which is convergent almost everywhere.
Consequently $K$ is relatively compact in $\eusm L^0$ for the topology
$\frak T_m$ of convergence in measure.

(b) If $K\subseteq\eusm L^0$ is countably compact for $\frak T_p$, then
it is compact for $\frak T_m$.

(c) Suppose that $K\subseteq\eusm L^0$ is countably compact for
$\frak T_p$ and that $\mu\{x:f(x)\ne g(x)\}>0$ for any distinct $f$,
$g\in K$.   Then the topologies $\frak T_m$ and $\frak T_p$ agree on $K$,
so both are compact and metrizable.

\proof{{\bf (a)} Since every sequence in $K$ must have a
$\frak T_p$-cluster point in $\eusm L^0$, 463K tells us that every
sequence in $K$ has a subsequence which is convergent almost everywhere,
therefore $\frak T_m$-convergent.   Now $K$ is relatively sequentially
compact in the pseudometrizable space $(\eusm L^0,\frak T_m)$, therefore
relatively compact (4A2Le again).

\medskip

{\bf (b)} As in (a), every sequence $\sequencen{f_n}$ in $K$ has a
subsequence $\sequencen{g_n}$ which is convergent almost everywhere.
But $\sequencen{g_n}$ has a $\frak T_p$-cluster point $g$ in $K$, and
now $g(x)=\lim_{n\to\infty}g_n(x)$ for every $x$ for which the limit is
defined;  accordingly $g_n\to g$ a.e., and $g$ is a $\frak T_m$-limit of
$\sequencen{g_n}$ in $K$.   Thus every sequence in $K$ has a $\frak
T_m$-cluster point in
$K$, and (because $\frak T_m$ is pseudometrizable) $K$ is
$\frak T_m$-compact.

\medskip

{\bf (c)} The point is that $K$ is sequentially compact under
$\frak T_p$.   \Prf\ Note that as $K$ is countably compact,
$\sup_{f\in K}|f(x)|$ is finite for every $x\in K$.   (I am passing over
the trivial
case $K=\emptyset$.)   If $\sequencen{f_n}$ is a sequence in $K$, then,
by (a), it has a subsequence $\sequencen{g_n}$ which is convergent a.e.
\Quer\ If $\sequencen{g_n}$ is not $\frak T_p$-convergent, then
there are a point $x_0\in X$ and two further subsequences
$\sequencen{g'_n}$,
$\sequencen{g''_n}$ of $\sequencen{g_n}$ such that
$\lim_{n\to\infty}g'_n(x_0)$, $\lim_{n\to\infty}g''_n(x_0)$ exist and
are different.   Now $\sequencen{g'_n}$,
$\sequencen{g''_n}$ must have cluster points $g'$, $g''\in K$ with
$g'(x_0)\ne g''(x_0)$.

However,

\Centerline{$g'(x)=\lim_{n\to\infty}g_n(x)=g''(x)$}

\noindent whenever the limit is defined, which is almost everywhere;  so
$g'\eae g''$.   And this contradicts the hypothesis that if two elements
of $K$ are equal a.e., they are identical.\  \BanG\  Thus
$\sequencen{g_n}$ is a $\frak T_p$-convergent subsequence of
$\sequencen{f_n}$.   As $\sequencen{f_n}$ is arbitrary, $K$ is
$\frak T_p$-sequentially compact.\ \Qed

Now 463Cd gives the result.
}%end of proof of 463L

\leader{463M}{Proposition} Let $X_0,\ldots,X_n$ be countably
compact topological spaces, each carrying a $\sigma$-finite 
perfect strictly positive measure which measures every Baire set.
Let $X$ be their product and $\CalBa(X_i)$ the Baire $\sigma$-algebra of
$X_i$ for each $i$.   
Then any separately continuous function $f:X\to\Bbb R$ is measurable
with respect to the $\sigma$-algebra
$\Tensorhat_{i\le n}\CalBa(X_i)$ generated by 
$\{\prod_{i\le n}E_i:E_i\in\CalBa(X_i)$ for $i\le n\}$.

\proof{ For $i\le n$ let $\mu_i$ be a $\sigma$-finite perfect
strictly positive measure on $X_i$ such that 
$\CalBa(X_i)\subseteq\dom\mu_i$;
let $\mu$ be the product measure on $X$.

\medskip

{\bf (a)} The proof relies on the fact that

\inset{($*$) if $g$, $g':X\to\Bbb R$ are distinct separately continuous
functions, then $\mu\{x:g(x)\ne g'(x)\}>0$;}

\noindent I seek to prove this, together with the stated result, by
induction on $n$.   The induction starts easily with $n=0$, so that $X$
can be identified with $X_0$, a separately continuous function on $X$ is
just a continuous function on $X_0$, and ($*$) is true because
$\mu=\mu_0$ is strictly positive.

\medskip

{\bf (b)} For the inductive step to $n+1$, given a separately continuous
function $f:X_0\times\ldots\times X_{n+1}\to\Bbb R$, set $f_t(y)=f(y,t)$
for every $y\in Y=X_0\times\ldots\times X_n$ and $t\in X_{n+1}$,
and $K=\{f_t:t\in X_{n+1}\}$   Then
every $f_t$ is separately continuous, therefore
$\Tensorhat_{i\le n}\CalBa(X_i)$-measurable, by the inductive 
hypothesis.   So $K$ consists of 
$\Tensorhat_{i\le n}\CalBa(X_i)$-measurable functions.  Moreover, again
because $f$ is separately continuous, the function $t\mapsto f_t(y)$ is
continuous for every $y$, that is, $t\mapsto f_t:X_{n+1}\to\BbbR^Y$ is
continuous;  it follows that $K$ is countably compact (4A2G(f-vi)).
Finally, by the inductive hypothesis ($*$),
$\nu\{y:f_t(y)\ne f_{t'}(y)\}>0$ whenever $t$, $t'\in X_{n+1}$ and
$f_t\ne f_{t'}$, where $\nu$ is the product measure on $Y$.

Since $\nu$ is perfect (451Ic) and 
$\Tensorhat_{i\le n}\CalBa(X_i)\subseteq\dom\nu$,
we can apply 463Lc to see that $K$ is metrizable for the
topology of pointwise convergence.   Let $\rho$ be a metric on $K$
inducing its
topology, and $\sequence{i}{g_i}$ a sequence running over a dense subset
of $K$.   (I am passing over the trivial case $K=\emptyset=X_{n+1}$.)
For $m$, $i\in\Bbb N$ set $E_{mi}=\{t:\rho(f_t,g_i)\le 2^{-m}\}$.
Because $t\mapsto f_t$ and $t\mapsto\rho(f_t,t_i)$ are continuous,
$E_{mi}\in\CalBa(X_{n+1})$.   Set
$f^{(m)}(y,t)=g_i(y)$ for $t\in E_{mi}\setminus\bigcup_{j<i}E_{mj}$
for $m$, $i\in\Bbb N$, $y\in Y$ and $t\in X_{n+1}$.   
Then $f^{(m)}:X\to\Bbb R$ is
$\Tensorhat_{i\le n+1}\CalBa(X_i)$-measurable because every $g_i$ is
$\Tensorhat_{i\le n}\CalBa(X_i)$-measurable and
every $E_{mi}$ belongs to $\CalBa(X_{n+1})$.
Also $\sequence{m}{f^{(m)}}\to f$ at every point, because
$\rho(f^{(m)}_t,f_t)\le 2^{-m}$ for every $m\in\Bbb N$ and
$t\in X_{n+1}$.   So $f$ is $\Tensorhat_{i\le n+1}\CalBa(X_i)$-measurable.

\medskip

{\bf (c)} We still have to check that ($*$) is true at the new level.
But if $h$, $h':X\to\Bbb R$ are distinct separately continuous
functions, then there are $t_0\in X_{n+1}$, $y_0\in Y$ such that
$h(y_0,t_0)\ne h'(y_0,t_0)$.   Let $G$ be an open set containing $t_0$
such that $h(y_0,t)\ne h'(y_0,t)$ whenever $t\in G$.   Then
$\nu\{y:h(y,t)\ne h'(y,t)\}>0$ for every $t\in G$, by the inductive
hypothesis, so

\Centerline{$\mu\{(y,t):h(y,t)\ne h'(y,t)\}
=\int\nu\{y:h(y,t)\ne h'(y,t)\}\mu_{n+1}(dt)>0$}

\noindent because $\mu_{n+1}$ is strictly positive.   Thus the induction
continues.
}%end of proof of 463M

\leader{463N}{Corollary} Let $X_0,\ldots,X_n$ be Hausdorff spaces with
product $X$.   Then every separately continuous function
$f:X\to\Bbb R$ is universally Radon-measurable\cmmnt{ in the sense of
434Ec}.

\proof{ Let $\mu$ be a Radon measure on $X$ and $\Sigma$ its
domain.

\medskip

{\bf (a)} Suppose first that the support $C$ of $\mu$ is compact.
For each $i\le n$, let $\pi_i:X\to X_i$ be the coordinate
projection, and $\mu_i=\mu\pi_i^{-1}$ the image Radon measure;  let
$Z_i$ be the support of $\mu_i$ and $Z=\prod_{i\le n}Z_i$.   Note that
$\pi_i[C]$ is compact and $\mu_i$-conegligible, so that 
$Z_i\subseteq\pi_i[C]$ is compact, for each $i$.   At the same time,
$\pi^{-1}[Z_i]$ is $\mu$-conegligible for each $i$, so that
$Z$ is $\mu$-conegligible.

By 463M, $f\restr Z$ is $\Tensorhat_{i\le n}\CalBa(Z_i)$-measurable;  
because $Z$ is conegligible, $f$ is $\Sigma$-measurable.

\medskip

{\bf (b)} In general, if $C\subseteq X$ is compact, then we can apply
(a) to the measure $\mu\LLcorner C$ (234M) to see that $f\restr C$ is
$\Sigma$-measurable.   
As $\mu$ is complete and locally determined and inner
regular with respect to the compact sets, $f$ is $\Sigma$-measurable 
(see 412Ja).   

As $\mu$ is arbitrary, $f$ is universally Radon-measurable.
}%end of proof of 463N

\exercises{\leader{463X}{Basic exercises $\pmb{>}$(a)}
%\spheader 463Xa
Let $(X,\Sigma,\mu)$ be a measure space, $\eusm L^0$ the
space of $\Sigma$-measurable real-valued functions on $X$, $\frak T_p$
the topology of pointwise convergence on $\eusm L^0$ and $\frak T_m$ the
topology of convergence in measure on $\eusm L^0$.   (i) Show that $\frak T_m\subseteq\frak T_p$ iff for every measurable set $E$ of finite
measure there is a countable set $D\subseteq E$ such that
$\mu^*D=\mu E$.  (ii) Show that $\frak T_p\subseteq\frak T_m$ iff
$0<\mu^*\{x\}<\infty$ for every $x\in X$.
%463A

\spheader 463Xb Let $(X,\Sigma,\mu)$ be a $\sigma$-finite measure space,
and $K\subseteq\eusm L^0$ a $\frak T_p$-countably compact set.
Show that the following are equiveridical:  (i) every sequence in $K$
has a subsequence which converges almost everywhere;  (ii) $K$ is
$\frak T_m$-compact;   (iii) $K$ is totally bounded for the uniformity
associated with the linear space topology $\frak T_m$.   Show that if
moreover the topology on $K$ induced by $\frak T_m$ is Hausdorff, then $K$ is $\frak T_p$-metrizable.
%463C

\spheader 463Xc(i) Show that there is a set of Borel measurable
functions on $[0,1]$ which is countably tight, compact and
non-metrizable
for the topology of pointwise convergence.   (ii) Show that there is a
strictly localizable measure space $(X,\Sigma,\mu)$ with a set $K$ of
measurable functions which is countably tight, compact, Hausdorff and
non-metrizable for both the topology of pointwise convergence and the
topology of convergence in measure.   \Hint{the one-point
compactification of any discrete space is countably tight.}
%463E

\spheader 463Xd Let $X$ be a topological space and
$K\subseteq C(X)$ a convex $\frak T_p$-compact set.   Show that if there is a strictly positive $\sigma$-finite topological measure on $X$, then $K$ is $\frak T_p$-metrizable.
%463G

\spheader 463Xe Use Koml\'os' theorem (276H) to shorten the proof of
463G.
%463G

\spheader 463Xf Let $(X,\Sigma,\mu)$ be any complete $\sigma$-finite
measure space.   Show that if $A\subseteq\eusm L^0$ is
$\frak T_m$-relatively compact, and $\sup_{f\in A}|f(x)|$ is finite for
every $x\in X$, then $A$ is $\frak T_p$-relatively countably compact in
$\eusm L^0$.
%463K

\spheader 463Xg Let $K$ be the set of non-decreasing functions from
$[0,1]$ to $\{0,1\}$.   Show that $K$, with its topology of pointwise
convergence, is homeomorphic to the split interval (419L).   Show that
(for any Radon measure $\mu$ on $[0,1]$) the identity map from
$(K,\frak T_p)$ to $(K,\frak T_m)$ is continuous.
%463L

\sqheader 463Xh Let $K$ be the set of non-decreasing functions from
$\omega_1$ to $\{0,1\}$.
%Show that $K$, with its topology of pointwise
%convergence, is homeomorphic to $\omega_1+1$.
Show that if $\mu$ is
the countable-cocountable measure on $\omega_1$ then $K$ is a
$\frak T_p$-compact set of measurable functions and is also
$\frak T_m$-compact, but the identity map from $(K,\frak T_p)$ to
$(K,\frak T_m)$ is not continuous.
%463L, 463Xg, 463F

\sqheader 463Xi Let $K$ be the set of functions $f:[0,1]\to\Bbb R$ such
that $\max(\|f\|_{\infty},\Var_{[0,1]}f)\le 1$, where $\Var_{[0,1]}f$ is
the variation
of $f$ (224A).   Show that $K$ is $\frak T_p$-compact and that (for any
Radon measure on $[0,1]$) the identity map from $(K,\frak T_p)$  to
$(K,\frak T_m)$ is continuous.
%463L, 463Xg

\sqheader 463Xj Let $A$ be the set of functions $f:[0,1]\to[0,1]$ such
that $\overline{\int}|f|d\mu\cdot\Var_{[0,1]}f\le 1$, where $\mu$ is
Lebesgue measure.   Show that every member of $A$ is measurable and that
every sequence in $A$ has a subsequence
which converges almost everywhere to a member of $A$, but that
$\int:A\to[0,1]$ is not $\frak T_p$-continuous, while $A$ is
$\frak T_p$-dense in $[0,1]^{[0,1]}$.
%463L, 463Xi

\spheader 463Xk Let $X$ be a Hausdorff space and $K\subseteq C(X)$ a
$\frak T_p$-compact set.   Show that if there is a strictly positive
$\sigma$-finite Radon measure on $X$ then $K$ is $\frak T_p$-metrizable.
%463L

\spheader 463Xl Let $(X,\Sigma,\mu)$ be a localizable measure space and
$K\subseteq\eusm L^0$ a non-empty $\frak T_p$-compact set.
Show that $\sup\{f^{\ssbullet}:f\in K\}$ is defined in $L^0(\mu)$.
%463L

\leader{463Y}{Further exercises (a)}
%\spheader 463Ya
Let $(X,\Sigma,\mu)$ be a probability space and $V$ a Banach space.   A
function $\phi:X\to V$ is {\bf scalarly measurable} (often called {\bf
weakly measurable}) if
$h\phi:X\to\Bbb R$ is $\Sigma$-measurable for every $h\in V^*$.
$\phi$ is {\bf Pettis integrable}, with {\bf indefinite
Pettis integral} $\theta:\Sigma\to V$, if $\int_Eh\phi\,d\mu$ is defined
and equal to $h(\theta E)$ for every $E\in\Sigma$ and every $h\in V^*$.
(i) Show that if $\phi$ is
scalarly measurable, then $K=\{h\phi:h\in V^*$, $\|h\|\le 1\}$ is a
$\frak T_p$-compact subset of $\eusm L^0$.   (ii) Show that if
$\phi$ is scalarly measurable, then it is Pettis integrable iff every
function in $K$ is integrable and $f\mapsto\int_Ef:K\to\Bbb R$ is
$\frak T_p$-continuous for every $E\in\Sigma$.   \Hint{4A4Cg.}   (iii)
In particular, if $\phi$ is bounded and scalarly measurable and the
identity map from $(K,\frak T_p)$ to $(K,\frak T_m)$ is continuous, then
$\phi$ is Pettis integrable.   (See {\smc Talagrand 84}, chap.\ 4.)

\spheader 463Yb Show that any Bochner integrable function (253Yf) is
Pettis integrable.
%463Ya

\spheader 463Yc Let $\mu$ be Lebesgue measure on $[0,1]$, and define
$\phi:[0,1]\to L^{\infty}(\mu)$ by setting
$\phi(t)=\chi[0,t]^{\ssbullet}$ for every $t\in[0,1]$.   (i) Show that
if $h\in L^{\infty}(\mu)^*$ and $\|h\|\le 1$, then $h\phi$ has variation
at most 1.  (ii) Show that $K=\{h\phi:h\in L^{\infty}(\mu)^*,\,\|h\|\le
1\}$ is a $\frak T_p$-compact set of Lebesgue measurable functions, and
that the identity map from $(K,\frak T_p)$ to $(K,\frak T_m)$ is
continuous, so that $\phi$ is Pettis integrable.   (iii) Show that
$\phi$ is not Bochner integrable.
%463Xi, 463Ya, 463Yb

\spheader 463Yd Let $(X,\Sigma,\mu)$ be a $\sigma$-finite measure space
and suppose that $\mu$ is inner regular with respect to some family
$\Cal E\subseteq\Sigma$ of cardinal at most $\omega_1$.   (Subject to
the continuum hypothesis, this is true for any subset of $\Bbb R$, for
instance.)   Show that if $K\subseteq\eusm L^0$ is
$\frak T_p$-compact then it is $\frak T_m$-compact.   (See
536C\Latereditions, or {\smc Talagrand 84}, 9-3-3.)
%463L

\spheader 463Ye Assume that the continuum hypothesis is true;  let
$\preccurlyeq$ be a well-ordering of $[0,1]$ with order type $\omega_1$
(4A1Ad).   Let $(Z,\nu)$ be the Stone space of the measure algebra of
Lebesgue measure on $[0,1]$, and $q:Z\to[0,1]$ the canonical \imp\ map
(416V). Let $g:[0,1]\to\coint{0,\infty}$ be any function.   Show that
there is a
function $f:[0,1]\times Z\to\coint{0,\infty}$ such that ($\alpha$) $f$
is continuous in the second variable ($\beta$) $f(t,z)=0$ whenever
$q(z)\preccurlyeq t$ ($\gamma$) $\int f(t,z)\nu(dz)=g(t)$ for every
$t\in[0,1]$.   Show that $f$ is universally measurable in the first
variable, but need not be $\tilde\Lambda$-measurable, where $\tilde\Lambda$
is the domain of the product Radon measure on
$[0,1]\times Z$.   Setting $f_z(t)=f(t,z)$, show that $K=\{f_z:z\in Z\}$
is a $\frak T_p$-compact set of Lebesgue measurable functions and that
$g$ belongs to the $\frak T_p$-closed convex hull of $K$ in
$\BbbR^{[0,1]}$.
%463N
}%end of exercises

\leader{463Z}{Problems} {\bf (a) A.Bellow's problem}
%463Za
Let $(X,\Sigma,\mu)$ be a probability space, and
$K\subseteq\eusm L^0$ a $\frak T_p$-compact set such that
$\{x:f(x)\ne g(x)\}$ is
non-negligible for any distinct functions $f$, $g\in K$, as in 463G and
463Lc.   Does it follow that $K$ is metrizable for $\frak T_p$?

\cmmnt{A positive answer would displace several of the arguments of
this section, and have other consequences (see 462Z, for instance).   It
is known that under any of a variety of special axioms (starting with
the continuum hypothesis) there is indeed a positive answer;
see \S536 in Volume 5, or {\smc Talagrand 84}, chap.\ 12.}

\spheader 463Zb Let $X\subseteq[0,1]$ be a set of outer Lebesgue measure
$1$, and $\mu$ the subspace measure on $X$, with $\Sigma$ its domain.
Let $K$ be a $\frak T_p$-compact subset of $\eusm L^0$.   Must
$K$ be $\frak T_m$-compact?

\spheader 463Zc Let $X_0,\ldots,X_n$ be compact Hausdorff spaces and
$f:X_0\times\ldots\times X_n\to\Bbb R$ a separately continuous function.
Must $f$ be universally measurable?

\endnotes{
\Notesheader{463} The relationship between the topologies $\frak T_p$
and $\frak T_m$ is complex, and I do not think that the results here are
complete;  in particular, we have a remarkable outstanding problem in
463Za.   Much of the work presented here has been stimulated by problems
concerning the integration of vector-valued functions.   I am keeping
this theory firmly in the `further exercises' (463Ya-463Yc), but it is
certainly the most important source of examples of pointwise compact
sets of measurable functions.   In particular, since the set
$\{h\phi:h\in V^*,\,\|h\|\le 1\}$ is necessarily convex whenever $V$ is
a Banach space and $\phi:X\to V$ is a function, we are led to look at
the special properties of convex sets, as in 463G.   There are obvious
connexions with the theory of measures on linear topological spaces,
which I will come to in \S466.

The dichotomy in 463K shows that sets of measurable functions on perfect
measure spaces are either `good' (relatively countably compact for
$\frak T_p$, relatively compact for $\frak T_m$) or `bad' (with neither
property).   It is known that the result is not true for arbitrary
$\sigma$-finite measure spaces (see \S464 below), but it is not clear
whether there are important non-perfect spaces in which it still applies
in some form;  see 463Zb.

Just as in \S462, many questions concerning the topology $\frak T_p$ on
$\BbbR^X$ can be re-phrased as questions about real-valued functions on
products $X\times K$ which are continuous in the second variable.   For
the topology of pointwise convergence on sets of measurable functions,
we find ourselves looking at functions which are measurable in the first
variable.   In this way we are led to such results as 463M-463N and
463Ye.   Concerning 463M and 463Zc, it is the case that
if $X$ and $Y$ are {\it any} compact Hausdorff spaces, and
$f:X\times Y\to\Bbb R$ is separately continuous, then $f$ is Borel
measurable\cmmnt{ ({\smc Burke \& Pol 05}, 5.2)}.

A substantial proportion of the questions which arise naturally in this
topic are known to be undecidable without using special axioms.   I am
avoiding such questions in this volume, but it is worth noting that the
continuum hypothesis, in particular, has many striking consequences
here, of which 463Ye is a sample.   It also decides 463Za and 463Zb (see
463Yd).

}%end of notes

\discrpage

