\frfilename{mt286.tex}
\versiondate{30.3.16}
\copyrightdate{2000}
\newdimen\sparewidth

\def\chaptername{Fourier analysis}
\def\sectionname{Carleson's theorem}

\def\energy{\mathop{\text{energy}}\nolimits}
\def\Innerprod#1#2{\bigl(#1\bigr|#2\bigr)}
\def\mass{\mathop{\text{mass}}\nolimits}
\def\recheck{\discrversionA{\immediate\write0{query}
    \global\advance\footnotenumber by 1
    \oldfootnote{$^{\the\footnotenumber}$}{recheck}}{}}
\def\vartildef{\tilde{\hbox{$f$}}}

%try:  f\in\eusm L^2
%      h rapidly decr test fn
%      g other

\newsection{286}

Carleson's theorem ({\smc Carleson 66}) was the (unexpected)
%Dorothy Maharam once said
solution to
a long-standing problem.   Remarkably, it can be proved by `elementary'
methods.   The hardest part of the work below, in 286J-286L, demands
only the laborious verification of inequalities.   How the inequalities
were chosen is a different matter;  for once, some of the ideas of
the proof are embodied in the statements of the lemmas.   The argument here is a
greatly expanded version of {\smc Lacey \& Thiele 00}.

The Hardy-Littlewood Maximal Theorem (286A) is important, and worth
learning even if you
leave the rest of the section as an unexamined monument.   I bring
286B-286D %286B 286C 286D
forward to the beginning of the section, even though they are little
more than worked exercises, because they also have potential uses in
other contexts.

%mention subsupersubscripts?

\cmmnt{The complexity of the argument is such that it is useful to introduce a
substantial number of special notations.   Rather than include
these in
the general index, I give a list in 286W.   Among them are ten
constants $C_1,\ldots,C_{10}$.
The values of these numbers are of no significance.   The method of
proof here is quite inappropriate if we want to estimate rates of
convergence.   I give recipes for the calculation of the $C_n$ only
for the sake of the linear logic in which this treatise is written, and
because they occasionally offer clues concerning the tactics being used.
}

In this section all integrals are with respect to Lebesgue measure $\mu$
on $\Bbb R$ unless otherwise stated.

\leader{286A}{The Maximal Theorem}
% Hardy \& Littlewood 30, but not best constant
Suppose that $1<p<\infty$ and that
$f\in\eusm L_{\Bbb C}^p(\mu)$\cmmnt{ (definition:  244P)}.   Set

\Centerline{$f^*(x)=\sup\{\Bover1{b-a}\int_a^b|f|:a\le x\le b$, $a<b\}$}

\noindent for $x\in\Bbb R$.   Then
$\|f^*\|_p\le\Bover{2^{1/p}p}{p-1}\|f\|_p$.

\proof{{\bf (a)} It is enough to consider the case $f=|f|$.   Note that
if $E\subseteq\Bbb R$ has finite measure, then

\Centerline{$\int_Ef
=\int(f\times\chi E)\times\chi E
\le\|f\times\chi E\|_p(\mu E)^{1/q}\le\|f\|_p(\mu E)^{1/q}$}

\noindent is finite, where $q=\Bover{p}{p-1}$, by H\"older's inequality
(244Eb).   Consequently, if $t>0$ and
$\int_Ef\ge t\mu E$, we must have
$t\mu E\le\|f\times\chi E\|_p(\mu E)^{1/q}$,
$t(\mu E)^{1/p}\le\|f\times\chi E\|_p$ and

\Centerline{$\mu E\le\Bover1{t^p}\|f\times\chi E\|^p_p
=\Bover1{t^p}\int_Ef^p$.}

\medskip

{\bf (b)} For $t>0$, set

\Centerline{$G_t=\{x:t(y-x)<\int_x^yf$ for some $y>x\}$.}

\medskip

\quad{\bf (i)} $G_t$ is an open set.  \Prf\ For any $y\in\Bbb R$,

\Centerline{$G_{ty}=\{x:x<y$, $t(y-x)<\int_x^yf\}$}

\noindent is open, because $x\mapsto t(y-x)$ and $x\mapsto\int_x^yf$ are
continuous (225A);  so $G_t=\bigcup_{y\in\Bbb R}G_{ty}$ is open.\ \Qed

\medskip

\quad{\bf (ii)} By 2A2I, there is a partition $\Cal C$ of $G_t$ into
open intervals.
Now $C$ is bounded and $t\mu C\le\int_Cf$ for every $C\in\Cal C$.

\medskip

\quad\Prf\grheada\ For $x\in C$, consider 
$F_x=\{y:y\ge x$, $t(y-x)\le\int_x^yf\}$.
$x\in F_x$
and $y-x\le\Bover1{t^p}\int_{-\infty}^{\infty}f^p$ for every $y\in F_x$, 
by (a), so
$F_x$ is bounded above.   Set $z_x=\sup F_x$.   Because
$y\mapsto t(y-x)-\int_x^yf$ is continuous, $z_x\in F_x$.   
\Quer\ If $z_x\in G_t$, there is a
$y>z_x$ such that $t(y-z_x)<\int_{z_x}^yf$;  but now 

\Centerline{$t(y-x)
\le\int_x^{z_x}f+\int_{z_x}^yf=\int_x^yf$}

\noindent and $y\in F_x$, which is impossible.\ \BanG\  Thus $z_x\notin G_t$
and $z_x\notin C$, so that $z_x$ is an upper bound of $C$.

\medskip

\qquad\grheadb\ This shows that 

\Centerline{$\sup C\le z_x\le x+\Bover1{t^p}\int_{-\infty}^{\infty}f^p$}

\noindent for every $x\in C$.   So in fact $C$ is
bounded and is of the form $\ooint{a,b}$ where $a<b$ in $\Bbb R$.   
\Quer\ If $t(b-a)>\int_a^bf$, there is an $x\in\ooint{a,b}$ such that
$t(b-x)>\int_x^bf$.   Now we know that $b\le z_x$ and $b\notin G_t$, 
so we have $t(z_x-b)\ge\int_b^{z_x}f$.   Adding,
$t(z_x-x)>\int_x^{z_x}f$ and $z_x\notin F_x$.\ \Bang

\medskip

\qquad\grheadc\ Thus $t\mu C\le\int_Cf$, as claimed.\ \Qed

\medskip

\quad{\bf (iii)} Accordingly, because $\Cal C$ is countable and
$f$ is non-negative, we can apply (a) in its full strength to see that

\Centerline{$\mu G_t=\sum_{C\in\Cal C}\mu C
\le\sum_{C\in\Cal C}\Bover1{t^p}\int_Cf^p
\le\Bover1{t^p}\int_{-\infty}^{\infty}f^p$}

\noindent is finite, and

\Centerline{$\int_{G_t}f=\sum_{C\in\Cal C}\int_Cf
\ge\sum_{C\in\Cal C}t\mu C=t\mu G_t$.}

\medskip

{\bf (c)} All this is true for every $t>0$.   Now if we set

\Centerline{$f_1^*(x)=\sup_{b>x}\Bover1{b-x}\int_b^af$}

\noindent for $x\in\Bbb R$, we have $\{x:f_1^*(x)>t\}=G_t$ for
every $t>0$.

For any $t>0$,

\Centerline{$\Bover1pt\mu G_t
=(1-\Bover1q)t\mu G_t
\le\int_{G_t}f-\Bover1q t\chi\Bbb R
\le\int_{-\infty}^{\infty}(f-\Bover1q t\chi\Bbb R)^+$.}

\noindent So

$$\eqalignno{\int_{-\infty}^{\infty}(f_1^*)^p
&=\int_0^{\infty}\mu\{x:f_1^*(x)^p>t\}dt\cr
\displaycause{see 252O}
&=p\int_0^{\infty}u^{p-1}\mu\{x:f_1^*(x)>u\}du\cr
\displaycause{substituting $t=u^p$}
&=p\int_0^{\infty}u^{p-1}\mu G_udu
\le p^2\int_0^{\infty}u^{p-2}
  \bigl(\int_{-\infty}^{\infty}(f-\Bover1q u\chi\Bbb R)^+\bigr)du\cr
&=p^2\int_{-\infty}^{\infty}\int_0^{\infty}
  \max(0,f(x)-\Bover1q u)u^{p-2}dudx\cr
\displaycause{by Fubini's theorem, 252B, because
$(x,u)\mapsto u^{p-2}\max(0,f(x)-\bover1qu)$ is measurable and
non-negative}
&=p^2\int_{-\infty}^{\infty}\int_0^{qf(x)}
  u^{p-2}(f(x)-\Bover1q u)dudx\cr
&=\Bover{p^2q^{p-1}}{p(p-1)}\int_{-\infty}^{\infty}f^p
=(\Bover{p}{p-1})^p\|f\|_p^p.\cr}$$

\medskip

{\bf (d)} Similarly, setting $f_2^*(x)=\sup_{a<x}\Bover1{x-a}\int_a^xf$
for $x\in\Bbb R$,
$\int_{-\infty}^{\infty}(f_2^*)^p\le(\Bover{p}{p-1})^p\|f\|_p^p$.   But
$f^*=\max(f_1^*,f_2^*)$.   \Prf\ Of course $f_1^*\le f^*$ and
$f_2^*\le f^*$.   But also, if $f^*(x)>t$, there must be a non-trivial
interval $I$ containing $x$ such that $\int_If>t\mu I$;  if $a=\inf I$
and $b=\sup I$, then either $\int_a^xf>(x-a)t$ and $f_2^*(x)>t$, or
$\int_x^bf>(b-x)t$ and $f_1^*(x)>t$.   As $x$ and $t$ are arbitrary,
$f^*=\max(f_1^*,f_2^*)$.\ \Qed

Accordingly

$$\eqalign{\|f^*\|_p^p
&=\int_{-\infty}^{\infty}(f^*)^p
=\int_{-\infty}^{\infty}\max((f_1^*)^p,(f_2^*)^p)\cr
&\le\int_{-\infty}^{\infty}(f_1^*)^p+(f_2^*)^p
\le 2(\Bover{p}{p-1})^p\|f\|^p_p.\cr}$$

\noindent Taking $p$th roots, we have the inequality we seek.
}%end of proof of 286A

\leader{286B}{Lemma} Let $g:\Bbb R\to\coint{0,\infty}$ be a function
which is non-decreasing on $\ocint{-\infty,\alpha}$, non-increasing on
$\coint{\beta,\infty}$ and constant on $[\alpha,\beta]$, where
$\alpha\le\beta$.   Then for any measurable function
$f:\Bbb R\to[0,\infty]$,
$\int_{-\infty}^{\infty}f\times g\le\int_{-\infty}^{\infty}g
  \cdot\sup_{a\le\alpha,b\ge\beta,a<b}\Bover1{b-a}\int_a^bf$.

\proof{ Set
$\gamma=\sup_{a\le\alpha,b\ge\beta,a<b}\Bover1{b-a}\int_a^bf$.   For
$n$, $k\in\Bbb N$ set
$E_{nk}=\{x:\alpha-2^n\le x\le\beta+2^n$, $g(x)\ge 2^{-n}(k+1)\}$, so
that $E_{nk}$
is either empty or a bounded interval including $[\alpha,\beta]$,
and $\int_{E_{nk}}f\le\gamma\mu E_{nk}$.   For
$n\in\Bbb N$, set $g_n=2^{-n}\sum_{k=0}^{4^n-1}\chi E_{nk}$;  then
$\sequencen{g_n}$ is a non-decreasing sequence of functions with
supremum $g$, and

$$\eqalign{\int_{-\infty}^{\infty} f\times g
&=\sup_{n\in\Bbb N}\int_{-\infty}^{\infty} f\times g_n
=\sup_{n\in\Bbb N}2^{-n}\sum_{k=0}^{4^n-1}\int_{E_{nk}}f\cr
&\le\sup_{n\in\Bbb N}2^{-n}\sum_{k=0}^{4^n-1}\gamma\mu E_{nk}
=\sup_{n\in\Bbb N}\gamma\int_{-\infty}^{\infty} g_n
=\gamma\int_{-\infty}^{\infty} g,\cr}$$

\noindent as claimed.
}%end of proof of 286B

\cmmnt{\medskip

\noindent{\bf Remark} Compare 224J.}

\vleader{48pt}{286C}{Shift, modulation and dilation}\cmmnt{ Some of
the
calculations below will be easier if we use the following formalism.}
For any function $f$ with domain included in $\Bbb R$, and
$\alpha\in\Bbb R$, we can define

\Centerline{$(S_{\alpha}f)(x)=f(x+\alpha)$,
\quad$(M_{\alpha}f)(x)=e^{i\alpha x}f(x)$,
\quad$(D_{\alpha}f)(x)=f(\alpha x)$}

\noindent whenever the right-hand sides are defined.   \cmmnt{In the
case of $S_{\alpha}f$ and $D_{\alpha}f$ it is sometimes convenient to
allow $\pm\infty$ as a value of the function.   We have the following
elementary facts.}

\spheader 286Ca $S_{-\alpha}S_{\alpha}f=f$, $D_{1/\alpha}D_{\alpha}f=f$
if $\alpha\ne 0$.

\spheader 286Cb
$S_{\alpha}(f\times g)=S_{\alpha}f\times S_{\alpha}g$,
$D_{\alpha}(f\times g)=D_{\alpha}f\times D_{\alpha}g$.

\spheader 286Cc $D_{\alpha}|f|=|D_{\alpha}f|$.

\spheader 286Cd If $f$ is integrable, then

\Centerline{$(M_{\alpha}f)\varsphat=S_{-\alpha}\varhatf$,
\quad$(S_{\alpha}f)\varsphat=M_{\alpha}\varhatf$,
\quad$(S_{\alpha}f)\varspcheck=M_{-\alpha}\varcheckf$;}

\noindent if moreover $\alpha>0$, then

\Centerline{$\alpha(D_{\alpha}f)\varsphat
=D_{1/\alpha}\varhatf$,
\quad$\alpha(D_{\alpha}f)\varspcheck
=D_{1/\alpha}\varcheckf$\dvro{.}{}}

\cmmnt{\noindent (283Cc-283Ce).}

\spheader 286Ce If $f$ belongs to
$\eusm L^1_{\Bbb C}=\eusm L^1_{\Bbb C}(\mu)$, so do $S_{\alpha}f$,
$M_{\alpha}f$ and (if $\alpha\ne 0$) $D_{\alpha}f$, and in this case

\Centerline{$\|S_{\alpha}f\|_1=\|M_{\alpha}f\|_1=\|f\|_1$,
\quad$\|D_{\alpha}f\|_1=\Bover1{|\alpha|}\|f\|_1$.}

\spheader 286Cf If $f$ belongs to
$\eusm L^2_{\Bbb C}$ so do $S_{\alpha}f$, $M_{\alpha}f$ and (if
$\alpha\ne 0$) $D_{\alpha}f$, and in this case

\Centerline{$\|S_{\alpha}f\|_2=\|M_{\alpha}f\|_2=\|f\|_2$,
\quad$\|D_{\alpha}f\|_2=\Bover1{\sqrt{|\alpha|}}\|f\|_2$.}

\spheader 286Cg If $h$ is a rapidly decreasing test
function\cmmnt{ (284A)}, so are
$M_{\alpha}h$ and $S_{\alpha}h$ and (if $\alpha\ne 0$) $D_{\alpha}h$.

\leader{286D}{Lemma} Suppose that $g:\Bbb R\to[0,\infty]$ is a
measurable function such that, for some constant $C\ge 0$,
$\int_Eg\le C\sqrt{\mu E}$ whenever $\mu E<\infty$.   Then $g$ is finite
almost everywhere and
$\biggerint_{-\infty}^{\infty}\Bover1{1+|x|}g(x)dx$ is finite.

\proof{ For any $n\ge 1$, set $E_n=\{x:|x|\le n,\,g(x)\ge n\}$;  then

\Centerline{$n\mu E_n\le\int_{E_n}g\le C\sqrt{\mu E_n}$,}

\noindent so $\mu E_n\le\Bover{C^2}{n^2}$ and

\Centerline{$\{x:g(x)=\infty\}=\bigcap_{n\ge 1}\bigcup_{m\ge n}E_m$}

\noindent has measure at most
$\inf_{n\ge 1}\sum_{m=n}^{\infty}\mu E_m=0$.

As for the integral,
set $G(x)=\int_0^xg$ for $x\ge 0$.   Then, for any $a\ge 0$,

$$\eqalignno{\int_0^a\Bover{g(x)}{1+x}dx
&=\Bover{G(a)}{1+a}+\int_0^a\Bover{G(x)}{(1+x)^2}dx\cr
\displaycause{225F}
&\le C\bigl(\Bover{\sqrt{a}}{1+a}
  +\int_0^a\Bover{\sqrt{x}}{(1+x)^2}dx\bigr)
\le C\bigl(1+\int_0^{\infty}\Bover{\sqrt{x}}{(1+x)^2}dx\bigr),\cr}$$

\noindent so

\Centerline{$\int_0^{\infty}\Bover{g(x)}{1+x}dx
\le C\bigl(1+\int_0^{\infty}\Bover{\sqrt{x}}{(1+x)^2}dx\bigr)$}


\noindent is finite.   Similarly,
$\biggerint_{-\infty}^0\Bover{g(x)}{1-x}dx$
is finite, so we have the result.
}%end of proof of 286D

\allowmorestretch{500}{
\leader{286E}{The Lacey-Thiele construction (a)}
Let $\Cal I$ be the family of all {\bf dyadic intervals} of the form
\penalty-100$\coint{2^kn,2^k(n+1)}$ where $k$, $n\in\Bbb Z$.   The essential
geometric property of $\Cal I$ is that if $I$, $J\in\Cal I$ then either
$I\subseteq J$ or $J\subseteq I$ or $I\cap J=\emptyset$.   Let $Q$ be
the set of all pairs $\sigma=(I_{\sigma},J_{\sigma})\in\Cal I^2$ such
that $\mu I_{\sigma}\cdot\mu J_{\sigma}=1$.   For $\sigma\in Q$, let
$k_{\sigma}\in\Bbb Z$
be such that $\mu J_{\sigma}=2^{k_{\sigma}}$ and
$\mu I_{\sigma}=2^{-k_{\sigma}}$;  let $x_{\sigma}$
be the midpoint of $I_{\sigma}$, $y_{\sigma}$ the midpoint of
$J_{\sigma}$, $J^l_{\sigma}\in\Cal I$ the left-hand half-interval of
$J_{\sigma}$, $J^r_{\sigma}\in\Cal I$ the
right-hand half-interval of $J_{\sigma}$, and $y^l_{\sigma}$ the lower
quartile of $J_{\sigma}$\cmmnt{, that is, the midpoint of
$J^l_{\sigma}$}.
}%end of allowmorestretch

\spheader 286Eb There is a rapidly decreasing test function $\phi$ such
that $\varhat{\phi}$ is real-valued and
$\chi[-\bover16,\bover16]\le\varhat{\phi}\le\chi[-\bover15,\bover15]$.
\prooflet{\Prf\ Look at parts (b)-(d) of the proof
of 284G.   The process there can be used to provide us with a smooth
function $\psi_1$ which is zero outside the interval
$[\bover16,\bover15]$ and strictly positive on
$\ooint{\bover16,\bover15}$;  multiplying by a suitable factor, we
can arrange that $\int_{-\infty}^{\infty}\psi_1=1$.   So if we set
$\psi_2(x)=1-\int_{-\infty}^x\psi_1$ for $x\in\Bbb R$, $\psi_2$ will be
smooth, and
$\chi\ocint{-\infty,\bover16}\le\psi_2\le\chi\ocint{-\infty,\bover15}$.
Now set $\psi_0(x)=\psi_2(x)\psi_2(-x)$ for $x\in\Bbb R$, and
$\phi=\varcheck{\psi}_0$;  $\varhat{\phi}=\psi_0$ (284C) will have
the required property.\ \Qed}
%wouldn't it be nicer if $\|\phi\|_2=1$?  but it is helpful at one
%point to have $\varhat\phi(0)=1$

For $\sigma\in Q$, set\cmmnt{ $\phi_{\sigma}
=2^{k_{\sigma}/2}M_{y^l_{\sigma}}S_{-x_{\sigma}}D_{2^{k_{\sigma}}}\phi$,
so that}

\Centerline{$\phi_{\sigma}(x)
=\sqrt{\mu J_{\sigma}}e^{iy^l_{\sigma}x}
  \phi((x-x_{\sigma})\mu J_{\sigma})$.}

\noindent\cmmnt{Observe that }$\phi_{\sigma}$ is a rapidly decreasing
test function.   \cmmnt{Now
$\varhat{\phi}_{\sigma}=2^{-k_{\sigma}/2}S_{-y^l_{\sigma}}
M_{-x_{\sigma}}D_{2^{-k_{\sigma}}}\varhat{\phi}$, that is,}

\Centerline{$\varhat{\phi}_{\sigma}(y)
=\sqrt{\mu I_{\sigma}}e^{-ix_{\sigma}(y-y^l_{\sigma})}
  \varhat{\phi}((y-y^l_{\sigma})\mu I_{\sigma})$,}

\noindent which is zero
unless\cmmnt{ $|y-y^l_{\sigma}|\le\bover15\mu J_{\sigma}$;  since the
length of $J^l_{\sigma}$ is $\bover12\mu J_{\sigma}$, this can be so
only when} $y\in J^l_{\sigma}$.   \cmmnt{We have the following simple
facts.}

\quad(i) $\|\phi_{\sigma}\|_2
=\cmmnt{\sqrt{\mu J_{\sigma}}\cdot\sqrt{\mu I_{\sigma}}\|\phi\|_2
=}\|\phi\|_2$
for every $\sigma\in Q$.

\quad(ii) $\|\varhat{\phi}_{\sigma}\|_1
=\cmmnt{\sqrt{\mu I_{\sigma}}\cdot\mu J_{\sigma}\|\varhat{\phi}\|_1
=}\sqrt{\mu J_{\sigma}}\|\varhat{\phi}\|_1$ for every $\sigma\in Q$.

\quad(iii) If $\sigma$, $\tau\in Q$ and
$J^l_{\sigma}\cap J^l_{\tau}=\emptyset$ then

\Centerline{$\innerprod{\phi_{\sigma}}{\phi_{\tau}}
=\innerprod{\varhat{\phi}_{\sigma}}{\varhat{\phi}_{\tau}}=0$\dvro{.}{,}}

\noindent\cmmnt{by 284Ob.   }(For $f$,
$g\in\eusm L_{\Bbb C}^2$, I
write $\innerprod{f}{g}$ for $\int_{-\infty}^{\infty}f\times\bar g$.)

\quad(iv) If $\sigma$, $\tau\in Q$ and
$J_{\sigma}\ne J_{\tau}$ and $J^r_{\sigma}\cap J^r_{\tau}$ is
non-empty, then\cmmnt{ $J^l_{\sigma}\cap J^l_{\tau}=\emptyset$ so}
$\innerprod{\phi_{\sigma}}{\phi_{\tau}}=0$.

\spheader 286Ec Set $w(x)=\Bover1{(1+|x|)^3}$ for $x\in\Bbb R$.
For $\sigma\in Q$, set\cmmnt{ $w_{\sigma}
=2^{k_{\sigma}}S_{-x_{\sigma}}D_{2^{k_{\sigma}}}w$, so that}

\Centerline{$w_{\sigma}(x)=w((x-x_{\sigma})\mu J_{\sigma})\mu J_{\sigma}
\le\mu J_{\sigma}=2^{k_{\sigma}}$}

\noindent for every
$x$.   Note that $w_{\sigma}=w_{\tau}$ whenever $I_{\sigma}=I_{\tau}$.

\leader{286F}{A partial order (a)} For $\sigma$, $\tau\in Q$ say that
$\tau\le\sigma$ if $J_{\tau}\subseteq J_{\sigma}$ and
$I_{\sigma}\subseteq I_{\tau}$.   Then $\le$ is a partial order on $Q$.
\cmmnt{We have the following elementary facts. }

\quad(i) If $\tau\le\sigma$, then $k_{\tau}\le k_{\sigma}$.

\quad(ii) If $\sigma$ and $\tau$ are incomparable\cmmnt{ (that is,
$\sigma\not\le\tau$ and $\tau\not\le\sigma$)}, then
$(I_{\sigma}\times J_{\sigma})\cap(I_{\tau}\times J_{\tau})$ is empty.
\prooflet{\Prf\ We may suppose that $k_{\sigma}\le k_{\tau}$.   If
$J_{\sigma}\cap J_{\tau}\ne\emptyset$, then
$J_{\sigma}\subseteq J_{\tau}$, because both are dyadic intervals, and
$J_{\sigma}$ is the shorter;  but as $\sigma\not\le\tau$, this means
that $I_{\tau}\not\subseteq I_{\sigma}$ and
$I_{\sigma}\cap I_{\tau}=\emptyset$.\ \Qed}

\quad(iii) If $\sigma$, $\sigma'$ are incomparable and both greater
than or equal to $\tau$, then
$I_{\sigma}\cap I_{\sigma'}=\emptyset$\cmmnt{, because
$J_{\tau}\subseteq J_{\sigma}\cap J_{\sigma'}$}.

\quad(iv) If $\tau\le\sigma$ and $k_{\tau}\le k\le k_{\sigma}$, then
there is
a (unique) $\upsilon$ such that $\tau\le\upsilon\le\sigma$ and
$k_{\upsilon}=k$.   \prooflet{(The point is that there is a unique
$I\in\Cal I$ such that $I_{\sigma}\subseteq I\subseteq I_{\tau}$ and
$\mu I=2^{-k}$;  and similarly there is just one candidate for
$J_{\upsilon}$.)}

\spheader 286Fb\dvro{ If}{ It will be convenient to have a shorthand for
the following:  if} $R\subseteq Q$, say that 

\Centerline{$R^+=\bigcup_{\tau\in R}\{\sigma:\tau\le\sigma\in Q\}$.}

 
\spheader 286Fc For $\tau\in Q$ set

\Centerline{$T_{\tau}=\{\sigma:\sigma\in Q$, $\tau\le\sigma$,
$J^r_{\tau}\subseteq J^r_{\sigma}\}$.}

\noindent Note that if $\sigma$, $\sigma'\in T_{\tau}$
and $k_{\sigma}\ne k_{\sigma'}$ then\cmmnt{ $J_{\sigma}\ne J_{\sigma'}$ and
$J^r_{\sigma}\cap J^r_{\sigma'}\ne\emptyset$, so}
$\innerprod{\phi_{\sigma}}{\phi_{\sigma'}}=0$\cmmnt{ (286E(b-iv))}.

\vleader{72pt}{286G}{}\cmmnt{ We shall need the results of some elementary
calculations.   The first four are nearly trivial.

\medskip

\noindent}{\bf Lemma} (a)
$\int_{-\infty}^{\infty}w_{\sigma}=\int_{-\infty}^{\infty}w=1$ for every
$\sigma\in Q$.

(b) For any $m\in\Bbb N$,
$\sum_{n=m}^{\infty}w(n+\bover12)\le\Bover1{2(1+m)^2}$.

(c) Suppose that $\sigma\in Q$ and that $I$ is an interval not
containing $x_{\sigma}$ in its interior.   Then
$\int_Iw_{\sigma}\ge w_{\sigma}(x)\mu I$, where $x$ is the midpoint of
$I$.

(d) For any $x\in\Bbb R$,
$\sum_{n=-\infty}^{\infty}w(x-n)\le 2$.

(e) There is a constant $C_1\ge 0$ such that
$|\phi(x)|\le C_1\min(w(3),w(x)^2)$ for every $x\in\Bbb R$ and

\Centerline{$|\phi_{\sigma}(x)|
\le C_1\sqrt{\mu I_{\sigma}}w_{\sigma}(x)
  \min(1,w_{\sigma}(x)\mu I_{\sigma})$}

\noindent for every $x\in\Bbb R$ and $\sigma\in Q$.

(f) There is a constant $C_2\ge 0$ such that
$\int_{-\infty}^{\infty}w(x)w(\alpha x+\beta)dx\le C_2w(\beta)$ whenever
$0\le\alpha\le 1$ and $\beta\in\Bbb R$.

(g) There is a constant $C_3\ge 0$ such that
$|\innerprod{\phi_{\sigma}}{\phi_{\tau}}|
\le C_3\sqrt{\mu I_{\sigma}}\sqrt{\mu J_{\tau}}\int_{I_{\tau}}w_{\sigma}$
whenever $\sigma$, $\tau\in Q$ and $k_{\sigma}\le k_{\tau}$.

(h) There is a constant $C_4\ge 0$ such that

\Centerline{$\sum_{\sigma\in Q,\sigma\ge\tau,k_{\sigma}=k}
\int_{\Bbb R\setminus I_{\tau}}w_{\sigma}\le C_4$}

\noindent whenever $\tau\in Q$ and $k\in\Bbb Z$.

\proof{{\bf (a)} Immediate from the definition in 286Ec,
the formulae in 286Ce and the fact that
$\biggerint_0^{\infty}\Bover1{(1+x)^3}dx=\Bover12$.

\medskip

{\bf (b)} The point is just that $w$ is convex on
$\ocint{-\infty,0}$ and
$\coint{0,\infty}$.   So we can apply 233Ib with $f(x)=x$, or
argue directly from the fact that
$w(n+\bover12)\le\bover12(w(n+\bover12+x)+w(n+\bover12-x))$ for
$|x|\le\bover12$, to see that $w(n+\bover12)\le\int_n^{n+1}w$
for every $n\ge 0$.   Accordingly

\Centerline{$\sum_{n=m}^{\infty}w(n+\bover12)
\le\int_m^{\infty}w=\Bover1{2(1+m)^2}$.}

\medskip

{\bf (c)} Similarly, because $I$ lies all on the same side of
$x_{\sigma}$, $w_{\sigma}$ is convex on $I$, so the same inequality
yields $w_{\sigma}(x)\mu I\le\int_Iw_{\sigma}$.

\medskip

{\bf (d)} Let $m$ be such that $|x-m|\le\bover12$.   Then,
using the same inequalities as before to estimate $w(x-n)$ for
$n\ne m$, we have

$$\eqalign{\sum_{n=-\infty}^{\infty}w(x-n)
&\le w(x-m)+\int_{-\infty}^{x-m-\bover12}w
  +\int_{x-m+\bover12}^{\infty}w\cr
&\le 1+\int_{-\infty}^{\infty}w
=2.\cr}$$

\medskip

{\bf (e)} Because
$\lim_{x\to\infty}x^6\phi(x)=\lim_{x\to-\infty}x^6\phi(x)=0$,
there is a $C_1>0$ such that
$|\phi(x)|\le C_1\min(w(3),w(x)^2)$ for every $x\in\Bbb R$.  Now
$|\phi(x)|\le C_1w(x)^2=C_1w(x)\min(1,w(x))$ for every $x$, so

$$\eqalign{|\phi_{\sigma}(x)|
&=\sqrt{\mu J_{\sigma}}|\phi((x-x_{\sigma})\mu J_{\sigma}|
\le C_1\sqrt{\mu J_{\sigma}}w((x-x_{\sigma})\mu J_{\sigma})
   \min(1,w((x-x_{\sigma})\mu J_{\sigma}))\cr
&=C_1\sqrt{\mu J_{\sigma}}w_{\sigma}(x)\mu I_{\sigma}
   \min(1,w_{\sigma}(x)\mu I_{\sigma})
=C_1\sqrt{\mu I_{\sigma}}w_{\sigma}(x)
   \min(1,w_{\sigma}(x)\mu I_{\sigma})\cr}$$

\noindent whenever $\sigma\in Q$ and $x\in\Bbb R$.

\medskip

{\bf (f)(i)} The first step is to note that

$$\bover{w(\bover12(1+\beta))}{w(\beta)}
=\bover{8(1+\beta)^3}{(3+\beta)^3}\le 8$$

\noindent for every $\beta\ge 0$.   Now
$\alpha w(\alpha+\alpha\beta)\le 4w(\beta)$ whenever $\beta\ge 0$ and
$\alpha\ge\bover12$.   \Prf\ For $t\ge\bover12$,

\Centerline{$\Bover{d}{dt}tw(t+t\beta)
=\Bover{1-2t(1+\beta)}{(1+t+t\beta)^4}\le 0$,}

\noindent so

\Centerline{$\alpha w(\alpha+\alpha\beta)
\le\bover12w(\bover12+\bover12\beta)\le 4w(\beta)$. \Qed}

Of course this means that

\Centerline{$\Bover1{\alpha}w(\Bover{1+\beta}{2\alpha})\le 8w(\beta)$}

\noindent whenever $\beta\ge 0$ and $0<\alpha\le 1$.

\medskip

\quad{\bf (ii)} Try $C_2=16$.   If $0<\alpha\le 1$ and $\beta\ge 0$, set
$\gamma=\Bover{1+\beta}{2\alpha}$.   Then, for any $x\ge-\gamma$,

\Centerline{$1+\alpha x+\beta
=(1+\beta)(1+\Bover{\alpha x}{1+\beta})\ge\Bover12(1+\beta)$,}

\noindent so $w(\alpha x+\beta)\le 8w(\beta)$ and

\Centerline{$\int_{-\gamma}^{\infty}w(x)w(\alpha x+\beta)dx
\le 8w(\beta)\int_{-\gamma}^{\infty}w\le 8w(\beta)$.}

\noindent On the other hand,

$$\eqalign{\int_{-\infty}^{-\gamma}w(x)w(\alpha x+\beta)dx
&\le w(\gamma)\int_{-\infty}^{\infty}w(\alpha x+\beta)dx\cr
&=\Bover1{\alpha}w(\Bover{1+\beta}{2\alpha})
  \int_{-\infty}^{\infty}w
\le 8w(\beta).\cr}$$

\noindent Putting these together,
$\int_{-\infty}^{\infty}w(x)w(\alpha x+\beta)dx
\le 16w(\beta)$;  and this is true whenever $0<\alpha\le 1$ and
$\beta\ge 0$.

\medskip

\quad{\bf (iii)} If $\alpha=0$, then

\Centerline{$\int_{-\infty}^{\infty}w(x)w(\alpha x+\beta)dx
=w(\beta)\int_{-\infty}^{\infty}w=w(\beta)\le C_2w(\beta)$}

\noindent for any $\beta$.   If $0<\alpha\le 1$ and $\beta<0$, then

$$\eqalignno{\int_{-\infty}^{\infty}w(x)w(\alpha x+\beta)dx
&=\int_{-\infty}^{\infty}w(-x)w(-\alpha x-\beta)dx\cr
\displaycause{because $w$ is an even function}
&=\int_{-\infty}^{\infty}w(x)w(\alpha x-\beta)dx
\le C_2w(-\beta)\cr
\displaycause{by (ii) above}
&=C_2w(\beta).\cr}$$

\noindent So we have the required inequality in all cases.

\medskip

{\bf (g)} Set $C_3=\max(C_1^2C_2,\|\phi\|_2^2/\int_{-1/2}^{1/2}w)$.

\medskip

\quad{\bf (i)} It is worth disposing immediately of the case
$\sigma=\tau$.   In this case,

\Centerline{$|\innerprod{\phi_{\sigma}}{\phi_{\tau}}|
=\|\phi_{\sigma}\|_2^2=\|\phi\|_2^2$,}

\noindent while

$$\eqalign{\int_{I_{\tau}}w_{\sigma}
&=\mu J_{\sigma}\int_{x_{\sigma}-\bover12\mu I_{\sigma}}
  ^{x_{\sigma}+\bover12\mu I_{\sigma}}w((x-x_{\sigma})\mu J_{\sigma})dx
=\int_{-1/2}^{1/2}w,\cr}$$

\noindent so certainly
$|\innerprod{\phi_{\sigma}}{\phi_{\tau}}|
\le C_3\int_{I_{\tau}}w_{\sigma}$.

\medskip

\quad{\bf (ii)} If $\sigma\ne\tau$ and $I_{\sigma}=I_{\tau}$ then
$J_{\sigma}\cap J_{\tau}=\emptyset$ so
$\innerprod{\phi_{\sigma}}{\phi_{\tau}}=0$, by 286E(b-iii).

\medskip

\quad{\bf (iii)} Now suppose that $I_{\sigma}\ne I_{\tau}$.   In this
case, because $\mu I_{\tau}\le\mu I_{\sigma}$,
$I_{\tau}$ must lie all on the same side of $x_{\sigma}$, so
$\int_{I_{\tau}}w_{\sigma}\ge w_{\sigma}(x_{\tau})\mu I_{\tau}$, by (c).
Accordingly

$$\eqalignno{|\innerprod{\phi_{\sigma}}{\phi_{\tau}}|
&\le\int_{-\infty}^{\infty}|\phi_{\sigma}|\times|\phi_{\tau}|
\le C_1^2\sqrt{\mu I_{\sigma}}\sqrt{\mu I_{\tau}}
  \int_{-\infty}^{\infty}w_{\sigma}\times w_{\tau}
\Displaycause{using (e) twice}
=C_1^2\sqrt{\mu J_{\sigma}}\sqrt{\mu J_{\tau}}\int_{-\infty}^{\infty}
  w((x-x_{\sigma})\mu J_{\sigma})w((x-x_{\tau})\mu J_{\tau})dx\cr&
=C_1^2\sqrt{\mu J_{\sigma}}\sqrt{\mu I_{\tau}}\int_{-\infty}^{\infty}
  w(x\mu J_{\sigma}\mu I_{\tau}
    +(x_{\tau}-x_{\sigma})\mu J_{\sigma})w(x)dx\cr&
\le C_1^2C_2\sqrt{\mu J_{\sigma}}\sqrt{\mu I_{\tau}}
  w((x_{\tau}-x_{\sigma})\mu J_{\sigma})
\Displaycause{by (f), since $\mu J_{\sigma}\mu I_{\tau}\le 1$}
\le C_3\sqrt{\mu I_{\sigma}}\sqrt{\mu I_{\tau}}w_{\sigma}(x_{\tau})
\le C_3\sqrt{\mu I_{\sigma}}\sqrt{\mu J_{\tau}}
   \int_{I_{\tau}}w_{\sigma},\cr}$$

\noindent as required.

\medskip

{\bf (h)} Set
$C_4=2\sum_{j=0}^{\infty}\int_{j+\bover12}^{\infty}w$;  this is
finite because $\int_{\alpha}^{\infty}w=\Bover1{2(1+\alpha)^2}$
for every $\alpha\ge 0$.

If $k<k_{\tau}$ then $k_{\sigma}\ne k$ for any $\sigma\ge\tau$, so the
result is trivial.   If $k\ge k_{\tau}$, then for each dyadic
subinterval $I$ of $I_{\tau}$ of length $2^{-k}$ there is exactly one
$\sigma\ge\tau$ such that $I_{\sigma}=I$, since $J_{\sigma}$ must be the
unique dyadic interval of length $2^k$ including $J_{\tau}$.
List these as
$\sigma_0,\ldots$ in ascending order of the centres $x_{\sigma_j}$, so
that if $I_{\tau}=\coint{m\mu I_{\tau},(m+1)\mu I_{\tau}}$ then
$x_{\sigma_j}=m\mu I_{\tau}+2^{-k}(j+\bover12)$, for
$j<2^{k-k_{\tau}}$.   Now

$$\eqalign{\sum_{j=0}^{2^{k-k_{\tau}}-1}
  \int_{-\infty}^{m\mu I_{\tau}}w_{\sigma_j}
&=2^k\sum_{j=0}^{2^{k-k_{\tau}}-1}\int_{-\infty}^{m\mu I_{\tau}}
  w(2^k(x-m\mu I_{\tau})-j-\Bover12)dx\cr
&=\sum_{j=0}^{2^{k-k_{\tau}}-1}
  \int_{-\infty}^0w(x-j-\Bover12)dx\cr
&\le\sum_{j=0}^{\infty}
  \int_{j+\bover12}^{\infty}w
=\Bover12C_4.\cr}$$

\noindent Similarly (since $w$ is an even function, so the whole picture
is symmetric about $x_{\tau}$)

\Centerline{$\sum_{j=0}^{2^{k-k_{\tau}}-1}
  \int_{(m+1)\mu I_{\tau}}^{\infty}w_{\sigma_j}
\le\Bover12C_4$,}

\noindent and

\Centerline{$\sum_{\sigma\ge\tau,k_{\sigma}=k}
  \int_{\Bbb R\setminus I_{\tau}}w_{\sigma}\le C_4$,}

\noindent as required.
}%end of proof of 286G

\leader{286H}{`Mass' and `energy'}\cmmnt{ ({\smc Lacey \& Thiele 00})}
If $P$ is a subset of $Q$, $E\subseteq\Bbb R$ is measurable,
$g:\Bbb R\to\Bbb R$ is measurable, and $f\in\eusm L^2_{\Bbb C}$, set

\Centerline{$\mass_{Eg}(P)
=\sup_{\sigma\in P,\tau\in Q,\tau\le\sigma}
  \int_{E\cap g^{-1}[J_{\tau}]}w_{\tau}
\le\sup_{\tau\in Q}\int_{-\infty}^{\infty}w_{\tau}=1$,}

\Centerline{$\Delta_f(P)
=\sum_{\sigma\in P}|\innerprod{f}{\phi_{\sigma}}|^2$,}

\Centerline{$\energy_f(P)
=\sup_{\tau\in Q}\sqrt{\mu J_{\tau}}\sqrt{\Delta_f(P\cap T_{\tau})}$.}

\noindent If $P'\subseteq P$ then $\mass_{Eg}(P')\le\mass_{Eg}(P)$ and
$\energy_f(P')\le\energy_f(P)$.   \cmmnt{Note that}
$\energy_f(\{\sigma\})=\sqrt{\mu J_{\sigma}}|\innerprod{f}{\phi_{\sigma}}|$
for any $\sigma\in Q$\cmmnt{, since if $\sigma\in T_{\tau}$ then
$\mu J_{\tau}\le\mu J_{\sigma}$}.
%end of 286H

\leader{286I}{Lemma} If $P\subseteq Q$ is finite and
$f\in\eusm L^2_{\Bbb C}$, then

(a) $\Delta_f(P)\le\|\sum_{\sigma\in P}
  \innerprod{f}{\phi_{\sigma}}\phi_{\sigma}\|_2\|f\|_2$,

(b) $\sum_{\sigma,\tau\in P,J_{\sigma}=J_{\tau}}
  \bigl|\innerprod{f}{\phi_{\sigma}}
    \innerprod{\phi_{\sigma}}{\phi_{\tau}}
    \innerprod{\phi_{\tau}}{f}\bigr|\le C_3\Delta_f(P)$.

\proof{{\bf (a)}

\Centerline{$\Delta_f(P)
=\sum_{\sigma\in P}
  \innerprod{f}{\phi_{\sigma}}\innerprod{\phi_{\sigma}}{f}
=\Innerprod{\sum_{\sigma\in P}
  \innerprod{f}{\phi_{\sigma}}\phi_{\sigma}}{f}
\le\|\sum_{\sigma\in P}
  \innerprod{f}{\phi_{\sigma}}\phi_{\sigma}\|_2\|f\|_2$}

\noindent by Cauchy's inequality (244Eb).

\medskip

{\bf (b)}

$$\eqalignno{\sum_{\Atop{\sigma,\tau\in P}{J_{\sigma}=J_{\tau}}}
  \bigl|\innerprod{f}{\phi_{\sigma}}
    \innerprod{\phi_{\sigma}}{\phi_{\tau}}
    \innerprod{\phi_{\tau}}{f}\bigr|
&\le\sum_{\Atop{\sigma,\tau\in P}{J_{\sigma}=J_{\tau}}}
  \Bover12\bigl(|\innerprod{f}{\phi_{\sigma}}|^2
             +|\innerprod{f}{\phi_{\tau}}|^2\bigr)
  |\innerprod{\phi_{\sigma}}{\phi_{\tau}}|\cr
\displaycause{because $|\xi\zeta|\le\bover12(|\xi|^2+|\zeta|^2)$ for
all complex numbers $\xi$, $\zeta$}
&=\sum_{\sigma\in P}\sum_{\Atop{\tau\in P}{J_{\sigma}=J_{\tau}}}
  |\innerprod{f}{\phi_{\sigma}}|^2
  |\innerprod{\phi_{\sigma}}{\phi_{\tau}}|\cr
&\le\sum_{\sigma\in P}|\innerprod{f}{\phi_{\sigma}}|^2
  \sum_{\Atop{\tau\in P}{J_{\sigma}=J_{\tau}}}
     C_3\int_{I_{\tau}}w_{\sigma}\cr
\displaycause{by 286Gg, since $k_{\sigma}=k_{\tau}$ if
$J_{\sigma}=J_{\tau}$}
&\le\sum_{\sigma\in P}|\innerprod{f}{\phi_{\sigma}}|^2
  C_3\int_{-\infty}^{\infty}w_{\sigma}\cr
\displaycause{because if $\tau$, $\tau'$ are distinct members of $P$ and
$J_{\tau}=J_{\tau'}$, then $I_{\tau}$ and $I_{\tau'}$ are disjoint}
&=C_3\sum_{\sigma\in P}|\innerprod{f}{\phi_{\sigma}}|^2
=C_3\Delta_f(P).\cr}$$
}%end of proof of 286I

\leader{286J}{Lemma} Set $C_5=2^{12}$.   If $P\subseteq Q$ is finite,
$E\subseteq\Bbb R$ is
measurable, $g:\Bbb R\to\Bbb R$ is measurable, and
$\gamma\ge\mass_{Eg}(P)$, then we can find 
$R\subseteq Q$ such that
$\gamma\sum_{\tau\in R}\mu I_{\tau}\le C_5\mu E$
and\cmmnt{ (in the notation of 286Fb)} 
$\mass_{Eg}(P\setminus R^+)\le\bover14\gamma$.

\proof{{\bf (a)} If $\gamma=0$ we can take $R=\emptyset$.
Otherwise, set 
$P_1=\{\sigma:\sigma\in P$, $\mass_{Eg}(\{\sigma\})>\bover14\gamma\}$.   
For each $\sigma\in P_1$ let
$\sigma'\in Q$ be such that $\sigma'\le\sigma$ and
$\int_{E\cap g^{-1}[J_{\sigma'}]}w_{\sigma'}>\bover14\gamma$.   Let
$R$ be the set of elements of $\{\sigma':\sigma\in P_1\}$
which are minimal for $\le$.   Then
$P\setminus R^+\subseteq\{\sigma:\mass_{Eg}(\{\sigma\})\le\bover14\gamma\}$
so $\mass_{Eg}(P\setminus R^+)\le\bover14\gamma$.

\medskip

{\bf (b)} For $k\in\Bbb N$ set

\Centerline{$R_k
=\{\tau:\tau\in R$,
  $\mu J_{\tau}\cdot\mu(E\cap g^{-1}[J_{\tau}]\cap I^{(k)}_{\tau})
  \ge 2^{2k-9}\gamma\}$,}

\noindent where $I^{(k)}_{\tau}$ is the half-open
interval with the same centre as
$I_{\tau}$ and $2^k$ times its length.   Now
$R=\bigcup_{k\in\Bbb N}R_k$.   \Prf\ Take $\tau\in R$.   If
$k\in\Bbb N$ and $x\in\Bbb R\setminus I^{(k)}_{\tau}$, then
$|x-x_{\tau}|\ge\bover12\mu I^{(k)}_{\tau}=2^{k-1}\mu I_{\tau}$, so

\Centerline{$w_{\tau}(x)
=w((x-x_{\tau})\mu J_{\tau})\mu J_{\tau}
\le w(2^{k-1})\mu J_{\tau}
=(1+2^{k-1})^{-3}\mu J_{\tau}$.}

\noindent Accordingly

$$\eqalign{\Bover14\gamma
&<\int_{E\cap g^{-1}[J_{\tau}]}w_{\tau}
=\int_{E\cap g^{-1}[J_{\tau}]\cap I_{\tau}}w_{\tau}
  +\sum_{k=0}^{\infty}\int_{E\cap g^{-1}[J_{\tau}]
    \cap I^{(k+1)}_{\tau}\setminus I^{(k)}_{\tau}}w_{\tau}\cr
&\le \mu J_{\tau}\cdot\mu(E\cap g^{-1}[J_{\tau}]\cap I_{\tau})
  +\sum_{k=0}^{\infty}(1+2^{k-1})^{-3}\mu J_{\tau}\cdot
   \mu(E\cap g^{-1}[J_{\tau}]\cap I^{(k+1)}_{\tau}).\cr}$$

\noindent It follows that either

\Centerline{$\mu J_{\tau}\cdot\mu(E\cap g^{-1}[J_{\tau}]\cap I_{\tau})
\ge\Bover18\gamma$}

\noindent and $\tau\in R_0$, or there is some $k\in\Bbb N$ such that

\Centerline{$(1+2^{k-1})^{-3}\mu J_{\tau}
  \cdot\mu(E\cap g^{-1}[J_{\tau}]\cap I^{(k+1)}_{\tau})
  \ge 2^{-k-4}\gamma$}

\noindent and

\Centerline{$\mu J_{\tau}
  \cdot\mu(E\cap g^{-1}[J_{\tau}]\cap I^{(k+1)}_{\tau})
\ge(1+2^{k-1})^32^{-k-4}\gamma
\ge 2^{2k-7}\gamma$,}

\noindent so that $\tau\in R_{k+1}$.\ \Qed

\medskip

{\bf (c)} For every $k\in\Bbb N$,
$\gamma\sum_{\tau\in R_k}\mu I_{\tau}\le 2^{11-k}\mu E$.   \Prf\ If
$R_k=\emptyset$, this is trivial.   Otherwise, enumerate $R_k$ as
$\langle\tau_j\rangle_{j\le n}$ in such a way that
$k_{\tau_j}\le k_{\tau_l}$ if $j\le l\le n$.   Define
$q:\{0,\ldots,n\}\to\{0,\ldots,n\}$ inductively by the rule

\Centerline{$q(l)
=\min(\{l\}\cup\{j:j<l$, $q(j)=j$,
$(I^{(k)}_{\tau_j}\times J_{\tau_j})
  \cap(I^{(k)}_{\tau_l}\times J_{\tau_l})\ne\emptyset\})$}

\noindent for each $l\le n$.   Note that, for $l\le n$, 
$q(q(l))=q(l)\le l$ and
$I^{(k)}_{\tau_{q(l)}}\cap I^{(k)}_{\tau_l}\ne\emptyset$,
so that

\Centerline{$I_{\tau_l}\subseteq I^{(k)}_{\tau_l}
\subseteq I^{(k+2)}_{\tau_{q(l)}}$, }

\noindent because $\mu I_{\tau_l}^{(k)}\le\mu I_{\tau_{q(l)}}^{(k)}$.
Moreover, if $j<l\le n$ and $q(j)=q(l)$, then both $J_{\tau_j}$ and
$J_{\tau_l}$ meet $J_{\tau_{q(j)}}$, therefore include it, and
$J_{\tau_j}\subseteq J_{\tau_l}$.   But as $\tau_j$ and $\tau_l$ are
distinct members of $R$, $\tau_j\not\le\tau_l$ and
$I_{\tau_j}\cap I_{\tau_l}$ must be empty.

Set $M=\{q(j):j\le n\}$.   We have

$$\eqalign{\gamma\sum_{\tau\in R_k}\mu I_{\tau}
&=\gamma\sum_{m\in M}\sum_{\Atop{j\le n}{q(j)=m}}\mu I_{\tau_j}
\le\gamma\sum_{m\in M}\mu I^{(k+2)}_{\tau_m}
=2^{k+2}\gamma\sum_{m\in M}\mu I_{\tau_m}\cr
&\le 2^{k+2}\sum_{m\in M}
  2^{9-2k}\mu(E\cap g^{-1}[J_{\tau_m}]\cap I^{(k)}_{\tau_m})\cr
&\le 2^{k+2}\cdot 2^{9-2k}\mu E
=2^{11-k}\mu E\cr}$$

\noindent because if $l$, $m\in M$ and $l<m$ then
$I^{(k)}_{\tau_l}\times J_{\tau_l}$ and
$I^{(k)}_{\tau_m}\times J_{\tau_m}$ are disjoint (since $q(m)=m$)),
so that $g^{-1}[J_{\tau_l}]\cap I^{(k)}_{\tau_l}$ and
$g^{-1}[J_{\tau_m}]\cap I^{(k)}_{\tau_m}$ are disjoint.\ \Qed

\medskip

{\bf (d)} Accordingly

\Centerline{$\gamma\sum_{\tau\in R}\mu I_{\tau}
\le\gamma\sum_{k=0}^{\infty}\sum_{\tau\in R_k}\mu I_{\tau}
\le 2^{12}\mu E$,}

\noindent as required.
}%end of proof of 286J

\leader{286K}{Lemma}
Set $C_6=4(C_3+4C_3\sqrt{2C_4})$.   Suppose that $P\subseteq Q$ is a finite set,
$f\in\eusm L^2_{\Bbb C}$, $\|f\|_2=1$ and $\gamma\ge\energy_f(P)$.
Then we can find $R\subseteq Q$ such that 
$\gamma^2\sum_{\tau\in R}\mu I_{\tau}\le C_6$ and 
$\energy_f(P\setminus R^+)\le\bover12\gamma$.

\proof{{\bf (a)} We may suppose that $\gamma>0$ and that
$P\ne\emptyset$, since otherwise we can take $R=\emptyset$.

\medskip

\quad{\bf (i)}
There are only finitely many sets of the form $P\cap T_{\tau}$
for $\tau\in Q$;  let $\tilde R\subseteq Q$ be a non-empty finite set
such that whenever $\tau\in Q$
and $P\cap T_{\tau}$ is not empty, there is a $\tau'\in \tilde R$ such that
$P\cap T_{\tau}=P\cap T_{\tau'}$ and $k_{\tau'}\ge k_{\tau}$;
this is possible because if $A\subseteq P$ is not empty then
$k_{\tau}\le\min_{\sigma\in A}k_{\sigma}$ whenever $T_{\tau}\supseteq A$.

\medskip

\quad{\bf (ii)} Choose $\tau_0,\tau_1,\ldots$, $P_0,P_1,\ldots$
inductively, as follows.   $P_0=P$.   Given that $P_j\subseteq P$ is
not empty, consider

\Centerline{$R_j
=\{\tau:\tau\in\tilde R$, $\Delta_f(P_j\cap T_{\tau})
  \ge\Bover14\gamma^2\mu I_{\tau}\}$.}

\noindent If $R_j=\emptyset$, stop the induction and set $n=j$ and
$R=\{\tau_l:l<j\}$.   Otherwise, among the members of
$R_j$ take one with $y_{\tau}$ as far to the left as possible, and call
it $\tau_j$;  set $P_{j+1}=P_j\setminus\{\tau_j\}^+$, and
continue.   Note that as $R_{j+1}\subseteq R_j$ for every $j$,
$y_{\tau_{j+1}}\ge y_{\tau_j}$ for every $j$.

The induction must stop at a finite stage because if it does not stop
with $n=j$ then
$\Delta_f(P_j\cap T_{\tau_j})>0$, so $P_j\cap T_{\tau_j}$ is not empty
and $P_{j+1}\subseteq P_j\setminus T_{\tau_j}$ is a proper subset of
$P_j$, while $P_0=P$ is finite.   Since $R_n=\emptyset$,

$$\eqalign{\energy_f(P\setminus R^+)
&=\energy_f(P_n)
=\sup_{\tau\in Q}\sqrt{\mu J_{\tau}}\sqrt{\Delta_f(P_n\cap T_{\tau})}\cr&
=\max_{\tau\in\tilde R}\sqrt{\mu J_{\tau}}\sqrt{\Delta_f(P_n\cap T_{\tau})}
\le\Bover12\gamma.\cr}$$

\medskip

\quad{\bf (iii)} Set
$P'_j=P_j\cap T_{\tau_j}\subseteq P_j\setminus P_{j+1}$ for $j<n$,
so that $\ofamily{j}{n}{P'_j}$ is disjoint, and
$P'=\bigcup_{j<n}P'_j\subseteq P$.
Then if $\sigma\in P'$, $j<n$ and $J_{\tau_j}\subseteq J^l_{\sigma}$,
$I_{\sigma}\cap I_{\tau_j}=\emptyset$.   \Prf\
Let $l<n$ be such that $\sigma\in P'_l$.   Then
$y_{\tau_j}\in J_{\tau_j}\subseteq J^l_{\sigma}$ and
$y_{\tau_l}\in J^r_{\tau_l}\subseteq J^r_{\sigma}$, so
$y_{\tau_j}<y_{\tau_l}$ and $j<l$.   Accordingly $P_{j+1}\supseteq P_l$
contains $\sigma$, so $\sigma\not\ge\tau_j$;  as
$J_{\tau_j}\subseteq J_{\sigma}$, $I_{\sigma}\not\subseteq I_{\tau_j}$,
while $\mu I_{\sigma}\le\mu I_{\tau_j}$, so $I_{\sigma}$ is disjoint from
$I_{\tau_j}$.\ \Qed

It follows that if $\sigma$, $\tau\in P'$ are distinct and
$J^l_{\sigma}\cap J^l_{\tau}$ is not empty, then
$I_{\sigma}\cap I_{\tau}=\emptyset$.   \Prf\ If $J_{\sigma}=J_{\tau}$
this is true just because $\sigma\ne\tau$.   Otherwise, since
$J_{\sigma}$ and $J_{\tau}$ intersect, one is included in the other;
suppose that $J_{\sigma}\subset J_{\tau}$.   Since $J_{\sigma}$ meets
$J^l_{\tau}$, $J_{\sigma}\subseteq J^l_{\tau}$.   Now let $j<n$ be such
that $\sigma\in P'_j$;  then $\sigma\ge\tau_j$, so
$J_{\tau_j}\subseteq J_{\sigma}\subseteq J^l_{\tau}$, and
$I_{\sigma}\cap I_{\tau}\subseteq I_{\tau_j}\cap I_{\tau}=\emptyset$ by
the last remark.\ \Qed

\medskip

{\bf (b)} Now let us estimate

\Centerline{$\gamma^2\sum_{j<n}\mu I_{\tau_j}
\le 4\sum_{j<n}\Delta_f(P'_j)=4\Delta_f(P')=4\alpha$}

\noindent say.   Because $\|f\|_2=1$, we have
$\alpha\le\|\sum_{\sigma\in P'}
  \innerprod{f}{\phi_{\sigma}}\phi_{\sigma}\|_2$ (286Ia).    So

$$\eqalignno{\alpha^2
&\le\|\sum_{\sigma\in P'}
  \innerprod{f}{\phi_{\sigma}}\phi_{\sigma}\|^2_2
=\sum_{\sigma,\tau\in P'}
  \innerprod{f}{\phi_{\sigma}}\innerprod{\phi_{\sigma}}{\phi_{\tau}}
  \innerprod{\phi_{\tau}}{f}\cr
&=\sum_{\Atop{\sigma,\tau\in P'}{J_{\sigma}=J_{\tau}}}
  \innerprod{f}{\phi_{\sigma}}\innerprod{\phi_{\sigma}}{\phi_{\tau}}
  \innerprod{\phi_{\tau}}{f}\cr
&\mskip100mu
 +\sum_{\Atop{\sigma,\tau\in P'}{J_{\sigma}\subseteq J^l_{\tau}}}
  \innerprod{f}{\phi_{\sigma}}\innerprod{\phi_{\sigma}}{\phi_{\tau}}
  \innerprod{\phi_{\tau}}{f}
 +\sum_{\Atop{\sigma,\tau\in P'}{J_{\tau}\subseteq J^l_{\sigma}}}
  \innerprod{f}{\phi_{\sigma}}\innerprod{\phi_{\sigma}}{\phi_{\tau}}
  \innerprod{\phi_{\tau}}{f}\cr}$$

\noindent because $\innerprod{\phi_{\sigma}}{\phi_{\tau}}=0$ unless
$J^l_{\sigma}\cap J^l_{\tau}\ne\emptyset$, as noted in 286E(b-iii).

Take
these three terms separately.   For the first, we have

\Centerline{$\sum_{\sigma,\tau\in P',J_{\sigma}=J_{\tau}}
  \bigl|\innerprod{f}{\phi_{\sigma}}
    \innerprod{\phi_{\sigma}}{\phi_{\tau}}
    \innerprod{\phi_{\tau}}{f}\bigr|
\le C_3\alpha$}

\noindent by 286Ib.   For the second term, we have

$$\eqalign{\sum_{\Atop{\sigma,\tau\in P'}{J_{\sigma}\subseteq J^l_{\tau}}}
  \bigl|\innerprod{f}{\phi_{\sigma}}
   \innerprod{\phi_{\sigma}}{\phi_{\tau}}
   \innerprod{\phi_{\tau}}{f}\bigr|
&\le\sum_{\sigma\in P'}|\innerprod{f}{\phi_{\sigma}}|
  \sum_{\Atop{\tau\in P'}{J_{\sigma}\subseteq J^l_{\tau}}}
   |\innerprod{\phi_{\sigma}}{\phi_{\tau}}
   \innerprod{\phi_{\tau}}{f}|\cr
&\le\sqrt{\sum_{\sigma\in P'}|\innerprod{f}{\phi_{\sigma}}|^2}
  \sqrt{\sum_{\sigma\in P'}
   \bigl(\sum_{\Atop{\tau\in P'}{J_{\sigma}\subseteq J^l_{\tau}}}
     |\innerprod{\phi_{\sigma}}{\phi_{\tau}}
     \innerprod{\phi_{\tau}}{f}|\bigr)^2}\cr
&=\sqrt{\alpha}\sqrt{\sum_{j<n}H_j},\cr}$$

\noindent where for $j<n$ I set

$$H_j=\sum_{\sigma\in P'_j}
   \bigl(\sum_{\Atop{\tau\in P'}{J_{\sigma}\subseteq J^l_{\tau}}}
     |\innerprod{\phi_{\sigma}}{\phi_{\tau}}
     \innerprod{\phi_{\tau}}{f}|\bigr)^2.$$

\noindent Now we can estimate $H_j$ by observing that, for any
$\tau\in P'$,

\Centerline{$|\innerprod{\phi_{\tau}}{f}|
=\sqrt{\mu I_{\tau}}\energy_f(\{\tau\})\le\gamma\sqrt{\mu I_{\tau}}$,}

\noindent while if $\sigma$, $\tau\in P'$ and
$J_{\tau}^l\supseteq J_{\sigma}$ then

\Centerline{$|\innerprod{\phi_{\sigma}}{\phi_{\tau}}|
\le C_3\sqrt{\mu I_{\sigma}}\sqrt{\mu J_{\tau}}\int_{I_{\tau}}w_{\sigma}$}

\noindent by 286Gg.   We also need to know that if
$\sigma\in P'_j$ and $\tau$, $\tau'$ are distinct elements of
$P'$ such that $J_{\sigma}\subseteq J^l_{\tau}\cap J^l_{\tau'}$, then
$I_{\tau}$, $I_{\tau'}$ and $I_{\tau_j}$ are all disjoint, by (a-iii)
above, because $J_{\tau_j}\subseteq J_{\sigma}$.   So we have

$$\eqalignno{H_j
&\le\sum_{\sigma\in P'_j}
   \bigl(\sum_{\Atop{\tau\in P'}{J_{\sigma}\subseteq J^l_{\tau}}}
    \gamma\sqrt{\mu I_{\tau}}
      \cdot C_3\sqrt{\mu I_{\sigma}}\sqrt{\mu J_{\tau}}
      \int_{I_{\tau}}w_{\sigma}\bigr)^2\cr
&=C_3^2\gamma^2\sum_{\sigma\in P'_j}\mu I_{\sigma}
   \bigl(\sum_{\Atop{\tau\in P'}{J_{\sigma}\subseteq J^l_{\tau}}}
     \int_{I_{\tau}}w_{\sigma}\bigr)^2
\le C_3^2\gamma^2\sum_{\sigma\in P'_j}\mu I_{\sigma}
   (\int_{\Bbb R\setminus I_{\tau_j}}w_{\sigma})^2\cr
&\le C_3^2\gamma^2\sum_{k=k_{\tau_j}}^{\infty}2^{-k}
   \sum_{\Atop{\sigma\in P'_j}{k_{\sigma}=k}}
   \int_{\Bbb R\setminus I_{\tau_j}}w_{\sigma}
   \cdot\int_{-\infty}^{\infty}w_{\sigma}
\le C_3^2\gamma^2\sum_{k=k_{\tau_j}}^{\infty}2^{-k}C_4\cr
\displaycause{by 286Ga and 286Gh, since $\sigma\ge\tau_j$ for every
$\sigma\in P'_j$}
&=C_3^2\gamma^22^{-k_{\tau_j}+1}C_4
=2C_3^2C_4\gamma^2\mu I_{\tau_j}.\cr}$$

\noindent Accordingly

\Centerline{$\sum_{j<n}H_j
\le 2C_3^2\gamma^2C_4\sum_{j<n}\mu I_{\tau_j}
\le 2C_3^2C_4\cdot 4\alpha$,}

\noindent and

\Centerline{$\sum_{\sigma,\tau\in P',J_{\sigma}\subseteq J^l_{\tau}}
  |\innerprod{f}{\phi_{\sigma}}
   \innerprod{\phi_{\sigma}}{\phi_{\tau}}
   \innerprod{\phi_{\tau}}{f}|
\le\sqrt{\alpha\sumop_{j<n}H_j}
\le 2C_3\alpha\sqrt{2C_4}$.}

Similarly,

\Centerline{$\sum_{\sigma,\tau\in P',J_{\tau}\subseteq J^l_{\sigma}}
  |\innerprod{f}{\phi_{\sigma}}
   \innerprod{\phi_{\sigma}}{\phi_{\tau}}
   \innerprod{\phi_{\tau}}{f}|
\le 2C_3\alpha\sqrt{2C_4}$;}

\noindent putting these together,

\Centerline{$\alpha^2
\le\alpha(C_3+4C_3\sqrt{2C_4})
=\Bover14\alpha C_6$}

\noindent and $\alpha\le\bover14C_6$.   But this means that

\Centerline{$\gamma^2\sum_{j<n}\mu I_{\tau_j}
\le 4\alpha\le C_6$,}

\noindent and $R=\{\tau_j:j<n\}$ has both the properties required.
}%end of proof of 286K

\leader{286L}{Lemma} Set

\Centerline{$C_7
=C_1\bigl(\Bover{7}2+\Bover87+\Bover{28}{w(3/2)}
  +\Bover{4\sqrt{14 C_3}}{w(3/2)}\bigr)$.}

\noindent Suppose that $P$ is a finite subset of $Q$ with a lower bound
$\tau$ in $Q$ for the ordering $\le$,
$E\subseteq\Bbb R$ is measurable,
$g:\Bbb R\to\Bbb R$ is measurable and $f\in\eusm L^2_{\Bbb C}$.   Then

\Centerline{$\sum_{\sigma\in P}|\innerprod{f}{\phi_{\sigma}}
  \int_{E\cap g^{-1}[J^r_{\sigma}]}\phi_{\sigma}|
\le C_7\energy_f(P)\mass_{Eg}(P)\mu I_{\tau}$.}

\proof{ Set $\gamma=\energy_f(P)$, $\gamma'=\mass_{Eg}(P)$.   If
$P=\emptyset$ the result is trivial, so suppose that $P\ne\emptyset$.

\medskip

{\bf (a)(i)} Note that $\bigcup_{\sigma\in P}I_{\sigma}\subseteq I_{\tau}$,
$J_{\tau}\subseteq\bigcap_{\sigma\in P}J_{\sigma}$ and
$k_{\tau}\le\min_{\sigma\in P}k_{\sigma}$.
So if $\sigma$, $\sigma'\in P$ are distinct and
$\mu I_{\sigma}=\mu I_{\sigma'}$, then $J_{\sigma}=J_{\sigma'}$ and
$I_{\sigma}\cap I_{\sigma'}=\emptyset$.

\medskip

\quad{\bf (ii)}
For a dyadic interval $I$ let
$I^*$ be the half-open interval with the same
centre as $I$ and three times its length.   Let $\Cal J$ be the family
of those $I\in\Cal I$ such that $I_{\sigma}\not\subseteq I^*$ for any
$\sigma\in P$ such that $\mu I_{\sigma}\le\mu I$.   Because $P$ is
finite, all sufficiently small intervals belong to $\Cal J$, and
$\bigcup\Cal J=\Bbb R$;  let $\Cal K$ be the set of maximal members of
$\Cal J$, so that $\Cal K$ is disjoint.  Then $\bigcup\Cal K=\Bbb R$.
\Prf\ The point is that $P\ne\emptyset$;  fix $\sigma\in P$ for the
moment.   If $I\in\Cal J$, consider for each $n\in\Bbb N$ the interval
$\tilde I^{(n)}\in\Cal I$ including $I$ with length $2^n\mu I$.   Then
there is some $n\in\Bbb N$ such that
$\mu\tilde I^{(n)}\ge\mu I_{\sigma}$ and
$I_{\sigma}\subseteq(\tilde I^{(n)})^*$, so that
$\tilde I^{(k)}\notin\Cal J$ for any
$k\ge n$, and there must be some $k<n$ such that
$\tilde I^{(k)}\in\Cal K$.   Thus
$I\subseteq\tilde I^{(k)}\subseteq\bigcup\Cal K$;  as $I$ is arbitrary,
$\bigcup\Cal K=\bigcup\Cal J=\Bbb R$.\ \Qed

\medskip

\quad{\bf (iii)} For $K\in\Cal K$, let $l_K\in\Bbb Z$ be such that
$\mu K=2^{-l_K}$.   If $l_K\ge k_{\tau}$, that is,
$\mu K\le\mu I_{\tau}$, then $K$ must lie within the half-open
interval $\hat I$
with centre $x_{\tau}$ and length $7\mu I_{\tau}$, since otherwise we
should have $I_{\tau}\cap\tilde K^*=\emptyset$, where $\tilde K$ is
the dyadic interval of length $2\mu K$ including $K$, and $\tilde K$
would belong to $\Cal J$.   But this means that

\Centerline{$\sum_{K\in\Cal K,\mu K\le\mu I_{\tau}}\mu K
  \le\mu\hat I=7\mu I_{\tau}$,}

\noindent because $\Cal K$ is disjoint.

\medskip

\quad{\bf (iv)} For any $l<k_{\tau}$, there are just three members $K$
of $\Cal K$ such that $l_K=l$.   \Prf\ If $I\in\Cal I$ and
$\mu I>\mu I_{\tau}$, then either $I_{\tau}\subseteq I^*$ or
$I_{\tau}\cap I^*=\emptyset$, and $I\in\Cal J$ iff $I_{\tau}\cap I^*$ is
empty.   This means that if $K\in\Cal I$ and $\mu K=2^{-l}$,
$K\in\Cal K$ iff $I_{\tau}\cap K^*$ is empty and
$I_{\tau}\subseteq\tilde K^*$.   So if
$I_{\tau}\subseteq\coint{2^{-l}n,2^{-l}(n+1)}$ and
$K=\coint{2^{-l}m,2^{-l}(m+1)}$, we shall have $K\in\Cal K$ iff

\Centerline{{\it either} $m=n-2$
{\it or} $m=n+2$
{\it or} $m=n-3$ is even
{\it or} $m=n+3$ is odd;}

\noindent which for any given $n$ happens for just three values of
$m$.\ \Qed

\medskip

{\bf (b)} For $\sigma\in P$, let $\zeta_{\sigma}$ be a complex number of
modulus $1$ such that
$\zeta_{\sigma}\innerprod{f}{\phi_{\sigma}}
   \int_{E\cap g^{-1}[J^r_{\sigma}]}\phi_{\sigma}$ is real and
non-negaive.   Set $W=P\times\Cal K$.   For $(\sigma,K)\in W$, set

\Centerline{$\alpha_{\sigma K}
=\innerprod{f}{\phi_{\sigma}}
  \int_{E\cap g^{-1}[J^r_{\sigma}]\cap K}\phi_{\sigma}$.}

\noindent The aim of the proof is to estimate

\Centerline{$\sum_{\sigma\in P}\bigl|\innerprod{f}{\phi_{\sigma}}
  \int_{E\cap g^{-1}[J^r_{\sigma}]}\phi_{\sigma}\bigr|
=\sum_{(\sigma,K)\in W}\zeta_{\sigma}\alpha_{\sigma K}$.}

\noindent It will be helpful to note straight away that

\Centerline{$\sum_{(\sigma,K)\in W}|\alpha_{\sigma K}|
\le\sum_{\sigma\in P}|\innerprod{f}{\phi_{\sigma}}|
  \int_{-\infty}^{\infty}|\phi_{\sigma}|$}

\noindent is finite.

Set

\Centerline{$W_0
=\{(\sigma,K):\sigma\in P$, $K\in\Cal K$,
$\mu I_{\sigma}\le\mu K\le\mu I_{\tau}\}$,}

\Centerline{$W_1
=\{(\sigma,K):\sigma\in P$, $K\in\Cal K$, $\mu I_{\tau}<\mu K\}$,}

\Centerline{$W_2
=\{(\sigma,K):\sigma\in P\setminus T_{\tau}$, $K\in\Cal K$,
$\mu K<\mu I_{\sigma}\}$,}

\Centerline{$W_3
=\{(\sigma,K):\sigma\in P\cap T_{\tau}$, $K\in\Cal K$,
$\mu K<\mu I_{\sigma}\}$.}

\noindent Because $\mu I_{\sigma}\le\mu I_{\tau}$
for every $\sigma\in P$,
$W=W_0\cup W_1\cup W_2\cup W_3$.   I will give estimates for

\Centerline{$\alpha_j
=\sum_{(\sigma,K)\in W_j}\zeta_{\sigma}\alpha_{\sigma K}$}

\noindent for each $j$;  the four components in the expression for
$C_7$ given above are bounds for $|\alpha_0|, |\alpha_1|$, $|\alpha_2|$
and $|\alpha_3|$ respectively.

\wheader{286L}{6}{2}{2}{36pt}

{\bf (c)(i)} If $K\in\Cal K$ and $l_K=l$, then for any $k\ge l$

\Centerline{$\sum_{\sigma\in P,k_{\sigma}=k}|\alpha_{\sigma K}|
\le 2^{-k}C_1\gamma\gamma'(1+2^{k-l})^{-2}
\le 2^{-k-2}C_1\gamma\gamma'$.}

\noindent\Prf\ For any $\sigma\in P$,

\Centerline{$|\innerprod{f}{\phi_{\sigma}}|
=\sqrt{\mu I_{\sigma}}\energy_f(\{\sigma\})
\le\gamma\sqrt{\mu I_{\sigma}}$}

\noindent as noted in 286H, and

$$\eqalignno{\int_{E\cap g^{-1}[J^r_{\sigma}]\cap K}|\phi_{\sigma}|
&\le C_1\mu I_{\sigma}\sqrt{\mu I_{\sigma}}
  \int_{E\cap g^{-1}[J^r_{\sigma}]\cap K}w_{\sigma}^2\cr
\displaycause{286Ge}
&\le C_1\mu I_{\sigma}\sqrt{\mu I_{\sigma}}
  \int_{E\cap g^{-1}[J_{\sigma}]}w_{\sigma}
  \cdot\sup_{x\in K}w_{\sigma}(x)\cr
&\le C_1\mu I_{\sigma}\sqrt{\mu I_{\sigma}}\gamma'
  \sup_{x\in K}w_{\sigma}(x)
=C_1\sqrt{\mu I_{\sigma}}\gamma'w(\mu J_{\sigma}\rho(x_{\sigma},K)),\cr}$$

\noindent where I write $\rho(x_{\sigma},K)$ for
$\inf_{x\in K}|x-x_{\sigma}|$.   So, for $k\ge l$,

$$\eqalign{\sum_{\Atop{\sigma\in P}{k_{\sigma}=k}}
  |\alpha_{\sigma K}|
&\le\sum_{\Atop{\sigma\in P}{k_{\sigma}=k}}
  C_1\gamma\gamma'\mu I_{\sigma}w(\mu J_{\sigma}\rho(x_{\sigma},K))\cr
&=2^{-k}C_1\gamma\gamma'
  \sum_{\Atop{\sigma\in P}{k_{\sigma}=k}}
    w(2^k\rho(x_{\sigma},K))
\le 2^{-k}C_1\gamma\gamma'
  \cdot 2\sum_{n=2^{k-l}}^{\infty}w(n+\hbox{$\bover12$})\cr}$$

\noindent because the $x_{\sigma}$, for $\sigma\in P$ and
$k_{\sigma}=k$, are all distinct (see (a-i) above)
and all a distance at least $\mu K=2^{k-l}2^{-k}$
from $K$ (because $I_{\sigma}\not\subseteq K^*$);  so there are at most
two such
$\sigma$ with $\rho(x_{\sigma},K)=2^{-k}(n+\bover12)$ for each
$n\ge 2^{k-l}$.   So we have

\Centerline{$\sum_{\sigma\in P,k_{\sigma}=k}
  |\alpha_{\sigma K}|
\le 2^{-k}C_1\gamma\gamma'(1+2^{k-l})^{-2}
\le 2^{-k-2}C_1\gamma\gamma'$}

\noindent by 286Gb.\ \Qed

\medskip

\quad{\bf (ii)} Now

$$\eqalign{|\alpha_0|
&\le\sum_{(\sigma,K)\in W_0}|\alpha_{\sigma K}|
=\sum_{\Atop{K\in\Cal K}{\mu K\le\mu I_{\tau}}}
  \sum_{\Atop{\sigma\in P}{\mu I_{\sigma}\le\mu K}}
   |\alpha_{\sigma K}|\cr
&=\sum_{\Atop{K\in\Cal K}{\mu K\le\mu I_{\tau}}}\sum_{k=l_K}^{\infty}
   \sum_{\Atop{\sigma\in P}{k_{\sigma}=k}}
   |\alpha_{\sigma K}|
\le\sum_{\Atop{K\in\Cal K}{\mu K\le\mu I_{\tau}}}\sum_{k=l_K}^{\infty}
   2^{-k-2}C_1\gamma\gamma'\cr
&=C_1\gamma\gamma'\sum_{\Atop{K\in\Cal K}{\mu K\le\mu I_{\tau}}}
  2^{-l_K-1}
=\Bover12C_1\gamma\gamma'
   \sum_{\Atop{K\in\Cal K}{\mu K\le\mu I_{\tau}}}\mu K
\le\Bover{7}2C_1\gamma\gamma'\mu I_{\tau}\cr}$$

\noindent by the formula in (a-iii).   This deals with $\alpha_0$.

\medskip

{\bf (d)} Next consider $W_1$.   We have

$$\eqalignno{|\alpha_1|
&\le\sum_{(\sigma,K)\in W_1}|\alpha_{\sigma K}|
=\sum_{l=-\infty}^{k_{\tau}-1}\sum_{k=k_{\tau}}^{\infty}
   \sum_{\Atop{K\in\Cal K}{l_K=l}}\sum_{\Atop{\sigma\in P}{k_{\sigma}=k}}
   |\alpha_{\sigma K}|\cr
&\le\sum_{l=-\infty}^{k_{\tau}-1}\sum_{k=k_{\tau}}^{\infty}
   \sum_{\Atop{K\in\Cal K}{l_K=l}}
   2^{-k}C_1\gamma\gamma'(1+2^{k-l})^{-2}\cr
\displaycause{by (c-i) above}
&=3C_1\gamma\gamma'\sum_{l=-\infty}^{k_{\tau}-1}\sum_{k=k_{\tau}}^{\infty}
   2^{-k}(1+2^{k-l})^{-2}\cr
\displaycause{by (a-iv)}
&\le 3C_1\gamma\gamma'\sum_{l=-\infty}^{k_{\tau}-1}
  \sum_{k=k_{\tau}}^{\infty}2^{-3k}2^{2l}
=3C_1\gamma\gamma'2^{2(k_{\tau}-1)}2^{-3k_{\tau}}
  \sum_{l=0}^{\infty}2^{-2l}\sum_{k=0}^{\infty}2^{-3k}\cr
&=\Bover34C_1\gamma\gamma'2^{-k_{\tau}}\cdot\Bover43\cdot\Bover87
=\Bover87C_1\gamma\gamma'\mu I_{\tau}.\cr}$$

\noindent This deals with $\alpha_1$.

\medskip

{\bf (e)} For $K\in\Cal K$, set $G_K
=K\cap E\cap\bigcup_{\sigma\in P,\mu I_{\sigma}>\mu K}g^{-1}[J_{\sigma}]$.
Then $\mu G_K\le 2\gamma'\mu K/w(\bover32)$.   \Prf\
If $\mu I_{\tau}\le\mu K$, then $G_K=\emptyset$, so we may suppose that
$\mu K<\mu I_{\tau}$.   Let $\tilde K\in\Cal I$ be the dyadic interval
containing $K$ and with twice the length.   Then
$\tilde K\notin\Cal J$, so there is a $\sigma\in P$ such that
$\tilde K^*\supseteq I_{\sigma}$ and

\Centerline{$\mu I_{\sigma}\le\mu\tilde K=2\mu K\le\mu I_{\tau}$.}

\noindent Let $\upsilon\in Q$ be such
that $\tau\le\upsilon\le\sigma$ and
$\mu I_{\upsilon}=2\mu K$ (286F(a-iv)).   Then
$I_{\upsilon}$ meets $\tilde K^*$, so $\tilde K$ is either equal to
$I_{\upsilon}$ or adjacent to it, and
$|x-x_{\upsilon}|\le\bover32\cdot\mu I_{\upsilon}$ for every
$x\in\tilde K$, therefore for every $x\in K$.   Accordingly

\Centerline{$w_{\upsilon}(x)\ge w(\bover32)\mu J_{\upsilon}
=w(\bover32)/2\mu K$}

\noindent for every
$x\in K$.   On the other hand, because $\sigma\in P$ and
$\upsilon\le\sigma$,
$\int_{E\cap g^{-1}[J_{\upsilon}]}w_{\upsilon}\le\gamma'$.   So

\Centerline{$\mu(E\cap g^{-1}[J_{\upsilon}]\cap K)
\le 2\gamma'\mu K/w(\bover32)$.}

Now suppose that $\sigma'\in P$ and $\mu I_{\sigma'}>\mu K$.   Then
$k_{\sigma'}\le k_{\upsilon}$ and
$J_{\sigma'}$ is the dyadic interval of length $2^{k_{\sigma'}}$
including $J_{\tau}$.   But $J_{\upsilon}$ is the dyadic interval of
length $2^{k_{\upsilon}}$ including $J_{\tau}$, so includes
$J_{\sigma'}$, and
$g^{-1}[J_{\sigma'}]\subseteq g^{-1}[J_{\upsilon}]$.   As $\sigma'$ is
arbitrary, $G_K\subseteq E\cap g^{-1}[J_{\upsilon}]\cap K$ and
$\mu G_K\le 2\gamma'\mu K/w(\bover32)$, as claimed.\ \Qed

\medskip

{\bf (f)(i)} If $\sigma$, $\upsilon\in P\setminus T_{\tau}$ and
$k_{\sigma}\ne k_{\upsilon}$, then
$J^r_{\sigma}\cap J^r_{\upsilon}=\emptyset$.   \Prf\
It is enough to consider the case $\mu J_{\sigma}<\mu J_{\upsilon}$, so
that $\mu J_{\sigma}\le\mu J^r_{\upsilon}$.   As $J_{\sigma}$ includes
$J_{\tau}$, but $J^r_{\upsilon}$ does not, $J_{\sigma}$ is disjoint from
$J^r_{\upsilon}$ and we have the result.\ \Qed

\medskip

\quad{\bf (ii)} For $x\in\Bbb R$, set

\Centerline{$v_2(x)=\bigl|\sum_{(\sigma,K)\in W_2}
   \zeta_{\sigma}\innerprod{f}{\phi_{\sigma}}\phi_{\sigma}(x)
   \chi(E\cap g^{-1}[J^r_{\sigma}]\cap K)(x)\bigr|$.}

\noindent (The sum is finite because there is at most one $K\in\Cal K$
containing $x$.)   Then for any $x\in\Bbb R$ there is a
$k\ge k_{\tau}$ such that

\Centerline{$v_2(x)
=\bigl|\sum_{\sigma\in P,k_{\sigma}=k}
   \zeta_{\sigma}\innerprod{f}{\phi_{\sigma}}\phi_{\sigma}(x)\bigl|$.}

\noindent\Prf\ If $v_2(x)=0$, any sufficiently large $k$ will serve.
Otherwise, $x\in E$ and we have a pair
$(\upsilon,L)\in W_2$ such that
$x\in g^{-1}[J^r_{\upsilon}]\cap L$.   Try $k=k_{\upsilon}$.
$L$ is the only member of $\Cal K$ containing $x$, so

\Centerline{$v_2(x)
=\bigl|\sum_{\sigma\in P_x}
  \innerprod{f}{\phi_{\sigma}}\phi_{\sigma}(x)\bigr|$,}

\noindent where $P_x=\{\sigma:\sigma\in P\setminus T_{\tau}$,
$\mu I_{\sigma}>\mu L$, $g(x)\in J^r_{\sigma}\}$.   Now if
$\sigma\in P$ and $k_{\sigma}=k$, then
$\mu I_{\sigma}=\mu I_{\upsilon}>\mu L$, $J_{\sigma}=J_{\upsilon}$ and
$J^r_{\sigma}=J^r_{\upsilon}$ does not include $J^r_{\tau}$, so that
$\sigma\in P\setminus T_{\tau}$, $g(x)\in J^r_{\sigma}$ and
$\sigma\in P_x$.   On the other hand, (i) above tells us that
$k_{\sigma}=k$ whenever $\sigma\in P\setminus T_{\tau}$ and
$g(x)\in J^r_{\sigma}$.   So
$P_x=\{\sigma:\sigma\in P$, $k_{\sigma}=k\}$ and

\Centerline{$v_2(x)
=\bigl|\sum_{\sigma\in P,k_{\sigma}=k}
   \zeta_{\sigma}\innerprod{f}{\phi_{\sigma}}\phi_{\sigma}(x)\bigl|$.
\Qed}

\medskip

\quad{\bf (iii)} It follows that $v_2(x)\le 2C_1\gamma$ for every
$x\in\Bbb R$.   \Prf\ If $v_2(x)=0$ this is trivial.   Otherwise, take
$k$ from (ii).   Then

$$\eqalignno{v_2(x)
&\le\sum_{\Atop{\sigma\in P}{k_{\sigma}=k}}
  |\innerprod{f}{\phi_{\sigma}}\phi_{\sigma}(x)|
\le\sum_{\Atop{\sigma\in P}{k_{\sigma}=k}}\sqrt{\mu I_{\sigma}}\gamma
  \cdot\sqrt{\mu I_{\sigma}}C_1w_{\sigma}(x)\cr
\displaycause{by 286H and 286Ge}
&=C_1\gamma\sum_{\Atop{\sigma\in P}{k_{\sigma}=k}}w(2^k(x-x_{\sigma}))
\le C_1\gamma\sum_{n=-\infty}^{\infty}w(2^kx-n-\hbox{$\bover12$})\cr
\displaycause{because the $x_{\sigma}$, for $\sigma\in P$ and
$k_{\sigma}=k$, are all distinct and of the form $2^{-k}(n+\bover12)$}
&\le 2C_1\gamma\cr}$$

\noindent by 286Gd.\ \Qed

\medskip

\quad{\bf (iv)} Note also that, if $v_2(x)>0$, there is a pair
$(\sigma,K)\in W_2$ such that $x\in g^{-1}[J_{\sigma}]\cap K$, so that
$\mu K<\mu I_{\sigma}\le\mu I_{\tau}$ and $x\in G_K$.   But now we have

$$\eqalignno{|\alpha_2|
&=\bigl|\sum_{(\sigma,K)\in W_2}
  \zeta_{\sigma}\innerprod{f}{\phi_{\sigma}}\int_{-\infty}^{\infty}
  \phi_{\sigma}\times\chi(E\cap g^{-1}[J^r_{\sigma}]\cap K)\bigr|\cr
&\le\int_{-\infty}^{\infty}v_2
\le\sum_{\Atop{K\in\Cal K}{\mu K<\mu I_{\tau}}}\int_{G_K}v_2
\le\sum_{\Atop{K\in\Cal K}{\mu K<\mu I_{\tau}}}
  \bover{4C_1\gamma\gamma'\mu K}{w(\bover32)}\cr
\displaycause{putting the estimates in (e) and (iii) just above together}
&\le\bover{28\cdot C_1\gamma\gamma'\mu I_{\tau}}{w(\bover32)}\cr}$$

\noindent by (a-iii).   This deals with $\alpha_2$.

\medskip

{\bf (g)} Set $P'=P\cap T_{\tau}$ and
$\tilde f=\sum_{\sigma\in P'}
\zeta_{\sigma}\innerprod{f}{\phi_{\sigma}}\phi_{\sigma}$.   Then

\Centerline{$\|\tilde f\|^2_2
\le C_3\gamma^2\mu I_{\tau}$.}

\noindent\Prf\ If $\sigma$, $\sigma'\in P'$ and
$k_{\sigma}\ne k_{\sigma'}$, then
$\innerprod{\phi_{\sigma}}{\phi_{\sigma'}}=0$ (286Fc).   While if
$k_{\sigma}=k_{\sigma'}$, then $J_{\sigma}=J_{\sigma'}$, by (a-i).   So

$$\eqalignno{\|\tilde f\|_2^2
&=\sum_{\sigma,\sigma'\in P'}
  \zeta_{\sigma}\innerprod{f}{\phi_{\sigma}}
  \innerprod{\phi_{\sigma}}{\phi_{\sigma'}}
  \innerprod{\phi_{\sigma'}}{f}\bar\zeta_{\sigma'}\cr
&\le\sum_{\Atop{\sigma,\sigma'\in P'}{J_{\sigma}=J_{\sigma'}}}
  \bigl|\innerprod{f}{\phi_{\sigma}}
  \innerprod{\phi_{\sigma}}{\phi_{\sigma'}}
  \innerprod{\phi_{\sigma'}}{f}\bigr|
\le C_3\Delta_f(P')\cr
\displaycause{286Ib}
&\le C_3\gamma^2\mu I_{\tau}\cr}$$

\noindent by the definition of `energy', because $P'\subseteq T_{\tau}$.\
\Qed

\medskip

{\bf (h)} For $m\in\Bbb N$, set

\Centerline{$\tilde f_m=\sum_{\sigma\in P',k_{\sigma}\le m}
  \zeta_{\sigma}\innerprod{f}{\phi_{\sigma}}\phi_{\sigma}$.}

\noindent Then whenever $x$, $x'\in\Bbb R$ and $|x-x'|\le 2^{-m}$,
$|\tilde f_m(x)|\le\bover12C_1\tilde f^*(x')$, where

\Centerline{$\tilde f^*(x')
=\sup_{a\le x'\le b,a<b}\Bover1{b-a}\int_a^b|\tilde f|$}

\noindent as in 286A.   \Prf\ {\bf (i)} Since $k_{\sigma}\ge k_{\tau}$
for every $\sigma\in P'$, we may take it that $m\ge k_{\tau}$.   Let
$\hat J$ be the dyadic interval
of length $2^m$ including $J_{\tau}$, and $\hat y$ its midpoint.  Set
$\psi=S_{-\hat y}D_{2^{-m}/3}\varhat{\phi}$, that is,
$\psi(y)=\varhat{\phi}(\bover132^{-m}(y-\hat y))$ for $y\in\Bbb R$.

\medskip

\quad{\bf (ii)} If $\sigma\in P'$ and $k_{\sigma}\le m$ and
$\varhat{\phi}_{\sigma}(y)\ne 0$,
then $y\in J^l_{\sigma}$.   But $J_{\sigma}\cap\hat J\supseteq J_{\tau}$
is not empty, so $J_{\sigma}\subseteq\hat J$,
$|y-\hat y|\le\bover122^m$, $|\bover132^{-m}(y-\hat y)|\le\bover16$ and
$\psi(y)=1$.

\medskip

\quad{\bf (iii)} If $\sigma\in P'$ and $k_{\sigma}>m$ and
$\varhat{\phi}_{\sigma}(y)\ne 0$, then
$J^r_{\sigma}\cap\hat J\supseteq J^r_{\tau}$ is non-empty, so
$\hat J\subseteq J^r_{\sigma}$ and $y\le y_{\sigma}\le\hat y$;  now

\Centerline{$\hat y-y
=(\hat y-y_{\sigma})+(y_{\sigma}-y)
\ge\Bover12\cdot 2^m+\Bover1{20}\mu J_{\sigma}
\ge\Bover35\cdot 2^m$,
\quad$|\Bover13\cdot 2^{-m}(y-\hat y)|\ge\Bover15$}

\noindent and $\psi(y)=0$.
\medskip

\quad{\bf (iv)}  What this means is that if $\sigma\in P'$ then

$$\eqalign{\varhat{\phi}_{\sigma}\times\psi
&=\varhat{\phi}_{\sigma}\text{ if }k_{\sigma}\le m,\cr
&=0\text{ if }k_{\sigma}>m,\cr}$$

\noindent so that $\varhat{\vartildef}_m=\psi\times\varhat{\vartildef}$.

\medskip

\quad{\bf (v)} By 283M,
$\tilde f_m=\Bover1{\sqrt{2\pi}}\tilde f*\varcheck{\psi}$,
where $\tilde f*\varcheck{\psi}$ is the convolution of $\tilde f$ and
the inverse Fourier transform $\varcheck{\psi}$ of $\psi$.   (Strictly
speaking, 283M, with the help of 284C, tells us that $\tilde f_m$ and
$\Bover1{\sqrt{2\pi}}\tilde f*\varcheck{\psi}$ have the same Fourier
transforms.   By 283G, they are equal almost everywhere;  by 255K,
the convolution is defined everywhere and is continuous;  so in fact
they are the same function.)   Now

\Centerline{$\varcheck{\psi}
=3\cdot 2^mM_{\hat y}D_{3\cdot 2^m}
\varhat{\phi}\varcheck{\phantom{\phi}}
=3\cdot 2^mM_{\hat y}D_{3\cdot 2^m}\phi$,}

\noindent that is,

\Centerline{$\varcheck{\psi}(x)
=3\cdot 2^me^{ix\hat y}\phi(3\cdot 2^mx)$}

\noindent for $x\in\Bbb R$.

\medskip

\quad{\bf (vi)} Set $w_1(x)=\min(w(3),w(x))$ for $x\in\Bbb R$, so
that $w_1$ is
non-decreasing on $\ocint{-\infty,-3}$, non-increasing on
$\coint{3,\infty}$, and constant on $[-3,3]$, and
$|\phi(x)|\le C_1w_1(x)$ for every $x$, by the choice of $C_1$
(286Ge).   Take $x$, $x'\in\Bbb R$ such that $|x-x'|\le 2^{-m}$.
Then

$$\eqalignno{|\tilde f_m(x)|
&\le\Bover1{\sqrt{2\pi}}\int_{-\infty}^{\infty}
  |\tilde f(x-t)||\varcheck{\psi}(t)|dt
=\Bover{3\cdot 2^m}{\sqrt{2\pi}}\int_{-\infty}^{\infty}
  |\tilde f(x-t)||\phi(3\cdot 2^mt)|dt\cr
&\le\Bover{3\cdot 2^m}{\sqrt{2\pi}}C_1\int_{-\infty}^{\infty}
  |\tilde f(x-t)|w_1(3\cdot 2^mt)dt
=\Bover{3\cdot 2^m}{\sqrt{2\pi}}C_1\int_{-\infty}^{\infty}
  |\tilde f(x+t)|w_1(3\cdot 2^mt)dt\cr
\displaycause{because $w_1$ is an even function}
&\le\Bover{3\cdot 2^m}{\sqrt{2\pi}}C_1
  \int_{-\infty}^{\infty}w_1(3\cdot 2^mt)dt
  \cdot\sup_{\Atop{a\le-2^{-m}}{b\ge 2^{-m}}}
  \Bover1{b-a}\int_a^b|\tilde f(x+t)|dt\cr
\displaycause{by 286B, because $t\mapsto w_1(3\cdot 2^mt)$ is
non-decreasing on $\ocint{-\infty,-2^{-m}}$, non-increasing on
$\coint{2^{-m},\infty}$ and constant on $[-2^{-m},2^{-m}]$}
&=\Bover1{\sqrt{2\pi}}C_1\int_{-\infty}^{\infty}w_1
  \cdot\sup_{\Atop{a\le x-2^{-m}}{b\ge x+2^{-m}}}
  \Bover1{b-a}\int_a^b|\tilde f|
\le\Bover12C_1\int_{-\infty}^{\infty}w
  \cdot\tilde f^*(x')\cr
\displaycause{because if $a\le x-2^{-m}$ and $b\ge x+2^{-m}$ then
$a\le x'\le b$}
&=\Bover12C_1\tilde f^*(x')\cr}$$

\noindent (286Ga), as required.\ \Qed

\medskip

{\bf (i)} For $x\in\Bbb R$, set

\Centerline{$v_3(x)=\bigl|\sum_{(\sigma,K)\in W_3}
   \zeta_{\sigma}\innerprod{f}{\phi_{\sigma}}\phi_{\sigma}(x)
   \chi(E\cap g^{-1}[J^r_{\sigma}]\cap K)(x)\bigr|$.}

\noindent Then whenever $L\in\Cal K$ and $x$, $x'\in L$,
$|v_3(x)|\le C_1\tilde f^*(x')$.   \Prf\ We may suppose that
$v_3(x)\ne 0$, so that, in particular, $x\in E$.   The only pairs
$(\sigma,K)$ contributing to the sum forming $v_3(x)$ are those in which
$x\in K$, so that $K=L$, and $g(x)\in J^r_{\sigma}$.   Moreover, since
we are looking only at $\sigma\in T_{\tau}$, so that
$J^r_{\tau}\subseteq J^r_{\sigma}$, $J^r_{\sigma}$ will always be
the dyadic interval of length $2^{k_{\sigma}-1}$ including $J^r_{\tau}$.
So these intervals are nested, and there will be some $m$ such that (for
$\sigma\in T_{\tau}$) $g(x)\in J^r_{\sigma}$ iff $k_{\sigma}\ge m$.
Accordingly

\Centerline{$v_3(x)
=\bigl|\sum_{\sigma\in P',m\le k_{\sigma}<l_L}
   \zeta_{\sigma}\innerprod{f}{\phi_{\sigma}}\phi_{\sigma}(x)\bigr|
=|\tilde f_{l_L-1}(x)-\tilde f_{m-1}(x)|$}

\noindent (we must have $m<l_L$ because $v_3(x)\ne 0$).   Now
$|x-x'|\le 2^{-l_L}\le 2^{-m}$, so (h) tells us that
both $|\tilde f_{l_L-1}(x)|$ and $|\tilde f_{m-1}(x)|$ are at most
$\bover12C_1\tilde f^*(x')$, and $v_3(x)\le C_1\tilde f^*(x')$, as
claimed.\ \Qed

It follows that $v_3(x)\le\Bover{C_1}{\mu L}\biggerint_L\tilde f^*$
for every $x\in L$.

\wheader{286L}{6}{2}{2}{108pt}

{\bf (j)} Now we are in a position to estimate

$$\eqalignno{|\alpha_3|
&=|\sum_{(\sigma,K)\in W_3}\zeta_{\sigma}\alpha_{\sigma K}|
\le\int_{-\infty}^{\infty}v_3
\le\sum_{\Atop{K\in\Cal K}{\mu K<\mu I_{\tau}}}\int_{G_K}v_3\cr
\displaycause{because if $v_3(x)\ne 0$ there are $(\sigma,K)\in W_3$
such that $x\in K$, and in this case $x\in G_K$ and
$\mu K<\mu I_{\sigma}\le\mu I_{\tau}$}
&\le\sum_{\Atop{K\in\Cal K}{\mu K<\mu I_{\tau}}}
  \mu G_K\cdot\Bover{C_1}{\mu K}\int_K\tilde f^*\cr
\displaycause{by (i) above, because $G_K\subseteq K$}
&\le C_1\sum_{\Atop{K\in\Cal K}{\mu K<\mu I_{\tau}}}
  \bover{2\gamma'}{w(\bover32)}\int_K\tilde f^*\cr
\displaycause{by (e)}
&\le\bover{2C_1\gamma'}{w(\bover32)}\int_{\hat I}\tilde f^*\cr
\displaycause{because if $\mu K<\mu I_{\tau}$ then $K\subseteq\hat I$, as
noted in (a-iii)}
&\le\bover{2C_1\gamma'}{w(\bover32)}\sqrt{\mu\hat I}
  \cdot\|\tilde f^*\|_2\cr
\displaycause{by Cauchy's inequality}
&\le\bover{2C_1\gamma'}{w(\bover32)}\sqrt{7\mu I_{\tau}}
  \cdot\sqrt{8}\|\tilde f\|_2\cr
\displaycause{by the Maximal Theorem, 286A}
&\le\bover{4C_1\gamma'\sqrt{14}}{w(\bover32)}\sqrt{\mu I_{\tau}}
  \cdot\gamma\sqrt{C_3\mu I_{\tau}}\cr
\displaycause{by (g)}
&=\bover{4C_1\sqrt{14 C_3}}{w(\bover32)}
  \gamma\gamma'\mu I_{\tau}.\cr}$$

\medskip

{\bf (k)} Assembling these,

$$\eqalign{\sum_{\sigma\in P}\bigl|\innerprod{f}{\phi_{\sigma}}
  \int_{E\cap g^{-1}[J^r_{\sigma}]}\phi_{\sigma}\bigr|
&=\sum_{\Atop{\sigma\in P}{K\in\Cal K}}\zeta_{\sigma}\alpha_{\sigma K}
=\sum_{j=0}^3\sum_{(\sigma,K)\in W_j}\zeta_{\sigma}\alpha_{\sigma K}
\le\sum_{j=0}^3|\alpha_j|\cr
&\le\Bover72\cdot C_1\gamma\gamma'\mu I_{\tau}
  +\Bover87\cdot C_1\gamma\gamma'\mu I_{\tau}
  +28\cdot C_1\gamma\gamma'\mu I_{\tau}/w(\hbox{$\bover32$})\cr
&\qquad\qquad\qquad\qquad
  +4\sqrt{14C_3}\cdot
      C_1\gamma\gamma'\mu I_{\tau}/w(\hbox{$\bover32$})\cr
&=C_7\gamma\gamma'\mu I_{\tau},\cr}$$

\noindent as claimed.
}%end of proof of 286L

\leader{286M}{The Lacey-Thiele lemma} Set $C_8=3C_7(C_5+C_6)$.   Then

\Centerline{$\sum_{\sigma\in Q}|\innerprod{f}{\phi_{\sigma}}
  \int_{E\cap g^{-1}[J^r_{\sigma}]}\phi_{\sigma}|
\le C_8$}

\noindent whenever $f\in\eusm L^2_{\Bbb C}$, $\|f\|_2=1$,
$\mu E\le 1$ and $g:\Bbb R\to\Bbb R$ is measurable.

\proof{%
{\bf (a)} The first step is to combine 286J and 286K, as follows:
if $P\subseteq Q$ is finite and
$\max(\sqrt{\mass_{Eg}(P)},\energy_f(P))\penalty-100\le\gamma$, there is an
$R\subseteq Q$ such that
$\gamma^2\sum_{\tau\in R}\mu I_{\tau}\le C_5+C_6$ and
$\max(\sqrt{\mass_{Eg}(P\setminus R^+)},
\energy_f(P\setminus R^+))
\le\bover12\gamma$.
\Prf\ Since $\mass_{Eg}(P)\le\gamma^2$, 286J tells us that there is an
$R_0\subseteq Q$ such that
$\gamma^2\sum_{\tau\in R_0}\mu I_{\tau}\le C_5$ and
$\mass_{Eg}(P\setminus R_0^+)\le\bover14\gamma^2$.
Turn to 286K:  since
$\energy_f(P\setminus R_0^+)\le\energy_f(P)\le\gamma$, we can find
$R_1\subseteq Q$ such that
$\gamma^2\sum_{\tau\in R_1}\mu I_{\tau}\le C_6$ and
$\energy_f((P\setminus R_0^+)\setminus R_1^+)\le\bover12\gamma$.
Set $R=R_0\cup R_1$.   Then

\Centerline{$\gamma^2\sum_{\tau\in R}\mu I_{\tau}\le C_5+C_6$, 
\quad$\mass_{Eg}(P\setminus R^+)\le\mass_{Eg}(P\setminus R_0^+)
\le\bover14\gamma^2$}

\noindent so
$\max(\sqrt{\mass_{Eg}(P\setminus R^+)},\energy_f(P\setminus R^+))
\le\bover12\gamma$.\ \Qed
\fontdimen3\tenrm=1.67pt

\medskip

{\bf (b)} Now take any finite $P\subseteq Q$.
Let $k\in\Bbb N$ be such that
$\max(\sqrt{\mass_{Eg}(P)},\energy_f(P))\le 2^k$.   By (a), we can
choose $\sequencen{P_n}$, $\sequencen{R_n}$ inductively such that
$P_0=P$ and, for each $n\in\Bbb N$,

\Centerline{$P_{n+1}=P_n\setminus R_n^+$,}

\Centerline{$2^{2k-2n}\sum_{\tau\in R_n}\mu I_{\tau}\le C_5+C_6$,
\quad$\max(\sqrt{\mass_{Eg}(P_n)},\energy_f(P_n))\le 2^{k-n}$.}

\noindent Since
$\energy_f(\{\sigma\})=\sqrt{\mu J_{\sigma}}|\innerprod{f}{\phi_{\sigma}}|>0$
whenever $\innerprod{f}{\phi_{\sigma}}\ne 0$ (286H),
$\innerprod{f}{\phi_{\sigma}}=0$ whenever
$\sigma\in\bigcap_{n\in\Bbb N}P_n$, and

$$\eqalignno{\sum_{\sigma\in P}\bigl|\innerprod{f}{\phi_{\sigma}}
  \int_{E\cap g^{-1}[J^r_{\sigma}]}\phi_{\sigma}\bigr|
&=\sum_{\sigma\in\bigcup_{n\in\Bbb N}P_n\setminus P_{n+1}}
  \bigl|\innerprod{f}{\phi_{\sigma}}
  \int_{E\cap g^{-1}[J^r_{\sigma}]}\phi_{\sigma}\bigr|\cr
&=\sum_{n=0}^{\infty}\sum_{\sigma\in P_n\setminus P_{n+1}}
  \bigl|\innerprod{f}{\phi_{\sigma}}
  \int_{E\cap g^{-1}[J^r_{\sigma}]}\phi_{\sigma}\bigr|\cr
&\le\sum_{n=0}^{\infty}\sum_{\tau\in R_n}
  \sum_{\Atop{\sigma\in P_n}{\sigma\ge\tau}}
     \bigl|\innerprod{f}{\phi_{\sigma}}
       \int_{E\cap g^{-1}[J^r_{\sigma}]}\phi_{\sigma}\bigr|\cr
&\le\sum_{n=0}^{\infty}\sum_{\tau\in R_n}C_7
  \energy_f(P_n)\mass_{Eg}(P_n)\mu I_{\tau}\cr
\displaycause{by 286L}
&\le C_7\sum_{n=0}^{\infty}2^{k-n}\min(1,2^{2k-2n})
  \sum_{\tau\in R_n}\mu I_{\tau}\cr
\displaycause{because $\mass_{Eg}(P_n)\le 1$ for every $n$, as noted in
286H}
&\le C_7\sum_{n=0}^{\infty}2^{k-n}\min(1,2^{2k-2n})2^{2n-2k}(C_5+C_6)
  \cr
&=C_7(C_5+C_6)\sum_{n=0}^{\infty}\min(2^{n-k},2^{k-n})\cr
&\le C_7(C_5+C_6)\sum_{n=-\infty}^{\infty}\min(2^n,2^{-n})
=3C_7(C_5+C_6).\cr}$$

\medskip

{\bf (c)} Since this true for every finite $P\subseteq Q$,

\Centerline{$\sum_{\sigma\in Q}|\innerprod{f}{\phi_{\sigma}}
  \int_{E\cap g^{-1}[J^r_{\sigma}]}\phi_{\sigma}|
\le 3C_7(C_5+C_6)=C_8$,}

\noindent as claimed.
}%end of proof of 286M

\leader{286N}{Lemma} Set $C_9=C_8\sqrt{2}$.   Suppose that
$f\in\eusm L^2_{\Bbb C}$, $g:\Bbb R\to\Bbb R$ is
measurable and $\mu F<\infty$.   Then

\Centerline{$\sum_{\sigma\in Q}|\innerprod{f}{\phi_{\sigma}}
  \int_{F\cap g^{-1}[J^r_{\sigma}]}\phi_{\sigma}|
\le C_9\|f\|_2\sqrt{\mu F}$.}

\proof{ This is trivial if $\|f\|_2=0$, that is, $f=0$ a.e.   So we may
take it that $\|f\|_2>0$.   Dividing both sides by $\|f\|_2$, we may
suppose that $\|f\|_2=1$.

Let $k\in\Bbb Z$ be such that $2^{k-1}<\mu F\le 2^k$.   We have a
permutation $\sigma\mapsto\sigma^*:Q\to Q$ defined by saying that
$\sigma^*=(2^{-k}I_{\sigma},2^kJ_{\sigma})$;  so that
$k_{\sigma^*}=k_{\sigma}+k$, $x_{\sigma^*}=2^{-k}x_{\sigma}$,
$y^l_{\sigma^*}=2^ky^l_{\sigma}$, $J^r_{\sigma^*}=2^kJ^r_{\sigma}$, and
for every $x\in\Bbb R$

$$\eqalign{\phi_{\sigma}(2^kx)
&=\sqrt{\mu J_{\sigma}}e^{2^kiy^l_{\sigma}x}
  \phi((2^kx-x_{\sigma})\mu J_{\sigma})\cr
&=\sqrt{\mu J_{\sigma}}e^{iy^l_{\sigma^*}x}
  \phi(2^{k_{\sigma}+k}(x-2^{-k}x_{\sigma}))\cr
&=2^{-k/2}\sqrt{\mu J_{\sigma^*}}e^{iy^l_{\sigma^*}x}
  \phi((x-x_{\sigma^*})\mu J_{\sigma^*})
=2^{-k/2}\phi_{\sigma^*}(x).\cr}$$

\noindent Write $\tilde F=2^{-k}F$, so that $\mu\tilde F\le 1$, and
$\tilde g(x)=2^kg(2^kx)$ for every $x$.   Then, for $\sigma\in Q$,

$$\eqalign{F\cap g^{-1}[J^r_{\sigma}]
&=\{x:x\in F,\,g(x)\in J^r_{\sigma}\}
=\{x:2^{-k}x\in\tilde F,\,2^{-k}\tilde g(2^{-k}x)\in J^r_{\sigma}\}\cr
&=\{x:2^{-k}x\in\tilde F,\,\tilde g(2^{-k}x)\in J^r_{\sigma^*}\}
=2^k\{x:x\in\tilde F,\,\tilde g(x)\in J^r_{\sigma^*}\}.\cr}$$

\noindent Write $\tilde f(x)=2^{k/2}f(2^kx)$, so that

\Centerline{$\|\tilde f\|_2=2^{k/2}\|D_{2^k}f\|_2=\|f\|_2=1$,}

\noindent while

\Centerline{$\innerprod{f}{\phi_{\sigma}}
=\int_{-\infty}^{\infty}f\times\bar\phi_{\sigma}
=2^k\int_{-\infty}^{\infty}
f(2^kx)\overline{\phi_{\sigma}(2^kx)}dx
=\innerprod{\tilde f}{\phi_{\sigma^*}}$}

\noindent for every $\sigma\in Q$.   Putting all these together,

$$\eqalignno{\sum_{\sigma\in Q}\bigl|\innerprod{f}{\phi_{\sigma}}
  \int_{F\cap g^{-1}[J^r_{\sigma}]}\phi_{\sigma}\bigr|
&=2^k\sum_{\sigma\in Q}\bigl|\innerprod{f}{\phi_{\sigma}}
  \int_{2^{-k}(F\cap g^{-1}[J^r_{\sigma}])}
     \phi_{\sigma}(2^kx)dx\bigr|\cr
&=2^{k/2}\sum_{\sigma\in Q}\bigl|\innerprod{\tilde f}{\phi_{\sigma^*}}
  \int_{\tilde F\cap\tilde g^{-1}[J^r_{\sigma^*}]}
    \phi_{\sigma^*}\bigr|\cr
&=2^{k/2}\sum_{\tau\in Q}\bigl|\innerprod{\tilde f}{\phi_{\tau}}
  \int_{\tilde F\cap\tilde g^{-1}[J^r_{\tau}]}\phi_{\tau}\bigr|\cr
&\le 2^{k/2}C_8\cr
\displaycause{by the Lacey-Thiele lemma, applied to $\tilde g$,
$\tilde F$ and $\tilde f$}
&\le C_9\sqrt{\mu F}
=C_9\|f\|_2\sqrt{\mu F}.\cr}$$
}%end of proof of 286N

\leader{286O}{Lemma}\dvArevised{2013} (a) For $z\in\Bbb R$, define
$\theta_z:\Bbb R\to[0,1]$ by setting

\Centerline{$\theta_z(y)=\varhat{\phi}(2^{-k}(y-\hat y))^2$}

\noindent whenever there is a dyadic interval $J\in\Cal I$ of length
$2^k$ such that $z$ belongs to the right-hand half of $J$ and $y$
belongs to the left-hand half of $J$ and $\hat y$ is the lower quartile
of $J$, and zero if there is no such $J$.   
Then $(y,z)\mapsto\theta_z(y)$ is Borel measurable,
$0\le\theta_z(y)\le 1$
for all $y$, $z\in\Bbb R$, and $\theta_z(y)=0$ if $y\ge z$.

(b) For $k\in\Bbb Z$, set $Q_k=\{\sigma:\sigma\in Q$, $k_{\sigma}=k\}$.
Let $[Q]^{<\omega}$ be the set of finite subsets of
$Q$, $[\Bbb Z]^{<\omega}$ the set of finite subsets of $\Bbb Z$
and $\Cal L$ the set of subsets $L$ of $Q$ such that $L\cap Q_k$ is
finite for every $k$.
If $K\in[\Bbb Z]^{<\omega}$ and $L\in\Cal L$, set

$$\eqalign{\Cal P_{KL}
&=\{P:P\in[Q]^{<\omega},\,P\cap Q_k\supseteq L\cap Q_k
  \text{ whenever }k\in\Bbb Z
\cr&\mskip200mu  
   \text{and either }k\in K\text{ or }P\cap Q_k\ne\emptyset\};\cr}$$
   
\noindent set

\Centerline{$\Cal F=\{\Cal P:\Cal P\subseteq[Q]^{<\omega}$ and there are 
  $K\in[\Bbb Z]^{<\omega}$, $L\in\Cal L$ such that 
  $\Cal P\supseteq\Cal P_{KL}\}$.}
  
\noindent Then $\Cal F$ is a filter on $[Q]^{<\omega}$ and

\Centerline{$2\pi\int_F(\varhat h\times\theta_z)\varspcheck
=\lim_{P\to\Cal F}\sum_{\sigma\in P,z\in J^r_{\sigma}}
   \innerprod{h}{\phi_{\sigma}}\int_F\phi_{\sigma}$}
   
\noindent for every $z\in\Bbb R$ and rapidly decreasing test function $h$ and measurable set
$F\subseteq\Bbb R$ of finite measure.

\proof{{\bf (a)(i)} I had better start by explaining why the recipe above
defines a function $\theta_z$.   Let $M$ be the set of those
$k\in\Bbb Z$ such that $z$ belongs to the right-hand half of the dyadic
interval $\hat J_k$ of length $2^k$ containing $z$.
For $k\in M$, let $\hat y_k$ be the midpoint of the left-hand half
$\hat J^l_k$ of $\hat J_k$, and set
$\psi_k(y)=\varhat{\phi}(2^{-k}(y-\hat y_k))^2$ for $y\in\Bbb R$;  then
$\psi_k$ is smooth and zero outside $\hat J^l_k$.   But now observe that
if $k$, $k'$ are distinct members of $M$, then $\hat J^l_k$ and
$\hat J^l_{k'}$ are disjoint, as remarked in
286E(b-iv).   So $\theta_z$ is just the sum $\sum_{k\in M}\psi_k$.
Because $\varhat{\phi}$ takes values in $[0,1]$, so does $\theta_z$.
If $y\ge z$, then of course $y\notin\hat J^l_k$ for any $k\in M$, so
$\theta_z(y)=0$.

\medskip

\quad{\bf (ii)} To see that $(y,z)\mapsto\theta_z(y)$ is Borel
measurable, observe that

\Centerline{$\{(y,z):\theta_z(y)\ge\gamma\}
=\bigcup_{\sigma\in Q}\{(y,z):
  \hat\phi((y-y^l_{\sigma})\mu I_{\sigma})^2\ge\gamma$,
  $z\in J^r_{\sigma}\}$}

\noindent for every $\gamma\in\Bbb R$.

\medskip

{\bf (b)(i)} $\emptyset$ belongs to both $[\Bbb Z]^{<\omega}$ and $\Cal L$ and
$[Q]^{<\omega}=\Cal P_{\emptyset\emptyset}$ belongs to $\Cal F$.
If $K\in[\Bbb Z]^{<\omega}$ and $L\in\Cal L$ then
$\bigcup_{k\in K}L\cap Q_k$ belongs to $\Cal P_{KL}$.   So no
$\Cal P_{KL}$ is empty and $\emptyset\notin\Cal F$.

If $\Cal P$, $\Cal P'\in\Cal F$, there are
$K$, $K'\in[\Bbb Z]^{<\omega}$ and $L$, $L'\in\Cal L$ 
such that $\Cal P_{KL}\subseteq\Cal P$ and $\Cal P_{K'L'}\subseteq\Cal P'$.
Now $K\cup K'\in[\Bbb Z]^{<\omega}$, $L\cup L'\in\Cal L$ and

\Centerline{$\Cal P_{K\cup K',L\cup L'}
\subseteq\Cal P_{KL}\cap\Cal P_{K'L'}\subseteq\Cal P\cap\Cal P'$,}

\noindent so $\Cal P\cap\Cal P'\in\Cal F$.

If $\Cal P\in\Cal F$ and $\Cal P\subseteq\Cal P'\subseteq[Q]^{<\omega}$,
then of course $\Cal P'\in\Cal F$.   So $\Cal F$ is a filter on
$[Q]^{<\omega}$.

\medskip

\quad{\bf (ii)} Now fix on $z\in\Bbb R$, a rapidly decreasing test function $h$ and a set $F$ of
finite measure.   Take $M$ and
$\psi_k$, $\hat J_k$, $\hat y_k$ for $k\in M$ from (a-i) above;
it will be convenient to set $\psi_k=0$ for $k\in\Bbb Z\setminus M$, so
that $\theta_z=\sum_{k\in\Bbb Z}\psi_k$.

For $k\in\Bbb Z$,

\Centerline{$2\pi\int_F(\varhat h\times\psi_k)\varspcheck
=\sum_{\sigma\in Q_k,z\in J^r_{\sigma}}
  \innerprod{h}{\phi_{\sigma}}\int_F\phi_{\sigma}$.}
  
\noindent\Prf\ If $k\notin M$, then $z\notin J^r_{\sigma}$ for any
$\sigma\in Q_k$, while $\psi_k=0$, so the result is trivial.   So I will
suppose that $k\in M$ and that $\hat y_k$ is defined.
If $\sigma\in Q_k$ and $z\in J^r_{\sigma}$, 
$y^l_{\sigma}=\hat y_k$ and $x_{\sigma}$ is
of the form $2^{-k}(n+\bover12)$ for some $n\in\Bbb Z$.   So

$$\eqalignno{\innerprod{h}{\phi_{\sigma}}
&=\int_{-\infty}^{\infty}
  \varhat h\times\bar\varhat{\phi}_{\sigma}\cr
\displaycause{284O}
&=\int_{-\infty}^{\infty}\varhat h(t)
  \cdot 2^{-k/2}e^{2^{-k}i(n+\bover12)(t-\hat y_k)}
  \varhat\phi(2^{-k}(t-\hat y_k))dt\cr
\displaycause{by the formula in 286Eb, because $\varhat{\phi}$ is
real-valued}
&=2^{k/2}\int_{-\infty}^{\infty}\varhat h(2^kt+\hat y_k)
  e^{i(n+\bover12)t}\varhat\phi(t)dt
=2^{k/2}\int_{-\pi}^{\pi}\varhat h(2^kt+\hat y_k)
  e^{i(n+\bover12)t}\varhat\phi(t)dt\cr
\displaycause{because $\varhat{\phi}(t)=0$ if $|t|\ge\bover15$}
&=2^{k/2}\int_{-\pi}^{\pi}g(t)e^{int}dt,\cr}$$

\noindent where $g(t)=\varhat h(2^kt+\hat y_k)e^{it/2}\varhat{\phi}(t)$
for $-\pi<t\le\pi$.   So if we set
$c_n=\Bover1{2\pi}\int_{-\pi}^{\pi}g(t)e^{-int}dt$, as in 282A, we have

\Centerline{$\innerprod{h}{\phi_{\sigma}}=2^{k/2}\cdot 2\pi c_{-n}$}

\noindent when $\sigma\in Q_k$, $z\in J^r_{\sigma}$
and $x_{\sigma}=2^{-k}(n+\bover12)$.
Note that as $g$ is smooth and zero outside
$[-\bover15,\bover15]$, $\sum_{n=-\infty}^{\infty}|c_n|<\infty$
(282Rb).

Now, for any $y\in\hat J^l_k$, writing
$R_k$ for 

\Centerline{$\{\sigma:\sigma\in Q_k$, $z\in J^r_{\sigma}\}
=\{\sigma:\sigma\in Q_k$, $J_{\sigma}=\hat J_k\}
=\{(I,\hat J_k):I\in\Cal I$, $\mu I=2^{-k}\}$,}

\noindent we have

$$\eqalignno{\sum_{\sigma\in R_k}
  \innerprod{h}{\phi_{\sigma}}\varhat{\phi}_{\sigma}(y)
&=\sum_{n=-\infty}^{\infty}2^{k/2}\cdot 2\pi c_{-n}
   \cdot 2^{-k/2}e^{-2^{-k}i(n+\bover12)(y-\hat y_k)}
   \varhat{\phi}(2^{-k}(y-\hat y_k))\cr
&=2\pi\varhat{\phi}(2^{-k}(y-\hat y_k))
   e^{-2^{-k-1}i(y-\hat y_k)}
   \sum_{n=-\infty}^{\infty}c_{-n}e^{-2^{-k}in(y-\hat y_k)}\cr
&=2\pi\varhat{\phi}(2^{-k}(y-\hat y_k))
   e^{-2^{-k-1}i(y-\hat y_k)}
   \sum_{n=-\infty}^{\infty}c_ne^{in 2^{-k}(y-\hat y_k)}\cr
&=2\pi\varhat{\phi}(2^{-k}(y-\hat y_k))
   e^{-2^{-k-1}i(y-\hat y_k)}g(2^{-k}(y-\hat y_k))\cr
\displaycause{by 282L(i), because $2^{-k}|y-\hat y_k|\le\bover14<\pi$
and $g$ is smooth}
&=2\pi e^{-2^{-k-1}i(y-\hat y_k)}\varhat{\phi}(2^{-k}(y-\hat y_k))
  \varhat h(y)e^{2^{-k-1}i(y-\hat y_k)}
  \varhat{\phi}(2^{-k}(y-\hat y_k))\cr
&=2\pi\varhat h(y)\psi_k(y).\cr}$$

\noindent On the other hand, if $y\in\Bbb R\setminus\hat J^l_k$,
$\psi_k(y)=\varhat{\phi}_{\sigma}(y)=0$ for every $\sigma\in R_k$, so
again
$\sum_{\sigma\in R_k}
  \innerprod{h}{\phi_{\sigma}}\varhat{\phi}_{\sigma}(y)
=2\pi\varhat h(y)\psi_k(y)$.

Next,

\Centerline{$\sum_{\sigma\in R_k}|\innerprod{h}{\phi_{\sigma}}|
=2\pi\cdot 2^{k/2}\sum_{n=-\infty}^{\infty}|c_n|$}

\noindent and

\Centerline{$\sup_{\sigma\in R_k}\int_{-\infty}^{\infty}
  |\varhat{\phi}_{\sigma}|
  =2^{k/2}\int_{-\infty}^{\infty}|\varhat{\phi}|$}

\noindent are finite, while of course $\varhat{\chi F}$ is bounded.
So

$$\eqalignno{2\pi\int_F(\varhat h\times\psi_k)\varspcheck
&=2\pi\innerprod{\varhat h\times\psi_k)\varspcheck}{\chi F}
=2\pi\innerprod{(\varhat h\times\psi_k)\varspcheck)\varsphat}
  {\varhat{\chi F}}\cr
\displaycause{284Ob again}
&=2\pi\innerprod{\varhat h\times\psi_k}{\varhat{\chi F}}
=\int_{-\infty}^{\infty}
    2\pi\varhat h\times\psi_k\times\overline{\varhat{\chi F}}
\cr&=\int_{-\infty}^{\infty}\sum_{\sigma\in R_k}
   \innerprod{h}{\phi_{\sigma}}\varhat{\phi}_{\sigma}
   \times\overline{\varhat{\chi F}}
=\sum_{\sigma\in R_k}\innerprod{h}{\phi_{\sigma}}\int_{-\infty}^{\infty}
   \varhat{\phi}_{\sigma}\times\overline{\varhat{\chi F}}\cr
\displaycause{226E}   
&=\sum_{\sigma\in R_k}\innerprod{h}{\phi_{\sigma}}
   \int_F{\phi}_{\sigma}
=\sum_{\sigma\in Q_k,z\in J^r_{\sigma}}\innerprod{h}{\phi_{\sigma}}
   \int_F{\phi}_{\sigma}.  \text{ \Qed}\cr}$$
   
\medskip

\quad{\bf (iii)} In the last sentence of the argument just above, 
I quoted B.Levi's
theorem in the form 226E, even though $R_k$ has a natural
enumeration, because I shall specifically want to say later that

\inset{\noindent
for every $\epsilon>0$ there is a finite $L_0\subseteq R_k$ such that

\centerline{$|2\pi\int_F(\varhat h\times\psi_k)\varspcheck
-\sum_{\sigma\in L}\innerprod{h}{\phi_{\sigma}}
   \int_F{\phi}_{\sigma}|\le\epsilon$}
   
\noindent whenever $L\subseteq R_k$ is finite and $L\supseteq L_0$;}

\noindent it follows at once that
   
\inset{\noindent
for every $\epsilon>0$ there is a finite $L_0\subseteq Q_k$ such that

\centerline{$|2\pi\int_F(\varhat h\times\psi_k)\varspcheck
-\sum_{\sigma\in L,z\in J^r_{\sigma}}\innerprod{h}{\phi_{\sigma}}
   \int_F{\phi}_{\sigma}|\le\epsilon$}
   
\noindent whenever $L\subseteq Q_k$ is finite and $L\supseteq L_0$.}

\medskip

\quad{\bf (iv)} Now let us consider $(\varhat h\times\theta_z)\varspcheck$.
Because every $\psi_k$ is non-negative, $\theta_z=\sum_{k\in\Bbb Z}\psi_k$
is bounded above by $1$, and $\varhat h$ is integrable,

$$\eqalign{\int_F(\varhat h\times\theta_z)\varspcheck
&=\int_{-\infty}^{\infty}\varhat h\times\theta_z\times\varhat{\chi F}
\cr&=\sum_{k\in\Bbb Z}\int_{-\infty}^{\infty}\varhat h\times\psi_k\times\varhat{\chi F}
=\sum_{k\in\Bbb Z}\int_F(\varhat h\times\psi_k)\varspcheck.\cr}$$

\noindent So here we can say

\inset{\noindent for every $\epsilon>0$ there is a 
$K_0\in[\Bbb Z]^{<\omega}$ such that 

\centerline{$|\int_F(\varhat h\times\theta_z)\varspcheck
  -\sum_{k\in K}\int_F(\varhat h\times\psi_k)\varspcheck|\le\epsilon$}

\noindent whenever $K\in[\Bbb Z]^{<\omega}$ and $K\supseteq K_0$.}  

\medskip

\quad{\bf (v)} To express the facts above in terms of a limit along the
filter $\Cal F$, we can argue as follows.   Take any $\epsilon>0$.
For each $k\in\Bbb Z$, (iii) tells us that there is a finite set
$L_k\subseteq Q_k$ such that 

\Centerline{$|2\pi\int_F(\varhat h\times\psi_k)\varspcheck
-\sum_{\sigma\in L',z\in J^r_{\sigma}}\innerprod{h}{\phi_{\sigma}}
   \int_F{\phi}_{\sigma}|\le 2^{-|k|}\epsilon$}
   
\noindent whenever $L'\subseteq Q_k$ is finite and $L'\supseteq L_k$;  of
course we can suppose that every $L_k$ is non-empty.
Set $L=\bigcup_{k\in\Bbb Z}L_k$, so that $L\cap Q_k=L_k$ is finite for each
$k$, and $L\in\Cal L$.   Next, there is a $K\in[\Bbb Z]^{<\omega}$ such
that

\Centerline{$|\int_F(\varhat h\times\theta_z)\varspcheck
  -\sum_{k\in K'}\int_F(\varhat h\times\psi_k)\varspcheck|
  \le\epsilon$}

\noindent whenever $K'\in[\Bbb Z]^{<\omega}$ and $K'\supseteq K$.
Take any $P\in\Cal P_{KL}$.   Setting $K'=\{k:P\cap Q_k\ne\emptyset\}$,
we have $K'\supseteq K$ and $P\cap Q_k\supseteq L\cap Q_k$ for
every $k\in K'$.   Accordingly

$$\eqalign{&|2\pi\int_F(\varhat h\times\theta_z)\varspcheck
   -\sum_{\sigma\in P,z\in J^r_{\sigma}}
      \innerprod{h}{\phi_{\sigma}}\int_F\phi_{\sigma}|
\cr&\mskip100mu
=|2\pi\int_F(\varhat h\times\theta_z)\varspcheck
   -\sum_{k\in K'}\sum_{\Atop{\sigma\in P\cap Q_k}{z\in J^r_{\sigma}}}
      \innerprod{h}{\phi_{\sigma}}\int_F\phi_{\sigma}|
\cr&\mskip100mu
\le 2\pi|\int_F(\varhat h\times\theta_z)\varspcheck
   -\sum_{k\in K'}\int_F(\varhat h\times\psi_k)\varspcheck|      
\cr&\mskip200mu
+\sum_{k\in K'}|2\pi\int_F(\varhat h\times\psi_k)\varspcheck
   -\sum_{\Atop{\sigma\in P\cap Q_k}{z\in J^r_{\sigma}}}
      \innerprod{h}{\phi_{\sigma}}\int_F\phi_{\sigma}|
\cr&\mskip100mu
\le 2\pi\epsilon+\sum_{k\in K'}2^{-|k|}\epsilon
\le (2\pi+3)\epsilon.\cr}$$

\noindent As $\Cal P_{KL}\in\Cal F$, and $\epsilon$ was arbitrary,

\Centerline{$2\pi\int_F(\varhat h\times\theta_z)\varspcheck
=\lim_{P\to\Cal F}\sum_{\sigma\in P,z\in J^r_{\sigma}}
   \innerprod{h}{\phi_{\sigma}}\int_F\phi_{\sigma}$}

\noindent as claimed.
}%end of proof of 286O
   
\leader{286P}{Lemma}\dvArevised{2013} 
Suppose that $h$ is a rapidly decreasing test function.   
For $x\in\Bbb R$, set

\Centerline{$Ah(x)
=\sup_{z\in\Bbb R}|2\pi(\varhat h\times\theta_z)\varspcheck(x)|$.}

\noindent Then $Ah:\Bbb R\to[0,\infty]$ is Borel measurable, and
$\int_FAh\le 4C_9\|h\|_2\sqrt{\mu F}$ whenever $\mu F<\infty$.

\proof{{\bf (a)} As $(\varhat h\times\theta_z)\varspcheck$ is continuous
for every $z$, $Ah$ is lower semi-continuous, therefore Borel measurable,
by 256Ma.   By 256Mb,

\Centerline{$\int_FAh
=\sup\{\int_F\sup_{i\le n}
  |2\pi(\varhat h\times\theta_{z_i})\varspcheck|:
  z_0,\ldots,z_n\in\Bbb R\}$.}

\medskip

{\bf (b)} Fix $z_0,\ldots,z_n\in\Bbb R$ for the moment.

\medskip

\quad{\bf (i)} Set  
$v_i=2\pi(\varhat h\times\theta_{z_i})\varspcheck$ for $i\le n$, and
$v=\sup_{i\le n}|v_i|$.   Set 
$E_i=\{x:v(x)=|v_i(x)|\}\setminus\bigcup_{j<i}\{x:v(x)=|v_j(x)|\}$ for
$i\le n$, so that $(E_0,\ldots,E_n)$ is a partition of $\Bbb R$ into
Borel sets, and 

\Centerline{$\int_Fv
=\int_F\sum_{i=0}^n|v_i|\times\chi E_i
=\int_F|\sum_{i=0}^nv_i\times\chi E_i|
\le 4|\int_{F'}\sum_{i=0}^nv_i\times\chi E_i|$}

\noindent for a suitable measurable $F'\subseteq F$ (246K).
Setting $g(x)=z_i$ for $x\in E_i$, $g:\Bbb R\to\Bbb R$ is Borel measurable.

\medskip

\quad{\bf (ii)} For each $i\le n$,

$$\eqalignno{\int_{F'}v_i\times\chi E_i
&=\int_{F'\cap E_i}v_i
=\lim_{P\to\Cal F}\sum_{\sigma\in P,z_i\in J^r_{\sigma}}
  \innerprod{h}{\phi_{\sigma}}\int_{F'\cap E_i}\phi_{\sigma}\cr
\displaycause{where $\Cal F$ is the filter on $[Q]^{<\omega}$ described in
286O}  
&=\lim_{P\to\Cal F}\int_{F'\cap E_i}
  \sum_{\sigma\in P,z_i\in J^r_{\sigma}}
  \innerprod{h}{\phi_{\sigma}}\phi_{\sigma}
\cr&=\lim_{P\to\Cal F}\int_{F'\cap E_i}
  \sum_{\sigma\in P,g(x)\in J^r_{\sigma}}
  \innerprod{h}{\phi_{\sigma}}\phi_{\sigma}(x)dx.\cr}$$
  
\noindent So

$$\eqalign{\int_{F'}\sum_{i=0}^nv_i\times\chi E_i  
&=\lim_{P\to\Cal F}\sum_{i=0}^n\int_{F'\cap E_i}
  \sum_{\sigma\in P,g(x)\in J^r_{\sigma}}
  \innerprod{h}{\phi_{\sigma}}\phi_{\sigma}(x)dx
\cr&=\lim_{P\to\Cal F}\int_{F'}
  \sum_{\sigma\in P,g(x)\in J^r_{\sigma}}
  \innerprod{h}{\phi_{\sigma}}\phi_{\sigma}(x)dx.\cr}$$
  
\noindent Now for any finite set $P\subseteq Q$,

\Centerline{$\int_{F'}\sum_{\sigma\in P,g(x)\in J^r_{\sigma}}
  \innerprod{h}{\phi_{\sigma}}\phi_{\sigma}(x)dx
=\sum_{\sigma\in P}\int_{F'\cap g^{-1}[J^r_{\sigma}]}
  \innerprod{h}{\phi_{\sigma}}\phi_{\sigma}$;}
  
\noindent if you like, you can think of this as an application of
Fubini's theorem, if you give counting measure to $Q$ and look at the 
function

$$\eqalign{(x,\sigma)
&\mapsto\innerprod{h}{\phi_{\sigma}}\phi_{\sigma}(x)
  \text{ if }x\in F',\,\sigma\in P\text{ and }g(x)\in J^r_{\sigma},
\cr&\mapsto 0\text{ otherwise}.\cr}$$  

\noindent But this means that

\Centerline{$|\int_{F'}\sum_{\sigma\in P,g(x)\in J^r_{\sigma}}
  \innerprod{h}{\phi_{\sigma}}\phi_{\sigma}(x)dx|
\le\sum_{\sigma\in P}|\innerprod{h}{\phi_{\sigma}}
   \int_{F'\cap g^{-1}[J^r_{\sigma}]}\phi_{\sigma}|
\le C_9\|h\|_2\sqrt{\mu F'}$}

\noindent by 286N.   Taking the limit as $P\to\Cal F$,

\Centerline{$|\int_{F'}\sum_{i=0}^nv_i\times\chi E_i|
\le C_9\|h\|_2\sqrt{\mu F'}$.}

\medskip

\quad{\bf (iii)} Thus we have

$$\eqalign{\int_F\sup_{i\le n}
   |2\pi(\varhat h\times\theta_{z_i})\varspcheck|
=\int_Fv
&\le 4|\int_{F'}\sum_{i=0}^nv_i\times\chi E_i|
\cr&\le 4C_9\|h\|_2\sqrt{\mu F'}
\le 4C_9\|h\|_2\sqrt{\mu F}.\cr}$$

\medskip

{\bf (c)} As $z_0,\ldots,z_n$ were arbitrary,
$\int_FAh\le 4C_9\|h\|_2\sqrt{\mu F}$, as claimed.
}%end of proof of 286P

\leader{286Q}{Lemma} For $\alpha>0$ and $y$, $z$, $\beta\in\Bbb R$, set
$\theta'_{z\alpha\beta}(y)=\theta_{\alpha z+\beta}(\alpha y+\beta)$.
Then

(a) the function $(\alpha,\beta,y,z)\mapsto\theta'_{z\alpha\beta}(y):
\ooint{0,\infty}\times\BbbR^3\to[0,1]$ is Borel measurable;

(b) $\theta'_{z\alpha\beta}(y)=0$ whenever $y\ge z$;

(c) for any rapidly decreasing test function $h$, and any $z\in\Bbb R$,

\Centerline{$2\pi|(\varhat h\times\theta'_{z\alpha\beta})\varspcheck|
\le D_{1/\alpha}AM_{\beta}D_{\alpha}h$}

\noindent\cmmnt{(in the notation of 286C)} at every point.

\proof{{\bf (a)} We need only recall that
$(y,z)\mapsto\theta_z(y):\BbbR^2\to\Bbb R$
is Borel measurable (286Oa), and that
$(\alpha,\beta,y,z)\mapsto\theta'_{z\alpha\beta}(y)$ is built up from
this, $+$ and $\times$.

\medskip

{\bf (b)} Again, this is immediate from 286Oa, because $\alpha>0$.

\medskip

{\bf (c)} Set $v=\alpha z+\beta$, so that
$\theta'_{z\alpha\beta}=D_{\alpha}S_{\beta}\theta_v$.   Then

$$\eqalign{\varhat h\times\theta'_{z\alpha\beta}
&=\varhat h\times D_{\alpha}S_{\beta}\theta_v
=D_{\alpha}S_{\beta}(S_{-\beta}D_{1/\alpha}\varhat h\times\theta_v)\cr
&=\alpha D_{\alpha}S_{\beta}
  (S_{-\beta}(D_{\alpha}h)\varsphat\times\theta_v)
=\alpha D_{\alpha}S_{\beta}
  ((M_{\beta}D_{\alpha}h)\varsphat\times\theta_v),\cr}$$

\noindent so

$$\eqalign{(\varhat h\times\theta'_{z\alpha\beta})\varspcheck
&=\alpha\bigl(D_{\alpha}S_{\beta}
  ((M_{\beta}D_{\alpha}h)\varsphat\times\theta_v)\bigr)\varspcheck\cr
&=D_{1/\alpha}\bigl(S_{\beta}
  ((M_{\beta}D_{\alpha}h)\varsphat\times\theta_v)\bigr)\varspcheck
=D_{1/\alpha}M_{-\beta}\bigl(
  (M_{\beta}D_{\alpha}h)\varsphat\times\theta_v\bigr)\varspcheck\cr}$$

\noindent and

\Centerline{$2\pi|(\varhat h\times\theta'_{z\alpha\beta})\varspcheck|
=2\pi D_{1/\alpha}\bigl|\bigl(
  (M_{\beta}D_{\alpha}h)\varsphat
  \times\theta_v\bigr)\varspcheck\bigr|
\le D_{1/\alpha}A(M_{\beta}D_{\alpha}h)$.}
}%end of proof of 286Q

\leader{286R}{Lemma} For any $y$, $z\in\Bbb R$,

\Centerline{$\tilde\theta_z(y)
=\int_1^2\Bover1{\alpha}\bigl(\lim_{n\to\infty}
  \Bover1n\int_0^n\theta'_{z\alpha\beta}(y)d\beta\bigr)d\alpha$}

\noindent is defined, and

$$\eqalign{\tilde\theta_z(y)&=\tilde\theta_1(0)>0\text{ if }y<z,\cr
&=0\text{ if }y\ge z.\cr}$$

\proof{{\bf (a)} The case $y\ge z$ is trivial, because if $y\ge z$ then
$\theta'_{z\alpha\beta}(y)=0$ for all $\alpha>0$ and
$\beta\in\Bbb R$ (286Qb) and $\tilde\theta_z(y)=0$.   For the rest of the proof,
therefore, I look at the case $y<z$.

\medskip

{\bf (b)(i)} Given $y<z\in\Bbb R$ and $\alpha>0$, set
$l=\lfloor\log_2(20\alpha(z-y))\rfloor$.   Then
$\theta'_{z,\alpha,\beta+2^l}(y)=\theta'_{z\alpha\beta}(y)$ for every
$\beta\in\Bbb R$.   \Prf\ If
$\theta'_{z\alpha\beta}(y)=\theta_{\alpha z+\beta}(\alpha y+\beta)$ is
non-zero, there must be $k$, $m\in\Bbb Z$ such that

\Centerline{$2^k(m+\Bover12)\le\alpha z+\beta<2^k(m+1)$}

\noindent and

\Centerline{$\varhat{\phi}(2^{-k}(\alpha y+\beta)-(m+\Bover14))^2
=\theta'_{z\alpha\beta}(y)\ne 0$,}

\noindent so

\Centerline{$2^km\le\alpha y+\beta\le 2^k(m+\Bover9{20})$}

\noindent because $\varhat\phi$ is zero outside $[-\bover15,\bover15]$.
In this case,
$\Bover1{20}\cdot 2^k<\alpha(z-y)$, so that $k\le l$.   We therefore
have

\Centerline{$2^k(m+2^{l-k}+\Bover12)\le\alpha z+\beta+2^l
<2^k(m+2^{l-k}+1)$,}

\Centerline{$2^k(m+2^{l-k})\le\alpha y+\beta+2^l
<2^k(m+2^{l-k}+\Bover12)$,}

\noindent so

\Centerline{$\theta'_{z,\alpha,\beta+2^l}(y)
=\varhat{\phi}(2^{-k}(\alpha y+\beta+2^l)-(m+2^{l-k}+\Bover14))^2
=\theta'_{z\alpha\beta}(y)$.}

\noindent Similarly,

\Centerline{$2^k(m-2^{l-k}+\Bover12)\le\alpha z+\beta-2^l
<2^k(m-2^{l-k}+1)$,}

\Centerline{$2^k(m-2^{l-k})\le\alpha y+\beta-2^l
<2^k(m-2^{l-k}+\Bover12)$,}

\noindent so

\Centerline{$\theta'_{z,\alpha,\beta-2^l}(y)
=\varhat{\phi}(2^{-k}(\alpha y+\beta-2^l)-(m-2^{l-k}+\Bover14))^2
=\theta'_{z\alpha\beta}(y)$.}

\noindent What this shows is that
$\theta'_{z,\alpha,\beta+2^l}(y)=\theta'_{z\alpha\beta}(y)$ if either is
non-zero, so we have the equality in any case.\ \Qed

\medskip

{\bf (ii)} It follows that $g(\alpha,y,z)
=\lim_{b\to\infty}\Bover1b\biggerint_0^b\theta'_{z\alpha\beta}(y)d\beta$ is
defined.   \Prf\ Set

\Centerline{$\gamma
=2^{-l}\int_0^{2^l}\theta'_{z\alpha\beta}(y)d\beta$.}

\noindent From (i) we see that

\Centerline{$\gamma
=2^{-l}\int_{2^lm}^{2^l(m+1)}\theta'_{z\alpha\beta}(y)d\beta$}

\noindent for every $m\in\Bbb Z$, and therefore that

\Centerline{$\gamma
=\Bover1{2^lm}\int_0^{2^lm}\theta'_{z\alpha\beta}(y)d\beta$}

\noindent for every $m\ge 1$.   Now $\theta'_{z\alpha\beta}(y)$ is always
greater than or equal to $0$, so if $2^lm\le b\le 2^l(m+1)$ then

\Centerline{$\Bover{m}{m+1}\gamma
=\Bover1{2^l(m+1)}\int_0^{2^lm}\theta'_{z\alpha\beta}
\le\Bover1b\int_0^b\theta'_{z\alpha\beta}
\le\Bover1{2^lm}\int_0^{2^l(m+1)}\theta'_{z\alpha\beta}
=\Bover{m+1}{m}\gamma$,}

\noindent which approach $\gamma$ as $b\to\infty$.\ \Qed

\medskip

{\bf (c)} Because $(\alpha,y)\mapsto\theta'_{z\alpha\beta}(y)$ is
always Borel measurable, each of the functions
$\alpha\mapsto\Bover1n\int_0^n\theta'_{z\alpha\beta}$, for
$n\ge 1$, is Borel measurable (putting 251M and 252P together), and
$\alpha\mapsto g(\alpha,y,z):\ooint{0,\infty}\to\Bbb R$ is Borel
measurable;  at the same time, since
$0\le\theta'_{z\alpha\beta}(y)\le 1$ for all $\alpha$ and $\beta$,
$0\le g(\alpha,y,z)\le 1$ for every $\alpha$, and
$\tilde\theta_z(y)=\int_1^2\Bover1{\alpha}g(\alpha,y,z)d\alpha$ is
defined in $[0,1]$.

\medskip

{\bf (d)} For any $y<z$, $\gamma\in\Bbb R$ and $\alpha>0$,
$g(\alpha,y+\gamma,z+\gamma)=g(\alpha,y,z)$.   \Prf\ It is enough to
consider the case $\gamma\ge 0$.   In this case

$$\eqalign{g(\alpha,y+\gamma,z+\gamma)
&=\lim_{b\to\infty}\Bover1b\int_0^b
  \theta'_{z+\gamma,\alpha,\beta}(y+\gamma)d\beta\cr
&=\lim_{b\to\infty}\Bover1b\int_0^b
  \theta_{\alpha z+\alpha\gamma+\beta}
    (\alpha y+\alpha\gamma+\beta)d\beta\cr
&=\lim_{b\to\infty}\Bover1b\int_{\alpha\gamma}^{b+\alpha\gamma}
  \theta_{\alpha z+\beta}(\alpha y+\beta)d\beta
=\lim_{b\to\infty}\Bover1b\int_{\alpha\gamma}^{b+\alpha\gamma}
  \theta'_{z\alpha\beta}(y)d\beta,\cr}$$

\noindent so

$$\eqalign{|g(\alpha,y+\gamma,z+\gamma)-g(\alpha,y,z)|
&=\lim_{b\to\infty}\Bover1b\bigl
  |\int_b^{b+\alpha\gamma}\theta'_{z\alpha\beta}(y)d\beta
  -\int_0^{\alpha\gamma}\theta'_{z\alpha\beta}(y)d\beta\bigr|\cr
&\le\lim_{b\to\infty}\Bover{2\alpha\gamma}{b}=0. \text{ \Qed}\cr}$$

It follows that whenever $y<z$ and $\gamma\in\Bbb R$,

\Centerline{$\tilde\theta_{z+\gamma}(y+\gamma)
=\int_{0}^{1}\Bover1{\alpha}g(\alpha,y+\gamma,z+\gamma)d\alpha
=\int_{0}^{1}\Bover1{\alpha}g(\alpha,y,z)d\alpha
=\tilde\theta_z(y)$.}

\medskip

{\bf (e)} The next essential fact to note is that $\theta_{2z}(2y)$ is
always equal to $\theta_z(y)$.   \Prf\ If $\theta_z(y)\ne 0$, then (as
in (b) above) there are $k$, $m\in\Bbb Z$ such that

\Centerline{$2^k(m+\Bover12)\le z<2^k(m+1)$,
\quad$2^km\le y<2^k(m+\Bover12)$,
\quad$\theta_z(y)=\varhat{\phi}(2^{-k}y-(m+\Bover14))^2$.}

\noindent In this case,

\Centerline{$2^{k+1}(m+\Bover12)\le 2z<2^{k+1}(m+1)$,
\quad$2^{k+1}m\le 2y<2^{k+1}(m+\Bover12)$,}

\noindent so

\Centerline{$\theta_{2z}(2y)
=\varhat{\phi}(2^{-k-1}\cdot 2y-(m+\Bover14))^2
=\theta_z(y)$.}

\noindent Similarly,

\Centerline{$2^{k-1}(m+\Bover12)\le \Bover12z<2^{k-1}(m+1)$,
\quad$2^{k-1}m\le \Bover12y<2^{k-1}(m+\Bover12)$,}

\noindent so

\Centerline{$\theta_{\bover12z}(\Bover12y)
=\varhat{\phi}(2^{-k+1}\cdot\Bover12y-(m+\Bover14))^2
=\theta_z(y)$.}

\noindent This shows that $\theta_{2z}(2y)=\theta_z(y)$ if either is
non-zero, and therefore in all cases.\ \Qed

Accordingly

\Centerline{$\theta'_{z,2\alpha,2\beta}(y)
=\theta_{2\alpha z+2\beta}(2\alpha y+2\beta)
=\theta_{\alpha z+\beta}(\alpha y+\beta)
=\theta'_{z\alpha\beta}(y)$}

\noindent for all $y$, $z$, $\beta\in\Bbb R$ and all $\alpha>0$.

\medskip

{\bf (f)} Consequently

$$\eqalign{g(2\alpha,y,z)
&=\lim_{b\to\infty}\Bover1b\int_0^b\theta'_{z,2\alpha,\beta}(y)d\beta
=\lim_{b\to\infty}\Bover2b\int_0^{b/2}
  \theta'_{z,2\alpha,2\beta}(y)d\beta\cr
&=\lim_{b\to\infty}\Bover2b\int_0^{b/2}
  \theta'_{z\alpha\beta}(y)d\beta
=\lim_{b\to\infty}\Bover1b\int_0^b\theta'_{z\alpha\beta}(y)d\beta
=g(\alpha,y,z)\cr}$$

\noindent whenever $\alpha>0$ and $y$, $z\in\Bbb R$.   It follows that

\Centerline{$\int_{\gamma}^{\delta}
   \Bover1{\alpha}g(\alpha,y,z)d\alpha
=\int_{\gamma}^{\delta}\Bover1{\alpha}g(2\alpha,y,z)d\alpha
=\int_{2\gamma}^{2\delta}\Bover1{\alpha}g(\alpha,y,z)d\alpha$}

\noindent whenever $0<\gamma\le\delta$, and therefore that

\Centerline{$\int_{\gamma}^{2\gamma}
  \Bover1{\alpha}g(\alpha,y,z)d\alpha
=\int_1^2\Bover1{\alpha}g(\alpha,y,z)d\alpha$}

\noindent for every $\gamma>0$.   \Prf\ Take $k\in\Bbb Z$ such that
$2^k\le\gamma<2^{k+1}$.   Then

$$\eqalign{\int_{\gamma}^{2\gamma}\Bover1{\alpha}g(\alpha,y,z)d\alpha
&=\int_{2^k}^{2^{k+1}}\Bover1{\alpha}g(\alpha,y,z)d\alpha
  -\int_{2^k}^{\gamma}\Bover1{\alpha}g(\alpha,y,z)d\alpha
  +\int_{2^{k+1}}^{2\gamma}\Bover1{\alpha}g(\alpha,y,z)d\alpha\cr
&=\int_{2^k}^{2^{k+1}}\Bover1{\alpha}g(\alpha,y,z)d\alpha
=\int_1^2\Bover1{\alpha}g(\alpha,y,z)d\alpha.\text{  \Qed}\cr}$$

\medskip

{\bf (g)} Now if $\alpha$, $\gamma>0$ and $y<z$,

\Centerline{$g(\alpha,\gamma y,\gamma z)
=\lim_{b\to\infty}\Bover1b\int_0^b
  \theta_{\alpha\gamma z+\beta}(\alpha\gamma y+\beta)d\beta
=g(\alpha\gamma,y,z)$.}

\noindent So if $\gamma>0$ and $y<z$,

$$\eqalign{\tilde\theta_{\gamma z}(\gamma y)
&=\int_1^2\Bover1{\alpha}g(\alpha,\gamma y,\gamma z)d\alpha
=\int_1^2\Bover1{\alpha}g(\alpha\gamma,y,z)d\alpha\cr
&=\int_{\gamma}^{2\gamma}\Bover1{\alpha}g(\alpha,y,z)d\alpha
=\int_1^2\Bover1{\alpha}g(\alpha,y,z)d\alpha
=\tilde\theta_z(y).\cr}$$

\noindent Putting this together with (d), we see that if $y<z$ then

\Centerline{$\tilde\theta_z(y)
=\tilde\theta_{z-y}(0)
=\tilde\theta_1(0)$.}

\medskip

{\bf (h)} I have still to check that $\tilde\theta_1(0)$ is not zero.
But suppose that $1\le\alpha<\bover76$ and that there is some
$m\in\Bbb Z$ such that $2(m+\bover1{12})\le\beta\le 2(m+\bover5{12})$.
Then $2(m+\bover12)\le\alpha+\beta<2(m+1)$, while
$|\bover12\beta-(m+\bover14)|\le\bover16$, so

\Centerline{$\theta_{\alpha+\beta}(\beta)
=\varhat{\phi}(\Bover12\beta-(m+\Bover14))^2
=1$.}

\noindent What this means is that, for $1\le\alpha<\bover76$,

$$\eqalign{g(\alpha,0,1)
&=\lim_{m\to\infty}\Bover1{2m}\int_0^{2m}
  \theta_{\alpha+\beta}(\beta)d\beta\cr
&\ge\lim_{m\to\infty}\Bover{1}{2m}
  \sum_{j=0}^{m-1}\mu[2(j+\Bover1{12}),2(j+\Bover5{12})]
=\Bover13.\cr}$$

\noindent So

\Centerline{$\tilde\theta_1(0)
=\int_1^2\Bover1{\alpha}g(\alpha,0,1)d\alpha
\ge\Bover13\int_1^{7/6}\Bover1{\alpha}d\alpha
>0$.}

\noindent This completes the proof.
}%end of proof of 286R

\leader{286S}{Lemma} Suppose that $h$ is a rapidly decreasing test function.

(a) For every $x\in\Bbb R$,

\Centerline{$(\tilde Ah)(x)
=\liminf_{n\to\infty}\Bover1n\int_1^2\Bover1{\alpha}\int_0^n
  (D_{1/\alpha}AM_{\beta}D_{\alpha}h)(x)d\beta d\alpha$}

\noindent is defined in $[0,\infty]$, and
$\tilde Ah:\Bbb R\to[0,\infty]$ is Borel measurable.

(b) $\int_F\tilde Ah\le 3C_9\|h\|_2\sqrt{\mu F}$ whenever $\mu F<\infty$.

(c) If $z\in\Bbb R$,
$2\pi|(\varhat h\times\tilde\theta_z)\varspcheck|\le\tilde Ah$ at every
point.

\proof{{\bf (a)} The point here is that the function

\Centerline{$(\alpha,\beta,x)
\mapsto(D_{1/\alpha}AM_{\beta}D_{\alpha}h)(x):
  \ooint{0,\infty}\times\BbbR^2\to[0,\infty]$}

\noindent is Borel measurable.   \Prf\

$$\eqalign{(D_{1/\alpha}AM_{\beta}D_{\alpha}h)(x)
&=(AM_{\beta}D_{\alpha}h)(\Bover{x}{\alpha})
\cr&=\sup_{z\in\Bbb R}
  |2\pi((M_{\beta}D_{\alpha}h)\varsphat\times\theta_z)\varspcheck
     (\Bover{x}{\alpha})|
\cr&=\Bover{2\pi}{\alpha}\sup_{z\in\Bbb R}
  |(S_{-\beta}D_{1/\alpha}\varhat h\times\theta_z)\varspcheck
     (\Bover{x}{\alpha})|.
\cr}$$
  
\noindent Now, for any $z\in\Bbb R$, 

\Centerline{$(S_{-\beta}D_{1/\alpha}\varhat h\times\theta_z)\varspcheck
     (\bover{x}{\alpha})
=\Bover1{\sqrt{2\pi}}\int_{-\infty}^{\infty}e^{ixy/\alpha}
   \varhat h(\bover{y-\beta}{\alpha})\theta_z(y)dy$.}

\noindent We know that $\varhat h$ is a rapidly decreasing test function, 
so there is some 
$\gamma\ge 0$ such that $|\varhat h(t)|\le\Bover{\gamma}{1+t^2}$ for every
$t\in\Bbb R$.   This means that if $\alpha>0$ and $\beta\in\Bbb R$
and $\sequencen{\alpha_n}$,
$\sequencen{\beta_n}$ are sequences in $\ocint{0,2\alpha}$ and
$[\beta-1,\beta+1]$, converging to $\alpha$, $\beta$ respectively, 
and we set $g(t)
=\sup_{n\in\Bbb N}|\varhat h(\Bover{t-\beta_n}{\alpha_n})\theta_z(t)|$,
then 

$$\eqalign{g(t)
&\le\Bover{4\gamma\alpha^2}{(|t|-|\beta|-1)^2}\text{ if }|t|\ge|\beta|+2,
  \cr
&\le\gamma\text{ otherwise},\cr}$$

\noindent and $g$ is integrable.   (Remember that 
$0\le\theta_z(y)\le 1$ for every $y$, as noted in 286Oa.)   So Lebesgue's
Dominated Convergence Theorem tells us that if
$\sequencen{\alpha_n}\to\alpha$ and $\sequencen{\beta_n}\to\beta$ and
$\sequencen{x_n}\to x$, then

\Centerline{$\int_{-\infty}^{\infty}e^{ix_ny/\alpha_n}
   \varhat h(\bover{y-\beta_n}{\alpha_n})\theta_z(y)dy
\to\int_{-\infty}^{\infty}e^{ixy/\alpha}
   \varhat h(\bover{y-\beta}{\alpha})\theta_z(y)dy$.}
   
\noindent Thus 
$(\alpha,\beta,x)
\mapsto(S_{-\beta}D_{1/\alpha}\varhat h\times\theta_z)\varspcheck
(\bover{x}{\alpha})$ is continuous;  and this is true for every
$z\in\Bbb R$.   Consequently

\Centerline{$(\alpha,\beta,x)\mapsto\sup_{z\in\Bbb R}
  |(S_{-\beta}D_{1/\alpha}\varhat h\times\theta_z)\varspcheck
     (\Bover{x}{\alpha})|$}   

\noindent and $(\alpha,\beta,x)
\mapsto(D_{1/\alpha}AM_{\beta}D_{\alpha}h)(x)$ are lower semi-continuous,
therefore Borel measurable, by 256Ma again.\ \Qed

It follows that the repeated integrals

\Centerline{$\int_1^2\Bover1{\alpha}\int_0^n
(D_{1/\alpha}AM_{\beta}D_{\alpha}h)(x)d\beta d\alpha$}

\noindent are defined in $[0,\infty]$ and are Borel measurable functions
of $x$ (252P again), so that $\tilde Af$ is Borel measurable.

\medskip

{\bf (b)} For any $n\in\Bbb N$,

$$\eqalignno{\int_F\Bover1n\int_1^2\Bover1\alpha\int_0^n
&(D_{1/\alpha}AM_{\beta}D_{\alpha}h)(x)d\beta d\alpha dx\cr
&=\Bover1n\int_1^2\Bover1\alpha\int_0^n\!\int_F
  (D_{1/\alpha}AM_{\beta}D_{\alpha}h)(x)dx d\beta d\alpha\cr
\displaycause{by Fubini's theorem, 252H}
&=\Bover1n\int_1^2\!\int_0^n\!\int_F\Bover1\alpha
  (AM_{\beta}D_{\alpha}h)(\Bover{x}{\alpha})dx d\beta d\alpha\cr
&=\Bover1n\int_1^2\!\int_0^n\int_{\alpha^{-1}F}
  (AM_{\beta}D_{\alpha}h)(x)dx d\beta d\alpha\cr
&\le\Bover1n\int_1^2\!\int_0^n 4C_9
  \|M_{\beta}D_{\alpha}h\|_2\sqrt{\mu(\alpha^{-1}F)} d\beta d\alpha\cr
\displaycause{286P}
&=4C_9\cdot\Bover1n\int_1^2\!\int_0^n
  \Bover1{\sqrt{\alpha}}\|h\|_2\cdot\Bover1{\sqrt{\alpha}}\sqrt{\mu F}
    d\beta d\alpha\cr
&=4C_9\|h\|_2\sqrt{\mu F}\cdot\Bover1n\int_1^2\Bover1{\alpha}\int_0^n
 d\beta d\alpha\cr
&=4C_9\|h\|_2\sqrt{\mu F}\ln 2
\le 3C_9\|h\|_2\sqrt{\mu F}.\cr}$$

\noindent So

$$\eqalignno{\int_F\tilde Ah
&=\int_F\liminf_{n\to\infty}\Bover1n\int_1^2\Bover1\alpha\int_0^n
  (D_{1/\alpha}AM_{\beta}D_{\alpha}h)(x)d\beta d\alpha dx\cr
&\le\liminf_{n\to\infty}\int_F\Bover1n\int_1^2\Bover1\alpha\int_0^n
  (D_{1/\alpha}AM_{\beta}D_{\alpha}h)(x)d\beta d\alpha dx\cr
\displaycause{by Fatou's lemma}
&\le 3C_9\|h\|_2\sqrt{\mu F}.\cr}$$

\medskip

{\bf (c)} For any $x\in\Bbb R$,

\Centerline{$\int_{-\infty}^{\infty}
  |\varhat h(y)|\int_1^2\Bover1{\alpha}
  \bigl(\sup_{n\in\Bbb N}\Bover1n\int_0^n
  \theta'_{z\alpha\beta}(y)d\beta\bigr)d\alpha dy
\le\ln 2\cdot\int_{-\infty}^{\infty}|\varhat h|$}

\noindent is finite.   So

$$\eqalignno{(\varhat h\times\tilde\theta_z)\varspcheck(x)
&=\Bover1{\sqrt{2\pi}}\int_{-\infty}^{\infty}
  e^{ixy}\varhat h(y)\tilde\theta_z(y)dy\cr
&=\Bover1{\sqrt{2\pi}}\int_{-\infty}^{\infty}
  e^{ixy}\varhat h(y)\int_1^2\Bover1{\alpha}\lim_{n\to\infty}
  \Bover1n\int_0^n\theta'_{z\alpha\beta}(y)d\beta d\alpha dy\cr
&=\Bover1{\sqrt{2\pi}}\lim_{n\to\infty}\int_{-\infty}^{\infty}
  e^{ixy}\varhat h(y)\int_1^2\Bover1{\alpha n}
  \int_0^n\theta'_{z\alpha\beta}(y)d\beta d\alpha dy\cr
\displaycause{by Lebesgue's Dominated Convergence Theorem}
&=\Bover1{\sqrt{2\pi}}\lim_{n\to\infty}\int_1^2\Bover1{\alpha n}
  \int_0^n\int_{-\infty}^{\infty}
  e^{ixy}\varhat h(y)\theta'_{z\alpha\beta}(y)dy d\beta d\alpha\cr
\displaycause{by Fubini's theorem}
&=\lim_{n\to\infty}\int_1^2\Bover1{\alpha n}
  \int_0^n(\varhat h\times\theta'_{z\alpha\beta})\varspcheck(x)
  d\beta d\alpha,\cr}$$

\noindent and

$$\eqalignno{2\pi|(\varhat h\times\tilde\theta_z)\varspcheck(x)|
&=2\pi\bigl|\lim_{n\to\infty}\int_1^2\Bover1{\alpha n}
  \int_0^n(\varhat h\times\theta'_{z\alpha\beta})\varspcheck(x)
  d\beta d\alpha\bigr|\cr
&\le 2\pi\liminf_{n\to\infty}\int_1^2\Bover1{\alpha n}
  \int_0^n|(\varhat h\times\theta'_{z\alpha\beta})\varspcheck(x)|
  d\beta d\alpha\cr
&\le\liminf_{n\to\infty}\int_1^2\Bover1{\alpha n}
  \int_0^n(D_{1/\alpha}AM_{\beta}D_{\alpha}h)(x)
  d\beta d\alpha\cr
\displaycause{286Qb}
&=(\tilde Ah)(x).\cr}$$
}%end of proof of 286S

\leader{286T}{Lemma} Set $C_{10}=3C_9/\pi\tilde\theta_1(0)$.   For
$f\in\eusm L^2_{\Bbb C}$, define $\hat Af:\Bbb R\to[0,\infty]$ by
setting

\Centerline{$(\hat Af)(y)
=\sup_{a\le b}\Bover1{\sqrt{2\pi}}|\int_a^be^{-ixy}f(x)dx|$}

\noindent for each $y\in\Bbb R$.   Then
$\int_F\hat Af\le C_{10}\|f\|_2\sqrt{\mu F}$ whenever $\mu F<\infty$.

\proof{{\bf (a)} As usual, the first step is to confirm that $\hat Af$
is measurable.   \Prf\ For $a\le b$,
$y\mapsto|\Bover1{\sqrt{2\pi}}\int_a^be^{-ixy}f(x)dx|$ is continuous (by
283Cf, since $f\times\chi[a,b]$ is integrable), so $\hat Af$ is lower
semi-continuous, therefore Borel measurable (256Ma once more).\ \Qed

\medskip

{\bf (b)} Suppose that $h$ is a rapidly decreasing test function.   Then

\Centerline{$(\hat Ah)(y)
\le\Bover1{\pi\tilde\theta_1(0)}(\tilde A\varcheck h)(-y)$}

\noindent for every $y\in\Bbb R$.   \Prf\ If $a\in\Bbb R$ then

$$\eqalignno{\Bover1{\sqrt{2\pi}}|\int_{-\infty}^ae^{-ixy}h(x)dx|
&=\Bover1{\tilde\theta_1(0)\sqrt{2\pi}}|\int_{-\infty}^{\infty}
  e^{-ixy}\tilde\theta_a(x)h(x)dx|\cr
\displaycause{286R}
&=\Bover1{\tilde\theta_1(0)}|(h\times\tilde\theta_a)\varspcheck(-y)|
=\Bover1{\tilde\theta_1(0)}
  |(\varcheck h\varhat{\phantom{h}}
    \times\tilde\theta_a)\varspcheck(-y)|\cr
\displaycause{284C once more}
&\le\Bover1{2\pi\tilde\theta_1(0)}(\tilde A\varcheck{h})(-y)\cr}$$

\noindent (286Sc).   So if $a\le b$ in $\Bbb R$,

\Centerline{$\Bover1{\sqrt{2\pi}}|\int_{a}^{b}e^{-ixy}h(x)dx|
\le\Bover1{\pi\tilde\theta_1(0)}(\tilde A\varcheck{h})(-y)$;}

\noindent taking the supremum over $a$ and $b$, we have the result.\
\Qed

It follows that

$$\eqalignno{\int_F\hat Ah
&\le\Bover1{\pi\tilde\theta_1(0)}\int_{-F}\tilde A\varcheck h
\le\Bover3{\pi\tilde\theta_1(0)}C_9\|h\|_2\sqrt{\mu(-F)}\cr
\displaycause{286Sb, 284Oa}
&=C_{10}\|h\|_2\sqrt{\mu F}.\cr}$$

\medskip

{\bf (c)} For general square-integrable $f$, take any $\epsilon>0$ and
any $n\in\Bbb N$.   Set

\Centerline{$(\hat A_nf)(y)
=\sup_{-n\le a\le b\le n}\Bover1{\sqrt{2\pi}}|\int_a^be^{-ixy}f(x)dx|$}

\noindent for each $y\in\Bbb R$.   Let $h$ be a rapidly decreasing test
function such that $\|f-h\|_2\le\epsilon$ (284N).   Then

\Centerline{$\hat Ah
\ge\hat A_nh
\ge\hat A_nf-\Bover{\sqrt{2n}}{\sqrt{2\pi}}\epsilon$}

\noindent (using Cauchy's inequality), so

\Centerline{$\int_F\hat A_nf
\le\int_F\hat Ah+\sqrt{\bover{n}{\pi}}\epsilon\mu F
\le C_{10}(\|f\|_2+\epsilon)\sqrt{\mu F}
  +\sqrt{\bover{n}{\pi}}\epsilon\mu F$.}

\noindent As $\epsilon$ is arbitrary,
$\int_F\hat A_nf\le C_{10}\|f\|_2\sqrt{\mu F}$;  letting $n\to\infty$,
we get $\int_F\hat Af\le C_{10}\|f\|_2\sqrt{\mu F}$.
}%end of proof of 286T

\vleader{72pt}{286U}{Theorem} If $f\in\eusm L^2_{\Bbb C}$ then

\Centerline{$g(y)
=\lim_{a\to-\infty,b\to\infty}
  \Bover1{\sqrt{2\pi}}\int_a^be^{-ixy}f(x)dx$}

\noindent is defined in $\Bbb C$ for almost every $y\in\Bbb R$, and $g$
represents the Fourier transform of $f$.

\proof{{\bf (a)} For $n\in\Bbb N$, $y\in\Bbb R$ set

\Centerline{$\gamma_n(y)=\sup_{a\le -n,b\ge n}\Bover1{\sqrt{2\pi}}
  \bigl|\int_a^be^{-ixy}f(x)dx-\int_{-n}^ne^{-ixy}f(x)dx\bigr|$.}

\noindent Then $g(y)$ is defined whenever
$\inf_{n\in\Bbb N}\gamma_n(y)=0$.   \Prf\ If
$\inf_{n\in\Bbb N}\gamma_n(y)=0$ and $\epsilon>0$, take $m\in\Bbb N$
such that $\gamma_m(y)\le\bover12\epsilon$;  then
$\Bover1{\sqrt{2\pi}}
|\int_a^be^{-ixy}f(x)dx-\int_{-n}^ne^{-ixy}f(x)dx|\le\epsilon$ whenever
$n\ge m$, $a\le-n$ and $b\ge n$.   But this means, first, that
$\sequencen{\int_{-n}^ne^{-ixy}f(x)dx}$ is a Cauchy sequence, so has a
limit $\zeta$ say, and, second, that
$\zeta=\lim_{a\to-\infty,b\to\infty}\int_a^be^{-ixy}f(x)dx$, so that
$g(y)=\Bover{\zeta}{\sqrt{2\pi}}$ is defined.\ \Qed

Also each $\gamma_n$ is lower-semicontinuous (cf.\ part (a) of the
proof of 286T).

\medskip

{\bf (b)} \Quer\ Suppose, if possible, that
$\{y:\inf_{n\in\Bbb N}\gamma_n(y)>0\}$ is not negligible.   Then

\Centerline{$\lim_{m\to\infty}\mu\{y:|y|\le m,\,
  \inf_{n\in\Bbb N}\gamma_n(y)\ge\Bover1m\}
>0$,}

\noindent so there is an $\epsilon>0$ such that

\Centerline{$F=\{y:|y|\le\Bover1{\epsilon},\,
  \inf_{n\in\Bbb N}\gamma_n(y)\ge\epsilon\}$}

\noindent has measure greater than $\epsilon$.   Let $n\in\Bbb N$ be
such that

\Centerline{$4C^2_{10}
  (\int_{-\infty}^{\infty}|f(x)|^2dx-\int_{-n}^n|f(x)|^2dx)
\le\epsilon^3$,}

\noindent and set $f_1=f-f\times\chi[-n,n]$;  then
$2C_{10}\|f_1\|_2\le\epsilon^{3/2}$.

We have

$$\eqalign{\gamma_n(y)
&=\sup_{a\le -n,b\ge n}\Bover1{\sqrt{2\pi}}
  \bigl|\int_a^be^{-ixy}f_1(x)dx-\int_{-n}^ne^{-ixy}f_1(x)dx\bigr|\cr
&\le 2\sup_{a\le b}\Bover1{\sqrt{2\pi}}
  |\int_a^be^{-ixy}f_1(x)dx|
\le 2(\hat Af_1)(y),\cr}$$

\noindent so that

$$\eqalignno{\epsilon\mu F
&\le\int_F\gamma_n
\le 2\int_F\hat Af_1
\le 2C_{10}\|f_1\|_2\sqrt{\mu F}\cr
\displaycause{286T}
&\le\epsilon^{3/2}\sqrt{\mu F}\cr}$$

\noindent and $\mu F\le\epsilon$;  but we chose $\epsilon$ so that
$\mu F$ would be greater than $\epsilon$.\ \Bang

\medskip

{\bf (c)} Thus $g(y)$ is defined for almost every $y\in\Bbb R$.  Now $g$
represents the Fourier transform of $f$.   \Prf\ Let $h$ be a rapidly
decreasing test function.   The restriction of $\hat Af$ to the set
on which it is finite is a tempered function, by 286D, so
$\int_{-\infty}^{\infty}(\hat Af)\times|h|$ is finite, by 284F.   Now

$$\eqalignno{\int_{-\infty}^{\infty}g\times h
&=\Bover1{\sqrt{2\pi}}\int_{-\infty}^{\infty}
  \bigl(\lim_{n\to\infty}\int_{-n}^ne^{-ixy}f(x)dx\bigr)h(y)dy\cr
&=\Bover1{\sqrt{2\pi}}\lim_{n\to\infty}\int_{-\infty}^{\infty}
  \int_{-n}^ne^{-ixy}f(x)h(y)dxdy\cr
\displaycause{because
$\Bover1{\sqrt{2\pi}}|\int_{-n}^ne^{-ixy}f(x)dx|\le\hat Af(y)$ for every
$n$ and $y$, so we can use Lebesgue's Dominated Convergence Theorem}
&=\Bover1{\sqrt{2\pi}}\lim_{n\to\infty}
  \int_{-n}^n\int_{-\infty}^{\infty}e^{-ixy}f(x)h(y)dydx\cr
\displaycause{because $\int_{-\infty}^{\infty}\int_{-n}^n|f(x)h(y)|dxdy$
is finite for each $n$}
&=\lim_{n\to\infty}
  \int_{-n}^nf\times\varhat{h}
=\int_{-\infty}^{\infty}f\times\varhat{h}\cr}$$

\noindent because $f\times\varhat{h}$ is certainly integrable.   As $h$
is arbitrary, $g$ represents the Fourier transform of $f$.\ \Qed
}%end of proof of 286U

\leader{286V}{Theorem} For any square-integrable complex-valued function
on $\ocint{-\pi,\pi}$, its sequence of
Fourier sums converges to it almost everywhere.

\proof{ Suppose that
$f\in\eusm L^2_{\Bbb C}(\mu_{\ocint{-\pi,\pi}})$.
Set $f_1(x)=f(x)$ for $x\in\dom f$, $0$ for
$x\in\Bbb R\setminus\ocint{-\pi,\pi}$;  then
$f_1\in\eusm L^2_{\Bbb C}(\mu)$.   Let $g\in\eusm L^2_{\Bbb C}(\mu)$
represent the inverse Fourier transform of $f_1$ (284O).   Then 286U
tells us that
$f_2(x)=\lim_{a\to\infty}\Bover1{\sqrt{2\pi}}\int_{-a}^ae^{-ixy}g(y)dy$
is defined for almost every $x$, and that $f_2$ represents the Fourier
transform of $g$, so is equal almost everywhere to $f_1$ (284Ib).

Now, for any $a\ge 0$, $x\in\Bbb R$,

$$\eqalignno{\int_{-a}^ae^{-ixy}g(y)dy
&=\innerprod{g}{h_{ax}}\cr
\displaycause{where $h_{ax}(y)=e^{ixy}$ if $|y|\le a$, $0$ otherwise}
&=\innerprod{f_2}{\varhat{h}_{ax}}\cr
\displaycause{284Ob}
&=\Bover1{\sqrt{2\pi}}\int_{-\infty}^{\infty}f_2(t)
  \overline{\int_{-\infty}^{\infty}e^{-ity}h_{ax}(y)dy}\,dt\cr
&=\Bover2{\sqrt{2\pi}}\int_{-\infty}^{\infty}
  \Bover{\sin(x-t)a}{x-t}f_2(t)dt
=\Bover2{\sqrt{2\pi}}\int_{-\pi}^{\pi}
  \Bover{\sin(x-t)a}{x-t}f(t)dt.\cr}$$

\noindent So

\Centerline{$f(x)=f_2(x)
=\lim_{a\to\infty}\Bover1{\pi}\int_{-\pi}^{\pi}
  \Bover{\sin(x-t)a}{x-t}f(t)dt$}

\noindent for almost every $x\in\ocint{-\pi,\pi}$.

On the other hand, writing $\sequencen{s_n}$ for the sequence of Fourier
sums of $f$, we have, for any $x\in\ooint{-\pi,\pi}$,

\Centerline{$s_n(x)
=\Bover1{2\pi}\int_{-\pi}^{\pi}f(t)
\Bover{\sin(n+{1\over 2}\pushbottom{3.5pt})(x-t)}
  {\sin{1\over 2}(x-t)}dt$}

\noindent for each $n$, by 282Da.   Now

$$\eqalign{\Bover1{2\pi}\int_{-\pi}^{\pi}f(t)
  &\Bover{\sin(n+{1\over 2}\pushbottom{3.5pt})(x-t)}
  {\sin{1\over 2}(x-t)}dt
-\Bover1{\pi}\int_{-\pi}^{\pi}f(t)
  \Bover{\sin(n+{1\over 2}\pushbottom{3.5pt})(x-t)}
  {x-t}dt\cr
&=\Bover1{\pi}\int_{-\pi}^{\pi}f(t)
  \bigl(\Bover{\sin(n+{1\over 2}\pushbottom{3.5pt})(x-t)}
    {2\sin{1\over 2}(x-t)}
  -\Bover{\sin(n+\bover12\pushbottom{3.5pt})(x-t)}{x-t}\bigr)dt\cr
&=\Bover1{\pi}\int_{x-\pi}^{x+\pi}\bigl
  (\Bover1{2\sin{1\over 2}t}-\Bover1{t}\bigr)
  f(x-t)\sin(n+\hbox{$\bover12$})t\,dt.\cr}$$

\noindent But if we look at the function

$$\eqalign{p_x(t)
&=\bigl(\Bover1{2\sin\bover12t}-\Bover1t\bigr)f(x-t)
  \text{ if }x-\pi<t<x+\pi\text{ and }t\ne 0,\cr
&=0\text{ otherwise},\cr}$$

\noindent $p_x$ is integrable, because $f$ is integrable over
$\ocint{-\pi,\pi}$ and
$\lim_{t\to 0}\Bover1{2\sin\bover12t}-\Bover1t=0$, so
$\sup_{t\ne 0,x-\pi\le t\le x+\pi}|\Bover1{2\sin\bover12t}-\Bover1t|$ is
finite.   (This is where we need to know that $|x|<\pi$.)   So

$$\eqalign{\lim_{n\to\infty}s_n(x)-\Bover1{\pi}\int_{-\pi}^{\pi}f(t)
  \Bover{\sin(n+{1\over 2}\pushbottom{3.5pt})(x-t)}
  {x-t}dt
=\lim_{n\to\infty}\int_{-\infty}^{\infty}
       p_x(t)\sin(n+\hbox{$\bover12$})t\,dt
=0\cr}$$

\noindent by the Riemann-Lebesgue lemma (282Fb).   But this means that
$\lim_{n\to\infty}s_n(x)=f(x)$ for any $x\in\ooint{-\pi,\pi}$ such that
$f(x)=\lim_{a\to\infty}\Bover1{\pi}\int_{-\pi}^{\pi}
  \Bover{\sin(x-t)a}{x-t}f(t)dt$, which is almost every
$x\in\ocint{-\pi,\pi}$.
}%end of proof of 286V

\cmmnt{
\vleader{108pt}{286W}{Glossary} The following special notations are
used in more than one paragraph of this section:

\def\vta{$\mu$ for Lebesgue measure on $\Bbb R$.}
\def\vtb{286A:  $f^*$.}
\def\vtc{286C:  $S_{\alpha}f$, $M_{\alpha}f$, $D_{\alpha}f$.}
\def\vtd{286Ea:  $\Cal I$, $Q$, $I_{\sigma}$, $J_{\sigma}$,
  $k_{\sigma}$, $x_{\sigma}$, $y_{\sigma}$, $J^l_{\sigma}$,
  $J^r_{\sigma}$, $y^l_{\sigma}$.}
\def\vte{286Eb:  $\phi$, $\phi_{\sigma}$, $\innerprod{f}{g}$.}
\def\vtf{286Ec:  $w$, $w_{\sigma}$.}
\def\vtg{286F:  $\le$, $R^+$, $T_{\tau}$.}
\def\vth{286G:  $C_1$, $C_2$, $C_3$, $C_4$.}
\def\vti{286H:  $\mass$, $\Delta_f$, $\energy$.}
\def\vtj{286J:  $C_5$.}
\def\vtk{286K:  $C_6$.}
\def\vtl{286L:  $C_7$.}
\def\vtm{286M:  $C_8$.}
\def\vtn{286N:  $C_9$.}
\def\vto{286O:  $\theta_z$, $\Cal F$.}
\def\vtp{286P:  $Ah$.}
\def\vtq{286Q:  $\theta'_{z\alpha\beta}$.}
\def\vtr{286R:  $\tilde\theta_z$.}
\def\vts{286S:  $\tilde Ah$.}
\def\vtt{286T:  $C_{10}$, $\hat Af$.}
\sparewidth=\pagewidth
\advance\sparewidth by -370pt
\divide\sparewidth by 4

\medskip

\halign{\hskip\sparewidth#\hfil&\hskip\sparewidth#\hfil
&\hskip\sparewidth#\hfil\cr
\vta&\vth&\vto\cr
\vtb&\vti&\vtp\cr
\vtc&\vtj&\vtq\cr
\vtd&\vtk&\vtr\cr
\vte&\vtl&\vts\cr
\vtf&\vtm&\vtt\cr
\vtg&\vtn\cr}
}%end of comment

\exercises{\leader{286X}{Basic exercises (a)}
%\spheader 286Xa
Use 284Oa and 284Xg to shorten part (c) of the proof of 286U.
%286U

\spheader 286Xb Show that if $\sequence{k}{c_k}$ is a sequence of
complex numbers such that $\sum_{k=0}^{\infty}|c_k|^2$ is finite, then
$\sum_{k=0}^{\infty}c_ke^{ikx}$ is defined in $\Bbb C$ for almost all
$x\in\Bbb R$.
%286V

\leader{286Y}{Further exercises (a)}
%\spheader 286Ya
Show that if $f$ is a square-integrable function on
$\BbbR^r$, where $r\ge 2$, then

\Centerline{$g(y)
=\Bover1{(\sqrt{2\pi})^r}
 \lim_{\alpha_1,\ldots,\alpha_r\to-\infty,
  \beta_1,\ldots,\beta_r\to\infty}
 \int_{a}^{b}e^{-iy\dotproduct x}f(x)dx$}

\noindent is defined in $\Bbb C$ for almost every $y\in\BbbR^r$, and
that $g$ represents the Fourier transform of $f$.
}%end of exercises

\endnotes{
\Notesheader{286} This is not the longest single section in this
treatise as a whole, but it is by a substantial margin the longest in
the present volume, and thirty pages of sub-superscripts must tax the
endurance of the most enthusiastic.   You will easily understand why
Carleson's theorem is not usually presented at this level.   But I am
trying in this book to present complete proofs of the principal
theorems, there is no natural place for Carleson's theorem in later
volumes as at present conceived, and it is (just) accessible at this
point;  so I take the space to do it here.

The proof here divides naturally into two halves:  the `combinatorial'
part in 286E-286M, up to the Lacey-Thiele lemma, followed by the
`analytic' part in 286N-286V, in which the averaging process

\Centerline{$\int_1^2\Bover1{\alpha}\lim_{b\to\infty}
\Bover1b\int_0^b\ldots d\beta d\alpha$}

\noindent is used to transform the geometrically coherent, but
analytically irregular, functions $\theta_z$ into the characteristic
functions $\Bover1{\tilde\theta_1(0)}\tilde\theta_z$.   From the
standpoint of ordinary Fourier analysis, this second part is essentially
routine;  there are many paths we could follow, and we have only to take
the ordinary precautions against illegitimate operations.   
\footnote{I ought at this
point to confess that I blundered badly in the 2001 edition of this
volume, and failed to notice my error until it was brought to my attention
by A.Derighetti at the end of 2013.   I hope that the version presented
here is essentially correct.}

Carleson ({\smc Carleson 66}) stated his theorem in the Fourier-series
form of 286V;  but it had long been understood that this was
equiveridical
with the Fourier-transform version in 286U.   There are of course many
ways of extending the theorem.   In particular, there are
corresponding
results for functions in $\eusm L^p$ for any $p>1$, and even for
functions $f$ such that $f\times\ln(1+|f|)\times\ln\ln\ln(16+|f|)$ is
integrable ({\smc Antonov 96}).   The methods here do not seem to
reach so far.   I ought also to remark that if we define $\hat Af$ as in
286T, then there is for every $p>1$ a constant $C$ such that
$\|\hat Af\|_p\le C\|f\|_p$ for every $f\in\eusm L^p_{\Bbb C}$
({\smc Hunt 67}, {\smc Mozzochi 71}, {\smc J{\o}rsboe \& Mejlbro 82},
{\smc Arias de Reyna 02}, {\smc Lacey 05}).

Note that the point of Carleson's theorem, in either form, is that we
take special limits.   In the formulae

\Centerline{$\varhatf(y)
=\Bover1{\sqrt{2\pi}}\lim_{a\to-\infty,b\to\infty}
\int_a^be^{-ixy}f(x)dx$,}

\Centerline{$f(x)=\lim_{n\to\infty}\sum_{-n}^nc_ke^{ikx}$,}

\ifdim\pagewidth>467pt\tenrmstretch{3pt}\fi
\noindent valid almost everywhere for square-integrable functions $f$,
we are not taking the ordinary integral
$\int_{-\infty}^{\infty}e^{-ixy}f(x)dx$ or the unconditional sum
$\sum_{k\in\Bbb Z}c_ke^{ikx}$.   If $f$ is not integrable, or
$\sum_{k=-\infty}^{\infty}|c_k|$ is infinite, these will not be defined
at even one point.
% Weyl's equidistribution theorem
Carleson's theorem makes sense only because we have a natural
preference for particular kinds of improper integral and conditional
sum.   So when we return, in Chapter 44 of Volume 4, to Fourier analysis
on general
topological groups, there will simply be no language in which to express
the theorem, and while versions have been proved for other groups (e.g.,
{\smc Schipp 78}),
they necessarily depend on some structure beyond the simple notion of
`locally compact Hausdorff abelian topological group'.   Even in
$\BbbR^2$, I understand that it is still unknown whether
\tenrmstretch{1.67pt}

\Centerline{$\lim_{a\to\infty}\Bover1{2\pi}\int_{B(\tbf{0},a)}
e^{-iy\dotproduct x}f(x)dx$}

\noindent will be defined a.e.\ for any square-integrable function $f$,
if we use ordinary Euclidean balls $B(\tbf{0},a)$ in place of the
rectangles in 286Ya.
}%end of notes

\frnewpage


