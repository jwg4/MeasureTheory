\frfilename{mt342.tex}
\versiondate{9.7.10}
\copyrightdate{2001}

\def\chaptername{The lifting theorem}
\def\sectionname{Compact measure spaces}

\newsection{342}

The next three sections amount to an extended parenthesis, showing how
the Lifting Theorem can be used to attack one of the fundamental
problems of
measure theory:  the representation of Boolean homomorphisms between
measure algebras by functions between appropriate measure spaces.   This
section prepares for the main idea by introducing the class of `locally
compact' measures (342Ad), with the associated concepts of `compact' and
`perfect' measures (342Ac, 342K).   These depend on the notions of
`inner regularity'
(342Aa, 342B) and `compact class' (342Ab, 342D).   I list the basic
permanence properties for compact and locally compact measures
(342G-342I)
and mention some of the compact measures which we have already seen
(342J). Concerning perfect measures, I content myself with the proof
that a locally
compact measure is perfect (342L).   I end the section with two examples
(342M, 342N).

\leader{342A}{Definitions (a)} Let $(X,\Sigma,\mu)$ be a measure space.
If $\Cal K\subseteq\Cal PX$, I will say that $\mu$ is {\bf inner
regular} with respect to $\Cal K$ if

\Centerline{$\mu E=\sup\{\mu K:K\in\Cal K\cap\Sigma$, $K\subseteq E\}$}

\noindent for every $E\in\Sigma$.

\cmmnt{Of course $\mu$ is inner regular with respect to $\Cal K$ iff
it is inner regular with respect to $\Cal K\cap\Sigma$.}

\spheader 342Ab A family $\Cal K$ of sets is a {\bf compact class} if
$\bigcap\Cal K'\ne\emptyset$ whenever $\Cal K'\subseteq\Cal K$ has the
finite intersection property.

Note that any subset of a compact class is again a compact class.
\cmmnt{(In particular, it is convenient to allow the empty set as a
compact class.)}

\spheader 342Ac A measure space $(X,\Sigma,\mu)$, or a measure $\mu$, is
{\bf compact} if $\mu$ is inner regular with respect to some compact
class of subsets of $X$.

\cmmnt{Allowing $\emptyset$ as a compact class, and interpreting
$\sup\emptyset$ as $0$ in (a) above,} $\mu$ is a compact
measure whenever $\mu X=0$.

\spheader 342Ad A measure space $(X,\Sigma,\mu)$, or a measure $\mu$, is
{\bf locally compact} if the subspace measure $\mu_E$ is compact
whenever $E\in\Sigma$ and $\mu E<\infty$.

\cmmnt{\medskip

\noindent{\bf Remark} I ought to point out that the original
definitions of `compact class' and `compact measure' ({\smc Marczewski
53}) correspond to what I will call `countably compact class' and
`countably
compact measure' in Volume 4.   For another variation on the concept
of `compact class' see condition ($\beta$) in 343B(ii)-(iii).

For examples of compact measure spaces see 342J and 342Xf.
}%end of comment

\leader{342B}{}\cmmnt{ I prepare the ground with some straightforward
lemmas.

\medskip

\noindent}{\bf Lemma} Let $(X,\Sigma,\mu)$ be a measure space, and
$\Cal K\subseteq\Sigma$ a set such that whenever $E\in\Sigma$ and
$\mu E>0$ there is a $K\in\Cal K$ such that $K\subseteq E$ and
$\mu K>0$.   Let $E\in\Sigma$.

(a) There is a countable disjoint set $\Cal K_1\subseteq\Cal K$ such
that $K\subseteq E$ for every $K\in\Cal K_1$ and
$\mu(\bigcup\Cal K_1)=\mu E$.

(b) If $\mu E<\infty$ then $\mu(E\setminus\bigcup\Cal K_1)=0$.

(c) In any case, there is for any $\gamma<\mu E$ a finite
disjoint $\Cal K_0\subseteq\Cal K$ such that $K\subseteq E$ for every
$K\in\Cal K_0$ and $\mu(\bigcup\Cal K_0)\ge\gamma$.

\proof{ Set
$\Cal K'=\{K:K\in\Cal K,\,K\subseteq E,\,\mu K>0\}$.   Let
$\Cal K^*$ be a maximal disjoint subfamily of $\Cal K'$.   If $\Cal K^*$
is uncountable, then there is some $n\in\Bbb N$ such that
$\{K:K\in\Cal K^*,\,\mu K\ge 2^{-n}\}$ is infinite, so that there is a
countable $\Cal K_1\subseteq\Cal K^*$ such that
$\mu(\bigcup\Cal K_1)=\infty=\mu E$.

If $\Cal K^*$ is countable, set $\Cal K_1=\Cal K^*$.   Then
$F=\bigcup\Cal K_1$ is measurable, and
$F\subseteq E$.   Moreover, there is no member of $\Cal K'$ disjoint
from $F$;  but this means that $E\setminus F$ must be negligible.   So
$\mu F=\mu E$, and (a) is true.   Now (b) and (c) follow at once,
because

\Centerline{$\mu(\bigcup\Cal K_1)
=\sup\{\mu(\bigcup\Cal K_0):\Cal K_0\subseteq\Cal K_1$ is finite$\}$.}
}%end of proof of 342B

\cmmnt{\medskip

\noindent{\bf Remark} This lemma can be thought of as more versions
of the principle of exhaustion;  compare 215A.
}

\leader{342C}{Corollary} Let $(X,\Sigma,\mu)$ be a measure space and
$\Cal K\subseteq\Cal PX$ a family of sets such that ($\alpha$)
$K\cup K'\in\Cal K$
whenever $K$, $K'\in\Cal K$ and $K\cap K'=\emptyset$ ($\beta$) whenever
$E\in\Sigma$ and $\mu E>0$, there is a $K\in\Cal K\cap\Sigma$ such that
$K\subseteq E$ and $\mu K>0$.   Then $\mu$ is inner regular with respect
to $\Cal K$.

\proof{ Apply 342Bc to $\Cal K\cap\Sigma$.
}

\vleader{60pt}{342D}{Lemma} Let $X$ be a set and $\Cal K$ a family of subsets
of $X$.

(a) The following are equiveridical:

\quad (i) $\Cal K$ is a compact class;

\quad (ii) there is a topology $\frak T$ on $X$ such that $X$ is compact
and every member of $\Cal K$ is a closed set for $\frak T$.

(b) If $\Cal K$ is a compact class, so are the families
$\Cal K_1=\{K_0\cup\ldots\cup K_n:K_0,\ldots,K_n\in\Cal K\}$ and
$\Cal K_2=\{\bigcap\Cal K':\emptyset\ne\Cal K'\subseteq\Cal K\}$.

\proof{{\bf (a)(i)$\Rightarrow$(ii)} Let $\frak T$ be the topology
generated
by $\{X\setminus K:K\in\Cal K\}$.   Then of course every member of
$\Cal K$ is closed for $\frak T$.   Let $\Cal F$ be an ultrafilter on
$X$.   Then $\Cal K\cap\Cal F$ has the finite intersection property;
because $\Cal K$ is a compact class, it has
non-empty intersection;  take $x\in X\cap\bigcap(\Cal K\cap\Cal F)$.   The
family

\Centerline{$\{G:G\subseteq X$, either $G\in\Cal F$ or $x\notin G\}$}

\noindent is easily seen to be a topology on $X$, and contains
$X\setminus K$ for every $K\in\Cal K$ (because if
$X\setminus K\notin\Cal F$ then
$K\in\Cal F$ and $x\in K$), so includes $\frak T$;  but this just means
that every $\frak T$-open set containing $x$ belongs to $\Cal F$, that
is, that $\Cal F\to x$.   As $\Cal F$ is arbitrary, $X$ is compact for
$\frak T$ (2A3R).

\medskip

\quad{\bf (ii)$\Rightarrow$(i)} Use 3A3Da.

\medskip

{\bf (b)} Let $\frak T$ be a topology on $X$ such that $X$ is compact
and every member of $\Cal K$ is closed for $\frak T$;  then the same is
true of every member of $\Cal K_1$ or $\Cal K_2$.
}%end of proof of 342D

\leader{342E}{Corollary} Suppose that $(X,\Sigma,\mu)$ is a measure
space and that $\Cal K$ is a compact class such that whenever
$E\in\Sigma$ and
$\mu E>0$ there is a $K\in\Cal K\cap\Sigma$ such that $K\subseteq E$ and
$\mu K>0$.   Then $\mu$ is compact.

\proof{ Set
$\Cal K_1=\{K_0\cup\ldots\cup K_n:K_0,\ldots,K_n\in\Cal K\}$.
By 342Db, $\Cal K_1$ is a compact class, and by 342C $\mu$ is inner
regular with respect to $\Cal K_1$.
}%end of proof of 342E

\leader{342F}{Corollary} A measure space $(X,\Sigma,\mu)$ is compact iff
there is a topology on $X$ such that $X$ is compact and $\mu$ is inner
regular with respect to the closed sets.

\proof{{\bf (a)} If $\mu$ is inner regular with respect to a compact
class $\Cal K$, then there is a compact topology on $X$ such that every
member of $\Cal K$ is closed (342Da);
now the family $\Cal F$ of closed sets includes
$\Cal K$, so $\mu$ is also inner regular with respect to $\Cal F$.

\medskip

{\bf (b)} If there is a compact topology on $X$ such that $\mu$ is inner
regular with respect to the family $\Cal K$ of closed sets, then this is
a compact class, so $\mu$ is a compact measure.
}%end of proof of 342F

\vleader{72pt}{342G}{}\cmmnt{ Now I look at the standard questions
concerning
preservation of the properties of `compactness' or `local compactness'
under the usual manipulations.

\medskip

\noindent}{\bf Proposition}
(a) Any measurable subspace of a compact measure space is compact.

(b) The completion and c.l.d.\ version of a compact measure space are
compact.

(c) A semi-finite measure space is compact iff its completion is compact
iff its c.l.d.\ version is compact.

(d) The direct sum of a family of compact measure spaces is compact.

(e) The c.l.d.\ product of two compact measure spaces is compact.

(f) The product of any family of compact probability spaces is compact.

\proof{{\bf (a)} Let $(X,\Sigma,\mu)$ be a compact measure space, and
$E\in\Sigma$.   If $\Cal K$ is a compact class such that $\mu$ is inner
regular with respect to $\Cal K$, then $\Cal K_E=\Cal K\cap\Cal PE$ is a
compact class (just because it is a subset of $\Cal K$) and the subspace
measure $\mu_E$ is inner regular with respect to $\Cal K_E$.

\medskip

{\bf (b)} Let $(X,\Sigma,\mu)$ be a compact measure space.   Write
$(X,\check\Sigma,\check\mu)$ for {\it either} the completion {\it or}
the c.l.d.\ version of $(X,\Sigma,\mu)$.   Let $\Cal K\subseteq\Cal PX$
be a compact class such that $\mu$ is inner regular with respect to
$\Cal K$. Then $\check\mu$ also is inner regular with respect to
$\Cal K$.   \Prf\ If
$E\in\check\Sigma$ and $\gamma<\check\mu E$ there is an $E'\in\Sigma$
such that $E'\subseteq E$ and $\mu E'>\gamma$;  if $\check\mu$ is the
c.l.d.\
version of $\mu$, we may take $\mu E'$ to be finite.   There is a
$K\in\Cal K\cap\Sigma$ such that $K\subseteq E'$ and $\mu K\ge\gamma$.
Now $\check\mu K=\mu K\ge\gamma$ and $K\subseteq E$ and
$K\in\Cal K\cap\check\Sigma$.\ \Qed

\medskip

{\bf (c)} Now suppose that $(X,\Sigma,\mu)$ is semi-finite;  again write
$(X,\check\Sigma,\check\mu)$ for {\it either} its completion {\it or}
its c.l.d.\ version.   We already know that if $\mu$ is compact, so is
$\check\mu$.   If $\check\mu$ is compact, let $\Cal K\subseteq\Cal PX$
be a compact class such that $\check\mu$ is inner regular with respect
to $\Cal K$.   Set
$\Cal K^*=\{\bigcap\Cal K':\emptyset\ne\Cal K'\subseteq\Cal K\}$;
then $\Cal K^*$ is a compact class (342Db).   Now $\mu$ is inner regular
with respect to $\Cal K^*$.   \Prf\ Take $E\in\Sigma$ and
$\gamma<\mu E$.
Choose $\sequencen{E_n}$, $\sequencen{K_n}$ as follows.   Because $\mu$
is semi-finite, there is an $E_0\subseteq E$ such that $E_0\in\Sigma$
and $\gamma<\mu E_0<\infty$.   Given $E_n\in\Sigma$ such that
$\mu E_n>\gamma$,
there is a $K_n\in\Cal K\cap\check\Sigma$ such that $K_n\subseteq E_n$
and $\check\mu K_n>\gamma$.   Now there is an $E_{n+1}\in\Sigma$ such
that $E_{n+1}\subseteq K_n$ and $\mu E_{n+1}>\gamma$.   Continue.   On
completing the induction, set
$K=\bigcap_{n\in\Bbb N}K_n=\bigcap_{n\in\Bbb N}E_n$, so that
$K\in\Cal K^*\cap\Sigma$ and $K\subseteq E$ and
$\mu K=\lim_{n\to\infty}\mu E_n\ge\gamma$.   As $E$ and $\gamma$ are
arbitrary, $\mu$ is inner regular with
respect to $\Cal K^*$.\ \QeD\   As $\Cal K^*$ is a compact class, $\mu$
is a compact measure.

\medskip

{\bf (d)} Let $\langle(X_i,\Sigma_i,\mu_i)\rangle_{i\in I}$ be a family
of compact measure spaces, with direct sum $(X,\Sigma,\mu)$.   We may
suppose that each $X_i$ is actually a subset of $X$, with $\mu_i$ the
subspace measure.   For each $i\in I$ let $\Cal K_i\subseteq\Cal PX_i$
be a compact class such that $\mu_i$ is inner regular with respect to
$\Cal K_i$.   Then $\Cal K=\bigcup_{i\in I}\Cal K_i$ is a compact class,
for if $\Cal K'\subseteq\Cal K$ has the finite intersection property,
then $\Cal K'\subseteq\Cal K_i$ for some $i$, so has non-empty
intersection.   Now if
$E\in\Sigma$ and $\mu E>0$ there is some $i\in I$ such that
$\mu_i(E\cap X_i)>0$, and we can find a
$K\in\Cal K_i\cap\Sigma_i\subseteq\Cal K\cap\Sigma$ such that
$K\subseteq E\cap X_i$ and $\mu_iK>0$, in which
case $\mu K>0$.   By 342E, $\mu$ is compact.

\medskip

{\bf (e)} Let $(X,\Sigma,\mu)$ and $(Y,\Tau,\nu)$ be two compact measure
spaces, with c.l.d.\ product measure $(X\times Y,\Lambda,\lambda)$.
Let $\frak T$, $\frak S$ be topologies on $X$, $Y$ respectively such
that $X$
and $Y$ are compact spaces and $\mu$, $\nu$ are inner regular with
respect to the closed sets.   Then the product topology on $X\times Y$
is compact (3A3J).

The point is that $\lambda$ is inner regular with respect to the family
$\Cal K$ of closed subsets of $X\times Y$.   \Prf\ Suppose that
$W\in\Lambda$ and $\lambda W>\gamma$.   Then there are $E\in\Sigma$,
$F\in\Tau$ such that $\mu E<\infty$, $\nu F<\infty$ and
$\lambda(W\cap(E\times F))>\gamma$ (251F).   Now
there are sequences $\sequencen{E_n}$, $\sequencen{F_n}$ in $\Sigma$,
$\Tau$ respectively such that

\Centerline{$(E\times F)\setminus W
\subseteq\bigcup_{n\in\Bbb N}E_n\times F_n$,}

\Centerline{$\sum_{n=0}^{\infty}\mu E_n\cdot\nu F_n
<\lambda((E\times F)\setminus W)
  +\lambda((E\times F)\cap W)-\gamma
=\lambda(E\times F)-\gamma$}

\noindent (251C).   Set

\Centerline{$W'
=(E\times F)\setminus\bigcup_{n\in\Bbb N}E_n\times F_n
=\bigcap_{n\in\Bbb N}((E\times(F\setminus F_n))
  \cup((E\setminus E_n)\times F))$.}

\noindent Then $W'\subseteq W$, and

\Centerline{$\lambda((E\times F)\setminus W')
\le\lambda(\bigcup_{n\in\Bbb N}E_n\times F_n)
\le\sum_{n=0}^{\infty}\mu E_n\cdot\nu F_n
<\lambda(E\times F)-\gamma$,}

\noindent so $\lambda W'>\gamma$.

Set $\epsilon=\bover14(\lambda W'-\gamma)/(1+\mu E+\mu F)$.   For each
$n$, we can find
closed measurable sets $K_n$, $K'_n\subseteq X$ and $L_n$,
$L'_n\subseteq Y$ such that

\Centerline{$K_n\subseteq E$,
\quad$\mu(E\setminus K_n)\le 2^{-n}\epsilon$,}

\Centerline{$L'_n\subseteq F\setminus F_n$,
\quad$\nu((F\setminus F_n)\setminus L'_n)\le 2^{-n}\epsilon$,}

\Centerline{$K'_n\subseteq E\setminus E_n$,
\quad$\mu((E\setminus E_n)\setminus K'_n)\le 2^{-n}\epsilon$,}

\Centerline{$L_n\subseteq F$,
\quad$\nu(F\setminus L_n)\le 2^{-n}\epsilon$.}

\noindent Set

\Centerline{$V
=\bigcap_{n\in\Bbb N}(K_n\times L'_n)\cup(K'_n\times L_n)
\subseteq W'\subseteq W$.}

\noindent   Now

$$\eqalign{W'\setminus V
&\subseteq\bigcup_{n\in\Bbb N}((E\setminus K_n)\times F)
   \cup(E\times((F\setminus F_n)\setminus L'_n))\cr
&\qquad\qquad\cup(((E\setminus E_n)\setminus K'_n)\times F)
   \cup(E\times(F\setminus L_n)),\cr}$$

\noindent so

$$\eqalign{\lambda(W'\setminus V)
&\le\sum_{n=0}^{\infty}\mu(E\setminus K_n)\cdot\nu F
   +\mu E\cdot\nu((F\setminus F_n)\setminus L'_n)\cr
&\qquad\qquad+\mu((E\setminus E_n)\setminus K'_n)\cdot\nu F
   +\mu E\cdot\nu(F\setminus L_n)\cr
&\le\sum_{n=0}^{\infty}2^{-n}\epsilon(2\mu E+2\mu F)
\le\lambda W'-\gamma,\cr}$$

\noindent and $\lambda V\ge\gamma$.   But $V$ is a countable
intersection of
finite unions of products of closed measurable sets, so is itself a
closed measurable set, and belongs to $\Cal K\cap\Lambda$.\ \Qed

Accordingly the product topology on $X\times Y$ witnesses that $\lambda$
is a compact measure.

\medskip

{\bf (f)} The same method works.   In detail:  let
$\langle(X_i,\Sigma_i,\mu_i)\rangle_{i\in I}$ be a family of compact
probability spaces, with product $(X,\Lambda,\lambda)$.   For each $i$,
let $\frak T_i$ be a topology on $X_i$ such that $X_i$ is compact and
$\mu_i$ is inner regular with respect to the closed sets.   Give $X$ the
product topology;  this is compact.   If $W\in\Lambda$ and $\epsilon>0$,
let $\sequencen{C_n}$ be a sequence of measurable cylinders (in the
sense of 254A) such that $X\setminus W\subseteq\bigcup_{n\in\Bbb N}C_n$
and $\sum_{n=0}^{\infty}\lambda C_n\le\lambda(X\setminus W)+\epsilon$.
Express each $C_n$ as $\prod_{i\in I}E_{ni}$ where $E_{ni}\in\Sigma_i$
for each $i$
and $J_n=\{i:E_{ni}\ne X_i\}$ is finite.   For $n\in\Bbb N$
set $\epsilon_n=2^{-n}\epsilon/(1+\#(J_n))$.   Choose closed
measurable sets $K_{ni}\subseteq X_i\setminus E_{ni}$ such that
$\mu_i((X_i\setminus E_{ni})\setminus K_{ni})\le\epsilon_n$ whenever
$n\in\Bbb N$ and $i\in J_n$.   For each $n\in\Bbb N$, set

\Centerline{$V_n=\bigcup_{i\in J_n}\{x:x\in X,\,x(i)\in K_{ni}\}$,}

\noindent so that $V_n$ is a closed measurable subset of $X$.   Observe
that

\Centerline{$X\setminus V_n=\{x:x(i)\in X\setminus K_{ni}$ for
$i\in J_n\}$}

\noindent includes $C_n$, and that

\Centerline{$\lambda(X\setminus(V_n\cup C_n))
\le\sum_{i\in J_n}\lambda\{x:x(i)\in X_i\setminus(K_{ni}\cup E_{ni})\}
\le\sum_{i\in J_n}\epsilon_n
\le 2^{-n}\epsilon$.}

Now set $V=\bigcap_{n\in\Bbb N}V_n$;  then $V$ is again a closed
measurable set, and

\Centerline{$X\setminus V
\subseteq\bigcup_{n\in\Bbb N}C_n\cup(X\setminus(C_n\cup V_n))$}

\noindent has measure at most

\Centerline{$\sum_{n=0}^{\infty}\lambda C_n+2^{-n}\epsilon
\le 1-\lambda W+\epsilon+2\epsilon$,}

\noindent so $\lambda V\ge\lambda W-3\epsilon$.
As $W$ and $\epsilon$ are arbitrary, $\lambda$ is inner regular with
respect to the closed sets, and is a compact measure.
}%end of proof of 342G

\leader{342H}{Proposition} (a) A compact measure space is locally
compact.

(b) A strictly localizable locally compact measure space is compact.

(c) Let $(X,\Sigma,\mu)$ be a measure space.   Suppose that whenever
$E\in\Sigma$ and $\mu E>0$ there is an $F\in\Sigma$ such that
$F\subseteq E$,
$\mu F>0$ and the subspace measure on $F$ is compact.   Then $\mu$ is
locally compact.

\proof{{\bf (a)} This is immediate from 342Ga and the definition of
`locally compact' measure space.

\medskip

{\bf (b)} Suppose that $(X,\Sigma,\mu)$ is a strictly localizable
locally compact measure space.   Let $\langle X_i\rangle_{i\in I}$ be a
decomposition of $X$, and for each $i\in I$ let $\mu_i$ be the subspace
measure on $X_i$.   Then $\mu_i$ is compact.   Now $\mu$ can be
identified
with the direct sum of the $\mu_i$, so itself is compact, by 342Gd.

\medskip

{\bf (c)} Write $\Cal F$ for the set of measurable sets $F\subseteq X$
such that the subspace measures $\mu_F$ are compact.   Take $E\in\Sigma$
with $\mu E<\infty$.   By 342Bb, there is a countable disjoint family
$\langle F_i\rangle_{i\in I}$ in $\Cal F$ such that $F_i\subseteq E$ for
each $i$,
and $F'=E\setminus\bigcup_{i\in I}F_i$ is negligible;  now this means
that $F'\in\Cal F$ (342Ac), so we may take it that
$E=\bigcup_{i\in I}F_i$.   In this case $\mu_E$ is isomorphic to the
direct sum of the measures
$\mu_{F_i}$ and is compact.   As $E$ is arbitrary, $\mu$ is locally
compact.
}%end of proof of 342H

\leader{342I}{Proposition} (a) Any measurable subspace of a locally
compact measure space is locally compact.

(b) A measure space is locally compact iff its completion is locally
compact iff its c.l.d.\ version is locally compact.

(c) The direct sum of a family of locally compact measure spaces is
locally compact.

(d) The c.l.d.\ product of two locally compact measure spaces is locally
compact.

\proof{{\bf (a)} Trivial:  if $(X,\Sigma,\mu)$ is locally compact, and
$E\in\Sigma$, and $F\subseteq E$ is a measurable set of finite measure
for the subspace measure on $E$, then $F\in\Sigma$ and $\mu F<\infty$,
so the subspace measure on $F$ is compact.

\medskip

{\bf (b)} Let $(X,\Sigma,\mu)$ be a measure space, and write
$(X,\check\Sigma,\check\mu)$ for {\it either} its completion {\it or}
its c.l.d.\ version.

\medskip

\quad{\bf (i)} Suppose that $\mu$ is locally compact, and that
$\check\mu F<\infty$.   Then there is an $E\in\Sigma$ such that
$E\subseteq F$ and $\mu E=\check\mu F$.   Let $\mu_E$ be the subspace
measure on $E$ induced by the
measure $\mu$;  then we are assuming that $\mu_E$ is compact.   Let
$\Cal K\subseteq\Cal PE$ be a compact class such that $\mu_E$ is inner
regular with respect to $\Cal K$.   Then, as in the proof of 342Gb, the
subspace measure $\check\mu_F$ on $F$ induced by $\check\mu$ is also
inner regular
with respect to $\Cal K$, so $\check\mu_F$ is compact;  as $F$ is
arbitrary, $\check\mu$ is locally compact.

\medskip

\quad{\bf (ii)} Now suppose that $\check\mu$ is locally compact, and
that $\mu E<\infty$.   Then the subspace measure $\check\mu_E$ is
compact.   But
this is just the completion of the subspace measure $\mu_E$, so $\mu_E$
is compact, by 342Gc;  as $E$ is arbitrary, $\mu$ is locally compact.

\medskip

{\bf (c)} Put (a) and 342Hc together.

\medskip

{\bf (d)} Let $(X,\Sigma,\mu)$ and $(Y,\Tau,\nu)$ be locally compact
measure spaces, with product $(X\times Y,\Lambda,\lambda)$.   If
$W\in\Lambda$ and
$\lambda W>0$, there are $E\in\Sigma$, $F\in\Tau$ such that
$\mu E<\infty$,
$\nu F<\infty$ and $\lambda(W\cap(E\times F))>0$.   Now the subspace
measure $\lambda_{E\times F}$ induced by $\lambda$ on $E\times F$ is
just the product of the subspace measures (251Q(ii-$\alpha$)), so is
compact, and the subspace measure $\lambda_{W\cap(E\times F)}$ is
therefore again compact, by
342Ga.   By 342Hc, this is enough to show that $\lambda$ is locally
compact.
}%end of proof of 342I

\leader{342J}{Examples} \cmmnt{It is time I listed some examples of
compact measure spaces.

\medskip

}{\bf (a)} Lebesgue measure on $\BbbR^r$ is
compact.   \prooflet{(Let $\Cal K$ be the family of subsets of $\BbbR^r$
which are compact for the usual topology.   By 134Fb, Lebesgue measure
is inner regular with respect to $\Cal K$.)}

\spheader 342Jb Similarly, any Radon measure on
$\BbbR^r$\cmmnt{ (256A)} is compact.

\spheader 342Jc If $(\frak A,\bar\mu)$ is any semi-finite measure
algebra, the standard measure\cmmnt{ $\nu$} on its Stone space $Z$ is 
compact.   \prooflet{(By
322Ra, $\nu$ is inner regular with respect to the family of
open-and-closed subsets
of $Z$, which are all compact for the standard topology of $Z$, so form
a compact class.)}

\spheader 342Jd The usual measure on $\{0,1\}^I$ is compact, for any set
$I$.   \prooflet{(It is obvious that the usual measure on $\{0,1\}$ is
compact;  now use 342Gf.)}

\cmmnt{\medskip

\noindent{\bf Remark} Actually all these measures are `Radon' in the sense
of Volume 4.}

\leader{342K}{}\cmmnt{ One of the most important properties of
(locally) compact measure spaces has been studied under the following
name.

\medskip

\noindent}{\bf Definition} Let $(X,\Sigma,\mu)$ be a measure space.
Then $(X,\Sigma,\mu)$, or $\mu$, is {\bf perfect} if whenever
$f:X\to\Bbb R$ is measurable, $E\in\Sigma$ and $\mu E>0$, then there
is a compact set $K\subseteq f[E]$ such that $\mu f^{-1}[K]>0$.

\leader{342L}{Theorem} A semi-finite locally compact measure space is
perfect.

\proof{ Let $(X,\Sigma,\mu)$ be a semi-finite locally compact measure
space, $f:X\to\Bbb R$ a measurable function, and $E\in\Sigma$ a set of
non-zero measure.   Because $\mu$ is semi-finite, there is an
$F\in\Sigma$ such that $F\subseteq E$ and $0<\mu F<\infty$.   Now the
subspace measure $\mu_F$ is compact;  let $\frak T$ be a topology on $F$
such that $F$ is compact and $\mu_F$ is inner regular with respect to
the family $\Cal K$ of closed sets for $\frak T$.

Let $\langle\epsilon_q\rangle_{q\in\Bbb Q}$ be a family of strictly
positive real numbers such that
$\sum_{q\in\Bbb Q}\epsilon_q<\bover12\mu F$.   (For
instance, you could set $\epsilon_{q(n)}=2^{-n-3}\mu F$ where
$\sequencen{q(n)}$ is an enumeration of $\Bbb Q$.)   For each
$q\in\Bbb Q$,
set $E_q=\{x:x\in F,\,f(x)\le q\}$, $E'_q=\{x:x\in F,\,f(x)>q\}$, and
choose $K_q$, $K'_q\in\Cal K\cap\Sigma$ such that $K_q\subseteq E_q$,
$K'_q\subseteq E'_q$, $\mu(E_q\setminus K_q)\le\epsilon_q$ and
$\mu(E'_q\setminus K'_q)\le\epsilon_q$.   Then
$K=\bigcap_{q\in\Bbb Q}(K_q\cup K'_q)\in\Cal K\cap\Sigma$,
$K\subseteq F$ and

\Centerline{$\mu(F\setminus K)
\le\sum_{q\in\Bbb Q}\mu(E_q\setminus K_q)+\mu(E'_q\setminus K'_q)
<\mu F$,}

\noindent so $\mu K>0$.

The point is that $f\restr K$ is continuous.   \Prf\ For any
$q\in\Bbb Q$, $\{x:x\in K,\,f(x)\le q\}=K\cap K_q$ and
$\{x:x\in K,\,f(x)>q\}=K\cap K'_q$.
If $H\subseteq\Bbb R$ is open and $x\in K\cap f^{-1}[H]$, take $q$,
$q'\in\Bbb Q$ such that $f(x)\in\ocint{q,q'}\subseteq H$;  then
$G=K\setminus(K_q\cup K'_{q'})$ is a relatively open subset of $K$
containing $x$ and included in $f^{-1}[H]$.   Thus $K\cap f^{-1}[H]$ is
relatively open in $K$;  as $H$ is arbitrary, $f\restr K$ is continuous.\
\Qed

Accordingly $f[K]$ is a continuous image of a compact set, therefore
compact;   it is
a subset of $f[E]$, and $\mu f^{-1}[f[K]]\ge\mu K>0$.   As $f$ and $E$
are arbitrary, $\mu$ is perfect.
}%end of proof of 342L

\leader{342M}{}\cmmnt{ I ought to give examples to distinguish between
the concepts introduced here, partly on general principles, but also
because it
is not obvious that the concept of `locally compact' measure space is
worth spending time on at all.   It is easy to distinguish between
`perfect' and
`(locally) compact';  `locally compact' and `compact' are harder to
separate.

\medskip

\noindent}{\bf Example} Let $X$ be an uncountable set and $\mu$ the
countable-cocountable measure on $X$\cmmnt{ (211R)}.   Then $\mu$ is
perfect but not compact or locally compact.

\proof{{\bf (a)} If $f:X\to\Bbb R$ is measurable and $E\subseteq X$ is
measurable, with measure greater than $0$, set
$A=\{\alpha:\alpha\in\Bbb R,\,\{x:x\in X,\,f(x)\le\alpha\}$ is
negligible$\}$.   Then $\alpha\in A$
whenever $\alpha\le\beta\in A$.   Since
$X=\bigcup_{n\in\Bbb N}\{x:f(x)\le n\}$, there is some $n$ such that
$n\notin A$, in which case $A$ is bounded
above by $n$.   Also there is some $m\in\Bbb N$ such that
$\{x:f(x)>-m\}$ is non-negligible, in which case it must be
conegligible, and $-m\in A$, so $A$ is non-empty.   Accordingly
$\gamma=\sup A$ is defined in $\Bbb R$.   Now for any $k\in\Bbb N$,
$\{x:f(x)\le \gamma-2^{-k}\}$ is negligible, so $\{x:f(x)<\gamma\}$ is
negligible.   Also, for any $k$, $\{x:f(x)\le\gamma+2^{-k}\}$ is
non-negligible, so $\{x:f(x)>\gamma+2^{-k}\}$ must be negligible;
accordingly, $\{x:f(x)>\gamma\}$ is negligible.   But this means that
$\{x:f(x)=\gamma\}$ is conegligible and has measure $1$.   Thus we have
a compact set
$K=\{\gamma\}$ such that $\mu f^{-1}[K]=1$, and $\gamma$ must belong to
$f[E]$.   As $f$ and $E$ are arbitrary, $\mu$ is perfect.

\medskip

{\bf (b)} $\mu$ is not compact.   \Prf\Quer\ Suppose, if possible, that
$\Cal K\subseteq\Cal PX$ is a compact class such that $\mu$ is inner
regular with respect to $\Cal K$.   Then for every $x\in X$ there is a
measurable set $K_x\in\Cal K$ such that $K_x\subseteq X\setminus\{x\}$
and $\mu K_x>0$, that is, $K_x$ is conegligible.   But this means that
$\{K_x:x\in X\}$ must have the finite intersection property;  as it also
has empty intersection, $\Cal K$ cannot be a compact class.\ \Bang\Qed

\medskip

{\bf (c)} Because $\mu$ is totally finite, it cannot be locally compact.
}%end of proof of 342M

\cmmnt{\medskip

\noindent{\bf Remark} See also 342X(n-viii).}

\leader{*342N}{Example} There is a complete locally determined
localizable locally compact measure space which is not compact.

\proof{{\bf (a)} I refer to the example of 216E.   In that construction,
we have a set $I$ and a family $\langle x_{\gamma}\rangle_{\gamma\in C}$
in $X=\{0,1\}^I$ such that for every $D\subseteq C$ there is an $i\in I$
such that $D=\{\gamma:x_{\gamma}(i)=1\}$;  moreover, $\#(C)>\frak c$.
The $\sigma$-algebra $\Sigma$ is the family of sets $E\subseteq X$ such
that for
every $\gamma$ there is a countable set $J\subseteq I$ such that
$\{x:x\restr J=x_{\gamma}\restr J\}$ is a subset of either $E$ or
$X\setminus E$;  and for $E\in\Sigma$, $\mu E$ is
$\#(\{\gamma:x_{\gamma}\in E\})$ if this is finite, $\infty$ otherwise.
Note that any subset of
$X$ determined by a countable set of coordinates belongs to $\Sigma$.

For each $\gamma\in C$, let $i_{\gamma}\in I$ be such that
$x_{\gamma}(i_{\gamma})=1$, $x_{\delta}(i_{\gamma})=0$ for
$\delta\ne\gamma$.   (In 216E I took $I$ to be $\Cal PC$, and
$i_{\gamma}$ would be $\{\gamma\}$.)   Set

\Centerline{$Y=\{x:x\in X,\,\{\gamma:\gamma\in C,\,x(i_{\gamma})=1\}$ is
finite$\}$.}

\noindent Give $Y$ its subspace measure $\mu_Y$ with domain $\Sigma_Y$.
Then $\mu_Y$ is complete, locally determined and localizable (214Ie).
Note that $x_{\gamma}\in Y$ for every $\gamma\in C$.

\medskip

{\bf (b)} $\mu_Y$ is locally compact.   \Prf\ Suppose that
$F\in\Sigma_Y$
and $\mu_YF<\infty$.   If $\mu_YF=0$ then surely the subspace measure
$\mu_F$ is compact.   Otherwise, we can express $F$ as $E\cap Y$ where
$E\in\Sigma$ and $\mu E=\mu_YF$.   Then
$D=\{\gamma:x_{\gamma}\in E\}=\{\gamma:x_{\gamma}\in F\}$ is finite.
For $\gamma\in D$ set

\Centerline{$G'_{\gamma}=\{x:x\in X,\,x(i_{\gamma})=1,\,x(i_{\delta})=0$
for
every $\delta\in D\setminus\{\gamma\}\}\in\Sigma$,}

\Centerline{$\Cal K_{\gamma}
=\{K:x_{\gamma}\in K\subseteq F\cap G'_{\gamma}\}$.}

\noindent Then each $\Cal K_{\gamma}$ is a compact class, and members of
different $\Cal K_{\gamma}$'s are disjoint, so
$\Cal K=\bigcup_{\gamma\in D}\Cal K_{\gamma}$ is a compact class.

Now suppose that $H$ belongs to the subpsace $\sigma$-algebra $\Sigma_F$
and $\mu_FH>0$.   Then there is a $\gamma\in D$ such that
$x_{\gamma}\in H$, so
that $H\cap G'_{\gamma}\in\Cal K\cap\Sigma_F$ and
$\mu_F(H\cap G'_{\gamma})>0$.   By 342E, this is enough to show that
$\mu_F$ is compact.
As $F$ is arbitrary, $\mu_Y$ is locally compact.\ \Qed

\medskip

{\bf (c)} $\mu_Y$ is not compact.   \Prf\Quer\ Suppose, if possible,
that $\mu_Y$ is inner regular with respect to a compact class
$\Cal K\subseteq\Cal PY$.   For each $\gamma\in C$ set
$G_{\gamma}=\{x:x\in X,\,x(i_{\gamma})=1\}$, so that
$x_{\gamma}\in G_{\gamma}\in\Sigma$ and
$\mu_Y(G_{\gamma}\cap Y)=1$.   There must
therefore be a $K_{\gamma}\in\Cal K$ such that
$K_{\gamma}\subseteq G_{\gamma}\cap Y$ and $\mu_YK_{\gamma}=1$ (since
$\mu_Y$ takes no value in
$\ooint{0,1}$).   Express $K_{\gamma}$ as $Y\cap E_{\gamma}$, where
$E_{\gamma}\in\Sigma$, and let $J_{\gamma}\subseteq I$ be a countable
set such that

\Centerline{$E_{\gamma}
\supseteq\{x:x\in X,\,x\restr J_{\gamma}=x_{\gamma}\restr J_{\gamma}\}$.}

At this point I call on the full strength of 2A1P.   There is a set
$B\subseteq C$, of cardinal greater than $\frak c$, such that
$x_{\gamma}\restr J_{\gamma}\cap J_{\delta}
=x_{\delta}\restr J_{\gamma}\cap J_{\delta}$ for all $\gamma$,
$\delta\in B$.   But this means that, for any
finite set $D\subseteq B$, we can define $x\in X$ by setting

$$\eqalign{x(i)
&=x_{\alpha}(i)\text{ if }\alpha\in D,\,i\in J_{\alpha},\cr
&=0\text{ if }i\in I\setminus\bigcup_{\alpha\in D}J_{\alpha}.\cr}$$

\noindent It is easy to check that
$\{\gamma:\gamma\in C,\,x(i_{\gamma})=1\}=D$, so that $x\in Y$;  but now

\Centerline{$x\in Y\cap\bigcap_{\alpha\in D}E_{\alpha}
=\bigcap_{\alpha\in D}K_{\alpha}$.}

What this shows is that $\{K_{\alpha}:\alpha\in B\}$ has the finite
intersection property.   It must therefore have non-empty intersection;
say

\Centerline{$y\in\bigcap_{\alpha\in B}K_{\alpha}
\subseteq\bigcap_{\alpha\in B}G_{\alpha}$.}

\noindent But now we have a member $y$ of $Y$ such that
$\{\gamma:y(i_{\gamma})=1\}\supseteq B$ is infinite, contrary to the
definition of $Y$.\ \Bang\Qed
}%end of proof of 342N

\exercises{
\leader{342X}{Basic exercises $\pmb{>}$(a)}
%\spheader 342Xa
Show that a measure space $(X,\Sigma,\mu)$ is semi-finite
iff $\mu$ is inner regular with respect to $\{E:\mu E<\infty\}$.
%342A

\spheader 342Xb Find a proof of 342B based on 215A.
%342B

\spheader 342Xc Let $(X,\Sigma,\mu)$ be a locally compact semi-finite
measure space in which all singleton sets are negligible.   Show that it
is atomless.
%342D

\spheader 342Xd Let $(X,\Sigma,\mu)$ be a measure
space, and $\nu$ an indefinite-integral measure over $\mu$
(234J\footnote{Formerly 2{}34B.}).
Show that $\nu$ is compact, or locally compact, if $\mu$ is.   \Hint{if
$\Cal K$ satisfies the conditions of 342E with respect to $\mu$, then it
satisfies them for $\nu$.}
%342E

\spheader 342Xe Let $f:\Bbb R\to\Bbb R$ be any non-decreasing function,
and $\nu_f$ the corresponding Lebesgue-Stieltjes measure.   Show
that $\nu_f$ is compact.   \Hint{256Xg.}
%342J

\spheader 342Xf Let $\mu$ be Lebesgue measure on $[0,1]$, $\nu$ the
countable-cocountable measure on $[0,1]$, and $\lambda$ their c.l.d.\
product.   Show that $\lambda$ is a compact measure.   \Hint{let
$\Cal K$ be the family of sets $K\times A$ where $A\subseteq[0,1]$ is
cocountable and $K\subseteq A$ is compact.}
%342J

\spheader 342Xg(i) Give an example of a compact probability space
$(X,\Sigma,\mu)$, a set $Y$ and a function $f:X\to Y$ such that the
image
measure $\mu f^{-1}$ is not compact.  (ii) Give an example of a compact
probability space $(X,\Sigma,\mu)$ and a $\sigma$-subalgebra $\Tau$ of
$\Sigma$ such that $(X,\Tau,\mu\restrp\Tau)$ is not compact.
\Hint{342Xf.}
%342Xf, 342J

\spheader 342Xh Let $(X,\Sigma,\mu)$ be a perfect measure space, and
$f:X\to\Bbb R$ a measurable function.   Show that the image measure
$\mu f^{-1}$ is inner regular with respect to the compact subsets of
$\Bbb R$, so is a compact measure.
%342K

\spheader 342Xi Let $(X,\Sigma,\mu)$ be a $\sigma$-finite
measure space.   Show that it is perfect iff for every measurable
$f:X\to\Bbb R$ there is a Borel set $H\subseteq f[X]$ such that
$f^{-1}[H]$ is conegligible in $X$.   \Hint{342Xh for `only if', 256C
for `if'.}
%342K, 342Xh

\spheader 342Xj Let $(X,\Sigma,\mu)$ be a complete totally finite
perfect measure space and $f:X\to\Bbb R$ a measurable function.   Show
that the image measure $\mu f^{-1}$ is a Radon measure, and is the only
Radon measure on $\Bbb R$ for which $f$ is \imp.   \Hint{256G.}
%342K

\spheader 342Xk Suppose that $(X,\Sigma,\mu)$ is a perfect measure
space.   (i) Show that if $(Y,\Tau,\nu)$ is a measure space, and $f:X\to
Y$ is a function such that $f^{-1}[F]\in\Sigma$ for every $F\in\Tau$ and
$f^{-1}[F]$ is $\mu$-negligible for every $\nu$-negligible set
$F$, then $(Y,\Tau,\nu)$ is perfect.   (ii) Show that if $\Tau$ is a
$\sigma$-subalgebra of $\Sigma$ then $(X,\Tau,\mu\restrp\Tau)$ is
perfect.
%342K

\spheader 342Xl Let $(X,\Sigma,\mu)$ be a perfect measure space such
that $\Sigma$ is the $\sigma$-algebra generated by a sequence of sets.
Show that $\mu$ is compact.   \Hint{if $\Sigma$ is generated by
$\{E_n:n\in\Bbb
N\}$, set $f=\sum_{n=0}^{\infty}3^{-n}\chi E_n$ and consider
$\{f^{-1}[K]:K\subseteq f[X]$ is compact$\}$.}
%342L

\spheader 342Xm Let $(X,\Sigma,\mu)$ be a
semi-finite measure space.   Show that $\mu$
is perfect iff $\mu\restrp\Tau$ is compact for every countably generated
$\sigma$-subalgebra $\Tau$ of $\Sigma$.
%342L

\spheader 342Xn Show that (i) a measurable subspace of a perfect measure
space is perfect (ii) a semi-finite measure space is perfect iff all its
totally finite subspaces are perfect (iii) the direct sum of any family
of perfect measure spaces is perfect (iv) the c.l.d.\ product of two
perfect measure spaces is perfect ({\it hint\/}: put 342Xm and 342Ge
together) (v) the product of any family of
perfect probability spaces is perfect (vi) a measure space is perfect
iff its completion is perfect (vii) the c.l.d.\ version of a perfect
measure space is perfect (viii) any purely atomic measure space is
perfect (ix) an indefinite-integral measure over a perfect measure is
perfect (x)\dvAnew{2010} a sum (234G\Latereditions) of perfect measures
is perfect.
%342L

\spheader 342Xo Let $\mu$ be Lebesgue measure on $\Bbb R$, $A$ a
subset of $\Bbb R$, and $\mu_A$ the subspace measure on $A$.   Show that
$\mu_A$ is compact iff it is perfect iff $A$ is Lebesgue measurable.
\Hint{if $\mu_A$ is perfect, consider the image measure $\mu_Ah^{-1}$ on
$\Bbb R$, where $h(x)=x$ for $x\in A$.}
%342L

\leader{342Y}{Further exercises (a)}
%\spheader 342Ya
Let $U$ be a Banach space such that there is a linear
operator $T:U^{**}\to U$, of norm at most $1$, such that $T\hat u=u$ for
every $u\in U$, writing $\hat u$ for the member of $U^{**}$
corresponding to $u$.   Show that the family of closed balls in $U$ is a
compact class.
%342D

\spheader 342Yb\dvAnew{2010} Give an example of a compact class
$\Cal K$ of subsets of $\Bbb N$
such that there is no compact Hausdorff topology on $\Bbb N$ for which
every member of $\Cal K$ is closed.
%mt34bits

\spheader 342Yc
Show that the space $(X,\Sigma,\mu)$ of 216E and 342N is a
compact measure space.   \Hint{use the usual topology on $X=\{0,1\}^I$.}
%342N

\spheader 342Yd Give an example of a compact complete locally
determined measure space which is not localizable.   \Hint{in 216D, add
a point to each horizontal and vertical section of $X$, so that all the
sections become compact measure spaces.}
%342N
}%end of exercises

\cmmnt{\Notesheader{342} The terminology I find myself using in this
section -- `compact', `locally compact', `perfect'
-- is not entirely satisfactory, in that it risks collision with the
same words applied to
topological spaces.   For the moment, this is not a serious problem;
but when in Volume 4 we come to the systematic analysis of spaces which
have both topologies and measures present, it will be necessary to watch
our language carefully.   Of course there are cases in which a `compact
class' of the sort discussed here can be taken to be the family of
compact sets for
some familiar topology, as in 342Ja-342Jd, but in others this is not so
(see 342Xf);  and even when we have a familiar compact class, the
topology constructed from it by the method of 342Da need not be one we
might expect.
(Consider, for instance, the topology on $\Bbb R$ for which the closed
sets are just the sets which are compact for the usual topology, together
with the set $\Bbb R$ itself.)

I suppose that `compact' and `perfect' measure spaces look reasonably
natural objects to study;  they offer to illuminate one of the basic
properties of Radon measures, the fact that (at least for totally finite
Radon measures on Euclidean space) the image measure of a Radon measure
under a measurable function is again Radon (256G, 342Xj).   Indeed this
was the original impetus for the
study of perfect measures ({\smc Gnedenko \& Kolmogorov 54}, {\smc
Sazonov 66}).   It is not obvious that there is any need to examine
`locally compact' measure spaces, but actually they are the chief
purpose of this
section, since the main theorem of the next section is an alternative
characterization of
semi-finite locally compact measure spaces (343B).   Of course you may
feel that the fact that `locally compact' and `compact' coincide for
strictly localizable spaces (342Hb) excuses you from troubling about the
distinction at first reading.

As with any new classification of measure spaces, it is worth finding
out how the classes of `compact' and `perfect' measure spaces
behave with respect to the standard constructions.   I run through the
basic facts in 342G-342I, 342Xd, 342Xk and 342Xn.   We can also look for
relationships between the new properties and those already studied.
Here, in fact, there is not much to be said;  342N and 342Yd show that
`compactness' is largely independent of the classification in \S211.
However there are
interactions with the concept of `atom' (342Xc, 342Xn(viii)).

I give examples to show that perfect measure spaces need not be locally
compact, and that locally compact measure spaces need not be compact
(342M, 342N).   The standard examples of measure spaces which are not
perfect are non-measurable subspaces (342Xo);  I will return to these in
the next section (343L-343M).

Something which is not important to us at the moment, but is perhaps
worth taking note of, is the following observation.   To determine
whether a
measure space $(X,\Sigma,\mu)$ is compact, we need only the structure
$(X,\Sigma,\Cal N)$, where $\Cal N$ is the $\sigma$-ideal of
negligible sets, since that is all that is referred to in the criterion
of 342E.   The same is true of local compactness, by 342Hc, and of
perfectness,
by the definition in 342K.   Compare 342Xd, 342Xk and 342Xn(ix).

Much of the material of this section will be repeated in Volume 4 as
part of a more systematic analysis of inner regularity.
}%end of notes

\discrpage

