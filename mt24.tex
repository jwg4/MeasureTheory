\frfilename{mt24.tex} 
\versiondate{15.11.13} 
\copyrightdate{1995} 
 
\def\chaptername{Function spaces} 
\def\sectionname{Introduction} 
 
\newchapter{24} 
 
The extraordinary power of Lebesgue's theory of integration is perhaps 
best demonstrated by its ability to provide structures relevant to 
questions quite different from those to which it was at first addressed. 
In this chapter I give the constructions, and elementary properties, of 
some of the fundamental spaces of functional analysis. 
 
I do not feel called on here to justify the study of normed spaces;  if 
you have not met them before, I hope that the introduction here will 
show at least that they offer a basis for a remarkable fusion of algebra 
and analysis.   The fragments of the theory of metric spaces, normed 
spaces and general topology which we shall need are sketched in 
\S\S2A2-2A5.  %2A2 2A3 2A4 2A5 
The principal `function spaces' described in this 
chapter in fact combine three structural elements:  they are 
(infinite-dimensional) linear spaces, they are metric spaces, with 
associated concepts of continuity and convergence, and they are ordered 
spaces, with corresponding notions of supremum and infimum.   The 
interactions between these three types of structure provide an 
inexhaustible wealth of ideas.   Furthermore, many of these ideas are 
directly applicable to a wide variety of problems in more or less 
applied mathematics, particularly in differential and integral 
equations, but more generally in any system with infinitely many degrees 
of freedom. 
 
I have laid out the chapter with sections on $L^0$ (the space of 
equivalence classes of all real-valued measurable functions, in which 
all the other spaces of the chapter are embedded), $L^1$ (equivalence 
classes of integrable functions), $L^{\infty}$ (equivalence classes of 
bounded measurable functions) and $L^p$ (equivalence classes of 
$p$th-power-integrable functions).   While ordinary functional analysis 
gives much more attention to the Banach spaces $L^p$ for  
$1\le p\le\infty$ than to $L^0$, from the special point of view of this book the space $L^0$ is at least as important and interesting as any of the others.   Following these four sections, I return to a study of the 
standard topology on $L^0$, the topology of `convergence in measure' 
(\S245), and then to two linked sections on uniform integrability and 
weak compactness in $L^1$ (\S\S246-247). 
 
There is a technical point here which must never be lost sight of. 
While it is customary and natural to call $L^1$, $L^2$ and the others 
`function spaces', their elements are not in fact functions, but 
equivalence classes of functions.   As you see from the language of the 
preceding paragraph, my practice is to scrupulously maintain the 
distinction;  I give my reasons in the notes to \S241. 
 
\discrpage 
 
