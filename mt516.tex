\frfilename{mt516.tex}
\versiondate{9.10.14}
\copyrightdate{2001}

\def\chaptername{Cardinal functions}
\def\sectionname{Precalibers}

\newsection{516}

In this section I will try to display the elementary connexions between
`precalibers', as defined in 511E, and the cardinal functions we have
looked at so far.
The first step is to generalize the idea of precaliber from partially
ordered sets to supported relations (516A);  the point is that
Galois-Tukey connections give us information on precalibers (516C), and
in particular give quick proofs that partially ordered sets, topological
spaces and Boolean algebras related in the canonical ways explored in
\S514 have many of the same precalibers (516G, 516H, 516M).
Much of the section is taken up with lists of expected facts, but for
some results the hypotheses need to be chosen with care.   I end with a
fundamental theorem on the saturation of product spaces (516T).

\leader{516A}{\bf Definition} If $(A,R,B)$ is a supported relation, a
{\bf precaliber triple} of $(A,R,B)$ is a triple
$\triplepc{\kappa}{\lambda}{\theta}$
where $\kappa$, $\lambda$ and $\theta$ are cardinals and whenever
$\ofamily{\xi}{\kappa}{a_{\xi}}$ is a family in $A$ then there is a set
$\Gamma\in[\kappa]^{\lambda}$ such that $\family{\xi}{\Gamma}{a_{\xi}}$
is $\hbox{$<$}\theta$-linked in the sense of 512Bc\cmmnt{, that is,
for every $I\in[\Gamma]^{<\theta}$ there is a $b\in B$ such that
$(a_{\xi},b)\in R$ for every $\xi\in I$}.   Similarly,
$(\kappa,\lambda,\theta)$ is a precaliber triple of $(A,R,B)$ if
whenever $\ofamily{\xi}{\kappa}{a_{\xi}}$ is a family in $A$ then there
is a set $\Gamma\in[\kappa]^{\lambda}$ such that
$\family{\xi}{\Gamma}{a_{\xi}}$ is $\theta$-linked;  that is,
if $\triplepc{\kappa}{\lambda}{\theta^+}$ is a precaliber triple.

Now $(\kappa,\lambda)$ is a {\bf precaliber pair} of $(A,R,B)$ if
$\triplepc{\kappa}{\lambda}{\omega}$ is a precaliber triple of
$(A,R,B)$, and $\kappa$ is a {\bf precaliber} of $(A,R,B)$ if
$(\kappa,\kappa)$ is a precaliber pair.

\leader{516B}{Elementary remarks}\cmmnt{ I ought perhaps to spell out
the following immediate consequences of the definitions.}
Let $(A,R,B)$ be a supported relation.

\spheader 516Ba If $\kappa'\ge\kappa$, $\lambda'\le\lambda$,
$\theta'\le\theta$ and $\triplepc{\kappa}{\lambda}{\theta}$ is a
precaliber triple of $(A,R,B)$, then
$\triplepc{\kappa'}{\lambda'}{\theta'}$ is a precaliber triple of
$(A,R,B)$.   So if $\kappa'\ge\kappa$, $\lambda'\le\lambda$ and
$(\kappa,\lambda)$ is a precaliber pair of $(A,R,B)$, then
$(\kappa',\lambda')$ is a precaliber pair of $(A,R,B)$.

\spheader 516Bb If $\theta>0$, then $\triplepc{0}{0}{\theta}$ is a
precaliber triple of
$(A,R,B)$ iff $B\ne\emptyset$.   If $A=\emptyset$ then
$\triplepc{\kappa}{\lambda}{\theta}$ is a precaliber triple of $(A,R,B)$
whenever $\kappa\ge 1$.   If $A\ne\emptyset$ and
$A\ne R^{-1}[B]$\cmmnt{, that
is, $\cov(A,R,B)=\infty$}, then the only precaliber triples of $(A,R,B)$
are of the form $\triplepc{\kappa}{0}{\theta}$.   If $A\ne\emptyset$ and
$\triplepc{\kappa}{\lambda}{\theta}$ is a precaliber triple of
$(A,R,B)$, then $\lambda\le\kappa$.   $\cov(A,R,B)=\infty$ iff $1$ is
not a precaliber of $(A,R,B)$.

\spheader 516Bc If $(\kappa,\lambda,\lambda)$ is a precaliber triple of
$(A,R,B)$ then $\triplepc{\kappa}{\lambda}{\theta}$ is a precaliber
triple of $(A,R,B)$ for every $\theta$;  in particular,
$(\kappa,\lambda)$ is a precaliber pair of $(A,R,B)$.

\spheader 516Bd If $\triplepc{\kappa}{\kappa}{\theta}$ is a precaliber
triple of $(A,R,B)$, so is $\triplepc{\cf\kappa}{\cf\kappa}{\theta}$.
\prooflet{\Prf\ If $\cf\kappa=\kappa$ there is nothing to prove.   If
$2\le\kappa<\omega$ and $A$ is empty the result is trivial.   If
$2\le\kappa<\omega$ and $A$ is not empty, then $B$ is not empty, so if
$\theta\le 1$ the result is trivial.   If $2\le\kappa<\omega$ and $A$ is
not empty and $\theta>1$, then $R^{-1}[B]=A$ so
$\triplepc{\cf\kappa}{\cf\kappa}{\theta}=\triplepc{1}{1}{\theta}$ is a
precaliber triple of $(A,R,B)$.

If $\kappa>\cf\kappa$ is infinite, let
$\ofamily{\xi}{\cf\kappa}{\gamma_{\xi}}$ be a strictly increasing family
with supremum $\kappa$.   For $\eta<\kappa$, set
$f(\eta)=\min\{\xi:\eta\le\gamma_{\xi}\}$.   If
$\ofamily{\xi}{\cf\kappa}{a_{\xi}}$ is a family in $A$, set
$a'_{\eta}=a_{f(\eta)}$ for each $\eta<\kappa$.   Then there is a
$\Gamma\in[\kappa]^{\kappa}$ such that
$\family{\eta}{\Gamma}{a'_{\eta}}$ is $\hbox{$<$}\theta$-linked.   Set
$\Gamma'=\{f(\eta):\eta\in\Gamma\}$;  then
$\family{\xi}{\Gamma'}{a_{\xi}}$ is $\hbox{$<$}\theta$-linked.   Also
$\Gamma$ must be cofinal with $\kappa$, so $\Gamma'$ is cofinal with
$\cf\kappa$ and $\#(\Gamma')=\cf\kappa$.   As
$\ofamily{\xi}{\cf\kappa}{a_{\xi}}$ is arbitrary,
$\triplepc{\cf\kappa}{\cf\kappa}{\theta}$ is a precaliber triple of
$(A,R,B)$.\ \Qed

}%end of prooflet
In particular, if $\kappa$ is a precaliber of $(A,R,B)$, so is
$\cf\kappa$.

\leader{516C}{Theorem} Suppose that $(A,R,B)$ and $(C,S,D)$ are
supported relations, and that $(A,R,B)\prGT(C,S,D)$.   Then
$\triplepc{\kappa}{\lambda}{\theta}$ or $(\kappa,\lambda,\theta)$ is a
precaliber triple of $(A,R,B)$
whenever it is a precaliber triple of $(C,S,D)$, so $(\kappa,\lambda)$
is a precaliber pair of $(A,R,B)$ whenever it is a precaliber pair of
$(C,S,D)$, and $\kappa$ is a precaliber of $(A,R,B)$ whenever it is a
precaliber of $(C,S,D)$.

\proof{ Let $(\phi,\psi)$ be a Galois-Tukey connection from $(A,R,B)$
to $(C,R,S)$.   If $\triplepc{\kappa}{\lambda}{\theta}$ is a precaliber
triple of
$(C,S,D)$, and $\ofamily{\xi}{\kappa}{a_{\xi}}$ is a family in $A$, then
there is a set $\Gamma\in[\kappa]^{\lambda}$ such that whenever
$I\in[\Gamma]^{<\theta}$ there is a $d\in D$ such that
$(f(a_{\xi}),d)\in S$ for every $\xi\in I$, and now
$(a_{\xi},g(d))\in R$ for every $\xi\in I$.   Thus
$\triplepc{\kappa}{\lambda}{\theta}$ is a precaliber triple of
$(A,R,B)$.   The results for
precaliber pairs and precalibers follow at once.
}%end of proof of 516C

\leader{516D}{Corollary} If $(A,R,B)\equivGT(C,S,D)$ then $(A,R,B)$ and
$(C,S,D)$ have the same precaliber triples, the same precaliber pairs
and the same precalibers.

\cmmnt{
\leader{516E}{Remark} Because all the definitions in 516A start
from precaliber triples $\triplepc{\kappa}{\lambda}{\theta}$, any
theorem about such precaliber triples is likely to lead at once to
corresponding results concerning precaliber triples
$(\kappa,\lambda,\theta)$, precaliber pairs and precalibers.   In the
rest of this section I shall not always take the space to spell these
out systematically, and when later I wish to use a fact about
precalibers I
may direct you, without comment, to a fact about precaliber triples or
pairs from which it may be deduced.
}%end of comment

\leader{516F}{}\cmmnt{ The next step is to check the connexion between
the definition in 516A and those of \S511.   But this is elementary.

\medskip

\noindent}{\bf Proposition} (a) If $P$ is a partially ordered set,
$\triplepc{\kappa}{\lambda}{\theta}$ or $(\kappa,\lambda,\theta)$ is a
precaliber triple of
$(P,\le,P)$ iff it is an upwards precaliber triple of $P$.

(b) If $\frak A$ is a Boolean algebra, then $\frak A$ and
$(\frak A^+,\Bsupseteqshort,\frak A^+)$ have the same precaliber
triples\cmmnt{, where $\frak A^+=\frak A\setminus\{0\}$}.

(c) If $(X,\frak T)$ is a topological space, then $X$ and
$(\frak T\setminus\{\emptyset\},\supseteq,
\frak T\setminus\{\emptyset\})$ have the same precaliber triples.

\proof{ Read the definitions in 511E and 516A.
}%end of proof of 516F

\leader{516G}{Corollary} Let $(P,\le)$ be a partially ordered set.

(a) If $Q$ is a cofinal subset of $P$, then $P$ and $Q$ have the same
upwards precaliber triples.

(b) Let $\frak T^{\uparrow}$ be the up-topology of
$P$\cmmnt{ (definition: 514L)}.    Then
$\triplepc{\kappa}{\lambda}{\theta}$ is an upwards precaliber triple for
$(P,\le)$ iff it is a precaliber triple for $(P,\frak T^{\uparrow})$.

\proof{{\bf (a)} By 513E(d-ii), $(P,\le,P)\equivGT(Q,\le,Q)$.

\medskip

{\bf (b)} By 514Na, $(P,\le,P)
\equivGT(\frak T^{\uparrow}\setminus\{\emptyset\},\supseteq,
\frak T^{\uparrow}\setminus\{\emptyset\})$.
}%end of proof of 516G

\leader{516H}{Corollary} Let $\frak A$ be a Boolean algebra.

(a) If $Z$ is the Stone space of $\frak A$, then $\frak A$ and $Z$ have
the same precaliber triples.

(b) If $\frak B$ is an order-dense subalgebra of $\frak A$, then
$\frak A$ and $\frak B$ have the same precaliber triples.

\proof{{\bf (a)} Write $\frak T$ for the topology of $Z$ and $\Cal E$
for the algebra of open-and-closed sets.   Because $Z$ is
zero-dimensional, $\Cal E^+$ is coinitial with
$\frak T\setminus\{\emptyset\}$, so
$(\frak A^+,\Bsupseteqshort,\frak A^+)
\cong(\Cal E^+,\supseteq,\Cal E^+)$
and $(\frak T\setminus\{\emptyset\},\supseteq,
\frak T\setminus\{\emptyset\})$ have the same precaliber triples, by
516Ga, inverted.

\medskip

{\bf (b)} $\frak B^+$ is coinitial with $\frak A^+$, so we can use the
same idea.
}%end of proof of 516H

\leader{516I}{Corollary} Let $(X,\frak T)$ be a topological space.

(a) If $Y$ is an open subspace of $X$, then every precaliber triple of
$X$ is a precaliber triple of $Y$.

(b) If $Y$ is a dense subspace of $X$, then every precaliber triple of
$X$ is a precaliber triple of $Y$.

(c) If $X$ is regular and $Y$ is a dense subspace of $X$, then $X$ and
$Y$ have the same precaliber triples.

(d) Suppose that $Y$ is a topological space, and that there is a
continuous surjection $f:X\to Y$ such that $\interior f[G]\ne\emptyset$
whenever $G\subseteq X$ is a non-empty open set.   Then every precaliber
triple of $X$ is a precaliber triple of $Y$.

\proof{{\bf (a)} Write $\frak S$ for the topology of $Y$.   For
$H\in\frak S\setminus\{\emptyset\}$, set $\phi(H)=H$;  for
$G\in\frak T\setminus\{\emptyset\}$, set $\psi(G)=G\cap Y$ if this is
non-empty, $G$ otherwise.   Then $(\phi,\psi)$ is a Galois-Tukey
connection from
$(\frak S\setminus\{\emptyset\},\supseteq,
\frak S\setminus\{\emptyset\})$ to
$(\frak T\setminus\{\emptyset\},\supseteq,
\frak T\setminus\{\emptyset\})$, so 516C and 516Fc give the result.

\medskip

{\bf (b)} Again write $\frak S$ for the topology of $Y$.   For
$H\in\frak S\setminus\{\emptyset\}$, set
$\phi(H)=X\setminus\overline{Y\setminus H}$, where the closure is taken
in $X$;  for $G\in\frak T\setminus\{\emptyset\}$, set $\psi(G)=G\cap Y$.
Then $(\phi,\psi)$ is a Galois-Tukey connection from
$(\frak S\setminus\{\emptyset\},\supseteq,
\frak S\setminus\{\emptyset\})$ to
$(\frak T\setminus\{\emptyset\},\supseteq,
\frak T\setminus\{\emptyset\})$, so again we have the result.

\medskip

{\bf (c)} If now $X$ is regular, then for each
$G\in\frak T\setminus\{\emptyset\}$ choose
$V_G\in\frak T\setminus\{\emptyset\}$ such that
$\overline{V}_G\subseteq G$ and set $\psi'(G)=V_G\cap Y$.   Then
$(\psi',\phi)$ is a Galois-Tukey connection from
$(\frak T\setminus\{\emptyset\},\supseteq,
\frak T\setminus\{\emptyset\})$ to
$(\frak S\setminus\{\emptyset\},\supseteq,
\frak S\setminus\{\emptyset\})$, so every precaliber triple of $Y$ is a
precaliber triple of $X$.

\medskip

{\bf (d)} Once more writing $\frak S$ for the topology of $Y$, set
$\phi(H)=f^{-1}[H]$ for every $H\in\frak S\setminus\{\emptyset\}$ and
$\psi(G)=\interior f[G]$ for every $G\in\frak T\setminus\{\emptyset\}$;
then again $(\phi,\psi)$ is a Galois-Tukey connection from
$(\frak S\setminus\{\emptyset\},\supseteq,
\frak S\setminus\{\emptyset\})$ to
$(\frak T\setminus\{\emptyset\},\supseteq,
\frak T\setminus\{\emptyset\})$.
}%end of proof of 516I

\cmmnt{\medskip

\noindent{\bf Remark} For variations on (b) and (d) here, see 516Xh and
516Oa.}

\leader{516J}{}\cmmnt{ Straightforward counting arguments give us
some connexions between precalibers and other cardinal functions, as
follows.

\medskip

\noindent}{\bf Proposition} Let $(A,R,B)$ be a supported relation.

(a) $\sat(A,R,B)$ is the least cardinal $\kappa$, if there is one, such
that $(\kappa,2)$ is a precaliber pair of $(A,R,B)$;  if there is no
such $\kappa$, $\sat(A,R,B)=\infty$.   In particular, if $\kappa\ge 2$
is a precaliber of $(A,R,B)$, then $\kappa\ge\sat(A,R,B)$.

(b) If $\kappa>\max(\omega,\lambda,\link_{<\theta}(A,R,B))$ then
$\triplepc{\kappa}{\lambda^+}{\theta}$ is a precaliber triple of
$(A,R,B)$.   In particular, if
$\kappa>\max(\omega,\lambda,\cov(A,R,B))$ then
$\triplepc{\kappa}{\lambda^+}{\theta}$ is a precaliber triple of
$(A,R,B)$ for every $\theta$.
%, and if
%$\kappa>\max(\omega,\lambda,\link_{<\omega}(A,R,B))$ then
%$(\kappa,\lambda^+)$ is a precaliber pair of $(A,R,B)$.

(c) If $\cf\kappa>\link_{<\theta}(A,R,B)$ then
$\triplepc{\kappa}{\kappa}{\theta}$ is a precaliber triple of $(A,R,B)$.
%In particular, if $\cf\kappa>\link_{<\omega}(A,R,B)$ then $\kappa$ is a
%precaliber of $(A,R,B)$.

\proof{{\bf (a)} If $(\kappa,2)$ is a precaliber pair of $(A,R,B)$, and
$\ofamily{\xi}{\kappa}{a_{\xi}}$ is any family in $A$, then there must
be a $\Gamma\in[\kappa]^2$ such that for every finite $I\subseteq\Gamma$
there is a $b\in B$ such that $(a_{\xi},b)\in R$ for every $\xi\in I$.
But this means that if $\Gamma=\{\xi,\eta\}$ then $\xi$, $\eta$ are
distinct members of $\kappa$ such that, for some $b\in B$, both
$(a_{\xi},b)$ and $(a_{\eta},b)$ belong to $R$.
As $\ofamily{\xi}{\kappa}{a_{\xi}}$ is arbitrary,
$\sat(A,R,B)\le\kappa$.

Conversely, any witness that $(\kappa,2)$ is not a precaliber pair of
$(A,R,B)$ will provide a witness that $\sat(A,R,B)>\kappa$.

Now if $\kappa\ge 2$ is a precaliber of $(A,R,B)$, that is,
$(\kappa,\kappa)$ is a precaliber pair, then $(\kappa,2)$ is a
precaliber pair of $(A,R,B)$, by 516Ba, so $\kappa\ge\sat(A,R,B)$.

\medskip

{\bf (b)} Write $\delta$ for $\link_{<\theta}(A,R,B)$, and let
$\ofamily{\eta}{\delta}{A_{\eta}}$ be a cover of $A$ by
$\hbox{$<$}\theta$-linked sets.   Let $\ofamily{\xi}{\kappa}{a_{\xi}}$
be any family in $A$.   For $\eta<\delta$ set
$C_{\eta}=\{\xi:a_{\xi}\in A_{\eta}\}$;  then
$\kappa=\bigcup_{\eta<\delta}C_{\eta}$ so there must be some
$\eta<\delta$ such that $\#(C_{\eta})>\lambda$.   Now if
$\Gamma\subseteq C_{\eta}$ is a set of size $\lambda^+$,
$\{a_{\xi}:\xi\in\Gamma\}$ is $\hbox{$<$}\theta$-linked in $(A,R,B)$.
As $\ofamily{\xi}{\kappa}{a_{\xi}}$ is arbitrary,
$\triplepc{\kappa}{\lambda^+}{\theta}$ is a precaliber triple of
$(A,R,B)$.

The special case is now elementary, if we remember that
$\link_{<\theta}(A,R,B)\le\cov(A,R,B)$ for every $\theta$ (512Bc).

\medskip

{\bf (c)} If $\link_{<\theta}(A,R,B)=0$ then $A=\emptyset$ and the
result is trivial.   Otherwise, $\cf\kappa\ge\omega$.   Choose $\delta$
and $\ofamily{\eta}{\delta}{C_{\eta}}$ as in (b) above.  Let
$\ofamily{\xi}{\kappa}{a_{\xi}}$ be any family in $A$.   Then there must
be some $\eta<\delta$ such that $\#(C_{\eta})=\kappa$, and
$\{a_{\xi}:\xi\in C_{\eta}\}$ is $\hbox{$<$}\theta$-linked in $(A,R,B)$.
As $\ofamily{\xi}{\kappa}{a_{\xi}}$ is arbitrary,
$\triplepc{\kappa}{\kappa}{\theta}$ is a precaliber triple of $(A,R,B)$.
%Taking $\theta=\omega$, we see that $\kappa$ is a precaliber of
%$(A,R,B)$ if $\cf\kappa>\link_{<\omega}(A,R,B)$.
}%end of proof of 516J

\leader{516K}{}\cmmnt{ For partially ordered sets, we have translations
of the results above, and a further useful fact.

\medskip

\noindent}{\bf Proposition} Let $P$ be a partially ordered set.

(a) $\sat^{\uparrow}(P)$ is the least cardinal $\kappa$ such that
$(\kappa,2)$ is an upwards precaliber pair of $P$.

(b) If $\kappa>\max(\omega,\lambda,\link^{\uparrow}_{<\theta}(P))$ then
$\triplepc{\kappa}{\lambda^+}{\theta}$ is an upwards precaliber triple
of $P$.   In particular, if $\kappa>\max(\omega,\lambda,\cf P)$ then
$\triplepc{\kappa}{\lambda^+}{\theta}$ is an upwards precaliber triple
of $P$ for every $\theta$, and if
$\kappa>\max(\omega,\lambda,\duparrow(P))$ then $(\kappa,\lambda^+)$
is an upwards precaliber pair of $P$.

(c) If $\cf\kappa>\cf P$ then $\triplepc{\kappa}{\kappa}{\theta}$ is an
upwards precaliber triple of $P$ for every $\theta$.   If
$\cf\kappa>\duparrow(P)$ then $\kappa$ is an
up-precaliber of $P$.

(d) If $\sat^{\uparrow}(P)\ge\omega$,
$(\sat^{\uparrow}(P),\omega)$ is an upwards precaliber pair of $P$.

\proof{{\bf (a)-(c)} We need only identify $\cf P$ with
$\cov(P,\le,P)\ge\sup_{\theta}\link_{<\theta}(P,\le,P)$ (512Bc) and
$\duparrow(P)$ with the centering number
$\link_{<\omega}(P,\le\nobreak,P)$, as in 512Ea.

\medskip

{\bf (d)(i)} Set $\kappa=\sat^{\uparrow}(P)$.   By 513Bb, $\kappa$ is a
regular uncountable cardinal.   The first thing to note is that if
$\ofamily{\xi}{\kappa}{p_{\xi}}$ is any family in $P$, then there is a
$\zeta<\kappa$ such that
$\{\xi:\xi<\kappa$,
$p_{\xi}$ and $p_{\zeta}$ are compatible upwards in $P\}$ has cardinal
$\kappa$.   \Prf\Quer\ Otherwise, for each $\zeta<\kappa$ there is an
$\alpha_{\zeta}<\kappa$ such that $p_{\zeta}$ and $p_{\xi}$ are
upwards-incompatible for every $\xi\ge\alpha_{\zeta}$.   Set
$C=\{\xi:\xi<\kappa$, $\alpha_{\eta}\le\xi$ for every $\eta<\xi\}$.   Then
$\#(C)=\kappa$ and $\family{\xi}{C}{p_{\xi}}$ is an up-antichain in $P$,
which is impossible.\ \Bang\Qed

\medskip

\quad{\bf (ii)} Now let $\ofamily{\xi}{\kappa}{p_{\xi}}$ be a
family in $P$.   Choose inductively sets $A_n\in[\kappa]^{\kappa}$,
ordinals $\zeta_n\in A_n$ and
families $\family{\xi}{A_n}{p_{n\xi}}$ in $P$, as follows.   $A_0=\kappa$,
$p_{0\xi}=p_{\xi}$ for each $\xi<\kappa$.   Given
$\family{\xi}{A_n}{p_{n\xi}}$, then by (i) there is a $\zeta_n\in A_n$ such
that

\Centerline{$A_{n+1}=\{\xi:\xi\in A_n$, $\xi\ne\zeta_n$,
$p_{n\xi}$ is compatible upwards with $p_{n,\zeta_n}\}$}

\noindent has cardinal
$\kappa$.   Now, for $\xi\in A_{n+1}$, let $p_{n+1,\xi}$ be an upper bound
of $\{p_{n,\zeta_n},p_{n\xi}\}$;  continue.

At the end of the induction, observe that $\sequencen{p_{n,\zeta_n}}$ is
non-decreasing.   At the same time, we see that $p_{\xi}\le p_{n\xi}$
whenever $n\in\Bbb N$ and $\xi\in A_n$.   So $\{p_{\zeta_n}:n\in\Bbb N\}$
is upwards-centered.   Also the $\zeta_n$ are all different, so
$\Gamma=\{\zeta_n:n\in\Bbb N\}$ is infinite.   As
$\ofamily{\xi}{\kappa}{p_{\xi}}$ is arbitrary,
$(\sat^{\uparrow}(P),\omega)$ is an upwards precaliber pair of $P$.
}%end of proof of 516K

\cmmnt{\medskip

\noindent{\bf Remark} There will be a stronger form of (d) in
517Fa below.
}

\leader{516L}{Corollary} Let $\frak A$ be a Boolean algebra.

(a) $\sat(\frak A)$ is the least cardinal $\kappa$ such
that $(\kappa,2)$ is a precaliber pair of $\frak A$.

(b) If $\kappa>\max(\omega,\lambda,\link_{<\theta}(\frak A))$ then
$\triplepc{\kappa}{\lambda^+}{\theta}$ is a precaliber triple of
$\frak A$.   In particular, if
$\kappa>\max(\omega,\lambda,\pi(\frak A))$ then
$\triplepc{\kappa}{\lambda^+}{\theta}$ is a precaliber triple of
$\frak A$ for every $\theta$, and if
$\kappa>\max(\omega,\lambda,d(\frak A))$ then $(\kappa,\lambda^+)$ is a
precaliber pair of $\frak A$.

(c) If $\cf\kappa>d(\frak A)$ then $\kappa$ is a precaliber
of $\frak A$.

(d) If $\frak A$ is infinite,
$(\sat(\frak A),\omega)$ is a precaliber pair of $\frak A$.

\proof{ Apply 516K, inverted, to $\frak A^+$, recalling that
$\pi(\frak A)=\ci(\frak A^+)$.
}%end of proof  of 516L

\leader{516M}{}\cmmnt{ When we turn to topological spaces, we can
refine the results slightly, using the following elementary facts.

\medskip

\noindent}{\bf Lemma} Let $(X,\frak T)$ be a topological space and
$\RO(X)$ its regular open algebra.   If $\kappa$, $\lambda$ and $\theta$
are cardinals, and $\theta\le\omega$, then the following are
equiveridical:

(i) $\triplepc{\kappa}{\lambda}{\theta}$ is a precaliber triple of
$(X,\frak T)$;

(ii) $\triplepc{\kappa}{\lambda}{\theta}$ is a precaliber triple of
$(\frak T\setminus\{\emptyset\},\ni,X)$;

(iii) $\triplepc{\kappa}{\lambda}{\theta}$ is a precaliber triple of
$\RO(X)$.

\proof{{\bf (a)(i)$\Rightarrow$(ii)} If we set $\phi(G)=G$ and choose a
point $\psi(G)\in G$ for every non-empty open set $G\subseteq X$, then
$(\phi,\psi)$ is a Galois-Tukey connection from
$(\frak T\setminus\{\emptyset\},\ni,X)$ to
$(\frak T\setminus\{\emptyset\},\supseteq,\frak T\setminus\{\emptyset\})$,
so any precaliber triple of the latter is a precaliber triple of the
former.

\medskip

{\bf (b)(ii)$\Rightarrow$(iii)} Assume (ii), and let
$\ofamily{\xi}{\kappa}{G_{\xi}}$ be a family in $\RO(X)^+$.   Then there
is a $\Gamma\in[\kappa]^{\lambda}$ such that $\bigcap_{\xi\in
I}G_{\xi}\ne\emptyset$ for every $I\in[\Gamma]^{<\theta}$.   But in this
case, because $I$ is finite, $\bigcap_{\xi\in I}G_{\xi}$ is a lower
bound for $\{G_{\xi}:\xi\in I\}$ in $\RO(X)^+$.   As
$\ofamily{\xi}{\kappa}{G_{\xi}}$ is arbitrary,
$\triplepc{\kappa}{\lambda}{\theta}$ is a precaliber triple of $\RO(X)$.

\medskip

{\bf (c)(iii)$\Rightarrow$(i)} Assume (iii), and let
$\ofamily{\xi}{\kappa}{G_{\xi}}$ be a family in
$\frak T\setminus\{\emptyset\}$.   Then there is a
$\Gamma\in[\kappa]^{\lambda}$ such that
$\bigcap_{\xi\in I}\interior\overline{G}_{\xi}\ne\emptyset$ for every
$I\in[\Gamma]^{<\theta}$.   But in this case, because $I$ is finite,
$\bigcap_{\xi\in I}G_{\xi}$ is not empty, and is a lower bound for
$\{G_{\xi}:\xi\in I\}$ in $\frak T\setminus\{\emptyset\}$.   As
$\ofamily{\xi}{\kappa}{G_{\xi}}$ is arbitrary,
$\triplepc{\kappa}{\lambda}{\theta}$ is a precaliber triple of
$(X,\frak T)$.
}%end of proof of 516M

\leader{516N}{Corollary} Let $X$ be a topological space.

(a) $\sat(X)$ is the least cardinal $\kappa$ such that
$(\kappa,2)$ is a precaliber pair of $X$.

(b) If $\kappa>\max(\omega,\lambda,d(X))$
then $(\kappa,\lambda^+)$ is a precaliber pair of $X$.

(c) If $\cf\kappa>d(X)$
then $\kappa$ is a precaliber of $X$.

(d) If $\sat(X)$ is infinite, then $(\sat(X),\omega)$ is
a precaliber pair of $X$.

\proof{ Here we need to know that $\sat(X)=\sat(RO(X))$ and
$d(X)\ge d(RO(X))$ (514H(b-i)).
}%end of proof of 516N

\leader{516O}{}\cmmnt{ The idea of 516M leads to further results about
precalibers of topological spaces.

\medskip

\noindent}{\bf Proposition} Let $(X,\frak T)$ be a topological space.

(a) If $Y$ is a continuous image of $X$ and $\theta\le\omega$, then
$\triplepc{\kappa}{\lambda}{\theta}$ is a precaliber triple of $Y$
whenever it is a precaliber triple of $X$.
%In particular, every
%precaliber pair of $X$ is a precaliber pair of $Y$.

(b) Suppose that $X$ is the product of a family $\familyiI{X_i}$ of
topological spaces.   If $\triplepc{\kappa}{\kappa}{\theta}$ is a
precaliber triple
of every $X_i$ and {\it either} $I$ is finite {\it or} $\theta\le\omega$
and $\kappa$ is a regular uncountable cardinal, then
$\triplepc{\kappa}{\kappa}{\theta}$ is a precaliber triple of $X$.
%In particular, a regular uncountable cardinal $\kappa$ is a precaliber
%of $X$ if it is a precaliber of every $X_i$.

\proof{{\bf (a)} Let $f:X\to Y$ be a continuous surjection.   Writing
$\frak S$ for the topology of $Y$, we have a Galois-Tukey connection
$(\phi,f)$ from $(\frak S\setminus\{\emptyset\},\ni,Y)$ to
$(\frak T\setminus\{\emptyset\},\ni,X)$, if we set $\phi(H)=f^{-1}[H]$
for $H\in\frak S\setminus\{\emptyset\}$.   Now if $\theta\le\omega$ and
$\triplepc{\kappa}{\lambda}{\theta}$ is a precaliber triple of
$(X,\frak T)$, it is a precaliber triple of
$(\frak T\setminus\{\emptyset\},\ni,X)$,
$(\frak S\setminus\{\emptyset\},\ni,Y)$ and $(Y,\frak S)$, using 516M
and 516C.

\medskip

{\bf (b)} If $X=\emptyset$ then
$\triplepc{\kappa}{\kappa}{\theta}$ is a precaliber triple of $X$ just
because $X=X_i$ for some $i$;  so let us suppose that $X\ne\emptyset$.

\medskip

\quad{\bf (i)} If $I=\{0,1\}$ then $\triplepc{\kappa}{\kappa}{\theta}$
is a precaliber triple of $X$.   \Prf\ Let
$\ofamily{\xi}{\kappa}{W_{\xi}}$ be a family of non-empty open sets in
$X$.   For each $\xi<\kappa$, let $G_{\xi 0}\subseteq X_0$ and
$G_{\xi 1}\subseteq X_1$ be non-empty open sets such that
$G_{\xi 0}\times G_{\xi 1}\subseteq W_{\xi}$.   Because
$\triplepc{\kappa}{\kappa}{\theta}$ is a precaliber triple of $X_0$,
there is a $\Gamma\in[\kappa]^{\kappa}$ such that
$H_K^{(0)}=\interior(\bigcap_{\xi\in K}G_{\xi 0})$ is non-empty for
every $K\in[\Gamma]^{<\theta}$.   Because
$\triplepc{\kappa}{\kappa}{\theta}$ is a precaliber triple of $X_1$,
there is a $\Delta\in[\Gamma]^{\kappa}$ such that
$H_K^{(1)}=\interior(\bigcap_{\xi\in K}G_{\xi 1})$ is non-empty for
every $K\in[\Delta]^{<\theta}$.   Now
$\bigcap_{\xi\in K}W_{\xi}\supseteq H_K^{(0)}\times H_K^{(1)}$ has
non-empty interior for every
$K\in[\Delta]^{<\theta}$.   As $\ofamily{\xi}{\kappa}{W_{\xi}}$ is
arbitrary, $\triplepc{\kappa}{\kappa}{\theta}$ is a precaliber triple of
$X$.\ \Qed

\medskip

\quad{\bf (ii)} If $I$ is finite, then
$\triplepc{\kappa}{\kappa}{\theta}$ is a precaliber triple of $X$.
\Prf\ Induce on $\#(I)$, using (i) for the inductive step.\ \Qed

\medskip

\quad{\bf (iii)} Now suppose that $I$ is infinite, $\kappa$ is regular
and uncountable and $\theta\le\omega$.   Then
$\triplepc{\kappa}{\kappa}{\theta}$ is a precaliber triple of $X$.
\Prf\ Let
$\ofamily{\xi}{\kappa}{W_{\xi}}$ be a family of non-empty open sets in
$X$.   Let $\Cal V$ be the standard base for the topology of $X$
consisting of sets of the form $\prod_{\xi<\kappa}U_{\xi}$ where
$U_{\xi}\subseteq X_{\xi}$ is open for every $\xi$ and
$\{\xi:U_{\xi}\ne X_{\xi}\}$ is finite.   For each $\xi<\kappa$ let
$W'_{\xi}\subseteq W_{\xi}$ be a non-empty member of $\Cal V$, so that
$W'_{\xi}$ is determined by a coordinates in a finite subset $I_{\xi}$
of $I$.   By the $\Delta$-system Lemma (4A1Db) there is a set
$A\subseteq\kappa$, with cardinal $\kappa$, such that
$\family{\xi}{A}{I_{\xi}}$ is a $\Delta$-system with root $J$ say.
For $\xi\in A$ express $W'_{\xi}$ as $U_{\xi}\cap V_{\xi}$ where
$U_{\xi}$ is determined by coordinates in $J$ and $V_{\xi}$ is
determined by coordinates in $I_{\xi}\setminus J$.   Now $U_{\xi}$ is of
the form $\pi_J^{-1}[H_{\xi}]$ where $H_{\xi}\subseteq\prod_{i\in J}X_i$
is a non-empty open set and $\pi_J:X\to\prod_{i\in J}X_i$ is the
canonical map.   By (ii), $\triplepc{\kappa}{\kappa}{\theta}$ is a
precaliber triple of $\prod_{i\in J}X_i$, so there is a
$\Gamma\in[A]^{\kappa}$ such that $\bigcap_{\xi\in K}H_{\xi}$ is
non-empty whenever $K\in[\Gamma]^{<\theta}$.    Now take any
$K\in[\Gamma]^{<\theta}$.   Then
$U=\pi_J^{-1}[\bigcap_{\xi\in K}H_{\xi}]$ is a non-empty
set determined by coordinates in $J$, while $V_{\xi}$ is a non-empty
open set determined by coordinates in $I_{\xi}\setminus J$ for each
$\xi\in K$;  because the $I_{\xi}\setminus J$ are disjoint and $K$ is
finite, $U\cap\bigcap_{\xi\in J}V_{\xi}$ is non-empty, and
$\bigcap_{\xi\in K}W_{\xi}$ is a non-empty set, necessarily open because
$K$ is finite.   As
$\ofamily{\xi}{\kappa}{W_{\xi}}$ is arbitrary,
$\triplepc{\kappa}{\kappa}{\theta}$ is a precaliber triple of $X$.\
\Qed
}%end of proof of 516O

\leader{516P}{Corollary} Let $\familyiI{P_i}$ be a family of non-empty
partially ordered sets, with upwards finite-support product
$P=\bigotimes^{\uparrow}_{i\in I}P_i$\cmmnt{ (definition:  514T)}.
If $\triplepc{\kappa}{\kappa}{\theta}$ is an upwards precaliber
triple of every $P_i$ and either $I$ is finite or $\theta\le\omega$ and
$\kappa$ is a regular uncountable cardinal, then
$\triplepc{\kappa}{\kappa}{\theta}$ is an upwards precaliber triple of
$P$.
%In particular, a regular infinite cardinal $\kappa$ is an upwards
%precaliber of $P$ if it is an upwards precaliber of every $P_i$.

\proof{ Suppose first that $\theta$ is countable.
By 516Gb and 516M we can identify the
relevant upwards precaliber triples of each $P_i$ and $P$ with the
precaliber triples of their regular open algebras.   But
$\RO^{\uparrow}(P)\cong\RO(\prod_{i\in I}P_i)$ (514Ua), so 516Ob gives
the result at once.

For finite $I$, $P^*=\prod_{i\in I}P_i$ is a cofinal subset of $P$
(514Ub), so that it has the same upwards precaliber triples (516Ga);  at
the same time, it is easy to see that the up-topology of $P^*$ is just
the product of the up-topologies on the $P_i$.   So this time we do not
need to look at regular open algebras and can use 516Gb and 516Ob
directly.
}%end of proof

\leader{516Q}{}\cmmnt{ For locally compact spaces, as usual, we have
further results.

\medskip

\noindent}{\bf Proposition} Let $X$ be a locally compact Hausdorff
topological space.

(a) $(\kappa,\lambda)$ is a precaliber pair of $X$ iff whenever
$\ofamily{\xi}{\kappa}{G_{\xi}}$ is a family of non-empty open
subsets of $X$, then there is an $x\in X$ such that
$\#(\{\xi:x\in G_{\xi}\})\ge\lambda$.

(b)\dvArevised{2014} Suppose that $\kappa$ is a regular infinite cardinal.
Then $\kappa$ is a precaliber of $X$ iff $\sat(X)\le\kappa$ and whenever
$\ofamily{\xi}{\kappa}{E_{\xi}}$ is a
non-decreasing family of nowhere dense subsets of $X$ then
$\bigcup_{\xi<\kappa}E_{\xi}$ has empty interior.

\proof{{\bf (a)(i)} The condition asserts that
$(\kappa,\lambda,\lambda)$ is a precaliber triple of
$(\frak T\setminus\{\emptyset\},\ni,X)$.   It follows at once that
$(\kappa,\lambda)$ is a precaliber pair of
$(\frak T\setminus\{\emptyset\},\ni,X)$
and therefore of $(X,\frak T)$, by 516M.

\medskip

\quad{\bf (ii)} Now suppose that $(\kappa,\lambda)$ is a precaliber pair
of $X$, and that $\ofamily{\xi}{\kappa}{G_{\xi}}$ is a family of
non-empty open subsets of $X$.   For each $\xi<\kappa$ choose a
non-empty relatively compact open set $H_{\xi}$ such that
$\overline{H}_{\xi}\subseteq G_{\xi}$.   Then there is a
$\Gamma\in[\kappa]^{\lambda}$ such that
$\{H_{\xi}:\xi\in\Gamma\}$ is centered.   In this case,
$\{\overline{H}_{\xi}:\xi\in\Gamma\}$ has the finite intersection
property, so has non-empty intersection.   If $x$ is any point of this
intersection, then $\{\xi:x\in G_{\xi}\}\supseteq\Gamma$ has cardinal at
least $\lambda$.

\medskip

{\bf (b)(i)} Suppose that $\kappa$ is a precaliber of $X$.   Then surely
$\sat(X)\le\kappa$ (516Ja).   If $\ofamily{\xi}{\kappa}{E_{\xi}}$ is a
non-decreasing family of nowhere dense subsets of $X$, take any
non-empty open set $G\subseteq X$.   For each $\xi<\kappa$,
$G_{\xi}=G\setminus\overline{E}_{\xi}$ is a non-empty open set, so by
(a) there is an $x\in X$ such that $\Gamma=\{\xi:x\in G_{\xi}\}$ has
cardinal $\kappa$.  But as $\ofamily{\xi}{\kappa}{G_{\xi}}$ is
non-increasing, this means that $\Gamma=\kappa$ and
$x\in G\setminus\bigcup_{\xi<\kappa}E_{\xi}$.   As $G$ is arbitrary,
$\bigcup_{\xi<\kappa}E_{\xi}$ has empty interior.

\medskip

\quad{\bf (ii)} Now suppose that the condition is satisfied.
Let $\ofamily{\xi}{\kappa}{G_{\xi}}$ be a family of non-empty open
subsets of $X$.   For $\xi<\kappa$ set
$H_{\xi}=\bigcup_{\eta\ge\xi}G_{\eta}$,
$W_{\xi}=X\setminus\overline{H}_{\xi}$.   By 5A4Bd, there is a set
$I\subseteq\kappa$ such that $\#(I)<\sat(X)$ and

\Centerline{$\overline{\bigcupop_{\xi\in I}W_{\xi}}
=\overline{\bigcupop_{\xi<\kappa}W_{\xi}}$.}

\noindent Because $\#(I)<\cf\kappa$, $\zeta=\sup I$ is less than
$\kappa$, and $H_{\zeta}\cap W_{\xi}=\emptyset$ for every $\xi\in I$, so
$H_{\zeta}\cap W_{\xi}=\emptyset$ for every $\xi<\kappa$, that is,
$H_{\zeta}\subseteq\overline{H}_{\xi}$ for every $\xi<\kappa$.

Setting

\Centerline{$E_{\xi}=H_{\zeta}\setminus H_{\xi}
\subseteq\overline{H}_{\xi}\setminus H_{\xi}$}

\noindent for each $\xi$, $\ofamily{\xi}{\kappa}{E_{\xi}}$ is a
non-decreasing family of nowhere dense sets, and cannot cover
$H_{\zeta}$.   If $x\in H_{\zeta}\setminus\bigcup_{\xi<\kappa}E_{\xi}$,
then $x\in H_{\xi}$ for every $\xi<\kappa$, so $\Gamma=\{\eta:x\in
G_{\eta}\}$ is cofinal with $\kappa$.   Because $\kappa$ is regular,
$\Gamma\in[\kappa]^{\kappa}$, and $\bigcap_{\xi\in I}G_{\xi}$ is
non-empty for every $I\in[\Gamma]^{<\omega}$.   As
$\ofamily{\xi}{\kappa}{G_{\xi}}$ is arbitrary, $\kappa$ is a precaliber
of $X$, by (a).
}%end of proof of 516Q

\leader{516R}{}\cmmnt{ We can use the last proposition to give
corresponding characterizations of precaliber pairs of Boolean algebras
in terms of their Stone spaces.

\medskip

\noindent}{\bf Corollary} Let $\frak A$ be a Boolean algebra and $Z$ its
Stone space.

(a) A pair $(\kappa,\lambda)$ of cardinals is a precaliber pair of
$\frak A$ iff whenever $\ofamily{\xi}{\kappa}{G_{\xi}}$ is a family of
non-empty open sets in $Z$ there is a $z\in Z$ such that
$\#(\{\xi:z\in G_{\xi}\})\ge\lambda$.

(b) Suppose that $\kappa\ge\sat(\frak A)$ is a regular infinite
cardinal.   Then $\kappa$ is a precaliber of $\frak A$ iff whenever
$\ofamily{\xi}{\kappa}{E_{\xi}}$ is a
non-decreasing family of nowhere dense subsets of $Z$ then
$\bigcup_{\xi<\kappa}E_{\xi}$ has empty interior.

\proof{ Put 516Ha and 516Q together.
}%end of proof of 516R

\leader{516S}{}\cmmnt{ I collect some further results relating
precalibers to the standard constructions involving Boolean algebras.

\medskip

\noindent}{\bf Proposition} Let $\frak A$ be a Boolean algebra.

(a) If $\frak B$ is a subalgebra of $\frak A$, then every
precaliber pair of $\frak A$ is a precaliber pair of $\frak B$.

(b) If $\frak B$ is a regularly embedded subalgebra of $\frak A$,
every precaliber triple of $\frak A$ is a precaliber triple of
$\frak B$.

(c) If $\frak B$ is a principal ideal of $\frak A$, every precaliber
triple of $\frak A$ is a precaliber triple of $\frak B$.

(d) If $\frak A$ is the simple product of a family
$\familyiI{\frak A_i}$ of Boolean algebras,
$\triplepc{\kappa}{\lambda}{\theta}$ is a precaliber triple
of every $\frak A_i$ and $\#(I)<\cf\kappa$, then
$\triplepc{\kappa}{\lambda}{\theta}$ is a precaliber triple of
$\frak A$.

\proof{{\bf (a)} The Stone space of $\frak B$ is a continuous image of
the Stone space of $\frak A$ (312Sa).   So all we have to do is to put
516Ha and 516Oa together.

\medskip

{\bf (b)} The embedding of $\frak B$ in $\frak A$ is order-continuous,
so corresponds to a continuous surjection from the Stone space $Z$ of
$\frak A$ onto the Stone space $W$ of $\frak B$ which takes non-empty
open sets to sets with non-empty interior (313R).   By 516Id, every
precaliber triple of $Z$ is a precaliber triple of $W$, so every
precaliber triple of $\frak A$ is a precaliber triple of $\frak B$.

\medskip

{\bf (c)} The Stone space of $\frak B$ can be identified with an open
subset of the Stone space of $\frak A$ (312T), so 516Ia gives the
result.

\medskip

{\bf (d)} If $I=\emptyset$ this is trivial;  suppose otherwise;  then
$\kappa$ is infinite.   Let $\ofamily{\xi}{\kappa}{a_{\xi}}$ be a family
of non-zero elements of $\frak A$.   For each $\xi<\kappa$ there is an
$i\in I$ such that $a_{\xi}(i)\ne 0$;  as $\cf\kappa>\#(I)$, there is an
$i\in I$ such that $C=\{\xi:a_{\xi}(i)\ne 0\}$ has cardinal $\kappa$.
Because $\triplepc{\kappa}{\lambda}{\theta}$ is a precaliber triple of
$\frak A_i$, there is a $\Gamma\in[C]^{\lambda}$ such that
$\{a_{\xi}(i):\xi\in J\}$ has a non-zero lower bound in $\frak A_i$ for
every $J\in[\Gamma]^{<\theta}$.   But now $\{a_{\xi}:\xi\in J\}$ has a
non-zero lower bound in $\frak A$ for every $J\in[\Gamma]^{<\theta}$.
As $\ofamily{\xi}{\kappa}{a_{\xi}}$ is arbitrary,
$\triplepc{\kappa}{\lambda}{\theta}$ is a precaliber triple of
$\frak A$.
}%end of proof of 516S

\leader{516T}{}\cmmnt{ A central problem from the very beginning of
set-theoretic topology concerns the saturation of product spaces.   Here
I describe one of the principal methods of showing that product spaces
have small saturation, in a form adapted to partially ordered sets.

\medskip

\noindent}{\bf Theorem} (a) Let $P$ and $Q$ be partially ordered
sets, and $\kappa$ a cardinal such that
$(\kappa,\sat^{\uparrow}(Q),2)$ is an upwards
precaliber triple of $P$.   Then $\sat^{\uparrow}(P\times Q)\le\kappa$.

(b) Let $\familyiI{P_i}$ be a family of non-empty partially ordered sets
with upwards finite-support product $P$.   Suppose that
$\kappa$ is a regular uncountable cardinal such that
$(\kappa,\kappa,2)$ is an
upwards precaliber triple of every $P_i$.   Then
$\sat^{\uparrow}(P)\le\kappa$.

\proof{{\bf (a)} \Quer\ Otherwise, there is an up-antichain
$\ofamily{\xi}{\kappa}{(p_{\xi},q_{\xi})}$ in $P\times Q$.   Let
$\Gamma\subseteq\kappa$ be a set of size $\sat^{\uparrow}(Q)$ such that
$\{p_{\xi}:\xi\in\Gamma\}$ is upwards-linked.   Then
$\family{\xi}{\Gamma}{q_{\xi}}$ must be an up-antichain in $Q$;  but
this is impossible.\ \Bang

\medskip

{\bf (b)} By 516P, $(\kappa,\kappa,2)$ is an upwards
precaliber triple of $P$.   So $(\kappa,2,2)$ and
$\triplepc{\kappa}{2}{\omega}$ also are (516Ba, 516Bc), and
$\sat^{\uparrow}(P)\le\kappa$ (516Ka).
}%end of proof of 516T

\leader{516U}{}\cmmnt{ It will be useful to be able to quote what
amounts to a simple special case of the above result.

\medskip

\noindent}{\bf Corollary} Let $\frak A$ be a Boolean algebra satisfying
Knaster's condition\cmmnt{ (511Ef)}
and $\frak B$ a ccc Boolean algebra.   Then their free
product $\frak A\otimes\frak B$ is ccc.

\proof{ By 516Ta, inverted,
$(\frak A\setminus\{0\})\times(\frak B\setminus\{0\})$ is downwards-ccc.
But $(a,b)\mapsto a\otimes b$ is an order-preserving bijection between
$(\frak A\setminus\{0\})\times(\frak B\setminus\{0\})$ and an order-dense
(that is, coinitial) subset of $(\frak A\otimes\frak B)\setminus\{0\}$
(315Kb);  so
$(\frak A\otimes\frak B)\setminus\{0\}$ is downwards-ccc
(513Gc, inverted), that is, $\frak A\otimes\frak B$ is ccc.
}%end of proof of 516U

\exercises{\leader{516X}{Basic exercises (a)}
%\spheader 516Xa
Let $(A,R,B)$ be a supported relation, and $n\ge 1$ an
integer.   Show that $n$ is a precaliber of $(A,R,B)$ iff
$\add(A,R,B)>n$.
%516B

\spheader 516Xb\dvAnew{2013}
Let $P$ and $Q$ be partially ordered sets, and $f:P\to Q$ a
surjection such that, for any finite set $I\subseteq P$, $I$ is bounded
above in $P$ iff $f[I]$ is bounded above in $Q$.   Show that $P$ and $Q$
have the same upwards precaliber pairs.
%516D

\spheader 516Xc(i) Show that if $P$ is a partially ordered set and
$\kappa>\cf P$ is an infinite cardinal such that $\cf\kappa$ is an
up-precaliber of $P$, then $\kappa$ is an up-precaliber of $P$.  (ii)
Show that if $\frak A$ is a Boolean algebra and
$\kappa>\pi(\frak A)$ is an infinite cardinal such that $\cf\kappa$ is a
precaliber of $\frak A$, then $\kappa$ is a precaliber of $\frak A$.
%516J

\spheader 516Xd Let $P$ be a partially ordered set and $\kappa$
an infinite cardinal.   Show that $\sat^{\uparrow}(P)\le\kappa$ iff
$(\kappa,\omega)$ is an upwards precaliber pair of
$P$.   \Hint{if $\kappa=\sat^{\uparrow}(P)$ and
$\ofamily{\xi}{\kappa}{p_{\xi}}$ is a family in $P$, choose $\xi_n$,
$q_n$ such that $p_{\xi_i}\le q_n$ for $i\le n$ and $\{\xi:q_n$ is
compatible upwards with $p_{\xi}\}$ is always cofinal with $\kappa$.}
%516K

\spheader 516Xe Let $(X,\frak T)$ be a topological space, $\RO(X)$ its
regular open algebra and $\frak G$ its category algebra (definition:
514I).   (i) Show that any precaliber triple of
$(X,\frak T)$ is also a precaliber triple of $\RO(X)$,
$(\frak T\setminus\{\emptyset\},\ni,X)$ and $\frak G$.   (ii) Show that
if $(X,\frak T)$ is regular, then $(X,\frak T)$ and $\RO(X)$ have the
same precaliber triples.   (iii) Show that if $(X,\frak T)$ is locally
compact and Hausdorff, then $(X,\frak T)$ and $\frak G$ have the same
precaliber triples.
%516M

\spheader 516Xf Let $(P,\le)$ be the totally ordered set $\omega_1$,
$\frak T^{\uparrow}$ its up-topology and $\RO^{\uparrow}(P)$ the regular
open algebra of $(P,\frak T^{\uparrow})$.   Show that
$(\omega_1,\omega_1,\omega_1)$ is a precaliber triple of
$\RO^{\uparrow}(P)$ but not of $(P,\le)$ or $(P,\frak T^{\uparrow})$.
%516G, 516M

\spheader 516Xg Let $\familyiI{\frak A_i}$ be a family of Boolean
algebras and $\frak A$ their free product.   Show that if
$\triplepc{\kappa}{\kappa}{\theta}$ is a precaliber
triple of every $\frak A_i$ and either $I$ is finite or
$\theta\le\omega$ and $\kappa$ is a regular infinite cardinal, then
$\triplepc{\kappa}{\kappa}{\theta}$ is a precaliber triple of $\frak A$.
%516O

\spheader 516Xh Suppose that $X$ is a topological space and $Y$ is a
dense subset of $X$ and $\theta\le\omega$.   Show that
$\triplepc{\kappa}{\lambda}{\theta}$ is a precaliber triple of $Y$  iff
it is a precaliber triple of $X$.
%516O

\spheader 516Xi Let $X$ be a locally compact Hausdorff space, and
$\kappa$ a precaliber of $X$.   Show that whenever
$\ofamily{\xi}{\kappa}{E_{\xi}}$ is a
non-decreasing family of nowhere dense subsets of $X$ then
$\bigcup_{\xi<\kappa}E_{\xi}$ has empty interior.
%516Q

\spheader 516Xj Prove 516Sa-516Sc without mentioning Stone spaces.
%516S

\spheader 516Xk(i) Let $X$ and $Y$ be topological spaces, and $\kappa$
a cardinal such that $(\kappa,\sat(Y),2)$ is a
precaliber triple of $X$.   Show that
$\sat(X\times Y)\le\kappa$.
(ii) Let $\familyiI{X_i}$ be a family of topological spaces
with product $X$.   Suppose that
$\kappa$ is a regular uncountable
cardinal such that $(\kappa,\kappa,2)$ is a
precaliber triple of every $X_i$.   Show that
$\sat(X)\le\kappa$.
%516T

\spheader 516Xl(i) Let $\frak A$ and $\frak B$ be Boolean algebras, and
$\frak A\otimes\frak B$ their free product.   Suppose that $\kappa$ is a
cardinal such that $(\kappa,\sat(\frak B),2)$ is a
precaliber triple of $\frak A$.   Show that
$\sat(\frak A\otimes\frak B)\le\kappa$.
(ii) Let $\familyiI{\frak A_i}$ be a family of Boolean algebras
with free product $\frak A$.   Suppose that
$\kappa$ is a regular uncountable
cardinal such that $(\kappa,\kappa,2)$ is a
precaliber triple of every $\frak A_i$.   Show that
$\sat(\frak A)\le\kappa$.
%516T

\spheader 516Xm Let $X$ and $Y$ be topological spaces.   Show that if
$\triplepc{\kappa}{\kappa'}{\theta}$ is a precaliber triple of $X$ and
$\triplepc{\kappa'}{\lambda}{\theta}$ is a precaliber triple of $Y$,
then $\triplepc{\kappa}{\lambda}{\theta}$ is a precaliber triple of
$X\times Y$.
%516T

%\leader{516Y}{Further exercises (a)}
}%end of exercises

\endnotes{
\Notesheader{516}
`Precaliber triples' are visibly complex.   With three cardinals in
action, there is a promise of a powerful method of describing special
features of a partially ordered set or Boolean algebra, but at the same
time a threat of alarming demands on our memory.   In fact none of the
arguments in this section are deep, and they are here mainly for
reference.
Some of the results depend in not-quite-obvious ways on the exact
hypotheses, and it will be useful later to have clear statements to
hand.   In the proofs I have emphasized Galois-Tukey connections
whenever possible;  at the cost of possibly tedious repetititions of
such formulae as $(\frak T\setminus\{\emptyset\},\supseteq,
\frak T\setminus\{\emptyset\})$ naming the supported relations involved,
they can save us the trouble of negotiating the quantifiers in the
definition

\Centerline{$\Forall\ofamily{\xi}{\kappa}{a_{\xi}}\in A^{\kappa}
\Exists\Gamma\in[\kappa]^{\lambda}
\Forall I\in[\Gamma]^{<\theta}\Exists b\in B\ldots$.}

\noindent But of course it is a useful exercise to find proofs from
first principles, not mentioning supported relations and not (for
instance) using Stone spaces to deal with Boolean algebras.

`Supported relations' form a materially more various class of structures
than partially ordered sets, topological spaces or Boolean algebras.
But the constructions already developed in this book (Stone spaces,
regular open algebras, up-topologies) give us functorial relations
between the last three categories which mean that from the point of view
of this section they are nearly the same.   So such results as 516T can
be expected to apply to topological spaces and Boolean algebras as well
(516Xk, 516Xl).   (But note 516Xf.)

Precaliber triples belong with saturation and linking numbers as
parameters describing the `breadth' of a topological space or Boolean
algebra;  see {\smc Comfort \& Negrepontis 82}.   In the first place,
they address a classic problem:  when is the product of ccc topological
spaces ccc?   (This is the case $\kappa=\omega_1$ of 516Xk.)   But with
the exception of saturation, there do not
appear to be simple connexions between precalibers and the cardinal
functions we have looked at so far.   Precalibers seem to correspond to
new features of the structures considered here.   When we come to look
at the most important objects of measure theory (in particular, measure
algebras), we shall find that their precalibers are relatively fluid;  I
mean that while cellularity, Maharam types and many linking numbers, for
instance, are determined by simple formulae in ZFC, precalibers are not.
}%end of notes

\discrpage

