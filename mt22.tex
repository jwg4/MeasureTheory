\frfilename{mt22.tex}
\versiondate{24.2.14}
\copyrightdate{1995}

\def\chaptername{The Fundamental Theorem of Calculus}
\def\sectionname{Introduction}

\newchapter{22}

In this chapter I address one of the most important properties of the
Lebesgue integral.   Given an integrable function $f:[a,b]\to\Bbb R$, we
can form its indefinite integral $F(x)=\int_a^xf(t)dt$ for $x\in[a,b]$.
Two questions immediately present themselves.   (i)  Can we expect to
have the derivative $F'$ of $F$ equal to $f$?   (ii)  Can we identify
which functions $F$ will appear as indefinite integrals?   Reasonably
satisfactory answers may be found for both of these questions:  $F'=f$
almost everywhere (222E) and indefinite integrals are the absolutely
continuous functions (225E).   In the course of dealing with them, we
need to develop a variety of techniques which lead to many striking
results both in the theory of Lebesgue measure and in other, apparently
unrelated, topics in real analysis.

The first step is `Vitali's theorem' (\S221), a remarkable
argument -- it is more a method than a theorem -- which uses the
geometric nature of the real line to extract disjoint subfamilies from
collections of intervals.   It is the foundation stone not only of the
results in \S222 but of all geometric measure theory, that is, measure
theory on spaces with a geometric structure.   I use it here to show
that monotonic functions are differentiable almost everywhere (222A).
Following this, Fatou's Lemma and Lebesgue's Dominated Convergence
Theorem are enough to show that the derivative of an indefinite integral
is almost everywhere equal to the integrand.   We find that some
innocent-looking manipulations of this fact take us surprisingly far;  I
present these in \S223.

I begin the second half of the chapter with a discussion of functions
`of bounded variation', that is, expressible as the difference of
bounded monotonic functions (\S224).   This is one of the least
measure-theoretic sections in the volume;  only in 224I and 224J are 
measure and integration even mentioned.   But this material is needed for 
Chapter 28 as well as for the next section, and is also
one of the basic topics of twentieth-century real analysis.   
\S225 deals with the
characterization of indefinite integrals as the `absolutely
continuous' functions.   In fact this is now quite easy;  it helps to
call on Vitali's theorem again, but everything else is a straightforward
application of methods previously used.   The second half of the section
introduces some new ideas in an attempt to give a deeper intuition into
the essential nature of absolutely continuous functions.   \S226 returns
to functions of bounded variation and their decomposition into
`saltus' and `absolutely continuous' and `singular' parts, the
first two being relatively manageable and the last looking something
like the Cantor function.



\discrpage

