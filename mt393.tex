\frfilename{mt393.tex}
\versiondate{11.5.08}
\copyrightdate{2007}

\def\chaptername{Measurable algebras}
\def\sectionname{Maharam submeasures}

\newsection{393}

Continuing our exploration of variations on the idea of `measurable
algebra', we come to sequentially order-continuous submeasures.
These are associated with `Maharam algebras' (393E), which share a great
many properties with measurable algebras;  for instance, the existence of a
standard topology defined by the algebraic structure (393G).   This
topology is intimately connected with the order*-convergence of sequences
introduced in \S367 (393L).   We can indeed characterize Maharam algebras
in terms of properties of
the order-sequential topology defined by this convergence
(393Q).

\leader{393A}{Definition}\dvAformerly{392G}
Let $\frak A$ be a Boolean algebra.   A {\bf
Maharam submeasure} or {\bf continuous outer measure} on $\frak A$ is a
totally finite submeasure $\nu:\frak A\to\coint{0,\infty}$ such that
$\lim_{n\to\infty}\nu a_n=0$ whenever $\sequencen{a_n}$ is a
non-increasing sequence in $\frak A$ with infimum $0$.

\leader{393B}{Lemma}\dvAformerly{392H}
Let $\frak A$ be a Boolean algebra and $\nu$ a
Maharam submeasure on $\frak A$.

(a) $\nu$ is sequentially order-continuous.

(b) $\nu$ is `countably subadditive', that is, whenever
$\sequencen{a_n}$ is a sequence in $\frak A$ and $a\in\frak A$ is such
that $a=\sup_{n\in\Bbb N}a\Bcap a_n$, then $\nu
a\le\sum_{n=0}^{\infty}\nu a_n$.

(c) If $\frak A$ is Dedekind $\sigma$-complete, then $\nu$ is
exhaustive.

\proof{{\bf (a)} (Of course $\nu$ is an order-preserving function, by
the definition of `submeasure';  so we can apply the ordinary definition
of `sequentially order-continuous' in 313Hb.)  (i) If $\sequencen{a_n}$
is a non-decreasing sequence in $\frak A$ with supremum $a$, then
$\sequencen{a_n\Bsetminus a}$ is a non-increasing sequence with infimum
$0$, so $\lim_{n\to\infty}\nu(a_n\Bsetminus a)=0$;  but as

\Centerline{$\nu a_n\le\nu a\le\nu a_n+\nu(a\Bsetminus a_n)$}

\noindent for every $n$, it follows that $\nu a=\lim_{n\to\infty}\nu
a_n$.   (ii) If $\sequencen{a_n}$ is a non-increasing sequence in $\frak
A$ with infimum $a$, then

\Centerline{$\nu a\le\nu a_n\le\nu a+\nu(a_n\Bsetminus a)\to \nu a$}

\noindent as $n\to\infty$.

\medskip

{\bf (b)} Set $b_n=\sup_{i\le n}a\Bcap a_i$;  then $\nu
b_n\le\sum_{i=0}^n\nu a_i$ for each $n$ (inducing on $n$), so that

\Centerline{$\nu a=\lim_{n\to\infty}\nu b_n\le\sum_{i=0}^{\infty}\nu
a_i$.}

\medskip

{\bf (c)} If $\sequencen{a_n}$ is a disjoint sequence in $\frak A$, set
$b_n=\sup_{i\ge n}a_i$ for each $n$;  then $\inf_{n\in\Bbb N}b_n=0$, so

\Centerline{$\limsup_{n\to\infty}\nu a_n\le\lim_{n\to\infty}\nu b_n=0$.}

}%end of proof of 393B

\leader{393C}{Proposition} Let $\frak A$ be a Dedekind $\sigma$-complete
Boolean algebra and $\nu$ a strictly positive Maharam submeasure on
$\frak A$.   Then $\frak A$ is ccc, Dedekind complete and \wsid, and
$\nu$ is order-continuous.

\proof{ By 393Bc, $\nu$ is exhaustive;  by 392Ca, $\frak A$ is ccc;  by
316Fa, $\frak A$ is Dedekind complete;
by 316Fc and 393Ba, $\nu$ is order-continuous

Now suppose that we have a sequence $\sequencen{A_n}$ of non-empty
downwards-directed subsets of $\frak A$, all with infimum $0$.   Let $B$
be the set

\Centerline{$\{b:b\in\frak A,\,\Forall \,n\in\Bbb
N\,\,\exists\,\,a\in A_n$ such that $a\Bsubseteq b\}$.}

\noindent As $\nu$ is order-continuous,
$\inf_{a\in A_n}\nu a=0$ for each $n$.   Given $\epsilon>0$, we can
choose $\sequencen{a_n}$ such that $a_n\in A_n$ and
$\nu a_n\le 2^{-n}\epsilon$
for each $n$;  now $b=\sup_{n\in\Bbb N}a_n$ belongs to $B$ and
$\nu b\le\sum_{n=0}^{\infty}\nu a_n\le 2\epsilon$.  Thus
$\inf_{b\in B}\nu b=0$.   Since $\nu$ is strictly
positive, $\inf B=0$.   As $\sequencen{A_n}$ is arbitrary, $\frak A$ is
\wsid.
}%end of proof of 393C

\leader{393D}{Theorem} Let $\frak A$ be a Boolean algebra.   Then it is
measurable iff it is Dedekind $\sigma$-complete and carries a uniformly
exhaustive strictly positive Maharam submeasure.

\proof{ If $\frak A$ is measurable, it surely satisfies the conditions,
since any totally finite measure on $\frak A$ is also a uniformly
exhaustive strictly positive Maharam submeasure.   If $\frak A$
satisfies the conditions, then it is \wsid, by 393C, so 392G gives the
result.
}%end of proof of 393D

\leader{393E}{Maharam algebras\dvAnew{2008} (a) Definition} A
{\bf Maharam algebra} is a
Dedekind $\sigma$-complete Boolean algebra $\frak A$ such that there is a
strictly positive Maharam submeasure on $\frak A$.

\spheader 393Eb Every measurable
algebra is a Maharam algebra, while every Maharam algebra is ccc and
\wsid\cmmnt{ (393C)}, therefore Dedekind complete.
A Maharam algebra $\frak A$ is measurable iff there is a strictly
positive uniformly exhaustive submeasure on $\frak A$.
\prooflet{(Put 393C and 392G together again.)}

\spheader 393Ec{\bf (i)}
A principal ideal in a Maharam algebra is a Maharam
algebra;  an order-closed subalgebra of a Maharam algebra is a Maharam
algebra.  \prooflet{\Prf\ Let $\frak A$ be a Maharam algebra and $\frak B$
{\it either} a principal ideal of $\frak A$ {\it or} an order-closed
subalgebra of $\frak A$.   Because $\frak A$ is Dedekind complete, so is
$\frak B$ (314Ea).   Let $\nu:\frak A\to\coint{0,\infty}$ be a
strictly positive Maharam submeasure.   Then $\nu\restr\frak B$ is a
strictly positive Maharam submeasure on $\frak A$, so $\frak B$ is a
Maharam algebra.\ \Qed}

\medskip

\quad{\bf (ii)} The simple product of a countable family of Maharam
algebras is a Maharam algebra.  \prooflet{\Prf\ Let $\familyiI{\frak A_i}$
be a countable family of Maharam algebras and $\frak A$ its simple product.
Then $\frak A$ is Dedekind complete (315De).   For each $i\in I$, let
$\nu_i$ be a strictly positive Maharam submeasure on $\frak A_i$.   Let
$\familyiI{\epsilon_i}$ be a family of strictly positive real numbers such
that $\sum_{i\in I}\epsilon_i$ is finite.   Set
$\nu a=\sum_{i\in I}\min(\epsilon_i,\nu_ia(i))$ for $a\in\frak A$;
then $\nu$ is a strictly positive Maharam submeasure on $\frak A$, so
$\frak A$ is a Maharam algebra.\ \Qed}

\leader{393F}{Lemma}\dvAformerly{393E}
Let $\frak A$ be a Dedekind $\sigma$-complete
Boolean algebra and $\nu$, $\nuprime$ two Maharam submeasures on $\frak A$
such that $\nu a=0$ whenever $\nuprime a=0$.   Then $\nu$ is absolutely
continuous with respect to $\nuprime$.

\proof{ (Compare 232Ba.)   \Quer\ Otherwise, we can find a sequence
$\sequencen{a_n}$ in
$\frak A$ such that $\nuprime a_n\le 2^{-n}$ for every $n$, but
$\epsilon=\inf_{n\in\Bbb N}\nu a_n>0$.   Set $b_n=\sup_{i\ge n}a_i$ for
each $n$, $b=\inf_{n\in\Bbb N}b_n$.   Then
$\nuprime b_n\le\sum_{i=n}^{\infty}2^{-i}\le 2^{-n+1}$ for each $n$
(393Bb),
so $\nuprime b=0$;  but $\nu b_n\ge\epsilon$ for each $n$, so
$\nu b\ge\epsilon$ (393Ba), contrary to the hypothesis.\ \Bang
}%end of proof of 393F

\leader{393G}{Proposition}\dvAnew{2008}
Let $\frak A$ be a Maharam algebra,
and $\nu$ and
$\nuprime$ two strictly positive Maharam submeasures on $\frak A$.
Then the
metrics they induce on $\frak A$ are uniformly equivalent, so we have a
topology and uniformity on $\frak A$ which we may call the
{\bf Maharam-algebra topology} and the {\bf Maharam-algebra uniformity}.

\proof{ By 393F, $\nu$ and $\nuprime$ are mutually absolutely continuous;
translating this with the formula of 392Ha, we see that the metrics are
uniformly equivalent, so induce the same topology and uniformity.   As
$\frak A$ does have a strictly positive Maharam submeasure, we may use it
to define the Maharam-algebra topology and uniformity of $\frak A$.
}%end of proof of 393G

\leader{393H}{Proposition}\dvAformerly{393B}
Let $\frak A$ be a Boolean algebra, and $\nu$
an exhaustive strictly positive totally finite submeasure on $\frak A$.
Let $\widehat{\frak A}$ be the metric completion of $\frak A$\cmmnt{, as
described in 392H}, and $\hat\nu$ the continuous extension of $\nu$ to
$\widehat{\frak A}$.   Then $\hat\nu$ is a Maharam submeasure, so
$\widehat{\frak A}$ is a Maharam algebra.

\proof{ (Compare 392Hd.)

\medskip

{\bf (a)} The point is that any non-increasing sequence
$\sequencen{a_n}$ in $\widehat{\frak A}$ is a Cauchy sequence for the
metric $\hat\rho$.   \Prf\ Let $\epsilon>0$.   For each $n\in\Bbb N$,
choose $b_n\in\frak A$ such that
$\hat\rho(a_n,b_n)\le 2^{-n}\epsilon$, and set $c_n=\inf_{i\le n}b_i$.
Then

\Centerline{$\hat\rho(a_n,c_n)
=\hat\rho(\inf_{i\le n}a_i,\inf_{i\le n}b_i)
\le\sum_{i=0}^n\hat\rho(a_i,b_i)
\le 2\epsilon$}

\noindent for every $n$.   Choose $\sequence{k}{n(k)}$ inductively so
that, for each $k\in\Bbb N$, $n(k+1)\ge n(k)$ and

\Centerline{$\nu(c_{n(k)}\Bsetminus c_{n(k+1)})
\ge\sup_{i\ge n(k)}\nu(c_{n(k)}\Bsetminus c_i)-\epsilon$.}

\noindent Then $\sequence{k}{c_{n(k)}\Bsetminus c_{n(k+1)}}$ is a
disjoint sequence in $\frak A$, so

$$\eqalign{\limsup_{k\to\infty}\sup_{i\ge n(k)}\hat\rho(a_{n(k)},a_i)
&\le 4\epsilon
  +\limsup_{k\to\infty}\sup_{i\ge n(k)}\hat\rho(c_{n(k)},c_i)\cr
&=4\epsilon
  +\limsup_{k\to\infty}\sup_{i\ge n(k)}\nu(c_{n(k)}\Bsetminus c_i)\cr
&\le 4\epsilon+\limsup_{k\to\infty}
    \nu(c_{n(k)}\Bsetminus c_{n(k+1)})+\epsilon
=5\epsilon.\cr}$$

\noindent As $\epsilon$ is arbitrary, $\sequencen{a_n}$ is Cauchy.\ \Qed

\medskip

{\bf (b)} It follows that $\widehat{\frak A}$ is Dedekind
$\sigma$-complete.   \Prf\ If $\sequencen{a_n}$ is any sequence in
$\widehat{\frak A}$, $\sequencen{b_n}=\sequencen{\inf_{i\le n}a_i}$ is a
Cauchy sequence with a limit $b\in\widehat{\frak A}$.   For any
$k\in\Bbb N$,

\Centerline{$\hat\nu(b\Bsetminus a_k)
=\lim_{n\to\infty}\hat\nu(b_n\Bsetminus a_k)=0$,}

\noindent so $b\Bsubseteq a_k$, because $\hat\nu$ is strictly positive.
While if $c\in\widehat{\frak A}$ is a lower bound for $\{a_n:n\in\Bbb
N\}$, we have $c\Bsubseteq b_n$ for every $n$, so

\Centerline{$\hat\nu(c\Bsetminus b)
=\lim_{n\to\infty}\hat\nu(c\Bsetminus b_n)=0$}

\noindent and $c\Bsubseteq b$.   Thus $b=\inf_{n\in\Bbb N}a_n$;  as
$\sequencen{a_n}$ is arbitrary, $\widehat{\frak A}$ is Dedekind
$\sigma$-complete (314Bc).\ \Qed

\medskip

{\bf (c)} We find also that $\hat\nu$ is a Maharam submeasure,
because if $\sequencen{a_n}$ is a non-increasing sequence in
$\widehat{\frak A}$ with infimum $0$, it must have a limit $a$ which (as
in (b) above) must be its infimum, that is, $a=0$;  consequently

\Centerline{$\lim_{n\to\infty}\hat\nu a_n=\hat\nu a=0$.}

\medskip

{\bf (d)} It follows at once that $\hat\nu$ is exhaustive
(393Bc), so that $\widehat{\frak A}$ is ccc (392Ca) and Dedekind
complete (316Fa).
}%end of proof of 393H

\leader{393I}{Proposition}\dvAnew{2008}
Let $\frak A$ be a Dedekind $\sigma$-complete
Boolean algebra and $\nu$ an atomless Maharam submeasure on $\frak A$.
Then for every $\epsilon>0$
there is a finite partition $C$ of unity in $\frak A$ such that
$\nu c\le\epsilon$ for every $c\in C$.

\proof{ Let $A\subseteq\frak A$ be a maximal disjoint set such that
$0<\nu a\le\epsilon$ for every $a\in A$.   As $\nu$ is exhaustive
(393Bc), $A$ is countable.   Set $c=1\Bsetminus\sup A$.
\Quer\ If $\nu c>0$, then
(because $\nu$ is atomless) we can choose inductively a sequence
$\sequencen{b_n}$ such that $b_0=c$,
$b_{n+1}\Bsubseteq b_n$, $\nu b_{n+1}>0$ and
$\nu(b_n\Bsetminus b_{n+1})>0$ for every $n\in\Bbb N$.   But now
$\sequencen{b_n\Bsetminus b_{n+1}}$ is a disjoint sequence of elements of
non-zero submeasure, so one of them has submeasure in $\ocint{0,\epsilon}$
and ought to have been added to $A$.\ \Bang

If $A$ is finite, we can set $C=A\cup\{c\}$ and stop.   Otherwise,
enumerate $A$ as $\sequencen{a_n}$ and set $c_n=\sup_{i\ge n}a_i$ for each
$n$;  then $\lim_{n\to\infty}\nu c_n=0$, so there is an $n$ such that
$\nu c_n\le\epsilon$, and we can set $C=\{a_i:i<n\}\cup\{c_n\Bcup c\}$.
}%end of proof of 393I

\leader{393J}{Lemma}\cmmnt{ ({\smc Maharam 47})} Let $\frak A$ be a
ccc Boolean algebra with a T$_1$ topology $\frak T$ such that
(i) $\Bcup:\frak A\times\frak A\to\frak A$ is continuous at $(0,0)$ (ii)
whenever $\sequencen{a_n}$ is a non-increasing
sequence in $\frak A$ with infimum $0$, then $\sequencen{a_n}\to 0$ for
$\frak T$.   Then $\frak A$ has a strictly positive Maharam submeasure.

\proof{{\bf (a)} For any $e\in\frak A\setminus\{0\}$, there is a Maharam
submeasure $\nu$ on $\frak A$ such that $\nu e>0$.

\medskip

\Prf{\bf (i)} Choose a sequence $\sequencen{G_n}$ of neighbourhoods
of $0$, as follows.   Because $\frak T$ is T$_1$,
$G_0=\frak A\setminus\{0\}$ is a
neighbourhood of $0$ not containing $e$.   Given $G_n$, choose a
neighbourhood $G_{n+1}$ of $0$ such that $G_{n+1}\subseteq G_n$ and
$a\Bcup b\Bcup c\in G_n$ whenever $a$, $b$, $c\in G_{n+1}$.   (Take
neighbourhoods $H$, $H'$ of $0$ such that $a\Bcup b\in G_n$ for $a$,
$b\in H$, $b\Bcup c\in H$ for $b$, $c\in H'$ and set
$G_{n+1}=H\cap H'\cap G_n$.)   Define $\nu_0:\frak A\to[0,1]$ by setting

$$\eqalign{\nu_0a
&=1\text{ if }a\notin G_0,\cr
&=2^{-n}\text{ if }a\in G_n\setminus G_{n+1},\cr
&=0\text{ if }a\in\bigcap_{n\in\Bbb N}G_n.\cr}$$

\noindent Then whenever $a_0,\ldots,a_r\in\frak A$, $n\in\Bbb N$ and
$\sum_{i=0}^r\nu_0a_i<2^{-n}$, $\sup_{i\le r}a_i\in G_n$.   To see this,
induce on $r$.   If $r=0$ then we have $\nu_0a_0<2^{-n}$ so
$a_0\in G_{n+1}\subseteq G_n$.   For the inductive step to $r\ge 1$,
there must
be a $k\le r$ such that $\sum_{i<k}\nu_0a_i<2^{-n-1}$ and
$\sum_{k<i\le n}\nu_0a_i<2^{-n-1}$ (allowing $k=0$ or $k=n$, in which
case one of the
sums will be zero).  (If $\sum_{i=0}^r\nu_0a_i<2^{-n-1}$, take $k=n$;
otherwise, take $k$ to be the least number such that
$\sum_{i=0}^k\nu_0a_i\ge 2^{-n-1}$.)   By the inductive hypothesis, and
because $0$ certainly belongs to $G_{n+1}$, $b=\sup_{i<k}a_i$ and
$c=\sup_{k<i\le r}a_i$ both belong to $G_{n+1}$;  but also
$\nu_0a_k<2^{-n}$ so $a_k\in G_{n+1}$.   Accordingly, by the choice of
$G_{n+1}$,

\Centerline{$\sup_{i\le r}a_i=b\Bcup a_k\Bcup c$}

\noindent belongs to $G_n$, and the induction continues.

\medskip

\quad{\bf (ii)} Set

\Centerline{$\nu_1a
=\inf\{\sum_{i=0}^r\nu_0a_i:a_0,\ldots,a_r\in\frak A$,
$a=\sup_{i\le r}a_i\}$}

\noindent for every $a\in\frak A$.   It is easy to see that
$\nu_1(a\Bcup b)\le\nu_1a+\nu_1b$ for all $a$, $b\in\frak A$;  also
$a\in G_n$ whenever $\nu_1a<2^{-n}$, so, in particular, $\nu_1e\ge 1$,
because $e\notin G_0$.

Set

\Centerline{$\nu a=\inf\{\nu_1b:a\Bcap e\Bsubseteq b\Bsubseteq e\}$}

\noindent for every $a\in\frak A$.   Then of course $0\le\nu a\le\nu b$
whenever $a\Bsubseteq b$, and

\Centerline{$\nu 0\le\nu_10\le\nu_00=0$,}

\noindent so $\nu 0=0$.   If $a$, $b\in\frak A$ and $\epsilon>0$, there
are $a'$, $b'$ such that $a\Bcap e\Bsubseteq a'\Bsubseteq e$,
$b\Bcap e\Bsubseteq b'\Bsubseteq e$, $\nu_1a'\le\nu a+\epsilon$ and
$\nu_1 b'\le\nu b+\epsilon$;  so that
$(a\Bcup b)\Bcap e\Bsubseteq a'\Bcup b'\Bsubseteq e$ and

\Centerline{$\nu(a\Bcup b)\le\nu_1(a'\Bcup b')\le\nu_1a'+\nu_1b'
\le\nu a+\nu b+2\epsilon$.}

\noindent As $\epsilon$, $a$ and $b$ are arbitrary, $\nu$ is a
submeasure.   Next, if $\sequence{i}{a_i}$ is any non-increasing
sequence in $\frak A$ with infimum $0$, $\sequence{i}{a_i\Bcap e}$ is
another, so converges to $0$ for $\frak T$.   If $n\in\Bbb N$ there is
an $m$ such that $a_i\Bcap e\in G_n$ for every $i\ge m$, so that

\Centerline{$\nu a_i\le\nu_1(a_i\Bcap e)\le\nu_0(a_i\Bcap e)\le 2^{-n}$}

\noindent for every $i\ge m$.   As $n$ is arbitrary,
$\lim_{i\to\infty}\nu a_i=0$;  as $\sequence{i}{a_i}$ is arbitrary,
$\nu$ is a Maharam submeasure.   Finally,

\Centerline{$\nu e=\nu_1e\ge 1$,}

\noindent so $\nu e\ne 0$.\ \Qed

\medskip

{\bf (b)} Write $C$ for the set of those $c\in\frak A$ such that
there is a strictly positive Maharam submeasure on the principal ideal
$\frak A_c$.   Then $C$ is order-dense in $\frak A$.  \Prf\ Take any
$e\in\frak A\setminus\{0\}$.   By (a), there is a
Maharam submeasure $\nu$ such that $\nu e>0$.   Set
$A=\{e\Bsetminus a:\nu a=0\}$.   Because $\nu$ is a submeasure, $A$ is
downwards-directed.
\Quer\ If $\inf A=0$ then, because $\frak A$ is ccc, there is a
non-increasing sequence
$\sequencen{a_n}$ in $A$ with infimum $0$;  because
$\nu$ is a Maharam submeasure,

\Centerline{$\nu e\le\inf_{n\in\Bbb N}\nu a_n+\nu(e\Bsetminus a_n)
=\inf_{n\in\Bbb N}\nu a_n=0$.  \Bang}

\noindent Thus $A$ has a non-zero lower bound $c$, and
$\nu\restrp\frak A_c$ is a strictly positive Maharam submeasure, while
$c\Bsubseteq e$.\ \Qed

\medskip

{\bf (c)} Because $\frak A$ is ccc, there is a sequence
$\sequencen{c_n}$ in $C$ with supremum $1$.   For each $n$, let $\nu_n$
be a strictly positive Maharam submeasure on $\frak A_{c_n}$;
multiplying by a scalar if necessary, we may suppose that
$\nu_nc_n\le 2^{-n}$.   We can therefore define $\nu:\frak A\to[0,2]$ by
setting $\nu a=\sum_{n=0}^{\infty}\nu_n(a\Bcap c_n)$ for every
$a\in\frak A$, and it is easy to check that $\nu$ is a strictly positive
Maharam submeasure on $\frak A$.
}%end of proof of 393J

\leader{*393K}{Theorem}
Let $\frak A$ be a ccc Dedekind complete Boolean algebra.
Then $\frak A$ is a Maharam algebra iff there is a Hausdorff linear
space topology $\frak T$ on $L^0(\frak A)$ such that for every
neighbourhood $G$ of $0$ there is a neighbourhood $H$ of $0$ such that
$u\in G$ whenever $v\in H$ and $|u|\le|v|$.

\proof{{\bf (a)} Suppose that $\frak A$ is a Maharam algebra;  let $\nu$ be
a strictly positive Maharam submeasure on $\frak A$.

\medskip

\quad{\bf (i)} For $u\in L^0=L^0(\frak A)$ set

\Centerline{$\tau(u)
=\inf\{\alpha:\alpha\ge 0,\,\nu\Bvalue{|u|>\alpha}\le\alpha\}$.}

Then

\Centerline{$\tau(u+v)\le\tau(u)+\tau(v)$,
\quad$\tau(\alpha u)\le\tau(u)$ if $|\alpha|\le 1$,
\quad$\lim_{\alpha\to 0}\tau(\alpha u)=0$}

\noindent for every $u$, $v\in L^0$.   \Prf\ (i) It will save a moment
if we observe that whenever $\beta>\tau(u)$ there is an $\alpha\le\beta$
such that $\nu\Bvalue{|u|>\alpha}\le\alpha$, so that

\Centerline{$\nu\Bvalue{|u|>\beta}\le\nu\Bvalue{|u|>\alpha}
\le\alpha\le\beta$.}

\noindent Also, because $\nu$ is sequentially order-continuous,

\Centerline{$\nu\Bvalue{|u|>\tau(u)}
=\lim_{n\to\infty}\nu\Bvalue{|u|>\tau(u)+2^{-n}}
\le\lim_{n\to\infty}\tau(u)+2^{-n}
=\tau(u)$.}

\noindent (ii) So

$$\eqalignno{\nu\Bvalue{|u+v|>\tau(u)+\tau(v)}
&\le\nu\Bvalue{|u|+|v|>\tau(u)+\tau(v)})\cr
&\le\nu(\Bvalue{|u|>\tau(u)}\Bcup\Bvalue{|v|>\tau(v)})\cr
\noalign{\noindent (364Ea)}
&\le\nu\Bvalue{|u|>\tau(u)}+\nu\Bvalue{|v|>\tau(v)}
\le\tau(u)+\tau(v),\cr}$$

\noindent and $\tau(u+v)\le\tau(u)+\tau(v)$.   (iii) If $|\alpha|\le 1$
then

\Centerline{$\nu\Bvalue{|\alpha u|>\tau(u)}
\le\nu\Bvalue{|u|>\tau(u)}
\le\tau(u)$,}

\noindent and $\tau(\alpha u)\le\tau(u)$.   (iv)
$\lim_{n\to\infty}\nu\Bvalue{|u|>n}=0$ because
$\sequencen{\Bvalue{|u|>n}}$ is a non-increasing sequence with infimum
$0$.   So if $\epsilon>0$, there is an $n\ge 1$ such that
$\nu\Bvalue{|u|>n\epsilon}\le\epsilon$, in which case
$\nu\Bvalue{|\alpha u|>\epsilon}\le\epsilon$ whenever
$|\alpha|\le\bover1n$, so that $\tau(\alpha u)\le\epsilon$ whenever
$|\alpha|\le\bover1n$.   As $\epsilon$ is arbitrary, $\lim_{\alpha\to
0}\tau(\alpha u)=0$.\ \Qed

\medskip

\quad{\bf (ii)} Accordingly we have a metric $(u,v)\mapsto\tau(u-v)$ which
defines a linear space topology $\frak T$ on $L^0$ (2A5B).   Now let $G$
be an open set containing $0$.   Then there is an $\epsilon>0$ such that
$H=\{u:\tau(u)<\epsilon\}$ is included in $G$.   If $v\in H$ and
$|u|\le|v|$, then

\Centerline{$\nu\Bvalue{|u|>\tau(v)}
\le\nu\Bvalue{|v|>\tau(v)}\le\tau(v)$,}

\noindent so $\tau(u)\le\tau(v)$ and $u\in H\subseteq G$.   So $\frak T$
satisfies all the conditions.

\medskip

{\bf (b)} Given such a topology $\frak T$ on $L^0$,
let $\frak S$ be the topology on $\frak A$ induced
by $\frak T$ and the function
$\chi:\frak A\to L^0$;  that is, $\frak S=\{\chi^{-1}[G]:G\in\frak T\}$.
Then $\frak S$ satisfies the conditions of 393J.   \Prf\ (i) Because
$\frak T$ is Hausdorff and $\chi$ is injective, $\frak S$ is Hausdorff,
therefore T$_1$.
(ii) If $0\in G\in\frak S$, there is an $H\in\frak T$ such that
$G=\chi^{-1}[H]$.   Now $0$ (the zero of $L^0$) belongs to $H$, so there
is an open set $H_1$ containing $0$ such that $u\in H$ whenever
$v\in H_1$ and $|u|\le|v|$.
Next, addition on $L^0$ is continuous for $\frak T$,
so there is an open set $H_2$ containing $0$ such that $u+v\in H_1$
whenever $u$, $v\in H_2$.   Consider $G'=\chi^{-1}[H_2]$.   This is an
open set in $\frak A$ containing $0_{\frak A}$, and if $a$, $b\in G'$ then

\Centerline{$|\chi(a\Bcup b)|\le\chi a+\chi b\in H_2+H_2\subseteq H_1$,}

\noindent so $\chi(a\Bcup b)\in H$ and $a\Bcup b\in G$.   As $G$ is
arbitrary, $\Bcup$ is continuous at $(0,0)$.   (iii) If
$\sequencen{a_n}$ is
a non-increasing sequence in $\frak A$ with infimum $0$,
$u_0=\sup_{n\in\Bbb N}n\chi a_n$ is defined in $L^0$ (use the criterion
of 364L(a-i):

\Centerline{$\inf_{m\in\Bbb N}\sup_{n\in\Bbb N}\Bvalue{n\chi a_n>m}
=\inf_{m\in\Bbb N}a_{m+1}=0$.)}

\noindent If $0\in G\in\frak S$, take $H\in\frak T$ such that
$G=\chi^{-1}[H]$, and $H_1\in\frak T$ such that $0\in H_1$ and $u\in H$
whenever $v\in H_1$ and $|u|\le|v|$.   Because scalar multiplication is
continuous for $\frak T$, there is a $k\ge 1$ such that
$\bover1ku_0\in H_1$.   For any $n\ge k$,
$\chi a_n\le\bover1ku_0$ so $\chi a_n\in H$
and $a_n\in G$.   As $G$ is arbitrary, $\sequencen{a_n}\to 0$ for
$\frak S$.   As $\sequencen{a_n}$ is arbitrary, condition (ii) in
the statement of 393J is satisfied.\ \Qed

So 393J tells us that $\frak A$ has a strictly positive Maharam
submeasure, and is a Maharam algebra.
}%end of proof of 393K

\leader{393L}{}\dvAformerly{3{}92K}\cmmnt{ I now turn to some very
remarkable
ideas relating the order*-convergence of \S367 to the questions here.

\medskip

\noindent}{\bf Definition} Let $P$ be a lattice, and consider the
relation `$\sequencen{p_n}$ order*-converges to $p$' as a relation
between $P^{\Bbb N}$ and $P$.
\dvro{There}{By 367Bc, this satisfies the hypothesis
of 3A3Pa, so there} is a unique topology on $P$ for which a set
$F\subseteq P$ is closed iff $a\in F$ whenever $\sequencen{a_n}$ is a
sequence in $F$ which order*-converges to $a$ in $P$.
I will call this topology the {\bf order-sequential topology} of $P$.

\cmmnt{{\bf Warning!} For the next few paragraphs I shall be closely
following the papers {\smc Balcar G{\l}owczy\'nski \& Jech 98}
and {\smc Balcar Jech \& Paz\'ak 05}.   I should therefore note
explicitly that if $\frak A$ is a Boolean algebra which is
neither Dedekind $\sigma$-complete nor ccc, my `order-sequential
topology' on $\frak A$ may not be identical to theirs.}

\leader{393M}{Lemma} Let $\frak A$ be a Boolean algebra.

(a) A sequence $\sequencen{a_n}$ order*-converges to $a\in\frak A$ iff
there is a partition $B$ of unity in $\frak A$ such that
$\{n:n\in\Bbb N$, $(a_n\Bsymmdiff a)\Bcap b\ne 0\}$ is finite for every
$b\in B$.

(b) If $\sequencen{a_n}$ order*-converges to $a$ and $c\in\frak A$, then
$\sequencen{a_n\Bcup c}$,
$\sequencen{a_n\Bcap c}$ and $\sequencen{a_n\Bsymmdiff c}$
order*-converge to
$a\Bcup c$, $a\Bcap c$ and $a\Bsymmdiff c$ respectively.

(c) The operations $\Bcap$, $\Bcup$ and $\Bsymmdiff$ are separately
continuous for the order-sequential topology.

(d) Every disjoint sequence in $\frak A$ is order*-convergent to $0$.

\proof{{\bf (a)} Let $\sequencen{a_n}$ be a sequence in $\frak A$ and
$a\in\frak A$;  set

\Centerline{$C=\{c:\,\Exists n\in\Bbb N$,
$c\Bsubseteq a_i$ for every $i\ge n\}$,}

\Centerline{$D=\{d:\,\Exists n\in\Bbb N$,
$a_i\Bsubseteq d$ for every $i\ge n\}$.}

\medskip

\quad{\bf (i)} If $\sequencen{a_n}$ order*-converges to $a$, then
$a=\sup C=\inf D$ (367Be).   Since

\Centerline{$\inf\{d\Bsetminus a:d\in D\}
=\inf\{a\Bsetminus c:c\in C\}=0$,}

\Centerline{$E=\{(d\Bsetminus a)\Bcup(a\Bsetminus c):c\in C$, $d\in D\}$}

\noindent also has infimum $0$ (313A, 313B).   So there is a partition $B$ of unity
such that for every $b\in B$ there is an $e\in E$ such that $b\Bcap e=0$.
Now, given $b\in B$, there are $c\in C$ and $d\in D$ such that
$b\Bcap(d\setminus c)=0$;  there are $n_1$, $n_2\in\Bbb N$ such that
$c\Bsubseteq a_n$ for $n\ge n_1$ and $a_n\Bsubseteq d$ for $n\ge n_2$;
so that $\{n:(a_n\Bsymmdiff a)\Bcap b\ne 0\}$
is bounded above by $\max(n_1,n_2)$ and is
finite.   So $B$ witnesses that the condition is satisfied.

\medskip

\quad{\bf (ii)} Now suppose that $B$ is a partition of unity such that
$\{n:(a_n\Bsymmdiff a)\Bcap b\ne 0\}$ is finite for every $b\in B$.
Then $a\Bcup(1\Bsetminus b)\in D$ for every $b\in B$, because
$\{n:a_n\notBsubseteq a\Bcup(1\Bsetminus b)\}
\subseteq\{n:(a_n\Bsymmdiff a)\Bcap b\ne 0\}$ is finite.   So any lower
bound for $D$ is also a lower bound for $\{a\Bcup(1\Bsetminus b):b\in B\}$
and is included in $a$.   Similarly, any upper bound for $C$ includes $a$;
as $c\Bsubseteq d$ whenever $c\in C$ and $d\in D$, $a=\sup C=\inf D$ and
$\sequencen{a_n}$ order*-converges to $a$.

\medskip

{\bf (b)} These are all immediate from (a), because

\Centerline{$(a_n\Bcup c)\Bsymmdiff(a\Bcup c)
\Bsubseteq a_n\Bsymmdiff a$,
\quad$(a_n\Bcap c)\Bsymmdiff(a\Bcap c)
\Bsubseteq a_n\Bsymmdiff a$,}

\Centerline{$(a_n\Bsymmdiff c)\Bsymmdiff(a\Bsymmdiff c)
=a_n\Bsymmdiff a$}

\noindent for every $n$.

\medskip

{\bf (c)} By (b), we can apply 3A3Pb to each of the functions
$a\mapsto a\Bcap b=b\Bcap a$,
$a\mapsto a\Bcup b=b\Bcup a$ and
$a\mapsto a\Bsymmdiff b=b\Bsymmdiff a$
to see that these are all continuous
for every $b\in\frak A$.

\medskip

{\bf (d)} If $\sequencen{a_n}$ is a disjoint sequence in $\frak A$, there
is a partition $B$ of unity in $\frak A$ containing every $a_n$ (311Gd);
now $B$ witnesses that the condition of (a) is satisfied.
}%end of proof of 393M

\leader{393N}{Proposition}\dvAnew{2008}
Let $\frak A$ be a Maharam algebra.   Then the
Maharam-algebra topology on $\frak A$ is the
order-sequential topology.

\proof{ Let $\frak T_o$ be the order-sequential topology
on $\frak A$, $\nu$ a strictly
positive Maharam submeasure on $\frak A$, $\rho$ the metric defined from
$\nu$ (392H) and $\frak T_M$ the Maharam-algebra
topology induced by $\rho$ (393G).

\medskip

{\bf (a)} If $\sequencen{a_n}$ is a sequence in $\frak A$ such that
$\sequencen{a_n}\to^*a$ in $\frak A$, then
$\lim_{n\to\infty}\rho(a_n,a)=0$.   \Prf\ By 393Mb,
$\sequencen{a_n\Bsymmdiff a}\to^*0$;  by 367Bf,
$0=\inf_{n\in\Bbb N}\sup_{i\ge n}(a_i\Bsymmdiff a)$, so

\Centerline{$\rho(a_n,a)\le\nu(\sup_{i\ge n}a_i\Bsymmdiff a)\to 0$}

\noindent as $n\to\infty$.\ \Qed

It follows that every $\frak T_M$-closed set is $\frak T_o$-closed,
and $\frak T_o\subseteq\frak T_M$.

\medskip

{\bf (b)} Conversely, suppose that $\sequencen{a_n}$ is a sequence in
$\frak A$ converging for $\frak T_M$ to $a\in\frak A$.   Then
$\sequencen{a_n}$ has a subsequence $\sequencen{a'_n}$ such that
$\rho(a'_n,a)\le 2^{-n}$ for every $n\in\Bbb N$.   In this case, setting
$b_m=\sup_{n\ge m}a'_n\Bsymmdiff a$ for each $m$,
$\nu b_m\le 2^{-m+1}$ for every $m$ (393Bb),
so $\inf_{m\in\Bbb N}b_m=0$, and
$\sequencen{a'_n\Bsymmdiff a}\to^*0$, that is, $\sequencen{a'_n}\to^*a$.

Thus every $\frak T_M$-convergent sequence has an order*-convergent
subsequence
with the same limit;  it follows that every $\frak T_o$-closed set is
$\frak T_M$-closed, that is, $\frak T_M\subseteq\frak T_o$.
}%end of proof of 393N

\leader{393O}{Proposition} Let $\frak A$ be a ccc Dedekind
$\sigma$-complete Boolean
algebra, with its order-sequential topology, and $\frak B$ a
subalgebra of $\frak A$.   Then the topological closure of $\frak B$ is the
smallest order-closed set including $\frak B$;
\cmmnt{in particular, }$\frak B$ is
order-closed iff it is topologically closed.

\proof{{\bf (a)} Let $\overline{\frak B}$ be the topological closure of
$\frak B$, and $\frak B^{\sim}$ the smallest order-closed set including
$\frak B$.

\medskip

\quad{\bf (i)}
Suppose that $\sequencen{b_n}$ is a non-decreasing sequence in
$\overline{\frak B}$ with supremum $b$ in $\frak A$;  then
$\sequencen{b_n}\to^*b$, by 367Bf or 367Xa.
So $b\in\overline{\frak B}$.  Similarly,
$\inf_{n\in\Bbb N}b_n\in\overline{\frak B}$ for every non-increasing
sequence in $\overline{\frak B}$.   Thus $\overline{\frak B}$ is
sequentially order-closed.   But this means that it is order-closed, by
316Fb.   So $\overline{\frak B}\supseteq\frak B^{\sim}$.

\medskip

\quad{\bf (ii)} By 313Fc, $\frak B^{\sim}$ is a subalgebra of $\frak A$.
Now suppose that $\sequencen{b_n}$ is a sequence in $\frak B^{\sim}$
which order*-converges to $a\in\frak A$.   Then
$c_{mn}=\sup_{m\le i\le n}b_i$ belongs to $\frak B^{\sim}$ whenever
$m\le n$;  as $\langle c_{mn}\rangle_{n\ge m}$ is non-decreasing,
$c_m=\sup_{i\ge m}b_i=\sup_{n\ge m}c_{mn}$ belongs to $\frak B^{\sim}$
for every $m\in\Bbb N$;  as $\sequence{m}{c_m}$ is non-increasing,
$\inf_{m\in\Bbb N}c_m\in\frak B^{\sim}$.   But $c=b$ (367Bf).
As $\sequencen{b_n}$ is arbitrary, $\frak B^{\sim}$ is closed for the
order-sequential topology, and must include $\overline{\frak B}$.

Thus $\overline{\frak B}=\frak B^{\sim}$, as claimed.

\medskip

{\bf (b)} Now

\Centerline{$\frak B\text{ is order-closed}
\iff\frak B=\frak B^{\sim}
\iff\frak B=\overline{\frak B}
\iff\frak B$ is topologically closed.}
}%end of proof of 393O

\leader{393P}{Lemma} Let $\frak A$ be a ccc \wsid\ Boolean algebra,
endowed with its order-sequential topology.

(a) If $\langle a_{mn}\rangle_{m,n\in\Bbb N}$, $\sequence{m}{a_m}$ and
$a$ are such that $\sequencen{a_{mn}}$ order*-converges to $a_m$ for
each $m$, while $\sequence{m}{a_m}$ order*-converges
to $a$, then there is a sequence $\sequence{m}{k(m)}$ in $\Bbb N$ such
that $\sequence{m}{a_{m,k(m)}}$
order*-converges to $a$.

(b) If $A\subseteq\frak A$ and $a\in\overline{A}$, there is a sequence
in $A$ which order*-converges to $a$.

(c) If $G$ is a neighbourhood of $0$ in $\frak A$ then there is an open
neighbourhood $H$ of $0$, included in $G$, such that
$[0,a]\subseteq H$ for every $a\in H$.

(d) For $A\subseteq\frak A$, set $\bigvee_0(A)=\{0\}$ and
$\bigvee_{n+1}(A)=\{a\Bcup b:a\in\bigvee_n(A)$, $b\in A\}$ for
$n\in\Bbb N$.

\quad(i) If $A\subseteq\frak A$ is such that $[0,a]\subseteq A$ for every
$a\in A$, and $n\in\Bbb N$, then $[0,a]\subseteq\bigvee_n(A)$ for every
$a\in\bigvee_n(A)$.

\quad(ii) If $H\subseteq\frak A$ is an open set containing $0$ such that
$[0,a]\subseteq H$ for every $a\in H$, then $\bigvee_{n+1}(H)$ is open
and $\overline{\bigvee_n(H)}\subseteq\bigvee_{n+1}(H)$ for every
$n\in\Bbb N$.

(e) Suppose that $\frak A$ is Dedekind $\sigma$-complete.
Then for every open set $G$ containing $0$ there is an open set $H$
containing $0$ such that $\bigvee_3(H)\subseteq\bigvee_2(G)$.

\proof{{\bf (a)} Let $C_m$, for $m\in\Bbb N$, be partitions of unity in
$\frak A$ such that

\Centerline{$\{m:(a_m\Bsymmdiff a)\Bcap c\ne 0\}$ is finite for every
$c\in C_0$,}

\Centerline{$\{n:(a_{mn}\Bsymmdiff a_m)\Bcap c\ne 0\}$ is finite whenever
$m\in\Bbb N$ and $c\in C_{m+1}$}

\noindent (393Ma).   Because $\frak A$ is \wsid, there is a partition $B$
of unity such that $\{c:c\in C_m$, $c\Bcap b\ne 0\}$ is finite whenever
$m\in\Bbb N$ and $b\in B$ (316H(ii)).
Because $\frak A$ is ccc, there is a sequence
$\sequencen{b_n}$ running over $B\cup\{0\}$.   Now, for each $m$, any
sufficiently large $k(m)$ will be such that
$(a_{m,k(m)}\Bsymmdiff a_m)\Bcap b_i=0$ for every $i\le m$.   In this case,
for any $i$,

\Centerline{$\{m:(a_{m,k(m)}\Bsymmdiff a)\Bcap b_i\ne 0\}
\subseteq\{m:m<i\}\cup\{m:(a_m\Bsymmdiff a)\cap b_i\ne 0\}$}

\noindent is finite, so $B$ witnesses that
$\sequence{m}{a_{m,k(m)}}\to^*a$ (393Ma, in the other direction).

\medskip

{\bf (b)} Let $A^{\sim}$ be the set of order*-limits of sequences in
$A$.   Of course $A^{\sim}$ must be
included in $\overline{A}$.   But from (a) we see that the limit
of any order*-convergent sequence in $A^{\sim}$ belongs to $A^{\sim}$.
So $A^{\sim}$ is closed and is equal to
$\overline{A}$.   Turning this round, we see that $\overline{A}$ is just
the set of order*-limits of sequences in $A$,
as claimed.

\medskip

{\bf (c)} Set $D=\{d:d\in\frak A$, $[0,d]\not\subseteq G\}$,
$H=\frak A\setminus\overline{D}$.   Since
$D\supseteq\frak A\setminus G$, $H$ is an open subset of $G$.

\Quer\ If $0\in\overline{D}$, then (b) tells us that there is a sequence
$\sequencen{d_n}$ in $D$ order*-converging to
$0$.   Now there is for each $n\in\Bbb N$ a $c_n\Bsubseteq d_n$ such
that $c_n\notin G$.   By 367Be or 393Ma,
$\sequencen{c_n}$ order*-converges to $0$, and
$0\in\overline{\frak A\setminus G}$;  but $G$ is supposed to be
a neighbourhood of $0$.\ \BanG\   Thus $0\in H$ and $H$ is a
neighbourhood of $0$.

\Quer\ If $a\in H$ and $b\in[0,a]\setminus H$, then $b\in\overline{D}$,
so there is a sequence $\sequencen{d_n}$ in
$D$ order*-converging to $b$.   In this case, $\sequencen{d_n\Bcup a}$
order*-converges to $b\Bcup a=a$, by
393Mb.   But also $[0,d_n\Bcup a]\supseteq[0,d_n]$ is not included in
$G$, so $d_n\Bcup a\in D$ for each $n$,
and $a\in\overline{D}$;  which is impossible.\ \BanG\   Thus
$[0,a]\subseteq H$ for every $a\in H$, and $H$ has the
properties declared.

\medskip

{\bf (d)(i)} This is an elementary induction on $n$.

\medskip

\quad{\bf (ii)} The point is that
$\bigvee_{n+1}(H)=\{a\Bsymmdiff b:a\in\bigvee_n(H)$, $b\in H\}$.
\Prf\ If $a\in\bigvee_n(H)$ and $b\in H$, then
$a\Bsetminus b\in\bigvee_n(H)$, by (i), and $b\setminus a\in H$, so
$a\Bsymmdiff b\in\bigvee_{n+1}(H)$.   On the other hand, if
$c\in\bigvee_{n+1}(H)$, it is expressible as
$a\Bcup b=a\Bsymmdiff(b\Bsetminus a)$ where $a\in\bigvee_n(H)$ and $b$ and
$b\Bsetminus a$ belong to $H$.\ \Qed

Since $\Bsymmdiff$ is separately continuous, it follows at once that

\Centerline{$\bigvee_{n+1}(H)
=\bigcup_{a\in\bigvee_n(H)}\{a\Bsymmdiff b:b\in H\}
=\bigcup_{a\in\bigvee_n(H)}\{b:a\Bsymmdiff b\in H\}$}

\noindent is open, because $\Bsymmdiff$ is separately continuous (393Mc).
Next, if $d\in\overline{\bigvee_n(H)}$, then
there is a sequence $\sequencen{d_n}$ in
$\bigvee_n(H)$ order*-converging to $d$, by (b).   Now
$\sequencen{d_n\Bsymmdiff d}\to^* 0$, by 393Mb, so
$\sequencen{d_n\Bsymmdiff d}$ converges topologically to $0$, by 3A3Pa, and
there is an $n\in\Bbb N$ such that $d_n\Bsymmdiff d\in H$;  in which case
$d=d_n\Bsymmdiff(d_n\Bsymmdiff d)$ belongs to $\bigvee_{n+1}(H)$.
As $d$ is arbitrary, $\overline{\bigvee_n(H)}\subseteq\bigvee_{n+1}(H)$.

\medskip

{\bf (e)} \Quer\ Suppose, if possible, otherwise.

\medskip

\quad{\bf (i)} Choose $H_n$, $a_n$, $b_n$ and $c_n$
inductively, as follows.   $H_0\subseteq G$ is to be an open neighbourhood
of $0$ such that $[0,a]\subseteq H_0$ whenever $a\in H_0$
((c) above).   Given that $H_n$ is an open set
containing $0$ and including $[0,a]$ whenever it contains $a$,
we are supposing that
$\bigvee_3(H_n)\not\subseteq\bigvee_2(G)$;
choose $a_n$, $b_n$, $c_n\in H_n$ such that
$a_n\Bcup b_n\Bcup c_n\notin\bigvee_2(G)$, and set

\Centerline{$H_{n+1}=\{a:a$, $a\Bcup a_n$, $a\Bcup b_n$ and
$a\Bcup c_n$ all belong to $H_n\}$,}

\noindent so that $H_{n+1}$ is an open set containing $0$,
and $[0,a]\subseteq H_{n+1}$ for every $a\in H_{n+1}$.
Continue.

\medskip

\quad{\bf (ii)} At the end of the induction, set
$F=\bigcap_{n\in\Bbb N}\overline{H}_n$ and
$a^*=\inf_{n\in\Bbb N}\sup_{i\ge n}a_i$.   Then $a^*\Bcup d\in F$ for every
$d\in F$.   \Prf\ For $m\le n\in\Bbb N$,
$\sup_{m\le i\le n}a_i\Bcup d\in H_m$ for every $d\in H_{n+1}$ (induce
downwards on $m$).   Because $\Bcup$ is separately continuous,
$\sup_{m\le i\le n}a_i\Bcup d\in\overline{H}_m$ for
every $d\in F$.   Letting $n\to\infty$,
$d\Bcup\sup_{i\ge m}a_i\in\overline{H}_m$ whenever $d\in F$ and
$m\in\Bbb N$.
Next, for any $b\in\frak A$, $\{a:a\Bcap b\in\overline{H}_m\}$ is a closed
set including $H_m$, so $a\Bcap b\in\overline{H}_m$ for every
$a\in\overline{H}_m$;  that is, $[0,a]\Bsubseteq\overline{H}_m$ for every
$a\in\overline{H}_m$.   As $a^*\Bsubseteq\sup_{i\ge m}a_i$,
$d\Bcup a^*\in\overline{H}_m$ for every
$d\in F$.   As $m$ is arbitrary, $d\Bcup a^*\in F$ for
every $d\in F$.\ \Qed

Similarly, setting $b^*=\inf_{n\in\Bbb N}\sup_{i\ge n}b_i$ and
$c^*=\inf_{n\in\Bbb N}\sup_{i\ge n}c_i$, $d\Bcup b^*$ and $d\Bcup c^*$
belong to $F$ for every $d\in F$;  and of course $0\in F$.
So $e=a^*\Bcup b^*\Bcup c^*$ belongs to
$F$.   For each $n\in\Bbb N$, $a_n\Bcup b_n\Bcup c_n\notin\bigvee_2(H_0)$;
but $[0,a]\subseteq\bigvee_2(H_0)$ for every $a\in\bigvee_2(H_0)$,
by (d-i),
so $\sup_{i\ge n}a_i\Bcup b_i\Bcup c_i\notin\bigvee_2(H_0)$.
Accordingly $e=\inf_{n\in\Bbb N}\sup_{i\ge n}a_i\Bcup b_i\Bcup c_i$
does not belong to the open set $\bigvee_2(H_0)$, and
$e\notin\overline{H}_0$, by (d-ii).
So $e\in F\setminus\overline{H}_0$;  which is impossible.\ \Bang
}%end of proof of 393P

\leader{393Q}{Theorem}\cmmnt{ ({\smc Balcar G{\l}owczy\'nski \& Jech
98}, {\smc Balcar Jech \& Paz\'ak 05})} Let
$\frak A$ be a Dedekind $\sigma$-complete Boolean algebra.
Then the following are equiveridical:

(i) $\frak A$ is a Maharam algebra;

(ii) $\frak A$ is ccc and the order-sequential topology is Hausdorff;

(iii) $\frak A$ is \wsid\ and
$\{0\}$ is a G$_{\delta}$ set for the order-sequential
topology of $\frak A$;

(iv) $\frak A$ is ccc and there is a T$_1$ topology on $\frak A$
such that ($\alpha$) $\Bcup:\frak A\times\frak A\to\frak A$ is
continuous at $(0,0)$ ($\beta$) whenever $\sequencen{a_n}$ is a
non-decreasing
sequence in $\frak A$ with infimum $0$, then $\sequencen{a_n}\to 0$.

\proof{{\bf (a)(i)$\Rightarrow$(ii)}
By 393Eb, $\frak A$ is ccc.
By 393N, the order-sequential topology is metrizable, therefore Hausdorff.

\medskip

{\bf (b)(ii)$\Rightarrow$(iii)} Suppose that the conditions of (ii) are
satisfied.   In the following argument,
all topological terms will refer to the
order-sequential topology on $\frak A$.

\medskip

\quad\grheada\ $\frak A$ is \wsid.   \Prf\ Let $\sequencen{A_n}$
be a sequence of partitions of unity in $\frak A$, and set

\Centerline{$D=\{d:d\in\frak A$, $\{a:a\in A_n$, $a\Bcap d\ne 0\}$ is
finite for every $n\in\Bbb N\}$.}

\noindent Take any
$c\in\frak A^+$.   Let $G$, $H$ be disjoint open sets
containing $0$, $c$ respectively.   Choose $\sequencen{c_n}$ inductively,
as follows.   $c_0=c$.   Given $c_n\in H$, let $\sequence{i}{a_{ni}}$ be a
sequence running over $A_n$, and set $c_{nj}=\sup_{i\le j}c_n\Bcap a_{ni}$;
then $\sequence{j}{c_{nj}}$ order*-converges to $c_n$ (367Bf/367Xa),
so there is a
$j_n$ such that $c_{nj_n}\in H$;  set $c_{n+1}=c_{nj_n}$, and continue.

This gives us a non-increasing sequence $\sequencen{c_n}$ in $H$.   Set
$d=\inf_{n\in\Bbb N}c_n$;  then $d\notin G$ so $d\ne 0$, while
$d\Bsubseteq\sup_{i\le j_n}a_{ni}$ for each $n$, so $d\in D$.

As $c$ is arbitrary, $D$ is order-dense in $\frak A$ and includes a
partition of unity.   As $\sequencen{A_n}$ is arbitrary, $\frak A$ is
\wsid\ (316H).\ \Qed

\medskip

\quad\grheadb\ For any $a\in\frak A^+$ there is a sequence
$\sequencen{H_n}$ of neighbourhoods of $0$ such that
$a\notBsubseteq\sup(\bigcap_{n\in\Bbb N}H_n)$.   \Prf\ For
$A\subseteq\frak A$ and $n\in\Bbb N$, define $\bigvee_n(A)$ as in
393Pd.   Let $G$, $G'$ be disjoint neighbourhoods of $0$ and $a$
respectively, and set $G_0=G\cap\{a\Bsymmdiff b:b\in G'\}$;  then
$G_0$ is a neighbourhood of $0$ (393Mc).
By 393Pc, we can find a neighbourhood $H_0$ of $0$ such that
$H_0\subseteq G_0$ and $[0,b]\subseteq H_0$ for every $b\in H_0$,
in which case $[0,b]\subseteq\bigvee_2(H_0)$ for every
$b\in\bigvee_2(H_0)$, while $a\notin\bigvee_2(H_0)$.
By 393Pe, we can choose neighbourhoods $H_n$ of $0$
such that $H_n\subseteq H_{n-1}$ and
$\bigvee_3(H_n)\subseteq\bigvee_2(H_{n-1})$ for every $n\ge 1$;
by 393Pc, we can suppose that $[0,b]\subseteq H_n$ whenever $b\in H_n$.
But this will ensure that $\bigvee_4(H_{n+2})\subseteq\bigvee_2(H_n)$ for
every $n$, so that $\bigvee_{2^k}(H_{2k})\subseteq\bigvee_2(H_2)$ for
every $k\ge 1$.   Set $F=\bigcap_{n\in\Bbb N}H_n$.   Then

\Centerline{$\bigvee_{2^k}(F)
\subseteq\bigvee_{2^k}(H_{2k})
\subseteq\bigvee_2(H_2)$}

\noindent for every $k\ge 1$.   Since $\sup F$ is the limit of a sequence
in $\bigcup_{k\ge 1}\bigvee_{2^k}(F)$,

\Centerline{$\sup F\in\overline{\bigvee_2(H_2)}\subseteq\bigvee_3(H_2)
\subseteq\bigvee_2(H_0)$}

\noindent (using 393P(d-ii) for the first inequality)
and cannot include $a$.\ \Qed

\medskip

\quad\grheadc\ Now consider the set $D$ of those $d\in\frak A$ such that
there is a sequence $\sequencen{H_n}$ of neighbourhoods of $0$ such that
$d\Bcap\sup(\bigcap_{n\in\Bbb N}H_n)=0$.
By ($\beta$), $D$ is order-dense, so
includes a partition of unity $A$.   $A$ is countable, so there is a
sequence $\sequencen{H_n}$ of neighbourhoods of $0$ such that
$d\Bcap\sup(\bigcap_{n\in\Bbb N}H_n)=0$ for every $d\in A$;  but this means
that $\bigcap_{n\in\Bbb N}H_n=\{0\}$.   So (iii) is true.

\medskip

{\bf (c)(iii)$\Rightarrow$(iv)} Now suppose that the conditions in (iii)
are satisfied.   As in (b),
all topological terms will refer to the order-sequential
topology on $\frak A$.

\medskip

\quad\grheada\ There is a non-increasing sequence $\sequencen{G_n}$ of open
neighbourhoods of $0$ such that
$\bigcap_{n\in\Bbb N}\overline{G}_n=\{0\}$.
\Prf\ Let $\sequencen{U_n}$
be a sequence of open sets with intersection $\{0\}$.   Set $G_0=\frak A$,
and for $n\in\Bbb N$ choose an open neighbourhood
$G_{n+1}$ of $0$, included in $U_n\cap G_n$, such that
$[0,a]\subseteq G_{n+1}$ for
every $a\in G_{n+1}$ (393P).   \Quer\ If
$0\ne d\in\bigcap_{n\in\Bbb N}\overline{G}_n$, then for each
$n\in\Bbb N$ we can find a sequence
$\sequence{i}{a_{ni}}$ in $G_n$ order*-converging to $d$ (393Pc).   By
393Pa, there is a sequence $\sequencen{k(n)}$
in $\Bbb N$ such that $\sequencen{a_{n,k(n)}}$ order*-converges to $d$.
Now $d=\sup_{n\in\Bbb N}\inf_{i\ge n}a_{i,k(i)}$ (367Bf), so there is an
$n\in\Bbb N$ such that
$c=\inf_{i\ge n}a_{i,k(i)}$ is non-zero.   But in this case we must have
$c\le a_{i,k(i)}\in G_i$ and $c\in G_i\subseteq U_j$ whenever
$i\ge\max(n,j+1)$, so $c=0$.\ \BanG\
Thus $\bigcap_{n\in\Bbb N}\overline{G}_n=\{0\}$, as
required.\ \Qed

\medskip

\quad\grheadb\ For every neighbourhood $G$ of $0$ there is a
neighbourhood $H$ of $0$ such that $a\Bcup b\in G$ for all
$a$, $b\in H$.   \Prf\Quer\ Otherwise, choose $\sequencen{H_n}$,
$\sequencen{a_n}$ and $\sequencen{b_n}$ inductively,
as follows.   Start with an open neighbourhood $H_0$ of $0$ such
that $H_0\subseteq G$ and $[0,a]\subseteq H_0$ for every $a\in H_0$.
Given that $H_n$ is an open neighbourhood of $0$, let $a_n$,
$b_n\in H_n$ be such that $a_n\Bcup b_n\notin G$.   Because
the maps $a\mapsto a\Bcup a_n$ and $a\mapsto a\Bcup b_n$ are continuous,
there is an open neighbourhood $H_{n+1}$ of $0$ such that $a\Bcup a_n$
and $a\Bcup b_n$ belong to $H_n$ for every
$a\in H_{n+1}$;  and we may suppose that $H_{n+1}\Bsubseteq G_n$.
Continue.

An easy induction on $k$ shows that $a\Bcup\sup_{n\le i\le n+k}a_i$ and
$a\Bcup\sup_{n\le i\le n+k}b_i$ belong to $H_n$
whenever $k\in\Bbb N$ and $a\in H_{n+k+1}$.   In particular,
$\sup_{n\le i\le n+k}a_i\in H_n$ for every $k$;
since $\sequence{k}{\sup_{n\le i\le n+k}a_i}$ is
order*-convergent to $\sup_{i\ge n}a_i$,
$\sup_{i\ge n}a_i\in\overline{H}_n\subseteq\overline{G}_n$ for every
$n$.   Set $a^*=\inf_{n\in\Bbb N}\sup_{i\ge n}a_i$.
Then $\sequencen{\sup_{i\ge n}a_i}\to^* a^*$, and
$\sup_{i\ge n}a_i\in\overline{G}_m$ whenever $n\ge m$, so
$a^*\in\overline{G}_m$ for every $m$, and $a^*=0$.

In the same way, $\inf_{n\in\Bbb N}\sup_{i\ge n}b_i=0$.   It follows
that $\inf_{n\in\Bbb N}c_n=0$, where
$c_n=\sup_{i\ge n}a_i\Bcup\sup_{i\ge n}b_i$ for each $n$.   But now
$\sequencen{c_n}$ is a non-increasing sequence
with infimum $0$, so order*-converges to $0$, and there must be an $n$
such that $c_n\in H_0$.   Since
$a_n\Bcup b_n\Bsubseteq c_n$, $a_n\Bcup b_n\in H_0\subseteq G$, contrary
to the choice of $a_n$ and $b_n$.\ \Bang\Qed

\medskip

\quad\grheadc\ $\frak A$ is ccc.   \Prf\ Let $\sequencen{U_n}$ be a
sequence of open sets with intersection $\{0\}$, and
$A\subseteq\frak A\setminus\{0\}$ a partition of unity.
If $\sequence{i}{a_i}$ is a sequence
of distinct elements of $A$, then $\sequence{i}{a_i}\to^*0$ (393Md);  so
$A\setminus U_n$ is finite for every $n$, and $A$ is countable.\ \Qed

\medskip

\quad\grheadd\ ($\beta$) means just that $\Bcup$ is continuous at
$(0,0)$.   Also a non-increasing sequence
with infimum $0$ order*-converges to $0$, so converges topologically to
$0$ (3A3Pa);  and the topology is certainly T$_1$.   So all the
conditions of (iv) are satisfied by the order-sequential topology.

\medskip

{\bf (d)(iv)$\Rightarrow$(i)} By 393J, there is a strictly positive
Maharam submeasure on $\frak A$;  as $\frak A$ is Dedekind
$\sigma$-complete, it is a Maharam algebra.
}%end of proof of 393Q

\leader{393R}{Definition}\dvAnew{2008}
Let $\frak A$ be a Boolean algebra.   Then
$\frak A$ is {\bf\sfcc} if $\frak A$ can be expressed as
$\bigcup_{n\in\Bbb N}A_n$ where no $A_n$ includes any infinite disjoint
set.

\leader{393S}{Theorem}\dvAnew{2008}\cmmnt{ ({\smc Todor\v{c}evi\'c 04})}
Let $\frak A$ be a
Boolean algebra.   Then $\frak A$ is a Maharam algebra iff it is
\sfcc, \wsid\ and Dedekind $\sigma$-complete.

\proof{ (B.Balcar){\bf (a)} If $\frak A$ is a Maharam algebra, then of
course it is Dedekind $\sigma$-complete, and we have known since 393C that
it is \wsid.   Also it carries a strictly positive exhaustive submeasure,
so is \sfcc.

Of course $\{0\}$ is a Maharam algebra.
For the rest of the proof, therefore, I suppose that $\frak A$ is a
non-trivial algebra satisfying the conditions, and seek to show that it is
a Maharam algebra.

\medskip

{\bf (b)(i)} Let $\sequencen{A_n}$ be a
sequence of sets, with union $\frak A^+$, such that no $A_n$
includes any infinite disjoint set.   For each $n$, set
$B_n=\bigcup_{m\le n}\bigcup_{a\in A_m}[a,1]$,  so that $B_n$ includes no
infinite
disjoint subset.   Now there is an $n$ such that $1$ is in the interior
of $B_n$ for the order-sequential topology.   \Prf\Quer\
Otherwise, of course $\frak A$ is ccc, so
there is for each $n\in\Bbb N$ a sequence $\sequence{i}{b_{ni}}$
in $\frak A\setminus B_n$ which is order*-convergent to $1$
(393Pb).  By 393Pa, there is a
sequence $\sequencen{k(n)}$ in $\Bbb N$ such that $\sequencen{b_{n,k(n)}}$
order*-converges to $1$.   As $1\ne 0$, there must be an $m\in\Bbb N$ such that
$c=\inf_{i\ge m}b_{i,k(i)}\ne 0$.   There is an $n$ such that
$c\in A_n$, in which case $b_{i,k(i)}\in B_n\subseteq B_i$ for
every $i\ge\max(m,n)$.\ \Bang\Qed

\medskip

\quad{\bf (ii)} Set $H=\interior B_n$.   Then there is a $c\in H$ such that for
every $d\in\frak A$ one of $c\Bcap d$, $c\Bsetminus d\notin H$.
\Prf\Quer\
Otherwise, we can choose a sequence $\sequence{i}{c_i}$ in $H$ such that
$c_0=1$ and, for each $i\in\Bbb N$, $c_{i+1}\Bsubseteq c_i$ and
$c_i\Bsetminus c_{i+1}\in H$.   But in this case
$\sequence{i}{c_i\Bsetminus c_{i+1}}$ is a disjoint sequence in $B_n$,
which is impossible.\ \Bang\Qed

\medskip

\quad{\bf (iii)} $0$ and $1$ can be separated by open sets.   \Prf\ Take
$H$ and $c$ from (ii).   Then $G_0=\{d:c\Bsetminus d\in H\}$ and
$G_1=\{d:c\Bcap d\in H\}$ are disjoint open sets containing $0$ and $1$
respectively.\ \Qed

\medskip

{\bf (b)} It follows that $\frak A$ is actually Hausdorff in the
order-sequential topology.   \Prf\ Let $a_0$, $a_1\in\frak A$ be such that
$b=a_1\Bsetminus a_0$ is non-zero.
Consider the principal ideal $\frak A_b$.   Like $\frak A$, this is \sfcc,
\wsid\ and Dedekind $\sigma$-complete.
By (a), there are disjoint subsets $U$, $V$
of $\frak A_b$, open for the order-sequential topology of $\frak A_b$, such
that $0\in U$ and $b\in V$.   The function
$a\mapsto a\Bcap b:\frak A\to\frak A_b$ is continuous for the
order-sequential topologies (3A3Pb), so
$G=\{a:a\Bcap b\in U\}$ and $H=\{a:a\Bcap b\in V\}$ are open.   Now $G$ and
$H$ are
disjoint open sets in $\frak A$ containing $a_0$, $a_1$ respectively.
As $a_0$ and $a_1$ are arbitrary, $\frak A$ is Hausdorff.\ \Qed

By 393Q, $\frak A$ is a Maharam algebra.
}%end of proof of 393S


\exercises{
\leader{393X}{Basic exercises $\pmb{>}$(a)}
%\sqheader 393Xa
Let $\frak A$ be the finite-cofinite algebra on an
uncountable set (316Yl).  (i) Set $\nu_10=0$, $\nu_1a=1$ for
$a\in\frak A\setminus\{0\}$.   Show that $\nu_1$ is a strictly positive
Maharam
submeasure but is not exhaustive.   (ii) Set $\nu_2a=0$ for finite $a$,
$1$ for cofinite $a$.   Show that $\nu_2$ is a uniformly exhaustive
Maharam submeasure but is not order-continuous.
%393B

\sqheader 393Xb Let $\frak A$ be a Boolean algebra and $\nu$ a
submeasure on $\frak A$.   Set $I=\{a:\nu a=0\}$.   Show that
(i) $I$ is an ideal of $\frak A$ (ii) there is a submeasure $\bar\nu$ on
$\frak A/I$ defined by setting $\bar\nu a^{\ssbullet}=\nu a$ for every
$a\in\frak A$ (iii) if $\nu$ is exhaustive, so is $\bar\nu$ (iv) if
$\nu$ is uniformly exhaustive, so is $\bar\nu$ (v) if $\nu$ is a Maharam
submeasure, $I$ is a $\sigma$-ideal (vi) if $\nu$ is a Maharam
submeasure and $\frak A$ is Dedekind $\sigma$-complete, $\bar\nu$ is a
Maharam submeasure.
%393B

\spheader 393Xc Let $\frak A$ be a Dedekind complete Boolean algebra and
$\nu$ an order-continuous submeasure on $\frak A$.   Show that $\nu$ has
a unique support $a\in\frak A$ such that $\nu\restrp\frak A_a$ is
strictly positive and $\nu\restrp\frak A_{1\Bsetminus a}$ is identically
zero.
%393B

\spheader 393Xd
Let $\frak A$ be a Boolean algebra and $\nu$ an exhaustive submeasure on
$\frak A$ such that
$\nu a=\lim_{n\to\infty}\nu a_n$ whenever $\sequencen{a_n}$ is a
non-decreasing sequence in $\frak A$ with supremum $a$.
Show that $\nu$ is a Maharam submeasure.
%393C

\spheader 393Xe Let $\frak A$ be a Dedekind $\sigma$-complete Boolean
algebra and $\nu$ a uniformly exhaustive Maharam submeasure on
$\frak A$.   Show that there is a non-negative countably additive
functional $\mu$ on $\frak A$ such that $\{a:\mu a=0\}=\{a:\nu a=0\}$.
\Hint{393Xb(vi).}
%393D, 393Xb

\spheader 393Xf Let $\frak A$ be a Maharam algebra with its Maharam-algebra
topology and uniformity.   (i) Let $B\subseteq\frak A$ be a non-empty
upwards-directed set.   For
$b\in B$ set $F_b=\{c:b\Bsubseteq c\in B\}$.
Show that $\{F_b:b\in B\}$ generates a Cauchy filter
$\Cal F(B\closeuparrow)$ on $\frak A$ which converges to $\sup B$.
(ii) Show that closed subsets of $\frak A$ are order-closed.
(iii) Show that an order-dense subalgebra of $\frak A$ must be dense in the
topological sense.
%393G

\spheader 393Xg Let $\frak A$ be a Maharam algebra.   Show that it is a
measurable algebra iff for every $A\subseteq\frak A$ including antichains
of all finite sizes there is a sequence in $A$ which is order*-convergent
to $0$.
%393G

\spheader 393Xh Let $\frak A$ be a Boolean algebra.   Suppose that
$\sequencen{a_n}$, $\sequencen{b_n}$ are sequences in $\frak A$
order*-converging to $a$, $b$ respectively.   Show that
$\sequencen{a_n\odot b_n}\to^*a\odot b$ when $\odot$ is any of the
operations $\Bcup$, $\Bcap$, $\Bsymmdiff$ or $\Bsetminus$.
%393M

\spheader 393Xi
Let $(\frak A,\bar\mu)$ be a semi-finite measure algebra.   Write
$\frak T_{\text{os}}$ for the order-sequential topology on $\frak A$ and
$\frak T_{\text{ma}}$ for the measure-algebra topology.
Show that $\frak T_{\text{os}}\supseteq\frak T_{\text{ma}}$,
with equality iff $(\frak A,\bar\mu)$ is $\sigma$-finite.
%393N

\spheader 393Xj ({\smc Jech 08}) Let $\frak A$ be a Dedekind
$\sigma$-complete Boolean algebra and $\sequencen{A_n}$ a sequence of
subsets of $\frak A$ such that ($\alpha$) for
every $n\in\Bbb N$, any antichain in $A_n$ has at most $n$ elements
($\beta$)
a sequence $\sequence{k}{a_k}$ in $\frak A$ is order*-convergent to $0$ iff
$\{k:a_k\in A_n\}$ is finite for every $n\in\Bbb N$.
(i) Show that $\frak A$ is ccc.   (ii) Show that $\frak A$ is \wsid.
\Hint{if $C_n$ is non-empty and
downwards-directed with infimum $0$ for each $n$, show that there is a
sequence $\sequencen{a_n}\to^*0$ such that $a_n\in C_n$ for every $n$.}
(iii) Show that $\frak A$ is a Maharam algebra.  \Hint{393S.}
(iv) Show that any Maharam submeasure on $\frak A$ is uniformly exhaustive.
(v) Show that $\frak A$ is a measurable algebra.
%393S

\leader{393Y}{Further exercises (a)}
%\spheader 393Ya
Let $\frak A$ be any Boolean algebra with a strictly positive Maharam
submeasure.   Show that $\frak A$ is weakly $\sigma$-distributive.
%393C

\spheader 393Yb Let $U$ be a Riesz space, with its order-sequential
topology.   (i) Show that addition and
subtraction are separately continuous.   (ii) Show that $U$ is
Archimedean iff scalar multiplication is separately continuous
as a function from $\Bbb R\times U$ to $U$, and that in
this case scalar multiplication is actually continuous.
%393M

\spheader 393Yc Let $\frak A$ be a Dedekind $\sigma$-complete
Boolean algebra, and give $L^0=L^0(\frak A)$ its order-sequential topology.
Suppose $h:\Bbb R\to\Bbb R$ is continuous, and let $\bar h:L^0\to L^0$ be
the corresponding function as defined in 364H.   Show that $\bar h$ is
continuous.
%393M

\spheader 393Yd Let $\frak A$
be a Maharam algebra.   Show that a topology $\frak T$ on $L^0(\frak A)$
defined by the method of 393K must be the order-sequential topology on
$L^0(\frak A)$.
%393N

\spheader 393Ye Let $U$ be a \wsid\ Riesz space with the countable sup
property, with its
order-sequential topology, and $A$ a subset of $U$.
Show that $\overline{A}$ is the set of
order*-limits of sequences in $A$.
%393P

\spheader 393Yf Let $U$ be a \wsid\ Dedekind complete Riesz space with the
countable sup property, endowed with its order-sequential topology,
and $\frak A$ its band algebra.   Show that the following are
equiveridical:   (i) $\frak A$ is a Maharam algebra;
(ii) $U$ is Hausdorff;
(iii) addition on $U$ is continuous at $(0,0)$;
(iv) $\vee:U\times U\to U$ is continuous at $(0,0)$.
%393Q

\spheader 393Yg Let $\frak G$ be the regular open algebra of
$\Bbb R$, with its order-sequential topology.   (i) Show that if $U$, $V$
are open sets in $\frak G$ containing $0_{\frak G}=\emptyset$ and
$1_{\frak G}=\Bbb R$ respectively, then $U\cap V\ne\emptyset$.
(ii) Show that if $U$ is an open set in $\frak G$ containing
$\emptyset$ then there are $G$, $H\in U$ such that
$H=\Bbb R\setminus\overline{G}$.
(iii) Show that $\{\emptyset\}$ is a G$_{\delta}$ set in $\frak G$.
(iv) Show that there is no non-zero Maharam submeasure on $\frak G$.
(v) Show that there is no non-zero countably additive functional on
$\frak G$.
%393Q

\spheader 393Yh In 393Xj, show that each of the sets $A_n$ must have
non-zero intersection number.
%393Xj 393S

\spheader 393Yi Let $\frak A$ be an atomless Boolean algebra with countable
Maharam type.   Show that there is a submeasure $\mu$ on
$\frak A$, order-continuous on the left, such that whenever
$a\in\frak A\setminus\{0\}$ there is a $b\Bsubseteq a$ such that
$\mu b<\mu a$.
%G\lowcy\'nski 08, 3.6  mt39bits
}%end of exercises

\cmmnt{\Notesheader{393}
For many years it was not known whether there were any Maharam algebras
which were not measurable algebras;  this was the famous `control measure
problem', eventually solved by M.Talagrand.   I will present his
example in the next section.   We now know that we have a larger class, but
it remains very poorly understood, and the material presented here must be
regarded as work in progress.   As in \S\S391-392, the stimulus for these
ideas has been the attempt to characterize measurable algebras in more or
less algebraic terms.   If we are prepared to allow
order*-convergence of sequences to be an `algebraic' notion, then 393Xj is
such a characterization;  but it shares with Kelley's criterion 391K the
need for a sequence $\sequencen{A_n}$, covering $\frak A^+$, with defined
properties.   The advance, if any, is that the properties ($\alpha$) and
($\beta$) of 393Xj are a good deal farther from any formula for a measure.

The first few results of this section, down to 393G, are concerned with
checking that Maharam algebras share properties with measurable algebras,
and the proofs use the same ideas, with occasional minor modifications.
In 393H we have to think a little, since exhaustivity is
less familiar, and harder to apply, than additivity.   From this
proposition we see that exhaustive submeasures are to uniformly exhaustive
submeasures something like
what Maharam algebras are to measurable algebras.
393K is a further example of a well-known construction -- this time,
convergence in measure -- which has a version based on Maharam algebras.

In \S367 I examined order*-convergence in Riesz spaces, without
explicitly discussing the associated topology, and in
393L-393Q %393L 393M 393O 393P 393N 393Q
here I look at
Boolean algebras.   In both cases the usefulness of the idea starts with
the fact that the algebraic operations are separately
continuous (367C, 393M), which is itself a consequence of the strong
distributive laws in 313A-313B and 352E.
It is easy to see that in a Maharam algebra the order-sequential topology
is the Maharam-algebra topology (393N).   What is remarkable is that
natural questions about the order-sequential topology lead to
characterizations of Maharam algebras (393Q).
This leads directly to an astonishing algebraic characterization of Maharam
algebras (393S).   (But once again we
need to hypothesize the existence of a
suitable sequence of sets covering $\frak A^+$.)
}

\discrpage

