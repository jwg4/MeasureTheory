\frfilename{mt493.tex}
\versiondate{4.1.13}
\copyrightdate{2001}

\def\Isom{\mathop{\text{Isom}}}

\def\chaptername{Further topics}
\def\sectionname{Extremely amenable groups}

\newsection{493}

A natural variation on the idea of `amenable group'\cmmnt{ (\S449)} is
the concept of
`extremely amenable' group (493A).   Expectedly, most of the ideas of
449C-449E %449C 449D 449E
can be applied to extremely amenable groups
(493B);  unexpectedly, we find not only that there are interesting
extremely amenable groups, but that we need some of the central ideas of
measure theory to study them.   I give a criterion for extreme
amenability of a group in terms of the existence of suitably
concentrated measures (493C) before turning to three examples:  measure
algebras under symmetric difference (493D), $L^0$ spaces (493E)
and isometry groups
of spheres in infinite-dimensional Hilbert spaces (493G).

\leader{493A}{Definition} Let $G$ be a topological group.   Then $G$ is
{\bf extremely amenable} or has the {\bf fixed point on compacta
property} if every continuous action of $G$ on a compact
Hausdorff space has a fixed point.

\leader{493B}{Proposition} (a) Let $G$ and $H$ be topological groups
such that there is a continuous surjective homomorphism from $G$ onto
$H$.   If $G$ is extremely amenable, so is $H$.

(b) Let $G$ be a topological group and suppose that there is a dense
subset $A$ of $G$ such that every finite subset of $A$ is included in an
extremely amenable subgroup of $G$.    Then $G$ is extremely
amenable.

(c) Let $G$ be a topological group with an extremely amenable normal
subgroup $H$ such that $G/H$ is extremely amenable.   Then $G$ is
extremely amenable.

(d) The product of any family of extremely amenable topological groups
is extremely amenable.

(e)\dvAformerly{4{}93C} Let $G$ be a topological group.   Then $G$ is
extremely amenable iff there is a point in the greatest ambit $Z$ of
$G$\cmmnt{ (definition:  449D)} which is fixed by the action of $G$ on
$Z$.

(f)\dvAformerly{4{}93Xa} Let $G$ be an extremely amenable topological group.
Then every dense subgroup of $G$ is extremely amenable.

\proof{ We can use the same arguments as in 449C-449F, %449C 449D 449E 449F
with some simplifications.

\medskip

{\bf (a)} As in 449Ca, let $\phi:G\to H$ be a continuous surjective
homomorphism, $X$ a non-empty compact Hausdorff space and
$\action:H\times X\to X$ a continuous action.   Let $\action_1$ be the
continuous action of $G$ on $X$ defined by the formula
$a\action_1x=\phi(a)\action x$.   Then any fixed point for $\action_1$
is a fixed point for $\action$.

\medskip

{\bf (b)} Let $X$ be a non-empty compact Hausdorff space and
$\action$ a continuous action of $G$ on $X$.   For $I\in[A]^{<\omega}$
let $H_I$ be an extremely amenable subgroup of $G$ including $I$.   The
restriction of the action to $H_I\times X$ is
a continuous action of $H_I$ on $X$, so

\Centerline{$\{x:a\action x=x$ for every $a\in I\}
\supseteq\{x:a\action x=x$ for every $a\in H_I\}$}

\noindent is closed and non-empty.   Because $X$ is compact, there is an
$x\in X$ such that $a\action x=x$ for every $a\in A$.
Now $\{a:a\action x=x\}$ includes the dense set $A$, so is the whole
of $G$, and $x$ is fixed under the action of $G$.   As $X$ and $\action$
are arbitrary, $G$ is extremely amenable.

\medskip

{\bf (c)} Let $X$ be a compact Hausdorff space and
$\action$ a continuous action of $G$ on $X$.   Set
$Q=\{x:x\in X,\,a\action x=x$ for every $a\in H\}$;  then $Q$ is a
closed subset of $X$ and, because $H$ is extremely amenable, is
non-empty.   Next, $b\action x\in Q$ for every $x\in Q$ and $b\in G$.
\Prf\ If $a\in H$, then $b^{-1}ab\in H$ and

\Centerline{$a\action(b\action x)=(ab)\action x
=(bb^{-1}ab)\action x=b\action((b^{-1}ab)\action x)=b\action x$.}

\noindent As $a$ is
arbitrary, $b\action x\in Q$.\ \QeD\    Accordingly we have a continuous
action of $G$ on the compact Hausdorff space $Q$.

If $b\in G$, $a\in H$ and $x\in Q$, then $(ba)\action x=b\action x$.
So we have an action of $G/H$ on $Q$ defined by saying that
$b^{\ssbullet}\action x=b\action x$ for every $b\in G$ and $x\in Q$, and
this is continuous for the quotient topology on $G/H$, as in the proof
of 449Cc.   Because $G/H$ is extremely
amenable, there is a point $x$ of $Q$ which is fixed
under the action of $G/H$.   So $b\action x=b^{\ssbullet}\action x=x$
for every
$b\in G$, and $x$ is fixed under the action of $G$.
As $X$ and $\action$ are arbitrary, $G$ is extremely amenable.

\medskip

{\bf (d)} By (c), the product of two extremely amenable topological
groups is extremely amenable, since each can be regarded as a normal
subgroup of the product.   It follows that the product of finitely
many extremely amenable topological groups is extremely amenable.   Now
let $\familyiI{G_i}$
be any family of extremely amenable topological groups with product $G$.
For finite $J\subseteq I$ let $H_J$ be the set of those $a\in G$ such
that $a(i)$ is the identity in $G_i$ for every $i\in I\setminus J$.
Then $H_J$ is isomorphic (as topological group) to $\prod_{i\in J}G_i$,
so is extremely amenable.   Since $\{H_J:J\in[I]^{<\omega}\}$ is an
upwards-directed
family of subgroups of $G$ with dense union, (b) tells us that $G$ is
extremely amenable.

\medskip

{\bf (e)} Repeat the arguments of 449E(i)$\Leftrightarrow$(ii),
noting that if
$z_0\in Z$ is a fixed point under the action of $G$ on $Z$, then its
images under the canonical maps $\phi$ of 449Dd will be fixed for other
actions.

\medskip

{\bf (f)} Again, the idea is to repeat the argument of 449F(a-ii).
As there, let $H$ be a dense subgroup of $G$,
$U$ the space of bounded real-valued functions on $G$ which are uniformly
continuous for the right uniformity, and $V$ the space of bounded
real-valued functions on $H$ which are uniformly continuous for the right
uniformity.   As in 449F(a-ii), we have an extension operator $T:V\to U$
defined
by saying that $Tg$ is the unique continuous extension of $g$ for every
$g\in V$;  and $b\action_lTg=T(b\action_lg)$ for every $b\in H$ and
$g\in V$.   Now $T$ is a Riesz homomorphism.   So if $z\in Z$ is fixed by
the action of $G$, that is, $z(a\action_lf)=z(f)$ for every $a\in G$ and
$f\in U$, then $zT:V\to\Bbb R$ is a Riesz homomorphism, with
$z(T\chi H)=1$, and $z(T(b\action_lg))=z(b\action_lTg)=z(Tg)$ whenever
$g\in V$ and $b\in H$.   Thus $zT$ is a fixed point of the greatest ambit
of $H$, and $H$ is extremely amenable.
}%end of proof of 493B

\leader{493C}{Theorem}\discrversionA{\footnote{Formerly 4{}93D;
revised 2009.}}{}
Let $G$ be a topological group and $\Cal B$ its
Borel $\sigma$-algebra.   Suppose that
for every $\epsilon>0$, open neighbourhood $V$ of the identity of $G$,
finite set $I\subseteq G$ and finite family $\Cal E$ of zero sets in $G$
there is a finitely additive functional $\nu:\Cal B\to[0,1]$ such that
$\nu G=1$ and

(i) $\nu(VF)\ge 1-\epsilon$ whenever $F\in\Cal E$ and
$\nu F\ge\bover12$,

%(ii) for every $a\in I$ there is a $b\in Va$ such that
(ii) for every $a\in I$ there is a $b\in aV$ such that
$|\nu(bF)-\nu F|\le\epsilon$ for every $F\in\Cal E$.

\noindent Then $G$ is extremely amenable.

\proof{{\bf (a)} Write $P$ for the set of finitely additive functionals
$\nu:\Cal B\to[0,1]$ such that $\nu G=1$.   If $V$ is an open
neighbourhood of the identity $e$ of $G$, $\epsilon>0$,
$I\in[D]^{<\omega}$ and $\Cal E$ is a finite family of zero sets in $G$,
let $A(V,\epsilon,I,\Cal E)$ be the set of those $\nu\in P$ satisfying
(i) and (ii) above.   Our hypothesis is that none of these sets
$A(V,\epsilon,I,\Cal E)$ are empty;  since
$A(V,\epsilon,I,\Cal E)\subseteq A(V',\epsilon',I',\Cal E')$ whenever
$V\subseteq V'$, $\epsilon\le\epsilon'$, $I\supseteq I'$ and
$\Cal E\supseteq\Cal E'$, there is an ultrafilter $\Cal F$ on $P$
containing all these sets.

Let $U$ be the space of bounded real-valued functionals on $G$ which are
uniformly continuous for the right uniformity on $G$.   If we identify
$L^{\infty}(\Cal B)$ with the space of bounded Borel measurable
real-valued functions on $G$ (363H), then $U$ is a Riesz subspace of
$L^{\infty}(\Cal B)$.   For each $\nu\in P$ we have a positive linear
functional $\dashint\,d\nu:L^{\infty}(\Cal B)\to\Bbb R$ (363L).   For
$f\in U$ set $z(f)=\lim_{\nu\to\Cal F}\dashint fd\nu$.

\medskip

{\bf (b)} $z:U\to\Bbb R$ is a Riesz homomorphism, and $z(\chi G)=1$.
\Prf\ Of course $z$ is a positive linear functional taking the value
$1$ at $\chi G$, just because all the integrals $\dashint\,d\nu$ are.   Now
suppose that $f_0$, $f_1\in U$ and $f_0\wedge f_1=0$.   Take any
$\epsilon>0$.   Then there is an open neighbourhood $V$ of $e$ such that
$|f_i(x)-f_i(y)|\le\epsilon$ whenever $xy^{-1}\in V$ and $i\in\{0,1\}$.
Set $F_i=\{x:f_i(x)=0\}$, $E_i=VF_i$ for each
$i$.   Then $F_0\cup F_1=X$, so $\nu F_0+\nu F_1\ge 1$ for every
$\nu\in P$, and there is a $j\in\{0,1\}$ such that
$A_0=\{\nu:\nu F_j\ge\bover12\}\in\Cal F$.   Next,

\Centerline{$A_1=\{\nu$:
 if $\nu F_j\ge\Bover12$ then $\nu(VF_j)\ge 1-\epsilon\}$}

\noindent belongs to $\Cal F$.   Accordingly
$\lim_{\nu\to\Cal F}\nu E_j\ge 1-\epsilon$.   As $f_j(x)\le\epsilon$ for
every $x\in E_j$,

\Centerline{$z(f_j)=\lim_{\nu\to\Cal F}\dashint f_jd\nu
\le\epsilon(1+\|f_j\|_{\infty})$.}

\noindent This shows that
$\min(z(f_0),z(f_1))\le\epsilon(1+\|f_1\|_{\infty}+\|f_2\|_{\infty})$.
As $\epsilon$ is arbitrary, $\min(z(f_0),z(f_1))=0$;  as $f_0$ and $f_1$
are arbitrary, $z$ is a Riesz homomorphism (352G(iv)).\ \Qed

Thus $z$ belongs to the greatest ambit $Z$ of $G$.

\medskip

{\bf (c)} $\lim_{\nu\to\Cal F}\dashint (a^{-1}\action_lf)d\nu
=\lim_{\nu\to\Cal F}\dashint fd\nu$ for every
non-negative $f\in U$ and $a\in G$.   \Prf\ Take any $\epsilon>0$.   Let
$V$ be an open neighbourhood of $e$ such that $|f(x)-f(y)|\le\epsilon$
whenever $x\in Vy$;  then

\Centerline{$\|a^{-1}\action_lf-b^{-1}\action_lf\|_{\infty}
=\sup_{x\in G}|f(ax)-f(bx)|\le\epsilon$}

\noindent whenever $b\in Va$.   For $n\in\Bbb N$ set
$F_n=\{x:x\in G,\,f(x)\ge n\epsilon\}$.   Set
$m=\lfloor\bover1\epsilon\|f\|_{\infty}\rfloor$, so that $F_n=\emptyset$
for every $n>m$.   Set $\delta=\bover1{m+1}$,

$$\eqalign{A
&=\{\nu:\text{ there is a }b\in Va\text{ such that }
|\nu(b^{-1}F_n)-\nu F_n|\le\delta\text{ for every }n\le m\}\cr
&=\{\nu:\text{ there is a }c\in a^{-1}V^{-1}\text{ such that }
|\nu(cF_n)-\nu F_n|\le\delta\text{ for every }n\le m\}
\in\Cal F.\cr}$$

\noindent Take any $\nu\in A$ and $b\in Va$
such that $|\nu(b^{-1}F_n)-\nu F_n|\le\delta$ for every $n\le m$.
Then, setting $g=\sum_{n=1}^m\epsilon\chi F_n$, we have
$g\in L^{\infty}(\Cal B)$ and $g\le f\le g+\epsilon\chi G$.   Since
$b^{-1}\action_lg$ (in the language of 4A5Cc)
% (a\action_lf)(y)=f(a^{-1}y)
is just $\sum_{n=1}^m\epsilon\chi(b^{-1}F_n)$, we have

$$\eqalignno{|\dashint a^{-1}\action_lf\,d\nu-\dashint f\,d\nu|
&\le\epsilon+|\dashint b^{-1}\action_lf\,d\nu-\dashint f\,d\nu|
\le 3\epsilon+|\dashint b^{-1}\action_lg\,d\nu-\dashint g\,d\nu|\cr
\displaycause{because $\|b^{-1}\action_lg-b^{-1}\action_lf\|_{\infty}
=\|g-f\|_{\infty}\le\epsilon$}
&\le 3\epsilon+\epsilon\sum_{n=1}^m|\nu F_n-\nu(b^{-1}F_n)|
\le 3\epsilon+m\epsilon\delta
\le 4\epsilon.\cr}$$

\noindent As $A\in\Cal F$,

\Centerline{$|\lim_{\nu\to\Cal F}
  \dashint a^{-1}\action_lfd\nu-\dashint fd\nu|
\le 5\epsilon$;}

\noindent as $\epsilon$ is arbitrary,
$\lim_{\nu\to\Cal F}\dashint a^{-1}\action_lf\,d\nu
=\lim_{\nu\to\Cal F}\dashint fd\nu$.\ \Qed

\medskip

{\bf (d)} Thus, for any $a\in G$,

\Centerline{$(a\action z)(f)
=z(a^{-1}\action_lf)
=\lim_{\nu\to\Cal F}\dashint a^{-1}\action_lf\,d\nu
=\lim_{\nu\to\Cal F}\dashint f\,d\nu
=z(f)$}

\noindent for every non-negative $f\in U$ and therefore for every $f\in U$,
and $a\action z=z$.   So
$z\in Z$ is fixed under the action of $G$ on $Z$;  by
493Ba, this is enough to ensure that $G$ is extremely amenable.
}%end of proof of 493C

\leader{493D}{}\dvAformerly{4{}93E}\cmmnt{ I
turn now to examples of extremely amenable
groups.   The first three are groups which we have already studied for
other reasons.

\medskip

\noindent}{\bf Theorem} Let $(\frak A,\bar\mu)$ be an atomless measure
algebra.   Then $\frak A$, with the group operation $\Bsymmdiff$ and the
measure-algebra topology\cmmnt{ (definition:  323A)}, is an extremely
amenable group.

\proof{{\bf (a)} To begin with let us suppose that $(\frak A,\bar\mu)$
is a probability algebra;  write $\sigma$ for the measure metric of
$\frak A$, so that $\sigma(a,a')=\bar\mu(a\Bsymmdiff a')$ for $a$,
$a'\in\frak A$.   I seek to apply 493C.

\medskip

\quad{\bf (i)} Let $V$ be an open neighbourhood of $0$ in $\frak A$,
$\epsilon\in\ooint{0,3}$,
$I\in[\frak A]^{<\omega}$ and $\Cal E$ a finite family of
zero sets in $\frak A$.   Let $\gamma>0$ be such that
$V\supseteq\{a:\bar\mu a\le 2\gamma\}$.   Let $\frak B_0$ be the finite
subalgebra of $\frak A$ generated by
$I$ and $B_0$ the set of atoms in $\frak B_0$.   Set

\Centerline{$t=\Bover1{\gamma}\ln\Bover3{\epsilon}$,
\quad$n=\lceil\max(t^2,\sup_{b\in B_0}\Bover1{\bar\mu b})\rceil$.}

\noindent Because $\frak A$ is atomless, we can split any member of
$\frak A\setminus\{0\}$ into two parts of equal measure (331C);  if,
starting from the disjoint set $B_0$, we successively split the largest
elements until we have a disjoint set $B$ with just $n$ elements, then
we shall have $\bar\mu b\le\Bover2n$ for every $b\in B$.   We have a
natural identification between $\{0,1\}^B$ and the subalgebra $\frak B$
of $\frak A$ generated by $B$, matching $x\in\{0,1\}^B$ with
$f(x)=\sup\{b:b\in B$, $x(b)=1\}$.   Writing $\rho$ for the normalized
Hamming metric on $\{0,1\}^B$ (492D), we have
$\sigma(f(x),f(y))\le 2\rho(x,y)$ for all $x$,
$y\in\{0,1\}^B$.   \Prf\ Set $J=\{b:b\in B,\,x(b)\ne y(b)\}$, so that

\Centerline{$\sigma(f(x),f(y))
=\bar\mu(f(x)\Bsymmdiff f(y))=\bar\mu(\sup J)
=\sum_{b\in J}\bar\mu b
\le\Bover2n\#(J)=2\rho(x,y)$.   \Qed}

\medskip

\quad{\bf (ii)} Let $\nu_B$ be the usual measure on $\{0,1\}^B$ and
set $\lambda E=\nu_B f^{-1}[E]$ for every Borel set $E\subseteq\frak A$.
Then $\lambda$ is a probability measure.   Note that
$f:\{0,1\}^B\to\frak B$
is a group isomorphism if we give $\{0,1\}^B$ the addition $+_2$
corresponding to its identification with $\Bbb Z_2^B$, and $\frak B$ the
operation $\Bsymmdiff$.   Because $\nu_B$ is translation-invariant for
$+_2$, its copy, the subspace measure $\lambda_{\frak B}$ on the
$\lambda$-conegligible finite set $\frak B$, is translation-invariant for
$\Bsymmdiff$.   But this means that
$\lambda\{b\Bsymmdiff d:d\in F\}=\lambda F$
whenever $b\in\frak B$ and $F\subseteq\frak B$, and therefore that
$\lambda\{b\Bsymmdiff d:d\in F\}=\lambda F$ whenever $b\in I$ and
$F\in\Cal E$.
This shows that $\lambda$ satisfies condition (ii) of 493C.

\medskip

\quad{\bf (iii)} Now suppose that $F\in\Cal E$ and that
$\lambda F\ge\bover12$.   Set $W=f^{-1}[F]$, so that $\nu_B W\ge\bover12$.
By 492D,

\Centerline{$\int e^{t\rho(x,W)}\nu_B(dx)\le 2e^{t^2/4n}
\le 2e^{1/4}\le 3$,}

\noindent so

\Centerline{$\nu_B\{x:\rho(x,W)\ge\gamma\}
=\nu_B\{x:t\rho(x,W)\ge\ln\bover3{\epsilon}\}
=\nu_B\{x:e^{t\rho(x,W)}\ge\bover3{\epsilon}\}
\le\epsilon$.}

\noindent Accordingly

$$\eqalignno{\lambda\{a\Bsymmdiff d:a\in V,\,d\in F\}
&\ge\lambda\{a:\sigma(a,F)\le 2\gamma\}
=\nu_B\{x:\sigma(f(x),F)\le 2\gamma\}\cr
&\ge\nu_B\{x:\sigma(f(x),f[W])\le 2\gamma\}
\ge\nu_B\{x:\rho(x,W)\le\gamma\}\cr
\displaycause{because $f$ is $2$-Lipschitz}
&\ge 1-\epsilon.\cr}$$

\noindent So $\lambda$ also satisfies (i) of 493C.

\medskip

\quad{\bf (iv)} Since $V$, $\epsilon$, $I$ and $\Cal E$ are arbitrary,
493C tells us that $\frak A$ is an extremely amenable group, at least
when $(\frak A,\bar\mu)$ is an atomless probability algebra.

\medskip

{\bf (b)} For the general case, observe first that if
$(\frak A,\bar\mu)$ is atomless and totally finite then
$(\frak A,\Bsymmdiffshort)$ is an extremely amenable group;
this is trivial
if $\frak A=\{0\}$, and otherwise there is a probability measure on
$\frak A$ which induces the same topology, so we can apply (a).   For a
general atomless measure algebra $(\frak A,\bar\mu)$, set
$\frak A^f=\{c:c\in\frak A,\,\bar\mu c<\infty\}$ and for $c\in\frak A^f$
let $\frak A_c$ be the principal ideal generated by $c$.   Then
$\frak A_c$ is a subgroup of $\frak A$ and the measure-algebra topology
of $\frak A_c$, regarded as a measure algebra in itself, is the subspace
topology induced by the measure-algebra topology of $\frak A$.   So
$\{\frak A_c:c\in\frak A^f\}$ is an upwards-directed family of extremely
amenable subgroups of $\frak A$ with union which is dense in $\frak A$,
so $\frak A$ itself is extremely amenable, by 493Bb.   This completes
the proof.
}%end of proof of 493D

\leader{493E}{Theorem}\dvAformerly{4{}93F}\cmmnt{ ({\smc Pestov 02})} Let
$(X,\Sigma,\mu)$ be an atomless measure
space.   Then $L^0(\mu)$, with the group operation $+$ and the topology
of convergence in measure, is an extremely amenable group.

\proof{ It will simplify some of the formulae if we move at once to the
space $L^0(\frak A)$, where $(\frak A,\bar\mu)$ is the measure algebra
of $(X,\Sigma,\mu)$;  for the identification of $L^0(\frak A)$ with
$L^0(\mu)$ see 364Ic;
for a note on convergence in measure in
$L^0(\frak A)$, see 367L;
of course $\frak A$ is atomless if $(X,\Sigma,\mu)$ is (322Bg).

\medskip

{\bf (a)} I seek to prove that $S(\frak A)$, with the
group operation of addition and the topology of convergence in measure,
is extremely amenable.
As in 493D, I start with the case in which $(\frak A,\bar\mu)$
is a probability algebra, and use 493C.

\medskip

\quad{\bf (i)} Take an open neighbourhood $V$ of $0$ in $S(\frak A)$, an
$\epsilon\in\ooint{0,3}$,
a finite set $I\subseteq S(\frak A)$ and a finite family
$\Cal E$ of zero sets in $S(\frak A)$.   Let $\gamma>0$ be such that
$u\in V$ whenever $u\in S(\frak A)$ and
$\bar\mu\Bvalue{u\ne 0}\le 2\gamma$.   Let $\frak B_0$ be a finite
subalgebra of $\frak A$ such that $I$ is included in the linear subspace
of $S(\frak A)$ generated by $\{\chi b:b\in\frak B_0\}$, and $B_0$ the
set of atoms of $\frak B_0$.   As in the proof of 493D, set

\Centerline{$t=\Bover1{\gamma}\ln\Bover3{\epsilon}$,
\quad$n=\lceil\max(t^2,\sup_{b\in B_0}\Bover1{\bar\mu b})\rceil$,}

\noindent and let $B\subseteq\frak A\setminus\{0\}$ be a partition of
unity with $n$ elements, refining $B_0$, such that
$\bar\mu b\le\bover2n$ for every $b\in B$.   We have a natural
identification between $\BbbR^B$ and the linear subspace of $S(\frak A)$
generated by $\{\chi b:b\in B\}$, matching $x\in\BbbR^B$ with
$f(x)=\sum_{b\in B}x(b)\chi b$, which is continuous if $\BbbR^B$ is
given its product topology.   Writing $\rho$ for the normalized Hamming
metric on $\BbbR^B$, we have

\Centerline{$\bar\mu\Bvalue{f(x)\ne f(y)}=\sum_{x(b)\ne y(b)}\bar\mu b
\le\Bover2n\#(\{b:x(b)\ne y(b)\})=2\rho(x,y)$}

\noindent for all $x$, $y\in\BbbR^B$.

\medskip

\quad{\bf (ii)} Set $\beta=\sup_{v\in I}\|v\|_{\infty}$ (if
$I=\emptyset$, take $\beta=0$).   Let $M>0$ be so large that
$(M+\beta)^n\le(1+\bover12\epsilon)M^n$.   On $\Bbb R$, write $\mu_L$
for Lebesgue measure and $\mu_L'$ for the
indefinite-integral measure over $\mu_L$ defined by the function
$\Bover1{2M}\chi[-M,M]$, so that $\mu_L'E=\Bover1{2M}\mu_L(E\cap[-M,M])$
whenever $E\subseteq\Bbb R$ and $E\cap[-M,M]$ is Lebesgue measurable.
Let $\lambda$, $\lambda'$ be the product measures on $\BbbR^B$ defined
from $\mu_L$ and $\mu'_L$.   Let $\nu$ be the Borel probability measure
on $S(\frak A)$ defined by setting $\nu F=\lambda'f^{-1}[F]$ for every
Borel set $F\subseteq S(\frak A)$.

Now $|\nu(v+F)-\nu F|\le\epsilon$ for every $v\in I$ and Borel set
$F\subseteq S(\frak A)$.   \Prf\ Because $B$ refines $B_0$, $v$ is
expressible as $f(y)$ for some $y\in\BbbR^B$;  because
$\|v\|_{\infty}\le\beta$, $|y(b)|\le\beta$ for every $b\in B$.   Because
$f:\BbbR^B\to S(\frak A)$ is linear, $f^{-1}[v+F]=y+f^{-1}[F]$.   Now

$$\eqalignno{|\nu(v+F)-\nu F|
&=|\lambda'f^{-1}[v+F]-\lambda'f^{-1}[F]|\cr
&=\Bover1{(2M)^n}
  \bigl|\lambda(f^{-1}[v+F]\cap[-M,M]^n)
    -\lambda(f^{-1}[F]\cap[-M,M]^n)\bigr|\cr
\displaycause{use 253I, or otherwise}
&=\Bover1{(2M)^n}
  \bigl|\lambda((y+f^{-1}[F])\cap[-M,M]^n)
      -\lambda(f^{-1}[F]\cap[-M,M]^n)\bigr|\cr
&=\Bover1{(2M)^n}
  \bigl|\lambda(f^{-1}[F]\cap([-M,M]^n-y))
      -\lambda(f^{-1}[F]\cap[-M,M]^n)\bigr|\cr
&\le\Bover1{(2M)^n}
  \lambda(([-M,M]^n-y)\symmdiff [-M,M]^n)\cr
&=\Bover2{(2M)^n}
  \lambda(([-M,M]^n-y)\setminus [-M,M]^n)\cr
&\le\Bover2{(2M)^n}
  \lambda(([-M-\beta,M+\beta]^n\setminus [-M,M]^n)\cr
&=\Bover2{M^n}((M+\beta)^n-M^n)
\le\epsilon.  \text{ \Qed}\cr}$$

\noindent So $\nu$ satisfies (ii) of 493C.

\medskip

\quad{\bf (iii)} Now suppose that $F\in\Cal E$ and $\nu F\ge\bover12$.
Set $W=f^{-1}[F]$, so that $\lambda'W\ge\bover12$.   Just as in the
proof of 493D,
$\int e^{t\rho(x,W)}\lambda'(dx)\le 2e^{t^2/4n}\le 3$, so

\Centerline{$\lambda'\{x:\rho(x,W)\ge\gamma\}
=\lambda'\{x:e^{t\rho(x,W)}\ge\bover3{\epsilon}\}\le\epsilon$,}

\noindent and

$$\eqalignno{\nu\{v+u:v\in V,\,u\in F\}
&\ge\nu\{w:\,\Exists u\in F,\,\bar\mu\Bvalue{u\ne w}\le 2\gamma\}\cr
&\ge\lambda'\{x:\rho(x,W)\le\gamma\}
\ge 1-\epsilon.\cr}$$

\noindent So $\nu$ also satisfies (i) of 493C.

\medskip

\quad{\bf (iv)} Since $V$, $\epsilon$, $I$ and $\Cal E$ are arbitrary,
493C tells us that $S(\frak A)$ is an extremely amenable group, at least
when $(\frak A,\bar\mu)$ is an atomless probability algebra.

\medskip

{\bf (b)} The rest of the argument is straightforward, as in 493D.
First, $S(\frak A)$ is extremely amenable whenever $(\frak A,\bar\mu)$
is an atomless totally finite measure algebra.   For a general atomless
measure algebra
$(\frak A,\bar\mu)$, set $\frak A^f=\{c:\bar\mu c<\infty\}$.   For each
$c\in\frak A^f$, let $\frak A_c$ be the corresponding principal ideal of
$\frak A$.   Then we can identify $S(\frak A_c)$, as topological group,
with the linear subspace of $L^0(\frak A)$ generated by
$\{\chi a:a\in\frak A_c\}$, and it is extremely amenable.   Since
$\{S(\frak A_c):c\in\frak A^f\}$ is an upwards-directed family of
extremely amenable subgroups of $L^0(\frak A)$ with dense union in
$L^0(\frak A)$, $L^0(\frak A)$ itself is extremely amenable, by 493Bb,
as before.
}%end of proof of 493E

\leader{493F}{}\dvAformerly{4{}93I}\cmmnt{ Returning to the ideas of
\S476, we find another remarkable example of an extremely amenable
topological group.   I recall the notation of 476I.    Let $X$ be a
(real) inner product space.   $S_X$ will be the unit sphere
$\{x:x\in X,\,\|x\|=1\}$.   Let $H_X$ be the isometry group of $S_X$
with its topology of pointwise convergence.
When $X$ is finite-dimensional, it is isomorphic, as inner product
space, to $\BbbR^r$, where $r=\dim X$.   In this case $S_X$ is
compact, so (if $r\ge 1$) has a unique $H_X$-invariant Radon probability
measure $\nu_X$, which is strictly positive, and is a multiple of
$(r-1)$-dimensional Hausdorff measure;  also $H_X$ is compact\cmmnt{
(441Gb)}, so has a unique Haar probability measure $\lambda_X$.

\medskip

\noindent}{\bf Lemma} For any $m\in\Bbb N$ and any $\epsilon>0$, there
is an $r(m,\epsilon)\ge 1$ such that whenever $X$ is a
finite-dimensional inner product space over $\Bbb R$ of dimension at least
$r(m,\epsilon)$,
$x_0,\ldots,x_{m-1}\in S_X$, $Q_1$, $Q_2\subseteq H_X$
are closed sets and $\min(\lambda_XQ_1,\lambda_XQ_2)\ge\epsilon$, then
there are $f_1\in Q_1$, $f_2\in Q_2$ such that
$\|f_1(x_i)-f_2(x_i)\|\le\epsilon$ for every $i<m$.

\proof{ Induce on $m$.
For $m=0$, the result is trivial.   For the
inductive step to $m+1$, take $r(m+1,\epsilon)>r(m,\bover13\epsilon)$
such that whenever $r(m+1,\epsilon)\le\dim X<\omega$ and $A_1$,
$A_2\subseteq S_X$ and $\min(\nu^*_XA_1,\nu^*_XA_2)\ge\bover12\epsilon$
then there are $x\in A_1$, $y\in A_2$ such that
$\|x-y\|\le\bover13\epsilon$;  this is possible by 476L.

Now take any inner product space $X$ over $\Bbb R$ of finite dimension
$r\ge r(m+1,\epsilon)$, closed sets $Q_1$, $Q_2\subseteq H_X$ such that
$\min(\lambda_XQ_1,\lambda_XQ_2)\ge\epsilon$, and
$x_0,\ldots,x_m\in S_X$.
Let $Y$ be the $(r-1)$-dimensional subspace
$\{x:x\in X,\,\innerprod{x}{x_m}=0\}$, so that
$\dim Y\ge r(m,\bover13\epsilon)$, and for $i<m$ let $y_i\in Y$ be a
unit vector such that $x_i$ is a linear combination of $y_i$ and $x_m$.
Set $H'_Y=\{f:f\in H_X,\,f(x_m)=x_m\}$;  then $f\mapsto f\restr S_Y$ is
a topological group isomorphism from $H'_Y$ to $H_Y$.   \Prf\ (i) If
$f\in H'_Y$ and $x\in S_X$, then

\Centerline{$x\in S_Y\iff\innerprod{x}{x_m}=0
\iff\innerprod{f(x)}{f(x_m)}=0
\iff f(x)\in S_Y$,}

\noindent so $f\restr S_Y$ is a permutation of $S_Y$ and
belongs to $H_Y$.   (ii) If $g\in H_Y$, we can define $f\in H'_Y$ by
setting $f(\alpha x+\beta x_m)=\alpha g(x)+\beta x_m$ whenever
$x\in S_Y$ and $\alpha^2+\beta^2=1$, and $f\restr S_Y=g$.   (iii) Note
that $H'_Y$ is a closed subgroup of $H_X$, so in itself is a compact
Hausdorff topological group.   Since the map
$f\mapsto f\restr S_Y:H'_Y\to H_Y$ is a bijective continuous group
homomorphism between compact Hausdorff topological groups, it is a
topological group isomorphism.\ \Qed

Let $\lambda'_Y$ be the Haar probability measure of $H'_Y$.   Then
$\lambda_XQ_1=\int\lambda'_Y(H'_Y\cap f^{-1}Q_1)\lambda_X(df)$ (443Ue),
so $\lambda_XQ'_1\ge\bover12\epsilon$, where
$Q'_1=\{f:\lambda'_Y(H'_Y\cap f^{-1}Q_1)\ge\bover12\epsilon\}$.
Similarly, setting
$Q'_2=\{f:\lambda'_Y(H'_Y\cap f^{-1}Q_2)\ge\bover12\epsilon\}$,
$\lambda_XQ'_2\ge\bover12\epsilon$.   Next, setting $\theta(f)=f(x_m)$
for $f\in H_X$, $\lambda_X\theta^{-1}$ is an $H_X$-invariant Radon
probability measure on $S_X$ (443Ub), so must be equal to $\nu_X$.
Accordingly

\Centerline{$\nu_X(\theta[Q'_j])=\lambda_X(\theta^{-1}[\theta[Q'_j]])
\ge\lambda_XQ'_j\ge\Bover12\epsilon$}

\noindent for both $j$.

We chose $r(m+1,\epsilon)$ so large that we can be sure that there are
$z_1\in\theta[Q_1']$, $z_2\in\theta[Q_2']$ such that
$\|z_1-z_2\|\le\bover13\epsilon$.   Let $h_1\in Q'_1$, $h_2\in Q'_2$ be
such that $h_1(x_m)=\theta(h_1)=z_1$ and $h_2(x_m)=z_2$.
Let $h\in H_X$ be such that $h(z_1)=z_2$ and
$\|h(x)-x\|\le\bover13\epsilon$ for every $x\in S_X$ (4A4Jg).   Set
$\tilde h_2=hh_1$, so that $\tilde h_2(x_m)=z_2$ and
$\|h_1(x)-\tilde h_2(x)\|\le\bover13\epsilon$ for every $x\in S_X$.
Note that $\tilde h_2^{-1}h_2\in H'_Y$, so that $\tilde h_2$ and $h_2$
belong to the same left coset of $H'_Y$, and

\Centerline{$\lambda'_Y(H'_Y\cap\tilde h_2^{-1}Q_2)
=\lambda'_Y(H'_Y\cap h_2^{-1}Q_2)\ge\Bover12\epsilon$}

\noindent by 443Qa.
%shows that we have a measure indep of choice of representative

At this point, recall that $\dim Y\ge r(m,\bover13\epsilon)$, and that
$\lambda'_Y$ is a copy of $\lambda_Y$, the Haar probability measure on
$Y$.   So we have
$g_1\in H'_Y\cap h_1^{-1}Q_1$, $g_2\in H'_Y\cap\tilde h_2^{-1}Q_2$ such
that $\|g_1(y_i)-g_2(y_i)\|\le\bover13\epsilon$ for every $i<m$.   We
have $f_1=h_1g_1\in Q_1$ and $f_2=\tilde h_2g_2\in Q_2$.   For any
$i<m$,

$$\eqalign{\|f_1(y_i)-f_2(y_i)\|
&\le\|h_1g_1(y_i)-h_1g_2(y_i)\|
  +\|h_1g_2(y_i)-\tilde h_2g_2(y_i)\|\cr
&\le\|g_1(y_i)-g_2(y_i)\|+\Bover13\epsilon
\le\Bover23\epsilon.\cr}$$

\noindent Also $g_1(x_m)=g_2(x_m)=x_m$, so

\Centerline{$\|f_1(x_m)-f_2(x_m)\|
=\|h_1(x_m)-\tilde h_2(x_m)\|\le\Bover13\epsilon$.}

\noindent If $i<m$, then
$x_i=\innerprod{x_i}{x_m}x_m+\innerprod{x_i}{y_i}y_i$, so
$f_j(x_i)=\innerprod{x_i}{x_m}f_j(x_m)+\innerprod{x_i}{y_i}f_j(y_i)$ for
both $j$ (476J) and

$$\eqalign{\|f_1(x_i)-f_2(x_i)\|
&\le\Bover13\epsilon|\innerprod{x_i}{x_m}|
  +\Bover23\epsilon|\innerprod{x_i}{y_i}| \cr
&\le\sqrt{(\Bover13\epsilon)^2+(\Bover23\epsilon)^2}
  \sqrt{\innerprod{x_i}{x_m}^2+\innerprod{x_i}{y_i}^2}
\le\epsilon.\cr}$$

\noindent So $f_1$ and $f_2$ witness that the induction proceeds.
}%end of proof of 493F

\leader{493G}{Theorem}\dvAformerly{4{}93J} Let
$X$ be an infinite-dimensional inner product
space over $\Bbb R$.
Then the isometry group $H_X$ of its unit sphere $S_X$, with
its topology of pointwise convergence, is extremely amenable.

\proof{{\bf (a)}
Let $\Cal Y$ be the family of finite-dimensional subspaces of
$X$.   For $Y\in\Cal Y$, write $Y^{\perp}$ for the orthogonal complement
of $Y$, so that $X=Y\oplus Y^{\perp}$ (4A4Jf).   For $q\in H_Y$
define $\theta_Y(q):S_X\to S_X$ by saying that
$\theta_Y(q)(\alpha y+\beta z)=\alpha q(y)+\beta z$ whenever $y\in S_Y$,
$z\in S_{Y^{\perp}}$ and $\alpha^2+\beta^2=1$.   Then
$\theta_Y:H_Y\to H_X$ is a injective group homomorphism.   Also it is
continuous, because $q\mapsto\alpha q(y)+\beta z$ is continuous for all
relevant $\alpha$, $\beta$, $y$ and $z$.

If $Y$, $W\in\Cal Y$ and $Y\subseteq W$ then
$\theta_Y[H_Y]\subseteq\theta_W[H_W]$.   \Prf\ For any $q\in H_Y$ we can
define $q'\in H_W$ by saying that
$q'(\alpha y+\beta x)=\alpha q(y)+\beta x$ whenever $y\in S_Y$,
$x\in S_{W\cap Y^{\perp}}$ and $\alpha^2+\beta^2=1$.   Now
$\theta_Y(q)=\theta_W(q')\in\theta_W[H_W]$.\ \Qed

Set $G^*=\bigcup_{Y\in\Cal Y}\theta_Y[H_Y]$, so that $G^*$ is a subgroup
of $H_X$.

\medskip

{\bf (b)} Let $V$ be an open neighbourhood of the identity in $G^*$
(with the subspace topology inherited from the topology of pointwise
convergence on $H_X$), $\epsilon>0$ and
$I\subseteq G^*$ a finite set.   Then there is a Borel probability
measure $\lambda$ on $G^*$ such that

\quad(i) $\lambda(fQ)=\lambda Q$ for every $f\in I$ and every closed set
$Q\subseteq G^*$,

\quad(ii) $\lambda(VQ)\ge 1-\epsilon$ whenever $Q\subseteq G^*$ is
closed and $\lambda Q\ge\bover12$.

\noindent\Prf\ We may suppose that $\epsilon\le\bover12$.   Let
$J\in[S_X]^{<\omega}$ and $\delta>0$ be such that
$f\in V$ whenever $f\in G^*$ and
$\|f(x)-x\|\le\delta$ for every $x\in J$.   We may suppose that $J$ is
non-empty;  set $m=\#(J)$.
Let $Y\in\Cal Y$ be such that $J\subseteq Y$ and
$I\subseteq\theta_Y[H_Y]$
and $\dim Y=r\ge r(m,\epsilon)$, as chosen in 493F.   (This is
where we need to know that $X$ is infinite-dimensional.)    Set
$\lambda F=\lambda_Y\theta_Y^{-1}[F]$ for every Borel set
$F\subseteq G^*$, where $\lambda_Y$ is the Haar probability measure of
$H_Y$, as before.

If $f\in I$ and $F\subseteq G^*$ is closed, then

\Centerline{$\lambda(fF)=\lambda_Y\theta_Y^{-1}[fF]
=\lambda_Y(\theta_Y^{-1}(f)\theta_Y^{-1}[F])
=\lambda_Y\theta_Y^{-1}[F]=\lambda F$.}

\noindent So $\lambda$ satisfies condition (i).

\Quer\ Suppose, if possible, that $Q_1\subseteq G^*$ is a closed set
such that $\lambda Q_1\ge\bover12$ and $\lambda(VQ_1)<1-\epsilon$;  set
$Q_2=G^*\setminus VQ_1$.   Then
$\theta_Y^{-1}[Q_1]$ and $\theta_Y^{-1}[Q_2]$ are subsets of $H_Y$ both
of measure at least $\epsilon$.   Set
$R_j=\{q:q\in H_Y,\,q^{-1}\in\theta_Y^{-1}[Q_j]\}$ for each $j$;
because
$H_Y$ is compact, therefore unimodular,

\Centerline{$\lambda_YR_j
=\lambda_Y\theta_Y^{-1}[Q_j]=\lambda Q_j\ge\epsilon$}

\noindent for both $j$.   Because $\dim Y\ge r(m,\epsilon)$,
there are $q_1\in R_1$, $q_2\in R_2$ such that
$\|q_1(x)-q_2(x)\|\le\epsilon$ for $x\in J$.   Set
$f=\theta_Y(q_2^{-1}q_1)$.   If $x\in J$, then

\Centerline{$\|f(x)-x\|=\|q_2^{-1}q_1(x)-x\|
=\|q_1(x)-q_2(x)\|\le\epsilon$.}

\noindent As this is true whenever $x\in J$ and $f\in V$.   On the other
hand, $\theta_Y(q_1^{-1})\in Q_1$ and $\theta_Y(q_2^{-1})\in Q_2$ and
$f\theta_Y(q_1^{-1})=\theta_Y(q_2^{-1})$, so
$\theta_Y(q_2^{-1})\in VQ_1\cap Q_2$, which is impossible.\ \Bang

Thus $\lambda$ satisfies (ii).\ \Qed

\medskip

{\bf (c)} By 493C, $G^*$ is extremely amenable.   But $G^*$ is dense in
$H_X$.   \Prf\ If $f\in H_X$ and $I\subseteq S_X$ is finite and not
empty, let $Y_1$ be the linear subspace of $X$ generated by $I$, and let
$(y_1,\ldots,y_m)$ be an orthonormal basis of $Y_1$.   Set $z_j=f(y_j)$
for each $j$, so that $(z_1,\ldots,z_m)$ is orthonormal (476J);  let $Y$
be the linear subspace of $X$ generated by
$y_1,\ldots,y_m,z_1,\ldots,z_m$.   Set $r=\dim Y$ and extend the
orthonormal sets $(y_1,\ldots,y_m)$ and $(z_1,\ldots,z_m)$ to
orthonormal bases $(y_1,\ldots,y_r)$ and $(z_1,\ldots,z_r)$ of $Y$.
Then we have an isometric linear operator $T:Y\to Y$ defined by saying
that $Ty_i=z_i$ for each $i$;  set $q=T\restr S_Y\in H_Y$.   By 476J,
$q(x)=f(x)$ for every $x\in I$, so $\theta_Y(q)$ agrees with $f$ on $I$,
while $\theta_Y(q)\in G^*$.   As $f$ and $I$ are arbitrary, $G^*$ is
dense in $G$.\ \Qed

So 493Bb tells us that $H_X$ is extremely amenable, and the proof is
complete.
}%end of proof of 493G

\leader{493H}{}\dvAformerly{4{}93K}\cmmnt{ The
following result shows why extremely amenable groups did not appear in
Chapter 44.

\medskip

\noindent}{\bf Theorem}\cmmnt{ ({\smc Veech 77})} If $G$ is a locally
compact Hausdorff topological group with more than one element, it is
not extremely amenable.
%Veech, th. 2.2.1

\proof{ If $G$ is compact, this is trivial, since the left action of $G$
on itself has no fixed point;  so let us assume henceforth that $G$ is
not compact.

\medskip

{\bf (a)} Let $Z$ be the greatest ambit of $G$, $a\mapsto\hat a:G\to Z$
the canonical map, and $U$ the space of
bounded right-uniformly continuous real-valued functions on $G$.   (I
aim to show that the action
of $G$ on $Z$ has no fixed point.)   Take any $z^*\in Z$.   Let $V_0$ be
a compact neighbourhood of the identity $e$ in $G$, and let
$B_0\subseteq G$ be a maximal set such that $V_0b\cap V_0c=\emptyset$
for all distinct $b$, $c\in B_0$.   Then for any $a\in G$ there is a
$b\in B_0$ such that $V_0a\cap V_0b\ne\emptyset$, that is,
$a\in V_0^{-1}V_0B_0$.   So if we set
$Y_0=\overline{\{\hat b:b\in B_0\}}\subseteq Z$,
$\{a\action y:a\in V_0^{-1}V_0,\,y\in Y_0\}$ is a compact subset of $Z$
including $\{\hat a:a\in G\}$, and is therefore the whole of $Z$
(449Dc).   Let $a_0\in V_0^{-1}V_0$, $y_0\in Y_0$ be such that
$a_0\action y_0=z^*$, and set $B_1=a_0B_0$, $V_1=a_0V_0a_0^{-1}$;  then
$z^*\in\overline{\{\hat b:b\in B_1\}}$ and $V_1b\cap V_1c=\emptyset$ for
all distinct $b$, $c\in B_1$.

\medskip

{\bf (b)} Because $V_1$ is compact and $G$ is not compact, there is an
$a_1\in G\setminus V_1$.   Let $V_2\subseteq V_1$ be a neighbourhood of
$e$ such that $a_1^{-1}V_2V_2^{-1}a_1\subseteq V_1$.   Then we can
express $B_1$ as $D_0\cup D_1\cup D_2$ where
$a_1D_i\cap V_2D_i=\emptyset$ for all $i$.   \Prf\ Consider
$\{(b,c):b,\,c\in B_1,\,a_1b\in V_2c\}$.   Because
$V_2c\cap V_2c'\subseteq V_1c\cap V_1c'=\emptyset$ for all distinct $c$,
$c'\in B_1$, this is the graph of a function $h:D\to B_1$ for some
$D\subseteq B_1$.   \Quer\ If $h$ is not injective, we have distinct
$b$, $c\in B_1$ and $d\in B_1$ such that $a_1b$ and $a_1c$ both belong
to $V_2d$.   But in this case $b$ and $c$ both belong to $a_1^{-1}V_2d$
and $bc^{-1}\in a_1^{-1}V_2dd^{-1}V_2^{-1}a_1\subseteq V_1$ and
$b\in V_1c$, which is impossible.\ \BanG\  At the same time, if
$b\in B_1$, then
$a_1b\notin V_2b$ because $a_1\notin V_2$, so $h(b)\ne b$ for every
$b\in D$.

Let $D_0\subseteq D$ be a maximal set such that
$h[D_0]\cap D_0=\emptyset$, and set $D_1=h[D_0]$,
$D_2=B_1\setminus(D_0\cup D_1)$.   Then $h[D_0]\cap D_0=\emptyset$ by
the choice of $D_0$;  $h[D\cap D_1]\cap D_1=\emptyset$ because $h$ is
injective and $D_1\subseteq h[D\setminus D_1]$;  and
$h[D\cap D_2]\subseteq D_0$ because if $b\in D\cap D_2$ there must have
been some reason why we did not put $b$ into $D_0$, and it wasn't
because
$b\in h[D_0]$ or because $h(b)=b$.   So $h[D_i]\cap D_i=\emptyset$ for
all $i$, which is what was required.\ \Qed

\medskip

{\bf (c)} Since $z^*\in\overline{\{\hat b:b\in B_1\}}$, there must be
some $j\le 2$ such that $z^*\in\overline{\{\hat b:b\in D_j\}}$.   Now
recall that the right uniformity on $G$, like any uniformity, can be
defined by some family of pseudometrics (4A2Ja).  There is therefore a
pseudometric $\rho$ on $G$ such that
$W_{\epsilon}=\{(a,b):a,\,b\in G,\,\rho(a,b)\le\epsilon\}$ is a member
of the right uniformity on $G$ for every $\epsilon>0$ and
$W_1\subseteq\{(a,b):ab^{-1}\subseteq V_2\}$.   If now we set

\Centerline{$f(a)=\min(1,\rho(a,D_j))
=\min(1,\inf\{\rho(a,b):b\in D_j\})$}

\noindent for $a\in G$, $f:G\to\Bbb R$ is bounded and uniformly
continuous for the right uniformity, so belongs to $U$.   On the other
hand, if $b$, $c\in D_j$, then $a_1b\notin V_2c$, that is,
$a_1bc^{-1}\notin V_2$ and $\rho(a_1b,c)>1$;  as $c$ is arbitrary,
$f(a_1b)=1$.

\medskip

{\bf (d)} Now $\hat b(f)=f(b)=0$ for every $b\in D_j$, so $z^*(f)=0$.
On the other hand, because $z\mapsto a_1\action z$ is continuous,

\Centerline{$a_1\action z^*\in\overline{\{a_1\action\hat b:b\in D_j\}}
=\overline{\{\widehat{a_1b}:b\in D_j\}}$,}

\noindent so

\Centerline{$(a_1\action z^*)(f)\ge\inf_{b\in D_j}\widehat{a_1b}(f)
=\inf_{b\in D_j}f(a_1b)=1$,}

\noindent and $a_1\action z^*\ne z^*$.   As $z^*$ is arbitrary, this
shows that the action of $G$ on $Z$ has no fixed point, and $G$ is not
extremely amenable.
}%end of proof of 493H

\exercises{\leader{493X}{Basic exercises (a)}
%\spheader 493Xa
Let $G$ be a Hausdorff topological group, and $\widehat{G}$
its completion with respect to its bilateral uniformity.   Show that $G$
is extremely amenable iff $\widehat{G}$ is.
%493Bb 493Bf 449Xi

\sqheader 493Xb Let $X$ be a set with more than one member and $\rho$
the zero-one metric on $X$.
Let $G$ be the isometry group of $X$ with the topology of
pointwise convergence.   Show that $G$ is not extremely
amenable.   \Hint{give $X$ a total
ordering $\le$, and let $x$, $y$ be any two points of $X$.   For
$a\in G$ set $f(a)=1$ if $a^{-1}(x)<a^{-1}(y)$, $-1$ otherwise.   Show
that, in the language of 449D, $f\in U$.   Show that if $\cycle{x\,y}$
is the transposition exchanging $x$ and $y$ then
$\cycle{x\,y}\action_l f=-f$, while $|z(f)|=1$ for every $z$ in the
greatest ambit of $G$.} (Compare 449Xh.)
%493Be 449Xh

\spheader 493Xc Show that under the conditions of 493C there is a
finitely additive functional $\nu:\Cal B\to[0,1]$ such that
$\nu(aF)=\nu F$ for every $a\in G$ and every zero set $F\subseteq G$,
while $\nu(VF)=1$ whenever $V$ is a neighbourhood of the identity, $F$
is a zero set and $\nu F>\bover12$.
%493C

\spheader 493Xd\dvAnew{2013} Prove 493G for infinite-dimensional
inner product spaces over $\Bbb C$.
%493G

\spheader 493Xe Let $X$ be any (real or complex)
inner product space.   Show that the
isometry group of $X$, with its topology of pointwise convergence, is
amenable.   \Hint{449Cd.}
%493G

\spheader 493Xf Let $X$ be a separable Hilbert space.   (i) Show that the
isometry group $G$ of its unit sphere, with its topology of pointwise
convergence, is a Polish group.   (ii) Show that if $X$ is
infinite-dimensional, then every countable discrete group can be embedded
as a closed subgroup of $G$, so that $G$ is an extremely amenable Polish
group with a closed subgroup which is not amenable.   (Cf.\ 449K.)
%493G

\spheader 493Xg\dvAnew{2010} If $X$ is a (real or complex)
Hilbert space, a bounded linear
operator $T:X\to X$ is {\bf unitary} if it is an invertible isometry.
Show that the set of unitary operators on $X$, with its strong operator
topology (3A5I),
is an extremely amenable topological group.
%493G out of order

\spheader 493Xh Let $G$ be a topological group carrying Haar measures.
Show that it is extremely amenable iff its topology is the indiscrete
topology.   \Hint{443L.}
%493H

\leader{493Y}{Further exercises (a)}
%\spheader 493Ya
For a Boolean algebra $\frak A$ and a group $G$ with
identity $e$, write $S(\frak A;G)$ for the set of partitions of unity
$\family{g}{G}{a_g}$ in $\frak A$ such that $\{g:a_g\ne 0\}$ is finite.
For $\family{g}{G}{a_g}$, $\family{g}{G}{b_g}\in S(A)$, write
$\family{g}{G}{a_g}\cdot\family{g}{G}{b_g}=\family{g}{G}{c_g}$ where
$c_g=\sup\{a_h\Bcap b_{h^{-1}g}:h\in G\}$ for $g\in G$.   (i) Show that
under this operation $S(\frak A;G)$ is a group.   (ii) Show that if we
write $h\chi a$ for the member $\family{g}{G}{a_g}$ of $S(\frak A;G)$
such that $a_h=a$ and $a_g=0$ for other $g\in G$, then
$g\chi a\cdot h\chi b=(gh)\chi(a\Bcap b)$, and $S(\frak A;G)$ is
generated by $\{g\chi a:g\in G,\,a\in\frak A\}$.   (iii) Show that if
$\frak A=\Sigma/\Cal I$ where $\Sigma$ is an algebra of subsets of a set
$X$ and $\Cal I$ is an ideal of $\Sigma$, then $S(\frak A;G)$ can be
identified with a space of equivalence classes in a suitable subgroup of
$G^X$.   (iv) Devise a universal mapping theorem for the construction
$S(\frak A;G)$ which matches 361F in the case $(G,\cdot)=(\Bbb R,+)$.
(v) Now suppose that
$(\frak A,\bar\mu)$ is a measure algebra and that $G$ is a topological
group.   Show that we have a topology on $S(\frak A;G)$,
making it a topological group,
for which basic neighbourhoods of the identity
$e\chi 1$ are of the form $V(c,\epsilon,U)=\{\family{g}{G}{a_g}:
\bar\mu(c\Bcap\sup_{g\in G\setminus U}a_g)\le\epsilon\}$ with
$\bar\mu c<\infty$, $\epsilon>0$ and $U$ a neighbourhood of the identity
in $G$.   (vi) Show that if $G$ is an amenable locally compact Hausdorff
group and $(\frak A,\bar\mu)$ is an atomless measure algebra, then
$S(\frak A;G)$ is extremely amenable.   \Hint{{\smc Pestov 02}.}
*(vi) Explore possible constructions of spaces $L^0(\frak A;G)$.
(See {\smc Hartman \& Mycielski 58}.)
%493E

}%end of exercises

\endnotes{
\Notesheader{493} In writing this section I have relied heavily on
{\smc Pestov 99} and {\smc Pestov 02}, where you may find many further
examples of extremely amenable groups.   It is a striking fact that
while the theories of locally compact groups and extremely amenable
groups are necessarily almost entirely separate (493H), both are
dependent on measure theory.   Curiously, what seems to have been the
first non-trivial extremely amenable group to be described was found in
the course of investigating the Control Measure Problem
({\smc Herer \& Christensen 75}).

The theory of locally compact groups has for seventy years now been a
focal point for measure theory.   Extremely amenable groups have not yet
had such an influence.   But they encourage us to look again at
concentration-of-measure theorems, which are of the highest importance
for quite separate reasons.   In all the principal examples of
this section, and again in the further example to come in \S494,
we need concentration of measure in product spaces
(493D-493E and 494J), permutation groups (494I) or on spheres in
Euclidean space
(493G).   493D and 493E are special cases of a general result in
{\smc Pestov 02} (493Ya(vi)) which itself extends an idea from
{\smc Glasner 98}.   I note that 493D needs only concentration of measure
in $\{0,1\}^I$, while 493E demands something rather closer to the full
strength of Talagrand's theorem 492D.

I have expressed 493G as a theorem about the isometry groups of spheres
in infinite-dimensional inner product spaces;  of course these are
isomorphic to the orthogonal groups of the whole spaces with their
strong operator topologies (476Xd).   Adapting the basic
concentration-of-measure theorem 476K
to the required lemma 493F involves an instructive application of ideas
from \S443.
}%end of notes

\discrpage

