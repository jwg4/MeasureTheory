\frfilename{mt56.tex} 
\versiondate{3.1.15} 
\copyrightdate{2006} 
 
\def\chaptername{Choice and determinacy} 
 
\newchapter{56} 
 
Nearly everyone reading this 
book will have been taking the axiom of choice for granted nearly all the 
time.   This is the home territory of twentieth-century abstract analysis, 
and the one in which the great majority of the results 
have been developed. 
But I hope that everyone is aware that there are other ways of doing 
things.   In this chapter I want to explore what seem to me to be the most 
interesting alternatives.   In one sense they are minor variations on the 
standard approach, since I keep strictly to ideas expressible  
within the framework of Zermelo-Fraenkel set theory;  but in other ways 
they are dramatic enough to rearrange our prejudices. 
The arguments I will present in this chapter are mostly not  
especially difficult by the 
standards of this volume, but they do depend on intuitions for 
which familiar results which are likely to remain valid under the new 
rules being considered.    
 
Let me say straight away that the real aim of the chapter is \S567, on 
the axiom of determinacy. 
The significance of this axiom is that it is (so far) the most striking 
rival to the axiom of choice, in that it leads us quickly to a large number 
of propositions directly contradicting familiar theorems;  for instance, 
every subset of the real line is now Lebesgue measurable (567G).   But we 
need also to know which theorems are still true, and the first six sections 
of the chapter are devoted to a discussion of what can be done in ZF alone 
(\S\S561-565) and with countable or dependent choice (\S566).   Actually 
\S\S562-565 are rather off the straight line to \S567, because they examine 
parts of real analysis in which the standard proofs depend only on 
countable choice or less;  but a great deal more can be done than most of 
us would expect, and the methods are instructive. 
 
Going into details, \S561 looks at basic facts from real analysis, 
functional analysis and general topology which can be proved in ZF.   \S562 
deals with `codable' Borel sets and functions, using Borel codes to keep 
track of constructions for objects, so that if we know a sequence of  
codes we can avoid having to make a sequence of choices.   A `Borel-coded 
measure' (\S563) is now one which behaves well with  
respect to codable sequences of 
measurable sets;  for such a measure we have an integral with 
versions of the convergence theorems (\S564), and Lebesgue measure fits 
naturally into the structure (\S565).   In \S566, with ZF + AC($\omega$), 
we are back in familiar territory, and most of the results of Volumes 1 and 
2 can be proved if we are willing to re-examine some definitions and 
hypotheses.   Finally, in \S567, I look at infinite games and half a dozen 
of the consequences of AD, with a postscript on determinacy in the context 
of ZF + AC. 
 
\discrpage 
 
