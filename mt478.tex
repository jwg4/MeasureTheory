\frfilename{mt478.tex}
\versiondate{4.6.09}

\def\energy{\mathop{\text{energy}}\nolimits}
\def\hp{\mathop{\text{hp}}\nolimits}
\def\varinnerprod#1#2{#1\dotproduct#2}
\def\zz{}

\newdimen\pagegoalminus
\newdimen\bigbaselineskip
\bigbaselineskip=36pt

\def\checkbreak{\noalign{\global\advance\pagetotalplus by \bigbaselineskip
%\immediate\write0{pagetotalplus \the\pagetotalplus,
%pagegoalminus \the\pagegoalminus,
%bigbaselineskip \the\bigbaselineskip}
\ifdim\pagetotalplus>\pagegoalminus\break
  \global\pagetotalplus=0pt\global\pagegoalminus=\pagegoal\fi}}


\def\chaptername{Geometric measure theory}
\def\sectionname{Harmonic functions}

\newsection{478}

In this section and the next I will attempt an introduction to potential
theory.   This is an enormous subject and my choice of results is
necessarily somewhat arbitrary.   My principal aim is to give the most
elementary properties of Newtonian capacity, which will appear in \S479.
It seems that this necessarily involves a basic understanding of harmonic
and superharmonic functions.   I approach these by the `probabilistic'
route, using Brownian motion as described in \S477.

The first few paragraphs, down to 478J, do not
in fact involve Brownian motion;
they rely on multidimensional advanced calculus and on the Divergence
Theorem.   \cmmnt{(The latter is applied only to continuously
differentiable functions and domains of very simple types, so
we need far less than the quoted result in 475N.)}   Defining `harmonic
function' in terms of average values over concentric
spherical shells (478B), the
first step is to identify this with the definition in terms of the
Laplacian differential operator (478E).   An essential result is a formula
for a harmonic function inside a sphere in terms of its values on the
boundary and the `Poisson kernel' (478Ib),
and we also need to understand the effects of smoothing
by convolution with appropriate functions (478J\cmmnt{, following
473D-473E}).
I turn to Brownian motion with Dynkin's formula (478K), relating the
expected value of $f(X_{\tau})$ for
a stopped Brownian process $X_{\tau}$
to an integral in terms of $\nabla^2f$.   This is already enough to deal
with the asymptotic behaviour
of Brownian paths, which depends in a striking way on the dimension of the
space (478M).

We can now approach Dirichlet's problem.   If we have a bounded open set
$G\subseteq\BbbR^r$, there is a family $\family{x}{G}{\mu_x}$ of
probability measures such that whenever
$f:\overline{G}\to\Bbb R$ is continuous and $f\restr G$ is harmonic, then
$f(x)=\int fd\mu_x$ for every $x\in G$ (478Pc).   So this
family of `harmonic measures' gives a formula continuously extending a
function on $\partial G$ to a harmonic function on $G$, whenever
such an extension exists (478S).   The method used expresses $\mu_x$ in
terms of
the distribution of points at which Brownian paths starting at $x$ strike
$\partial G$, and relies on Dynkin's formula through Theorem 478O.   The
strong Markov property of Brownian motion now enables us to relate harmonic
measures associated with different sets (478R).

\leader{478A}{Notation} $r\ge 1$ will be an integer;
if you find it easier to focus on one dimensionality at a time, you should
start with $r=3$, because $r=1$ is too easy and $r=2$ is exceptional.
$\mu$ will be Lebesgue measure on $\BbbR^r$, and
$\|\,\|$ the Euclidean norm on $\BbbR^r$;  $\nu$ will be normalized
$(r-1)$-dimensional Hausdorff measure on $\BbbR^r$.
In the elementary case $r=1$, $\nu$ will be counting measure on $\Bbb R$.

\cmmnt{$\beta_r$ will be the volume of the unit ball in $\BbbR^r$,
that is,}

$$\eqalign{\beta_r
&=\Bover{1}{k!}\pi^k\text{ if }r=2k\text{ is even},\cr
&=\Bover{2^{2k+1}k!}{(2k+1)!}\pi^k\text{ if }r=2k+1\text{ is odd}.\cr}$$

\cmmnt{\noindent Recall that}

$$\eqalign{\nu(\partial B(\tbf{0},1))=r\beta_r
&=\Bover{2}{(k-1)!}\pi^k\text{ if }r=2k\text{ is even},\cr
&=\Bover{2^{2k+1}k!}{(2k)!}\pi^k\text{ if }r=2k+1\text{ is odd}
  \dvro{.}{}\cr}$$

\cmmnt{\noindent (265F/265H).}

In the formulae below, there are repeated expressions of the form
$\Bover1{\|x-y\|^{r-1}}$, $\Bover1{\|x-y\|^{r-2}}$;  in these, it will
often be
convenient to interpret $\bover10$ as $\infty$\cmmnt{, so that we have
$[0,\infty]$-valued functions defined everywhere}.

It will be convenient to do some
calculations in the one-point compactification $\BbbR^r\cup\{\infty\}$ of
$\BbbR^r$\cmmnt{ (3A3O)}.
For a set $A\subseteq\BbbR^r$\cmmnt{, write $\overline{A}^{\infty}$
and $\partial^{\infty}A$ for its closure and boundary taken in
$\BbbR^r\cup\{\infty\}$;  that is,}

\Centerline{$\overline{A}^{\infty}=\overline{A}$,
\quad$\partial^{\infty}A=\partial A$}

\noindent if $A$ is bounded, and

\Centerline{$\overline{A}^{\infty}=\overline{A}\cup\{\infty\}$,
\quad$\partial^{\infty}A=\partial A\cup\{\infty\}$}

\noindent if $A$ is unbounded.
\cmmnt{Note that }$\overline{A}^{\infty}$ and
$\partial^{\infty}A$ are always compact.
In this context I will take $x+\infty=\infty$ for every $x\in\BbbR^r$.

$\mu_W$ will be $r$-dimensional Wiener measure on
$\Omega=C(\coint{0,\infty};\BbbR^r)_0$, the space of continuous functions
$\omega$ from
$\coint{0,\infty}$ to $\BbbR^r$ such that $\omega(0)=0$\cmmnt{ (477D)},
endowed with the topology of uniform convergence on compact sets;  $\Sigma$
will be the domain of $\mu_W$.
\cmmnt{The probabilistic notations $\Expn$ and $\Pr$ will always be
with respect to $\mu_W$ or some directly associated probability.}
$\mu_W^2$ will be the product measure on $\Omega\times\Omega$.
I will write $X_t(\omega)=\omega(t)$ for $t\in\coint{0,\infty}$ and
$\omega\in\Omega$, and if $\tau:\Omega\to[0,\infty]$ is a function,
I will write
$X_{\tau}(\omega)=\omega(\tau(\omega))$ whenever $\omega\in\Omega$ and
$\tau(\omega)$ is finite.

\cmmnt{As in 477Hc,} I will write $\Sigma_t$ for
the $\sigma$-algebra of sets $F\in\Sigma$
such that $\omega'\in F$ whenever $\omega\in F$, $\omega'\in\Omega$ and
$\omega'\restr[0,t]=\omega\restr[0,t]$, and
$\Sigma^+_t=\bigcap_{s>t}\Sigma_s$.   $\Tau_{[0,t]}$ will be the
$\sigma$-algebra of subsets of $\Omega$ generated by $\{X_s:s\le t\}$.

\leader{478B}{Harmonic and superharmonic functions} Let $G\subseteq\BbbR^r$
be an open set and $f:G\to[-\infty,\infty]$ a function.

\spheader 478Ba $f$ is {\bf superharmonic} if
$\Bover1{\nu(\partial B(x,\delta))}\int_{\partial B(x,\delta)}fd\nu$ is
defined in $[-\infty,\infty]$ and
less than or equal to $f(x)$ whenever $x\in G$, $\delta>0$ and
$B(x,\delta)\subseteq G$.

\spheader 478Bb $f$ is {\bf subharmonic} if $-f$ is superharmonic, that is,
$\Bover1{\nu(\partial B(x,\delta))}\int_{\partial B(x,\delta)}fd\nu$ is
defined in $[-\infty,\infty]$ and
greater than or equal to $f(x)$ whenever $x\in G$, $\delta>0$ and
$B(x,\delta)\subseteq G$.

\spheader 478Bc $f$ is {\bf harmonic} if it is both superharmonic and
subharmonic, that is,
$\Bover1{\nu(\partial B(x,\delta))}\int_{\partial B(x,\delta)}fd\nu$
is defined and equal to $f(x)$ whenever $x\in G$, $\delta>0$ and
$B(x,\delta)\subseteq G$.

\leader{478C}{Elementary facts} Let $G\subseteq\BbbR^r$ be an open set.

\spheader 478Ca If $f:G\to[-\infty,\infty]$ is a function,
then $f$ is superharmonic iff $-f$ is subharmonic.

\spheader 478Cb If $f$, $g:G\to\coint{-\infty,\infty}$
are superharmonic functions,
then $f+g$ is superharmonic.   \prooflet{\Prf\ If $x\in G$, $\delta>0$ and
$B(x,\delta)\subseteq G$, then $\int_{\partial B(x,\delta)}fd\nu$ and
$\int_{\partial B(x,\delta)}g\,d\nu$ are defined in
$\coint{-\infty,\infty}$, so $\int_{\partial B(x,\delta)}f+g\,d\nu$ is
defined and is

\Centerline{$\int_{\partial B(x,\delta)}fd\nu+
\int_{\partial B(x,\delta)}g\,d\nu
\le\nu(\partial B(x,\delta))\,(f(x)+g(x))$.  \Qed}}

\spheader 478Cc If $f$, $g:G\to[-\infty,\infty]$ are superharmonic
functions, then $f\wedge g$ is superharmonic.
\prooflet{\Prf\ If $x\in G$, $\delta>0$ and
$B(x,\delta)\subseteq G$, then $\int_{\partial B(x,\delta)}fd\nu$ and
$\int_{\partial B(x,\delta)}g\,d\nu$ are defined in
$[-\infty,\infty]$, so $\int_{\partial B(x,\delta)}f\wedge g\,d\nu$ is
defined and is at most

\Centerline{$\min\bigl(\int_{\partial B(x,\delta)}fd\nu,
\int_{\partial B(x,\delta)}g\,d\nu\bigr)
\le\nu(\partial B(x,\delta))\,\min(f(x),g(x))$.  \Qed}}

\woddheader{478Cd}{0}{0}{0}{48pt}

\spheader 478Cd Let $f:G\to\Bbb R$ be a harmonic function which is
locally integrable with respect to Lebesgue measure on $G$\cmmnt{ (that
is, every point of $G$ has a neighbourhood $V$ such that $\int_Vfd\mu$ is
defined and finite)}.   Then

\Centerline{$f(x)=\Bover1{\mu B(x,\delta)}\int_{B(x,\delta)}fd\mu$}

\noindent whenever $x\in G$, $\delta>0$ and $B(x,\delta)\subseteq G$.
\prooflet{\Prf\ By 265G,

\Centerline{$\int_{B(x,\delta)}fd\mu
=\int_0^t\int_{\partial B(x,t)}fd\nu\,dt
=\int_0^t\nu(\partial B(x,\delta))f(x)dt
=\beta_r\delta^rf(x)$.   \Qed}

\noindent}So $f$ is continuous.   \prooflet{\Prf\ If $x\in G$, take
$\delta>0$ such that $B(x,2\delta)\subseteq G$, and set $f_1(y)=f(y)$ for
$y\in B(x,2\delta)$, $0$ for $y\in\BbbR^r\setminus B(x,2\delta)$.   Set
$g=\Bover1{\mu B(\tbf{0},\delta)}\chi B(\tbf{0},\delta)$.
Then $f_1$ is integrable, so the convolution $f_1*g$ is continuous (444Rc).
Also, for any $y\in B(x,\delta)$,

$$\eqalign{(f_1*g)(y)
&=\int f_1(z)g(y-z)\mu(dz)
=\int_{B(y,\delta)}\Bover{f_1(z)}{\mu B(y,\delta)}\mu(dz)\cr
&=\Bover1{\mu B(y,\delta)}\int_{B(y,\delta)}f(z)\mu(dz)
=f(y),}$$

\noindent so $f$ is continuous at $x$.\ \Qed}

%do we really need to assume "locally integrable"?  could it be a
%consequence of "measurable harmonic real-valued"? query

\leader{478D}{Maximal \dvrocolon{principle}}\cmmnt{ One of the
fundamental properties of harmonic functions will hardly be used in the
exposition here, but I had better give it a suitably prominent place.

\medskip

\noindent}{\bf Proposition} Let $G\subseteq\BbbR^r$ be a
non-empty open set.  Suppose that
$g:\overline{G}^{\infty}\to\ocint{-\infty,\infty}$ is
lower semi-continuous,
$g(y)\ge 0$ for every $y\in\partial^{\infty}G$, and $g\restr G$ is
superharmonic.   Then $g(x)\ge 0$ for every $x\in G$.

\proof{ \Quer\ Otherwise, set
$\gamma=\inf_{x\in G}g(x)=\inf\{g(y):y\in\overline{G}^{\infty}\}$.
Because $\overline{G}^{\infty}$ is compact and
$g$ is lower semi-continuous, $K=\{x:x\in G$, $g(x)=\gamma\}$ is
non-empty and compact (4A2B(d-viii)).
Let $x\in K$ be such that $\|x\|$ is maximal, and $\delta>0$
such that $B(x,\delta)\subseteq G$.   Then
$\Bover1{\nu(\partial B(x,\delta))}\int_{\partial B(x,\delta)}g\,d\nu
\le g(x)$.   But $g(y)\ge g(x)$ for
every $y\in\partial B(x,\delta)$ and

\Centerline{$\{y:y\in\partial B(x,\delta)$, $g(y)>g(x)\}
\supseteq\{y:y\in\partial B(x,\delta)$, $(y-x)\dotproduct x\ge 0\}$}

\noindent is not $\nu$-negligible, so this is impossible.\ \Bang
}%end of proof of 478D

\leader{478E}{Theorem} Let $G\subseteq\BbbR^r$ be an open set and
$f:G\to\Bbb R$ a function with continuous second derivative.
Write $\nabla^2f$ for its Laplacian
$\diverg\grad f=\sum_{i=1}^r\Pdd{f}{\xi_i}$.

(a) $f$ is superharmonic iff $\nabla^2f\le 0$ everywhere in $G$.

(b) $f$ is subharmonic iff $\nabla^2f\ge 0$ everywhere in $G$.

(c) $f$ is harmonic iff $\nabla^2f=0$ everywhere in $G$.

\proof{{\bf (a)(i)} For $x\in G$ set

\Centerline{$R_x=\rho(x,\BbbR^r\setminus G)
=\inf_{y\in\BbbR^r\setminus G}\|x-y\|$,}

\noindent counting $\inf\emptyset$ as $\infty$;  for $0<\gamma<R_x$ set

\Centerline{$g_x(\gamma)
=\Bover1{\gamma^{r-1}}\int_{\partial B(x,\gamma)}f(y)\nu(dy)
=\int_{\partial B(\tbf{0},1)}f(x+\gamma z)\nu(dz)$.}

\noindent Because $f$ is
continuously differentiable, $g'_x(\gamma)$ is defined and equal to
$\int_{\partial B(\tbf{0},1)}\Pd{}{\gamma}f(x+\gamma z)\nu(dz)$ for
$\gamma\in\ooint{0,R_x}$.

Set $\phi=\grad f$, so that $\nabla^2f=\diverg\phi$.
Each ball $B(x,\gamma)$ has
finite perimeter;  its essential boundary is its ordinary boundary;
the Federer exterior normal
$v_y$ at $y$ is $\Bover1{\gamma}(y-x)$;  and if $y=x+\gamma z$, where
$\|z\|=1$, then
$\phi(y)\dotproduct v_y$ is $\Pd{}{\gamma}f(x+\gamma z)$.
So the Divergence Theorem (475N) tells us that

$$\eqalignno{\int_{B(x,\gamma)}\nabla^2fd\mu
&=\int_{\partial B(x,\gamma)}\phi(y)\dotproduct v_y\nu(dy)\cr
&=\gamma^{r-1}\int_{\partial B(\tbf{0},1)}
  \Pd{}{\gamma}f(x+\gamma z)\nu(dz)
=\gamma^{r-1}g'_x(\gamma).\cr}$$

\medskip

\quad{\bf (ii)} If $\nabla^2f\le 0$ everywhere in $G$, and
$B(x,\gamma)\subseteq G$, then $g'_x(t)\le 0$ for $0<t\le \gamma$, so

\Centerline{$g_x(\gamma)\le\lim_{t\downarrow 0}g_x(t)=r\beta_rf(x)$;}

\noindent as $x$ and $\gamma$ are arbitrary, $f$ is superharmonic.

\medskip

\quad{\bf (iii)} If $f$ is superharmonic, and $x\in G$, then

\Centerline{$g_x(\gamma)\le r\beta_rf(x)=\lim_{t\downarrow 0}g_x(t)$}

\noindent for every $\gamma\in\ooint{0,R_x}$.   So there must be arbitrarily
small $\gamma >0$ such that $g'_x(\gamma)\le 0$ and
$\int_{B(x,\gamma)}\nabla^2fd\mu\le 0$;  as $\nabla^2f$ is continuous,
$(\nabla^2f)(x)\le 0$.

\medskip

{\bf (b)-(c)} are now immediate.
}%end of proof of 478E

\leader{478F}{Basic examples} (a) For any $y$, $z\in\BbbR^r$,

\Centerline{$x\mapsto\Bover1{\|x-z\|^{r-2}}$,
\quad$x\mapsto\Bover{(x-z)\dotproduct y}{\|x-z\|^r}$,}

\Centerline{$x\mapsto\Bover{\|y-z\|^2-\|x-y\|^2}{\|x-z\|^r}
=2\Bover{(x-z)\dotproduct(y-z)}{\|x-z\|^r}-\Bover1{\|x-z\|^{r-2}}$}

\noindent are harmonic on $\BbbR^r\setminus\{z\}$.

(b) For any $z\in\BbbR^2$,

\Centerline{$x\mapsto\ln\|x-z\|$}

\noindent is harmonic on $\Bbb R^2\setminus\{z\}$.


\proof{ The Laplacians are easiest to calculate when $z=0$, of course,
but in any case you only have to get the algebra right to apply 478Ec.
}%end of proof of 478F

\cmmnt{\medskip

\noindent{\bf Remark} The function
$x\mapsto\Bover{\|y-z\|^2-\|x-y\|^2}{\|x-z\|^r}$ is the {\bf Poisson
kernel};  see 478I below.}

\leader{478G}{}\cmmnt{ We shall
need a pair of exact integrals involving the functions here, with an easy
corollary.

\medskip

\noindent}{\bf Lemma}
(a) $\Bover1{\nu(\partial B(\tbf{0},\delta))}\biggerint_{\partial B(\tbf{0},\delta)}
\Bover1{\|x-z\|^{r-2}}\nu(dz)
=\Bover1{\max(\delta,\|x\|)^{r-2}}$ whenever $x\in\BbbR^r$
and $\delta>0$.

(b) $\Bover1{\nu(\partial B(\tbf{0},\delta))}\biggerint_{\partial B(\tbf{0},\delta)}
  \Bover{|\delta^2-\|x\|^2|}{\|x-z\|^r}\nu(dz)
=\Bover1{\max(\delta,\|x\|)^{r-2}}$ whenever $x\in\BbbR^r$,
$\delta>0$ and $\|x\|\ne\delta$.

(c) $\biggerint_{B(\tbf{0},\delta)}\Bover1{\|x-z\|^{r-2}}\mu(dz)
\le\Bover12r\beta_r\delta^2$
whenever $x\in\BbbR^r$ and $\delta>0$.

\proof{{\bf (a)(i)} The first thing to note is that there is a function
$g:\coint{0,\infty}\setminus\{\delta\}\to\coint{0,\infty}$ such that
$g(\|x\|)=\Bover1{\nu(\partial B(\tbf{0},\delta))}
\biggerint_{\partial B(\tbf{0},\delta)}\Bover1{\|x-z\|^{r-2}}\nu(dz)$
whenever $\|x\|\ne\delta$.   \Prf\ If $\|x\|=\|y\|$, then
there is an orthogonal transformation $T:\BbbR^r\to\BbbR^r$ such that
$Tx=y$, so that

$$\eqalignno{\int_{\partial B(\tbf{0},\delta)}\Bover1{\|y-z\|^{r-2}}\nu(dz)
&=\int_{\partial B(\tbf{0},\delta)}\Bover1{\|y-Tz\|^{r-2}}\nu(dz)\cr
\displaycause{because $T$ is an automorphism of
$(\BbbR^r,B(\tbf{0},\delta),\nu)$}
&=\int_{\partial B(\tbf{0},\delta)}\Bover1{\|Tx-Tz\|^{r-2}}\nu(dz)
=\int_{\partial B(\tbf{0},\delta)}\Bover1{\|x-z\|^{r-2}}\nu(dz).
\text{ \Qed}\cr}$$

\woddheader{478G}{4}{2}{2}{60pt}

\quad{\bf (ii)} Now suppose that $0<\gamma<\delta$.   Then

$$\eqalignno{g(\gamma)
&=\Bover1{\nu(\partial B(\tbf{0},\gamma))}\int_{\partial B(\tbf{0},\gamma)}
  g(\gamma)\nu(dx)\cr
&=\Bover1{\nu(\partial B(\tbf{0},\gamma))}\int_{\partial B(\tbf{0},\gamma)}
   \Bover1{\nu(\partial B(\tbf{0},\delta))}\int_{\partial B(\tbf{0},\delta)}
   \Bover1{\|x-z\|^{r-2}}\nu(dz)\nu(dx)\cr
&=\Bover1{\nu(\partial B(\tbf{0},\delta))}\int_{\partial B(\tbf{0},\delta)}
   \Bover1{\nu(\partial B(\tbf{0},\gamma))}
   \int_{\partial B(\tbf{0},\gamma)}\Bover1{\|x-z\|^{r-2}}\nu(dx)\nu(dz)\cr
&=\Bover1{\nu(\partial B(\tbf{0},\delta))}\int_{\partial B(\tbf{0},\delta)}
   \Bover1{\|z\|^{r-2}}\nu(dz)\cr
\displaycause{because the function $x\mapsto\Bover1{\|x-z\|^2}$ is harmonic
in $\BbbR^r\setminus\{z\}$, by 478Fa}
&=\Bover1{\delta^{r-2}}.\cr}$$

\medskip

\quad{\bf (iii)} Next, if $\gamma>\delta$,

$$\eqalignno{g(\gamma)
&=\Bover1{\nu(\partial B(\tbf{0},\delta))}\int_{\partial B(\tbf{0},\delta)}
   \Bover1{\nu(\partial B(\tbf{0},\gamma))}
   \int_{\partial B(\tbf{0},\gamma)}\Bover1{\|x-z\|^{r-2}}\nu(dx)\nu(dz)\cr
\displaycause{as in (ii)}
&=\Bover1{\nu(\partial B(\tbf{0},\delta))}\int_{\partial B(\tbf{0},\delta)}
   \Bover1{\gamma^{r-2}}\nu(dz)\cr
\displaycause{by (ii), with $\gamma$ and $\delta$ interchanged}
&=\Bover1{\gamma^{r-2}}.\cr}$$

\medskip

\quad{\bf (iv)} So we have the result if $\|x\|\ne\delta$.   If
$\|x\|=\delta$ and $r\ge 2$,
set $x_n=(1+2^{-n})x$ for each $n\in\Bbb N$.   If
$z\in\partial B(\tbf{0},\delta)$,
$\sequencen{\|x_n-z\|}$ is a decreasing sequence with limit $\|x-z\|$, so
$\sequencen{\Bover1{\|x_n-z\|^{r-2}}}$ is a non-decreasing sequence with
limit $\Bover1{\|x-z\|^{r-2}}$.   By B.Levi's theorem,

$$\eqalign{\Bover1{\nu(\partial B(\tbf{0},\delta))}\int_{\partial B(\tbf{0},\delta)}
  \Bover1{\|x-z\|^{r-2}}\nu(dz)
&=\lim_{n\to\infty}\Bover1{\nu(\partial B(\tbf{0},\delta))}
  \int_{\partial B(\tbf{0},\delta)}\Bover1{\|x_n-z\|^{r-2}}\nu(dz)\cr
&=\lim_{n\to\infty}\Bover1{\|x_n\|^{r-2}}
=\Bover1{\|x\|^{r-2}}
=\Bover1{\delta^{r-2}}.\cr}$$

\noindent Finally, if $r=1$ and $\|x\|=|x|=\delta$, we are just trying to
take the average of $|x-\delta|$ and  $|x-(-\delta)|$, which will be
$\delta=\Bover1{\delta^{r-2}}$.

\medskip

{\bf (b)} We can follow the same general line.

\medskip

\quad{\bf (i)}
Define $f:\BbbR^r\setminus\partial B(\tbf{0},\delta)\to\Bbb R$ by setting
$f(x)=\Bover1{\nu(\partial B(\tbf{0},\delta))}\int_{\partial B(\tbf{0},\delta)}
\Bover{\|x\|^2-\delta^2}{\|x-z\|^r}\nu(dz)$ when
$\|x\|\ne\delta$.   Then $f$ is harmonic.   \Prf\ If $x$ and $\gamma>0$ are
such that $B(x,\gamma)\subseteq\BbbR^r\setminus\partial B(\tbf{0},\delta)$, then

$$\eqalignno{\Bover1{\nu(\partial B(x,\gamma))}
   &\int_{\partial B(x,\gamma)}f(y)\nu(dy)\cr
&=\Bover1{\nu(\partial B(x,\gamma))}\int_{\partial B(x,\gamma)}
   \Bover1{\nu(\partial B(\tbf{0},\delta))}\int_{\partial B(\tbf{0},\delta)}
   \Bover{\|y\|^2-\delta^2}{\|y-z\|^r}
   \nu(dz)\nu(dy)\cr
&=\Bover1{\nu(\partial B(\tbf{0},\delta))}\int_{\partial B(\tbf{0},\delta)}
   \Bover1{\nu(\partial B(x,\gamma))}\int_{\partial B(x,\gamma)}
   \Bover{\|y\|^2-\delta^2}{\|y-z\|^r}
   \nu(dy)\nu(dz)\cr
&=\Bover1{\nu(\partial B(\tbf{0},\delta))}\int_{\partial B(\tbf{0},\delta)}
   \Bover{\|x\|^2-\delta^2}{\|x-z\|^r}\nu(dz)\cr
\displaycause{because the functions
$y\mapsto\Bover{\|y\|^2-\delta^2}{\|y-z\|^r}$ are harmonic when
$\|z\|=\delta$, by 478Fa}
&=f(x).  \text{\Qed}\cr}$$

\noindent
Since $f$ is smooth, $\nabla^2f=0$ everywhere off $\partial B(\tbf{0},\delta)$,
by 478Ec.

\medskip

\quad{\bf (ii)} As before, we have a function
$h:\coint{0,\infty}\setminus\{\delta\}\to\coint{0,\infty}$ such that
$f(x)=h(\|x\|)$ whenever $\|x\|\ne\delta$.   If $0<\gamma<\delta$ then

$$\eqalign{h(\gamma)
&=\Bover1{\nu(\partial B(\tbf{0},\gamma))}
  \int_{\partial B(\tbf{0},\gamma)}h(\gamma)\nu(dy)
=\Bover1{\nu(\partial B(\tbf{0},\gamma))}
  \int_{\partial B(\tbf{0},\gamma)}f(y)\nu(dy)\cr
&=f(0)
=\Bover1{\nu(\partial B(\tbf{0},\delta))}\int_{\partial B(\tbf{0},\delta)}
  \Bover{-\delta^2}{\|z\|^r}\nu(dz)
=-\Bover1{\delta^{r-2}},\cr}$$

\noindent and

\Centerline{$\Bover1{\nu(\partial B(\tbf{0},\delta))}\int_{\partial B(\tbf{0},\delta)}
  \Bover{|\delta^2-\|x\|^2|}{\|x-z\|^r}\nu(dz)
=-h(\gamma)=\Bover1{\delta^{r-2}}$}

\noindent if $\|x\|=\gamma<\delta$.

\medskip

\quad{\bf (iii)} For $\gamma>\delta$ I start with an elementary estimate.
If $\|x\|=\gamma>\delta$ then
$\Bover{\|x\|^2-\delta^2}{\|x-z\|^r}$ lies between
$\Bover{\gamma^2-\delta^2}{(\gamma+\delta)^r}$ and
$\Bover{\gamma^2-\delta^2}{(\gamma-\delta)^r}$ for every
$z\in\partial B(x,\delta)$, so that $\gamma^{r-2}f(x)$ lies
between
$\Bover{1-(\delta/\gamma)^2}{(1+(\delta/\gamma))^r}$
and $\Bover{1-(\delta/\gamma)^2}{(1-(\delta/\gamma))^r}$,
and is approximately $1$ if $\gamma$ is large.

\medskip

\quad{\bf (iv)} Now we can use the Divergence Theorem
again, as follows.   If $\delta<\gamma<\beta$ consider the region
$E=B(\tbf{0},\beta)\setminus B(\tbf{0},\gamma)$ and the function $\phi=\grad f$.
As $f$ is smooth, $\phi$ is defined everywhere off
$\partial B(\tbf{0},\delta)$, and $\phi(x)=\Bover{h'(\|x\|)}{\|x\|}x$
at every $x\in\BbbR^r\setminus\partial B(\tbf{0},\delta)$.
The essential boundary of $E$ is
$\partial E=\partial B(\tbf{0},\gamma)\cup\partial B(\tbf{0},\beta)$;  the Federer
exterior normal at $x\in\partial B(\tbf{0},\gamma)$ is $v_x=-\Bover1{\gamma}x$
and at $x\in\partial B(\tbf{0},\beta)$ it is $v_x=\Bover1{\beta}x$;
and $\diverg\phi=\nabla^2f$ is zero everywhere on $E$.
So 475N tells us that

$$\eqalign{0
&=\int_{\partial B(\tbf{0},\beta)}\phi(x)\dotproduct v_x\nu(dx)
   +\int_{\partial B(\tbf{0},\gamma)}\phi(x)\dotproduct v_x\nu(dx)\cr
&=\int_{\partial B(\tbf{0},\beta)}h'(\beta)\nu(dx)
   -\int_{\partial B(\tbf{0},\gamma)}h'(\gamma)\nu(dx)\cr
&=r\beta_r\beta^{r-1}h'(\beta)
   -r\beta_r\gamma^{r-1}h'(\gamma).\cr}$$

\noindent
This shows that $h'(\gamma)$ is inversely proportional to $\gamma^{r-1}$.

\medskip

\quad{\bf (v)} If $r\ge 3$,
there are $\alpha$, $\beta\in\Bbb R$ such that
$h(\gamma)=\alpha+\Bover{\beta}{\gamma^{r-2}}$ for every $\gamma>\delta$.
But since (iii) shows us that
$\lim_{\gamma\to\infty}\gamma^{r-2}h(\gamma)=1$,
we must have $h(\gamma)=\Bover1{\gamma^{r-2}}$
for $\gamma>\delta$, as declared.   If $r=2$, then we can express $h$ in
the form $h(\gamma)=\alpha+\beta\ln\gamma$;  this time,
$\lim_{\gamma\to\infty}h(\gamma)=1$, so once more
$h(\gamma)=1=\Bover1{\gamma^{r-2}}$ for every $\gamma$.

\medskip

\quad{\bf (vi)} Finally, if $r=1$ and $|x|>\delta$, then, as in
(a-iv) above,
$\Bover1{\nu(\partial B(\tbf{0},\delta))}\biggerint_{\partial B(\tbf{0},\delta)}
  \Bover{|\delta^2-\|x\|^2|}{\|x-z\|^r}\nu(dz)$ is the average of
$\Bover{x^2-\delta^2}{|x-\delta|}=|x+\delta|$ and
$\Bover{x^2-\delta^2}{|x+\delta|}=|x-\delta|$, so is
$|x|=\Bover1{|x|^{r-2}}$.

\medskip

{\bf (c)} This follows easily from (a);

$$\eqalign{\int_{B(\tbf{0},\delta)}\Bover1{\|x-z\|^{r-2}}\mu(dz)
&=\int_0^{\delta}\int_{\partial B(\tbf{0},t)}\Bover1{\|x-z\|^{r-2}}\nu(dz)dt
   \cr
&=\int_0^{\delta}\Bover{r\beta_rt^{r-1}}
   {\max(t,\|x\|)^{r-2}}dt
\le\int_0^{\delta}r\beta_rt\,dt
=\Bover12r\beta_r\delta^2.\cr}$$
}%end of proof of 478G

\leader{478H}{Corollary} If $r\ge 2$, then
$x\mapsto\Bover1{\|x-z\|^{r-2}}:\BbbR^r\to[0,\infty]$ is superharmonic
for any $z\in\BbbR^r$.

\woddheader{478H}{0}{0}{0}{40pt}

\proof{ If $\delta>0$ and $x\in\BbbR^r$,

$$\eqalign{\Bover1{\nu\partial B(x,\delta)}
  \int_{\partial B(x,\delta)}\Bover1{\|y-z\|^{r-2}}\nu(dy)
&=\Bover1{\nu\partial B(\tbf{0},\delta)}
  \int_{\partial B(\tbf{0},\delta)}\Bover1{\|y+x-z\|^{r-2}}\nu(dy)\cr
&=\Bover1{\max(\delta,\|x-z\|)^{r-2}}
\le\Bover1{\|x-z\|^{r-2}}.\cr}$$
}%end of proof of 478H

\leader{478I}{}\cmmnt{ The Poisson kernel gives a basic method of
building and describing harmonic functions.

\medskip

\noindent}{\bf Theorem} Suppose that $y\in\BbbR^r$ and $\delta>0$;
let $S=\partial B(y,\delta)$\cmmnt{ be the sphere with centre $y$ and
radius $\delta$}.

(a) Let $\zeta$ be a totally finite Radon measure on $S$, and define $f$ on
$\BbbR^r\setminus S$ by setting

\Centerline{$f(x)
=\Bover1{r\beta_r\delta}\int_S
  \Bover{|\delta^2-\|x-y\|^2|}{\|x-z\|^r}\zeta(dz)$}

\noindent for $x\in\BbbR^r\setminus S$.
Then $f$ is continuous and harmonic.

(b) Let $g:S\to\Bbb R$ be a $\nu_S$-integrable function, where $\nu_S$ is
the subspace measure on $S$ induced by $\nu$, and define
$f:\BbbR^r\to\Bbb R$ by setting

$$\eqalign{f(x)
&=\Bover1{r\beta_r\delta}\int_S
  g(z)\Bover{|\delta^2-\|x-y\|^2|}{\|x-z\|^r}\nu(dz)
  \text{ if }x\in\BbbR^r\setminus S,\cr
&=g(x)\text{ if }x\in S.\cr}$$

\quad(i) $f$ is continuous and harmonic in $\BbbR^r\setminus S$.

\quad(ii) If $r\ge 2$, then

\Centerline{$\liminf_{z\in S,z\to z_0}g(x)=\liminf_{x\to z_0}f(x)$,
\quad$\limsup_{x\to z_0}f(x)=\limsup_{z\in S,z\to z_0}g(x)$}

\noindent for every $z_0\in S$.

\quad(iii) $f$
is continuous at any point of $S$ where $g$ is continuous, and if $g$ is
lower semi-continuous then $f$ also is.

\quad(iv) $\sup_{x\in\BbbR^r}f(x)=\sup_{z\in S}g(z)$ and
$\inf_{x\in\BbbR^r}f(x)=\inf_{z\in S}g(z)$.

\proof{{\bf (a)} $f$ is continuous just because
$x\mapsto\Bover{\delta^2-\|x-y\|^2}{\|x-z\|^r}$ is continuous for each
$z\in S$ and uniformly bounded for $z\in S$ and
$x$ running over any compact set not meeting $S$.

Suppose that $\|x-y\|<\delta$
and $\eta>0$ is such
that $B(x,\eta)\cap S=\emptyset$.   Then

$$\eqalignno{
\Bover1{\nu(\partial B(x,\eta))}\int_{\partial B(x,\eta)}fd\nu
&=\Bover1{\nu(\partial B(x,\eta))}\int_{\partial B(x,\eta)}
  \Bover1{r\beta_r\delta}
    \int_S\Bover{\delta^2-\|w-y\|^2}{\|w-z\|^r}\zeta(dz)\nu(dw)\cr
&=\Bover1{r\beta_r\delta}\int_S\Bover1{\nu(\partial B(x,\eta))}
    \int_{\partial B(x,\eta)}
    \Bover{\delta^2-\|w-y\|^2}{\|w-z\|^r}\nu(dw)\zeta(dz)\cr
&=\Bover1{r\beta_r\delta}\int_S
    \Bover{\delta^2-\|x-y\|^2}{\|x-z\|^r}\zeta(dz)\cr
\displaycause{because $w\mapsto\Bover{\delta^2-\|w-y\|^2}{\|w-z\|^r}$
is harmonic on $\BbbR^r\setminus\{z\}$ whenever $z\in S$, by 478Fa}
&=f(x).\cr}$$

\noindent As $x$ and $\eta$ are arbitrary, $f$ is harmonic on
$\interior B(y,\delta)$.   Similarly, it is harmonic on
$\BbbR^r\setminus B(y,\delta)$.
\medskip

{\bf (b)(i)} Applying (a) to the indefinite-integral measures over
the subspace measure $\nu_S$
defined by the positive and negative parts of $g$, we see that $f$ is
continuous and harmonic in $\BbbR^r\setminus S$.

\medskip

\quad{\bf (ii)}\grheada\ If $x\notin S$,

$$\eqalign{\int_S\Bover{|\delta^2-\|x-y\|^2|}{\|x-z\|^r}\nu(dz)
&=\int_{\partial B(\tbf{0},\delta)}
   \Bover{|\delta^2-\|x-y\|^2|}{\|x-y-z\|^r}\nu(dz)\cr
&=\Bover{\nu(\partial B(\tbf{0},\delta))}{\max(\delta,\|x-y\|)^{r-2}}\cr}$$

\noindent by 478Gb.   In particular, if
$x$ is close to, but not on, the sphere $S$,
$\int_S\Bover{|\delta^2-\|x-y\|^2|}{\|x-z\|^r}\nu(dz)$ is
approximately
$\Bover{\nu(\partial B(\tbf{0},\delta)}{\delta^{r-2}}=r\beta_r\delta$.

\medskip

\qquad\grheadb\ Set $M=\int_S|g|d\nu$, and take $z_0\in S$;   set
$\gamma=\limsup_{x\in S,x\to z_0}g(x)$.   If $\gamma=\infty$ then certainly
$\limsup_{x\to z_0}f(x)\le\gamma$.   Otherwise, take $\eta>0$.
Let $\alpha_0\in\ooint{0,\delta}$ be such that

\Centerline{$|\Bover1{r\beta_r\delta}\int_S
  \Bover{|\delta^2-\|x-y\|^2|}{\|x-z\|^r}\nu(dz)-1|\le\eta$}

\noindent whenever $0<|\delta-\|x-y\||\le\alpha_0$, and
$g(z)\le\gamma+\eta$
whenever $z\in S$ and $0<\|z-z_0\|\le 2\alpha_0$.
Let $\alpha\in\ocint{0,\alpha_0}$ be such that
$(2\delta+\alpha_0)(M+|\gamma|)\alpha\nu S
\le 2^rr\beta_r\delta\alpha_0^r\eta$.

If $\|x-z_0\|\le\alpha$ and $\|x-y\|\ne \delta$,
then $|\delta -\|x-y\||\le\|x-z_0\|\le\alpha_0$ and
$|\delta^2-\|x-y\|^2|\le\|x-z_0\|(2\delta +\alpha_0)$, so

$$\eqalignno{f(x)-\gamma
&\le\eta|\gamma|+\Bover1{r\beta_r\delta}\int_S
  (g(z)-\gamma)\Bover{|\delta^2-\|x-y\|^2|}{\|x-z\|^r}\nu(dz)\cr
&\le\eta|\gamma|
  +\Bover1{r\beta_r\delta}\int_{S\cap B(z_0,2\alpha_0)}
    (g(z)-\gamma)\Bover{|\delta^2-\|x-y\|^2|}{\|x-z\|^r}\nu(dz)\cr
&\mskip100mu
  +\Bover1{r\beta_r\delta}\int_{S\setminus B(z_0,2\alpha_0)}
    (|g(z)|+|\gamma|)\Bover{|\delta^2-\|x-y\|^2|}{\|x-z\|^r}\nu(dz)\cr
&\le\eta|\gamma|
  +\Bover{\eta}{r\beta_r\delta}\int_S
    \Bover{|\delta^2-\|x-y\|^2|}{\|x-z\|^r}\nu(dz)\cr
&\mskip100mu
  +\Bover1{r\beta_r\delta}\int_{S\setminus B(z_0,2\alpha_0)}
    (|g(z)|+|\gamma|)\Bover{\|x-z_0\|(2\delta +\alpha_0)}{2^r\alpha_0^r}
    \nu(dz)\cr
\displaycause{because $r\ge 2$, so $\nu\{z_0\}=0$}
&\le\eta|\gamma|
  +\eta(1+\eta)+(M+|\gamma|)
      \Bover{\alpha(2\delta +\alpha_0)}{2^rr\beta_r\delta\alpha_0^r}\nu S
  \cr
&\le (|\gamma|+1+\eta+1)\eta.\cr}$$

\noindent Also, of course, $f(x)-\gamma=g(x)-\gamma\le\eta$ if
$0<\|x-z_0\|\le\alpha$ and $\|x-y\|=\delta$.   As $\eta$
is arbitrary, $\limsup_{x\to z_0}f(x)\le\gamma$.
In the other direction, $\limsup_{x\to z_0}f(x)\ge\gamma$ just because $f$
extends $g$.

\medskip

\qquad\grheadc\
Similarly, or applying ($\beta$) to $-g$,
$\liminf_{x\to z_0}f(x)=\liminf_{x\in S,x\to z_0}g(x)$.

\medskip

\quad{\bf (iii)}\grheada\ If $r\ge 2$, it follows at once from (ii)
that if $g$ is continuous at $z_0\in S$, so is $f$, and
that if $g$ is lower semi-continuous (so that
$g(z_0)\le\liminf_{x\in S,x\to z_0}g(x)$ for every $z_0\in S$) then $f$
also is lower semi-continuous.

\medskip

\qquad\grheadb\ If $r=1$, then $S=\{y-\delta,y+\delta\}$ and
$\nu\{y-\delta\}=\nu\{y+\delta\}=1$, so

$$\eqalign{f(x)
&=\Bover1{2\delta}\bigl(
   \Bover{|\delta^2-(x-y)^2|}{|x-(y-\delta)|}g(y-\delta)
     +\Bover{|\delta^2-(x-y)^2|}{|x-(y+\delta)|}g(y+\delta)\bigr)\cr
&=\Bover1{2\delta}\bigl(|x-(y+\delta)|g(y-\delta)
           +|x-(y-\delta)|g(y+\delta)\bigr)\cr}$$

\noindent for $x\in\Bbb R\setminus S$, and
$\lim_{x\mapsto y\pm\delta}f(x)=g(y\pm\delta)$, so $f$ is continuous.

\medskip

\quad{\bf (iv)} If $g(z)\le\alpha<\infty$ for every $z\in S$, then

$$\eqalign{f(x)
&=\Bover1{r\beta_r\delta}\int_S
   g(z)\Bover{|\delta^2-\|x-y\|^2|}{\|x-z\|^r}\nu(dz)
\le\Bover{\alpha}{r\beta_r\delta}\int_{\partial B(\tbf{0},\delta)}
   \Bover{|\delta^2-\|x-y\|^2|}{\|x-y-z\|^r}\nu(dz)\cr
&=\Bover{\alpha}{r\beta_r\delta}
   \cdot\Bover{\nu(\partial B(\tbf{0},\delta))}{\max(\delta,\|x-y\|)^{r-2}}
=\Bover{\alpha\delta^{r-2}}{\max(\delta,\|x-y\|)^{r-2}}
\le\alpha\cr}$$

\noindent for every $x\in\BbbR^r\setminus S$.   So
$\sup_{x\in\BbbR^r}f(x)=\sup_{z\in S}g(z)$;  similarly,
$\inf_{x\in\BbbR^r}f(x)=\inf_{z\in S}g(z)$.
}%end of proof of 478I

\leader{478J}{Convolutions and smoothing:  Proposition} (a) Suppose that
$f:\BbbR^r\to[0,\infty]$ is Lebesgue measurable, and $G\subseteq\BbbR^r$
an open set such that $f\restr G$ is superharmonic.   Let
$h:\BbbR^r\to[0,\infty]$ be a Lebesgue integrable function, and $f*h$ the
convolution of $f$ and $h$.   If
$H\subseteq G$ is an open set such that $H-\{z:h(z)\ne 0\}\subseteq G$,
then $(f*h)\restr H$ is superharmonic.

(b) Suppose, in (a), that
$h(y)=h(z)$ whenever $\|y\|=\|z\|$ and that
$\int_{\BbbR^r}h\,d\mu\le 1$.   If $x\in G$ and $\gamma>0$ are such that
$B(x,\gamma)\subseteq G$ and $h(y)=0$ whenever $\|y\|\ge\gamma$,
then $(f*h)(x)\le f(x)$.

(c) Let $f:\BbbR^r\to[0,\infty]$ be a lower semi-continuous
function, and $\sequencen{\tilde h_n}$ the sequence of 473E.
If $G\subseteq\BbbR^r$ is an open set such that $f\restr G$ is
superharmonic, then $f(x)=\lim_{n\to\infty}(f*\tilde h_n)(x)$ for every
$x\in G$.

\proof{{\bf (a)} If $x\in H$ and $\delta>0$ are such that
$B(x,\delta)\subseteq H$, then

$$\eqalignno{\int_{\partial B(x,\delta)}(f*h)(y)\nu(dy)
&=\int_{\partial B(x,\delta)}\int_{\BbbR^r}f(y-z)h(z)\mu(dz)\nu(dy)\cr
&=\int_{\BbbR^r}h(z)\int_{\partial B(x,\delta)}f(y-z)\nu(dy)\mu(dz)\cr
&=\int_{\BbbR^r}h(z)\int_{\partial B(x-z,\delta)}f(y)\nu(dy)\mu(dz)\cr
&\le r\beta_r\delta^{r-1}\int_{\BbbR^r}h(z)f(x-z)\mu(dz)\cr
\displaycause{because if $h(z)\ne 0$ then $B(x-z,\delta)=B(x,\delta)-z$ is
included in $G$}
&=\nu(\partial B(x,\delta))\cdot(f*h)(x).\cr}$$

\medskip

{\bf (b)} Let $g:\coint{0,\infty}\to[0,\infty]$ be such that
$h(y)=g(\|y\|)$ for every $y$.   Then

$$\eqalign{(f*h)(x)
&=\int_{\BbbR^r}f(y)h(x-y)\mu(dy)
=\int_0^{\gamma}\int_{\partial B(x,t)}f(y)g(t)\nu(dy)dt\cr
&\le\int_0^{\gamma}r\beta_rt^{r-1}f(x)g(t)dt
=f(x)\int_{\BbbR^r}h\,d\mu
\le f(x).\cr}$$

\medskip

{\bf (c)} By (b), $(f*\tilde h_n)(x)\le f(x)$ for every sufficiently
large $n$, so $\limsup_{n\to\infty}(f*\tilde h_n)(x)\le f(x)$.
In the other
direction, if $x\in G$ and $\alpha<f(x)$, there is a $\delta>0$ such
that $B(x,\delta)\subseteq G$ and
$f(y)\ge\alpha$ for every $y\in B(x,\delta)$.   Now there is an
$m\in\Bbb N$ such that $\tilde h_n(y)=0$ whenever $n\ge m$ and
$\|y\|\ge\delta$;  so that

$$\eqalign{(f*\tilde h_n)(x)
&=\int f(y)\tilde h_n(x-y)\mu(dy)
=\int_{B(x,\delta)}f(y)\tilde h_n(x-y)\mu(dy)\cr
&\ge\alpha\int_{B(x,\delta)}\tilde h_n(x-y)\mu(dy)
=\alpha\cr}$$

\noindent whenever $n\ge m$.   As $\alpha$ is arbitrary,
$f(x)=\lim_{n\to\infty}(f*\tilde h_n)(x)$.
}%end of proof of 478J

\leader{478K}{Dynkin's formula:  Lemma}
Let $\mu_W$ be $r$-dimensional Wiener measure on
$\Omega=C(\coint{0,\infty};\BbbR^r)_0$;  set
$X_t(\omega)=\omega(t)$ for $\omega\in\Omega$ and $t\ge 0$.
Let $f:\BbbR^r\to\Bbb R$ be a three-times-differentiable
function such that $f$ and its
first three derivatives are continuous and bounded.

(a) $\Expn(f(X_t))=f(0)+\Bover12\Expn(\int_0^t(\nabla^2f)(X_s)ds)$ for
every $t\ge 0$.

(b) If $\tau:\Omega\to\coint{0,\infty}$ is a stopping time adapted to
$\langle\Sigma^+_t\rangle_{t\ge 0}$ and $\Expn(\tau)$ is finite, then

\Centerline{
$\Expn(f(X_{\tau}))=f(0)+\Bover12\Expn(\int_0^{\tau}(\nabla^2f)(X_s)ds)$.}

\proof{{\bf (a)(i)} We need a special case of the multidimensional
Taylor's theorem.   If $f:\BbbR^r\to\Bbb R$ is three times differentiable
and $x=(\xi_1,\ldots,\xi_r)$, $y=(\eta_1,\ldots,\eta_r)\in\BbbR^r$,
then there is a $z$ in the line segment $[x,y]$ such that

$$\eqalign{f(y)&=f(x)+\sum_{i=1}^r(\eta_i-\xi_i)\Pd{f}{\xi_i}(x)
             +\Bover12\sum_{i=1}^r\sum_{j=1}^r(\eta_i-\xi_i)(\eta_j-\xi_j)
	        \Pde{f}{\xi_i}{\xi_j}(x)\cr
&\mskip100mu
+\Bover16\sum_{i=1}^r\sum_{j=1}^r\sum_{k=1}^r
	        (\eta_i-\xi_i)(\eta_j-\xi_j)(\eta_k-\xi_k)
      \Bover{\partial^3f}{\partial\xi_i\partial\xi_j\partial\xi_k}(z).\cr}$$

\noindent\Prf\ Set
$g(\beta)=f(\beta y+(1-\beta)x)$ for $\beta\in\Bbb R$.   Then $g$ is three
times differentiable, with

$$\eqalign{g'(\beta)
&=\sum_{k=1}^r(\eta_k-\xi_k)\Pd{f}{\xi_k}(\beta y+(1-\beta)x),\cr
g''(\beta)
&=\sum_{j=1}^r\sum_{k=1}^r(\eta_j-\xi_j)(\eta_k-\xi_k)
         \Pde{f}{\xi_j}{\xi_k}(\beta y+(1-\beta)x),\cr
g'''(\beta)	
&=\sum_{i=1}^r\sum_{j=1}^r\sum_{k=1}^r
	        (\eta_i-\xi_i)(\eta_j-\xi_j)(\eta_k-\xi_k)
      \Bover{\partial^3f}{\partial\xi_i\partial\xi_j\partial\xi_k}
        (\beta y+(1-\beta)x).\cr}$$

\noindent Now by Taylor's theorem with remainder, in one dimension, there
is a $\beta\in\ooint{0,1}$ such that

\Centerline{$g(1)=g(0)+g'(0)+\Bover12g''(0)+\Bover16g'''(\beta)$}

\noindent and all we have to do is to
set $z=\beta y+(1-\beta)x$ and substitute in the values for
$g(1),\ldots,g'''(\beta)$.\ \Qed

\medskip

\quad{\bf (ii)} Let $M\ge 0$ be such that
$\|\Bover{\partial^3f}{\partial\xi_i\partial\xi_j\partial\xi_k}\|_{\infty}
\le M$ whenever $1\le i$, $j$, $k\le r$.   Let $K$ be
$\Expn((\sum_{i=1}^r|Z_i|)^3)$ when $Z_1,\ldots,Z_r$ are independent
real-valued
random variables with standard normal distribution.   (To see that this is
finite, observe that

$$\eqalignno{\Expn((\sum_{i=1}^r|Z_i|)^3)
&\le\Expn(r^3\max_{i\le r}|Z_i|^3)
\le r^3\,\Expn(\sum_{i=1}^r|Z_i|^3)
=r^4\,\Expn(|Z|^3)\cr
\displaycause{where $Z$ is a random variable with standard normal
distribution}
&=\Bover{2r^4}{\sqrt{2\pi}}\int_0^{\infty}t^3e^{-t^2/2}dt
<\infty.)\cr}$$

\noindent For any $x$, $y\in\BbbR^r$ we have

$$\eqalign{\bigl|f(y)-f(x)-\sum_{i=1}^r(\eta_i-\xi_i)\Pd{f}{\xi_i}(x)
             &+\Bover12\sum_{i=1}^r\sum_{j=1}^r(\eta_i-\xi_i)(\eta_j-\xi_j)
	        \Pde{f}{\xi_i}{\xi_j}(x)\bigr|\cr
&\le\Bover{M}6(\sum_{i=1}^r|\eta_i-\xi_i|)^3.\cr}$$
		
\noindent If
$0\le s\le t$ and $\omega\in\Omega$, then

$$\eqalign{\bigl|f(\omega(t))-f(\omega(s))
  &-\sum_{i=1}^r\Pd{f}{\xi_i}(\omega(s))(\omega_i(t)-\omega_i(s))\cr
&\mskip100mu
   -\Bover12\sum_{i=1}^r\sum_{j=1}^r\Pde{f}{\xi_i}{\xi_j}(\omega(s))
      (\omega_i(t)-\omega_i(s))(\omega_j(t)-\omega_j(s))\bigr|\cr
&\le\Bover{M}6(\sum_{i=1}^r|\omega_i(t)-\omega_i(s)|)^3,\cr}$$

\noindent writing $\omega_1,\ldots,\omega_r\in C(\coint{0,\infty})_0$
for the coordinates of
$\omega\in\Omega$.   Integrating with respect to $\omega$, we have

$$\eqalignno{|\Expn(f(X_t)-&f(X_s)-\Bover12(t-s)(\nabla^2f)(X_s))|\cr
&=\bigl|\Expn(f(X_t)-f(X_s))
  -\sum_{i=1}^r\Expn(\Pd{f}{\xi_i}(X_s))\Expn(X^{(i)}_t-X^{(i)}_s)\cr
&\mskip100mu
  -\Bover12\sum_{i=1}^r\Expn(\Pdd{f}{\xi_i}(X_s))
      \Expn(X^{(i)}_t-X^{(i)}_s)^2\cr
&\mskip100mu
  -\Bover12\sum_{i=1}^r\sum_{j\ne i}\Expn(\Pde{f}{\xi_i}{\xi_j})(X_s)
      \Expn(X^{(i)}_t-X^{(i)}_s)\Expn(X^{(j)}_t-X^{(j)}_s)\bigr|\cr
\displaycause{writing $X^{(i)}_t(\omega)=\omega_i(t)$ for
$1\le i\le r$, and recalling that $\Expn(X^{(i)}_t-X^{(i)}_s)=0$ for
every $i$, while $\Expn(X^{(i)}_t-X^{(i)}_s)^2=t-s$}
&=\bigl|\Expn(f(X_t)-f(X_s))
  -\sum_{i=1}^r\Expn((X^{(i)}_t-X^{(i)}_s)\Pd{f}{\xi_i}(X_s))\cr
&\mskip100mu
  -\Bover12\sum_{i=1}^r\Expn((X^{(i)}_t-X^{(i)}_s)^2\Pdd{f}{\xi_i}(X_s))\cr
&\mskip100mu
  -\Bover12\sum_{i=1}^r\sum_{j\ne i}
     \Expn((X^{(i)}_t-X^{(i)}_s)(X^{(j)}_t-X^{(j)}_s)
        \Pde{f}{\xi_i}{\xi_j}(X_s))\bigr|\cr
\displaycause{because for any $i\le r$ the random variables
$\Pd{f}{\xi_i}(X_s)$ and $X^{(i)}_t-X^{(i)}_s$ are independent, while
for any distinct $i$, $j\le r$ the random variables
$\Pde{f}{\xi_i}{\xi_j}(X_s)$, $X^{(i)}_t-X^{(i)}_s$ and
$X^{(j)}_t-X^{(j)}_s$ are independent}
&=\bigl|\Expn\bigl(f(X_t)-f(X_s)
  -\sum_{i=1}^r(X^{(i)}_t-X^{(i)}_s)\Pd{f}{\xi_i}(X_s)\cr
&\mskip100mu  -\Bover12\sum_{i=1}^r\sum_{j=1}^r
     (X^{(i)}_t-X^{(i)}_s)(X^{(j)}_t-X^{(j)}_s)
        \Pde{f}{\xi_i}{\xi_j}(X_s)\bigr)\bigr|\cr
&\le\Bover{M}6\Expn((\sum_{i=1}^r|X^{(i)}_t-X^{(i)}_s|)^3)
\le\Bover{MK}6(t-s)^{3/2}\cr}$$

\noindent because the $X^{(i)}_t-X^{(i)}_s$ are independent random
variables all with the same distribution as $\sqrt{t-s}\,Z$ where $Z$ is
standard normal.

\woddheader{478K}{4}{2}{2}{36pt}

{\bf (iii)} Now fix $t\ge 0$ and $n\ge 1$;  set $s_k=\Bover{k}{n}t$ for
$k\le n$.   Set

\Centerline{$g_n(\omega)
=\sum_{k=0}^{n-1}(s_{k+1}-s_k)(\nabla^2f)(\omega(s_k))$}

\noindent for $\omega\in\Omega$.   Then

$$\eqalign{
\bigl|\int\bigl(f(\omega(t))-&f(0)
  -\Bover12g_n(\omega)\bigr)\mu_W(d\omega)\bigr|\cr
&=\bigl|\sum_{k=0}^{n-1}\Expn(f(X_{s_{k+1}})-f(X_{s_k})
  -\Bover12(s_{k+1}-s_k)(\nabla^2f)(X_{s_k}))\bigr|\cr
&\le\sum_{k=0}^{n-1}\Bover{MK}6(\Bover{t}n)^{3/2}
=\Bover{MKt\sqrt t}{6\sqrt n}.\cr}$$

\noindent On the other hand,

\Centerline{$\lim_{n\to\infty}g_n(\omega)
=\int_0^t(\nabla^2f)(\omega(s))ds$}

\noindent for every $\omega$ (the Riemann integral
$\int_0^t(\nabla^2f)(\omega(s))ds$ is defined because
$s\mapsto(\nabla^2f)(\omega(s))$ is continuous), and
$|g_n(\omega)|\le t\|\nabla^2f\|_{\infty}<\infty$ for every $\omega$, so by
Lebesgue's Dominated Convergence Theorem

$$\eqalign{\Expn(f(X_t)-f(0))
&=\Bover12\lim_{n\to\infty}\int g_n(\omega)\mu_W(d\omega)
=\Bover12\int\lim_{n\to\infty}g_n(\omega)\mu_W(d\omega)\cr
&=\Bover12\int\int_0^t(\nabla^2f)(\omega(s))ds\,\mu_W(d\omega)
=\Bover12\Expn\bigl(\int_0^t(\nabla^2f)(X_s)ds\bigr)\cr}$$

\noindent as claimed.

\medskip

{\bf (b)(i)}
Consider first the case in which $\tau$ takes values in a
finite set $I\subseteq\coint{0,\infty}$.
In this case we can induce on $\#(I)$.   If $I=\{t_0\}$ then

\Centerline{
$\Expn(f(X_{\tau}))
=\Expn(f(X_{t_0}))
=f(0)+\Bover12\Expn(\int_0^{t_0}(\nabla^2f)(X_s)ds)
=f(0)+\Bover12\Expn(\int_0^{\tau}(\nabla^2f)(X_s)ds)$}

\noindent by (a).   For the inductive step to $\#(I)>1$, set $t_0=\min I$,
$E=\{\omega:\tau(\omega)=t_0\}$ and

$$\eqalign{\phi(\omega,\omega')(t)
&=\omega(t)\text{ if }t\le t_0,\cr
&=\omega(t_0)+\omega'(t-t_0)\text{ if }t\ge t_0,\cr}$$

\noindent for $\omega$, $\omega'\in\Omega$, so that $\phi$ is \imp\ (477G).
Set

\Centerline{$\sigma_{\omega}(\omega')
=\tau(\phi(\omega,\omega'))-t_0$}

\noindent for $\omega$, $\omega'\in\Omega$.   If
$\omega\in\Omega\setminus E$,
$\sigma_{\omega}$ is a stopping time adapted to
$\langle\Sigma_t^+\rangle_{t\ge 0}$, taking fewer than $\#(I)$ values.
\Prf\ Suppose that $t>0$ and $F=\{\omega':\sigma_{\omega}(\omega')<t\}$.
If $\omega'\in F$, $\tilde\omega'\in\Omega$ and
$\tilde\omega'\restr[0,t]=\omega'\restr[0,t]$,
then $\tau(\phi(\omega,\omega'))<t+t_0$, while
$\phi(\omega,\tilde\omega')(s)=\phi(\omega,\omega')(s)$ whenever
$s\le t+t_0$;  so that

\Centerline{$\sigma_{\omega}(\tilde\omega')+t_0
=\tau(\phi(\omega,\tilde\omega'))<t+t_0$}

\noindent and $\tilde\omega'\in F$.   Thus $F\in\Sigma_t$;  as $t$ is
arbitrary, $\sigma_{\omega}$ is adapted to
$\langle\Sigma_t^+\rangle_{t\ge 0}$ (455Lb).
Also every value of $\sigma_{\omega}$ belongs to
$\{t-t_0:t\in I$, $t>t_0\}$ which is smaller than $I$.\ \Qed

Writing $\int\ldots d\omega$ and
$\int\ldots d\omega'$ for integration with respect to $\mu_W$,

\global\pagegoalminus=\the\pagegoal
\global\pagetotalplus=\the\pagetotal
\global\advance\pagegoalminus by -\pageshrink
\global\advance\pagetotalplus by \bigbaselineskip

$$\eqalignno{\Expn\bigl(\int_0^{\tau}&(\nabla^2f)(X_s)ds\bigr)\cr
\global\advance\pagetotalplus by \bigbaselineskip
&=\int_{\Omega}\int_0^{\tau(\omega)}(\nabla^2f)(\omega(s))
  ds\,d\omega\cr
\global\advance\pagetotalplus by \bigbaselineskip
&=\int_{\Omega}\int_{\Omega}\int_0^{\tau(\phi(\omega,\omega'))}
  (\nabla^2f)(\phi(\omega,\omega')(s))
  ds\,d\omega'd\omega\cr
\checkbreak
&=\int_E\int_{\Omega}\int_0^{t_0}
    (\nabla^2f)(\omega(s))ds\,d\omega'd\omega\cr
\global\advance\pagetotalplus by \bigbaselineskip
&\mskip100mu  +\int_{\Omega\setminus E}\int_{\Omega}
    \int_0^{t_0+\sigma_{\omega}(\omega')}
    (\nabla^2f)(\phi(\omega,\omega')(s))
    ds\,d\omega'd\omega\cr
\checkbreak
&=\int_E\int_0^{t_0}(\nabla^2f)(\omega(s))ds\,d\omega
  +\int_{\Omega\setminus E}\int_{\Omega}\int_0^{t_0}
    (\nabla^2f)(\phi(\omega,\omega')(s))
    ds\,d\omega'd\omega\cr
\global\advance\pagetotalplus by \bigbaselineskip
&\mskip100mu  +\int_{\Omega\setminus E}\int_{\Omega}
    \int_{t_0}^{t_0+\sigma_{\omega}(\omega')}
    (\nabla^2f)(\phi(\omega,\omega')(s))
    ds\,d\omega'd\omega\cr
&=\int_{\Omega}\int_0^{t_0}(\nabla^2f)(\omega(s))ds\,d\omega\cr
\global\advance\pagetotalplus by \bigbaselineskip
&\mskip100mu
  +\int_{\Omega\setminus E}\int_{\Omega}
    \int_{t_0}^{t_0+\sigma_{\omega}(\omega')}
    (\nabla^2f)(\omega(t_0)+\omega'(s-t_0))
    ds\,d\omega'd\omega\cr
\checkbreak
&=\int_{\Omega}\int_0^{t_0}(\nabla^2f)(\omega(s))ds\,d\omega
  +\int_{\Omega\setminus E}\int_{\Omega}\int_0^{\sigma_{\omega}(\omega')}
    (\nabla^2f)(\omega(t_0)+\omega'(s))
    ds\,d\omega'd\omega\cr
\checkbreak
&=2\int_{\Omega}f(\omega(t_0))-f(0)d\omega\cr
\global\advance\pagetotalplus by \bigbaselineskip
&\mskip100mu
  +2\int_{\Omega\setminus E}\int_{\Omega}
    f(\omega(t_0)+\omega'(\sigma_{\omega}(\omega')))-f(\omega(t_0))
    d\omega'd\omega\cr
\displaycause{\global\advance\pagetotalplus by \baselineskip
applying the inductive hypothesis to the function
$x\mapsto f(\omega(t_0)+x)$ and the stopping time $\sigma_{\omega}$}
\checkbreak
&=2\int_E f(\omega(t_0))-f(0)d\omega
  +2\int_{\Omega\setminus E}\int_{\Omega}
    f(\omega(t_0)+\omega'(\sigma_{\omega}(\omega')))-f(0)
    d\omega'd\omega\cr
\checkbreak
&=2\int_E\int_{\Omega}
   f\bigl(\phi(\omega,\omega')(\tau(\phi(\omega,\omega')))\bigr)
     -f(0)d\omega'd\omega\cr
\global\advance\pagetotalplus by \bigbaselineskip
&\mskip100mu  +2\int_{\Omega\setminus E}\int_{\Omega}
    f\bigl(\phi(\omega,\omega')(\tau(\phi(\omega,\omega')))\bigr)
      -f(0)d\omega'd\omega\cr
\checkbreak
&=2\int_{\Omega}\int_{\Omega}
    f\bigl(\phi(\omega,\omega')(\tau(\phi(\omega,\omega')))\bigr)
      -f(0)d\omega'd\omega\cr
\checkbreak
&=2\int_{\Omega}f(\omega(\tau(\omega)))-f(0)d\omega
=2\bigl(\Expn(f(X_{\tau}))-f(0)\bigr).\cr}$$

\noindent Turning this around, we have the formula we want, so the
induction proceeds.

\medskip

\quad{\bf (ii)} Now suppose that every value of
$\tau$ belongs to an infinite set of
the form $\{t_n:n\in\Bbb N\}\cup\{\infty\}$ where $\sequencen{t_n}$ is
a strictly increasing sequence in $\coint{0,\infty}$.
In this case, for $n\in\Bbb N$, define

\Centerline{$\tau_n(\omega)=\min(\tau(\omega),t_n)$}

\noindent for $\omega\in\Omega$, so that $\tau_n$ takes
values in the finite set $\{t_0,\ldots,t_n\}$, and

$$\eqalign{\{\omega:\tau_n(\omega)<t\}
&=\{\omega:\tau(\omega)<t\}\in\Sigma_t\text{ if }t\le t_n,\cr
&=\Omega\in\Sigma_t\text{ if }t>t_n.\cr}$$
Now $\tau$ is finite a.e., so
$\tau\eae\lim_{n\to\infty}\tau_n$;  it follows that

\Centerline{$f(X_{\tau}(\omega))
=f(\omega(\tau(\omega)))
=\lim_{n\to\infty}f(\omega(\tau_n(\omega)))
=\lim_{n\to\infty}f(X_{\tau_n}(\omega))$}

\noindent for almost
every $\omega$;  because $f$ is bounded,

\Centerline{$\Expn(f(X_{\tau}))=\lim_{n\to\infty}\Expn(f(X_{\tau_n}))$.}

\noindent On the other side,

\Centerline{$\int_0^{\tau(\omega)}(\nabla^2f)(\omega(s))ds
=\lim_{n\to\infty}\int_0^{\tau_n(\omega)}(\nabla^2f)(\omega(s))ds$}

\noindent for almost
every $\omega$.   At this point, recall that we are supposing
that $\tau$ has finite expectation and that $\nabla^2f$ is bounded.
So

\Centerline{$|\int_0^{\tau_n(\omega)}(\nabla^2f)(\omega(s))ds|
\le\int_0^{\tau(\omega)}|(\nabla^2f)(\omega(s))ds|
\le\|\nabla^2f\|_{\infty}\tau(\omega)$}

\noindent for every $\omega$, and the dominated convergence theorem assures
us that

$$\eqalign{\Expn\bigl(\int_0^{\tau}(\nabla^2f)(X_s)ds\bigr)
&=\int_{\Omega}\int_0^{\tau(\omega)}(\nabla^2f)(\omega(s))ds\,d\omega\cr
&=\lim_{n\to\infty}
  \int_{\Omega}\int_0^{\tau_n(\omega)}(\nabla^2f)(\omega(s))ds\,d\omega\cr
&=\lim_{n\to\infty}\Expn\bigl(\int_0^{\tau_n}(\nabla^2f)(X_s)ds\bigr).
\cr}$$

\noindent Accordingly

$$\eqalignno{\Expn(f(X_{\tau}))
&=\lim_{n\to\infty}\Expn(f(X_{\tau_n}))
=\lim_{n\to\infty}f(0)
  +\Bover12\Expn\bigl(\int_0^{\tau_n}(\nabla^2f)(X_s)ds\bigr)\cr
\displaycause{by (i)}
&=f(0)+\Bover12\Expn\bigl(\int_0^{\tau}(\nabla^2f)(X_s)ds\bigr),\cr}$$

\noindent as required.

\medskip

\quad{\bf (iii)} Suppose just that $\tau$ has finite
expectation.   This time, for $n\in\Bbb N$, define a stopping time $\tau_n$
by saying that

$$\eqalign{\tau_n(\omega)&=2^{-n}k\text{ if }k\ge 1
  \text{ and }2^{-n}(k-1)\le\tau(\omega)<2^{-n}k,\cr
&=\infty\text{ if }\tau(\omega)=\infty.\cr}$$

\noindent If $t>0$, set $t'=2^{-n}k$ where $2^{-n}k<t\le 2^{-n}(k+1)$;
then

\Centerline{$\{\omega:\tau_n(\omega)<t\}
=\{\omega:\tau(\omega)<t'\}\in\Sigma_{t'}\subseteq\Sigma_t$.}

\noindent So $\tau_n$ is a stopping time adapted to
$\langle\Sigma_t^+\rangle_{t\ge 0}$;  as $\tau_n\le 2^{-n}+\tau$,
$\Expn(\tau_n)<\infty$.   Again we have
$\tau(\omega)=\lim_{n\to\infty}\tau_n(\omega)$ for every
$\omega$.   The arguments of (ii) now tell us that, as before,

\Centerline{$f(X_{\tau}(\omega))=\lim_{n\to\infty}f(X_{\tau_n}(\omega))$}

\noindent(because $f$ is continuous),

\Centerline{$\int_0^{\tau(\omega)}(\nabla^2f)(\omega(s))ds
=\lim_{n\to\infty}\int_0^{\tau_n(\omega)}(\nabla^2f)(\omega(s))ds$}

\noindent for almost every $\omega$, so that

\Centerline{$\Expn(f(X_{\tau}))=\lim_{n\to\infty}\Expn(f(X_{\tau_n}))$,}

\Centerline{$\Expn(\int_0^{\tau}(\nabla^2f)(X_s)ds)
=\lim_{n\to\infty}\Expn(\int_0^{\tau_n}(\nabla^2f)(X_s)ds)$.}

\noindent (This time, of course, we need to check that

\Centerline{$|\int_0^{\tau_n(\omega)}(\nabla^2f)(\omega(s))ds|
\le\|\nabla^2f\|_{\infty}\tau_0(\omega)$}

\noindent for almost every $\omega$, to confirm that we have dominated
convergence.)   So once again the desired formula can be got by taking the
limit of a sequence of equalities we already know.
}%end of proof of 478K

\leader{478L}{Theorem} Let $\mu_W$ be $r$-dimensional Wiener measure on
$\Omega=C(\coint{0,\infty};\BbbR^r)_0$,
$f:\BbbR^r\to[0,\infty]$ a lower semi-continuous superharmonic
function, and $\tau:\Omega\to[0,\infty]$ a stopping time adapted to
$\langle\Sigma_t^+\rangle_{t\ge 0}$.   Set
$H=\{\omega:\omega\in\Omega$, $\tau(\omega)<\infty\}$.   Then

\Centerline{$f(x)
  \ge\int_Hf(x+\omega(\tau(\omega)))\mu_W(d\omega)$}

\noindent for every $x\in\BbbR^r$.

\proof{{\bf (a)} To begin with, suppose that $f$ is
real-valued and bounded.
Let $\sequence{m}{\tilde h_m}$ be the sequence of 473E/478J, and
for $m\in\Bbb N$ set $f_m=f*\tilde h_m$.   Then each $f_m$ is non-negative,
smooth with
bounded derivatives of all orders (473De) and superharmonic (478Ja), so
$\nabla^2f_m\le 0$ (478Ea).   Set $\tau_n(\omega)=\min(n,\tau(\omega))$ for
$n\in\Bbb N$ and $\omega\in\Omega$;  then each $\tau_n$ is a stopping time,
adapted to $\langle\Sigma^+_t\rangle_{t\ge 0}$, with finite expectation.
In the language of 478K,

\Centerline{$\Expn(f_m(x+X_{\tau_n}))
=f_m(x)+\Bover12\Expn\bigl(\int_0^{\tau_n}(\nabla^2f_m)(x+X_s)ds\bigr)
\le f_m(x)$}

\noindent whenever $m$, $n\in\Bbb N$.   Letting $n\to\infty$,

$$\eqalignno{\int_Hf_m(x+\omega(\tau(\omega)))\mu_W(d\omega)
&=\lim_{n\to\infty}\int_Hf_m(x+\omega(\tau_n(\omega)))\mu_W(\omega)\cr
\displaycause{because $f_m$, and every $\omega\in\Omega$, are continuous,
and $\tau(\omega)=\lim_{n\to\infty}\tau_n(\omega)$ for every $\omega$}
&\le\liminf_{n\to\infty}
   \int_{\BbbR^r}f_m(x+\omega(\tau_n(\omega)))\mu_W(d\omega)\cr
\displaycause{because $f_m$ is non-negative}
&\le f_m(x)\cr}$$

\noindent by Fatou's Lemma.   Now $f=\lim_{m\to\infty}f_m$ (478Jc), so

$$\eqalign{\int_Hf(x+\omega(\tau(\omega)))\mu_W(\omega)
&\le\liminf_{m\to\infty}
    \int_Hf_m(x+\omega(\tau(\omega)))\mu_W(\omega)\cr
&\le\liminf_{m\to\infty}f_m(x)
=f(x),\cr}$$

\noindent which is what we need to know.

\medskip

{\bf (b)} For the general case, set $g_k=f\wedge k\chi\BbbR^r$ for each
$k\in\Bbb N$.   Then $g_k$ is non-negative,
lower semi-continuous, superharmonic (478Cc) and bounded.   So

$$\eqalign{\int_Hf(x+\omega(\tau(\omega)))\mu_W(d\omega)
&=\lim_{k\to\infty}\int_Hg_k(x+\omega(\tau(\omega)))\mu_W(d\omega)\cr
&\le\lim_{k\to\infty}g_k(x)
=f(x).\cr}$$
}%end of proof of 478L

\leader{478M}{Proposition} (a) If $r=1$, then $\{\omega(t):t\ge 0\}=\Bbb R$
for almost every $\omega\in\Omega$.

(b) If $r\le 2$, then
$\{\omega(t):t\ge 0\}$ is dense in $\BbbR^2$ for almost every
$\omega\in\Omega$.

(c) If $r\ge 2$, then
for every $z\in\BbbR^2$, $z\notin\{\omega(t):t>0\}$ for almost
every $\omega\in\Omega$.

(d) If $r\ge 3$, then
$\lim_{t\to\infty}\|\omega(t)\|=\infty$ for
almost every $\omega\in\Omega$.

\proof{{\bf (a)} Suppose that
$\alpha$, $\beta>0$ and that $\tau$ is the Brownian exit time from
$\ooint{-\alpha,\beta}$;  then $\tau$ is a stopping time adapted to
$\langle\Sigma_t\rangle_{t\ge 0}$ (477Ic).   Now
$\tau$ is almost everywhere finite and
$\Pr(X_{\tau}=\beta)=\Bover{\alpha}{\alpha+\beta}$.   \Prf\
Since
$\Pr(|X_t|\le\max(\alpha,\beta))\to 0$ as $t\to\infty$, $\tau$ is finite
a.e., and $\Pr(X_{\tau}=\beta)+\Pr(X_{\tau}=-\alpha)=1$.   Set
$\tau_n(\omega)=\min(n,\tau(\omega))$ for each $n$, and $f(x)=x$ for
$x\in\Bbb R$.   Then 478K tells us that

\Centerline{$\Expn(X_{\tau_n})=\Expn(f(X_{\tau_n}))
=f(0)+\Bover12\Expn(\int_0^{\tau_n}(\nabla^2f)(X_s)ds)=f(0)=0$.}

\noindent Since $\sequencen{X_{\tau_n}}$ is a uniformly bounded sequence
converging almost everywhere to $X_{\tau}$,

\Centerline{$\beta\Pr(X_{\tau}=\beta)-\alpha\Pr(X_{\tau}=-\alpha)
=\Expn(X_{\tau})=\lim_{n\to\infty}\Expn(X_{\tau_n})=0$,}

\noindent and $\Pr(X_{\tau}=\beta)=\Bover{\alpha}{\alpha+\beta}$.\ \Qed

Letting $\alpha\to\infty$, we see that
$\Pr(\Exists t\ge 0,\,X_t=\beta)=1$.   Similarly, $-\alpha$ lies on
almost every sample path.

Thus almost every sample path must pass through every point of $\Bbb Z$;
since sample paths are continuous, they almost all cover $\Bbb R$.

\medskip

{\bf (b)} For $r=1$ this is covered by (a);  take $r=2$.
Suppose that $z\in\Bbb R^2$ and that
$\delta>0$.   Then almost every sample path meets $B(z,\delta)$.
\Prf\ If $\delta\ge\|z\|$ this is trivial.
Otherwise, take $R>\|z\|$ and let $\tau$ be
the Brownian exit time from $G=\interior B(z,R)\setminus B(z,\delta)$.
We have $\Pr(\|X_t\|\le R+\|z\|)\to 0$ as $t\to\infty$
(because $\Pr(\|X_t\|\le\alpha)\le\Pr(|Z|\le\Bover{\alpha}{\sqrt{t}})$
where
$Z$ is a standard normal random variable), so $\tau$ is finite a.e.
Once again, set $\tau_n(\omega)=\min(n,\tau(\omega))$ for $n\in\Bbb N$ and
$\omega\in\Omega$;  this time,
take $f(x)=\ln\|x-z\|$ for $x\in\BbbR^2\setminus\{z\}$.   Then

\Centerline{$\Expn(f(X_{\tau_n}))=f(0)=\ln\|z\|$}

\noindent(use 478Fb), so

$$\eqalign{\ln R\cdot\Pr(X_{\tau}\in\partial B(z,R))
  &+\ln\delta\cdot\Pr(X_{\tau}\in\partial B(z,\delta)\cr
&=\Expn(f(X_{\tau}))
=\lim_{n\to\infty}\Expn(f(X_{\tau_n}))
=\ln\|z\|\cr}$$

\noindent and

\Centerline{$\Pr(X_{\tau}\in\partial B(z,\delta))
=\Bover{\ln R-\ln\|z\|}{\ln R-\ln\delta}$.}

\noindent Letting $R\to\infty$, we see that
$\Pr(\Exists t\ge 0,\,\omega(t)\in B(z,\delta))=1$;  that is, almost every
sample path meets $B(z,\delta)$.\ \Qed

Letting $B(z,\delta)$ run over a
sequence of balls constituting a network for the topology of $\BbbR^2$,
we see that almost every path meets every non-empty open set and is dense
in $\BbbR^2$.

\medskip

{\bf (c)(i)} Consider first the case $r=2$.

\medskip

\qquad\grheada\ Suppose that $z\ne 0$.   In this case,
take $\delta$, $R$ such that
$0<\delta<\|z\|<R$ and let $\tau$ be
the Brownian exit time from $G=\interior B(z,R)\setminus B(z,\delta)$,
as in the proof of (b).   As before, we have

\Centerline{$\Pr(X_{\tau}\in\partial B(z,\delta))
=\Bover{\ln R-\ln\|z\|}{\ln R-\ln\delta}$.}

\noindent This time, looking at the limit as $\delta\downarrow 0$, we see
that

\Centerline{$\{\omega:$ there is a $t\ge 0$ such that $\omega(t)=z$ but
$\|\omega(s)-z\|<R$ for every $s\le t\}$}

\noindent is negligible.   Taking the union of these sets over large
integer $R$, we see that

\Centerline{$\{\omega:$ there is a $t\ge 0$ such that $\omega(t)=z\}$}

\noindent is negligible, as required.

\medskip

\qquad\grheadb\ As for $z=0$, take any $\epsilon>0$.   Then

$$\eqalignno{
\mu_W\{\omega:&\text{ there is some }t\ge\epsilon
  \text{ such that }\omega(t)=0\}\cr
&=\mu_W^2\{(\omega,\omega'):
  \text{ there is some }t\ge 0
  \text{ such that }\omega'(t)=-\omega(\epsilon)\}\cr
&=\mu_W^2\{(\omega,\omega'):\omega(\epsilon)\ne 0
  \text{ and there is some }t\ge 0
  \text{ such that }\omega'(t)=-\omega(\epsilon)\}\cr
\displaycause{because the distribution of $X_{\epsilon}$ is atomless, so
$\{\omega:\omega(\epsilon)=0\}$ is negligible}
&=0\cr}$$

\noindent by ($\alpha$).   Taking the union over rational $\epsilon>0$,
$\{\omega:\text{ there is some }t>0\text{ such that }\omega(t)=0\}$ is
negligible.

\medskip

\quad{\bf (ii)} If $r>2$, set $Tx=(\xi_1,\xi_2)$ for
$x=(\xi_1,\ldots,\xi_r)\in\BbbR^r$.   Then
$\omega\mapsto T\omega:
C(\coint{0,\infty};\BbbR^r)_0\to C(\coint{0,\infty};\BbbR^2)_0$ is
\imp\ for $r$-dimensional Wiener measure $\mu_{Wr}$ on
$C(\coint{0,\infty};\BbbR^r)_0$ and
two-dimensional Wiener measure $\mu_{W2}$ on
$C(\coint{0,\infty};\BbbR^2)_0$, by 477D(c-i) or otherwise.
So

\Centerline{$\{\omega:z\in\omega[\,\ooint{0,\infty}\,]\}
\subseteq\{\omega:Tz\in(T\omega)[\,\ooint{0,\infty}\,]\}$}

\noindent is negligible.

\medskip

{\bf (d)(i)} Fix $\gamma\in\coint{0,\infty}$ and $\epsilon>0$ for the
moment.
Set $g(x)=\int\Bover1{\|y\|^{r-2}}\tilde h_0(x-y)\mu(dy)$ for
$x\in\BbbR^r$, where $\tilde h_0$ is the function of 473E;
then $g$ is smooth (473De), strictly positive and superharmonic (478Ja).
In addition, we have the following.

\medskip

\qquad\grheada\ All the derivatives of $g$ are bounded.   \Prf\
As shown in the proof of 473De, $\Pd{g}{\xi_i}(x)
=\int\Bover1{\|y\|^{r-2}}\Pd{}{\xi_i}\tilde h_0(x-y)\mu(dy)$ for
$1\le i\le r$ and $x\in\BbbR^r$.
Inducing on the order of $D$, and using 478Gc at the last step,
we see that

$$\eqalign{(Dg)(x)
&=\int\Bover1{\|y\|^{r-2}}(D\tilde h_0)(x-y)\mu(dy)
=\int\Bover1{\|x-y\|^{r-2}}(D\tilde h_0)(y)\mu(dy)\cr
&=\int_{B(\tbf{0},1)}\Bover1{\|x-y\|^{r-2}}(D\tilde h_0)(y)\mu(dy)\cr
&\le\|D\tilde h_0\|_{\infty}\int_{B(\tbf{0},1)}\Bover1{\|x-y\|^{r-2}}\mu(dy)
\le\Bover12r\beta_r\|D\tilde h_0\|_{\infty}\cr}$$

\noindent for
any partial differential operator $D$ and any $x\in\BbbR^r$.\ \Qed

\medskip

\qquad\grheadb\ $\lim_{\|x\|\to\infty}g(x)=0$, because

\Centerline{$g(x)
\le\|\tilde h_0\|_{\infty}\int_{B(\tbf{0},1)}\Bover1{\|x-y\|^{r-2}}\mu(dy)
\le\|\tilde h_0\|_{\infty}\Bover{\beta_r}{(\|x\|-1)^{r-2}}$}

\noindent whenever $\|x\|>1$.

\medskip

\qquad\grheadc\ $g(x)=g(y)$ whenever $\|x\|=\|y\|$.   \Prf\ Let
$T:\BbbR^r\to\BbbR^r$ be an orthogonal transformation such that $Tx=y$.
Then

$$\eqalignno{g(x)
&=\int\Bover1{\|x-z\|^{r-2}}\tilde h_0(z)\mu(dz)
=\int\Bover1{\|T(x-z)\|^{r-2}}\tilde h_0(Tz)\mu(dz)\cr
\displaycause{because $\tilde h_0T=\tilde h_0$}
&=\int\Bover1{\|y-Tz\|^{r-2}}\tilde h_0(Tz)\mu(dz)
=\int\Bover1{\|y-z\|^{r-2}}\tilde h_0(z)\mu(dz)\cr
\displaycause{because $T$ is an automorphism of $(\BbbR^r,\mu)$}
&=g(y).  \text{  \Qed}\cr}$$

\medskip

\quad{\bf (ii)} Let $\beta>0$ be the common value of $g(y)$ for
$\|y\|=\gamma$.  Take $x\in\BbbR^r$ such that $\|x\|>\gamma$,
and $n\in\Bbb N$.   Define

\Centerline{$\tau(\omega)=\min(\{n\}\cup\{t:\|x+\omega(t)\|\le\gamma\})$}

\noindent for $\omega\in\Omega$.   Then $\tau$ is a bounded stopping time
adapted to $\langle\Sigma_t\rangle_{t\ge 0}$, so

$$\eqalign{\beta\Pr(\tau<n)
&\le\Expn(g(x+X_{\tau}))\cr
&=g(x)+\Bover12\Expn(\int_0^{\tau}(\nabla^2g)(x+X_s)ds)
\le g(x).\cr}$$

\noindent Letting $n\to\infty$, we see that

\Centerline{$\mu_W\{\omega:\|x+\omega(t)\|\le\gamma$ for some $t\ge 0\}
\le\Bover1{\beta}g(x)$.}

\medskip

\quad{\bf (iii)} Now let $n>\gamma$ be an integer such that
$\Bover1{\beta}g(x)\le\epsilon$
whenever $\|x\|\ge n$.   As in (a) and (b-i) above,
$\lim_{t\to\infty}\Pr(\|X_t\|\le n)=0$;  take $m\in\Bbb N$ such that
$\Pr(\|X_m\|\le n)\le\epsilon$.   Let $\sigma$ be the stopping time with
constant value $m$, with $\phi_{\sigma}:\Omega\times\Omega\to\Omega$ the
corresponding \imp\ function (477G).   Set
$F=\{\omega:\|\omega(m)\|>n\}$.   Now

$$\eqalignno{\Pr(\|X_t\|\le\gamma&\text{ for some }t\ge m)\cr
&=\mu_W^2\{(\omega,\omega'):\|\phi_{\sigma}(\omega,\omega')(t)\|\le\gamma
  \text{ for some }t\ge m\}\cr
\displaycause{where $\mu_W^2$ is the product measure on
$\Omega\times\Omega$}
&=\mu_W^2\{(\omega,\omega'):\|\omega(m)+\omega'(t-m)\|\le\gamma
  \text{ for some }t\ge m\}\cr
&\le\mu_W^2\{(\omega,\omega'):\|\omega(m)\|\le n\text{ or }
  \|\omega(m)\|\ge n\cr
&\mskip200mu
  \text{and }\|\omega(m)+\omega'(t)\|\le\gamma\text{ for some }t\ge 0\}\cr
&\le\mu\{\omega:\|\omega(m)\|\le n\}\cr
&\mskip50mu
  +\int_F\mu_W\{\omega':\|\omega(m)+\omega'(t)\|\le\gamma\text{ for some }
     t\ge 0\}\mu_W(d\omega)\cr
&\le\epsilon
  +\int_F\Bover1{\beta}g(\omega(m))\mu_W(d\omega)
\le\epsilon+\epsilon\mu_WF
\le 2\epsilon.\cr}$$

\noindent As $\epsilon$ is arbitrary,
$\Pr(\liminf_{t\to\infty}\|X_t\|<\gamma)=0$;  as $\gamma$ is arbitrary,
$\Pr(\lim_{t\to\infty}\|X_t\|=\infty)=1$.
}%end of proof of 478M

\cmmnt{\medskip

\noindent{\bf Remark} In 479R I will show that there is a
surprising difference between the cases $r=3$ and $r\ge 4$.
}

\vleader{48pt}{478N}{Wandering paths} 
Let $G\subseteq\BbbR^r$ be an open set, and for $x\in G$ set

\Centerline{$F_x(G)=\{\omega:$ either $\tau_x(\omega)<\infty$ or
$\lim_{t\to\infty}\|\omega(t)\|=\infty\}$}

\noindent where $\tau_x$ is the
Brownian exit time from $G-x$.   I will say that
$G$ has {\bf few wandering paths} if $F_x(G)$ is conegligible for every
$x\in G$.   In this case we can
be sure that, if $x\in G$, then for almost every $\omega$ {\it either}
$\lim_{t\to\infty}\|\omega(t)\|=\infty$ {\it or} $\omega(t)\notin G-x$ for
some $t$.   So
we can speak of $X_{\tau_x}(\omega)=\omega(\tau_x(\omega))$, taking this to
be $\infty$ if $\omega\in F_x(G)$ and
$\tau_x(\omega)=\infty$;  and $\omega$ will be
continuous on $[0,\tau_x(\omega)]$ for every $\omega\in F_x(G)$.
\cmmnt{We find that }$X_{\tau_x}:\Omega\to\partial^{\infty}(G-x)$
is Borel
measurable.   \prooflet{\Prf\ $\tau_x$ is the Brownian hitting time to the
closed set $\BbbR^r\setminus(G-x)$, so is a stopping time adapted to
$\langle\Tau_{[0,t]}\rangle_{t\ge 0}$ (477Ic).   Let
$\Cal B(\Omega)$ be the
Borel $\sigma$-algebra of $\Omega$ for the topology of uniform convergence
on compact sets;  then $\Tau_{[0,t]}\subseteq\Cal B(\Omega)$ for every
$t\ge 0$.   The function

\Centerline{$(t,\omega)\mapsto X_t(\omega):
\coint{0,\infty}\times\Omega\to\BbbR^r$}

\noindent is continuous, therefore
$\Cal B(\coint{0,\infty})\tensorhat\Cal B(\Omega)$-measurable
(4A3D(c-i));  so $X_{\tau_x}$ is $\Cal B(\Omega)$-measurable (455Ld).\
\Qed}

From 478M, we see that if $r\ge 3$
then any open set in $\BbbR^r$ will have
few wandering paths, while if $r\le 2$ then $G$ will have few wandering
paths whenever it is not dense in $\BbbR^r$.
Note that if $G\subseteq\BbbR^r$ is open, $H$ is a component of $G$, and
$x\in H$, then the exit times from $H-x$ and $G-x$ are the same\cmmnt{,
just because sample paths are continuous}, and $F_x(G)=F_x(H)$.
It follows\cmmnt{ at once}
that if $G$ has more than one component then it has few wandering
paths.

\leader{478O}{Theorem} Let $G\subseteq\BbbR^r$ be an open set with few
wandering paths and
$f:\overline{G}^{\infty}\to\Bbb R$ a bounded lower semi-continuous function
such that $f\restr G$ is superharmonic.   Take $x\in G$ and let
$\tau:\Omega\to[0,\infty]$ be the
Brownian exit time from $G-x$\cmmnt{ (477Ia)}.
Then $f(x)\ge\Expn(f(x+X_{\tau}))$.

\proof{ It will be enough to deal with the case $f\ge 0$.

\medskip

{\bf (a)} Extend $f$ to a function
$\tilde f:\BbbR^r\cup\{\infty\}\to\Bbb R$ by setting $\tilde f(x)=0$
for $x\notin\overline{G}^{\infty}$.   Since $f$ is bounded, so is
$\tilde f$.
Let $\sequencen{\tilde h_n}$ be the sequence of 473E/478Jc, and for
$n\in\Bbb N$ set $f_n=(\tilde f\restr\BbbR^r)*\tilde h_n$.
Also, for $n\in\Bbb N$, set

\Centerline{$G_n
=\{y:y\in G$, $\|y\|<n$, $\rho(y,\BbbR^r\setminus G)>\Bover1{n+1}\}$}

\noindent (interpreting $\rho(y,\emptyset)$ as $\infty$ if $G=\BbbR^r$),
and let $\tau_n$ be the Brownian exit time from $G_n-x$.

\medskip

{\bf (b)} For $y\in G$, $f_n(y)\le f(y)$ for all sufficiently large
$n$ and $f(y)=\lim_{n\to\infty}f_n(y)$ (478Jb).   Also $f_n\restr G_n$ is
superharmonic (478Ja).   Each $f_n$ is smooth with bounded
derivatives of all orders (473De), and $(\nabla^2f_n)(y)\le 0$ for
$y\in G_n$ (478Ea).

If $m\ge n$,

\Centerline{$\Expn(f_m(x+X_{\tau_n}))
=f_m(x)+\Bover12\Expn(\int_0^{\tau_n}(\nabla^2f_m)(x+X_s)ds)
\le f_m(x)$}

\noindent (478K).   Consequently

$$\eqalign{\Expn(f(x+X_{\tau_n}))
&\le\liminf_{m\to\infty}\Expn(f_m(x+X_{\tau_n}))\cr
&\le\liminf_{m\to\infty}f_m(x)
=f(x).\cr}$$

{\bf (c)} For every $\omega\in\Omega$,
$\sequencen{\tau_n(\omega)}$ is a non-decreasing sequence with limit
$\tau(\omega)$.   \Prf\ Since
$\tau_n(\omega)\le\tau_{n+1}(\omega)\le\tau(\omega)$ for every $n$,
$t=\lim_{n\in\Bbb N}\tau_n(\omega)$ is defined in $[0,\infty]$.   If
$t=\infty$ then surely $t=\tau(\omega)$.   Otherwise,
$\omega(t)=\lim_{n\to\infty}\omega(\tau_n(\omega))\notin G-x$, so again
$t=\tau(\omega)$.\ \Qed

Consequently

\Centerline{$f(x+\omega(\tau(\omega)))
\le\liminf_{n\to\infty}f(x+\omega(\tau_n(\omega)))$}

\noindent for almost every $\omega$.   \Prf\ In the language of
478N, we can suppose that $\omega\in F_x(G)$, so that
$\omega(\tau(\omega))=\lim_{n\to\infty}\omega(\tau_n(\omega))$
in $\BbbR^r\cup\{\infty\}$, and
we can use the fact that $f$ is lower semi-continuous.\ \QeD\
So

\Centerline{$\Expn(f(x+X_{\tau}))
\le\liminf_{n\to\infty}\Expn(f(x+X_{\tau_n}))\le f(x)$}

\noindent as required.
}%end of proof of 478O

\leader{478P}{Harmonic measures (a)}
Let $A\subseteq\BbbR^r$ be an analytic
set and $x\in\BbbR^r$.   Let $\tau:\Omega\to[0,\infty]$ be the Brownian
hitting time to $A-x$\cmmnt{ (477I)}.   Then $\tau$ is
$\Sigma$-measurable, where $\Sigma$ is the domain of
$\mu_W$\cmmnt{ (455Ma)}.   Setting $H=\{\omega:\tau(\omega)<\infty\}$,
$X_{\tau}:H\to\BbbR^r$ is $\Sigma$-measurable.
\prooflet{\Prf\ By 4A3Wc,
$(\omega,t)\mapsto\omega(t)$ is
$\Sigma\tensorhat\Cal B(\coint{0,\infty})$-measurable, while
$\omega\mapsto(\omega,\tau(\omega))$ is
$(\Sigma,\Sigma\tensorhat\Cal B(\coint{0,\infty}))$-measurable.   So
$\omega\mapsto\omega(\tau(\omega))$ is $\Sigma$-measurable on $H$.\ \Qed}

Consider the function
$\omega\mapsto x+\omega(\tau(\omega)):H\to\BbbR^r$.
This induces a Radon image measure $\mu_x$ on $\BbbR^r$ defined by saying
that

\Centerline{$\mu_xF
=\mu_W\{\omega:\omega\in H$, $x+\omega(\tau(\omega))\in F\}
=\Pr(x+X_{\tau}\in F)$}

\noindent whenever this is defined.   \cmmnt{ Because
every $\omega\in\Omega$ is
continuous,} $X_{\tau}(\omega)\in\partial(A-x)$ for every $\omega\in H$,
and $\partial A$ is conegligible for $\mu_x$.   I will call $\mu_x$ the
{\bf harmonic measure for arrivals in $A$ from $x$}.   Of course
$\mu_x\BbbR^r$ is the Brownian hitting probability of $A$.

Note that if $F\subseteq\BbbR^r$ is closed and $x\in\BbbR^r\setminus F$,
then the Brownian hitting time to $F-x$ is the same as the Brownian hitting
time to $\partial F-x$,\cmmnt{ because all paths are continuous,}
so that the harmonic measure for arrivals in $F$
from $x$ coincides with the harmonic measure for arrivals in $\partial F$
from $x$.

\spheader 478Pb\cmmnt{ We now have an easy corollary of 478L.}   Let
$A\subseteq\BbbR^r$ be an analytic set, $x\in\BbbR^r$, and $\mu_x$
the harmonic measure for arrivals in $A$ from $x$.   If
$f:\BbbR^r\to[0,\infty]$ is a lower semi-continuous superharmonic function,
$f(x)\ge\int fd\mu_x$.   \prooflet{\Prf\ Let $\tau$ be the Brownian hitting
time to $A-x$, and $H=\{\omega:\tau(\omega)<\infty\}$.   Then

$$\eqalignno{\int fd\mu_x
&=\int_Hf(x+\omega(\tau(\omega)))d\mu_W\cr
\displaycause{because $\mu_x$ is the image measure of
the subspace measure $(\mu_W)_H$ under
$\omega\mapsto x+\omega(\tau(\omega))$}
&\le f(x)\cr}$$

\noindent by 478L.\ \Qed}

\spheader 478Pc We can re-interpret 478O in this language.   Let
$G\subseteq\BbbR^r$ be an open set with few wandering paths, and $x\in G$.
Let $\mu_x$ be the harmonic measure for arrivals in $\BbbR^r\setminus G$
from $x$.   In this case, taking $\tau$ to be the Brownian exit time from
$G-x$ and $H=\{\omega:\tau(\omega)<\infty\}$,\cmmnt{ we know that}
$\lim_{t\to\infty}\|\omega(t)\|=\infty$ for almost every
$\omega\in\Omega\setminus H$.   If
$f:\partial^{\infty}G\to[-\infty,\infty]$ is a function, then

$$\eqalignno{\Expn(f(x+X_{\tau}))
&=\int_Hf(x+X_{\tau}(\omega))\mu_W(d\omega)
   +f(\infty)\mu_W(\Omega\setminus H)\cr
\displaycause{counting $f(\infty)$ as zero if $G$ is bounded}
&=\int fd\mu_x+f(\infty)(1-\mu_x\BbbR^r)\cr}$$

\noindent if either integral is defined in
$[-\infty,\infty]$\cmmnt{ (235J\formerly{2{}35L})}.
In particular, if $f:\overline{G}^{\infty}\to\Bbb R$ is a bounded lower
semi-continuous function and $f\restr G$ is superharmonic, then
$f(x)\ge\int fd\mu_x+f(\infty)(1-\mu_x\BbbR^r)$\cmmnt{, by 478O}.
Similarly, if
$f:\overline{G}^{\infty}\to\Bbb R$ is continuous and $f\restr G$ is
harmonic, then $f(x)=\int fd\mu_x+f(\infty)(1-\mu_x\BbbR^r)$ for every
$x\in G$.

\spheader 478Pd Suppose that $\sequencen{A_n}$ is a non-decreasing sequence
of analytic subsets of $\BbbR^r$, with union $A$.   For $x\in\BbbR^r$, let
$\mu^{(n)}_x$, $\mu_x$ be the harmonic measures for arrivals in $A_n$, $A$
respectively from $x$.   Then $\mu_x$ is the limit
$\lim_{n\to\infty}\mu^{(n)}_x$ for the narrow topology on the space of
totally finite Radon measures on $\BbbR^r$\cmmnt{ (437Jd)}.
\prooflet{\Prf\ Let $\tau_n$, $\tau$ be the Brownian hitting times for
$A_n-x$, $A-x$ respectively.   Then $\sequencen{\tau_n}$ is a
non-increasing sequence with limit $\tau$.   Since every $\omega\in\Omega$
is continuous, $X_{\tau}(\omega)=\lim_{n\to\infty}X_{\tau_n}(\omega)$
whenever $\tau(\omega)<\infty$.   Set

\Centerline{$H_n=\{\omega:\tau_n(\omega)<\infty\}$ for each $n$,
$H=\{\omega:\tau(\omega)<\infty\}=\bigcup_{n\in\Bbb N}H_n$.}

\noindent If $f\in C_b(\BbbR^r)$, then
$f(x+X_{\tau}(\omega))=\lim_{n\to\infty}f(x+X_{\tau_n}(\omega))$ for every
$\omega\in H$, so

\Centerline{$\int fd\mu_x
=\int_Hf(x+X_{\tau})d\mu_W
=\lim_{n\to\infty}\int_{H_n}f(x+X_{\tau_n})d\mu_W
=\lim_{n\to\infty}\int fd\mu^{(n)}_x$.}

\noindent As $f$ is arbitrary, $\mu_x=\lim_{n\to\infty}\mu^{(n)}_x$
(437Kc).\ \Qed}

\leader{478Q}{}\cmmnt{ It is generally difficult to find formulae
describing
harmonic measures.   Theorem 478I, however, gives us a technique for
an important special case.

\wheader{478Q}{4}{2}{2}{60pt}

\noindent}{\bf Proposition} Let $S$ be the sphere $\partial B(y,\delta)$,
where $y\in\BbbR^r$ and $\delta>0$.   For $x\in\BbbR^r\setminus S$, let
$\zeta_x$ be the indefinite-integral measure over $\nu$ defined by the
function

$$\eqalign{z
&\mapsto\Bover{|\delta^2-\|x-y\|^2|}{r\beta_r\delta\|x-z\|^r}
  \text{ if }z\in S,\cr
&\mapsto 0\text{ if }z\in\BbbR^r\setminus S.\cr}$$

(a) If $x\in\interior B(y,\delta)$,
then the harmonic measure $\mu_x$ for arrivals in $S$ from $x$
is $\zeta_x$.

(b) In particular, the harmonic measure $\mu_y$ for arrivals in $S$ from
$y$ is $\Bover1{\nu S}\nu\LLcorner S$.

(c) Suppose that $r\ge 2$.   If $x\in\BbbR^r\setminus B(y,\delta)$,
then the harmonic measure $\mu_x$ for arrivals in $S$ from
$x$ is $\zeta_x$.\cmmnt{  In particular,}
$\mu_x\BbbR^r=\Bover{\delta^{r-2}}{\|x-y\|^{r-2}}$.

\proof{{\bf (a)} If $g\in C_b(\BbbR^r)$,
then 478Ib tells us that we have a continuous function $f_g$ which extends
$g$, is harmonic on $\BbbR^r\setminus S$ and is such that
$f_g(x)=\int g\,d\zeta_x$.   Now $G=\interior B(y,\delta)$ is bounded,
so it has
few wandering paths (478N) and the harmonic measure $\mu_x$ is defined,
with $f_g(x)=\int f_gd\mu_x$, by 478Pc.   But this means that

\Centerline{$\int g\,d\mu_x=\int f_gd\mu_x=f_g(x)=\int g\,d\zeta_x$.}

\noindent As $g$ is arbitrary, $\mu_x=\zeta_x$ (415I), as claimed.

\medskip

{\bf (b)} If $x=y$ then

\Centerline{$\Bover{\delta^2-\|x-y\|^2}{r\beta_r\delta\|x-z\|^r}
=\Bover{\delta^2}{r\beta_r\delta^{r+1}}
=\Bover1{\nu S}$}

\noindent if $z\in S$.   Since $\nu\LLcorner S$ is the indefinite-integral
measure over $\nu$ defined by $\chi S$ (234M\formerly{2{}34E}),
we have the result.

\medskip

{\bf (c)(i)} To see that $\zeta_x$ is the harmonic measure,
we can use the same argument as in (a), with decorations.
If $g\in C_b(\BbbR^r)$, then 478Ib gives us a bounded
continuous function $f_g$, harmonic on $H=\BbbR^r\setminus B(y,\delta)$,
such that $f_g$ agrees with $g$ on $S$, and $f_g(x)=\int g\,d\zeta_x$.
$S$ is conegligible for both $\mu_x$ and $\zeta_x$.

\medskip

\qquad\grheada\ If $r\ge 3$, then

\Centerline{$\limsup_{\|x\|\to\infty}|f_g(x)|
\le\Bover{\nu S}{r\beta_r\delta}
  \limsup_{\|x\|\to\infty}
    \Bover{\|x-y\|^2-\delta^2}{(\|x\|-\delta-\|y\|)^r}
=0$.}

\noindent So setting $f_g(\infty)=0$, $f_g:\overline{H}^{\infty}\to\Bbb R$
is continuous and bounded, and harmonic on $H$;  so that

\Centerline{$\int g\,d\zeta_x
=f_g(x)
=\int f_gd\mu_x+f_g(\infty)(1-\mu_x\BbbR^r)
=\int gd\mu_x$.}

\noindent As in (a), we conclude that $\zeta_x=\mu_x$.

\medskip

\qquad\grheadb\ If $r=2$, then by 478Mb we see that almost every
$\omega\in\Omega$ takes values in $B(y,\delta)-x$;  so $\mu_x\BbbR^r=1$.
Set $\underline{f}\vthsp_g(x)=f_g(x)$ for $x\in\BbbR^2$,
$\underline{f}\vthsp_g(\infty)=\liminf_{\|x\|\to\infty}f_g(x)$.   Then
$\underline{f}\vthsp_g$ is lower semi-continuous on
$\overline{H}^{\infty}$ and harmonic on $H$, so

\Centerline{$\int g\,d\zeta_x
=f_g(x)
=\underline{f}\vthsp_g(x)
\ge\int\underline{f}\vthsp_gd\mu_x
   +\underline{f}\vthsp_g(\infty)(1-\mu_x\BbbR^r)
=\int g\,d\mu_x$.}

\noindent Applying the same argument to $-g$, we see that
$\int g\,d\zeta_x\le\int g\,d\mu_x$, so in fact the integrals are equal,
and we have the result in this case also.

\medskip

\quad{\bf (ii)} Now

$$\eqalignno{\mu_x\BbbR^r
&=\zeta_xS
=\int_S\Bover{\|x-y\|^2-\delta^2}{r\beta_r\delta\|x-z\|^r}\nu(dz)\cr
&=\int_{\partial B(\tbf{0},\delta)}
  \Bover{\|x-y\|^2-\delta^2}{r\beta_r\delta\|x-y-z\|^r}\nu(dz)
=\Bover{\nu(\partial B(\tbf{0},\delta))}{r\beta_r\delta\|x-y\|^{r-2}}\cr
\displaycause{478Gb}
&=\Bover{\delta^{r-2}}{\|x-y\|^{r-2}}.\cr}$$
}%end of proof of 478Q

\leader{478R}{Theorem} Let $A$, $B\subseteq\BbbR^r$ be analytic sets
with $A\subseteq B$.
For $x\in\BbbR^r$, let $\mu^{(A)}_x$, $\mu^{(B)}_x$ be the harmonic
measures for arrivals in $A$, $B$ respectively from $x$.   Then, for any
$x\in\BbbR^r$,
$\family{y}{\BbbR^r}{\mu^{(A)}_y}$ is a disintegration of $\mu^{(A)}_x$
over $\mu^{(B)}_x$.

\proof{{\bf (a)} Let $\tau$ be the
Brownian hitting time to $B-x$, and $\tau'$ the hitting time
to $A-x$; then $\tau(\omega)\le\tau'(\omega)$ for every $\omega$.
If $\tau(\omega)<\infty$, set $f(\omega)=x+\omega(\tau(\omega))$, so that
$\mu_x^{(A)}$ is the image measure $(\mu_W)_Hf^{-1}$, where
$H=\{\omega:\tau(\omega)<\infty\}$.
Define $\phi_{\tau}:\Omega\times\Omega\to\Omega$ as in
477G, so that

$$\eqalign{\phi_{\tau}(\omega,\omega')(t)
&=\omega(t)\text{ if }t\le\tau(\omega),\cr
&=\omega(\tau(\omega))+\omega'(t-\tau(\omega))
   \text{ if }t\ge\tau(\omega).\cr}$$

\noindent Then we have $x+\phi_{\tau}(\omega,\omega')(t)\in A$ iff
$t\ge\tau(\omega)$ and
$f(\omega)+\omega'(t-\tau(\omega))\in A$;   so if we write
$\sigma_{\omega}$ for the Brownian hitting time to $A-f(\omega)$ when
$\omega\in H$,
$\tau'(\phi_{\tau}(\omega,\omega'))=\tau(\omega)+\sigma_{\omega}(\omega')$.

\medskip

{\bf (b)} Now suppose that $E\subseteq\BbbR^r$ is a Borel set.   Then

$$\eqalignno{\mu_x^{(A)}(E)
&=\mu_W\{\omega:\tau'(\omega)<\infty,
  \,x+\omega(\tau'(\omega))\in E\}\cr
&=\mu_W^2\{(\omega,\omega'):
  \tau'(\phi_{\tau}(\omega,\omega'))<\infty,\,
  x+\phi_{\tau}(\omega,\omega')(\tau'(\phi_{\tau}(\omega,\omega')))\in E\}
  \cr
&=\mu_W^2\{(\omega,\omega'):\tau(\omega)<\infty,\,
  \sigma_{\omega}(\omega')<\infty,\,
  f(\omega)+\omega'(\sigma_{\omega}(\omega'))\in E\}\cr
&=\int_H\mu_W\{\omega':\sigma_{\omega}(\omega')<\infty,\,
  f(\omega)+\omega'(\sigma_{\omega}(\omega'))\in E\}\mu_W(d\omega)\cr
&=\int_H\mu^{(A)}_{f(\omega)}(E)\mu_W(d\omega)
=\int\mu^{(A)}_y(E)\mu_x^{(B)}(dy).
}$$

\noindent The definition in 452E demands that this formula should be valid
whenever $E$ is measured by $\mu_x^{(A)}$;  but in general there will be
Borel sets $E'$, $E''$ such that $E'\subseteq E\subseteq E''$ and
$\mu^{(A)}_x(E')=\mu^{(A)}_x(E)=\mu^{(A)}_x(E'')$,
in which case we must have
$\mu^{(A)}_y(E')=\mu^{(A)}_y(E)=\mu^{(A)}_y(E'')$ for $\mu^{(B)}_x$-almost
every $y$, and again
$\mu_x^{(A)}(E)=\int\mu^{(A)}_y(E)\mu^{(B)}_x(dy)$.
}%end of proof of 478R

\leader{478S}{Corollary} Let $A\subseteq\BbbR^r$ be an analytic set,
and $f:\partial A\to\Bbb R$ a bounded universally measurable function.
For $x\in\BbbR^r\setminus\overline{A}$ set $g(x)=\int fd\mu_x$, where
$\mu_x$ is the harmonic measure for arrivals in $A$ from
$x$.   Then $g$ is harmonic.

\proof{ Suppose that $\delta>0$ is such that
$B(x,\delta)\cap\overline{A}=\emptyset$, and set
$S=\partial B(x,\delta)=\partial(\BbbR^r\setminus\interior B(x,\delta))$.
Then the harmonic measure for arrivals in
$\BbbR^r\setminus\interior B(x,\delta)$ from $x$ is
$\Bover1{\nu S}\nu\LLcorner S$ (478Qb).   So

$$\eqalignno{g(x)
&=\int fd\mu_x
=\int_S\Bover1{\nu S}\int fd\mu_y\nu(dy)\cr
\displaycause{478R, 452F}
&=\Bover1{\nu S}\int_Sg(y)\nu(dy).\cr}$$

\noindent As $x$ and $\delta$ are arbitrary, $g$ is harmonic.
}%end of proof of 478S

\leader{478T}{Corollary} Let $A\subseteq\BbbR^r$ be an analytic set,
and for $x\in\BbbR^r$ let $\mu_x$ be the harmonic measure for arrivals in
$A$ from $x$.   Then $x\mapsto\mu_x$ is continuous on
$\BbbR^r\setminus\overline{A}$ for the
total variation metric on the set of totally finite
Radon measures on $\BbbR^r$\cmmnt{ (definition:  437Qa)}.

\proof{ Take any $y\in\BbbR^r\setminus\overline{A}$.   Let $\delta>0$ be
such that $B(y,\delta)\cap A=\emptyset$, and set
$S=\partial B(y,\delta)$.   For $x\in\interior B(y,\delta)$,
let $\zeta_x$ be the harmonic measure for arrivals in
$\BbbR^r\setminus B(y,\delta)$ from $x$, so that $\zeta_x$ is the
indefinite-integral measure over $\nu$ defined by the function

$$\eqalign{z
&\mapsto\Bover{\delta^2-\|x-y\|^2}{r\beta_r\delta\|x-z\|^r}
  \text{ if }z\in S,\cr
&\mapsto 0\text{ if }z\in\BbbR^r\setminus S\cr}$$

\noindent (478Qa).   Then, for any $x\in\interior B(y,\delta)$,
$\family{z}{\BbbR^r}{\mu_z}$ is a disintegration of $\mu_x$
over $\zeta_x$.   So if $E\subseteq\BbbR^r$ is a Borel set,

\Centerline{$\mu_xE
=\int\mu_z(E)\,\zeta_x(dz)
=\Bover1{r\beta_r\delta}\int_S\mu_z(E)\,
   \Bover{\delta^2-\|x-y\|^2}{\|x-z\|^r}\nu(dz)$.}

\noindent But this means that

\Centerline{$|\mu_x(E)-\mu_y(E)|
\le\Bover{\nu S}{r\beta_r\delta}
  \sup_{z\in S}\bigl|\Bover{\delta^2-\|x-y\|^2}{\|x-z\|^r}
      -\Bover{\delta^2}{\|y-z\|^r}\bigr|$;}

\noindent as $E$ is arbitrary, the distance from $\mu_x$ to
$\mu_y$ is at most

\Centerline{$\delta^{r-2}
  \sup_{z\in S}\bigl|\Bover{\delta^2-\|x-y\|^2}{\|x-z\|^r}
      -\Bover{\delta^2}{\|y-z\|^r}\bigr|$,}

\noindent which is small if $x$ is close to $y$.
}%end of proof of 478T

\leader{478U}{}\cmmnt{ A variation on the technique of 478R enables us to
say something about Brownian paths starting from a point in the essential
closure of a set.

\medskip

\noindent}{\bf Proposition} Suppose that $A\subseteq\BbbR^r$ and that
$0$ belongs to the essential closure\cmmnt{ $\clstar A$} of $A$\cmmnt{ as
defined in 475B}.   Then the outer Brownian hitting probability
$\hp^*(A)$ of $A$\cmmnt{ (477Ia)} is $1$.

\proof{{\bf (a)} Take that $\alpha\in\ooint{0,1}$ such that
$\Bover{1-\alpha^2}{(1+\alpha)^r}=\Bover12$.
Suppose that $E\subseteq\BbbR^r$ is analytic, and that
$0<\delta_0<\ldots<\delta_n$ are such that
$\delta_i\le\alpha\delta_{i+1}$ for $i<n$.   For $i\le n$, let $\tau_i$
be the Brownian hitting time to $S_i=\partial B(\tbf{0},\delta_i)$.   Then

\Centerline{$\mu_W\{\omega:
   \omega(\tau_i(\omega))\notin E$ for every $i\le n\}
\le\prod_{i=0}^n(1-\Bover{\nu(E\cap S_i)}{2\nu S_i})$.}

\noindent\Prf\ Induce on $n$.   If $n=0$, then

$$\eqalignno{\mu_W\{\omega:\omega(\tau_0(\omega))\notin E\}
&=1-\mu_0^{(S_i)}(E)\cr
\displaycause{where $\mu_0^{(S_i)}$ is the harmonic measure for arrivals in
$S_i$ from $0$}
&=1-\Bover{\nu(E\cap S_i)}{\nu S_i}
\le 1-\Bover{\nu(E\cap S_i)}{2\nu S_i}\cr}$$

\noindent(478Qb).   For the inductive step to $n+1\ge 1$, let
$\phi:\Omega\times\Omega\to\Omega$ be the \imp\ function
corresponding to the stopping time $\tau_n$ as in 477G;  when
$\tau_n(\omega)$ is finite, set
$y(\omega)=\omega(\tau_n(\omega))\in S_n$.   Since
$\tau_i(\omega)<\tau_{i+1}(\omega)$ whenever $i\le n$ and
$\tau_i(\omega)$ is finite,

\Centerline{$\tau_i(\phi(\omega,\omega'))=\tau_i(\omega)$}

\noindent for $i\le n$ and $\omega$, $\omega'\in\Omega$.   As for
$\tau_{n+1}(\phi(\omega,\omega'))$, this is infinite if
$\tau_n(\omega)=\infty$, and otherwise is
$\sigma_{y(\omega)}(\omega')$, where $\sigma_y$
is the Brownian hitting time of $S_{n+1}-y$.   Now if $y\in S_n$, then

$$\eqalignno{\mu_y^{(S_{n+1})}(E)
&=\int_{E\cap S_{n+1}}
  \Bover{\delta_{n+1}^2-\delta_n^2}{r\beta_r\delta_{n+1}\|x-y\|^r}
  \nu(dx)\cr
\displaycause{478Qa}
&\ge\Bover{\delta_{n+1}^2-\delta_n^2}
   {r\beta_r\delta_{n+1}(\delta_{n+1}+\delta_n)^r}
   \nu(E\cap S_{n+1})\cr
&\ge\Bover{1-\alpha^2}
   {r\beta_r\delta_{n+1}^{r-1}(1+\alpha)^r}
   \nu(E\cap S_{n+1})
=\Bover{\nu(E\cap S_{n+1})}{2\nu S_{n+1}}.\cr}$$

\noindent Consequently

$$\eqalignno{
&\mu_W\{\omega:\omega(\tau_i(\omega))\notin E\text{ for every }i\le n+1\}
  \cr
&\mskip50mu=(\mu_W\times\mu_W)\{(\omega,\omega'):
  \phi(\omega,\omega')(\tau_i(\phi(\omega,\omega')))\notin E
  \text{ for every }i\le n+1\}\cr
&\mskip50mu=(\mu_W\times\mu_W)\{(\omega,\omega'):
  \omega(\tau_i(\omega))\notin E
  \text{ for every }i\le n,
  \,\omega'(\sigma_{y(\omega)}(\omega'))\notin E\}\cr
&\mskip50mu
=\int_V\mu_W\{\omega':\omega'(\sigma_{y(\omega)}(\omega'))\notin E\}
   \mu_W(d\omega)\cr
\displaycause{setting $V=\{\omega:\omega(\tau_i(\omega))\notin E$ for every
$i\le n\}$}
&\mskip50mu\le\mu_WV\cdot\sup_{y\in S_n}(1-\mu_y^{(S_{n+1})}E)\cr
&\mskip50mu\le\mu_WV\cdot(1-\Bover{\nu(E\cap S_{n+1})}{2\nu S_{n+1}})
\le\prod_{i=0}^{n+1}(1-\Bover{\nu(E\cap S_i)}{2\nu S_i})\cr}$$

\noindent by the inductive hypothesis.   So the induction continues.\ \Qed

\medskip

{\bf (b)} In particular, under the conditions of (a),
$\hp(E)\ge 1-\prod_{i=0}^n(1-\Bover{\nu(E\cap S_i)}{2\nu S_i})$.
Now suppose that $A\subseteq\BbbR^r$ and that $0\in\clstar A$.
Let $E\supseteq A$ be an analytic set such that $\hp(E)=\hp^*(A)$
(477Id).   Then $0\in\clstar E$;  set
$\gamma=\Bover13\limsup_{\delta\downarrow 0}
   \Bover{\mu(E\cap B(\tbf{0},\delta))}{\mu B(\tbf{0},\delta)}>0$.
For any $\delta>0$, there is a $\delta'\in\ocint{0,\delta}$ such that
$\Bover{\nu(E\cap\partial B(\tbf{0},\delta'))}{\nu\partial B(\tbf{0},\delta')}
\ge 2\gamma$.   \Prf\ Let $\beta\in\ocint{0,\delta}$ be such that
$\mu(E\cap B(\tbf{0},\beta))\ge 2\gamma\mu B(\tbf{0},\beta)$.   Then

\Centerline{$\int_0^{\beta}\nu(E\cap\partial B(\tbf{0},t))dt
\ge 2\gamma\int_0^{\beta}\nu\partial B(\tbf{0},t)dt$,}

\noindent so there must be a $\delta'\in\ocint{0,\beta}$ such that
$\nu(E\cap\partial B(\tbf{0},\delta'))\ge 2\gamma\nu\partial B(\tbf{0},\delta')$.\ \Qed

We can therefore find, for any $n\in\Bbb N$, $0<\delta_0<\ldots<\delta_n$
such that $\delta_i\le\alpha\delta_{i+1}$ for every $i<n$ (where $\alpha$
is chosen as in (a) above) and
$\nu(E\cap\partial B(\tbf{0},\delta_i))\ge 2\gamma\nu\partial B(\tbf{0},\delta_i)$ for
every $i$.   As noted at the beginning of this part of the proof, it
follows that $\hp(E)\ge 1-(1-\gamma)^{n+1}$.   As this is true for every
$n\in\Bbb N$, $\hp(E)=1$, so $\hp^*(A)=1$, as claimed.
}%end of proof of 478U

\leader{*478V}{Theorem} (a)
Let $G\subseteq\BbbR^r$ be an open set with few wandering paths and
$f:\overline{G}^{\infty}\to\Bbb R$ a continuous function such that
$f\restr G$ is harmonic.   For $x\in\BbbR^r$ let
 $\tau_x:\Omega\to[0,\infty]$ be the Brownian exit time from
$G-x$.   Set

$$\eqalign{g_{\tau_x}(\omega)
&=f(x+\omega(\tau_x(\omega)))\text{ if }\tau_x(\omega)<\infty,\cr
&=f(\infty)\text{ if }\lim_{t\to\infty}\|\omega(t)\|=\infty
\text{ and }\tau_x(\omega)=\infty.\cr}$$

\noindent Then $f(x)=\Expn(g_{\tau_x})$.

(b) Now suppose that $\sigma$ is a stopping time
adapted to $\langle\Sigma_t\rangle_{t\ge 0}$ such that
$\sigma(\omega)\le\tau_x(\omega)$ for every $\omega$.   Set

$$\eqalign{g_{\sigma}(\omega)
&=g_{\tau_x}(\omega)\text{ if }\sigma(\omega)=\tau_x(\omega)=\infty,\cr
&=f(x+\omega(\sigma(\omega)))\text{ otherwise}.\cr}$$

\noindent As in 455Lc, set
$\Sigma_{\sigma}=\{E:E\in\dom\mu_W$,
$E\cap\{\omega:\sigma(\omega)\le t\}\in\Sigma_t$ for every $t\ge 0\}$.
Then $g_{\sigma}$ is a conditional expectation of
$g_{\tau_x}$ on $\Sigma_{\sigma}$.

\proof{{\bf (a)(i)} Of course if $x\notin G$ then $\tau_x(\omega)=0$
and $g_{\tau_x}(\omega)=f(x)$ for
every $\omega$ and the result is trivial.   So we can suppose that
$x\in G$.   Note next that if there is any $\omega$ such that
$\lim_{t\to\infty}\|\omega(t)\|=\infty$ and $\tau_x(\omega)=\infty$, then
$G$ must be unbounded, so $f(\infty)$ will be defined.   Because
$G$ has few wandering paths, $g_{\tau_x}$ is defined almost everywhere.

\medskip

\quad{\bf (ii)} Let $m\in\Bbb N$ be such that
$\rho(x,\BbbR^r\setminus G)>\bover1{m+1}$ and $\|x\|<m$;
for $n\ge m$, set
$G_n=\{y:\|y\|<n$, $\rho(y,\BbbR^r\setminus G)>\bover1{n+1}\}$,
let $\tau_{xn}'$ be the
Brownian exit time from $G_n-x$ and set
$\tau_{xn}(\omega)=\min(n,\tau_{xn}'(\omega))$ for every $\omega$.
Note that by 477I(c-i), $x+\omega(\tau_{xn}(\omega))\in\overline{G}_n$
for every $\omega$.

By 477I(c-iii) and 455L(c-v), $\tau_{xn}$ is a stopping time adapted to
$\langle\Sigma_t\rangle_{t\ge 0}$.   Let $\sequencen{\tilde h_n}$ be the
sequence of 473E, and for $n\ge m$ set $f_n=\tilde f*\tilde h_n$, where
$\tilde f$ is the extension of $f\restr\overline{G}$ to
$\BbbR^r$ which takes the value $0$
on $\BbbR^r\setminus\overline{G}$.   Then $f_n$ is smooth and bounded.
By 478Jb
(applied in turn to the functions $\tilde f+M\chi\BbbR^r$
and $-\tilde f+M\chi\BbbR^r$ where $M=\sup_{y\in G}|f(y)|$,
which of course are both superharmonic on $G$),
$f_n$ agrees with $f$ on $G_n$, so that $f_n\restr G_n$ is harmonic and
$\nabla^2f_n$ is zero on $G_n$ (478Ec).
Also, because both $f_n$ and $f$ are continuous, they agree on
$\overline{G}_n$ and
$f(x+\omega(\tau_{xn}(\omega)))
=f_n(x+\omega(\tau_{xn}(\omega)))$
for every $\omega$.

If $n\ge m$, $\omega\in\Omega$ and $0\le s<\tau_{xn}(\omega)$, then
$x+\omega(s)\in G_n$ so $(\nabla^2f_n)(x+\omega(s))=0$.   Dynkin's formula
(478K), applied to the function $y\mapsto f_n(x+y)$,
therefore tells us that
$f(x)=f_n(x)
=\int f_n(x+\omega(\tau_{xn}(\omega)))\mu_W(d\omega)$.

\medskip

\quad{\bf (iii)} If $\omega\in\Omega$ and $t<\tau_x(\omega)$,
then the compact set
$x+\omega[\,[0,t]\,]$ is included in the open set $G$ and there is an
$n\ge\max(m,t)$ such that it is included in $G_n$.   So
$\lim_{n\to\infty}\tau_{xn}(\omega)=\tau_x(\omega)$ and, because $f$ is
continuous on $\overline{G}^{\infty}$,

\Centerline{$g_{\tau_x}(\omega)
=\lim_{n\to\infty}f(x+\omega(\tau_{xn}(\omega)))
=\lim_{n\to\infty}f_n(x+\omega(\tau_{xn}(\omega)))$}

\noindent
for almost every $\omega$.   Since
$\|f_n\|_{\infty}\le\|f\|_{\infty}<\infty$ for every $n$,
Lebesgue's Dominated Convergence Theorem tells us that

\Centerline{$\Expn(g_{\tau_x})
=\lim_{n\to\infty}\int f_n(x+\omega(\tau_{xn}(\omega)))
  \mu_W(d\omega)
=f(x)$,}

\noindent as required.

\medskip

{\bf (b)(i)} If $\omega_0$, $\omega_1\in\Omega$, $\sigma(\omega_0)=t$ and
$\omega_1\restr[0,t]=\omega_0\restr[0,t]$, then
$\sigma(\omega_1)=t$.
\Prf\ The set $\{\omega:\sigma(\omega)\le t\}$ belongs
to $\Sigma_t$;  as it contains $\omega_0$, it contains $\omega_1$, and
$\sigma(\omega_1)\le t$.   But now $\omega_0$ agrees with $\omega_1$ on
$[0,\sigma(\omega_1)]$, so $\sigma(\omega_1)\ge\sigma(\omega)=t$.\ \Qed

If $H\in\Sigma_{\sigma}$ then $\omega_0\in H$ iff $\omega_1\in H$.
\Prf\ For every $t\ge 0$,
$H\cap\{\omega:\sigma(\omega)\le t\}$ belongs to
$\Sigma_t$, so contains $\omega_0$ iff it contains $\omega_1$.\ \Qed

\medskip

\quad{\bf (ii)} Of course
$E_{\infty}=\{\omega:\sigma(\omega)=\infty\}$ belongs to $\Sigma_{\sigma}$,
because it has empty intersection with every set
$\{\omega:\sigma(\omega)\le t\}$.

\leaveitout{Also
$E_1=\{\omega:\tau(\omega)=\sigma(\omega)<\infty\}$ belongs to
$\Sigma_{\sigma}$.   \Prf\ For any $t\ge 0$,

$$\eqalign{E_1\cap\{\omega:\sigma(\omega)\le t\}
&=\{\omega:\sigma(\omega)\le t\}\cap\{\omega:\tau(\omega)\le t\}\cr
&\mskip50mu
  \setminus\bigcup_{q\in[0,t]\cap\Bbb Q}
      (\{\omega:\sigma(\omega)\le q<\tau(\omega)\}
         \cup\{\omega:\tau(\omega)\le q<\sigma(\omega)\})\cr
&\in\Sigma_t.  \text{ \Qed}\cr}$$

\noindent So

\Centerline{$E_0=\{\omega:\sigma(\omega)=0<\tau(\omega)\}
=\{\omega:\sigma(\omega)\le 0\}\setminus E_1$}

\noindent belongs to $\Sigma_{\sigma}$.
}

\medskip

\quad{\bf (iii)} $g_{\sigma}$ is $\Sigma_{\sigma}$-measurable.   \Prf\
For $n\in\Bbb N$, $\omega\in\Omega$ and $t\ge 0$, set

$$\eqalign{h_n(t,\omega)
&=f(\omega(2^{-n}\lfloor 2^nt\rfloor))
  \text{ if }\omega(2^{-n}\lfloor 2^nt\rfloor)\in G,\cr
&=0\text{ otherwise}.\cr}$$

\noindent Then $h_n$ is
$(\Cal B(\coint{0,\infty})\times\Sigma)$-measurable,
so if we set $h(t,\omega)=\lim_{n\to\infty}h_n(t,\omega)$ when this is
defined, $h$ also will be
$(\Cal B(\coint{0,\infty})\times\Sigma)$-measurable, and
$\omega\mapsto h(\sigma(\omega),\omega)$ is $\Sigma$-measurable.
Now, because
$\sigma\le\tau$, $g_{\sigma}(\omega)=h(\sigma(\omega),\omega)$ for almost
every $\omega\in\Omega\setminus E_{\infty}$;
because $\mu_W$ is complete,
$g_{\sigma}$ is $\Sigma$-measurable.   But now observe that if $t\ge 0$ and
$\alpha\in\Bbb R$,
$\{\omega:\sigma(\omega)\le t$, $g_{\sigma}(\omega)\ge\alpha\}$ belongs to
$\Sigma$ and is determined by coordinates less than or equal to $t$, so
belongs to $\Sigma_t$.   As $t$ is arbitrary,
$\{\omega:g_{\sigma}(\omega)\ge\alpha\}\in\Sigma_{\sigma}$;  as
$\alpha$ is arbitrary, $g_{\sigma}$ is $\Sigma_{\sigma}$-measurable.\ \Qed

\medskip
	
\quad{\bf (iv)} As in 477G, define
$\phi_{\sigma}:\Omega\times\Omega\to\Omega$ by saying that

$$\eqalign{\phi_{\sigma}(\omega,\omega')(t)
&=\omega(t)\text{ if }t\le\sigma(\omega),\cr
&=\omega(\sigma(\omega))+\omega'(t-\sigma(\omega))
   \text{ if }t\ge\sigma(\omega).\cr}$$

\noindent Then 477G tells us that $\phi_{\sigma}$ is \imp.

\medskip

\quad{\bf (v)} $\tau_x(\phi_{\sigma}(\omega,\omega'))
=\sigma(\omega)+\tau_{x+\omega(\sigma(\omega))}(\omega')$
for all $\omega$, $\omega'\in\Omega$.   \Prf\ If
$\sigma(\omega)=\infty$ then $\phi_{\sigma}(\omega,\omega')=\omega$
and

\Centerline{$\tau_x(\phi_{\sigma}(\omega,\omega'))
=\tau_x(\omega)=\sigma(\omega)$.}

\noindent If $\sigma(\omega)=\tau_x(\omega)$ is finite then
$\omega(\sigma(\omega))\notin G-x$ and
$\phi_{\sigma}(\omega,\omega')\restr[0,\tau_x(\omega)]
=\omega\restr[0,\tau_x(\omega)]$, so

\Centerline{$\tau_x(\phi_{\sigma}(\omega,\omega'))
=\tau_x(\omega)=\sigma(\omega)
=\sigma(\omega)+\tau_{x+\omega(\sigma(\omega))}(\omega')$.}

\noindent If $\sigma(\omega)<\tau_x(\omega)$ then
$\omega(t)=\phi_{\sigma}(\omega,\omega')(t)$ belongs to $G-x$ for every
$t\le\sigma(\omega))$ and

$$\eqalign{\tau_x(\phi_{\sigma}(\omega,\omega'))
&=\inf\{t:t\ge\sigma(\omega),\,
   \omega(\sigma(\omega))+\omega'(t-\sigma(\omega))\notin G-x\}\cr
&=\sigma(\omega)+\inf\{t:\omega'(t)\notin G-x-\omega(\sigma(\omega))\}
=\sigma(\omega)+\tau_{x+\omega(\sigma(\omega))}(\omega').
\text{ \Qed}\cr}$$

\noindent Consequently, if $\omega\in\Omega$, $\sigma(\omega)<\infty$ and
$y=\omega(\sigma(\omega))$,

\Centerline{$\phi_{\sigma}(\omega,\omega')
  (\tau_x(\phi_{\sigma}(\omega,\omega')))
=\phi_{\sigma}(\omega,\omega')
  (\sigma(\omega)+\tau_{x+y}(\omega'))
=y+\omega'(\tau_{x+y}(\omega'))$}

\noindent whenever either $\tau_x(\phi_{\sigma}(\omega,\omega'))$ or
$\tau_{x+y}(\omega')$ is finite,

\Centerline{$g_{\tau_x}(\phi_{\sigma}(\omega,\omega'))
=f(x+\phi_{\sigma}(\omega,\omega')
  (\tau_x(\phi_{\sigma}(\omega,\omega'))))
=f(x+y+\omega'(\tau_{x+y}(\omega')))$}

\noindent for almost every $\omega'$.

\medskip

\quad{\bf (vi)} If $H\in\Sigma_{\sigma}$ then
$\phi_{\sigma}^{-1}[H]=H\times\Omega$.   \Prf\ If $\omega$,
$\omega'\in\Omega$ then
$\phi_{\sigma}(\omega,\omega')\restr[0,\sigma(\omega)]
=\omega\restr[0,\sigma(\omega)]$, so by (i) above
$\phi_{\sigma}(\omega,\omega')\in H$ iff $\omega\in H$.\ \Qed

If $H\cap E_{\infty}=\emptyset$, we now have

$$\eqalignno{\int_Hg_{\tau_x}
&=\int_{\phi_{\sigma}^{-1}[H]}g_{\tau_x}(\phi_{\sigma}(\omega,\omega'))
  d(\omega,\omega')\cr
&=\int_H\int f(x+\omega(\sigma(\omega))
   +\omega'(\tau_{x+\omega(\sigma(\omega))}(\omega)))
  d\omega'd\omega\cr
&=\int_Hf(x+\omega(\sigma(\omega)))d\omega\cr
\displaycause{by (a) above}
&=\int_Hg_{\sigma}.\cr}$$

\noindent Of course we also have $\int_Hg_{\tau_x}=\int_Hg_{\sigma}$ if
$H\subseteq E_{\infty}$.   So $\int_Hg_{\sigma}=\int_Hg_{\tau_x}$ for every
$H\in\Sigma_{\sigma}$, and
$g_{\sigma}$ is a conditional expectation of
$g_{\tau_x}$ on $\Sigma_{\sigma}$.
}%end of proof of 478V

\leaveitout{\medskip

\noindent{\bf Remarks} Will this now cover 478R?

Anyway we can do quite a bit better;  e.g., assuming only that
$f:G\to\Bbb R$ is harmonic, but that there is a bound on
$f(\omega(t))$ for $\omega\in\Omega$, $t<\tau(\omega)$.
}

\exercises{\leader{478X}{Basic exercises (a)}
%\spheader 478Xa
Let $G\subseteq\BbbR^r$ be an open set, and
$\sequencen{f_n}$ a sequence of superharmonic functions from $G$ to
$[0,\infty]$.   Show that
$\liminf_{n\to\infty}f_n$ is superharmonic.
%478B

\spheader 478Xb Let $G\subseteq\BbbR^r$ be an open set, and $f:G\to\Bbb R$
a continuous harmonic function.   Show that $f$ is smooth.   \Hint{put 478I
and 478D together.}
%478I
%won't "Lebesgue measurable" do in place of "continuous"?

\spheader 478Xc Let $G\subseteq\BbbR^r$ be an open set, and
$f:G\to[0,\infty]$ a lower semi-continuous superharmonic function.
Show that there is are sequences $\sequencen{G_n}$, $\sequencen{f_n}$ such
that (i) $\sequencen{G_n}$ is a non-decreasing sequence of open sets with
union $G$ (ii) for each $n\in\Bbb N$,
$f_n:G_n\to\coint{0,\infty}$ is a bounded smooth superharmonic
function and $f_n\le f\restr G_n$ (iii) $f(x)=\lim_{n\to\infty}f_n(x)$ for
every $x\in G$.
%478J

\sqheader 478Xd Let $\langle X_t\rangle_{t\ge 0}$ be Brownian motion in
$\BbbR^r$, and $\delta>0$.   Let $\tau$ be the Brownian hitting time to
$\{x:\|x\|\ge\delta\}$.   Show that $\Expn(\tau)=\Bover{\delta^2}{r}$.
\Hint{in 478K, take $f(x)=\|x\|^2$.}
%478K

\spheader 478Xe Show that
$\hp^*:\Cal P\BbbR^r\to[0,1]$ is an outer regular Choquet
capacity (definition:  432J) iff $r\ge 3$.
\Hint{if $r\ge 3$, $\mu_W$ is inner regular with respect to
$\{K:K\subseteq\Omega$,
$\lim_{t\to\infty}\inf_{\omega\in K}\|\omega(t)\|=\infty\}$.}
%477I %478M

\spheader 478Xf Show that an open subset of $\Bbb R$ has few wandering
paths iff it is not $\Bbb R$ itself.
%478N

\spheader 478Xg Suppose $r=2$.  (i)
Show that if $x\in\BbbR^2\setminus\{0\}$, then
the Brownian exit time from $\BbbR^2\setminus\{x\}$ is infinite a.e.
\Hint{use the
method of part (b-ii) of the proof of 478M to show that if $R>\|x\|$ and
$\delta>0$ is small enough then most sample paths meet $\partial B(x,R)$
before they meet $B(x,\delta)$.}
(ii) Show that if $G\subseteq\BbbR^2$ is an open set
with countable complement then $G$ does not have few wandering paths.
%478N

\spheader 478Xh Suppose that $r\ge 3$.   Let $\sequencen{A_n}$ be a
non-increasing sequence of analytic sets in $\BbbR^r$ such that $A_0$ is
bounded and $\bigcap_{n\in\Bbb N}\bar A_n=\bigcap_{n\in\Bbb N}A_n$,
and $x\in\BbbR^r$.   Let $\mu_x^{(n)}$, $\mu_x$ be the
harmonic measures for arrivals in $A_n$, $\bigcap_{m\in\Bbb N}A_m$
from $x$.   Show that
$\mu_x=\lim_{n\to\infty}\mu^{(n)}_x$ for the narrow topology on the set of
totally finite Radon measures on $\BbbR^r$.   \Hint{478Xe, 478Pd.}
%478Xe 478M 478P

\spheader 478Xi Let $A\subseteq\Bbb R$ be an analytic set, $x\in\Bbb R$ and
$\mu_x$ the harmonic measure for arrivals in $A$ from $x$.   For
$y\in\Bbb R$ let $\delta_y$ be the Dirac measure on $\Bbb R$ concentrated
at $y$.   Show that (i) if $A$ is empty, then $\mu_x$ is the zero measure;
(ii) if $A\ne\emptyset$ but $A\cap\coint{x,\infty}=\emptyset$ then
$\mu_x=\delta_{\sup A}$;
(iii) if $A\ne\emptyset$ but $A\cap\ocint{-\infty,x}=\emptyset$ then
$\mu_x=\delta_{\inf A}$;  (iv) if $A$ meets both $\ocint{-\infty,x}$ and
$\coint{x,\infty}$, and $y=\sup(A\cap\ocint{-\infty,x})$,
$z=\inf(A\cap\coint{x,\infty})$, then $\mu_x=\delta_x$ if $y=z=x$, and
otherwise $\mu_x=\Bover{z-x}{z-y}\delta_y+\Bover{x-y}{z-y}\delta_z$.
%478P

\spheader 478Xj Prove 478Qb by a symmetry argument not involving the
calculations of 478I.
%478Q

\sqheader 478Xk Let $G\subseteq\BbbR^r$ be an open set, and for $x\in G$
let $\mu_x$ be the
harmonic measure for arrivals in $\BbbR^r\setminus G$ from $x$.
Show that for any bounded universally measurable function
$f:\partial G\to\Bbb R$, the function
$x\mapsto\int fd\mu_x:G\to\Bbb R$ is continuous and harmonic.
%478R

\spheader 478Xl (i)
Suppose that $r=2$, and that $x$, $y$, $z\in\BbbR^r$ are
such that $\|x\|<1=\|z\|$ and $y=0$.   Identify $\BbbR^2$ with $\Bbb C$,
and express $x$, $z$ as $\gamma e^{i\theta}$ and $e^{it}$ respectively.
Show that, in the language of 272Yg,
$\Bover{\|y-z\|^2-\|x-y\|^2}{\|x-z\|^r}=A_{\gamma}(\theta-t)$.
(ii) Compare 478I(b-iii) with 272Yg(iii).
%478I out of order query

\leader{478Y}{Further exercises (a)}%
%\spheader 478Ya
(i) Show that there is a function $f:\Bbb R\to\Bbb Q$ which
is `harmonic' in the sense of 478B, but is not continuous.   \Hint{take $f$
to be a linear operator when $\Bbb R$ is regarded as a linear space over
$\Bbb Q$.}   (ii) Show that if the continuum hypothesis is true, there is a
surjective
function $f:\BbbR^2\to\{0,1\}$ which is `harmonic' in the sense of 478B.
%478B

\spheader 478Yb Let $G\subseteq\BbbR^r$ be a connected open set, and
$f:G\to[0,\infty]$ a superharmonic Lebesgue measurable function which is
not everywhere infinite.   Show that $f$ is locally integrable.
%478C mt47bits

\spheader 478Yc Let $G\subseteq\BbbR^2$ be an open set, and $f:G\to\Bbb C$ a
function which is analytic when regarded as a function of a complex
variable.   Show that $\Real f$ is harmonic.   \Hint{The
non-trivial part is the theorem that $f$ has continuous second partial
derivatives.}
%478F

\spheader 478Yd Define $\psi:\BbbR^r\setminus\{0\}\to\BbbR^r$ by setting
$\psi(x)=\Bover{x}{\|x\|^2}$.
For a $[-\infty,\infty]$-valued function $f$ defined on a subset of
$\BbbR^r$, set $f^*(x)=\Bover1{\|x\|^{r-2}}f(\psi(x))$ for
$x\in\psi^{-1}[\dom f]\setminus\{0\}$.
(This is the {\bf Kelvin transform} of $f$ relative
to the sphere $\partial B(\tbf{0},1)$.)   (i) Show that if $f$ is real-valued
and twice continuously differentiable, then
$(\nabla^2f^*)(x)=\Bover1{\|x\|^{r+2}}(\nabla^2f)(\psi(x))$
for $x\in\dom f^*$.
%Armitage 1.6.3;  Axler \& Ramey 95
(ii) Show that if $\dom f$ is open and $f$ is non-negative,
lower semi-continuous and superharmonic, then $f^*$ is
superharmonic.  %478Xc
%478J

\spheader 478Ye Let $G\subseteq\BbbR^2$ be an open set.   Show that $G$ has
few wandering paths iff there is an $x\in\BbbR^2$ such that
$\hp((\BbbR^2\setminus(G\cup\{x\}))-x)>0$.
%478N mt47bits

\spheader 478Yf Show that 478K remains true if we replace
`three-times-differentiable function such that $f$ and its first three
derivatives are continuous and bounded' with
`twice-differentiable function such that $f$ and its first two
derivatives are continuous and bounded'.
%478K

\spheader 478Yg Let $f:\BbbR^r\to\Bbb R$ be a twice-differentiable
function such that $f$ and its first two
derivatives are continuous and bounded.   Show that

\Centerline{$\lim_{\delta\downarrow 0}\Bover1{\delta^2}
  \bigl(f(x)-\Bover1{\mu B(x,\delta)}\int_{B(x,\delta)}fd\mu\bigr)
=-\Bover{(\nabla^2f)(x)}{2(r+2)}$}

\noindent for every $x\in\BbbR^r$.
%478Yf 478K mt47bits

\spheader 478Yh Let $G\subseteq\BbbR^r$ be an open set and $f:G\to\Bbb R$ a
continuous function such that

\Centerline{$\lim_{\delta\downarrow 0}\Bover1{\delta^2}
  \bigl(f(x)-\Bover1{\mu B(x,\delta)}\int_{B(x,\delta)}fd\mu\bigr)
=0$}

\noindent for every $x\in G$.   Show that $f$ is harmonic.
%478Yg 478K apply 478Yg to smoothed versions of $f$

\spheader 478Yi Let $f:\BbbR^r\to[0,\infty]$ be a lower semi-continuous
superharmonic function.   Show
that $f\omega$ is continuous for $\mu_W$-almost every $\omega\in\Omega$.
%478L mt47bits

\spheader 478Yj Suppose that $A\subseteq\BbbR^r$.
Show that $x\mapsto\hp^*(A-x)$ is lower
semi-continuous at every point of $\BbbR^r\setminus A$, and continuous at
every point of $\BbbR^r\setminus\overline{A}$.
%if x_n\to x  there is an analytic  A'\supseteq A  such that
%\hp(A'-x_n)=\hp^*(A-x_n)$ for every  n , so enough to consider analytic
%A;  now use 477Ie
%478T

\spheader 478Yk Suppose that $A\subseteq\BbbR^r$ is such that
$\inf_{\delta>0}\hp^*(A\cap B(\tbf{0},\delta))>0$.   Show that $\hp^*(A)=1$.
%478U

\spheader 478Yl Let $\mu_W$ be three-dimensional Wiener measure on
$\Omega=C(\coint{0,\infty};\BbbR^3)_0$, and $e$ a unit vector in
$\BbbR^3$.   Set $Y_t(\omega)=\Bover1{\|\omega(t)-e\|}$ for
$\omega\in\Omega$ and $t\in\coint{0,\infty}$ such that $\omega(t)\ne e$.
(i) Show that if $R>1$ then

$$\eqalign{\Expn(Y_t)
&=\Bover1{(\sqrt{2\pi t})^3}
  \int\Bover{\exp(-\|x\|^2/2t)}{\|x-e\|}\mu(dx)\cr
&\le\Bover1{(\sqrt{2\pi t})^3}
  \int_{B(\tbf{0},1)}\Bover1{\|x\|}\mu(dx)
+\Bover1{(\sqrt{2\pi t})^3}
  \int_{B(e,R)\setminus B(e,1)}e^{-\|x\|^2/2t}\mu(dx)
+\Bover1R
\to\Bover1R\cr}$$

\noindent as $t\to\infty$, so that $\lim_{t\to\infty}\Expn(Y_t)=0$ and
$\langle Y_t\rangle_{t\ge 0}$ is not a martingale.    (ii) Show that if
$n\ge 1$ and $\tau_n$ is the Brownian hitting time to $B(e,2^{-n})$,
then $\langle Y_{t\wedge\tau_n}\rangle_{t\ge 0}$ is a martingale, where
$t\wedge\tau_n$ is the stopping time $\omega\mapsto\min(t,\tau_n(\omega))$.
(iii) Show that $\sequencen{\tau_n(\omega)}$ is a strictly increasing
sequence with limit $\infty$ for almost every $\omega$.
($\langle Y_t\rangle_{t\ge 0}$ is a `local martingale'.)
%478Pc Bain 5.2 out of order query
}%end of exercises

\endnotes{
\Notesheader{478} I find that books are still being published on the
subject of potential theory which ignore Brownian motion.   To my eye,
Newtonian potential, at least (and this is generally acknowledged to be the
core of the subject), is an essentially geometric concept, and
random walks are an indispensable tool for understanding it.
So I am giving these priority, at the
cost of myself ignoring Green's functions.

The definitions in 478B are already unconventional;  most authors take it
for granted that harmonic functions should be finite-valued and continuous
(see 478Cd).   All the work of this section refers to measurable functions.
But there are things which can be said about non-measurable functions
satisfying the definitions here (478Ya).   Let me draw your attention to
478Fa and 478H.   If we want to say that $x\mapsto\Bover1{\|x\|^{r-2}}$ is
harmonic, we have to be careful not to define it at $0$.   If (for
$r\ge 3$) we allow $\Bover1{0^{r-2}}=\infty$, we get a superharmonic
function.
If (for $r=1$) we allow $\Bover1{0^{-1}}=0$, we get a subharmonic function.
The slightly paradoxical phenomenon of 478Yl is another manifestation of
this.

I hope that using the operations $\overline{\phantom{x}}^{\infty}$ and
$\partial^{\infty}$ does not make things more difficult.   The point is
that by compactifying $\BbbR^r$ we get an efficient way of talking about
$\lim_{\|x\|\to\infty}f(x)$ when we need to.   This is particularly
effective for Brownian paths, since in three and more dimensions almost all
paths go off to infinity (478Md).   In two dimensions the situation is more
complex (478Mb-478Mc),
and we have to consider the possibility that a path $\omega$ in an
open set may be `wandering', in the sense that it neither strikes the
boundary nor goes to infinity, and $\lim_{t\to\infty}f(\omega(t))$ may fail
to exist even for the best-behaved functions $f$.   Of course this already
happens in one dimension, but only when $G=\Bbb R$, and classical potential
theory (though not, I think, Brownian potential theory) is nearly trivial
in the one-dimensional case.

Many readers will also find that setting $r=3$ and $r\beta_r=4\pi$ will
make the formulae easier to digest.   I allow for variations in $r$ partly
in order to cover the case $r=2$ (in this section, though not in the
next, many of the ideas translate directly into the one- and
two-dimensional cases), and partly because it is not always easy to guess
at a formula for $r\ge 4$ from the formula for $r=3$.   There is
little extra work to be done, given that \S\S472-475 cover the general
case.

I call 478K a `lemma' because I have made no attempt to look for weakest
adequate hypotheses;  of course we don't really need third derivatives
(478Yf).
The `theorems' are 478L and 478O, where the hypotheses seem to mark
natural boundaries of the arguments given.   In 478O %maybe elsewhere
I use a language which is both unusual and slightly contorted, in
order to do as much as possible without splitting the cases $r\le 2$ from
the rest.   Of course any result involving the notion of `few wandering
paths' really has two forms;  one when $r\ge 3$, so that there is no
restriction on the open set and we are genuinely making use of the
one-point
compactification of $\BbbR^r$, and one when $r\le 2$, in which essentially
all our paths are bounded.

Theorem 478O leads directly to a solution of Dirichlet's problem, in the
sense that, for an open set $G$ with few wandering paths, we have a
family of measures enabling us to
calculate the values within $G$ of a continuous function on
$\overline{G}^{\infty}$ which satisfies Laplace's equation inside $G$
(478Pc).   We do not get a satisfactory existence theorem;  we can
use harmonic measures to generate
many harmonic functions on $G$ (478S), but
we do not get good information on their behaviour near $\partial G$, and
are left guessing at which continuous functions on $\partial^{\infty}G$
will be extended continuously.   The method does, however, make it clear
that what matters is the geometry of the boundary;  we need to know
whether, starting from a point near the boundary, a random walk will hit
the boundary soon.   So at least we can see from 478M that (if $r\ge 2$) an
isolated point of $\partial G$ will be at worst an irrelevant distraction.
For $r\ge 3$ the next section will give some useful information
(479P {\it
et seq.}), though I shall not have space for a proper analysis.

The idea of 478Vb is that we have a particularly dramatic kind of
martingale.   Writing $S$ for the set of stopping times $\sigma\le\tau$,
it is easy to see that the family
$\family{\sigma}{S}{g_{\sigma}}$ is a martingale in the sense that
$\Sigma_{\sigma}\subseteq\Sigma_{\sigma'}$
and $g_{\sigma}$ is a conditional expectation of $g_{\sigma'}$ on
$\Sigma_{\sigma}$  whenever $\sigma\le\sigma'$ in $S$.
}

\discrpage


%variation of $\mu_x^{(G)}$ as $G$ varies;  narrow convergence for
%increasing unions?
