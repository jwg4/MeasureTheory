\frfilename{mt415.tex}
\versiondate{7.4.05}
\copyrightdate{2000}

\def\chaptername{Topologies and measures I}
\def\sectionname{Quasi-Radon measure spaces}
\def\undphi{\underline{\phi}}

\newsection{415}

We are now I think ready to draw together the properties of inner
regularity and $\tau$-additivity.   Indeed this section will unite
several of the
themes which have been running through the treatise so far:  (strict)
localizability, subspaces and products as well as the new concepts of
this chapter.   In
these terms, the principal results are that a quasi-Radon space is
strictly localizable (415A), any subspace of a quasi-Radon space is
quasi-Radon (415B), and the product of a family of strictly positive
quasi-Radon probability measures on separable metrizable spaces is
quasi-Radon (415E).
I describe a basic method of constructing quasi-Radon measures
(415K), with details of one of the standard ways of applying it (415L,
415N) and some notes on how to specify a quasi-Radon measure uniquely
(415H-415I).   I spell out useful results on
indefinite-integral measures (415O) and $L^p$ spaces (415P), and end the
section with a discussion of the Stone space $Z$ of a localizable
measure algebra $\frak A$ and an important relation in
$Z\times X$ when $\frak A$ is the measure algebra of a quasi-Radon
measure space $X$ (415Q-415R).

It would be fair to say that the study of quasi-Radon spaces for their
own sake is a minority interest.   If you are not already well
acquainted
with Radon measure spaces, it would make good sense to read this section
in parallel with the next.   In particular, the constructions of 415K
and 415L derive much of their importance from the corresponding
constructions in \S416.

\leader{415A}{Theorem} A quasi-Radon measure space is strictly
localizable.

\proof{ This is a special case of 414J.
}%end of proof of 415A

\leader{415B}{Theorem} Any subspace of a quasi-Radon measure space is
quasi-Radon.

\proof{ Let $(X,\frak T,\Sigma,\mu)$ be a quasi-Radon measure space and
$(Y,\frak T_Y,\Sigma_Y,\mu_Y)$ a subspace with the induced topology and
measure.   Because $\mu$ is complete, locally determined and localizable
(by 415A), so is $\mu_Y$ (214Ie).   Because $\mu_Y$ is semi-finite and
$\mu$ is an effectively locally finite $\tau$-additive topological
measure, so is
$\mu_Y$ (414K).   Because $\mu$ is inner regular with respect to the
closed sets and $\mu_Y$ is semi-finite, $\mu_Y$ is inner regular with
respect to the relatively closed subsets of $Y$ (412Pc).   So $\mu_Y$ is
a quasi-Radon measure.
}%end of proof of 415B

\leader{415C}{}\cmmnt{ In regular topological spaces, the condition
`inner regular with respect to the closed sets' in the definition of
`quasi-Radon measure' can be weakened or omitted.

\medskip

\noindent}{\bf Proposition} Let $(X,\frak T)$ be a regular topological
space.

(a) If $\mu$ is a complete locally determined effectively locally finite
$\tau$-additive topological measure on $X$, inner regular with respect
to the Borel sets, then it is a quasi-Radon measure.

(b) If $\mu$ is an effectively locally finite $\tau$-additive Borel
measure on $X$, its c.l.d.\ version is a quasi-Radon measure.

\proof{{\bf (a)} By 414Mb, $\mu$ is inner regular with respect to the
closed sets, which is the only feature missing from the given
hypotheses.

\medskip

{\bf (b)} The c.l.d.\ version of $\mu$ satisfies the hypotheses of
(a).
}%end of proof of 415C

\leader{415D}{}\cmmnt{ In separable metrizable spaces, among others, we
can even omit $\tau$-additivity.

\medskip

\noindent}{\bf Proposition} Let $(X,\frak T)$ be a hereditarily
Lindel\"of topological space\cmmnt{;  e.g., a separable metrizable
space (4A2P(a-iii)), indeed any space with a countable network (4A2Nb)}.

(i) If $\mu$ is a complete effectively locally finite measure on $X$,
inner regular with respect to the Borel sets, and its domain includes
a base for $\frak T$, then it is a quasi-Radon measure.

(ii) If $\mu$ is an effectively locally finite Borel measure on $X$,
then its completion is a quasi-Radon measure.

(iii) Any quasi-Radon measure on $X$ is $\sigma$-finite.

(iv) If $X$ is regular, any quasi-Radon measure on $X$ is completion
regular.

\proof{{\bf (a)} The basic fact we need is that if $\Cal G$ is any
family of open sets in $X$, then there is a countable
$\Cal G_0\subseteq\Cal G$
such that $\bigcup\Cal G_0=\bigcup\Cal G$ (4A2H(c-i)).
Consequently any effectively locally finite measure $\mu$ on $X$ is
$\sigma$-finite.   \Prf\ Let $\Cal G$ be the family of measurable open
sets of finite measure.   Let $\Cal G_0\subseteq\Cal G$ be a countable
set with the same union as $\Cal G$.   Then
$E=X\setminus\bigcup\Cal G_0$ is
measurable, and $E\cap G=\emptyset$ for every $G\in\Cal G$, so
$\mu E=0$; accordingly $\Cal G_0\cup\{E\}$ is a countable cover of $X$
by sets of finite measure, and $\mu$ is $\sigma$-finite.\ \Qed

Moreover, any measure on $X$ is $\tau$-additive.   \Prf\ If $\Cal G$ is
a non-empty upwards-directed family of open measurable sets, there is a
sequence $\sequencen{G_n}$ in $\Cal G$ with union $\bigcup\Cal G$.   If
$n\in\Bbb N$ there is a $G\in\Cal G$ such that
$\bigcup_{i\le n}G_i\subseteq G$, so

\Centerline{$\mu(\bigcup\Cal G)=\mu(\bigcup_{n\in\Bbb N}G_n)
=\sup_{n\in\Bbb N}\mu(\bigcup_{i\le n}G_i)\le\sup_{G\in\Cal G}\mu G$.}

\noindent As $\Cal G$ is arbitrary, $\mu$ is $\tau$-additive.\ \Qed

\medskip

{\bf (b)(i)} Now let $\mu$ be a complete effectively locally finite
measure on $X$, inner regular with respect to the Borel sets, and with
domain $\Sigma$ including a base for the topology of $X$.   If
$H\in\frak T$, then
$\Cal G=\{G:G\in\Sigma\cap\frak T,\,G\subseteq H\}$ has union $H$,
because $\Sigma\cap\frak T$ is a base for $\frak T$;  but in this case
there is a
countable $\Cal G_0\subseteq\Cal G$ such that $H=\bigcup\Cal G_0$, so
that $H\in\Sigma$.   Thus $\mu$ is a topological measure.   We know also
from (a) that it is $\tau$-additive and $\sigma$-finite, therefore
locally determined.   By 415Ca, it is a quasi-Radon measure.

\medskip

\quad{\bf (ii)} If $\mu$ is an effectively locally finite Borel measure
on $X$, then its completion $\hat\mu$ satisfies the conditions of (i),
so is a quasi-Radon measure.

\medskip

\quad{\bf (iii)} If $\mu$ is a quasi-Radon measure on $X$, it is surely
effectively locally finite, therefore $\sigma$-finite.

\medskip

\quad{\bf (iv)} If $X$ is regular, then every closed set is a zero set
(4A2H(c-ii)), so any measure which is inner regular with respect to the
closed sets is completion regular.
}%end of proof of 415D

\leader{415E}{}\cmmnt{ I am delaying most of the theory of products of
(quasi-\nobreak)Radon measures to \S417.   However, there is one result
which is
so important that I should like to present it here, even though some of
the ideas will have to be repeated later.

\medskip

\noindent}{\bf Theorem} Let $\familyiI{(X_i,\frak T_i,\Sigma_i,\mu_i)}$
be a family of separable metrizable quasi-Radon probability spaces such
that every $\mu_i$ is strictly positive, and $\lambda$ the product
measure on $X=\prod_{i\in I}X_i$.   Then

(i) $\lambda$ is a completion regular quasi-Radon measure;

(ii) if $F\subseteq X$ is a closed self-supporting set, there is a
countable set $J\subseteq I$ such that $F$ is determined by coordinates
in $J$, so $F$ is a zero set.

\proof{{\bf (a)} Write $\Lambda$ for the domain of $\lambda$, and
$\Cal U$ for the family of subsets of $X$ of the form
$\prod_{i\in I}G_i$ where $G_i\in\frak T_i$ for every $i\in I$ and
$\{i:G_i\ne X_i\}$
is finite.   Then $\Cal U$ is a base for the topology of $X$, included
in $\Lambda$.
For $J\subseteq I$ let $\lambda_J$ be the product measure on
$X_J=\prod_{i\in J}X_i$ and $\Lambda_J$ its domain.   Write
$\Cal U^{(J)}$ for the family of subsets of $X_J$ of the form
$\prod_{i\in J}G_i$ where
$G_i\in\frak T_i$ for each $i\in J$ and $\{i:G_i\ne X_i\}$ is finite.

\medskip

{\bf (b)} Consider first the case in which $I$ is countable.   In this
case $X$ also is separable and metrizable (4A2P(a-v)), while $\Lambda$
includes a base for its topology.   Also $\lambda$ is a complete
probability measure
and inner regular with respect to the closed sets (412Ua), so must be
a quasi-Radon measure, by 415D(i).

\medskip

{\bf (c)} Now consider uncountable $I$.   The key to the proof is the
following fact:  if $\Cal V\subseteq\Cal U$ has union $W$, then
$W\in\Lambda$ and $\lambda W=\sup_{V\in\Cal V^*}\lambda V$, where
$\Cal V^*$ is the set of unions of finite subsets of $\Cal V$.

\medskip

\Prf\ {\bf (i)} By 215B(iv), there is a countable set
$\Cal V_1\subseteq\Cal V$ such that $\lambda(U\setminus W_1)=0$ for
every $U\in\Cal V$, where $W_1=\bigcup\Cal V_1$.
Every member of $\Cal U$ is determined by coordinates in some finite set
(see 254M for this concept), so there is a countable set $J\subseteq I$
such that every member of $\Cal V_1$ is determined by coordinates in
$J$, and $W_1$ also is determined by coordinates in $J$.
Let $\pi_J:X\to X_J$ be the canonical
map.   Because it is an open map (4A2B(f-i)), $\pi_J[W]$ and
$\pi_J[W_1]$ are open in $X_J$, and belong to $\Lambda_J$, by (b).

\medskip

\quad{\bf (ii)} \Quer\ Suppose, if possible, that
$\lambda_J\pi_J[W]>\lambda_J\pi_J[W_1]$.   Since
$\pi_J[W]=\bigcup\{\pi_J[U]:U\in\Cal V\}$, while $\lambda_J$ is
quasi-Radon and all the sets $\pi_J[U]$ are open, there must
be some $U\in\Cal V$
such that $\lambda_J(\pi_J[U]\setminus\pi_J[W_1])>0$ (414Ea).   Now
$\pi_J$ is \imp\ (254Oa), so

\Centerline{$0<
\lambda\pi_J^{-1}[\pi_J[U]\setminus\pi_J[W_1]]
=\lambda(\pi_J^{-1}[\pi_J[U]]\setminus\pi_J^{-1}[\pi_J[W_1]])
=\lambda(\pi_J^{-1}[\pi_J[U]]\setminus W_1)$,}

\noindent because $W_1$ is determined by coordinates in $J$.

At this point note that $U$ is of the form $\prod_{i\in I}G_i$, where
$G_i\in\frak T_i$ for each $I$, so we can express $U$ as $U'\cap U''$,
where $U'=\pi_J^{-1}[\pi_J[U]]$ and
$U''=\pi_{I\setminus J}^{-1}[\pi_{I\setminus J}[U]]$.
$U'$ is determined by coordinates in $J$ and $U''$ is determined by
coordinates in $I\setminus J$.   In this case

\Centerline{$\lambda(U\setminus W_1)
=\lambda(U''\cap U'\setminus W_1)
=\lambda U''\cdot\lambda(U'\setminus W_1)$,}

\noindent because $U''$ is determined by coordinates in $I\setminus J$
and $U'\setminus W_1$ is determined by coordinates in $J$,
and we can identify $\lambda$ with the product
$\lambda_{I\setminus J}\times\lambda_J$
(254N).   But now recall that every $\mu_i$ is strictly positive.
Since $U$ is
surely not empty, no $G_i$ can be empty and no $\mu_iG_i$ can be $0$.
Consequently $\prod_{i\in I}\mu_iG_i>0$ (because only finitely
many terms in the product are less than $1$) and $\lambda U>0$;  more to
the point, $\lambda U''>0$.   Since we chose $U$ so that
$\lambda(U'\setminus W_1)>0$, we have $\lambda(U\setminus W_1)>0$.   But
this contradicts the first sentence of (i) just above.\ \Bang

\medskip

\quad{\bf (iii)} Thus $\lambda_J\pi_J[W]=\lambda_J\pi_J[W_1]$.   But
this means that $\lambda\pi_J^{-1}[\pi_J[W]]=\lambda W_1$.   Since
$\lambda$ is complete and $W_1\subseteq W\subseteq\pi_J^{-1}[\pi_J[W]]$,
$\lambda W$ is defined and equal to $\lambda W_1$.

Taking $\sequencen{V_n}$ to be a sequence running over
$\Cal V_1\cup\{\emptyset\}$, we have

\Centerline{$\lambda W=\lambda W_1
=\lambda(\bigcup_{n\in\Bbb N}V_n)
=\sup_{n\in\Bbb N}\lambda(\bigcup_{i\le n}V_i)
\le\sup_{V\in\Cal V^*}\lambda V
\le\lambda W,$}

\noindent so $\lambda W=\sup_{V\in\Cal V^*}\lambda V$, as
required.\ \Qed

\medskip

{\bf (d)} Thus we see that $\lambda$ is a topological measure.   But it
is also $\tau$-additive.   \Prf\ If $\Cal W$ is an upwards-directed
family of open sets in $X$ with union $W^*$, set

\Centerline{$\Cal V=\{U:U\in\Cal U,\,\exists\,W\in\Cal W,\,U\subseteq
W\}$.}

\noindent Then $W^*=\bigcup\Cal V$, so $\lambda W^*=\sup_{V\in\Cal
V^*}\lambda V$, where $\Cal V^*$ is the set of finite unions of members
of $\Cal V$.   But because $\Cal W$ is upwards-directed, every member of
$\Cal V^*$ is included in some member of $\Cal W$, so

\Centerline{$\lambda W^*=\sup_{V\in\Cal V^*}\lambda V
\le\sup_{W\in\Cal W}\lambda W\le\lambda W^*$.}

\noindent As $\Cal W$ is arbitrary, $\lambda$ is $\tau$-additive.\ \Qed

\medskip

{\bf (e)} As in (b) above, we know that $\lambda$ is a complete
probability
measure and is inner regular with respect to the closed sets, so it is a
quasi-Radon measure.   Because $\lambda$ is inner regular with respect
to the zero sets (412Ub), it is completion regular.

\medskip

{\bf (f)} Now suppose that $F\subseteq X$ is a closed self-supporting
set.   By 254Oc, there is a set $W\subseteq X$, determined by
coordinates in some countable set $J\subseteq I$, such that
$W\symmdiff F$ is negligible.   \Quer\ Suppose, if possible, that
$x\in F$ and $y\in X\setminus F$ are such that $x\restr J=y\restr J$.
Then there is a $U\in\Cal U$ such that $y\in U\subseteq X\setminus F$.
As in (b-ii) above, we can express $U$ as $U'\cap U''$ where $U'$,
$U''\in\Cal U$ are determined by coordinates in $J$ and $I\setminus J$
respectively.   In this case,

$$\eqalign{\lambda(F\cap U)
&=\lambda(W\cap U)
=\lambda(W\cap U')\cdot\lambda U''\cr
&=\lambda(F\cap U')\cdot\lambda U''
>0,\cr}$$

\noindent because $x\in F\cap U'$ and $F$ is self-supporting, while
$U''\ne\emptyset$ and $\lambda$ is strictly positive.   But $F\cap
U=\emptyset$, so this is impossible.\ \Bang

Thus $F$ is determined by coordinates in the countable set $J$.
Consequently it is of the form $\pi_J^{-1}[\pi_J[F]]$.   But
$\pi_J[X\setminus F]$ is open (4A2B(f-i) again), 
so its complement $\pi_J[F]$
is closed.   Now $X_J$ is metrizable (4A2P(a-v)), so $\pi_J[F]$ is a
zero set (4A2Lc) and $F$ is a zero set (4A2C(b-iv)).
}%end of proof of 415E

\leader{415F}{Corollary} (a) If $Y$ is either
$\coint{0,1}$ or $\ooint{0,1}$, endowed with Lebesgue measure, and $I$
is any set, then $Y^I$, with the product topology and measure, is a
quasi-Radon measure space.

(b) If $\familyiI{\nu_i}$ is a family of probability distributions on
$\Bbb R$\cmmnt{, in the sense of \S271 (that is, Radon probability
measures)}, and every $\nu_i$ is strictly positive, then the product
measure on $\BbbR^I$ is a quasi-Radon measure.

\cmmnt{\medskip

\noindent{\bf Remark} See also 416U below, and 453I, where there is
an alternative proof of the main step in 415E, applicable to some
further cases.   Yet another approach, most immediately applicable to
$\coint{0,1}^I$, is in 443Xp.   For further facts about these product
measures, see \S417, particularly 417M.
}%end of comment

\leader{415G}{Comparing quasi-Radon measures:  Proposition} Let $X$ be a
topological space, and $\mu$, $\nu$ two quasi-Radon measures on $X$.
Then the following are equiveridical:

(i) $\mu F\le\nu F$ for every closed set $F\subseteq X$;

(ii) $\dom\nu\subseteq\dom\mu$ and $\mu E\le\nu E$ for every
$E\in\dom\nu$.

\noindent If $\nu$ is locally finite, we can add

(iii) $\mu G\le\nu G$ for every open set $G\subseteq X$;

(iv) there is a base $\Cal U$ for the topology of $X$ such that
$G\cup H\in\Cal U$ for all $G$, $H\in\Cal U$ and $\mu G\le\nu G$ for
$G\in\Cal U$.

\proof{{\bf (a)} Of course (ii)$\Rightarrow$(i).   Suppose that (i) is
true.   Observe that if $E\in\dom\mu\cap\dom\nu$ (for instance, if
$E\subseteq X$ is Borel), then

\Centerline{$\mu E
=\sup\{\mu F:F\subseteq E\text{ is closed}\}
\le\sup\{\nu F:F\subseteq E\text{ is closed}\}
=\nu E$.}

\noindent Set $\Cal H=\{H:H\subseteq X$ is open, $\nu H<\infty\}$, and
$W=\bigcup\Cal H$.   Then $\nu(X\setminus W)=0$, because $\nu$ is
effectively locally finite, so $\mu(X\setminus W)=0$.   Set
$\Cal F=\{F:F\subseteq X$ is closed, $\mu F<\infty\}$.

Take any $E\in\dom\nu$.   If $F\in\Cal F$ and $\epsilon>0$, then
$\mu(F\cap W)=\mu F$, so there is an $H\in\Cal H$ such that
$\mu(F\cap H)\ge\mu F-\epsilon$.   Now there are
closed sets $F_1\subseteq F\cap H\cap E$,
$F_2\subseteq F\cap H\setminus E$ such that
$\nu F_1+\nu F_2\ge\nu(F\cap H)-\epsilon$, that is,
$\nu((F\cap H)\setminus(F_1\cup F_2))\le\epsilon$, so that
$\mu((F\cap H)\setminus(F_1\cup F_2))\le\epsilon$ and
$\mu F_1+\mu F_2\ge\mu(F\cap H)-\epsilon$.   This means that

\Centerline{$\mu_*(F\cap E)+\mu_*(F\setminus E)
\ge\mu(F\cap H)-\epsilon
\ge\mu F-\epsilon$.}

\noindent As $\epsilon$ is arbitrary,
$\mu_*(F\cap E)+\mu_*(F\setminus E)\ge\mu F$;  as $\mu$ is inner regular
with respect to $\Cal F$, $\mu$ measures $E$, by 413F(vii).

Thus $\dom\nu\subseteq\dom\mu$;  and we have already observed that
$\mu E\le\nu E$ whenever $E$ is measured by both.

\medskip

{\bf (b)} The first sentence in the proof of (a) shows that
(i)$\Rightarrow$(iii), and (iii)$\Rightarrow$(iv) is trivial.   If (iv)
is true and
$G\subseteq X$ is open, then $\Cal V=\{V:V\in\Cal U,\,V\subseteq G\}$ is
upwards-directed and has union $G$, so

\Centerline{$\mu G
=\sup_{V\in\Cal V}\mu V
\le\sup_{V\in\Cal V}\nu V
=\nu G$.}

\noindent Thus (iv)$\Rightarrow$(iii).

Now assume that $\nu$ is locally finite and that (iii) is true.
\Quer\ Suppose, if possible, that $F\subseteq X$ is a closed set such
that $\nu F<\mu F$.   Then $\Cal H$, as defined in part (a) of the
proof, is upwards-directed and has union $X$, so there is an
$H\in\Cal H$ such that $\nu F<\mu(F\cap H)$.   Now there is a closed set
$F'\subseteq H\setminus F$ such that

\Centerline{$\nu F'>\nu(H\setminus F)-\mu(F\cap H)+\nu F
\ge\nu H-\mu(F\cap H)$.}

\noindent Set $G=H\setminus F'$, so that $F\cap H\subseteq G$ and

\Centerline{$\nu G
=\nu H-\nu F'
<\mu(F\cap H)
\le\mu G$,}

\noindent which is impossible.\ \Bang

This shows that (provided that $\nu$ is locally finite)
(iii)$\Rightarrow$(i).
}%end of proof of 415G

\leader{415H}{Uniqueness of quasi-Radon measures:  Proposition} Let
$(X,\frak T)$ be a topological space and $\mu$, $\nu$ two quasi-Radon
measures on $X$.   Then the following are equiveridical:

(i) $\mu=\nu$;

(ii) $\mu F=\nu F$ for every closed set $F\subseteq X$;

(iii) $\mu G=\nu G$ for every open set $G\subseteq X$;

(iv) there is a base $\Cal U$ for the topology of $X$ such that
$G\cup H\in\Cal U$ for every $G$, $H\in\Cal U$ and
$\mu\restr\Cal U=\nu\restr\Cal U$;

(v) there is a base $\Cal U$ for the topology of $X$ such that
$G\cap H\in\Cal U$ for every $G$, $H\in\Cal U$ and
$\mu\restr\Cal U=\nu\restr\Cal U$.

\proof{ Of course (i) implies all the others.   (ii)$\Rightarrow$(i) is
immediate from 415G (see also 412L).   If (iii) is true, then, for any
closed set $F\subseteq X$,

$$\eqalign{\mu F
&=\sup\{\mu(G\cap F):G\in\frak T,\,\mu G<\infty\}\cr
&=\sup\{\mu G-\mu(G\setminus F):G\in\frak T,\,\mu G<\infty\}\cr
&=\sup\{\nu G-\nu(G\setminus F):G\in\frak T,\,\nu G<\infty\}
=\nu F;\cr}$$

\noindent so (iii)$\Rightarrow$(ii).   (iv)$\Rightarrow$(iii) by the
argument of (iv)$\Rightarrow$(iii) in the proof of 415G.

Finally, suppose that (v) is true.   Then
$\mu(G_0\cup\ldots\cup G_n)=\nu(G_0\cup\ldots\cup G_n)$ for all
$G_0,\ldots,G_n\in\Cal U$.
\Prf\ Induce on $n$.   For the inductive step to $n\ge 1$, if any $G_i$
has infinite measure (for either measure) the result is trivial.
Otherwise,

$$\eqalign{\mu(G_0\cup\ldots\cup G_n)
&=\mu(\bigcup_{i<n}G_i)+\mu G_n-\mu(\bigcup_{i<n}G_n\cap G_i)\cr
&=\nu(\bigcup_{i<n}G_i)+\nu G_n-\nu(\bigcup_{i<n}G_n\cap G_i)
=\nu(G_0\cup\ldots\cup G_n).\text{ \Qed}\cr}$$

\noindent So $\mu$ and $\nu$ agree on the base
$\{G_0\cup\ldots\cup G_n:G_0,\ldots,G_n\in\Cal U\}$, and (iv) is true.
}%end of proof of 415H

\leader{415I}{Proposition} Let $X$ be a completely regular topological
space and $\mu$, $\nu$ two quasi-Radon measures on $X$ such that
$\int fd\mu=\int fd\nu$ whenever $f:X\to\Bbb R$ is a
bounded continuous function
integrable with respect to both measures.   Then $\mu=\nu$.

\proof{ \Quer\ Otherwise, there is an open set $G\subseteq X$ such that
$\mu G\ne\nu G$;  suppose $\mu G<\nu G$.   Because $\nu$ is effectively
locally finite, there is an open set $G'\subseteq G$ such that
$\mu G<\nu G'<\infty$.   Now the cozero sets form a base for the
topology of
$X$, so $\Cal H=\{H:H\subseteq G'$ is a cozero set$\}$ has union $G'$;
as $\nu$ is $\tau$-additive, there is an $H\in\Cal H$ such that
$\nu H>\mu G$.   Express $H$ as $\{x:g(x)>0\}$ where
$g:X\to\coint{0,\infty}$
is continuous.   For each $n\in\Bbb N$, set $f_n=ng\wedge\chi X$;  then
$\sequencen{f_n}$ is a non-decreasing sequence with limit $\chi H$, so
there is an $n\in\Bbb N$ such that $\int f_nd\nu>\mu G\ge\int f_nd\mu$.
But $f_n$ is both $\mu$-integrable and $\nu$-integrable because $\mu G$
and $\nu H$ are both finite.\ \Bang
}%end of proof of 415I

\leader{415J}{Proposition} Let $X$ be a regular topological space, $Y$ a
subspace of $X$, and $\nu$ a quasi-Radon measure on $Y$.
Then there is a quasi-Radon measure $\mu$ on $X$ such that
$\mu E=\nu(E\cap Y)$ whenever $\mu$ measures $E$, that is, $Y$ has full
outer measure in $X$ and $\nu$ is the subspace measure on $Y$.

\proof{ Write $\Cal B$ for the Borel $\sigma$-algebra of $X$, and set
$\mu_0E=\nu(E\cap Y)$ for every $E\in\Cal B$.   Then it is easy to see
that $\mu_0$ is a $\tau$-additive Borel measure on $X$.   Moreover,
$\mu_0$ is effectively locally finite.   \Prf\ If $E\in\Cal B$ and
$\mu_0E>0$, there is a relatively open set $H\subseteq Y$ such that
$\nu H<\infty$ and $\nu(H\cap E\cap Y)>0$.   Now $H$ is of the form
$G\cap Y$ where $G\subseteq X$ is open, and we have
$\mu_0G=\nu H<\infty$, $\mu_0(E\cap G)=\nu(H\cap E\cap Y)>0$.\ \Qed

By 415Cb, the c.l.d.\ version $\mu$ of $\mu_0$ is a quasi-Radon measure
on $X$.   If $E\in\dom\mu$, then $E\cap Y\in\dom\nu$.   \Prf\ Let
$\Cal F_Y$ be the set of relatively closed subsets of $Y$ of finite
measure for $\nu$.   If $F\in\Cal F_Y$, it is expressible as $F'\cap Y$
where $F'$ is a closed subset of $X$, and $\mu F'=\mu_0F'=\nu F$ is
finite.   So there are $E_1$, $E_2\in\Cal B$ such that
$E_1\subseteq E\cap F'\subseteq E_2$ and
$\mu E_1=\mu(E\cap F')=\mu E_2$.   Accordingly

\Centerline{$E_1\cap Y\subseteq E\cap Y\cap F\subseteq E_2\cap Y$}

\noindent and

\Centerline{$\nu(E_1\cap Y)=\nu(E_2\cap Y)=\mu(E\cap F')$}

\noindent is finite.   This means
that $E\cap Y\cap F\in\dom\nu$;  because $\nu$ is complete and locally
determined and inner regular with respect to $\Cal F_Y$,
$E\cap Y\in\dom\nu$, by 412Ja.\ \Qed

If $E\in\dom\mu$, then

$$\eqalign{\mu E
&=\sup\{\mu F:F\subseteq E\text{ is closed}\}\cr
&=\sup\{\nu(F\cap Y):F\subseteq E\text{ is closed}\}
\le\nu(E\cap Y).\cr}$$

\noindent On the other hand, if $\gamma<\nu(E\cap Y)$, there is a
relatively open set $H\subseteq Y$ such that $\nu H<\infty$ and
$\nu(E\cap Y\cap H)\ge\gamma$ (412F).   Let $G\subseteq X$ be an open
set such that $G\cap Y=H$.   Then

\Centerline{$\mu E
\ge\mu G-\mu(G\setminus E)
=\nu H-\nu(H\setminus E)
=\nu(E\cap Y\cap H)\ge\gamma$.}

\noindent As $\gamma$ is arbitrary, $\mu E=\nu(E\cap Y)$.

Thus $\mu E=\nu(E\cap Y)$ whenever $\mu$ measures $E$.   So if $E$,
$F\in\dom\mu$ and $E\cap Y\subseteq F$,

\Centerline{$\mu E=\nu(E\cap Y)\le\nu(F\cap Y)=\mu F$;}

\noindent as $F$ is arbitrary, $\mu^*(E\cap Y)=\mu E$;  as $E$ is
arbitrary, $Y$ has full outer measure in $X$.   Moreover, if $\mu_Y$ is
the subspace measure on $Y$, $\mu_YH=\mu^*H=\nu H$ whenever
$H\in\dom\mu_Y$, that is, $H=E\cap Y$ for some $E\in\dom\mu$.   Now
$\mu_Y$, like $\nu$, is a quasi-Radon measure on $Y$ (415B), and they
agree on the (relatively) closed subsets of $Y$, so are equal, by 415H.
}%end of proof of 415J

\leader{415K}{}\cmmnt{ I come now to a couple of basic results on the
construction of quasi-Radon measures.  The first follows 413J.

\medskip

\noindent}{\bf Theorem} Let $X$ be a topological space
and $\Cal K$ a family of closed subsets of $X$ such that

\inset{$\emptyset\in\Cal K$,

($\dagger$) $K\cup K'\in\Cal K$ whenever $K$, $K'\in\Cal K$ are
disjoint,

($\ddagger$) $F\in\Cal K$ whenever $K\in\Cal K$ and
$F\subseteq K$ is closed.}

\noindent Let $\phi_0:\Cal K\to\coint{0,\infty}$ be a
functional such that

\inset{($\alpha$) $\phi_0 K=\phi_0 L+\sup\{\phi_0 K':K'\in\Cal K,\,
K'\subseteq K\setminus L\}$ whenever $K$, $L\in\Cal K$ and
$L\subseteq K$,

($\beta$) $\inf_{K\in\Cal K'}\phi_0K=0$ whenever
$\Cal K'$ is a non-empty downwards-directed subset of $\Cal K$ with
empty intersection,

($\gamma$) whenever $K\in\Cal K$ and $\phi_0K>0$, there is an
open set $G$ such that the supremum
$\sup_{K'\in\Cal K,K'\subseteq G}\phi_0K'$ is finite,
while $\phi_0K'>0$ for some $K'\in\Cal K$ such that
$K'\subseteq K\cap G$.}

\noindent Then there is a unique quasi-Radon measure on $X$ extending
$\phi_0$ and inner regular with respect to $\Cal K$.

\proof{ By 413J, there is a complete locally determined measure $\mu$ on
$X$, inner regular with respect to $\Cal K$, and
extending $\phi_0$;  write $\Sigma$ for the domain of $\mu$.   If
$F\subseteq X$ is closed, then $K\cap F\in\Cal K\subseteq\Sigma$ for
every $K\in\Cal K$, so $F\in\Sigma$, by 413F(ii);   accordingly every
open set is measurable.   Because $\mu$ is inner regular with respect to
$\Cal K$ it is surely inner regular with respect to the closed sets.
If $E\in\Sigma$ and $\mu E>0$, there is a $K\in\Cal K$ such that
$K\subseteq E$ and $\mu K>0$;  now ($\gamma$) tells us that
there is an open set $G$ such that $\mu G<\infty$ and $\mu(G\cap K)>0$,
so that $\mu(G\cap E)>0$.   As $E$ is arbitrary, $\mu$ is effectively
locally finite.   Now suppose that $\Cal G$ is a non-empty
upwards-directed family of open sets with union $H$, and that
$\gamma<\mu H$.   Then there is a $K\in\Cal K$ such that $K\subseteq H$
and $\mu K>\gamma$.   Applying the hypothesis ($\beta$) to
$\Cal K'=\{K\setminus G:G\in\Cal G\}$, we see that
$\inf_{G\in\Cal G}\mu(K\setminus G)=0$, so that

\Centerline{$\sup_{G\in\Cal G}\mu G\ge\sup_{G\in\Cal G}\mu(K\cap G)
=\mu K\ge\gamma$.}

\noindent As $\Cal G$ and $\gamma$ are arbitrary, $\mu$ is
$\tau$-additive.   So $\mu$ is a quasi-Radon measure.
}%end of proof of 415K

\leader{415L}{Proposition} Let $(X,\Sigma_0,\mu_0)$ be a measure space
and $\frak T$ a topology on $X$ such that $\mu_0$ is $\tau$-additive,
effectively locally finite and inner regular with respect to the closed
sets, and $\Sigma_0$ includes a base for $\frak T$.
Then $\mu_0$ has a unique extension to a quasi-Radon measure $\mu$ on
$X$ such that

\inset{(i) $\mu F=\mu_0^*F$ whenever $F\subseteq X$ is closed and
$\mu_0^*F<\infty$,}

\inset{(ii) $\mu G=(\mu_0)_*G$ whenever $G\subseteq X$ is open,}

\inset{(iii) the embedding $\Sigma_0\embedsinto\Sigma$ identifies the
measure algebra $(\frak A_0,\bar\mu_0)$ of $\mu_0$ with an order-dense
subalgebra
of the measure algebra $(\frak A,\bar\mu)$ of $\mu$, so that the
subrings
$\frak A_0^f$, $\frak A^f$ of elements of finite measure coincide, and
$L^p(\mu_0)$ may be identified with $L^p(\mu)$ for $1\le p<\infty$,}

\inset{(iv) whenever $E\in\Sigma$ and $\mu E<\infty$, there is an
$E_0\in\Sigma_0$ such that $\mu(E\symmdiff E_0)=0$,}

\inset{(v) for every $\mu$-integrable real-valued function $f$ there is
a
$\mu_0$-integrable function $g$ such that $f=g\,\,\mu$-a.e.}

\noindent If $\mu_0$ is complete and locally determined, then we have

\inset{(i)$'$ $\mu F=\mu_0^*F$ for every closed $F\subseteq X$.}
\noindent If $\mu_0$ is localizable, then we have

\inset{(iii)$'$ $\frak A_0=\frak A$, so that $L^0(\mu)\cong L^0(\mu_0)$ and
$L^{\infty}(\mu)\cong L^{\infty}(\mu_0)$,}

\inset{(iv)$'$ for every $E\in\Sigma$ there is an $E_0\in\Sigma_0$ such
that
$\mu(E\symmdiff E_0)=0$,}

\inset{(v)$'$ for every $\Sigma$-measurable real-valued function $f$
there
is a $\Sigma_0$-measurable real-valued function $g$ such that
$f=g\,\,\mu$-a.e.}

\proof{{\bf (a)} Let $\Cal K$ be the set of closed subsets of $X$ of
finite outer measure for $\mu_0$.   Note that $\mu_0$ is inner regular
with respect to $\Cal K$, because it is inner regular with respect to
the closed sets and also with respect to the sets of finite measure.

It is obvious from its definition that $\Cal K$ satisfies ($\dagger$)
and ($\ddagger$) of 415K.   For $K\in\Cal K$, set $\phi_0K=\mu_0^*K$.
Then $\phi_0$ satisfies ($\alpha$)-($\gamma$) of 415K.

\medskip

\Prf\ \grheada\ If $K$, $L\in\Cal K$ and $L\subseteq K$, take measurable
envelopes $E_0$, $E_1\in\Sigma_0$ of $K$, $L$ respectively.   (i) Let
$\epsilon>0$.   Because $\mu_0$ is inner regular with respect to the
closed sets, there is a closed set $F\in\Sigma_0$ such that $F\subseteq
E_0\setminus E_1$ and
$\mu F\ge\mu_0(E_0\setminus E_1)-\epsilon$.   Set $K'=F\cap K$.   Then
$K'\in\Cal K$ and

\Centerline{$\phi_0K'=\mu_0^*(F\cap K)=\mu_0(F\cap E_0)=\mu_0 F
\ge\mu_0E_0-\mu_0E_1-\epsilon=\phi_0K-\phi_0L-\epsilon$.}

\noindent As $\epsilon$ is arbitrary, we have

\Centerline{$\phi_0K
\le\phi_0L+\sup\{\phi_0K':K'\in\Cal K,\,K'\subseteq K\setminus L\}$.}

\noindent (ii) On the other hand, \Quer\ suppose, if possible, that
there is a closed set $K'\subseteq K\setminus L$ such that
$\mu_0^*L+\mu_0^*K'>\mu^*K$.   Let $E_2$ be a measurable envelope of
$K'$, so that $\mu_0E_1+\mu_0E_2>\mu_0E_0$;  since

\Centerline{$\mu_0(E_1\setminus
E_0)=\mu_0^*(L\setminus E_0)=\mu_0^*\emptyset=0$,
\quad$\mu_0(E_2\setminus E_0)=\mu_0^*(K'\setminus E_0)=0$,}

\noindent $\mu_0(E_1\cap E_2)>0$.   Because $\mu_0$
is effectively locally finite, there is a measurable open set $G_0$, of
finite measure, such that $\mu_0(G_0\cap E_1\cap E_2)>0$.   Set

\Centerline{$\Cal G=\{G\cup G':G$, $G'\in\Sigma_0\cap\frak T$,
$G\subseteq G_0\setminus L$, $G'\subseteq G_0\setminus K'\}$.}

\noindent Then $\Cal G$ is an upwards-directed family of measurable open
sets, and because $\Sigma_0$ includes a base for the topology of $X$,
its
union is $(G_0\setminus L)\cup(G_0\setminus K')=G_0$.   So there is an
$H\in\Cal G$ such that
$\mu_0H>\mu_0G_0-\mu_0(E_1\cap E_2)$, that is, there are open sets
$G$, $G'\in\Sigma_0$ such that $G\subseteq G_0\setminus L$, $G'\subseteq
G_0\setminus K'$ and $\mu_0((G\cup G')\cap E_1\cap E_2))>0$.
But we must have

\Centerline{$\mu_0(G\cap E_1)=\mu_0^*(G\cap L)=0$,
\quad$\mu_0(G'\cap E_2)=\mu_0^*(G'\cap K')=0$,}

\noindent so this is impossible.\ \Bang

Accordingly

\Centerline{$\phi_0K\ge\phi_0L+\sup\{\phi_0K':K'\in\Cal K,\,K'\subseteq
K\setminus L\}$,}

\noindent so that $\phi_0$ satisfies condition $(\alpha)$ of 415K.

\medskip

\quad\grheadb\ Let $\Cal K'\subseteq\Cal K$ be a non-empty
downwards-directed family with empty intersection.   Fix $K_0\in\Cal K'$
and $\epsilon>0$.   Let $E_0$ be a measurable envelope of $K_0$ and
$G_0$ a measurable open set of finite measure such that $\mu_0(G_0\cap
E_0)\ge\mu_0E_0-\epsilon$.   Then

\Centerline{$\Cal G=\{G:G\in\Sigma_0\cap\frak T$, $G\subseteq
G_0\setminus
K$ for some $K\in\Cal K'$ such that $K\subseteq K_0\}$}

\noindent is an upwards-directed family of measurable open sets, and its
union is $G_0\setminus\bigcap\Cal K'=G_0$, again because $\Sigma_0$
includes
a base for the topology $\frak T$.   So there is a $G\in\Cal G$ such
that
$\mu_0G\ge\mu_0G_0-\epsilon$.   Let $K\in\Cal K'$ be such that
$K\subseteq K_0$ and $G\cap K=\emptyset$;  then

\Centerline{$\phi_0K=\mu_0^*K\le\mu_0(E_0\setminus
G)\le\mu_0(E_0\setminus
G_0)+\mu_0(G_0\setminus G)\le 2\epsilon$.}

\noindent As $\epsilon$ is arbitrary, $\inf_{K\in\Cal K'}\phi_0K=0$.

\medskip

\quad\grheadc\ If $K\in\Cal K$ and $\phi_0K>0$, let $E_0$ be a
measurable envelope of $K$.   Then there is a measurable open set $G$ of
finite measure such that $\mu_0(G\cap E_0)>0$.   Of course
$\sup_{K'\in\Cal
K,K'\subseteq G}\phi_0K'\le\mu_0G<\infty$;  but also there is a
measurable closed set $K'\subseteq G\cap E_0$ such that $\mu_0K'>0$, in
which case $\phi_0(K\cap K')=\mu_0(E_0\cap K')>0$.   So $\phi_0$
satisfies condition ($\gamma$).\ \Qed

\medskip

{\bf (b)} By 415K, $\phi_0$ has an extension to a quasi-Radon measure
$\mu$
on $X$ which is inner regular with respect to $\Cal K$.   Write $\Sigma$
for the domain
of $\mu$.   Note that, for $K\in\Cal K$,

\Centerline{$\mu K=\phi_0K=\mu_0^*K$,}
\noindent so we can already be sure that the conclusion (i) of this
theorem is satisfied.  Now $\mu$ extends $\mu_0$.

\medskip
\Prf{\bf (i)} Take any $K\in\Cal K$.
Let $E_0\in\Sigma_0$ be a measurable envelope of $K$ for the measure
$\mu_0$.   If $E\in\Sigma_0$, then surely

$$\eqalign{\mu_*(K\cap E)
&=\sup\{\mu K':K'\in\Cal K,\,K'\subseteq K\cap E\}\cr
&=\sup\{\mu_0^*K':K'\in\Cal K,\,K'\subseteq K\cap E\}
\le\mu_0^*(K\cap E).\cr}$$

\noindent On the other hand, given
$\gamma<\mu_0^*(K\cap E)=\mu_0(E_0\cap E)$, there is a closed set
$F\in\Sigma_0$ such that $F\subseteq E_0\cap E$
and $\mu_0F\ge\gamma$, so that

\Centerline{$\mu_*(K\cap E)
\ge\mu(K\cap F)
=\mu_0^*(K\cap F)
=\mu_0(E_0\cap F)\ge\gamma$.}

\noindent Thus $\mu_*(K\cap E)=\mu_0^*(K\cap E)$ for every $K\in\Cal K$ and
$E\in\Sigma_0$.

\woddheader{415L}{4}{2}{2}{36pt}

\quad{\bf (ii)} If $K\in\Cal K$ and $E\in\Sigma_0$ then

\Centerline{$\mu_*(K\cap E)+\mu_*(K\setminus E)
=\mu_0^*(K\cap E)+\mu_0^*(K\setminus E)=\mu_0^*K=\mu K$.}

\noindent Because $\mu$ is complete and locally determined and inner
regular with respect to $\Cal K$, $E\in\Sigma$ (413F(iv)).   Thus
$\Sigma_0\subseteq\Sigma$.

\medskip

\quad{\bf (iii)} For any $E\in\Sigma_0$, we now have

$$\eqalignno{\mu E
&=\sup\{\mu K:K\in\Cal K,\,K\subseteq E\}
=\sup\{\mu_0^*K:K\in\Cal K,\,K\subseteq E\}\cr
&\le\mu_0E
=\sup\{\mu_0K:K\in\Cal K\cap\Sigma_0,\,K\subseteq E\}
\le\mu E.\cr}$$

\noindent As $E$ is arbitrary, $\mu$ extends $\mu_0$.\ \Qed

\medskip

{\bf (c)} Because $\Sigma_0\cap\frak T$ is a base for $\frak T$, closed
under finite unions, $\mu$ is unique, by 415H(iv).

\medskip

{\bf (d)} Now for the conditions (i)-(v).   I have already noted that
(i) is guaranteed by the construction.   Concerning (ii), if
$G\subseteq X$ is
open, we surely have $(\mu_0)_*G\le\mu_*G=\mu G$ because $\mu$ extends
$\mu_0$.   On the other hand, writing
$\Cal G=\{G':G'\in\Sigma_0\cap\frak T,\,G'\subseteq G\}$, $\Cal G$ is
upwards-directed and has union $G$, so

\Centerline{$\mu G=\sup_{G'\in\Cal G}\mu G'
=\sup_{G'\in\Cal G}\mu_0G'\le(\mu_0)_*G$.}

\noindent So (ii) is true.

Because $\mu$ extends $\mu_0$, the embedding $\Sigma_0\embedsinto\Sigma$
corresponds to a measure-preserving embedding of $\frak A_0$ as a
$\sigma$-subalgebra of $\frak A$.   To see that $\frak A_0$ is
order-dense in $\frak A$, take any
non-zero $a\in\frak A$.   This is expressible as $E^{\ssbullet}$ for some
$E\in\Sigma$ with $\mu E>0$.   Now there is a $K\in\Cal K$ such that
$K\subseteq E$ and $\mu K>0$.   There is an $E_0\in\Sigma_0$ which is a
measurable envelope for $K$ with respect to $\mu_0$, so that

\Centerline{$\mu E_0=\mu_0E_0=\mu_0^*K=\mu K$.}

\noindent But this means that

\Centerline{$0\ne E_0^{\ssbullet}=K^{\ssbullet}
\Bsubseteq E^{\ssbullet}=a$}

\noindent in $\frak A$, while $E_0^{\ssbullet}\in\frak A_0$.   As $a$ is
arbitrary, $\frak A_0$ is order-dense in $\frak A$.

If $a\in\frak A^f$, then $B=\{b:b\in\frak A_0,\,b\Bsubseteq a\}$ is
upwards-directed and $\sup_{b\in B}\bar\mu_0b\le\bar\mu a$ is finite;
accordingly $B$ has a supremum in $\frak A_0$ (321C), which must also be
its supremum in
$\frak A$, which is $a$ (313O, 313K).   So $a\in\frak A_0$.   Thus
$\frak A^f$ can be identified with $\frak A_0^f$.   But this means that,
for any $p\in\coint{1,\infty}$, $L^p(\mu)\cong L^p(\frak A,\bar\mu)$ is
identified
with $L^p(\frak A_0,\bar\mu_0)\cong L^p(\mu)$ (366H).   This proves
(iii).

Of course (iv) and (v) are just translations of this.   If $E\in\Sigma$
and $\mu E<\infty$, then $E^{\ssbullet}\in\frak A^f\subseteq\frak A_0$,
that is, there is an $E_0\in\Sigma_0$ such that $\mu(E\symmdiff E_0)=0$.
If $f$ is $\mu$-integrable, then $f^{\ssbullet}\in L^1(\mu)=L^1(\mu_0)$,
that is, there is a $\mu_0$-integrable function $f_0$ such that
$f=f_0\,\,\mu$-a.e.

\medskip

{\bf (e)} If $\mu_0$ is complete and locally determined and
$F\subseteq X$ is an arbitrary closed set, then

\Centerline{$\mu_0^*F=\sup_{K\in\Cal K}\mu_0^*(F\cap K)
=\sup_{K\in\Cal K}\mu(F\cap K)=\sup_{K\in\Cal K,K\subseteq F}\mu K
=\mu F$}

\noindent by 412Jc, because $\mu$ and $\mu_0$ are both inner regular
with respect to $\Cal K$.

\medskip

{\bf (f)} If $\mu_0$ is localizable, $\frak A_0$ is Dedekind complete;
as it is order-dense in $\frak A$, the two must coincide (314Ib).
Consequently

\Centerline{$L^0(\mu)\cong L^0(\frak A)=L^0(\frak A_0)\cong L^0(\mu_0)$,
\quad$L^{\infty}(\mu)\cong L^{\infty}(\frak A)
=L^{\infty}(\frak A_0)\cong L^{\infty}(\mu_0)$.}

\noindent So (iii)$'$ is true;  now (iv)$'$ and (v)$'$ follow at once.
}%end of proof of 415L

\leader{415M}{Corollary} Let $(X,\frak T)$ be a regular topological
space and $\mu_0$ an effectively locally finite $\tau$-additive measure
on $X$, defined on the $\sigma$-algebra generated by a base
for $\frak T$.   Then $\mu_0$ has a unique extension to a quasi-Radon
measure on $X$.

\proof{ By 414Mb, $\mu_0$ is inner regular with respect to the closed
sets.   So 415L gives the result.
}%end of proof of 415M

\leader{415N}{Corollary} Let $(X,\frak T)$ be a completely regular
topological space, and $\mu_0$ a $\tau$-additive effectively locally
finite Baire measure on $X$.   Then $\mu_0$ has a unique extension to a
quasi-Radon measure on $X$.

\proof{ This is a special case of 415M, because the domain $\CalBa(X)$ of
$\mu_0$ is generated by the family of
cozero sets, which form a base for $\frak T$ (4A2Fc).
}%end of proof of 415N

\leader{415O}{Proposition} (a) Let $(X,\frak T)$ be a topological space,
and $\mu$, $\nu$ two quasi-Radon measures on $X$.   Then $\nu$ is an
indefinite-integral measure over $\mu$\cmmnt{ (definition:
234J\footnote{Formerly 2{}34B.})} iff
$\nu F=0$ whenever $F\subseteq X$ is closed and $\mu F=0$.

(b) Let $(X,\frak T,\Sigma,\mu)$ be a
quasi-Radon measure space, and $\nu$ an indefinite-integral measure over
$\mu$.   If $\nu$ is effectively locally 
finite it is a quasi-Radon measure.

\proof{{\bf (a)} If $\nu$ is an indefinite-integral measure over $\mu$,
then of course it is zero on all $\mu$-negligible closed sets.   So let
us suppose that the condition is satisfied.   Write $\Sigma=\dom\mu$ and
$\Tau=\dom\nu$.

\medskip

\quad{\bf (i)} If $E\subseteq X$ is a $\mu$-negligible Borel set it is
$\nu$-negligible, because every closed subset of $E$ must be
$\mu$-negligible, therefore $\nu$-negligible, and $\nu$ is inner regular
with respect to the closed sets.
In particular, taking $U^*$ to be the union of the family
$\Cal U=\{U:U\in\frak T$, $\mu U<\infty\}$,
$\nu(X\setminus U^*)=\mu(X\setminus U^*)=0$ because $\mu$ is effectively
locally finite.
Also, of course, taking $V^*$ to be the union of the family
$\Cal V=\{V:V\in\frak T$, $\nu V<\infty\}$, $\nu(X\setminus V^*)=0$
because $\nu$ is effectively locally finite.   Setting
$\Cal G=\Cal U\cap\Cal V$ and $G^*=\bigcup\Cal G$, we have
$G^*=U^*\cap V^*$, so $G^*$ is $\nu$-conegligible.

\medskip

\quad{\bf (ii)} In fact, every $\mu$-negligible set $E$ is
$\nu$-negligible.   \Prf\Quer\ Otherwise, $\nu^*(E\cap G^*)>0$.
Because the subspace measure $\nu_E$ is quasi-Radon (415B), there is a
$G\in\Cal G$ such that $\nu^*(E\cap G)>0$.   But there is an
F$_{\sigma}$ set
$H\subseteq G\setminus E$ such that $\mu H=\mu(G\setminus E)$, and now
$E\cap G$ is included in the $\mu$-negligible Borel set $G\setminus H$,
so that $\nu(E\cap G)=\nu(G\setminus H)=0$.\ \Bang\Qed

\medskip

\quad{\bf (iii)} Let $\Cal K$ be the family of closed subsets $F$ of $X$
such that either $F$ is included in some member of
$\Cal G$ or $F\cap G^*=\emptyset$.   If $E\in\dom\mu$ and
$\mu E>0$, then there is an $F\in\Cal K$ such that $F\subseteq E$ and
$\mu F>0$.   \Prf\ If $\mu(E\setminus G^*)>0$ take any closed set
$F\subseteq E\setminus G^*$ with $\mu F>0$.   Otherwise,
$\mu(E\cap G^*)>0$.   Because the subspace measure
$\mu_E$ is quasi-Radon, there is a $G\in\Cal G$ such that $\mu(E\cap
G)>0$;  and now we can find a closed set
$F\subseteq E\cap G$ with $\mu F>0$, and $F\in\Cal K$.\ \Qed

\medskip

\quad{\bf (iv)} By 412Ia, there is a decomposition $\familyiI{X_i}$ for
$\mu$ such that every $X_i$ except perhaps one belongs to $\Cal K$ and
that exceptional one, if any, is $\mu$-negligible.   Now
$\familyiI{X_i}$ is a decomposition for $\nu$.   \Prf\ Every $X_i$ is
measured by $\nu$ because it is either closed or $\mu$-negligible, and
of finite measure for $\nu$ because it is either $\nu$-negligible
or included in a member of $\Cal G$.   If
$E\subseteq X$ and $\nu E>0$, then $\nu(E\cap G^*)>0$, so there must be
some $G\in\Cal G$ such that
$\nu(E\cap G)>0$.   Now $J=\{i:i\in I$, $\mu(X_i\cap G)>0\}$ is
countable, and $\nu(G\setminus\bigcup_{i\in J}X_i)
=\mu(G\setminus\bigcup_{i\in J}X_i)=0$, so there is an $i\in J$ such
that $\nu(X_i\cap E)>0$.   By 213Ob, $\familyiI{X_i}$ is a decomposition
for $\nu$.\ \Qed

\medskip

\quad{\bf (v)} It follows that $\Sigma\subseteq\Tau$.   \Prf\ If
$E\in\Sigma$, then for every $i\in I$ there is an F$_{\sigma}$ set
$H\subseteq E\cap X_i$ such that $E\cap X_i\setminus H$ is
$\mu$-negligible, therefore $\nu$-negligible, and $E\cap X_i\in\Tau$.
As $i$ is arbitrary, $E\in\Tau$.\ \QeD\   In fact, $\nu$ is the
completion of $\nu\restr\Sigma$.   \Prf\ If $F\in\Tau$, then for every
$i\in I$ there is an F$_{\sigma}$ set $H_i\subseteq F\cap X_i$ such that
$F\cap X_i\setminus H_i$ is $\nu$-negligible.   Set
$H=\bigcup_{i\in I}H_i$;  because $H\cap X_i=H_i$ belongs to $\Sigma$
for every $i$, $H\in\Sigma$;  and
$\nu(F\setminus H)=\sum_{i\in I}\nu(F\cap X_i\setminus H)=0$.
Similarly, there is an $H'\in\Sigma$ such that $H'\subseteq X\setminus
F$ and $\nu((X\setminus F)\setminus H')=0$, so that $H\subseteq
F\subseteq X\setminus H'$ and $\nu((X\setminus H')\setminus H)=0$.   So
$F$ is measured by the completion of $\nu\restr\Sigma$.   Since $\nu$
itself is complete, it must be the completion of $\nu\restr\Sigma$.\
\Qed

\medskip

\quad{\bf (vi)} By (iv), $\nu$ is inner regular with respect to
$\{E:E\in\Sigma$, $\mu E<\infty\}$.   By 234O\footnote{Formerly 2{}34G.},
$\nu$ is an indefinite-integral measure over $\mu$.

\medskip

{\bf (b)} Let $f\in\eusm L^0(\mu)$ be a non-negative function such that
$\nu F=\int f\times\chi F\,d\mu$ whenever this is defined.   Because
$\mu$ is complete and locally determined, so is $\nu$
(234Nb\footnote{Formerly 2{}34F.}).
Because $\mu$ is an effectively locally finite $\tau$-additive
topological measure, $\nu$ is a $\tau$-additive topological measure
(414H).   Because $\mu$ is inner regular with respect to the closed
sets, so is $\nu$ (412Q).   Since we are assuming in the hypotheses that
$\nu$ is effectively locally finite, it is a quasi-Radon measure.
}%end of proof of 415O

\leader{415P}{Proposition} Let $(X,\frak T,\Sigma,\mu)$ be a quasi-Radon
measure space.

(a) Suppose that $(X,\frak T)$ is completely regular.   If
$1\le p<\infty$ and $f\in\eusm L^p(\mu)$, then for any $\epsilon>0$
there is a bounded continuous
function $g:X\to\Bbb R$ such that $\mu\{x:g(x)\ne 0\}<\infty$ and
$\|f-g\|_p\le\epsilon$.

(b) Suppose that $(X,\frak T)$ is regular and Lindel\"of.   Let
$f\in\eusm L^0(\mu)$ be locally integrable.   Then for any $\epsilon>0$
there is a continuous function $g:X\to\Bbb R$ such that
$\|f-g\|_1\le\epsilon$.

\proof{{\bf (a)} Write $\eusm C$ for the set of bounded continuous
functions $g:X\to\Bbb R$ such that $\{x:g(x)\ne 0\}$ has finite measure.
Then $\eusm C$ is a linear subspace of $\Bbb R^X$ included in
$\eusm L^p=\eusm L^p(\mu)$.
Let $\eusm U$ be the closure of $\eusm C$ in $\eusm L^p$, that is, the
set of $h\in\eusm L^p$ such that for every $\epsilon>0$ there is a
$g\in\eusm C$ such that $\|h-g\|_p\le\epsilon$.   Then $\eusm U$ is
closed under addition and scalar multiplication.   Also
$\chi E\in\eusm U$ whenever $\mu E<\infty$.   \Prf\ Let $\epsilon>0$.
Set $\delta=\bover14\epsilon^{1/p}$.   Write $\Cal G$ for the family of
open sets of finite measure.   Because $\mu$ is effectively locally
finite, there is a $G\in\Cal G$ such that $\mu(E\setminus G)\le\delta$.
Let $F\subseteq G\setminus E$ be a closed set such that
$\mu F\ge\mu(G\setminus E)-\delta$;  then
$\mu(E\symmdiff(G\setminus F))\le 2\delta$.   Write
$\Cal H$ for the family of cozero sets.   Because $\frak T$ is
completely regular, $\Cal H$ is a base
for $\frak T$;  because $\Cal H$ is closed under finite unions
(4A2C(b-iii)) and $\mu$ is $\tau$-additive, there is an $H\in\Cal H$
such that
$H\subseteq G\setminus F$ and $\mu H\ge\mu(G\setminus F)-\delta$, so
that $\mu(E\symmdiff H)\le 3\delta$.   Express $H$ as $\{x:g(x)>0\}$
where $g:X\to\Bbb R$ is a continuous function.   For each $n\in\Bbb N$,
set $g_n=ng\wedge\chi X\in\eusm C$;  then

\Centerline{$|\chi E-g_n|^p\le\chi(E\symmdiff H)+(\chi H-g_n)^p$}

\noindent for every $n$, so

\Centerline{$\biggerint|\chi E-g_n|^p
\le\mu(E\symmdiff H)+\int(\chi H-g_n)^p
\to\mu(E\symmdiff H)$}

\noindent as $n\to\infty$, because $g_n\to\chi H$.   So there is an
$n\in\Bbb N$ such that
$\int|\chi E-g_n|^p\le 4\delta$, that is, $\|\chi E-g_n\|_p\le\epsilon$.
As $\epsilon$ is arbitrary, $\chi E\in\eusm U$.\ \Qed

Accordingly every simple function belongs to $\eusm U$.   But if
$f\in\eusm L^p$ and $\epsilon>0$, there is a simple function $h$ such
that $\|f-h\|_p\le\bover12\epsilon$ (244Ha);  now there is a
$g\in\eusm C$ such that $\|h-g\|_p\le\bover12\epsilon$ and
$\|f-g\|_p\le\epsilon$,
as claimed.

\medskip

{\bf (b)} This time, write $\Cal G$ for the family of open subsets of
$X$ such that $\int_Gf$ is finite, so that $\Cal G$ is an open cover of
$X$.   As $X$ is paracompact (4A2H(b-i)), there is a locally finite
family $\Cal G_0\subseteq\Cal G$ covering $X$, which must be countable
(4A2H(b-ii)).

Let $\family{G}{\Cal G_0}{\epsilon_G}$ be a family
of strictly positive real numbers such that
$\sum_{G\in\Cal G_0}\epsilon_G\le\epsilon$ (4A1P).   Since $X$ is
completely regular (4A2H(b-i)), we can apply (a) to see that, for each
$G\in\Cal G_0$, there is a continuous function $g_G:X\to\Bbb R$ such
that $\int|g_G-f\times\chi G|\le\epsilon_G$.   Next, because $X$ is
normal (4A2H(b-i)), there is a family $\family{G}{\Cal G_0}{h_G}$ of
continuous
functions from $X$ to $[0,1]$ such that $h_G\le\chi G$ for every
$G\in\Cal G_0$ and $\sum_{G\in\Cal G_0}h_G(x)=1$ for every $x\in X$
(4A2F(d-viii)).

Set $g(x)=\sum_{G\in\Cal G_0}g_G(x)h_G(x)$ for every $x\in X$.   Because
$\Cal G_0$ is locally finite, $g:X\to\Bbb R$ is continuous (4A2Bh).
Now

$$\eqalign{\int|f-g|
&=\int|\sum_{G\in\Cal G_0}(f-g_G)\times h_G|
\le\sum_{G\in\Cal G_0}\int|(f-g_G)\times h_G|\cr
&\le\sum_{G\in\Cal G_0}\int_G|f-g_G|
\le\sum_{G\in\Cal G_0}\epsilon_G
\le\epsilon,\cr}$$

\noindent as required.
}%end of proof of 415P

\allowmorestretch{468}{
\leader{415Q}{}\cmmnt{ Recall (411P) that if $(\frak A,\bar\mu)$ is a
localizable measure algebra, with Stone space $(Z,\frak S,\Tau,\nu)$,
then $\nu$ is a strictly positive completion regular quasi-Radon
measure, inner regular with respect to the open-and-closed sets (which
are all compact).   The following construction is primarily important
for Radon measure spaces
(see 416V), but is also of interest for general quasi-Radon spaces.

\medskip

\noindent}{\bf Proposition} Let $(X,\frak T,\Sigma,\mu)$ be a
quasi-Radon measure space and $(\frak A,\bar\mu)$ its measure algebra.
Let $(Z,\frak S,\Tau,\nu)$ be the Stone space of $(\frak A,\bar\mu)$.
For $E\in\Sigma$ let $E^*\subseteq Z$ be the open-and-closed set
corresponding to the image $E^{\ssbullet}$ of $E$ in $\frak A$.   Define
$R\subseteq Z\times X$ by saying that $(z,x)\in R$ iff $x\in F$ whenever
$F\subseteq X$ is closed and $z\in F^*$.   Set $Q=R^{-1}[X]$.
}%end of allowmorestretch

(a) $R$ is a closed subset of $Z\times X$.

(b) For any $E\in\Sigma$, $R[E^*]$ is the smallest closed set such that
$\mu(E\setminus R[E^*])=0$.
In particular, if $F\subseteq X$ is closed then $R[F^*]$ is the
self-supporting closed set included in $F$ such that
$\mu(F\setminus R[F^*])=0$;  and $R[Z]$ is the support of $\mu$.

(c) $Q$ is of full outer measure for $\nu$.

(d) For any $E\in\Sigma$, $R^{-1}[E]\symmdiff(Q\cap E^*)$ is negligible;
consequently $\nu^*R^{-1}[E]=\mu E$ and
$R^{-1}[E]\cap R^{-1}[X\setminus E]$ is negligible.

(e) For any $A\subseteq X$, $\nu^*R^{-1}[A]=\mu^*A$.

(f) If $(X,\frak T)$ is regular, then $R^{-1}[G]$ is relatively open in
$Q$ for every open set $G\subseteq X$, $R^{-1}[F]$ is relatively closed
in $Q$ for every closed set $F\subseteq X$ and
$R^{-1}[X\setminus E]=Q\setminus R^{-1}[E]$ for every Borel set $E\subseteq X$.

\proof{{\bf (a)}

$$R=\bigcap_{F\subseteq X\text{ is closed}}((Z\setminus F^*)\times
X)\cup(Z\times F)$$

\noindent is an intersection of closed sets, therefore closed.

\medskip

{\bf (b)} Let $\Cal G$ be the family of open sets $G\subseteq X$ such
that $\mu(E\cap G)=0$, and $G_0=\bigcup\Cal G$;  then $G_0\in\Cal G$
(414Ea).   Set $F_0=X\setminus G_0$, so that $F_0$ is the smallest
closed set such that $E\setminus F_0$ is negligible, and
$F_0^*\supseteq E^*$.   If $(z,x)\in R$ and $z\in E^*$ we must have
$x\in F_0$.   Thus $R[E^*]\subseteq F_0$.
On the other hand, if $x\in F_0$, and $G$ is an open set containing $x$,
then $G\notin\Cal G$ so $\mu(G\cap E)>0$ and $(E\cap G)^*\ne\emptyset$.
Accordingly $\{(G\cap E)^*:x\in G\in\frak T\}$ is a downwards-directed
family of non-empty open-and-closed sets in the compact space $Z$ and
has non-empty
intersection, containing a point $z$ say.   If $H\subseteq X$ is closed
and $z\in H^*$, then $X\setminus H$ is open and $z\notin(X\setminus
H)^*$, so $x$ cannot belong to $X\setminus H$, that is, $x\in H$;  as
$H$ is arbitrary, $(z,x)\in R$ and $x\in R[E^*]$;  as $x$ is arbitrary,
$R[E^*]=F_0$, as claimed.

Of course, when $E$ is actually closed, $R[E^*]=F_0\subseteq E$.
Taking $E=X$ we see that $R[Z]=R[X^*]$ is the support of $\mu$.

\medskip

{\bf (c)} If $W\in\Tau$ and $\nu W>0$, there is a non-empty
open-and-closed set $U\subseteq W$, by 322Ra, which must be of the form
$E^*$ for some $E\in\Sigma$.   By (b), $R[E^*]$ cannot be empty;  but
$E^*\subseteq W$, so $R[W]\ne\emptyset$, that is, $W\cap Q\ne\emptyset$.
As $W$ is arbitrary, $Q$ has full outer measure in $Z$.

\medskip

{\bf (d)(i)} Let $\Cal F$ be the set of closed subsets of $X$ included
in $E$.   Then $\sup_{F\in\Cal F}F^{\ssbullet}=E^{\ssbullet}$ in
$\frak A$ (412N), so $E^*\setminus\bigcup_{F\in\Cal F}F^*$ is nowhere
dense and negligible.   Now for each $F\in\Cal F$, $R[F^*]\subseteq F$,
so $Q\cap F^*\subseteq R^{-1}[F]\subseteq R^{-1}[E]$.   Accordingly

\Centerline{$Q\cap E^*\setminus R^{-1}[E]
\subseteq E^*\setminus\bigcup_{F\in\Cal F}F^*$}

\noindent is nowhere dense and negligible.

\medskip

\quad{\bf (ii)} \Quer\ Suppose, if
possible, that $\nu^*(R^{-1}[E]\setminus E^*)>0$.   Then there is an
open-and-closed set $U$ of finite measure such that
$\nu^*(R^{-1}[E]\cap U\setminus E^*)>0$ (use 412Jc).   Express
$U$ as $H^*$, where $\mu H<\infty$, and let $F\subseteq H\setminus E$ be
a closed set such that
$\mu((H\setminus E)\setminus F)<\nu^*(R^{-1}[E]\cap H^*\setminus E^*)$.
Then we must have $\nu^*(R^{-1}[E]\cap F^*)>0$.
But $R[F^*]\subseteq F\subseteq X\setminus E$ so
$F^*\cap R^{-1}[E]=\emptyset$,
which is impossible.\ \BanG\

\medskip

\quad{\bf (iii)} Putting these together,
$(Q\cap E^*)\symmdiff R^{-1}[E]$ is negligible.

\medskip

\quad{\bf (iv)} It follows at
once that (because $Z$ is a measurable envelope for $Q$)

\Centerline{$\nu^*R^{-1}[E]=\nu^*(Q\cap E^*)=\nu E^*=\mu E$.}

\noindent Moreover, applying the result to $X\setminus E$,

\Centerline{$R^{-1}[X\setminus E]\cap R^{-1}[E]
\subseteq(R^{-1}[X\setminus E]\symmdiff(Q\cap(X\setminus E)^*))
\cup(R^{-1}[E]\symmdiff(Q\cap E^*))$}

\noindent is negligible.

\medskip

{\bf (e)(i)} Take $E\in\Sigma$ such that $A\subseteq E$ and
$\mu E=\mu^*A$;  then $R^{-1}[A]\subseteq R^{-1}[E]$, so

\Centerline{$\nu^*R^{-1}[A]\le\nu^*R^{-1}[E]=\mu E=\mu^*A$.}

\medskip

\quad{\bf (ii)} \Quer\ Suppose, if possible, that
$\nu^*R^{-1}[A]<\mu^*A$.   Let $W\in\Tau$ be a measurable envelope of
$F\in\Sigma$ such that $\nu(W\symmdiff F^*)=0$.   Since

\Centerline{$\mu F=\nu F^*=\nu W<\mu^*A$,}

\noindent $\mu^*(A\setminus F)>0$;  let $G$ be a measurable envelope of
$A\setminus F$ disjoint from $F$.   Then $G^*\cap F^*=\emptyset$ so

\Centerline{$\nu(G^*\setminus W)=\nu G^*=\mu G>0$}

\noindent and there is a non-empty open-and-closed
$V\subseteq G^*\setminus W$;  let $H\in\Sigma$ be such that
$H\subseteq G$ and $V=H^*$.   In this case, $R[V]$ is closed and
$\mu(H\setminus R[V])=0$, by (b), so that $H\cap R[V]$ is measurable,
not negligible, and included in $G$.   But $H\cap R[V]\cap A$ is empty,
because $V\cap R^{-1}[A]$ is empty, so
$\mu^*(H\cap R[V]\cap A)<\mu(H\cap R[V])$, and $G$ cannot be a
measurable envelope of $A\setminus F$.\ \Bang

Thus $\nu^*R^{-1}[A]=\mu^*A$, as claimed.

\medskip

{\bf (f)} Suppose now that $(X,\frak T)$ is regular.

\medskip

\quad{\bf (i)} If $G\subseteq X$ is open, $R^{-1}[G]\cap
R^{-1}[X\setminus G]=\emptyset$.   \Prf\ If $z\in R^{-1}[G]$, then there
is an $x\in G$ such that $(z,x)\in R$.   Let $H$ be an open set
containing $x$
such that $\overline{H}\subseteq G$.   Then $x\notin X\setminus H$ so
$z\notin(X\setminus H)^*$, that is, $z\in H^*$.   But

\Centerline{$R[H^*]\subseteq R[\overline{H}^*]\subseteq
\overline{H}\subseteq G$,}

\noindent so $H^*\cap R^{-1}[X\setminus G]=\emptyset$ and
$z\notin R^{-1}[X\setminus G]$.\ \Qed

\medskip

\quad{\bf (ii)} It is easy to check that

$$\eqalign{\Cal E
&=\{E:E\subseteq X,\,R^{-1}[E]\cap R^{-1}[X\setminus E]=\emptyset\}\cr
&=\{E:E\subseteq X,\,R^{-1}[X\setminus E]=Q\setminus R^{-1}[E]\}\cr}$$

\noindent is a $\sigma$-algebra of subsets of $X$ (indeed, an algebra
closed under arbitrary unions), just because $R\subseteq Z\times X$ and
$R^{-1}[X]=Q$.   Because it contains all open sets, $\Cal E$ must
contain
all Borel sets.

\medskip

\quad{\bf (iii)} Now suppose once again that $G\subseteq X$ is open and
that $z\in R^{-1}[G]$.   As in (i) above, there is an open set
$H\subseteq
G$ such that $z\in H^*\subseteq Z\setminus R^{-1}[X\setminus G]$, so
that $z\in H^*\cap Q\subseteq R^{-1}[G]$.   As $z$ is arbitrary,
$R^{-1}[G]$ is relatively open in $Q$.

\medskip

\quad{\bf (iv)} Finally, if $F\subseteq X$ is closed,
$R^{-1}[F]=Q\setminus
R^{-1}[X\setminus F]$ is relatively closed in $Q$.
}%end of proof of 415Q

\leader{415R}{Proposition} Let $(X,\frak T,\Sigma,\mu)$ be a Hausdorff
quasi-Radon measure space and $(Z,\frak S,\Tau,\nu)$ the Stone space of
its measure algebra.   Let $R\subseteq Z\times X$ be the relation
described in 415Q.   Then

(a) $R$ is (the graph of) a function $f$;

(b) $f$ is \imp\ for the subspace measure $\nu_Q$ on $Q=\dom f$, and in
fact $\mu$ is the image measure $\nu_Qf^{-1}$;

(c) if $(X,\frak T)$ is regular, then $f$ is continuous.

\proof{{\bf (a)} If $z\in Z$ and $x$, $y\in X$ are distinct, let $G$,
$H$ be disjoint open sets containing $x$, $y$ respectively.   Then

\Centerline{$(X\setminus G)^*\cup(X\setminus H)^*
=((X\setminus G)\cup(X\setminus H))^*=Z$,}

\noindent defining $^*$ as in 415Q, so $z$ must belong to at least one
of $(X\setminus G)^*$, $(X\setminus H)^*$.   In the former case
$(z,x)\notin R$ and in the latter case $(z,y)\notin R$.   This shows
that $R$ is a
function; to remind us of its new status I will henceforth call it $f$.
The domain of $f$ is just $Q=R^{-1}[X]$.

\medskip

{\bf (b)} By 415Qd, $f$ is \imp\ for $\nu_Q$ and $\mu$.   Suppose that
$A\subseteq X$ and $f^{-1}[A]$ is in the domain $\Tau_Q$ of $\nu_Q$,
that is, is of the form $Q\cap U$ for some $U\in\Tau$.   Take any
$E\in\Sigma$
such that $\mu E>0$;  then either $\nu(E^*\cap U)>0$ or
$\nu(E^*\setminus U)>0$.   ($\alpha$) Suppose that $\nu(E^*\cap U)>0$.
Because $\nu$ is inner regular with respect to the open-and-closed sets,
there is an
$H\in\Sigma$ such that $H^*\subseteq E^*\cap U$ and $\mu H=\nu H^*>0$.
Now there is a closed set $F\subseteq E\cap H$ with $\mu F>0$.   In this
case, $f[F^*]\subseteq F\subseteq E$, by 415Qb,
while $F^*\cap Q\subseteq U\cap Q=f^{-1}[A]$,
so $f[F^*]\subseteq E\cap A$.   But this means that

\Centerline{$\mu_*(E\cap A)\ge\mu f[F^*]=\mu F>0$.}

\noindent ($\beta$) If $\nu(E^*\setminus U)>0$, then the same arguments
show that $\mu_*(E\setminus A)>0$.   ($\gamma$) Thus
$\mu_*(E\cap A)+\mu_*(E\setminus A)>0$ whenever $\mu E>0$.   Because
$\mu$ is complete and locally determined, $A\in\Sigma$ (413F(vii)).

Thus we see that $\{A:A\subseteq X,\,f^{-1}[A]\in\Tau_Q\}$ is included
in $\Sigma$, and $\mu$ is the image measure $\nu_Qf^{-1}$.

\medskip

{\bf (c)} If $\frak T$ is regular, then 415Qf tells us that $f$ is
continuous.
}%end of proof of 415R

\exercises{
\leader{415X}{Basic exercises $\pmb{>}$(a)}
%\spheader 415Xa
Let $(X,\frak T,\Sigma,\mu)$ be a quasi-Radon measure space and
$E\in\Sigma$ an atom for the measure.   Show that there is a closed set
$F\subseteq E$
such that $\mu F>0$ and $F$ is an atom of $\Sigma$, in the sense that
the only measurable subsets of $F$ are $\emptyset$ and $F$.
\Hint{414G.}   Show that $\mu$ is atomless iff all countable subsets of
$X$ are negligible.
%- %  maybe look for other versions, e.g. for \tau-additive measures

\spheader 415Xb Let
$\langle(X_i,\frak T_i,\Sigma_i,\mu_i)\rangle_{i\in I}$ be any family of
quasi-Radon measure spaces.   Show that the direct
sum measure on $X=\{(x,i):i\in I,\,x\in X_i\}$ is a
quasi-Radon measure when $X$ is given its disjoint union topology.
%- %

\spheader 415Xc Let $\frak S$ be the {\bf right-facing Sorgenfrey
topology} or {\bf lower limit topology} on $\Bbb R$, that is, the
topology generated by the half-open intervals of
the form $\coint{\alpha,\beta}$.   Show that Lebesgue measure is
completion regular and
quasi-Radon for $\frak S$.   \Hint{114Yj or 221Yb, or 419L.}
%- %

\spheader 415Xd Let $X$ be a topological space and $\mu$ a complete
measure on $X$, and suppose that there is a conegligible closed
measurable set
$Y\subseteq X$ such that the subspace measure on $Y$ is quasi-Radon.
Show that $\mu$ is quasi-Radon.
%- %

\spheader 415Xe Let $(X,\frak T,\Sigma,\mu)$ be a quasi-Radon measure
space.   Show that $\mu$ is inner regular with respect to the family of
self-supporting closed sets included in open sets of finite measure.
%- %

\spheader 415Xf Let $(X,\frak T,\Sigma,\mu)$ be a quasi-Radon measure
space.   Show that whenever
$E\in\Sigma$ and $\epsilon>0$ there is an open
set $G$ such that $\mu G\le\mu E+\epsilon$ and $E\setminus G$ is
negligible.
%- %

\spheader 415Xg Find a compact Hausdorff quasi-Radon measure space which
is not $\sigma$-finite.
%-

\spheader 415Xh Let $(X,\frak T,\Sigma,\mu)$ be an atomless quasi-Radon
measure space which is outer regular with respect to the open sets.
Show that it is $\sigma$-finite.   \Hint{if not, take a decomposition
$\familyiI{X_i}$ in which every $X_i$ except one is self-supporting, and
a set $A$ meeting every $X_i$ in just one point.}
%415A

\spheader 415Xi Let $(X,\Sigma,\mu)$ be a $\sigma$-finite measure space
in which $\Sigma$ is countably generated as a $\sigma$-algebra.   Show
that, for a suitable topology on $X$, the completion of $\mu$ is a
quasi-Radon measure.   \Hint{take the topology generated by a countable
subalgebra of $\Sigma$, and use the arguments of 415D.}
%415D

\spheader 415Xj Let $\familyiI{(X_i,\frak T_i,\Sigma_i,\mu_i)}$
be a family of quasi-Radon probability spaces such
that every $\mu_i$ is strictly positive, and $\lambda$ the product
measure on $X=\prod_{i\in I}X_i$.   Show that if every $\frak T_i$ has a
countable network, $\lambda$ is a quasi-Radon measure.
%415E

\spheader 415Xk Let $\familyiI{X_i}$ be a family of separable metrizable
spaces, and $\mu$ a quasi-Radon measure on $X=\prod_{i\in I}X_i$.   Show
that $\mu$ is completion regular iff every self-supporting closed set in
$X$ is determined by coordinates in a countable set.  \Hint{4A2Eb.}
%415E

\spheader 415Xl Find two quasi-Radon measures $\mu$, $\nu$ on the unit
interval such that $\mu G\le\nu G$ for every open set $G$ but there is a
closed set $F$ such that $\nu F<\mu F$.
%415G

\spheader 415Xm Let $X$ be a topological space and $\mu$, $\nu$ two
quasi-Radon measures on $X$.   (i) Suppose that $\mu F=\nu F$ whenever
$F\subseteq X$ is closed and both $\mu F$ and $\nu F$ are finite.   Show
that $\mu=\nu$.
(ii) Suppose that $\mu G=\nu G$ whenever $G\subseteq X$ is open and both
$\mu G$ and $\nu G$ are finite.   Show that $\mu=\nu$.
%415H

\spheader 415Xn In 415L, write $\tilde\mu_0$ for the c.l.d.\ version of
$\mu_0$ (213E).   Show that $\mu$ extends $\tilde\mu_0$.   Show that
$\tilde\mu_0$ is $\tau$-additive and inner regular with respect to the
closed sets.
%415L

\spheader 415Xo Let $(X,\frak T,\Sigma,\mu)$ be a $\sigma$-finite
paracompact Hausdorff quasi-Radon measure space, and
$f\in\eusm L^0(\mu)$ a locally integrable function.   Show that for any
$\epsilon>0$ there is a continuous function $g:X\to\Bbb R$ such that
$\int|f-g|\le\epsilon$.
%415P

\sqheader 415Xp Let $(X,\frak T,\Sigma,\mu)$ be a $\sigma$-finite
completely regular quasi-Radon measure space.   (i) Show that for every
$E\in\Sigma$ there is an $F$ in the Baire $\sigma$-algebra $\CalBa(X)$ of $X$ such that $\mu(E\symmdiff F)=0$.   \Hint{start with an open set $E$ of finite measure.}   (ii) Show that for every $\Sigma$-measurable
function $f:X\to\Bbb R$ there is a $\CalBa(X)$-measurable function equal
almost everywhere to $f$.
%+

\spheader 415Xq Let $(X,\Sigma,\mu)$ be a measure space and $f$ a
$\mu$-integrable real-valued function.   Show that there is a unique
quasi-Radon measure $\lambda$ on $\Bbb R$ such that $\lambda\{0\}=0$ and
$\lambda\coint{\alpha,\infty}=\mu^*\{x:x\in\dom f$, $f(x)\ge\alpha\}$,
$\lambda\ocint{-\infty,-\alpha}=\mu^*\{x:x\in\dom f$, $f(x)\le-\alpha\}$
whenever $\alpha>0$;  and that $\int h\,d\lambda=\int hfd\mu$ whenever
$h\in\eusm L^0(\lambda)$ and $h(0)=0$ and either integral is defined in
$[-\infty,\infty]$.   \Hint{set $\lambda E=\mu^*f^{-1}[E\setminus\{0\}]$
for Borel sets $E\subseteq\Bbb R$, and use 414Mb, 414O and
235Gb\formerly{2{}35I}.}
%+

\spheader 415Xr Let $(X,\frak T,\Sigma,\mu)$ be a $\sigma$-finite
quasi-Radon measure space with $\mu X>0$.   Show that there is a
quasi-Radon probability measure on $X$ with the same measurable sets and the same negligible sets as $\mu$.
%415B

\spheader 415Xs Let $(X,\frak T,\Sigma,\mu)$ be a quasi-Radon measure
space, $(\frak A,\bar\mu)$ its measure algebra, and $\frak A^f$ the ideal
$\{a:a\in\frak A$, $\bar\mu a<\infty\}$.   Show that
$\{G^{\ssbullet}:G\in\frak T$, $\mu G<\infty\}$ is dense in $\frak A^f$ for
the strong measure-algebra topology (323Ad).
%out of order query;  used in 443B

\spheader 415Xt\dvAnew{2010}
Find a second-countable Hausdorff topological space $X$
with a $\tau$-additive Borel probability measure which is not inner 
regular with respect to the closed sets.
%out of order query

\leader{415Y}{Further exercises (a)}
%\spheader 415Ya
Give an example of two quasi-Radon measures $\mu$, $\nu$
on $\Bbb R$ such that their sum, as defined in
234G\footnote{Formerly 1{}12Xe.}, %until 2008
is not effectively locally finite, therefore not a quasi-Radon measure.
%- %

\spheader 415Yb
Show that any quasi-Radon measure space is isomorphic, as topological
measure space, to a subspace of a compact quasi-Radon measure space.
\Hint{if $X$ is a T$_1$ quasi-Radon measure space, let $\hat{X}$ be its
Wallman compactification ({\smc Engelking 89}, 3.6.21).}
%- %

\spheader 415Yc Let $(X,\frak T,\Sigma,\mu)$ be a quasi-Radon measure
space.   Show that the following are equiveridical:  (i) $\mu$ is outer
regular with respect to the open sets;  (ii) every negligible subset of
$X$ is included in an open set of finite measure;   (iii)
$\{x:\mu\{x\}=0\}$ can be covered by a sequence of open sets of finite
measure.
%- %

\spheader 415Yd Show that $+:\Bbb R\times\Bbb R\to\Bbb R$ is continuous
for the right-facing Sorgenfrey topology.
%415Xc

\spheader 415Ye Let $r\ge 1$.   On $\BbbR^r$ let $\frak S$ be the
topology generated by the half-open intervals $\coint{a,b}$ where $a$,
$b\in\BbbR^r$ (definition:  115Ab).   (i) Show that $\frak S$ is the
product topology if each
factor is given the right-facing Sorgenfrey topology (415Xc).   (ii)
Show that Lebesgue measure is quasi-Radon for $\frak S$.   \Hint{induce
on $r$.   See also 417Yi.}
%415Xc %mt41bits

\spheader 415Yf Let $Y\subseteq[0,1]$ be a set of full outer measure and
zero inner measure for Lebesgue measure $\mu$.   Give $[0,1]$ the
topology $\frak T$ generated by $\frak S\cup\{Y\}$ where $\frak S$ is
the usual topology.    Show that the subspace measure $\nu=\mu_Y$ is
quasi-Radon for the subspace topology $\frak T_Y$, but that there is no
measure $\lambda$ on $X$ which is quasi-Radon for $\frak T$ and such
that the subspace measure $\lambda_Y$ is equal to $\nu$.
%415J

\spheader 415Yg Find a base $\Cal U$ for the topology of
$X=\{0,1\}^{\Bbb N}$ and two totally finite (quasi-\nobreak)Radon
measures
$\mu$, $\nu$ on $X$ such that $G\cap H\in\Cal U$ for all $G$,
$H\in\Cal U$, $\mu G\le\nu G$ for every $G\in\Cal U$, but $\nu X<\mu X$.
%415G

\spheader 415Yh Let $X$ be a topological space and $\Cal G$ an open
cover of $X$.   Suppose that for each $G\in\Cal G$ we are given a
quasi-Radon measure $\mu_G$ on $G$ such that $\mu_G(U)=\mu_H(U)$
whenever $G$, $H\in\Cal G$ and $U\subseteq G\cap H$ is open.   Show that
there is a unique quasi-Radon measure on $X$ such that each $\mu_G$ is
the subspace measure on $G$.   \Hint{if $\family{G}{\Cal G}{\mu_G}$ is a
maximal family with the given properties, then $\Cal G$ is
upwards-directed.}
%415H

\spheader 415Yi Let $(X,\Sigma,\mu)$ be a measure space and $\frak T$ a
topology on $X$, and suppose that there is a family
$\Cal U\subseteq\Sigma\cap\frak T$ such that

\inset{$\mu U<\infty$ for
every $U\in\Cal U$,

for every $U\in\Cal U$,
$\frak T\cap\Sigma\cap\Cal PU$
is a base for the subspace topology of $U$,

if $\Cal G$ is an
upwards-directed family in $\frak T\cap\Sigma$ and
$\bigcup\Cal G\in\Cal U$, then
$\mu(\bigcup\Cal G)=\sup_{G\in\Cal G}\mu G$,

$\mu$ is inner
regular with respect to the closed sets,

if $E\in\Sigma$ and
$\mu E>0$ then there is a $U\in\Cal U$ such that $\mu(E\cap U)>0$.}

\noindent Show that $\mu$ has an extension to a quasi-Radon measure on $X$.
%415L

\spheader 415Yj Let $(X,\frak T,\Sigma,\mu)$ be a quasi-Radon measure
space such that $\frak T$ is normal (but not necessarily Hausdorff or
regular).   Show that if $1\le p<\infty$, $f\in\eusm L^p(\mu)$ and
$\epsilon>0$, there is a bounded continuous function $g:X\to\Bbb R$ such
that $\|f-g\|_p\le\epsilon$ and $\{x:g(x)\ne 0\}$ has finite measure.
%415P

\spheader 415Yk Let $(X,\frak T,\Sigma,\mu)$ be a completely regular
quasi-Radon measure space and suppose that we are given a uniformity
defining the topology $\frak T$.
Show that if $1\le p<\infty$, $f\in\eusm L^p(\mu)$ and $\epsilon>0$,
there is a bounded uniformly continuous function $g:X\to\Bbb R$ such
that $\|f-g\|_p\le\epsilon$ and $\{x:g(x)\ne 0\}$ has finite measure.
%415P

\spheader 415Yl Let $(X,\frak T,\Sigma,\mu)$ be a completely regular
quasi-Radon measure space and $\tau$ an extended Fatou norm on
$L^0(\mu)$ such that (i) $\tau\restr L^{\tau}$ is an
order-continuous norm (ii) whenever $E\in\Sigma$ and $\mu E>0$ there is
an open set $G$ such that $\mu(E\cap G)>0$ and
$\tau(\chi G^{\ssbullet})<\infty$.   Show that
$L^{\tau}\cap\{f^{\ssbullet}:f:X\to\Bbb R$ is continuous$\}$ is
norm-dense in $L^{\tau}$.
%415P

\spheader 415Ym Let $(X,\frak T,\Sigma,\mu)$ be a quasi-Radon measure
space.   Show that $\mu$ is a compact measure (definition:  342Ac or
451Ab) iff
there is a locally compact topology $\frak S$ on $X$ such that
$(X,\frak S,\Sigma,\mu)$ is quasi-Radon.
%+

}%end of exercises

\endnotes{\Notesheader{415}
415B is particularly important because a very high proportion of the
quasi-Radon measure spaces we study are actually subspaces of Radon
measure spaces.   I would in fact go so far as to say that when you have
occasion to
wonder whether all quasi-Radon measure spaces have a property, you
should as a matter of habit look first at subspaces of Radon measure
spaces;  if
the answer is affirmative for them, you will have most of what you want,
even if the generalization to arbitrary quasi-Radon spaces gives
difficulties.
Of course the reverse phenomenon can also occur.   Stone spaces (411P)
can be thought of as quasi-Radon compactifications of Radon measure
spaces (416V).   But this is relatively rare.   Indeed the reason why I
give so few examples of quasi-Radon spaces at this point is just that
the natural ones arise from Radon measure spaces.   Note however that
the quasi-Radon product of an uncountable family of Radon
probability spaces need not be
Radon (see 417Xq), so that 415E here and 417O below are sources of
non-Radon quasi-Radon
measure spaces.   Density and lifting topologies can also provide us
with quasi-Radon measure spaces (453Xd, 453Xg).

415K is the second in a series of inner-regular-extension theorems;
there will be a third in 416J.

I have been saying since Volume 1 that the business of measure theory,
since Lebesgue's time, has been to measure as many sets and integrate as
many functions as possible.   I therefore take seriously any theorem
offering a
canonical extension of a measure.   415L and its corollaries can all be
regarded as improvement theorems, showing that a good measure can be
made even better.   We have already had such improvement theorems in
Chapter 21:  the completion and c.l.d.\ version of a measure (212C,
213E).   In all such
theorems we need to know exactly what effect our improvement is having
on the other constructions we are interested in;  primarily, the measure
algebra and the function spaces.   The machinery of Chapter 36 shows
that if we understand the measure algebra(s) involved then the function
spaces will give us no further surprises.   Completion of a measure does
not affect the
measure algebra at all (322Da).   Taking the c.l.d.\ version does not
change $\frak A^f=\{a:\bar\mu a<\infty\}$ or $L^1$ (213Fc, 213G, 322Db,
366H), but can affect the rest of the measure algebra and
therefore $L^0$ and $L^{\infty}$. In this respect, what we might call
the `quasi-Radon version' behaves like
the c.l.d.\ version (as could be expected, since the quasi-Radon version
must itself be complete and locally determined;  cf.\ 415Xn).
The archetypal application of 415L is 415N.   We shall see later how
Baire measures arise naturally when studying Banach spaces of continuous
functions (436E).   415N will be one of the keys to applying the
general theory of
topological measure spaces in such contexts.   A virtue of Baire
measures is that inner regularity with respect to closed sets comes
almost free
(412D);  but there can be unsurmountable difficulties if we wish to
extend them to Borel measures (439M), and it is important to know
that $\tau$-additivity, even in the relatively weak form allowed by the
definition I use here (411C), is often enough to give a canonical
extension to
a well-behaved measure defined on every Borel set.   In 415C we have
inner regularity for a different reason, and the measure is already
known to be
defined on every Borel set, so in fact the quasi-Radon version of the
measure is just the c.l.d.\ version.

One interpretation of the Lifting Theorem is that for a complete
strictly localizable measure space $(X,\Sigma,\mu)$ there is a function
$g:X\to Z$, where $Z$ is the Stone space of the measure algebra of
$\mu$, such
that $E\symmdiff g^{-1}[E^*]$ is negligible for every $E\in\Sigma$,
where $E^*\subseteq Z$ is the open-and-closed set corresponding to the
image of $E$ in the measure algebra (341Q).   For a Hausdorff
quasi-Radon measure
space we have a function $f:Q\to X$, where $Q$ is a dense subset of $Z$,
such that $(Q\cap E^*)\symmdiff f^{-1}[E]$ is negligible for every
$E\in\Sigma$ (415Qd, 415R);  moreover, there is a canonical construction
for this function.   For completeness' sake, I have given the result for
general, not necessarily Hausdorff, spaces $X$ (415Q);  but evidently it
will be of greatest interest for regular Hausdorff spaces (415Rc).
Perhaps I should remark that in the most important applications, $Q$ is
the whole
of $Z$ (416Xx).   Of course the question arises, whether $fg$ can be
the identity.   ($Z$ typically has larger cardinal than $X$, so asking
for $gf$ to be the identity is a bit optimistic.)   This is in fact an
important question;  I will return to it in 453M.
}%end of notes

\discrpage

