\frfilename{mt215.tex} 
\versiondate{13.11.13} 
\copyrightdate{2000} 
      
\def\chaptername{Taxonomy of measure spaces} 
\def\sectionname{$\sigma$-finite spaces and the principle of exhaustion} 
      
\newsection{215} 
      
I interpolate a short section to deal with some useful facts which might 
get lost if buried in one of the longer sections of this chapter.   The 
great majority of the applications of measure theory involve 
$\sigma$-finite spaces, to the point that many authors skim over any 
others.   I 
myself prefer to signal the importance of such concepts by explicitly 
stating just which theorems apply only to the restricted class of 
spaces.   But undoubtedly some facts about $\sigma$-finite spaces need 
to be grasped early on.   In 215B I give a list of properties 
characterizing $\sigma$-finite spaces.   Some of these make better sense 
in the light of the principle of exhaustion (215A).   I take the 
opportunity to include a fundamental fact about atomless measure spaces 
(215D). 
      
\leader{215A}{The principle of \dvrocolon{exhaustion}}\cmmnt{ The 
following is an example of the use of one of the most important methods 
in measure theory. 
      
\medskip 
      
\noindent}{\bf Lemma} Let $(X,\Sigma,\mu)$ be any measure space and 
$\Cal E\subseteq\Sigma$ a non-empty set such that 
$\sup_{n\in\Bbb N}\mu F_n$ is finite for every non-decreasing sequence 
$\sequencen{F_n}$ in $\Cal E$. 
      
(a) There is a 
non-decreasing sequence $\sequencen{F_n}$ in $\Cal E$ such that, for 
every $E\in\Sigma$, {\it either} there is an $n\in\Bbb N$ such that 
$E\cup F_n$ is not included in any member of $\Cal E$ {\it or}, setting 
$F=\bigcup_{n\in\Bbb N}F_n$, 
      
\Centerline{$\lim_{n\to\infty}\mu(E\setminus F_n) 
=\mu(E\setminus F)=0$.} 
      
\noindent In particular, if $E\in\Cal E$ and 
$E\supseteq F$, then 
$E\setminus F$ is negligible. 
      
(b) If $\Cal E$ is upwards-directed, then there is a non-decreasing 
sequence $\sequencen{F_n}$ in $\Cal E$ such that, setting 
$F=\bigcup_{n\in\Bbb N}F_n$, $\mu F=\sup_{E\in\Cal E}\mu E$ and 
$E\setminus F$ is negligible for every $E\in\Cal E$, so that $F$ is an 
essential supremum of $\Cal E$ in $\Sigma$\cmmnt{ in the sense of 
211G}. 
      
(c) If the union of any non-decreasing sequence in $\Cal E$ belongs to 
$\Cal E$, then there is an $F\in\Cal E$ such that $E\setminus F$ is 
negligible whenever $E\in\Cal E$ and $F\subseteq E$. 
      
\proof{{\bf (a)} Choose $\sequencen{F_n}$, $\sequencen{\Cal E_n}$ and 
$\sequencen{u_n}$ inductively, as follows.   Take $F_0$ to be any member 
of $\Cal E$.   Given $F_n\in\Cal E$, set 
$\Cal E_n=\{E:F_n\subseteq E\in\Cal E\}$ and 
$u_n=\sup\{\mu E:E\in\Cal E_n\}$ in $[0,\infty]$, and choose 
$F_{n+1}\in\Cal E_n$ such that $\mu F_{n+1}\ge\min(n,u_n-2^{-n})$; 
continue. 
      
Observe that this construction yields a non-decreasing sequence 
$\sequencen{F_n}$ in $\Cal E$.   Since $\Cal E_{n+1}\subseteq\Cal E_n$ 
for every $n$, $\sequencen{u_n}$ is non-increasing, and has a limit $u$ 
in $[0,\infty]$.   Since $\min(n,u-2^{-n})\le\mu F_{n+1}\le u_n$ for 
every $n$, $\lim_{n\to\infty}\mu F_n=u$.   Our hypothesis on $\Cal E$ 
now tells us that $u$ is finite. 
      
If $E\in\Sigma$ is such that for every $n\in\Bbb N$ there is an 
$E_n\in\Cal E$ such that 
$E\cup F_n\subseteq E_n$, then $E_n\in\Cal E_n$, so 
      
\Centerline{$\mu F_n\le\mu(E\cup F_n)\le\mu E_n\le u_n$} 
      
\noindent for every $n$, and $\lim_{n\to\infty}\mu(E\cup F_n)=u$.   But 
this means that 
      
\Centerline{$\mu(E\setminus F) 
\le\lim_{n\to\infty}\mu(E\setminus F_n) 
=\lim_{n\to\infty}\mu(E\cup F_n)-\mu F_n 
=0$,} 
      
\noindent as stated.   In particular, this is so if $E\in\Cal E$ and 
$E\supseteq F$. 
      
\medskip 
      
{\bf (b)} Take $\sequencen{F_n}$ from (a).   If $E\in\Cal E$, then 
(because $\Cal E$ is upwards-directed) $E\cup F_n$ is included in some 
member of $\Cal E$ for every $n\in\Bbb N$;  so we must have the second 
alternative of (a), and $E\setminus F$ is negligible.    It follows that 
      
\Centerline{$\sup_{E\in\Cal E}\mu E 
\le\mu F 
=\lim_{n\to\infty}\mu F_n 
\le\sup_{E\in\Cal E}\mu E$,} 
      
\noindent so $\mu F=\sup_{E\in\Cal E}\mu E$. 
      
If $G$ is any measurable set such that $E\setminus F$ is negligible for 
every $E\in\Cal E$, then $F_n\setminus G$ is negligible for every $n$, 
so that $F\setminus G$ is negligible;  thus $F$ is an essential supremum 
for $\Cal E$. 
      
\medskip 
      
{\bf (c)} Again take $\sequencen{F_n}$ from (a), and set 
$F=\bigcup_{n\in\Bbb N}E_n$.   Our hypothesis now is that $F\in\Cal E$, 
so has both the properties declared. 
}%end of proof of 215A 
      
\leader{215B}{}\cmmnt{ $\sigma$-finite spaces are so important that I 
think it is worth spelling out the following facts. 
      
\medskip 
      
\noindent}{\bf Proposition} Let $(X,\Sigma,\mu)$ be a semi-finite 
measure space.   Write $\Cal N$ for the family of $\mu$-negligible sets 
and $\Sigma^f$ for the family of measurable sets of finite measure. 
Then the following are equiveridical: 
      
(i) $(X,\Sigma,\mu)$ is $\sigma$-finite; 
      
(ii) every disjoint family in $\Sigma^f\setminus\Cal N$ is countable; 
      
(iii) every disjoint family in $\Sigma\setminus\Cal N$ is countable; 
      
(iv) for every $\Cal E\subseteq\Sigma$ there is a countable set 
$\Cal E_0\subseteq\Cal E$ such that $E\setminus\bigcup\Cal E_0$ is 
negligible for every $E\in\Cal E$; 
      
(v) for every non-empty upwards-directed $\Cal E\subseteq\Sigma$ there 
is a non-decreasing sequence $\sequencen{F_n}$ in $\Cal E$ such that 
$E\setminus\bigcup_{n\in\Bbb N}F_n$ is negligible for every 
$E\in\Cal E$; 
      
(vi) for every non-empty $\Cal E\subseteq\Sigma$, there is a 
non-decreasing sequence $\sequencen{F_n}$ in $\Cal E$ such that 
$E\setminus\bigcup_{n\in\Bbb N}F_n$ is negligible whenever $E\in\Cal E$ 
and $E\supseteq F_n$ for every $n\in\Bbb N$; 
      
(vii) {\it either} $\mu X=0$ {\it or} there is a probability measure 
$\nu$ on $X$ with the same domain and the same negligible sets as $\mu$; 
      
(viii) there is a measurable integrable function $f:X\to\ocint{0,1}$; 
      
(ix) either $\mu X=0$ or there is a measurable function 
$f:X\to\ooint{0,\infty}$ such that $\int fd\mu=1$. 
      
\proof{{\bf (i)$\Rightarrow$(vii) and (viii)} If $\mu X=0$, (vii) is 
trivial and we can take $f=\chi X$ in (viii).   Otherwise, let 
$\sequencen{E_n}$ be a disjoint sequence in 
$\Sigma^f$ covering $X$.   Then it is easy to see that there is a 
sequence $\sequencen{\alpha_n}$ of strictly positive real numbers such 
that $\sum_{n=0}^{\infty}\alpha_n\mu E_n=1$.   Set 
$\nu E=\sum_{n=0}^{\infty}\alpha_n\mu(E\cap E_n)$ for $E\in\Sigma$; 
then $\nu$ is a probability measure with domain $\Sigma$ and the same 
negligible sets as $\mu$.   Also 
$f=\sum_{n=0}^{\infty}\min(1,\alpha_n)\chi E_n$ is a strictly positive 
measurable integrable function. 
      
\medskip 
      
{\bf (vii)$\Rightarrow$(vi) and (v)} Assume (vii), and let $\Cal E$ be a 
non-empty family of measurable sets.   If $\mu X=0$ then (vi) and (v) 
are certainly true.   Otherwise, let $\nu$ be a probability measure with 
domain $\Sigma$ and the same negligible sets as $\mu$.   Since 
$\sup_{E\in\Cal E}\nu E\le 1$ is finite, we can apply 215Aa to find a 
non-decreasing sequence $\sequencen{F_n}$ in $\Cal E$ such that 
$E\setminus\bigcup_{n\in\Bbb N}F_n$ is negligible whenever $E\in\Cal E$ 
includes $\bigcup_{n\in\Bbb N}F_n$;  and if $\Cal E$ is 
upwards-directed, $E\setminus\bigcup_{n\in\Bbb N}F_n$ will be negligible 
for every $E\in\Cal E$, as in 215Ab. 
      
\medskip 
      
{\bf (vi)$\Rightarrow$(iv)} Assume (vi), and let $\Cal E$ be any subset 
of $\Sigma$.   Set 
      
\Centerline{$\Cal H=\{\bigcup\Cal E_0:\Cal E_0\subseteq\Cal E$ is 
countable$\}$.} 
      
\noindent By (vi), there is a sequence $\sequencen{H_n}$ in $\Cal H$ 
such that $H\setminus\bigcup_{n\in\Bbb N}H_n$ is negligible whenever 
$H\in\Cal H$ and $H\supseteq H_n$ for every $n\in\Bbb N$.   Now we can 
express each $H_n$ as $\bigcup\Cal E'_n$, where 
$\Cal E'_n\subseteq\Cal E$ is countable;  setting 
$\Cal E_0=\bigcup_{n\in\Bbb N}\Cal E'_n$, $\Cal E_0$ is countable. 
If $E\in\Cal E$, then 
$E\cup\bigcup_{n\in\Bbb N}H_n=\bigcup(\{E\}\cup\Cal E_0)$ belongs to 
$\Cal H$ and includes every $H_n$, so that 
$E\setminus\bigcup\Cal E_0=E\setminus\bigcup_{n\in\Bbb N}H_n$ is 
negligible.   So $\Cal E_0$ has the property we need, and (iv) is true. 
      
\medskip 
      
{\bf (iv)$\Rightarrow$(iii)} Assume (iv).   If $\Cal E$ is a disjoint 
family in $\Sigma\setminus\Cal N$, take a countable 
$\Cal E_0\subseteq\Cal E$ such that $E\setminus\bigcup\Cal E_0$ is 
negligible for every $E\in\Cal E$.   Then $E=E\setminus\bigcup\Cal E_0$ 
is negligible for every $E\in\Cal E\setminus\Cal E_0$;  but this just 
means that $\Cal E\setminus\Cal E_0$ is empty, so that 
$\Cal E=\Cal E_0$ is countable. 
      
\medskip 
      
{\bf (iii)$\Rightarrow$(ii)} is trivial. 
      
\medskip 
      
{\bf (ii)$\Rightarrow$(i)} Assume (ii).   Let $\frak P$ be the set of 
all disjoint subsets of $\Sigma^f\setminus\Cal N$, ordered by 
$\subseteq$.   Then $\frak P$ is a partially ordered set, not empty (as 
$\emptyset\in \frak P$), and if $\frak Q\subseteq \frak P$ is non-empty 
and totally ordered then it has an upper bound in $\frak P$.   \Prf\ Set 
$\Cal E=\bigcup \frak Q$, the union of all the disjoint families 
belonging to $\frak Q$.   If $E\in\Cal E$ then $E\in\Cal C$ for some 
$\Cal C\in \frak Q$, so $E\in\Sigma^f\setminus\Cal N$.  If $E$, 
$F\in\Cal E$ and $E\ne F$, then there are $\Cal C$, $\Cal D\in \frak Q$ 
such that $E\in\Cal C$, $F\in\Cal D$;  now $\frak Q$ is totally ordered, 
so one of $\Cal C$, $\Cal D$ is larger than the other, and in either 
case $\Cal C\cup\Cal D$ is a member of $\frak Q$ containing both $E$ and 
$F$.   But since any member of $\frak Q$ is a disjoint collection of 
sets, $E\cap F=\emptyset$.   As $E$ and $F$ are arbitrary, $\Cal E$ is 
a disjoint family of sets and belongs to $\frak P$.   And of course 
$\Cal C\subseteq\Cal E$ for every $\Cal C\in \frak Q$, so $\Cal E$ is an 
upper bound for $\frak Q$ in $\frak P$.\ \Qed 
      
By Zorn's Lemma (2A1M), $\frak P$ has a maximal element $\Cal E$ say. 
By (ii), $\Cal E$ must be countable, so $\bigcup\Cal E\in\Sigma$.   Now 
$H=X\setminus\bigcup\Cal E$ is negligible.   \Prf\Quer\ Suppose, if 
possible, otherwise.   Because $(X,\Sigma,\mu)$ is semi-finite, there is 
a set $G$ of finite measure such that $G\subseteq H$ and $\mu G>0$, that 
is, $G\in\Sigma^f\setminus\Cal N$ and $G\cap E=\emptyset$ for every 
$E\in\Cal E$.  But this means that $\{G\}\cup\Cal E$ is a member of 
$\frak P$ strictly larger than $\Cal E$, which is supposed to be 
impossible.\ \Bang\Qed 
      
Let $\sequencen{X_n}$ be a sequence running over $\Cal E\cup\{H\}$. 
Then $\sequencen{X_n}$ is a cover of $X$ by a sequence of measurable 
sets of finite measure, so $(X,\Sigma,\mu)$ is $\sigma$-finite. 
      
\medskip 
      
{\bf (v)$\Rightarrow$(i)} If (v) is true, then we have a sequence 
$\sequencen{E_n}$ in $\Sigma^f$ such that 
$E\setminus\bigcup_{n\in\Bbb N}E_n$ is negligible for every 
$E\in\Sigma^f$.   Because $\mu$ is semi-finite, 
$X\setminus\bigcup_{n\in\Bbb N}E_n$ must be negligible, so $X$ is 
covered by a countable family of sets of finite measure and $\mu$ is 
$\sigma$-finite. 
      
\medskip 
      
{\bf (viii)$\Rightarrow$(ix)} If $\mu X=0$ this is trivial.   Otherwise, 
if $f$ is a strictly positive measurable integrable function, then 
$c=\int f>0$ (122Rc), so $\Bover1cf$ is a strictly positive measurable 
function with integral $1$. 
      
{\bf (ix)$\Rightarrow$(i)} If $f:X\to\ooint{0,\infty}$ is measurable and 
integrable, $\sequencen{\{x:f(x)\ge 2^{-n}\}}$ is a sequence of sets of 
finite measure covering $X$. 
}%end of proof of 215B 
      
\leader{215C}{Corollary} Let $(X,\Sigma,\mu)$ be a $\sigma$-finite 
measure space, and suppose that $\Cal E\subseteq\Sigma$ is any non-empty 
set. 
      
(a) There is a non-decreasing sequence $\sequencen{F_n}$ in $\Cal E$ 
such that, for every $E\in\Sigma$, {\it either} there is an $n\in\Bbb N$ 
such that $E\cup F_n$ is not included in any member of $\Cal E$ {\it or} 
$E\setminus\bigcup_{n\in\Bbb N}F_n$ is negligible. 
      
(b) If $\Cal E$ is upwards-directed, then there is a non-decreasing 
sequence $\sequencen{F_n}$ in $\Cal E$ such that 
$\bigcup_{n\in\Bbb N}F_n$ is an essential supremum of $\Cal E$ in 
$\Sigma$. 
      
(c) If the union of any non-decreasing sequence in $\Cal E$ belongs to 
$\Cal E$, then there is an $F\in\Cal E$ such that $E\setminus F$ is 
negligible whenever $E\in\Cal E$ and $F\subseteq E$. 
      
\proof{ By 215B, there is a totally finite measure $\nu$ on $X$ with the 
same measurable sets and the same negligible sets as $\mu$.   Since 
$\sup_{E\in\Cal E}\nu E$ is finite, we can apply 215A to $\nu$ to obtain 
the results. 
}%end of proof of 215C 
      
\leader{215D}{}\cmmnt{ As a further example of the use of the 
principle of exhaustion, I give a fundamental fact about atomless 
measure spaces. 
      
\medskip 
      
\noindent}{\bf Proposition} Let $(X,\Sigma,\mu)$ be an atomless measure 
space.   If $E\in\Sigma$ and $0\le\alpha\le\mu E<\infty$, there is an 
$F\in\Sigma$ such that $F\subseteq E$ and $\mu F=\alpha$. 
      
\proof{{\bf (a)} We need to know that if $G\in\Sigma$ is non-negligible 
and $n\in\Bbb N$, then there is an $H\subseteq G$ such that 
$0<\mu H\le 2^{-n}\mu G$.   \Prf\ Induce on $n$.   For $n=0$ this is 
trivial.   For the inductive step to $n+1$, use the inductive hypothesis 
to find $H\subseteq G$ such that $0<\mu H\le 2^{-n}\mu G$.   Because 
$\mu$ is atomless, there is an $H'\subseteq H$ such that $\mu H'$, 
$\mu(H\setminus H')$ are both defined and non-zero.   Now at least one 
of them has measure less than or equal to $\bover12\mu H$, so gives us a 
subset of $G$ of non-zero measure less than or equal to 
$2^{-n-1}\mu G$.\ \Qed 
      
It follows that if $G\in\Sigma$ has non-zero finite measure and 
$\epsilon>0$, there is a measurable set $H\subseteq G$ such that 
$0<\mu H\le\epsilon$. 
      
\medskip 
      
{\bf (b)} Let $\Cal H$ be the family of all those $H\in\Sigma$ such that 
$H\subseteq E$ and $\mu H\le\alpha$.   If $\sequencen{H_n}$ is any 
non-decreasing sequence in $\Cal H$, then 
$\mu(\bigcup_{n\in\Bbb N}H_n)=\lim_{n\to\infty}\mu H_n\le\alpha$, so 
$\bigcup_{n\in\Bbb N}H_n\in\Cal H$.   So 215Ac tells us that there is an 
$F\in\Cal H$ such that $H\setminus F$ is negligible whenever 
$H\in\Cal H$ and $F\subseteq H$.   \Quer\ Suppose, if possible, that 
$\mu F<\alpha$.   By (a), there is an 
$H\subseteq E\setminus F$ such that $0<\mu H\le\alpha-\mu F$.   But in 
this case $H\cup F\in\Cal H$ and $\mu((H\cup F)\setminus F)>0$, which is 
impossible.\ \Bang 
      
So we have found an appropriate set $F$. 
}%end of proof of 215D 
      
\leader{*215E}{}\cmmnt{ One of the basic properties of Lebesgue measure is
that singleton subsets (and therefore countable subsets) are negligible.
This is of course associated with the fact that Lebesgue measure is
atomless (211Md).   It is easy to construct measures for which singleton
sets are negligible but there are atoms (e.g., the countable-cocountable
measures of 211R).   It takes a little more ingenuity to construct
atomless measures in which not all singleton sets are negligible
(216Ye).   The following result gives conditions under which this
can't happen.

\medskip

\noindent}{\bf Proposition}\dvAnew{2015} 
Let $(X,\Sigma,\mu)$ be an atomless measure space and $x\in X$.   

(a) If $\mu^*\{x\}$ is finite then $\{x\}$ is negligible.

(b) If $\mu$ has locally determined negligible sets then $\{x\}$ is
negligible.

(c) If $\mu$ is localizable then $\{x\}$ is negligible.

\proof{{\bf (a)} \Quer\ Otherwise, $0<\mu^*\{x\}<\infty$.   Let $E$ be a
set containing $x$ such that $\mu E<2\mu^*\{x\}$.   By 215D, there is an
$F\subseteq E$ such that 

\Centerline{$\mu F=\bover12\mu E=\mu(E\setminus F)<\mu^*\{x\}$.}

\noindent So neither $F$ nor $E\setminus F$ can contain $x$;  but
$x\in E$.\ \Bang

\medskip

{\bf (b)} For any set $E$ of finite measure, $E\cap\{x\}$ is negligible; 
for if it is not empty then $x\in E$ and we can apply (a).   As
$\mu$ has locally determined negligible sets, $\{x\}$ is negligible.

\medskip

{\bf (c)} Let $\Cal E\subseteq\Sigma^f$ be a maximal family such that
$\mu E<\infty$ for every $E\in\Cal E$ and $\mu(E\cap F)=\emptyset$ whenever
$E$, $F\in\Cal E$ are distinct.   For each $E\in\Cal E$ and $n\in\Bbb N$,
use 215D $n$ times to find a partition
$E_{n0}$, $E_{n1},\ldots,E_{nn}$ of $E$
such that $\mu E_{ni}=\bover1{n+1}\mu E$ for every $i\le n$.   Next, for
$i\le n\in\Bbb N$, let $H_{ni}$ be an essential supremum of 
$\{E_{ni}:E\in\Cal E\}$.   For each $n\in\Bbb N$ and $E\in\Cal E$.

\Centerline{$\mu(E\setminus\bigcup_{i\le n}H_{ni})
\le\sum_{i=0}^n\mu(E_{ni}\setminus H_{ni})=0$.}

\noindent So if $F\in\Sigma$, $\mu F<\infty$
and $F\cap\bigcup_{i\le n}H_{ni}=\emptyset$, we shall have $\mu(E\cap F)=0$
for every $E\in\Cal E$;  as $\Cal E$ was maximal, $F$ must be negligible.
We are supposing that $\mu$ is semi-finite, so 
$X\setminus\bigcup_{i\le n}H_{ni}$ is negligible.

It follows that if there is any $n$ such that 
$x\notin\bigcup_{i\le n}H_{ni}$ then $\{x\}$ is negligible and we can stop.
Consider $H=\bigcap\{H_{ni}:i\le n\in\Bbb N$, $x\in H_{ni}\}$.
\Quer\ If $\mu H>0$, there is an $F\subseteq H$ such that $0<\mu F<\infty$.
By the maximality of $\Cal E$, there is an $E^*\in\Cal E$ such that
$\mu(F\cap E^*)>0$.   Let $n$ be such that 
$\mu(F\cap E^*)>\bover1{n+1}\mu E^*$.
There is an $i\le n$ such that $x\in H_{ni}$, so that
$F\subseteq H_{ni}$, while $\mu(F\cap E^*_{ni})\le\bover1{n+1}\mu E^*$,
so $F'=F\cap E^*\setminus E^*_{ni}$ has non-zero measure.   Consider
$G=H_{ni}\setminus F'$.   If $E\in\Cal E$, then either
$E=E^*$ and 

\Centerline{$E_{ni}\setminus G=E^*_{ni}\setminus G
=E^*_{ni}\setminus H_{ni}$}

\noindent is negligible, or $E\ne E^*$ and 

\Centerline{$E_{ni}\setminus G
\subseteq(E_{ni}\setminus H_{ni})\cup(E\cap E^*)$}

\noindent is negligible.   Because $H_{ni}$ is an essential supremum of
$\{E_{ni}:E\in\Cal E\}$, $H_{ni}\setminus G$ must be negligible and
$F'$ is negligible;  which is impossible.\ \Qed

Thus $H$ is a negligible set containing $x$ and $\{x\}$ is negligible in
this case also.
}%end of proof of 215E

\exercises{\leader{215X}{Basic exercises (a)} 
%\spheader 215Xa 
Let $(X,\Sigma,\mu)$ be a measure space and $\Phi$ a 
non-empty set of $\mu$-integrable real-valued functions from $X$ to 
$\Bbb R$.   Suppose that $\sup_{n\in\Bbb N}\int f_n$ is finite for every 
sequence $\sequencen{f_n}$ in $\Phi$ such that $f_n\leae f_{n+1}$ 
for every $n$.   Show that there is a sequence $\sequencen{f_n}$ in 
$\Phi$ such that $f_n\leae f_{n+1}$ for every $n$ and, for every 
integrable real-valued function $f$ on $X$, {\it either} 
$f\leae\sup_{n\in\Bbb N}f_n$ {\it or} there is an 
$n\in\Bbb N$ such that no member of $\Phi$ is greater than or equal to 
$\max(f,f_n)$ almost everywhere. 
%215A 
      
\sqheader 215Xb Let $(X,\Sigma,\mu)$ be a measure space.   (i) Suppose 
that $\Cal E$ is a non-empty upwards-directed subset of $\Sigma$ such 
that $c=\sup_{E\in\Cal E}\mu E$ is finite.   Show that 
$E\setminus\bigcup_{n\in\Bbb N}F_n$ is negligible whenever $E\in\Cal E$ 
and $\sequencen{F_n}$ is a sequence in $\Cal E$ such that 
$\lim_{n\to\infty}\mu F_n=c$.   (ii) Let $\Phi$ be a non-empty set of 
integrable functions on $X$ which is upwards-directed in the sense that 
for all $f$, $g\in\Phi$ there is an $h\in\Phi$ such that 
$\max(f,g)\leae h$, and suppose that $c=\sup_{f\in\Phi}\int f$ is 
finite.   Show that $f\leae\sup_{n\in\Bbb N}f_n$ whenever $f\in\Phi$ 
and $\sequencen{f_n}$ is a sequence in $\Phi$ such that 
$\lim_{n\to\infty}\int f_n=c$. 
%215A 
      
\spheader 215Xc Use 215A to shorten the proof of 211Ld. 
      
\spheader 215Xd Give an example of a (non-semi-finite) measure space 
$(X,\Sigma,\mu)$ satisfying conditions (ii)-(iv) of 215B, but not (i). 
%215B 
      
\sqheader 215Xe Let $(X,\Sigma,\mu)$ be an atomless $\sigma$-finite 
measure space.   Show that for any $\epsilon>0$ there is a disjoint 
sequence $\sequencen{E_n}$ of measurable sets with measure at most 
$\epsilon$ such that $X=\bigcup_{n\in\Bbb N}E_n$. 
%215D 
      
\spheader 215Xf Let $(X,\Sigma,\mu)$ be an atomless strictly localizable 
measure space.   Show that for any $\epsilon>0$ there is a decomposition 
$\familyiI{X_i}$ of $X$ such that $\mu X_i\le\epsilon$ for every 
$i\in I$. 
      
\leader{215Y}{Further exercises (a)} 
%\spheader 215Ya 
Let $(X,\Sigma,\mu)$ be a $\sigma$-finite measure space 
and $\langle f_{mn}\rangle_{m,n\in\Bbb N}$, $\sequence{m}{f_m}$, $f$ 
measurable real-valued functions defined almost everywhere in $X$ and 
such that $\sequencen{f_{mn}}\to f_m$ a.e.\ for 
each $m$ and $\sequence{m}{f_m}\to f$ a.e.   Show that there is a 
strictly increasing sequence $\sequence{m}{n_m}$ in $\Bbb N$ such that 
$\sequence{m}{f_{m,n_m}}\to f$ a.e.   (Compare 134Yb.) 
%215B 
      
\spheader 215Yb Let $(X,\Sigma,\mu)$ be a $\sigma$-finite measure space. 
Let $\sequencen{f_n}$ be a sequence of measurable real-valued functions 
such that $f=\lim_{n\to\infty}f_n$ is defined almost everywhere in $X$. 
Show that there is a non-decreasing sequence $\sequence{k}{X_k}$ of 
measurable subsets of $X$ such that $\bigcup_{k\in\Bbb N}X_k$ is 
conegligible in $X$ and $\sequencen{f_n}\to f$ uniformly on every $X_k$, 
in the sense that for any $\epsilon>0$ there is an $m\in\Bbb N$ such 
that $|f_j(x)-f(x)|$ is defined and less than or equal to $\epsilon$ 
whenever $j\ge m$ and $x\in X_k$. 
      
(This is a version of Egorov's theorem.) 
%215B 
      
\spheader 215Yc Let $(X,\Sigma,\mu)$ be a totally finite measure space 
and $\sequencen{f_n}$, $f$ measurable real-valued functions defined 
almost everywhere in $X$.   Show that $\sequencen{f_n}\to f$ a.e.\ iff 
there is a sequence $\sequencen{\epsilon_n}$ of strictly positive real 
numbers, converging to $0$, such that 
      
\Centerline{$\lim_{n\to\infty}\mu^*(\bigcup_{k\ge n}\{x:x\in\dom 
f_k\cap\dom f,\,|f_k(x)-f(x)|\ge\epsilon_n\})=0$.} 
%215B 
      
\spheader 215Yd Find a direct proof of (v)$\Rightarrow$(vi) in 215B. 
\Hint{given $\Cal E\subseteq\Sigma$, use Zorn's Lemma to find a maximal 
totally ordered $\Cal E'\subseteq\Cal E$ such that 
$E\symmdiff F\notin\Cal N$ for any distinct $E$, $F\in\Cal E'$, and 
apply (v) to $\Cal E'$.} 

}%end of exercises 
      
\endnotes{ 
\Notesheader{215} 
The common ground of 215A, 215B(vi), 215C and 215Xa is actually one of 
the most fundamental ideas in measure theory.   It appears in such 
various forms that it is often easier to prove an application from first 
principles than to explain how it can be reduced to the versions here. 
But I will try henceforth to signal such applications as they arise, 
calling the 
method (the proof of 215Aa or 215Xa) the `principle of exhaustion'. 
One point which is perhaps worth noting here is the inductive 
construction of the sequence $\sequencen{F_n}$ in the proof of 215Aa. 
Each $F_{n+1}$ is chosen {\it after} the preceding one.   It is this 
which makes it possible, in the proof of 215B(vii)$\Rightarrow$(vi), to 
extract a suitable sequence $\sequencen{F_n}$ directly.   In many 
applications (starting with what is surely the most important one in the 
elementary theory, the Radon-Nikod\'ym theorem of \S232, or with part 
(i) of the proof of 211Ld), this 
refinement is not needed;  we are dealing with an upwards-directed set, 
as in 215B(v), and can choose the whole sequence $\sequencen{F_n}$ at 
once, no term interacting with any other, as in 215Xb.   The axiom of 
`dependent choice', which asserts that we can construct sequences 
term-by-term, is known to be stronger than the axiom of `countable 
choice', which asserts only that we can choose countably many objects 
simultaneously. 
      
In 215B I try to indicate the most characteristic properties of 
$\sigma$-finiteness;  in particular, the properties which distinguish 
$\sigma$-finite measures from other strictly localizable measures. 
This result is in a way more abstract than the manipulations in the 
rest of the section.   Note that it makes an essential use of the axiom 
of choice in the form of Zorn's Lemma.   I spent a paragraph in 134C 
commenting on the distinction between `countable choice', which is 
needed for anything which looks like the standard theory of Lebesgue 
measure, and the full axiom of choice, which is relatively little used 
in the elementary theory.   The implication (ii)$\Rightarrow$(i) of 215B 
is one of the points where we do need something beyond countable choice. 
(I should perhaps remark that the whole theory of 
non-$\sigma$-finite measure spaces looks very odd without the general 
axiom of choice.) 
Note also that in 215B the proofs of (i)$\Rightarrow$(vii) and 
(vii)$\Rightarrow$(vi) are the only points where anything so vulgar as a 
number appears.   The conditions (iii), (iv), (v) and (vi) are linked in 
ways that have nothing to do with measure theory, and involve only with 
the structure $(X,\Sigma,\Cal N)$.   (See 215Yd here, and 316D-316E in 
Volume 3.)   There are similar conditions relating to measurable 
functions 
rather than measurable sets;  for a fairly abstract example, see 241Ye. 
      
In 215Ya-215Yc are three more standard theorems on 
almost-everywhere-convergent sequences which depend on 
$\sigma$- or total finiteness. 
}%end of notes 
      
\discrpage 
      
