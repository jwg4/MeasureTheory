\frfilename{mt222.tex}
\versiondate{20.11.03/18.10.04}
\copyrightdate{2004}


%$(\DiniD^+f)(x)=(\DiniD^-f)(x)
%=(\Dinid^+f)(x)=(\Dinid^-f)(x)$

\def\chaptername{The Fundamental Theorem of Calculus}
\def\sectionname{Differentiating an indefinite integral}
     
\newsection{222}
     
I come now to the first of the two questions mentioned in the
introduction to this chapter:  if $f$ is an integrable function on
$[a,b]$, what is $\bover{d}{dx}\int_a^xf$?   It turns out that this
derivative exists and is equal to $f$ almost everywhere (222E).   The
argument is based on a striking property of monotonic functions:  they
are differentiable almost everywhere (222A), and we can bound the
integrals of their derivatives (222C).
     
\leader{222A}{Theorem} Let $I\subseteq\Bbb R$ be an interval and
$f:I\to\Bbb R$ a monotonic function.   Then $f$ is differentiable almost 
everywhere in $I$.
     
\cmmnt{\medskip
     
\noindent{\bf Remark} If I seem to be speaking of a measure on
$\Bbb R$ without naming it, as here, I mean Lebesgue measure.
}%end of comment
     
\proof{ As usual, write $\mu^*$ for Lebesgue outer measure
on $\Bbb R$, $\mu$ for Lebesgue measure.
     
\medskip
     
{\bf (a)} To begin with (down to the end of (c) below), let us suppose that $f$ is non-decreasing and $I$ is a bounded open interval
on which $f$ is bounded;  say $|f(x)|\le M$ for $x\in I$.
For any closed subinterval $J=[a,b]$ of $I$, write
$f^*(J)$ for the open interval $\ooint{f(a),f(b)}$.   For $x\in I$, write
     
\Centerline{$\DiniD f(x)=\limsup_{h\to 0}{1\over h}(f(x+h)-f(x))$,
\quad$\Dinid f(x)=\liminf_{h\to 0}{1\over h}(f(x+h)-f(x))$,}
     
\noindent allowing the value $\infty$ in both cases.   Then $f$ is
differentiable at $x$ iff $\DiniD f(x)=\Dinid f(x)\in\Bbb R$.
Because surely $\DiniD f(x)\ge \Dinid f(x)\ge 0$,  $f$ will be
differentiable at $x$ iff $\DiniD f(x)$ is finite and $\DiniD f(x)\le \Dinid f(x)$.
     
I therefore have to show that the sets
     
\Centerline{$\{x:x\in I,\,\DiniD f(x)=\infty\}$,
\quad$\{x:x\in I,\,\DiniD f(x)>\Dinid f(x)\}$}
     
\noindent are negligible.
     
\medskip
     
{\bf (b)} Let us take $A=\{x:x\in I,\,\DiniD f(x)=\infty\}$ first.   Fix an
integer $m\ge 1$ for the moment, and set
     
\Centerline{$A_m=\{x:x\in I,\,\DiniD f(x)>m\}\supseteq A$.}
     
\noindent Let $\Cal I$ be the family of non-trivial closed intervals
$[a,b]\subseteq I$ such that $f(b)-f(a)\ge m(b-a)$;   then
$\mu f^*(J)\ge m\mu J$ for every $J\in\Cal I$.   If $x\in A_m$,
then for any $\delta>0$ we have an $h$ with $0<|h|\le\delta$ and
${1\over h}(f(x+h)-f(x))>m$, so that
     
\Centerline{$[x,x+h]\in\Cal I$ if $h>0$,
\quad$[x+h,x]\in\Cal I$ if $h<0$;}
     
\noindent
thus every member of $A_m$ belongs to arbitrarily small intervals in $\Cal I$.   By Vitali's theorem (221A), there is a countable disjoint set $\Cal I_0\subseteq\Cal I$ such that
$\mu(A\setminus\bigcup\Cal I_0)=0$.
Now, because $f$ is non-decreasing,
$\langle f^*(J)\rangle_{J\in\Cal I_0}$ is disjoint, and all the $f^*(J)$ are included in $[-M,M]$, so $\sum_{J\in\Cal I_0}\mu f^*(J)\le 2M$
and $\sum_{J\in\Cal I_0}\mu J\le 2M/m$.   Because
$A_m\setminus\bigcup\Cal I_0$ is negligible,
     
     
\Centerline{$\mu^*A\le\mu^*A_m\le\Bover{2M}{m}$.}
     
\noindent As $m$ is arbitrary, $\mu^*A=0$ and $A$ is negligible.
     
\medskip
     
{\bf (c)} Now consider $B=\{x:x\in I,\,\DiniD f(x)>\Dinid f(x)\}$.   For $q$, $q'\in\Bbb Q$ with $0\le q<q'$, set
     
     
\Centerline{$B_{qq'}=\{x:x\in I,\,\Dinid f(x)< q,\,\DiniD f(x)>q'\}$.}
     
\noindent  Fix such $q$, $q'$ for the moment, and write
$\gamma=\mu^*B_{qq'}$.  Take any $\epsilon>0$, and let $G$ be an open
set including $B_{qq'}$ such that $\mu G\le\gamma+\epsilon$ (134Fa).
Let $\Cal J$ be the set of non-trivial closed intervals
$[a,b]\subseteq I\cap G$ such that $f(b)-f(a)\le q(b-a)$;  this time  $\mu f^*(J)\le q\mu J$ for $J\in\Cal J$.   Then every member of $B_{qq'}$ is included
in arbitrarily small members of $\Cal J$, so there is a countable
disjoint $\Cal J_0\subseteq\Cal J$ such that $B_{qq'}\setminus\bigcup\Cal
J_0$ is
negligible.   Let $L$ be the set of endpoints of members of $\Cal J_0$;
then $L$ is a countable union of doubleton sets, so is countable, therefore
negligible.  Set
     
\Centerline{$C=B_{qq'}\cap\bigcup\Cal J_0\setminus L$;}
     
\noindent  then
$\mu^*C=\gamma$.   Let $\Cal I$ be the set of non-trivial closed
intervals $J=[a,b]$ such that (i) $J$ is
included in one of the members of $\Cal J_0$ (ii)
$f(b)-f(a)\ge q'(b-a)$;  now $\mu f^*(J)\ge q'\mu J$ for every $J\in\Cal
I$.   Once again, because every member of $C$ is an interior point of
some member of $\Cal J_0$, every point of $C$ belongs to arbitrarily
small members of $\Cal I$;  so there is a countable disjoint $\Cal
I_0\subseteq\Cal I$ such that $\mu(C\setminus\bigcup\Cal I_0)=0$.
     
As in (b) above,
     
\Centerline{$\gamma q'\le q'\mu(\bigcup\Cal I_0)
=\sum_{I\in\Cal I_0}q'\mu I
\le\sum_{I\in\Cal I_0}\mu f^*(I)
=\mu(\bigcup_{I\in\Cal I_0}f^*(I))$.}
     
\noindent   On the other hand,
     
$$\eqalign{\mu(\bigcup_{J\in\Cal J_0}f^*(J))
&=\sum_{J\in\Cal J_0}\mu f^*(J)
\le q\sum_{J\in\Cal J_0}\mu J
=q\mu(\bigcup\Cal J_0)\cr
&\le q\mu(\bigcup\Cal J)
\le q\mu G
\le q(\gamma+\epsilon).\cr}$$
     
     
\noindent But $\bigcup_{I\in\Cal I_0}f^*(I)\subseteq\bigcup_{J\in\Cal
J_0}f^*(J)$, because every member of $\Cal I_0$ is included in a member
of $\Cal J_0$, so $\gamma q'\le q(\gamma+\epsilon)$ and
$\gamma\le \epsilon q/(q'-q)$.
As $\epsilon$ is arbitrary, $\gamma=0$.
     
Thus every $B_{qq'}$ is negligible.   Consequently $B=\bigcup_{q,q'
\in\Bbb Q,0\le q<q'}B_{qq'}$ is negligible.
     
\medskip
     
{\bf (d)} This deals with the case of a bounded open interval on which
$f$ is bounded and non-decreasing.
Still for non-decreasing $f$, but for an arbitrary interval $I$, observe
that $K=\{(q,q'):q,\,q'\in I\cap\Bbb Q,\,q<q'\}$ is countable and that
$I\setminus\bigcup_{(q,q')\in K}\ooint{q,q'}$ has at most two points
(the endpoints of $I$, if any), so is negligible.   If
we write $S$ for the set of points of $I$ at which $f$ is not
differentiable, then from (a)-(c) we see that $S\cap\ooint{q,q'}$ is
negligible for every
$(q,q')\in K$, so that $S\cap\bigcup_{(q,q')\in K}\ooint{q,q'}$ is
negligible and $S$ is negligible.
     
\medskip
     
{\bf (e)} Thus we are done if $f$ is non-decreasing.   For
non-increasing
$f$, apply the above to $-f$, which is differentiable at exactly the
same points as $f$.
}%end of proof of 222A
     
\leader{222B}{Remarks}\dvro{ If}{ {\bf (a)} I note that in the above argument I am 
using such formulae as $\sum_{J\in\Cal I_0}\mu f^*(J)$.
This is because Vitali's theorem leaves it open whether the families $\Cal I_0$ will 
be finite or infinite.   The sum must be
interpreted along the lines laid down in
112Bd in Volume 1;  generally, $\sum_{k\in K}a_k$, where $K$ is an
arbitrary set and every $a_k\ge 0$, is to be
$\sup_{L\subseteq K\text{ is finite}}\sum_{k\in L}a_k$, with the convention that $\sum_{k\in\emptyset}a_k=0$.
Now, in this context, if} $(X,\Sigma,\mu)$ is a measure space, $K$ is a
countable set, and $\family{k}{K}{E_k}$ is a family in $\Sigma$,
     
\Centerline{$\mu(\bigcup_{k\in K}E_k)\le\sum_{k\in K}\mu E_k$,}
     
\noindent with equality if $\family{k}{K}{E_k}$ is disjoint.
\prooflet{\Prf\ If $K=\emptyset$,
this is trivial.   Otherwise, let $n\mapsto k_n:\Bbb N\to K$ be a
surjection, and set
     
\Centerline{$K_n=\{k_i:i\le n\}$, \quad$G_n=\bigcup_{i\le n}E_{k_i}
=\bigcup_{k\in K_n}E_k$}
     
\noindent for each $n\in\Bbb N$.   Then $\sequencen{G_n}$
is a non-decreasing sequence with union $E=\bigcup_{k\in K}E_k$, so
     
\Centerline{$\mu E=\lim_{n\to\infty}\mu G_n=\sup_{n\in\Bbb N}\mu G_n$;}
     
\noindent and
     
\Centerline{$\mu G_n\le\sum_{k\in K_n}\mu E_k\le\sum_{k\in K}\mu E_k$}
     
\noindent for
every $n$, so $\mu E\le\sum_{k\in K}\mu E_k$.   If the $E_k$ are
disjoint, then $\mu G_n$ is precisely $\sum_{k\in K_n}\mu E_k$ for
each $n$;  but as $\sequencen{K_n}$ is a non-decreasing sequence of
sets with union $K$, every finite subset of $K$ is included in some
$K_n$, and
     
\Centerline{$\sum_{k\in K}\mu E_k=\sup_{n\in\Bbb N}\sum_{k\in K_n}
\mu E_k=\sup_{n\in\Bbb N}\mu G_n=\mu E$,}
     
\noindent as required.\ \Qed}
     
\cmmnt{
\spheader 222Bb Some readers will prefer to re-index sets
regularly, so that
all the sums they need to look at will be of the form $\sum_{i=0}^n$ or
$\sum_{i=0}^{\infty}$.   In effect, that is what I did in Volume 1, in
the proof of 114Da/115Da, when showing that Lebesgue outer measure is
indeed an outer measure.
The disadvantage of this procedure in the context of 222A is that we
must continually check that it doesn't matter whether we have a finite or 
infinite sum
at any particular moment.   I believe that it is worth taking the
trouble to learn the technique sketched here, because it very frequently
happens that we wish to consider unions of sets indexed by sets other 
than $\Bbb N$ and $\{0,\ldots,n\}$.
     
\spheader 222Bc Of course the argument above can be shortened if
you know a tiny bit
more about countable sets than I have explicitly stated so far.
But note that the value assigned to
$\sum_{k\in K}a_k$ must not depend on which enumeration
$\sequencen{k_n}$ we pick on.
}%end of comment
     
     
\leader{222C}{Lemma} Suppose that $a\le b$ in $\Bbb R$, and that
$F:[a,b]\to\Bbb R$ is a non-decreasing function.   Then $\int_a^bF'$
exists and is at most $F(b)-F(a)$.
     
\medskip
     
\def\tmpins{}
     
\noindent{\bf Remark}\cmmnt{ I discussed
integration over subsets at length in \S131 and \S214.   For measurable
subsets, which are sufficient for our needs in this
chapter, we have a simple description:  if $(X,\Sigma,\mu)$ is a measure
space, $E\in\Sigma$ and $f$ is a real-valued function, then
$\int_Ef=\int\tilde f$ if the latter integral exists, where
$\dom\tilde f=(E\cap\dom f)\cup(X\setminus E)$ and $\tilde f(x)=f(x)$ if 
$x\in E\cap\dom f$, $0$ if $x\in X\setminus E$ (apply 131Fa to
$\tilde f$).   It follows at once that if now $F\in\Sigma$ and $F\subseteq E$, $\int_Ff=\int_Ef\times\chi F$.
     
} I write $\int_a^xf$ to mean
$\int_{\coint{a,x}}f$\cmmnt{, which (because $\coint{a,x}$ is measurable) can be dealt with as described above}.
\cmmnt{Note that, as long as we are dealing with Lebesgue measure, so that $[a,x]\setminus\ooint{a,x}=\{a,x\}$ is negligible, there is no need to distinguish between $\int_{[a,x]}$, $\int_{\ooint{a,x}}$,
$\int_{\coint{a,x}}$, $\int_{\ocint{a,x}}$;  for other measures on
$\Bbb R$ we may
need to take more care.   I use half-open intervals to make it obvious
that $\int_a^xf+\int_x^yf=\int_a^yf$ if $a\le x\le y$, because
     
\Centerline{$f\times\chi\coint{a,y}
=f\times\chi\coint{a,x}\,+\,f\times\chi\coint{x,y}$.}
}%end of comment
     
\proof{{\bf (a)} The result is trivial if $a=b$;  let us suppose that
$a<b$.   By 222A, $F'$ is defined almost everywhere in $[a,b]$.
     
\medskip
     
{\bf (b)} For each $n\in\Bbb N$, define a simple function
$g_n:\coint{a,b}\to\Bbb R$ as follows.   For $0\le k<2^n$, set
$a_{nk}=a+2^{-n}k(b-a)$, $b_{nk}=a+2^{-n}(k+1)(b-a)$,
$I_{nk}=\coint{a_{nk},b_{nk}}$.   For each $x\in\coint{a,b}$, take that
$k<2^n$ such that $x\in I_{nk}$, and set
     
\Centerline{$g_n(x)=\Bover{2^n}{b-a}(F(b_{nk})-F(a_{nk}))$}
     
\noindent for $x\in I_{nk}$, so that $g_n$ gives the slope of the chord
of the graph of $F$ defined by the endpoints of $I_{nk}$.   Then
     
\Centerline{$\int_a^b g_n=\sum_{k=0}^{2^n-1}F(b_{nk})-F(a_{nk})
=F(b)-F(a)$.}
     
\medskip
     
{\bf (c)} On the other hand, if we set
     
\Centerline{$C=\{x:x\in\ooint{a,b},\,F'(x)$ exists$\}$,}
     
\noindent then $[a,b]\setminus C$ is negligible, by 222A, and
$F'(x)=\lim_{n\to\infty}g_n(x)$ for every $x\in C$.   \Prf\ Let
$\epsilon>0$.   Then there is a $\delta>0$ such that $x+h\in [a,b]$ and
$|F(x+h)-F(x))-hF'(x)|\le\epsilon|h|$ whenever $|h|\le\delta$.   Let
$n\in\Bbb N$ be such that $2^{-n}(b-a)\le\delta$.   Let
$k<2^n$ be such that $x\in I_{nk}$.   Then
     
\Centerline{$x-\delta\le a_{nk}\le x<b_{nk}\le x+\delta$,
\quad$g_n(x)=\Bover{2^n}{b-a}(F(b_{nk})-F(a_{nk}))$.}
     
\noindent Now we have
     
$$\eqalign{|g_n(x)-F'(x)|
&=|\Bover{2^n}{b-a}(F(b_{nk})-F(a_{nk}))-F'(x)|\cr
&=\Bover{2^n}{b-a}|F(b_{nk})-F(a_{nk})-(b_{nk}-a_{nk})F'(x)|\cr
&\le\Bover{2^n}{b-a}\bigl(|F(b_{nk})-F(x)-(b_{nk}-x)F'(x)|\cr
&\qquad\qquad\qquad\qquad   +|F(x)-F(a_{nk})-(x-a_{nk})F'(x)|\bigr)\cr
&\le \Bover{2^n}{b-a}(\epsilon|b_{nk}-x|+\epsilon|x-a_{nk}|)
=\epsilon.\cr}$$
     
\noindent And this is true whenever $2^{-n}\le\delta$,
that is, for all $n$ large enough.   As $\epsilon$ is arbitrary,
$F'(x)=\lim_{n\to\infty}g_n(x)$.\ \Qed
     
\medskip
     
{\bf (d)} Thus $g_n\to F'$ almost everywhere in $[a,b]$.   By Fatou's
Lemma,
     
\Centerline{$\int_a^bF'
=\int_a^b\liminf_{n\to\infty}g_n
\le\liminf_{n\to\infty}\int_a^bg_n
=\lim_{n\to\infty}\int_a^b g_n
=F(b)-F(a)$,}
     
\noindent as required.
}%end of proof of 222C
     
\cmmnt{
\medskip
     
\noindent{\bf Remark} There is a generalization of this result in 224I.
}
     
\leader{222D}{Lemma} Suppose that $a<b$ in $\Bbb R$, and that $f$, $g$
are real-valued functions, both integrable over $[a,b]$, such that
$\int_a^xf=\int_a^xg$ for
every $x\in[a,b]$.   Then $f=g$ almost everywhere in
$[a,b]$.
     
\proof{ The point is that
     
\Centerline{$\int_Ef=\int_a^bf\times\chi E
=\int_a^bg\times\chi E=\int_Eg$}
     
\noindent for any
measurable set $E\subseteq\coint{a,b}$.
     
\Prf\ {\bf (i)} If $E=\coint{c,d}$ where $a\le c\le d\le b$,
then
     
\Centerline{$\int_Ef=\int_a^df-\int_a^cf
=\int_a^dg-\int_a^cg=\int_Eg$.}
     
\medskip
     
\quad{\bf (ii)} If $E=\coint{a,b}\cap G$ for some open set
$G\subseteq\Bbb R$, then for each $n\in\Bbb N$ set
     
\Centerline{$K_n
=\{k:k\in\Bbb Z,\,|k|\le 4^n,\,\coint{2^{-n}k,2^{-n}(k+1)}\subseteq
G\}$,}
     
\Centerline{$H_n
=\bigcup_{k\in K_n}\coint{2^{-n}k,2^{-n}(k+1)}\cap\coint{a,b}$;}
     
\noindent then
$\sequencen{H_n}$ is a non-decreasing sequence of measurable sets with
union $E$, so $f\times\chi E=\lim_{n\to\infty}f\times\chi H_n$, and (by
Lebesgue's Dominated Convergence Theorem, because $|f\times\chi
H_n|\le|f|$ almost everywhere for every $n$, and $|f|$ is integrable)
     
\Centerline{$\int_Ef=\lim_{n\to\infty}\int_{H_n}f$.}
     
\noindent At the same time, each $H_n$ is a finite disjoint union of
half-open intervals in $\coint{a,b}$, so
     
\Centerline{$\int_{H_n}f
=\sum_{k\in K_n}\int_{\coint{2^{-n}k,2^{-n}(k+1)}\cap\coint{a,b}}f
=\sum_{k\in K_n}\int_{\coint{2^{-n}k,2^{-n}(k+1)}\cap\coint{a,b}}g
=\int_{H_n}g$,}
     
\noindent and
     
\Centerline{$\int_Eg=\lim_{n\to\infty}\int_{H_n}g
=\lim_{n\to\infty}\int_{H_n}f=\int_Ef$.}
     
\medskip
     
\quad{\bf (iii)} For general measurable $E\subseteq\coint{a,b}$, we can
choose for each $n\in\Bbb N$ an open set $G_n\supseteq E$ such that
$\mu G_n\le\mu E+2^{-n}$ (134Fa).   Set $G_n'=\bigcap_{m\le n}G_m$, $E_n=\coint{a,b}\cap G'_n$ for each $n$,
     
\Centerline{$F=\coint{a,b}\cap\bigcap_{n\in\Bbb N}G_n
=\bigcap_{n\in\Bbb N}\coint{a,b}\cap G_n'=\bigcap_{n\in\Bbb N}E_n$.}
     
\noindent Then $E\subseteq F$ and
     
\Centerline{$\mu F\le\inf_{n\in\Bbb N}\mu G_n=\mu E$,}
     
\noindent so $F\setminus E$ is negligible and
$f\times\chi(F\setminus E)$ is zero almost everywhere;  consequently $\int_{F\setminus E}f=0$
and $\int_Ff=\int_Ef$.  On the other hand,
     
\Centerline{$f\times\chi F=\lim_{n\to\infty}f\times\chi E_n$,}
     
\noindent so by Lebesgue's Dominated Convergence Theorem again
     
\Centerline{$\int_Ef=\int_Ff=\lim_{n\to\infty}\int_{E_n}f$.}
     
\noindent Similarly
     
\Centerline{$\int_Eg=\lim_{n\to\infty}\int_{E_n}g$.}
     
\noindent But by part (ii) we have $\int_{E_n}g=\int_{E_n}f$ for every
$n$, so $\int_Eg=\int_Ef$, as required.\ \Qed
     
By 131Hb, $f=g$ almost everywhere in $\coint{a,b}$, and therefore almost everywhere in $[a,b]$.
}%end of proof of 222D
     
\leader{222E}{Theorem} Suppose that $a\le b$ in $\Bbb R$ and that $f$ is
a real-valued function which is integrable over $[a,b]$.
Then $F(x)=\int_a^xf$ exists in $\Bbb R$ for every $x\in [a,b]$, and the derivative
$F'(x)$ exists and is equal to $f(x)$ for almost every $x\in[a,b]$.
     
\proof{{\bf (a)} For most of this proof (down to the end of
(c) below) I suppose that $f$ is non-negative.   In this case,
     
     
\Centerline{$F(y)=F(x)+\int_x^yf\ge F(x)$}
     
\noindent whenever $a\le x\le y\le b$;
thus $F$ is non-decreasing and therefore differentiable almost
everywhere in
$[a,b]$, by 222A.
     
By 222C we know also that $\int_a^xF'$ exists and is less than or equal
to $F(x)-F(a)=F(x)$ for every $x\in[a,b]$.
     
\medskip
     
{\bf (b)} Now suppose, in addition, that $f$ is bounded;  say $0\le
f(t)\le
M$ for every $t\in\dom f$.   Then $M-f$ is integrable over $[a,b]$;  let
$G$ be its indefinite integral, so that $G(x)=M(x-a)-F(x)$ for every
$x\in[a,b]$.   Applying (a) to $M-f$ and $G$, we have $\int_a^xG'\le
G(x)$ for every $x\in[a,b]$;  but of course $G'=M-F'$, so
$M(x-a)-\int_a^xF'\le M(x-a)-F(x)$, that is, $\int_a^xF'\ge F(x)$ for
every $x\in[a,b]$.   Thus $\int_a^xF'=\int_a^xf$ for every
$x\in[a,b]$.    Now 222D tells us that $F'=f$ almost
everywhere in $[a,b]$.
     
\medskip
     
{\bf (c)} Thus for bounded, non-negative $f$ we are done.   For
unbounded $f$, let $\sequencen{f_n}$ be a non-decreasing sequence of
non-negative simple functions converging to $f$ almost everywhere in
$[a,b]$, and let $\sequencen{F_n}$ be the corresponding indefinite
integrals.   Then for any $n$ and any $x$, $y$ with $a\le x\le y\le b$,
we have
     
\Centerline{$F(y)-F(x)=\int_x^yf\ge\int_x^yf_n=F_n(y)-F_n(x)$,}
     
\noindent so that
$F'(x)\ge F_n'(x)$ for any $x\in\ooint{a,b}$ where both are defined, and
$F'(x)\ge f_n(x)$ for almost every $x\in[a,b]$.   This is true for every
$n$, so $F'\ge f$ almost everywhere, and $F'-f\ge 0$ almost everywhere.
On the other hand, as noted in (a),
     
\Centerline{$\int_a^bF'\le F(b)-F(a)=\int_a^bf$,}
     
\noindent so $\int_a^bF'-f\le 0$.   It follows
that $F'\eae f$ (that is, that $F'=f$ almost everywhere in $[a,b]$)(122Rd).
     
\medskip
     
{\bf (d)} This completes the proof for non-negative $f$.   For general
$f$, we can express $f$ as $f_1-f_2$ where $f_1$, $f_2$ are
non-negative integrable functions;  now $F=F_1-F_2$ where $F_1$, $F_2$
are the corresponding indefinite integrals, so
$F'\eae F_1'-F_2'\eae f_1-f_2$, and $F'\eae f$.
}%end of proof of 222E
     
\leader{222F}{Corollary} Suppose that $f$ is any real-valued function which is integrable over $\Bbb R$, and set $F(x)=\int_{-\infty}^xf$ for every $x\in\Bbb R$.   Then $F'(x)$ exists and is equal to $f(x)$ for almost every $x\in\Bbb R$.
     
\proof{ For each $n\in\Bbb N$, set
     
\Centerline{$F_n(x)=\int_{-n}^xf$}
     
\noindent for $x\in[-n,n]$.   Then $F_n'(x)=f(x)$ for almost every
$x\in[-n,n]$.   But $F(x)=F(-n)+F_n(x)$ for every $x\in[-n,n]$, so
$F'(x)$ exists and is equal to $F'_n(x)$ for every $x\in\ooint{-n,n}$
for which $F'_n(x)$ is defined;  and $F'(x)=f(x)$ for almost every
$x\in[-n,n]$.   As $n$ is arbitrary, $F'\eae f$.
}%end of proof of 222F
     
\leader{222G}{Corollary} Suppose that $E\subseteq\Bbb R$ is a measurable
set and that $f$ is a real-valued function which is integrable over $E$.
Set $F(x)=\int_{E\cap\ooint{-\infty,x}}f$ for $x\in\Bbb R$.   Then
$F'(x)=f(x)$ for almost every $x\in E$, and $F'(x)=0$ for almost every
$x\in\Bbb R\setminus E$.
     
\proof{ Apply 222F to $\tilde f$, where $\tilde f(x)=f(x)$ for $x\in
E\cap\dom f$ and $\tilde f(x)=0$ for $x\in\Bbb R\setminus E$, so that
$F(x)=\int_{-\infty}^x\tilde f$ for every $x\in\Bbb R$.
}%end of proof of 222G
     
\leader{222H}{}\cmmnt{ The result that $\bover{d}{dx}\int_a^xf=f(x)$
for almost every $x$ is satisfying, but is no substitute for the more
elementary result that this equality is valid at any point at which $f$
is continuous.
     
\medskip
     
\noindent}{\bf Proposition} Suppose that $a\le b$ in $\Bbb R$ and that
$f$ is a real-valued function which is integrable over $[a,b]$.   Set
$F(x)=\int_a^xf$ for $x\in[a,b]$.   Then $F'(x)$ exists and is equal to
$f(x)$ at any point $x\in\dom(f)\cap\ooint{a,b}$ at which $f$ is
continuous.
     
\proof{ Set $c=f(x)$.  Let $\epsilon>0$.   Let $\delta>0$ be such that
$\delta\le\min(b-x,x-a)$ and $|f(t)-c|\le\epsilon$ whenever $t\in\dom f$
and $|t-x|\le\delta$.   If $x<y\le x+\delta$, then
     
\Centerline{$|\Bover{F(y)-F(x)}{y-x}-c|
=\Bover1{y-x}|\int_x^yf-c|
\le\Bover1{y-x}\int_x^y|f-c|
\le\epsilon$.}
     
\noindent Similarly, if $x-\delta\le y<x$,
     
\Centerline{$|\Bover{F(y)-F(x)}{y-x}-f(x)|
=\Bover1{x-y}|\int_y^xf-c|
\le\Bover1{x-y}\int_y^x|f-c|
\le\epsilon$.}
     
\noindent As $\epsilon$ is arbitrary,
     
\Centerline{$F'(x)=\lim_{y\to x}\Bover{F(y)-F(x)}{y-x}=c$,}
     
\noindent as required.
}%end of proof of 222H
     
     
\leader{222I}{Complex-valued functions} \cmmnt{So far in this section, I
have taken every $f$ to be real-valued.   The extension to
complex-valued $f$ is just a matter of applying the above results to the
real and imaginary parts of $f$.   Specifically, we have the following.

\medskip

}{\bf (a)} If $a\le b$ in $\Bbb R$ and $f$ is a
complex-valued function which is integrable over $[a,b]$, then
$F(x)=\int_a^xf$ is defined in $\Bbb C$ for every $x\in[a,b]$, and its
derivative $F'(x)$ exists and is equal to $f(x)$ for almost every
$x\in[a,b]$;  moreover, $F'(x)=f(x)$ whenever
$x\in\dom(f)\cap\ooint{a,b}$ and $f$ is continuous at $x$.
     
\spheader 222Ib If $f$ is a complex-valued function which is
integrable over $\Bbb R$, and $F(x)=\int_{-\infty}^xf$ for each 
$x\in\Bbb R$, then
$F'$ exists and is equal to $f$ almost everywhere in $\Bbb R$.
     
\spheader 222Ic If $E\subseteq\Bbb R$ is a measurable set and
$f$ is a complex-valued function which is integrable over $E$, and
$F(x)=\int_{E\cap\ooint{-\infty,x}}f$ for each $x\in\Bbb R$, then
$F'(x)=f(x)$ for almost every $x\in E$ and $F'(x)=0$ for almost every
$x\in\Bbb R\setminus E$.
     
\leader{*222J}{The Denjoy-Young-Saks \dvrocolon{theorem}}\cmmnt{ The 
next result 
will not be used, on present plans, anywhere in this treatise.   It
is however central to parts of real analysis for which this volume is
supposed to be a foundation, and while the argument
requires a certain sophistication it is not really a large step from
Lebesgue's theorem 222A.   I must begin with some notation.
     
\medskip

\noindent}{\bf Definition} Let $f$ be any real function, and
$A\subseteq\Bbb R$ its domain.   Write  

\Centerline{$\tilde A^+
=\{x:x\in A$, $\ocint{x,x+\delta}\cap A\ne\emptyset$ 
for every $\delta>0\}$,}

\Centerline{$\tilde A^-
=\{x:x\in A$, $\coint{x-\delta,x}\cap A\ne\emptyset$ 
for every $\delta>0\}$.}
     
\noindent Set

\Centerline{$\DiniD^+(x)
=\limsup_{y\in A,y\downarrow x}\Bover{f(y)-f(x)}{y-x}
=\inf_{\delta>0}\sup_{y\in A,x<y\le x+\delta}\Bover{f(y)-f(x)}{y-x}$,}

\Centerline{$\Dinid^+(x)
=\liminf_{y\in A,y\downarrow x}\Bover{f(y)-f(x)}{y-x}
=\sup_{\delta>0}\inf_{y\in A,x<y\le x+\delta}\Bover{f(y)-f(x)}{y-x}$}

\noindent for $x\in\tilde A^+$, and

\Centerline{$\DiniD^-(x)
=\limsup_{y\in A,y\uparrow x}\Bover{f(y)-f(x)}{y-x}
=\inf_{\delta>0}\sup_{y\in A,x-\delta\le y<x}\Bover{f(y)-f(x)}{y-x}$,}

\Centerline{$\Dinid^-(x)
=\liminf_{y\in A,y\uparrow x}\Bover{f(y)-f(x)}{y-x}
=\sup_{\delta>0}\inf_{y\in A,x-\delta\le y<x}\Bover{f(y)-f(x)}{y-x}$}

\noindent for $x\in\tilde A^-$, all defined in $[-\infty,\infty]$.   
(These are the four {\bf Dini derivates} of $f$.\cmmnt{ You will also
see $D^+$, $d^+$, $D^-$, $d^-$ used in place of my $\DiniD^+$, 
$\Dinid^+$, $\DiniD^-$ and $\Dinid^-$.})

Note that we surely have $(\Dinid^+f)(x)\le(\DiniD^+f)(x)$
for every $x\in\tilde A^+$, while 
$(\Dinid^-f)(x)\le(\DiniD^-f)(x)$
for every $x\in\tilde A^-$.   The ordinary derivative
$f'(x)$ is defined and equal to $c\in\Bbb R$ iff 
($\alpha$) $x$ belongs to some open interval included in $A$ ($\beta$)
$(\DiniD^+f)(x)=(\Dinid^+f)(x)=(\DiniD^-f)(x)
=(\Dinid^+f)(x)=c$.

\leader{*222K}{Lemma} Let $A$ be any subset of $\Bbb R$, and define 
$\tilde A^+$ and $\tilde A^-$ as in 222J.   Then $A\setminus\tilde A^+$ and
$A\setminus\tilde A^-$ are countable, therefore negligible.

\proof{ We have

\Centerline{$A\setminus\tilde A^+
=\bigcup_{q\in\Bbb Q}\{x:x\in A$, $x<q$, $A\cap\ocint{x,q}=\emptyset\}$.}

\noindent But for any $q\in\Bbb Q$ there can be at most one $x\in A$ such
that $x<q$ and $\ocint{x,q}$ does not meet $A$, 
so $A\setminus\tilde A^+$ is a
countable union of finite sets and is countable.   Similarly,

\Centerline{$A\setminus\tilde A^-
=\bigcup_{q\in\Bbb Q}\{x:x\in A$, $q<x$, $A\cap\coint{q,x}=\emptyset\}$}

\noindent is countable.
}%end of proof of 222K

\leader{*222L}{Theorem} Let $f$ be any real function, and $A$ its domain.  
Then for almost every $x\in A$

\qquad{\it either} all four Dini derivates of $f$ at $x$ are defined, 
finite and equal

\qquad{\it or} $(\DiniD^+f)(x)=(\Dinid^-f)(x)$ is finite,
$(\Dinid^+f)(x)=-\infty$ and $(\DiniD^+f)(x)=\infty$

\qquad{\it or} $(\Dinid^+f)(x)=(\DiniD^-f)(x)$ is finite,
$(\DiniD^+f)(x)=\infty$ and $(\Dinid^-f)(x)=-\infty$

\qquad{\it or} $(\DiniD^+f)(x)=(\DiniD^-f)(x)=\infty$ and
$(\Dinid^+f)(x)=(\Dinid^-f)(x)=-\infty$.

\proof{{\bf (a)} Set $\tilde A=\tilde A^+\cap\tilde A^-$, 
defining $\tilde A^+$ and
$\tilde A^-$ as in 222J, so that $\tilde A$ is a cocountable subset of $A$
and all four Dini derivates are defined on $\tilde A$.   For $n\in\Bbb N$,
$q\in\Bbb Q$ set 

\Centerline{$E_{qn}=\{x:x\in\tilde A$, $x<q$, $f(y)\ge f(x)-n(y-x)\}$
for every $y\in A\cap[x,q]\}$.}

\noindent Observe that 

\Centerline{$\bigcup_{n\in\Bbb N,q\in\Bbb Q}E_{qn}=\{x:x\in\tilde A$,
$(\Dinid^+f)(x)>-\infty\}$.}

\noindent For those $q\in\Bbb Q$, $n\in\Bbb N$ such that $E_{qn}$ is not
empty, set $\beta_{qn}=\sup E_{qn}\in\ocint{-\infty,q}$, 
$\alpha_{qn}=\inf E_{qn}\in[-\infty,\beta_{qn}]$,
and for $x\in\ooint{\alpha_{qn},\beta_{qn}}$ set
$g_{qn}(x)=\inf\{f(y)+ny:y\in A\cap[x,q]\}$.   Note that if 
$x\in E_{qn}\setminus\{\alpha_{qn},\beta_{qn}\}$ then $g_{qn}(x)=f(x)+nx$
is finite;  also $g$ is monotonic, therefore finite everywhere in
$\ooint{\alpha_{qn},\beta_{qn}}$, and of course $g_{qn}(x)\le f(x)+nx$ for
every $x\in A\cap\ooint{\alpha_{qn},\beta_{qn}}$.   

By 222A, almost every point of
$\ooint{\alpha_{qn},\beta_{qn}}$ belongs to $F_{qn}=\dom g'_{qn}$;  in
particular, $E_{qn}\setminus F_{qn}$ is negligible.   Set
$h_{qn}(x)=g_{qn}(x)-nx$ for $x\in\ooint{\alpha_{qn},\beta_{qn}}$;  then
$h$ is differentiable at every point
of $F_{qn}$.   Now if $x\in E_{qn}\cap F_{qn}$,
we have $h_{qn}(x)=f(x)$, while $h_{qn}(x)\le f(x)$ for 
$x\in A\cap\ooint{\alpha_{qn},\beta_{qn}}$;  it follows that

$$\eqalign{(\Dinid^+f)(x)
&=\sup_{\delta>0}\inf_{y\in A\cap\ocint{x,x+\delta}}
  \Bover{f(y)-f(x)}{y-x}\cr
&\ge\sup_{\delta>0}\inf_{y\in A\cap\ocint{x,x+\delta}}
  \Bover{h_{qn}(y)-h_{qn}(x)}{y-x}\cr
&\ge\sup_{0<\delta<\beta_{qn}-x}\inf_{y\in\ocint{x,x+\delta}}
  \Bover{h_{qn}(y)-h_{qn}(x)}{y-x}
=h_{qn}'(x),\cr}$$
  
\noindent and similarly
$(\DiniD^-f)(x)\le h'_{qn}(x)$.   On the other hand,
if $x\in\tilde E_{qn}^+$, then

$$\eqalign{(\Dinid^+f)(x)
&=\sup_{\delta>0}\inf_{y\in A\cap\ocint{x,x+\delta}}
    \Bover{f(y)-f(x)}{y-x}\cr
&\le\sup_{\delta>0}\inf_{y\in E_{qn}\cap\ocint{x,x+\delta}}
    \Bover{f(y)-f(x)}{y-x}\cr
&=\sup_{\delta>0}\inf_{y\in E_{qn}\cap\ocint{x,x+\delta}}
    \Bover{h_{qn}(y)-h_{qn}(x)}{y-x}\cr
&\le\inf_{\delta>0}\sup_{y\in E_{qn}\cap\ocint{x,x+\delta}}
    \Bover{h_{qn}(y)-h_{qn}(x)}{y-x}\cr
&\le\inf_{0<\delta<\beta_{qn}-x}\sup_{y\in\ocint{x,x+\delta}}
    \Bover{h_{qn}(y)-h_{qn}(x)}{y-x}    
=h'_{qn}(x).\cr}$$

\noindent Putting these together, we see that if 
$x\in F_{qn}\cap\tilde E_{qn}^+$ then
$(\Dinid^+f)(x)=h'_{qn}(x)\ge(\DiniD^-f)(x)$.

Conventionally setting $F_{qn}=\emptyset$ if $E_{qn}$ is empty, the last
sentence is true for all $q\in\Bbb Q$ and $n\in\Bbb N$, while
$A\setminus\tilde A$ and
$\bigcup_{q\in\Bbb Q,n\in\Bbb N}E_{qn}\setminus(F_{qn}\cap\tilde E_{qn}^+)$
are negligible, and $(\Dinid^+f)(x)=-\infty$ for every
$x\in\tilde A\setminus\bigcup_{q\in\Bbb Q,n\in\Bbb N}E_{qn}$.   So we see
that, for almost every $x\in A$, either $(\Dinid^+f)(x)=-\infty$ or
$\infty>(\Dinid^+f)(x)\ge(\DiniD^-f)(x)$.

\medskip

{\bf (b)} Reflecting the above argument left-to-right or up-to-down, we see
that, for almost every $x\in A$,

\Centerline{either $(\Dinid^-f)(x)=-\infty$ or
$\infty>(\Dinid^-f)(x)\ge(\DiniD^+f)(x)$,}

\Centerline{either $(\DiniD^+f)(x)=\infty$ or
$-\infty<(\DiniD^+f)(x)\le(\Dinid^-f)(x)$,}

\Centerline{either $(\DiniD^-f)(x)=\infty$ or
$-\infty<(\DiniD^-f)(x)\le(\Dinid^+f)(x)$,}

\noindent and also

\Centerline{$(\Dinid^+f)(x)\le(\DiniD^+f)(x)$,
\quad$(\Dinid^-f)(x)\le(\DiniD^-f)(x)$.}

\noindent For such $x$, therefore,

\Centerline{$(\Dinid^+f)(x)>-\infty
\Longrightarrow (\DiniD^-f)(x)\le(\Dinid^+f)(x)<\infty
\Longrightarrow (\Dinid^+f)(x)=(\DiniD^-f)(x)\in\Bbb R$,}

\noindent and similarly

\Centerline{$(\Dinid^-f)(x)>-\infty
\Longrightarrow(\Dinid^-f)(x)=(\DiniD^+f)(x)\in\Bbb R$,}

\Centerline{$(\DiniD^+f)(x)<\infty
\Longrightarrow(\DiniD^+f)(x)=(\Dinid^-f)(x)\in\Bbb R$,}

\Centerline{$(\DiniD^-f)(x)<\infty
\Longrightarrow(\DiniD^-f)(x)=(\Dinid^+f)(x)\in\Bbb R$.}

\noindent So we have

\Centerline{either $(\Dinid^+f)(x)=(\DiniD^-f)(x)$ is finite
or $(\Dinid^+f)(x)=-\infty$ and
$(\DiniD^-f)(x)=\infty$,}

\Centerline{either $(\Dinid^-f)(x)=(\DiniD^+f)(x)$ is finite
or $(\Dinid^-f)(x)=-\infty$ and
$(\DiniD^+f)(x)=\infty$.}

\noindent These two dichotomies lead to four possibilities;  and since 

\Centerline{$(\Dinid^+f)(x)=(\DiniD^-f)(x)$ is finite,
\quad$(\Dinid^-f)(x)=(\DiniD^+f)(x)$ is finite}

\noindent can be true together only when all four derivates are equal and
finite, we have the four cases listed in the statement of the theorem.
}%end of proof of 222L


\exercises{
\leader{222X}{Basic exercises $\pmb{>}$(a)}
%\sqheader 222Xa
Let $F:[0,1]\to[0,1]$ be the Cantor function (134H).
Show that $\int_0^1F'=0<F(1)-F(0)$.
%222C
     
\sqheader 222Xb Suppose that $a<b$ in $\Bbb R$ and that $h$ is a
real-valued function such that $\int_x^yh$ exists and is non-negative
whenever $a\le x\le y\le b$.   Show that $h\ge 0$ almost everywhere in $[a,b]$.
%222D
     
\sqheader 222Xc Suppose that $a<b$ in $\Bbb R$ and that $f$, $g$ are
integrable complex-valued functions on $[a,b]$ such that
$\int_a^xf=\int_a^xg$ for every $x\in [a,b]$.   Show that $f=g$ almost everywhere in $[a,b]$.
%222I
     
\sqheader 222Xd Suppose that $a<b$ in
$\Bbb R$ and that $f$ is a
real-valued function which is integrable over $[a,b]$.   Show that the
indefinite integral $x\mapsto\int_a^xf$ is continuous.
%222+
     
\leader{222Y}{Further exercises (a)} 
%\spheader 222Ya
Let $\sequencen{F_n}$ be a sequence
of non-negative, non-decreasing functions on $[0,1]$ such that
$F(x)=\sum_{n=0}^{\infty}F_n(x)$ is finite for every $x\in[0,1]$.   Show
that $\sum_{n=0}^{\infty}F'_n(x)=F'(x)$ for almost every $x\in[0,1]$.
({\it Hint\/}:  take $\sequence{k}{n_k}$ such that
$\sum_{k=0}^{\infty}F(1)-G_k(1)<\infty$, where
$G_k=\sum_{j=0}^{n_k}F_j$, and set
$H(x)=\sum_{k=0}^{\infty}F(x)-G_k(x)$.   Observe that
$\sum_{k=0}^{\infty}F'(x)-G'_k(x)\le H'(x)$ whenever all the derivatives
are defined, so that $F'=\lim_{k\to\infty}G'_k$ almost everywhere.)
%222A
     
\spheader 222Yb Let $F:[0,1]\to\Bbb R$ be a continuous non-decreasing
function.   (i)   Show that if $c\in\Bbb R$ then
$C=\{(x,y):x,\,y\in[0,1],\,F(y)-F(x)=c\}$ is connected.   ({\it Hint\/}:
A set $A\subseteq\BbbR^r$ is {\bf connected} if there is no continuous
surjection $h:A\to\{0,1\}$.   Show that if $h:C\to\{0,1\}$ is continuous
then it is of the form $(x,y)\mapsto h_1(x)$ for some continuous
function $h_1$.)  (ii) Now suppose that $F(0)=0$, $F(1)=1$ and that
$G:[0,1]\to[0,1]$ is a second continuous non-decreasing function with
$G(0)=0$, $G(1)=1$.   Show that for any $n\ge 1$ there are $x$,
$y\in[0,1]$ such that $F(y)-F(x)=G(y)-G(x)=\bover1n$.
%222+
     
\spheader 222Yc Let $f$, $g$ be non-negative integrable functions on
$\Bbb R$, and $n\ge 1$.   Show that there are $u<v$ in
$[-\infty,\infty]$ such that $\int_u^vf=\bover1n\int f$ and
$\int_u^vg=\bover1n\int g$.
%222Yb 222+
     
\spheader 222Yd Let $f:\Bbb R\to\Bbb R$ be measurable.   Show that 
$H=\dom f'$ is a measurable set and that $f'$ is a measurable function.
%222+

\spheader 222Ye Construct a Borel measurable
function $f:[0,1]\to\{-1,0,1\}$ such that
each of the four possibilities described in Theorem 222L occurs on a set 
of measure $\bover14$.
%222L
}%end of exercises
     
\endnotes{
\Notesheader{222} I have relegated to an exercise (222Xd) the
fundamental fact that an indefinite integral $x\mapsto\int_a^xf$ is
always continuous;  this is not strictly speaking needed in this
section, and a much stronger result is given in 225E.   There is also
much more to be said about monotonic functions, to which I will return
in \S224.   What we need here is the fact that they are differentiable
almost everywhere (222A), which I prove by applying Vitali's theorem
three times, once in part (b) of the proof and twice in part (c).
Following this, the arguments of 222C-222E form a fine series of
exercises in the central ideas of Volume 1, using the concept of
integration over a (measurable) subset, Fatou's Lemma (part (d) of the
proof of 222C), Lebesgue's Dominated Convergence Theorem (parts (ii)
and (iii) of the proof of 222D) and the approximation of Lebesgue
measurable sets by open sets (part (iii) of the proof of 222D).   Of
course knowing that $\bover{d}{dx}\int_a^xf=f(x)$ almost everywhere is not at all
the same thing as knowing that this holds for any particular $x$, and
when we come to differentiate any particular indefinite integral we
generally turn to 222H first;  the point of 222E is that it applies to
wildly discontinuous functions, for which more primitive methods give no
information at all.

The Denjoy-Young-Saks theorem (222L) is one of the starting points of a
flourishing theory of `typical' phenomena in real analysis.   
It is easy to build a 
function $f$ with any prescribed set of values for $(\DiniD^+f)(0)$,
$(\Dinid^+f)(0)$, $(\DiniD^-f)(0)$ and $(\Dinid^-f)(0)$ (subject, of
course, to the requirements $(\Dinid^+f)(0)\le(\DiniD^+f)(0)$ and
$(\Dinid^-f)(0)\le(\DiniD^-f)(0)$).   But 222L tells us that such
combinations as $(\Dinid^-f)(x)=(\Dinid^+f)(x)=\infty$ (what we might call
`$f'(x)=\infty$') can occur only on negligible sets.   The four easily
realized possibilities in 222L (see 222Ye) 
are the only ones which can appear
at points which are `typical' for the given function, from the point of
view of Lebesgue measure.   For a monotonic
function, 222A tells us more:  at `typical' points for a monotonic
function, the function is actually differentiable.   In the next section we
shall see some more 
ways of generating negligible and conegligible sets from
a given set or function, leading to further refinements of the idea.
}%end of comment
     
\discrpage
     
