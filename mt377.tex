\frfilename{mt377.tex}
\versiondate{30.12.09}
\copyrightdate{2008}

\def\chaptername{Linear operators between function spaces}
\def\sectionname{Function spaces of reduced products}

\newsection{*377\dvAnew{2008}}

In \S328 I introduced `reduced products' of probability algebras.   In this
section I seek to describe the function spaces of reduced products as
images of subspaces of products of function spaces of the original
algebras.   I add a group of universal mapping theorems associated
with the constructions of projective and inductive limits of directed
families of probability algebras (377G-377H).

\leader{377A}{Proposition} If $\familyiI{\frak A_i}$ is a
non-empty family of Boolean
algebras with simple product $\frak A$, then $L^{\infty}(\frak A)$ can be
identified, as normed space and $f$-algebra, with the subspace
$W_{\infty}$ of $\prod_{i\in I}L^{\infty}(\frak A_i)$
consisting of families $u=\familyiI{u_i}$ such that
$\|u\|_{\infty}=\sup_{i\in I}\|u_i\|_{\infty}$ is finite.

\proof{{\bf (a)} I begin by noting that $W_{\infty}$ is, in itself, an
Archimedean
$f$-algebra and $\|\,\|_{\infty}$ is a Riesz norm on $W_{\infty}$.
\Prf\ $W_{\infty}$ is a solid linear subspace of
$\prod_{i\in I}L^{\infty}(\frak A_i)$, so inherits a Riesz space structure
(352K, 352Ja).   Now it is easy to check that
$e=\familyiI{\chi 1_{\frak A_i}}$ is an order unit in $W_{\infty}$ and
that $\|\,\|_{\infty}$ is the corresponding order-unit norm (354F-354G).
Finally, because $W_{\infty}$ is the solid linear subspace
of $\prod_{i\in I}L^{\infty}(\frak A_i)$ generated by
$e$, and $e$ is the multiplicative identity of
$\prod_{i\in I}L^{\infty}(\frak A_i)$, $W_{\infty}$ is closed under
multiplication, and is an $f$-algebra.\ \Qed

\medskip

{\bf (b)} We have a natural function
$\theta:\frak A\to W_{\infty}$ defined by saying that
$\theta a=\familyiI{\chi a_i}$ whenever
$a=\familyiI{a_i}\in\frak A$.   Clearly $\theta$ is additive and
$\|\theta a\|_{\infty}\le 1$ for every $a\in\frak A$;  moreover,
$\theta a\wedge\theta b=0$ when $a$, $b\in\frak A$ are disjoint.
By 363E, we have a corresponding Riesz homomorphism
$T:L^{\infty}(\frak A)\to W_{\infty}$ of norm at most $1$.

\medskip

{\bf (c)} In fact $\|Tw\|_{\infty}=\|w\|_{\infty}$ for every
$w\in L^{\infty}(\frak A)$.   \Prf\ If $w=0$, this is trivial.
If $w\in S(\frak A)\setminus\{0\}$, express it as
$\sum_{k=0}^n\alpha_k\chi a^{(k)}$ where $\langle a^{(k)}\rangle_{k\le n}$
is a disjoint family of non-zero elements.
Expressing each $a^{(k)}$ as $\familyiI{a_{ki}}$,

\Centerline{$Tw=\familyiI{\sum_{k=0}^n\alpha_k\chi a_{ki}}$.}

\noindent There must be a $j$ such that $|\alpha_j|=\|w\|_{\infty}$;  now
there is an $i$ such that $a_{ji}\ne 0$;  as
$\langle a_{ki}\rangle_{k\le n}$ is disjoint,

\Centerline{$\|Tw\|_{\infty}
\ge\|\sum_{k=0}^n\alpha_k\chi a_{ki}\|_{\infty}
\ge|\alpha_j|=\|w\|_{\infty}$.}

If now $w$ is any member of $L^{\infty}(\frak A)$,

$$\eqalign{\|w\|_{\infty}
&=\sup\{\|w'\|_{\infty}:w'\in S(\frak A),\,|w'|\le|w|\}\cr
&=\sup\{\|Tw'\|_{\infty}:w'\in S(\frak A),\,|w'|\le|w|\}
\le\|Tw\|_{\infty}\cr}$$

\noindent because $T$ is a Riesz homomorphism.\ \Qed

Thus $T$ is norm-preserving, therefore injective.

\medskip

{\bf (d)} Next, $T$ is surjective.   \Prf\
Suppose that $\familyiI{u_i}\in W_{\infty}^+$ is non-negative, and that
$\epsilon>0$.   Let $n\in\Bbb N$ be such that
$n\epsilon\ge\sup_{i\in I}\|u_i\|_{\infty}$, and for $k\le n$, $i\in I$ set
$a_{ki}=\Bvalue{u_i>k\epsilon}$.   Set
$w=\epsilon\sum_{k=1}^n\chi(\familyiI{a_{ki}})$.   Then
$w\in L^{\infty}(\frak A)$ and $Tw=\familyiI{v_i}$, where
$v_i=\epsilon\sum_{k=1}^n\chi a_{ki}$, so that $v_i\le u_i$ and
$\|u_i-v_i\|_{\infty}\le\epsilon$, for every $i\in I$.
Thus $\|Tw-\familyiI{u_i}\|_{\infty}\le\epsilon$.

As $\familyiI{u_i}$ and $\epsilon$ are arbitrary,
$T[L^{\infty}(\frak A)]\cap W_{\infty}^+$ is norm-dense in $W_{\infty}^+$.
But $T$ is an isometry and $L^{\infty}(\frak A)$ is norm-complete, so
$T[L^{\infty}(\frak A)]$ is closed in $W_{\infty}$ and includes
$W_{\infty}^+$ and therefore $W_{\infty}$;  that is, $T$ is surjective.\
\Qed

So $T$ is a norm-preserving bijective Riesz homomorphism, that is, a normed
Riesz space isomorphism.   Finally, by 353Pd or otherwise, $T$ is
multiplicative, so is an $f$-algebra isomorphism.
}%end of proof of 377A

\leader{377B}{Theorem} Let $\familyiI{(\frak A_i,\bar\mu_i)}$ be a
non-empty
family of probability algebras, and $(\frak B,\bar\nu)$ a probability
algebra.   Let $\frak A$ be the simple product of $\familyiI{\frak A_i}$,
and $\pi:\frak A\to\frak B$ a Boolean homomorphism such that
$\bar\nu\pi(\familyiI{a_i})\le\sup_{i\in I}\bar\mu_ia_i$ whenever
$\familyiI{a_i}\in\frak A$.
Let $W_0$ be the subspace of $\prod_{i\in I}L^0(\frak A_i)$ consisting
of families $\familyiI{u_i}$ such that
$\inf_{k\in\Bbb N}\sup_{i\in I}\bar\mu_i\Bvalue{|u_i|>k}=0$.

(a) $W_0$ is
a solid linear subspace and a subalgebra of $\prod_{i\in I}L^0(\frak A_i)$,
and there is a unique Riesz homomorphism
$T:W_0\to L^0(\frak B)$ such that
$T(\familyiI{\chi a_i})=\chi\pi(\familyiI{a_i})$ whenever
$\familyiI{a_i}\in\frak A$.   Moreover, $T$ is multiplicative,
and $\Bvalue{Tu>0}\Bsubseteq\pi(\familyiI{\Bvalue{u_i>0}})$ whenever
$u=\familyiI{u_i}$ belongs to $W_0$.

(b) If $h:\Bbb R\to\Bbb R$ is a continuous function, and we write
$\bar h$ for the corresponding maps from $L^0$ to itself for any of the
spaces $L^0=L^0(\frak A_i)$, $L^0=L^0(\frak B)$\cmmnt{ (364H)}, then
$\familyiI{\bar h(u_i)}\in W_0$ and
$T(\familyiI{\bar h(u_i)})=\bar h(Tu)$ whenever
$u=\familyiI{u_i}$ belongs to $W_0$.

\proof{{\bf (a)} For
$u=\familyiI{u_i}\in\prod_{i\in I}L^0(\frak A_i)$ and $k\in\Bbb N$, set
$\gamma_k(u)=\sup_{i\in I}\bar\mu_i\Bvalue{|u_i|>k}$.

\medskip

\quad{\bf (i)} $W_0$ is a solid linear
subspace and subalgebra of the $f$-algebra
$\prod_{i\in I}L^0(\frak A_i)$.   \Prf\ For $k\in\Bbb N$ and $u$,
$v\in\prod_{i\in I}L^0(\frak A_i)$,

\Centerline{$\gamma_k(u)\le\gamma_k(v)$ whenever $|u|\le|v|$,}

\Centerline{$\gamma_{2k}(u+v)\le\gamma_k(u)+\gamma_k(v)$,}

\Centerline{$\gamma_{k^2}(u\times v)\le\gamma_k(u)+\gamma_k(v)$}

\noindent for all $u$, $v\in\prod_{i\in I}L^0(\frak A_i)$ and $k\in\Bbb N$.
So $W_0$ is solid, is closed under addition, and is closed under
multiplication.\ \Qed

\medskip

\quad{\bf (ii)}
Let $W_{\infty}\subseteq W_0$ be the set of families
$\familyiI{u_i}\in\prod_{i\in I}L^{\infty}(\frak A_i)$ such that
$\sup_{i\in I}\|u_i\|_{\infty}$ is finite;  by 377A, we can identify
$W_{\infty}$ with $L^{\infty}(\frak A)$.   We therefore have a
corresponding multiplicative
Riesz homomorphism $S:W_{\infty}\to L^{\infty}(\frak B)$ such
that $S(\familyiI{\chi a_i})=\chi\pi(\familyiI{a_i})$ whenever
$\familyiI{a_i}\in\frak A$ (363F);  note that
$S(\familyiI{\chi 1_{\frak A_i}})=\chi 1_{\frak B}$.

\medskip

\quad{\bf (iii)} If $u=\familyiI{u_i}\in W_{\infty}$ and $k\in\Bbb N$, then
$\Bvalue{Su>k}\Bsubseteq\pi(\familyiI{\Bvalue{u_i>k}})$.   \Prf\ Setting
$a_i=\Bvalue{u_i>k}$, we have
$u_i\times\chi(1_{\frak A_i}\Bsetminus a_i)\le k\chi 1_{\frak A_i}$ for
every $i$.   Set $a=\familyiI{a_i}$.
Since $S$ is a multiplicative Riesz homomorphism,

$$\eqalignno{Su\times\chi(1_{\frak B}\Bsetminus\pi a)
&=Su\times\chi\pi(\familyiI{1_{\frak A_i}\Bsetminus a_i})
=S(\familyiI{u_i})\times S(\familyiI{\chi(1_{\frak A_i}\Bsetminus a_i)})\cr
&=S(\familyiI{u_i}\times\familyiI{\chi(1_{\frak A_i}\Bsetminus a_i}))
=S(\familyiI{u_i\times\chi(1_{\frak A_i}\Bsetminus a_i)})\cr
&\le S(\familyiI{k\chi 1_{\frak A_i}})
=k\chi 1_{\frak B}\cr}$$

\noindent and $\Bvalue{Su>k}\Bsubseteq\pi a$, as claimed.\ \Qed


\medskip

\quad{\bf (iv)} If $u=\familyiI{u_i}\in W_0^+$, then
$\sup\{Sv:v\in W_{\infty}$, $0\le v\le u\}$ is defined in
$L^0(\frak B)$.   \Prf\ Set $A_u=S[W_{\infty}\cap[0,u]\,]$.
Because $W_{\infty}\cap[0,u]$ is upwards-directed, so is $A$.
If $v=\familyiI{v_i}\in W_{\infty}\cap[0,u]$, then
$\Bvalue{Sv>k}\Bsubseteq\pi(\familyiI{\Bvalue{v_i>k}})$, by (iii), so

\Centerline{$\bar\nu\Bvalue{Sv>k}\le\sup_{i\in I}\bar\mu_i\Bvalue{v_i>k}
\le\gamma_k(u)$.}

\noindent Thus $\bar\nu\Bvalue{w>k}\le\gamma_k(u)$ for every $w\in A$.
Since $u\in W_0$, $\lim_{k\to\infty}\gamma_k(u)=0$;  so 364L(a-ii)
tells us that $\sup A_u$ is defined in $L^0(\frak B)$.\ \QeD\

By 355F, there is a Riesz homomorphism $T:W_0\to L^0(\frak B)$
extending $S$ and such that $Tu=A_u$ for every
$u\in W_0^+$.   By 353Pd, $T$ is multiplicative.

\medskip

\quad{\bf (v)} Because $T$ is multiplicative, we can repeat the
calculations of (iii), with $T$ in place of $S$, to see that

\Centerline{$\Bvalue{Tu>k}\Bsubseteq\pi(\familyiI{\Bvalue{u_i>k}})$}

\noindent whenever $u=\familyiI{u_i}\in W_0$;  in particular,
$\Bvalue{Tu>0}\Bsubseteq\pi(\familyiI{\Bvalue{u_i>0}})$.

\medskip

\quad{\bf (vi)} To see that $T$ is uniquely defined, let
$T':W_0\to L^0(\frak B)$ be another Riesz homomorphism agreeing with $T$ on
families of the form $\familyiI{\chi a_i}$.
Then $T$ and $T'$ agree on $W_{\infty}\cong L^{\infty}(\frak A)$,
by the uniqueness guaranteed in
363Fa, and $T'$ also is multiplicative, by 353Pd again.   As in (v), we
therefore have

\Centerline{$\Bvalue{Tu>k}\Bcup\Bvalue{T'u>k}
\Bsubseteq\pi(\familyiI{\Bvalue{u_i>k}})$,
\quad$\bar\nu(\Bvalue{Tu>k}\Bcup\Bvalue{T'u>k})\le\gamma_k(u)$}

\noindent whenever $u\in W_0$ and $k\in\Bbb N$.

Suppose that $u\in W_0^+$ and $\epsilon>0$.   Then there is a
$k\in\Bbb N$ such that $\gamma_k(u)\le\epsilon$.   Set
$v_i=u_i\wedge k\chi 1_{\frak A_i}$ for each $i$, and
$v=\familyiI{v_i}$.   Then $Tv=T'v$, so

\Centerline{$\bar\nu\Bvalue{|Tu-T'u|>0}
\le\bar\nu(\Bvalue{Tu-Tv>0}\Bcup\Bvalue{T'u-T'v>0})
\le\gamma_0(u-v)=\gamma_k(u)\le\epsilon$.}

\noindent As $\epsilon$ is arbitrary,
$Tu=T'u$;  as $u$ is arbitrary, $T=T'$.

\medskip

{\bf (b)(i)} If $\epsilon>0$, there is a $k\in\Bbb N$ such that
$\bar\mu_i\Bvalue{|u_i|>k}\le\epsilon$ for every $i\in I$.   Now there is
an $l\in\Bbb N$ such that $|h(t)|\le l$ whenever $|t|\le k$.   So
$\Bvalue{|\bar h(u_i)|>l}\Bsubseteq\Bvalue{|u_i|>k}$ and
$\bar\mu_i\Bvalue{|\bar h(u_i)|>l}\le\epsilon$ for every $i\in I$.
As $\epsilon$ is arbitrary, $\familyiI{\bar h(u_i)}\in W_0$.

\medskip

\quad{\bf (ii)} Again take any $\epsilon>0$.   Let $k\in\Bbb N$ be such
that $\bar\mu_ia_i\le\epsilon$ for every $i\in I$, where
$a_i=\Bvalue{|u_i|>k}$.   By the Stone-Weierstrass theorem in the form
281E, there is a polynomial $g:\Bbb R\to\Bbb R$ such that
$|g(t)-h(t)|\le\epsilon$ whenever $|t|\le k$.   Setting
$v_i=\bar h(u_i)$, $v'_i=\bar g(u_i)$, $v=\familyiI{u_i}$ and
$v'=\familyiI{v'_i}$, we have
$\Bvalue{|v_i-v'_i|>\epsilon}\Bsubseteq a_i$ for every $i$
(use 364Ib for a quick check of the calculation).   Because $T$ is
multiplicative (and $T(\familyiI{\chi 1_{\frak A_i}})=\chi 1_{\frak B}$),
$Tv'=\bar g(Tu)$.   So

$$\eqalignno{\Bvalue{|Tv-\bar h(Tu)|>2\epsilon}
&\Bsubseteq\Bvalue{T|v-v'|>\epsilon}
   \Bcup\Bvalue{|\bar g(Tu)-\bar h(Tu)|>\epsilon}\cr
&\Bsubseteq\pi(\familyiI{\Bvalue{|v_i-v'_i|>\epsilon}})
   \Bcup\Bvalue{|Tu|>k}\cr
\displaycause{using (b)}
&\Bsubseteq\pi(\familyiI{a_i})\cr}$$

\noindent (see (a-v) above),
which has measure at most $\epsilon$.   As $\epsilon$ is
arbitrary, $Tv=\bar h(Tu)$, as claimed.
}%end of proof of 377B

\leader{377C}{Theorem} Let $\familyiI{(\frak A_i,\bar\mu_i)}$ be a
non-empty
family of probability algebras, $(\frak B,\bar\nu)$ a probability algebra,
and $\pi:\prod_{i\in I}\frak A_i\to\frak B$ a Boolean homomorphism such
that $\bar\nu\pi(\familyiI{a_i})\le\sup_{i\in I}\bar\mu_ia_i$ whenever
$\familyiI{a_i}\in\prod_{i\in I}\frak A_i$.
Let $W_0\subseteq\prod_{i\in I}L^0(\frak A_i)$ and $T:W_0\to L^0(\frak B)$
be as in 377B.
Suppose {\it either} that every $\frak A_i$ is atomless {\it or} that
there is an ultrafilter $\Cal F$ on $I$ such that
$\bar\nu\pi(\familyiI{a_i})=\lim_{i\to\Cal F}\bar\mu_ia_i$ whenever
$\familyiI{a_i}$ in $\prod_{i\in I}\frak A_i$.
For $1\le p\le\infty$
let $W_p$ be the subspace of $\prod_{i\in I}L^0(\frak A_i)$
consisting of families $\familyiI{u_i}$ such that
$\sup_{i\in I}\|u_i\|_p$ is finite.   Then
$T[W_p]\subseteq L^p(\frak B,\bar\nu)$, and
$\|Tu\|_p\le\sup_{i\in I}\|u_i\|_p$ whenever $u=\familyiI{u_i}$ belongs to
$W_p$.

\proof{{\bf (a)} I should begin by explaining why $W_1\subseteq W_0$.
All we need to observe is that if $u=\familyiI{u_i}$ belongs to $W_1$,
so that $\gamma=\sup_{i\in I}\|u_i\|_1$ is finite, then

\Centerline{$\inf_{k\ge 1}\sup_{i\in I}\bar\mu_i\Bvalue{u_i>k}
\le\inf_{k\ge 1}\Bover{\gamma}k=0$,}

\noindent so $u\in W_0$.   Of course we now have $W_p\subseteq W_1$ for
$p\ge 1$, because every $(\frak A_i,\bar\mu_i)$ is a probability algebra.

\medskip

{\bf (b)} I start real work on the proof
with a note on the case in which every $\frak A_i$
is atomless.   Suppose that this is so, and that we are given a family
$\familyiI{a_i}\in\prod_{i\in I}\frak A_i$ and $\gamma\in\Bbb Q\cap[0,1]$.
Then there is a family $\familyiI{a'_i}$ such that
$a'_i\Bsubseteq a_i$ and $\bar\mu_ia'_i=\gamma\mu_ia_i$ for every $i\in I$,
and

\Centerline{$\gamma\bar\nu\pi(\familyiI{a_i})
\le\bar\nu\pi(\familyiI{a'_i})$.}

\Prf\ For each $i\in I$, we can find a non-decreasing
family $\langle a_{it}\rangle_{t\in[0,1]}$ in $\frak A_i$ such that
$a_{i1}=a_i$ and $\bar\mu_ia_{it}=t\bar\mu a_i$ for every $t\in[0,1]$.
Set $b(t)=\pi(\familyiI{a_{it}})$ and
$\beta(t)=\bar\nu b(t)$ for $t\in[0,1]$;  then
$\beta(s)\le\beta(t)\le\beta(s)+t-s$ for $0\le s\le t\le 1$, because

\Centerline{$\beta(t)-\beta(s)
=\bar\nu\pi(\familyiI{a_{it}\Bsetminus a_{is}})
\le\sup_{i\in I}\bar\mu_i(a_{it}\Bsetminus a_{is})
=(t-s)\sup_{i\in I}\bar\mu_ia_i
\le t-s$.}

\noindent Let $n\ge 1$ be such that $\Bover1n\le\epsilon$ and $m=n\gamma$
is an integer, and set
$\alpha_i=\beta(\bover{i+1}n)-\beta(\bover{i}n)$ for $i<n$;  then

\Centerline{$\sum_{i=0}^{n-1}\alpha_i=\beta(1)=\bar\nu b(1)$.}

\noindent Consider the possible values of
$\gamma_K=\sum_{k\in K}\alpha_k$ for
sets $K\in[n]^m$.
(I am thinking of $n$ as the set $\{0,1,\ldots,n-1\}$.)
The average value of $\gamma_K$ over all $m$-element subsets of $n$ is just
$\bover{m}n\beta(1)=\gamma\beta(1)$, so there is some $K$ such that
$\gamma_K\ge\gamma\beta(1)$.

Set

\Centerline{$a'_i=\sup_{k\in K}a_{i,(k+1)/n}\Bsetminus a_{i,k/n}$}

\noindent for $i\in I$.   Then $\bar\mu_ia'_i=\gamma\bar\mu_ia_i$ for
every $i$, while

\Centerline{$\bar\nu\pi(\familyiI{a'_i})
=\sup_{k\in K}\bar\nu(b(\bover{k+1}n)\Bsetminus b(\bover{k}n))
=\sum_{k\in K}\alpha_k$}

\noindent is at least $\gamma\beta(1)$, as required.\ \Qed

\medskip

{\bf (c)} We find now that under either of the hypotheses proposed,

\Centerline{$\sum_{k=0}^n\gamma_k\bar\nu\pi(\familyiI{a_{ki}})
\le\sup_{i\in I}\sum_{k=0}^n\gamma_k\mu_ia_{ki}$}

\noindent whenever $\gamma_0,\ldots,\gamma_n\ge 0$ are rational and
$\langle a_{ki}\rangle_{k\le n}$ is a disjoint family in $\frak A_i$ for
each $i\in I$.

\medskip

\Prf{\bf (i)} Consider first the case in which every $\frak A_i$ is
atomless and every $\gamma_k$ is between $0$ and $1$.   In this case, given
$\epsilon>0$, (b) above tells us that
we can find $a'_{ki}\Bsubseteq a_{ki}$, for $i\in I$ and
$k\le n$, such that $\bar\mu_ia'_{ki}=\gamma_k\bar\mu_ia_{ki}$ and

\Centerline{$\gamma_k\bar\nu\pi(\familyiI{a_{ki}})
\le\bar\nu\pi(\familyiI{a'_{ki}})$.}

\noindent Set $c_i=\sup_{k\le n}a'_{ki}$ for $i\in I$;  then

$$\eqalign{\sum_{k=0}^n\gamma_k\bar\nu\pi(\familyiI{a_{ki}})
&\le\sum_{k=0}^n\bar\nu\pi(\familyiI{a'_{ki}})
=\bar\nu\pi(\sup_{k\le n}\familyiI{a'_{ki}})
=\bar\nu\pi(\familyiI{c_i})\cr
&\le\sup_{i\in I}\bar\mu_ic_i
=\sup_{i\in I}\sum_{k=0}^n\bar\mu_ia'_{ki}
=\sup_{i\in I}\sum_{k=0}^n\gamma_k\bar\mu_ia_{ki},\cr}$$

\noindent as required.

\medskip

\quad{\bf (ii)} Because $T$ is linear, it follows at once that the result
is true for any rational $\gamma_0,\ldots,\gamma_n\ge 0$, if every
$\frak A_i$ is atomless.

\medskip

\quad{\bf (iii)} Now consider the case in which there is an ultrafilter
$\Cal F$ on $I$ such that
$\bar\nu\pi(\familyiI{a_i})=\lim_{i\to\Cal F}\bar\mu_ia_i$ for every
$\familyiI{a_i}$.   In this case, given $\epsilon>0$, the set

\Centerline{$J=\{j:j\in I$,
$\bar\nu(\familyiI{a_{ki}})\le\bar\mu_ja_{kj}+\epsilon$
for every $k\le n\}$}

\noindent belongs to $\Cal F$ and is not empty.   Take any $j\in J$;  then

\Centerline{$\sum_{k=0}^n\gamma_k\bar\nu\pi(\familyiI{a_{ki}})
\le\sum_{k=0}^n\gamma_k(\bar\mu_ja_{kj}+\epsilon)
\le\epsilon\sum_{k=0}^n\gamma_k
  +\sup_{i\in I}\sum_{k=0}^n\gamma_k\bar\mu_ia_{ki}$.}

\noindent As $\epsilon$ is arbitrary, we again have the result.\ \Qed

\medskip

{\bf (d)} Next,
$\int Tu\le\sup_{i\in I}\int u_i$ whenever $u=\familyiI{u_i}$ belongs to
$W_{\infty}^+$.   \Prf\ Let $\epsilon>0$ and let $n\in\Bbb N$ be such that
$\|u_i\|_{\infty}\le n\epsilon$ for every $i\in I$.   For $i\in I$ and
$k\le n$, set
$a_{ki}=\Bvalue{u_i>k\epsilon}\Bsetminus\Bvalue{u_i>(k+1)\epsilon}$;
for $i\in I$, set $u'_i=\sum_{k=0}^nk\epsilon\chi a_{ki}$;
then $u'_i\le u_i\le u'_i+\epsilon\chi 1_{\frak A_i}$.   Setting
$u'=\familyiI{u'_i}$, $Tu\le Tu'+\epsilon\chi 1_{\frak B}$, so

$$\eqalignno{\int Tu-\epsilon
&\le\int Tu'
=\int\sum_{k=0}^nk\epsilon\chi\pi(\familyiI{a_{ki}})\cr
&=\sum_{k=0}^nk\epsilon\bar\nu\pi(\familyiI{a_{ki}})
\le\sup_{i\in I}\sum_{k=0}^nk\epsilon\bar\mu_ia_{ki}\cr
\displaycause{by (c)}
&=\sup_{i\in I}\int u'_i
\le\sup_{i\in I}\int u_i.\cr}$$

\noindent As $\epsilon$ is arbitrary, we have the result.\ \Qed

\medskip

{\bf (d)} It follows that $Tu\in L^1(\frak B,\bar\nu)$ and
$\int Tu\le\sup_{i\in I}\int u_i$ whenever $u=\familyiI{u_i}$ belongs to
$W_1^+$ and $u\ge 0$.
\Prf\ Set $\gamma=\sup_{i\in I}\int u_i$.   Let
$\epsilon>0$.   Set $\gamma'=\gamma/\epsilon$.
For $i\in I$ set
$v_i=u_i\wedge\gamma'\chi 1_{\frak A_i}$;  set $v=\familyiI{v_i}$.   Then
$v\in W_{\infty}$ and

\Centerline{$\int Tv\le\sup_{i\in I}\int v_i
\le\sup_{i\in I}\int u_i=\gamma$}

\noindent by (c) above.   Also
$\Bvalue{Tu-Tv>0}\Bsubseteq\pi(\familyiI{\Bvalue{u_i>\gamma'}})$,
by 377Ba, so

\Centerline{$\bar\nu\Bvalue{Tu-Tv>0}
\le\sup_{i\in I}\bar\mu_i\Bvalue{u_i>\gamma'}\le\epsilon$.}

Thus for each $n\in\Bbb N$ we can find a $w_n\in L^{\infty}(\frak B)$ such
that $0\le w_n\le Tu$, $\int w_n\le\gamma$ and
$\bar\nu\Bvalue{Tu-w_n>0}\le 2^{-n}$.
Set $w'_n=\inf_{i\ge n}w_i$ for each
$n$;  then $\sequencen{w'_n}$ is a non-decreasing sequence with
supremum $Tu$ in $L^0(\frak B)$,
while $\int w'_n\le\gamma$ for every $n$.   Consequently
$Tu\in L^1(\frak B,\bar\nu)$ and $\int Tu\le\gamma$, as claimed.\ \Qed

\medskip

{\bf (e)} Because $T$ is a Riesz homomorphism, $Tu\in L^1(\frak B,\bar\nu)$
and $\|Tu\|_1=\int T|u|$ is at most
$\sup_{i\in I}\int|u_i|=\sup_{i\in I}\|u_i\|_1$ for every $u\in W_1$.

\medskip

{\bf (f)} Now suppose that $p\in\ooint{1,\infty}$ and that
$u=\familyiI{u_i}$ belongs to $W_p$.   In this case, $\familyiI{|u_i|^p}$
belongs to $W_1$, so $T(\familyiI{|u_i|^p})\in L^1(\frak B,\bar\nu)$ and
$\int T(\familyiI{|u_i|^p})\le\sup_{i\in I}\int|u_i|^p$.   By 377Bb,
with $h(t)=|t|^p$, $T(\familyiI{|u_i|^p})=|Tu|^p$.   So
$Tu\in L^p(\frak B,\bar\nu)$ and

\Centerline{$\|Tu\|_p
=(\int|Tu|^p)^{1/p}
\le\sup_{i\in I}(\int|u_i|^p)^{1/p}
=\sup_{i\in I}\|u_i\|_p$}

\noindent as claimed.
}%end of proof of 377C

\leader{377D}{}\cmmnt{ The original motivation for the work of this
section was to understand the function spaces associated with the reduced
products of \S328.   For these we have various simplifications in addition
to that observed in 377C.

\medskip

\noindent}{\bf Theorem} Let $\familyiI{(\frak A_i,\bar\mu_i)}$ be a
family of probability algebras, $\Cal F$ an ultrafilter on $I$,
and $(\frak B,\bar\nu)$ a probability
algebra.   Let $\frak A$ be the simple product $\prod_{i\in I}\frak A_i$
and $\pi:\frak A\to\frak B$ a Boolean homomorphism such that
$\bar\nu\pi(\familyiI{a_i})=\lim_{i\to\Cal F}\bar\mu_ia_i$
whenever $\familyiI{a_i}\in\frak A$.   Let
$W_0\subseteq\prod_{i\in I}L^0(\frak A_i)$ and $T:W_0\to L^0(\frak B)$ be
as in 377B-377C.

(a) If $u=\familyiI{u_i}$ belongs to $W_0$ and $\{i:i\in I$, $u_i=0\}\in\Cal F$, then
$Tu=0$.

(b) For $1\le p\le\infty$, write $W_p$ for the set of those families
$\familyiI{u_i}\in\prod_{i\in I}L^p(\frak A_i,\bar\mu_i)$ such that
$\sup_{i\in I}\|u_i\|_p$ is finite.   Then $Tu\in L^p(\frak B,\bar\nu)$
and $\|Tu\|_p\le\lim_{i\to\Cal F}\|u_i\|_p$ whenever
$u=\familyiI{u_i}$ belongs to $W_p$.

(c) Let $W_{ui}$ be the subspace of
$\prod_{i\in I}L^1(\frak A_i,\bar\mu_i)$
consisting of families $\familyiI{u_i}$ such
that $\inf_{k\in\Bbb N}\sup_{i\in I}\int(|u_i|-k\chi 1_{\frak A_i})^+=0$.
Then $\int Tu=\lim_{i\to\Cal F}\int u_i$ and
$\|Tu\|_1=\lim_{i\to\Cal F}\|u_i\|_1$
whenever $u=\familyiI{u_i}$ belongs to $W_{ui}$.

(d) Suppose now that $\pi[\frak A]=\frak B$.

\quad(i) $T[W_0]=L^0(\frak B)$.

\quad(ii) $T[W_{ui}]=L^1(\frak B,\bar\nu)$.

\quad(iii) If $p\in[1,\infty]$, then $T[W_p]=L^p(\frak B,\bar\nu)$ and for
every $w\in L^p(\frak B,\bar\nu)$ there is a
$u=\familyiI{u_i}$ in $W_p$ such that $Tu=w$ and
$\sup_{i\in I}\|u_i\|_p=\|w\|_p$.

\proof{{\bf (a)} Setting

$$\eqalign{a_i&=1_{\frak A_i}\text{ if }u_i\ne 0,\cr
&=0\text{ if }u_i=0,\cr}$$

\noindent $\familyiI{a_i}\in\frak A$ and
$\bar\nu\pi(\familyiI{a_i})=\lim_{i\to\Cal F}\bar\mu_ia_i=0$, so
$\pi(\familyiI{a_i})=0$.   Accordingly

\Centerline{$Tu=T(\familyiI{u_i\times\chi a_i})
=Tu\times T(\familyiI{\chi a_i})=Tu\times\chi\pi(\familyiI{a_i})=0$.}

\medskip

{\bf (b)} Suppose that $u=\familyiI{u_i}\in W_p$ and that $J\in\Cal F$.
Set

$$\eqalign{v_i&=u_i\text{ if }i\in J,\cr
&=0\text{ if }i\in I\setminus J;\cr}$$

\noindent then, putting (a) and 377C together,

\Centerline{$\|Tu\|_p=\|Tv\|_p\le\sup_{i\in I}\|v_i\|_p
=\sup_{i\in J}\|u_i\|_p$.}

\noindent As $J$ is arbitrary,
$\|Tu\|_p\le\lim_{i\to\Cal F}\|u_i\|_p$.

\medskip

{\bf (c)(i)} Clearly $W_{ui}$ is a
solid linear subspace of $W_1$.
Suppose that $u=\familyiI{u_i}\in W_{ui}^+$ and $\epsilon>0$.
Let $n\ge 1$ be such that
$\int(u_i-n\epsilon\chi 1_{\frak A_i})^+\le\epsilon$ for every $i\in I$.
For $i\in I$ and $k\le n$, set $a_{ki}=\Bvalue{u_i>k\epsilon}$;
set $v_i=\sum_{k=1}^nk\epsilon\chi a_{ki}$, so that

\Centerline{$v_i\le u_i
\le v_i+\epsilon\chi 1_{\frak A_i}+(u_i-n\epsilon\chi 1_{\frak A_i})^+$,
\quad$\int u_i\le\int v_i+2\epsilon$.}

\noindent If $v=\familyiI{v_i}$, then

$$\eqalign{\int Tu
&=\|Tu\|_1
\le\lim_{i\to\Cal F}\|u_i\|_1
=\lim_{i\to\Cal F}\int u_i\cr
&\le 2\epsilon+\lim_{i\to\Cal F}\int v_i
=2\epsilon+\sum_{k=1}^nk\epsilon\lim_{i\to\Cal F}\bar\mu_ia_{ki}\cr
&=2\epsilon+\sum_{k=1}^nk\epsilon\bar\nu\pi(\familyiI{a_{ki}})
=2\epsilon+\int\sum_{k=1}^nk\epsilon\chi\pi(\familyiI{a_{ki}})\cr
&=2\epsilon+\int Tv
\le 2\epsilon+\int Tu.\cr}$$

\noindent As $\epsilon$ is arbitrary, $\int Tu=\lim_{i\to\Cal F}\int u_i$.

\medskip

\quad{\bf (ii)} It follows at once that
$\int Tu=\lim_{i\to\Cal F}\int u_i$
and that

\Centerline{$\|Tu\|_1=\int|Tu|=\int T|u|=\lim_{i\to\Cal F}\int|u_i|
=\lim_{i\to\Cal F}\|u_i\|_1$}

\noindent whenever $u=\familyiI{u_i}\in W_{ui}$.

\medskip

{\bf (d)(i)}\grheada\
Let $T_{\pi}:L^{\infty}(\frak A)\to L^{\infty}(\frak B)$
be the Riesz homomorphism associated with the Boolean homomorphism
$\pi:\frak A\to\frak B$.   Since $\pi$ is surjective,
363Fd tells us that $T_{\pi}$ is surjective.
Identifying $W_{\infty}$ with
$L^{\infty}(\frak A)$, and $T\restr W_{\infty}$ with
$T_{\pi}$, as in part (a) of the
the proof of 377B, we see that $T[W_{\infty}]=L^{\infty}(\frak B)$.
Moreover, 363Fd tells us also that if $w\in L^{\infty}$ there is a
$v\in L^{\infty}(\frak A)$ such that $T_{\pi}v=w$ and
$\|v\|_{\infty}=\|w\|_{\infty}$;  translating this into terms of
$W_{\infty}$, we have a $u=\familyiI{u_i}\in W_{\infty}$ such that
$Tu=w$ and $\sup_{i\in I}\|u_i\|_{\infty}=\|w\|_{\infty}$.

It will be useful to know that if $b\in\frak B$ and $\epsilon>0$ there is a
family $\familyiI{a_i}\in\frak A$ such that $\pi(\familyiI{a_i})=b$ and
$\sup_{i\in I}\bar\mu_ia_i\le\bar\nu b+\epsilon$.   \Prf\ By hypothesis,
there is a family $\familyiI{a'_i}\in\frak A$ such that
$\pi(\familyiI{a'_i})=b$, and $\bar\nu b=\lim_{i\to\Cal F}\bar\mu_ia_i$.
Set

$$\eqalign{a_i
&=a'_i\text{ if }\bar\mu_ia_i\le\bar\nu b+\epsilon,\cr
&=0\text{ for other }i\in I.\cr}$$

\noindent Then $\lim_{i\to\Cal F}\bar\mu_i(a'_i\Bsymmdiff a_i)=0$ so
$\pi(\familyiI{a'_i\Bsymmdiff a_i})=0$ and $\pi(\familyiI{a_i})=b$, while
$\bar\mu_ia_i\le\bar\nu b+\epsilon$ for every $i\in I$.\ \Qed

\medskip

\qquad\grheadb\ Now suppose that $w\in L^0(\frak B)^+$.   For each
$n\in\Bbb N$, set $w_n=w\wedge n\chi 1_{\frak B}$ and
let $u^{(n)}=\familyiI{u_{ni}}\in W_{\infty}$ be such that
$Tu^{(n)}=w_{n+1}-w_n$ and $\|u_{ni}\|_{\infty}\le 1$ for
every $i\in I$.   Next, for each $n$, set $b_n=\Bvalue{w_{n+1}-w_n>0}$,
and let
$\familyiI{a_{ni}}\in\frak A$ be such that $\pi(\familyiI{a_{ni}})=b_n$ and
$\sup_{i\in I}\bar\mu_ia_{ni}\le\bar\nu b_n+2^{-n}$.   If we set
$a'_{ni}=\inf_{m\le n}a_{mi}$ and
$u'_{ni}=u_{ni}\times\chi a'_{ni}$, we shall have

$$\eqalign{T(\familyiI{u'_{ni}})
&=T(\familyiI{u_{ni}})\times\chi\pi(\familyiI{a'_{ni}})\cr
&=(w_{n+1}-w_n)\times\inf_{m\le n}\chi b_m
=w_{n+1}-w_n\cr}$$

\noindent for every $n$.   Also, for each $i\in I$, $\sequencen{a'_{ni}}$
is non-increasing and

\Centerline{$\lim_{n\to\infty}\bar\mu_ia'_{ni}
\le\lim_{n\to\infty}\bar\nu b_n+2^{-n}=0$.}

\noindent So $v_i=\sup_{n\in\Bbb N}\sum_{m=0}^nu'_{ni}$ is defined in
$L^0(\frak A_i)$, and

\Centerline{$
\inf_{k\in\Bbb N}\sup_{i\in I}\bar\mu_i\Bvalue{v_i>k}
\le\inf_{k\in\Bbb N}\sup_{i\in I}\bar\mu_ia'_{ki}=0$.}

\noindent Thus $v=\familyiI{v_i}$ belongs to $W_0$ and we can speak of
$Tv$.   Of course

\Centerline{$Tv\ge\sum_{m=0}^nT(\familyiI{u'_{ni}})=w_{n+1}$}

\noindent for every $n$, so $Tv\ge w$.   On the other hand, for any
$n\in\Bbb N$,

\Centerline{$\Bvalue{v_i-\sum_{m=0}^nu'_{ni}>0}\Bsubseteq a'_{ni}$}

\noindent for every $i$, so $\Bvalue{Tv-w_{n+1}>0}\Bsubseteq b_n$, by 377B;  as
$\inf_{n\in\Bbb N}b_n=0$, $Tv=\sup_{n\in\Bbb N}w_n=w$.

\medskip

\qquad\grheadc\ Thus $T[W_0]\supseteq L^0(\frak B)^+$;  as $T$ is linear,
$T[W_0]=L^0(\frak B)$.

\medskip

\quad{\bf (ii)} Now suppose that $w\in L^1(\frak B,\bar\nu)^+$.
In this case, repeat the process of (i-$\beta$) above.
This time, observe that as $\chi b_{n+1}\le w_{n+1}-w_n$ for every $n$,
$\sum_{n=0}^{\infty}\bar\nu b_n\le 1+\int w$ is finite.   Consequently, in
the first place,

\Centerline{$\sum_{n=0}^{\infty}\int u'_{ni}
\le\sum_{n=0}^{\infty}\bar\mu_ia_{ni}
\le\sum_{n=0}^{\infty}\bar\nu b_n+2^{-n}$}

\noindent is finite, and $v_i\in L^1(\frak A_i,\bar\mu_i)$, for every
$i\in I$.   But also, for any $k\in\Bbb N$ and $i\in I$,

\Centerline{$\int(v_i-k\chi 1_{\frak A_i})^+
\le\sum_{n=k}^{\infty}\int u'_{ni}
\le\sum_{n=k}^{\infty}\bar\nu b_n+2^{-n}
\to 0$}

\noindent $k\to\infty$.   So $v\in W_{ui}$ and $w\in T[W_{ui}]$.   Because
$W_{ui}$ is a linear subspace of $W_0$, $T[W_{ui}]=L^1(\frak B,\bar\nu)$.

\medskip

\quad{\bf (iii)}\grheada\ If $p=\infty$ the result has already been dealt
with in (i-$\alpha$) above.

\medskip

\qquad\grheadb\ For the case $p=1$, take $w\in L^1(\frak B,\bar\nu)$.
Let $v=\familyiI{v_i}\in W_{ui}$ be such that $Tv=w$.   For $i\in I$ set

$$\eqalign{u_i&=\Bover{\|w\|_1}{\|v_i\|_1}v_i
  \text{ if }\|v_i\|_1>\|w\|_1,\cr
&=v_i\text{ otherwise}.\cr}$$

\noindent Then

\Centerline{$\bar\mu\Bvalue{(|u_i|-k>0}
\le\bar\mu\Bvalue{|v_i|-k>0}$}

\noindent for all $k\in\Bbb N$ and $i\in I$, so
$u=\familyiI{u_i}\in W_{ui}$.   Since $\lim_{i\to\Cal F}\|v_i\|_1=\|w\|_1$,
by (c) above, $\lim_{i\to\Cal F}\|u_i-v_i\|_1=0$ and $Tu=Tv=w$, by (b).
And of course $\|u_i\|_1\le\|w\|_1$ for every $i$.

\medskip

\qquad\grheadc\ Now suppose that $1<p<\infty$ and that
$w\in L^p(\frak B,\bar\nu)$.   By ($\beta$), there is a
$v=\familyiI{v_i}\in W_1$ such that $Tv=|w|^p$ and
$\sup_{i\in I}\|v_i\|_1=\|w\|_p^p$.   Set $v'_i=|v_i|^{1/p}$ for each
$i$;  then $v'=\familyiI{v'_i}\in W_p$ and $Tv'=|w|$, by 377Bb.   Next, $w$
is expressible as $|w|\times\tilde w$, where
$\tilde w\in L^{\infty}(\frak B)$ and $\|\tilde w\|_{\infty}\le 1$.
There is a $\tilde v=\familyiI{\tilde v_i}\in W_{\infty}$ such that
$T\tilde v=\tilde w$ and $\sup_{i\in I}\|\tilde v_i\|_{\infty}=1$.
Set $u_i=v'_i\times\tilde v_i$ for each $i$;  then $u=\familyiI{u_i}$
belongs to $W_p$, $\|u_i\|_p\le\|w\|_p$ for every $i$, and $Tu=w$.
}%end of proof of 377D

\leader{377E}{Proposition} Let $(\frak A,\bar\mu)$ and $(\frak B,\bar\nu)$
be probability algebras, $I$ a set and $\Cal F$ an ultrafilter on $I$.
Let $\pi:\frak A^I\to\frak B$ be a Boolean homomorphism such that
$\bar\nu\pi(\familyiI{a_i})=\lim_{i\to\Cal F}\bar\mu a_i$ whenever
$\familyiI{a_i}\in\frak A^I$.   Let $W_0$ be the set of families in
$L^0(\frak A)^I$ which are bounded for the topology of convergence in
measure on $L^0(\frak A)$.

(a)(i) $W_0$ is a solid linear subspace and a subalgebra of
$L^0(\frak A)^I$,
and there is a unique multiplicative Riesz homomorphism
$T:W_0\to L^0(\frak B)$ such that
$T(\familyiI{\chi a_i})=\chi\pi(\familyiI{a_i})$ whenever
$\familyiI{a_i}\in\frak A^I$.

\quad(ii) $\Bvalue{Tu>0}\Bsubseteq\pi(\familyiI{\Bvalue{u_i>0}})$ whenever
$u=\familyiI{u_i}$ belongs to $W_0$.

\quad(iii) If $h:\Bbb R\to\Bbb R$ is a continuous function, and we write
$\bar h$ for the corresponding maps from $L^0$ to itself for either of the
spaces $L^0=L^0(\frak A)$, $L^0=L^0(\frak B)$, then
$\familyiI{\bar h(u_i)}\in W_0$ and
$T(\familyiI{\bar h(u_i)})=\bar h(Tu)$ whenever
$u=\familyiI{u_i}$ belongs to $W_0$.

(b)(i) For $1\le p\le\infty$
let $W_p$ be the subspace of $L^p(\frak A,\bar\mu)^I$
consisting of $\|\,\|_p$-bounded families.   Then
$T[W_p]\subseteq L^p(\frak B,\bar\nu)$, and
$\|Tu\|_p\le\lim_{i\to\Cal F}\|u_i\|_p$
whenever $u=\familyiI{u_i}$ belongs to $W_p$.

\quad(ii) Let $W_{ui}$ be the subspace of
$L^1(\frak A_i,\bar\mu_i)^I$
consisting of uniformly integrable families.
Then $\int Tu=\lim_{i\to\Cal F}\int u_i$ and
$\|Tu\|_1=\lim_{i\to\Cal F}\|u_i\|_1$
whenever $u=\familyiI{u_i}$ belongs to $W_{ui}$.

(c)(i) We have a measure-preserving Boolean homomorphism
$\tilde\pi:\frak A\to\frak B$ defined by setting
$\tilde\pi a=\pi(\familyiI{a})$ for each $a\in\frak A$.

\quad(ii) Let $P_{\tilde\pi}:L^1(\frak B,\bar\nu)\to L^1(\frak A,\bar\mu)$
be the conditional-expectation operator corresponding to
$\tilde\pi:\frak A\to\frak B$\cmmnt{ (365P)}.
If $\familyiI{u_i}$ is a uniformly
integrable family in $L^1(\frak A)$, then $P_{\tilde\pi}T(\familyiI{u_i})$
is the limit $\lim_{i\to\Cal F}u_i$ for the weak topology of
$L^1(\frak A,\bar\mu)$.

\quad(iii) Suppose that $1<p<\infty$ and that $\familyiI{u_i}$ is a
bounded family
in $L^p(\frak A,\bar\mu)$.   Then $P_{\tilde\pi}T(\familyiI{u_i})$ is the
limit $\lim_{i\to\Cal F}u_i$ for the weak topology of
$L^p(\frak A,\bar\mu)$.

\proof{{\bf (a)} The point is that a family $\familyiI{u_i}$ in
$L^0(\frak A)$ is bounded for the topology of convergence in measure iff
$\inf_{k\in\Bbb N}\sup_{i\in I}\bar\mu\Bvalue{|u_i|>k}=0$.
\Prf\ (i) If $\familyiI{u_i}$ is bounded in this sense, take any
$\epsilon>0$.   Then
$G=\{u:u\in L^0(\frak A)$, $\bar\mu\Bvalue{|u|>1}\le\epsilon\}$ is a
neighbourhood of $0$ in $L^0(\frak A)$, so there is a $k\in\Bbb N$ such
that $u_i\in kG$, that is, $\bar\mu\Bvalue{|u_i|>k}\le\epsilon$, for every
$i\in I$.   So $\{u_i:i\in I\}$ satisfies the condition.   (ii) If
$\{u_i:i\in I\}$ satisfies the condition, and $G$ is a neighbourhood of
$0$ in $L^0(\frak A)$, then there is an $\epsilon>0$ such that $G$ includes
$\{u:\bar\mu\Bvalue{|u|>\epsilon}\le\epsilon\}$ (367L).   Now there is
a $k\in\Bbb N$ such that $\bar\mu\Bvalue{|u_i|>k}\le\epsilon$ for every
$i\in I$, in which case, setting $n=\lceil k/\epsilon\rceil$, we have
$u_i\in nG$ for every $i\in I$.
As $G$ is arbitrary, $A$ is bounded.\ \Qed

So we just have a special case of 377B.

\medskip

{\bf (b)} Similarly, the condition
`$\inf_{k\in\Bbb N}\sup_{i\in I}\int(|u_i|-k\chi 1_{\frak A_i})^+=0$'
translates into
`$\{u_i:i\in I\}$ is uniformly integrable' (cf.\ 246Bd), so we are
looking at a special case of 377Db-377Dc.

\medskip

{\bf (c)(i)} $\tilde\pi$ is a Boolean homomorphism just because the
function taking $a\in\frak A$ into the constant family with value $a$ is a
Boolean homomorphism from $\frak A$ to $\frak A^I$.   The formula
`$\bar\nu\pi(\familyiI{a_i})=\lim_{i\to\Cal F}\bar\mu a_i$' now ensures
that $\tilde\pi$ is measure-preserving.

\medskip

\quad{\bf (ii)} By the defining formula for $P_{\tilde\pi}$ (365Pa),

$$\eqalignno{\int_aP_{\tilde\pi}T(\familyiI{u_i})
&=\int T(\familyiI{u_i})\times\chi\tilde\pi(a)
=\int T(\familyiI{u_i})\times\chi\pi(\familyiI{a})\cr
&=\int T(\familyiI{u_i})\times T(\familyiI{\chi a})\cr
&=\int T(\familyiI{u_i\times\chi a})
=\lim_{i\to\Cal F}\int u_i\times\chi a\cr
\displaycause{because $\{u_i\times\chi a:i\in I\}$ is uniformly integrable}
&=\lim_{i\to\Cal F}\int_au_i\cr}$$

\noindent for every $a\in\frak A$.   It follows that
$P_{\tilde\pi}T(\familyiI{u_i})=\lim_{i\to\Cal F}u_i$.   \Prf\ We have

\Centerline{$\int P_{\tilde\pi}T(\familyiI{u_i})\times v
=\lim_{i\to\Cal F}\int u_i\times v$}

\noindent whenever $v=\chi a$, for any $a\in\frak A$;  by linearity,
whenever $v\in S(\frak A)$, the space of $\frak A$-simple functions;
and by continuity, whenever $v\in L^{\infty}(\frak A)$ (because
$\{u_i:i\in I\}$ is $\|\,\|_1$-bounded, and $S(\frak A)$ is
$\|\,\|_{\infty}$-dense in $L^{\infty}(\frak A)$).
Since $L^{\infty}(\frak A)$ can be
identified with the dual of $L^1(\frak A,\bar\mu)$ (365Mc), we have the
required weak convergence.\ \Qed

\medskip

\quad{\bf (iii)} If $\{u_i:i\in I\}$ is $\|\,\|_p$-bounded,
where $1<p<\infty$, then it is uniformly integrable.   \Prf\
Set $q=\Bover{p}{p-1}$.   If $k\ge 1$,

\Centerline{$\inf_{k\ge 1}\sup_{i\in I}\int(|u_i|-k\chi 1_{\frak A})^+
\le\inf_{k\ge 1}\Bover1{k^{p-1}}\sup_{i\in I}\|u_i\|_p^p=0$.  \Qed}

\noindent So

\Centerline{$\int P_{\tilde\pi}T(\familyiI{u_i})\times v
=\lim_{i\to\Cal F}\int u_i\times v$}

\noindent for every $v\in S(\frak A)$, and therefore for every
$v\in L^q(\frak A,\bar\mu)$, since $v$ can be $\|\,\|_q$-approximated by
members of
$S(\frak A)$ (366C).   Since $L^q(\frak A,\bar\mu)$ can be identified with
$L^p(\frak A,\bar\mu)^*$, we again have weak convergence.
}%end of proof of 377E

\leader{377F}{}\cmmnt{ Finally, I come to a result which depends on the
special properties of reduced products of probability algebras.

\medskip

\noindent}{\bf Proposition} Let $(\frak A,\bar\mu)$ and
$(\frak A',\bar\mu')$ be probability algebras, $I$ a set and $\Cal F$ an
ultrafilter on $I$;  let $(\frak B,\bar\nu)$ and $(\frak B',\bar\nu')$ be
the reduced powers $(\frak A,\bar\mu)^I|\Cal F$,
$(\frak A',\bar\mu')^I|\Cal F$\cmmnt{ as described in
328A-328C}, %328A 328B 328C
with corresponding homomorphisms $\pi:\frak A^I\to\frak B$ and
$\pi':{\frak A'}^I\to\frak B'$.

(a) Writing $W_0$, $W_0'$ for the spaces of topologically
bounded families in $L^0(\frak A)^I$, $L^0(\frak A')^I$ respectively,
we have unique Riesz homomorphisms $T:W_0\to L^0(\frak B)$ and
$T':W_0'\to L^0(\frak B')$ such that
$T(\familyiI{\chi a_i})=\chi\pi(\familyiI{a_i})$,
$T'(\familyiI{\chi a'_i})=\chi\pi'(\familyiI{a'_i})$ whenever
$\familyiI{a_i}\in\frak A^I$ and $\familyiI{a'_i}\in(\frak A')^I$.

(b) Suppose that $S:L^1(\frak A,\bar\mu)\to L^1(\frak A',\bar\mu')$ is a
bounded linear operator.   Then we have a unique bounded linear operator
$\hat S:L^1(\frak B,\bar\nu)\to L^1(\frak B',\bar\nu')$ such that
$\hat ST(\familyiI{u_i})=T'(\familyiI{Su_i})$ whenever $\familyiI{u_i}$
is a uniformly integrable family in $L^1(\frak A,\bar\mu)$.

(c) The map
$S\mapsto\hat S$ is a norm-preserving Riesz homomorphism from
$\eurm B(L^1(\frak A,\bar\mu);L^1(\frak A',\bar\mu'))$ to
$\eurm B(L^1(\frak B,\bar\nu);\penalty-100L^1(\frak B',\bar\nu'))$.


\proof{{\bf (a)} Once again, this is nothing but a specialization of the
corresponding fragments of 377Ba and 377Ea.

\medskip

{\bf (b)} Write $W_{ui}$ for the space of uniformly integrable families
in $L^1_{\bar\mu}=L^1(\frak A,\bar\mu)$.   If $\familyiI{u_i}\in W_{ui}$,
then $\familyiI{Su_i}$ is uniformly integrable in
$L^1_{\bar\mu'}=L^1(\frak A',\bar\mu')$ (because $\{u_i:i\in I\}$ and
$\{Su_i:i\in I\}$ are relatively weakly compact, as in 247D),
so belongs to $W'_0$, and we can speak of
$T'(\familyiI{Su_i})$.   If moreover $T(\familyiI{u_i})=0$, then
$\lim_{i\to\Cal F}\|u_i\|_1=0$ (377E(b-ii)), so
$\lim_{i\to\Cal F}\|Su_i\|_1=0$ and $T'(\familyiI{Su_i})=0$.
Finally, $T[W_{ui}]=L^1_{\bar\nu}=L^1(\frak B,\bar\nu)$ by 377D(d-ii).
So the given formula defines a linear operator
$\hat S:L^1_{\bar\nu}\to L^1_{\bar\nu'}=L^1(\frak B',\bar\nu')$.
Next, if $w\in L^1_{\bar\nu}$, we can take any family
$\familyiI{u_i}\in W_{ui}$ such that $T(\familyiI{u_i})=w$, and

$$\eqalignno{\|\hat Sw\|_1
&=\|T'(\familyiI{Su_i})\|_1
=\lim_{i\to\Cal F}\|Su_i\|_1\cr
\displaycause{377E(b-ii)}
&\le\|S\|\lim_{i\to\Cal F}\|u_i\|_1
=\|S\|\|w\|_1.\cr}$$

\noindent As $w$ is arbitrary, $\hat S$ is a bounded linear operator, and
$\|\hat S\|\le\|S\|$.   On the other hand, if
$u\in L^1_{\bar\mu}$ and $\|u\|_1\le 1$,
$\|T(\familyiI{u})\|_1\le 1$ so

\Centerline{$\|\hat S\|\ge\|\hat ST(\familyiI{u})\|
=\|T'(\familyiI{Su})\|_1=\|Su\|_1$;}

\noindent as $u$ is arbitrary, $\|\hat S\|\ge\|S\|$.

\medskip

{\bf (c)(i)} Recall from 371D that the Banach space
$\eurm B(L^1_{\bar\mu};L^1_{\bar\mu'})$ of continuous
linear operators is also the Dedekind complete Riesz space
$L^{\sim}(L^1_{\bar\mu};L^1_{\bar\mu'})$ of order-bounded linear operators,
and its norm is a Riesz norm;  similarly,
$\eurm B(L^1_{\bar\nu};L^1_{\bar\nu'})
=L^{\sim}(L^1_{\bar\nu};L^1_{\bar\nu'})$.
We have already seen that $S\mapsto\hat S$ is norm-preserving, and it is
clearly linear.   If $w\in(L^1_{\bar\nu})^+$, then, by
377D(d-ii), $w=T(\familyiI{u_i})$ for a family
$\familyiI{u_i}\in W_{ui}$;  since $T$ is a Riesz homomorphism,
$w=w^+=T(\familyiI{u_i^+}$;  so that
if $S\ge 0$ we shall have $\hat Sw=T'(\familyiI{Su_i^+})\ge 0$.
This shows
that $\hat S\ge 0$ whenever $S\ge 0$, so that $S\mapsto\hat S$ is a
positive linear operator.

\medskip

\quad{\bf (ii)} To show that $S\mapsto\hat S$ is a Riesz homomorphism, I
argue as follows.   Take any bounded linear operator
$S:L^1_{\bar\mu}\to L^1_{\bar\mu'}$ and $\epsilon>0$.   Then

\Centerline{$B=\{\sum_{k=0}^n|Sv_k|:
  v_0,\ldots,v_k\in(L^1_{\bar\mu})^+$,
$\sum_{k=0}^nv_k=\chi 1_{\frak A}\}$}

\noindent is an upwards-directed set in $L^1_{\bar\mu'}$ with
supremum $|S|(\chi 1_{\frak A})$ (371A, part (b) of the proof of 371B).
So we can find $v_0,\ldots,v_n\in(L^1_{\bar\mu})^+$ such that
$\sum_{k=0}^nv_k=\chi 1_{\frak A}$ and $\|v'\|_1\le\epsilon$, where
$v'=|S|(\chi 1_{\frak A})-\sum_{k=0}^n|Sv_k|\ge 0$.

Next, if $0\le u\le\chi 1_{\frak A}$ in $L^1_{\bar\mu}$, set
$u'=\chi 1_{\frak A}-u$;  we have

$$\eqalign{|S|(\chi 1_{\frak A})-v'
&=\sum_{k=0}^n|Sv_k|
\le\sum_{k=0}^n|S(u\times v_k)|+\sum_{k=0}^n|S(u'\times v_k)|\cr
&\le|S|(u)+|S|(u')
=|S|(\chi 1_{\frak A}).\cr}$$

\noindent So $|S|(u)-\sum_{k=0}^n|S(u\times v_k)|\le v'$ and
$\||S|(u)-\sum_{k=0}^n|S(u\times v_k)|\|_1\le\epsilon$.

Now take any $w\in L^1_{\bar\nu}$ such that $0\le w\le\chi 1_{\frak B}$.
Again because $T$ is a Riesz homomorphism and
$T(\familyiI{\chi 1_{\frak A}})=\chi 1_{\frak B}$, we can express $w$ as
$T(\familyiI{u_i})$ where
$0\le u_i\le\chi 1_{\frak A}$ for every $i$.   Consequently, setting
$v'_i=|S|u_i-\sum_{k=0}^n|S(u_i\times v_k)|$ for each $i$,
and $w'=T'(\familyiI{v'_i})$,

$$\eqalign{|S|\sphat\mskip4mu(w)
&=T'(\familyiI{|S|u_i})
=T'(\familyiI{\sum_{k=0}^n|S(u_i\times v_k)|+v'_i})\cr
&=\sum_{k=0}^n|T'(\familyiI{S(u_i\times v_k)})|+T'(\familyiI{v'_i})\cr
&=\sum_{k=0}^n|\hat ST(\familyiI{u_i\times v_k})|+w'
=\sum_{k=0}^n|\hat S\bigl(T(\familyiI{u_i})\times T(\familyiI{v_k})\bigr)|
   +w'\cr
&\le\sum_{k=0}^n
  |\hat S|\bigl(T(\familyiI{u_i})\times T(\familyiI{v_k})\bigr)
   +w'
=|\hat S|(w)+w'\cr}$$

\noindent because

\Centerline{$\sum_{k=0}^nT(\familyiI{v_k})=T(\familyiI{\chi 1_{\frak A}})
=\chi 1_{\frak B}$.}

\noindent But we also have
$\|w'\|_1=\lim_{i\to\Cal F}\|v'_i\|_1\le\epsilon$, while
$|\hat S|\le|S|\sphat$.   So we conclude that
$\||S|\sphat\mskip4mu(w)-|\hat S|(w)\|_1\le\epsilon$;  as $\epsilon$ is arbitrary,
$|S|\sphat\mskip4mu(w)=|\hat S|(w)$.

This is true whenever $0\le w\le\chi 1_{\frak B}$.   But as both
$|S|\sphat\mskip4mu$ and $|\hat S|$ are continuous linear operators, and
$L^{\infty}(\frak B)$ is dense in $L^1_{\bar\nu}$, $|S|\sphat=|\hat S|$.
As $S$ is arbitrary, we have a Riesz homomorphism (352G).
}%end of proof of 377F

\leader{377G}{Projective limits:  Proposition}
Let $(I,\le)$, $\familyiI{(\frak A_i,\bar\mu_i)}$ and
$\langle\pi_{ij}\rangle_{i\le j}$ be such that $(I,\le)$ is a non-empty
upwards-directed partially ordered
set, every $(\frak A_i,\bar\mu_i)$ is a probability
algebra, $\pi_{ij}:\frak A_j\to\frak A_i$ is a measure-preserving
Boolean homomorphism whenever $i\le j$ in $I$, and
$\pi_{ik}=\pi_{ij}\pi_{jk}$ whenever $i\le j\le k$.   Let
$(\frak C,\bar\lambda,\familyiI{\pi_i})$ be the corresponding projective
limit\cmmnt{ (328I)}.   Write $L^1_{\bar\mu_i}$ for
$L^1(\frak A_i,\bar\mu_i)$ and $L^1_{\bar\lambda}$ for
$L^1(\frak C,\bar\lambda)$.   For $i\le j$ in $I$, let
$T_{ij}:L^1_{\bar\mu_j}\to L^1_{\bar\mu_i}$ and
$P_{ij}:L^1_{\bar\mu_i}\to L^1_{\bar\mu_j}$ be the
norm-preserving Riesz homomorphism and the positive linear operator
corresponding to $\pi_{ij}:\frak A_j\to\frak A_i$\cmmnt{ (365O, 365P)},
and $T_i:L^1_{\bar\lambda}\to L^1_{\bar\mu_i}$,
$P_i:L^1_{\bar\mu_i}\to L^1_{\bar\lambda}$ the operators corresponding to
$\pi_i:\frak C\to\frak A_i$.   Let $X$ be any set.

(a) Suppose that for each $i\in I$ we are given a function
$S_i:L^1_{\bar\mu_i}\to X$ such that $S_iT_{ij}=S_j$ whenever $i\le j$
in $I$.   Then there is a unique function $S:L^1_{\bar\lambda}\to X$ such
that $S=S_iT_i$ for every $i\in I$.

(b) Suppose that for each $i\in I$ we are given a function
$S_i:X\to L^1_{\bar\mu_i}$ such that $T_{ij}S_j=S_i$ whenever $i\le j$
in $I$.   Then there is a unique function $S:X\to L^1_{\bar\lambda}$ such
that $T_iS=S_i$ for every $i\in I$.

(c) Suppose that $X$ is a topological space, and
for each $i\in I$ we are given a norm-continuous function
$S_i:L^1_{\bar\mu_i}\to X$ such that $S_jP_{ij}=S_i$ whenever $i\le j$
in $I$.   Then there is a unique function $S:L^1_{\bar\lambda}\to X$ such
that $SP_i=S_i$ for every $i\in I$.

(d) Suppose that for each $i\in I$ we are given a function
$S_i:X\to L^1_{\bar\mu_i}$ such that $P_{ij}S_i=S_j$ whenever $i\le j$ in
$I$.   Then there is a unique function $S:X\to L^1_{\bar\lambda}$ such that
$S=P_iS_i$ for every $i\in I$.

\ifwithproofs\medskip

\noindent{\bf proof: preliminary remarks (i)} It will be helpful to
recall some basic facts from \S\S328 and 365.
If $i\le j$ in $I$, then by the definition of `projective limit' we have
$\pi_{ij}\pi_j=\pi_i$ so
$T_{ij}T_j=T_i$ and $P_jP_{ij}=P_i$.
Also $P_{ij}T_{ij}$ is the identity operator on $L^1_{\bar\mu_j}$, and
$P_iT_i$ is the identity operator on $L^1_{\bar\lambda}$.

\medskip

\quad{\bf (ii)}
At a deeper level, we have useful concretizations of
$(\frak C,\bar\lambda)$, as follows.   Fix $i\in I$ for the moment.
For $j\ge i$, set $\frak B_j=\pi_{ij}[\frak A_j]$,
$\bar\nu_j=\bar\mu_i\restrp\frak B_j$;  then
$\frak B_j$ is a closed subalgebra of $\frak A_i$, isomorphic (as
probability algebra) to $\frak A_j$.   If $u\in L^1_{\bar\mu_i}$ and
$b\in\frak B_j$, set $b'=\pi_{ij}^{-1}b\in\frak A_j$;  then

\Centerline{$\int_bu=\int_{\pi_{ij}b'}u=\int_{b'}P_{ij}u
=\int_bT_{ij}P_{ij}u$;}

\noindent thus $T_{ij}P_{ij}$ is the conditional expectation
$P_{\frak B_j}:L^1_{\bar\mu_i}\to L^1(\frak B_j,\bar\nu_j)$,
identifying $L^1(\frak B_j,\bar\nu_j)$ with
$L^1_{\bar\mu_i}\cap L^0(\frak B_j)$ as in 365Ra.

If $k\ge j$, then $\pi_{ik}=\pi_{ij}\pi_{jk}$ so
$\frak B_k\subseteq\frak B_j$;  thus $\Bbb B=\{\frak B_j:j\ge i\}$ is
downwards-directed.   Set $\frak D=\bigcap\Bbb B$,
$\bar\nu=\bar\mu\restrp\frak D$.

For $k\ge i$, set $\phi_k=\pi_{ik}^{-1}\restrp\frak D:\frak D\to\frak A_k$;
then $\phi_k$ is a measure-preserving Boolean homomorphism, and
$\phi_j=\pi_{jk}\phi_k$ whenever $i\le j\le k$.   We can therefore define
$\phi_j:\frak D\to\frak A_j$, for any $j\in I$, by saying that
$\phi_j=\pi_{jk}\phi_k$ whenever $k\in I$ is greater than or equal to both
$i$ and $j$, and we shall have $\phi_j=\pi_{jk}\phi_k$ whenever
$j\le k$ in $I$.   \Prf\ If $j\in I$ and
$k_0$, $k_1$ are two upper bounds of $\{i,j\}$
in $I$, take an upper bound $k$ of $\{k_0,k_1\}$;  then

\Centerline{$\pi_{jk_0}\phi_{k_0}
=\pi_{jk_0}\pi_{k_0k}\phi_k
=\pi_{jk}\phi_k
=\pi_{jk_1}\pi_{k_1k}\phi_k
=\pi_{jk_1}\phi_{k_1}$,}

\noindent so $\phi_j$ is well-defined.   If $j$, $k\in I$ and
$j\le k$, let $k'$ be an upper bound of $\{i,k\}$;  then

\Centerline{$\pi_{jk}\phi_k
=\pi_{jk}\pi_{kk'}\phi_{k'}
=\pi_{jk'}\phi_{k'}
=\phi_j$.  \Qed}

\noindent Of course every $\phi_j$ is a measure-preserving Boolean
homomorphism.

By the definition of $(\frak C,\bar\lambda)$, there is a
measure-preserving Boolean homomorphism $\phi:\frak D\to\frak C$ such that
$\pi_j\phi=\phi_j$ for every $j\in I$.   In this case, $\pi_i\phi=\phi_i$
is the identity embedding of $\frak D$ in $\frak A_i$, and
$\pi_i[\frak C]=\frak D$.   Accordingly $P_{\frak D}=T_iP_i$.   By
the generalized reverse martingale theorem 367Qa,
$T_iP_i$ is the limit of $P_{\frak B}$ as $\frak B$ decreases in
$\Bbb B$, in the sense that for every $u\in L^1(\frak A_i)$ and
$\epsilon>0$ there is a $j\ge i$ in $I$ such that

\Centerline{$\|T_iP_iu-T_{ik}P_{ik}u\|_1
=\|P_{\frak D}u-P_{\frak B_k}u\|_1\le\epsilon$}

\noindent whenever $k\ge j$ in $I$.   If we write $\Cal F(I\closeuparrow)$ for
the filter on $I$ generated by $\{\{k:k\ge j\}:j\in I\}$, we have

\Centerline{$T_iP_iu=\lim_{j\to\Cal F(I\uparrow)}T_{ij}P_{ij}u$,}

\noindent for the norm in $L^1_{\bar\mu_i}$, for every
$u\in L^1_{\bar\mu_i}$.

Now let us turn to (a)-(d) as listed above.

\medskip

{\bf (a)} All we have to know is that

\Centerline{$S_iT_i=S_iT_{ij}T_j=S_jT_j$}

\noindent whenever $i\le j$ in $I$;  because $I$ is upwards-directed,
$S_iT_i=S_jT_j$ for all $i$, $j\in I$, and we have a sound definition for
$S$.

\medskip

{\bf (b)} The point is that $T_iP_iS_i=S_i$ for every
$i\in I$.   \Prf\ For $j\ge i$,

\Centerline{$T_{ij}P_{ij}S_i=T_{ij}P_{ij}T_{ij}S_j=T_{ij}S_j=S_i$.}

\noindent If $x\in X$,

\Centerline{$T_iP_iS_ix=\lim_{j\to\Cal F(I\uparrow)}T_{ij}P_{ij}S_ix
=S_ix$.  \Qed}

\noindent If now $i\le j$ in $I$,

\Centerline{$P_iS_i
=P_jP_{ij}T_{ij}S_j=P_jS_j$.}

\noindent As $I$ is upwards-directed, $P_iS_i=P_jS_j$ for all
$i$, $j\in I$;  write $S$ for this common value.   Then

\Centerline{$T_iS=T_iP_iS_i=S_i$}

\noindent for every $i\in I$.
As $T_i$ is injective for every $i\in I$,
the formula uniquely defines the function $S$.

\medskip

{\bf (c)} This time, we have $S_iT_iP_i=S_i$ for every $i\in I$.
\Prf\ For any $u\in L^1_{\bar\lambda}$,

$$\eqalignno{S_iT_iP_iu
&=\lim_{j\to\Cal F(I\uparrow)}S_iT_{ij}P_{ij}u\cr
\displaycause{because $S_i$ is continuous}
&=\lim_{j\to\Cal F(I\uparrow)}S_jP_{ij}T_{ij}P_{ij}u
=\lim_{j\to\Cal F(I\uparrow)}S_jP_{ij}u
=S_iu.  \text{ \Qed}\cr}$$

\noindent If $i\le j$ in $I$,

\Centerline{$S_iT_i=S_jP_{ij}T_{ij}T_j=S_jT_j$;}

\noindent consequently $S_iT_i=S_jT_j$ for all $i$, $j\in I$, and we can
call this common function $S$.   In this case, $SP_i=S_iT_iP_i=S_i$ for
every $i\in I$.   Since $P_i[L^1_{\bar\mu_i}]=L^1_{\bar\lambda}$,
this defines $S$ uniquely.

\medskip

{\bf (d)} As in (a), all we have to check is that if $i\le j$ in $I$ then

\Centerline{$P_jS_j=P_jP_{ij}S_i=P_iS_i$.}
\else\fi%end of proof of 377G

\leader{377H}{Inductive limits:  Proposition}
Let $(I,\le)$, $\familyiI{(\frak A_i,\bar\mu_i)}$ and
$\langle\pi_{ji}\rangle_{i\le j}$ be such that $(I,\le)$ is a non-empty
upwards-directed partially ordered
set, every $(\frak A_i,\bar\mu_i)$ is a probability
algebra, $\pi_{ji}:\frak A_i\to\frak A_j$ is a measure-preserving
Boolean homomorphism whenever $i\le j$ in $I$, and
$\pi_{ki}=\pi_{kj}\pi_{ji}$ whenever $i\le j\le k$.   Let
$(\frak C,\bar\lambda,\familyiI{\pi_i})$ be the corresponding inductive
limit\cmmnt{ (328H)}.   Write $L^1_{\bar\mu_i}$ for
$L^1(\frak A_i,\bar\mu_i)$ and $L^1_{\bar\lambda}$ for
$L^1(\frak C,\bar\lambda)$.   For $i\le j$ in $I$, let
$T_{ji}:L^1_{\bar\mu_i}\to L^1_{\bar\mu_j}$ and
$P_{ji}:L^1_{\bar\mu_j}\to L^1_{\bar\mu_i}$ be the Riesz homomorphism
and the positive linear operator corresponding to
$\pi_{ji}:\frak A_i\to\frak A_j$, and
$T_i:L^1_{\bar\mu_i}\to L^1_{\bar\lambda}$,
$P_i:L^1_{\bar\lambda}\to L^1_{\bar\mu_i}$ the operators corresponding to
$\pi_i:\frak A_i\to\frak C$.   Let $X$ be a set.

(a) Suppose that for each $i\in I$ we are given a function
$S_i:L^1_{\bar\mu_i}\to X$ such that
$S_jT_{ji}=S_i$ whenever $i\le j$
in $I$.   Then there is a
function $S:L^1_{\bar\lambda}\to X$ such
that $S_i=ST_i$ for every $i\in I$.

(b) Suppose that for each $i\in I$ we are given a function
$S_i:X\to L^1_{\bar\mu_i}$ such that $T_{ji}S_i=S_j$ whenever $i\le j$
in $I$.   Then there is a unique function $S:X\to L^1_{\bar\lambda}$ such
that $T_iS_i=S$ for every $i\in I$.

(c) Suppose that for each $i\in I$ we are given a function
$S_i:L^1_{\bar\mu_i}\to X$ such that
$S_iP_{ji}=S_j$ whenever $i\le j$ in $I$.   Then there is a unique
function $S:L^1_{\bar\lambda}\to X$ such
that $S=S_iP_i$ for every $i\in I$.

(d) Suppose that for each $i\in I$ we are given a function
$S_i:X\to L^1_{\bar\mu_i}$ such that $P_{ji}S_j=S_i$ whenever $i\le j$ in
$I$, and that

\Centerline{$\inf_{k\in\Bbb N}\sup_{i\in I}
  \int(|S_ix|-k\chi 1_{\frak A_i})^+=0$}

\noindent for every $x\in X$.
Then there is a unique function $S:X\to L^1_{\bar\lambda}$ such that
$S_i=P_iS$ for every $i\in I$.

\proof{ We can follow the same programme as in
the proof of 377G, but with a couple of new twists.

\medskip

{\bf preliminary remarks (i)}
If $i\le j$ in $I$, then by the definition of `inductive limit' we have
$\pi_j\pi_{ji}=\pi_i$ so
$T_jT_{ji}=T_i$ and $P_{ji}P_j=P_i$.
$P_{ji}T_{ji}$ and $P_iT_i$ are the identity
operator on $L^1_{\bar\mu_i}$.

\medskip

\quad{\bf (ii)} Let $\Cal F(I\closeuparrow)$ be the filter on $I$ generated by
$\{\{k:k\ge j\}:j\in I\}$.   Then
$\lim_{i\to\Cal F(I\uparrow)}T_iP_iu=u$ for every $u\in L^1_{\bar\lambda}$.
\Prf\ Setting $\frak B_i=T_i[\frak A_i]$ for each $i\in I$,
$\Bbb B=\{\frak B_i:i\in I\}$ is an upwards-directed family of closed
subalgebras of $\frak C$;  set $\frak D=\overline{\bigcup\Bbb B}$ and
$\bar\nu=\bar\lambda\restrp\frak D$, so that $(\frak D,\bar\nu)$ is a
probability algebra.   Since $\pi_i:\frak A_i\to\frak D$ is a
measure-preserving Boolean homomorphism and $\pi_i=\pi_j\pi_{ji}$ whenever
$i\le j$ in $I$, there is a measure-preserving Boolean homomorphism
$\phi:\frak C\to\frak D$ such that $\phi\pi_i=\pi_i$ for every $i$.
But this means that $\frak C=\frak D$.

As in 377G, we can identify
each $T_iP_i:L^1_{\bar\lambda}\to L^1_{\bar\lambda}$
with the conditional expectation $P_{\frak B_i}$.
This time, 367Qb tells us that $P_{\frak B}u\to P_{\frak D}u=u$ as
$\frak B$ increases through $\Bbb B$, that is,
$u=\lim_{i\to\Cal F(I\uparrow)}T_iP_iu$, for every
$u\in L^1_{\bar\lambda}$.\ \Qed

\medskip

{\bf (a)} The point is that if $i$, $j\in I$, $u\in L^1_{\bar\mu_i}$,
$v\in L^1_{\bar\mu_j}$ and $T_iu=T_jv$, then $S_iu=S_jv$.   \Prf\
Let $k\in I$ be such that $i\le k$ and $j\le k$.   Then

\Centerline{$T_kT_{ki}u=T_iu=T_jv=T_kT_{kj}v$;}

\noindent since $T_k$ is injective, $T_{ki}u=T_{kj}v$.   Accordingly

\Centerline{$S_iu=S_kT_{ki}u=S_kT_{kj}v=S_jv$.  \Qed}

\noindent There is therefore a function
$S':\bigcup_{i\in I}S_i[L^1_{\bar\mu_i}]\to X$ defined by saying that
$S(T_iu)=S_iu$ whenever $i\in I$ and $u\in L^1_{\bar\mu_i}$;  extending
$S'$ arbitrarily to a function $S:L^1_{\bar\lambda}\to X$, we get the
result.

\medskip

{\bf (b)} All we have to do is to check that if $i\le j$ in $I$, then

\Centerline{$T_iS_i=T_jT_{ji}S_i=T_jS_j$.}

\medskip

{\bf (c)} In this case, we have

\Centerline{$S_jP_j=S_iP_{ji}P_j=S_iP_i$}

\noindent whenever $i\le j$ in $I$.

\medskip

{\bf (d)(i)} For each $x\in X$,
$\{T_iS_ix:i\in I\}\subseteq L^1_{\bar\lambda}$ is uniformly integrable.
\Prf\ If $k\in\Bbb N$ and $i\in I$,

\Centerline{$|T_iS_ix|
\le T_i(|S_ix|-k\chi 1_{\frak A_i})^++T_i(k\chi 1_{\frak A_i})
\le T_i(|S_ix|-k\chi 1_{\frak A_i})^++k\chi 1_{\frak C}$,}

\noindent so

\Centerline{$\int(|T_iS_ix|-k\chi 1_{\frak C})^+
\le\int T_i(|S_ix|-k\chi 1_{\frak A_i})^+
=\int(|S_ix|-k\chi 1_{\frak A_i})^+$.}

\noindent Accordingly

\Centerline{$\inf_{k\in\Bbb N}\sup_{i\in I}
  \int(|T_iS_ix|-k\chi 1_{\frak C})^+
\le\inf_{k\in\Bbb N}\sup_{i\in I}\int(|S_ix|-k\chi 1_{\frak A_i})^+
=0$.  \Qed}

\medskip

\quad{\bf (ii)} Fix an ultrafilter $\Cal G$ on $I$ including
$\Cal F(I\closeuparrow)$.   For each $x\in X$, $\{T_iS_ix:i\in I\}$ is
relatively weakly compact in $L^1_{\bar\lambda}$, so
$Sx=\lim_{i\to\Cal G}T_iS_ix$ is defined for the weak topology on
$L^1_{\bar\lambda}$.   Now for any $i\in I$,

$$\eqalignno{P_iSx
&=\lim_{j\to\Cal G}P_iT_jS_jx\cr
\displaycause{for the weak topology on $L^1_{\bar\mu_i}$}
&=\lim_{j\to\Cal G}P_{ji}P_jT_jS_jx\cr
\displaycause{because $\{j:j\ge i\}\in\Cal G$}
&=\lim_{j\to\Cal G}P_{ji}S_jx
=S_ix.\cr}$$

\medskip

\quad{\bf (iii)} To see that $S$ is uniquely defined, it is enough to
recall that

\Centerline{$Sx
=\lim_{i\to\Cal F(I\uparrow)}T_iP_iSx
=\lim_{i\to\Cal F(I\uparrow)}T_iS_ix$}

\noindent is uniquely defined by the family $\familyiI{S_ix}$, for every
$x\in X$.
}%end of proof of 377H

\exercises{\leader{377X}{Basic exercises (a)}
%\spheader 377Xa
In 377B, show that $\familyiI{u_i}\in\prod_{i\in I}L^0(\frak A_i)$ belongs
to $W_0$ iff $\{u_i^*:i\in I\}$ is bounded above in $L^0(\frak A_L)$, where
$\frak A_L$ is the measure algebra of Lebesgue measure on
$\coint{0,\infty}$, and
$u_i^*$ is the decreasing rearrangement of $u_i$ for each $i$ (373C).
%377B

\spheader 377Xb In 377D, suppose that $u=\familyiI{u_i}$ and
$v=\familyiI{v_i}$ belong to $W_2$, and that at least one of
$|u|^2$, $|v|^2$ belongs to $W_{ui}$.
Show that $\innerprod{Tu}{Tv}=\lim_{i\to\Cal F}\innerprod{u_i}{v_i}$.
%377D

\spheader 377Xc Let $\familyiI{(\frak A_i,\bar\mu_i)}$ be a family of
probability algebras, and suppose that we have
$u_i\in L^1(\frak A_i,\bar\mu_i)$ for each $i$.   Show that the following
are equiveridical:  (i)
$\inf_{k\in\Bbb N}\sup_{i\in I}\int(|u_i|-k\chi 1_{\frak A_i})^+=0$;
(ii) $\{u_i^*:i\in I\}$ is uniformly integrable in $L^1(\mu_L)$, where
$\mu_L$ is Lebesgue measure on $\coint{0,\infty}$, and $u_i^*$ is the
decreasing rearrangement of $u_i$ for each $i\in I$.
%377D

\spheader 377Xd Take any $p\in\ooint{1,\infty}$.   Show that 377G remains
true if we replace every `$L^1$' by `$L^p$'.
%377G

\spheader 377Xe Take any $p\in\ooint{1,\infty}$.   Show that 377H remains
true if we replace every `$L^1$' by `$L^p$' and in
part (d) we replace
`$\inf_{k\in\Bbb N}\sup_{i\in I}\int(|S_ix|-k\chi 1_{\frak A_i})^+=0$' by
`$\sup_{i\in I}\|S_ix\|_p<\infty$'.
%377G

\spheader 377Xf In 377Ha, suppose that $X$ has a metric $\rho$ under which
it is complete, and that $\familyiI{S_i}$ is uniformly equicontinuous in
the sense that for every $\epsilon>0$ there is a $\delta>0$ such that
$\rho(S_iu,S_iv)\le\epsilon$ whenever $i\in I$, $u$, $v\in L^1_{\bar\mu_i}$
and $\|u-v\|_1\le\delta$.   Show that there is a unique continuous function
$S:L^1_{\bar\lambda}\to X$ such that $S_i=ST_i$ for every $i\in I$.
%377H

\leader{377Y}{Further exercises (a)}
%\spheader 377Ya
Find a non-empty family $\familyiI{(\frak A_i,\bar\mu_i)}$
of probability algebras, a probability algebra $(\frak B,\bar\nu)$, a
Boolean homomorphism $\pi:\prod_{i\in I}\frak A_i\to\frak B$ such
that $\bar\nu\pi(\familyiI{a_i})\le\sup_{i\in I}\bar\mu_ia_i$ whenever
$\familyiI{a_i}\in\prod_{i\in I}\frak A_i$, and an element
$u=\familyiI{u_i}$ of $W_0^+$, as described in 377B, such that
$\|Tu\|_1>\sup_{i\in I}\|u_i\|_1$, where $T:W_0\to L^0(\frak B)$ is
the Riesz homomorphism of 377B-377C.   \Hint{$\#(I)=2$.}
%377C

\spheader 377Yb Show that if, in 377Gc, we omit the hypothesis that the
$S_i$ are to be continuous, then the result can fail.
%377G mt37bits

\spheader 377Yc Let $\familyiI{U_i}$ be a non-empty family of $L$-spaces
and $\Cal F$ an ultrafilter on $I$.   (i) Show that $\prod_{i\in I}U_i$ is a
Dedekind complete Riesz space (see 352K) in which
$W_{\infty}=\{\familyiI{u_i}:\sup_{i\in I}\|u_i\|<\infty\}$ is a solid
linear subspace.   (ii) Let $W_0\subseteq W_{\infty}$ be
$\{\{\familyiI{u_i}:\sup_{i\in I}\|u_i\|<\infty$,
$\lim_{i\to\Cal F}\|u_i\|=0\}$;  show that $W_0$ is a solid linear subspace
of $W_{\infty}$.   (iii) Let $U$ be the quotient Riesz space
$W_{\infty}/W_0$ (352U).   Show that $U$ is an $L$-space under the norm
$\|\familyiI{u_i}^{\ssbullet}\|=\lim_{i\to\Cal F}\|u_i\|$ for
$\familyiI{u_i}\in W_{\infty}$.
%377D out of order query

\spheader 377Yd Let $V$ be a normed space, and suppose that for every
finite-dimensional subspace $V_0$ of $V$ there are an $L$-space $U$ and a
norm-preserving linear map $T:V_0\to U$.   Show that there are an $L$-space
$U$ and a norm-preserving linear map $T:V\to U$.
%377Yc
}%end of exercises


\endnotes{
\Notesheader{377} Although my main target in this section has been to
understand the function spaces of reduced products of probability algebras,
I have as usual felt that the ideas are clearer if each is developed in
a context closer to the most general case in which it is applicable.
Only in part (b) of the proof of 377C, I think, does this involve us in
extra work.

The new techniques of this section are forced on us by the fact that we are
looking at Boolean homomorphisms $\pi:\prod_{i\in I}\frak A_i\to\frak B$
which are not normally sequentially order-continuous.    While we have
a natural Riesz homomorphism from $L^{\infty}(\prod_{i\in I}\frak A_i)$ to
$L^{\infty}(\frak B)$, as in 363F, we cannot expect
a similar operator from the whole of
$L^0(\prod_{i\in I}\frak A_i)\cong\prod_{i\in I}L^0(\frak A_i)$ to
$L^0(\frak B)$.   However the condition
`$\bar\nu\pi(\familyiI{a_i})\le\sup_{i\in I}\bar\mu_ia_i$' ensures that
there is a space $W_0\subseteq\prod_{i\in I}L^0(\frak A_i)$ on which an
operator to $L^0(\frak B)$ can be defined, and which is large enough to
give us a method of investigating the spaces $L^p(\frak B,\bar\nu)$ as
images of subspaces $W_p$ of products
$\prod_{i\in I}L^p(\frak A_i,\bar\mu_i)$.

In 377E, the case $p=1$ is
special because we can identify $W_{ui}$ as the space of relatively weakly
compact families in $L^1(\frak A,\bar\mu)$, and for such a family
$u=\familyiI{u_i}$ we have $\|Tu\|_1=\lim_{i\to\Cal F}\|u_i\|_1$.
So the Banach space $L^1(\frak B,\bar\nu)$ is a kind of reduced power,
describable in terms of the normed space $L^1(\frak A,\bar\nu)$.   For
other $L^p$ spaces we need to know something more, e.g., the lattice
structure, if we are to identify those $u\in W_p$ such that $Tu=0$.
The difference becomes significant when we come to look at morphisms of
$L^p(\frak B,\bar\nu)$ corresponding to morphisms of
$L^p(\frak A,\bar\mu)$, as in 377F.

In 377G-377H I give a string of results which are visibly mass-produced.
What is striking is that in eight cases out of eight we have a
straightforward formula corresponding to the idea that
$(\frak C,\bar\lambda)$ is a limit of $\familyiI{(\frak A_i,\bar\mu_i)}$.
What is curious is that in two of the eight cases (377Gc, 377Hd)
we have to impose different special conditions on the functions $S_i$ which
the target $S$ is supposed to approximate, and in just one case (377Ha) the
target $S$ is not uniquely defined in the absence of further constraints
(377Xf).   I think the ideas take up
enough room when given only in their application to $L^1$ spaces, but of
course there are versions, only slightly modified, which apply to other
$L^p$ spaces (377Xd-377Xe).

The repeated conditions of the form

\Centerline{$\inf_{k\in\Bbb N}\sup_{i\in I}\bar\mu_i\Bvalue{|u_i|>k}=0$,}

\Centerline{$\inf_{k\in\Bbb N}\sup_{i\in I}
  \int(|u_i|-k\chi 1_{\frak A_i})^+=0$,}

\noindent (377B, 377Dc, 377Hd) both have expressions in terms of
decreasing rearrangements (377Xa, 377Xc).   The latter is clearly
associated with uniform integrability and weak compactness, and
unsurprisingly we use it to show that a weak limit will be defined.
The former is there to ensure that a set appearing in an $L^0$ space will
be bounded above, so that we can apply 355F to extend a Riesz homomorphism.
}%end of notes

\discrpage

