\frfilename{mt5a3.tex}
\versiondate{3.7.13}
\copyrightdate{2007}

\Loadfourteens

\def\chaptername{Appendix}
\def\sectionname{Forcing}

%\def\uBp{\Cal{UB}\eurm{p}}
\def\uBp{\Cal{U}\mskip-1mu\widehat{\Cal B}}

\newsection{5A3}

My discussion of forcing is based on {\smc Kunen 80};  in particular,
I start from pre-ordered sets rather than Boolean algebras, and the class
$V^{\Bbb P}$ of terms in a forcing language will consist of subsets of
$V^{\Bbb P}\times P$.   I find however that I wish to diverge almost
immediately from standard formulations in a technical respect, which I
describe in 5A3A, introducing what I call `forcing notions'.
I do not refer to generic filters or models of ZFC, preferring to express
all results in terms of the forcing relation (5A3C).   I give some space to
the interpretation of names (5A3E, 5A3H) and, in particular, to names for
real numbers derived from elements of $L^0(\RO(\Bbb P))$ (5A3L).

\leader{5A3A}{Forcing notions (a)}
A {\bf forcing notion} is a quadruple
$\Bbb P=(P,\le,\Bbbone,\uparrow)$ or $\Bbb P=(P,\le,\Bbbone,\downarrow)$
where $(P,\le)$ is a pre-ordered set (that is, $\le$ is a transitive
reflexive relation on $P$), $\Bbbone\in P$, and

\inset{if $\Bbb P=(P,\le,\Bbbone,\uparrow)$ then $\Bbbone\le p$ for every
$p\in P$,

if $\Bbb P=(P,\le,\Bbbone,\downarrow)$ then $p\le\Bbbone$ for every
$p\in P$.}

\noindent In this context members of $P$ are\cmmnt{ commonly} called
{\bf conditions}.

\spheader 5A3Ab\cmmnt{ I had better try to explain what I am doing here.
The problem is the following.
Consider two of the standard examples of pre-ordered set in this
context.   For a set $I$,
$\Fn_{<\omega}(I;\{0,1\})$ is the set of functions from finite subsets of
$I$ to $\{0,1\}$;  for a non-trivial Boolean algebra $\frak A$,
$\frak A^+$ is the set of non-zero elements of $\frak A$.
In each case, we have a relevant direction.   In
$\Fn_{<\omega}(I;\{0,1\})$, a condition $p$ is stronger than a condition
$q$ if $p$ extends $q$, that is, if $p\supseteq q$;
in $\frak A^+$, $p$ is stronger than $q$ if $p\Bsubseteq q$.   So the
forcing notions, in the terminology I have chosen, are

\Centerline{$(\Fn_{<\omega}(I;\{0,1\}),\subseteq,\emptyset,\uparrow)$,
\quad$(\frak A^+,\Bsubseteqshort,1_{\frak A},\downarrow)$.}

\noindent Generally,} I will say that a forcing notion
$(P,\le,\Bbbone,\uparrow)$ is {\bf active upwards}, while
$(P,\le,\Bbbone,\downarrow)$ is {\bf active downwards}.

\cmmnt{\spheader 5A3Ac Of course this is unconventional.
It is much more usual
to take all forcing notions to be active in the same direction (usually
downwards) and to use local definitions (e.g., saying that `$p\le q$ if $p$
extends $q$') to ensure that this will be appropriate.

However the great majority of forcing notions, like the two examples in (b)
above, come with structures which strongly suggest a natural interpretation
of `$\le$';  and these structures are not arbitrary, but are essential to
our intuitive conception of the pre- or partial order we are studying.
I prefer, therefore, to maintain the notation I would use for the
same objects
in any other context, and to indicate separately the orientation which is
relevant when using them to build a forcing language.
}

\spheader 5A3Ad\cmmnt{ This approach demands further changes in the
language.}   It
will no longer be helpful to talk about conditions in $P$ being `larger' or
`less than' others.   Instead, I will use the word `{\bf stronger}';  if
$\Bbb P=(P,\le,\Bbbone,\uparrow)$, then $p\in P$ will be stronger than
$q\in P$ if $p\ge q$;  if
$\Bbb P=(P,\le,\Bbbone,\downarrow)$, then $p\in P$ will be stronger than
$q\in P$ if $p\le q$.
\cmmnt{(So $p$ will be stronger than $\Bbbone$ for every $p\in P$.)}

Similarly,\cmmnt{ the words `cofinal' and `coinitial' are now
inappropriate, and} I
will turn to the word `dense'\cmmnt{, as favoured by most authors
discussing forcing};  if $\Bbb P=(P,\le,\Bbbone,\updownarrows)$
is a forcing notion, a subset $Q$ of
$P$ is {\bf dense} if for every $p\in P$ there is a $q\in Q$ such that $q$
is stronger than $p$.   In the same way,\cmmnt{ I can say that} two
conditions $p$,
$q$ in $P$ are `compatible' if there is an $r\in P$ stronger than both.
We\cmmnt{ shall} have a standard topology on $P$ generated by sets of
the form
$\{q:q$ is stronger than $p\}$, and a corresponding regular open algebra
$\RO(\Bbb P)$, as in 514M.   An antichain for $\Bbb P$
will be a set $A\subseteq P$ such that any
two distinct conditions in $A$ are incompatible, and
$\Bbb P$ will be ccc if every antichain for $\Bbb P$ is countable.
The `saturation' $\sat\Bbb P$ of $\Bbb P$ will be the least cardinal
$\kappa$ such that there is no antichain of size $\kappa$.

\leader{5A3B}{Forcing languages} Let
$\Bbb P=(P,\le,\Bbbone,\updownarrows)$
be a forcing notion.

\spheader 5A3Ba The class of {\bf $\Bbb P$-names}, that is,
terms of the forcing language defined by $\Bbb P$, is

\Centerline{$V^{\Bbb P}=\{A:A$ is a set and
$A\subseteq V^{\Bbb P}\times P\}$}

\noindent\cmmnt{({\smc Kunen 80}, VII.2.5)\footnote{For once, I rely on
the Axiom of
Foundation;  to determine whether a set $A$ belongs to $V^{\Bbb P}$, we
need to induce on the rank of $A$.}.   In this context,}
I will say that the {\bf domain} of a name $A\in V^{\Bbb P}$ is the set
$\dom A\subseteq V^{\Bbb P}$ of first members of elements of $A$.

\spheader 5A3Bb For any set $X$, $\check X$ will be the $\Bbb P$-name
$\{(\check x,\Bbbone):x\in X\}\in V^{\Bbb P}$\cmmnt{ ({\smc Kunen 80},
VII.2.10)}.

\leader{5A3C}{The Forcing Relation}\cmmnt{ ({\smc Kunen 80}, VII.3.3)}
Suppose that $\Bbb P=(P,\le,\Bbbone,\updownarrows)$
is a forcing notion, $p\in P$, $\phi$, $\psi$ are formulae of set theory,
and $\dot x_0,\ldots,\dot x_n\in V^{\Bbb P}$.

\spheader 5A3Ca $p\VVdP\,\dot x_0=\dot x_1$ iff

\inset{whenever $(\dot y,q)\in\dot x_0$ and $r\in P$ is
stronger than both $p$ and $q$, there are a $(\dot y',q')\in\dot x_1$ and
an $r'$ stronger than both $r$ and $q'$ such that
$r'\VVdP\,\dot y=\dot y'$,

whenever $(\dot y,q)\in\dot x_1$ and $r\in P$ is
stronger than both $p$ and $q$, there are a $(\dot y',q')\in\dot x_0$ and
an $r'$ stronger than both $r$ and $q'$ such that
$r'\VVdP\,\dot y=\dot y'$.}

\cmmnt{\noindent Note that $p\VVdP\,\dot x=\dot x$ for every $\Bbb P$-name 
$\dot x$ and every $p\in P$ (induce on the rank of $\dot x$).}

\spheader 5A3Cb $p\VVdP\,\dot x_0\in\dot x_1$ iff

\inset{whenever $q\in P$ is stronger than $p$ there are a
$(\dot y,q')\in\dot x_1$ and an $r$ stronger than both $q$ and $q'$ such
that $r\VVdP\,\dot x_0=\dot y$.}

\spheader 5A3Cc
$p\VVdP\,\phi(\dot x_0,\ldots,\dot x_n)\,
\&\,\psi(\dot x_0,\ldots,\dot x_n)$ iff

\Centerline{$p\VVdP\,\phi(\dot x_0,\ldots,\dot x_n)$ and
  $p\VVdP\,\psi(\dot x_0,\ldots,\dot x_n)$.}

\spheader 5A3Cd $p\VVdP\,\neg\phi(\dot x_0,\ldots,\dot x_n)$ iff

\Centerline{there is no $q$ stronger than $p$ such that
$q\Vdash_{\Bbb P}\,\phi(\dot x_0,\ldots,\dot x_n)$.}

\spheader 5A3Ce $p\VVdP\,\Exists x,\,\phi(x,\dot x_0,\ldots,\dot x_n)$
iff 

\inset{for
every $q$ stronger than $p$ there are an $r$ stronger than $q$ and a
$\dot y\in V^{\Bbb P}$ such that
$r\VVdP\,\phi(\dot y,\dot x_0,\ldots,
\dot x_n)$.\cmmnt{\footnote{\smallerfonts This formulation
is appropriate if we wish to explore forcing without using the axiom of
choice.   Subject to AC, we have an alternative condition:
$p\eightVVdash_{\Bbb P}\,\Exists x,\,\phi(x,\dot x_0,\ldots,\dot x_n)$
iff there is a $\dot y\in V^{\Bbb P}$ such that
$p\eightVVdash_{\Bbb P}\,\phi(\dot y,\dot x_0,\ldots,\dot x_n)$.}}}


\spheader 5A3Cf\cmmnt{ In this context} I will write $\VVdP$ for
$\Bbbone\VVdP$.

\leader{5A3D}{The Forcing Theorem} If $\phi$ is any theorem of ZFC, and
$\Bbb P$ is any forcing notion, then $\VVdP\,\phi$.
\prooflet{({\smc Kunen 80}, VII.4.2.)}

\leader{5A3E}{More notation}\cmmnt{ In 5A3C I took it for granted that
every
formula of set theory would have a version in $V^{\Bbb P}$.   I should
perhaps explain some of the versions I have in mind.}
Let $\Bbb P=(P,\le,\Bbbone,\updownarrows)$ be a forcing notion.

\spheader 5A3Ea If $\dot y_0$, $\dot y_1\in V^{\Bbb P}$ then
$\dot x=\{(\dot y_0,\Bbbone),(\dot y_1,\Bbbone)\}\in V^{\Bbb P}$, and

\Centerline{$\VVdP\,\dot x=\{\dot y_0,\dot y_1\}$\dvro{.}{;}}

\noindent\cmmnt{so we have a suitable formal expression for pair sets in
$V^{\Bbb P}$. }Similarly, if we think of the formula $(x,y)$ as being an
abbreviation for $\{\{x\},\{x,y\}\}$, we get a $\Bbb P$-name

\Centerline{$\dot z
=\{(\{(\dot y_0,\Bbbone)\},\Bbbone),
  (\{(\dot y_0,\Bbbone),(\dot y_1,\Bbbone)\},\Bbbone)\}$}

\noindent such that

\Centerline{$\VVdP\,\dot z=(\dot y_0,\dot y_1)$.}

\spheader 5A3Eb Now let
$\familyiI{\dot x_i}$ be a family of $\Bbb P$-names.   Then

\Centerline{$\dot f=\{((\varchecki,\dot x_i),\Bbbone):i\in I\}$}

\noindent is a $\Bbb P$-name, provided that in the formula
$((\varchecki,\dot x_i),\Bbbone)$ we interpret the inner pair $(\,,\,)$ of
brackets as an ordered pair in $V^{\Bbb P}$, that is, as

\Centerline{$\{(\{(\varchecki,\Bbbone)\},\Bbbone),
  (\{(\varchecki,\Bbbone),(\dot x_i,\Bbbone)\},\Bbbone)\}$}

\noindent and the outer pair of brackets $(\,,\Bbbone)$
as an ordered pair in the ordinary universe.   In this case,

\Centerline{$\VVdP\,\dot f$ is a function with domain $\check I$,}

\noindent and

\Centerline{$\VVdP\,\dot f(\varchecki)=\dot x_i$}

\noindent for every $i\in I$.\cmmnt{ For obvious reasons I do not wish
to spell
this procedure out every time, and} I will use the\cmmnt{ rather
elliptic} formula

\Centerline{$\family{i}{\check I}{\dot x_i}$}

\noindent to signify the $\Bbb P$-name $\dot f$.

\spheader 5A3Ec Similarly,
$\dot T=\{(\dot x_i,\Bbbone):i\in I\}$ is a $\Bbb P$-name such that
$\VVdP\,\dot x_i\in\dot T$
for every $i\in I$, and whenever $p\in\Bbb P$ and $\dot x$ is a
$\Bbb P$-name such that $p\VVdP\,\dot x\in\dot T$, there are an $i\in I$
and a $q$ stronger than $p$ such that $q\VVdP\,\dot x=\dot x_i$;  I will
write $\{\dot x_i:i\in\check I\}$ for $\dot T$.

\spheader 5A3Ed
In the same spirit, if I have a family
$\familyiI{\dot x_i}$ of $\Bbb P$-names for real numbers between $0$ and $1$,
I will allow myself to write
`$\sup_{i\in\check I}\dot x_i$' to signify a $\Bbb P$-name such that

\Centerline{$\VVdP\,\sup_{i\in\check I}\dot x_i
=\sup\{\dot x_i:i\in\check I\}$\dvro{.}{,}}

\noindent\cmmnt{without taking the trouble to spell out any exact
formula to
represent the supremum.  }I will do the same for limits of sequences;
if $\sequencen{\dot x_n}$ is a sequence of $\Bbb P$-names for real numbers,
and

\Centerline{$\VVdP\,\sequencen{\dot x_n}$ is convergent,}

\noindent then I will write `$\lim_{n\to\infty}\dot x_n$' to mean a
$\Bbb P$-name $\dot x$ such that

\Centerline{$\VVdP\,\sequencen{\dot x_n}\to\dot x\in\Bbb R$.}

\cmmnt{\noindent Of course this is tolerable only because it is
possible to set
out a general rule for constructing a suitable name $\dot x\in V^{\Bbb P}$
from the given sequence $\sequencen{\dot x_n}$.}

\vleader{72pt}{5A3F}{Boolean truth values} Let
$\Bbb P$ be a forcing notion and $P$ its set of conditions.

\spheader 5A3Fa If $\phi$ is a
formula of set theory, and $\dot x_0,\ldots,\dot x_n\in V^{\Bbb P}$, then

\Centerline{$\{p:p\in P$, $p\VVdP\,\phi(\dot x_0,\ldots,\dot x_n)\}$}

\noindent is a regular open set in $P$\cmmnt{ (use 514Md)};
I will denote it $\Bvalue{\phi(\dot x_0,\ldots,\dot x_n)}$.

\spheader 5A3Fb If $\phi$ and $\psi$ are formulae of set theory
and $\dot x_0,\ldots,\dot x_n\in V^{\Bbb P}$, then

\Centerline{
$\Bvalue{\phi(\dot x_0,\ldots,\dot x_n)\,\&\,\psi(\dot x_0,\ldots,\dot x_n)}
=\Bvalue{\phi(\dot x_0,\ldots,\dot x_n)}
\cap\Bvalue{\psi(\dot x_0,\ldots,\dot x_n)}$}

\noindent and

\Centerline{$\Bvalue{\neg\phi(\dot x_0,\ldots,\dot x_n)}
=P\setminus\overline{\Bvalue{\phi(\dot x_0,\ldots,\dot x_n)}}$\dvro{.}{,}}

\cmmnt{\noindent the complement of
$\Bvalue{\phi(\dot x_0,\ldots,\dot x_n)}$ in $\RO(\Bbb P)$.}

\spheader 5A3Fc If $\phi$ is a formula of set theory, $A$ is a set,
and $\dot x_0,\ldots,\dot x_n$ are $\Bbb P$-names, then

\Centerline{$\Bvalue{\exists\,x\in\check A,\,\phi(x,\dot x_0,\ldots,\dot x_n)}
=\interior\overline{\bigcupop_{a\in A}
   \Bvalue{\phi(\check a,\dot x_0,\ldots,\dot x_n)}}$,}

\noindent the supremum of
$\{{\Bvalue{\phi(\check a,\dot x_0,\ldots,\dot x_n}}:a\in A\}$ in
$\RO(\Bbb P)$.  \prooflet{(Use 5A3Ce.)}

\leader{5A3G}{Concerning $\pmb{\var2spcheck}$s (a)}
\cmmnt{The reader is entitled to an attempt at
consistency on the following point of notation, among others.}
For any set $X$ and any forcing notion $\Bbb P$ there is a
corresponding $\Bbb P$-name $\check X$\cmmnt{ (5A3Bb)}.   We start with
$\check\emptyset=\emptyset$.   If $1=\{\emptyset\}$ is the next von Neumann
ordinal, we get a name

\Centerline{$\check 1
=\{(\check\emptyset,\Bbbone)\}=\{(\emptyset,\Bbbone)\}$;}

\noindent and\cmmnt{ we can check directly from 5A3C that}

\Centerline{$\VVdP\,\check 1=\{\emptyset\}$,}

\noindent that is\cmmnt{, if you like},

\Centerline{$\VVdP\,\check 1=1$,}

\noindent where in this formula the first $1$ is interpreted in the
ordinary universe and the second is interpreted in the forcing language.
Similarly, if we take `$2$' to be an abbreviation for
`$\{\emptyset,\{\emptyset\}\}$',\cmmnt{ we have}

\Centerline{$\VVdP\,\check 2=2$,}

\noindent and so on.   Indeed\cmmnt{ we get}

\Centerline{$\VVdP\,\check\omega$ is the first infinite ordinal,}

\Centerline{$\VVdP\,\check\Bbb Q$ is the set of rational numbers,}

\noindent so the same convention would lead to

\Centerline{$\VVdP\,\check\omega=\omega$, $\check\Bbb Q=\Bbb Q$.}

\noindent\cmmnt{(This formula does not depend on which construction of
the set of
rational numbers we use, provided that we use the same method both in the
ordinary universe and in the forcing language.)  Of course it is
{\it not} the case (except for forcing notions of particular types) that

\Centerline{$\VVdP\,\check\omega_1$ is the first uncountable ordinal,
\quad$\VVdP\,\check\Bbb R$ is the set of real numbers.}}

\cmmnt{\spheader 5A3Gb For `absolute' objects, therefore,
like $\omega+7$,
appearing in sentences of a forcing language, I shall have a
choice between formulations

\Centerline{$(\omega+7)\var2spcheck$}

\noindent (working directly from 5A3Bb),

\Centerline{$\omega+7$}

\noindent (regarding the phrase `$\omega+7$' as an abbreviation for an
expression in set theory which can be evaluated either in the ordinary
universe or in the forcing language), or

\Centerline{$\check\omega\mskip5mu\check+\mskip5mu\check 7$}

\noindent (regarding $\omega$, $+$ and $7$ as sets to which the rule of
5A3Bb can be applied, and then interpreting the combination in the forcing
language).   The least cluttered version, $\omega+7$, looks better, and
this will ordinarily be my choice.   But it means that when you see the
symbol $\Bbb Q$ in a sentence of the forcing language, it is likely to mean
two things at once, a superposition of `the set of rational numbers' and
`the $\Bbb P$-name $\check{\Bbb Q}$'.
}

\leader{5A3H}{Names for functions} Let $\Bbb P$ be a forcing notion,
$P$ its set of conditions, and
$R\subseteq V^{\Bbb P}\times V^{\Bbb P}\times P$ a set.
Consider the $\Bbb P$-names

\Centerline{$\dot f=\{((\dot x,\dot y),p):(\dot x,\dot y,p)\in R\}$,}

\Centerline{$\dot A=\{(\dot x,p):(\dot x,\dot y,p)\in R\}$,
\quad$\dot B=\{(\dot y,p):(\dot x,\dot y,p)\in R\}$.}

(a) The following are equiveridical:

\quad(i) $\VVdP\,\dot f$ is a function;

\quad(ii) whenever $(\dot x_0,\dot y_0,p_0)$,
$(\dot x_1,\dot y_1,p_1)$ belong to $R$,
$p\in\Bbb P$ is stronger than both $p_0$ and $p_1$
and $p\VVdP\,\dot x_0=\dot x_1$, then $p\VVdP\,\dot y_0=\dot y_1$.

(b)\dvArevised{2013} In this case,

\Centerline{$\VVdP\,\dom\dot f=\dot A$ and $\dot f[\dot A]=\dot B$,}

\noindent and the following are equiveridical:

\quad(i) $\VVdP\,\dot f$ is injective;

\quad(ii) whenever $(\dot x_0,\dot y_0,p_0)$,
$(\dot x_1,\dot y_1,p_1)$ belong to $R$,
$p\in\Bbb P$ is stronger than both $p_0$ and $p_1$
and $p\VVdP\,\dot y_0=\dot y_1$, then $p\VVdP\,\dot x_0=\dot x_1$.


\cmmnt{\medskip

\noindent{\bf Remark} In the formula for $\dot f$ here\cmmnt{, as in
that of 5A3Eb}, the brackets take
different meanings at different points.   In the expression
$((\dot x,\dot y),p)$, the
inner brackets must be interpreted in the forcing language, while the outer
brackets, like the brackets in the expression $(\dot x,\dot y,p)\in R$, are
interpreted in the ordinary universe\dvro{.}{;}\cmmnt{ {\smc Kunen 80}
might write

\Centerline{$\dot f=\{($op$(\dot x,\dot y),p):(\dot x,\dot y,p)\in R\}$.}}
}

\proof{ Elementary.
}%end of proof of 5A3H

\leader{5A3I}{Regular open algebras} If
$\Bbb P=(P,\le,\Bbbone,\updownarrows)$ is a forcing notion with regular
open algebra $\RO(\Bbb P)$, then we have a natural map
$\iota:P\to\RO(\Bbb P)^+$ defined by saying that

\Centerline{$\iota(p)
=\interior\overline{\{q:q\text{ is stronger than }p\}}$}

\noindent for $p\in P$\cmmnt{ ({\smc Kunen 80}, II.3.3)};
and (allowing for the possible reversal of the direction of $\Bbb P$)
$\iota$ is a dense
embedding of the pre-ordered set $(P,\le)$ in the partially ordered set
$(\RO(\Bbb P)^+,\subseteq)$\cmmnt{,
in the sense of {\smc Kunen 80}, \S VII.7}.
Consequently, taking $\widehat{\Bbb P}$ to be the forcing notion
$(\RO(\Bbb P)^+,\subseteq,P,\downarrow)$,\cmmnt{ we shall have}

\Centerline{$\VVdP\,\phi$ if and only if
$\VVdash_{\widehat{\Bbb P}}\,\phi$}

\noindent for every sentence $\phi$ of
set theory\cmmnt{ ({\smc Kunen 80},
VII.7.11)}.   It follows that if two forcing notions
have isomorphic regular open algebras, then they force exactly the same
theorems of set theory.

\leader{5A3J}{}\cmmnt{ The following technical device will be useful at
one point.

\medskip

\noindent}{\bf Definition} Let $\Bbb P$ be a forcing notion.
I will say that a
$\Bbb P$-name $\dot X$ is {\bf discriminating} if whenever $(\dot x,p)$ and
$(\dot y,q)$ are distinct members of $\dot X$, and $r$ is stronger than
both $p$ and $q$, then $r\VVdP\,\dot x\ne\dot y$.

\leader{5A3K}{Lemma} Let $\Bbb P$ be a forcing notion, and $P$ its
set of conditions.

(a) For any $\Bbb P$-name $\dot X$,
there is a discriminating $\Bbb P$-name $\dot X_1$
such that $\VVdP\,\dot X_1=\dot X$.

(b) Let $\dot X$ be a
discriminating $\Bbb P$-name, and $f:\dot X\to V^{\Bbb P}$ a function.
Let $\dot g$ be the $\Bbb P$-name

\Centerline{$\{((\dot x,f(\dot x,p)),p):
(\dot x,p)\in\dot X\}$.\cmmnt{\footnote{Once again I present a formula
in which some ordered pairs
are to be interpreted in the ordinary universe, but another is to be
interpreted in the forcing language.}}}

\noindent Then

\Centerline{$\VVdP\,\dot g$ is a function with domain $\dot X$.}

\proof{{\bf (a)(i)} Set

\Centerline{$\dot X_0=\{(\dot x,q):$ there is some
$p\in P$ such that $(\dot x,p)\in\dot X$ and $q\in P$
is stronger than $p\}$.}

\noindent Then $\VVdP\,\dot X=\dot X_0$.   \Prf\ Because
$\dot X\subseteq\dot X_0$, $\VVdP\,\dot X\subseteq\dot X_0$.   In the other
direction, if $\dot x$ is a $\Bbb P$-name and $p$ is such that
$p\VVdP\,\dot x\in\dot X_0$, there are an $(\dot x_1,p_1)\in\dot X_0$ and a
$q$, stronger than both $p$ and $p_1$, such that $q\VVdP\,\dot x=\dot x_1$.
Now there is a $p_2$ such that $(\dot x_1,p_2)\in\dot X$ and
$p_1$ is stronger than $p_2$.   In this case, $q$ is stronger than $p_2$
and $p_2\VVdP\,\dot x_1\in\dot X$, so $q\VVdP\,\dot x=\dot x_1\in\dot X$,
while $q$ is stronger than $p$.   As $\dot x$ and $p$ are arbitrary,
$\VVdP\,\dot X_0\subseteq\dot X$.\ \Qed

Let $\dot X_1\subseteq\dot X_0$ be a maximal discriminating name.

\medskip

\quad{\bf (ii)} Because $\dot X_1\subseteq\dot X_0$,
$\VVdP\,\dot X_1\subseteq\dot X_0=\dot X$.   But we also have
$\VVdP\,\dot X\subseteq\dot X_1$.   \Prf\ Suppose that $\dot x$
is a $\Bbb P$-name and $p\in P$ is such that
$p\VVdP\,\dot x\in\dot X$.    Then there must be an
$(\dot x_1,p_1)\in\dot X$ and a $q$ stronger than both $p$ and $p_1$
such that $q\VVdP\,\dot x=\dot x_1$.   In this case,
$(\dot x_1,q)\in\dot X_0$.   By the maximality of $\dot X_1$, there are a
$(\dot y,q')\in\dot X_1$ and an $r$ stronger than both $q$ and $q'$
such that $r\VVdP\,\dot x_1=\dot y$.   Now $r$ is stronger than $p$ and

\Centerline{$r\VVdP\,\dot x=\dot x_1=\dot y\in\dot X_1$.}

\noindent As $\dot x$ and $p$ are arbitrary,
$\VVdP\,\dot X\subseteq\dot X_1$.\ \Qed

So $\VVdP\,\dot X_1=\dot X$, as required.

\medskip

{\bf (b)} Consider

\Centerline{$\{(\dot x,f(\dot x,p),p):(\dot x,p)\in\dot X\}
\subseteq V^{\Bbb P}\times V^{\Bbb P}\times P$.}

\noindent If $(\dot x_0,p_0)$ and $(\dot x_1,p_1)$ belong to $\dot X$,
$p$ is stronger than both $p_0$ and $p_1$, and $p\VVdP\,\dot x_0=\dot x_1$,
then $(\dot x_0,p_0)=(\dot x_1,p_1)$, because $\dot X$ is a discriminating
name;  so $p\VVdP\,f(\dot x_0,p_0)=f(\dot x_1,p_1)$.   By
5A3H,

\Centerline{$\VVdP\,\dot g$ is a function and $\dom\dot g=\dot X$.}
}%end of proof of 5A3K

\leader{5A3L}{Real numbers in forcing languages}
Let $\Bbb P$ be any forcing notion, and $P$ its set of conditions.

\spheader 5A3La\cmmnt{ I have tried to avoid committing myself to any
declaration
of what a real number actually `is';  in fact I believe that at the deepest
level this should be regarded as
an undefined concept, and that the descriptions offered by
Weierstrass and Dedekind are essentially artificial.   But if we are to
make sense of real analysis in forcing models we must fix on some
formulation, so} I will say that a real number is the set of strictly
smaller rational numbers.\cmmnt{ (I leave it to you to decide whether a
rational number is an equivalence class of pairs of integers, or a
coprime pair $(m,n)$
where $m\in\Bbb Z$ and $n\in\Bbb N\setminus\{0\}$, or something else
altogether, provided only that you fix on a construction expressible by a
formula of set theory.)}   \cmmnt{Observe that this has the desirable
effect that}

\Centerline{$\VVdP\,\check\alpha$ is a real number}

\noindent for every real number $\alpha$.

\spheader 5A3Lb Consider the Dedekind complete Boolean algebra
$\RO(\Bbb P)$ and the corresponding space
$L^0=L^0(\RO(\Bbb P))$\cmmnt{ as defined in 364A}.

\medskip

\quad{\bf (i)} For every $u\in L^0$, set

\Centerline{$\vec u
=\{(\check\alpha,p):\alpha\in\Bbb Q$, $p\in\Bvalue{u>\alpha}\}$.}

\noindent Then

\Centerline{$\VVdP\,\vec u$ is a real number.}

\prooflet{\noindent\Prf\ Of course
$\VVdP\,\vec u\subseteq\Bbb Q$.   If $p\in P$, there are an
$n\in\Bbb Z$ and a $q$ stronger than $p$ such that $q\in\Bvalue{u>n}$,
in which case $(\check n,q)\in\vec u$ and $q\VVdP\,\vec u\ne\emptyset$;
accordingly $\VVdP\,\vec u\ne\emptyset$.   Again, if $p\in P$, there
are an $n\in\Bbb N$ and a $q$ stronger than $p$ such that
$q\in\Bvalue{u\le n}$, in which case $\alpha\le n$ whenever
$(\check\alpha,r)\in\vec u$ and $r$ is stronger than $q$, so that

\Centerline{$q\VVdP\,\check n$ is an upper bound for $\vec u$;}

\noindent accordingly $\VVdP\,\vec u$ is bounded above.

If $p\in P$ and $\dot\alpha$, $\dot\beta$ are $\Bbb P$-names such that

\Centerline{$p\VVdP\,\dot\alpha\in\Bbb Q$,
$\dot\alpha\le\dot\beta\in\vec u$,}

\noindent then for any $q$ stronger than $p$ there are an $r$ stronger than
$q$ and $\alpha$, $\beta\in\Bbb Q$ such that

\Centerline{$r\VVdP\,\dot\alpha=\check\alpha$, $\dot\beta=\check\beta$}

\noindent and $(\check\beta,r)\in\vec u$.   In this case,
$\alpha\le\beta$, $r\in\Bvalue{u>\beta}\subseteq\Bvalue{u>\alpha}$,
$(\check\alpha,r)\in\vec u$ and $r\VVdP\,\dot\alpha=\check\alpha\in\vec u$.
As $q$ is arbitrary,

\Centerline{$p\VVdP\,\dot\alpha\in\vec u$;}

\noindent as $p$, $\dot\alpha$ and $\dot\beta$ are arbitrary,

\Centerline{$\VVdP\,\alpha\in\vec u$ whenever $\alpha\in\Bbb Q$ and
$\alpha\le\beta\in\vec u$.}

If $p\in P$ and $\dot\alpha$ is a $\Bbb P$-name such that

\Centerline{$p\VVdP\,\dot\alpha\in\vec u$,}

\noindent then for any $q$ stronger than $p$ there are an $r$ stronger than
$q$ and an $\alpha\in\Bbb Q$ such that $r\VVdP\,\dot\alpha=\check\alpha$
and $r\in\Bvalue{u>\alpha}$.   Now there are a $\beta\in\Bbb Q$ and an $r'$
stronger than $r$ such that $\beta>\alpha$ and $r'\in\Bvalue{u>\beta}$;  in
which case $r'\VVdP\,\dot\alpha<\check\beta\in\vec u$.   As $q$ is
arbitrary,

\Centerline{$p\VVdP\,\dot\alpha$ is not the greatest member of $\vec u$;}

\noindent as $p$ and $\dot\alpha$ are arbitrary,

\Centerline{$\VVdP\,\vec u$ has no greatest member, and is a real number.
\Qed}
}%end of prooflet

\medskip

\quad{\bf (ii)}\cmmnt{ Observe next
that} $\Bvalue{\vec u>\check\alpha}=\Bvalue{u>\alpha}$ for every
$\alpha\in\Bbb Q$.   \prooflet{\Prf\ For $p\in P$,

$$\eqalign{p\in\Bvalue{\vec u>\check\alpha}
&\iff p\VVdP\,\check\alpha<\vec u\cr
&\iff p\VVdP\,\check\alpha\in\vec u\cr
&\iff\text{ for every }q\text{ stronger than }p
\text{ there are }q'\in P,\,\beta\in\Bbb Q\cr
&\mskip50mu\text{ and an }r\text{ stronger than both }q\text{ and }q'\cr
&\mskip50mu
\text{ such that }(\check\beta,q')\in\vec u
\text{ and }r\VVdP\,\check\beta=\check\alpha\cr
&\iff\text{ for every }q\text{ stronger than }p
\text{ there is an }r\text{ stronger than }q\cr
&\mskip50mu
\text{ such that }(\check\alpha,r)\in\vec u\cr
&\iff\text{ for every }q\text{ stronger than }p
\text{ there is an }r\text{ stronger than }q\cr
&\mskip50mu
\text{ such that }r\in\Bvalue{u>\alpha}\cr
&\iff p\in\Bvalue{u>\alpha}\cr}$$

\noindent (514Md, because $\Bvalue{u>\alpha}$ is a regular open subset of
$P$).\ \Qed}

\medskip

\quad{\bf (iii)} In the other direction,
if we have a $\Bbb P$-name $\dot x$ for a real number (that is, a
$\Bbb P$-name such that
$\VVdP\,\dot x$ is a real number), then there is a unique
$u\in L^0$ such that $\VVdP\,\dot x=\vec u$.   \prooflet{\Prf\
For every $\alpha\in\Bbb Q$ we have a
Boolean value $\Bvalue{\dot x>\check\alpha}$ belonging to
$\RO(\Bbb P)$ (5A3F).   It is easy to see that

\Centerline{$\Bvalue{\dot x>\check\alpha}
=\sup_{\beta\in\Bbb Q,\beta>\alpha}\Bvalue{\dot x>\check\beta}$}

\noindent for every $\alpha\in\Bbb Q$,

\Centerline{$\inf_{n\in\Bbb Z}\Bvalue{\dot x>\check n}=0$,
\quad$\sup_{n\in\Bbb Z}\Bvalue{\dot x>\check n}=1$.}

\noindent We therefore have a unique $u\in L^0$ such that

\Centerline{$\Bvalue{u>\alpha}
=\Bvalue{\dot x>\check\alpha}$}

\noindent for every $\alpha\in\Bbb R$ (364Ae).   Now 
$\Bvalue{\vec u>\check\alpha}=\Bvalue{\dot x>\check\alpha}$ for every
$\alpha\in\Bbb Q$, that is,

\Centerline{$p\VVdP\,\vec u>\check\alpha$ iff 
$p\VVdP\,\dot x>\check\alpha$}  

\noindent for every $\alpha\in\Bbb Q$ and $p\in P$, that is,

\Centerline{$p\VVdP\,\check\alpha\in\vec u$ iff 
$p\VVdP\,\check\alpha\in\dot x$}

\noindent for every $\alpha\in\Bbb Q$ and $p\in P$, that is (since both
$\vec u$ and $\dot x$ are $\Bbb P$-names for subsets of $\Bbb Q$),

\Centerline{$\VVdP\,\vec u=\dot x$.  \Qed}}

\medskip

\quad{\bf (iv)} It follows that if $\dot x$ is a $\Bbb P$-name and
$p\in P$ is such that $p\VVdP\,\dot x\in\Bbb R$, then there is a
$u\in L^0$ such that $p\VVdP\,\dot x=\vec u$.   \prooflet{(For there is a
$\Bbb P$-name $\dot y$ such that $\VVdP\,\dot y\in\Bbb R$ and
$p\VVdP\,\dot x=\dot y$.)}

\medskip

\quad{\bf (v)} If $\alpha\in\Bbb R$, then
$(\alpha\chi 1)\sspvec=\check\alpha$.
\prooflet{\Prf\ For any $\beta\in\Bbb Q$ and
$p\in P$,

$$\eqalign{\Bvalue{(\alpha\chi 1)\sspvec>\check\beta}
=\Bvalue{\alpha\chi 1>\beta}
&=1\text{ if }\beta<\alpha,\,0\text{ otherwise}\cr
&=\Bvalue{\check\alpha>\check\beta}\text{ in either case}.
  \text{ \Qed}\cr}$$
}

\spheader 5A3Lc Suppose that $u$, $v\in L^0$.

\medskip

\quad{\bf (i)} $\Bvalue{\vec u<\vec v}=\Bvalue{v-u>0}$.   \prooflet{\Prf

$$\eqalignno{\Bvalue{\vec u<\vec v}
&=\Bvalue{\Exists\alpha\in\Bbb Q,\,\vec u\le\alpha<\vec v}\cr
&=\sup_{\alpha\in\Bbb Q}\Bvalue{\vec u\le\check\alpha<\vec v}\cr
\displaycause{taking the supremum in $\RO(\Bbb P)$, 5A3Fc}
&=\sup_{\alpha\in\Bbb Q}(\Bvalue{\vec v>\check\alpha}
  \Bsetminus\Bvalue{\vec u>\check\alpha})\cr
\displaycause{taking the relative complements in $\RO(\Bbb P)$}
&=\sup_{\alpha\in\Bbb Q}(\Bvalue{v>\alpha}
  \Bsetminus\Bvalue{u>\alpha})
=\Bvalue{v-u>0}.  \text{ \Qed}\cr}$$
}%end of prooflet

\medskip

\quad{\bf (ii)} In particular, if $u\le v$ in $L^0$, then

\Centerline{$\VVdP\,\vec u\le\vec v$ in $\Bbb R$\dvro{.}{}}

\noindent\prooflet{since $\Bvalue{\vec v<\vec u}=0$.  }In the same way,

\Centerline{$\Bvalue{\vec u=\vec v}=\Bvalue{u=v}$}

\noindent for any $u$, $v\in L^0$.

\medskip

\quad{\bf (iii)} $\VVdP\,(u+v)\sspvec=\vec u+\vec v$.
\prooflet{\Prf\ For any $\alpha\in\Bbb Q$,

$$\eqalignno{\Bvalue{\vec u+\vec v>\check\alpha}
&=\Bvalue{\Exists\,\beta\in\Bbb Q,\,\vec u>\beta,\,
  \vec v>\check\alpha-\beta}\cr
&=\sup_{\beta\in\Bbb Q}\Bvalue{\vec u>\check\beta,\,
  \vec v>\check\alpha-\check\beta}\cr
&=\sup_{\beta\in\Bbb Q}(\Bvalue{\vec u>\check\beta}\Bcap
  \Bvalue{\vec v>(\alpha-\beta)\var2spcheck})\cr
&=\sup_{\beta\in\Bbb Q}
  (\Bvalue{u>\beta}\Bcap\Bvalue{v>\alpha-\beta})\cr
&=\Bvalue{u+v>\alpha}\cr
\displaycause{364D}
&=\Bvalue{(u+v)\sspvec>\check\alpha}.  \text{ \Qed}\cr}$$
}

\medskip

\quad{\bf (iv)} $\VVdP\,(u\times v)\sspvec=\vec u\vec v$.
\prooflet{\Prf\
If $u$, $v\ge 0$ in $L^0$ and $\alpha\ge 0$ in $\Bbb Q$,

$$\eqalignno{\Bvalue{\vec u\vec v>\check\alpha}
&=\Bvalue{\Exists\,\beta\in\Bbb Q,\,\beta>0,\,\vec u>\beta,\,
  \vec v>\Bover{\check\alpha}{\beta}}\cr
&=\sup_{\beta\in\Bbb Q,\beta>0}\Bvalue{\vec u>\check\beta,\,
  \vec v>\Bover{\check\alpha}{\check\beta}}\cr
&=\sup_{\beta\in\Bbb Q,\beta>0}\Bvalue{\vec u>\check\beta}\cap
  \Bvalue{\vec v>(\Bover{\alpha}{\beta})\var2spcheck}\cr
&=\sup_{\beta\in\Bbb Q,\beta>0}\Bvalue{u>\beta}\cap
  \Bvalue{v>\Bover{\alpha}{\beta}}\cr
&=\Bvalue{u\times v>\alpha}\cr
&=\Bvalue{(u\times v)\sspvec>\check\alpha}.\cr}$$

\noindent So in this case

\Centerline{$\VVdP\,(u\times v)\sspvec=\vec u\vec v$.}

\noindent Since we have an appropriate distributive law in $L^0$, it
follows from (iii) that the same is true for general $u$, $v\in L^0$.\
\Qed}

\medskip

\quad{\bf (v)} If $\alpha\in\Bbb R$, then
$\VVdP\,(\alpha u)\sspvec=\check\alpha\vec u$.  \prooflet{(Put (iv) and
(b-v) together.)}

\spheader 5A3Ld Suppose that $\familyiI{u_i}$ is a non-empty family in
$L^0$ with supremum $u\in L^0$.   Then

\Centerline{$\VVdP\,\vec u=\sup_{i\in\check I}\vec u_i$ in $\Bbb R$.}

\prooflet{\noindent\Prf\ By (c-ii),

\Centerline{$\VVdP\,\vec u_i\le\vec u$}

\noindent for every $i\in I$, so

\Centerline{$\VVdP\,\sup_{i\in\check I}\vec u_i\le\vec u$.}

\noindent In the other direction, \Quer\ suppose, if possible, that

\Centerline{$\notVVdash_{\Bbb P}\,\vec u$ is the least upper bound of
$\{\vec u_i:i\in\check I\}$.}

\noindent Then there are a $p\in P$ and an $\alpha\in\Bbb Q$ such that

\Centerline{$p\VVdP\,\check\alpha<\vec u$ is an upper bound for
$\{\vec u_i:i\in\check I\}$.}

\noindent In this case,

\Centerline{$p\in\Bvalue{\vec u>\check\alpha}=\Bvalue{u>\alpha}
=\sup_{i\in I}\Bvalue{u_i>\alpha}
=\interior\overline{\bigcupop_{i\in I}\Bvalue{u_i>\alpha}}$}

\noindent (314P, 364L(a-ii)).
There are therefore a $q$ stronger than $p$ and an
$i\in I$ such that $q\in\Bvalue{u_i>\alpha}$;  but in this case
$q\VVdP\,\vec u_i>\check\alpha$, which is impossible, because
$p\VVdP\,\vec u_i\le\check\alpha$.\ \BanG\  So

\Centerline{$\VVdP\,\vec u=\sup_{i\in\check I}\vec u_i$.  \Qed}
}

\spheader 5A3Le Suppose that $\sequencen{u_n}$ is a sequence in
$L^0$, order*-convergent\cmmnt{ (in the sense of \S367)} to $u\in L^0$.
Then

\Centerline{$\VVdP\,\vec u=\lim_{n\to\infty}\vec u_n$.}

\prooflet{\noindent\Prf\ By 367Gb, $u=\inf_{n\in\Bbb N}v_n$ where
$v_n=\sup_{m\ge n}u_m$ for every $n\in\Bbb N$.   Now (d) tells us that

\Centerline{$\VVdP\,\vec v_n=\sup_{m\ge\check n}\vec u_m$}

\noindent for every $n\in\Bbb N$, and also that

$$\eqalign{\VVdP\,\vec u
&=-(-\vec u)
=-(-u)\sspvec
=-\sup_{n\in\Bbb N}(-v_n)\sspvec\cr
&=-\sup_{n\in\Bbb N}(-\vec v_n)
=\inf_{n\in\Bbb N}\vec v_n
=\lim_{n\to\infty}\vec u_n. \text{ \Qed}\cr}$$
}

\leader{5A3M}{Forcing with Boolean algebras}
Suppose that $\frak A$ is a Dedekind complete
Boolean algebra, not $\{0\}$.   \cmmnt{As noted in 5A3Ab,}
$\Bbb P=(\frak A^+,\Bsubseteqshort,1_{\frak A},\downarrow)$ is a forcing
notion.   We have a natural
isomorphism between $\RO(\Bbb P)$ and $\frak A$, matching
each $G\in\RO(\Bbb P)$ with $\sup G\in\frak A$\cmmnt{ (514Sb)};
\cmmnt{by 514M(d-ii),} $\sup G$ will belong to $G$ unless
$G=\emptyset$.   \cmmnt{In this context,} I
will usually identify the two algebras, so that
$\Bvalue{\phi}$ becomes
$\sup\{a:a\in\frak A^+$, $a\VVdP\,\phi\}$,
and\cmmnt{ we shall have} $\Bvalue{\phi}\VVdP\,\phi$ except when
$\VVdP\,\neg\phi$.

The identification of $\RO(\Bbb P)$ with $\frak A$ itself
simplifies\cmmnt{ some
of} the discussion in 5A3L.   We have a $\Bbb P$-name $\vec u$ associated
with each $u\in L^0(\frak A)$, and the formula

\Centerline{$\Bvalue{\vec u=\vec v}=\Bvalue{u=v}$}

\noindent of 5A3L(c-ii) turns into

\Centerline{$u\times\chi a=v\times\chi a\iff a\VVdP\,\vec u=\vec v$}

\noindent whenever $u$, $v\in L^0(\frak A)$ and $a\in\frak A^+$.  
\prooflet{\Prf\

$$\eqalignno{u\times\chi a=v\times\chi a
&\iff (u-v)\times\chi a=0\cr
&\iff |u-v|\wedge\chi a=0\cr
\displaycause{because $L^0(\frak A)$ is an $f$-algebra, by
364C, so we can use 353Pb, or otherwise}
&\iff\Bvalue{|u-v|>0}\Bcap\Bvalue{\chi a>0}=0\cr
\displaycause{364L(b-ii)}
&\iff a\Bcap\Bvalue{|u-v|>0}=0\cr
&\iff a\Bcap\Bvalue{|u-v|\sspvec>0}=0\cr
&\iff a\Bcap\Bvalue{|\vec u-\vec v|>0}=0\cr
\displaycause{assembling facts from 5A3L}
&\iff a\Bcap\Bvalue{\vec u\ne\vec v}=0\cr
&\iff a\VVdP\vec u=\vec v.  \text{ \Qed}\cr}$$
}

\leader{5A3N}{Ordinals and cardinals} Let $\Bbb P$ be a forcing notion, and
$P$ its set of conditions.

\spheader 5A3Na For any ordinal $\alpha$,

\Centerline{$\VVdP\,\check\alpha$ is an ordinal;}

\noindent\cmmnt{moreover, }if $p\in P$ and $\dot x$ is a
$\Bbb P$-name such that

\Centerline{$\VVdP\,\dot x$ is an ordinal,}

\noindent there are a $q$ stronger than $p$ and an ordinal $\alpha$ such
that

\Centerline{$q\VVdP\,\dot x=\check\alpha$}

\cmmnt{\noindent ({\smc Jech 03}, 14.23).}  %can't find this in Kunen 80.

\spheader 5A3Nb If $\Bbb P$ is ccc, then

\Centerline{$\VVdP\,\check\kappa$ is a cardinal}

\noindent for every cardinal\cmmnt{ (that is, initial
ordinal)} $\kappa$\cmmnt{ ({\smc Kunen 80}, VII.5.6;  {\smc Jech 03},
14.34)}.   In particular,

\Centerline{$\VVdP\,\check\omega_1$ is a cardinal, so is the first
uncountable cardinal,}

\noindent and we can write

\Centerline{$\VVdP\,\omega_1=\check\omega_1$, $\omega_2=\check\omega_2$}

\noindent etc.\cmmnt{, if we are sure of being understood.}

\spheader 5A3Nc Suppose that $\Bbb P$ is ccc, and that we have a set $A$,
a $\Bbb P$-name $\dot X$ and a cardinal $\kappa$ such that

\Centerline{$\VVdP\,\dot X\subseteq\check A$ and $\#(\dot X)\le\check\kappa$.}

\noindent Then
there is a set $B\subseteq A$ such that $\#(B)\le\max(\omega,\kappa)$ and

\Centerline{$\VVdP\,\dot X\subseteq\check B$.}

\prooflet{\noindent\Prf\ Let $\dot f$ be a $\Bbb P$-name such that

\Centerline{$\VVdP\,\dot f$ is an injective function with domain $\dot X$ and
$\dot f[\dot X]\subseteq\check\kappa$.}

\noindent For $\xi<\kappa$ set

\Centerline{$B_{\xi}=\{a:a\in A$ and there is a $p\in P$
such that $p\VVdP\,\check a\in\dot X$ \& $\dot f(\check a)=\check\xi\}$.}

\noindent For each $a\in B_{\xi}$, choose $p_{\xi a}\in P$ such that

\Centerline{$p_{\xi a}\VVdP\,\check a\in\dot X$ and
$\dot f(\check a)=\check\xi$;}

\noindent then if $a$, $b\in B_{\xi}$ and $q$ is stronger than both
$p_{\xi a}$ and $p_{\xi b}$, we have

\Centerline{$q\VVdP\,
\dot f(\check a)=\dot f(\check b)$ so $\check a=\check b$}

\noindent and $a=b$.   Thus $\family{a}{B_{\xi}}{p_{\xi a}}$ is an
antichain in $\Bbb P$ and $B_{\xi}$ must be countable;  setting
$B=\bigcup_{\xi<\kappa}B_{\xi}$, $B\subseteq A$ and
$\#(B)\le\max(\omega,\kappa)$.

Now suppose that $p\in P$ and that $\dot x$ is a $\Bbb P$-name such that
$p\VVdP\,\dot x\in\dot X$.   Then

\Centerline{$p\VVdP\,\dot x\in\check A$ and
$\dot f(\dot x)\in\check\kappa$,}

\noindent so
there are a $q$ stronger than $p$, an $a\in A$ and a $\xi<\kappa$ such that

\Centerline{$q\VVdP\,\dot x=\check a$ and $\dot f(\dot x)=\check\xi$.}

\noindent Now $a\in B_{\xi}\subseteq B$, so $q\VVdP\,\dot x\in\check B$.
As $p$ and $\dot x$ are arbitrary,

\Centerline{$\VVdP\,\dot X\subseteq\check B$,}

\noindent as required.\ \Qed}

\spheader 5A3Nd If $\Bbb P$ is ccc, then

\Centerline{$\VVdP\,\cff[\check I]^{\le\omega}
=(\cff[I]^{\le\omega})\var2spcheck$}

\noindent for every set $I$.   \prooflet{\Prf\ Write $\kappa$ for
$\cff[I]^{\le\omega}$.   (i) Let
$\Cal K\subseteq[I]^{\le\omega}$ be a cofinal family with
$\#(\Cal K)=\kappa$.   Then
$\VVdP\,\check\kappa$ is a cardinal, so

\Centerline{$\VVdP\,\check{\Cal K}\subseteq[\check I]^{\le\omega}$
and $\#(\check{\Cal K})=\check\kappa$.}

\noindent If $\dot J$ is a $\Bbb P$-name such that

\Centerline{$\VVdP\,\dot J\in[\check I]^{\le\omega}$,}

\noindent then by (c) there is a countable set $K\subseteq I$ such that
$\VVdP\,\dot J\subseteq\check K$;  now there is an $L\in\Cal K$ such that
$K\subseteq L$, and

\Centerline{$\VVdP\,
\dot J\subseteq\check K\subseteq\check L\in\check{\Cal K}$.}

\noindent As $\dot J$ is arbitrary,

\Centerline{$\VVdP\,\check{\Cal K}$ is cofinal with
$[\check I]^{\le\omega}$ and $\cff[\check I]^{\le\omega}\le\check\kappa$.}

\noindent(ii) \Quer\ If

\Centerline{$\neg\VVdP\,\check\kappa\le\cff[\check I]^{\le\omega}$,}

\noindent then there are a $p\in P$ and an ordinal $\delta$ such
that

\Centerline{$p\VVdP\,
\cff[\check I]^{\le\omega}=\check\delta<\check\kappa$.}

\noindent Now there must be a family $\ofamily{\xi}{\delta}{\dot J_{\xi}}$
of $\Bbb P$-names such that

\Centerline{$p\VVdP\,
\{\dot J_{\xi}:\xi<\check\delta\}$
is cofinal with $[\check I]^{\le\omega}$.}

\noindent By (c) again, there must be for each $\xi<\kappa$ a countable
$K_{\xi}\subseteq I$ such that
$p\VVdP\,\dot J_{\xi}\subseteq\check K_{\xi}$.   Because
$\delta<\cff[I]^{\le\omega}$, there is a $K\in[I]^{\le\omega}$ such that
$K\not\subseteq K_{\xi}$ for every $\xi<\delta$.   In this case,

\Centerline{$p\VVdP\,\check K\in[\check I]^{\le\omega}$ so there is a
$\xi<\check\delta$ such that $\check K\subseteq\dot J_{\xi}$,}

\noindent and there must be a $\xi<\delta$ and a $q$ stronger than $p$ such
that

\Centerline{$q\VVdP\,\check K\subseteq\dot J_{\xi}
\subseteq\check K_{\xi}$.}

\noindent But this implies that $K\subseteq K_{\xi}$, which isn't so.\
\Bang

We conclude that

\Centerline{$\VVdP\,\check\kappa\le\cff[\check I]^{\le\omega}$
and $\check\kappa=\cff[\check I]^{\le\omega}$.
\Qed}}

\vleader{72pt}{5A3O}{Iterated forcing}\cmmnt{ ({\smc Kunen 80}, VIII.5.2)}
If $\Bbb P$ is a forcing notion and $P$ its set of conditions,
and we have a quadruple
$\dot{\Bbb Q}=(\dot Q,\dot\le,\dot 1,\dot\epsilon)$
of $\Bbb P$-names such that $(\dot 1,\Bbbone_{\Bbb P})\in\dot Q$ and

\doubleinset{$\VVdP\,\dot\le$ is a pre-order on $\dot Q$,
$\dot\epsilon$ is a direction of activity
and every member of $\dot Q$ is stronger than $\dot 1$,}

\noindent then $\Bbb P*\dot{\Bbb Q}$ is the forcing notion defined by
saying that its conditions are objects of the form $(p,\dot q)$ where

\Centerline{$p\in P$,
\quad$\dot q\in\dom\dot Q$,
\quad$p\VVdP\,\dot q\in\dot Q$,}

\noindent and that $(p,\dot q)$ is stronger than $(p',\dot q')$ if
$p$ is stronger than $p'$ and

\Centerline{$p\VVdP\,\dot q$ is stronger than $\dot q'$.}

\noindent(\cmmnt{Strictly speaking, }I should add that
$\Bbbone_{\Bbb P*\dot\Bbb Q}
=(\Bbbone_{\Bbb P},\dot 1)$.)\cmmnt{\footnote{This
formulation gives us the freedom to take $\dot\epsilon$ to be
non-trivial.
I do not mean to suggest that it would be reasonable to take
advantage of this.}}

\leader{5A3P}{Martin's axiom} Let $\kappa$ be a regular uncountable
cardinal such that $2^{\lambda}\le\kappa$ for every $\lambda<\kappa$.
Then there is a ccc forcing notion $\Bbb P$ such that

\Centerline{$\VVdP\,\frak m=\frak c=\check\kappa$.}

\prooflet{\noindent ({\smc Kunen 80}, VIII.6.3;  {\smc Jech 03}, 16.13).}

\leader{5A3Q}{Countably closed forcings (a)} Let $\Bbb P$ be a forcing
notion, and $P$ its set of conditions.
$\Bbb P$ is {\bf countably closed} if
whenever $\sequencen{p_n}$ is a sequence in $P$ such that $p_{n+1}$ is
stronger than $p_n$ for every $n$, there is a $p\in P$ which is stronger
than every $p_n$.

\spheader 5A3Qb If $\Bbb P$ is a countably closed forcing notion, then
$\VVdP\,\Cal P\Bbb N=(\Cal P\Bbb N)\var2spcheck$.
\prooflet{\Prf\ Let $P$ be the set of
conditions of $\Bbb P$.   If $p\in P$
and $\dot x$ is a $\Bbb P$-name such that
$p\VVdP\,\dot x\subseteq\Bbb N$, choose $\sequencen{p_n}$ inductively in
$P$ such that $p_0=p$ and, for each $n\in\Bbb N$, $p_{n+1}$ is stronger
than $p_n$ and either $p_{n+1}\VVdP\,\check n\in\dot x$ or
$p_{n+1}\VVdP\,\check n\notin\dot x$.   Let $q\in P$ be stronger than every
$p_n$, and set $A=\{n:q\VVdP\,\check n\in\dot x\}$.   Then
$q\VVdP\,\check n\notin\dot x$ for every $n\in\Bbb N\setminus A$, so
$q\VVdP\,\dot x=\check A\in(\Cal P\Bbb N)\var2spcheck$.   As $p$ and $\dot x$
are arbitrary,
$\VVdP\,\Cal P\Bbb N\subseteq(\Cal P\Bbb N)\var2spcheck$;  of course the
reverse inequality is trivial.\ \Qed

}
Consequently $\VVdP\,\Bbb R=\check\Bbb R$.   \prooflet{\Prf\ Of course
$\VVdP\,\Cal P\Bbb Q=(\Cal P\Bbb Q)\var2spcheck$,
and now it is easy to see that

$$\eqalign{\VVdP\,\Bbb R
&=\{A:\emptyset\ne A\subseteq\Bbb Q,
  \,A\text{ is bounded above and has no greatest element},\cr
&\mskip 200mu
  q\in A\text{ whenever }q\le q'\in A\}\cr
&=\check\Bbb R.  \text{ \Qed}\cr}$$

\noindent}Similarly, $\VVdP\,[0,1]=[0,1]\var2spcheck$.

%\exercises{\leader{5A3X}{Basic exercises (a)}

%\leader{5A3Y}{Further exercises (a)}

%}%end of exercises


\endnotes{
\Notesheader{5A3} In terms of the discussion in {\smc Kunen 80}, \S VII.9,
you will see that I follow an extreme version of the `syntactical'
approach to forcing.
In the first place, this is due to a philosophical prejudice;  I do not
believe in models of ZF.   But it seems to me that quite apart from this
there is a fundamental difference between the sentences

\Centerline{$\frak m=\frak c$}

\noindent and

\Centerline{$\VVdP\,\,\frak m=\frak c$}

\noindent associated with the fact that the symbols $\frak m$, $\frak c$
and even $=$ must be reinterpreted in the second version.   I have tried in
this section to develop a language which can express and accommodate the
difference.   It puts a substantial burden on the reader, especially in
such formulae as $\sup_{i\in\check I}\dot x_i$ (5A3E) and
$((\dot y,f(\dot y,p)),p)$ (5A3K), where you may have to read quite
carefully to determine which parts of the formulae are supposed to be in
the forcing language, and which are in the ordinary language of set theory.
There is an additional complication in 5A3L, where I use the same symbol
$\Bvalue{\mskip15mu}$ for two quite different functions;
but here at least the
objects $\Bvalue{u>\alpha}$, $\Bvalue{\vec u>\check\alpha}$ belong to the
same set $\RO(\Bbb P)$, even if the formulae inside the brackets have to be
parsed by very different rules.
I hope that the clue of a superscripted letter $\dot x$ or
$\check I$ or $\vec u$ will alert you to the need for thought.
Once we have grasped this nettle, however, we are in a position to move
between the two languages, as in 5A3K;  and statements of results such as
5A3P can be shortened by taking it for granted that the preamble
`$2^{\lambda}\le\kappa$ for every $\lambda<\kappa$' refers to the
ground universe, while the conclusion `$\frak m=\frak c=\check\kappa$' is
to be interpreted in the forcing universe.

Of course a large number of different types of forcing notion have been
described and investigated.   In 5A3N I mention some basic facts about ccc
forcings.   Another important class is that of countably closed forcings
(5A3Q).
}%end of notes

\discrpage



